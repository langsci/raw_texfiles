\documentclass[output=paper,colorlinks,citecolor=brown]{langscibook}
\ChapterDOI{10.5281/zenodo.11091819}
\title{Introduction} 
\author{Christopher R. Green\affiliation{Syracuse University} and Samson Lotven \affiliation{Indiana University}}

\abstract{}


\IfFileExists{../localcommands.tex}{
   \addbibresource{../localbibliography.bib}
   \usepackage{langsci-optional}
\usepackage{langsci-gb4e}
\usepackage{langsci-lgr}

\usepackage{listings}
\lstset{basicstyle=\ttfamily,tabsize=2,breaklines=true}

%added by author
% \usepackage{tipa}
\usepackage{multirow}
\graphicspath{{figures/}}
\usepackage{langsci-branding}

   
\newcommand{\sent}{\enumsentence}
\newcommand{\sents}{\eenumsentence}
\let\citeasnoun\citet

\renewcommand{\lsCoverTitleFont}[1]{\sffamily\addfontfeatures{Scale=MatchUppercase}\fontsize{44pt}{16mm}\selectfont #1}
  
   %% hyphenation points for line breaks
%% Normally, automatic hyphenation in LaTeX is very good
%% If a word is mis-hyphenated, add it to this file
%%
%% add information to TeX file before \begin{document} with:
%% %% hyphenation points for line breaks
%% Normally, automatic hyphenation in LaTeX is very good
%% If a word is mis-hyphenated, add it to this file
%%
%% add information to TeX file before \begin{document} with:
%% %% hyphenation points for line breaks
%% Normally, automatic hyphenation in LaTeX is very good
%% If a word is mis-hyphenated, add it to this file
%%
%% add information to TeX file before \begin{document} with:
%% \include{localhyphenation}
\hyphenation{
affri-ca-te
affri-ca-tes
an-no-tated
com-ple-ments
com-po-si-tio-na-li-ty
non-com-po-si-tio-na-li-ty
Gon-zá-lez
out-side
Ri-chárd
se-man-tics
STREU-SLE
Tie-de-mann
}
\hyphenation{
affri-ca-te
affri-ca-tes
an-no-tated
com-ple-ments
com-po-si-tio-na-li-ty
non-com-po-si-tio-na-li-ty
Gon-zá-lez
out-side
Ri-chárd
se-man-tics
STREU-SLE
Tie-de-mann
}
\hyphenation{
affri-ca-te
affri-ca-tes
an-no-tated
com-ple-ments
com-po-si-tio-na-li-ty
non-com-po-si-tio-na-li-ty
Gon-zá-lez
out-side
Ri-chárd
se-man-tics
STREU-SLE
Tie-de-mann
}
   \boolfalse{bookcompile}
   \togglepaper[1]%%chapternumber
}{}

\begin{document}
\maketitle

\section{Introduction}
The papers in this volume contribute to our understanding of linguistic phenomena that span disciplines from phonetics to phonology and morphology, and on to discourse analysis and pragmatics. They further provide theoretical, typological, and descriptive insights centered upon several African languages, but particularly those native to Ghana and surrounding countries. What ties these contributions together is that they are inspired by and make reference -- whether explicitly or implicitly -- to a sizable body of research produced by Ghanaian linguists, and especially by Dr. Samuel Gyasi Obeng, Distinguished Professor of Linguistics at Indiana University. 

\begin{sloppypar}
Obeng's contributions to linguistics are distinctly noteworthy and unique among African and Africanist linguists in their breadth and their extension beyond linguistics to other disciplines. Initially trained in descriptive fieldwork by the eminent Ghanaian linguist, Professor Florence Abena Dolphyne, Obeng later pursued training in phonetics at University of York in England and at the University of California -- Los Angeles, before further expanding his research program to political discourse analysis, indirectness, and his theory of \textit{Language \& Liberty}. Other subjects upon which Obeng theorized and published include conversational prosody, onomasiology, and creole genesis, just to name a few. For each of these topical areas, Obeng’s point of departure has been his own diverse language experiences in Ghana, and particularly those involving his mother tongue Akan. His work inspired a generation of linguists whose research focus is either primarily or entirely centered upon African languages. Indeed, many of the young linguists he trained have established and grown linguistics departments and programs throughout West Africa. 
\end{sloppypar}

One might ask how such a diverse research trajectory possibly came to be, but Obeng's upbringing as the son of a village chief in Asuom, Ghana, undoubtedly played a role. As intimated in the poem that he contributed to this volume (see preface), Obeng grew up in trying times and within a national education system that did nothing to celebrate Ghanaian languages and cultures. Despite facing ridicule and antipathy from his colleagues and elders, he held fast to his love for the diversity of Ghanaian languages, including Akan, and the intricacies of the practices surrounding its use in a variety of social and political contexts.

Indeed, having been raised with daily exposure to a litany of discursive political strategies for activities such as turn taking, apologizing, face saving, cross-examination (among a host of other interactional pragmatic phenomena), and above all ``speaking the unspeakable'', Obeng's first research interests centered on the area of conversational phonetics, or what have come to be called ``voice-prints'' in conversational interaction. His 1988 dissertation \textit{Conversational strategies: Towards a phonological description of Projection in Akyem-Twi} analyzed the use of Firthian ``prosodies'' like silence, pitch, tempo, and length, among other phonetic correlates to negotiate their interactions.

This work carried forward and expanded upon an already strong tradition of descriptive and theoretical linguistics in Ghana built by the likes of Dolphyne, but also Africanists like Mary Esther Kropp Dakubu, an American linguist who spent her professional life in Ghana and working on Ghanaian languages. Alan Stewart Duthie, a Scottish linguist, similarly devoted himself to Ghana and the exploration of its languages. Other prominent Ghanaian linguists who have contributed to this tradition include Felix Ameka, Gilbert Ansre, Adams Bodomo, James Essegbey, Kwesi Yankah, George Akanlig-Pare, and Kofi Agyekum. A new generation of Ghanaian linguists, many of whom were trained by this cadre, includes many already well-known names -- Clement Appah, Mercy Bobuafor, Seth Ofori, Nana Aba Appiah Amfo, Fusheini Hudu, Esther Manu-Barfo, Michael Obiri-Yeboah, Augustina Owusu, Eyo Mensah -- and certainly others too numerous to list. We, of course, are very happy to have had several of these individuals contribute to this volume, whether as authors or reviewers.

This research tradition, having originated in Ghana, has since transcended national boundaries via these individuals' scholarship. Scholars like Ameka, Bodomo, and Obeng, in particular, not only have supervised many of those listed above, but have also had a lasting influence on colleagues at universities around the globe, and a body of students who have become Africanist linguists. Several of these individuals, trained by these eminent Ghanaian linguists, are contributors to this volume.


\section{Structure of this volume}

The remainder of this volume contains 12 chapters from 16 contributors all of whom have drawn inspiration -- methodological, analytical, empirical, or otherwise -- from the foundational work described above. The first six chapters specifically concern themselves with languages native to Ghana. The two chapters that follow reflect extensions of the discourse methodologies developed for Ghanaian languages, but in the context of Nigeria. The last four chapters celebrate the challenges and outcomes of linguistic fieldwork on African languages more broadly. 

Reflecting on an ongoing research program developed over several decades, de Jong (Chapter 2) reveals the unique timing relationships between consonants and vowels in Akan. Motivated by the uncommonly large inventory of fricatives in the language, as well as the distribution of secondary labial and palatal features vis-à-vis flanking vowels, de Jong's analysis takes into account phonology, historical linguistics, and phonetics, presenting a path from observation to experimentation. Akan appears to have undergone two processes of incorporating vocalic features into the consonantal system, first losing some instances of /u/ in favor of secondary labial articulations, and then, more recently, losing some instances of /i/ in favor of secondary palatal articulations. Such diachronic changes are reflected in modern co-occurrence restrictions. This chapter also describes ongoing research involving the rhythmic alignment of these secondary articulations within the syllable.

\begin{sloppypar}
Ofori (Chapter 3) untangles the complex phenomenon of glide-onset formation in Akan CV\textsubscript{1}V\textsubscript{2} words. Prompted by noted shortcomings in the literature concerning the emergence of [j]- vs. [w]-onsets in the language, Ofori argues that two phonotactic constraints are the overarching drivers of glide choice. While seeking to avoid sequences of V\textsubscript{[+High]}V\textsubscript{[-High]} overall, often accomplished via loss of the first vowel, the language is also faced with the need to avoid losing this vowel's contrastive place features. It is shown that the choice of alternation, and indeed of glide-onset, depends not only on underlying characteristics of V\textsubscript{1}, but also the [Labial] specification of a preceding consonant. In some instances, secondary influences emerge based on characteristics of V\textsubscript{2}. Intriguing distinctions reveal themselves, such as the fact that high back vowels that differ only in their [ATR] specification participate differently in the process. Ultimately, Ofori illustrates that the phenomenon of glide formation is far more intricate than the published literature thus far would lead one to believe.
\end{sloppypar}

Lotven \& Ajibade (Chapter 4) probe the relationship between ideophones and morphology in two isolating West African languages: Gengbe, a Gbe language (Niger-Congo) spoken primarily in Togo, but also in Benin and Ghana; and Yoruba (Niger-Congo), spoken primarily in Nigeria, but also in Benin and Togo. This paper considers various word formation processes such as compounding, reduplication, and tonal morphology which play a part in the production of ideophones and prosaic words (non-ideophones) alike. In doing so, they offer evidence that research which sets aside ideophones in the course of linguistic work risks ignoring rich swaths of the lexicons of these and other West African Languages. Such comparison suggests that ideophones, with their extra-linguistic baggage and depictive-rather-than-descriptive function, behave very much like other occupants of the lexicon in their derivational morphology.

The next chapter explores the ways in which a society views its relationship with the natural world, as reflected in the metaphors its people sew into proverbs, poems, and every day conversation. Agyekum (Chapter 5) offers the reader a view of Akan society in this local-ecological and culturally-specific mirror by examining plant metaphors (plantosemy) in various domains of daily and ritualized experience. Such cultural concepts analogize parts of life and human activity to the life cycles and features of various Ghanaian plants. By laying bare the ideas that underlie these expressions, this research illuminates both the unique ecological interconnections of the Akan and the universality of the human-plant relationships on which these expressions are built. Such research is also of documentary value, since Agyekum notes that many of these metaphors are falling out of use among younger speakers increasingly familiar with metaphors involving non-native plants.

As part of a larger language documentation project on the moribund Ghanaian language Dompo, Manu-Barfo (Chapter 6) shines a light on the daily and ritualized interactions that oil the cogs of communication. The routine parts of human interaction are a network of unwritten rules governing communication. Their adherence or violation is glaringly obvious to those who grew up within the community of their practice. This chapter includes discussions of the different expectations that native Dompo speakers have for greetings that happen at different parts of the day, in different locations, and situations. It also covers expressions of gratitude, congratulations, and sympathy, as well as non-verbal aspects of greeting and greeting taboos. For a language at such an advanced state of decline and attrition, such daily expressions play an outsized role in conversation, since they may even be known and used by community members not fluent in Dompo. And, as this language disappears, this chapter contributes to the preservation of its memory and the memories of its last three speakers, consulted for this project.

In their cross-linguistic survey of 31 Niger-Congo languages from 14 distinct groups, Schaefer \& Egbohare (Chapter 7) cover three major languages of Ghana (Akan, Ga, and Ewe), as well as three additional ``Togo Mountain Languages'' (Logba, Tafi, and Avatime) that are spoken in the border regions across Togo and Ghana. The focus of this typological survey is `give' predications, or constructions that encode changes of possession between a Theme argument and a Recipient argument. Revealed in this survey is that three constructions are attested among these groups and that they are distributed areally. Moreover, it is shown that two of the three linkage types that involve `give' verbs ``couple'' such that they are found within the same languages. Elsewhere, in languages that do not encode change of possession with `give,' only a single predication type, or linkage, is possible. A few atypical linkage types are also reported and their unusual characteristics discussed in light of the broader categories.

The next two chapters of the volume take us outside of Ghana itself, illustrating the inspiration that Ghanaian scholars like Obeng and Agyekum have had on the analysis of African political discourse elsewhere in West Africa, and here, specifically in the Nigerian political arena. Drawing upon discursive tools like indirectness, intertextuality, and (more broadly) interactional pragmatics, Alabi (Chapter 8) analyzes in detail aspects of an open letter written by one former president of Nigeria -- Chief Olusegun Obasanjo -- to then-president of Nigeria Dr. Goodluck Jonathan. Alabi illustrates a variety of powerful discursive mechanisms that Obasanjo employed to draw attention to shortcomings and apparent falsehoods that he attributed to Jonathan's words and actions as he debated running for a second term as Nigeria's president. Obasanjo is shown to have drawn upon his own publically-spoken words, as well as those of Jonathan, to support and provide legitimacy to his writing of the open letter. In doing so, he further invokes proverbs and Biblical verses to situate his aversion to Jonathan pursuing a second term in office. Alabi hypothesizes that this action by Obasanjo may have ultimately played a key role in Jonathan losing the election.

Working within the area of onomastics, Ehineni (Chapter 9) uses the social and behavioral context of Yoruba naming traditions to offer insights into the Yoruba culture and people. A Yoruba child may have five or six different names, proposed to the parents by various family members -- the many names expressing diverse rhetorical and pragmatic goals. Yoruba names may indicate the place someone was born, or events in the life of the family during the time of birth. They may offer information about how birth occurred, the birth order of children, or previously deceased siblings. Names also tell something of a family's lineage, beliefs, and professions. Ehineni illustrates that names in the Yoruba tradition are communicative acts to those with the context to understand their significance.

While best known for his work on political discourse analysis, always underlying Obeng's work has been a deep commitment to the description and documentation of African languages, particularly those that are threatened with endangerment. This tradition and this commitment, as introduced above, carries forward his own training, as he has instilled these values in dozens of students whom he has mentored, especially those in Ghana. In bringing this fieldwork tradition to the US, Obeng -- alongside Africanist colleagues Paul Newman, Roxana Ma Newman, and Robert Botne -- pressed and challenged their students to re-think how to do ``fieldwork'', given ever-dwindling and increasingly competitive funding, sometimes dangerous or uncertain political conditions in many parts of the world, and also decreased time available to spend working \textit{in situ}. In doing so, students' attention was brought to the value of working with diaspora communities and field methods courses to gain insight into lesser-known languages and thus to form and test hypotheses about them. The last four chapters of this volume celebrate the challenges of fieldwork on African languages.

Newman (Chapter 10) revisits apparent morphophonological ``oddities'' from three Chadic langauges -- Hausa, Kanakuru, and Tera -- that have long been treated as exceptions among experts on these languages. He illustrates that with further insight into the historical context now known about these languages, behaviors and characteristics previously deemed divergent find a ready explanation. As Newman suggests, such apparent oddities should not be relegated to the descriptive and analytical sidelines for long, as they have the potential to shed important light on other aspects of the language that might have thus far escaped notice.  

Hantgan, Green, \& Contreras Roa (Chapter 11) reflect upon a connection to earlier research on complex Ghanaian vowel harmony systems which laud the benefits of focusing on a targeted array of related constructs -- in essence, the Firthian-style analytical ``piece'' -- as a means to unravel the intricacies of the language's vocalic system and how it functions more broadly. In this chapter, Hantgan et al. delve into the surprising behavior of Bondu So verbs -- particularly a subset of inflectional forms of the Past and Chaining stems -- for which stems with a [+ATR] vowel and that end in a sonorant preclude suffixation in some contexts. Their results, while preliminary, suggest that there are acoustic correlations between the presence of the [ATR] feature on these stem vowels and increased length in stem-final sonorants. They argue that these results implicate the ability of [ATR] to license the projection of sonorant moras and thus illustrate a different type of ``contextual weight'' than has otherwise been reported in the literature.

Stepping away from West Africa, Okelo (Chapter 12) revisits the challenging morphophonological behavior of Dholuo pluralization. Despite being better described than many African languages, Okelo points out that a satisfying analysis of how Dholuo nouns form their plurals remains elusive, in part because proposed analyses typically only account for a subset of attested patterns. By dividing nouns into classes by pluralization strategy, Okelo separately tackles three types of suffixation, two types of suppletion, and also pluralization by morphological subtraction. Her proposed analysis of six interrelated alternations via \textit{-e} expands upon accounts offered elsewhere in the literature that posit an abstract underlying form of this suffix that never surfaces but nonetheless would appear to have a clear and consistent influence on several outcomes. Ultimately, her analysis stipulates that metrification and phonotactics are key factors that, taken together, permit a unified account of observed outcomes.

In the final chapter of this volume, Green \& Smith (Chapter 13) present data highlighting the tonal and morphological behavior of two Maay dialects -- Kenyan Maay and Baydhabo (Baidoa) Maay -- and, more specially, their nominal systems. Green \& Smith argue that although both varieties exhibit restricted tone systems, what is known about these and related languages diachronically suggests that they have come to exhibit more stress-like properties over time, yet have done so in different ways. The authors posit that in approaching a ``pivot'' point between tonehood and stresshood, the varieties remain definitionally tonal by different parameters. This, it is shown, appears to correlate with their somewhat divergent historical responses to the loss of tone bearing units at the right edge of words. Another variety of Maay (Lower Jubba Maay) and Somali are presented for further comparison.

{\sloppy\printbibliography[heading=subbibliography,notkeyword=this]}
\end{document}
