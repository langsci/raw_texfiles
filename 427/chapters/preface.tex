\addchap{Preface}
\begin{refsection}

\noindent In addition to his many contributions to the field of African linguistics, among Samuel Gyasi Obeng's most treasured hobbies and past times is writing short stories and poetry. Indeed, an accomplished poet, Obeng has published three collections of his poems: \textit{Yɛse Yɛsee}, `Rumor mongering' (1993, Bureau of Ghana Languages), \textit{Voices from the graves: Words of wisdom and caution from the departed} (2008, Author House), and \textit{A nation in crises} (2022, Page Publishing). We are grateful to Sam for kindly agreeing to provide a poem for this volume, and one that offers deep, personal insight into his upbringing, early education, and the love that he has for his language and for Ghanaian culture.
\vspace{.2in}

\hfill \textit{crg \& sl}
\vspace{.2in}

%content goes here
\noindent \textbf{Sunset in the morning}\\
\noindent \textit{Samuel Gyasi Obeng}
\vspace{.2in}

\noindent It almost ended before it began\\
It should have ended at dawn on a night\\
\hspace*{10mm}of the thirty-sixth moon after my birth\\
No one gave me a chance at survival\\
Not even the Salvation Army clergy,\\
\hspace*{10mm}nor the Nzema woman who brought me back from Hades.\\
The clergy named me \textit{Rejected-by-Death}\\
For Death did not like me and so refused to take me with him\\
My father, Gyasi, Brave Buffalo, named me \textit{Ugly-and-Useless}\\
And via that name hid me from Death and the Ancestral Phantoms\\
Yes, the Death-Prevention name gave me the opportunity and power to live.\\
Death, however, did not spare my two younger sisters, \\
Ama Oyuo, \textit{The-Black-Antelope}, and Ama Fanosaara, \textit{Take-Her-As-She-Is} \\
For, the Squad-of-Killers sent by Death via the measles epidemic\\
\hspace*{10mm}had no mercy on them.\\
\newpage
\noindent I witnessed their small coffins being carried away to the cemetery\\
\hspace*{10mm}one after the other \\
\noindent That same day; that Tuesday in June, just before \textit{Ohumuu},\\
\hspace*{10mm}the Festival-of-Harvest. \\

\noindent One Monday in a September, a bearded man, Tikya Brako,\\
\hspace*{10mm}hauled me from his mother’s arms.\\
I saw a tear in my Mom’s eyes but somehow clung to the man,\\
\hspace*{10mm}Tikya Brako \\
Later in life, I understood the meaning\\
\hspace*{10mm}of the tear that was shed.\\
\noindent Having lost two of her children,\\
\hspace*{10mm}she did not want the third being hauled away.\\
Tikya Brako assured my mom with the words,\\
\hspace*{10mm}\textit{I’ll bring him back at noon; he must begin school}.\\
Tikya Brako carried me on his shoulder and bid my mom farewell.\\
As we left the house, the house’s gate squeaked and closed after us,\\ 
As if to say, \textit{Isn’t he too young to begin Grade One}?\\
On arrival at the school, I did not see much of a school\\
Seven black slates, each with a chalk resting on it;\\
Scattered in the dust, yawning and staring at me.\\

\noindent This is the Salvation Army Primary School,\\
\hspace*{10mm}Tikya Brako whispered.\\
You and your friends are going to be the pioneers of the school\\
And your classroom is this huge and comfortable space \\
Under these two tall mango trees;\\
\hspace*{10mm}you’ll get mangoes in the weeks to come.\\
The morning winds swung the mango trees’ branches to and fro \\
And beckoned me to sit in the dust that danced with the winds.\\
Tikya Brako made six other trips\\
Each time carrying a poor little soul, \\
Yes; they were all ‘infants.’\\
The oldest, Owusu, was six\\
And I, Kwadwo Nyankomago, \textit{Second-After-Twins}, was four.\\

\noindent Five of boys and two girls, Boatemaa and Dokuaa,
Constituted the Unit,\\
\hspace*{10mm}the first batch of pupils, of that new school.\\
\newpage

\noindent The three other boys, Kwadwo Mòtó, Kwame Dukuro and Yaw Minta,\\
\hspace*{10mm}were somehow all emotionally and probably mentally challenged.\\
\noindent There were no special schools those days\\
\hspace*{10mm}for the mentally and emotionally challenged\\
And no other school would take them during those \textit{Days and Years of Ignorance}\\
So the Salvation Army accepted them,\\
\hspace*{10mm}as it would of others in the years that followed. \\
Those days, the belief was that giving such children an education\\
\hspace*{10mm}was against the wishes of the gods\\
For they were believed to be agents of terror\\
\hspace*{10mm}and torture of the other children. \\
\noindent And even though they may have unleashed terror and torture\\
\hspace*{10mm}every now and then,\\
They made the classroom fun for one and all.\\

\noindent Our teacher, Tikya Brako, had a Seventh-Grade education\\
But he gave his all to the best of his ability\\
Often times making what will now constitute a professional malpractice\\
By composing songs that we sang in class\\
To ridicule one of the emotionally challenged classmates, Kwame Dukuro. \\
This song has stuck with me, that I still sing in tears and in shame\\
The words of the song in the indigenous language, Akan, were: \\
\hspace*{10mm}\textit{Mankɔ sukuua, anka meyɛ aboa, keteke!}\\
\hspace*{10mm}‘Had it not been for school/education, I would have been an animal’\\
\hspace*{10mm}\textit{Kwame Dukuro anka meyɛ aboa, keteke!}\\
\hspace*{10mm}‘Kwame Dukuro, I would have been an animal.’\\

\noindent Two years of Middle School at a Presbyterian Middle School\\
Seven years at a Presbyterian run Secondary School,\\
\hspace*{10mm}five in my village, Asuom,\\
Before been blown by the winds of education\\
To learn more about the white man’s ways and thinking\\
\hspace*{10mm}in College and Graduate school.\\
Stubbornness or love for my culture caused me to study my own tongue\\ \hspace*{10mm}all those however many years\\
Subjected to ridicule by my classmates,\\
\hspace*{10mm}especially during Middle and Secondary school\\
And punished by the teachers if I spoke my own tongue at school.\\
\newpage

\noindent I persisted in learning and speaking that on-campus tabooed tongue\\
\hspace*{10mm}Akan, into College\\
Where I ended up studying Linguistics\\
Which in my Ghanaian culture is often interpreted as, \\
A chief’s spokesperson, \textit{Ɔkyeame}.\\
A proud, Ɔkyeame; Spokesperson, Orator, Editor\\
\hspace*{10mm}and Educator of Chiefs, Kings and intellectual giants!\\

\noindent I was given no chance of survival\\
I was given a Death-Prevention name that made me live\\
In dust under two mango trees \\
\hspace*{10mm}I first breathed an education that was termed formal\\
Nurtured by a teacher who gave his all despite his many hiccups\\
Immersed in foreign religions and their denominations \\
Yet continued to love and learn my culture and native tongue\\
Shaped by the culture that gave birth to me\\
Yet respectful of all cultures that have nurtured my professional life\\
A professional life that has become a web of constellation\\
A professional life that has taken me away from my homeland\\
\hspace*{10mm}to a land called Home-of-the-Brave\\
Yes, it takes the brave bird to take a flight to the unknown!\\

\noindent I end my song, my poem and my story with proverbs, that have guided my personal and professional journeys:\\

\noindent\textit{Wohu anomaa korɔ a, ɛnto no tuo.}\\
If you see a lone bird, do not shoot it.\\ \\
\hspace*{10mm}\textit{Do not persecute the defenseless.}\\

\noindent\textit{Ɔhohoɔ nte nea ne bɔtɔ teɛ.}\\ 
A visitor does not hear what his luggage hears.\\ \\
\hspace*{10mm}\textit{Don’t be naive about the good that people say in your presence;}\\
\hspace*{10mm}\textit{their insults are said behind your back.}\\

\newpage
\noindent\textit{Wohwɛ asubɔnten nkyɛnmkyɛn apopɔbibire a, anka worennom nsuo da.}\\
If you look at the spirogyra on a river’s banks, you’ll never drink from the river.\\ \\
\hspace*{10mm}\textit{Be welcoming to one and all}.\\

\noindent\textit{Sɛ obi hwɛ wo na wo se fifi a, hwɛ no berɛ a, ne se retutu.}\\
If someone nurtures you when you're teething, you must look after her when she's losing her teeth.\\ \\
\hspace*{10mm}\textit{Return a favor to those who helped you early in life.}\\

\noindent\textit{Wote sɛ obi retu ne ba fo a, na wode bi ato wo kotodwe mu.}\\
When you hear someone advising her child, keep some in your knees so that you walk everywhere with it.\\ \\
\hspace*{10mm}\textit{Learn from those who know better than you}.\\

\noindent\textit{Sɛ obi yɛ wo bɔne apem ne papa baako a, hwɛ papa korɔ no so na wo ne no nyɛ adwuma}.\\
If someone does a thousand wrongs against you and does one right, look at that one right and word with him.\\ \\
\hspace*{10mm}\textit{Wrongs are easy to count but goodness is easily forgotten}.\\

\noindent\textit{Mepɛ me sankuo dɛ nti na sɛ merebɔ a, na makata m’ani and makyea m’ano no.}\\
It is because I want my hand piano to sound good that is why I close my eyes and twist my lips while playing it.\\ \\
\hspace*{10mm}\textit{Create the necessary conditions to ensure success of your endeavors}.\\

\noindent\textit{Obi nsi ne ho hene, enti kae na ma w’ani nsɔ nea nipa ayɛ ama wo.}\\
No one crowns himself chief, so remember and appreciate what people do for you.\\ \\
\hspace*{10mm}\textit{Be appreciative of the help you receive from others because you got to}\\
\hspace*{10mm}\textit{where you are now because of other’s generosity and assistance.}\\
\newpage

\noindent\textit{Sɛ yaw ne abɛbrɛsɛ ka w’ani ɔbra ahyɛaseɛ a, ɛma wo atimpirim,}\\
\textit{denhyɛ, yiedie ne nkɔsoɔ.}\\
A life characterized by sunset in the morning survives and endures.\\ \\
\hspace*{10mm}\textit{Tough beginnings prepare one for ultimate success in life.}\\

%{\sloppy\printbibliography[heading=subbibliography]}
\end{refsection}

