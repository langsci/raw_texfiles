\documentclass[output=paper,colorlinks,citecolor=brown]{langscibook}
\ChapterDOI{10.5281/zenodo.11091829}

\author{Esther Desiadenyo Manu-Barfo\affiliation{University of Ghana, Legon\\La Trobe University, Melbourne, Australia}}
\title{The socio-pragmatics of routine expressions in Dompo} 
\abstract{The basic communicative mechanisms used by members of a speech community provide insights into their culture. Greetings, expressions of gratitude, congratulations, sympathies, and farewells all form part of the formulaic and socially specific expressions that people understand and respond to accordingly in a community. This paper describes routine expressions in Dompo, a \is{language endangerment} moribund language spoken in the North-Western corner of the Bono Region of Ghana. In 2018, four of the six remaining speakers of Dompo provided data for this project as part of a larger description of the language. Presently, there are three remaining speakers of the language. Some non-linguistic cues, as well as circumstances that condition these expressions and the underlying meanings they convey are discussed. The paper establishes that routine expressions form an integral part of the existence of the Dompo people, having social and cultural norms attached to them. With so few native speakers left, this paper provides a snapshot of some of the most common and accessible conversations available at such a late stage of language shift and attrition.}

\IfFileExists{../localcommands.tex}{
   \addbibresource{../localbibliography.bib}
   \usepackage{langsci-optional}
\usepackage{langsci-gb4e}
\usepackage{langsci-lgr}

\usepackage{listings}
\lstset{basicstyle=\ttfamily,tabsize=2,breaklines=true}

%added by author
% \usepackage{tipa}
\usepackage{multirow}
\graphicspath{{figures/}}
\usepackage{langsci-branding}

   
\newcommand{\sent}{\enumsentence}
\newcommand{\sents}{\eenumsentence}
\let\citeasnoun\citet

\renewcommand{\lsCoverTitleFont}[1]{\sffamily\addfontfeatures{Scale=MatchUppercase}\fontsize{44pt}{16mm}\selectfont #1}
  
   %% hyphenation points for line breaks
%% Normally, automatic hyphenation in LaTeX is very good
%% If a word is mis-hyphenated, add it to this file
%%
%% add information to TeX file before \begin{document} with:
%% %% hyphenation points for line breaks
%% Normally, automatic hyphenation in LaTeX is very good
%% If a word is mis-hyphenated, add it to this file
%%
%% add information to TeX file before \begin{document} with:
%% %% hyphenation points for line breaks
%% Normally, automatic hyphenation in LaTeX is very good
%% If a word is mis-hyphenated, add it to this file
%%
%% add information to TeX file before \begin{document} with:
%% \include{localhyphenation}
\hyphenation{
affri-ca-te
affri-ca-tes
an-no-tated
com-ple-ments
com-po-si-tio-na-li-ty
non-com-po-si-tio-na-li-ty
Gon-zá-lez
out-side
Ri-chárd
se-man-tics
STREU-SLE
Tie-de-mann
}
\hyphenation{
affri-ca-te
affri-ca-tes
an-no-tated
com-ple-ments
com-po-si-tio-na-li-ty
non-com-po-si-tio-na-li-ty
Gon-zá-lez
out-side
Ri-chárd
se-man-tics
STREU-SLE
Tie-de-mann
}
\hyphenation{
affri-ca-te
affri-ca-tes
an-no-tated
com-ple-ments
com-po-si-tio-na-li-ty
non-com-po-si-tio-na-li-ty
Gon-zá-lez
out-side
Ri-chárd
se-man-tics
STREU-SLE
Tie-de-mann
}
   \boolfalse{bookcompile}
   \togglepaper[2]%%chapternumber
}{}

\begin{document}
\SetupAffiliations{mark style=none}
\maketitle

\section{Introduction}

The communities\il{Dompo|(} we come from and live in propel us to conform to their conventional norms and customs. These invisible guidelines are part of the community and have been part of them for a long time, having been passed down from one generation to the next. Some of these norms may require that members of a society undergo orientation and initiation to better understand them, and this social competency enables people to be  fully considered bona fide community members. For instance, in some communities across Africa, to be ushered into adulthood, it is required of both males and females to undergo certain rites. In some ethnic groups in Eastern and Southern Africa, males must undergo a circumcision rite to usher them into manhood (\cite{BarkerRicardo05,Siweyaetal2018}), and females in some African societies, as well as some North American Indian cultures, may go through certain rites to be ushered into womanhood (\cite{Crentsil2015, Gyan2017, Johnson2018, Markstrom2008}). Such rituals are often performed by females at their menarche (i.e., the first occurrence of menstruation). In general, the groups of people who will be undergoing such \is{puberty rites} rites have been conscientized about them and have even been groomed to undergo them. The aforementioned initiation rites for both genders largely involve other members of the community who either actively participate in the enactment process or serve as observers. On the other hand, there are other rites that do not involve rigorous initiations. 

These are considered practices and social customs that are taught to children from a young age by their immediate families. They include routine expressions such as greetings, expressions of gratitude, congratulations, sympathies, and farewells. They are classified as routine expressions because they form part of one's daily interactions with other people. They are a necessary component of locutionary exchanges with members of our society and are “crucial for the establishment and maintenance of interpersonal relationships” (\cite[171]{Wójtowicz2021}). While different speech communities may employ varied verbal and non-verbal cues to accompany these expressions, it should be noted that many linguistic communities also share several similarities when it comes to the norms regulating the expression of daily ritual exchanges. For instance, in many African cultures where greetings may be accompanied by a handshake, it is \is{politeness} considered anathema to use the left hand in doing so. In fact, it is a taboo in most African cultures to use the left hand to make gestures such as pointing, to interact with people, to give and receive items,\ and to eat and drink (\citealt{Alhassan2018, AmekaandBreeveld2004, KitaEssegbey2001, Wójtowicz2021}). This is because the left hand is deemed dirty or filthy, since it is used to perform ablutions (\cite[170]{AmekaandBreeveld2004}). 

In this paper, I elaborate on the routine expressions of greetings, congratulations, gratitude, sympathy, and farewells in the critically endangered language Dompo. I discuss their usefulness in serving as a testament to the cordial relationships that exist among members of the Dompo community. Their absence is generally an indication of a strain in relationship. I further instantiate the various forms that these routine expressions take and additionally draw on the different aspects of similarities and variations that characterize them in some African languages. Most routine expressions are accompanied by non-verbal cues. I further explore these as well and the social dynamics that exist between interlocutors. Finally, I argue that with the tremendous decline in the active usage of Dompo among the very few remaining native speakers, various aspects of routine expressions form part of the everyday repertoire of some of the non-fluent natives. For the handful of natives who remember some vocabulary and a few sentences in Dompo, many of these are from routine expressions. It is particularly important to document routine expressions in a moribund language such as Dompo precisely because they make up an important and substantial proportion of the communication that occurs between fluent speakers, as well as between fluent and non-fluent speakers.

\begin{sloppypar}
Data on Dompo greetings and routine expressions presented below were collected in Dompofie and form part of a larger corpus of data drawn upon for the write-up of the only descriptive grammar of the language \citep{Manu-Barfo2020}. Our native speakers provided data on the greeting systems in the language through elicitation procedures and staged communicative events in 2018. I also observed how greetings were done among the speakers and asked follow-up questions to gain more insight. The sessions were audio and video recorded, and later analysed, glossed, and translated into English.
\end{sloppypar}

\section{About the moribund Dompo language}

Endangered languages \is{language endangerment} are once-active languages with communities of native speakers who took pride in the knowledge of their language, culture, and heritage. The dynamics of our societies, coupled with some natural and unnatural causes force languages to compete against each other (\cite[56]{Sallabank2012}). Languages that are winning the race are those deemed to have more economic, political, and cultural value. Their speakers are in the majority, and their languages have more of an appeal to attract speakers of minority languages. These and many associated factors are leading to the loss of an estimated half of the world's approximately 6,000--7,000 languages by the next century (\cite[1]{GrenobleandWhaley2006}). 

One such language on the brink of extinction is Dompo, which has three remaining speakers at the time of writing. Dompo is spoken in a small village called Dompofie which means ‘home of the Dompo people’. Dompofie is in the North-Western part of the Bono Region of Ghana. The language was thought to be extinct until the first linguistic work on it was conducted in 1999 by \citet{Blench2007}. In his work, he recorded about 60--70 speakers who had some command of the language and about 10 people who could remember some obscure words. Over the years, the number of speakers has dwindled tremendously, \is{language endangerment} largely due to death and emigration from Dompofie (\cite{Manu-Barfo2020}). 

In 2016, when work on the descriptive grammar of the language began, there were six remaining speakers. Three speakers have passed on, bringing the current number of remaining speakers to three. Dompo is believed by some to be a language isolate (\cite{Blench2007, Dimmendaal2011}) based on three claims made by \citet{Blench2007}, the last of which advances the position that Dompo may be a language of unknown origin, relexified from \ili{Gonja} and other languages. The two others stipulate that Dompo is related to Gonja based on the various lexical forms they share, and lastly, that Dompo is a dialect of Gonja that has been heavily influenced by other languages. More intensive research on Dompo has, however, concluded that Dompo should rather be classified as part of the North Guang language group based on some phonological, lexical, and social evidence (\cite{Manu-Barfo2019, Manu-Barfo2020}). This conclusion aligns with the classifications of Dompo made by \citet{Güldemann2018}, \citet{Painter1967}, and  \citet{Ethnologue21}. The native Dompo speakers and their language have been submerged by the dominant Na\-faanra language and its speakers. The Dompos claim they are the autochthonous settlers of the lands (\cite{Goody1964, Stahl91}). The Nafaanras, however, dispute this claim by also asserting that when they first arrived in the area, they found no other group settled there (\cite{Owusuh1976}). The speakers of Nafaanra are believed to have migrated from Kakala, near Bontuku in the eastern part of Ivory Coast (\cite{Ameyaw65}). The speakers settled in the villages around Dompofie and subsequently in Dompofie. Nafaanra is a Gur language and belongs to the Senufo language family in Ivory Coast and Mali (\cite{Jordan1980, Ethnologue21}). Nafaanra itself has gained so much prominence over the Dompo language that it is the language used across all spheres in Dompofie. It has become the first language of the progeny of the native Dompo speakers. Dompo is currently only used on very rare occasions such as during their annual festival when homage is paid to the ancestors in the language, during funerals, marriage, and puberty ceremonies, when some songs are sung in the language and when rituals are held to appease the gods.  

According to some Dompo speakers who were witnesses to the causes of the decline in the use of the language over the years, the shift in use of Dompo to Nafaanra can been attributed to several fundamental causes. The chief of Dompofie explains that there was a period in the history of the Dompo people where there were very strict law makers who insisted that the Dompo natives should marry among themselves to keep their language active, so as to transmit it to the next generation. However, when those law makers passed on, the Dompos began marrying individuals from other tribes, particularly the Nafaanras. A woman who experienced the effect of this shared that both her parents were Dompos, but her father also married a \ili{Nafaanra} woman who came to reside in the same household with them. It happened that whenever her parents were speaking Dompo, or her mother and other Dompo speakers were conversing, the Nafaanra woman would accuse them of gossiping about her. Over time, the frequency with which Dompo was used reduced, leading to a breakdown in the transmission of the language in the household. The Nafaanra language was, however, gaining more ground, in turn, because its speakers were increasing in number in the Dompo community. Another opinion was given by three siblings who all confirmed that when they were young, their parents deliberately did not actively transmit Dompo to them because their parents didn’t want them to understand their conversations. In a similar manner, other children who could speak Dompo were discouraged \is{language endangerment} from interacting and teaching Dompo to their Nafaanra peers. The oldest remaining speaker of Dompo attributes the decline of the language to the fact that teens were not respectful of the elderly and were not willing to help them out on the farms. He noted that one easy way the language could be transmitted to the younger generation was if they walked to the farms with the elderly Dompo speakers and if they worked on the farms with them. He pointed out that that was how he taught the youngest speaker of the language (who is now deceased). Other Dompo speakers and non-Dompo-speaking community members also shared that they witnessed the Dompo language used as a secret language to gossip about those who could not speak it.

\section{Greetings in the African milieu and in the Dompo community}

Greetings form part of those essential daily exchanges that characterize our existence as humans. \citet[14]{Egblewogbe1990} suggests that “greetings are regarded as a means of establishing social contact and acknowledging the social presence of others”. Both \citet{Egblewogbe1990} and \citet{Foley1997} establish that greetings are used to acknowledge the social presence of someone, to maintain the bond between interlocutors; and greetings encircle fundamental aspects of culture.  Members of a society are mandated to initiate greeting exchanges and receive responses during an encounter. 

Greetings in most African contexts are influenced by social factors such as age, gender, religion, occupation, and social status \parencites[14]{Egblewogbe1990}[37]{Nwoye1993}. Each society may have different approaches to how greetings are conducted by indicating who should be greeted, the person to initiate the greeting, and what verbal and non-verbal acts should go with the greetings. In most communities, the person who encounters another person or a group of people first initiates the greeting, irrespective of the former’s age. Generally, in the social order, the onus falls on the youngest of the group in the hierarchy to initiate a greeting, with some appropriate interval of questioning to follow. In the Dompo community as well, young people are required to greet the elderly first and to wait for the older interlocutor to make enquiries about the health of the younger interlocutor. In contrast, young people in Persian and Hausa societies can ask about the welfare of the older interlocutor and that of their family after initiating a greeting (\cite[263]{Chamo2015}, \cite[300]{Moradi2017}). Whereas the onus lies on women in the Yoruba and Kisukuma family hierarchy to greet a man first (\cite[2]{Akindele1990}, \cite[93]{Batibo2009}), in the Dompo community, such an onus does not exist. Rather, age takes precedence over gender in a greeting exchange. Observe below the various greeting exchanges used during the different periods of the day in Dompo. Note also that the initiators are either the younger person in the greeting exchange or the person who first sees the other as they approach.

\subsection{Morning greetings}

\textit{Klà} is the Dompo word for \textit{greet} or \textit{greetings}. It is also used as a morning greeting. It is further used in two ways in the language. In its first form, it occurs after the kinship name or the title of the person who is being greeted as is depicted in (\ref{Ex.fathergreet}). As seen here, a kinship title, such as `mother' or `father', is added to greetings in Dompo; it does not matter if the person one is greeting is the biological mother or father. The exchange in (\ref{ex:fathergreet2}) is between the author and the oldest male speaker of the language. It took place one morning when I was doing my fieldwork in Dompofie.\footnote{I lived with the Dompo speakers in Dompofie for about 10 months during my fieldwork. Greetings were one of the first linguistic exchanges I learnt from the speakers. It provided the grounds for the speakers to always want to engage further because they saw the enthusiasm with which I learnt different aspects of their culture.} 

\ea \label{Ex.fathergreet}
\gll Ntrô,	 klà.\\
father	greet\\
\glt  ‘Greetings, father.’
\z

\noindent In its other form, \textit{klà} serves as a logical object that can be given to someone, as depicted in (\ref{ex:fathergreet2}). 

\ea \label{ex:fathergreet2}
\begin{xlist}
\ex 
 \gll Ntrô,	é	hã́	wó	klá.\\
        father	\textsc{1sg}	give	\textsc{2sg}	greet\\
        \glt  ‘Father, I greet you.’ (lit. father, I give you greetings) \jambox*{\textit{Speaker A}}
\ex
\gll Klà	mí	bí.	ó	kàà	fóófó	à?\\
        greet	\textsc{1sg}	child	\textsc{2sg}	wake	well	\textsc{q}\\
        \glt  ‘Greetings my child (response). Did you wake up well?’ \jambox*{\textit{Speaker B}}
\ex 
 \gll ɛ̃̀ɛ̃́,	é	kàà	fóófó.\\
\textsc{intj}	\textsc{1sg}	wake	well\\
\glt  ‘Yes, I woke up well.’ \jambox*{\textit{Speaker A}}
\ex
\gll Á	hã́	Bwàrèngò	lɛ̀ɛ́.\\
	\textsc{1pl}	give	God		thank\\
\glt  ‘We thank God (for life).’ \jambox*{\textit{Speaker B}}
\end{xlist}
\z

As can be inferred from the dialogue above, morning greetings in Dompo express appreciation for the lives of both interlocutors and give thanks to God for the blessing of another day. Greetings are moreover viewed as something that can be given to the subject, perhaps as sustenance, and are expressed as something to be eaten or treasured. When one greets another in the morning, the words serve to nourish the body of the person being greeted. It is a blessing to be alive and to have someone to interact with. This buttresses the important nature of greetings in Dompo and alludes to the fact that humans need these physical interactions to feel loved and appreciated. 

\subsection{Afternoon greetings}\largerpage

Afternoon greetings in Dompo serve to further recognize the presence of the interactants. They might have been away during the morning, and so seeing each other gives them the opportunity to touch base and to further talk about what might have ensued during the period before meeting once again. Below is a brief exchange between two friends who met on the road while going to conduct their individual business.\footnote{\textit{Ama} is the name for a female born on Saturday in the Akan culture of Ghana.} 

\ea \label{ex:afternoongreet}
\begin{xlist}
\ex 
 \gll Ámá, 	ápáá.\\
	Ama	afternoon\\
\glt  ‘Good afternoon, Ama.’ \jambox*{\textit{Speaker A}}
\ex
\gll Pàà,		mí	nákpáá.\\
	afternoon	\textsc{1sg}	friend\\
\glt  ‘Good afternoon, my friend.’\jambox*{\textit{Speaker B}}
\ex 
 \gll Lànɔ̀	lìrè	à?\\
	house	exit	\textsc{q}\\
\glt  ‘(Is everything ok) at the house you come from?’ \jambox*{\textit{Speaker B}}
\ex
\gll  ɔ̀ɔ́.\\
	\textsc{intj}\\
\glt  ‘Yes.’ \jambox*{\textit{Speaker A}}
\end{xlist}
\z

In this exchange, speaker B’s question about the home of speaker A is deeper than its literal meaning. Speakers B wants to find out if everyone and everything in the home of speaker A is okay. If something eventful had happened in the home of speaker A, she would have prolonged the interaction by detailing it out to speaker B. In this case, it can be assumed that there was no such thing and so the interaction was cut short and both interlocutors went their way. 

\subsection{Evening greetings}

Evening greetings in Dompo give interlocutors the opportunity to unwind and discuss the activities of the day, which mostly would be what happened on their respective farms. Observe the exchange below between the older male Dompo speaker, who is speaker B and the younger, who is speaker A.\footnote{Note that `father' has the forms \textit{trò} and \textit{ntrô}, used interchangeably.}

\ea \label{ex:eveninggreet}
\begin{xlist}
\ex 
    \gll Ntrô,		áséé.\\
    father		evening\\
    \glt		‘Good evening, father.’ \jambox*{\textit{Speaker A}}
\ex
    \gll Sèè,		trò. \\
    evening	father\\
    \glt		‘Good evening, father.’ \jambox*{\textit{Speaker B}}
\ex
    \gll Bí	lɛ́ɛ́		ndɔ̀ɔ̀	rɔ́.\\
    \textsc{2pl}	welcome	farm	in\\
    \glt ‘Welcome from work.’ \jambox*{\textit{Speaker B}}
\ex 
    \gll Lɛ̀ɛ̀,	ntrô. \\
		thank,	father\\
    \glt ‘Thank you, father.’ \jambox*{\textit{Speaker A}}
\ex
    \gll Ntí	sʊ̃̀mɪ̀	wá	dù?\\
			how	work	\textsc{def}	plant		\\	
	\glt	‘How did the work go?’ \jambox*{\textit{Speaker B}}
\ex 
   \gll Á	tĩ́		mú.\\
			\textsc{1pl}	be.able	\textsc{3sg}\\
    \glt ‘We managed it (e.g., the scheduled weeding/planting).’ \jambox*{\textit{Speaker A}}
\end{xlist}
\z

The kinship title \textit{trò/ntrô} `father' in (\ref{ex:eveninggreet}) is used to refer to an older male -- the respondent, though older than the initiator, refers to the latter using the same title. This shows mutual respect between the interactants. B’s utterance, \textit{bí lɛ́ɛ́ ndɔ̀ɔ̀ rɔ́} is meant to welcome A and his family from the farm. In Dompofie, women and children often accompany their husbands and fathers to the farm. Thus, B is further thanking God for bringing the family from the farm to their home. B then goes ahead to ask about what ensued on the farm, and A responds that all the work they had planned to do that day was accomplished. The conversation could go on about what ensued in speaker B’s day as well. Other topics may come up for discussion and depending on the time they have to spare, the conversation could go on for hours. 

\subsection{Night farewells}

Night farewells take place between interlocutors who are taking leave of each other to go to their respective homes. They bid good night by praying for God’s protection throughout the night and by expressing hope that He ushers them into a new day. The following exchange is between the author, who is speaker A, and the oldest Dompo speaker, who is speaker B.

\ea \label{ex:nightgreet}
\begin{xlist}
\ex 
    \gll Mín		yá	dèhè.\\
		\textsc{1sg.fut}	go	sleep\\
	\glt	‘I am going to sleep.’
\jambox*{\textit{Speaker A}}
\ex
    \gll Bwàrè	yí	nò	ó	déhè	lɛ̀lɛ̀.\\
		God	let	\textsc{conj}	\textsc{2sg}	sleep	well\\
	\glt	‘May God let you sleep well.’
\jambox*{\textit{Speaker B}}
\ex
   \gll Nɛ́	kádé	ŋkɛ̀,		ànín		srɛ́.\\
		\textsc{cond}	dawn	tomorrow	\textsc{1pl.fut}	meet\\
	\glt	‘When day breaks, we will meet.’
 \jambox*{\textit{Speaker B}}
\ex 
    \gll Yòò,	Bwàrèn	bú	àni.\\
		okay	God		cover	\textsc{1pl}	\\
        \glt ‘Okay, God cover us.’
\jambox*{\textit{Speaker A}}
\ex
   \gll  Bwàrèn	há	àní	kɛ̀.\\
		God		give	\textsc{1pl}	tomorrow\\
	\glt	‘God give us tomorrow.’
 \jambox*{\textit{Speaker B}}
\end{xlist}
\z

Though the oldest Dompo speaker is a traditionalist and believes in a lesser god, on all occasions when we had to bid good night, he referred to the supreme God.  

Additionally, greetings are not only used to perform the role of phatic communion, but some are also used to convey information (\cite[64]{Duranti1997}, \cite[151]{Malinowski1972}, \cite[37]{Nwoye1993}). The communal nature of the remaining Dompo society makes greetings not only a phatic gesture but one that elicits further personal details about the interlocutors. The information greetings relay usually transcends the interactants merely recognizing each other, transitioning to exchanges about their personal lives (\cite{Couplandetal1992}). Greetings between interlocutors in Dompo can be of any length, depending on the relationship between them. If the interlocutors are not very well acquainted, the exchange can be brief. This is, however, very rare because of the close-knit nature of the small community. Most of the exchanges go beyond the normal greetings and responses, and largely border on the well-being of the interlocutors and their respective families. For the women, it may go into a discussion of the welfare of some of their children who might be living outside Dompofie and even to some petty gossip about what is happening in the lives of some members of the community. The length of greetings in Dompo may also be dependent on when last the interlocutors interacted. If it has been a while, then there will be a cause to catch up on all the eventful things that have happened in their individual lives. Observe the greeting exchange in (\ref{ex:nightgreet2}) between two women, the enquiries they make about each other and their respective families, as well as the formulaic responses that follow. 

\ea \label{ex:nightgreet2}
\begin{xlist}
\ex 
    \gll Mí	nákpáá, 	é	hã́	wó	klá. \\
	\textsc{1sg}	friend		\textsc{1sg}	give	\textsc{2sg}	greet\\
\glt ‘My friend, I greet you.’ \jambox*{\textit{Speaker A}}
\ex
    \gll Klà,	mí	nákpáá.\\
	greet	\textsc{1sg}	friend\\
	\glt ‘Greetings, my friend.’ \jambox*{\textit{Speaker B}}
\ex
   \gll Ó	kàà	fóófó	à?  \\
	\textsc{2sg}	wake	well	\textsc{q}\\
\glt 	‘Did you wake up well?’ \jambox*{\textit{Speaker B}}
\ex 
    \gll ɛ̀ɛ́,	é	kàà	fóófó.\\
	Yes	\textsc{1sg}	wake	well		\\
	\glt ‘Yes, I woke up well.’  \jambox*{\textit{Speaker A}}
\ex
 \gll Ó	klú		nɛ̀	mbìà		áná	bí	kàà	fóófó	à?\\
	\textsc{2sg}	husband	\textsc{conj}	\textsc{pl}-child	\textsc{quant}	\textsc{3pl}	wake	well	\textsc{q}\\
\glt	‘Did your husband and children wake up well? \jambox*{\textit{Speaker A}}

\ex 
    \gll ɛ̀ɛ́,	bí	kàà	fóófó. \\
	Yes	\textsc{3pl}	wake	well		\\
\glt	‘Yes, they woke up well.’ \jambox*{\textit{Speaker B}}
 \ex 
    \gll Bà-shìà	wɔ̀	ó	lándɔ̀	bí	kàà	fóófó	à? \\
	\textsc{pl}-person	be.at	\textsc{2sg}	house	\textsc{3pl}	wake	well	\textsc{q}\\
	\glt ‘Did the people in your house wake up well?
	\jambox*{\textit{Speaker B}}
 \ex 
    \gll ɛ̀ɛ́,	bì	wùrɔ̀.	Bí	kàà	fóófó. \\
	Yes	\textsc{3pl}	be.in	\textsc{3pl}	wake	well\\
	\glt ‘Yes, they are in (the house). They woke up well.’
	\hfill \raisebox{1.2\baselineskip}[0pt][0pt]{\textit{Speaker A}}
 \ex 
   \gll Á	hã́	Bwàrèngò	lɛ̀ɛ́.\\
	\textsc{1pl}	give	God		thank\\
\glt 	‘We thank God (for life).’ \jambox*{\textit{Speaker B}}
\end{xlist}
\z

Furthermore, greetings are a component of the politeness strategies that are taught to members of a community from a young age with the aim of training them to become respectful and responsible members. Greetings are thus embedded in the concept of \isi{politeness} (\cite{BrownandLevinson1987, Obeng1996, Obeng1999}). Greetings indicate that the initiator is well grounded in the cultural ethics of the society he/she belongs to. Knowledge of good procedures of greetings classifies one as having good manners, politeness, and respect towards others. Ignorance of customs shows that a person is uncouth, uncaring, and lacks education (\cite[174]{Wójtowicz2021}). 

Another aspect of politeness that greetings relate to is the concept of  ‘face' (\cite{Goffman1967}). \citet[311]{BrownandLevinson1987} assert that face is “the public self-image that every member wants to claim for himself”. This includes the desire of the interactants to be approved of or appreciated. Thus, in a greeting exchange, the interlocutors appeal to each other’s positive face by recognizing their presence. Any contrary act will lead to the loss of one’s positive face, and to embarrassment.

\citet[20]{Spolsky1998} describes greetings as the “basic oil of social relations”. In the Dompo community, a person who passes by others without greeting is considered uncultured and disrespectful. If he/she is young, an older person may point out the anomaly and correct the former there and then. On the other hand, if the person who does not greet is a person whose social status renders them beyond reproach, the one who is not greeted may harbor the malice of not being greeted, and may talk to others about the slight. A person who does not respond to greetings in the Dompo community is considered proud and hostile towards the one who greeted. Similarly, (\cite[301]{Moradi2017}) records that in \ili{Persian} society, people who fail to respond to greetings are not only considered worse than impolite, but arrogant and hostile. In most African communities, one’s ability to greet properly indicates respect and concern for the well being of others (\cite[335]{Schleicher1997}). She indicates that in the \ili{Yoruba} community, greetings can be exchanged several times in a day among members. The frequency of interaction is a visible manifestation of the love and concern people have and feel for one another. In the \ili{Ewe} society of Ghana, a premium is placed on greetings because they establish mutual interest, respect, and goodwill (\cite[384]{Dzameshie2002}). In the \ili{Limba} community of Sierra Leone, greetings are performed as a formal act to show commitment and to acknowledge others (\cite[544]{Finnegan1969}). Thus, when two people who were initially on good terms are spotted not exchanging greetings, it automatically signals that the relationship between them has gone sour. 

In a social setting, greetings are language events that comprise participators, time and place, message, situation, and function (\cite[14]{Egblewogbe1990}). \citet[1]{Akindele1990} posits that greeting is “informed by the rules of conduct and is an inevitable part of everyday conversation”. He further observes that greetings are an essential part of social intercourse and serve as a foundation for establishing social connections. Greetings are used as a conversational opener or closer, and their introduction maintains the fluidity of the interaction. Their format is embedded with turn-taking that signifies commonality, mutual respect, and reciprocity among interlocutors (\cite{Agyekum2008, Duranti1997}). In the Dompo community, greetings are expected at all times and in all social situations that involve the meeting of two or more people. Greetings are required from one who enters a house, meets others on the road, in the market, at the farm, and during occasions such as festivals, marriage, and funeral ceremonies. The illustrations below in Dompo depict several of the various social situations and occasions when greetings are exchanged. 

\subsection{Greetings when entering someone's home}

In the Dompo community, a person is mandated to call the attention of others in the house they are visiting from a short distance away, preferably at the door, before entering -- if there is a response to do so. The conventional attention seeking marker, \textit{àgòò} is used to fulfill this request. The response to this greeting is \textit{àmɛ̀ɛ̀}, which is given by someone in the house and indicates permission to enter the house. This bipartite greeting and response is a borrowing into Dompo. It is used in most of the languages spoken in the Southern part of Ghana. It is used for the same purpose in \ili{Ga} (\cite{BerryKotei69}), \ili{Ewe} (\cite{Ameka2009}), and \ili{Akan} (\cite{Ofori2011}). Alternatively, in Dompo, the greeting exchange in (\ref{ex:enterhome}) can also take place when one is entering another’s house. 

\ea \label{ex:enterhome}
\begin{xlist}
\ex 
    \gll É	bá.\\
			\textsc{1sg}	come \\
	\glt	‘I come (I am here).’
 \jambox*{\textit{Speaker A}}
\ex
   \gll  Ó	mnɛ́?\\
			\textsc{2sg}	who\\
		\glt ‘Who is it?’
 \jambox*{\textit{Speaker B}}
\ex
   \gll Mì	Kòfí	nà \\
			\textsc{1sg}	Kòfí	\textsc{foc}		\\
	\glt		‘It is me, Kofi.’
 \jambox*{\textit{Speaker A}}
\end{xlist}
\z

In this exchange, Speaker B was likely expecting some visitors and asked to verify their identities. 

\subsection{Greetings when welcoming people into homes}

Visiting people is part of our social function as human beings. In the Dompo community, members create opportunities to visit one another. The rate of the visits increases depending on the occasion, such as marriage, naming ceremonies, or funerals in the homes of members. When a visitor is ushered into someone's home, the host offers their guest a seat. An offering of water then follows. Especially for people who have travelled a distance to pay a visit, they are given some time to rest before other exchanges take place. The host then goes on to ask about the purpose of the visit, and further exchanges ensue about the well-being of their individual families and interesting things that might be happening in the guest’s village. This is demonstrated in the exchange below.

\ea \label{ex:welcomehome}
\begin{xlist}
\ex 
    \gll Lɛ̀ɛ̀lɛ̀.\\
			welcome\\
	\glt	‘Welcome.’
 \jambox*{\textit{Host}}
\ex
   \gll  Lɛ̀ɛ̀	trò/níí.\\
			thank	father/mother\\
		\glt ‘Thank you, father/mother.’
 \jambox*{\textit{Guest}}
\ex
   \gll Lànɔ̀	mánɪ̀ɛ́?\footnotemark[4]\\
    house	matter	\\
\glt			‘How are issues at home?’
 \jambox*{\textit{Host}}
 \ex
   \gll ó	  Káwí		mánɪ̀ɛ́?\\
   \textsc{2sg}	village.name		matter	\\
\glt			‘How are issues in your village?’
 \jambox*{\textit{Guest}}
\end{xlist}
\z

\subsection{Greeting people working on a farm}

Farming is a major part of the livelihood of the Dompo people. The farms of the inhabitants are located several kilometers from the Dompo settlement. They thus trek or cycle these long distances to their farms in the early morning and return late in the afternoon. It is considered good neighborliness to acknowledge and encourage others who are working on the farm. The following conversational exchange in (\ref{ex:farmgreeting}) may ensue in such situations.\footnotetext[4]{\textit{Mánɪ̀ɛ́} is a borrowing from the Akan word \textit{àmànɪ̀ɛ́} which has the same meaning.}

\ea \label{ex:farmgreeting}
\begin{xlist}
\ex 
    \gll É	hã́	bán	klá.\\
						\textsc{1sg}	give	\textsc{2pl}	greet\\
	\glt	‘I give you greetings.’
 \jambox*{\textit{Speaker A}}
\ex
   \gll  Klà	trò/níí.\\
			greet	father/mother\\
		\glt ‘Greetings, father/mother.’
  \jambox*{\textit{Speaker B}}
\ex
   \gll Bí	bɔ́	kó.\\
    			\textsc{2pl}	do	\textsc{indef}	\\
\glt			‘You have done some work.’
  \jambox*{\textit{Speaker A}}
 \ex
   \gll Ó	lɛ̀ɛ́	nà.\\
   			\textsc{2sg}	thank	\textsc{foc}	\\
\glt			‘Thank you.’
 \jambox*{\textit{Speaker B}}
\end{xlist}
\z

These types of greetings are usually brief and might not extend to incorporate other topics because the interlocuters are likely on their way to a farm or some other place.

\subsection{Greeting people when they are eating}

When someone receives a visitor just when they are about to eat, the guest greets the host first and follows this with a funny remark about being strategic, and lucky enough to come at the right time when there is food. The host thus performs the cultural norm routine of inviting the guest to join him or her in eating. If the guest wants to eat, their may join the host in eating. The guest cannot blatantly say no to the request. The polite way of declining the offer is to urge the host to continue eating, or by stating that one’s hands have joined in the eating -- a metaphorical way of saying that the guest's hands are helping the host finish the food. In the event that the host has little food available, or is almost done eating, they may still invite the guest to join in the eating. Often, the guest responds by urging the host to finish up, with the excuse of having already eaten. Observe the exchange below.\footnote{I thank a reviewer for pointing out that ‘enter hand’ is an idiom for ‘eat’. There is no mention of the word for `food' \textit{jí sɔ} in the Dompo sentence, probably because ‘enter hand’ may also have the same connotation in the language.}

\ea \label{ex:eatinggreeting}
\begin{xlist}
\ex 
    \gll Ó	wúrá	hálé.\\
						\textsc{2sg}	enter	hand\\
	\glt	‘Your hand has entered food.’
  \jambox*{\textit{Guest}}
\ex
   \gll  Bà	nà	á	wúrá	hálé.\\
						come	\textsc{conj}	\textsc{1pl}	enter	hand\\
		\glt ‘Come let us eat.’
   \jambox*{\textit{Host}}
\ex
   \gll Mí	hálé	wɔ̀	mù	rɔ̀.\\
    						\textsc{1sg}	hand	be.at	\textsc{3sg}	in	\\
\glt			‘My hands are in it.’
   \jambox*{\textit{Guest}}
 \ex
   \gll É	jí	drà.\\
   						\textsc{1sg}	eat	already		\\
\glt			‘I have already eaten.’
   \jambox*{\textit{Guest}}
\end{xlist}
\z

There are other expressions among the Dompos that indicate that members of the community are connected in their interpersonal relationships. These expressions are discussed in Section \ref{Other routine expressions in Dompo}.

\subsection{Other routine expressions in Dompo}
\label{Other routine expressions in Dompo}

\subsubsection{Exchanges during the expression of gratitude}

The Dompos believe that showing gratitude for things they receive from someone leaves more room for other good things to happen for both parties. They sing praises upon the giver and pray for God’s blessings upon their life, so more of such gifts will come from the person in the future. The morning after a gift is received, it is the culture of the Dompos to visit the home of the giver to offer their gratitude for what was given to them. After greetings have been exchanged, the interaction in (\ref{ex:gratitudegreeting}) may take place, where (\ref{ex:gratitudegreeting}a--d) are possible statements made by the recipient, depending on the circumstance.\footnote{The word for God has the variant forms \textit{Bwàrè, Bwàrèn, Bùàrèngò}, and \textit{Gbwàrèngò}. These are used interchangeably in speech.}

\ea \label{ex:gratitudegreeting}
\begin{xlist}
\ex
   \gll  Gbàrèngo 	chá	wò.	Kó	ó	léé	nɛ̀ 	ò 	nyá.\\
						God		help	\textsc{2sg}	\textsc{indef}	\textsc{2sg}	want	\textsc{conj}	\textsc{2sg}	get\\
		\glt ‘May God help you. Whatever you want may you get it.’
\jambox*{\textit{Recipient}}
\ex
   \gll Bwàrèn	hã́	ó	ŋkpà	kán	kísì	ó	lɔ́.\\
    						God		give	\textsc{2sg}	life	\textsc{sc}	hate	\textsc{2sg}	ill	\\
\glt			‘God give you life and hate your illness.’
% \hfill \raisebox{1.2\baselineskip}[0pt][0pt]{\textit{}
 \ex
   \gll Bwàrèn	yílí	ó	kàmnɛ̀	nò	ó	má	lɔ́.\\
   						God		stand	\textsc{2sg}	back	\textsc{conj}	\textsc{2sg}	\textsc{neg}	ill		\\
\glt			‘God stand behind you so that you don’t get ill.’
% \hfill \raisebox{1.2\baselineskip}[0pt][0pt]{\textit{Recipient}}
 \ex 
    \gll Chòsò	ndré		kó	ó	bɔ́	kán	hã̀	mí wá	ó	bɔ́	kó.\\
			pass	yesterday	\textsc{indef}	\textsc{2sg}	do	\textsc{sc}	give	\textsc{2sg}		\textsc{def}	\textsc{2sg}	do	\textsc{indef}		\\
	\glt	‘Thank you for what you did for me yesterday. You did something for me.’ 
% \hfill \raisebox{.1\baselineskip}[0pt][0pt]{\textit{}
  \ex
   \gll Àmín.	Ó	má	hã́	mí	lɛ̀ɛ́,	hã̀	Gbàrèngò	lɛ̀ɛ́.\\
   						Amen  \textsc{2sg}	\textsc{neg}	give	\textsc{1sg}	thank	give	God		thank		\\
\glt			‘Amen. Don’t give me thanks, give thanks to God.’
   \jambox*{\textit{Giver}}
\end{xlist}
\z

\subsubsection{Exchanges during the expression of congratulations}

When good fortune falls upon any member of the Dompo community, the person is celebrated. Such happenings include childbirth, marriage, and the passing of a major exam. Individual successes are celebrated because it is believed each member of the community helped one way or the other to help attain them. For instance, when a woman is pregnant, others help take care of her by giving her food items, fetching water for her, and offering advice on how to carry herself during the pregnancy period. Women who are about to get married are also presumed to have been groomed to be suitable women for marriage by the community. Some exchanges that might take place include the exchanges in (\ref{ex:congratsgreeting}) and (\ref{ex:congratsgreeting2}). The first is between a visitor and a host congratulating the former after the birth of a child. The second is a congratulatory message to someone who has achieved success in an exam.

\ea \label{ex:congratsgreeting}
\begin{xlist}
\ex
   \gll Bí	lé	nfɔ̀ɔ́	nà.\\
\textsc{2pl}	come	far	\textsc{foc}\\
   \glt `You have come far.'   \jambox*{\textit{Visitor}}
   \ex
   \gll Bwàrèn	yí	nɛ̀	bì chɪ̀ná	kán	kùrè	mbì		lɛ́lɛ́	kán	hã́	ání.\\
God	let	\textsc{conj}	\textsc{2sg} live	\textsc{sc}	bear	\textsc{pl}.child	good	\textsc{sc}	give	\textsc{2pl}\\
   \glt `God let you live to bear good children for us.' 
\ex 
\gll Bwàrè	yí	nɛ̀	bán	kánú	bɔ́	kòólè.\\
God	let	\textsc{conj}	\textsc{2pl}	mouth	make	one\\
\glt  `God let you be united as one.'
\ex 
\gll Àmín.		Ó	lɛ̀ɛ́	nà.\\
Amen		\textsc{2sg}	thank	\textsc{foc}\\
\glt ‘Amen. Thank you.’  \jambox*{\textit{Host}}
\end{xlist}
\z

\ea \label{ex:congratsgreeting2}
\begin{xlist}
\ex
   \gll Ó	bɔ́	kó.\\
\textsc{2sg}	do	\textsc{indef}\\
   \glt `You have done well.'    \jambox*{\textit{Visitor}}
   \ex
   \gll Ó	yà	kɛ̃́ɛ̃́	kán	yɔ̀	ó	nyíírɔ́.\\
\textsc{2sg}	go	write	\textsc{sc}	go	\textsc{2sg}	face\\
   \glt `You have gone forward (by passing your exams).' 
   \ex 
   \gll Gbàrèngo	yí	nɛ̀	ò	yɔ́ 	ó	nyíírɔ́	ló. \\
   God		let	\textsc{conj}	\textsc{2sg}	go	\textsc{2sg}	face	like.that \\
\glt `May God let you move forward like that.'
\ex 
\gll Àmín.\\
Amen\\
\glt `Amen.'   \jambox*{\textit{Host}}
\end{xlist} 
\z

\subsubsection{Exchanges during the expression of sympathy}

When misfortunes such as accidents or death occur in any household in the Dompo community, other members rally their support behind the affected family. They pay visits to show solidarity with them and to offer words of comfort and encouragement. In the exchange in (\ref{ex:sympathygreeting}), the host responds that they (the bereaved family) have heard and received all the words of comfort from the visitors and thank them for it. They also acknowledge that these words are the only ones that can sustain them during their difficult period. 

\ea \label{ex:sympathygreeting}
\begin{xlist}
\ex
   \gll Bwàlɛ̀.\\
sorry\\
   \glt `Sorry.'    \jambox*{\textit{Visitor}}
   \ex
   \gll Bí	yìɛ́	mú	kán	hã́	Gbárèngò.\\
\textsc{2pl}	leave	\textsc{3sg}	\textsc{sc}	give	God\\
   \glt `Leave it (the situation) to God.' 
   \ex 
   \gll Á	nú.	Bí	lɛ̀ɛ́	nà. \\
   \textsc{1pl}	hear	\textsc{3pl}	thank	\textsc{foc}\\
\glt `We have heard. We thank you.'   \jambox*{\textit{Host}}
\end{xlist} 
\z


\subsubsection{Exchanges during the expression of farewell}

There are certain exchanges that might ensue when a guest is taking leave of the host. In a similar light, when a person is travelling out of the community, there are some farewell messages that might be said, which largely border on God’s protection for the person travelling. Such an exchange is found in (\ref{ex:farewellgreeting}). 

\ea \label{ex:farewellgreeting}
\begin{xlist}
\ex
   \gll Yɔ̀-m	bà.\\
go-\textsc{sc}	come\\
   \glt `Go and come back.'    \jambox*{\textit{Host}}
   \ex
   \gll Gbàrèngò	hã́	wó	kló	lɛ̀lɛ̀.\\
God		give	\textsc{2sg}	car	good \\
   \glt `May God give you a good car to travel in.' 
   \ex 
   \gll Gbàrèngò	yí	nó	ò	yá	lùá	nɔ́	jɔ̀ɔ̀. \\
  God		let	\textsc{conj}	\textsc{2sg}	go	arrive	there	smooth	\\
\glt `May God let you arrive (at your destination) smoothly.'  
\ex 
\gll Àmín.\\
Amen\\
\glt `Amen.'   \jambox*{\textit{Guest}}
\end{xlist} 
\z


\section{Greeting taboos in Dompo}

Just as greetings are expected in all social encounters, there are a few situations in which their usage is prohibited in Dompo culture, as is the case for many cultures in West Africa. In the Dompo community, when a person \is{taboo} is on the way to use the toilet, they neither greet nor respond to greetings from anyone. The person can, however, respond to greetings after using the toilet. In Dompofie, there are separate pit latrines built by the community for both men and women, which are located at some distance from the settlement. A few other \is{politeness} households may also come together to build one behind a home. Thus, one way of knowing when a person is going to use the toilet is by the direction they are going. One who meets another going in the direction of the toilet and not holding any farming tools passes by without greeting. The rationale behind this greeting prohibition is that the person going to the toilet is unclean and can only be interacted with when they have rid themself of the filth they were carrying (\cite{Agyekum2008, Dzameshie2002, Egblewogbe1990}). 

In the Dompo community, the majority of bath houses are built outside homes and in such a way that the head of the person bathing may be visible to passersby. People are thus tempted to greet when they encounter someone taking their bath. It is, however, a taboo for a man who is not married to greet a bathing woman and further engage in conversation with her. The same applies if the one bathing is a man and the potential greeter is a woman. Otherwise, community members may assume that they are having an affair. It is, however, acceptable for there to be such an encounter and a conversational exchange between same sex people. 

Lastly, in the Dompo community, the elders are neither greeted nor respond to any greetings when they are on their way to perform rituals to appease the gods. 

The elders of the Dompo community have stated that these cultural norms and values regarding the performance and non-performance of greetings are not being adhered to by the younger generation. Some Dompos attribute this attrition to the advent and influence of modernity, which has made their society more liberal and prone to outside negative influence.   

\section{Non-verbal aspects of Dompo greetings}

In speech societies, the modes of carrying out greetings may be verbal, non-verbal, or both. Cross-linguistically, these verbal acts are often complemented by non-verbal gestures such as shaking one's hand, waving one's hand, nodding one's head, bending one's knee, and/or making certain facial expressions (\cite{Agyekum2008, Dzameshie2002, Egblewogbe1990, Nwoye1993}). These non-verbal gestures, which may have some slight variations across cultures of the world, are heavily intertwined across the African continent. The onus lies on \is{non-verbal communication} every member of the community to properly train younger members in their use. In the Dompo community, and similar to what happens in the Ewe community, a young girl or woman bends one or both knees in deference when greeting an older man/woman (\cite{Egblewogbe1990, Manu-Barfo2020}). This non-verbal gesture is comparable to the Hausa child who squats when greeting the parents, the Igbo child who bows down, and the Yoruba boy who prostrates while the girl kneels when greeting their parents or elders (\cite{Schleicher1997}). 

Furthermore, a young man in the Dompo community is mandated to remove his hat and/or spectacles when greeting an elderly person. In the past, anyone who came around to interact with the elders in the community had to take off their shoes before offering any form of greetings. This is also the norm among the Baatonu tribe of Northern Benin, where subjects remove their shoes upon entering the chief’s palace; there, men further prostrate, and women get down on their knees with their arms on the ground (\cite{Schottman1995}). If a person is far off and is not visible enough to exchange verbal greetings, one can simply raise the right hand and wave at the person to acknowledge their presence. A smile, indicating the pleasure of seeing the person also accompanies the hand wave. 

Finally, depending on when last two people saw each other, they are likely to exchange hugs alongside greetings in the Dompo community. Hugs show the warmth and love that they feel towards each other, and confirm how well their bodies have been since their last encounter.  


\section{Summary and conclusion}

The prevalence of greetings in the Dompo community can be attributed to the fact that it is close-knit and increasingly small unit, with its few members sharing strong relationships and common interests. Greetings transcend the role of phatic communion, establishing and maintaining strong ties between community members. Greetings thus serve to empower members to carry out their social mandate of being each other’s keeper by checking up on and being concerned about the welfare of others. The lack of a greeting exchange between two interlocutors automatically implies a sour relationship. This act deviates from the oneness the community portrays, and, as such, the people involved may be urged to settle their differences to enable peace and love to prevail in the community. 

In the Dompo community, just as has been recorded in other African cultures, age supersedes gender when it comes to the initiation of greetings. Thus, a younger person always initiates a greeting exchange first. Greetings are conducted at different times and under different conditions. This reflects the fact that languages may have different ways of expressing events. However, the underlying situations that condition these greetings are the same in most African countries. For instance, all languages have greetings that pertain to different times of the day and ones used at different occasions. 

Inasmuch as greetings are a requirement in most social situations, there are a few occasions where they are not \is{politeness} needed. In the Dompo culture, one going to the toilet is not to be greeted. A person taking his/her bath who is of the opposite sex should not be greeted, and elders going to perform rituals are also not to be greeted. Greetings in the Dompo culture are buttressed by \is{non-verbal communication} non-verbal gestures. Young children and women are expected to bend their knees while greeting older people. Young men are also supposed to remove their caps and spectacles while greeting in the Dompo community.

Greetings have been explored in many languages across the world. By this paper, I have added to the discussion on the topic by providing data from an underdescribed and moribund language. Similarities and differences with other languages in these greeting exchanges have also been noted and described. In documenting a language with so few remaining speakers, routine expressions such as these show warmth and solidarity, offering a peek into the customs of a once-larger speaker community.\il{Dompo|)}



\section*{Abbreviations}
\begin{tabularx}{.5\textwidth}[t]{@{}lQ}
1 & first person \\
2  & second person \\
3 & third person \\
 \textsc{cond} & conditional \\
 \textsc{conj} & conjunction \\
 \textsc{det} & determiner \\
 \textsc{foc} & focus \\
 \textsc{fut} & future \\
\textsc{indef} & indefinite \\
\end{tabularx}%
\begin{tabularx}{.5\textwidth}[t]{lQ@{}}
\textsc{intj} & interjection \\
\textsc{neg} & negative \\
\textsc{part} & particle \\
\textsc{pl} & plural \\
\textsc{q} & question marker \\
\textsc{quant} & quantifier \\
\textsc{sc} & serial connector \\
\textsc{sg} & singular \\
\end{tabularx}


\section*{Acknowledgements}
I would like to appreciate the efforts of every member of the Dompo community in making the documentation of their language possible. Special thanks go to my main consultants, Mr. Kosi Mila, Madam Abena Kuma, and Nana Shiembor Agba II (Mr. Emmanuel Dwirah). I offer this and other works on Dompo to the memory of two of my main consultants, Mr. Daniel Kofi Nakpa and Madam Afia Nimena, and Madam Mariama Sima (who couldn't partake in the project, but is a speaker of Dompo). We lost them in 2021 and 2022, respectively. I am also indebted to La Trobe University for funding my research and to my thesis supervisors, Prof. Stephen Morey, Dr. Lauren Gawne, and Prof. David Bradley.

%\section*{Contributions}
%John Doe contributed to conceptualization, methodology, and validation.
%Jane Doe contributed to the writing of the original draft, review, and editing.

{\sloppy\printbibliography[heading=subbibliography,notkeyword=this]}
\end{document}
