\documentclass[output=paper,colorlinks,citecolor=brown]{langscibook}
\ChapterDOI{10.5281/zenodo.11091841}

\author{Beatrice Ng'uono Okelo\affiliation{Indiana University}}
\title{Morphophonology of Dholuo noun pluralization} 
\abstract{There are several types of plural formation observed for Dholuo nouns. The type of plural formation selected by a particular singular noun is unpredictable and therefore encoded in the lexicon. As such, speakers must memorize which plural form is applicable to which noun (\cite{Tucker1994}). Even though there are many descriptive studies on Dholuo, there is little in-depth published work focusing on the morphophonology of plural formation in the language. Even the most sophisticated of analyses encounter problems when a full range of data is taken into consideration. This paper aims to fill this lacuna by (i) identifying and classifying a full range of Dholuo plural noun formation types and (ii) providing a detailed synthesis and analysis of the various phonological and morphophonological processes that take place in their formation.}

\IfFileExists{../localcommands.tex}{
   \addbibresource{../localbibliography.bib}
   % add all extra packages you need to load to this file

\usepackage{tabularx,multicol}
\usepackage{url}
\urlstyle{same}

\usepackage{listings}
\lstset{basicstyle=\ttfamily,tabsize=2,breaklines=true}

\usepackage{langsci-basic}
\usepackage{langsci-optional}
\usepackage{langsci-lgr}
\usepackage{langsci-osl}
% \usepackage{./langsci/styles/langsci-lgr}
% \usepackage{./langsci/styles/langsci-osl}
% \usepackage{langsci-gb4e}

\usepackage{tikz}
\usetikzlibrary{patterns,calc}
\pgfdeclarepatternformonly{south east lines}{\pgfqpoint{-0pt}{-0pt}}{\pgfqpoint{3pt}{3pt}}{\pgfqpoint{3pt}{3pt}}{
    \pgfsetlinewidth{0.6pt}
    \pgfpathmoveto{\pgfqpoint{0pt}{3pt}}
    \pgfpathlineto{\pgfqpoint{3pt}{0pt}}
    \pgfpathmoveto{\pgfqpoint{.2pt}{-.2pt}}
    \pgfpathlineto{\pgfqpoint{-.2pt}{.2pt}}
    \pgfpathmoveto{\pgfqpoint{3.2pt}{2.8pt}}
    \pgfpathlineto{\pgfqpoint{2.8pt}{3.2pt}}
    \pgfusepath{stroke}}
    
\usepackage{stmaryrd}
\usepackage{wasysym}
\usepackage{multirow}
\usepackage{caption}
\usepackage{subcaption}
\usepackage{mathrsfs}
\usepackage{qtree}

\usepackage{linguex}


   %pminos do not split footnotes
% \interfootnotelinepenalty=10000 %Footnote in Laporte chapters has to be split SN


%\DeclareIndexNameFormat{default}{%
%\nameparts{#1}%
%\usebibmacro{index:name}%
%{\index[names]}%
%{\namepartfamily}%
%{\namepartgiveni}%
% {}% L1
% {}% L2
%{\namepartprefix}% generates spurious space L3
%{\namepartsuffix}% generates spurious space L4
%}

%  {\DeclareIndexNameFormat{default}{%
%     \usebibmacro{index:name}{\index[names]}{#1}{#3}{#5}{#7}}}

%\DeclareIndexNameFormat{default}{%
%  \usebibmacro{index:name}{\sindex[nom]}{#1}{#3}{#5}{#7}}

%\DeclareIndexNameFormat{default}{%
%  \usebibmacro{index:name}{\sindex[person]}{#1}{#3}{#5}{#7}}
%\DeclareIndexNameFormat{default}{%
%\nameparts{#1} \usebibmacro{index:name}{\sindex[person]]}{\namepartfamily}{‌​\namepartgiven}{\nam‌​epartprefix}{\namepa‌​rtsuffix}}

%\newcommand{\smiley}{:)}

%\renewbibmacro*{index:name}[5]{%
%\usebibmacro{index:entry}{#1}%
%{\iffieldundef{usera}{}{\thefield{usera}\actualoperator}\mkbibindexname{#2}{#3}{#4}{#5}}}

% \newcommand{\noop}[1]{}

%remove for final
%\overfullrule=1mm

\newcommand{\tobi}[2]}}
\renewcommand{\S}[1]{\tobi{#1}{\textsc{*}}}

% this volume references
% puts: [this volume]
% already defined: \citetv
%\newcommand{\citepv}[1]{(\citeauthor{#1} \citeyear*{#1} [this volume])}
\newcommand{\citealtv}[1]{\citeauthor{#1} \citeyear*{#1} [this volume]}

%parentheses around example number
\newcommand{\pref}[1]{(\ref{#1})}

% in-text examples

\newcommand{\lnex}[1]{\textit{#1}} %target lang word
\newcommand{\lnlit}[1]{(lit.: `#1')} %literal reading
\newcommand{\lnlat}[1]{(#1)} % latinization
\newcommand{\lntrans}[1]{`#1'} %translation
\newcommand{\lnexl}[2]%
{\lnex{#1}{} \lnlat{#2}} % ex with latinization
\newcommand{\lnexlat}[3]{\lnex{#1}{} \lnlat{#2}{} \lntrans{#3}} % ex with latinization and tranl.

%ch01
\newcommand{\co}[1]{\mbox{\textbf{#1}}}

%ch09

\newcommand{\cyrbulg}[1]{\begin{otherlanguage*}{bulgarian}#1\end{otherlanguage*}}


%ch10
\newcommand{\nlp}{{\small NLP}}
\newcommand{\mwe}{{\small MWE}}
\newcommand{\rae}{{\small RAE}}
\newcommand{\lvc}{{\small LVC}}
\newcommand{\pos}{{\small P}o{\small S}}
%\newcommand{\todo}[1]{ \textcolor{red}{#1} }

%\renewcommand{\labelenumi}{\theenumi}
%\ainamefmt{{vv}{ll}{, ff}{, jj}} % fullname

\newcommand{\biberror}[1]{{\color{red}#1}}

\newcommand{\osenovaitem}{--~}
   %% hyphenation points for line breaks
%% Normally, automatic hyphenation in LaTeX is very good
%% If a word is mis-hyphenated, add it to this file
%%
%% add information to TeX file before \begin{document} with:
%% %% hyphenation points for line breaks
%% Normally, automatic hyphenation in LaTeX is very good
%% If a word is mis-hyphenated, add it to this file
%%
%% add information to TeX file before \begin{document} with:
%% %% hyphenation points for line breaks
%% Normally, automatic hyphenation in LaTeX is very good
%% If a word is mis-hyphenated, add it to this file
%%
%% add information to TeX file before \begin{document} with:
%% \include{localhyphenation}
\hyphenation{
    Beck-man
    Ngu-yen
    back-chan-nel
    back-chan-nels
    mo-not-o-nous
    ste-reo-typ-i-cal
}

\hyphenation{
    Beck-man
    Ngu-yen
    back-chan-nel
    back-chan-nels
    mo-not-o-nous
    ste-reo-typ-i-cal
}

\hyphenation{
    Beck-man
    Ngu-yen
    back-chan-nel
    back-chan-nels
    mo-not-o-nous
    ste-reo-typ-i-cal
}

   \boolfalse{bookcompile}
   \togglepaper[2]%%chapternumber
}{}

\begin{document}
\SetupAffiliations{mark style=none}
\maketitle

\section{Introduction}

This paper\il{Dholuo|(} seeks to examine the morphophonology of plural noun formation in Dholuo. The term \textit{Dholuo} refers to one of the many languages of the Luo group within the Western Nilotic branch of the Nilo-Saharan language family. The Luo ethnic group can be viewed as comprising a sub-family of diverse ethnolinguistically affiliated languages. Luos inhabit an area from Southern Sudan, through Northern Uganda and Eastern Congo (DRC), into Western Kenya towards the upper tip of Tanzania. In addition to Dholuo, other \ili{Luo languages} include Lang’o, Dhopadhola, Acholi (spoken in Uganda), Alur (spoken in Uganda and DRC), and Shilluk, Burun, Maban, Luwo, Thuri, and Anuak (spoken in South Sudan). According to anthropologists and ethnolinguists, notably \citet{Ndeda2019}, the Luo of Kenya are also referred to as River-Lake Nilotes because they come from Nyanza Province in the Western region of Kenya, a region that is close to Lake Victoria and is also surrounded by many rivers. The two other divisions of Nilotes in East Africa are the Plain Nilotes and the Highland Nilotes.

This paper focuses on Dholuo as spoken by the Luo of Kenya, called \textit{Joluo}. There are two main Dholuo dialects: Trans-Yala Dholuo and South Nyanza Dholuo, though several other sub-dialects are mentioned in \citet{Tucker1994}. The dialect represented in this work is South Nyanza Dholuo, which is the native dialect of the author and the dialect that is spoken in various parts of the South Nyanza region of Kenya and most parts of Central Nyanza. It is the dialect that is most commonly spoken among the Luo people of Kenya and used in educational materials, most Dholuo media houses, and in the Dholuo Bible.

Plural formation in Dholuo is complex, as there is not one specific rule that can be used to derive plural nouns from corresponding singular noun forms in the language. Most singular nouns in Dholuo have plural forms, though some nouns (e.g., verbal nouns and abstract nouns) do not. Different approaches have been taken to account for the formation of plural nouns in Dholuo. For instance, \isi{voicing polarity}, as proposed by \citet[16]{Alderete1999}, states that “[voice] specification for the stem-final obstruent in the singular is reversed in the corresponding plural.” This can be seen as problematic, primarily because exceptions to this generalization abound. For example, while such a ``polarity'' analysis might seem to apply in pairs like \textit{bat} `arm' and \textit{bede} `arms' (voiceless \textsc{SG}, voiced \textsc{PL}) and \textit{ogudu} `hat' and \textit{ogute} `hats' (voiced \textsc{SG}, voiceless \textsc{PL}) that Alderete provides, it cannot account for pairs like \textit{chik} `law' and \textit{chike} `laws', where no voicing alternation occurs.

Onyango's \citeyearpar{Onyango2016} discussion of Dholuo plural formation focuses on place of articulation. The analysis provided describes a process of articulatory harmony that must hold between corresponding segments in singulars vs. plurals. The data analyzed in Onyango's study propose that the stem-final consonant of a singular noun and that of its plural counterpart must share the same place of articulation. However, there are ample instances in which this does not hold, as seen for [l] vs. [k] in \textit{dɪɛl} `goat' and \textit{diek} `goats'.

\begin{sloppypar}
While it is true that voicing polarity and \isi{articulatory harmony} do apply in at least some instances of Dholuo plural formation, there are several other morphophonological processes that extend beyond these two approaches. These include instances of suffixation, prefixation and substitution, subtraction, and suppletion. To best understand the characteristics of these various pluralization strategies, I propose that it is best to divide Dholuo nouns into several morphological classes (one might call them noun classes) based on their plural formation. I shall define each class based on the primary morphophonological process(es) that the nouns within the class undergo in plural formation. There are also instances of ``irregular'' plural formation that defy clear categorization. Other processes such as vowel harmony, vowel deletion, consonant alternations, tone assimilation, and other tone alternations further complicate the phenomena.
\end{sloppypar}

The remainder of this paper is laid out as follows. In Section \ref{04_Section2}, I provide a brief description of some segmental and tonal matters that are relevant to this research. Section \ref{04_Section3} briefly states the methodology followed and the organization of data. Section \ref{04_Section4} provides a detailed look at the different morphological classes of plural formation, as well as a possible analysis to address the many interrelated instances of pluralization by the suffix \textit{-e}. Subsections cover pluralization via three different suffixes, as well as by suppletion and subtraction. Concluding remarks and a summary of findings are provided in Section \ref{04_Section5}.

\section{Overview of Dholuo phonology} \label{04_Section2}

In this section, I provide a brief overview of some phonological details that are relevant to this study. As introduced above, because plural formation variously involves alternations in consonants, vowels, and tones, it is important to have a baseline understanding of the language's sound inventory. This section contains information about Dholuo's consonant inventory and vowel inventory, as well as the role played by the feature [ATR] in the language. It concludes with a brief introduction to the language's tonal system. 

\subsection{Consonants}
There are 26 consonant phonemes in Dholuo, as shown in \tabref{tab:CPhonemes}.\footnote{Bilab - Bilabial, LabDent - Labiodental, IntDent - Interdental, Alv - Alveolar, PalAlv - Palatoalveolar, Pal - Palatal, Vel - Velar, Glot - Glottal} In the language's orthography, each consonant phoneme is typically represented using a single alphabetic symbol akin to its IPA counterpart. The exception to this are eleven consonants that are represented as follows:\\

\begin{tabularx}{.50\textwidth}{ll}
interdental fricatives - /θ, ð/ & \textit{th, dh}\\
palato-alveolar affricates - /tʃ, dʒ/ & \textit{ch, j} \\
palatal nasal - /ɲ/ & \textit{ny} \\
velar nasal - /ŋ/ & \textit{ng'} \\
prenasalized consonants - /ᵐb, ⁿð, ⁿd, ᶮdʒ, ᵑg/ & \textit{mb, ndh, nd, nj, ng}\\ \\
\end{tabularx}
 \begin{table}
\caption{Dholuo consonant phonemes}
\label{tab:CPhonemes}
 \begin{tabular}{lcccccccc}
  \lsptoprule
& Bilab & LabDent & IntDent & Alv & PalAlv & Pal & Vel & Glot \\
\midrule
 Stop & p b &  &  & t d &  &  &  k g & \\
 Fricative & & f & θ ð & s &  &    &  & h \\
 Affricate & &&& & tʃ dʒ &&& \\
 Nasal & m &&& n && ɲ & ŋ & \\
 Prenasal & ᵐb && ⁿð & ⁿd & ᶮdʒ && ᵑg & \\
 Lateral &&&&l &&&& \\
 Trill &&&&r &&&& \\
 Glide & w&&& &&j&& \\
  \lspbottomrule
 \end{tabular}
\end{table}   

\begin{sloppypar}
\citet{Gregersen1961} does not list \isi{prenasalized consonants} as phonemes in Dholuo‘s inventory. Rather, he refers to them as “clusters”.
\citet{Onyango2016} recognizes these segments, but he lists them only in the phonetic inventory. \citet{Oduor2002} and \citet{Njuki2016} claim they are prenasalized stops that function as unit phonemes. \citet{Ombijah2020} says that they are phonemic units as well, contending that they can appear as onsets or codas.
\end{sloppypar}

\citet[17]{Okoth-Okombo1997} discusses these segments at length and calls them nasal-stop “compounds”. It is said that they function as unit phonemes: “Foreigners usually find Dholuo nasal-stop compounds hard to pronounce in a native-like manner. Take, for instance, the compound \textit{mb}, as in \textit{mbaka} ‘conversation’. To produce a native-like quality of \textit{mb}, the lips should be brought together without muscular tension and kept together until the release-stage for \textit{b}. The nasal quality must characterize the whole of the compound segment, although the final release is mainly through the mouth. In principle the same procedure can be used to produce all the nasal-stop compounds of Dholuo, the only variation being in the point of articulation”. 


\subsection{Vowels}
Most native Dholuo speakers would consider the language to have five vowels -- \textit{a, e, i, o, u} -- but this is because the feature \is{ATR} [ATR], which is both contrastive and pertinent to \isi{vowel harmony}, is not overtly represented in the orthography. \citet[18]{Okoth-Okombo1997} states that “Dholuo orthography underdifferentiates vowel phonemes”, noting that “the whole system uses only five symbols, one for /a/ and one for each of the four [phonemic] pairs” discussed later in this section. Given that [ATR] is contrastive in Dholuo, specifically for high and mid vowels, there are in fact nine phonemic vowels, as shown in \tabref{tab:VPhonemes}, which can be divided based on their specification for the feature [ATR].

\begin{table}
\caption{Dholuo vowel phonemes}
\label{tab:VPhonemes}
 \begin{tabular}{lcccc}
  \lsptoprule
& \multicolumn{2}{c}{[+ATR]} & \multicolumn{2}{c}{[-ATR]}  \\
& Front & Back & Front & Back \\
\midrule
High & i & u & ɪ & ʊ \\
Mid & e & o & ɛ & ɔ \\
Low &&  & & a\\
  \lspbottomrule
 \end{tabular}
\end{table}   

For each of the four [ATR] \is{ATR} vowel pairs, one easily finds minimal pairs illustrating that the corresponding vowels are contrastive and thus separate phonemes. \tabref{tab:ATRComparison} provides examples of each. I have provided both the orthographic representations and their corresponding IPA transcriptions. To clearly demonstrate that the vowel pairs are contrastive based on their ATR qualities and not as a result of tonal differences, the selected pairs share the same tone as well.

\begin{table}
\caption{Vowel phonemes minimal pairs on [ATR]}
\label{tab:ATRComparison}
 \begin{tabular}{lllll}
  \lsptoprule
& Phonemes & Orthography & IPA & Gloss \\
\midrule
a) & /i/ vs. /ɪ/ & \textit{pith} & [pìθ] & `slope' \\
&&& [pɪ̀θ] & `rearing of animals/poultry' \\
b) & /e/ vs. /ɛ/ & \textit{ler} & [lér] & `vein' \\
&&& [lɛ́r] & `cleanliness' \\
c) & /o/ vs. /ɔ/ & \textit{romo} & [ròmò] & `meet, have a meeting'\\
&&& [rɔ̀mɔ̀] & `be enough, fill up (with food)' \\
d) & /u/ vs. /ʊ/ & \textit{bur} & [bùr] & `hole (in ground)' \\
&&& [bʊ̀r] & `boil (of skin)' \\
  \lspbottomrule
 \end{tabular}
\end{table}   

\subsubsection{Status of [a]}

As shown in \tabref{tab:VPhonemes}, unlike other Dholuo vowels, /a/ does not have a contrastive [+ATR] counterpart. However, an [+ATR] counterpart to this vowel, namely [ɐ], does arise, though not consistently, in [+ATR] vocalic environments for some speakers in some instances. As such, one would need to posit that it is simply an allophonic variant of /a/ for these speakers. \citet[61]{Barasa2018} makes a similar claim to explain the vocalic facts in \ili{Ateso}, an Eastern Nilotic language. Just as in Dholuo, his study found that the presence of this variant “is conditioned by neighbouring [+ATR] vowels or glides, and hence does not have phonemic status; instead, it is treated as an allophone of /a/”.

According to \citet{BorowskyAvery2009}, an ATR distinction in Dholuo for the vowel /a/ appears not to be salient for most speakers. In my experience as a linguist and as a native speaker of the language, I concur with this finding both from the standpoint of production and perception. I assume, however, that given the allophonic alternations that this vowel participates in for some speakers, /a/ in Dholuo is best treated as underlyingly [-ATR]. It typically displays neutral behaviour in that it can occur within a word with any other vowel, regardless of the vowel’s ATR quality. \citet[37--38]{Casali2003} shares similar sentiments in this regard, arguing that the vowel /a/ in Dholuo “does not function as a genuinely [+ATR] vowel in the language; it is still clearly a [-ATR] vowel in a number of respects.” To support his argument, he further states that vowel /a/ ``has the voice quality characteristics of other [-ATR] vowels'' and, from an empirical perspective, he points out that ``roots containing only /a/ take [-ATR] rather than [+ATR] forms of harmonizing affixes''.

Likewise, in reporting on the behavior of the vowel /a/, \citet[80]{Ojal2015} affirms that “this vowel is a neutral one in Dholuo, and like other neutral vowels, it is underlyingly [-ATR]”. In coming to this conclusion, Ojal tested this vowel's behavior with infinitives. The infinitive suffix in Dholuo is [-o] or [-ɔ], depending on the ATR nature of the stem vowels, with the former patterning with [+ATR] stems and the latter with [-ATR] stems. In his analysis, \citet[79]{Ojal2015} compared the behavior of the vowel /a/ with [+ATR] stem vowels. Based on his findings, he claims that “the vowel /a/ is the only vowel in Dholuo that lacks the [+ATR] suffix [-o]”. and “this confirms Casali’s assertion that roots containing only /a/ take [-ATR] rather than [+ATR] forms of harmonizing affixes”.

\subsubsection{/a/ to [e] alternations}

Another matter pertaining to the vowel /a/ concerns alternations between /a/ and [e]. In sections below, it will be shown that many instances of this alternation are found in the presence of different plural suffixes. For example, if the stem vowel of a singular noun is underlyingly /a/, it alternates to [e] when a plural suffix containing a front vowel -- \textit{-e}, \textit{-ni}, or \textit{-i} -- is added to the stem. It is therefore important to consider why this is so in order to fully account for the language's plural formation. 

In his study on the status of the vowel /a/ in Dholuo, \citet{Ojal2015} uses a list of nouns (singular and plural) and verbs (imperatives and gerunds) to examine the occurrence of the vowel /a/ with the plural suffixes \textit{-e} and \textit{-ni}, alongside the verbal-noun suffix \textit{-o}. These three suffixes are inherently [+ATR]. The vowel /a/ was the stem vowel for all the nouns and verbs in the data. His findings, specifically for nouns and verbs from the \is{variation} South Nyanza Dholuo dialect, reveal that “/a/ becomes [e] when [+ATR] suffixes /-ni/, /-e/ and /-o/ are attached to a root containing it.” He concludes that “this is a clear case of vowel raising or shift where /a/, a low vowel has been raised to [e], a mid-high vowel” (p. 78). He adds that “this raising or shift is occasioned by the fact that there is need for harmony since the suffixes are inherently [+ATR] while the root vowel [a] lacks a [+ATR] harmony counterpart” (p. 78). As such, one can view this as the vowel [e] filling a gap as the arguably closest [+ATR] vowel to /a/. I concur with Ojal’s argument regarding /a/ to [e] ``harmony''. This explanation will serve as a point of reference whenever there is an instance of /a/ to [e] alternation below. 

\subsection{Tone}

Dholuo is a tone language wherein tone is both lexical (i.e., can be used to distinguish meaning in words) and grammatical (i.e., can convey grammatical distinctions, such as in tense/aspect or part of speech). Remarkably little detail is known about its tone system, however, and even basic facts about its characteristics are often conflicting (cf. \citealt{Gregersen1961, Okoth-Okombo1982, Okoth-Okombo1997, Ombijah2020, Tucker1994}). With this in mind, I assume that vowels are the tone bearing units, and there are four primary tones on these vowels: High tone (v́), Low tone (v̀), Falling tone (v̂), and Rising tone (v̌). Besides these, \citet{Tucker1994} establishes a variety of allotones in Dholuo, for instance, downstepped high tone, undulating tone, extra high tone, extra low tone, extra high falling, and low descending tone. In this paper, the only allotone that I will assume is downstepped High tone (v́ꜜv́), which is non-automatic, as it is usually caused by \isi{Obligatory Contour Principle} (OCP; \citealt{Leben1973}) effects on High tones associated with adjacent vowels. Just like in the case of ATR feature, Dholuo orthography does not overtly represent tone.

\section{Methodology} \label{04_Section3}

This study employs a qualitative research strategy, and the data used are drawn from a sample of native Dholuo nouns, including several that are borrowings from \ili{Swahili} and \ili{English}. As mentioned above, the data are specifically representative of the South Nyanza dialect of Dholuo which is spoken in various parts of the South Nyanza region of Kenya and most parts of Central Nyanza. I rely mainly on my intuitions as a native Dholuo speaker in presenting and analyzing the data.

The data are organized and analyzed in various morphological classes which are defined upon the primary morphophonological process(es) that the nouns within the class undergo during plural formation. 

The morphophonological analysis of native Dholuo nouns offered below is primarily descriptive. That said, it expands upon previous works that have sought to analyze the sometimes seemingly unusual alternations witnessed between the language's singular and plural nouns. Besides attempting to offer a more comprehensive overview of the various pluralization patterns found in the language, a key contribution of this chapter is my proposal of a novel analysis of \textit{-e} pluralization that involves the suffix /-tE/.

\section{Data and analysis} \label{04_Section4}\largerpage

In this section, I provide a description and analysis of the various phonological and morphophonological processes that take place in the formation of plural nouns in Dholuo. I begin with discussing plural formation via suffixation, which is by far the most common strategy. There are three different suffixes employed, though suffixation via \textit{-e} is the most widely attested among the three. I later turn to other strategies for pluralization, namely subtraction and stem suppletion. Data are presented using IPA notation. Vowels within a word are considered to have the same \isi{ATR} feature specification, as is characteristic in Dholuo morphophonology (\cite{Okelo2020}). Tones are marked according to the conventions defined above. 

\subsection{Pluralization by \textit{-i} and \textit{-ni}}

As stated above, there are three suffixes involved in Dholuo plural formation. Two of these, \textit{-i} and \textit{-ni}, exhibit fairly straightforward behavior such that they do not involve the same type and number of consonant \is{consonant mutation} mutations seen below for \textit{-e} suffixation. Pluralization via \textit{-i} and \textit{-ni} are discussed first in this section before turning to the more complex behavior of \textit{-e} pluralization.

Examples of \textit{-i} pluralization are in \tabref{tab:ISuffixation}. These nouns consistently realize a voiced stem-final consonant in the plural, which is voiceless in the singular when no suffix is present. I assume, therefore, that the stem-final consonant in these stems is underlyingly voiced that the outcomes in the singular as involving a straightforward instance of \is{final devoicing} word-final devoicing, and, more specifically, they involve voicing \isi{neutralization}.

\begin{table}
\caption{\textit{-i} suffixation}
\label{tab:ISuffixation}
 \begin{tabular}{llll}
  \lsptoprule
& Singular & Plural &  \\
\midrule
a.	&rʊ̀àθ&	rùèðì&	`bull' \\
b.&	rùòθ&	rúóðî&	`king/chief' \\
c.	&gùòk&	gúógî&	`dog' \\
d.	&dʒʊ̀ɔ̀k	&dʒùògì	&`spirit' \\
e.&	òⁿdíèk&	òⁿdíégî&	`hyena' \\
f.&	mùòk	&múógî&	`antbear/aardvark' \\
g.&	dàk	&dègì&	`pot' \\
h.&	ɔ̀t&	ùdì &	`house' \\
  \lspbottomrule
 \end{tabular}
\end{table}  

Note that if the stem vowel of a singular noun is /a/, it alternates to [e] in the plural. Along similar lines, [-ATR] stem vowels in the \is{ATR} singular become [+ATR] when appearing before plural \textit{-i}. Somewhat surprisingly, however, there are occasions in which a [-ATR] mid vowel will alternate in both [ATR] and height, as in \tabref{tab:ISuffixation}h.

Here and elsewhere, it will become clear that alternations in tone between singular and plural forms involving \textit{-i} and \textit{-e} are typically not predictable. For example, in \tabref{tab:ISuffixation}, sometimes the tone on noun stems changes from Low to High (as in \tabref{tab:ISuffixation}b,c), whereas in others, like (\tabref{tab:ISuffixation}g,h), there is no tone alternation. As will be shown below for \textit{-e} pluralization, still other stem tone alternations are attested, such as High to Low, or Rising to High. Pluralization involving \textit{-ni}, however, always results in an alternation of the \is{grammatical tone} stem tone to Low tone.

Pluralization by \textit{-ni}, as seen in \tabref{tab:NiSuffixation}, is even simpler than by \textit{-i}, as it involves no alternations affecting stem consonants. That said, a stem-final vowel, as seen in singular forms, is lost upon addition of the plural suffix. Others who have described Dholuo pluralization have noted this vowel loss (e.g., \citealt{Gregersen1961, Okoth-Okombo1982}), but have simply stipulated that it occurs. I will tentatively assume that this loss is grounded in \isi{metrification} whereby the second short vowel in a sequence of three is lost. It will be shown that this assumption stems from alternations involving \textit{-e} pluralization discussed in the next subsection.

\begin{table}
\caption{\textit{-ni} suffixation}
\label{tab:NiSuffixation}
 \begin{tabular}{llll}
  \lsptoprule
& Singular & Plural &  \\
\midrule
a.	&kóᵐbé&	kòᵐbnì&	`hole in tree'\\
b.&	tʃùlá&	tʃùlnì&	`island'\\
c.&	hónó&	hònnì&	`miracle'\\
d.&	ⁿdɪ́gà &	ⁿdìgnì&	`bicycle'\\
e.&	àgúlú&	àgùlnì&	`clay pot'\\
f.&	kʊ́bέ&	kùbnì	&`jerrican'\\
g.&	lǎw&	lèwnì&	`clothing/garment'\\
h.&	ɔgáⁿdá	&ògèⁿdnì	&`kingdom'\\
i.&	àgwátá&	àgùètnì&	`calabash'\\
j.&	lúáⁿdá&	lùèⁿdnì&	`rock'\\
k.&	ⁿdàrà	&ⁿdèrnì&	`road'\\
  \lspbottomrule
 \end{tabular}
\end{table}   

Other predictable vocalic and harmonic alternations are triggered by the \textit{-ni} suffix, and as stated above, the presence of \textit{-ni} entails Low tone on the stem.

\subsection{Pluralization by \textit{-e}}\largerpage

Pluralization by the addition of the suffix \textit{-e} is by far the most common strategy encountered in Dholuo. It can be thought of as a default strategy of sorts, as it is productive and applicable in the incorporation of loanwords (\cite[91]{Gregersen1961}). That said, the process is more complex than simply adding this suffix. In addition to now expected alternations affecting vowels, as introduced above, many instances of \textit{-e} pluralization involve some alternation or mutation \is{consonant mutation} that affects the last consonant of the stem. As will be seen in the remainder of this subsection, there are no fewer than ten unique outcomes to be discussed that fall under the heading of \textit{-e} pluralization, as previewed in Table  \ref{tab:ESuffixation}. 

\begin{table}
\caption{\textit{-e} suffixation}
\label{tab:ESuffixation}
 \begin{tabular}{llll}
  \lsptoprule
& Singular & Plural &  \\
\midrule
a.&	tʃɪ́k&	tʃɪ̀kὲ&	`law/regulation'\\
b.	&òsìkí	&òsíkê	&`tree stump'\\
c.&	gɔ̀t	&gɔ̀dὲ	&`hill/mountain' \\
d.	&ògûdú&	ògútê	&`hat/cap' \\
e.&	kùdnì&	kútê	&`worm' \\
f.&	bùᵑgú&	búᵑgê&	`bush/forest'\\			
g.&kɔ́m	&kɔ̀ᵐbὲ&	`chair'\\
h.	&mʊ̀mâ&	mʊ̀ᵐbὲ	&`Bible'\\
i.	&tɔ́l	&tɔ̀ⁿdὲ&	`rope'\\
j.&	lér&	lètʃè	&`vein'\\
  \lspbottomrule
 \end{tabular}
\end{table}   

Given the number and sometimes unusual nature of the alternations involved in \textit{-e} pluralization, morphophonologists have been both intrigued and puzzled by their outcomes and have offered several ways to analyze them formally. One such treatment by \citet{Alderete1999} involving \textit{consonant voice polarity} was \is{voicing polarity} introduced above and problematized as failing to address those cases where no such polarity occurs. \citet[245]{Trommer2011} instead argues, drawing inspiration from perspectives raised in \citet{Tucker1994}, that rather than simply involving the suffix \textit{-e}, pluralization of this type involves an abstract suffix /-Ce/ whose surface realization (in both quality and voice) depends on featural properties \is{abstractness} of the stem consonant. While it is far beyond the scope of this paper to provide a formal analysis of all outcomes weighed against Trommer's, I submit that a somewhat simpler solution can be posited if one considers this suffix to be underlyingly /-tE/.

\subsubsection{Voiceless stems with no voice alternation}\largerpage

The simplest instances of \textit{-e} pluralization involve no voicing alternation of the final consonant in the stem. Examples are provided in \tabref{tab:NoAlternation}. As pointed out by \citet{Tucker1994}, all such non-alternating stems involve an underlying stem-final voiceless consonant, a perspective which I adopt here. Note that here and elsewhere, though I use the term \textit{stem-final}, I extend this to mean the last consonant in a stem, which emerges as stem final as a result of stem-final vowel loss, a process also posited elsewhere (\citealt{Trommer2011, Tucker1994}). 

\begin{table}
\caption{C- and V-final stems with no voicing alternation}
\label{tab:NoAlternation}
 \begin{tabular}{llll}
  \lsptoprule
& Singular & Plural &  \\
\midrule
a.&	tʃɪ́k&	tʃɪkὲ&	`law/regulation'\\
b. & làk & lékê & `tooth' \\
c. & wátʃ & wètʃè & `word, information, news' \\
d.&	òsíép&	òsìèpè	&`friend'\\
e. &òfìs &òfísê & `office' \\  
f. &sɪ̀ⁿdàn & sìⁿdénê &needle'  \\
&&& \\ 
g.	&òsìkí	&òsíkê	&`tree stump'\\
h. &òdʒìkò &òdzíkê    &`spoon' \\
i.& òfúkò &òfúkê &`pocket' \\
j. & dìrísà &dìrísê &`window' \\
k.& dùkà & dúkê & `store' \\
  \lspbottomrule
 \end{tabular}
\end{table}   

C-final stems of this type undergo word-final devoicing vacuously, while corresponding V-final stems realize no voicing alternation. Upon pluralization, I assume that /-tE/ is added to the stem, though the suffixal consonant \is{abstractness} is ultimately lost. The alternation is simple for C-final stems, though the same occurs for V-final stems following the noted stem-final vowel loss. That is, stem-final vowel loss precedes suffixal consonant loss. The importance of the suffixal consonant to my analysis will become clear below. Briefly here, however, I assume that there is no voicing alternation because both the suffixal vowel and the stem vowel are voiceless. A different outcome occurs under other conditions. The surface quality of the suffixal vowel, of course, is predictable by \is{vowel harmony} ATR harmony.

\subsubsection{C-final obstruent stems: voiceless SG, voiced PL}

For consonant-final stems with a final obstruent that is voiced in the plural and alternates to voiceless in the singular, I assume that this obstruent is voiced underlyingly. Examples of this type are shown in \tabref{tab:FinalDeVoicing}. Here, the alternation to voiceless in the singular is due to word-final devoicing when the plural suffix is not present. Like in the plurals described just above, the suffixal consonant is not realized. It will be shown in the next subsection, however, that in V-final stems of this type, the suffixal consonant finally exerts its influence.

\begin{table}[H]
\caption{C-final obstruent stems - voiceless SG, voiced PL}
\label{tab:FinalDeVoicing}
 \begin{tabular}{llll}
  \lsptoprule
& Singular & Plural &  \\
\midrule
a.&	gɔ̀t	&gɔ̀dὲ	&`hill/mountain' \\
b.&	wàt	&wédê	&`relative' \\
c.&	bàt	&bédê&	`arm' \\
d.&	tʃàk&	tʃégê&	`milk' \\
e. & gùòk & gúógî & `dog' \\
f.	&lʊ̀θ&	lʊ̀ðὲ	&`staff/rod' \\
g.&	pìθ	&pìðè&	`slope (inclined surface)' \\
h.&	pùθ	&púðê&	`crippled person' \\
  \lspbottomrule
 \end{tabular}
\end{table}   

\subsubsection{V-final obstruent stems: voiced SG, voiceless PL}
\begin{sloppypar}
Singular/Plural pairs with consonant-final stems in \tabref{tab:FinalDeVoicing} can be directly compared to those in \tabref{tab:V-FinalDevoicing} which have vowel-final stems. The two types are markedly different from one another. Whereas the former have a voiceless consonant in the singular and a voiced consonant in the plural, the latter realize the exact opposite outcome: voiced singular, voiceless plural.
\end{sloppypar}

\begin{table}
\caption{V-final stems - voiced SG, voiceless PL}
\label{tab:V-FinalDevoicing}
 \begin{tabular}{llll}
  \lsptoprule
& Singular & Plural &  \\
\midrule
a.	&lwέdɔ̂ &	lúétê	&`hand' \\
b.&	kìdí	&kítê&	`stone' \\
c.	&ògûdú&	ògútê	&`hat/cap' \\
d.&	tùgò&	túkê	&`game/sport' \\
e.&	ɔ̀sɔ̂gɔ̀&	ɔ̀sɔ́kê&	`weaver bird' \\
f.&	tʃògó	&tʃókê&`bone' \\
g.&	àdʒʊ́ɔ̀gá&	àdʒùókê&	`diviner' \\
h.	&pùòðó&	pùóθê	&`farmland' \\
i.	&àɪ̂ðá&	àɪ́θê&	`squirrel' \\
j.	&kʊ́ðɔ̂&	kʊ̀θɛ&	`thorn' \\
  \lspbottomrule
 \end{tabular}
 \end{table}
 
I posit that this difference in stem shape is closely tied to the alternations that they participate in. Under the assumption of the suffix /-tE/, I would argue that C-final stems simply involve loss of the suffixal consonant: /gɔd+tE/  \rightarrow\ [gɔ̀dɛ̀] (\ref{tab:FinalDeVoicing}a). V-final stems, on the other hand, following a rule of \is{metrification} metrically-conditioned vowel loss introduced above, witness a rule of voicing agreement (analogous to that proposed by \cite{Trommer2011}) that is active, but only in derived environments. As such, one arrives at a derivation like /kidi+tE/ \rightarrow\ |kid+tE| \rightarrow\ |kit+tE| \rightarrow [kítê] (\ref{tab:V-FinalDevoicing}b). The suffixal consonant is arguably lost, but not before it exerts its influence on the voicing specification of the newly stem-final consonant.

Such an approach would offer an explanation for seemingly unrelated singular/plural pairs like	\textit{kùdnì/kútê} `worm/worms' and \textit{kògnò/kókê} `nail/nails (of fingers or toes)'.  Under the analysis just proposed, one could posit that the underlying form of `worms' is /kudn+tE/. Via a rule of nasal deletion between the adjacent stops, a derived environment \is{rule ordering} is created whereafter the suffixal stop is able to affect devoicing on the remaining stem consonant.

\subsubsection{Prenasalized stop stems: No voicing alternation}

Stems whose final consonant is a prenasalized stop constitute a separate sub-group that exhibits unique properties. These stems undergo no consonant alternations or mutations between the singular and plural that affect their stem consonants. Representative examples are shown in \tabref{tab:NCstems}.

\begin{table}
\caption{NC-final stems}
\label{tab:NCstems}
 \begin{tabular}{llll}
  \lsptoprule
& Singular & Plural &  \\
\midrule
a.&	bùᵑgú&	búᵑgê&	`bush/forest'\\			
b.&	rɔ́ᵐbɔ̂	&ròᵐbè&	`sheep'\\
c. & tàᵑgì & téᵑgê & `tank' \\
d. & rùᵑgú & rúᵑgê & `club (weapon)' \\
  \lspbottomrule
 \end{tabular}
\end{table}   

The behavior of these nouns is easily captured in the analysis under development thus far in this paper. Despite being V-final stems, and presumably undergoing stem-final vowel loss, the /-tE/ suffix realizes no effect on the stem consonant. I assume that this situation arises because \is{prenasalized consonants} pre-nasalized consonants are complex segments for which an alternation in voice would entail subsequent alternations that would be too phonologically or \is{phonotactics} phonotactically costly to be realized, and thus they remain unaffected.


\subsubsection{Nasal stems}

Noun stems whose final consonant is a nasal also support the proposed analysis with a plural suffix whose underlying form is /-tE/. As seen in the C-final nasal stems in \tabref{tab:FinalNasalAlt}, all such stems that end in a singleton nasal in the singular are realized with a prenasalized stop in the plural. 

\begin{table}
\caption{C-final stems - nasal alternations}
\label{tab:FinalNasalAlt}
 \begin{tabular}{llll}
  \lsptoprule
& Singular & Plural &  \\
\midrule
a.&kɔ́m	&kɔ̀ᵐbὲ&	`chair'\\
b.&	θìm&θìᵐbè	&`wilderness'\\
c.&	tɪ́m&	tɪ̀ᵐbὲ	&`behavior, manners'\\
d.&	nàm	&néᵐbê	&`lake'\\
e.&	lùm &	lúᵐbê	&`grass'\\
f.&	jɪ́ên&	jɪ̀ὲⁿdὲ	&`tree'\\
g.&	pìèn	&píéⁿdê	&`skin, hide, leather'\\
h.&	pìɲ	&pɪ̀ᶮdʒὲ	&`country, nation'\\
i.&	lwέɲ &	lwὲᶮdʒὲ	&`war, battle'\\
j.& 	tɔ́ŋ&	tɔ̀ᵑgὲ	&`egg' or `spear'\\
k.&	wàŋ	&wéᵑgê	&`eye'\\
  \lspbottomrule
 \end{tabular}
\end{table}   

\begin{sloppypar}
The alternations seen here are easily accounted for if one appeals to some rather uncontroversial phonotactic properties of Dholuo. Indeed, reference works like \citet{Tucker1994} and formal works like \citet{Trommer2011} similarly appeal to phonotactic restrictions (or constraints) in the language to explain a variety of other outcomes in the language's morphophonology. If one posits that Dholuo avoids nasal+voiceless stop sequences, and also that any sequence of nasal\,+\,voiced stop must be homorganic, the outcomes in \tabref{tab:FinalNasalAlt} easily emerge. Via the \is{underspecification} proposed suffix /-tE/, the stem-final nasal could simply be seen as progressively assimilating the suffixal consonant to its specification for both voicing and place of articulation.
\end{sloppypar}

Outcomes like those seen for V-final nasal stems in \tabref{tab:StemNasalAlt} require little additional explanation, provided that one recognizes stem-final vowel loss, as discussed here and elsewhere. Such vowel loss would simply \is{rule ordering} feed the downstream assimilations just mentioned.

\begin{table}
\caption{V-final stems - nasal alternations}
\label{tab:StemNasalAlt}
 \begin{tabular}{llll}
  \lsptoprule
& Singular & Plural &  \\
\midrule
a.	&mʊ̀mâ&	mʊ̀ᵐbὲ	&`Bible'\\
b.&	òlèmò&	òléᵐbê&	`fruit'\\
c.&	jàmɔ̀&	jéᵐbê&	`wind'\\
d.&	pɪ̀nɔ̀&	pɪ́ⁿdê	&`wasp'\\
e.	&sɪ́gáná&	sìgèⁿdè 	&`story'\\
f.&	kɔ̀ŋɔ̀&	kɔ́ᵑgê	&`alcoholic drink'\\
g.&	ɲɪ̀ɲɔ̀&	ɲɪ́ᶮdʒê&	`iron (metal)'\\
h.&	lwàŋnɪ̀&	lúéᵑgê&	`housefly'\\
  \lspbottomrule
 \end{tabular}
\end{table}   

\subsubsection{Stems with /l/ to [ⁿd] alternations}
Alternations discussed in this subsection, as well as those in the next subsection, are admittedly more complex, but still, they are transparently accounted for by the proposed analysis. Alternations seen in the examples in \tabref{tab:LateralAlt} involve what I will call ``lateral stems'', namely those whose final consonant is [l]. In the plural, one observes an alternation to [ⁿd]. 

\begin{table}
\caption{Stem-final lateral alternations}
\label{tab:LateralAlt}
 \begin{tabular}{llll}
  \lsptoprule
& Singular & Plural &  \\
\midrule
a.	&tɔ́l	&tɔ̀ⁿdὲ&	`rope'\\
b.&	ɔ̀gwàl	&ògúéⁿdê	&`frog'\\
c.&	thùòl&	thúóⁿdê&	`snake'\\
d.&	bùl	&bùⁿdè	&`drum'\\
e.	&tɪὲlɔ̀&	tíéⁿdé	&`foot/leg'\\
f.	&àðôlá	&àðòⁿdè	&`wound'\\
  \lspbottomrule
 \end{tabular}
\end{table}   

The alternations seen here also have an explanation grounded in Dholuo's \isi{phonotactics}. If one surveys the language's syllable contact sequences and its broader syllable phonotactics, it can be seen that liquid-stop sequences like *[lt] and *[rt] do not appear. In morphologically-derived environments where such sequences might be expected to appear, they are repaired. The chosen repair for *[lt] is [ⁿd], which involves two steps, though the order in which the steps occur is not entirely clear. I will assume that *[lt] is first repaired by progressive voicing assimilation to *ld, though this is subsequently repaired to [ⁿd] given that, as just stated, liquid-stop sequences, in general, are dispreferred in the language.

\subsubsection{Stems with /r/ to [tʃ] alternations}\largerpage

Last among instances of \textit{-e} pluralization are those nouns with stem-final /r/. These nouns witness an alternation to [tʃ] in their plural. Alternations like this are similar in ways to those described just above which repair an illicit *[lt] sequence. Like such sequences, a *[rt] sequence is similarly disallowed in Dholuo. Examples of nouns exhibiting these alternations are in \tabref{tab:StemRhoticAlt}.

\begin{table}
\caption{Stem rhotic alternations}
\label{tab:StemRhoticAlt}
 \begin{tabular}{llll}
  \lsptoprule
& Singular & Plural &  \\
\midrule
a.&	ɔ̀ŋὲr&	ɔ̀ŋέtʃê&	`monkey'\\
b.&	lér&	lètʃè	&`vein'\\
c.&	síbúòr&	síbúótʃé	&`lion'\\
d.&	bùr	&bùtʃè	&`hole'\\
e.&	bʊ̀r	&bʊ̀tʃὲ&	`boil'\\
f.	&dὲrɔ̀	&détʃê&	`granary'\\
g.&	kùèro	&kùètʃè&	`taboo'\\
h.&	ᵑgéró&	ᵑgètʃè	&`proverb'\\
i.&	jùòrò	&júótʃê	&`sibling-in-law'\\
j.&	bùrà	&bútʃê	&`meeting'\\
k.&àòrà	&àótʃê	&`river'\\
l.	&tʃíró&	tʃìtʃè&	`market'\\
m.&	màsɪ̂rà	&màsɪ̀tʃὲ	&`tragedy'\\
  \lspbottomrule
 \end{tabular}
\end{table}   

From a featural standpoint, and once again assuming /-tE/, one can understand these outcomes such that an alternation from /r/ to [tʃ] involves an alternation in the features [continuant] and [sonorant]. Alternations in both features emerge in favor of the suffixal consonant which is [-continuant, -sonorant]. That said, [r] is [+high], and this appears to be maintained on the resulting consonant, as affricates and other similarly palatalized sounds are often considered to be [+high]. If this is true, then the outcome is motivated by \isi{phonotactics} and thus by featural preferences, with two features contributed by the suffixal consonant, but one from the stem consonant. 

\subsection{Plural formation via suppletion}

Having described the many outcomes involving pluralization via suffixation in the sections above, focus turns now to other mechanisms of pluralization. In this section, I discuss Dholuo plural nouns whose formation involves the process of suppletion, whether involving the first element of a compound or the stem itself. Instances of \isi{suppletion} in \isi{compounding} are partial or ``weak'' -- they involve forms that are clearly morphologically related, yet deriving one form from the other is not strictly phonological. Instances of stem suppletion in Dholuo pluralization are either ``weak'' or ``strong''. I begin with a brief illustration of the comparatively less extensive matter of stem suppletion before turning to compounds.

\subsubsection{Stem suppletion}

As mentioned, stem \isi{suppletion} between singular and plural nouns may involve stems that are either demonstrably related to one another, as in \tabref{tab:WeakStem}, or otherwise clearly derived from entirely different stems, as in \tabref{tab:StrongStem}. The former represents a case of weak suppletion, while the latter is strong suppletion.

\begin{table}
\caption{Weak stem rhotic suppletion}
\label{tab:WeakStem}
 \begin{tabular}{llll}
  \lsptoprule
& Singular & Plural &  \\
\midrule
a.	&wuòrò&	wúónê	&`father'\\
b.&	mìjò	&mínê	&`mother'\\
c.	&pì&	pígê	&`water (body of)'\\
d.	&túô	&tuòtʃè	&`disease'\\
e.&	ðɪ̀àŋ	&ðòk	&`cow'\\
f.&	jò	&jòrè&	`way/path'\\
g.&	dɪ̀ὲl&	dìèk	&`goat'\\
h.&	lɪ̀ὲl	&líétê	&`funeral, grave'\\
  \lspbottomrule
 \end{tabular}
\end{table}   

\begin{table}
\caption{Strong stem suppletion}
\label{tab:StrongStem}
 \begin{tabular}{llll}
  \lsptoprule
& Singular & Plural &  \\
\midrule
a.&ðákɔ̂&	món	&`woman'\\
b.&	dálâ	&mìèr	&`homestead, usually in a village'\\
c.&	ŋàtɔ̀&	jì	&`person/people'\\
  \lspbottomrule
 \end{tabular}
\end{table}   

\subsubsection{Suppletion in compounds}

\citet[115--116]{Gregersen1961} points out the difficulty of distinguishing between \is{compounding} compounds and phrases in Dhouluo. Even for constructs that he defines as compounds, he often refers to their first element as a ``prefix''. With this inherent challenge in mind, several instances of plural formation involving these constructs are discussed below. In each, the first element, which is the head of the resulting construction, is inflected for plural number. The second, non-head element is largely unaffected as a result of the process, except tonally in some instances.

The first two instances of \isi{suppletion} discussed here do appear more prefix-like in that involved morphemes appear attenuated relative to the noun from which they are known to be derived. The first involves alternations affecting what I shall call the \isi{agentive} prefix \textit{dʒà-}, which alternates to \textit{dʒɔ̀-}. The other involves alternations affecting what I shall call the \isi{diminutive} prefix \textit{ɲà-}, which alternates with \textit{ɲí-}. Both are characterized as weak suppletion given their shared consonantal content, though the involved vocalic alternation is challenging to motivate phonologically in that it involves a morphologically-conditioned height mutation.

Starting with the former, agentive nouns and nouns for professions are formed by the addition of the agentive prefix to nouns, verbal nouns, or adjectives. The resulting noun means `a person or people of', `a person or people who come(s) from a geographical location X', or `a person who does some job, profession, or action'. The person/people referred to is/are usually male. As such, they function as the masculine counterparts of diminutive \textit{ɲà-/ɲí-} prefixation, which is typically used for females, as discussed below.

The singular agentive prefix \textit{dʒà-} is derived from the noun \textit{dʒàl} or \textit{dʒàgɔ̀} `person', while its plural counterpart arguably derives from \textit{dʒɔ̀gò} `the people' or `those people'. Thus, whether one considers the operation to be truly prefixation following by inflection for number, or otherwise a type of \isi{compounding} is open to interpretation. As elsewhere, stems further undergo various processes ranging from suffixation and subtraction, to consonantal, vocalic, and tonal alternations, all of which are covered elsewhere in this chapter. Examples of these nouns are given in \tabref{tab:AgentiveDza}.

\begin{table}
\caption{Agentive nouns with \textit{dʒà-/dʒɔ-}}
\label{tab:AgentiveDza}
 \begin{tabular}{llll}
  \lsptoprule
& Singular & Plural &  \\
\midrule
a.	&dʒàlúô	&dʒòlúô&	`Luo person'\\
b.&	dʒàsêᵐbò	&dʒòàsêᵐbò&	`[male] person from Asembo'\\
c.&	dʒàkɔ́m	&dʒɔ̀kɔ́m&	`chairperson'\\
d.&	dʒàkéꜜnó &	dʒòkéꜜnó&	`treasurer'\\
e.&	dʒàwèr&	dʒòwèr&	`singer'\\
f.&	dʒàpùr	&dʒòpùr	&`farmer'\\
g.&	dʒàgéꜜdó	&dʒògéꜜdó&	`mason' \\
h.&	dʒàlúꜜpó&	dʒòlúꜜpó&	`fisherman'\\
i.&	dʒàθìèθ	&dʒòθìèθ	&`doctor/medicine man'\\
j.	&dʒàpúóᶮdʒ	&dʒòpúòᶮdʒ(è) &	`teacher'\\
k.	&dʒàsʊ̀ᵑgá&	dʒɔ̀sʊ̀ᵑgá&	`boastful person'\\
l. &	dʒàwùòrò	&dʒòwúòtʃê 	&`selfish person'\\
  \lspbottomrule
 \end{tabular}
\end{table}   

Based on the nouns from which these prefixes are derived, we can assume that the vowels of both prefixes are underlyingly [-ATR]. However, the examples given illustrate, at least for the plural suffix, that they alternate regressively to the ATR quality of the vowels within the stem that they are attached to. This behavior is in line with \isi{ATR} harmony, as seen elsewhere in Dholuo. The low vowel of the singular prefix notably does not alternate to [e]. I assume that perhaps the failure of this alternation to occur is that it does not apply to prefixal elements, or otherwise does not cross a word boundary, rather applying only to stems and suffixes.\largerpage[-1]

Analogous outcomes are found with the prefix \textit{ɲà-}, which serves several purposes in Dholuo. One of its functions is as a diminutive morpheme, in which case, when it is attached to an animate noun, it forms the diminutive form of the noun in question. The prefix itself is derived from \textit{ɲàθɪ̂}, `child, young one'. Its plural is \textit{ɲɪ́θɪ̂}. Accordingly, a plural diminutive noun requires the prefix {ɲí-}. As was the case above for \textit{dʒa-}, it is unclear whether it is best to treat the formation of these words as an instance of affixation or \isi{compounding}. Either way, the witnessed alternation in \textit{ɲá-/ɲì-} can be considered a case of weak suppletion. Note that the singular prefix has an optional variant \textit{ɲàr-} that occurs before vowel-initial stems. Examples of singular and plural nouns formed with this prefix are in \tabref{tab:DimNya}.

\begin{table}
\caption{Diminutive nouns with \textit{ɲà-/ɲí-}}
\label{tab:DimNya}
 \begin{tabular}{llll}
  \lsptoprule
& Singular & Plural &  \\
\midrule
a.&	ɲàgwὲnɔ̀&	ɲígúén	&`chick'\\
b.&	ɲàrɔ́ᵐbɔ̂ &	ɲírôᵐbè	&`lamb'\\
c.&	ɲàrwaθ&	ɲírúêði	&`bull-calf'\\
d.&	ɲàrɔ̀jà&	ɲírôji	&`calf (of a cow)'\\
e.&	ɲàdɪὲl & 	ɲídíêk	&`kid (goat)'\\
f.&	ɲàguòk&	ɲígúógî&	`puppy'\\
g.&	ɲàkwàrɔ&	ɲɪ́kwâjɔ̀	&`grandchild'\\
h.&	ɲàθɪ̂&	ɲɪ́θɪ̂ndɔ̀	&`child (of a human)'\\
i.&	ɲàr asêᵐbò/ɲasêᵐbò &	ɲí àsêᵐbò	&`[female] person from Asembo'\\
j.&	ɲàr sàkwà/ɲàsàkwà	&ɲí sâkwà&	`[female] person from Sakwa'\\
  \lspbottomrule
 \end{tabular}
\end{table}   

As these examples illustrate, the prefix \textit{ɲà-} may also be attached to a locative noun to express the idea `a [female] person from place/region X'. To be clear, one prefix \textit{ɲà-} is derived from the word \textit{ɲàθɪ̂} `child or young one [of]'. This is the \textit{ɲà-} that functions as a \isi{diminutive}. A second prefix, \textit{ɲà}- is derived from the word \textit{ɲàr} `daughter of'. It is the latter that is attached to a locative noun to express the idea `a [female] person from place/region X', and in the plural, it is \textit{ɲí} `the daughters of'. These morphemes \textit{ɲà-} or \textit{ɲàr} and \textit{ɲí(-)} are the feminine counterparts of the morphemes \textit{dʒa-} and \textit{dʒɔ-} that were discussed above, specifically when the latter pair of morphemes is used to refer to a person or people from a particular geographical location. The morpheme \textit{ɲí} occurs as a word (i.e., a free morpheme) when used with nouns and as a prefix when used together with bound stems. 

Other examples of apparent \isi{compounding} can be found that largely align with the behavior discussed for the agentive and diminutive, though their first element or head is arguably less attenuated. While not discussed in works like \citet{Gregersen1961}, these nonetheless exhibit properties that he otherwise attributes to compounds. Two of these have in common that their complement is a verbal noun. One involves the noun \textit{gìr}, which generally refers to a `tool/item used for X'. Consider \textit{gɪ̀r ɔ̀t} `household item' or `tool for (a/the) house' which is composed of this noun and the word for `house'. When pluralized, the head noun is closely related \textit{gìk}. Both are subject to regressive ATR harmony. Additional examples are in \tabref{tab:GirCompound}.

\begin{table}
\caption{Compounds with \textit{gìr/gìk}}
\label{tab:GirCompound}
 \begin{tabular}{llll}
  \lsptoprule
& Singular & Plural &  \\
\midrule
a.&	gìr pùoðó	&gìk púôðó	&`farm supply/tool'\\
b.	&gìr tʃíró&	gìk tʃíró	&`market supply'\\
c.&	gɪ̀r ɔ̀hálá&	gɪ̀k ɔ́hálâ	&`business item/commodity'\\
d.&	gɪ̀r ɔ̀t	&gɪ̀k ɔ́t	&`household item'\\
e.&	gìr tùgò	&gìk túgô	&`something to play with'\\
f.&	gìr tùéꜜŋó	&gìk tùéꜜŋó	&`tool for sewing'\\
g.&	gìr pùóꜜᶮdʒó&	gìk pùóꜜᶮdʒó&	`teaching aid/supply'\\
h.&	gìr tèdò	&gìk tédô&	`cooking utensil/appliance'\\
  \lspbottomrule
 \end{tabular}
\end{table}   

Compounds \is{compounding} involving \textit{kàr} `a place for' are similar in that they take a verbal noun complement. Here, however, forming the plural involves much stronger \isi{suppletion} of the stem to \textit{kùòⁿdé}. Examples are provided in \tabref{tab:KarCompound}.

\begin{table}
\caption{Compounds with \textit{kàr/kùòⁿdé}}
\label{tab:KarCompound}
 \begin{tabular}{llll}
  \lsptoprule
& Singular & Plural &  \\
\midrule
a.	&kàr nɪ̀ⁿdɔ̀&	kùòⁿdé nɪ̀ⁿdɔ̀	&`place  for sleeping/sleeping area'\\
b.&	kàr tèdò&	kùòⁿdé tèdò	&`place for cooking/cooking area'\\
c.	&kàr tʃíéꜜmó	&kùòⁿdé tʃíéꜜmó	&`eating area/location'\\
d.	&kàr bʊ̀ðɔ̀	&kùòⁿdé bʊ̀ðɔ̀	&`resting area/place for relaxing'\\
e.&	kàr ròmò	&kùòⁿdé ròmò	&`meeting place/venue'\\
f.&	kàr pì	&kùòⁿdé pì	&`place where water is stored/fetched/sold'\\
g.	&kàr péᶮdʒ&	kùòⁿdé péᶮdʒ	&`exam location/venue'\\
  \lspbottomrule
 \end{tabular}
\end{table}  

A final compound of this type involves singular \textit{wùòn} and its plural counterpart \textit{wêg} which selects a nominal complement and derives nouns meaning `owner of X'. Examples are in \tabref{tab:WuonCompound}.

\begin{table}
\caption{Compounds with \textit{wùòn/wêg}}
\label{tab:WuonCompound}
 \begin{tabular}{llll}
  \lsptoprule
& Singular & Plural &  \\
\midrule
a.	&wùòn ɔ̀t	&wêg ɔ̂t	&`house owner'\\
b.&	wùòn kɔ̀m&	wêg kɔ́m	&`owner of chair'\\
c.&	wùòn pùòðó	&wêg púôðó	&`farmland owner'\\
d.&	wùòn lewni	&wêg léwnî	&`owner of clothing'\\
e.&	wùòn gùòk	&wêg gúôk	&`dog owner'\\
f.	&wùòn tɪ́gɔ̂	&wêg tɪ́gɔ̂&	`necklace owner'\\
  \lspbottomrule
 \end{tabular}
\end{table}  

These are unique in that the nominal complement can optionally be pluralized to yield a ``double plural'', a possibility mentioned in passing in \citet{Gregersen1961}. Consider, for example, \textit{wùòn pùòðó} `farmland owner', which can be realized \textit{wêg púôðó} `farmland owners', with only the head being pluralized, or as \textit{wêg pùóθê} `owners of farmlands', with both nouns pluralized. 

\subsection{Plural formation by subtraction}

A final type of plural formation to be discussed in Dholuo involves shortening or truncation of a singular noun form by eliding its final vowel. This morphological process is referred to as \isi{subtraction}. Most nouns in this category usually occur in their plural forms, and thus, one can perhaps assume that the plural noun forms the base upon which a singular noun is formed. One possible explanation for this is that, under normal circumstances, a person does not (for example) own only one fowl (especially in Luoland), and birds are usually seen in groups, rather than individually. Similar reasoning could be extended to most (but not all) other nouns behaving in this way. \tabref{tab:Subtraction} below shows a few examples of plural nouns formed by the process of subtraction. 

\begin{table}
\caption{Plural formation via subtraction}
\label{tab:Subtraction}
 \begin{tabular}{llll}
  \lsptoprule
& Singular & Plural &  \\
\midrule
a.	&kwànɔ̀&	kwán&	`count/total number/calculation' \\
b.&	únô	&ùn&	`thick rope' \\
c.	&wɪ̀ɲɔ́	&wɪ̀ɲ&	`bird' \\
d.&	wùotʃὲ	&wúótʃ	&`shoe' \\
e.	&gwɔ̀ɲɔ́	&gwɔ́ɲ&	`rash' \\
f.&	gwὲnɔ̀&	gúén&	`fowl' \\
  \lspbottomrule
 \end{tabular}
\end{table}  

\section{Summary and concluding remarks} \label{04_Section5}

This paper has examined plural formation in Dholuo by providing a description and analysis of various morphological and morphophonological processes that take place in the inflection of nouns for number. I have proposed, following others' perspectives on the language, that there are several morphological classes that dictate a given noun's plural formation. Notably, I have proposed a novel possibility to unify and explain six interrelated patterns involving \textit{-e}, or better yet /-tE/, suffixation. 

This analysis and other findings of this study build on previous research on Dholuo plural formation and also make a resourceful contribution to future research on the language, and perhaps other Nilotic languages. By providing a detailed synthesis of the various patterns and processes involved in number inflection, this study fills a gap left in most studies previously done in Dholuo plural formation that often discuss only a portion of the possible patterns, or otherwise fail to take into account all alternations involved in a given pattern. This study thus can serve as a point of reference not only for future linguistic research, but also current and future speakers and learners of Dholuo.\il{Dholuo|)} 

\section*{Abbreviations}
\begin{tabularx}{.45\textwidth}{@{}lQ}
ATR & Advanced Tongue Root\\
PL  & Plural\\
SG & Singular \\
\end{tabularx}


\section*{Acknowledgments}
Writing this paper has been both a learning process and a rewarding experience. I would first like to express my utmost appreciation to Prof. Samuel Obeng for his guidance and invaluable support during my PhD studies at Indiana University. To the editors of this volume, Christopher Green and Samson Lotven, I am thankful to you both for the tremendous amount of time and effort that you’ve put into this project to make the writing and publishing of this book a reality. I would also like to express my sincere thanks to the reviewers for all the time they took to read my paper and others’ as well. Thank you very much for your useful comments and suggestions, and detailed feedback. 

%\section*{Contributions}
%John Doe contributed to conceptualization, methodology, and validation.
%Jane Doe contributed to the writing of the original draft, review, and editing.

{\sloppy\printbibliography[heading=subbibliography,notkeyword=this]}
\end{document}
