\documentclass[output=paper,colorlinks,citecolor=brown]{langscibook}
\ChapterDOI{10.5281/zenodo.11091843}

\author{Christopher R. Green\affiliation{Syracuse University} and Katrina Smith\affiliation{University of Florida}}
\title{Poised to pivot: Kenyan Maay's restricted tone system} 
\abstract{This paper explores characteristics of languages with restricted tone systems, with a focus on dialects of the Cushitic language Maay. Languages with restricted tone systems, referred to by a variety of terms such as \textit{reduced tone}, \textit{pitch accent}, and \textit{non-stress accent}, among others, display several stress-like properties despite remaining definitionally tonal. We discuss two Maay dialects (Kenyan Maay and Baydhabo Maay) that have been on similar pathways toward stresshood, though each has retained different properties of tone systems. We present the Maay facts and compare these dialects' tonal behavior to that of other closely related languages with similarly restricted systems. Through consideration of relevant phonological and morphological processes in these languages, we examine the dividing line between tone and stress, in service of better understanding observed variation between restricted tone systems and pathways to stresshood. We propose an analysis that accounts for the ``near pivot'' status of these Maay tonal systems.}

\IfFileExists{../localcommands.tex}{
   \addbibresource{../localbibliography.bib}
   \usepackage{langsci-optional}
\usepackage{langsci-gb4e}
\usepackage{langsci-lgr}

\usepackage{listings}
\lstset{basicstyle=\ttfamily,tabsize=2,breaklines=true}

%added by author
% \usepackage{tipa}
\usepackage{multirow}
\graphicspath{{figures/}}
\usepackage{langsci-branding}

   
\newcommand{\sent}{\enumsentence}
\newcommand{\sents}{\eenumsentence}
\let\citeasnoun\citet

\renewcommand{\lsCoverTitleFont}[1]{\sffamily\addfontfeatures{Scale=MatchUppercase}\fontsize{44pt}{16mm}\selectfont #1}
  
   %% hyphenation points for line breaks
%% Normally, automatic hyphenation in LaTeX is very good
%% If a word is mis-hyphenated, add it to this file
%%
%% add information to TeX file before \begin{document} with:
%% %% hyphenation points for line breaks
%% Normally, automatic hyphenation in LaTeX is very good
%% If a word is mis-hyphenated, add it to this file
%%
%% add information to TeX file before \begin{document} with:
%% %% hyphenation points for line breaks
%% Normally, automatic hyphenation in LaTeX is very good
%% If a word is mis-hyphenated, add it to this file
%%
%% add information to TeX file before \begin{document} with:
%% \include{localhyphenation}
\hyphenation{
affri-ca-te
affri-ca-tes
an-no-tated
com-ple-ments
com-po-si-tio-na-li-ty
non-com-po-si-tio-na-li-ty
Gon-zá-lez
out-side
Ri-chárd
se-man-tics
STREU-SLE
Tie-de-mann
}
\hyphenation{
affri-ca-te
affri-ca-tes
an-no-tated
com-ple-ments
com-po-si-tio-na-li-ty
non-com-po-si-tio-na-li-ty
Gon-zá-lez
out-side
Ri-chárd
se-man-tics
STREU-SLE
Tie-de-mann
}
\hyphenation{
affri-ca-te
affri-ca-tes
an-no-tated
com-ple-ments
com-po-si-tio-na-li-ty
non-com-po-si-tio-na-li-ty
Gon-zá-lez
out-side
Ri-chárd
se-man-tics
STREU-SLE
Tie-de-mann
}
   \boolfalse{bookcompile}
   \togglepaper[2]%%chapternumber
}{}

\begin{document}
\maketitle

\section{Introduction}

Prosodic typology \is{prosodic typology} has long drawn a distinction between two main types of proso\-dic systems: tone and stress (accent). As recognized in the published literature (\citealt{Beckman1986,Hyman2009, McCawley1970, McCawley1978, Mous2021,vdH2011}, among others), however, this simple dichotomy falls short of capturing the true diversity of prosodic systems found in the world's languages. This issue is explored in Hyman's \citeyearpar{Hyman2006} cross-linguistic survey of \textit{word-prosodic typology} which illustrates an extensive array of languages whose prosodic systems ``pick and choose'' (p. 225) or select properties common to other tone or stress systems, rather than fitting fully or neatly into one category or the other. Various terms have been used to discuss and categorize these languages with hybrid prosodic systems, with \textit{pitch accent} and \textit{non-stress-accent} \citep{Beckman1986} being the best known among them. From the standpoint of \isi{prosodic typology}, however, not all of these prosodically hybrid systems exhibit the same properties. \citet{Hyman2009} argues, therefore, that such languages do not constitute a definable type of their own and are instead better treated as having ``restricted'' or ``reduced'' tonal systems. 

While \citet{ClementsGoldsmith1984} use ``transitional'' to describe such systems, the term ``pivot'' system has also been used in reference to restricted tonal systems that could be analyzable as either tone or stress (\citealt[228]{Hyman2006}, \citeyear[234]{Hyman2009}). The best known example of such a system is in Nubi [iso 639-3: kcn], a Sudanese Arabic-lexified creole spoken in Kenya and Uganda. Gussenhoven \citeyearpar{Gussenhoven2006} explains that \ili{Nubi} is neither a typical stress nor tone language. For example, it has no ``lexically idiosyncratic'' pitch marking, and metrically designated pitch and stress align with one another precisely, meaning that tone and stress cannot be disentangled from one another, or are redundant. For these and other reasons, Nubi's prosodic system is said to be analyzable either as a tone system or a stress system \citep{Gussenhoven2006,Hyman2009}.\footnote{Alongside Nubi, other languages discussed in the aforementioned works by Hyman are said to have pivot systems, but we would contend that they are different -- they do not display the same analytically ambiguous properties that Nubi does. Rather, they are still criterially tonal, albeit in a highly restricted sense. We believe that it is worthwhile to differentiate these non-analytically-ambiguous systems from analytically-ambiguous systems like Nubi's, in order to make clear that they are still best treated as tonal. That said, some such systems, like those found in dialects of Maay, exhibit properties that closely approach the tone/stress pivot. As such, we refer to these as ``near pivot'' systems.}

To be considered a tonal language, the bar is set fairly low. The well-known criterion that Hyman \citeyearpar{Hyman2001} adapts from Welmers \citeyearpar{Welmers1959} is that a language has a tonal system if it encodes pitch contrasts for ``at least some morphemes." These pitch contrasts must be phonologized along one or more featural dimensions and accordingly must be active in some way in the phonological grammar. Beyond these basic requirements, however, tonal systems vary widely in the properties that they exhibit. For example, tonal \is{prominence} prominences may occur once, more than once, or even not at all in a given word. The prominence bearer itself may be the syllable or the mora, and tones may or may not be metrically organized (see \citealt{AkinlabiLiberman2001,Green2015,Leben1997,Leben2003,Pearce2013}). These are just a few of the dimensions of tone system variation discussed in \citet{Hyman2009}.

While there is just one definitional parameter for tone systems, the properties of stress systems are more stringently defined. \citet{Hyman2009} proposes that, in total, there are six parametric settings that are ``definitional'' and thus \textit{required} of a stress system. Notable among these are that stress systems must exhibit prominence that is both \is{privativity} privative and \is{obligatoriness} obligatory -- there is one, and only one stress in a given word -- and that it is realized on syllables, rather than on moras. Languages with stress systems also \is{metrification} metrically arrange syllables which bear \isi{prominence} relations to one another. One potential dimension of variation, however, is that the phonetic correlates of stress may involve length, intensity, vowel quality measures (F1, F2), and pitch, or some combination of these. 

With this imbalance in definitional parameters, it is easy to see how variation in the properties of tonal systems might abound relative to stress systems. As should be clear from the preceding discussion, a given tonal system may exhibit some (or even several) stress-like properties, and this is precisely what occurs in languages with restricted tonal systems. These languages tend to display multiple stress-like characteristics, but yet they are unamenable to a stress-based analysis, minimally due to their maintenance of some phonologized featural contrast.


%Categorical stresshood, however, is more stringently defined. To see how this is so, consider that in arguing for a property-driven typology of prosodic systems, Hyman \citeyearpar{Hyman2009} defines and compares prototypical tone and stress systems along six dimensions, as in \tabref{tab:stresstone}.

%\begin{table}
%\caption{Stress vs. tone characteristics, adapted from Hyman \citeyearpar{Hyman2009}}
%\label{tab:stresstone}
% \begin{tabularx}{.8\textwidth}{lll}
 % \lsptoprule
 %         & \textit{stress} & \textit{tone} \\
 % \midrule
 %         \textit{form} & structural & featural\\
  %        \textit{function}& contrastive & distinctive\\
   %%      \textit{prominence bearer} & syllable & mora\\
     %     \textit{level} & lexical & URs\\
      %    \textit{domain}& output word & input morpheme\\
 % \lspbottomrule
% \end{tabularx}
%\end{table}

%Unlike stress systems, only one of the parameters – \textit{form} – is said to truly define tone systems. In terms of this parameter, by virtue of being \textit{structural}, stress is organizational: languages with stress systems metrically arrange syllables which bear prominence relations to one another. For tone systems, although they too may also exhibit a degree of metrical organization (e.g., \citealt{AkinlabiLiberman2001,Green2015,Leben1997,Leben2003,Pearce2013}), what sets them apart is their \textit{featural} encoding of pitch contrasts (see the criterion above). Rather than simply being present vs. absent, pitch contrasts are phonologized along one or more featural dimensions. These features are accordingly active in some way in the phonological grammar. 

%It is easy to see, with only the \textit{form} parameter being definitional of tonal systems, how variation in the properties of tonal systems might abound. A given tonal system may exhibit all or some subset of the remaining prototypical tonal properties above, or they may instead display some (or even several) stress-like properties. Indeed, languages with restricted tonal systems tend to display multiple stress-like characteristics, but yet they are unamenable to a stress-based analysis minimally due to their maintenance of some featural contrast, but often at least one other more tone-like characteristic (e.g., use of a moraic prominence bearer, non-obligatoriness of prominence in some subset of the lexicon, or the admission of more than one prominence in some words).

When it comes to languages with restricted or near pivot systems, particularly those that have arguably changed diachronically \is{diachrony} from being more prototypically tonal to a point where they are one or two parameters away from stresshood, there are several important questions to be asked that can provide insight into changes in and the typology and diachrony of prosodic systems more broadly. For example, for a given restricted tonal system, what property or properties prevent one from classifying it as a stress system? Can a given system's prosodic holdout be predicted from other properties of its grammar? A larger goal would be to assess what property/properties are more common prosodic ``holdouts'' than others. In the case of Maay, we shall see that pathways toward stresshood in different dialects appear to correlate with different responses taken to the historical loss of tone bearing units at the right edge of a word.

In the remainder of this paper, we begin by providing some necessary details of the prosodic properties of Maay, a Lowland East Cushitic language whose dialects exhibit markedly different tonal systems. Two of these dialects -- Kenyan Maay (KM) and Baydhabo Maay (BM) -- can be described as highly reduced tonal systems. As we shall see, these systems have changed diachronically to a point where they now stand at the line that separates tone and stress, yet they have done so in different ways. We illustrate that the ways in which they differ prosodically correlate with other aspects of their grammars. In doing so, Maay appears to be following a broader trend noted elsewhere that some \ili{Cushitic languages} ``have moved, or may be moving toward a predictable, syllable-counting stress-based (prosodic) system'' \citep{Appleyard1991}. For additional context, we also discuss closely related languages whose properties flank those of KM and BM. One of these is Lower Jubba Maay, another variety of Maay which has lost its tonal system entirely in favor of a syllable-based stress system with no pitch correlates. The other is Somali, whose prosodic system is at least marginally more tonal, such that is displays fewer characteristics otherwise typically associated with stress systems.

\section{Background on Maay and Cushitic tone}

Maay [iso 639-3: ymm] is also known as Af-Maay, Maay Maay, or Rahanweyn. In some earlier literature (e.g., \citealt{Biber1982,Saeed1982}), it has also been called Central \ili{Somali}. Though internal classifications differ between scholars, most would agree that Maay is rightly classified among other Lowland East Cushitic languages, the best described among which is Somali [iso 639-3: som]. In addition, Maay and Somali are closely related to the \il{Digil languages} so-called \textit{Digil} languages (Dabarre [iso 639-3: dbr], Garre [iso 639-3: gex], Jiiddu [iso 639-3: jii], and Tunni [iso 639-3: tqq]), as well as to \ili{Ashraaf} (also called Marka or Shingani), which has not been assigned an iso code. 

Compared to Somali and to other East Cushitic languages like Afar and Oromo, there has been little linguistic research conducted on Maay. Perhaps best known is Saeed's \citeyearpar{Saeed1982} sketch of ``Central Somali'' which describes the variety of Maay spoken near Baydhabo (Baidoa), Somalia. Another series of articles and chapters by \citet{paster07,OptionalmultiplepluralmarkinginMaayPaster10,Paster2018} and colleagues (\citealt{ComfortPaster2009,PasterRanero2015}) describe aspects of Lower Jubba Maay phonology and morphology. 

Of particular interest to the goals of this paper is Biber \citeyearpar{Biber1982}, which describes aspects of Maay's tonal system, focused on the dialect spoken in Mandera, Kenya. In addition, Smith's \citeyearpar{Smith2022} MA thesis is on the nominal tonal system of the Maay spoken in Dadaab, Kenya. The current paper builds upon the description of Kenyan Maay offered by Smith, and the data herein are drawn from an expansion of Smith's corpus. These data are from a female L1 speaker of the language who was raised through age 15 in Dadaab, Kenya, and thereafter resettled in the US with her family in 2013. She has since lived in sizable diaspora communities of Maay speakers. She currently lives in a close-knit community of approximately 200 Maay speakers in Syracuse, New York. Maay remains the primary language of her household and day-to-day interactions with family and other close relations. She is also fluent in English and has some conversational proficiency in Somali and reading proficiency in Arabic. While we recognize that there are potential shortcomings of having only a single speaker's productions represented here, we take care to point out that our data closely match what is reported for Kenyan Maay in Biber \citeyearpar{Biber1982}, both segmentally and tonally, though our data are more extensive and cover more contexts.\footnote{Throughout the paper, data are presented as they occur in the source cited. High (H) tone is indicated by an acute accent, as is customary in the Africanist tone literature. Data are presented in broad phonemic transcription, as there is no standard orthography for Maay. Morpheme boundaries are indicated by a hyphen “-” wherever relevant.}

In what follows, in addition to addressing our broader typological questions about the pathways of diachronic change in prosodic systems, we hope that this paper will contribute specifically to the microtypology of Cushitic prosodic systems. Fortunately, compared to what is known about most aspects of Cushitic phonology, there are reliable (though sometimes preliminary) descriptions of the prosodic characteristics of many of these languages that can serve as a basis for comparison to what we have observed for Maay. As one might glean from this parenthetical, however, there is much work that remains to be done on the subject. Notably, there is a fair amount of disagreement about the nature of Cushitic prosodic systems, namely whether they involve tone, stress, or ``pitch (tonal) accent.'' The Cushitic survey by Tosco \citeyearpar{Tosco2000} proposes that a divide can be made within the group such that many languages in the group, including the Highland East Cushitic sub-group, are stress (accent) languages. Others in the group, including most Lowland East \ili{Cushitic languages}, are instead tone (accent) languages, at least synchronically. For further information on this prosodic divide, the reader is encouraged to consult the typological survey of Cushitic prosodic systems and a summary of competing viewpoints on the matter in Mous \citeyearpar{Mous2021}, which is replete with references to many relevant works on the subject. 

\section{Tone in Maay}

\subsection{Overview}

Reconstructions of Proto-Lower East Cushitic (\citealt{Appleyard1991,Lamberti1986}) propose that the language exhibited a tonal contrast on the final two moras of most noun stems and that this contrast correlated with \isi{grammatical gender} -- grammatically masculine nouns had H tone on the penultimate mora (e.g., *gaála `he-camel') while feminine nouns had H tone on the final mora (e.g., *kimbiró `bird'). Based on broad comparison between Cushitic languages, these same scholars argue that some modern East Cushitic languages exhibit evidence of a \is{diachrony} historical \textit{accent shift}, ultimately resulting in a leftward retraction of H tone by one mora. This shift was accompanied by subsequent final vowel loss in some instances, thus at least partially obscuring the process. 

While many (but not all) languages tonally maintain this historical grammatical gender contrast, not all do so in the same way. \ili{Somali}, for example, witnessed both accent shift and final vowel loss. It maintains a contrast analogous to what is proposed for the Proto language: H tone is found on the penultimate mora of masculine words (e.g., \textit{gáal} `he-camel') and on the final mora of feminine words (e.g., \textit{shimbír} `bird'). Maay, on the other hand, witnessed no such uniform ``shift'', yet it did experience final vowel loss. Compare KM \textit{gaál} `he-camel', which maintains H tone on the same mora as the Proto language, with \textit{shímbir} `bird', where H tone surprisingly surfaces on the first mora of the stem, seemingly in the opposite distribution relative to Somali. As we shall see, however, this is not always the case. To explore Maay tone further, our initial focus will be on Kenyan Maay (KM) before turning to Baydhabo Maay (BM) and Lower Jubba Maay (LJM). 

\subsection{Kenyan Maay}

The\il{Kenyan Maay|(} varieties of KM described in \citet{Biber1982} and \citet{Smith2022} share many of the same properties, though ultimately the variety of word shapes and phrase types covered in Smith are more extensive than those in Biber. As illustrated in Smith \citeyearpar{Smith2022}, H tone location is diagnostic of a noun's grammatical gender, except in the case of monosyllabic CVC noun stems -- nouns of this shape in both gender series have H tone on their vowel, as in masculine \textit{mə́s} `snake' (\tabref{tab:MFIsolation}f) and feminine \textit{lə́f} `bone' (\tabref{tab:MFIsolation}l). The examples in \tabref{tab:MFIsolation} further illustrate this point, where notably the location of High tone in grammatically masculine nouns with consonant-final stems is invariable, surfacing on the stem's final vocalic mora. In grammatically feminine nouns, the location of H tone differs, but predictably so, based on stem shape and context. In addition, determiner shape and type also play a role in dictating surface H tone location in these nouns. 

\begin{table}
\caption{Tone in Masculine and Feminine nouns -- C-final, isolation}
\label{tab:MFIsolation}
 \begin{tabular}{llllll}
  \lsptoprule
     \multicolumn{3}{l}{Masculine}  & \multicolumn{3}{l}{Feminine} \\
  \midrule
     a. & saháŋ & `plate' &g. & bɪ́laŋ & `woman' \\
     b. & ʕonʊ́k & `child' &h. & ɲáɲur & `cat' \\
     c. & wəreék & `circle' &i. & daróor & `cloud' \\
     d. & gaál & `camel' & j. & wéel & `calf'\\
     e. & tuqæǽŋ& `store' & k. &  búúr & `heel' \\
     f. & mə́s & `snake' & l. & lə́f & `bone' \\
  \lspbottomrule
 \end{tabular}
\end{table}

As seen in \tabref{tab:MFIsolation}, in isolation, most consonant-final Feminine nouns have High tone on their penultimate vocalic mora (\tabref{tab:MFIsolation}g--j). There is variation for CVVC nouns, however, which surface either as \textit{cv́vc}, similar to what occurs for other stem shapes (\tabref{tab:MFIsolation}j), or instead \textit{cv́v́c}  (\tabref{tab:MFIsolation}k), with a flat High span across the entire monosyllabic word. We return to this matter of variation below. As seen thus far, however, the tonal distinction between most Masculine and Feminine stems of the same shape is lexically determined and thus idiosyncratic. 

Exceptions to this surface tonal contrast between Masculine and Feminine nouns are seen only in limited contexts, and notably in nouns with vowel-final stems in isolation. Examples can be seen in \tabref{tab:MFVfinalIsolation}, where the tonal distinction between them is \is{neutralization} neutralized, as in (\tabref{tab:MFVfinalIsolation}a) vs. (\tabref{tab:MFVfinalIsolation}e), etc. 

\begin{table}
\caption{Tone in Masculine and Feminine nouns -- V-final, isolation}
\label{tab:MFVfinalIsolation}
 \begin{tabular}{llllll}
  \lsptoprule
  \multicolumn{3}{l}{Masculine}  &  \multicolumn{3}{l}{Feminine}\\
  \midrule
 a. & dʊβʊ́ & `bull' & e. &  ʃaqə́ & `work'\\
 b. & wəβɛ́ & `river' & f. & boðə́ & `thigh' \\
 c. & waxtə́ & `time' & g. & toorə́ & `knife' \\
 d. & hoŋgʊɾɨ́ & `meal' & h. & aayó & `mother' \\
  \lspbottomrule
 \end{tabular}
\end{table}

In some but not all other contexts, for nouns of this same shape, a surface tonal contrast between these nouns emerges. For example, in \tabref{tab:MFVFinalDet}, we see that before a basic definite determiner - \textit{Cə} - the tonal distinction remains neutralized. However, before \textit{Cii}, the remote definite determiner (\textsc{rem}), whose use is appropriate only with referents that were previously active in the discourse, a surface tonal contrast appears. Here and elsewhere, determiners may be indicated as beginning with \textit{C}, a consonant placeholder for a grammatical gender prefix. The basic forms of these agreement prefixes are \textit{k} (Masculine) and \textit{t} (Feminine), though they have other predictable variants.

\begin{table}
\caption{Tone in Masculine and Feminine nouns -- V-final, with determiners}
\label{tab:MFVFinalDet}
 \begin{tabular}{lllll}
  \lsptoprule
  &     Masculine  &  & Feminine &  \\
  \midrule
Basic &guɾú-ɣə & `the house'  & qahwə́-ðə & `the coffee' \\
Remote & guɾú-ɣii & `the (\textsc{rem}) house' & qáhwə-ðii & `the (\textsc{rem}) coffee' \\
  \lspbottomrule
 \end{tabular}
\end{table}

Given the static nature of High tone in Masculine nouns and its predictable distribution in Feminine nouns, Smith \citeyearpar{Smith2022} analyzes KM's behavior in terms of a \is{privativity} privative High vs. toneless contrast for Masculine vs. Feminine nouns, respectively. More specifically, she argues that Masculine nouns are underlyingly specified for High tone on their final vocalic mora in all instances, while Feminine nouns are instead toneless. Due to a requirement for \is{obligatoriness} obligatory High tone on all nouns, however, a High tone is provided by the phonology and assigned to Feminine nouns based on several interrelated factors pertaining to word shape. Consider, for example, the nouns modified by the basic definite determiner in \tabref{tab:MFBasicDef}. 

\begin{table}
\caption{Tone in Masculine and Feminine nouns -- basic definite}
\label{tab:MFBasicDef}
 \begin{tabular}{lllll}
  \lsptoprule
  &      Masculine  &  & Feminine &  \\
  \midrule
 C-final& fərə́s-kə & `the horse' & bɪʃɪŋ-tə́  & `the lip' \\
 V-final & rootɨ́-ɣə & `the bread' & bɛɛsɨ́-ðə & `the money' \\
  \lspbottomrule
 \end{tabular}
\end{table}

As expected, Masculine nouns have H tone on the final vocalic mora of the stem. Feminine nouns, however, have H tone on the determiner if the stem is C-final and on the final stem vowel if the stem is V-final. For nouns modified by the proximal demonstrative determiner in \tabref{tab:MFProxDem}, H tone surfaces on the determiner for all Feminine nouns.

\begin{table}
\caption{Tone in Masculine and Feminine nouns -- proximal demonstrative}
\label{tab:MFProxDem}
 \begin{tabular}{lllll}
  \lsptoprule
  &      Masculine  &  & Feminine &  \\
  \midrule
C-final & fərə́s-ɣəŋ & `this horse' & bɪʃɪŋ-tə́ŋ & `this lip' \\
V-final & rootɨ́-ɣəŋ & `this bread' & qahwə-ðə́ŋ & `this coffee' \\
  \lspbottomrule
 \end{tabular}
\end{table}

Yet another outcome for Feminine nouns is seen in the presence of a \is{possession} possessive determiner. In \tabref{tab:MFPosssesive}, for the first person singular, H tone in Feminine contexts surfaces on the determiner in C-final nouns. For V-final nouns, however, H instead surfaces on the stem. The difference, compared to basic definites in \tabref{tab:MFBasicDef}, where H also surfaces on the stem, is that High is now in penultimate, rather than final position.\footnote{A reviewer asks whether there is any evidence that diphthongs are bimoraic such that H could be said to be assigned to the penultimate mora of the determiner in forms like `my lip' in \tabref{tab:MFPosssesive}. We tentatively follow \citet{Smith2022} in assuming that this is true, and also that word-final consonants appear to ``count'' in the calculation of tone assignment. We believe that this plays a key role in dictating why H tone assignment differs in C-final vs. V-final Feminine nouns under different conditions. Unfortunately, independent evidence for this assertion is yet difficult to come by for Maay. For closely related \ili{Somali}, however, the role of moras, both vocalic and consonantal, is well established. This is evidenced from a variety of outcomes pertaining to reduplication (\citealt{Orwin1996}) and poetic metrics (\citealt{Orwin2001}), as well as word minimality and syllable shape distribution (\citealt{Green2022}).} Taken together, stem shape, as well as determiner shape and type, play a role in dictating surface H tone location in Feminine nouns. 

\begin{table}
\caption{Tone in Masculine and Feminine nouns -- first singular possessive}
\label{tab:MFPosssesive}
 \begin{tabular}{lllll}
  \lsptoprule
  &      Masculine  &  & Feminine &  \\
  \midrule
C-final & fərə́s-key & `my horse' & bɪʃɪŋ-téy & `my lip' \\
V-final & rootɨ́-ɣey & `my bread' & qáhwə-ðey & `my coffee' \\
  \lspbottomrule
 \end{tabular}
\end{table}

These data and generalizations strongly suggest that KM has a restricted tonal system. The language exhibits a privative tonal contrast between H and toneless stems. High tone is predictably assigned to underlyingly toneless stems based on stem shape and the presence vs. absence of modifiers of different types. High tone is also obligatory \is{obligatoriness} on all content words (i.e., nouns and verbs), except in one syntactic context, namely on post-verbal objects, regardless of their focus status. Such a postlexical, syntactically-defined instance of non-obligatoriness is reminiscent of \ili{Nubi}, for which Gussenhoven \citeyearpar{Gussenhoven2006} argues that phrasal phonology affects a ``deaccenting rule'' on gerunds when they take an object. This differs from Somali, for example, where the two instances in which High tone is absent are morphologically determined \citep{GreenLampitelli2022,hyman81}.

Despite the fact that a H may appear on a stem in some contexts, or on a modifier in others, there are never any instances in which a H appears on both at once. In this way, H in KM is \is{culminativity} culminative in all contexts that we have explored. This analysis is further substantiated by looking at nouns and certain modifiers, including plural suffixes and other derivational suffixes. Most telling are \textit{-yaál} plurals, like those in \tabref{tab:MFYaal}, whose suffix is underlyingly tonal. 

\begin{table} 
\small
\caption{Tone in Masculine and Feminine nouns -- \textit{-yaál} plurals}
\label{tab:MFYaal}
 \begin{tabular}{llllll}
  \lsptoprule
  \multicolumn{3}{l}{Masculine}  & \multicolumn{3}{l}{Feminine}\\
  \midrule
a. & fəɾə́s & `horse' & d. &gə́laŋ & `arm' \\
b.&fəɾəs-yaál & `horses' &e.& gəlaɲ-yaál& `arms' \\
c.&fəɾəs-yaál-kii & `the (\textsc{rem}) horses' & f.&gəlaɲ-yaál-kii & `the (\textsc{rem}) arms' \\
  \lspbottomrule
 \end{tabular}
\end{table}

The \textit{-yaál} suffix has a static H, analogous to, and in the same position as H on Masculine noun stems. The suffix itself is also grammatically masculine; it requires masculine agreement on any modifying determiner. In the presence of this plural suffix, High tone fails to be realized on the noun stem. This includes Masculine nouns that otherwise exhibit a lexical High tone -- when a second High is contributed by the plural \is{pluralization} suffix, only one tone (the suffixal High) survives.

Similar outcomes obtain when a derivational suffix with a static H tone modifies the stem. When forming \isi{agentive} nouns and gerunds, H appears on the suffix, effectively overriding the stem High. This is particularly apparent for Masculine nouns given their otherwise static lexical High tone (cf. \textit{degaál} `fight', \textit{degaal-ə́} `killer', and \textit{degaal-ə-mó} `fighting'). Alternation of the stem High is also observed in nominal compounds where, in a N1 + N2 \is{compounding} compound, a single H appears on N2. 

What remains an outstanding issue is defining the domain of \isi{culminativity}. Importantly, one must ask whether determiners are prosodified with the stem that they modify (e.g., perhaps as clitics or suffixes), or whether they (or at least some of them) should instead be considered separate \is{prosodification} prosodic words.

While it is true that individual determiners behave differently in terms of their ability to bear a H tone, this may simply be a consequence of their shape, as opposed to their wordhood status. Unlike \ili{Somali}, where some determiners function as independent words (\citealt{GreenMorrison2016}), our KM language consultant was reluctant to use determiners in this way. Moreover, it is worthwhile to note that in Saeed \citeyearpar[90--91]{Saeed1982}, all determiners are listed with a ``hyphen'' (e.g., \textit{-kaas/\nobreakdash-taas}), suggesting that they were analyzed as affixal in that research. We assume, therefore, that a noun and its modifying determiner form a single word, and thus, we take the domain of H tone culminativity to be the word. We do so in full acknowledgment that the matter deserves further attention.

Another matter of importance \is{prominence} concerns KM's \textit{prominence bearer}. Though there are various stress-like properties manifested in KM's restricted tonal system, the language still clearly counts moras for the purpose of High tone assignment, and the mora remains the surface prominence bearer. Recall that this property of tone systems differs from the definitional requirement of stress systems that the syllable be the prominence bearer. There is some evidence, however, that the identity of the language's prominence bearer is in flux. More specifically, we find that for Feminine nouns where one might expect the phonology to assign a H that would result in a rising contour, KM conspires to avoid this outcome in several ways. 

For example, vowel-final Feminine stems of the shape \textit{cvvcv́} (as in \tabref{tab:MFVfinalIsolation}), when modified by the remote definite determiner (where we would otherwise expect H tone retraction to yield \textit{cvv́cv}), surface with one of \is{variation} three different lexically-determined outcomes:

\begin{itemize}
    \item High remains on the final vowel, as in \textit{bɛɛsɨ́-ðii} `the (\textsc{rem}) money'
    \item High retracts and also decontours, as in \textit{tɛ́ɛ́sɨ-ðii} `the (\textsc{rem}) fly' 
    \item High retracts to the first mora of the first syllable, yielding a fall, as in \textit{tóorə-ðii} `the (\textsc{rem}) knife' 
\end{itemize}

These facts taken together suggest that derived rising contours are dispreferred in the language and that their creation is avoided whenever possible. Despite this, however, such contours are not necessarily banned outright in KM. There is no rising contour avoidance in Masculine nouns: H tone on consonant-final Masculine stems in isolation is always on the final mora, including on the second mora of a long vowel, which yields a rising contour. This High tone is also immune to alternation in the presence of determinersː cf. \textit{gaál} `camel' and \textit{gaál-kii} `the (\textsc{rem}) camel'.\footnote{Recall that some suffixes attract H tone from stems across the board, similar to what occurs in forming \is{compounding} compounds. Interrogative determiners, like in \ili{Somali} (\citealt[245]{greensomgram}, \citealt[114]{saeed1999}), also behave exceptionally in that they attract H tone in all instances.} Likewise, as seen above, such contours are not disallowed in the formation of \textit{-yaál} plurals, which behave in many ways like other Masculine nouns. 

The noted generalization about the \is{tonotactics} avoidance of derived contours finds further support in the behavior of some Feminine nouns in isolation, and extends even to falling contours. Recall from \tabref{tab:MFIsolation} that some monosyllabic Feminine \textit{cvvc}-shaped nouns surface with a flat \textit{cv́v́c} tonal contour, rather than an otherwise expected falling \textit{cv́vc} sequence. These tonal sequences are considered derived under an analysis where Feminine nouns are underlyingly toneless and H tone is assigned only later by the phonology.

Given the behavior of Masculine nouns and \textit{-yaál} plurals in various contexts discussed above, we assume that the maintenance of these contours is indicative of the language prioritizing tonal faithfulness, despite contours being otherwise dispreferred in the language. When tonal faithfulness is not at play, as in Feminine nouns, contours are actively selected against. As such, the alternations discussed above concerning High tone assignment in modified Feminine nouns appear to be driven by markedness (i.e., the avoidance of tonal contours), and they have implications for our understanding of the KM prosodic system overall.{\interfootnotelinepenalty=10000\footnote{KM's strong preference to avoid rising contours is perhaps not unexpected given that such contours are known to be marked relative to falling contours cross-linguistically. Contours, in general, are of course more marked than level tones. For more on tonal \isi{markedness}, see reference works like \citet{Gordon2001}, \citet{Hyman2009b}, and \citet[27--30]{Yip2002}, among many others.}}

When left to the phonology, as opposed to the lexicon, it would appear that KM's tone assignment algorithm requires little explicit reference to the mora. The location of High tone assignment could easily be defined by syllable (penultimate or final), with the key considerations being stem and determiner shape. In some instances, an otherwise expected moraic tonal contrast on long vowels is being leveled via decontouring, though as we have seen, there remain some instances where a fall emerges across a long vowel, instead of decontouring, as one way to avoid creating a rise: cf. \textit{toorə́} `knife' and \textit{tóorə-ðii} `the (\textsc{rem}) knife', *\textit{toórə-ðii}. Elsewhere, however, even falling contours are leveled, as we have seen in some monosyllabic Feminine nouns in isolation. Of course, reference to the mora remains indispensable to describe the tonal behavior of Masculine nouns. Despite this, however, if the classification of KM's prosodic systems as `near pivot' rests in large part on its moraic \isi{prominence} bearer, one might question whether such outcomes (and instances of variation) in Feminine nouns portend the language's eventual restructuring in the direction of syllables.\largerpage

Synchronically, KM's tonal system may not be actively seeking out moras as tone bearing units, though the mora's role in this regard remains as a residue of the lexicon. Put another way, faithfulness to underlying stem tone is effectively preserving the mora as the language's tone bearing unit, but only in a subset of the lexicon.

With the behavior of Kenyan Maay now established, we turn our attention to \ili{Baydhabo Maay}. As will be seen, this second dialect of Maay has also developed more stress-like properties over time, but it has done so in a way that differs from KM.\il{Kenyan Maay|)}

\subsection{Baydhabo (Baidoa) Maay}

Saeed \citeyearpar{Saeed1982} reports that for \textit{Central Somali}, i.e., Maay, the language's tonally-encoded \isi{grammatical gender} contrast in nouns, unlike what is observed in Somali and in KM, has nearly been \is{neutralization} neutralized. Illustrative examples from Saeed \citeyearpar{Saeed1982} are shown in \tabref{tab:MFSaeed}. 

\begin{table}
\caption{Tone in Masculine and Feminine nouns -- Baydhabo Maay}
\label{tab:MFSaeed}
 \begin{tabular}{llllll}
  \lsptoprule
    \multicolumn{3}{l}{Masculine}  &  \multicolumn{3}{l}{Feminine} \\
  \midrule
  a.& boodə́ & `thigh' & g.& osbə́ & `salt' \\
b. & kirkirə́ & `wild pig' & h. & ɖilmaaɲə́ & `mosquito' \\
c. & gorgór & `vulture' & i. &aftíin & `light' \\
d. & fárow & `zebra' & j. &ókun & `egg' \\
e. & búr & `flour' & k. &bad & `ocean' \\
f. & wéer & `jackal' & l. & buur & `mountain' \\
  \lspbottomrule
 \end{tabular}
\end{table}

We see here that most grammatically masculine (Masculine) and feminine (Feminine) nouns have a single High tone on their final syllable, regardless of word shape. This applies, regardless of whether the stem is consonant- or vowel-final, provided that the noun stem is larger than one syllable (\tabref{tab:MFSaeed}a--c, g--i). Another smaller group of nouns has a High on the penultimate syllable, but once again, there is no contrast between Masculine and Feminine nouns (\tabref{tab:MFSaeed}d, j). For these nouns, grammatical gender is recoverable only via agreement on determiners and verbs. Monosyllabic nouns differ in that a tonal contrast is maintained between the presence (on Masculine nouns) vs. absence (on Feminine nouns) of tone (\tabref{tab:MFSaeed}e--f, k--l). Thus, for larger noun stems, tone location is lexically idiosyncratic, while for monosyllabic stems, the same can be said about presence vs. absence of tone. 

These examples show that there is no need to invoke moras to analyze \ili{Baydhabo Maay} (BM) tone assignment. Indeed, the mora is never mentioned in Saeed's description. It is worthwhile to note that Saeed \citeyearpar[8]{Saeed1982} does mention pitch, length, and loudness as phonetic correlates of ``stress'' on syllables with a High tone: ``This tone opposition could possibly be described in terms of accent or stress. Prominent \is{prominence} syllables are higher in pitch, slightly longer than their non-prominent counterparts (whether short or long), and louder.''

In comparing these examples in BM to those from KM above, it should be clear that while the varieties share several properties, they are distinct from one another prosodically. While KM has \is{obligatoriness} obligatory tone and at least residual reference to the mora for tone assignment, BM fails to exhibit High tone on some content words, as seen in (\tabref{tab:MFSaeed}e--f). Where H tone is present, no reference to the mora is necessary to describe its distribution. 

With this in mind, one way to view BM relative to KM is that BM, as a result of the historical final vowel loss described above, has begun to neutralize its lexical tonal contrast rather than to reconfigure it. Where \isi{neutralization} occurs, that is, in monosyllabic nouns, it obviates tonal reference to the mora. Recall that in each variety, reference to the mora is still required outside the tonal system (there is a length contrast for vowels and some consonants). Viewed in this way, BM could also be seen as having a ``near pivot'' system, but one in which the parametric ``hold out'' relates instead to obligatoriness of tone. Moreover, according to Saeed's description of BM, pitch and other acoustic properties that align with it as \isi{prominence} correlates are redundant, just as what occurs in Nubi's ``pivot'' system.

For KM, one could easily imagine the language extending its decontouring rule to Masculine nouns, and for BM, one could similarly imagine the emergence of obligatory H tone on monosyllabic Feminine nouns. Both varieties could perhaps be seen as `poised to pivot' from tone to stress. They might not necessarily do so, though another dialect of Maay appears to have done so.

\subsection{Lower Jubba Maay}

In the interest of arriving at a more complete picture of Maay's potential path(s) to stresshood, it is worthwhile to note that \ili{Lower Jubba Maay} (LJM), as described in works by Paster and colleagues (\citealt{ComfortPaster2009,paster07, OptionalmultiplepluralmarkinginMaayPaster10, Paster2018, PasterRanero2015}), has already passed the pivot point. These works report obligatory, syllable-based \isi{prominence} on all content words; and moreover, it is claimed that there are no pitch correlates associated with this prominence. LJM is thus rightfully analyzable as a stress accent language. The data in \tabref{tab:DialectComp} provide a few illustrative comparisons between LJM, BM, and KM. We follow the aforementioned works on LJM in indicating stress by underlining the stressed syllable.

\begin{table}
\caption{Select comparisons between LJM, BM, and KM}
\label{tab:DialectComp}
 \begin{tabular}{lllll}
  \lsptoprule
    & LJM & BM & KM & Gloss \\
  \midrule
  a. & w\underline{e}l & wéel & wéel & `calf' \\
   b. & usb\underline{o} & usbə́ & usbə́ & `salt' \\
 c. & bood\underline{o} & boodə́ & bodə́ & `thigh' (BM/KM)\\
   &&&& `heel' (LJM) \\
 d. & l\underline{u}k & lug & lúk & `leg' \\
  e. & r\underline{oo}p & róob & roóp & `rain' \\
  \lspbottomrule
 \end{tabular}
\end{table}

While this comparison highlights the high degree of lexical similarity shared by the three Maay dialects, it also shows some ways in which they depart from one another prosodically. Comparison (\tabref{tab:DialectComp}a), for a monosyllabic masculine noun, and also (\tabref{tab:DialectComp}b--c) for V-final nouns, show that there are sometimes close prosodic correlations between the three varieties. However, comparison (\tabref{tab:DialectComp}d) shows that the varieties sometimes differ markedly from one another -- some monosyllabic feminine CVC nouns which are toneless in BM are instead toned in KM. Comparison (\tabref{tab:DialectComp}e) further shows a masculine CVVC noun which exhibits the basic gender-neutralized pattern in BM but the static rising contour expected of KM nouns of this type. 

\section{Next steps and concluding remarks}

In this paper, we have focused on properties -- both synchronic and diachronic~-- of two ``near pivot'' restricted tone systems in dialects of Maay. For each, we have seen that their synchronic prosodic parameters position them on the cusp of stresshood. Our primary goal has been to explore whether some aspect of their grammar -- their phonology, lexicon, etc. -- may have contributed to their particular paths \is{diachrony} from tonehood toward stresshood. Though undoubtedly tentative, we have identified that, following a reported historical \is{prosodification} prosodic erosion at the right edge of the noun stem via final vowel loss, these dialects have made  different adaptations to compensate for the loss of final tone bearing units. Kenyan Maay has maintained a grammatical tonal contrast between stem types by reconfiguring the historical penultimate vs. ultimate distribution of tones. Baydhabo Maay has lost this tonal distinction in nearly every instance. From a parametric standpoint, Kenyan Maay marginally maintains the mora as its surface prominence bearer while Baydhabo Maay instead fails to require an obligatory tonal prominence in a small subset of the lexicon. Thus, we have observed two synchronic near pivot points that would appear to have arisen in systems with different diachronic phonological developments. 

Let us consider briefly a comparison to \ili{Somali}. We discussed above that Somali's prosodic system is tonal, but to quote Hyman \citeyearpar[216]{Hyman2009}, it is ``far from anyone's ideal or prototypical tone system.'' Despite this, and when compared to KM and BM, Somali's prosodic system seems rather robust. Somali has maintained the Proto Lower East Cushitic tonal contrast and has done so via leftward tone shift by one mora. It also exhibits a privative \is{privativity} tonal contrast with High tone remaining both culminative \is{culminativity} and \is{obligatoriness} obligatory in the lexicon. However, under two morphological conditions -- in non-focused subjects and indicative (realis) mood verbs -- High tone is lost. High tone is assigned by mora count, and the mora is the surface \isi{prominence} bearer, though there is variation whereby rising contours are flattened (\citealt{banti88}); falling contours, however, are widely attested. Based on these characteristics, one could assert that Somali exhibits fewer stress-like properties than either Maay variety, though it ultimately has more in common with KM than with BM. Both Somali and KM have obligatory prominence in the lexicon whose primary correlate is pitch, and both rely on the mora as the tone bearing unit. Relatedly, both do realize some instances of decontouring, though this occurs less commonly in Somali. 

\begin{sloppypar}
The question that remains is whether or not the commonalities between Somali and KM truly extend from both varieties having reconfigured (and thus, maintained) the tonal contrast reconstructed for the Proto language. These two languages, as we have seen, differ from BM, which is has not reconfigured the Proto tonal contrast. At the very least, these correlations are intriguing and worthy of further exploration. If and whether similar correlations obtain more broadly must await future research on other restricted tone languages. Other Lowland East Cushitic languages could certainly offer clues in this regard and provide an ideal testing ground to delve more deeply into this line of inquiry.
\end{sloppypar}

%something about Somali maintaining tonal contrast, etc.; the contrast is still robust, as opposed to BM. 
%Somali > shift to maintain > obligatoriness, but morphological (there is some decontouring of rising), but falls are maintained. Pitch is the only consistent prosodic correlate of prominence. The mora TBU is more robust in Somali, but obligatoriness is absolute in the lexicon and only lost morphologically. In this way, Somali looks somewhat more like KM, and is definitely more tone-like than either of the Maay varieties.
%BM > loss > obligatoriness, but lexical
%KM > maintain/reconfigure > mora to syl

\section*{Abbreviations}
\begin{multicols}{2}
\begin{tabbing}
mmm   \= grammar \kill
BM    \> Baydhabo (Baidoa) Maay \\
H     \> high tone \\
KM    \> Kenyan Maay \\
L     \> low tone \\
LJM   \> Lower Jubba Maay \\
REM   \> remote definite determiner 
\end{tabbing}
\end{multicols}

\section*{Acknowledgments}
The authors are grateful to Samson Lotven and to three reviewers for their constructive and critical feedback on this paper. We also thank audience members at ACAL 53 and members of the International Research Network's Somali Working Group for their comments and suggestions on aspects of our data and analysis. Special thanks go to Habiba Noor, the Syracuse Northside Learning Center, and to the Syracuse University Department of Languages, Literatures \& Linguistics for their financial support of this research. 

\printbibliography[heading=subbibliography,notkeyword=this]
\end{document}
