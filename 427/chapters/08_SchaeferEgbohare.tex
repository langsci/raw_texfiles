\documentclass[output=paper,colorlinks,citecolor=brown]{langscibook}
\ChapterDOI{10.5281/zenodo.11091831}

\author{Ronald Schaefer\affiliation{Southern Illinois University Edwardsville} and Francis Egbokhare\affiliation{University of Ibadan}}
\title{Argument linkage for Niger-Congo `give’} 
\abstract{Relative to the Niger-Congo languages of West Africa, we survey ‘give’ predications, i.e., those involving physical change of possession of a theme object to a recipient. Linkage of theme (T) and recipient (R) arguments to a predication is of three predication types: ditransitive verb, verb-verb, and verb-oblique. Across West Africa, linkage types couple. They reflect two areal zones: the Bandama-Tano-Volta watershed and the Niger River delta. Both show ditransitive linkage (V\textsubscript{1} NP\textsubscript{R} NP\textsubscript{T}) and verb-verb linkage (V\textsubscript{2} NP\textsubscript{T} V\textsubscript{1} NP\textsubscript{R}). Outside these zones, single linkage types predominate, including verb-oblique (V\textsubscript{2} NP\textsubscript{T} OBL NP\textsubscript{R}). Where available, we assess correlations between predication form (linear order of arguments, linkage type) and predication function (possession change vs. transfer without possession change). In ditransitive only languages, adjacency of R to ‘give’ tends to convey possession change, while transfer emerges with T adjacency. For languages that couple ditransitive and verb-verb, verb-verb conveys transfer while ditransitive expresses possession change.}

\IfFileExists{../localcommands.tex}{
   \addbibresource{../localbibliography.bib}
   % add all extra packages you need to load to this file

\usepackage{tabularx,multicol}
\usepackage{url}
\urlstyle{same}

\usepackage{listings}
\lstset{basicstyle=\ttfamily,tabsize=2,breaklines=true}

\usepackage{langsci-basic}
\usepackage{langsci-optional}
\usepackage{langsci-lgr}
\usepackage{langsci-osl}
% \usepackage{./langsci/styles/langsci-lgr}
% \usepackage{./langsci/styles/langsci-osl}
% \usepackage{langsci-gb4e}

\usepackage{tikz}
\usetikzlibrary{patterns,calc}
\pgfdeclarepatternformonly{south east lines}{\pgfqpoint{-0pt}{-0pt}}{\pgfqpoint{3pt}{3pt}}{\pgfqpoint{3pt}{3pt}}{
    \pgfsetlinewidth{0.6pt}
    \pgfpathmoveto{\pgfqpoint{0pt}{3pt}}
    \pgfpathlineto{\pgfqpoint{3pt}{0pt}}
    \pgfpathmoveto{\pgfqpoint{.2pt}{-.2pt}}
    \pgfpathlineto{\pgfqpoint{-.2pt}{.2pt}}
    \pgfpathmoveto{\pgfqpoint{3.2pt}{2.8pt}}
    \pgfpathlineto{\pgfqpoint{2.8pt}{3.2pt}}
    \pgfusepath{stroke}}
    
\usepackage{stmaryrd}
\usepackage{wasysym}
\usepackage{multirow}
\usepackage{caption}
\usepackage{subcaption}
\usepackage{mathrsfs}
\usepackage{qtree}

\usepackage{linguex}


   %pminos do not split footnotes
% \interfootnotelinepenalty=10000 %Footnote in Laporte chapters has to be split SN


%\DeclareIndexNameFormat{default}{%
%\nameparts{#1}%
%\usebibmacro{index:name}%
%{\index[names]}%
%{\namepartfamily}%
%{\namepartgiveni}%
% {}% L1
% {}% L2
%{\namepartprefix}% generates spurious space L3
%{\namepartsuffix}% generates spurious space L4
%}

%  {\DeclareIndexNameFormat{default}{%
%     \usebibmacro{index:name}{\index[names]}{#1}{#3}{#5}{#7}}}

%\DeclareIndexNameFormat{default}{%
%  \usebibmacro{index:name}{\sindex[nom]}{#1}{#3}{#5}{#7}}

%\DeclareIndexNameFormat{default}{%
%  \usebibmacro{index:name}{\sindex[person]}{#1}{#3}{#5}{#7}}
%\DeclareIndexNameFormat{default}{%
%\nameparts{#1} \usebibmacro{index:name}{\sindex[person]]}{\namepartfamily}{‌​\namepartgiven}{\nam‌​epartprefix}{\namepa‌​rtsuffix}}

%\newcommand{\smiley}{:)}

%\renewbibmacro*{index:name}[5]{%
%\usebibmacro{index:entry}{#1}%
%{\iffieldundef{usera}{}{\thefield{usera}\actualoperator}\mkbibindexname{#2}{#3}{#4}{#5}}}

% \newcommand{\noop}[1]{}

%remove for final
%\overfullrule=1mm

\newcommand{\tobi}[2]}}
\renewcommand{\S}[1]{\tobi{#1}{\textsc{*}}}

% this volume references
% puts: [this volume]
% already defined: \citetv
%\newcommand{\citepv}[1]{(\citeauthor{#1} \citeyear*{#1} [this volume])}
\newcommand{\citealtv}[1]{\citeauthor{#1} \citeyear*{#1} [this volume]}

%parentheses around example number
\newcommand{\pref}[1]{(\ref{#1})}

% in-text examples

\newcommand{\lnex}[1]{\textit{#1}} %target lang word
\newcommand{\lnlit}[1]{(lit.: `#1')} %literal reading
\newcommand{\lnlat}[1]{(#1)} % latinization
\newcommand{\lntrans}[1]{`#1'} %translation
\newcommand{\lnexl}[2]%
{\lnex{#1}{} \lnlat{#2}} % ex with latinization
\newcommand{\lnexlat}[3]{\lnex{#1}{} \lnlat{#2}{} \lntrans{#3}} % ex with latinization and tranl.

%ch01
\newcommand{\co}[1]{\mbox{\textbf{#1}}}

%ch09

\newcommand{\cyrbulg}[1]{\begin{otherlanguage*}{bulgarian}#1\end{otherlanguage*}}


%ch10
\newcommand{\nlp}{{\small NLP}}
\newcommand{\mwe}{{\small MWE}}
\newcommand{\rae}{{\small RAE}}
\newcommand{\lvc}{{\small LVC}}
\newcommand{\pos}{{\small P}o{\small S}}
%\newcommand{\todo}[1]{ \textcolor{red}{#1} }

%\renewcommand{\labelenumi}{\theenumi}
%\ainamefmt{{vv}{ll}{, ff}{, jj}} % fullname

\newcommand{\biberror}[1]{{\color{red}#1}}

\newcommand{\osenovaitem}{--~}
   %% hyphenation points for line breaks
%% Normally, automatic hyphenation in LaTeX is very good
%% If a word is mis-hyphenated, add it to this file
%%
%% add information to TeX file before \begin{document} with:
%% %% hyphenation points for line breaks
%% Normally, automatic hyphenation in LaTeX is very good
%% If a word is mis-hyphenated, add it to this file
%%
%% add information to TeX file before \begin{document} with:
%% %% hyphenation points for line breaks
%% Normally, automatic hyphenation in LaTeX is very good
%% If a word is mis-hyphenated, add it to this file
%%
%% add information to TeX file before \begin{document} with:
%% \include{localhyphenation}
\hyphenation{
    Beck-man
    Ngu-yen
    back-chan-nel
    back-chan-nels
    mo-not-o-nous
    ste-reo-typ-i-cal
}

\hyphenation{
    Beck-man
    Ngu-yen
    back-chan-nel
    back-chan-nels
    mo-not-o-nous
    ste-reo-typ-i-cal
}

\hyphenation{
    Beck-man
    Ngu-yen
    back-chan-nel
    back-chan-nels
    mo-not-o-nous
    ste-reo-typ-i-cal
}

   \boolfalse{bookcompile}
   \togglepaper[2]%%chapternumber
}{}

\begin{document}
\maketitle

\section{Introduction}

Recent studies of argument structure have investigated ‘give’ verbs from typological and crosslinguistic perspectives (\citealt{Bouveret2021, Malchukovetal, Newman1996, Newman1998}). An explicitly areal dimension to ‘give’ studies was added by Comrie's \citeyearpar{Comrie2012} assessment of ‘give’ constructions across Europe and North-Central Asia. 

For this paper, we undertake an areal survey of ‘give’ among 14 Niger-Congo language groups. Relevant constructions, which may consist of one- or two-element predications, convey physical transfer in which an agent causes a theme to become possessed by an animate recipient. In other words, someone transfers something to someone else, and thereby loses possession of it. 

Relative to West Africa, we assembled a convenience sample of ‘give’ constructions from the Niger-Congo language groups in Table \ref{tab:Groups}. Countries in which the surveyed languages from these groups are spoken are provided for reference.

\begin{table}
\caption{Language groups in this survey and countries where they are spoken}
\label{tab:Groups}
 \begin{tabular}{ll}
  \lsptoprule
 Language group & Countries \\
 \midrule
 Atlantic & Senegal \\
 Delta Cross & Nigeria \\
 Dogon & Burkina Faso, Mali \\
 Edoid & Nigeria \\
 Gur & Burkina Faso, Ghana \\
 Igboid & Nigeria \\
 Ijoid & Nigeria \\
 Jukunoid & Nigeria \\
 Kru & Ivory Coast \\
 Kwa & Benin, Ghana, Ivory Coast, Togo \\
 Mande & Burkina Faso, Gambia, Guinea, \\
 & Guinea-Bissau, Mali, Senegal \\
 Nupoid & Nigeria \\
 Senufo & Mali\\
 Yoruboid & Nigeria \\
  \lspbottomrule
 \end{tabular}
\end{table}

\begin{sloppypar}
Our survey encompasses languages from near the headwaters of the Niger River in the Guinea Highlands to the tributaries of Nigeria’s Niger Delta around Port Harcourt. Across these languages, we scrutinize the linkage relation of theme and recipient arguments to their predication.
\end{sloppypar}

\section{Ditransivity and its nature}

Regarding ditransitive predications and their cross-linguistic realization, \citet[148]{Whaley1997} has footnoted, “It is an intriguing fact that the verb form meaning ‘give’ is commonly employed in serial constructions to mark benefactive”. A similar statement could be advanced about the marking of ‘give’ recipient. Indeed, \citet[149--154]{HeineKuteva2002} and  \citet{Kutevaetal2019} find that in many genetically unrelated languages the marker for a benefactive or recipient argument has \is{grammaticalization} grammaticalized from a verb meaning ‘give’.

In these general statements, there are two dimensions that should not be lost. What Whaley alludes to and Heine and Kuteva recognize are two dimensions of grammatical analysis that apply to ‘give’ verbs and their framing of theme and recipient. One concerns a contrast in the \isi{linear ordering} of theme and recipient arguments, as illustrated in (\ref{EX:ThRecip1}) with \textsubscript{R} signaling recipient and \textsubscript{T} theme.

\ea \label{EX:ThRecip1}
\ea NP\textsubscript{R} NP\textsubscript{T}
\ex NP\textsubscript{T} NP\textsubscript{R}
\z
\z

%\begin{tabularx}{.50\textwidth}{llll}
% &	NP\textsubscript{R} NP\textsubscript{T}&  & NP\textsubscript{T} NP\textsubscript{R} \\ \\
%\end{tabularx}

A second dimension has to do with how these arguments relate to their predication. In West Africa, three patterns are pertinent. In one, a ‘give’ verb, as a ditransitive predicate, links to two arguments: recipient and theme. In a second pattern, a ‘give’ verb links to only a single argument. Another distinct verb, ‘take’ for example, links to the remaining argument. In a third pattern, a non-`give' verb links to one argument, while the second argument is linked to a non-verbal oblique marker, such as an adposition.

\begin{sloppypar}
Undergirding these predication types is the potentially \is{variation} variable nature of theme and recipient linkage. There are three linkage types. Ditransitive linkage relies on a single verb. Verb-verb linkage associates each argument to a distinct element of a predication, both of which are verbs. And verb-oblique linkage fixes each argument to a distinct predicate element, one of which is a verb and the other is a non-verb, grammatical form. In our survey, one or more of these linkage types, shown in (\ref{EX:ThRecip2}), identifies argument-predicate relations for ‘give’ events. In each, V\textsubscript{1} represents a ‘give’ verb, and V\textsubscript{2} is a verb with a meaning distinct from ‘give’ (either ‘take’ or some other transitive verb). OBL is a non-verbal grammatical form.
\end{sloppypar}

\ea \label{EX:ThRecip2}
\ea V\textsubscript{1} NP\textsubscript{R} NP\textsubscript{T}
\ex V\textsubscript{2} NP\textsubscript{T} V\textsubscript{1} NP\textsubscript{R}
\ex V\textsubscript{2} NP\textsubscript{T} OBL NP\textsubscript{R} 
\z
\z


%\begin{tabularx}{.50\textwidth}{llllll}
%& V\textsubscript{1} NP\textsubscript{R} NP\textsubscript{T}	&	&	V\textsubscript{2} NP\textsubscript{T} V\textsubscript{1} NP\textsubscript{R}	&	&	V\textsubscript{2} NP\textsubscript{T} OBL NP\textsubscript{R} \\
%\end{tabularx}

\section{Language samples and analysis}

Comparative analyses of ‘give’ events coding recipient and theme arguments among West Africa’s Niger-Congo languages are limited at best (\cite{Ameka2013}). Due to the very preliminary nature of this investigation and our modest sample of languages, we do not claim our sample to be more than it is. We do, however, assume that the data from each language is representative of its group and is suggestive of essential linkage types for ‘give’ events in West Africa. We hold to this assumption even though coverage provided ‘give’ clauses in the grammatical literature we consulted is uneven. Across statements, constraints affecting ‘give’ clauses are inconsistently discussed. Nonetheless, we highlight these constraints whenever they occur in order to provide a flavor of the range of issues that interlace with our principal concern: argument linkage relations and their familial and areal \is{areality} distribution. With this as background, we turn to ‘give’ clauses among the fourteen language groups in our sample.

\subsection{Kwa}

The \ili{Kwa languages} we assess exist along a coastal axis from Baule in the west to Tafi and Fongbe in the east. They are spoken in a block of nation states that include Ivory Coast, Ghana, Togo, and the Republic of Benin. Relative to ‘give’ and its theme and recipient arguments, all exhibit ditransitive linkage and verb-verb linkage. None show verb-oblique linkage. A more general discussion of serial verb types in Kwa is found in \citet{Shluinsky2017}.

\ili{Baule}, spoken in southeastern Ivory Coast, evinces the verb \textit{man} ‘give’ \citep{CreisselsKouadio2010}. Although additional comments by Creissels and Kouadio suggest that \textit{man} may be limited to gift giving and similar rituals, thus restricting theme argument character, we leave for another time the distinction between ‘give something for someone to possess’ and ‘present/offer a gift to someone’. Across West African languages, information bearing on this distinction is simply not available. \ili{Baule} \textit{man} occurs as a simple predicate and as the second element of a complex predicate. As a simple ditransitive, \textit{man} takes a recipient argument and a theme argument (\ref{ex:Kwa1}a). Argument order is restricted to NP\textsubscript{R} NP\textsubscript{T}. In addition, \textit{man} occurs in a verb-verb predicate, which similarly restricts argument order. Within a single clause, \textit{man} and its recipient argument follow the verb \textit{fa} ‘take’ with a theme argument (\ref{ex:Kwa1}b). These examples strongly suggest that Baule exhibits the ditransitive pattern V\textsubscript{1} NP\textsubscript{R} NP\textsubscript{T} and the verb-verb pattern V\textsubscript{2} NP\textsubscript{T} V\textsubscript{1} NP\textsubscript{R}.\footnote{Here and elsewhere, morphological glosses and tone marking are provided as in the cited source. High tone is marked with an acute diacritic, Mid tone with a macron, and Low tone with a grave diacritic, as is conventional. We have taken the liberty to indicate downstep with a superscript ꜝ. In some traditions, it is conventional to indicate \textit{open} or \textit{lax} vowels with an underline, but we have chosen to convert these to the appropriate IPA vowel symbol. The reader is encouraged to consult the source for more detail.}

\ea \label{ex:Kwa1} Baule (adapted from \citealt[11]{CreisselsKouadio2010})
\begin{xlist}
\ex 
 \gll Kuàkú				màn-nìn		mín		sìkàá.\\
	Kouakou	give-\textsc{pfv}		1\textsc{sg}		money\\
\glt  ‘Kouakou gave me money.’ 
\ex
\gll Kuàkú			fà-lì 				sìkàá			màn-nìn		mín.\\
	Kouakou	take-\textsc{pfv}		money		give-\textsc{pfv}		1\textsc{sg}	\\
\glt  ‘Kouakou gave me money.’ 
\end{xlist}
\z

We note here a point regarding argument occurrence which is also applicable to other languages in our sample. Baule \textit{man} can maintain its ditransitive character when followed by a single noun phrase that expresses a theme argument. Its recipient argument can be understood from context. The positioning of only a recipient after \textit{man} is not open to such contextualization; recipient-only structures with verb \textit{man} are ungrammatical (\ref{ex:Kwa2}b).

\ea \label{ex:Kwa2} Baule (adapted from \citealt[18]{CreisselsKouadio2010})
\begin{xlist}
\ex[]{
 \gll Kuàkú			màn-nìn		sìkàá.\\
	Kouakou	give-\textsc{pfv}		money\\
\glt  ‘Kouakou gave money (to someone).’}
\ex[*]{
\gll Kuàkú				màn-nìn		kòfí.\\
	Kouakou	give-\textsc{pfv}		kofi	\\
\glt ‘Kouakou gave Kofi (something).’ [intended]}
\end{xlist}
\z

\ili{Akan}, spoken over much of southern Ghana, has two verbs, \textit{ma} and \textit{kye} with English ‘give’ as translation (\cite[23]{Osam2004}). Of these, \textit{kye} is associated with ritual activity (Reginald Duah, personal communication). \textit{Ma} occurs as a simple predicate and as one element of a complex predicate. As a ditransitive predicate, \textit{ma} precedes arguments for recipient and theme (\ref{ex:Kwa3}a). Argument order \is{linear ordering} is restricted to NP\textsubscript{R} NP\textsubscript{T}. In a verb-verb predication, \textit{ma} follows verb \textit{de} ‘take’. \textit{De} takes a theme argument and \textit{ma}, a recipient (\ref{ex:Kwa3}b); argument order is restricted. Based on \textit{ma} behavior, Akan exhibits the linkage types ditransitive V\textsubscript{1} NP\textsubscript{R} NP\textsubscript{T} and verb-verb V\textsubscript{2} NP\textsubscript{T} V\textsubscript{1} NP\textsubscript{R}.
\newpage

\ea\label{ex:Kwa3}Akan (Reginald Duah, personal communication)
\begin{xlist}
\ex
\gll Kofi		ma-a					maame		no			oguan.\\
					Kofi	give-\textsc{comp}	woman		the		sheep\\
\glt				‘Kofi gave the woman a sheep / a sheep to the woman.’
\ex
\gll 	Kofi		de		oguan	no		ma-a						maame			no.\\
					Kofi	take sheep		the	give-\textsc{comp}		woman			the\\
\glt					‘Kofi gave the sheep to the woman / the woman the sheep.’
\end{xlist}
\z

The verb \textit{kye} in Akan exhibits linkage types similar to \textit{ma}. As a ditransitive predicate, \textit{kye} takes recipient and theme in that order (\ref{ex:Kwa4}a). \textit{Kye} also shows a verb-verb predication with the verb \textit{de} ‘take’ (\ref{ex:Kwa4}b). \textit{Kye} and its recipient argument follow \textit{de} and its theme argument. Distinct orders for theme and recipient characterize \textit{kye} linkage types: ditransitive V\textsubscript{1} NP\textsubscript{R} NP\textsubscript{T} and verb-verb V\textsubscript{2} NP\textsubscript{T} V\textsubscript{1} NP\textsubscript{R}.

\ea \label{ex:Kwa4} Akan (\citealt[23]{Osam2004})
\begin{xlist}
\ex[]{
\gll Abena		kye-e						abofra		no		sika. \\
	Abena		give-\textsc{comp}		child			the	money \\
\glt				‘Abena gave the child money / gave money to the child.’}
\ex[]{
\gll	Abena		de			sika			no		kye-e					abofra		no.\\
		Abena		take		money	the	give-\textsc{comp}	child			the \\
\glt	‘Abena gave the child the money / the money to the child.’}
\ex[*]{
\gll Abena		kye-e					abofra		no		sika				no.\\
			Abena	give-\textsc{comp}	child			the	money		the \\
\glt		‘Abena gave the child the money / the money to the child.’ [intended]}
\end{xlist}
\z

Various constraints affect the realization of \textit{ma} and \textit{kye} arguments, as is the case for some other languages in our sample. Nevertheless, none of these constraints undermine our basic contention that Akan ‘give’ verbs manifest ditransitive and verb-verb linkage. Among constraints on ditransitive linkage is a \isi{definiteness prohibition} on the second argument, the theme. It rejects both the definite article \textit{no} (\ref{ex:Kwa5}a) and third person definite pronoun \textit{no} (\ref{ex:Kwa5}b). No similar prohibition exists for verb-verb linkage.\largerpage

\ea \label{ex:Kwa5} Akan (Reginald Duah, personal communication)
\begin{xlist}
\ex[*]{
\gll Kofi		ma-a					maame		no			oguan		no.\\
			Kofi	give-\textsc{comp}	woman		the		sheep			the\\
\glt						‘Kofi gave the woman the sheep  / the  sheep to the woman.’ [intended]}
\ex[*]{
\gll Kofi			ma-a					maame		no.\\
						Kofi		give-\textsc{comp}	woman		it \\
\glt					‘Kofi gave it to the woman / the woman it.’ [intended]}
\end{xlist}
\z

Another Kwa language of southern Ghana is \ili{Ga}. Its ‘give’ equivalent is the verb \textit{han} (\cite{Dakubu2003, Dakubu2004, Dakubu2009}). \textit{Han} occurs as a simple ditransitive predicate and as one element of a complex predicate. Ditransitive \textit{han} appears with recipient and theme arguments (\ref{ex:Kwa6}a). Argument order is limited to NP\textsubscript{R} NP\textsubscript{T}, and similar to Akan, ditransitive \textit{han} prohibits theme definiteness (\ref{ex:Kwa6}b). In a verb-verb predication, \textit{han} and its recipient argument follow the verb \textit{ke} ‘move’ and its theme argument (\ref{ex:Kwa6}c). No alternative order for predication arguments is acceptable. Linkage patterns for Ga are ditransitive V\textsubscript{1} NP\textsubscript{R} NP\textsubscript{T} and verb-verb V\textsubscript{2} NP\textsubscript{T} V\textsubscript{1} NP\textsubscript{R}.

\ea \label{ex:Kwa6} Ga (\citealt[116-117]{Dakubu2004})
\begin{xlist}
\ex[]{
\gll Oto		hán		è-bí				ꜝe		tsòbí. \\
					Oto		give		3\textsc{s}-child	\textsc{def}	doll \\
\glt				‘Oto gave his child a doll.’}
\ex[*]{
\gll			Oto			hán		è-bí				ꜝe		tsòbí	!e.\\
						Oto		give		3\textsc{s}-child	\textsc{def}	doll		\textsc{def} \\
\glt						‘Oto gave his child the doll.’ [intended]}
\ex[]{ 
\gll	Oto		ke				tsobí		!e		han 		è-bi				!e.\\
					Oto		move		doll			\textsc{def}	give		3\textsc{s}-child	\textsc{def} \\
\glt ‘Oto gave the doll to his child / gave his child the doll.’}
\end{xlist}
\z

A Kwa language in our sample spoken in southeastern Ghana and southwestern Togo is \ili{Ewe}. It employs the ‘give’ verb \textit{na} (\citealt{Dzameshie2004}), apparently associated with ritual giving. As a predicate, \textit{na} appears in ditransitive and verb-verb linkages. Ewe \textit{na}, as a ditransitive predicate, permits a recipient and a theme argument, although not necessarily in that order. Either NP\textsubscript{R} NP\textsubscript{T} or NP\textsubscript{T} NP\textsubscript{R} argument order is grammatically sanctioned (\ref{ex:Kwa7}a--b). In addition, Ewe \textit{na} occurs in a verb-verb predication, where the verb \textit{tso} ‘take’ and its theme argument precede \textit{na} and its recipient argument (\ref{ex:Kwa7}c). Ewe exhibits verb-verb V\textsubscript{2} NP\textsubscript{T} V\textsubscript{1} NP\textsubscript{R}, where argument order is restricted and ditransitive V\textsubscript{1} NP\textsubscript{R} NP\textsubscript{T} or V\textsubscript{1} NP\textsubscript{T} NP\textsubscript{R}, where argument order is unrestricted.\largerpage

\ea \label{ex:Kwa7} Ewe (\citealt[143, 145, 156]{Essegbey1999})
\begin{xlist}
\ex
\gll Kofí		ná			ga				amí.\\
					Kofi	give		money	Ami\\
\glt		‘Kofi gave money to Ami.’
\ex
\gll 	Kofí		ná			amí		ga.\\
					Kofi	give		Ami		money	\\
\glt					‘Kofi gave Ami money.’
\ex 
\gll 	Kofí		tsó			ga				lá		ná			nyónuví	áɖé.\\
					Kofi	take		money	\textsc{def}	give		girl				\textsc{speci}\\
\glt					‘Kofi gave the money to a certain girl.’
\end{xlist}
\z

The distinct orders of recipient and theme under ditransitive linkage in \ili{Ewe} are not semantically equivalent, as shown by an appended clause explicitly rejecting possession, (\ref{ex:Kwa8}a--b), from \citet{Essegbey1999}. Positioning the recipient adjacent to the verb conveys a change of \isi{possession}, i.e., the recipient accepts the transferred object and takes possession of it. An appended clause rejecting possession renders the multi-clause structure only marginally acceptable (\ref{ex:Kwa8}a). On the other hand, a non-adjacent recipient, i.e., positioned after the theme, articulates transfer without change of possession. The non-adjacent argument can both be recipient and yet not take possession of the transferred entity. An appended clause rejecting possession of theme is thus acceptable (\ref{ex:Kwa8}b).

\ea \label{ex:Kwa8} Ewe (\citealt[164]{Essegbey1999})
\begin{xlist}
\ex[?]{
\gll Kofí		ná			amí		ga				gaké	mé-xɔ-e		o.\\
	Kofi	give		Ami		money	but		\textsc{neg}-receive-3\textsc{sg}	\textsc{neg}\\
\glt						‘Kofi gave Ami money but she did not take it.’ [marginal]}
\ex[]{
\gll 	Kofí		ná			ga 			amí		gaké	mé-xɔ-e		o.\\
					Kofi	give		money	Ami		but		\textsc{neg}-receive-3\textsc{sg}	\textsc{neg}\\
\glt					‘Kofi gave money to Ami but she did not take it.’}
\end{xlist}
\z

Another Kwa language in our sample is \ili{Fongbe}. It is spoken in southern parts of Togo and the Republic of Benin. Fongbe exhibits the ‘give’ verb \textit{na} (\citealt{Lefebvre_Brousseau2002}). As with Ewe, the behavior of this verb allows one to tease apart simple transfer from possession or ownership change.\largerpage[-2]

Fongbe \textit{na} occurs as a ditransitive predicate and as an element in a verb-verb predicate. As a ditransitive predicate, \textit{na} permits recipient followed by theme as well as theme followed by recipient. Either NP\textsubscript{R} NP\textsubscript{T} or NP\textsubscript{T} NP\textsubscript{R} argument order is grammatically sanctioned (\ref{ex:Kwa9}a--b). In addition, Fongbe \textit{na} occurs in a predication with verb-verb linkage. \textit{Na} and its recipient argument follow the verb \textit{tso} ‘take’ and its theme (\ref{ex:Kwa9}c).

\ea \label{ex:Kwa9} Fongbe (\citealt[447-448]{Lefebvre_Brousseau2002})
\begin{xlist}
\ex
\gll Kɔ̀kú		ná			àsíbá			àsɔ́n.\\
					Koku		give		Asiba			crab\\
\glt					‘Koku gave Asiba crab.’
\ex
\gll 	Kɔ̀kú		ná			àsɔ́n	àsíbá.\\
					Koku		give		crab		Asiba\\
\glt					‘Koku gave Asiba crab.’
\ex
\gll	Kɔ̀kú 		tsɔ́			àsɔ́n	ɔ́			ná 		àsíbá.\\
					Koku		take		crab		the	give Asiba\\
\glt					‘Koku gave the crab to Asiba.’
\end{xlist}
\z

In \ili{Fongbe}, ditransitive predications with \textit{na} and one or another of its verb argument orders correlate with distinct semantic interpretations, too. These denote transfer of object as opposed to \isi{possession} change of object (\citealt{Lefebvre_Brousseau2002}). Ditransitive linkage manifests equivalent grammatical standing relative to an inference of inchoative possession change. Both (\ref{ex:Kwa10}a) and (\ref{ex:Kwa11}a) allow, respectively, (\ref{ex:Kwa10}b) and (\ref{ex:Kwa11}b), statements of inference holding that the recipient has come into “possession” of the theme.\largerpage[-2]

\ea \label{ex:Kwa10} Fongbe (\citealt[447]{Lefebvre_Brousseau2002})
\begin{xlist}
\ex
\gll Kɔ̀kú		ná			àsíbá		àsɔ́n.\\
	 Koku		give		Asiba		crab\\
\glt ‘Koku gave Asiba crab.’
\ex
\gll	Àsɔ́n	húzú		/		nyí		àsíbá		tɔ̀n.\\
	crab		become	/	be			Asiba		\textsc{gen} \\
\glt ‘The crab has become / is Asiba’s.’
\end{xlist}
\ex \label{ex:Kwa11} Fongbe (\citealt[447]{Lefebvre_Brousseau2002})
\begin{xlist}
\ex
\gll Kɔ̀kú		ná			àsɔ́n	àsíbá.\\
	 Koku		give		crab		Asiba\\
\glt ‘Koku gave Asiba crab.’
\ex
\gll 	Àsɔ́n	húzú		/		nyí		àsíbá		tɔ̀n.\\
		crab		become	/	be			Asiba	\textsc{gen}\\
\glt    ‘The crab has become / is Asiba’s.’
\end{xlist}
\z

However, when one adjoins a ditransitive or a verb-verb linkage structure to a grammatically appended ‘but’ clause, that negates possession change; grammaticality is not symmetrical. Ditransitive linkage, regardless of argument order, is unacceptable with an explicit rejection of possession change in a ‘but’ clause where the theme is possessum and recipient is possessor (\ref{ex:Kwa12}a--b). On the other hand, a verb-verb linkage joined to the same explicit rejection of possession change clause is grammatical. Verb-verb linkage therefore appears compatible with transfer of theme but not possession change of theme.

\ea \label{ex:Kwa12} Fongbe (\citealt[473]{Lefebvre_Brousseau2002})
\begin{xlist}
\ex[*]{
\gll Kɔ̀kú			ná			àsíbá		àsɔ́n	vɔ́		àsɔ́n	nyí	àsíbá		tɔ̀n		àá.\\
	Koku		give		Asiba		crab		but	crab		be		Asiba		\textsc{gen}		\textsc{neg} \\
\glt ‘Koku gave Asiba crab, but the crab is not hers.’ [intended]}
\ex[*]{ 
\gll Kɔ̀kú			ná			àsɔ́n	àsíbá		vɔ́		àsɔ́n	nyí	àsíbá		tɔ̀n		àá.\\
	Koku		give		crab		Asiba		but	crab		be		Asiba		\textsc{gen}	\textsc{neg}\\
\glt			‘Koku gave Asiba crab, but the crab is not hers.’ [intended]}
\ex[]{
\gll 	Kɔ̀kú		tsɔ́			àsɔ́n	ɔ́			ná			Àsíbá vɔ́		àsɔ́n	ɔ́			nyí	àsíbá		tɔ̀n		àá.\\
	Koku		take		crab		the	give		Asiba	
	but	crab		the	be		Asiba		\textsc{gen}	\textsc{neg}\\
\glt ‘Koku gave the crab to Asiba, but the crab is not Asiba’s.’}
\end{xlist}
\z

A final group of Kwa languages in our sample is spoken in the mountainous terrain of southeastern Ghana and southwestern Togo. The ``Togo Mountain Languages'' \il{Togo Mountain Languages} in our sample include Logba, Tafi, and Avatime. Their ‘give’ verbs exhibit behavior consistent with other Kwa languages, although these languages uniformly restrict argument order in ditransitive predications.

In \ili{Avatime}, the verb \textit{ki} ‘give’ participates in two predication types \citep{Defina2016}. In ditransitive linkage, \textit{ki} takes recipient and theme arguments whose order is restricted to NP\textsubscript{R} NP\textsubscript{T} (\ref{ex:Kwa13}a). In a complex predicate, \textit{ki} and its recipient argument follow \textit{ko} ‘take’ and its theme argument. The argument order is NP\textsubscript{T} NP\textsubscript{R} (\ref{ex:Kwa13}b--c). For Avatime, ‘give’ linkage patterns restrict argument order asymmetrically. For ditransitive it is V\textsubscript{1} NP\textsubscript{R} NP\textsubscript{T}, and for verb-verb it is V\textsubscript{2} NP\textsubscript{T} V\textsubscript{1} NP\textsubscript{R}.\largerpage

\ea \label{ex:Kwa13} Avatime (\citealt[42-42, 57]{Defina2016})
\begin{xlist}
\ex
\gll A-kɪ=yɛ	lɪ-ba=lɛ.\\
C\textsubscript{1}\textsc{s}.\textsc{pfv}-give=C\textsubscript{1}\textsc{s.obj}	C\textsubscript{3}\textsc{s}-hoe=\textsc{def}\\
\glt 	‘He gave him the hoe.’
\ex
\gll A-kɔ̀	lɪ-ba=lɛ		kɪ́=yɛ.\\
C\textsubscript{1}\textsc{s.pfv}-take		C\textsubscript{3}\textsc{s}-hoe=\textsc{def}		give=C\textsubscript{1}\textsc{s.obj}\\
\glt			‘He gave him the hoe.’
\ex	
\gll A-kɔ̀	kʊ̀-sà=a					kɪ́ ó-dzɛ=ɛ.\\
	C\textsubscript{1}\textsc{s.pfv}-take	C\textsubscript{5}\textsc{s}-cloth=\textsc{def}		give		C\textsubscript{1}\textsc{s}-woman=\textsc{def}\\
\glt 	‘He gave the cloth to the woman.’
\end{xlist}
\z

Other Togo Mountain Languages such as \ili{Logba} (\cite{Dorvlo2008}) and \ili{Tafi} (\cite{Bobuafor2013}) exhibit similar predications. In Logba, the verb \textit{ta} ‘give’ occurs as a ditransitive predicate and as one element of a verb-verb predicate. As a ditransitive, \textit{ta} takes a recipient argument and a theme argument (\ref{ex:Kwa14}a), in that order. \textit{ta} also follows the verb \textit{mi} ‘take’ in another predication type: \textit{Ta} and its recipient argument follow \textit{mi} and its theme argument (\ref{ex:Kwa14}b). Linkage types in \ili{Logba} are ditransitive V\textsubscript{1} NP\textsubscript{R} NP\textsubscript{T} and verb-verb V\textsubscript{2} NP\textsubscript{T} V\textsubscript{1} NP\textsubscript{R}. Both predications restrict argument order,   although the results are dissimilar.

\ea \label{ex:Kwa14} Logba (\citealt[137]{Dorvlo2008}, personal communication)
\begin{xlist}
\ex
\gll Howusu		ó-tá							Asafo			e-feshi.\\
	Howusu		\textsc{sm.sg}-give		Asafo			\textsc{cm}-sheep\\
\glt ‘Howusu gave Asafo sheep.’
\ex	
\gll Howusu		o-mi	e-feshi=e				ta			Asafo.\\
	Howusu		\textsc{sm.sg}-take		\textsc{cm}-sheep=\textsc{def}		give		Asafo \\
\glt	‘Howusu gave the sheep to Asafo.’
\end{xlist}
\z

A composite set of predication patterns evident in each Kwa language of our sample is shown in Table \ref{tab:LinkageKwa}. Two linkage types stand out. Uniformly, a V\textsubscript{1} ‘give’ verb exhibits ditransitive linkage V\textsubscript{1} NP\textsubscript{R} NP\textsubscript{T} and verb-verb linkage V\textsubscript{2} NP\textsubscript{T} V\textsubscript{1} NP\textsubscript{R}. Argument order \is{linear ordering} across these patterns is not similar. Verb-verb is restricted to NP\textsubscript{T} NP\textsubscript{R} order for all sample languages, whereas ditransitive is restricted for some languages but not for others. Ewe and Fongbe, in particular, allow ditransitive linkage where argument order appears syntactically unrestricted, i.e., both NP\textsubscript{R} NP\textsubscript{T} and NP\textsubscript{T} NP\textsubscript{R} occur. The remaining languages, Akan, Baule, and Ga, as well as Logbo, Tafi, and Avatime rely on ditransitive NP\textsubscript{R} NP\textsubscript{T}, where argument order is restricted.

\begin{table}
\caption{Linkage types for ‘give’ in Kwa}
\label{tab:LinkageKwa}
 \begin{tabular}{lccc}
  \lsptoprule
 &  V\textsubscript{1} NP\textsubscript{R} NP\textsubscript{T}	&V\textsubscript{1} NP\textsubscript{T} NP\textsubscript{R} &	V\textsubscript{2} NP\textsubscript{T} V\textsubscript{1} NP\textsubscript{R}  \\
  \midrule
Baule &	+ &	--	&+\\
Akan&	+&--	&+\\
Ga	&+&	--&	+\\
Ewe&	+&	+&	+\\
Fongbe&	+&	+&	+\\
	&&&		\\
Logba&	+	&--&	+\\
Tafi	&+&	--&	+\\
Avatime&	+&	--&	+\\
  \lspbottomrule
 \end{tabular}
\end{table}

\subsection{Gur}

Two languages in our sample belong to the Gur group: Dagaare and Kasem are spoken in Northern Ghana and southern Burkina Faso. As with Kwa, each Gur language articulates two predication patterns that link recipient and theme arguments.

\ili{Dagaare} exhibits the ‘give’ verb \textit{ko}. It occurs as a simple predicate and as one element of a complex verb-verb predicate. As a ditransitive predicate, \textit{ko} takes a recipient and a theme argument (\ref{ex:Gur1}a), only in that order. In a non-ditransitive predication, \textit{ko} takes a recipient argument and follows the verb \textit{de} ‘take’ and its theme argument (\ref{ex:Gur1}b). Verb-verb argument order is exclusively NP\textsubscript{T} NP\textsubscript{R}. Dagaare linkage patterns are ditransitive V\textsubscript{1} NP\textsubscript{R} NP\textsubscript{T} and verb-verb V\textsubscript{2} NP\textsubscript{T} V\textsubscript{1} NP\textsubscript{R}. According to data available in \citet{Bodomo1997}, each linkage type restricts argument order differently.

\ea \label{ex:Gur1} Dagaare (\citealt[105]{Bodomo1997})
\begin{xlist}
\ex
\gll O			ko						la			dere		a			gane. \\
		he		give.\textsc{perf}	\textsc{fact}	Dere	\textsc{def}	book\\
\glt						‘He gave Dere the book.’
\ex
\gll O			de	la			a			gane	ko			dere.\\
he		take.\textsc{perf}	\textsc{fact}	\textsc{def}	book	give		Dere\\
\glt				‘He gave Dere the book.’
\end{xlist}
\z

\ili{Kasem}, another Gur language, displays similar linkage and argument order patterns in ditransitive and verb-verb predications. It provides a ‘give’ verb \textit{pV}, variously \is{variation} realized as \textit{pe} or \textit{pa} depending on tense/aspect selection (\citealt{Hewer1983, Nabaarese2013}, personal communication). \textit{pV} occurs as a simple ditransitive predicate taking arguments for recipient and theme (\ref{ex:Gur2}a) in the order NP\textsubscript{R} NP\textsubscript{T}. \textit{pV} also appears in a complex predicate with the verb \textit{kwe} ‘take’ and its theme argument. \textit{pV} and its recipient argument follow \textit{kwe} and its theme argument (\ref{ex:Gur2}b--c). Like Dagaare, Kasem manifests the linkage types ditransitive V\textsubscript{1} NP\textsubscript{R} NP\textsubscript{T} and verb-verb V\textsubscript{2} NP\textsubscript{T} V\textsubscript{1} NP\textsubscript{R}, each relying on a distinct argument order.

\ea \label{ex:Gur2} Kasem (Nabaarese, personal communication)
\begin{xlist}
\ex
\gll Kofi		pɛ									ama		pɛɛne.\\
	Kofi	\textsc{comp}.give.to		Ama	pen\\
\glt						‘Kofi gave Ama a pen.’
\ex
\gll	Kofi		kwè						amo		pɛɛne		ɔ				pa				ama.\\
	Kofi	\textsc{comp}.take	my		pen			3\textsc{sg}		give.to	Ama \\
\glt						‘Kofi gave my pen to Ama.’
\ex
\gll Kofi		wora		ɔ			kwe-a			amo		pɛɛne		ɔ			pa-e					ama.\\
Kofi	\textsc{cont}		3\textsc{sg}	take-\textsc{cont}	my	pen	3\textsc{sg}	give.to-\textsc{cont}	Ama\\
\glt						‘Kofi is giving Ama my pen.’
\end{xlist}
\z

We note in (\ref{ex:Gur2}b--c) that \ili{Kasem}, unlike Dagaare, requires explicit subject indexing across its verb-verb predicate. Third person singular \textit{ɔ} agrees in number and person with the subject phrase \textit{Kofi}, preceding \textit{kwe}. A final example illustrates that indexing in Kasem is not a function of tense/aspect conditions, as one might conjecture. (\ref{ex:Gur2}c), which displays continuous (\textsc{cont}) aspect and a third person subject, requires indexing on the second verb. Subject indexing is required on all instances of verb-verb linkage (\ref{ex:Gur2}b--c) in Kasem, regardless of tense-aspect. No indexing appears in ditransitive V\textsubscript{1} NP\textsubscript{R} NP\textsubscript{T} (\ref{ex:Gur2}a).

Table \ref{tab:LinkageGur} summarizes linkage types applicable to ‘give’ predications in \ili{Dagaare} and Kasem. Differentially restricting argument order across predications, each Gur language manifests ditransitive V\textsubscript{1} NP\textsubscript{R} NP\textsubscript{T} as well as verb-verb V\textsubscript{2} NP\textsubscript{T} V\textsubscript{1} NP\textsubscript{R}.

\begin{table}
\caption{Linkage types for ‘give’ in Gur}
\label{tab:LinkageGur}
 \begin{tabular}{lccc}
  \lsptoprule
 &  V\textsubscript{1} NP\textsubscript{R} NP\textsubscript{T}	&V\textsubscript{1} NP\textsubscript{T} NP\textsubscript{R} &	V\textsubscript{2} NP\textsubscript{T} V\textsubscript{1} NP\textsubscript{R}  \\
  \midrule
Dagaare &	+ &	--	&+\\
Kasem&	+&--	&+\\
  \lspbottomrule
 \end{tabular}
\end{table}

\subsection{Atlantic and Kru}

In our survey, there are languages from both the Atlantic and Kru groups. Jóola Banjal and Diola-Fogny from Atlantic are spoken in Senegal. Vata, an Eastern Kru dialect, also identified as Dida, is spoken in southwestern Ivory Coast, west of the Bandama River. Regardless of group, each language exhibits a single linkage type for recipient and theme.

\ili{Jóola Banjal} has the ‘give’ verb \textit{sen} (\cite{Bassène2007}). It occurs only in a ditransitive predication. \textit{Sen} takes recipient and theme arguments in a flexible linear order. Jóola Banjal ditransitive linkage is either V\textsubscript{1} NP\textsubscript{T} NP\textsubscript{R} (\ref{ex:Atl1}a) or V\textsubscript{1} NP\textsubscript{R} NP\textsubscript{T} (\ref{ex:Atl1}b). A similar constraint operated in the Kwa languages \ili{Ewe} and \ili{Fongbe}. Jóola Banjal, however, displays neither verb-verb linkage nor verb-oblique linkage.

\ea \label{ex:Atl1} Jóola Banjal (adapted from \citealt[130]{Bassène2007})
\begin{xlist}
\ex
\gll ∅-aare			axu			na-sen-e				fu-mangu			a-nynyil			axu.\\
		\textsc{cl}1-woman		\textsc{cl1.dem}		\textsc{s3s}-give-\textsc{tam}		\textsc{cl}7-mango		\textsc{cl}1-child		\textsc{cl1.dem4}\\
\glt					‘The woman gave a mango to the child.’
\ex
\gll 	∅-aare			axu			na-sen-e			a-nynyil			axu				fu-mangu.\\
	\textsc{cl}1-woman		\textsc{cl1.dem}		\textsc{s3s}-give-\textsc{tam}	\textsc{cl1}-child		\textsc{cl1.dem4}		\textsc{cl7}-mango\\
 \glt ‘The woman gave the child a mango.’
\end{xlist}
\z

\ili{Diola-Fogny} exhibits the ‘give’ verb \textit{sɛn} (\cite{Sapir1965}) . It occurs in a ditransitive predication that shows recipient and theme arguments. The verb \textit{sɛn} grammatically sanctions only the argument order NP\textsubscript{R} NP\textsubscript{T} (\ref{ex:Atl2}a--b), with no mention made of the order NP\textsubscript{T} NP\textsubscript{R}. Ditransitive linkage for Diola-Fogny is limited to V\textsubscript{1} NP\textsubscript{R} NP\textsubscript{T}. It fails to display verb-verb or verb-oblique linkage.

\ea \label{ex:Atl2} Diola-Fogny (\cite[30]{Sapir1965})
\begin{xlist}
\ex
\gll Ni-sɛn-ɔ				ebe.\\
						I-give-him		cow\\
\glt						‘I gave a cow to him.’
\ex
\gll Na-sɛn-ɔm-ɔ.\\
						he-give-me-him\\
\glt 						‘He gave him to me.’
\end{xlist}
\z

\ili{Vata}, a Kru language, employs the ‘give’ verb \textit{nyɛ} (\citealt{Koopman1984}). Its positioning relative to theme and recipient arguments is a function of the presence of a segmental auxiliary. When a future marker like \textit{ka} occurs in auxiliary position, \textit{nyɛ} follows its arguments (\ref{ex:Atl3}a--b). Basic constituents of the clause are ordered SOV. In perfective and imperfective aspect, where no segmental auxiliary occurs, \textit{nyɛ} precedes theme and recipient arguments (\ref{ex:Atl3}c). Constituent word order is then SVO.

\ea \label{ex:Atl3} Vata (adapted from \citealt[29, 157]{Koopman1984})
\begin{xlist}
\ex
\gll N		ká						yɔ-ɔ			saká	nyɛ.\\
						I		\textsc{fut.a}		child		rice		give\\
\glt						‘I will give rice to the child.’
\ex
\gll 	N		ká				saká	yɔ-ɔ 	nyɛ.\\
						I		\textsc{fut.a}	rice		child	give\\
\glt						‘I will give rice to the child.’
\ex
\gll 	À			nyɛ̄		à			nɔ̄					dàlà.\\
						we		give		our	mother		money\\
\glt						‘We gave money to our mother.’
\end{xlist}
\z

The verb \textit{nyɛ} orders recipient and theme arguments differently depending on the basic order of its clause. When a segmental auxiliary is present, argument order can be either NP\textsubscript{R} NP\textsubscript{T} or NP\textsubscript{T} NP\textsubscript{R} (\ref{ex:Atl3}a--b). A similar versatility of argument order operated in Atlantic \ili{Jóola Banjal} and in \il{Kwa languages} Kwa. When no overt auxiliary occurs in Vata, argument order appears limited to NP\textsubscript{R} NP\textsubscript{T} (\ref{ex:Atl3}c). Regardless of auxiliary occurrence, Vata manifests ditransitive linkage, which is either V\textsubscript{1} NP\textsubscript{R} NP\textsubscript{T} under SVO or the contrasting NP\textsubscript{R} NP\textsubscript{T} V\textsubscript{1} or NP\textsubscript{T} NP\textsubscript{R} V\textsubscript{1} under SOV. Vata fails to display the linkage types verb-verb or verb-oblique.

An inventory of linkage types for ‘give’ predications in Atlantic and Kru is presented in Table \ref{tab:LinkageAtl}. The only linkage type sanctioned is ditransitive. For \ili{Diola-Fogny}, ditransitive V\textsubscript{1} NP\textsubscript{R} NP\textsubscript{T} is restricted to order NP\textsubscript{R} NP\textsubscript{T}, as is Vata under perfective or imperfective \isi{aspect}. Jóola Banjal and Vata are not similarly constrained under non-perfective/imperfective aspects. Their ditransitive linkage permits either argument order NP\textsubscript{R} NP\textsubscript{T} or NP\textsubscript{T} NP\textsubscript{R}.

\vfill
\begin{table}[H]
\caption{Linkage types for ‘give’ in Atlantic and Kru}
\label{tab:LinkageAtl}
 \begin{tabular}{lcc}
  \lsptoprule
 &  V\textsubscript{1} NP\textsubscript{R} NP\textsubscript{T}	&V\textsubscript{1} NP\textsubscript{T} NP\textsubscript{R} 	\\
  \midrule
Jóola Banjal &	+ &	+\\
Diola-Fogny &	+&--	\\
&& \\
Vata &+&+ \\
  \lspbottomrule
 \end{tabular}
\end{table}
\vfill\pagebreak


\subsection{Dogon}

The \ili{Dogon languages} are spoken primarily in Mali, although some communities may also exist in Burkina Faso. As with Mande and Ijoid, the status of Dogon as part of Niger-Congo is controversial since it lacks typical Niger-Congo features such as SVO word order, noun class affixes, and verb suffixes of derivation such as causative (see \citealt{WilliamsonBlench2000}). For basic sentences, Dogon languages exhibit SOV constituent order.

Our assessment of Dogon is based on the grammars of \citet{Heath2016, Heath2017a, Heath2017b, Heath2017c}. In the expression of ‘give’ events, Dogon languages exhibit linkage patterns that are exclusively ditransitive. Argument order in this single linkage type can vary for some languages between NP\textsubscript{T} NP\textsubscript{R} and NP\textsubscript{R} NP\textsubscript{T}. Bunoge shows both NP\textsubscript{T} NP\textsubscript{R} and NP\textsubscript{R} NP\textsubscript{T} in (\ref{ex:Dogon1}) from \citet{Heath2017a}. \ili{Dogul Dom} manifests NP\textsubscript{R} NP\textsubscript{T} in (\ref{ex:Dogon2}) from \citet{Heath2016}, while both \ili{Najamba} (\ref{ex:Dogon3}) and \ili{Yorno So} (\ref{ex:Dogon4}) from \citet{Heath2017c} display only order NP\textsubscript{T} NP\textsubscript{R}.

\ea \label{ex:Dogon1} Bunoge (\cite[215, 324] {Heath2017a})
\begin{xlist}
\ex
\gll Bármà			à-ŋgù				ŋ̀						tábè.\\
						pot					3\textsc{pl=acc}		1\textsc{sgsbj}		give.\textsc{pfv} \\
\glt						‘I gave him/her a pot.’
\ex
\gll ʔÁ:mádù		ŋgù			tɔ́ndí-gè		ŋ̀						tábè.\\
						Amadou 	\textsc{acc} 		money			1\textsc{sgsbj}		give.\textsc{pfv}\\
\glt 						‘I gave the money to Amadou.’
\end{xlist}
\z

\ea \label{ex:Dogon2} Dogul Dom (\cite [186] {Heath2016}) \\
\gll Ó=ỳ					bú:dù			ǹdɛ̀-ŋ.\\
					2\textsc{sg=acc}	money		give.\textsc{pfv-1sgsbj}\\
\glt					‘I gave the money to you.’
\z

\ea \label{ex:Dogon3} Najamba (\citealt[245]{Heath2017c}) \\
\gll Ŋ̀gwɛ̌ː mí						yɛ̀							mó						á:màdù 	gì			ǹdɛ-̀ḿ.\\
				dog		\textsc{1sgposs}		\textsc{psm.ansg}		\textsc{def.ansg}		Amadu		\textsc{acc}		give.\textsc{pfv-1sgsbj}\\
\glt				‘I gave Amadou my dog.
\z

\ea \label{ex:Dogon4} Yorno So (\citealt[360]{Heath2017c}) \\
\gll Sǔm					lɛ́y			sày		mí-ỳ			òb-ù-∅.\\
				hundred		two		only		1\textsc{sg-acc}		give-\textsc{pfv-3sgsbj}\\
\glt				‘He only gave me two hundred riyals.’
\z

None of the available Dogon grammars concentrated on word order for ‘give’ events. Nonetheless, linear order\is{linear ordering} of arguments in one language was flexible. As we found earlier, a similar flexibility of argument order in ditransitives appeared in Ewe, Fongbe, Jóola Banjal, and Vata. It remains to be determined whether all Dogon languages exhibit flexibility for the order of theme and recipient arguments and whether this correlates with any specific meaning change. This issue aside, no Dogon language displayed linkage type verb-verb or verb-oblique, as Table \ref{tab:LinkageDogon} indicates.

\begin{table}
\caption{Linkage types for ‘give’ in Dogon}
\label{tab:LinkageDogon}
 \begin{tabular}{lcc}
  \lsptoprule
 &  V\textsubscript{1} NP\textsubscript{R} NP\textsubscript{T}	&V\textsubscript{1} NP\textsubscript{T} NP\textsubscript{R} 	\\
  \midrule
Bunoge &	+ &	+\\
Dogul Dom &	+&?	\\
Najamba &+& ? \\
Yorno So &+&? \\
  \lspbottomrule
 \end{tabular}
\end{table}

\subsection{Mande and Senufo}

The Mande and Senufo languages occupy western sectors of West Africa. Mande languages in our sample are Mandinka and Bambara. Mande is spoken in the countries of Guinea, Guinea-Bissau, Senegal, and The Gambia. Bambara, on the other hand, is spoken in Mali, Senegal, and Burkina Faso. Senufo in our sample is represented by Supyire; it is spoken in southeastern Mali. All Mande and Senufo languages in our sample order basic sentence constituents as SOV. They also manifest a uniform linkage for recipient and theme.

From the Mande group, \ili{Mandinka} and Bambara articulate ‘give’ predications with a verb-oblique linkage. Argument order is NP\textsubscript{R} NP\textsubscript{T} or NP\textsubscript{T} NP\textsubscript{R}, although never with a predication defined by a single verb.

Mandinka exhibits the verbs \textit{dii} and \textit{so}, which are translated as ‘give’ (\cite{Creissels2015}). Each occurs with an adposition in a complex predicate where argument order for theme and recipient can contrast. Consistent with SOV word order, \textit{dii} takes a preceding theme argument. It is followed by a phrase in which the postposition \textit{la} is preceded by a recipient argument (\ref{ex:Mande1}a). Verb \textit{dii} manifests the verb-oblique linkage NP\textsubscript{T} V\textsubscript{1} NP\textsubscript{R} OBL, where arguments are ordered as NP\textsubscript{T} NP\textsubscript{R}. In contrast, Mandinka \textit{so} ‘give’ takes a preceding recipient as one argument of its verb-oblique linkage; the theme argument follows the verb and is marked by the postposition \textit{la} (\ref{ex:Mande1}b).

\ea \label{ex:Mande1} Mandinka (\cite [225] {Creissels2015})
\begin{xlist}
\ex
\gll Kew-ó				ye								kód-ôo				díi			mus-óo					la.\\
	man-\textsc{def}	\textsc{cmp.pos.tr}		money-\textsc{def}	give		woman-\textsc{def}	\textsc{obl}\\
\glt 						‘The man gave money to the woman.’
\ex 
\gll Kew-ó				ye								mus-ôo					só			kód-óo					la.\\
man-\textsc{def}	\textsc{cmp.pos.tr}		woman-\textsc{def}	give		money-\textsc{def}		\textsc{obl} \\
\glt 						‘The man gave money to the woman.’
\end{xlist}
\z

Overall, Mandinka manifests only verb-oblique linkage. Its two ‘give’ verbs sanction argument orders that contrast: NP\textsubscript{T} NP\textsubscript{R} for \textit{dii} and NP\textsubscript{R} NP\textsubscript{T} for \textit{so}. Oblique is realized by postposition \textit{la} in either instance. Depending on verb selection, linkage for Mandinka is NP\textsubscript{T} V\textsubscript{1} NP\textsubscript{R} OBL or NP\textsubscript{R} V\textsubscript{1} NP\textsubscript{T} OBL.

\ili{Bambara} exhibits two ‘give’ verbs as well, \textit{di} and \textit{sɔn} (\cite{Creissels2007}). Both verbs are limited to verb-oblique linkage with theme and recipient arguments, although argument order is determined by the verb.

The verb \textit{di} exhibits verb-oblique linkage; \textit{di} takes a preceding theme argument. Associated with \textit{di} is the postposition \textit{ma} and its preceding recipient. The verb \textit{sɔn} also manifests verb-oblique linkage. In contrast to \textit{di}, \textit{sɔn} takes a recipient as its preceding argument. These elements are followed by the postposition \textit{ma} and its preceding theme argument. Regardless of the ‘give’ verb employed, Bambara manifests only verb-oblique linkage. However, argument order in this linkage type is conditioned by verb selection. When \textit{di} occurs (\ref{ex:Mande2}a), verb-oblique linkage is realized as NP\textsubscript{T} V\textsubscript{1} NP\textsubscript{R} OBL and its ordering of arguments. When \textit{sɔn} appears (\ref{ex:Mande2}b), verb-oblique linkage has the form NP\textsubscript{R} V\textsubscript{1} NP\textsubscript{T} OBL, with argument order NP\textsubscript{R} NP\textsubscript{T}.

\ea \label{ex:Mande2} Bambara (\cite [8-9] {Creissels2007})
\begin{xlist}
\ex
\gll Mùso						yé				dúmuni		dí			ń		mà.\\
						woman.\textsc{def}		\textsc{pf.pos}	food.\textsc{def}	give		1\textsc{s}	\textsc{postp}\\
\glt						‘The woman gave the food to me.’
\ex
\gll 	Mùso						yé				ń		sɔ́n		dúmuni		ná. \\
						woman.\textsc{def}		\textsc{pf.pos}	1\textsc{s}	give		food.\textsc{def}	\textsc{postp} \\
\glt					 	‘The woman gave me the food.’
\end{xlist}
\z

\begin{sloppypar}
\ili{Supyire}, a Senufo language, also exhibits SOV word order for basic constituents, and it reveals only one ‘give’ verb (\cite{Carlson1994}). The verb \textit{kan} and distinct postpositions occur in verb-oblique linkage predications. Although Supyire articulates the verb-oblique type of linkage, like the \ili{Mande languages}, argument order contrasts are determined by distinct postpositions. In one predication, \textit{kan} precedes postposition \textit{a}, and \textit{kan} takes a theme argument, whereas the postposition \textit{a} accepts a recipient (\ref{ex:Mande3}a). In a second predication, \textit{kan} precedes postposition \textit{na}, and \textit{kan} takes a recipient, while postposition \textit{na} accepts a theme (\ref{ex:Mande3}b). Accordingly, Supyire permits only one linkage type: verb-oblique. Nonetheless, argument order within this linkage type contrasts as NP\textsubscript{R} NP\textsubscript{T} or NP\textsubscript{T} NP\textsubscript{R}. Argument order is determined by selection of postposition. Verb-oblique linkage is NP\textsubscript{T} V\textsubscript{1} NP\textsubscript{R} OBL for \textit{kan a} but NP\textsubscript{R} V\textsubscript{1} NP\textsubscript{T} OBL for \textit{kan na}.
\end{sloppypar}

\ea \label{ex:Mande3} Supyire (\cite [270, 400] {Carlson1994})
\begin{xlist}
\ex
\gll Kà		nɔ̀gɔ̀-lyèngí		sì				ngkùù		kan		u			à. \\
		and	man-old.\textsc{def}	\textsc{narr}		chicken	give		him	to \\
\glt						‘Then my father gave a chicken to him.’
\ex 
\gll Mii	a				u			kan		biki	na.\\
						I			\textsc{perf}	him	give		pen	at	\\
\glt 						‘I have given him a pen.’
\end{xlist}
\z

A view of Mande and Senufo linkage types is shown in Table \ref{tab:LinkageMande}. As we have seen, languages in these two groups employ only one linkage type for ‘give’: verb-oblique. Argument order in this linkage type varies according to the selection of either main verb or postposition. In Mande, distinct argument orders within verb-oblique linkage correlate with contrasting ‘give’ verbs. In Senufo, distinct argument orders within verb-oblique linkage correlate with contrasting postpositions. In \ili{Bambara}, distinct argument orders within verb-oblique correlate with contrasting patterns of selection for both verb and postposition.

\vfill
\begin{table}[H]
\caption{Linkage types for ‘give’ in Mande and Senufo}
\label{tab:LinkageMande}
 \begin{tabular}{lcc}
  \lsptoprule
 &  NP\textsubscript{T} V\textsubscript{1} NP\textsubscript{R} OBL	&NP\textsubscript{R} V\textsubscript{1} NP\textsubscript{T} OBL \\
  \midrule
Mandinka &	+ dii…la &	+ so…la\\
Bambara &	 + di…ma & + sɔn…na	\\
&& \\
Supyire &+ kan…a& + kan…na \\
  \lspbottomrule
 \end{tabular}
\end{table}
\vfill\pagebreak

\subsection{Delta Cross and Ijoid}

Additional languages in our sample come from the Delta Cross group of East Benue Congo and non-Benue Congo Ijoid. All are spoken in eastern Nigeria’s Niger Delta. Even though order of basic constituents varies across these groups, two linkage types are evident for ‘give’ predications and their expression of recipient and theme arguments.

\ili{Delta Cross languages} in our sample include \ili{Obolo} and Kana. Obolo has the ‘give’ verb \textit{nyi} (\citealt{Aaron1999, Faraclas1984}). It manifests ditransitive linkage and verb-verb linkage, both of which restrict argument order. As a ditransitive predicate, \textit{nyi} takes two arguments, a recipient followed by a theme (\ref{ex:Delta1}a). Argument order is restricted to NP\textsubscript{R} NP\textsubscript{T}. In a verb-verb predicate, \textit{nyi} combines with verb \textit{sa} ‘take’. \textit{Nyi} takes a recipient argument and \textit{sa}, a theme (\ref{ex:Delta1}b). Within this last predication type, argument order is restricted to NP\textsubscript{T} NP\textsubscript{R}. Overall, Obolo shows linkage types where argument order is differently restricted: ditransitive V\textsubscript{1} NP\textsubscript{R} NP\textsubscript{T} compared to verb-verb V\textsubscript{2} NP\textsubscript{T} V\textsubscript{1} NP\textsubscript{R}.

\ea \label{ex:Delta1} Obolo (\citealt[53, 83]{Aaron1999})
\begin{xlist}
\ex
\gll Kpèé-yáká				í-nyí				ɔ́mɔ́		mkpɔ́		géègè.\\
						\textsc{cpl-neg.inc}		\textsc{nsp}-give		3\textsc{sg}		thing		any\\
\glt						‘They didn’t give him anything anymore.’
\ex
\gll mèí-ní-gwɔ́						èsé					í-sà					í-nyí				gwújà	yà.\\
						that.\textsc{nsp-inc}-scoop		crayfish		\textsc{nsp}-take		\textsc{nsp}-give		child		\textsc{cpl-neg.inc} \\
\glt 			‘that she scooped some crayfish and gave it to the boy.’
\ex 
\gll í-ꜝnéè-nìí-sà							àchá	yà		í-nyí.\\
  \textsc{3sg-redup-inc}-take		hoe		\textsc{ddemsg}		\textsc{nsp}-give \\
\glt 				‘then he gave the hoe (to the people).’
\end{xlist}
\z

The Delta Cross language \ili{Kana} manifests ditransitive linkage and verb-verb linkage as well. In each linkage type, argument order is constrained, although order in ditransitive predications is atypical. Kana has a ‘give’ verb \textit{ne} (\cite{Ikoro1996}). \textit{Ne} occurs as a ditransitive predicate taking a theme and a recipient (\ref{ex:Delta2}a). Argument order is limited to NP\textsubscript{T} NP\textsubscript{R}, which is relatively unusual for ditransitive linkage in our survey. In a predication with verb-verb linkage, \textit{ne} follows the verb \textit{su} ‘take’. \textit{Ne} takes a recipient argument and \textit{su}, a theme (\ref{ex:Delta2}b). Argument order in this verb-verb predication is again restricted to NP\textsubscript{T} NP\textsubscript{R}. Like \ili{Obolo}, Kana manifests the linkage types ditransitive V\textsubscript{1} NP\textsubscript{T} NP\textsubscript{R} as well as verb-verb V\textsubscript{2} NP\textsubscript{T} V\textsubscript{1} NP\textsubscript{R}. Both linkage types in Kana restrict arguments to the same linear order.

\ea \label{ex:Delta2} Kana (\cite [254]{Ikoro1996})
\begin{xlist}
\ex
\gll Bàrìlè		é-nɛ				péé		yɔ.\\
			Barile	\textsc{df}-give	goat		oraclist\\
\glt			‘Barile will present an oraclist a goat / a goat to an oraclist.’
\ex
\gll 	Bàrìlè		é-sú				péé		nɛ			yɔ.\\
		Barile	\textsc{df}-take	goat		give		oraclist\\
\glt ‘Barile will give / present a goat to an oraclist.’
\end{xlist}
\z

The final language group of the Niger Delta in our sample is \il{Ijoid languages} Ijoid. As with Dogon, Mande, and Senufo, word order of basic clause constituents in Ijoid is SOV. Our sample includes data from the Kolokuma dialect of Ịjọ\il{Ijo@Ịjọ} (\cite{Williamson1965}). Ịjọ exhibits ditransitive and verb-verb linkage types. Like most other languages in the Niger Delta area, it does not show verb-oblique linkage.

Ịjọ has the ‘give’ verb \textit{piri}. It occurs in a ditransitive predicate where a theme argument precedes a recipient argument (\ref{ex:Delta3}a). Argument order for this predication, NP\textsubscript{T} NP\textsubscript{R}, is the mirror image of SVO languages of a similar linkage type in our sample. In Ịjọ, no other argument order is sanctioned. In a verb-verb linkage, \textit{piri} appears with transitive verb \textit{aki} ‘take’. \textit{Piri} and its preceding recipient follow \textit{aki} and its preceding theme (\ref{ex:Delta3}b). Verb-verb linkage in Ịjọ limits argument order to NP\textsubscript{T} NP\textsubscript{R}. As with other languages from the Niger Delta, Ịjọ restricts theme and recipient to a single order regardless of linkage as ditransitive NP\textsubscript{T} NP\textsubscript{R} V\textsubscript{1} or verb-verb NP\textsubscript{T} V\textsubscript{2} NP\textsubscript{R} V\textsubscript{1}.

\ea \label{ex:Delta3} Ịjọ (\cite[54]{Williamson1965})
\begin{xlist}
\ex
\gll Erí		opúru-mɔ-nɪ̀		tɔboʊ́		pɪ̀rɪ-mɪ.\\
	he		crayfish-\textsc{pl}-\textsc{lk}		boy			give-\textsc{spa}\\
\glt			‘He gave the boy the crayfish.’
\ex 
\gll Erí		opúru-mɔ		àkɪ		tɔboʊ́		pɪ̀rɪ-mɪ.\\
				he		crayfish-\textsc{pl}	take		boy			give-\textsc{spa} \\
\glt						‘He gave the crayfish to the boy.’
\end{xlist}
\z

A summary of ‘give’ predications in the Delta Cross and Ijoid groups is presented in Table \ref{tab:LinkageDelta}. Two linkage types are evident: ditransitive and verb-verb. In the easternmost part of the Niger Delta, Delta Cross languages exhibit ditransitive linkage where argument order varies: V\textsubscript{1} NP\textsubscript{R} NP\textsubscript{T} for Obolo but V\textsubscript{1} NP\textsubscript{T} NP\textsubscript{R} for Kana. Verb-verb linkage among these languages is confined to V\textsubscript{2} NP\textsubscript{T} V\textsubscript{1} NP\textsubscript{R}. Ịjọ, spoken across the Niger Delta but notably in its westernmost regions, manifests the same linkage types but with argument order in the ditransitive being the mirror image of Delta Cross Obolo. Ịjọ shows NP\textsubscript{T} NP\textsubscript{R} V\textsubscript{1} for ditransitive linkage and NP\textsubscript{T} V\textsubscript{2} NP\textsubscript{R} V\textsubscript{1} for verb-verb linkage. As a point of comparison with respect to linkage types, Delta Cross and Ijoid strongly resemble Kwa and Gur. All exhibit linkage patterns that are ditransitive and verb-verb.

\begin{table}
\caption{Linkage types for ‘give’ in Delta Cross and Ijoid}
\label{tab:LinkageDelta}
 \begin{tabular}{lcccc}
  \lsptoprule
 &  V\textsubscript{1} NP\textsubscript{R} NP\textsubscript{T}	&V\textsubscript{1} NP\textsubscript{T} NP\textsubscript{R} &	V\textsubscript{2} NP\textsubscript{T} V\textsubscript{1} NP\textsubscript{R}  &
NP\textsubscript{T} NP\textsubscript{R}\
V\textsubscript{1} \\
\midrule
Obolo & + & & +&\\
Kana & & + & +  &\\
&&&NP\textsubscript{T} V\textsubscript{2} NP\textsubscript{R} V\textsubscript{1}& \\
Ịjọ &&  & + & + \\
  \lspbottomrule
 \end{tabular}
\end{table}

\subsection{Benue Congo}

The remaining languages in our sample belong to the Benue Congo group, as spoken in Nigeria. They include Yukuben from Jukunoid, Ebira from Nupoid, Igbo from Igboid, Yoruba from Yoruboid, and several from Edoid. Among these languages, there are three linkage types: ditransitive, verb-verb and verb-oblique.

\ili{Jukunoid languages} are spoken east of the Niger-Benue River confluence and northeast of Igboid toward the Cameroon highlands. In our sample, \ili{Yukuben} represents Jukunoid. It displays ditransitive linkage. It does so with the verb \textit{nda} ‘give’ (\cite{Anyanwu2013}). As a ditransitive predicate, \textit{nda} takes a recipient argument followed by a theme argument (\ref{ex:BC1}). Such predications restrict argument order since they allow only NP\textsubscript{R} NP\textsubscript{T}. Jukunoid thus exhibits ditransitive linkage V\textsubscript{1} NP\textsubscript{R} NP\textsubscript{T}. It does not manifest verb-verb or verb-oblique linkage.

\ea \label{ex:BC1} Yukuben (\cite[204, 270]{Anyanwu2013})
\begin{xlist}
\ex
\gll Ndà		íí-dúng			e-mi.\\
						give		\textsc{cl}-child		\textsc{cl}-breast\\
\glt						‘Give the child a breast.’
\ex
\gll 	È-yí						ndà		mú		bà-tr			nyí?\\
						\textsc{pref}-who		give		2\textsc{sg}		\textsc{cl}-cloth	\textsc{det}\\
\glt						‘Who gave you that garment?’
\end{xlist}
\z

The Nupoid language \ili{Ebira} is spoken in the Nigerian middle belt immediately west and south of the Niger-Benue River confluence. It is located some distance from Jukunoid, but directly east of Yoruba and north of Edoid. Ebira exhibits predications whose linkage types are ditransitive and verb-verb, as was the case for Kwa and Gur, as well as for Delta Cross and Ijoid. Ebira has the ‘give’ verb \textit{yí} (\cite{Adive1984}). As a ditransitive predicate, \textit{yi} takes a recipient argument followed by a theme argument (\ref{ex:BC2}a). In a verb-verb predication, \textit{yi} follows the verb \textit{si} ‘take’. \textit{Yi} and its recipient follow \textit{si} and its theme argument (\ref{ex:BC2}b). Both predication types limit argument order: ditransitive is NP\textsubscript{R} NP\textsubscript{T}, and verb-verb is NP\textsubscript{T} NP\textsubscript{R}. As for linkage type, Ebira shows ditransitive V\textsubscript{1} NP\textsubscript{R} NP\textsubscript{T} and verb-verb V\textsubscript{2} NP\textsubscript{T} V\textsubscript{1} NP\textsubscript{R}. Ebira does not exhibit verb-oblique linkage.

\ea \label{ex:BC2} Ebira (\cite[132]{Adive1984})
\begin{xlist}
\ex
\gll Ìzé		ộ			yị́			ozí			ịsá.\\
						Ize	she		give		child	food\\
\glt						‘Ize gave the child food / fed the child.’
\ex
\gll Ìzé	ộ		sị́			ịsá			yị́			ozí.\\
						Ize	she	take		food	give		child\\
\glt						‘Ize gave the child food.’
\end{xlist}
\z

Turning to Igboid, \ili{Igbo} is spoken over a large area east of the Niger River and in parts of the Niger Delta. It features predications that are a ditransitive or verb-verb, just as found in \ili{Ebira}, Delta Cross, and Ijoid. Igbo has the ‘give’ verb \textit{nye} (\cite{Uwalaka1988}). As a ditransitive predicate, \textit{nye} takes an immediately following recipient argument, in turn followed by a theme argument (\ref{ex:BC3}a). The argument order is NP\textsubscript{R} NP\textsubscript{T}. \textit{Nye} also participates in verb-verb linkage with the verb \textit{wee} ‘take’. The verb \textit{nye} and its recipient argument follow \textit{wee} and its theme argument (\ref{ex:BC3}b). The linear order of arguments is NP\textsubscript{T} NP\textsubscript{R}. Although argument order varies according to linkage type, Igbo displays both ditransitive V\textsubscript{1} NP\textsubscript{R} NP\textsubscript{T} and verb-verb V\textsubscript{2} NP\textsubscript{T} V\textsubscript{1} NP\textsubscript{R}. There is no evidence that Igbo shows verb-oblique linkage. As with most other language descriptions in our sample, no mention is made of a potential functional difference between grammatically sanctioned linkage types.

\ea \label{ex:BC3} Igbo (\cite[122]{Uwalaka1988}, personal communication)
\begin{xlist}
\ex
\gll Àdha		nyè-rè				ucè		ego.\\
						Adha		give-\textsc{past}		Uce		money\\
\glt						‘Adha gave Uce money.’
\ex 
\gll Àdha		wèe-re				ego			nye		ucè.\\
						Adha		take-\textsc{past}		money	give	Uce \\
\glt ‘Adha gave money to Uce.’
\end{xlist}
\z

In contrast to Igbo, \ili{Yoruba} is spoken in southwestern Nigeria, including the border region with the Republic of Benin. Different from other Benue-Congo languages, Yoruba uses ‘give’ expressions with predications that are either verb-oblique or verb-verb. Its ‘give’ verb is \textit{fun} (\citealt{Atoyebietal2010, Lord1993}). In a verb-oblique predication, \textit{fun} appears with the adposition \textit{ni}. The verb \textit{fun} and its recipient argument precede the oblique marker \textit{ni} and its theme argument (\ref{ex:BC4}a). In a verb-verb predication, \textit{fun} follows the verb \textit{mu} ‘take.’ \textit{Fun} takes a recipient argument, while \textit{mu} takes a theme argument (\ref{ex:BC4}b). Linear order in both linkage types is constrained. Verb-oblique employs order NP\textsubscript{R} NP\textsubscript{T}, while verb-verb utilizes NP\textsubscript{T} NP\textsubscript{R}. Setting order restrictions aside, Yoruba manifests verb-oblique linkage V\textsubscript{1} NP\textsubscript{R} OBL NP\textsubscript{T} and verb-verb linkage V\textsubscript{2} NP\textsubscript{T} V\textsubscript{1} NP\textsubscript{R}. Yoruba does not show ditransitive linkage.

\ea \label{ex:BC4} Yoruba (\citealt[2, 148]{Atoyebietal2010})
\begin{xlist}
\ex
\gll Bólá		fún		adé		ní		ìwé.\\
						Bola	give		Ade		\textsc{seca}	book\\
\glt						‘Bola gave Ade a book.’
\ex 
\gll 	Bólá		mú		ìwé			fún		adé.\\
						Bola	take		book		give		Ade\\
\glt 							‘Bola gave a book to Ade.’
\end{xlist}
\z

Our sample includes four \ili{Edoid languages}, viz. Bini, Esan, Degema, and Emai (\cite{Elugbe1989}). Most are spoken west of the Niger River, east of the Yoruba, south of the Ebira, and west of the Igbo. Among Edoid languages, verb-oblique predications dominate. However, Degema, spoken on an island in the Niger Delta, exhibits both verb-oblique and ditransitive linkage.

We start with \ili{Emai} and its ‘give’ expression, drawing upon our own previous research. Emai shows only verb-oblique predications. It exhibits a ritual event verb \textit{kuee} with the two senses ‘give a present, provide a welcome gift or offering to someone’, as well as ‘betroth to someone, become engaged for marriage’. With either sense, \textit{kuee} occurs in a verb-oblique predication with the \isi{applicative} adposition \textit{li/ni} (\textit{ni} is reserved for clause final position when the applicative complement appears in focus or question word position, or when its complement is a pronoun). The verb \textit{kuee} consistently takes a theme argument that is followed by \textit{li/ni} and its recipient argument (\ref{ex:BC5}a--b).

\ea \label{ex:BC5} Emai (\citealt[241]{SchaeferEgbokhare2007})
\begin{xlist}
\ex
\gll Òjè		kúéé	òkpàn		lí			í\textsuperscript{ǃ}ré.\\
	Oje:\textsc{prx} \textsc{pst}:present:\textsc{pfv}		gourd			\textsc{app}	visitors\\
\glt						‘Oje has presented a gourd to his visitors.’
\ex 
\gll Yàn				kúéé	ólì		òkpòsò		lí			òhí.\\
	\textsc{3pl:prx} \textsc{pst}:present:\textsc{pfv}	the	woman		\textsc{app}	Ohi.\\
\glt			‘They have betrothed the woman to Ohi.’
\end{xlist}
\z

For other less ritualized events, \ili{Emai} has no single ‘give’ verb (\cite{SchaeferEgbokhare2007, SchaeferEgbokhare2010, SchaeferEgbokhare2017}). Instead, it deploys a variety of transitive verbs of handling that convey object manipulation such as \textit{zɛ} ‘scoop’ (\ref{ex:BC6}a), \textit{roo} ‘pick out’ (\ref{ex:BC6}b), \textit{vo} ‘fetch’ (\ref{ex:BC6}c), and \textit{nwu} ‘carry’ (\ref{ex:BC6}d). Each occurs with the 
\isi{applicative} adposition \textit{li/ni} in verb-oblique predication V NP\textsubscript{T} OBL NP\textsubscript{R}, where argument order is confined to NP\textsubscript{T} NP\textsubscript{R}. This was also the case for argument order with \textit{kuee}.

\ea \label{ex:BC6} Emai (\citealt[518-525]{SchaeferEgbokhare2007})
\begin{xlist}
\ex
\gll Ɔ́lì		òkpòsò					zɛ́	émàè		lí			ɔ́lì		ɔ̀nwìmè.\\
						the	woman:\textsc{prx}	\textsc{pst}:scoop:\textsc{pfv}		food		\textsc{app}	the	farmer \\
\glt 						‘The woman has given food to the farmer.’
\ex
\gll 	Ɔ́lì		òkpòsò					róó							ɔ́lì		ùhàì			lí			ɔ́lì		ɔ̀nwìmè.\\
						the	woman:\textsc{prx}	\textsc{pst}:pick:\textsc{pfv}	the	arrow		\textsc{app}	the	farmer\\
\glt						‘The man has given the arrow to the farmer.’
\ex 
\gll 	Ɔ́lí		ɔ́mɔ̀hè			vó									óràn		lí			ɔ́lí		ókpósɔ́díɔ̀n.\\
						the	man:\textsc{prx}	\textsc{pst}:fetch:\textsc{pfv}	wood		\textsc{app}	the	old.woman \\
\glt 						‘The man has given wood to the old woman.’
\ex 
\gll 	Ɔ́lì		òkpòsò					nwú								émà		lí			ɔ́lì		ɔ̀nwìmè. \\
						the	woman:\textsc{prx}	\textsc{pst}:carry:\textsc{pfv}	yam		\textsc{app}	the	farmer \\
\glt 						‘The woman has given yam to the farmer.’
\end{xlist}
\z

It is important to note that Emai handling verbs, when not in a predication with \textit{li/ni}, do not denote ‘give’. As the sole predicate, these verbs convey various manners for handling entities, such as scooping, picking out, fetching, or carrying, as in (\ref{ex:BC7}).

\ea \label{ex:BC7} Emai (\citealt[518-525]{SchaeferEgbokhare2007})
\begin{xlist}
\ex
\gll Ɔ́lì		òkpòsò					zɛ́										émàè.\\
						the	woman:\textsc{prx}	\textsc{pst}:scoop:\textsc{pfv}		food\\
\glt						‘The woman has scooped food.’
\ex
\gll 	Ɔ́lì		òkpòsò					róó								ùhàì.\\
						The	woman:\textsc{prx}	\textsc{pst}:pick:\textsc{pfv}		arrow\\
\glt ‘The man has picked out an arrow.’
\ex
\gll 	Ɔ́lí		ɔ́mɔ̀hè			vó									óràn.\\
						the	man:\textsc{prx}	\textsc{pst}:fetch:\textsc{prv}	wood\\
\glt					‘The man has fetched wood.’
\ex
\gll 	Ɔ́lì		òkpòsò					nwú								émà.\\
						the	woman:\textsc{prx}	\textsc{pst}:carry:\textsc{pfv}	yam\\
\glt					‘The woman has carried yam.’
\end{xlist}
\z

It is equally important to point out that the oblique marker \textit{li/ni} has no counterpart appearing as a synchronic verb. There are no transitive or ditransitive predications shaped by \textit{li/ni} as the sole verb, especially one that might mean ‘give’ or something similar; see attempted forms in (\ref{ex:BC8}).

\ea \label{ex:BC8} Emai (\citealt[518-525]{SchaeferEgbokhare2007})
\begin{xlist}
\ex[*]{
\gll Ɔ́lì		òkpòsò					lí							émàè.\\
						the	woman:\textsc{prx}	\textsc{pst}:give:\textsc{pfv}		food\\
\glt						‘The woman has given food.’ [intended]}
\ex[*]{
\gll Ɔ́lì		òkpòsò					lí			ójé				émàè.\\
						the	woman:\textsc{prx}	\textsc{pst}:give:\textsc{pfv}	Oje	food\\
\glt						‘The woman has given Oje food.’ [intended]}
\end{xlist}
\z

\ili{Emai} exhibits only the verb-oblique linkage V\textsubscript{1} NP\textsubscript{T} OBL NP\textsubscript{R}. As already shown, this linkage type restricts argument order to NP\textsubscript{T} NP\textsubscript{R}. Emai manifests neither ditransitive nor verb-verb linkage.

The most populous of the Edoid languages, spoken south of Emai but north of the western-most area of the greater Niger Delta region, is \ili{Bini}. It is limited to verb-oblique predications for ‘give’. Bini predicates in this domain consist of a verb of handling and an oblique marker (\cite{Agheyisi1990}). For instance, verb \textit{rhie} ‘pick out’ and its argument precede the adposition \textit{ne}. \textit{Rhie} accepts a theme argument; \textit{ne} follows with a recipient argument (\ref{ex:BC9}a). Argument order \is{linear ordering} in Bini predications of this sort is limited to NP\textsubscript{T} NP\textsubscript{R}. Overall, Bini displays verb-oblique linkage V\textsubscript{1} NP\textsubscript{T} OBL NP\textsubscript{R}. Linkage possibilities for ‘give’ are further restricted, since there is neither ditransitive linkage (\ref{ex:BC9}b) nor verb-verb linkage. Virtually identical statements can be made about \ili{Esan}, an Edoid neighbor to the north of Bini and east of Emai (\cite{Ejele1986}).

\ea \label{ex:BC9} Bini (\cite[92]{Agheyisi1990})
\begin{xlist}
\ex[]{
\gll Òzó		rhíè					ùkpòn	mwɛ́n		nè		ɔ̀sè			ɔ́rè.\\
						Ozo		pick.out		cloth		my			to		friend		his\\
\glt						‘Ozo gave my cloth to his friend.’}
\ex[*]{
\gll 	Òzó		rhíè		ɔ̀sè			ɔ́rè		ùkpòn	mwɛ́n. \\
							Ozo		give		friend		his			cloth		my \\
\glt 		‘Ozo gave his friend my cloth.’ [intended]}
\end{xlist}
\z

An Edoid language whose speakers occupy parts of a Niger Delta island is Degema (\cite{Kari2004}). It shows verb-oblique linkage and ditransitive linkage; it also restricts argument order in both linkage types. \ili{Degema} articulates ‘give’ with the verb \textit{kiye} as a ditransitive predicate taking a recipient and theme argument. The linear order of arguments is restricted to NP\textsubscript{R} NP\textsubscript{T}. Another argument order in Degema appears with a verb-oblique predication, which consists of the ‘give’ verb \textit{kiye} and the adposition \textit{mu}. \textit{Kiye} takes a theme argument and \textit{mu} a recipient. Predications of this sort restrict argument order to NP\textsubscript{T} NP\textsubscript{R}. Compared to other Edoid languages, Degema shows not only the verb-oblique linkage V\textsubscript{1} NP\textsubscript{T} OBL NP\textsubscript{R} but also the ditransitive linkage V\textsubscript{1} NP\textsubscript{R} NP\textsubscript{T}. It fails to show verb-verb linkage.

\ea \label{ex:BC10} Degema  (\cite[194]{Kari2004})
\begin{xlist}
\ex
\gll Ohoso			ɔ=kɪ́yé=n						ɔ́mó		yɔ		ɔ́sama.\\
						Ohoso		\textsc{3sgscl}=give=\textsc{fe}	child	\textsc{def}	shirt\\
\glt						‘Ohoso gave the boy a shirt.’
\ex
\gll Ohoso		ɔ=kɪ́yé=n						ɔsama	mʊ́	ɔ́mó			yɔ.	\\
						Ohoso	\textsc{3sgscl}=give=\textsc{fe}	shirt			to		child		\textsc{def}\\
\glt						‘Ohoso gave a shirt to the boy.’
\end{xlist}
\z

Predication types evident in our Benue Congo sample appear in Table \ref{tab:LinkageBC}. All three linkage types are evident; this is the only West African region we have surveyed where this is the case. Ditransitive linkage V\textsubscript{1} NP\textsubscript{R} NP\textsubscript{T} is found in Yukuben, Ebira, Igbo, and Degema. Verb-verb linkage is evident in Ebira, Igbo, and Yoruba. Verb-oblique linkage occurs in Yoruba, as well as Bini, Esan, Emai, and Degema. In addition, argument order within each linkage type is restricted: NP\textsubscript{R} NP\textsubscript{T} for ditransitive, NP\textsubscript{T} NP\textsubscript{R} for verb-verb, and NP\textsubscript{T} NP\textsubscript{R} for verb-oblique. There is also the uniquely ordered NP\textsubscript{R} NP\textsubscript{T} for verb-oblique in Yoruba. Overall, Benue Congo exhibits the linkage types ditransitive V\textsubscript{1} NP\textsubscript{R} NP\textsubscript{T}, verb-verb V\textsubscript{2} NP\textsubscript{T} V\textsubscript{1} NP\textsubscript{R} and verb-oblique, either V\textsubscript{2} NP\textsubscript{T} OBL NP\textsubscript{R} or V\textsubscript{2} NP\textsubscript{R} OBL NP\textsubscript{T}.

\begin{table}
\small
\caption{Linkage types for ‘give’ in Benue Congo}
\label{tab:LinkageBC}
 \begin{tabular}{lcccc}
  \lsptoprule
 &  V\textsubscript{1} NP\textsubscript{R} NP\textsubscript{T}
 & V\textsubscript{2} NP\textsubscript{T} V\textsubscript{1} NP\textsubscript{R}
 & V\textsubscript{1} NP\textsubscript{R} OBL NP\textsubscript{T}
 & V\textsubscript{2} NP\textsubscript{T} OBL NP\textsubscript{R}  \\
\midrule
Yukuben & +& -- &-- &-- \\
Igbo &+& +&--&-- \\
Ebira&+ &+& --& -- \\
Yoruba& -- &+& + &-- \\
&&&& \\
Emai& -- &-- &--& + \\
Bini& -- &-- &--& + \\
Esan& -- &--& --&+ \\
Degema &+& --& --& + \\
  \lspbottomrule
 \end{tabular}
\end{table}


\section{Discussion}\label{sec:8:4}

Our survey of argument linkage for ‘give’ predications has netted three primary findings. Firstly, linkage types are not evenly distributed across the language groups of Niger-Congo. Secondly, some linkage types appear across physically adjacent language groups and so reflect potential areal zones of convergence. Linkage coupling or its absence, in fact, allows one to identify two primary areal zones in West Africa. Thirdly, the frequency with which a recipient argument abuts a ‘give’ verb reflects an adjacency condition that, while relatively uniform in distribution across language groups, is attenuated for ditransitive linkage in one areal subzone.

In what follows, we first note the frequency of the three linkage types relative to language groups and their members. For each type, we identify geographically \is{areality} defined areas within West Africa where a linkage type is or is not employed. Using a straightforward frequency count, it is ditransitive linkage that is most extensively distributed in our sample of language groups. Less widely distributed are predications that consist of two elements. Multi-element linkage patterns are also not evenly distributed across language groups. Verb-verb linkage is nearly twice as frequent as verb-oblique linkage.

\subsection{Linkage type distribution}

Occurrence of the three linkage types by language group is summarized in Table~\ref{tab:LinkagebyGroup}. Ditransitive linkage occurs in 11 of 14 groups, although not exclusively so for many of them. Ditransitive linkage is evident across the central West African states of Burkina Faso, Ivory Coast, Ghana, Mali, Togo, and the Republic of Benin. It is also found on either side of this block. Toward the west, ditransitive linkage occurs in Senegal and western Ivory Coast. Toward the east, it appears in eastern Nigeria, including the Niger River delta, and near the Niger-Benue confluence. Ditransitive linkage was not found among any of the Mande and Senufo groups of Mali, north of the central block, or in Guinea-Bissau and Gambia, west of the central block. It was also lacking in southwestern Nigeria, where neither Yoruba nor any upland Edoid language exhibited ditransitive linkage.

\begin{table}
\caption{Linkage types aligned according to language group}
\label{tab:LinkagebyGroup}
 \begin{tabular}{llll}
  \lsptoprule
 & V\textsubscript{1} & V\textsubscript{2}-V\textsubscript{1} & V\textsubscript{2}-OBL  \\
\midrule
Kwa (8)	& BA, AK, GA, EW &	BA, AK, GA, EW	 & \\
& 	FB, LO, TA, AV &	FB, LO, TA, AV	 & \\
Gur (2) & 	DA, KA	&  DA, KA	& \\
Atlantic (2) & 	JB, DF		 & & \\
Dogon (4)	&BN, DD, NJ, YS&&\\		
Kru (1)	&VA		&& \\
Mande (2)	& & &		MA, BB\\
Senufo (1)	& & &		SU \\
Jukunoid (1) &	YU	 && \\
Ijoid (1) &	IJ&	IJ	& \\
Delta Cross (2)	& OB, KA& 	OB, KA	&\\
Igboid (1) &	IG	&IG	& \\
Nupoid (1)	& EB&	EB	& \\
Yoruboid (1)& &		YR	&YR\\
Edoid (4) &	DE	& &	EM, BI, ES, DE \\
\midrule
\# of groups &	11&	7&	4 \\
  \lspbottomrule
 \end{tabular}
\end{table}

Verb-verb linkage is found in 7 of 14 groups, three fewer than ditransitive linkage. Verb-verb dominates in the central block of nations including eastern Ivory Coast, Ghana, Togo, Burkina Faso, and the Republic of Benin. It fails to occur to the west and northwest of this block. To its east, verb-verb linkage appears not only in the Niger Delta but in southwestern Nigeria with Yoruba. Verb-verb linkage, however, is not found among the \ili{Edoid languages} of southwestern Nigeria nor the \ili{Jukunoid languages} spoken in the Nigerian middle belt as it nears the Cameroonian highlands.


The last of the linkage types, verb-oblique, is found in only 4 of 14 groups. These are Mande and Senufo, north and west of the central block, as well as Yoruba and Edoid of southwestern Nigeria, directly east of the central block.

\subsection{Linkage type combinations}

Our next point of analysis concerns linkage combinations within and across language groups. Table \ref{tab:LinkageCoupling} provides an overall impression of language groups and their members that couple linkage types or that maintain a single linkage type. There are three patterns where linkage types couple: ditransitive plus verb-verb (D-VV); verb-verb plus verb-oblique (VV-VOBL); and ditransitive plus verb-oblique (D-VOBL). By far the most widely distributed of these linkage couples is D-VV, i.e., ditransitive with verb-verb.

\begin{table}
\caption{Linkage coupling according to language group}
\label{tab:LinkageCoupling}
 \begin{tabular}{llll}
  \lsptoprule
 & V\textsubscript{1} + V\textsubscript{2}-V\textsubscript{1} & V\textsubscript{2} V\textsubscript{1} + V\textsubscript{2}-OBL & V\textsubscript{1} + V\textsubscript{2}-OBL  \\
\midrule
Kwa	&  BA, KA, GA, EN		& & \\
	&FG, LO, TA, AV	&& \\	
Gur &	DA, KA		&& \\
Atlantic		&&& \\	
Dogon 			&&&\\
Kru 			&&&\\
Mande			&&&\\
Senufo			&&&\\
Jukunoid			&&&\\
Ijoid &	IJ		 && \\
Delta Cross &	OB, KA		&& \\
Igboid & 	IG		&& \\
Nupoid	& EB		&& \\
Yoruboid & &	YR	& \\
Edoid 	& & &		DE\\
\midrule
\# of groups &	6	&1&	1\\
  \lspbottomrule
 \end{tabular}
\end{table}

D-VV coupling occurs in the greatest number of language groups. It is found in 6 of the 14. It is, again, the central block of nation states that constitute one area where D-VV is exclusively found. The Kwa and Gur language groups, encompassing the Bandama-Tano-Volta (BTV) watershed define this area. Another area rich in D-VV coupling is the Niger Delta, encompassing \il{Delta Cross languages} Delta Cross and \ili{Ijoid languages} as well as Igboid. Outside the Niger Delta, it is only Nupoid in our survey that shows D-VV coupling; it is spoken near the Niger-Benue River confluence.

VV-VOBL coupling occurs in only one language group. \ili{Yoruba} is the single language in our sample that combines verb-verb and verb-oblique linkage predications. D-VOBL coupling is also very limited in our sample. It appears in only one language group, viz. \il{Edoid languages} Edoid, and only in one of its four languages: \ili{Degema}. As mentioned earlier, Degema is spoken on an island in the Niger Delta, an area where language groups tend to exhibit at least ditransitive linkage. There are 7 groups that fail to exhibit linkage coupling altogether or do so extremely sparingly, as shown in Table \ref{tab:LinkageFail}.

\begin{table}
\caption{Language group members failing to couple linkage types}
\label{tab:LinkageFail}
 \begin{tabular}{llll}
  \lsptoprule
 & V\textsubscript{1} & V\textsubscript{2}V\textsubscript{1} & V\textsubscript{1} OBL \\
\midrule
Kwa	&  		& & \\
Gur &		&& \\
Atlantic 		& JB, DF&& \\	
Dogon 			& BN, DD, NJ, YS&&\\
Kru 			& VA&&\\
Mande			& && MA, BB\\
Senufo			&&&SU \\
Jukunoid			&YU &&\\
Ijoid &			 && \\
Delta Cross &			&& \\
Igboid & 		&& \\
Nupoid	&		&& \\
Yoruboid & &		& \\
Edoid 	& & &	EM, BI, ES\\
\midrule
\# of groups &	4	&0&	3\\
  \lspbottomrule
 \end{tabular}
\end{table}

Non-coupling groups include Atlantic, Dogon, Kru, Mande, Senufo, Jukunoid, and most of Edoid except Degema of the Niger Delta. None of these non-coupling language groups appears within the BTV watershed. Atlantic is far to the west in Senegal, and Kru is found in western Ivory Coast. Dogon is spoken to the north and immediate west of the BTV area in Mali and possibly Burkino Faso. Mande also occurs to the north and immediately west of the BTV area in Mali but also in Guinea-Bissau. Nearly all of Edoid is spoken west of the Niger River in southwestern Nigeria. Jukunoid appears in the Nigerian middle belt near the Cameroonian highlands.

Overall, we find that the propensity for D-VV coupling is extremely strong in our sample. This propensity, however, does not correlate with a contiguous land mass. Rather, there appear to be two major areas of D-VV coupling in West Africa. One is the BTV watershed that includes southeastern Ivory Coast, Ghana, Togo, Burkina Faso, and the Republic of Benin, where Kwa and Gur languages are spoken. The other is the Niger Delta, where Ijoid, Delta Cross, and Igboid languages converge.

Between these two major zones of D-VV coupling is western Nigeria, homeland of the Yoruba. It is perhaps not surprising that, in this location, Yoruba coupling would be unique in our sample. \ili{Yoruba} VV-VOBL coupling employs one multi-element linkage type found to its east and one to its west. To its immediate west is the verb-verb linkage of Kwa and Gur. To its immediate east is the verb-oblique linkage of upland Edoid. Yoruba VV-VOBL combines these.

Two more \is{areality} convergence zones can be gleaned from Table \ref{tab:LinkageFail}, although not from linkage coupling. Instead, these two areas show verb-oblique linkage, exclusively so or nearly so. One of these areal zones is immediately north and west of the BTV watershed. This is where the Mande and Senufo languages are spoken. A second area where verb-oblique dominates is immediately west of the Niger River where one finds the bulk of \il{Edoid languages} Edoid and Yoruba. Interestingly, the languages in these two areas, although similar with respect to linkage types, are dissimilar with respect to the order of basic constituents within a clause. Mande and Senufo are SOV, while Edoid and Yoruba are SVO. What we should make of this is not yet clear, but it seems relevant to the overall distribution of areal linguistic patterns in West Africa.

Finally, we note that verb-verb is the only linkage type that is not exclusive to a single language group in our sample. It consistently couples with another linkage type, primarily ditransitive but also verb-oblique. Verb-verb appears to be a linkage type whose dependency requires further attention, especially when dependency is not a characteristic of ditransitive or verb-oblique linkage.\largerpage

To start, we suggest that this dependency may be related to a sometimes-noted function for verb-verb linkage in our survey. Although inconsistently identified, the meaning relationship between the distinct forms D and VV appears to correlate, at least in some languages, with distinct functions. That is, VV expresses object transfer, while D signals object transfer with possession change. Both linkage types have in common a V\textsubscript{1} verb ‘give’. Such a relationship is not unlike predication pairs in Edoid where there also exists a verb that is common to predications that are functionally distinct with respect to a feature of tertiary aspect (\cite{DesclésGuentchéva2012}).{\interfootnotelinepenalty=10000\footnote{Tertiary aspect is a term used to identify aspectual values beyond viewpoint aspect and Aktionsart.}} For example, common to \ili{Emai} predications (\ref{ex:khuae}a) and (\ref{ex:khuae}b) is the verb \textit{khuae} ‘raise.’ Predication (\ref{ex:khuae}a) means that the direct object of \textit{khuae} simply changes its position in a vertical manner. The extent of positional change is not coded linguistically. In contrast, predication (\ref{ex:khuae}b), where \textit{khuae} is preceded by its erstwhile direct object and by verb \textit{nwu} ‘carry/take’, signals that position change has occurred and that it has achieved its maximum extent, e.g., ‘up to arm’s length’. No further positional change is possible.

\ea \label{ex:khuae} Emai (\citealt[732]{SchaeferEgbokhare2017})
\begin{xlist}
\ex
\gll Òjè				khúáé							ɔ́lì		ùkòdò.\\
	Oje:\textsc{prx}	\textsc{pst}:raise:\textsc{pfv}		the	pot\\
\glt						‘Oje has raised the pot.’	
\ex 	
\gll Òjè				nwú								ɔ́lì		ùkòdò	khúáé.\\
						Oje:\textsc{prx}	\textsc{pst}:carry:\textsc{pfv}	the	pot			raise\\
\glt						‘Oje has raised the pot up to arm’s length.’	
\end{xlist}
\z

By analogy, it may be that in our survey of West Africa that D and VV relate via a feature in which D denotes maximal transfer, i.e., possession/ownership change, while VV expresses \is{possession} simple transfer that is non-maximal. It seems possible that the various languages in this survey showing both D and VV predications exploit a feature of tertiary aspect that allows one to profile possession or transfer. More detailed studies within Niger-Congo will be required to determine whether this might be the case.

\subsection{Adjacency of linkage elements}

Our final point of analysis concerns a possible adjacency relationship between NP\textsubscript{R} (recipient, R in Table \ref{tab:LinkageRecipient}) and one or another predication element, V\textsubscript{1}, V\textsubscript{2}, or OBL. There are five adjacency possibilities in our sample where either a verb or an oblique marker can stand adjacent to a recipient argument. The five are V\textsubscript{1} NP\textsubscript{R} for ditransitive linkage, V\textsubscript{1} NP\textsubscript{R} or V\textsubscript{2} NP\textsubscript{R} for verb-verb linkage, and OBL NP\textsubscript{R} or V\textsubscript{1} NP\textsubscript{R} for verb-oblique linkage. In Table \ref{tab:LinkageRecipient}, each is presented to allow for mirror image occurrence of recipient across SVO and SOV languages. We consider recipient rather than theme since animate entities have been central to the comparison of three argument predications (\cite{Gruber1992}) and their typology (\citealt{Kittilä2006, MargettsAustin2007}).

A review of Table \ref{tab:LinkageRecipient} reveals that not all adjacency conditions are realized in our sample. Of course, not all languages in our sample exploit each of the linkage types verb-oblique, verb-verb, and ditransitive. Nonetheless, within each linkage type, one can assess cross-linguistic tendencies for recipient placement.

Under ditransitive linkage, V\textsubscript{1} NP\textsubscript{R} / NP\textsubscript{R} V\textsubscript{1} adjacency is found in 11 of 14 language groups. It occurs in the two major areal zones we have previously identified within West Africa: the BTV watershed and the Niger Delta. In fact, all of eastern Nigeria including the Niger-Benue confluence zone and the middle belt extending from there to the Cameroonian highlands shows V\textsubscript{1} NP\textsubscript{R} / NP\textsubscript{R} V\textsubscript{1} adjacency.

There are two areas where ditransitive V\textsubscript{1} NP\textsubscript{R} adjacency does not appear. One is the area between the BTV Watershed and eastern Nigeria demarcated by the river Niger. Essentially this is western Nigeria; it includes \ili{Yoruba} and the greater part of  \il{Edoid languages} Edoid. The second area is north and immediately west of the central block; it includes Guinea-Bissau, The Gambia, Mali, and their Mande and Senufo languages.

Each of these areas where ditransitive is not exploited has its own adjacency conditions. In the area north and immediately west of the BTV Watershed, the Mande and Senufo languages show alternating adjacency conditions for recipient. They exhibit verb-oblique NP\textsubscript{R} V1 adjacency or verb-oblique NP\textsubscript{R} OBL adjacency. In western Nigeria there are three adjacency conditions for recipient. Yoruba relies on verb-verb V\textsubscript{1} NP\textsubscript{R} adjacency and verb-oblique V\textsubscript{1} NP\textsubscript{R}. Edoid, in contrast, exploits only verb-oblique OBL NP\textsubscript{R} adjacency.

Highlighting these two areas draws attention to adjacency conditions that are infrequent in Table \ref{tab:LinkageRecipient}. The verb-oblique V\textsubscript{1} NP\textsubscript{R} adjacency is limited to Yoruba. It represents the only group in our sample of 14 that employs this adjacency condition in addition to verb-verb V\textsubscript{1} NP\textsubscript{R} adjacency. The other less frequent condition is OBL NP\textsubscript{R} and its mirror image NP\textsubscript{R} OBL; they occur, respectively, among the Edoid of western Nigeria and the Mande and Senufo groups north and west of the BTV Watershed.

Interestingly, the only possible adjacency condition that failed to occur in our sample was verb-verb V\textsubscript{2} NP\textsubscript{R}. Recipient is never adjacent to V\textsubscript{2} position in a verb-verb linkage. It will be important to determine whether this is true of all languages in West Africa. Whatever the case, verb-verb V\textsubscript{2} NP\textsubscript{R} absence will require future attention.

\begin{sidewaystable}
\caption{Adjacency of ‘give’ and recipient (R) in each linkage type according to group, regardless of basic constituent order}
\label{tab:LinkageRecipient}
 \begin{tabular}{llllll}
  \lsptoprule
& 	DITR &	V-V\textsubscript{1} & 	V\textsubscript{2}-V\textsubscript{1}&	V-OBL & V\textsubscript{1}-OBL \\
 & V\textsubscript{1} R & V\textsubscript{1} R & V\textsubscript{2} R & OBL R & V\textsubscript{1} R \\ 
& R V\textsubscript{1SOV} & R V\textsubscript{1SOV} & R V\textsubscript{2SOV} & R OBL\textsubscript{SOV} & R V\textsubscript{1SOV} \\
\midrule
Kwa	& BA, AK, GA, EW &	BA, AK, GA, EW	 & & & \\		
	& FG, LO, TA, AV	&FG, LO, TA, AV	 &&&\\		
Gur	& DA, KA &	DA, KA	& && \\		
Atlantic&	JB, DF&JB, DF		 &&& \\	
Dogon\textsubscript{SOV} &	BN, DD, NJ, YS		&&&& \\		
Kru\textsubscript{SVO/SOV} &	VA	&&&& \\			 
Mande\textsubscript{SOV}	& & 	&	&	MA, BB&	MA, BB \\
Senufo\textsubscript{SOV} && 		&	&	SU&	SU \\
Jukunoid	& YU	 &&&& \\			
Ijoid\textsubscript{SOV}& 	IJ	&IJ	&&&\\	
Delta Cross &	OB, KA	&OB, KA		&&&\\	
Igboid&	IG&	IG		&&&\\	
Nupoid	&EB&	EB&&&\\			
Yoruboid &	&	YR		&&&	YR \\
Edoid &	DE		&&	& EM, BI, ES, DE	& \\
\midrule
\# of groups 	&11&	8&	0	&3	&3 \\
  \lspbottomrule
 \end{tabular}
\end{sidewaystable}   


\section{Conclusion}

We have surveyed ‘give’ predications that express possession change among 14 West African language groups. We sought to identify linkage types for theme and recipient arguments. Two primary areal zones emerged where linkage types couple for adjacent groups. One is the BTV watershed; the other is the Niger Delta. In these two areas, linkage types for ‘give’ couple.

The strength of these two zones is suggested by two atypical linkage patterns in our sample. In the BTV watershed, ditransitive and verb-verb linkage are typical. In the area immediately east of the BTV, Yoruba shows verb-verb linkage. It is the only language in our sample that exhibits the coupling of verb-verb linkage and verb-oblique linkage. Recall that verb-oblique characterizes Edoid even further to the east than \ili{Yoruba}. The other atypical linkage is shown by \ili{Degema}, an Edoid language spoken on an island in the Niger Delta. In the Delta zone, ditransitive linkage and verb-verb linkage are typical. Degema exhibits ditransitive linkage. It also couples ditransitive with verb-oblique linkage, which characterizes the remainder of Edoid. No other language in our sample exhibits a coupling of linkage types ditransitive and verb-oblique.

As a final analytic point, an areal \is{areality} subzone evident in our survey is defined not by a linkage type but by an adjacency condition defined by verb and recipient. Relative to ditransitive linkage, Kwa and Gur languages most often restricted argument order to recipient first and theme second. However, two Kwa languages, \ili{Ewe} and \ili{Fongbe}, exhibited flexible theme recipient order. They permitted both NP\textsubscript{R} NP\textsubscript{T} and NP\textsubscript{T} NP\textsubscript{R} order. What we do not know is if the semantic condition governing possession/non-possession change in Ewe and Fongbe also applies to other languages in Kwa and perhaps Gur.

Although ditransitive ‘give’ and its linkage patterns were central to our analysis, we hesitate to suggest that other trivalent verbs in each of our language groups would exhibit similar behavior. In part, we say this because some languages in our sample, Emai for instance, which manifested verb-oblique linkage exclusively, have trivalent verbs in domains of information transfer, contact, and possession exchange, among others. These domains require cross-linguistic attention (\cite{Gruber1992}) if we are to understand ditransitivity in West Africa and its possible role in Niger-Congo and areal studies. We should not assume that ‘give’ is a reliable guide to broad patterning of \isi{trivalent predicates}. Its atypical syntactic nature as a trivalent verb has been previously noted (\citealt{BorgComrie1984, Kittilä2006}). It may be, however, that it is precisely this atypical syntactic and semantic nature that allows ‘give’ equivalents to exhibit variable linkage types.

Clearly, there are additional issues, particularly regarding the grammatical realization of theme and recipient, that need to be scrutinized before we can begin to fully account for areal patterns affecting ‘give’ verbs in West Africa. Perhaps most importantly, the restricted language sample serving the present analysis should be augmented to include other languages in West Africa. Meanwhile though, we can broaden our investigation within existing language groups in order to determine the extent of linkage types and possible points of contact with other genetically related groups. In this paper, we hope to have shown that further inquiry of verbs, their predications, and their linkage types could prove extremely useful to our delineation of areal relations in West Africa, as well as Niger-Congo. These areal relations might then be useful for comparison to genetic relations, which are most often based on lexical features.


\section*{Abbreviations and orthographic conventions}

Orthographic conventions for languages in this paper derive from their respective sources with this additional note on vowels: orthographic underlines, are often used in sources for half open vowels, as in o̠ for [ɔ], but we have chosen to write these with IPA equivalents throughout this chapter. Tone is represented as in the original; if tone is not marked in the original, it is not marked here. Abbreviations used in morphological glosses for grammatical morphemes follow their respective sources.

Abbreviations for the 31 language names in Tables in Section~\ref{sec:8:4} are as follows:

\begin{multicols}{3}
\begin{tabbing}
MM \= Jóola Banjal\kill
AK \> Akan \\
AV \> Avatime \\
BA \> Baule \\
BB \> Bambara \\
BI \> Bini \\
BN \> Bunoge \\
DA \> Dagaare \\
DD \> Dogul Dom \\
DE \> Degema \\
DF \> Diola-Fogny \\
EB \> Ebira \\
EM \> Emai \\
ES \> Esan \\
EW \> Ewe \\
FG \> Fongbe \\
GA \> Ga \\
IB \> Igbo \\
IJ \> Ịjọ \\
JB \> Jóola Banjal \\
KN \> Kana \\
KS \> Kasem \\
LO \> Logba \\
MA \> Mandinka \\
NJ \> Najamba \\
OB \> Obola \\
SU \> Supyire \\
TA \> Tafi \\
VA \> Vata \\
YK \> Yukuben \\
YR \> Yoruba \\
YS \> Yorno So
\end{tabbing}
\end{multicols}

\noindent Abbreviations compiled from examples drawn from other sources, some of which do not following Leipzig Glossing conventions:

\begin{multicols}{2}
\begin{tabbing}
MMMMMMM \= distal demonstrative singular \kill
\GreenSC{1, 2, 3} \> 1st, 2nd, 3rd person \\
\GreenSC{ACC} \>	accusative \\
\GreenSC{ANSG} \>animate singular\\
\GreenSC{APP} \>applicative\\
\GreenSC{AUX, A} \>	auxiliary\\
\GreenSC{CL\#} \>	class number\\
\GreenSC{CM} \>class marker\\
\GreenSC{CMP} \>completive aspect\\
\GreenSC{COMP} \>	complementizer\\
\GreenSC{CONT} \>continuous\\
\GreenSC{CPL} \>common plural prefix\\
\GreenSC{D} \>ditransitive\\
\GreenSC{DDEMSG} \>distal demonstrative \\
				\> singular\\
\GreenSC{DEF} \>definite\\
\GreenSC{DEM} \>demonstrative\\
\GreenSC{DF} \>definite future\\
\GreenSC{FACT} \>factitive\\
\GreenSC{FE} \>factive enclitic\\
\GreenSC{FUT} \>future\\
\GreenSC{GEN} \>genitive\\
\GreenSC{INC} \> inceptive\\
\GreenSC{LK} \>linker\\
\GreenSC{NARR} \>narrative auxiliary\\
\GreenSC{NEG} \>negative\\
\GreenSC{NP} \>noun  phrase\\
\GreenSC{NSP} \>neutralized subject\\
			  \> prefix\\
\GreenSC{OBJ}\>	object\\
\GreenSC{OBL}\>	oblique\\
\GreenSC{PART}\>	particle\\
\GreenSC{PAST}\>	past tense\\
\GreenSC{PERF, PFV,}  \>perfective\\
\GreenSC{PF} \> \\
\GreenSC{PL} \>plural\\
\GreenSC{POS} \>positive\\
\GreenSC{POSS} \>possessor\\
\GreenSC{PREF} \>	prefix\\
\GreenSC{PRX} \>proximal\\
\GreenSC{PSM} \>possessum\\
\GreenSC{R} \>recipient\\
\GreenSC{SBJ} \>	subject\\
\GreenSC{SEC} \>secondative\\
\GreenSC{SG, S} \>singular\\
\GreenSC{SGSCL} \>singular subject clitic\\
\GreenSC{SM} \>subject marker\\
\GreenSC{SPA} \>simple past\\
\GreenSC{SPECI} \>specific\\
\GreenSC{T} \>theme\\
\GreenSC{TAM} \>tense-aspect-modality\\
\GreenSC{TR} \>transitive\\
\GreenSC{V} \>verb\\
\end{tabbing}
\end{multicols}

\section*{Acknowledgments}

An initial draft of this paper was prepared while Schaefer spent a semester at the University of Ghana in Legon and its Department of Linguistics. It was stimulated in part by a question about areal studies in West Africa posed by Masha Koptjevskaja-Tamm, and it benefited from encouragement by Samuel Obeng. While we thank these individuals for their inspiration, data presentation and interpretation remain our responsibility.

%\section*{Contributions}
%John Doe contributed to conceptualization, methodology, and validation.
%Jane Doe contributed to the writing of the original draft, review, and editing.

{\printbibliography[heading=subbibliography,notkeyword=this]}
\end{document}
