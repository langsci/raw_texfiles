\documentclass[output=paper,colorlinks,citecolor=brown]{langscibook}
\ChapterDOI{10.5281/zenodo.11091839}

\author{Abbie Hantgan\affiliation{CNRS-LLACAN} and Christopher R. Green\affiliation{Syracuse University} and Leonardo Contreras Roa\affiliation{Université de Picardie Jules Verne}}
\title{Coerced weight and its consequences in Bondu So verbs}
\abstract{Bondu So (Dogon, Mali) displays [ATR] vowel harmony that interacts with root-final consonants. This study provides evidence from five verb paradigms to show that the quality of a stem suffix, and its absence or presence, is determined by a combination of factors including [ATR], sonorancy, and prosody. Preliminary phonetic results show that, on average, root-final sonorants following [+ATR] vowels are longer than those following [−ATR] vowels. This differs from what is reported for other languages, as these consonants are neither inherently moraic, nor do they receive a mora due to their position alone. This finding suggests that, in Bondu So, [+ATR] licenses sonorant moraicity. We argue that these instances of coerced moraicity explain otherwise unexpected patterns of suffixation among these verb paradigms.}

\IfFileExists{../localcommands.tex}{
   \addbibresource{../localbibliography.bib}
   % add all extra packages you need to load to this file

\usepackage{tabularx,multicol}
\usepackage{url}
\urlstyle{same}

\usepackage{listings}
\lstset{basicstyle=\ttfamily,tabsize=2,breaklines=true}

\usepackage{langsci-basic}
\usepackage{langsci-optional}
\usepackage{langsci-lgr}
\usepackage{langsci-osl}
% \usepackage{./langsci/styles/langsci-lgr}
% \usepackage{./langsci/styles/langsci-osl}
% \usepackage{langsci-gb4e}

\usepackage{tikz}
\usetikzlibrary{patterns,calc}
\pgfdeclarepatternformonly{south east lines}{\pgfqpoint{-0pt}{-0pt}}{\pgfqpoint{3pt}{3pt}}{\pgfqpoint{3pt}{3pt}}{
    \pgfsetlinewidth{0.6pt}
    \pgfpathmoveto{\pgfqpoint{0pt}{3pt}}
    \pgfpathlineto{\pgfqpoint{3pt}{0pt}}
    \pgfpathmoveto{\pgfqpoint{.2pt}{-.2pt}}
    \pgfpathlineto{\pgfqpoint{-.2pt}{.2pt}}
    \pgfpathmoveto{\pgfqpoint{3.2pt}{2.8pt}}
    \pgfpathlineto{\pgfqpoint{2.8pt}{3.2pt}}
    \pgfusepath{stroke}}
    
\usepackage{stmaryrd}
\usepackage{wasysym}
\usepackage{multirow}
\usepackage{caption}
\usepackage{subcaption}
\usepackage{mathrsfs}
\usepackage{qtree}

\usepackage{linguex}


   %pminos do not split footnotes
% \interfootnotelinepenalty=10000 %Footnote in Laporte chapters has to be split SN


%\DeclareIndexNameFormat{default}{%
%\nameparts{#1}%
%\usebibmacro{index:name}%
%{\index[names]}%
%{\namepartfamily}%
%{\namepartgiveni}%
% {}% L1
% {}% L2
%{\namepartprefix}% generates spurious space L3
%{\namepartsuffix}% generates spurious space L4
%}

%  {\DeclareIndexNameFormat{default}{%
%     \usebibmacro{index:name}{\index[names]}{#1}{#3}{#5}{#7}}}

%\DeclareIndexNameFormat{default}{%
%  \usebibmacro{index:name}{\sindex[nom]}{#1}{#3}{#5}{#7}}

%\DeclareIndexNameFormat{default}{%
%  \usebibmacro{index:name}{\sindex[person]}{#1}{#3}{#5}{#7}}
%\DeclareIndexNameFormat{default}{%
%\nameparts{#1} \usebibmacro{index:name}{\sindex[person]]}{\namepartfamily}{‌​\namepartgiven}{\nam‌​epartprefix}{\namepa‌​rtsuffix}}

%\newcommand{\smiley}{:)}

%\renewbibmacro*{index:name}[5]{%
%\usebibmacro{index:entry}{#1}%
%{\iffieldundef{usera}{}{\thefield{usera}\actualoperator}\mkbibindexname{#2}{#3}{#4}{#5}}}

% \newcommand{\noop}[1]{}

%remove for final
%\overfullrule=1mm

\newcommand{\tobi}[2]}}
\renewcommand{\S}[1]{\tobi{#1}{\textsc{*}}}

% this volume references
% puts: [this volume]
% already defined: \citetv
%\newcommand{\citepv}[1]{(\citeauthor{#1} \citeyear*{#1} [this volume])}
\newcommand{\citealtv}[1]{\citeauthor{#1} \citeyear*{#1} [this volume]}

%parentheses around example number
\newcommand{\pref}[1]{(\ref{#1})}

% in-text examples

\newcommand{\lnex}[1]{\textit{#1}} %target lang word
\newcommand{\lnlit}[1]{(lit.: `#1')} %literal reading
\newcommand{\lnlat}[1]{(#1)} % latinization
\newcommand{\lntrans}[1]{`#1'} %translation
\newcommand{\lnexl}[2]%
{\lnex{#1}{} \lnlat{#2}} % ex with latinization
\newcommand{\lnexlat}[3]{\lnex{#1}{} \lnlat{#2}{} \lntrans{#3}} % ex with latinization and tranl.

%ch01
\newcommand{\co}[1]{\mbox{\textbf{#1}}}

%ch09

\newcommand{\cyrbulg}[1]{\begin{otherlanguage*}{bulgarian}#1\end{otherlanguage*}}


%ch10
\newcommand{\nlp}{{\small NLP}}
\newcommand{\mwe}{{\small MWE}}
\newcommand{\rae}{{\small RAE}}
\newcommand{\lvc}{{\small LVC}}
\newcommand{\pos}{{\small P}o{\small S}}
%\newcommand{\todo}[1]{ \textcolor{red}{#1} }

%\renewcommand{\labelenumi}{\theenumi}
%\ainamefmt{{vv}{ll}{, ff}{, jj}} % fullname

\newcommand{\biberror}[1]{{\color{red}#1}}

\newcommand{\osenovaitem}{--~}
   %% hyphenation points for line breaks
%% Normally, automatic hyphenation in LaTeX is very good
%% If a word is mis-hyphenated, add it to this file
%%
%% add information to TeX file before \begin{document} with:
%% %% hyphenation points for line breaks
%% Normally, automatic hyphenation in LaTeX is very good
%% If a word is mis-hyphenated, add it to this file
%%
%% add information to TeX file before \begin{document} with:
%% %% hyphenation points for line breaks
%% Normally, automatic hyphenation in LaTeX is very good
%% If a word is mis-hyphenated, add it to this file
%%
%% add information to TeX file before \begin{document} with:
%% \include{localhyphenation}
\hyphenation{
    Beck-man
    Ngu-yen
    back-chan-nel
    back-chan-nels
    mo-not-o-nous
    ste-reo-typ-i-cal
}

\hyphenation{
    Beck-man
    Ngu-yen
    back-chan-nel
    back-chan-nels
    mo-not-o-nous
    ste-reo-typ-i-cal
}

\hyphenation{
    Beck-man
    Ngu-yen
    back-chan-nel
    back-chan-nels
    mo-not-o-nous
    ste-reo-typ-i-cal
}

   \boolfalse{bookcompile}
   \togglepaper[23]%%chapternumber
}{}

\begin{document}
\maketitle

\section{Introduction}
\label{sec-intro}
Bondu So\footnote{Elsewhere in the literature, the language name is written as Bondu-so in error. This paper\il{Bondu So|(} rectifies this issue with the updated orthographic convention of separating the form \textit{so}, meaning `word,' in line with other Dogon language names such as Tommo So and Donno So.} is a Dogon language of central-eastern Mali spoken by approximately 8,000 people. There are an estimated 21 Dogon languages spoken across the Bandiagara Escarpment that runs parallel to the Niger River. Bondu So speakers live in the northwestern quadrant of the cliff range. In Dogon languages, noun and verb stems are composed of a root plus derivational and inflectional suffixes.\footnote{For the purposes of this paper, we define a root as the underlying and most basic form of the word, without any inflectional affixes. Most roots are monosyllabic, but some have frozen derivational suffixes that we consider part of the root. A stem, as we define it for Bondu So, is a root plus inflectional or productive derivational suffixes.} The basic word order is SOV. Dogon languages exhibit fusional-agglutinative, suffixing morphology. Bondu So, in particular, features a robust system of classificatory suffixes on nouns with agreement on adjectives, determiners, and numerals. TAM and person are marked on verbs through suffixation. Of particular interest to this paper is that Dogon languages are subject to \isi{vowel harmony}. Bondu So, in particular, exhibits sometimes opaque patterns of \is{ATR} ATR harmony.\footnote{Throughout this paper, we describe patterns relative to [+ATR] and [−ATR] feature \is{feature binarity} specifications, which suffice for our analytical purposes here. Bondu So vowel harmony has been described elsewhere with unary features under the assumption of featural \isi{privativity} (\citealt{GreenHantgan}).}

Bondu So has two main varieties, Kindige and Najamba, that are described in  \citet{hantgan2013kindigue} and \citet{HeathBS2017}, respectively. This study concentrates on the Kindige variety but includes some comparisons to Najamba where relevant. Pertinent to this investigation is that Heath, in his description of Najamba, analyzes the final vowel of nouns and verbs -- not only in Najamba, but throughout the Dogon languages -- as an integral part of the stem. Hantgan and subsequent studies \citep{HantganDavis, GreenHantgan} interpret the final vowel of a verb stem as either an inflectional suffix or the result of epenthesis. Importantly, per our analysis, it is clear that the quality of these vowels is predictable based on properties of the preceding root. Under Heath's analysis, the language's vowel inventory comprises only the surface inventory [i e ɛ a ɔ o u], and stem-final vowels are not predictable. On the other hand, under Hantgan's analysis, a broader phonemic inventory of ten vowels is proposed as /i ɪ e ɛ a̘ a ɔ o ʊ u/, which she argues is crucial to motivating the \is{abstractness} surface realization of these stem-final vowels from a standpoint of ATR harmony. Key data substantiating this analytical choice are presented in \citet{HantganDavis}.

The current investigation relies on the analysis of root shapes as being either consonant or vowel final; this paper only deals with consonant-final roots. Whereas any consonant may occupy the final position of a noun or verb root, stem-final codas can only be filled by a nasal or a liquid. It is additionally argued that properties of root vowels and root-final consonants determine the stem-final vowel's presence or absence, and its quality and length.

Our goal, in addition to providing an overview of the morphophonological patterning of the interactions between the feature [ATR] and root-final consonants among five Bondu So verb paradigms, is to present a pilot phonetic study aimed at elucidating certain articulatory aspects of the language's vowels and sonorant consonants. Our findings thus far illustrate that, on average, root-final sonorants following a [+ATR] vowel are longer than those following a [−ATR] vowel. We argue that their behavior – correlations between tongue root gesture, sonorant moraicity, sonorant lengthening, and precluded suffixation – supports a phonological account in line with that proposed by \citet{GreenHantgan}, though the current analysis differs in some assumptions about the language's morphology. We appeal to \isi{mora licensing} and prosodic \isi{minimality} in arguing that [+ATR] root vowels license the projection of a mora from a sonorant coda. The presence of this mora, in turn, precludes the expression of an otherwise expected suffix in order to maintain stem size requirements. 

The remainder of this paper is organized as follows: in \sectref{sec-over}, we present an overview of Bondu So Perfective and Past stems which show the interaction between ATR vowel harmony and root-final consonants on stem-final vowels. We complement these stems with data from the so-called \textit{Chaining} stem in phrasal contexts to show how our analysis of moraic sonorants supplements \citeauthor{HeathBS2017}'s \citeyearpar{HeathBS2017} description of \textit{post-sonorant high vowel deletion}. Following this, we augment previously published data with new forms from the Infinitive and Nominative stems in \sectref{sec-epenth} in order to compare the role of suffixal vowels with that of epenthetic vowels in terms of ATR harmony and its interaction with root-final consonants. Next, \sectref{sec-meth} describes the methods used for this study, including the details of the data collection. Following the methodological overview, we present the results from our pilot phonetic study in \sectref{sec-res}. Lastly, in \sectref{sec-diss}, we discuss both theoretical and descriptive implications of our findings, as they apply cross-linguistically as well as specifically to other Dogon languages.

\section{Overview}
\label{sec-over}

\subsection{Previous studies}
\label{subsec-prev}

Two formal studies \citep{HantganDavis, GreenHantgan} detail aspects of Bondu So vowel harmony. Here, we provide only a basic overview of the hitherto described patterns using two verb stems: the perfective aspect and past tense in \sectref{subsec-pfv} and~\sectref{subsec-pst}, respectively. These are presented first as they display similar properties and illustrate root-controlled ATR harmony wherein the value of the root vowel spreads to the suffix.

\subsection{Perfective stems}
\label{subsec-pfv}

Examples of Perfective stems with simplified glosses are provided in \tabref{tab:pfv:obs}. These are representative of the 3\textsuperscript{rd} person, which is marked by a -VV suffix in all instances.\footnote{By this interpretation, we diverge from \citeauthor{HeathBS2017}'s \citeyearpar[10]{HeathBS2017} analysis of Najamba wherein he states that the 3\textsuperscript{rd} person singular perfective aspect is a ``zero suffix.''} The quality of the perfective suffix vowel is \textit{front} in the 3\textsuperscript{rd} person singular and \textit{back} in the 3\textsuperscript{rd} person plural.

\begin{table}
\caption{Perfective 3\textsuperscript{rd} person stems with obstruent-final roots}
\label{tab:pfv:obs}
\begin{tabular}{ *2{l@{~~}lll} } 
    \lsptoprule
    \multicolumn{4}{c}{+ATR}&\multicolumn{4}{c}{−ATR}\\\cmidrule(lr){1-4}\cmidrule(lr){5-8}
    & singular & plural & Gloss & & singular & plural & gloss \\
    \midrule
    a. 	&	bèdʒ-éè	& bèdʒ-óò &	bury	&	aa.	& nɛ̀ɡ-ɛ́ɛ̀	& nɛ̀ɡ-ɔ́ɔ̀ &	lick	\\
    b.	&	póɡ-èè	& póɡ-òò &	dump	&   bb.	& dɔ̀ɡ-ɛ́ɛ̀	& dɔ̀ɡ-ɔ́ɔ̀ & abandon	\\\addlinespace
    c.	&	íb-èè	& íb-òò &	take	&   cc.	& bìmb-ɛ́ɛ̀	& bìmb-ɔ́ɔ̀ &	scrub	\\
    d.	&	kúmb-èè	& kúmb-òò  &  hold 	&   dd.	& ɡùb-ɛ́ɛ̀	& ɡùb-ɔ́ɔ̀  &	hang   \\\addlinespace
    e.	&	páɡ-èè	& páɡ-òò  &  tie 	&   ee.	& dʒàmb-ɛ́ɛ̀	& dʒàmb-ɔ́ɔ̀  &	betray   \\
    \lspbottomrule
\end{tabular}
\end{table}

The stems in \tabref{tab:pfv:obs} illustrate pertinent aspects of the Bondu So morphophonological system. First, the Perfective suffix is realized by four \is{allomorphy} allomorphs whose tone and quality are dictated by characteristics of the root to which it attaches. Kindige Perfective stems are susceptible to depressor \is{depressor effects} consonants, and thus, an otherwise expected HL tonal melody is realized as LHL when the root begins with a voiced consonant (whether obstruent or sonorant). Root shapes may be VC, CVC, or CVNC, where an initial consonant may be a either a voiced or voiceless obstruent or sonorant. Stem-internal (i.e., root-final) consonants must be voiced.\footnote{It is likely that some roots have underlyingly voiceless consonants that surface voiced intervocalically in verb stems.} Recall that the language is SOV, and thus, the Perfective stem occurs phrase-finally. \citet{HantganDavis} argue that the quality of the Perfective suffix is determined by ATR harmony. More specifically, it is dictated by the [$\pm$ATR] specification of the root vowel which, in cases of underlyingly [−ATR] high (\tabref{tab:pfv:obs}cc--dd) and [+ATR] low (\tabref{tab:pfv:obs}e) vowels, is opaque.

Perfective stems with sonorant-final roots, like those in \tabref{tab:pfv:son}, realize different outcomes compared to obstruent-final roots. Precisely, in Kindige, nasal-final roots with [+ATR] vowels have high vowel suffixes in the Perfective. In Najamba, all stems pattern like those shown for Kindige in Table~\ref{tab:pfv:obs}, as do liquid- (\tabref{tab:pfv:son}d--dd) and glide-final (\tabref{tab:pfv:son}e--ee) roots in both varieties. \citet{GreenHantgan}, in their \is{feature geometry} feature geometric approach to Bondu So vowel harmony, attribute Perfective stem suffixal vowel raising in nasal-final roots to feature spreading. In their approach, nasals, unlike most other consonants, possess the vocalic height feature [closed], which licenses [ATR] spreading from the root to the nasal. A saturated Height node -- specified for both [\textsc{closed}] and [ATR] -- thereafter spreads in its entirety to suffixal vowels, resulting in high, [+ATR] vowels.

\begin{table}
\caption{Perfective 3\textsuperscript{rd} person stems with sonorant-final roots}
\label{tab:pfv:son}
\begin{tabular}{ *2{l@{~~}lll} } 
    \lsptoprule
    \multicolumn{4}{c}{+ATR}&\multicolumn{4}{c}{−ATR}\\\cmidrule(lr){1-4}\cmidrule(lr){5-8}
    & singular & plural & gloss & & singular & plural & gloss \\
    \midrule
    a.	&	ɡòm-íì	&	ɡòm-úù	&	remove	&	aa.	&	ɡɔ̀m-ɛ́ɛ̀	&	ɡɔ̀m-ɔ́ɔ̀	&	reek	\\
    b.	&	mı̀n-íì	&	mı̀n-úù	&	wait	&	bb.	&	mìn-ɛ́ɛ̀	&	mìn-ɔ́ɔ̀	&	swallow	\\
    c.	&	dààn-íì	&	dààn-úù	&	grill	&	cc.	&	dʒàŋ-ɛ́ɛ̀	&	dʒàŋ-ɔ́ɔ̀	&	study	\\\addlinespace
    d.	&	pór-\textbf{èè}	&	pór-\textbf{òò}	&	let escape	&	dd.	&	bɛ̀l-ɛ́ɛ̀	&	bɛ̀l-ɔ́ɔ̀	&	pick fruit	\\
    e.	&	ɡìj-\textbf{éè}	&	ɡìj-\textbf{óò}	&	dance	&	ee.	&	ɡìj-ɛ́ɛ̀	&	ɡìj-ɔ́ɔ̀	&	kill	\\
    \lspbottomrule
\end{tabular}
\end{table}

Although not all root types are shown in Table~\ref{tab:pfv:son}, any nasal /n m ŋ/ found in the language, preceded by any [+ATR] vowel /i e a̘ o u/ takes the high vowel allomorph of the Perfective suffix. Liquids /r l/ and the glide /j/ pattern with other obstruents in having no raising effect on the final vowel of the Perfective stem. The Past stem, on the other hand, witnesses different outcomes, as detailed in the following subsection. 

\subsection{Past stems}
\label{subsec-pst}
A second verb stem that demonstrates interactions between root vowels and consonants, and suffixal vowels is found in the Past tense.\footnote{We agree with \citet[220]{HeathBS2017} who considers the stem of the Past tense paradigm (which he calls the ``Past perfect'') to be a combination of the ``Chaining'' stem plus the clitic \textit{=b-}, where person and tense are indicated through a VV suffix on the clitic.} The Past tense is formed by a verb root, and sometimes a suffixal vowel whose patterning resembles, in some ways, that of the Perfective stem but diverges from it in others. These divergences provide insight into the finer details of the morphophonology of Bondu So. Of particular interest is the behavior of [+ATR] vowel roots with sonorant codas, as they surprisingly fail to appear with the Past tense suffix. 

Representative Past stem examples from the same roots shown above for the Perfective are given in their 3\textsuperscript{rd} singular and plural forms in \tabref{tab:pst}. Note that, due to space restrictions, the clitic \is{cliticization} that follows each Past stem is omitted: the 3\textsuperscript{rd} singular Past tense clitic is consistently [=bɛɛ], and the plural is [=bɔɔ], with tonal melodies overlaid as in the Perfective stem. Therefore, to illustrate, `s/he buried' is realized as \textit{bèdʒ-í=bɛ̀ɛ̀}, whereas `they removed' would be \textit{ɡòm=bɔ́ɔ̀}.\footnote{A reviewer asks whether these sonorant moras are a result of the preservation of a lost suffixal vocalic mora. We view the  suffixal vowel in these instances as an allomorph of the Past morpheme that appears only when an underlying [−ATR] root vowel cannot license a mora on a sonorant coda. As such, mora insertion, rather than preservation, is better in line with this analysis.}

\begin{table}
\caption{Past 3\textsuperscript{rd} person stems}
\label{tab:pst}
\begin{tabular}{ *2{l@{~~}lll} } 
    \lsptoprule
    \multicolumn{4}{c}{+ATR}&\multicolumn{4}{c}{−ATR}\\\cmidrule(lr){1-4}\cmidrule(lr){5-8}
    & singular & plural & gloss & & singular & plural & gloss \\
    \midrule
    a. 	&	bèdʒ-í	& bèdʒ-ú &	bury	&	aa.	& dʒàmb-ɛ́	& dʒàmb-ɔ́  &	betray	\\
    b.	&	nòj-í	&	nòj-ú	&	sleep	&	bb.	&	ɡìj-ɛ́	&	ɡìj-ɔ́	&	kill	\\\addlinespace
    c.	&	dààn	&	dààn	&	grill	&	cc.	&	dʒàŋ-ɛ́	&	dʒàŋ-ɔ́	&	study	\\
    d.	&	ɡòm	&	ɡòm	&	remove	&	dd.	&	ɡɔ̀m-ɛ́	&	ɡɔ̀m-ɔ́	&	reek	\\
    e.	&	pór	&	pór	&	let escape	&	ee.	&	bɛ̀l-ɛ́	&	bɛ̀l-ɔ́	&	pick fruit	\\
    \lspbottomrule
\end{tabular}
\end{table}

The Past stem deviates from the Perfective stem in that it is non-phrase final, and so the addition of the Past tense clitic involves three notable differences. First, phrase-final tone lowering no longer applies, and thus each stem-final vowel (where these occur) carries a High tone. Furthermore, whereas the quality of the stem-final vowel was raised to high only after nasal-final roots with [+ATR] vowels in the Perfective, in the Past stem, it is obstruent-final roots following [+ATR] vowels that realize a raised suffixal vowel.\footnote{According to \citet{GreenHantgan}, the difference in behavior, as compared to the Perfective, rests in the featural specification of the suffixal vowel involved.} Finally, and perhaps most strikingly: nasal- and liquid-final roots containing a [+ATR] vowel take no suffix.

\subsection{Chaining stems}
\label{subsec-chn}

We follow \citet{HeathBS2017} in labeling the verb stem which is used in verb chains as the \textit{Chaining stem}. The Chaining stem itself is identical to the 3\textsuperscript{rd} person singular Past stem; \citet[220]{HeathBS2017} considers them one and the same. The only difference between them is that the Past stem is followed by the \is{cliticization} clitic \textit{=b-}, which is inflected for person. The Chaining stem, however, must be followed by another verb on which inflection is marked. Thus, the complete Chaining paradigm involves a verb root, suffixed with a short vowel, and then followed by an additional verb. Example sentences for verbs with [+ATR] and [−ATR] vowels are provided in (\ref{ex-able-sleep}) and (\ref{ex-able-study}), respectively, followed by the verb `can.' The Chaining stems themselves are identical to those shown for the 3\textsuperscript{rd} person singular in \tabref{tab:pst} above.
%page 365 NJ grammar as FUT for 17.5.3 Capacitative constructions - always in FUT rather than PRES

\begin{multicols}{2}
\ea \label{ex-able-sleep}
\gll nòj-í                           dʒá-mbò-m\\
     sleep-\textsc{chn} can-\textsc{fut}-\textsc{1sg}\\
\glt `I can sleep.'
\z
\ea \label{ex-able-study}
\gll dʒàŋɡ-ɛ́                           dʒá-mbò-m\\
     study-\textsc{chn} can-\textsc{fut}-\textsc{1sg}\\
\glt `I can study.'
\z
\end{multicols}
\il{Bondu So}

As with the Past stem, no final vowel suffix is present following sonorants in [+ATR] contexts, in both varieties of Bondu So. Examples of the Chaining stem in phrasal contexts illustrating minimal pairs of nasal-final roots are shown in (\ref{ex-able-wait}) and (\ref{ex-able-swallow}).

\begin{multicols}{2}
\ea \label{ex-able-wait}
\gll mìn                           dʒá-mbò-m\\
     wait.\textsc{chn} can-\textsc{fut}-\textsc{1sg}\\
\glt `I can wait.'
\z
\ea \label{ex-able-swallow}
\gll mìn-ɛ́                           dʒá-mbò-m\\
     study-\textsc{chn} can-\textsc{fut}-\textsc{1sg}\\
\glt `I can swallow.'
\z
\end{multicols}
\il{Bondu So}

\citet[34]{HeathBS2017} attributes the absence of the final vowel in these stems to a rule of \textit{post-sonorant high vowel deletion}. While this is, of course, a descriptively accurate statement of the process, it begs the question of why high vowels would be susceptible to deletion following sonorants in the first place. It also does not address the predictable correlation between these deletions and the quality of the root vowel that we address here.

\citet{GreenHantgan} offer one possible solution to these alternations in Past and Chaining stems that is based on prosodic \isi{minimality}. Crucially, they assume that verb roots can be CVC-shaped, but that stems must be minimally bimoraic. In most instances, they argue, bimoraicity is achieved via the addition of an epenthetic vowel, the quality of which is determined in part by featural characteristics of the root vowel and root-final consonant, in a way that is analogous but not identical to outcomes for the Perfective. Epenthesis is said to be required after all obstruent-final roots and also after sonorant-final roots containing a [−ATR] vowel. The exceptional cases, as we have shown, are sonorant-final roots containing a [+ATR] vowel. They further contend that the reason for the divergent behavior in suffix-less stems rests in the ability of [+ATR] root vowels to license the projection of a mora from the sonorant coda. This mora, in turn, satisfies the minimality condition, thus obviating the epenthetic vowel.

While certain analytical assumptions that we adopt below differ from earlier work on Bondu So \citep{GreenHantgan, HantganDavis}, we agree in principle with Green \& Hantgan's argument for a prosodic basis to these Past and Chaining stem alternations. The data presented below are from a pilot acoustic study designed to test the proposal that coda liquids and nasals are moraic following [+ATR] vowels. The data reveal that, on average, root-final sonorants following [+ATR] vowels are significantly longer than those following [−ATR] vowels. We take this as promising evidence in support of the hypothesis that, in Bondu So, the [+ATR] feature indeed \is{mora licensing} licenses sonorant moraicity.

\section{Epenthetic vowels}
\label{sec-epenth}

In this paper, although we disagree with \citet{GreenHantgan} concerning their proposal that Past and Chaining stems involve an epenthetic vowel, we nonetheless find evidence for such vowels elsewhere in Bondu So. More specifically, we find alternations in two verbal contexts -- the Infinitive and Nominative stems -- that are more clearly epenthetic in nature. 

To begin, Infinitive and Nominative stems involve regressive spreading such that the vowels of the verb stem surface [+ATR], making these stand out relative to other contexts discussed thus far.\footnote{The Infinitive stem label is somewhat speculative and its suitability will be the subject of future research. Correlates of what we refer to as the ``Infinitive'' suffix are not well represented among descriptions of other Dogon languages. The only instance of this suffix, or one like it, mentioned in the literature is in \citet[120]{Culy1994} for \ili{Donno So}. However, there, the suffix is described alternatively as a participle or an infinitive.} In other ways, however, the behavior of the Infinitive and Nominative stems resembles that of the Perfective and Past stems. Both stems are followed by a \is{cliticization} clitic: \textit{=loŋ} $\sim$\textit{=doŋ} and \textit{=le} $\sim$\textit{=de}, respectively. All verb roots with final obstruents are followed by a vowel, which is [i] in all instances. Roots ending in sonorants differ somewhat -- a vowel is present in most instances (78\%) of the Infinitive stems, and is absent in most instances (24\%) of the Nominative stems in our dataset. In some instances, the same verb was realized with both an epenthetic vowel and without one, even for the same speaker at different tokens in our dataset. 

Despite this \isi{variation}, what is important for our purposes here is that when a vowel is present, it is always [i]. Recall that this unified vowel quality differs from the Perfective, Past, and Chaining forms, where the height of stem-final vowels (where present) alternated depending on the characteristics of the root. Examples of sonorant-final roots across vowel heights in Infinitive and Nominative stems are provided in \tabref{tab:inf-nom:VS}.

\begin{table}
\small
\caption{Sonorant-final roots in Infinitive and Nominative stems}
\label{tab:inf-nom:VS}
\begin{tabular}{l@{~~}lll l@{~~}lll} 
\lsptoprule
\multicolumn{4}{c}{+ATR}&\multicolumn{4}{c}{−ATR}\\\cmidrule(lr){1-4}\cmidrule(lr){5-8}
& Infinitive & Nominative & Gloss & & Infinitive & Nominative & Gloss \\\midrule
a.	&	ɡóm=dòŋ	&	ɡóm=dè	&	remove	&	aa.	&	kún-í=lòŋ	&	kún=dè	&	fatten	\\
b.	&	némbil=lòŋ  	&	némbíl=lè	&	beg	&	bb.	&	bél=lòŋ	&	bél=lè	&	pick fruit	\\
c.	&	ín-í=lòŋ	&	ín=dè	&	go	&	cc.	&	mín-í=lòŋ	&	mín=dè	&	swallow	\\
d.	&	bár=lòŋ	&	bár=lè	&	help	&	dd.	&	sár=lòŋ	&	sár=lè	&	ask	\\
\lspbottomrule
\end{tabular}
\end{table}

We believe it is pertinent that these stems with [+ATR] vowels and sonorant-final roots in some ways resemble what occurs in the Past and Chaining stems: a stem-final sonorant coda is permissible, but only following a [+ATR] vowel. Yet, in the Past and Chaining stems, the motivation for the appearance of a vowel following obstruent-final roots is categorical and clear -- to prevent impermissible consonant contact. What is not immediately apparent, however, is what drives the presence or absence of the epenthetic vowel among permissible clusters \is{phonotactics} such as those shown for the Infinitive and Nominative stems in \tabref{tab:inf-nom:VS}, for example (\ref{tab:inf-nom:VS}a) and (\ref{tab:inf-nom:VS}d) versus (\ref{tab:inf-nom:VS}c) and (\ref{tab:inf-nom:VS}cc).\footnote{Additionally, although only alveolar nasals are shown here, each instance of a velar nasal in the coda of the root also takes an epenthetic vowel across both paradigms.} One possibility is that the tendency of Infinitive stems to take an epenthetic vowel even after some sonorants may be due to the CVC shape of the clitic, the presence of which may lead to a dispreferred phonotactic sequence, or perhaps relatedly, to a problematic prominence clash. This would not be an issue for the Nominative. We leave the matter of what conditions and presence vs. absence of the epenthetic vowel in Infinitive and Nominative stems to future research. 

\section{Methodology}
\label{sec-meth}

The data upon which this study is based were gathered by the first author from 2008--2010 in Douentza, Mali, for the purpose of investigating acoustic correlates of [ATR]. They represent the productions of one male speaker, who was between the ages of 30 and 40 (specific ages among the Dogon are not calculated) at the time of recording.\footnote{This speaker was the only person available and able at the time to travel to the city to make the recordings in a relatively quiet setting with electricity. Future studies will be based on a larger group of speakers including female speakers as well as males.} Verbs were elicited from a list drawn from a dictionary of the language. Recordings were made in a quiet room of a home with a Marantz handheld recorder and microphone.

All of the data are available in the \href{https://bang.huma-num.fr/ressources.html}{Supplemental Materials}. The present study is focused on verb stems: the Perfective, Past, and Chaining stems, which all consist of a verb root plus a suffixal vowel, as well as the Infinitive and Nominative stems, which are formed by a verb root and an epenthetic vowel. The details of each stem are discussed above in \sectref{sec-over} and \sectref{sec-epenth}.

For the phonetic portion of the study, a total of 308 verb tokens were analyzed for root-vowel quality; and then a subset of 294 sonorant-final verb tokens from across 5 paradigms (\textsc{pfv, pst, chn, inf, nom}) were considered, all produced by the one, previously-mentioned, male speaker consulted for this study.\footnote{The other 14 stems were obstruent-final.} Verbs were phonetically and phonemically transcribed and manually aligned in Praat \citep{Praat2022}. Vowels in roots were transcribed as /a̘, e, i, o, u/ [+ATR] and /a, ɛ, ɪ, ɔ, ʊ/ [−ATR]. Sonorants in the dataset include the nasals and liquids [m n ŋ l r]. Each verb stem was annotated at the level of the phoneme, root, stem, and gloss.

An ad-hoc Praat script was created to extract the following acoustic data from root-final sonorants and their preceding vowels (e.g., from both /o/ and /m/ in /ɡom/ `remove', and from both /ɪ/ and /l/ in /ambɪl/ `lower price'):
\begin{itemize}
    \item duration (s)
    \item F1, F2 and F3 values (Hz)
    \begin{itemize}
        \item point values at 25\%, 50\% and 75\% of the of the segment
        \item mean values throughout the entire segment
        \end{itemize}
    \item center of gravity (Hz)
    \begin{itemize}
    \item with a 50--11000\,Hz Hann band filter (100\,Hz smoothing)
    \item taken from the central 20\% portion of the segment
    \end{itemize}
\end{itemize}

Aside from being the acoustic correlate of the articulatory feature [$\pm$\textsc{open}], we were particularly interested in F1 because it has been identified as one of the main acoustic cues associated with ATR \citep{hess1992assimilatory, koffi2016acoustic, olejarczuk2019acoustic}. These studies identify other less statistically reliable phonetic factors that can be related to ATR as well, such as F1 bandwidth and voice quality (e.g., creaky or breathy voice). The latter can be a physiological consequence of tongue root lowering or raising but is not always a reliable factor on its own, since vocal fold tension is not solely dependent on tongue root position. The same applies to other measures such as the first harmonics H1--H2 \citep{yang2021measuring}, which we did not measure since we wanted to concentrate on the resonance aspects of advanced tongue root position.

Additionally, we chose to analyze \isi{center of gravity} (COG) -- the spectral mean~-- of root-final sonorants. Although this measure is usually studied as an acoustic correlate for the place of articulation of fricatives \citep{figueroa2021}, studies on ATR harmony have used it as well, as it can account for variations in vocal tract resonances in cases where there is overlap in representations of the acoustic space of vowels that rely only on F1 and F2 \citep{anderson2007, kingston1997}. In other words, as a result of the interaction between [\textsc{open}] and [ATR], F1 means can sometimes be neutralized and result in similar values in spite of differences in tongue root position. COG data can provide a bigger picture of the distribution of frequencies and their intensities across the whole spectrum.

\begin{sloppypar}
In order to target spectral differences associated to vowel resonances, we adapted the COG extraction method by applying a Hann band filter from 50\,Hz to 11000\,Hz. Articulatorily speaking, lower COG values account for bigger resonance spaces inside of the vocal tract, which can be achieved by the modification of tongue root position, as interpreted phonologically as [$\pm$ATR], but also by other means such as larynx lowering or spreading of the faucial pillars.
\end{sloppypar}

Finally, we calculated \isi{Pillai scores} to determine how these acoustic measures overlap in [$\pm$ATR] vowel pairs. The Pillai score, also known as the Pillai-Bartlett Trace \citep{pillai1955}, is a statistical measure that is used to evaluate the overlap between two populations. The score ranges from 0 to 1, where 0 indicates complete overlap (the two populations are identical), and 1 indicates no overlap at all (the two populations are completely distinct). In phonetics, this measure has often been used to assess the overlap between two vowel classes to determine whether pairs of vowels can be considered as merged or split \citep{freeman2023production, kelley2020comparison, mairanoetal2019}. The Pillai score is calculated using a Multivariate Analysis of Variance (MANOVA) test. This test takes at least two continuous dependent variables and evaluates whether they come from the same distribution in a multivariate space. In the context of vowel analysis, these dependent variables could be different measures of the vowel characteristics such as frequency (F1 and F2), duration, or center of gravity (COG). The \texttt{manova()} function in R is typically used to perform this test, and the Pillai score is part of the output from this function when a summary statistic is requested.

\section{Results}
\label{sec-res}

\subsection{Acoustic analyses}
\label{subsec-acous}

In this section, we report the acoustic analyses of the dataset of word-list recordings to determine if there are durational, spectral, or other phonetic cues associated with ATR that may provide insight into its apparent ability to license mora projection from root-final sonorants.

\subsection{Vowels}
\label{subsec-vow}

Formant data show the surface differences between [+ATR] and [−ATR] vowels among Perfective, Past, and Chaining stems as illustrated in \figref{fig:vowel-space}. Infinitive and Nominative stems were excluded from the vowel space analysis because they are all [+ATR] due to spreading from their respective clitics. Further, for this part of the study, both obstruent and sonorant-final verb roots were incorporated. 

The \isi{phonetic merger} between the high vowels (front and back) /ɪ, ʊ/ to [i,~u] appears complete, whereas the mid vowels [ɛ, ɔ] and [e, o] remain divergent between [+ATR] and [−ATR] qualities. The strikingly fronted [u] tokens in the chart are explored further below. Unfortunately, our dataset contains only two roots with the [+ATR] mid front vowel [e]. However, the mid back vowels illustrate that the contrast still exists between [+ATR] and [−ATR] vowels, which can be compared with the high back vowels. 

\begin{figure}
    \centering
    \includegraphics[scale=0.75]{figures/80-chn-pfv-pst_vowel-space.pdf}
    \caption{F1 vs F2 Vowel space of CHN, PFV, and PST paradigms by [ATR] specification}
    \label{fig:vowel-space}
\end{figure}

On the other hand, acoustic qualities for the low vowels /a̘, a/ to [a], are somewhat more variable, but nonetheless, their merger appears complete on the F1/F2 dimension. This can be seen in the synthetic chart in \figref{fig:vowel-space-means} which more clearly illustrates the differences in vowel spaces in terms of means and standard error.

\begin{figure}
    \centering
    \includegraphics[scale=0.75]{figures/80-chn-pfv-pst_vowel-space_means.pdf}
    \caption{F1 vs F2 Vowel space of CHN, PFV, and PST paradigms by [ATR] specification, means and standard error}
    \label{fig:vowel-space-means}
\end{figure}

\figref{fig:vowel-space-means} shows that the high vowels have merged on the surface, even though, based on the phonological patterning in Bondu So, the \is{abstractness} underlyingly contrast between these vowels still plays a role in the language. The low vowel is merged between the [+ATR] and [−ATR] qualities to surface only as [−ATR]. In [+ATR] conditions, F1 values are overall slightly higher, which coincides with the acoustic features of an open vowel, whereas in the considerably less common [−ATR] condition in our dataset, F1 values resemble those of a mid or open-mid vowel [ɐ~{\sim}~ə].\footnote{Most -- if not all, assuming that some loanwords have yet to be detected -- of the occurrences of [−ATR] low vowels in the dataset were found in loanwords.} 

Another interesting finding is the broad distributions of [u, ʊ] across the F2 axis, as represented by the wide horizontal error bars in \figref{fig:vowel-space-means}. We initially hypothesized that this could have been on account of noise in the recording or of problems during data extraction: formant identification algorithms sometimes wrongly interpret F1 and F2 peaks in close proximity across the spectrum and merge them. This might result in F2 interpreted as F1, F3 interpreted as F2, and so on. However, auditory and spectral slice observation of the extreme occurrences of these vowels, [u] in \textit{tún-ìì} `s/he put' and [ʊ] in \textit{ɡùb-ɛ́ɛ̀} `s/he hung,' confirm clear perceptive fronting of [u] which could be heard as [y] in \textit{tún-ìì}. 

As shown in \figref{fig:u-fronting}, formant peaks were distinct and had not been incorrectly merged. Whether this fronting is the effect of a [+ATR] articulation or of a regressive assimilation of place of articulation towards the locus of the subsequent /n/ in \textit{tún} needs to be determined with further analyses and comparable data.

\begin{figure}[h]
    \centering
    \includegraphics[scale=0.75]{figures/tunii_gUbEE.pdf}
    \caption{Spectral slices showing F1 and F2 for fronted [u{$\sim$}y] in tún-ìì (top) and [ʊ] in ɡùb-ɛ́ɛ̀ (bottom)}
    \label{fig:u-fronting}
\end{figure}

Duration values for \is{vowel length} individual vowels are reported in \figref{fig:vowel-duration}. Marked differences are observed in front vowel pairs, while all other vowel pairs display similar duration values. These preliminary data suggest that the target acoustic correlate which differentiates the [+ATR] and [−ATR] vowels in Bondu So is not length, however future studies will consider a broader range of data to gain a more complete picture from the range of vowel features.

Finally, COG \is{center of gravity} data, illustrated in \figref{fig:vowel-cog}A, show overall lower COG values in [+ATR] \is{ATR} vowel spectra, which is consistent with the articulatory features of advanced tongue root position: slower frequencies are reinforced due to there being a larger resonance space inside of the vocal tract. This results in a lower spectral mean.
\pagebreak

\begin{figure}[H]
    \centering
    \includegraphics[scale=0.75]{figures/90-vowel-duration.pdf}
    \caption{$\pm$ATR Vowel durations across all paradigms}
    \label{fig:vowel-duration}
\end{figure}\largerpage[2]

\begin{figure}[H]
    \centering
    \includegraphics[scale=0.75]{figures/80-cog-vowels-dark.pdf}
    \caption{Center of gravity of root vowels by [ATR], means and standard error}
    \label{fig:vowel-cog}
\end{figure}\pagebreak

Results by vowel in \figref{fig:vowel-cog}B show the individual contribution of each [$\pm$ATR] vowel pair to this difference: although most vowel pairs have the expected higher COG values in the [−ATR] condition, the pairs [a, a̘] and [ʊ, u] have the highest \is{contrast preservation} contrast. The only pair that shows the opposite trend, i.e. higher COG values in the [+ATR] condition, is [o, ɔ]. However, these vowels show a marked difference in vowel space with no overlap in terms of F1 and F2, as shown in \figref{fig:vowel-space-means} above.

As described in the methodology, the COG \is{center of gravity} data provide a broader perspective on the distribution of frequencies and their intensities across the entire spectrum, capturing spectral differences associated with vowel resonances. Notably, the pairs /a, a̘/ and /ʊ, u/ exhibited the highest contrast in COG values despite showing no marked contrasts in F1 and F2 values (\figref{fig:vowel-space-means}). These results are of interest to the interpretation of the Bondu So vowel inventory as our phonological analysis posits a contrast between /a, a̘/ and /ʊ, u/, as well as between /ɪ, i/, despite these vowels exhibiting F1/F2 mergers, as seen above. It may be that such subtle acoustic cues, as contributed by COG, contribute to the maintenance of a contrast between these vowels. This matter will be of key importance in our future research.

The COG values of vowels in the dataset are of further relevance to this study for two reasons: first, these results support our proposition that the feature implicated in the phenomena under study may reasonably be [ATR]. Secondly, as is discussed in the following section, the COG of root-final sonorants \textit{agrees} with the [ATR] value of the preceding vowel: the COG of root-final sonorants in the [+ATR] condition is lower than that of the [−ATR] condition. Taken together, these findings would appear to support a phonological analysis involving the spreading of the [ATR] feature, not only from vowels to other vowels in the stem, but also to sonorants. 

In summary, we calculated MANOVA tests to obtain the \isi{Pillai scores} for all vowel pairs so that we could determine to what extent they overlap. We added dependent variables progressively to assess the cumulative contribution of each one, starting with F1 and F2 (\tabref{tab:f1-f2}), then adding duration (\tabref{tab:f1-f2-duration}) and then adding COG (\tabref{tab:f1-f2-duration-cog}). These tables show the Pillai scores for different vowel pairs with associated degrees of freedom (Df), F-statistics (approx F), numerator degrees of freedom (num Df), denominator degrees of freedom (den Df), and $p$-values (Pr(>F)).

\begin{table}
\begin{subtable}{\textwidth}
\centering
\caption{Pillai scores for (F1, F2) \textasciitilde ATR}
\begin{tabular}{lc S[table-format=1.7] S[table-format=4.5] c r S[table-format=1.7{***}]}
\lsptoprule
Vowel pair & {Df} & {Pillai} & {approx F} & {num Df} & {den Df} & {Pr(>F)} \\
\midrule
a, a̘ & 1 & 0.013126 & 0.53202 & 2 & 80 & 0.5895 \\
e, ɛ & 1 & 0.42747  & 4.1065  & 2 & 11 & 0.04655 {*} \\
i, ɪ & 1 & 0.0049833 & 0.09766 & 2 & 39 & 0.9072 \\
o, ɔ & 1 & 0.73795  & 15.489  & 2 & 11 & 0.0006325 {***} \\
u, ʊ & 1 & 0.054655 & 0.26017 & 2 & 9  & 0.7765 \\
\lspbottomrule
\end{tabular}
\label{tab:f1-f2}
\end{subtable}
\medskip\\
\begin{subtable}{\textwidth}
\centering
\caption{Pillai scores for (F1, F2, duration) \textasciitilde ATR}
\begin{tabular}{lc S[table-format=1.7] S[table-format=4.5] c r S[table-format=1.7{***}]}
\lsptoprule
Vowel pair & {Df} & {Pillai} & {approx F} & {num Df} & {den Df} & {Pr(>F)} \\
\midrule
a, a̘ & 1 & 0.015991 & 0.42794 & 3 & 79 & 0.7335 \\
e, ɛ & 1 & 0.51257  & 3.5052  & 3 & 10 & 0.05731 \\
i, ɪ & 1 & 0.30903  & 5.6651  & 3 & 38 & 0.002615 {**} \\
o, ɔ & 1 & 0.74097  & 9.5353  & 3 & 10 & 0.002794 {**} \\
u, ʊ & 1 & 0.056648 & 0.16013 & 3 & 8  & 0.9202 \\
\lspbottomrule
\end{tabular}
\label{tab:f1-f2-duration}
\end{subtable}
\medskip\\
\begin{subtable}{\textwidth}
\centering
\caption{Pillai scores for (F1, F2, duration) \textasciitilde ATR}
\begin{tabular}{lc S[table-format=1.7] S[table-format=4.5] c r S[table-format=1.7{***}]}
\lsptoprule
Vowel pair & {Df} & {Pillai} & {approx F} & {num Df} & {den Df} & {Pr(>F)} \\
\midrule
a, a̘ & 1 & 0.065554 & 1.368 & 4 & 78 & 0.2527 \\
e, ɛ & 1 & 0.52434  & 2.4803  & 4 & 9 & 0.1186 \\
i, ɪ & 1 & 0.31339  & 4.2219  & 4 & 37 &  0.006479 {**} \\
o, ɔ & 1 & 0.74691  & 6.6403  & 4 & 9 & 0.009001  {**} \\
u, ʊ & 1 & 0.10446 & 0.20413 & 4 & 7  & 0.9282 \\
\lspbottomrule
\end{tabular}
\label{tab:f1-f2-duration-cog}
\end{subtable}
\caption{Pillai scores for various factors \textasciitilde ATR}
\label{tab:main}
\end{table}


The highest \isi{Pillai scores} were obtained for vowel pairs /o, ɔ/ and /e, ɛ/, respectively. They rise slightly as the variables' duration and COG are added to the MANOVA analysis, but overall they remain constant. This means that, all variables considered, /o, ɔ/ show little overlap, i.e. they are two distinct vowels acoustically speaking, and /e, ɛ/ show a slight overlap. These values are statistically significant for both vowel pairs in \tabref{tab:f1-f2}, but retain their significance only for /o, ɔ/ in the remaining two tables. The only Pillai score that increases substantially when adding duration (\tabref{tab:f1-f2-duration}) and COG (\tabref{tab:f1-f2-duration-cog}) is that of /i, ɪ/, and significantly so. However, the 0.3 Pillai score it attains is rather low and implies a high degree of overlap between these two vowels.

Taken together, these exploratory phonetic measurements of vowels in Bondu So provide some hints in support of the phonological hypothesis that there was once a 10-vowel system in the language with contrasts for [$\pm$ATR] at all vowel heights. The COG measurements of individual vowels, in particular, suggest that there may still be a signal of the contrast, even among the vowels that appear to be completely merged according to formant values. In the following section, we show further support for the supposition that there exists an underlying divergence between [+ATR] and [−ATR] vowels in the language as witnessed in the behavior of root-final sonorants. 

\subsection{Consonants}
\label{subsec-cons}

Root-final sonorants were, across all the paradigms considered in this study, longer following [+ATR] vowels than [−ATR] ones, as displayed in \figref{fig:conson-dur-all}. This is a significant finding that, at the very least, indicates that a correlation exists between these vowels and the articulatory realization of the sonorants that follow them. As we shall see, effects differed somewhat between liquids and \is{consonant length} nasals, by TAM context, and by syllable position of the sonorant (onset vs. coda), but the overall effect is promisingly robust.\footnote{A reviewer asks whether there are also acoustic and/or durational differences that affect other root-final sonorants in the [+ATR] vs. [−ATR] conditions. We have not yet explored this possibility, as our focus has been on the properties of sonorants and their divergent phonological behavior. Nonetheless, the data to do so are in hand, though this must await a future study.}

\begin{figure}
    \centering
    \includegraphics[scale=0.75]{figures/80-all-paradigm_duration_nasals-liquids.pdf}  
    \caption{Root-final sonorant durations by [ATR] across all paradigms}
    \label{fig:conson-dur-all}
\end{figure}
%crg

The results are divided by paradigm below, beginning with the Perfective and Past stems illustrated in \figref{fig:conson-dur-chn-pfv}. As seen in the figure, in Perfective stems, sonorants trend longer following [+ATR] vowels than following [−ATR] ones, but this does not appear significant based on the current pilot data set. The effect does appear somewhat clearer following nasals as opposed to liquids. For example, root-final [l] in [−ATR] \textit{í\textbf{l}-ὲὲ} (ascend-\textsc{PFV}) is shorter than root-final [l] in [+ATR]  \textit{ú\textbf{l}-èè }(spit/vomit-\textsc{PFV}), but for root-final nasals, the difference is greater. That is, root-final [n] in [−ATR] \textit{dì\textbf{n}-έὲ} (find-\textsc{PFV}) is considerably shorter than root-final [m] in [+ATR] \textit{dà\textbf{m}-íì} (speak-\textsc{PFV}).

Unlike the Perfective, there are clear and significant differences observed for root-final nasals in the Past stems, which are undeniably longer in [+ATR] conditions. The effects in the corresponding liquid-final roots were largely absent, at least for this paradigm. Nonetheless, recall from \figref{fig:conson-dur-all} that the effect of [+ATR] in the liquid context was at least weakly significant overall.

\begin{figure}
    \centering
    \includegraphics[scale=0.85]{figures/80-pfv-pst_duration_nasals-liquids.pdf}    
    \caption{Root-final sonorant duration by [ATR] in Perfective stem (syllable onset) and Past stem (syllable coda)}
    \label{fig:conson-dur-chn-pfv}
\end{figure}

One might question why a weak duration effect is noted even in Perfective stems relative to ATR status, but we find that this is unsurprising. After all, as \citet{GreenHantgan} have argued, it is clear that [+ATR] still spreads to these consonants given their behavior in \isi{vowel harmony}; suffixal vowels in the Perfective are consistently raised following roots with [+ATR] vowels.

Recall from \sectref{sec-over} that the Perfective and Past stems differ from one another in that a root-final nasal occupies a syllable onset in the former context yet a coda position in the latter. To illustrate this with the example cited above, `s/he spoke' syllabifies as [dà.\textbf{m}íì] whereas `speech' is [dá\textbf{m}.dè]. Given this fact, we must ask ourselves whether and how syllabification contributes to the duration differences that have revealed themselves, strongly in some instances, and less so in others. 

The effects in \figref{fig:dur-epenth-atr}a, from Infinitive and Nominative stems, suggest that a sonorant's status as onset vs. coda does affect its duration, at least in the case of nasals. Recall that the Infinitive and Nominative both contain vowels that are all [+ATR] due to harmonization from a dominant clitic \is{cliticization} vowel, and that root-final sonorants in these paradigms appear in these syllable positions under different conditions.  There is a strong effect in this regard for nasals, but the liquid results are inconclusive.

\begin{figure}
    \centering
    \includegraphics[scale=0.75]{figures/80-duration-epenthesis-vs-atr.pdf}    
    \caption{Root-final [+ATR] sonorant duration in either coda (N) or onset (Y) position among Nominative and Infinitive stems compared to duration of [+ATR] sonorants as codas in Chaining and Past stems}
    \label{fig:dur-epenth-atr}
\end{figure}

We compare the results from the Perfective and Past stems with those in \figref{fig:dur-epenth-atr}b, for Chaining and Past stems, where the [ATR] specification determines whether a stem surfaces with a final vowel (CVC-i) or a final sonorant (CVSon). Across both stems, root-final sonorants are consistently and significantly longer following [+ATR] vowels. Again, and more broadly, we see that sonorants (and particularly nasals) are, irrespective of their syllabic position, longer following [+ATR] vowels.\footnote{A reviewer asks whether or not there is a difference in duration based on nasal place or vowel quality. In response to this, we feel that our dataset is too limited at this time to make any real predictions based on place or quality but will investigate these questions further in subsequent work.}

One challenge inherent in interpreting these results is that Bondo So provides us no means of directly comparing the duration of coda sonorants in [+ATR] vs. [−ATR] contexts. This is because [−ATR] roots, such as in the Past and Chaining stems, surface with suffixal vowels. Despite this, given sonorants' rather consistent articulatory correlates related to duration, regardless of context and syllable position, we would argue that this provides compelling support for the phonological patterning of sonorants as moraic when they appear in syllable codas in [+ATR] contexts. 

As we next illustrate, duration \is{consonant duration} is not the only phonetic correlate that distinguishes sonorants from one another following [+ATR] vs. [−ATR] vowels. \figref{fig:conson-cog-all} shows COG results for root-final nasals [m, n, ŋ] and liquids [r, l] across all five paradigms. The COG of root-final nasals and liquids following [+ATR] vowels is consistently lower than those of nasals and liquids following [−ATR] vowels. These results display similar trends as those of vowel COG shown above in \sectref{subsec-vow}, as well as those of prior studies focusing on ATR differences in vowels. Thus, we take this as further evidence that there is a surface distinction between [+ATR] vs. [−ATR] sonorants in Bondu So.

\begin{figure}
    \centering
    \includegraphics[scale=0.9]{figures/80-all-paradigms_cog_nasals-liquids-dark.pdf}
    \caption{Root-final sonorant center of gravity by [ATR] across all paradigms}
    \label{fig:conson-cog-all}
\end{figure}

The interplay of this phonetic \is{center of gravity} evidence (COG, F1, Duration) supports the hypothesis that there are articulatory differences in the production of both sonorants and vowels under the two ATR conditions. These differences are not only observed throughout the articulation of vowels, but would also seem to spread over the articulation of the sonorant consonants that follow them.

\section{Discussion and next steps}
\label{sec-diss}

\subsection{Contributions to the literature}
\label{subsec-contr}

The outcomes presented here contribute to a small but growing literature on the properties of sonorants variously described as [+ATR], \textit{tense}, or \textit{fortis}. More specifically, if our interpretation of the Bondu So facts is correct, this language provides an example of a phonology wherein a segmental feature like [ATR] has the ability to coerce \is{coerced weight} the morafication of sonorants. So too do these outcomes have implications for how best to understand core morphological and phonological characteristics of Bondu So, such as the structure of verb roots and the roles played by moras and metrification in morphophonological processes in this language. These outcomes may also provide new insight into how best to view similar phenomena in other Dogon languages. 

\subsection{Implications for phonological theory}
\label{subsec-imp}

Our phonetic findings, though preliminary, are intriguing such that they offer a quantifiable illustration that sonorants following [+ATR] vowels are different from those found after [−ATR] vowels. As noted above, \citet{HeathBS2017} states that high vowels are lost after sonorants in Bondu So, but this offers little explanation for why the process occurs and, moreover, for why it occurs only in some instances and not in others. 

The crux of the analysis argued for here, following that suggested in \citet{GreenHantgan}, is that [+ATR] vowels spread their [ATR] feature specification to sonorants. Sonorants are argued to license [ATR] by virtue of possessing Vowel-Manner features and, moreover, a location within their \is{feature geometry} vocalic geometry's Height node to which the feature can spread. \citet{GreenHantgan}, in turn, contend that the presence of [ATR] within the sonorant geometry licenses the projection of a mora, thereby satisfying a bimoraic stem \is{minimality} condition in the language.

While it is certainly unsurprising that coda consonants, and sonorant codas in particular, can be associated with a mora, the mechanism by which the mora appears to have come about in Bondu So is of interest. In contemporary versions of moraic theory, coda consonants are either inherently moraic \citep{Hyman1985} or receive a mora by rule \citep{Hayes1995}, but neither of these neatly applies to Bondu So. There is no evidence to suggest that all codas are moraic. Rather, it is this particular configuration ([+ATR] vowel plus sonorant) that leads to these sonorants' moraicity -- sonorants are coerced to be moraic in the presence of [ATR], which we contend implicates feature spreading. As such, the Bondu So outcomes add to our inventory of \is{contextual weight} known \textit{contextual} or \textit{context-dependent} weight phenomena, but differ from well-known instances of such phenomena wherein moraicity depends, for example, on word position \citep{Hayes1994,Hayes1995,RosenthallVdH1999}. In other instances, codas may differ in their weightfulness depending on their sonority \citep{Zec1995}. Weight may depend on prosodic necessity, as in \ili{Kashmiri}, where CVC syllables pattern as heavy only in the absence of another heavy stressable (i.e., CVV) syllable within a word \citep{Moren2000}, or \ili{Jóola Eegimaa} wherein \citet{HantganSagnaDavis} posit only voiced plosives as being moraic in coda position. 

The ability \is{tense/fortis sonorants} for \textit{tense} or \textit{fortis} sonorants, whether explicitly or implicitly, to be associated with [+ATR] is not entirely new either, but it is yet to be fully explored. One explicit discussion of such consonants is found in \citet{carnie2002} who, citing \citet{NiChiosain1991} and several others, analyzes two alternations -- vowel lengthening and diphthongization -- before \textit{tense} or \textit{long} sonorants associated with [+ATR] in some varieties of \ili{Irish}. Examples in \tabref{tab:CLIrish} are adapted from \citet{carnie2002}; we follow the practice adopted in the source material to represent \textit{tense} sonorants phonetically with a capital letter.

\begin{table}
\caption{Compensatory lengthening and diphthongization in Irish \citep{carnie2002}}
\label{tab:CLIrish}
 \begin{tabular}{lllll}
  \lsptoprule
  & Surface & Orthography &   &  \\
  \midrule
 a. &[f\/iːL] & fill & `bend' (V) &\\
 b. & [f\/iːLte] & fillte & `bent'& \\
 c. &[f\/iLə] & filleadh &`bend' (N) & \\\addlinespace
 d. & [pauL] & poll& `hole' & /poL/ \\
 e. & [paiL] & poill & `holes' & /peL/ \\
  \lspbottomrule
 \end{tabular}
\end{table}

Although the details vary by dialect, the gist of the phenomena shown in \tabref{tab:CLIrish} is that \textit{tense} \is{tense/fortis sonorants} sonorants, when in coda position, trigger \isi{compensatory lengthening} of a preceding vowel, but not when in an onset. Diphthongization occurs under similar conditions. There are similar outcomes reported by \citet{Archangelietal2011} for \ili{Scottish Gaelic}. 

The argument for the correlation between moraicity and [+ATR] in Irish is best substantiated by the quality of resulting diphthongs. The presence of [+ATR], as donated by the sonorants, entails raising to a high vowel, given phonotactic constraints on the language's vocalic inventory. High vowels do not diphthongize as in \tabref{tab:CLIrish}d--e, but rather lengthen, as in \tabref{tab:CLIrish}a--b. Carnie proposes that these outcomes, taken together, implicate a moraicity contrast in Irish sonorants~-- in coda position, an underlyingly moraic sonorant vacates its mora, leading to compensatory lengthening. 

\begin{sloppypar}
Analogous outcomes involving compensatory lengthening before moraic sonorants are reported for \ili{Quiaviní Zapotec} in \citet{uchiharabaez2016}, though no necessary connection to ATR is proposed. For Quiaviní Zapotec, it is argued that the language encodes a \textit{fortis/lenis} contrast in all consonants (including sonorants) that is based on the presence vs. absence of a mora, respectively. Evidence for this contrast, in part, comes from the fact that some coda consonants (i.e. \textit{fortis} consonants) block compensatory lengthening of short vowels, which otherwise occurs to achieve bimoraic minimality.
\end{sloppypar}

Both the analyses described in this subsection propose coda moraicity and, in the case of Irish, a connection between moraicity and the presence of [+ATR] to explain alternations affecting stem vowels. Importantly, both appeal to the contrastive, underlying presence of moras which are given up or vacated under some conditions. Our analysis of the Bondu So alternations is reminiscent of, but not identical to, these other outcomes: there is a correlation between moraicity and [+ATR] that has both featural and prosodic consequences. Bondu So differs, however, in that there is no evidence that codas of any type are underlyingly moraic. Rather, sonorant codas are \is{contextual weight} contextually moraic only under certain featural conditions. As such, the phonetic effects detailed above are accordingly witnessed on the sonorants themselves, and secondarily make themselves apparent by precluding affixation.

\subsection{Implications for the description of Dogon languages}
\label{subsec-imp-Dog}

We hope that it is clear, based on the discussion thus far, that moras undeniably play an important, although not always immediately apparent, role in Bondu So morphophonology. In fact, because sonorants may not appear in codas in [−ATR] contexts, the language provides us no means of direct comparison. However, given their rather consistent acoustic correlates related to \is{consonant duration} duration, and regardless of syllable position, we believe that the vocalic alternations observed in Past and Chaining stems are perhaps more telling than they appear, as they require one to revisit core assumptions about the language's structural characteristics, and notably the very shape of its verb stems. Recall that \citet{HeathBS2017} assumes that all verb stems are vowel-final, but that some lose high vowels by rule. In their appeal to coerced coda moraicity, \citet{GreenHantgan} instead posited that verb roots can be consonant-final, with any following stem vocalic material being either affixal or epenthetic. We propose another alternative here which effectively draws upon the strengths of both earlier analyses. Like Heath, we propose that the vowels most often seen in the Past and Chaining stems are lexical, though we treat them as affixes, rather than as part of the root. Additionally, like \citet{ GreenHantgan}, we believe that the noted alternations are prosodically motivated.

Based on a broader comparison of both Bondu So varieties (Najamba and Kindige), we believe that this proposition is further supported if one considers parallels between the Perfective and Past stems, as well as the Chaining stem. The Chaining and Past stems are formed periphrastically from a verb base and following enclitic or subsequent verb; \citet{HeathBS2017} often refers to the Chaining form as the \textit{bare} stem. The Perfective requires no such clitic, and, in Kindige, its vocalic suffix differs only marginally (in quality and length) from what occurs in the same position in the Chaining and Past forms.

Though it is well beyond the scope of this paper to treat the matter fully, we would propose that these outcomes find a coherent explanation in a shared morphological origin for these vowels. If one takes the pre-clitic vowel found in most Chaining/Past stems as basic, its quality is readily predictable from the established stem-controlled alternations attributed to vowel harmony; see \citet{GreenHantgan} for a detailed featural analysis. Briefly here, in most instances, the base-final vowel surfaces [i]/[u] after [+ATR] root vowels and [ɛ]/[ɔ] after [−ATR] root vowels. The exception, as we have seen, is [+ATR] roots ending in sonorants, after which the suffix fails to appear. This outcome, we reiterate, is due to a bimoraic stem requirement in the Past/Chaining context, which we assume pertains to subcategorization requirements of the accompanying clitics.

The Perfective's exponence, we would argue, entails the addition of a single feature, [\textsc{open}] (in accordance with Green \& Hantgan's previous analysis), to the same suffixal vowel, and the addition of a mora. Other alternations in quality are thereafter predictable, once again, from the stem-controlled harmonic alternations established elsewhere in the literature. Metrically, the Perfective presumably yields stems composed of an \is{metrification} unbalanced (CV.CVV) iamb, while the Past and Chaining forms, instead, are (CV.CV)=(CVV) or (CV\textsubscript{[+ATR]}C\textsubscript{[sonorant]})=(CVV), the latter being representative of roots with [+ATR] vowels and sonorants, as we have discussed throughout this paper. Arriving at a better understanding of these patterns would benefit from a closer study of metrification across the language, but it may be that metrically unbalanced feet are possible only by virtue of their phrase-final position.

\subsection{Next steps}\label{subsec-next}\largerpage

The planned next phase of this project is to resume fieldwork with a larger number of Bondu So speakers from both varieties Kindige and Najamba and thus to collect a more robust number of tokens to further substantiate the claims that we make here. Further, we will obtain and analyze acoustic data for other verb paradigms in the language, including those which were discussed in previous studies such as the Imperative and Mediopassive stems, which involve suffix-controlled [−ATR] spreading, as well as noun stems with noun class suffixes, which also involve harmony patterns, to determine if the same patterns discussed here also apply to nominal stems.

We believe that these data and the analysis we propose here complement prior work on Bondu So, and on Dogon languages, in general. The current study contributes to our understanding of the diachrony of the Bondu So morpho-phonological system, and, in the near future, it will add to the comparative literature on Dogon languages within the context of other West African languages and their phonological patterns. Unfortunately, some experimental methods that might shed more detailed light on the phenomena described here, as well as others, will understandably remain out of our reach, given the remoteness of the geographic area in which Bondu So is spoken. Nonetheless, we will continue to pursue acoustic correlates of [ATR] among vowels and consonants, not only in Bondu So, but in other Dogon languages as well.\il{Bondu So|)}

\section*{Abbreviations}
\begin{tabularx}{.5\textwidth}{@{}lQ@{}}
\textsc{PFV} & Perfective \\
\textsc{PST} & Past \\
\textsc{CHN} & Chaining \\
\textsc{INF} & Infinitive \\
\textsc{NOM} & Nominative \\
\end{tabularx}

\section*{Acknowledgments}
We are particularly grateful to Bondu So speakers, HS and IS, for their patience and collaboration with this research. We are also indebted to three reviewers and additional comments from Samson Lotven that have allowed us to improve this work. We appreciate the critical but helpful feedback from audience members at ACAL 54 that provided a somewhat different perspective on our data than we had received from others. Of course, any remaining shortcomings are ours to deal with. Many thanks to Gary Mullin for assistance with formatting of figures.

This study has received funding from the European Research Council (ERC) under the European Union’s Horizon Europe Framework Programme (HORIZON) grant number 101045195, as well as the United States National Science Foundation (NSF), award number 1263150.

\section*{Contributions}
AH contributed to writing of the original draft, review, and editing.
CG contributed to writing, theoretical underpinnings, and relevance to the literature on other types of coerced moraicity.
LCR contributed to writing, methodology, and performed the acoustic analysis. 

%\section*{Contributions}
%John Doe contributed to conceptualization, methodology, and validation.
%Jane Doe contributed to the writing of the original draft, review, and editing.

{\sloppy\printbibliography[heading=subbibliography,notkeyword=this]}
\end{document}
