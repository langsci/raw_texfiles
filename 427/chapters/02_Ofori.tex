\documentclass[output=paper,colorlinks,citecolor=brown]{langscibook}
\ChapterDOI{10.5281/zenodo.11091823}

\author{Seth Antwi Ofori\affiliation{University of Ghana, Legon}}
\title{Glide-onset formation between vowels in Akan} \label{Ch_Ofori}
\abstract{This paper examines complex alternations involved in glide-onset formation between vowels in Akan. A glide-onset formed between vowels is realized as either [w] or [j], largely aligning with the place specification of an abutting vowel. I propose that two inviolable phonotactic constraints underlie the [w] and [j] glide-onset formation process -- *V\textsubscript{1[+High]}V\textsubscript{2[−High]} 
%-- avoid a high/non-high vowel sequence
and *CʷV\textsubscript{[Labial]} -- 
%avoid a labialized-labial CV sequence. The latter impermissible structure might arise in a context where a V\textsubscript{1[+Hi, +Lab]} is followed by a V\textsubscript{2[−Hi]}. Removing such a V\textsubscript{1} has the potential to be semantically costly, however, hence the need to preserve its contrastive labiality by displacing it elsewhere. 
which are sometimes in conflict with one another. To avoid such impermissible structures, glide-onsets are formed that ultimately preserve V\textsubscript{1}'s contrastive features.
%As such, in a bid to avoid one phonotactic constraint (i.e., *V\textsubscript{[+Hi]}V\textsubscript{[−Hi]}), another dispreferred sequence (i.e., CʷV\textsubscript{[Lab]}) would be created, though this is ultimately resolved through V\textsubscript{1} delabialization. A glide-onset is formed, achieving phonotactic well-formedness while preserving the V\textsubscript{1}’s contrastive features. This illustrates that the syllable is indeed a unit of organization of segmental units in Akan. 
It will be shown that, surprisingly, /u/ and /ʊ/ as V\textsubscript{1}s behave differently in glide-onset formation.}

\IfFileExists{../localcommands.tex}{
   \addbibresource{../localbibliography.bib}
   % add all extra packages you need to load to this file

\usepackage{tabularx,multicol}
\usepackage{url}
\urlstyle{same}

\usepackage{listings}
\lstset{basicstyle=\ttfamily,tabsize=2,breaklines=true}

\usepackage{langsci-basic}
\usepackage{langsci-optional}
\usepackage{langsci-lgr}
\usepackage{langsci-osl}
% \usepackage{./langsci/styles/langsci-lgr}
% \usepackage{./langsci/styles/langsci-osl}
% \usepackage{langsci-gb4e}

\usepackage{tikz}
\usetikzlibrary{patterns,calc}
\pgfdeclarepatternformonly{south east lines}{\pgfqpoint{-0pt}{-0pt}}{\pgfqpoint{3pt}{3pt}}{\pgfqpoint{3pt}{3pt}}{
    \pgfsetlinewidth{0.6pt}
    \pgfpathmoveto{\pgfqpoint{0pt}{3pt}}
    \pgfpathlineto{\pgfqpoint{3pt}{0pt}}
    \pgfpathmoveto{\pgfqpoint{.2pt}{-.2pt}}
    \pgfpathlineto{\pgfqpoint{-.2pt}{.2pt}}
    \pgfpathmoveto{\pgfqpoint{3.2pt}{2.8pt}}
    \pgfpathlineto{\pgfqpoint{2.8pt}{3.2pt}}
    \pgfusepath{stroke}}
    
\usepackage{stmaryrd}
\usepackage{wasysym}
\usepackage{multirow}
\usepackage{caption}
\usepackage{subcaption}
\usepackage{mathrsfs}
\usepackage{qtree}

\usepackage{linguex}


   %pminos do not split footnotes
% \interfootnotelinepenalty=10000 %Footnote in Laporte chapters has to be split SN


%\DeclareIndexNameFormat{default}{%
%\nameparts{#1}%
%\usebibmacro{index:name}%
%{\index[names]}%
%{\namepartfamily}%
%{\namepartgiveni}%
% {}% L1
% {}% L2
%{\namepartprefix}% generates spurious space L3
%{\namepartsuffix}% generates spurious space L4
%}

%  {\DeclareIndexNameFormat{default}{%
%     \usebibmacro{index:name}{\index[names]}{#1}{#3}{#5}{#7}}}

%\DeclareIndexNameFormat{default}{%
%  \usebibmacro{index:name}{\sindex[nom]}{#1}{#3}{#5}{#7}}

%\DeclareIndexNameFormat{default}{%
%  \usebibmacro{index:name}{\sindex[person]}{#1}{#3}{#5}{#7}}
%\DeclareIndexNameFormat{default}{%
%\nameparts{#1} \usebibmacro{index:name}{\sindex[person]]}{\namepartfamily}{‌​\namepartgiven}{\nam‌​epartprefix}{\namepa‌​rtsuffix}}

%\newcommand{\smiley}{:)}

%\renewbibmacro*{index:name}[5]{%
%\usebibmacro{index:entry}{#1}%
%{\iffieldundef{usera}{}{\thefield{usera}\actualoperator}\mkbibindexname{#2}{#3}{#4}{#5}}}

% \newcommand{\noop}[1]{}

%remove for final
%\overfullrule=1mm

\newcommand{\tobi}[2]}}
\renewcommand{\S}[1]{\tobi{#1}{\textsc{*}}}

% this volume references
% puts: [this volume]
% already defined: \citetv
%\newcommand{\citepv}[1]{(\citeauthor{#1} \citeyear*{#1} [this volume])}
\newcommand{\citealtv}[1]{\citeauthor{#1} \citeyear*{#1} [this volume]}

%parentheses around example number
\newcommand{\pref}[1]{(\ref{#1})}

% in-text examples

\newcommand{\lnex}[1]{\textit{#1}} %target lang word
\newcommand{\lnlit}[1]{(lit.: `#1')} %literal reading
\newcommand{\lnlat}[1]{(#1)} % latinization
\newcommand{\lntrans}[1]{`#1'} %translation
\newcommand{\lnexl}[2]%
{\lnex{#1}{} \lnlat{#2}} % ex with latinization
\newcommand{\lnexlat}[3]{\lnex{#1}{} \lnlat{#2}{} \lntrans{#3}} % ex with latinization and tranl.

%ch01
\newcommand{\co}[1]{\mbox{\textbf{#1}}}

%ch09

\newcommand{\cyrbulg}[1]{\begin{otherlanguage*}{bulgarian}#1\end{otherlanguage*}}


%ch10
\newcommand{\nlp}{{\small NLP}}
\newcommand{\mwe}{{\small MWE}}
\newcommand{\rae}{{\small RAE}}
\newcommand{\lvc}{{\small LVC}}
\newcommand{\pos}{{\small P}o{\small S}}
%\newcommand{\todo}[1]{ \textcolor{red}{#1} }

%\renewcommand{\labelenumi}{\theenumi}
%\ainamefmt{{vv}{ll}{, ff}{, jj}} % fullname

\newcommand{\biberror}[1]{{\color{red}#1}}

\newcommand{\osenovaitem}{--~}
   %% hyphenation points for line breaks
%% Normally, automatic hyphenation in LaTeX is very good
%% If a word is mis-hyphenated, add it to this file
%%
%% add information to TeX file before \begin{document} with:
%% %% hyphenation points for line breaks
%% Normally, automatic hyphenation in LaTeX is very good
%% If a word is mis-hyphenated, add it to this file
%%
%% add information to TeX file before \begin{document} with:
%% %% hyphenation points for line breaks
%% Normally, automatic hyphenation in LaTeX is very good
%% If a word is mis-hyphenated, add it to this file
%%
%% add information to TeX file before \begin{document} with:
%% \include{localhyphenation}
\hyphenation{
    Beck-man
    Ngu-yen
    back-chan-nel
    back-chan-nels
    mo-not-o-nous
    ste-reo-typ-i-cal
}

\hyphenation{
    Beck-man
    Ngu-yen
    back-chan-nel
    back-chan-nels
    mo-not-o-nous
    ste-reo-typ-i-cal
}

\hyphenation{
    Beck-man
    Ngu-yen
    back-chan-nel
    back-chan-nels
    mo-not-o-nous
    ste-reo-typ-i-cal
}

   \boolfalse{bookcompile}
   \togglepaper[2]%%chapternumber
}{}

\begin{document} 
\SetupAffiliations{mark style=none}
\maketitle

\section{Introduction}
In Akan,\il{Akan|(} a Niger-Congo language of the New Kwa sub-branch, forms in which one might expect a V\textsubscript{[+High]}V\textsubscript{[−High]} sequence are instead produced with a glide-onset inserted between the vowels. I will henceforth call the process \textit{glide-onset formation}. Data for the current study were gathered through direct data elicitation from 30 native speakers of the language, alongside my intuition as a native speaker of Akan. Written sources such as \citet{deJongObeng2000}, \citet{Dolphyne1988}, and \citet{Ofori2006, Ofori2008, Ofori2013, Ofori2018, Ofori2019} were also useful in determining the underlying representations of the data that were collected. The following is a brief introduction of the problem.

\citet[8--14]{Dolphyne1988} lists and describes the \is{phonotactics} vowel sequences in \figref{fig:AkanVowels} as permissible in Akan. Here, and in tables elsewhere throughout this chapter, grey cells show impermissible forms. The sequences under consideration here are bolded (and later presented in \tabref{tab:AkanVowels2}).


\begin{figure}
\small
\caption{Vowel sequences in Akan (\cite[8--14]{Dolphyne1988})}
\label{fig:AkanVowels}
\fittable{\begin{tabular}{cccc|c|c|c|c|c|c|c|c|c|}
  \hline
%line1
 \multicolumn{4}{|c|}{\multirow{4}{*}{}} & \multicolumn{4}{c|}{V\textsubscript{2}+High}&\multicolumn{5}{c|}{V\textsubscript{2}$-$High}\\
\cline{5-13}
%line2
  \multicolumn{4}{|c|}{}   & \multicolumn{2}{c|}{\multirow{2}{*}{$-$Round}} & \multicolumn{2}{c|}{\multirow{2}{*}{+Round}} & \multicolumn{4}{c|}{$-$Low} & \multicolumn{1}{c|}{\multirow{2}{*}{+Low}} \\
\cline{9-12}
%line3
 \multicolumn{4}{|c|}{}  & \multicolumn{2}{c|}{} & \multicolumn{2}{c|}{}  &  \multicolumn{2}{c|}{$-$Round} & \multicolumn{2}{c|}{+Round} & \multicolumn{1}{c|}{}\\
\hline
%line4
  \multicolumn{4}{|c|}{}                  & i & ɪ & u & ʊ & e & ɛ & o & ɔ & a \\
\hline
%line5
\multicolumn{1}{|c|}{\multirow{4}{*}{V\textsubscript{1}+High}} &  \multicolumn{2}{c|}{\multirow{2}{*}{$-$Round}} & i & ii &\cellcolor{lightgray}  &\cellcolor{lightgray} & \cellcolor{lightgray} & \textbf{ie} & \textbf{iɛ} & \textbf{io} & \textbf{iɔ} & \textbf{ia} \\
\cline{4-4}
%line6
\multicolumn{1}{|c|}{} & \multicolumn{2}{c|}{} & ɪ & \cellcolor{lightgray} & ɪɪ &\cellcolor{lightgray} &\cellcolor{lightgray} &\cellcolor{lightgray} & \textbf{ɪɛ} &\cellcolor{lightgray} &\cellcolor{lightgray} & \textbf{ɪa} \\
 \cline{2-4}
 %line7
 \multicolumn{1}{|c|}{}& \multicolumn{2}{c|}{\multirow{2}{*}{+Round}} & u & ui & \cellcolor{lightgray}& uu & \cellcolor{lightgray}&  \textbf{ue} &\cellcolor{lightgray} & \textbf{uo} & \cellcolor{lightgray}& \textbf{ua} \\
 \cline{4-4}
 %line8
 \multicolumn{1}{|c|}{} & \multicolumn{2}{c|}{} & ʊ & & ʊɪ & \cellcolor{lightgray}& ʊʊ &\cellcolor{lightgray} & \textbf{ʊɛ} &\cellcolor{lightgray} & \textbf{ʊɔ} & \textbf{ʊa} \\
 \cline{1-4}
%line9
\multicolumn{1}{|c|}{\multirow{5}{*}{V\textsubscript{1}$-$High}} & \multicolumn{1}{c|}{\multirow{4}{*}{-Low}} & \multicolumn{1}{c|}{\multirow{2}{*}{$-$Round}} & e & ei & \cellcolor{lightgray} & \cellcolor{lightgray} & \cellcolor{lightgray} & ee & \cellcolor{lightgray} & \cellcolor{lightgray}& \cellcolor{lightgray}& \cellcolor{lightgray} \\
\cline{4-4}
%line10
\multicolumn{1}{|c|}{} & \multicolumn{1}{c|}{} & \multicolumn{1}{c|}{} & ɛ & \cellcolor{lightgray} & ɛɪ & \cellcolor{lightgray} & \cellcolor{lightgray} & \cellcolor{lightgray} & ɛɛ & \cellcolor{lightgray} & \cellcolor{lightgray}& \cellcolor{lightgray} \\
\cline{3-4}
%line11
\multicolumn{1}{|c|}{} & \multicolumn{1}{c|}{} & \multicolumn{1}{c|}{\multirow{2}{*}{+Round}} & o & oi & \cellcolor{lightgray}& \cellcolor{lightgray}& \cellcolor{lightgray}& \cellcolor{lightgray}& \cellcolor{lightgray}& oo & \cellcolor{lightgray} & \cellcolor{lightgray} \\
\cline{4-4}
%line12
\multicolumn{1}{|c|}{} & \multicolumn{1}{c|}{} & \multicolumn{1}{c|}{} & ɔ & \cellcolor{lightgray} & ɔɪ & \cellcolor{lightgray}& \cellcolor{lightgray}& \cellcolor{lightgray}& \cellcolor{lightgray}& \cellcolor{lightgray}& ɔɔ& \cellcolor{lightgray} \\
\cline{2-4}
%line13
\multicolumn{1}{|c|}{} & \multicolumn{2}{c|}{+Low} & a & \cellcolor{lightgray} & aɪ & \cellcolor{lightgray} & \cellcolor{lightgray} & \cellcolor{lightgray} & \cellcolor{lightgray} & \cellcolor{lightgray} & \cellcolor{lightgray} & aa \\\hline
 \end{tabular}}
\end{figure}

Given the possible sequences in \figref{fig:AkanVowels}, the generalizations in (\ref{ex:VVsequences}) can be stated, based on \citet{Dolphyne1988}. 

\ea \label{ex:VVsequences}
\begin{xlist}
\ex $\lbrack \alpha F \rbrack \lbrack \alpha F \rbrack$: identical vowel sequences are allowed \\
\ex $\lbrack +\text{High} \rbrack \lbrack −\text{High} \rbrack $: high followed by non-high is allowed \\
\ex $\lbrack −\text{High} \rbrack \lbrack +\text{High} \rbrack $: non-high followed by high is allowed \\
\ex $\lbrack +\text{High} \rbrack \lbrack +\text{High} \rbrack $: two high vowels (round followed by non-round) are allowed \\
\end{xlist}
\z

It is not explicitly stated in Dolphyne’s book at what level of representation these vowel sequences are acceptable, and whether the vowel sequences in (\ref{ex:VVsequences}b), in particular, are the same at both the underlying and surface levels of representation in Akan. The current study seeks to bring determinacy to this matter by providing the relevant phonetic evidence that these vowel sequences are only present underlyingly. On the surface, however, there are markedness and sonority-based syllable sequencing constraints coupled with the need to preserve segmental/feature contrasts that militate against a surface V\textsubscript{1[+High]}V\textsubscript{2[−High]} sequence. As such, a glide-onset must be formed between the involved vowels to satisfy all these conditions. Therefore, the claim in this paper is that, on the surface, there is glide-onset formation between an underlying V\textsubscript{1[+High]}V\textsubscript{2[−High]} sequence. One goal of this paper is to establish the rules leading up to glide-onset formation which achieve phonotactic well-formedness while also \is{contrast preservation} preserving underlyingly contrastive distinctions. The interactions that result in glide-onset formation, have implications for markedness theory,\is{markedness} perspectives on the sonority scale, and for syllable theory, including for the Syllable Contact Law \is{Syllable Contact Law} (\cite{MurrayVennemann1983}).      

For expository purposes, Tables \ref{tab:Table3} through \ref{tab:Table7} illustrate instances of glide-onset formation in Akan that will be further explored below. Here and elsewhere, underlying (phonological) forms are given between slashed brackets, and surface (phonetic forms) are not bracketed.%\footnote{With the exception of [ia], [ua], [iɛ], [iɔ], and [uɔ], the general requirement is for adjacent vowels to agree in \is{ATR} [±ATR].} 
 
As seen in \tabref{tab:Table3}, glide-onset formation is realized as [j] when the V\textsubscript{1} is an underlyingly high coronal vowel. It can also be realized as [w], when V\textsubscript{1} is an underlyingly high labial (i.e., round) vowel, as in Tables \ref{tab:Table4} and \ref{tab:Table5}.

\begin{table}
\begin{floatrow}
\ttabbox
 {\begin{tabular}{l@{~~}lll}
  \lsptoprule
a. & /ɛ̀fíé/	& èfíjé	&‘house’\\
b. & /àbìɛ̀sá/	 & àbìjɛ̀sá	&‘three’\\
c. &/ɛ̀fɪ̀ɛ́/ & ɛ̀fɪ̀jɛ́  &	‘vomit’\\
d. &/bìó/ & bìjó & ‘again, further-\\
   &      &      & \quad more’\\
e. & /àpɪ́á/ & àpɪ́já	& ‘itchy powdered \\
   &        &       & \quad substance’\\
  \lspbottomrule
 \end{tabular}}
{\caption{[j] glide-onset formation when V\textsubscript{1[+High]} is /i/ or /ɪ/}\label{tab:Table3}}

\ttabbox
 {\begin{tabular}{l@{~~}lll}
  \lsptoprule
a. & /ɛ̀bʊ́ɔ́/	&ɛ̀bʊ́wɔ́	&‘stone’\\
b.& /ɛ̀tʊ́ɔ́/	&ɛ̀tʊ́wɔ́	&‘butt’\\
c.& /ɛ̀kʊ́ɔ́/	&ɛ̀kʊ́wɔ́	&‘buffalo’\\
d. & /kʊ̀á/	&kʊ̀wá	&‘bend over’\\
e.& /bʊ̀á/	&bʊ̀wá	&‘help’\\
f.&  /tʊ̀á/	 &tʊ̀wá	&‘enjoin’\\
  \lspbottomrule
 \end{tabular}}
{\caption{[w] glide-onset formation when V\textsubscript{1[+High]} is /ʊ/}\label{tab:Table4}}
\end{floatrow}
\end{table}   

\begin{table}
\captionsetup{margin=.05\linewidth}
\begin{floatrow}
\ttabbox
 {\begin{tabular}{l@{~~}lll}
  \lsptoprule
a.& /bùá/ &bùwá	&‘answer’\\
b.& /pùé/	 &pùwé	&‘go out’\\
c. &/èbúó/ &	èbúwó	&‘coop’\\
  \lspbottomrule
 \end{tabular}}
{\caption{[w] glide-onset formation when V\textsubscript{1[+High]} is /u/}\label{tab:Table5}}

\ttabbox
 {\begin{tabular}{l@{~~}llll}
  \lsptoprule
a.& /bùá/ &bùwá	&‘answer’\\
b.& /pùé/	 &pùwé	&‘go out’\\
c. &/èbúó/ &	èbúwó	&‘coop’\\
  \lspbottomrule
 \end{tabular}}
{\caption{[j] glide-onset formation in Akuapem when V\textsubscript{1[+High]} is /ʊ/}\label{tab:Table6}}
\end{floatrow}
\end{table}   

As will become clear, typical outcomes of glide insertion like those just shown are not realized in all instances. For example, forms in \tabref{tab:Table6}, which are found only in the Akuapem \il{Akuapem (Akwapem)} dialect, involve additional rules of \isi{delabialization} and \isi{labio-palatalization} that affect an underlying /ʊ/, yielding unique surface forms. 


In addition, there is variation in forms with underlying V\textsubscript{1} /u/. While a [w] is often found after a labial consonant, as in \tabref{tab:Table5}, other forms with [j] are also attested in \tabref{tab:Table7}a--b in this context; these further entail vowel \isi{delabialization} to [i]. After a non-labial consonant, there are alternative forms with delabialization to [ɪ], with subsequent \isi{labio-palatalization} (\tabref{tab:Table7}c--h).

\begin{table}
\caption{Alternative glide insertion for V\textsubscript{1} /u/}
\label{tab:Table7}
 \begin{tabular}{lllll}
  \lsptoprule
a.& /bùá/ &	bìjá	&‘answer’\\
b. &/pùé/&	pìjé&	‘go out’\\
c.& /tùá/	&tᶣìjá 	&	‘settle debt’\\
d.& /ètúó/	&ètᶣíjó 	&	‘gun’\\
e. &/kúá/&	kᶣíjá	&‘farming’\\
f.& /èkúó/&	èkᶣíjó	&	‘group/association’\\
g.& /dùé/&	dᶣìjé&	‘bid apologies’\\
h.& /àdùɔ̀wɔ̀tɕᶣɪ́/	&æ̀dɥìjɔ̀wɔ̀tɕɥɪ́&	‘eighty’\\
  \lspbottomrule
 \end{tabular}
\end{table} 

The remainder of this paper is organized as follows. \sectref{02_Section2} defines the relevant phonological background for the analysis introduced above. \sectref{SEC:DataAkan} presents and analyzes the data within rule-based phonology. \sectref{02_Section4} connects findings in \sectref{SEC:DataAkan} with principles of markedness, sonority and syllable theories, and research on Akan phonology.

\section{Phonological background}\label{02_Section2}\largerpage[2]

\tabref{tab:VowelFeatures} shows features that I assume to be associated with the nine vowel pho\-nemes in Akan. Redundant features appear in parentheses.\footnote{There is an additional low vowel represented as [æ], which is underlyingly /a/ but appears allophonically in \is{ATR} [+ATR] contexts.} Certain feature definitions are particularly important for this paper. Notably, I extend place features to vowels. Doing so follows \citet{Hume1992} and provides a unified and more meaningful account of vowel-consonant and consonant-vowel feature interactions in this study. Note that one could alternatively analyze [\pm Labial] vowels [\pm Round], but given their interaction with Labial consonants, the former is analytically preferable. A \is{feature binarity} binary \pm\ distinction for [Labial] place for consonants is necessary, as consonant-based phonological processes actively reference these values independently. The same is not the case for vowels, and therefore [Labial] is treated privatively for vowels. Since there is no evidence to suggest its binarity, the [Coronal] feature is also treated \is{privativity} privatively. Privative features, when present, are marked with “\ding{51}” in \tabref{tab:VowelFeatures}. In general, I assume featural binarity unless there is reason to posit otherwise.

% This vowel is realized [e] in the \ili{Fante} dialect of Akan \il{Fante (Akan dialect) For example, the phrase \textit{da bi} `some day' is underlyingly /da bi/ and realized [dæ bi] in Twi but [de bi] in Fante.

\begin{table}
\caption{The Akan vowel feature matrix}
\label{tab:VowelFeatures}
 \begin{tabular}{lccccccccc}
  \lsptoprule
        & i & ɪ & e & ɛ & a & u & ʊ & o & ɔ \\
        \midrule
\pm High & + & +  & − & (−) & + & + & − & − & − \\
\pm Low & (−) & (−) & − & − & + & (−) & (−) & − & − \\
\pm ATR & + & − & + & − & − & + & − & + & − \\
Labial &  & & &  & &  \ding{51}\ &  \ding{51} &  \ding{51} & \ding{51} \\
Coronal &  \ding{51} &  \ding{51} &  \ding{51} &  \ding{51} & & & & & \\
  \lspbottomrule
 \end{tabular}
\end{table} 

The feature classes defined in \tabref{tab:VowelClasses} provide a reference to be used throughout this paper, with V\textsubscript{1} and V\textsubscript{2} indicated for convenience. Phonotactic constraints refer to these classes, and they aid in defining the context and processes that underlie alternations.

\begin{table}
\caption{Vowel feature classes}
\label{tab:VowelClasses}
\begin{tabular}{ ll ll }
\lsptoprule
\multicolumn{2}{c}{V\textsubscript{1}} & \multicolumn{2}{c}{V\textsubscript{2}}\\\cmidrule(lr){1-2}\cmidrule(lr){3-4}
 i ɪ u ʊ& [+High]               &  e ɛ a o ɔ& [−High]\\
 u ʊ    & [+High, Labial]       & o ɔ &  [−High, Labial]\\
 i ɪ    & [+High, Coronal]      & e ɛ &  [−High, Coronal]\\
 i u    & [+High, +ATR]         & a&  [+Low] \\
 ɪ ʊ    & [+High, −ATR]         & & \\
\lspbottomrule
\end{tabular}
\end{table} 

In Akan, there is a constraint \is{phonotactics} forbidding \is{ATR} unadvanced [−ATR] and advanced [+ATR] vowels from co-occurring either within words (roots, stems, and compound words) or between words in a phrase. This phonotactic state of affairs is typically resolved by favoring [+ATR] over [−ATR] such that an unadvanced vowel becomes advanced. The process is called ``[+ATR] harmony'', and is caused by \is{vowel harmony} the \textit{[+ATR] harmony rule} (\cite{Dolphyne1988}), in recognition of the direction of sound change. Generally speaking, [−ATR] vowels, /ɪ, ʊ, ɛ, ɔ, a/, in a given domain, are pronounced [i, u, e, o, æ], respectively, without a meaning change. For example, there is regressive [+ATR] harmony in which the future marker /bɛ-/ is pronounced [be-] before a [+ATR] vowel (e.g., /bɛ-di/ → [bedi] ‘will eat’).\largerpage[2]

Progressive [+ATR] harmony also occurs in Akan, but strictly between two vowels without an intervening consonant, the second of which is [−Low]. For example, /ɔ-di-ɪ-ɛ/ (3\GreenSC{SG}-eat-\GreenSC{PST-EMP}) is realized [odiijɛ] `s/he ate it'. Here, [+ATR] harmony spreads from /i/, the root vowel, to /ɪ/, the past/perfect marker, but does not spread to /ɛ/, the emphatic marker. Such outcomes are relevant to the current paper in that [j] glide-onset formation precedes and blocks the spread of [+ATR] to /ɛ/.{\interfootnotelinepenalty=10000\footnote{The [+ATR] harmony process will only be discussed in this paper where its application interacts with processes that contribute to glide-onset formation as described above. The interested reader could consult \citet[14--18]{Dolphyne1988} for when and how the [+ATR] harmony rule applies in Akan.}}

\tabref{tab:AkanConsonants} is based on \citet[29, 48]{Dolphyne1988} and provides readers with information concerning the consonantal phonemes of Akan. The phonetic realization(s) of each phoneme appear(s) in square brackets, where relevant.%\footnote{In this table, [ts] and [dz] are alveolar affricates in Fante, \is{variation} variants of \textit{t} and \textit{d}, respectively, before coronal vowels. [ʨ] \textit{ky} and [ʥ] \textit{gy} are prepalatal affricates derived from \textit{k} and \textit{g}, respectively, that appear before coronal vowels. [ʨɥ] \textit{tw} and [ʥɥ] \textit{dw} are labialized prepalatal affricates, derived from \textit{ku} and \textit{gu}, occurring before a coronal vowel. [kw] and [gw] instead arise from \textit{ku} and \textit{gu}, respectively, followed by the low vowel, /a/. [ɕ] \textit{hy} occurs before coronal vowels while [ɕɥ] \textit{hw} occurs before non-labial vowels. [ɲ] \textit{ny} occurs before non-labial vowels, and [ɲɥ] \textit{nw} occurs before non-low vowels. [ŋ] \textit{n} appears in word-final position in Akuapem – alveolar /n/ is also realized [ŋ] before velar consonants. [ŋw] \textit{nw} occurs before the low vowel or high labial vowels. [j] is orthographically \textit{y}. /w/ is realized [ɥ] before coronal vowels.}  

\begin{table}
\caption{Akan consonant system (\cite[29, 48]{Dolphyne1988})}
\label{tab:AkanConsonants}
\begin{tabular}{l c c c@{~~}c c}
\lsptoprule
%line1
 & \multicolumn{1}{c}{Labial} & \multicolumn{1}{c}{Coronal} & \multicolumn{2}{c}{Dorsal} & \multicolumn{1}{c}{Glottal}\\
\midrule
%line2
Plosive & \\
\quad Voiceless & p & t [t, ts] & \multicolumn{2}{c}{k [k, kw, tɕ, tɕɥ]} & \multicolumn{1}{c}{} \\ 
%line3
\quad Voiced    & b & d [d, r, l, dz] & \multicolumn{2}{c}{g [g, tw, dʑ, dʑɥ]} & \multicolumn{1}{c}{} \\
\midrule
%line4
                & & & \multicolumn{1}{c}{Prepalatal} & \multicolumn{1}{c}{Velar} & \multicolumn{1}{c}{} \\
\midrule
Fricative & f & s & ɕ ɕɥ & & h \\
Nasal     & m & n & ɲ ɲɥ &ŋ ŋw &  \\
Lateral   &  & l & & &  \\
Trill     &  & r & & & \\
Glide     &  &  & j &w [w, ɥ] &  \\
\lspbottomrule
\end{tabular}
\end{table}

In this paper, I am concerned with only a subset of these consonants. In several instances, the phonemes /t, d, s, k, g/ behave differently from /p, b, f, m, w/ in that they undergo labialization triggered by V\textsubscript{1[+High/Labial]}. An opposing \isi{delabialization} process also applies in the language, which derives the coronal vowels [i, ɪ] from the labial vowels /u, ʊ/, respectively. 
I view the latter as a repair that is necessary to ensure that a CʷV\textsubscript{[Labial]} sequence does not occur on the surface. 

Depending on the quality of the following vowel, a C\textsubscript{[−Labial]} may be labialized (i.e., [Cʷ]), or ultimately labio-palatalized ([Cᶣ]). In both instances, the trigger is V\textsubscript{[Labial]}, though this vowel ultimately loses its labiality by rule to become coronal. The processes taken together displace the vowel’s labiality onto the consonant while subsequently satisfying \is{Obligatory Contour Principle} an Obligatory Contour Principle (OCP, \citealt{Leben1973}) constraint on adjacent segments specified for [Labial]. Vowel delabialization yields an environment which, in turn, sets the stage for \isi{labio-palatalization}, with the former, in essence, feeding the latter. 

Independent evidence for these outcomes is seen in that /u/ is \is{variation} optionally delabialized when preceded by an inherently labial consonant (e.g., /bùá/ → [bùwá] $\sim$ [bìjá] ‘to answer’). As seen in this example, such inputs have two non-contrastive output forms: i) V\textsubscript{1} [u] with a [w] glide-onset, and ii) V\textsubscript{1} [i] with a [j] glide-onset. Employing the feature [Labial] for vowels and consonants allows one to capture both the aforementioned phenomena as instances of \is{Obligatory Contour Principle} [Labial][Labial] dissimilation whereby the second instance of the feature is removed. As discussion of these outcomes continues, the feature classes in \tabref{tab:ConsClasses} will prove important in capturing the outcomes witnessed in Akan. The focus here is on consonants, but some vowels are included to illustrate the classes within which consonants and vowels pattern in the language’s phonological processes.

\begin{table}
\caption{Feature classes}
\label{tab:ConsClasses}
\begin{tabular}{lll}
\lsptoprule\relax
[+Labial] & Consonants: p, b, f, m, w & Labial \\\relax
[+Labial] & Labialized consonants: Cʷ & Labialized\\\relax
[+Labial] & Labio-palatalized consonants: Cᶣ  & Labio-palatalized  \\\relax
[−Labial] & Consonants: t, d, s, j, k, g & Non-labial\\\relax
[Coronal] & Consonants: t, d, s, j & Coronal \\\relax
[Dorsal]  & Consonants: k, g & Dorsal\\\relax
[Labial]  & Vowels: u, ʊ, o, ɔ & Labial \\\lspbottomrule
\end{tabular}
\end{table} 

With reference to the feature classes in \tabref{tab:ConsClasses}, the glide-onset [w] patterns with [+Labial] consonants and labial vowels, while the [j] glide-onset patterns with [−Labial] consonants and vowels. In addition, the [j] glide-onset patterns with [Coronal] consonants. Where there is the need to discuss /k, g/ separately, [Dorsal] is used. 

It is clear that the vowels [i, ɪ] are coronal given their role in [j] glide-onset formation. Also, /e, ɛ/ are coronal, which is supported by the fact that even when they are in V\textsubscript{2} position, they dictate [j] glide-formation when V\textsubscript{1} is /ʊ/, and also when [w] glide-onset formation would be \is{phonotactics} phonotactically and/or \is{semantic avoidance} semantically costly. It is typically, but not always, in the absence or inability of [j] glide-onset formation to apply that a [w] glide-onset is formed. As will be shown, context is everything.\largerpage

With the relevant features and feature classes defined, we can now turn to more details of the Akan data. To aid in doing so, \tabref{tab:AkanVowels2} presents a subset of the vowel sequences that were given in \figref{fig:AkanVowels} in order to focus strictly on those relevant to the current study. These sequences can be described as V\textsubscript{1[+High]}V\textsubscript{2[−High]}. 


\begin{table}
\caption{Vowel sequences under consideration}
\label{tab:AkanVowels2}
\begin{tabular}{l@{~~}l cccccc}
  \lsptoprule
  & & & \multicolumn{5}{c}{V\textsubscript{2} −High}\\\cmidrule(lr){4-8}
  & & & \multicolumn{4}{c}{−Low} & \\\cmidrule(lr){4-7}
  & & & \multicolumn{2}{c}{−Round} & \multicolumn{2}{c}{+Round} & +Low\\\cmidrule(lr){4-8}
 &        &   & e & ɛ & o & ɔ & a \\\midrule
\multirow{4}{*}{\rotatebox{90}{V\textsubscript{1} +High}} & −Round & i & ie & iɛ & io & iɔ & ia \\
 &        & ɪ &\cellcolor{lightgray} & ɪɛ &\cellcolor{lightgray} &\cellcolor{lightgray} & ɪa \\
 & +Round & u &  ue &\cellcolor{lightgray} & uo & \cellcolor{lightgray}& ua \\
 &        & ʊ & \cellcolor{lightgray} & ʊɛ &\cellcolor{lightgray} & ʊɔ & ʊa \\
  \lspbottomrule
 \end{tabular}
\end{table}

Important to this paper is that a glide is formed between these vowel sequences and comes to serve as an onset to V\textsubscript{2[−High]}: coronal vowels introduce a [j] glide-onset, and labial vowels introduce a [w] glide-onset. The morphological structure of a word plays no role in glide-onset formation or in processes associated with it. As a requirement, a glide-onset must agree with an adjacent surface vowel for place, either [Coronal] or [Labial], with a preference for the former over the latter. This preference is partly based on \isi{contrast preservation}, but perhaps also in markedness. [j] is coronal, whereas [w] is labial, with the former being less sonorous, and perhaps a better syllable onset in the language. But, although this is the preferred outcome, others are observed that are predicated on several interrelated featural and phonotactic factors. The description and analysis in \sectref{SEC:DataAkan} aims to establish how context affects a given rule’s application, and how rule interactions derive output forms.  

\section{A rule-based analysis of glide-onset formation}\label{SEC:DataAkan}\largerpage

This section analyzes data on glide-onset formation in Akan drawing upon concepts and principles of rule-based phonology. Data are sub-divided into two broad categories, namely [j] glide-onset formation (\sectref{SEC:JFormation}) and [w] glide-onset formation (\sectref{SEC:Wformation}). Rules and their relevant refinements are fully expressed in \sectref{SEC:Rules}. 

As elsewhere in this chapter, input forms are given in slashed brackets. The second column of each table shows the output for a given input. As will be shown, for some inputs, onset formation is more complex than simple glide insertion. There are indeed instances in which the onset formation trigger (a V\textsubscript{1[+High]}) is altered by rule before a glide-onset is formed. There are also situations in which a preferred alternation is not phonotactically and/or semantically feasible, leading to another outcome instigated by a non-high, coronal V\textsubscript{2}. 

\subsection{Domains of [j] glide-onset formation} \label{SEC:JFormation}

This section focuses on instances when onset-formation is realized as [j] between a V\textsubscript{1[+High]}V\textsubscript{2[−High]} vowel sequence. The goal here is to establish requirements for the formation of the [j] glide-onset and to formulate linear rules to formalize the outcomes.

As shown in Tables \ref{tab:Tableie_ia} and \ref{tab:Tableie_io}, Akan avoids a V\textsubscript{1[+High]}V\textsubscript{2[−High]} vowel sequence by introducing [j], the coronal glide. 

\begin{table}
\caption{Input /ɪɛ/ and /ɪa/ sequences}
\label{tab:Tableie_ia}
 \begin{tabular}{llll}
  \lsptoprule
a.	&/ɛ̀fɪ̀ɛ́/ 	&ɛ̀fɪ̀jɛ́	&‘vomit’\\
b.	&/tɪ̀ɛ́/	&tɪ̀jɛ́&	‘discipline’\\
c.	&/àpɪ́á/	&àpɪ́já	&‘itchy powdered substance’\\
d.&	/tɪ̀á/ &	tɪ̀já&	‘shout at’\\
  \lspbottomrule
 \end{tabular}
\end{table}   

\begin{table}
\caption{Input /ie/, /io/, and /ia/ sequences}
\label{tab:Tableie_io}
 \begin{tabular}{llll}
  \lsptoprule
a.& 	/èfíé/&	èfíjé	&‘house’\\
b.&	/tìé/ &	tìjé	&‘listen’\\
c.	&/bìé/&	bìjé&	‘open’\\
d.&	/pìé/&	pìjé	&‘go out’\\
e.	&/bìó/	&bìjó	&‘again, furthermore’\\
f.	&/àbìɛ̀sá/&	æ̀bìjɛ̀sá	&‘three’\\
g.	&/ànìɛ̀dɪ́ń/&	æ̀nìjɛ̀dɪ́ń	&‘persistence’ \\
h.&	/pìá/&	pìjá	&‘push’\\
i.&	/àfìá/&æ̀fìjá	&‘Friday female name’\\

  \lspbottomrule
 \end{tabular}
\end{table}   

Prosodically, the augmented [j] comes to form an onset for the syllable containing V\textsubscript{2[−High]}. A common characteristic of the vowel sequences in these tables is that V\textsubscript{1} is both [+High] and [Coronal] – the only notable featural difference between the vowels in V\textsubscript{1} is their [ATR] status, which is \is{ATR} [−ATR] in \tabref{tab:Tableie_ia}, but [+ATR] in \tabref{tab:Tableie_io}. The V\textsubscript{2}s, which are all [−High], can be of either [ATR] value.

Though I have not listed them in the table, the forms in \tabref{tab:Tableie_io}, /bìé/ ‘to open’ and /pìé/ ‘to go out', are sometimes realized as the variants [bùwé] ‘to open’ and [pùwé] ‘to go out', respectively, with a [w] glide onset, with no change in meaning. While each retains its V\textsubscript{2}, it is the V\textsubscript{1}s that witness \is{variation} variation. This \isi{variation}, in turn, underlies their difference in glide-onset formation: a labial V\textsubscript{1[+High]} will select [w], whereas a coronal V\textsubscript{1[+High]} will select [j]. The latter outcome is relevant in Tables \ref{tab:Tableie_ia} and \ref{tab:Tableie_io}, given that the glide that is inserted and the V\textsubscript{1[+High]} that precedes it are both coronal.

It is unclear at this point how best to treat this variation. One plausible explanation is that these items have a V\textsubscript{1[+High]} that is \is{underspecification} underspecified for place. Coronality might be assigned by default, resulting in the [j] glide-onset, while the [w] glide-onset is derived via V\textsubscript{1} labialization from the preceding consonant. This is a matter to be explored in future research given that their counterparts in \tabref{tab:Tableie_io} with V\textsubscript{1} /i/ are always realized with a [j] glide-onset. From the data, it is clear that V\textsubscript{2[−High]} plays no role in this matter. Rule (\ref{rule:JFormation}) captures [j]-onset formation in these forms.

\ea \label{rule:JFormation}
\begin{xlist}
     j glide-onset formation: \\
∅ → [j] / V\textsubscript{1[+High, Coronal]} \_\_ V\textsubscript{2[−High]} \\
\end{xlist}
\z

It was mentioned that the presence of a labial V\textsubscript{1} often yields a [w] glide-onset to prevent a V\textsubscript{1[+High]}V\textsubscript{2[−High]} sequence, but this is not what obtains in Tables \ref{tab:TableueNonLab} through \ref{tab:TableUE}. Although V\textsubscript{1[+High]} is underlyingly labial (either /u/ or /ʊ/), it loses its \is{delabialization} labiality to a preceding non-labial consonant, becoming coronal and triggering a [j] glide-onset. /ʊ/ only appears as V\textsubscript{1[+High]} in \tabref{tab:TableUE} and is always followed by /ɛ/. /u/ is the V\textsubscript{1[+High]} for the input forms in Tables \ref{tab:TableueNonLab} through \ref{tab:Tableua2}, with a non-high V\textsubscript{2}.

\begin{table}
\begin{floatrow}
\ttabbox
 {\begin{tabular}{llll}
  \lsptoprule
a. & /tùé/ &	tᶣìjé&	‘pierce’\\
b. & /dùé/ &	dᶣìjé	&‘sorry’\\
  \lspbottomrule
 \end{tabular}}
{\caption{Input /ue/ sequences with non-Labial C\label{tab:TableueNonLab}}}

\ttabbox
 {\begin{tabular}{llll}
  \lsptoprule
a.&	/ètúó/ &	ètᶣíjó&	‘gun’\\
b.&	/èǹsúó/ 	&èǹsᶣíjó	&‘water’\\
c.	&/èkúó/	&èkᶣíjó&	‘group’\\
d. 	&/àdùɔ̀wɔ̀tɕɥɪ́/ &	æ̀dᶣìjɔ̀wɔ̀tɕɥɪ́	&‘80’\\
  \lspbottomrule
 \end{tabular}}
{\caption{Input /uo/ sequences with non-Labial C}\label{tab:Tableuo2}}
\end{floatrow}
\end{table}   

\begin{table}
\begin{floatrow}
\ttabbox
 {\begin{tabular}{llll}
  \lsptoprule
a.& 	/tùá/&	tᶣìjá	&‘settle debt’\\
b.&	/dùá/&	dᶣɪ̀já&	‘plant’\\
c.	&/sùá/&	sᶣɪ̀já	&‘imitate’\\
d.&	/kúá/&	kᶣíjá	&‘farming’\\
  \lspbottomrule
 \end{tabular}}
{\caption{Input /ua/ sequences with non-Labial C}\label{tab:Tableua2}}

\ttabbox
 {\begin{tabular}{llll}
  \lsptoprule
a. 	&/tʊ̀ɛ́/	&tᶣɪ̀jɛ́	&‘remove from fire’\\
b. &	/sʊ̀ɛ́/ &	sᶣɪ̀jɛ́	&‘offload’\\
  \lspbottomrule
 \end{tabular}}
{\caption{Input /ʊɛ/ sequences with non-Labial C}\label{tab:TableUE}}
\end{floatrow}
\end{table}   

The following are shared characteristics of forms in Tables \ref{tab:TableueNonLab} through \ref{tab:TableUE}, aside from their shared V\textsubscript{1[+High, Labial]}V\textsubscript{2[−High]} vowel sequences: i) the consonant after which the glide-onset is added is underlyingly non-labial, ii) the underlying consonants become \is{labio-palatalization} labio-palatalized, iii) the underlying V\textsubscript{1[+High, Labial]} becomes coronal, and iv) onset-formation yields [j], triggered by a derived coronal V\textsubscript{1}. Given the place alternation of labial V\textsubscript{1}s to coronal, and the context in which Rule (\ref{rule:JFormation}) applies, the rule responsible for the labial-to-coronal alternation must apply before the glide formation rule. I call this the V\textsubscript{1} delabialization rule. 

There are other instances in which the consonant preceding V\textsubscript{1[+High]} becomes labio-palatalized, and this can only apply after V\textsubscript{1} delabialization. In Akan, a labial vowel can only labialize, but not labio-palatalize a consonant. For example, the stems \textit{àkɔ́} ‘has gone’ and \textit{àba}́ ‘has come’ merge to derive the expression:\textit{ àkwáába}́ [àkʷáábá] ‘welcome'. Here, /ɔ/ would labialize /k/ to derive the intermediate output |akʷɔaba|. To avoid the *kʷɔ sequence, /ɔ/ is deleted, and the low vowel that follows it \is{compensatory lengthening} lengthens to compensate for this deletion to ultimately yield [àkʷáábá]. The [w] that is superimposed on /k/ in [àkʷáábá] does not become labio-palatalized because the succeeding vowel is not (or does not become) a high coronal. In Akan, it is only a labial glide that can become labio-palatalized before a coronal vowel. The word /wɪ/ ‘to chew’ is pronounced [ɥɪ] – that is, /w/ (labio)palatalizes to [ɥ] before the coronal vowel /ɪ/. 

From the examples above, it can be argued that there are three rules that apply consecutively (in a feeding relation) \is{feeding interaction} to render labio-palatalized a non-labial consonant preceding V\textsubscript{1[+Hi]}. The three rules are (ii) consonant labialization, (ii) V\textsubscript{1} delabialization, and (iii) labio-palatalization. According to the consonant labialization rule, a non-labial consonant followed by the vowel-sequence, V\textsubscript{1[+Hi, Lab]} V\textsubscript{2[−Hi]} becomes labialized (i.e., Cʷ). 

For the current study, one must formulate two separate consonant labialization rules, one for when /u/ is V\textsubscript{1}, with V\textsubscript{2} specified broadly as V\textsubscript{2[−Hi]} (see \ref{rule:JonsetV1Labial}a), and a second consonant labialization rule for when V\textsubscript{1} is /ʊ/, with a V\textsubscript{2} that is strictly /ɛ/ as in (see \ref{rule:JonsetV1Labial}b). This is necessary because forms with /ʊ/ as V\textsubscript{1}, as in \tabref{tab:TableUE}, opt out of [j] glide-onset formation when V\textsubscript{2} is /a/ or /ɔ/. Forms with V\textsubscript{1} /u/ do not.

A consonant labialization rule focusing strictly on the /ʊɛ/ vowel sequence (Rule \ref{rule:JonsetV1Labial}b) ensures that output forms with /ʊa/ and /ʊɔ/ sequences are not wrongly predicted to have a [j] glide-onset. That is, Rule (\ref{rule:JonsetV1Labial}a) has diverse V\textsubscript{2}s and, therefore, is more productive than Rule (\ref{rule:JonsetV1Labial}b) whose application is restricted to when the sequence following the underlyingly non-labial consonant is /ʊɛ/.     

\ea \label{rule:JonsetV1Labial}
\begin{xlist}
\ex Consonant labialization: \\
    C\textsubscript{[−Labial]} → C\textsubscript{[−Labial]}ʷ /\_\_V\textsubscript{1[+High, Labial, +ATR]}V\textsubscript{2[−High]} \\
   \ex  Consonant labialization: \\
    C\textsubscript{[−Labial]} → C\textsubscript{[−Labial]}ʷ /\_\_V\textsubscript{1[+High, Labial, +ATR]}V\textsubscript{2[−High, Coronal]} \\
   \ex V\textsubscript{1} delabialization (avoidance of CʷV\textsubscript{[Labial]}):\\
   V\textsubscript{[Labial]} → V\textsubscript{[Coronal]} / C\textsubscript{[−Labial]}ʷ\_\_ \\
   \ex Labio-palatalization:\\
   Cʷ →  Cᶣ / C \_\_ V\textsubscript{1[Coronal]}\\ 
   \ex j glide-onset formation: \\
 ∅ → [j] / V\textsubscript{1[+High, Coronal]} \_\_ V\textsubscript{2[−High]} \\
\end{xlist}
\z

The two rules of consonant labialization, in effect, derive an impermissible consonant-vowel sequence, CʷV\textsubscript{[+High, Labial]}. This, in turn, motivates the application of the V\textsubscript{1} delabialization (Rule \ref{rule:JonsetV1Labial}c) to derive the intermediate output,\\
CʷV\textsubscript{[+High, Coronal]}. This sequence then motivates \isi{labio-palatalization} of the labialized consonant (Rule \ref{rule:JonsetV1Labial}d) and [j] glide-onset formation (Rule \ref{rule:JonsetV1Labial}e). The labio-palatalization and [j] glide-onset rules must apply after the V\textsubscript{1} delabialization rule; labio-palatalization and the [j] onset-formation rules need not be ordered crucially.     

In \tabref{tab:TableUE}, it was shown that a non-labial consonant preceding /ʊ/ is realized with labialization as a \isi{secondary articulation}: C\textsubscript{[−Labial]} → C\textsubscript{[−Labial]}ʷ. In these instances, V\textsubscript{1} /ʊ/ delabializes to [ɪ], and consequently, a [j] glide is formed to avoid the [+High][−High] vowel-sequence, in that order. 

A different procedure occurs in words like those in \tabref{tab:TableUELabial}, however. Consonant labialization and V\textsubscript{1} delabialization do not apply to yield [j] onset-formation, and yet a [j]-onset is formed nonetheless. The consonants preceding /ʊ/ in \tabref{tab:TableUELabial} are underlyingly labial, which is the only way that these forms differ from those in \tabref{tab:TableUE}.   

\begin{table}
\caption{Input /ʊɛ/ sequences with a preceding labial consonant}
\label{tab:TableUELabial}
 \begin{tabular}{llll}
  \lsptoprule
a.&	/bʊ̀ɛ́/	&bʊ̀jɛ́	&‘be crunchy’ \\
b.	& /fʊ̀ɛ́/	&fʊ̀jɛ́	&‘be ill’\\
  \lspbottomrule
 \end{tabular}
\end{table}   

Comparing Tables \ref{tab:TableUE} and \ref{tab:TableUELabial}, one can see that consonant labialization, with its associated V\textsubscript{1} \isi{delabialization}, does not apply when the consonant preceding /ʊ/ is underlyingly labial. It is expected, in the absence of consonant labialization and resultant V\textsubscript{1} delabialization, that [w] will be formed (i.e., to derive: *[bʊ̀wɛ́] and *[fʊ̀wɛ́]) given that V\textsubscript{1} remains /ʊ/ and labial, yet this does not happen. Rather, the V\textsubscript{2} /ɛ/ dictates onset formation. Hence, these output forms emerge with the [j] onset. Note that *[wɛ] is not a permissible phonetic sequence/syllable in Akan, and any attempt to avoid this structure by extending labiality further to /ɛ/ would therefore yield attested Akan words: [bʊ̀wɔ́] ‘stone’ and [fʊ̀ɔ́] ‘buffalo'. To avoid such an outcome, a [j]-onset, triggered by /ɛ/, is the preferred option. Rule (\ref{rule:JFormationEpsilon}) shows insertion of [j] after C\textsubscript{[+Labial]}V\textsubscript{[Labial]} and before V\textsubscript{2} /ɛ/.

\ea \label{rule:JFormationEpsilon}
\begin{xlist}
V\textsubscript{2}/ɛ/, [j] onset-glide formation: \\
   ∅ → [j] / C\textsubscript{[+Labial]}V\textsubscript{1[+High, Labial, −ATR]} \_\_ V\textsubscript{2[−High, Coronal]}\\
\end{xlist}
\z

The data in \tabref{tab:TableUELabial} suggest that consonant labialization with V\textsubscript{1} delabialization fails to apply when the consonant preceding /ʊ/ is underlyingly labial. It will also be shown in Tables \ref{tab:TableUa} and \ref{tab:TableueUa} that onset-formation between a /ʊa/ sequence, irrespective of the place of articulation of the preceding consonant, is typically [w]. The output forms in \tabref{tab:TableUAAkuap}, observed in the Akuapem dialect of Akan, might seem to contradict these two positions. However, this is not entirely true, as these forms are simply \is{variation} variants of those seen elsewhere, which are observed for all dialects of Akan. In other words, the underlyingly /ʊa/ word-forms in Akuapem \il{Akuapem (Akwapem)} optionally involve consonant labialization, which triggers the processes that derive the forms in \tabref{tab:TableUAAkuap}.

\begin{table}
\caption{Input /ʊa/ sequences in Akuapem Twi}
\label{tab:TableUAAkuap}
 \begin{tabular}{llll}
  \lsptoprule
a. &	/bʊ̀á/&	bᶣɪ̀já&	‘help’\\
b.&	/fʊ̀á/&	fᶣɪ̀já&	‘agree with’\\
c.	&/tʊ̀á/&	tᶣɪ̀já&	‘enjoin’\\
d.	&/sʊ̀á/	&sᶣɪ̀já&	‘carry over head’\\
e. &	/kʊ̀á/	&kᶣɪ̀.já&	‘bend over’\\
  \lspbottomrule
 \end{tabular}
\end{table}   

\begin{sloppypar}
\noindent Given these facts, I propose the optional consonant-labialization rule in (\ref{rule:UAAkuapem}), which applies only in the Akuapem dialect of Akan.  
\end{sloppypar}

\ea \label{rule:UAAkuapem}
\begin{xlist}
Consonant labialization (in Akuapem): \\
 C → [Cʷ] / \_\_  V\textsubscript{1[+High, Labial, −ATR]}V\textsubscript{2[+Low]}\\
\end{xlist}
\z

With this optional consonant labialization rule now added, the V\textsubscript{1} delabialization rule proposed above in (\ref{rule:JonsetV1Labial}c) must be refined to accommodate the instances of V\textsubscript{1} delabialization seen in \tabref{tab:TableUAAkuap}. The only way to do this is to remove the feature [−Labial], as in the amended Rule (\ref{rule:V1LabialRevised}), so that the V\textsubscript{1} delabialization rule applies regardless of whether a \is{secondary articulation} labialized consonant is underlyingly labial or non-labial. 

\ea \label{rule:V1LabialRevised}
\begin{xlist}
V\textsubscript{[Labial]} → V\textsubscript{[Coronal]} / Cʷ \_\_ \\
\end{xlist}
\z

I have illustrated thus far that there are three contexts triggering [j] onset formation: i) a V\textsubscript{1[+High]} that is underlyingly coronal (Tables \ref{tab:Tableie_ia} and \ref{tab:Tableie_io}), ii) a derived coronal vowel from an underlyingly high labial vowel (Tables \ref{tab:TableueNonLab} through \ref{tab:TableUE}), and iii) a coronal V\textsubscript{2[−High]} preceded by C\textsubscript{[+Labial]}V\textsubscript{[+High, Labial, −ATR]} (\tabref{tab:TableUELabial}). Thus, a derived coronal-trigger of the [j]-onset emerges after consonant labialization and V\textsubscript{1} delabialization. 

In addition, I have proposed three consonant labialization rules. In two of them, the consonant preceding a labial V\textsubscript{1} is underlyingly non-labial. The third is an optional consonant labialization rule found only in Akuapem that affects any consonant before a V\textsubscript{1}/ʊ/V\textsubscript{2}/a/ sequence. To these, one can also add a V\textsubscript{1} delabialization rule which targets either /u/ or /ʊ/ when preceded by a labialized consonant. 

The rule relationships identified here are of the \is{feeding interaction} feeding type, which holds among consonant labialization, V\textsubscript{1} delabialization, and [j] onset-formation and labio-palatalization, in that order. The V\textsubscript{1} delabialization rule applies to avoid a [Labialized][Labial] (consonant-vowel) sequence, and the [j]-onset is formed to prevent a [+High][−High] vowel-sequence. 

The vowel /u/ triggers consonant labialization and also undergoes V\textsubscript{1} delabialization to condition the [j] glide-onset more than /ʊ/. Cross-dialectally, V\textsubscript{1} delabialization involving /ʊ/ applies \is{variation} when the surrounding sounds (both consonants and vowels) are underlyingly coronal.



\subsection{Domains of [w] glide-onset formation} \label{SEC:Wformation}

This subsection discusses input vowel sequences that result in [w] glide-onset formation. The first portion focuses on data in Tables \ref{tab:Tableuo} through \ref{tab:Tableue} which involve vowel sequences preceded by a labial consonant. In most instances, the expected [w] glide-onset emerges. However, \tabref{tab:Tableue2} is given for comparison as it contains data discussed earlier where the language distinctly but unexpectedly opts out of [w] onset-formation.

\begin{table}
\captionsetup{margin=.05\linewidth}
\begin{floatrow}
\ttabbox{\begin{tabular}{llll}
  \lsptoprule
a.	&/èbúó/ &	èbúwó	& ‘coop’\\
b.&	/àfúó/ &	æ̀fúwó	 &‘farm’\\
  \lspbottomrule
 \end{tabular}}
{\caption{Input /uo/ sequences}\label{tab:Tableuo}}

\ttabbox{\begin{tabular}{llll}
  \lsptoprule
a. &	/ɛ̀bʊ́ɔ́/	& ɛ̀bʊ̀wɔ́	& ‘stone’\\
b. & 	/ɛ̀fʊ́ɔ́/	& ɛ̀fʊ́wɔ́	& ‘buffalo’\\
  \lspbottomrule
 \end{tabular}}
{\caption{Input /ʊɔ/ sequences}\label{tab:TableUO}}
\end{floatrow}
\vskip1.5\baselineskip
\begin{floatrow}
\ttabbox{\begin{tabular}{llll}
  \lsptoprule
a. & 	/bùá/ & 	bùwá	 & ‘answer’\\
b. & 	/fúá/ & 	fúwá	& ‘single’\\
  \lspbottomrule
 \end{tabular}}
{\caption{Input /ua/ sequences}\label{tab:Tableua}}

\ttabbox{\begin{tabular}{llll}
  \lsptoprule
a. &  	/bʊ̀á/ & 	bʊ̀wá	& ‘help’\\
b. & 	/fʊ̀á/ & 	fʊ̀wá	& ‘agree with’\\
  \lspbottomrule
 \end{tabular}}
{\caption{Input /ʊa/ sequences}\label{tab:TableUa}}
\end{floatrow}
\vskip1.5\baselineskip
\begin{floatrow}
\ttabbox{\begin{tabular}{llll}
  \lsptoprule
a.	& /pùé/	& pùwé	& ‘go out’\\
b. & 	/bùé/	& bùwé	& ‘open’\\
  \lspbottomrule
 \end{tabular}}
{\caption{Input /ue/ sequences}\label{tab:Tableue}}

\ttabbox{\begin{tabular}{llll}
  \lsptoprule
a. &	/bʊ̀ɛ́/	& bʊ̀jɛ́  & 	‘be crunchy’\\
b. & 	/fʊ̀ɛ́/ & 	fʊ̀jɛ́	& ‘be ill’\\
  \lspbottomrule
 \end{tabular}}
{\caption{Input /ʊɛ/ sequences}\label{tab:Tableue2}}
\end{floatrow}
\end{table}

Data from Tables \ref{tab:Tableuo} through \ref{tab:TableUa}, and also \tabref{tab:Tableue} below, exhibit the following shared properties: i) V\textsubscript{1[+High]} is underlyingly high and labial, ii) the consonant preceding V\textsubscript{1} is labial, and iii) V\textsubscript{2} is [−High]. Rule (\ref{rule:WOnsetRule}) accounts for [w] glide-onset formation in these instances and is not restricted to V\textsubscript{1} being only /ʊ/ or /u/.

\ea \label{rule:WOnsetRule}
\begin{xlist}
w glide-onset formation:	\\
	∅ → [w] /C\textsubscript{[+Labial]}V\textsubscript{1[+High, Labial]} \_\_ V\textsubscript{2[−High]} \\
\end{xlist}
\z


This rule applies broadly, when C is underlyingly labial, V\textsubscript{1} is high and labial, and V\textsubscript{2} is [−High]. Exceptions to it contain inputs with V\textsubscript{2}/ɛ/ (as in \tabref{tab:Tableue2}) which seem immune to it. To account for such outcomes, Rule (\ref{rule:JFormationEpsilon}),  proposed above, must be ordered before Rule (\ref{rule:WOnsetRule}). In these instances involving /ɛ/, the grammar prioritizes [j] onset-formation rule over [w] onset-formation. The derivations in \tabref{tab:Ordering1} illustrates this outcome. 

 \begin{table}
\caption{Effect of V\textsubscript{2} /ɛ/}
\label{tab:Ordering1}
 \begin{tabular}{lcc}
  \lsptoprule
     & /bʊ̀ɛ́/ & /bùé/ \\
     \midrule
    Rule \ref{rule:JFormationEpsilon} & bʊ̀jɛ́ & -- \\
    Rule \ref{rule:WOnsetRule} & -- & bùwé \\
    \midrule
    & [bʊ̀jɛ́] & [bùwé] \\
    & `be crunchy' & `open' \\
  \lspbottomrule
 \end{tabular}
 \end{table}

In \tabref{tab:Ordering1}, the /ɛ/-specific [j]-onset rule bleeds the [w]-onset rule, yielding /bʊ̀ɛ́/ → [bʊ̀jɛ́]. Reversing the rules would derive unattested *[bʊ̀wɛ́]. The [j]-onset rule does not apply to /bùé/ whose V\textsubscript{2} is [+ATR], \is{ATR} resulting instead in [bùwé].

The last of the [w]-onset cases to be analyzed involve the forms in 
Tables \ref{tab:TableueUO} and \ref{tab:TableueUa}. The vowel sequences are underlyingly /ʊɔ/ and /ʊa/, just like those in Tables \ref{tab:Tableua} and \ref{tab:TableUa}, but they differ in that they are preceded by non-labial consonants. 

 \begin{table}
\caption{Input /ʊɔ/ sequences}
\label{tab:TableueUO}
 \begin{tabular}{llll}
  \lsptoprule
a. &	/ɛ̀tʊ́ɔ́/ &	ɛ̀tʊ̀wɔ́	&‘butt’\\
b. & 	/ɛ̀kʊ́ɔ́/ & 	ɛ̀kʊ̀wɔ́	&  ‘buffalo’ \\
  \lspbottomrule
 \end{tabular}
 \end{table}

  \begin{table}
\caption{Input /ʊa/ sequences}
\label{tab:TableueUa}
 \begin{tabular}{llll}
  \lsptoprule
a. & 	/tʊ̀á/	&  tʊ̀wá	&  ‘enjoin’\\
b. & 	/sʊ̀á/ & 	sʊ̀wá	& ‘carry over head’\\
c. &  	/kʊ̀á/ & 	kʊ̀wá	& ‘bend over’\\
  \lspbottomrule
 \end{tabular}
 \end{table}

\noindent Rule (\ref{rule:WOnsetRuleU}) captures [w] onset-formation as it applies to forms in Tables \ref{tab:TableueUO} and \ref{tab:TableueUa}.

\ea \label{rule:WOnsetRuleU}
\begin{xlist}
w glide-onset formation for C\textsubscript{[−Labial]}V\textsubscript{1}/ʊ/:	\\
	∅ → [w] /C\textsubscript{[−Labial]}V\textsubscript{1[+High, Labial, −ATR]} \_\_ V\textsubscript{2[−High]} \\
\end{xlist}
\z

We can see that this rule is restricted to applying to V\textsubscript{1} /ʊ/ because vowel-sequences /uo/ and /ua/, when preceded by non-labial consonants, submit instead to the [j]-onset rule. The outcomes in \tabref{tab:Ordering2} illustrate key differences that extend from these differing vocalic environments, which are captured by \isi{rule ordering}.  

 \begin{table}
\caption{Effect of V\textsubscript{1} /ʊ/ vs. /u/ and C labiality}
\label{tab:Ordering2}
 \begin{tabular}{lcccccc}
  \lsptoprule
   & /(ɛ)tʊ̀ɔ́/	&/tʊ̀á/&	/bʊ̀á/	&/bʊ̀ɛ́/ &	/bùá/ 	&/bùé/\\
   \midrule
    Rule \ref{rule:JFormationEpsilon} & -- & -- & -- & bʊ̀jɛ́ & -- & -- \\
     Rule \ref{rule:WOnsetRule} & -- & -- & bʊ̀wá & -- & bùwá & bùwé \\
     Rule \ref{rule:WOnsetRuleU} & tʊ̀wɔ́ & tʊ̀wá & -- & -- & -- & -- \\  
     \midrule
     & [tʊ̀wɔ́]& [tʊ̀wá] & [bʊ̀wá] & [bʊ̀jɛ́] & [bùwá] & [bùwé] \\
     & `butt' & `join' & `help' & `be crunchy' & `help' & `open' \\
  \lspbottomrule
 \end{tabular}
 \end{table}

\subsection{Rules and their refinements}  \label{SEC:Rules}

\tabref{tab:RuleSummary} presents, in no particular order, a summary of the rules proposed thus far to be responsible for [j] and [w] onset-formation between high and non-high vowel-sequences in Akan; any ordering relationships that hold between rules are discussed below. In the interest of space, feature names are abbreviated. Taken together, these repair strategies in Akan prevent *V\textsubscript{1[+High]}V\textsubscript{2[−High]} vowel sequences and *[Labialized][Labial] \is{phonotactics} consonant-vowel sequences in CVV words. For ease of reference, rules are numbered here according where they were first discussed earlier in this chapter. Rules contributing to [j] onset-formation are Rules (\ref{rule:JFormation}) and (\ref{rule:JFormationEpsilon}), and those responsible for [w] onset-formation are Rules (\ref{rule:WOnsetRule}) and (\ref{rule:WOnsetRuleU}). Thus, there are two rules directly responsible for forming each onset; one occurs in a very specific vocalic environment, and one that applies more broadly.

\begin{table}
\small\tabcolsep=.66\tabcolsep
\caption{Rule summary}
\label{tab:RuleSummary}
 \begin{tabular}{lll}
  \lsptoprule
  Rule & Name & Context \\
  \midrule
\ref{rule:JFormationEpsilon}& V\textsubscript{2} /ɛ/-conditioned [j]-onset &
∅ → [j] / C\textsubscript{[+Lab]}V\textsubscript{1[+Hi, Lab, −ATR]} 
\_\_ V\textsubscript{2[−Hi, Cor]}\\
\ref{rule:JFormation} & V\textsubscript{1[+Hi, Cor]} [j]-onset &
∅ → [j] / V\textsubscript{1[+Hi, Cor]} \_\_ V\textsubscript{2[−Hi]} \\
\ref{rule:WOnsetRuleU} & [w]-onset, non-Labial C, V\textsubscript{1}/ʊ/ & ∅ → [w]/C\textsubscript{[−Lab]}V\textsubscript{1[+Hi, Lab, −ATR]} \_\_ V\textsubscript{2[−Hi]} \\
\ref{rule:WOnsetRule} & [w]-onset, Labial C & ∅ → [w]/C\textsubscript{[+Lab]}V\textsubscript{1[+Hi, Lab]} \_\_ V\textsubscript{2[−Hi]} \\
\ref{rule:JonsetV1Labial}a & Consonant labialization 1 &  C\textsubscript{[−Lab]}→C\textsubscript{[−Lab]}ʷ/\_\_V\textsubscript{1[+Hi, Lab, +ATR]}V\textsubscript{2[−Hi]} \\
\ref{rule:JonsetV1Labial}b & Consonant labialization 2 &  C\textsubscript{[−Lab]}→C\textsubscript{[−Lab]}ʷ/\_\_V\textsubscript{1[+Hi, Lab, −ATR]}V\textsubscript{2[−Hi, Cor]} \\
\ref{rule:UAAkuapem} & C-Labialization (Akuapem), /ʊa/ & C → [Cʷ]/\_\_  V\textsubscript{1[+Hi, Lab, −ATR]}V\textsubscript{2[+Low]}\\
\ref{rule:V1LabialRevised} & V\textsubscript{1} delabialization & V\textsubscript{[Lab]} → [Cor]/Cʷ \_\_ \\
\ref{rule:JonsetV1Labial}d & Labio-palatalization &
 Cʷ →  Cᶣ/\_\_ V\textsubscript{1[+Hi, Cor]}\\ 
\lspbottomrule
 \end{tabular}
 \end{table}

In the remainder of this section, I illustrate various interactions that arise between rules in \tabref{tab:RuleSummary}. It has been shown thus far that rule ordering is critical in a number of instances. For example it was shown in \tabref{tab:Ordering1} that Rule (\ref{rule:JFormationEpsilon}) must crucially precede Rule (\ref{rule:WOnsetRule}). I also discussed that Rule (\ref{rule:UAAkuapem}), which applies only in the Akuapem dialect, sets the stage for V\textsubscript{1} delabialization via Rule (\ref{rule:V1LabialRevised}), and thereafter [j] glide-onset formation (Rule \ref{rule:JFormation}) and labio-palatalization (Rule \ref{rule:JonsetV1Labial}d).

In Rule (\ref{rule:JFormationEpsilon}), [j] onset-formation is conditioned by a non-high coronal V\textsubscript{2} when the preceding CV sequence is /C\textsubscript{[+Labial]}ʊ/. Given what occurs elsewhere in Akan, other potential outcomes could have been: i) for V\textsubscript{1} /ʊ/ to delabialize to [ɪ] and then for [ɪ] to have conditioned the [j]-onset (i.e., fʊ̀ɛ́ →  fɪ̀jɛ́ ‘vomit’), or ii) for V\textsubscript{1} /ʊ/ to have conditioned a [w]-onset, which would thereafter trigger an alternation of /ɛ/ to [ɔ] as in: bʊ̀ɛ́ → bʊ̀wɛ́ → bʊ̀wɔ́). Both outcomes would have been semantically costly, however. That is, [j] onset-formation conditioned by V\textsubscript{2} /ɛ/ blocks a V\textsubscript{1[Labial]} \isi{delabialization} rule and also a [w] onset-formation rule, which I have demonstrated above disrupts the \is{semantic avoidance} semantic identity of C\textsubscript{[+Labial]} /ʊ̀ɛ́/ inputs. As such, /ʊ/ as V\textsubscript{1} must retain its \is{contrast preservation} labiality and must not be allowed to trigger the [w] glide-onset. 

The [j] glide-formation rule (Rule \ref{rule:JFormation}) applies in contexts where such semantic issues are not relevant. That is, the rule applies when V\textsubscript{1[+High, Labial]} delabialization (via Rule \ref{rule:V1LabialRevised}) has applied to V\textsubscript{1} /ʊ/ or /u/, changing them to a coronal, [ɪ] or [i], respectively, or when V\textsubscript{1} is inherently coronal. Note that the consonant labialization rules in (\ref{rule:JonsetV1Labial}a) and (\ref{rule:JonsetV1Labial}b) create the context required for the application of Rule (\ref{rule:V1LabialRevised}); these labialization rules apply when the initial consonant of the word is [−Labial]. 

A third consonant labialization rule (Rule \ref{rule:UAAkuapem}) applies only in the Akuapem dialect and affects both labial and non-labial consonants before the vowel sequence /ʊa/. This rule is just like the labialization rules (\ref{rule:JonsetV1Labial}a) and (\ref{rule:JonsetV1Labial}b) in that it feeds V\textsubscript{1[+High, +Labial]} \isi{delabialization}. It can therefore be said that C\textsubscript{[−Labial]} labialization, via whichever of the three labialization rules applies in a given environment, is a process that \is{feeding interaction} feeds V\textsubscript{1[Labial]} delabialization which, in turn, \is{rule ordering} feeds [j] glide-onset formation triggered by V\textsubscript{1[+High, Coronal]}. 

In sum, consonant labialization applies to preserve V\textsubscript{1} labiality,  V\textsubscript{1} delabialization prevents an impermissible CʷV\textsubscript{1[Labial]} sequence, and [j] glide-onset formation prevents *V\textsubscript{1[+High]}V\textsubscript{2[−High]} sequence. As shown, the vowel /u/ is more susceptible to these processes than /ʊ/.  

Concerning the two [w] onset-formation rules, Rules (\ref{rule:WOnsetRule}) and (\ref{rule:WOnsetRuleU}), a [w] glide-onset is formed when V\textsubscript{1[+High]} is underlyingly labial and remains so on the surface. Rule (\ref{rule:WOnsetRuleU}) is restricted to applying when V\textsubscript{[+High]} is /ʊ/ and the consonant preceding it is non-labial. In Rule (\ref{rule:JonsetV1Labial}b), the environment conditioning consonant labialization is similar to that of [w] glide-onset formation. For both, the initial consonant is non-labial, and V\textsubscript{1} is /ʊ/. The only difference between them is the content of V\textsubscript{2} which is broadly [−High] in Rule (\ref{rule:WOnsetRule}), while narrowly [−High, -Coronal] in Rule (\ref{rule:JonsetV1Labial}b). As such, the narrower rule must precede the broader rule, as shown in \tabref{tab:Ordering3}, so that /tʊ̀ɛ́/ ‘to remove from fire’ can be spared of [w] glide-onset formation. Even reversing the order of the two rules in \tabref{tab:Ordering3} would yield an unattested form for /tʊ̀ɛ́/ → *[tʊ̀wɛ́], illustrating that other factors must be at play. 

\begin{table}
\caption{Consonant labialization precedes [w] glide-onset}
\label{tab:Ordering3}
 \begin{tabular}{lccc}
  \lsptoprule
   & /tʊ̀ɛ́/	&/(ɛ̀)tʊ́ɔ́/&	/tʊ̀á/	\\
   \midrule
   Rule \ref{rule:JonsetV1Labial}b & tʷʊ̀ɛ́ & -- & -- \\
   Rule \ref{rule:WOnsetRuleU} & -- & (ɛ̀)tʊ́wɔ́ & tʊ̀wá\\
   \midrule
   & *[tʷʊ̀ɛ́] & [(ɛ̀)tʊ́wɔ́] & [tʊ̀wá] \\
   & `remove from fire' & `butt' & `enjoin' \\
  \lspbottomrule
 \end{tabular}
 \end{table}

\tabref{tab:Ordering4} shows the attested outcome [tᶣɪ̀jɛ́] is created owing to the downstream effects of V\textsubscript{1} \isi{delabialization}, consonant \isi{labio-palatalization}, and [j] glide-onset formation, which must consecutively apply and act upon the output of Rule (\ref{rule:JonsetV1Labial}b). The input forms /(ɛ̀)tʊ́ɔ́/ ‘butt’ and /tʊ̀á/ ‘to enjoin’ are not subject to V\textsubscript{1} delabialization, consonant labio-palatalization, and [j] glide-onset formation, as they have not undergone consonant labialization, on which \is{feeding interaction} these rules depend.    

\begin{table}
\caption{Feeding effects}
\label{tab:Ordering4}
 \begin{tabular}{lccc}
  \lsptoprule
   & /tʊ̀ɛ́/	&/(ɛ̀)tʊ́ɔ́/&	/tʊ̀á/	\\
   \midrule
   Rule \ref{rule:JonsetV1Labial}b & tʷʊ̀ɛ́ & -- & -- \\
   Rule \ref{rule:WOnsetRuleU} & -- & (ɛ̀)tʊ́wɔ́ & tʊ̀wá\\
    Rule \ref{rule:V1LabialRevised} & tʷɪ̀ɛ́ & -- & -- \\
    Rule \ref{rule:JonsetV1Labial}d & tᶣɪ̀ɛ́ & -- &  -- \\
    Rule \ref{rule:JFormation} & tᶣɪ̀jɛ́ & -- & -- \\
    \midrule
   & *[tᶣɪ̀jɛ́] & [(ɛ̀)tʊ́wɔ́] & [tʊ̀wá] \\
   & `remove from fire' & `butt' & `enjoin' \\
  \lspbottomrule
 \end{tabular}
 \end{table}

The broader [w] glide-onset rule, Rule (\ref{rule:WOnsetRule}), applies in instances where V\textsubscript{1[+High]} and the consonant preceding it are [+Labial]. Notably, V\textsubscript{2} must be a non-high vowel. The context of the narrower [j] onset formation rule triggered by V\textsubscript{2} /ɛ/, Rule (\ref{rule:JFormationEpsilon}), is similar, illustrating that the latter must be ordered before the former in order for it to have any visible effect. This \is{rule ordering} ordering, yielding [bʊ̀jɛ́] `to be crunchy' from input /bʊ̀ɛ́/ illustrates the necessity of this ordering in \tabref{tab:Ordering5}. Other inputs whose V\textsubscript{2} differs are not affected by the more stringent rule.

\begin{table}
\caption{Ordering [j] glide-onset before [w] glide-onset}
\label{tab:Ordering5}
\small
 \begin{tabular}{lcccccc}
  \lsptoprule
 & /bʊ̀ɛ́/ & /(ɛ̀)bʊ́ɔ́/ & /bùá/ & /bùé/ & /(è)búó/ & /bùá/	\\
 \midrule
  Rule \ref{rule:JFormationEpsilon} & bʊ̀jɛ́ & -- & -- & -- & -- & --  \\ 
 Rule \ref{rule:WOnsetRule} & -- & (ɛ̀)bʊ̀wɔ́ & bʊ̀wá& bùwé & (è)búwó & bùwá \\
 \midrule
   & [bʊ̀jɛ́] & [(ɛ̀)bʊ̀wɔ́] & [bʊ̀wá]& [bùwé] & [(è)búwó] & [bùwá] \\
   & `be crunchy' & `stone' & `help' & `open' & `coop' & `answer' \\
  \lspbottomrule
 \end{tabular}
 \end{table}

What I hope is clear is that V\textsubscript{1} is the primary conditioner of glide-onset formation. V\textsubscript{2} has a limited, but nonetheless significant, role to play. Concerning the two high labial V\textsubscript{1}s, /ʊ/ generally conditions [w] glide-onset formation. It only conditions consonant labialization (with accompanying V\textsubscript{1} delabialization, consonant labio-palatalization, and [j] glide-onset formation rules) when the preceding consonant is non-labial and the succeeding vowel is /ɛ/. There are cases where /ɛ/ is the V\textsubscript{2} (i.e., in C\textsubscript{+Labial]}/ʊɛ/ words), and therefore V\textsubscript{2} steps in to trigger a [j] glide-onset when V\textsubscript{1} cannot act in either of the ways described above (i.e., when neither consonant labialization nor the [w] glide-onset formation rule could apply). V\textsubscript{1} /u/ generally conditions consonant labialization, followed by V\textsubscript{1} delabialization, consonant labio-palatalization, and [j] glide-onset formation. It only conditions [w] glide-onset formation when the preceding consonant is labial. 

In \il{Akuapem (Akwapem)} Akuapem, consonant labialization \is{variation} and its accompanying rules, on the one hand, and the rule of [w] glide-onset formation, on the other, apply independently (irrespective on the place of articulation of the initial consonant) to derive variant output realizations in /Cʊ̀á/ input cases. That is, these forms undergo either [j] glide-onset formation with the initial consonant becoming labio-palatalized, or [w] glide-onset formation with the initial consonant being intact.\footnote{The approach taken here has been rule-based. In non-linear phonological terms, consonant labialization and both [j] and [w] glide-onset formation, could be conceived of and represented by place feature spreading – not a complete spreading of a vowel as one reviewer suggested – either regressively or progressively.}
  
\section{Conclusion} \label{02_Section4}

In this section, I situate the findings presented above within phonological theory to shed more light on the motivations for the observed vowel and consonant behaviors as well as the prosodic aspects of the glide-onset formations under consideration in this paper. 

The following is a brief overview of some essential principles of \isi{markedness} theory and \isi{sonority} theory to which the findings relate. The theory of markedness (\citealt{deLacy2002, deLacy2006, Lombardi2002}) posits that “not all elements in a phonological system are of equal status” (\citealt{Rice2007}). The unmarked/marked distinction between any two segments often dictates which phonological processes they can be subject to. The tendency is to preserve marked units over unmarked ones in situations where one of them must be deleted in a given phonological domain. In terms of height (from marked to unmarked) the following relationships can be proposed for Akan vowels: [−High] $\gg$ [+High] and [+Low] $\gg$ [−Low]. Taken together, high vowels are the least marked, and so forth, as vowel height moves to mid and low: \textit{a} $\gg$ \textit{o, ɔ, ɛ, e} $\gg$ \textit{ʊ, u, i, ɪ}). For consonant place, coronal is unmarked. 

As noted by \citet[178--179]{Zec2007}, “[s]onority … steers the crucial aspects of syllable internal segment sequencing” [and that] “[t]he second mode of constraining sonority is syntagmatic in nature.” Argued to underlie intra- and inter-syllabic organization of segments, the \isi{sonority} scale/hierarchy has been represented as follows (from the most to least sonorous): V (low $\gg$ mid $\gg$ high) $>$ L (rhotics $\gg$ laterals) $>$ N (nasals) $>$ O (voiced fricatives $\gg$ voiced stops $\gg$ voiceless fricatives $\gg$ voiceless stops) (\cite[178]{Zec2007}). As reiterated by Zec, “[b]y taking into account the ordering [as given above], the arrangement of segments within the syllable follows a clear pattern: the most sonorous segment occupies the nucleus, while the less sonorous ones occur towards the margins. [Syntagmatically], [c]onstraints on sonority distance have the task to optimize the sonority slope between margins and peaks, both within and across syllables''. I would argue that Syllable Contact Law (SCL, \cite{MurrayVennemann1983}), focusing on vowel sequences between syllables, motivates onset formation in Akan.

Prosodically, what begins as a /CV.V/ syllable sequence (with the second onsetless syllable being marked) undergoes glide-onset formation to become [CV.CV]. The outcome is two unmarked CV syllables. What is fascinating is how the language employs an unmarked prosodic strategy to obviate the phonotactically ill-formed *[+High][−High] \is{phonotactics} vowel sequence. In achieving this objective, templates have to be modified (e.g., a C has to be inserted on the CV-tier) to allow for the onset and glide formations (i.e., the prosodic and segmental/featural remedies, respectively) which apply concurrently to right this phonotactic blunder. To meet this objective, there is alignment (i.e., association), re-alignment (i.e., re-association), and de-association of units of the different levels of input representation, which could easily be illustrated in a non-linear representational model.\largerpage 

This re-organization of features and prosodic units is required to prevent a \is{markedness} marked [+High][−High] phonotactic sequence, in a context where the feature [+High] (i.e., the unmarked value) is lexically significant and must be preserved. The unmarked status of [+High] is established based on how sounds with this feature respond when adjacent to those with the feature [−High]. The word, \textit{paɪ}‘split', when reduplicated to \textit{paɪpaɪ}, is pronounced [paapaɪ]. That is, /ɪ/ (being unmarked in Akan) is lost in pronunciation, and /a/ (being marked in Akan) is doubled to compensate for the loss. 

Also contributing to the observed outcomes is that, historically, the sounds [tɕ, dʑ, ɥ, ɕ] are said to have been originally /k, g, w, h/, respectively, word-initially before V\textsubscript{1[+High, Coronal, +ATR]}V\textsubscript{2[−High]}. The V\textsubscript{1} /i/ was deleted alongside the coronalization of the two plosives /k, g/ to [tɕ, dʑ], respectively. Thus, \is{ATR} what was once a disyllabic root became monosyllabic. Interestingly, the [+ATR] feature of the deleted /i/ still triggers regressive [+ATR] harmony in the language. The so-called historical reduction was made possible by the fact that the high vowel was not contrastive in said domain. Words in the language that are argued to have been subjected to the two processes are: \textit{agya} [æʥa] ‘father', \textit{gya} [ʥa] ‘leave', and \textit{egya} [eʥa] ‘fire', as well as, \textit{twa} [ʨɥa], ‘to cut', and \textit{dwa} [ʥɥa] ‘to peel off'. \Citet{deJongObeng2000} (also, see \citetv{chapters/12_deJong}) use the \is{palatalization} term \textit{palatalization} to refer to the above historical process in Akan.

\begin{sloppypar}
Via this process, an \is{markedness} unmarked [+High] vowel was deleted, and a marked [−High] vowel preserved, illustrating another instance of faithfulness to the marked, masking of the unmarked, submergence of the unmarked (\citealt{deLacy2002,deLacy2006,Rice1999,Rice2002}). The impermissible [+High][−High] vowel sequence cannot be prevented by deleting the unmarked high vowel in the data under consideration in the current paper. This is because the unmarked high vowel is lexically significant – i.e., its deletion would create another attested word. The formation of a glide-onset, therefore, precludes the unmarked [+High] vowel from deletion.%I view the unmarked [+High] vowel deletion between an affricate and a non-high vowel, as discussed above, as a synchronic rather than a diachronic process.
\end{sloppypar}

Based on the current study, I would argue that glide-onset formation is a strategy for preservation \is{contrast preservation} of the unmarked. The unmarked high vowel is under threat of deletion by a succeeding non-high vowel. The argument in this paper is that principles of sonority underlie the determination of what is marked and unmarked in Akan grammar and that the two factors combined are significant in the construction of syllable-sequences in Akan grammar. Glide-onset formation applies to prevent an unmarked/less-sonorous vowel from the encroachment of, and subsequent deletion by, a marked/more-sonorous \is{sonority} vowel in sequence. With V\textsubscript{1[+High]} being less sonorous than V\textsubscript{2}, the [j] or [w] glide is inserted to create a syllable onset for the second vowel.\largerpage

My analytical position is that sonority is not only significant in the organization of segments within syllables (\citealt{Bybee1976, Clements1990, Jespersen1904, Selkirk1984, Steriade1982, Vennemann1972, Zec1988}), and in constraining syllable contact sequences via the Syllable Contact Law (\cite{MurrayVennemann1983}), but it is also equally relevant in the sequencing of vowels at the syllable-boundary in open syllable languages like Akan. Thus, it would appear that Syllable Contact Law motivates glide-onset formation. Glide-onset formation is necessitated by the fact that V\textsubscript{1}, which ends the first syllable, is less sonorous than a following V\textsubscript{2}. The argument, therefore, is that glide formation applies to ensure that the left edge of the second syllable is lower in sonority than the right edge of the first syllable. Glides are well-suited for this because they are lower on the sonority scale than the high vowels which end the first syllable, but it should also be noted that liquids, nasals, obstruents are also lower on the sonority scale than high vowels, and, on that basis, one might consider them equally plausible candidates for the V\textsubscript{2} onset slot. So, this leads to a question of why glides are preferred over non-glides preceding V\textsubscript{2}. 

The answer would appear to rest in the fact that glide-onset formation must be pursued in ways that \is{contrast preservation} preserve meaning, and the grammar meets this objective by employing more predictable, meaning-preserving processes, which the [w] and [j] glide-onset formation can achieve. That is, the transition from a less sonorous vocalic unit to the more sonorous \is{sonority} vocalic unit of abutting syllables must be avoided, and is so achieved through glide-onset formation conditioned by an abutting vowel in the sequence in this preference order: [i, ɪ] $\gg$ [ɛ, e] $\gg$ [u, ʊ]. The conditioning vowel must be preserved, to the extent possible, to maintain the root's lexical identity.

This (re)syllabification effort is significant for achieving both phonotactic and functional/semantic well-formedness. In the end, contrastive units/features are preserved, and conditions regulating the sequencing of phonological (segmental and featural) units (which are markedness- and sonority-based) under the umbrella term, phonotactics, are observed. 

That V\textsubscript{2[−High]} is more sonorous than V\textsubscript{1[+High]}, and the outcomes surrounding them as presented here, is conceived as a fortitioning strategy aimed at preserving an unmarked but contrastive feature. This occurs predictably in all instances. It is guaranteed by the extension a vowel's place feature -- either [Coronal] or [Labial] -- to the newly created onset slot. We see the tendency of /u/ (but not /ʊ/) to lose its labiality, and, therefore, for the [j] glide-onset to be generated often for forms with underlying /u/. This trend can be viewed as principled if one bears in mind that /u/ is higher than /ʊ/, and therefore less marked based on the reasoning above. The vowel trapezium in \citet[7]{Dolphyne1988} supports these findings. Indeed, what the current study has done is to have offered the relevant empirical phonological support for Dolphyne's Akan vowel chart.\largerpage

Lastly, on the semantic/functional side of the argument, glide-onset formation applies when the repair of phonotactically impermissible vowel sequences via loss of one of the two segments would be functionally/semantically costly. Therefore, glide-onset formation comes across as the more convenient, alternative repair strategy that significantly preserves essential elements of the two vowels without the output resulting in any meaning difference. Thus, Akan's grammar appears to restrict the application of phonological to specific units and domains, and demands that certain rules be ordered, so preserve both meaning and phonotactic well-formedness.\il{Akan|)}



\section*{Abbreviations}
\begin{multicols}{2}
\begin{tabbing}
mmm \= Consonant\kill
ATR \> Advanced Tongue Root \\
C \> Consonant \\
Cor  \> Coronal \\
Dor \> Dorsal \\
EMP \> Emphatic \\
Hi \> High \\
Lab \> Labial \\
Lo \> Low \\
PST \> Past \\
Rd \> Round \\
SG \> Singular \\
V \> Vowel 
\end{tabbing}
\end{multicols}



\section*{Acknowledgements}

When I was growing up in Ghana in the 80s, there was a program on national radio called “The Everyday English” on which Dr. Samuel Gyasi Obeng was the radio teacher. We were always glued to our radios each morning to learn from him about the intricacies of the English language and how to use it. We so admired him, and little did some of us know that we would someday meet him and that we would become colleagues in Linguistics. Prof. Obeng is a great teacher and colossus in matters of general and African linguistics, and it is befitting that his contributions to our field be recognized. May God continue to bless this fine scholar and gentleman with good health and in all his endeavors. 

I am also very thankful to the editorial team and the two anonymous reviewers for their comments and suggestions on the drafts that have finally developed into the current paper. Chris and Samson, I really appreciate your attention to detail and clarity, your patience and hard work in putting this volume together.

%\section*{Contributions}
%John Doe contributed to conceptualization, methodology, and validation.
%Jane Doe contributed to the writing of the original draft, review, and editing.

{\sloppy\printbibliography[heading=subbibliography,notkeyword=this]}
\end{document}
