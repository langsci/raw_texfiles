\documentclass[output=paper,colorlinks,citecolor=brown]{langscibook}
\ChapterDOI{10.5281/zenodo.11091825}

\author{Samson Lotven\affiliation{Indiana University} 
and Matthew Ajibade\affiliation{Indiana University}}
\title{Morphology in Gengbe and Yoruba ideophones} 
\abstract{Ideophones depict events and states, filling the cracks between linguistic iconicity and arbitrariness. More than extemporaneous sound effects, ideophones are words, stored in the mental lexicons of speakers, and as such, despite their often exceptional properties, ideophones behave very much like other words, especially in their derivational morphology -- word formation that happens in the lexicon. To highlight the similarities in morphological phenomena between ideophones and other words in the lexicon, we consider compounding, reduplication, and tone in the derivational morphology of Gengbe and Yoruba. In these and other West African languages, often used as examples of isolating languages, we take this opportunity to highlight morphological processes where we find them, revealing complexity in the word formation patterns of ideophones and in the morphology of isolating languages.}

\IfFileExists{../localcommands.tex}{
   \addbibresource{../localbibliography.bib}
   % add all extra packages you need to load to this file

\usepackage{tabularx,multicol}
\usepackage{url}
\urlstyle{same}

\usepackage{listings}
\lstset{basicstyle=\ttfamily,tabsize=2,breaklines=true}

\usepackage{langsci-basic}
\usepackage{langsci-optional}
\usepackage{langsci-lgr}
\usepackage{langsci-osl}
% \usepackage{./langsci/styles/langsci-lgr}
% \usepackage{./langsci/styles/langsci-osl}
% \usepackage{langsci-gb4e}

\usepackage{tikz}
\usetikzlibrary{patterns,calc}
\pgfdeclarepatternformonly{south east lines}{\pgfqpoint{-0pt}{-0pt}}{\pgfqpoint{3pt}{3pt}}{\pgfqpoint{3pt}{3pt}}{
    \pgfsetlinewidth{0.6pt}
    \pgfpathmoveto{\pgfqpoint{0pt}{3pt}}
    \pgfpathlineto{\pgfqpoint{3pt}{0pt}}
    \pgfpathmoveto{\pgfqpoint{.2pt}{-.2pt}}
    \pgfpathlineto{\pgfqpoint{-.2pt}{.2pt}}
    \pgfpathmoveto{\pgfqpoint{3.2pt}{2.8pt}}
    \pgfpathlineto{\pgfqpoint{2.8pt}{3.2pt}}
    \pgfusepath{stroke}}
    
\usepackage{stmaryrd}
\usepackage{wasysym}
\usepackage{multirow}
\usepackage{caption}
\usepackage{subcaption}
\usepackage{mathrsfs}
\usepackage{qtree}

\usepackage{linguex}


   %pminos do not split footnotes
% \interfootnotelinepenalty=10000 %Footnote in Laporte chapters has to be split SN


%\DeclareIndexNameFormat{default}{%
%\nameparts{#1}%
%\usebibmacro{index:name}%
%{\index[names]}%
%{\namepartfamily}%
%{\namepartgiveni}%
% {}% L1
% {}% L2
%{\namepartprefix}% generates spurious space L3
%{\namepartsuffix}% generates spurious space L4
%}

%  {\DeclareIndexNameFormat{default}{%
%     \usebibmacro{index:name}{\index[names]}{#1}{#3}{#5}{#7}}}

%\DeclareIndexNameFormat{default}{%
%  \usebibmacro{index:name}{\sindex[nom]}{#1}{#3}{#5}{#7}}

%\DeclareIndexNameFormat{default}{%
%  \usebibmacro{index:name}{\sindex[person]}{#1}{#3}{#5}{#7}}
%\DeclareIndexNameFormat{default}{%
%\nameparts{#1} \usebibmacro{index:name}{\sindex[person]]}{\namepartfamily}{‌​\namepartgiven}{\nam‌​epartprefix}{\namepa‌​rtsuffix}}

%\newcommand{\smiley}{:)}

%\renewbibmacro*{index:name}[5]{%
%\usebibmacro{index:entry}{#1}%
%{\iffieldundef{usera}{}{\thefield{usera}\actualoperator}\mkbibindexname{#2}{#3}{#4}{#5}}}

% \newcommand{\noop}[1]{}

%remove for final
%\overfullrule=1mm

\newcommand{\tobi}[2]}}
\renewcommand{\S}[1]{\tobi{#1}{\textsc{*}}}

% this volume references
% puts: [this volume]
% already defined: \citetv
%\newcommand{\citepv}[1]{(\citeauthor{#1} \citeyear*{#1} [this volume])}
\newcommand{\citealtv}[1]{\citeauthor{#1} \citeyear*{#1} [this volume]}

%parentheses around example number
\newcommand{\pref}[1]{(\ref{#1})}

% in-text examples

\newcommand{\lnex}[1]{\textit{#1}} %target lang word
\newcommand{\lnlit}[1]{(lit.: `#1')} %literal reading
\newcommand{\lnlat}[1]{(#1)} % latinization
\newcommand{\lntrans}[1]{`#1'} %translation
\newcommand{\lnexl}[2]%
{\lnex{#1}{} \lnlat{#2}} % ex with latinization
\newcommand{\lnexlat}[3]{\lnex{#1}{} \lnlat{#2}{} \lntrans{#3}} % ex with latinization and tranl.

%ch01
\newcommand{\co}[1]{\mbox{\textbf{#1}}}

%ch09

\newcommand{\cyrbulg}[1]{\begin{otherlanguage*}{bulgarian}#1\end{otherlanguage*}}


%ch10
\newcommand{\nlp}{{\small NLP}}
\newcommand{\mwe}{{\small MWE}}
\newcommand{\rae}{{\small RAE}}
\newcommand{\lvc}{{\small LVC}}
\newcommand{\pos}{{\small P}o{\small S}}
%\newcommand{\todo}[1]{ \textcolor{red}{#1} }

%\renewcommand{\labelenumi}{\theenumi}
%\ainamefmt{{vv}{ll}{, ff}{, jj}} % fullname

\newcommand{\biberror}[1]{{\color{red}#1}}

\newcommand{\osenovaitem}{--~}
   %% hyphenation points for line breaks
%% Normally, automatic hyphenation in LaTeX is very good
%% If a word is mis-hyphenated, add it to this file
%%
%% add information to TeX file before \begin{document} with:
%% %% hyphenation points for line breaks
%% Normally, automatic hyphenation in LaTeX is very good
%% If a word is mis-hyphenated, add it to this file
%%
%% add information to TeX file before \begin{document} with:
%% %% hyphenation points for line breaks
%% Normally, automatic hyphenation in LaTeX is very good
%% If a word is mis-hyphenated, add it to this file
%%
%% add information to TeX file before \begin{document} with:
%% \include{localhyphenation}
\hyphenation{
    Beck-man
    Ngu-yen
    back-chan-nel
    back-chan-nels
    mo-not-o-nous
    ste-reo-typ-i-cal
}

\hyphenation{
    Beck-man
    Ngu-yen
    back-chan-nel
    back-chan-nels
    mo-not-o-nous
    ste-reo-typ-i-cal
}

\hyphenation{
    Beck-man
    Ngu-yen
    back-chan-nel
    back-chan-nels
    mo-not-o-nous
    ste-reo-typ-i-cal
}

   \boolfalse{bookcompile}
   \togglepaper[2]%%chapternumber
}{}

\begin{document}
\SetupAffiliations{mark style=none}
\maketitle

\section{Introduction}
Like the category of \textit{isolating languages}, defined by a dearth of inflectional morphology, the category of \textit{ideophones} is also said to “display little morphology'', \citep[185]{Childs1994}, or even “exceptional morphology'', the example given by \citet[167]{Klamer2001} being reduplication. Yet, there is more to morphology than inflection, and reduplication is not so exceptional, especially in West African languages. In responding to similar assertions, such as \citet{Johnson1976} and \citet{Kunene1965}, \citet[26]{Ameka2001} notes that it is not useful to highlight that ideophones do not display the morphology of inflected words in languages where there is little inflection: “I suspect that many of the features that have been noted for ideophones co-vary in similar ways with the typological properties of the languages in which they occur''. 

\begin{sloppypar}
As researchers increasingly consider ideophones within their linguistic descriptions and analyses, academic focus has broadened to include how ideophones fit into grammar and more broadly into linguistic typology (\citealt{Newman2001, Voeltz_Killian-Hatz2001}). In pursuing this concern, we present examples of word formation patterns found in two isolating West African languages, Gengbe and Yoruba. Our goal is to highlight similarities between the derivational morphology of prosaic words (non-ideophones) and that of ideophones, with particular focus given to \isi{compounding}, \isi{reduplication}, and tonal morphology. In doing so, we examine different types of non-arbitrariness in Gengbe and Yoruba derivational morphology, including qualitative \isi{iconicity} (depiction based on the sounds of a word), quantitative iconicity (depiction based on the shape of a word), and systematicity (regularity within the language system), distinctions discussed in \citet{CarlingandJohansson2015} and \citet{Dingemanseetal2015}. 
\end{sloppypar}

After this brief introduction, Section \ref{Sec-LangBackgrounds} introduces these two languages, their morphology, and the sources for data presented in this paper. Section \ref{sec_derivatioanlgengbe} offers examples of word formation in ideophones, discussing different types of non-arbitrariness and working from word formation that is not clearly morphological towards examples of patterns more similar to prosaic word morphology. Section \ref{Sec_GengbeConclusion} concludes the paper by considering similarities between the morphological processes discussed and the importance of including ideophones in morphological description and analysis.
 
\section{Language backgrounds and data}\label{Sec-LangBackgrounds}

\subsection{Gengbe}

Gengbe [iso 639-3: gej], alongside other \ili{Gbe languages} (Niger-Congo, Kwa), is spoken in Togo, Benin, and Ghana by multilingual populations. As with other Gbe languages, such as \ili{Ewe}, the basic word order in a Gengbe sentence is Subject-Verb-Object, but syntactic movement such as topicalization and focusing, as well as morphological processes like nominal and adjectival reduplication, can generate a surface Subject-Object-Verb order. Similar to Ewe, the Gengbe syllable may be Vowel or Syllabic Nasal only (V or N), Consonant-Vowel (CV), Consonant-Liquid-Vowel (CLV), or Consonant-Glide-Vowel (CGV). All syllables either have (H)igh tone, realized as H or Rising pitch, or are toneless, realized with (M)id or (L)ow pitch \citep{Bole-Richard1983}. Additional analysis of the phonetics and phonology of this variety of Gengbe can be found in \citet{Lotven&Obeng2018} and \citet{Lotven2020}, as can the conventions of the practical orthography used here.

\subsection{Gengbe ideophones}
\label{Gengbe Ideophones}

It\il{Gengbe|(} has been long noted that Gbe languages are rich with ideophones, and some of the earliest linguistic research on ideophones is from \citet{Schlegel1857}, who devoted a chapter to the subject in his grammar of the Gbe language \ili{Ewe}, a research program continued by \citet{Westermann1905, Westermann1907}. \citet{Ameka2001}, describes ideophones in Ewe as belonging to no one grammatical word class, and rather that they can be found in syntactic positions typically occupied by nouns, verbs, adjectives, adverbs, intensifiers, and interjections. Gengbe is similar to Ewe in this respect (as is Yoruba). 

In the Spring of 2015, Dr. Samuel Gyasi Obeng and the first author, working with Gabriel Mawusi, a native speaker of Gengbe from Batonou (a village near Glidji, Togo), transcribed, discussed, and audio recorded (in isolation) over 80 sound symbolic expressions in the Indiana University Department of Linguistics, certainly a fraction of those found in the language. The data presented here were elicited alongside discussions of other adverbials, so all Gengbe examples presented here can at least function as adverbs, and some can likely occupy more grammatical roles, as outlined in \citet{Ameka2001}. Using the distinction from \citet{Martin1975} made in discussing \ili{Japanese}, this list includes both phonomimes (phon) -- ideophones depicting sounds such as \textit{sss} (the sound of a snake moving) and phenomimes (phen) -- those depicting actions or states, such as \textit{jɔ̀::} (the motion of a snake moving). Examples of phenomena in Gengbe are from this list (included in the Appendix). 

\subsection{Gengbe morphology}
\label{Gengbe Morphology}

\citet[1]{Essegbey2006} calls \ili{Ewe} “an isolating language with agglutinating features”. Gbe languages generally display little inflectional morphology, and, with the exception of case-marked pronouns, they lack verbal and nominal inflection \citep[32--33] {Aboh2004}. This description is apt for Gengbe as well, with Gengbe derivational morphology making use of compounding, reduplication (full and partial), and tone, as exemplified below.

Compounding is commonplace in Gengbe, for example in the word meaning ‘pen’ \textit{è-sì-nṹ-ŋlɔ̃̀-tí} (\GreenSC{NML}-water-thing-write-stick), and nouns typically have a lexically-determined nominal prefix \textit{è-} or \textit{à-} (glossed \GreenSC{NML}), which is deleted during compounding. \citet{Abaglo&Archangeli1989} and \citet{Beavon-Hamm2020} analyze these prefixes in Gen and other \ili{Gbe languages}, where they are obligatory for monosyllabic nouns, as working to satisfy a disyllabic minimal word condition, but free to drop when the word is sufficiently long. For this reason, in nouns, the presence of only one nominal prefix is diagnostic of a compound. However, verb-object sequences, as will be seen in example (\ref{Ex.Iwilleat}), may also involve deletion of the nominal prefix after a predicate, and further, ideophones do not appear with nominal prefixes. So, in this domain of word formation at least, the dividing line between phrase- and word-level phonology is blurry. 

In the nominal domain, compounding in the language is trending towards affixation, as noted by \citet{Essegbey2006} for Gbe more broadly. Some common morphemes have undergone grammaticalization, taking on more functional and conventionalized usages. For example, the word \textit{(è)tɔ́} ‘father, parent, owner’ may be used as an agentive following nouns like `head' in ‘leader’ \textit{è-tà-tɔ́} (\GreenSC{NML}-head-father) and `oil' in ‘oil seller’ \textit{à-mĩ̀-tɔ́} (\GreenSC{NML}-oil-father), and as a nominalizer following adjectives and verbs, as in ‘redness’ \textit{d͡ʒɛ̃̌-tɔ́} (red-father) and ‘smartness’ \textit{nṹ-ɲã́-tɔ́} (thing-know-father).\largerpage

Reduplication \is{reduplication} is productively used in Gengbe for two patterns illustrated in Table \ref{tab:RedupGengbe}. The stem in isolation is given in the first column. Full reduplication used to indicate pluractional and emphatic/intensive forms is given in the second column, and partial reduplication used to derive nouns and adjectives from verbs is given in the third column. For monosyllabic predicates, the nominal/adjectival pattern only fully reduplicates Consonant-(Glide)-Vowel syllables, as in (\ref{tab:RedupGengbe}a--c). Consonant-Liquid-Vowel syllables do not copy the liquid in the reduplicant, as in (\ref{tab:RedupGengbe}d--e). Note that this is different from Ewe, which simplifies both CLV and CGV syllables to CV reduplicants \citep{Stemberger&Lewis1986}. Examples (\ref{tab:RedupGengbe}e--f) show how the nominal/adjectival pattern avoids reduplicants of the form CLV, but can reduplicate multi-syllabic (CVCV) forms. This distinction can help us to separate serial verb or verb-particle constructions, like \textit{d͡ʒɾà ɖó} (repair do), from derivational morphology, as in \textit{mã̀-sà} (\GreenSC{NEG}-sell) `unsold'.\footnote{In addition to these phonological features, the nominal/adjectival forms involve word order differences. While the reduplicated forms in the former pattern do not have an effect on word order, those in the latter pattern do. For nominal and adjectival forms, the head is pre-posed. The order of Verb and Object in a Gengbe verb phrase is usually V-O, yet in such a noun phrase it is O-V, as in ‘gift, giving’ \textit{è-nṹ-nã́-nã́} (\GreenSC{NML}-thing-give-\GreenSC{REDUP}), wherein the reduplicated form can be interpreted as an adjective ‘a given thing = gift’ or as a noun ‘the giving of a thing = giving.’}

\begin{table}
\caption{Reduplication in Gengbe}
\label{tab:RedupGengbe}
 \begin{tabularx}{.8\textwidth}{llll}
  \lsptoprule
    &  Isolation & Pluractional/Emphatic & Nominal/Adjectival \\
  \midrule
a. & \textit{βù}  & \textit{βù βù}   & \textit{βù-βù} or \textit{βù-βǔ}  \\
& `open' & `many people open,'  & `opening, opened' \\
& & `many things opened' & \\

b. & \textit{vɔ̃̌} & \textit{vɔ̃̌ vɔ̃̌} & \textit{vɔ̃̌-vɔ̃́}  \\
& `be scared' & `very scared'  & `scared, fear' \\

c. & \textit{bjǒ} & \textit{bjǒ bjǒ} & \textit{bjǒ-bjó} \\
& `ask' & `repeatedly ask' & `asking, asked' \\

d. & \textit{gblɛ̃̌}  & \textit{gblɛ̃̌ g͡blɛ̃̌}  & \textit{g͡bɛ̃̌-g͡blɛ̃́} \\
& `spoil' & `completely spoiled' & `spoilage, spoiled' \\

e. & \textit{d͡ʒɾà ɖó} & \textit{dʒɾà ɖó d͡ʒɾà ɖó} & \textit{dʒà-d͡ʒɾà ɖó} \\
& `repair' & `repeatedly repair' & `repairing, repaired' \\

f. & \textit{mã̀-sà} &  & \textit{mã̀-sà-mã̀-sà} or \\
& `not sell' & - & \textit{mã̀-sà-mã̀-sǎ} \\
&  &  & `unsold' \\

  \lspbottomrule
 \end{tabularx}
\end{table}

These two patterns are distinct. First, while the nominal/adjectival pattern involves one base and one prefixed reduplicant, the pluractional/emphatic pattern can be reduplicated indefinitely and creatively. Next, the tone pattern for High tone stems in the nominal/adjectival is Rising-High, the result of a regular phonological rule~-- a prohibition on a sequence of Rises that is active within words~-- while the pluractional/emphatic pattern shows a series of Rises, as in (\ref{tab:RedupGengbe}b--d) \citep{LotvenandBerkson2019}. These observations evince different structures: the pluractional/emphatic pattern is syntactic reduplication, or iteration of the same syntactic constituent, while the nominal/adjectival pattern is morphological, with the reduplicant integrated into the same prosodic word as the stem.\largerpage[2]

There are some other reduplicative \is{reduplication} patterns in the language, for example, a productive pattern with \textit{síá} ‘all', as in ‘everything’ \textit{è-nṹ-síâ-nṹ} (\GreenSC{NML}-thing-all-thing), with deletion of \GreenSC{NML} marking and \isi{compensatory lengthening} of the preceding /a/, preserving the High-Low tone pattern (\textit{è-nṹ-síá-è-nṹ} \rightarrow \textit{è-nṹ-síâ-nṹ}). These and other reduplicative patterns are described for \ili{Ewe} in \citet{Ameka1999}. Far from exceptional, reduplication offers some of the most productive morphological patterns in Gengbe, with unique morpho-phonological processes that shine a light on the structure of the language. 

Tone matters to Gengbe morphology as well, for example, in distinguishing the 1st person plural \textit{mĩ́} from the 2nd person plural \textit{mĩ̀}, the 3rd person singular subject \textit{é}, from the object \textit{è}, and ‘I will’ from ‘I should’ in (\ref{Ex.Iwilleat}).\footnote{The \textsc {3sg} clitic assimilates in height, nasality, and ATR to the previous vowel but retains Low tone, for example in \textit{kè-è} `spread-\textsc{3sg},' \textit{tú-ì} `close-\textsc{3sg},' and \textit{sɛ̃́-ɛ̃̀} `bear-\textsc{3sg} (fruit)'} Tone is also part of the nominal/adjectival reduplicative template discussed above.

\ea \label{Ex.Iwilleat}
\gll \textit{mã̌}/\textit{mã́ ɖù nṹ}\\
I.will/I.should eat thing\\
\glt  ‘I will/should eat.’
\z

In addition, there is an alternative reduplicative template for Low tone Gengbe verbs wherein the second syllable has Rising tone, as in \textit{kù.kù} or \textit{kù.kǔ} ‘uprooting/uprooted,’ as is also shown in Table (\ref{tab:RedupGengbe}a). This pattern is similar to reduplicative templates in \ili{Ewe}, where a suffixed High tone lengthens syllables with High tone and derives Rising tone from Low tone syllables \citep{Ameka2012}.

Although Gengbe has few instances of inflectional morphology, its derivational morphology makes use of compounding, reduplication, and tone. Similar phenomena are found in Yoruba.

\subsection{Yoruba}

Yoruba [iso 639-3: yor] is a Benue-Congo language spoken in Southwestern Nigeria, Benin, and Togo. Similar to Gengbe, Yoruba has been described as having a rich inventory of ideophones, even in early works such as \citet{Awoyale1978, Awoyale1981, Awoyale1989} and \citet{Rowlands1970}, but also \citet{Akanbi2014}. Yoruba data for this paper are from these sources as well as from the second author, a native speaker of Yoruba; we direct the reader to these sources for further research on Yoruba ideophones, as well as for lists of examples. We use the Yoruba orthography for the examples in this paper. Yoruba has 7 oral vowels [i e ɛ a ɔ o u] and 5 nasal vowels [ĩ ɛ̃ ã ɔ̃ ũ], with [ã] and [ɔ̃] being allophones of the same phoneme. The mid vowels /ɛ/ and /ɔ/ are represented orthographically as 〈ẹ〉 and 〈ọ〉, respectively, while nasality is represented with 〈n〉 after the vowel, such that /ĩ/ is represented orthographically as 〈in〉. Yoruba has 3 tones: High, Mid, and Low. High tone is conventionally marked with an acute accent (x́), Low tone with a grave accent (x̀), and Mid tone is unmarked.\il{Gengbe|)} 

\subsection{Yoruba morphology}

Similar to Gengbe,\il{Yoruba|(} typological work has categorized Yoruba as an isolating language due to its limited inflectional morphology, and Yoruba morphology is characterized by compounding, affixation, and reduplication (\citealt{Adewole1995, PulleyblankandAkinlabi1988, Schleicher1987}). Tonal morphology, or word formation that uses tone as part or all of its exponence, is commonplace in Yoruba. In addition to playing a role in the reduplicative templates to be discussed in this chapter, tone can mark features in the pronominal clitic system \citep{AkinlabiandLeiberman2001}. For example, the 2nd person singular subject pronoun \textit{o} is distinct from the 3rd person \textit{ó}, and the 3rd person plural subject pronoun \textit{wọ́n} is distinct from the pronoun \textit{wọn}. 

Yoruba commonly derives words through \isi{compounding}. Compounding can result in the formation of complex verbs through Verb+Verb or Verb+Noun constructions (\citealt{SchleicherandSchleicher1990, Seidl2000}) or in the creation of complex nouns through Noun+Noun constructions (\citealt{Adewole1995}; \citealt{Eleshin2021}). As a result of a vowel hiatus constraint in the language, compounding can create an environment for vowel assimilation across morpheme boundaries (\citealt{Adewole1995}) or vowel elision, especially in compound verbs (\citealt{SchleicherandSchleicher1990, Seidl2000}). Compound verbs are created through Verb+Noun constructions, as in \textit{wíjọ́} ‘complain', derived from \textit{wí} ‘speak’ + \textit{ẹjọ́} ‘case', or \textit{wolé} ‘inspect a house', derived from \textit{wo} ‘look’ and \textit{ilé} ‘house', both with elision of the noun’s initial vowels, \textit{e-} or \textit{i-}, respectively. Compound noun examples (Noun+Noun constructions) include the following words starting with the noun \textit{owo} ‘money’: (1) \textit{owoorí} ‘tax', with \textit{orí} ‘head', (2), \textit{owooṣù} ‘salary', with \textit{oṣù} ‘month', and (3) \textit{owoolé} (\textit{owo}+\textit{ilé}) ‘rent', with \textit{ilé} ‘house'. For the second members of these compounds, there is an assimilation of the initial vowel to the preceding vowel quality. In addition to orthographic conventions for compounds and phrases in Yoruba, compounds can often be identified phonologically by their hiatus resolution strategies, and semantically by their strict collocations with often conventionalized meanings. 

As an example of affixation, deverbal noun constructions typically take nominalizing prefixes, as in \textit{a-kọ̀wé} (\GreenSC{AG}-write) 'secretary' and \textit{ì-kọ̀wé} (\GreenSC{INS}-write) `pen.' With the base verb \textit{kọ̀wé} found in both examples, the former employs the agentive prefix \textit{a-} while the latter has the instrumental prefix \textit{ì-}. Some prefixes have more than one functional usage, as with \textit{ì-}, which is an instrument for performing an action in the example above, but which can indicate the act of performing the action in other contexts (\citealt{Adewole1995, Awobuluyi2008, Bamgbose1990}). 

Reduplication \is{reduplication} functions productively in several Yoruba word formation processes (\citealt{Ehineni2017, Pulleyblank2009}), as illustrated in Table \ref{tab:RedupYoruba}. Notably, prosaic words may fully reduplicate for intensity, as in the adverbs in (\ref{tab:RedupYoruba}a--b), or to form agentive (nominal) constructions from verbs, as in (\ref{tab:RedupYoruba}c--d). Partial reduplication occurs in examples (\ref{tab:RedupYoruba}e--f), as these deverbal nouns reduplicate only the first consonant of the word, followed invariably by the vowel /i/ with High tone. In examples (\ref{tab:RedupYoruba}g--h), the morpheme \textit{kí} ‘any’ intervenes between the stem and reduplicant (with assimilation and elision), similar to the \textit{síá} ‘all’ construction in Gengbe, discussed in Section \ref{Gengbe Morphology}.

\begin{table}
\caption{Yoruba Reduplication}
\label{tab:RedupYoruba}
 \begin{tabularx}{.9\textwidth}{lllll}
  \lsptoprule
   & Stem    & Gloss       & Reduplicated form & Gloss         \\
   \midrule
a. & \textit{tayọ̀}   & `joyfully'    & \textit{tayọ̀tayọ̀}        & `very joyfully' \\
b. & \textit{díẹ̀}    & `little'      & \textit{díẹ̀díẹ̀ }         & `very little'   \\
c. & \textit{pẹja }   & 'kill fish   & \textit{pẹjapẹja }         & `fisherman'     \\
d. & \textit{paná}    &`quench fire' & \textit{panápaná}        & `fireman'       \\
e. & \textit{jẹ  }    & `eat'         & \textit{jíjẹ }             & `eating'        \\
f. & \textit{gbàdúrà} & `pray'        & \textit{gbígbàdúrà}        & `praying'       \\
g. & \textit{ọmọ}     & `child'       & \textit{ọmọkọ́mọ}          & `any child'     \\
h. & \textit{ilé}     & `house'       & \textit{ilékílé  }         & `any house'    \\
 \lspbottomrule
 \end{tabularx}
\end{table}

The ‘any’ structure in (\ref{tab:RedupYoruba}g--h) can also extend to some ideophones such as \textit{tìmùtìmù-kí-tìmùtìmù} ‘any mattress', where ‘mattress’ is the already reduplicated ideophone \textit{tìmùtìmù}. \citet{Awoyale1981} notes that this type of reduplication and infixation is permitted only for ideophones which are already functioning as nouns in Yoruba.
 
Though lacking inflectional morphology, Gengbe and Yoruba are rich with derivational morphology, making use of compounding, affixation, reduplication, and tone. The following section explores similar processes used in forming ideophones in these two languages.\il{Yoruba|)}

\section{Derivational morphology in Gengbe and Yoruba ideophones} \label{sec_derivatioanlgengbe}

\subsection{Qualitative iconicity and related word forms}
\begin{sloppypar}
Like prosaic words, some pairs or sets of ideophones have related forms and meanings, often differing minimally in their depictions \citep{Diffloth1972}.\footnote{While prosaic words are largely used to \textit{describe}, representing events and states with arbitrary symbols, ideophones \textit{depict}, offering vivid, gradient, and specific representations of events and states to be interpreted. Further discussion is found in \citet{Dingemanse2015}, especially as it relates to ideophones.} These iconic relationships between the sounds of a word and its meaning are referred to by \citet{CarlingandJohansson2015} as qualitative \isi{iconicity}. Such associations may be language specific, such as phonoaesthemes~-- combinations of sounds with similar meaning \citep{Firth1930}~-- or they may be universal, such as the frequency code, which relates smallness with high pitch and high vowels cross-linguistically (\citealt{Ohala1984, Ohala1994}).
\end{sloppypar}

Paradigmatic differences in consonants, vowels, and tone all play a part in influencing the depiction offered by an ideophone. In Yoruba, for example, \textit{gbìm} and \textit{kìm} both describe degrees of heart palpitation, with the former associated with a heavier heartbeat than the latter. In Gengbe, \textit{pájì} and \textit{tájì} evoke the sound of a slap, where the former slap struck the front of the recipient’s face, and the latter struck the person’s cheek. Likewise, a punch to the stomach or cheek \textit{g͡bùm} differs from a punch to the ribs \textit{gùm}, which differs from a punch to the face \textit{gìm} -- which High tone renders ineffectual as an attack (\textit{gím}). Tone alone may make the difference between depictions. For example, in Yoruba, \textit{táló} describes the sound a tiny object makes when it drops into a river, while Rising tone in \textit{tàlǒ} indicates that the object created a ripple. Similarly reliant on tone for interpretation, \textit{káló} evokes the sound or motion of food being quickly swallowed, and \textit{kàlò} is associated with the metaphorical swallowing of money, such as a gambler or wasteful person may do. 

If we define morphological patterns as regular differences in word form associated with regular differences in word meaning, these types of alternations do not clearly meet the mark. For such morphological analysis, one might split onsets or tones from rhymes and assign them meanings. Yet, each of these units does not clearly relate to an individual piece of the meaning, and such sound-meaning correspondences are not usually systematic enough to be considered morphology, rather than, say, sound symbolism \citep{Hintonetal1994}. In other words, the above examples appear more like observations of individual relationships between word forms, rather than productive or predictable patterns of form-meaning correspondence, which can be seen elsewhere in the morphology of ideophones. However, it is worth noting that minor changes in meaning associated with minor changes in word form are also typical in, for example, the pronoun systems of Gengbe and Yoruba, where tone also plays idiosyncratic roles; and all morphology varies in productivity.

\subsection{Quantitative iconicity and event semantics}

The event or state that an ideophone \is{iconicity} depicts may be tied to its syllable and/or word structure. These types of syntagmatic alternations and shape-based depictions fall into the category that \citet{CarlingandJohansson2015} call quantitative iconicity, and while such processes are available to other word categories, ideophones are often this type of creative-yet-conventionalized expression. For example, in Gengbe, the continuous blowing of wind is lengthened iconically \textit{βù::}, but when wind hits the resistance of trees, and there is a repetitive rather than continuous quality to its sound/motion, a coronal stop is added to form two syllables, and reduplication is employed \textit{βùdùβùdù}. Lengthening and reduplication are used in both Gengbe and Yoruba to similar effect, yet Gengbe ideophone \isi{reduplication} is likely syntactic, while later in Section \ref{Redupandtemplatic}, we present some examples of morphological reduplication in Yoruba.

In Gengbe, when a person walks in water, sliding forward without raising their feet, a single syllable is reduplicated \textit{vlàvlà}, but when that person moves their legs through deeper water, this exertion is depicted with a lengthening of the reduplicated rhyme from one to two syllables through glide insertion \textit{vlàjàvlàjà}. When the person picks their feet out of the water, and there is a \is{iconicity} distinct raising and lowering of the leg through the surface of the water, the action is depicted with syllables containing onset stops and a Low-High-Low-High tone pattern, \textit{dàbúdàbú} -- similarly, elephants in mud walk \textit{d͡ʒàgùd͡ʒàgù}. Lengthening through glide insertion seems to be associated with slow or protracted movement, such as in the motion of a clumsy person walking leisurely \textit{βlàjàβlàjà}.

Reduplication \is{reduplication} and lengthening are productively used in the intensification of Yoruba ideophones as well. For example, the act of staring at something is intensified through lengthening \textit{sì::}, as is the depiction of a finished item \textit{po::}, or a single punctual event, such as the sound that comes from hitting two things together \textit{gbà::}. \citet{Awoyale1978} notes that Yoruba ideophones can be reduplicated for recurrent actions, such as \textit{gbì} (a single heavy tread of one's boots) and \textit{gbìgbì(gbì...)} (numerous heavy treads of one's boots). Some ideophones, such as \textit{gbà::} (two things hitting against each other), can be lengthened for intensity \textit{gbà::} or reduplicated indefinitely \textit{gbà::gbà::(gbà::)} to convey the number of occurrences, in this case strikes.

\begin{sloppypar}
Some relationships between word formation and event semantics walk us more clearly into the domain of regular, predictable morphology, for example, in the choice of reduplication or lengthening for intensification in Gengbe. To compare this with other morphological processes of the language, we first consider how a predicate’s lexical \isi{aspect} influences the aspectual morphology it takes. Bare Gengbe predicates describing events have a default past-time interpretation, while bare predicates describing states have a default present-time interpretation. \ili{Gbe languages} often use reduplication in forming progressive constructions, varying also along the divide between events and states. \citet{Essegbey1999}, writing on \ili{Ewe}, claims that true stative verbs are identified as those that resist reduplication for progressive aspect. A similar correspondence between lexical aspect and reduplication can be found in Gengbe progressive constructions as well. 
\end{sloppypar}

For ideophones, the choice of lengthening or reduplication also reveals information about the lexical aspect of the depicted event or state. The productive link between these processes can be illustrated with a stem and its possible modifications, shedding a light on regular morphological patterns among ideophones. For example, in Gengbe, a single step by a big man \textit{kìm} can be reduplicated (indefinitely) to indicate the \is{iconicity} many steps involved in walking, \textit{kìmkìm(kìm)}, while the duration of an object sitting still is intensified through lengthening \textit{kpó::}. Likewise, slow or quiet actions are emphasized or exaggerated  through lengthening \textit{blèù::}, and intense darkness lengthens \textit{jìbɔ̀::}. 

The link between repetition of punctual events and reduplication, as well as between lengthening and slow motions or states, is illustrated with the Gengbe ideophone \textit{kɛ́dɛ́}. A single quiet or gentle motion is lengthened, \textit{kɛ́dɛ́::}, a hunter stalking prey carefully is lengthened and repeated, \textit{kɛ́dɛ́:: kɛ́dɛ́::}, and quick movements that are not so quiet skip the lengthening and surface with reduplication only, \textit{kɛ́dɛ́kɛ́dɛ́}. Not all reduplicated ideophones have a clear stem form from which they are derived, as some expressions are obligatorily reduplicated, however, \citet{Ameka2001} has noted that the general link between event repetition and reduplication in \ili{Ewe} ideophones holds in these forms as well.

These examples show that depictive lengthening and reduplication may be linked to lexically stored and aspectually bound differences in ideophones, similar to the influence of lexical aspect on verbal morphology. Some of these transformations are less productive, such as glide insertion, while others are more productive, such as full reduplication and lengthening for intensification. Reduplication, in some functions, is linked to punctual event repetition, and it is worth noting that this iteration is similar to compounding if we consider that each reduplicant depicts a different event, for example, a punch in a boxing match or a step through mud, repeated. In Gengbe, such full reduplication is likely syntactic iteration rather than morphological concatenation, as these processes can be applied creatively and indefinitely (unlike the one base + one reduplicant patterns given in Table \ref{tab:RedupGengbe}) and fail to exhibit word-level phonological rules such as a prohibition \is{Obligatory Contour Principle} on a series of Rising tones. The following section considers more clearly compositional ideophone compounds.

\subsection{Compounding and complex events}

Some events are composed of more than one part, so the iconic depiction of events leads to some ideophones that are decomposable. In Gengbe, the ideophone \textit{βù::g͡bìm} can be split in two, where the first syllable indicates the motion of a large object falling, and the second denotes the sound of it hitting the ground. Similarly, in Yoruba, \textit{gbù:gbà:} describes a car that loses control and hits something, where \textit{gbù:} depicts the wobbling movement of the car, and \textit{gbà:} denotes both the action and sound of the collision. 

Compounding \is{compounding} in ideophones can depict different participant involvement as well. In Gengbe, the motion of a slim person taking a step/walking is \textit{srá/srásrá}, while the motion of a large person taking a step/walking is \textit{g͡bì/g͡bìg͡bì}. When these individuals are stomping through mud side-by-side, their steps are intermingled \textit{g͡bìsrág͡bìsrá}. These are examples of collocations -- words that often occur together -- but they are not clearly morphologically bound into a prosodic word. We offer no evidence of clearly-absent word-level phonology which would suggest a morphological analysis, so these examples may, like ideophone reduplication in the language, be better described as phrases rather than words.

Yoruba offers some conventionalized examples of compounding in ideophones. For example, \textit{gbẹ̀dẹ̀múkẹ́} depicts a festive mood with maximal enjoyment and entertainment, where \textit{gbẹ̀dẹ̀} depicts a simple/easy situation or event, and \textit{múkẹ́} depicts a relaxing mood or situation. As another example, with some metaphorical extension, we can understand \textit{gbangbakedere} to depict a secret that was exposed, composed of \textit{gbangba}, which depicts openness (especially a part of a building), and \textit{kedere}, which depicts ‘clearly.' 

As with simplex ideophones, these complex ideophones function like words rather than phrases, and unlike phrases, they can occupy the positions of various categories, as illustrated by (\ref{InlineEx1}) below. \textit{Gbẹ̀dẹ̀múkẹ́} acts as noun in (\ref{InlineEx1}a) (only nouns can occupy this position in Yoruba, \citealt{Bamgbose1990}), an adverb in (\ref{InlineEx1}b), and an adjective in (\ref{InlineEx1}c) with a difference in tone that does not affect interpretation. \textit{Gbangbakedere} acts as noun in (\ref{InlineEx1}d) but as an adverb in (\ref{InlineEx1}e). 

\ea \label{InlineEx1}
\begin{xlist}
    \ex
    \gll Gbẹ̀dẹ̀múkẹ́    ni     gbogbo wa    wà.\\
    \textsc{ideo}            \textsc{comp}   all    \textsc{1pl}    be \\
   \glt  `We are all in a festive/relaxing mood.'

    \ex	
    \gll Ó  rọ̀    gbẹ̀dẹ̀múkẹ́.\\
        \textsc{3sg} simple \textsc{ideo}\\
    \glt   `It is simple in a fun way.'

        \ex 
        \gll Ẹ		kú	    gbẹdẹmukẹ	òpin ọ̀sẹ̀.\\
        \textsc{2pl} greet   \textsc{ideo}	    end week\\
   \glt     `Have a happy fun weekend!'

        \ex 
        \gll Gbangbakedere	ni	  l-ójú	   Olódùmarè.\\
        \textsc{ideo}		    \textsc{comp}  before-eyes   God\\
   \glt     `It is wide-open before God.'

        \ex 
        \gll Ó	ti	       hàn	    gbangbakedere.\\
    \textsc{3sg}	\textsc{pfv}       expose	\textsc{ideo}\\
   \glt     `It has been exposed openly/It is no longer a secret.'
\end{xlist}
\z

These examples, taken in conjunction with those of prosaic compounds in Section \ref{Sec-LangBackgrounds} show how compounding applies across the lexicons of these two languages. We now turn to patterns of reduplication that reveal further complexities, as well as more clearly morphological phenomena: templatic and tonal morphology in ideophones.

\subsection{Reduplication and templatic morphology}
\label{Redupandtemplatic}

Reduplication \is{reduplication} is commonplace in Gengbe and Yoruba and is used, among other functions, to form nouns and adjectives from Gengbe verbs and to mark emphatic and agentive forms in Yoruba. In this subsection, we highlight regular, productive patterns in Yoruba ideophone reduplication, contrasting the full reduplication found in pluractional formation with a tonal template that marks counter-expectation and overwrites lexical tone. Such templates are non-arbitrary, but not necessarily iconic, in that they are predictable and regular patterns within the language system, though the nature of their tone and shape does not necessarily depict events and states. \citet{Dingemanseetal2015} refer to this type of non-arbitrariness \is{iconicity} as ``systematicity''.

Commonly marked with reduplication across the world’s languages, pluractionality refers to the expression of multiplicity, usually of occurrence or participant (\citealt{Newman1980, Newman2012}). In Table \ref{tab:RedupYoruba} above, full reduplication of Yoruba adverbs was shown to indicate intensity, and full reduplication of verbs was shown to indicate agentivity. As shown in Table \ref{tab:PluractionalityYoruba} below, full reduplication of Yoruba ideophones indicates pluractionality. While these forms may be used to depict many flat, huge, fat, small, or bulging things, pluractional meaning is context-dependent. For example, (\ref{tab:PluractionalityYoruba}a) could be used to depict multiple flat objects, or multiple people making something flat, like a team of workers flattening a section of road.

\begin{table}
\caption{Pluractionality in Yoruba}
\label{tab:PluractionalityYoruba}
 \begin{tabular}{llll}
  \lsptoprule
   & Stem & Pluractional form        & Gloss   \\
   \midrule
a. & \textit{pẹlẹbẹ}         & \textit{pẹlẹbẹpẹlẹbẹ}             & `flat'    \\
b. & \textit{kọ̀bìtì}         & \textit{kọ̀bìtìkọ̀bìtì }          & `huge'    \\
c. & \textit{bẹ̀rẹ̀kẹ̀tẹ̀}       & \textit{bẹ̀rẹ̀kẹ̀tẹ̀bẹ̀rẹ̀kẹ̀tẹ̀} & `fat'     \\
d. & \textit{ríndín }        & \textit{ríndínríndín}             & `small'   \\
e. & \textit{rògòdò}         & \textit{rògòdòrògòdò }            & `bulging' \\
\lspbottomrule
 \end{tabular}
\end{table}

The pattern in Table \ref{tab:PluractionalityYoruba} contrasts with the counter-expectation \is{templatic morphology} template illustrated in Table \ref{tab:CounterexpectationYoruba}, which replaces the tone pattern \is{grammatical tone} of the stem. This overwrite of the underlying tone pattern happens regardless of the stem tone or number of syllables, and the resulting pattern is that of all Low tone on the first copy of the stem and all Mid tone on the second. In (\ref{tab:CounterexpectationYoruba}a--c), Low tone stems with two, three, and four syllables, surface with this Low-Mid pattern, as do Mid and High tone stems in (\ref{tab:CounterexpectationYoruba}d--e). Neither stem copy always surfaces faithfully, so it is not possible to determine which is the base and which is the reduplicant for this construction. While the stems in (\ref{tab:CounterexpectationYoruba}a) and (\ref{tab:CounterexpectationYoruba}e) are tonal minimal pairs, their reduplicated forms are identical, and ambiguity is avoided through predicate choice in two collocations -- \textit{dùn} ‘sweet’ usually precedes the former, and \textit{rí} ‘appear’ usually precedes the latter. This prosodic template is reminiscent of the nominalizing reduplication pattern in (\ref{tab:RedupYoruba}e--f) where vowel quality is overwritten, surfacing consistently as /i/ for the reduplicant.

\begin{table}
\caption{Counter-expectation in Yoruba}
\label{tab:CounterexpectationYoruba}
 \begin{tabularx}{.9\textwidth}{lllll}
  \lsptoprule
 &  Stem & Gloss & Reduplicated Form    & Gloss           \\
 \midrule
a. & \textit{rìndìn}         & `sweet' & \textit{rìndìnrindin}         & `unusually sweet' \\
b. & \textit{kọ̀bìtì  }      & `huge'  & \textit{kọ̀bìtìkọbiti }       & `unusually huge'  \\
c. & \textit{bẹ̀rẹ̀kẹ̀tẹ̀}   & `fat'   & \textit{bẹ̀rẹ̀kẹ̀tẹ̀bẹrẹkẹtẹ} & `unusually fat'   \\
d. & \textit{pẹlẹbẹ  }       & `flat'  & \textit{pẹ̀lẹ̀bẹ̀pẹlẹbẹ }     & `unusually flat'  \\
e. & \textit{ríndín }        & `small' & \textit{rìndìnrindin     }    & `unusually small' \\
\lspbottomrule
 \end{tabularx}
\end{table}

\begin{sloppypar}
To draw a comparison with the ‘any’ \isi{reduplication} in (\ref{tab:RedupYoruba}g--h), it is worth also mentioning another reduplicative template that applies to limited derived ideophones in Yoruba. Some ideophones, such as \textit{kẹ̀ǹbẹ̀} ‘loose/spacious', have a CVNCV shape with all Low tones, a pattern that may involve reduplication, as in \textit{gbòǹgbò} ‘deeply rooted’. These ideophone examples are not readily decomposable into constituent morphemes, but others are derived from monosyllabic, prosaic predicates, such as \textit{gàǹgà} `conspicuously tall', which is derived from the prosaic verb \textit{ga} ‘tall', and \textit{làǹlà} `conspicuously heavy and big’, derived from \textit{ńlá} ‘big’. These examples show two different paths to mimicking the CVNCV \is{templatic morphology} shape of ideophones. For \textit{ga/gàǹgà}, an \textit{-n-} is inserted between the base and reduplicant, while in \textit{ńlá/làǹlà}, the initial \textit{n-} is deleted, while the medial \textit{n-} remains. In addition to conforming to this CVNCV template through insertion or deletion, both examples show that the underlying tones of the base, here Mid or High respectively, are overwritten by Low tone across the entire word. Such reduplication offers an example of systematicity, as a template that even prosaic words can conform to, through different means, to make their form non-arbitrarily linked to meaning, in this case \textit{conspicuousness}. Further evidence of morphology in reduplication is seen in other West African languages like \ili{Siwu} and \ili{Emai}, which \citet{Dingemanse2015} uses to argue that reduplication bridges description and depiction.
\end{sloppypar}

This snapshot of word formation processes, from the more iconic and idiosyncratic to the more systematic and productive, offers a look at the parallels between word formation in ideophones compared to the rest of the lexicon. These parallels include the use of compounding, affixation, reduplication, and tone in deriving meaning, and although some processes appear more syntactic, others suggest morphological processes active in the ideophones of Gengbe and Yoruba. 

\section{Conclusion} \label{Sec_GengbeConclusion}
\begin{sloppypar}
Since ideophones, like other words, inhabit the lexicon, derivational morphology~-- word formation that occurs there~-- is a natural point of comparison. Stepping past spontaneous and performative vocal gestures, we examined the fuzzy line between depictive word formation strategies and the conventionalized forms that offer us evidence for derivational morphology. In doing so, we considered various word-formation strategies, including those linked to qualitative and quantitative iconicity, as well as to systematicity, and those making use of compounding, reduplication, and tone.
\end{sloppypar}

Qualitative \isi{iconicity}, or connections between the sounds of a word and its meaning, are commonplace and conventionalized, yet when regularity is discovered, patterns are often treated within the domain of sound symbolism, rather than morphology. In this discussion, a parallel was made between form-meaning (in this case, tone-meaning) correspondences that alter the details of events and states depicted by ideophones and those that mark the interpretation of person and number in the pronoun systems of Gengbe and Yoruba.

Quantitative iconicity, or connections between word shape and meaning, was also discussed, where regular links can be found between a morphophonological process and the event or state depicted. Lengthening is associated with slow movements and states in Gengbe, while reduplication is associated with repetition and pluractionality, as it is in \ili{Ewe}, Yoruba, and many other languages. Much like individual predicates, which are compatible with different morphological structures based on their lexical aspect, the word formation processes available to ideophones (lengthening and reduplication) are also dependent on lexically-defined aspectual differences between them, offering another point of similarity between ideophones and other lexical items. 

Compounding -- combining words to derive complex concepts -- is also used in Gengbe and Yoruba ideophones. Compounds range from those that depict complex events, to those which are less clearly compositional, some involving metaphorical extension. Such compounding is available across the lexicon, to ideophones and prosaic words alike.

Reduplication -- copying all or part of a morpheme -- is particularly widespread in Gengbe and Yoruba, and is employed by derivational processes across the lexicon. We offered examples of full and partial reduplication, some with infixation and some conforming to particular tonal or prosodic templates, to highlight the complexities found in the morphology of ideophones and prosaic words alike in West African languages. We discussed systematicity, or non-arbitrariness, linked to regularity within the language system itself, and offered examples in Yoruba of an ideophone template that can even accept some prosaic words as stems. A possible continuation of this research could examine whether ideophones favor templatic and replacive strategies, those forcing stems to conform to a specified form. In future research, formalisms used to analyze reduplication and non-concatenative word formation, such as Construction Morphology (\citealt{Booij2010, Goldberg2006}) and Optimality Theory (\citealt{McCarthy&Prince1995, PrinceandSmolensky1993}), are available for more pointed comparison between the patterns found in ideophones and those found elsewhere. 

In focusing on these two isolating languages, we look to treat word formation in the lexicon as inclusive of ideophones. Previous research has been hampered by such omissions -- consider the following quote, ``Although it is conceivable that ideophonic expressions, particularly those employed to describe physical objects (e.g., \textit{roboto} `round') are an important source of adjectives in the language, ideophones are not considered in this study" (\citealt[86]{Madugu1976}). Asides like these bring truth to the assertion by \citet{Dingemanse2018} that the inclusion or exclusion of ideophones in analysis reshapes typology. Typology relies on more than just a list of which phenomena appear in which languages; it thrives on an understanding of how those phenomena work within as well as across languages.

Emphasis on the extralinguistic tendencies of ideophones sets them apart from the “regular” grammar of languages and may relegate them to exceptions or footnotes. Yet, language is naturally both arbitrary and non-arbitrary, and our linguistic analyses should be built to accommodate the whole lexicon, not just the places we can most readily build rules around. It is in this leaky corner of the grammar where ideophones thrive, and where they have much to give to the study of language. 

\section*{Abbreviations}
\begin{multicols}{2}
\begin{tabbing}
mmmm \= agentive\kill
\textsc{1, 2, 3} \> 1st, 2nd, 3rd person \\
\textsc{ag} \> agentive\\
\textsc{comp} \> complementizer \\
\textsc{ideo} \> ideophone\\
\textsc{ins} \> instrumental \\
\textsc{neg} \> negative\\
\textsc{nml} \> nominalizing prefix\\
\textsc{pfv} \> perfective\\
\textsc{pl} \> plural \\
\textsc{sg} \> singular\\ 
\textsc{redup} \> reduplicant 
\end{tabbing}
\end{multicols}

\section*{Acknowledgments}

This research would not be possible without Dr. Samuel Gyasi Obeng, our mentor and friend, who deserves credit both for his guidance and for his work on the list of Gengbe ideophones in the Appendix. We also thank Gabriel Mawusi for the Gengbe ideophone data in this chapter, who offered his intuitions and voice to the project. In revising this chapter, we owe a debt of gratitude to the excellent and detailed feedback from Christopher R. Green, Mark Dingemanse, and an anonymous reviewer.

\section*{Appendix: Gengbe ideophones}

\begin{tabbing}
glòd͡ʒòglòd͡ʒò \= (phen) speaking bitterly\kill
\textit{bàdàbàdà}                         \>(phen) speaking bitterly                                                                                \\
\textit{bɛ̂}                                \>(phon) goat bleat                                                                                       \\
\textit{ɕɥǐm}                                \>(phen) bird/plane rising; car speeding by; fast moving object                                \\
\textit{ɕɥǐmɕɥǐm}                           \>(phen) object repeatedly moving fast or rising                                             \\
\textit{dàbúdàbú}                         \>(phon) person walking in water and raising feet out of water                                               \\
\textit{d͡ʒàgùd͡ʒàgù}                     \>(phon) heavy person or elephant walking through mud                                                     \\
\textit{d͡ʒéd͡ʒé}                           \>(phon) people chattering loudly with tense or negative feeling                                          \\
\textit{d͡ʒì}                                \>(phen) light person taking a step                                                                        \\
\textit{d͡ʒìd͡ʒì}                           \>(phon) light person walking                                                                                 \\
\textit{d͡ʒɾà::}                             \>(phen/phon) large animal dashing away                                                                   \\
\textit{d͡ʒɾàʔ}                              \>(phen/phon) lion lunging                                                                                \\
\textit{g͡bã̀ĩ́:}                           \>(phon) large bell                                                                                         \\
\textit{g͡bàm̀}                              \>(phon) crash (lengthen [m] for a bigger crash)                                                         \\
\textit{g͡bì}                                \>(phon/phen) heavy person stepping in sand/mud/tall grass                                                  \\
\textit{g͡bì}                                \>(phon) large thing hitting the ground                                                                     \\
\textit{g͡bìg͡bì}                           \>(phon/phen) heavy person walking in sand/mud/tall grass                                                   \\
\textit{g͡bìm}                               \>(phon) heavy thing hitting the ground                                                                   \\
\textit{g͡bìsɾág͡bìsɾá}                   \>(phen/phon) a heavy and a light person walking together  \\
\> in sand/mud/tall grass                             \\
\textit{g͡bóg͡bó}                           \>(phon) people chattering loudly with chaotic but \\
\> positive feeling                                       \\
\textit{g͡bùm}                               \>(phon) loud gunshot (lengthen {[}m{]} for echo)                                                          \\
\textit{g͡bùm}                               \>(phon) punch to the stomach or cheek                                                                        \\
\textit{gìdìgìdì}                         \>(phen) squirming                                                                                        \\
\textit{gím}                                 \>(phon) ineffective punch                                                                                \\
\textit{gìm}                                 \>(phon) punch to the face                                                                                    \\
\textit{gìmgìm}                             \>(phen) heavy person walking quickly                                                                       \\
\textit{glàd͡ʒàglàd͡ʒà}                   \>(phen/phon) car on rough road; horse gallop; \\
\> clumsy person running                                      \\

\textit{glàmàglàmà}                       \>(phen/phon) person or object swaying side to side                                                       \\
\textit{glòd͡ʒò}                            \>(phen/phon) big truck stopping                                                                          \\
\textit{glòd͡ʒòglòd͡ʒò}                   \>(phen/phon) drunk person walking; car on a rough road; \\
\> truck passing                               \\
\textit{gɾàgɾà}                             \>(phen/phon) large animal running                                   \\
\textit{gɾɛ̀ù gɾɛ̀ù}                        \>(phon) person crunching something with teeth                                                            \\
\textit{gùdùgùdù}                         \>(phon) fast running water; boiling water; big river running       \\
\textit{gùm}    \>(phon) punch to the ribs                                                                                    \\
\textit{jɔ̀::}                                \>(phen) snake moving; water flowing                                                                      \\
\textit{kàtà}                               \>(phon) raindrop                                                                                         \\
\textit{kàtàkàtà}                         \>(phon) rainfall, especially on a roof                                                                    \\
  
\textit{kɛ́dɛ̂::}                             \>(phen) quiet/gentle action; setting something down; \\
\> opening a door                             \\
\textit{kɛ́dɛ́::kɛ́dɛ́::}                       \>(phen) hunter or predator stalking prey                                                                 \\
\textit{kɛ́dɛ́kɛ́dɛ́}                         \>(phen/phon) moving quickly and not quietly                                                              \\
\textit{kìm}                                 \>(phen) heavy person taking a step                                                                            \\
\textit{kìmkìm}                             \>(phen) heavy person walking quickly; \\
\> mid size person walking forcefully                                       \\
\textit{k͡pò}                                \>(phen) chopping meat                                                                                    \\
\textit{k͡pó::}                               \>(phen) object sitting still                                                                             \\
\textit{k͡pó::kpó::}                          \>(phen) sneaking quietly so as not to sleeping people                                                    \\
\textit{kɾìkɾì}                             \>(phen/phon) light animal (e.g. mouse) moving in the forest                                           \\
\textit{mĩ́ã̀ò}                            \>(phon) cat cry                                                                                          \\
\textit{mũ̂}                              \>(phon) cow moo                                                                                          \\
\textit{ɲâ::ɲâ::} \>(phen) sick/tired/injured person or animal walking \\
\> also \textit{blèù::, blèùblèù}\\
\textit{ɳã́ŋṹɳã́ŋṹ}                     \>(phen) person chewing                                                                                   \\
\textit{pájì}                               \>(phon) slap to the front of the face                                                                    \\
\textit{pê}                                \>(phon) small gunshot with echo                                                                          \\
\textit{pépé}                               \>(phon) small gunshot                                                                                    \\
\textit{pɛ́}                                  \>(phon) small thing falling or hitting the ground                                                                              \\
\textit{pípí}                    \>(phon) car horn                                                                                         \\
\textit{pô}                                \>(phon) slap on the bottom with cupped hand                                                              \\
\textit{ɸím}                                 \>(phen) light person taking a step                                                                           \\
\textit{ɸímɸím}                             \>(phen) light person walking quickly                                                                         \\
\textit{sɾá}                                 \>(phen) light person stepping in sand/mud/tall grass                                                      \\
\textit{sɾásɾá}                             \>(phen/phon) light person walking in sand/mud/tall grass                                                  \\
\textit{sss}                                  \>(phon) snake moving through grass                                                                                      \\
\textit{ʃuɾǐʃuɾǐ}                           \>(phon) people speaking a language that is not understood                                         \\
\textit{tâ}                                \>(phon) slap on the bottom with open hand                                                                \\
\textit{tájì}                               \>(phon) slap to the cheek                                                                                \\
\textit{tɛ́tɛ́}                               \>(phon) popping sound like a fire crackling or oil in a pan                                              \\
\textit{t͡ʃákút͡ʃákú}                     \>(phen/phon) person/cow/goat chewing                                                                     \\
\textit{t͡ʃàt͡ʃà}                           \>(phen/phon) cutting through grass/woods; chopping meat                                             \\
\textit{t͡ʃɛ̀t͡ʃɛ̀}                           \>(phon) leaves rustling                                                                                  \\
\textit{t͡ʃì::}                            \>(phon) frying sound                                                                                     \\
\textit{t͡ʃɾàt͡ʃɾà}                         \>(phen/phon) cutting quickly/intensely through grass/woods                                            \\
\textit{vlàjàvlàjà}                       \>(phon)  walking in tall grass; moving legs through water                                       \\
\textit{vlàvlà}                             \>(phen)  person walking in water and not raising feet                                                       \\
\textit{wô}                                \>(phon) dog bark                                                                                         \\
\textit{zɾìm}                                \>(phon) cutting                                                                                          \\
\textit{βì::}                                \>(phen) person/thing falling                                                                                    \\
\textit{βĩ́::βĩ́::}                       \>(phen) light person walking slowly                                                                          \\
\textit{βlàjàβlàjà}                       \>(phen/phon) clumsy person walking leisurely                                                             \\
\textit{βlìβlì}                             \>(phen/phon) people/animals/cars moving in a group                                                       \\
\textit{βù::}                                \>(phen/phon) wind blowing, big fire burning                                                              \\
\textit{βùdùβùdù}                         \>(phen/phon) wind meeting resistance (e.g. trees)                                         \\
                                                                
\textit{βùù g͡bìm}                         \>(phen/phon) large thing falling over (e.g. elephant, car, building) 
\end{tabbing}


{\sloppy\printbibliography[heading=subbibliography,notkeyword=this]}
\end{document}
