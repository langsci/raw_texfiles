\documentclass[output=paper,colorlinks,citecolor=brown]{langscibook}
\ChapterDOI{10.5281/zenodo.11091835}

\author{Taiwo Ehineni\affiliation{Harvard University}}
\title{Names as communicative acts: A study of Yoruba names} 
\abstract{This paper discusses the nature of names from an ethnopragmatic framework, with the aim of explicating how names are formed, their various cultural contexts as well as the critical functions they play in specific ethnolinguistic ecologies. Drawing theoretical perspectives from Samuel Gyasi Obeng’s work on African anthroponymy, data from the Yoruba context of naming are analyzed and discussed. It is argued that names are not just linguistic tags or labels, they have deep sociocultural undertones. They may show family situation, circumstances of birth, religious orientation, death situation, and profession. As Yoruba names reflect, names are communicative acts whose varied functions and meanings are informed by their contexts, situated with\-in specific cultural frames. Furthermore, I opine that names are not just linguistic forms in a sense of structure, but crucially, they are communicative acts in the sense of meaning.}

\IfFileExists{../localcommands.tex}{
   \addbibresource{../localbibliography.bib}
   % add all extra packages you need to load to this file

\usepackage{tabularx,multicol}
\usepackage{url}
\urlstyle{same}

\usepackage{listings}
\lstset{basicstyle=\ttfamily,tabsize=2,breaklines=true}

\usepackage{langsci-basic}
\usepackage{langsci-optional}
\usepackage{langsci-lgr}
\usepackage{langsci-osl}
% \usepackage{./langsci/styles/langsci-lgr}
% \usepackage{./langsci/styles/langsci-osl}
% \usepackage{langsci-gb4e}

\usepackage{tikz}
\usetikzlibrary{patterns,calc}
\pgfdeclarepatternformonly{south east lines}{\pgfqpoint{-0pt}{-0pt}}{\pgfqpoint{3pt}{3pt}}{\pgfqpoint{3pt}{3pt}}{
    \pgfsetlinewidth{0.6pt}
    \pgfpathmoveto{\pgfqpoint{0pt}{3pt}}
    \pgfpathlineto{\pgfqpoint{3pt}{0pt}}
    \pgfpathmoveto{\pgfqpoint{.2pt}{-.2pt}}
    \pgfpathlineto{\pgfqpoint{-.2pt}{.2pt}}
    \pgfpathmoveto{\pgfqpoint{3.2pt}{2.8pt}}
    \pgfpathlineto{\pgfqpoint{2.8pt}{3.2pt}}
    \pgfusepath{stroke}}
    
\usepackage{stmaryrd}
\usepackage{wasysym}
\usepackage{multirow}
\usepackage{caption}
\usepackage{subcaption}
\usepackage{mathrsfs}
\usepackage{qtree}

\usepackage{linguex}


   %pminos do not split footnotes
% \interfootnotelinepenalty=10000 %Footnote in Laporte chapters has to be split SN


%\DeclareIndexNameFormat{default}{%
%\nameparts{#1}%
%\usebibmacro{index:name}%
%{\index[names]}%
%{\namepartfamily}%
%{\namepartgiveni}%
% {}% L1
% {}% L2
%{\namepartprefix}% generates spurious space L3
%{\namepartsuffix}% generates spurious space L4
%}

%  {\DeclareIndexNameFormat{default}{%
%     \usebibmacro{index:name}{\index[names]}{#1}{#3}{#5}{#7}}}

%\DeclareIndexNameFormat{default}{%
%  \usebibmacro{index:name}{\sindex[nom]}{#1}{#3}{#5}{#7}}

%\DeclareIndexNameFormat{default}{%
%  \usebibmacro{index:name}{\sindex[person]}{#1}{#3}{#5}{#7}}
%\DeclareIndexNameFormat{default}{%
%\nameparts{#1} \usebibmacro{index:name}{\sindex[person]]}{\namepartfamily}{‌​\namepartgiven}{\nam‌​epartprefix}{\namepa‌​rtsuffix}}

%\newcommand{\smiley}{:)}

%\renewbibmacro*{index:name}[5]{%
%\usebibmacro{index:entry}{#1}%
%{\iffieldundef{usera}{}{\thefield{usera}\actualoperator}\mkbibindexname{#2}{#3}{#4}{#5}}}

% \newcommand{\noop}[1]{}

%remove for final
%\overfullrule=1mm

\newcommand{\tobi}[2]}}
\renewcommand{\S}[1]{\tobi{#1}{\textsc{*}}}

% this volume references
% puts: [this volume]
% already defined: \citetv
%\newcommand{\citepv}[1]{(\citeauthor{#1} \citeyear*{#1} [this volume])}
\newcommand{\citealtv}[1]{\citeauthor{#1} \citeyear*{#1} [this volume]}

%parentheses around example number
\newcommand{\pref}[1]{(\ref{#1})}

% in-text examples

\newcommand{\lnex}[1]{\textit{#1}} %target lang word
\newcommand{\lnlit}[1]{(lit.: `#1')} %literal reading
\newcommand{\lnlat}[1]{(#1)} % latinization
\newcommand{\lntrans}[1]{`#1'} %translation
\newcommand{\lnexl}[2]%
{\lnex{#1}{} \lnlat{#2}} % ex with latinization
\newcommand{\lnexlat}[3]{\lnex{#1}{} \lnlat{#2}{} \lntrans{#3}} % ex with latinization and tranl.

%ch01
\newcommand{\co}[1]{\mbox{\textbf{#1}}}

%ch09

\newcommand{\cyrbulg}[1]{\begin{otherlanguage*}{bulgarian}#1\end{otherlanguage*}}


%ch10
\newcommand{\nlp}{{\small NLP}}
\newcommand{\mwe}{{\small MWE}}
\newcommand{\rae}{{\small RAE}}
\newcommand{\lvc}{{\small LVC}}
\newcommand{\pos}{{\small P}o{\small S}}
%\newcommand{\todo}[1]{ \textcolor{red}{#1} }

%\renewcommand{\labelenumi}{\theenumi}
%\ainamefmt{{vv}{ll}{, ff}{, jj}} % fullname

\newcommand{\biberror}[1]{{\color{red}#1}}

\newcommand{\osenovaitem}{--~}
   %% hyphenation points for line breaks
%% Normally, automatic hyphenation in LaTeX is very good
%% If a word is mis-hyphenated, add it to this file
%%
%% add information to TeX file before \begin{document} with:
%% %% hyphenation points for line breaks
%% Normally, automatic hyphenation in LaTeX is very good
%% If a word is mis-hyphenated, add it to this file
%%
%% add information to TeX file before \begin{document} with:
%% %% hyphenation points for line breaks
%% Normally, automatic hyphenation in LaTeX is very good
%% If a word is mis-hyphenated, add it to this file
%%
%% add information to TeX file before \begin{document} with:
%% \include{localhyphenation}
\hyphenation{
    Beck-man
    Ngu-yen
    back-chan-nel
    back-chan-nels
    mo-not-o-nous
    ste-reo-typ-i-cal
}

\hyphenation{
    Beck-man
    Ngu-yen
    back-chan-nel
    back-chan-nels
    mo-not-o-nous
    ste-reo-typ-i-cal
}

\hyphenation{
    Beck-man
    Ngu-yen
    back-chan-nel
    back-chan-nels
    mo-not-o-nous
    ste-reo-typ-i-cal
}

   \boolfalse{bookcompile}
   \togglepaper[2]%%chapternumber
}{}

\begin{document}
\SetupAffiliations{mark style=none}
\maketitle

\section{Introduction}

The study of anthroponyms,\il{Yoruba|(} specifically African anthroponyms, has been investigated by a diverse range of scholars, creating a comprehensive body of knowledge that traces the sociocultural influences, linguistic complexities, and functional roles of names in African societies. Key contributions include research by \citet{Agyekum2006Sociolinguistics}, \citet{Makondo2009}, \citet{Mmadike2014}, and \citet{Obeng1997, Obeng1998, Obeng1999, Obeng2001}, among others. In this introduction, I review the contributions of these authors before turning to Yoruba naming in the remainder of this chapter.
 
\citet{Agyekum2006Sociolinguistics} embarked on an in-depth examination of personal names within the Akan community of Ghana. His study offered a sociolinguistic and cultural anthropological perspective, exploring the typology of Akan names, such as day names (names based on days of the week on which a child is born), circumstantial names (names based on parental experience, birth order, etc.), family names, and religious names, as well as their significance as markers of sociocultural identity. Agyekum's work elucidated the sociocultural implications and functional roles of names within the \ili{Akan} community. Makondo's study \citeyearpar{Makondo2009} paralleled Agyekum's approach, focusing on Shona anthroponyms -- an exploration underscored by a functional perspective. His research shed light on the Shona naming system and the social variables that shape it, using a pragma-semantic decompositional approach. His research accentuated how naming practices could provide insight into the sociocultural contexts of African societies.

Taking a linguistic lens, \citet{Mmadike2014} explored the linguistic processes that guide the formation of Àlà forenames given to male children among the Igbo ethnic group of Nigeria. His work detailed the simplification process that de-sententializes a sentence to generate a concise lexical form, using mechanisms like nominalization, vowel assimilation, and segment deletion. Mmadike's work contributed to the understanding of the linguistic complexities in Àlà names and the practical necessity for brevity that drives this process.

\begin{sloppypar}
Additionally, \citet{Obeng1997} explicates the linguistic formation of \is{hypocoristics} hypocoristic names (diminutive forms of names, including pet names) in Akan where he identifies morphological and phonological processes that occur in them. He argues that hypocoristization often involves morphological processes such as compounding and reduplication as well as (morpho)phonological processes such as deletion, vowel harmony, and tonal change. Thus, hypocoristic names, Obeng notes, offer a rich revelation of morphological status marking. Furthermore, Obeng underscores the different contexts in which the names are used, which could be among equals as well as in superior-to-subordinate or subordinate-to-superior interactive contexts. Obeng indicates that in a superior-to-subordinate context, hypocoristic names express affection, tenderness, playfulness, warmth, the idea of being loved or worth caring for. They may also denote the smallness of the referent. However, while hypocoristic names may indicate playfulness, oneness or solidarity among peers, they may also express disrespectfulness, unusual familiarity with the referent and unusual friendliness on the part of the speaker if used by a social subordinate in reference or in addressing a social superior. Thus, function is informed by context. The communicative contexts of use determine the functions the names perform and how discourse participants respond to these uses. 
\end{sloppypar}

Crucially, Obeng’s paper provides invaluable insights into the structure and functions of hypocoristic day-name formation. At the level of structure, Obeng examines morphological and morphophonological processes, while in terms of function, he discusses the different sociolinguistic roles that these names play and the contextual factors that influence their use in \ili{Akan} society. Obeng’s linguistic analyses provide rich insights into the structure of Akan hypocoristic forms and demonstrate how various linguistic processes operate in other naming and social contexts.

In a more detailed linguistic analysis, \citet{Obeng2001} provides a rich discussion of the structure of personal names in different African communities. Obeng not only identifies and categorizes different African personal names based on their structures and contexts, he explores and explicates the basic linguistic processes – both morphological and phonological – that these names utilize in their formation. For instance, he reveals that some African day-names are compounds formed through certain affixes. In Akan names, Obeng illustrates that the \textit{kwa-} prefix is affixed to a root to form masculine day-names, while the formation of feminine day-names involve both \textit{a-} prefix and \textit{-wa} suffix, as in examples such as \textit{Kwadwo} `Monday-born male child', \textit{Kwabena}, `Tuesday-born male child', \textit{Adwoa} `Monday-born female child', and \textit{Abenaa} `Tuesday-born female child' (\citealt[10]{Obeng2001}). Note that although  \textit{-wa} indicates feminine day names, the glide /w/ is subsequently deleted \citeyearpar[11]{Obeng2001}. 

When contextualizing the above linguistic processes in the Yoruba linguistic landscape, we find comparable phenomena. For instance, the \isi{vowel harmony} and tonal change processes cited by Obeng in Akan names, also occur in Yoruba names. In Yoruba names like \textit{Gbẹ́bẹ̀} [gbɛ́bɛ̀] meaning ‘take plea’. Both vowels are Retracted Tongue Root (RTR) vowels, reflecting harmony in the distribution of vowels within the name. Similarly, in the name \textit{Tèmi} [tèmī] meaning ‘mine’, we find \isi{ATR} (Advanced Tongue Root) vowels. Tonal change, another essential aspect of word formation patterns in Akan naming, is revealed in the formation process of Yoruba names as well. This phenomenon is observable in hypocoristic names such as \textit{Ádéade}̀ derived from [ādé], meaning ‘crown’, where the high tone on the second \is{grammatical tone} syllable [dé] changes to low in the last syllable [dè]. Similarly, \textit{Gbémisiọlá} meaning ‘lift me to wealth’, may undergo tonal change, with the mid tone on the penultimate syllable [sī-...ọ̄lá] becoming a high tone [ɡbémisọ́lá]. These tonal alterations often arise due to the addition of affixes in forming names; i.e., these phonological changes are morphologically induced. Intriguingly, these processes often do not adhere to the regular language rules. For instance, the anticipated tonal patterns in names may be altered, where names being formed might adopt new tones, such as a high tone turning into a low tone. This insight, gleaned from Obeng’s study, underlines the unique linguistic approach to the study of African names, suggesting that names may exhibit processes not typically found in the general language.

As noted by Obeng’s numerous publications (as well as others previously mentioned), names depict a symbiotic relationship between language and culture. Working within this conceptual framework, this chapter identifies the naming contexts and the various factors within the Yoruba ethnic community that influence and/or motivate the formation of these anthroponyms. The Yoruba names that this study focuses on were collected through personal observation during participation in various naming ceremonies in South-West Nigeria. Over 100 names were collected, and their meanings and functions are identified based on further discussions with native speakers. In Section \ref{SEC2-NamingContext}, I discuss the naming context among the Yoruba. Section \ref{SEC3-Communicative Acts} will explore the meanings of these names and the communicative acts they perform in detail. A conclusion is provided in Section~\ref{SEC-Conclusion}.  

\section{The naming context of Yoruba} \label{SEC2-NamingContext}

As noted by \citet{Ehineni2019}, the giving of names is an important socio-cultural facet of the Yoruba community; hence it is always accompanied by ceremonial activities. Naming is a symbolic event that is usually historically constructed, socially maintained, and based on shared assumptions and expectations of members of a particular community. It is a ritual that is based on historical traditions passed down from generation to generation and a communal festive occasion celebrated jointly by relatives, friends, neighbors, acquaintances, and well-wishers. In Yoruba society, naming is referred to as \textit{ìsọmọlórúkọ}, which literally translates to ‘giving a child a name’. Traditionally, the naming ceremony usually takes place on the seventh day after birth, if the child is a girl, but if the child is a boy, it occurs on the ninth day. However, in the case of twins, the seventh day is also the day of naming, if the twins are both females, but the naming ceremony is held on the eighth, if the children are male and female twins, while if both twins are male, then the naming takes place on the ninth day (\cite{Ilésanmí1987}). 

The difference in days of naming is based on the traditional Yoruba belief that females have seven ribs while males have nine (\cite{Akinnaso1980}). This is based on the Yoruba belief that males are physically stronger than women. Furthermore, several studies on naming in Yoruba culture have also identified the use of symbolic elements, such as \textit{omi} ‘water’, \textit{epo} ‘palm oil’, \textit{orógbó} ‘bitter kola’,  \textit{obì} ‘kola nuts’, \textit{ataare} ‘alligator pepper’,  \textit{àádùn} ‘grounded roasted corn made into paste with palm oil’, \textit{iyọ̀} ‘salt’, \textit{ìrèké} ‘sugar cane’, \textit{ọtí }‘liquor’, and so on (\citealt{Adeoye1979, DaramolaJeje1967, Ilésanmí1987, Ladeleetal1986, Ogunbowale1979}). 

These ``symbolic elements'' embody greater cultural association beyond their basic denotations. For example, \textit{omi} `water' is essential for life and might be used symbolically to wish someone a life full of essential blessings or fluidity. \textit{Epo} `palm oil' is a staple in many African diets and may symbolize wealth, nourishment, or the richness of life. \textit{Obì} `kola nuts' and \textit{orógbó} `bitter kola' are traditionally used in Yoruba ceremonies and could symbolize hospitality, respect, or community bonds. \textit{Ataare} `alligator pepper' is often associated with spiritual rituals and might symbolize protection or spiritual strength. In this case, the ``symbolic'' usage of these terms in naming can serve to bestow upon the named individual the positive traits, blessings, or protections that these cultural items represent. Thus, these names function not just as labels, but as a form of prayer, aspiration, or spiritual protection for the individual. 

It should be noted that all these traditional elements are important to Yoruba people, and as \citet[116]{Akinyemi2005} points out, despite longstanding contact with Europeans and the adoption of the Western calendar, the naming ceremony takes place on the morning of the eighth day after the birth of a child, with a party or social gathering following later in the afternoon. That is, unlike the customary practice in American and European societies, where parents are expected to provide a name for their child on the day of birth, the Yoruba people must not announce a child's name until a week after their birth.

Tradition allows parents, grandparents, great grandparents, relations, and family friends to give names to a newborn during the naming ceremony (\citealt[116]{Akinyemi2005}). Hence, a Yoruba child may have as many as five or six names. Of course, it is ultimately the biological parents who decide on the name that a child will eventually use (\cite[116]{Akinyemi2005}). It should be noted, however, that many factors come into play when making this decision, as discussed later in this paper.

\section{Communicative acts in Yoruba names} \label{SEC3-Communicative Acts}

\subsection{Naming contexts in the Yoruba tradition}

The Yoruba have a popular maxim which says \textit{ilé làá wò kató sọ ọmọlórúkọ}, meaning ‘the condition of the home determines a child’s name’. This maxim emphasizes the indispensability of the social or circumstantial context when naming a child (\cite{Ehineni2019}). Consequently, names are informed by sociocultural considerations. Also, \citet[163]{Obeng1998} observed that names may “reflect their users' geographical environment where a child or its parents inhabit, as well as their fears, religious beliefs and philosophy of life and death. Children’s names may even provide insights into important cultural or socio-political events at the time of their birth”. Hence, among the Yoruba, names may communicate significant information, such as parental experience, birth circumstance, religious affiliation, professional history, or proximity to the death of another member of the immediate or extended family, among others. This communicative function is explored in the subsections below.

\subsection{Parental experiences in the family}

The family situation refers to concurrent experiences or events in the life of the family when the child was born. These experiences also include those of the parents. For instance, if a child was born after a major breakthrough in a family or during a successful achievement by the parents, the child could be given the following names:

\ea \label{Experience1}
\begin{xlist}
\ex	\textit{Ayọ̀dipúpọ}̀  	‘joy has become much’\\
\ex	\textit{Ọlámilékan} 	‘my wealth has been added to’\\
\ex	\textit{Ayọ̀dipúpọ̀}	‘wealth has become much’\\
\ex	\textit{Ayọ̀mikún} 	‘my joy is now full’\\
\ex	\textit{Ọládipúpọ̀}	‘wealth has become much’\\
\ex	\textit{Iremidé}	‘my goodness has come’\\
\end{xlist}
\z

These names are given to commemorate good moments in a family’s life. Parents usually give these names to express emotions such as joy and happiness. Also, these names are given to show that the child was born at a time when things were going well in a family. Thus, a name is given to mark progress or positive developments. Names may also communicate parents’ tough times or misfortunes.

\ea \label{Experience2}
\begin{xlist}
\ex	\textit{Fìjàbí}		‘born while in conflict’\\
\ex	\textit{Ayésòro} 	 ‘life is difficult’\\
\ex	\textit{Ayéjùsùnlé} 	‘life is not worth relying on’\\
\ex	\textit{Ajéníyà}		‘success has suffering’\\
\end{xlist}
\z

These names are motivated by hard times and problems endured by parents during the time of the birth of a child. For instance, the name \textit{Fìjàbí} ‘born while in conflict’, is a name given to a child born in times of war or during rivalry between the parents or community and other people or communities. As \citet{Akinyemi2005} observes, names are a public means through which Yoruba people document their histories; the names given in this section document parents’ experiences.

\subsection{Circumstance of birth}

The Yoruba believe that the way a child is born may say a lot about their destiny (\cite{Ehineni2019}). Names given based on the birth circumstance describe the physical condition of a child at delivery, such as their posture or position during the birthing process. A child born with an unusual posture at birth, such as a breech presentation, may be said to have \textit{orúkọ àmútọrunwá} `a name brought from heaven'. Hence, a name is given to indicate this uniqueness. Names may also be based on the circumstances surrounding the birth such as the place or time period of birth, such as during festivals or sacred days. Examples of such names are given as follows:

\ea \label{Circumstance1}
\begin{xlist}
\ex	\textit{Ìgè} 	    ‘born feet first’\\
\ex	\textit{Àjàyí}		‘born face down’\\
\ex	\textit{Ọ̀kẹ́}		‘born with amniotic sac’\\
\ex	\textit{Òjó}		‘male-child born with umbilical cord twined around neck’\\
\ex	\textit{Àìná} 	 	‘female-child born with umbilical cord twined around neck’\\
\end{xlist}
\z

Another aspect of circumstancial names is the period of birth. These names are given to indicate the time or moment when a child is born. These names inform about traditional festivals, religious celebrations or other social events (coronation, war, etc.) as provided below.

\ea \label{Circumstance2}
\begin{xlist}
\ex	\textit{Abọ́dúndé}         ‘come with the new year’\\
\ex	\textit{Abíọ́yè}		‘born during coronation’\\
\ex	\textit{Abíogun}	‘born during war’\\
\end{xlist}
\z

The name \textit{Abíọ́dún} may be used when a child is born during any of the traditional festivals (\textit{Egúngún} `masquerade' festival, \textit{Iṣu túntún} `new yam' festival, \textit{Iléyá} `homecoming' festival, etc). Names such as \textit{Abíogun}, etc. are given to denote that a child was born during fierce moments of war. It should be noted that children given such ``war names'' are perceived as potentially strong and vigorous, and they are therefore able to confront and deal with any tough challenges they will face later in life. According to \citet[364]{Blum1997}, suggests that names are believed to influence a child's destiny, correlating with the time and place of their birth. This view also applies to the Yoruba context where it is believed that by designating the period of birth, the names may influence the personality or destiny of the child.

\subsection{Order of children in the family}

There are names in the Yoruba tradition that reflect the order in which children are born into a family. For instance, the first child in a family may be called \textit{Àlàbí} ‘first to be born’. However, a common situation in which names are positionally given to children based on the order of birth among the Yoruba is in the case of twins. Children who are born twins and those who follow or are born after them are given names that show the order in which they follow the twins. 

\ea \label{Order1}
\begin{xlist}
\ex	\textit{Táyéwò}		‘have the first taste of the world’\\
\ex		\textit{Kẹ́hìndé}	‘one who comes later’\\
\ex		\textit{Idòwú}		‘one that comes after twins’\\
\ex		\textit{Àlabá} 		‘one that survives for us to meet’\\
\ex		\textit{Ìdògbé}		‘one born third after twins’ \\
\ex  \textit{Ìdòkún}   `one born fourth after twins'
\end{xlist}
\z

The first of the twins is called \textit{Táyéwò} (\textit{also known as Táíwò}) while \textit{Kẹ́hìndé} is given to the second twin. The next child born after the twins is called \textit{Ìdòwú}, meaning ‘one that comes after twins'. The child born after is \textit{Ìdòwú} called \textit{Àlàbá}, while the child born after \textit{Àlàbá} is called \textit{Ìdògbé}. The child born after \textit{Ìdògbé} is called \textit{Ìdòkún}. These names are positionally determined and reflect the order of birth of children. It is also important to note that for twins, especially concerning the issue of birth order, while \textit{Táyéwò} is born before \textit{Kẹ́hìndé}, \textit{Kẹ́hìndé} is considered \textit{Táyéwò's} senior. This is because the Yoruba believe that \textit{Táyéwò} [tọ́-ayé-wò] meaning ‘taste the world to see' is a forerunner for \textit{Kẹ́hìndé} `last to come'. \textit{Táyéwò} comes first to taste the world to see if it is a place to live and after tasting the world to see how it is, \textit{Táyéwò} informs \textit{Kẹ́hìndé} with a good report about the world. Hence, \textit{Kẹ́hìndé} comes after \textit{Táyéwò}. Thus, \textit{Táyéwò} is a messenger, who visited the world on errands for \textit{Kẹ́hìndé}. The Yoruba believe it is the older that sends the younger on errands. This belief makes \textit{Táyéwò} the younger (although the first born of the twins), while \textit{Kẹ́hìndé} (although the last of the twins) is considered the elder. Thus, birth order names also foreground Yoruba psychology and philosophy.

\subsection{Gender}

Yoruba names may also communicate the sex of the child – either male or female.  For instance, if a woman witnessed several losses of children at childbirth, a surviving male child born after that experience is named \textit{Àjàní} ‘fight to have’, while a surviving female child is called \textit{Àbẹ̀bí} `child who was begged to be born'. It should be noted that a male child may not be given \textit{Àbẹ̀bí}, neither can a female child bear \textit{Àjàní} – it is culturally inappropriate. More examples of these names are as follows, with masculine names in (\ref{MasculineNames}) and feminine names in (\ref{FeminineNames}):

\ea \label{MasculineNames}
\begin{xlist}
\ex	\textit{Bánkọ́lé}	‘build a house for me’\\
\ex	\textit{Akin}		‘strong one’\\
\ex	\textit{Àjàní}		‘fight to have’\\
\ex	\textit{Àkànní}		‘meet to have’ \\
\ex	\textit{Wálé}       	‘come home’\\
\ex	\textit{Gbádé}       	‘take crown’\\
\end{xlist}
\ex \label{FeminineNames}
\begin{xlist}
\ex \textit{Títílayọ̀}		‘forever is joy’\\
\ex	\textit{Àdùnní}		‘sweet to have’\\
\ex	\textit{Àríkẹ́}		‘see to pamper’\\
\ex	\textit{Àjíkẹ́}		‘wake up to pamper’ \\
\ex	\textit{Wùnmí}		‘desire me’\\
\ex	\textit{Yẹ́misí}		‘honor me’ \\
\end{xlist}
\z

What is important to note in describing the nature of gender-oriented names in Yoruba is that feminine names often reflect ideas such as sweetness, pampering, desiring, etc., while masculine names may be characterized with ideas involving action or responsibility. For instance, \textit{Bánkọ́lé} ‘build a house for me’ in Yoruba, is a name accompanied with a social responsibility given to a male child at birth to build a house for his parents.

\subsection{Religious affiliation}

A name may show the religious orientation of a person's family. This connection is made, traditionally, through the reference to deities recognized in each clan. There are different deities that are revered in the Yoruba traditional society and names may be given to reflect beliefs in these deities. A description of these deities is given in (\ref{Yoruba Dieties}) from \citet[78]{Ehineni2019}.


\ea \label{Yoruba Dieties}
\begin{xlist}
\ex \textit{Ogún} `god of iron'\\
\ex \textit{Ifá} `god of wisdom'\\
\ex \textit{Sàngó}	`god of lightning and thunder'\\
\ex \textit{Èṣù} `god of roads and schemes'\\
\ex \textit{Ọ̀sanyín} `god of the forest'\\
\ex \textit{Ọ̀ya} `goddess of fertility' \\
\ex \textit{Yemoja} `goddess of the sea' \\
\ex \textit{Ọ̀sún} `goddess of beauty and love'\\
\ex \textit{Ajé} `god of wealth'\\

\end{xlist}
\z

As \citet{Ehineni2019} explained, these deities are believed to be supernatural beings with extraordinary powers. They are therefore worshipped by people who desire their blessings. Each deity has a shrine with priests or a priestess that people meet for spiritual consultations. Also, every clan or lineage in the Yoruba community has a family deity that is worshipped and venerated via the ancestors. Thus, when a child is born into such a family, a name is given to reflect the deity that is worshipped. Hence, names may express people’s beliefs in these deities. 

\ea \label{DietyNames}
\begin{xlist}
\ex	\textit{Ògúnnọ́wọ̀}		‘Ògún has respect’\\
\ex	\textit{Ògúnrótìmí}		‘Ògún stands with me’\\
\ex	\textit{Fábùnmi}		‘Ifá gives me’\\
\ex	\textit{Fákọ̀yà}		‘Ifá rejects suffering’ \\
\ex	\textit{Èṣùsànyà}		`Èṣù repaid my suffering’\\
\ex	\textit{Èṣùgbèmí}		‘Èṣù supports me’\\
\ex	\textit{Ọ̀ṣuntókun}		‘Ọ̀ṣun is up to the river’\\
\ex	\textit{Ọ̀ṣundáre}		‘Ọ̀ṣun justifies'\\
\ex	\textit{Ṣàngódélé}		‘Ṣàngó arrived home’\\
\ex	\textit{Ṣàngógbàmí}		‘Ṣàngó saves me’\\
\ex	\textit{Ọyadìran}		‘Ọya becomes a vision’\\
\ex	\textit{Ọyadárà}		‘Ọya performs wonders’\\
\end{xlist}
\z

Essentially, names are ``pointers to their users'" religious beliefs and practices (\cite[144]{Obeng2001}), and through names ``Africans are able to reveal how the natural and the supernatural function together to construct an individual’s fate and destiny'' (\cite[144]{Obeng2001}). This idea expressed by Obeng is reflected in the meanings of the names provided above. 

\subsection{Profession}

Names may also showcase professions ancestrally associated with a family. \citet[118]{Akinyemi2005} explains that Yoruba society is characterized by all kinds of professions, and even though these professions vary according to gender, each has a prefix that can be added to names to reflect the professional affiliation of the name bearer's family. For instance, as Akinyemi illustrates, the prefix \textit{Akin} in \textit{Akinjídé} ‘the strong one has arrived’ may only be used in naming male children born into a family of warriors (\citealt[118]{Akinyemi2005}). It should be noted that what is meant by the designation ‘warriors’ is that there is a specialized group of people who are trained to fight for a community against external forces. Among the Yoruba, there used to be inter-ethnic wars where different clans fought against one another to show superiority and assert domination over others. Historically, the Yoruba, being a significant ethnic group, maintained a cadre of highly trained warriors proficient in various weapons, including spears, arrows, traditional firearms, and cutlasses. This warrior class held a revered and respected position within Yoruba society, ensuring the protection of various sub-communities. Consequently, children born into these families received names that signified their heritage as descendants of warriors (\cite{Ehineni2019}). 

It is a spiritualized belief of people in a given profession that their vocation makes them successful in all areas of their endeavors - including childbirth.  Children born into a family of hunters may have the prefix \textit{Ọdẹ} `hunter' in their names. Similarly, in a family of drummers, the word \textit{Àyàn} `drum', is often prefixed to their names. Additionally, a name may show the profession of art and sculpture. For instance, \textit{Ọ̀nàmúyiwá}, ‘Artwork has brought this child’ reflects a belief that artistic talent played a role in making the birth of the child possible, while \textit{Ọ̀nàyẹmí}, ‘Artwork befits me’, demonstrates a family’s pride in their art profession.

\subsection{Death situation}

The Yoruba believe that if a mother suffers repeated infant mortality, then the reason is that the child’s mother in the underworld does not want the child to stay in the world of the living. Such a child is regarded as an \textit{àbíkú} ‘born-to-die’ (\cite{Ehineni2019}). As \citet{Akinyemi2005} notes, the Yoruba believe that the giving of such death situation names can prevent the untimely death of born children. This practice has also been observed in other African communities that by ``giving children death-prevention names, the children’s biological parents hope that even if the members of the spirit world recognize the children eventually, they will \is{death-prevention names} be so angry (because of the ugly nature of the death-prevention name) that they will not call the child to the spirit world. Specifically, the spiritual parents will be `incapacitated' by the death-prevention names and this will enable the child to live'' (\cite[166]{Obeng1998}). In many cases, a name may be given as an address to death as follows:

\ea \label{DeathNames1}
\begin{xlist}
\ex	\textit{Kúforíjì}			‘death, forgive this one’\\
\ex	\textit{Kúfisílẹ̀}			‘death, leave this one alone’\\
\ex	\textit{Kújẹ́nyọ̀}		‘death, allow me to rejoice’\\
\ex	\textit{Kúmáyọ̀mí}		‘death, don’t make jest of me’  \\
\ex	\textit{Kúmápàyí}		 ‘death, don’t kill this one\\
\end{xlist}
\z

In some cases, names may be presented in the form of a command to the child to stay, especially when the parents are deeply frustrated after losing many children. These command-like names are given as follows:

\ea \label{DeathNames2}
\begin{xlist}
\ex	\textit{Málọmọ́}		‘don’t go again’\\
\ex	\textit{Máku}			‘don’t die’\\
\ex	\textit{Dúrójayé}		‘wait to enjoy life’\\
\ex	\textit{Dúrósimí}		‘wait to bury me’\\
\ex	\textit{Bánjókòó}		‘sit down or stay with me’\\
\ex	\textit{Bámitálẹ́}		‘stay with me till the night’\\
\end{xlist}
\z

Furthermore, these death-situation names may also convey nagging or nastiness, which reflects the observation about death-prevention names that ``they [death-prevention names] may be nasty names of migrant labourers, dangerous animals, nasty objects, filthy places and expressions of emotions'' (\cite[221]{Agyekum2006Sociolinguistics}; see also \cite{Obeng2001, Obeng1998}). Examples of this group of death-situation names are provided below.

\ea \label{DeathNames3}
\begin{xlist}
\ex	\textit{Akísàátán}		‘no more rags’\\
\ex	\textit{Ìgbẹ́kọ̀yí}		‘the bush rejects this’\\
\ex	\textit{Emèrè}			‘the bewitched’\\
\ex	\textit{Kílànkó}			‘what are we collecting'\\
\end{xlist}
\z

 The name \textit{Akísàátán} ‘no more rags’ is a nagging name to inform the \textit{àbíkú} child that the parents do not have clothing material in which to bury them, if they die. Also, by using the word ‘rags’, the child is being humiliated as someone who deserves only rags, not good clothes. \textit{Ìgbẹ́kòyí} ‘even the bush rejects this one’ is another nasty name to insult the \textit{àbíkú} child that they would be rejected in case they die again. 

\subsection{Praise or eulogy}

Personal names may be given to convey certain attributes peculiar to a specific family. An aspect of personal names where the function of praise is largely manifest is totemic names. Totemic names may be referred to as \textit{oríkì orílé} (\cite{Oduyoye1972}). The term \textit{oríkì orílé} literally means ‘praise belonging to family’ (\cite{Ehineni2019}). These names often derive from totems such as \textit{ọ̀kín} ‘peacock’, \textit{erin} ‘elephant', \textit{òpó} ‘staff’, and \textit{oyin} ‘honey’, which are used to categorize family character. For instance, in the family of kings, apart from a kingship personal name, a child may also be given a totemic personal name based on the totem associated with dominion in the Yoruba community. In other words, such a child may be referred to as \textit{Adeníyí Erinfọlámí}, meaning ‘crown has honor’ and ‘the elephant that breathes with wealth’, respectively (see \cite{Ehineni2019} for more discussion). Basically, these names are given to children to eulogize and valorize certain qualities of the family into which such children are born. 

\subsection{Commentary on human nature}

Names may express certain ideals or morality. They may be referred to as proverbial names since they express cultural truisms and inform about human nature in general (\cite{Obeng2001}). Examples of Yoruba names with such a communicative goal are \textit{Olówóòkéré}, ‘the wealthy one is not a small person’, \textit{Ọláòṣebikan} ‘wealth alone does not solve a problem’, or \textit{Olówópọ̀rọ̀kú} ‘the wealthy one destroys poverty’. These names may also ask rhetorical questions to call people to logical or critical thinking. Among such names are: \textit{Tanimọ̀la} ‘who knows tomorrow?’, \textit{Tántọ́lọ́un} ‘who’s like the supreme God?’, \textit{Tanimọ̀ọ́wò} ‘who knows how to cater for them?’ Furthermore, these names may also be clipped as \textit{Ẹ̀hìnẹni} ‘one’s back’ (from \textit{ẹ̀hìnẹniníbanikalẹ́} ‘one’s back stands with one till the end’, \textit{Báọ̀kú} ‘as long as we are not dead’ (from \textit{báòkúiseòtán} ‘as long as we are not dead work does not end'). It is also important to note that the message in these names may be indirectly communicated – for instance, in \textit{Tanimọ̀la} ‘who knows tomorrow?’ a message that no one knows everything is indirectly conveyed; see \citet[49--68]{Obeng1994, Obeng2001} for a more detailed discussion of indirectness. More importantly, Yoruba proverbial names demonstrate that names are deeply rooted in the sociocultural context of the Yoruba people, and it is difficult to understand the names without a thorough knowledge of the Yoruba society, beliefs, philosophy, and psychology.

\section{Conclusion} \label{SEC-Conclusion}

This study underscores the profound significance of names, which are formed in response to pivotal cultural contexts. Our examination of personal names reveals their multifunctional roles within society, bolstering their communicative relevance. Names are potent tools for expressing individuals' identities, social statuses, experiences, and emotions. Furthermore, they can serve an educational purpose, imparting instructions, or lessons to community members, thereby demonstrating their didactic value. Such an investigation into naming practices underscores the intimate interweaving of language and culture, reaffirming that language is profoundly rooted in a given ethnic community's cultural beliefs, traditions, and practices. They echo and shape one another in an inseparable relationship.\il{Yoruba|)}



{\sloppy\printbibliography[heading=subbibliography,notkeyword=this]}
\end{document}
