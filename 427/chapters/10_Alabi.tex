\documentclass[output=paper,colorlinks,citecolor=brown]{langscibook}
\ChapterDOI{10.5281/zenodo.11091833}

\author{Victor Temitope Alabi\affiliation{Brown University}}
\title{Intertextuality in discourse: Chief Obasanjo’s open letter to Dr. Jonathan} 
\abstract{This study employs the concept of intertextuality to examine the open letter written by Chief Olusegun Obasanjo to the then-president of Nigeria, Dr. Goodluck Jonathan. The open letter titled \textit{Before it is too late} had two main themes: first, that Dr. Jonathan had failed as a leader, and second, that he should not seek re-election for a second term in office. Direct quotations and allusions were the forms of intertextuality employed in the open letter. I conduct a descriptive analysis by classifying intertextuality in the open letter into five categories: reference to a speech earlier delivered by Chief Obasanjo, reference to a speech earlier delivered by Dr. Jonathan, reference to quotations from published texts, reference to English and Yorùbá proverbs, and reference to the Bible. I propose that the use of intertextuality in the open letter, as well as access to classified information in the letter made available on the Internet to the public, may have contributed to Dr. Jonathan’s loss in the 2015 general elections in Nigeria.}

\IfFileExists{../localcommands.tex}{
   \addbibresource{../localbibliography.bib}
   % add all extra packages you need to load to this file

\usepackage{tabularx,multicol}
\usepackage{url}
\urlstyle{same}

\usepackage{listings}
\lstset{basicstyle=\ttfamily,tabsize=2,breaklines=true}

\usepackage{langsci-basic}
\usepackage{langsci-optional}
\usepackage{langsci-lgr}
\usepackage{langsci-osl}
% \usepackage{./langsci/styles/langsci-lgr}
% \usepackage{./langsci/styles/langsci-osl}
% \usepackage{langsci-gb4e}

\usepackage{tikz}
\usetikzlibrary{patterns,calc}
\pgfdeclarepatternformonly{south east lines}{\pgfqpoint{-0pt}{-0pt}}{\pgfqpoint{3pt}{3pt}}{\pgfqpoint{3pt}{3pt}}{
    \pgfsetlinewidth{0.6pt}
    \pgfpathmoveto{\pgfqpoint{0pt}{3pt}}
    \pgfpathlineto{\pgfqpoint{3pt}{0pt}}
    \pgfpathmoveto{\pgfqpoint{.2pt}{-.2pt}}
    \pgfpathlineto{\pgfqpoint{-.2pt}{.2pt}}
    \pgfpathmoveto{\pgfqpoint{3.2pt}{2.8pt}}
    \pgfpathlineto{\pgfqpoint{2.8pt}{3.2pt}}
    \pgfusepath{stroke}}
    
\usepackage{stmaryrd}
\usepackage{wasysym}
\usepackage{multirow}
\usepackage{caption}
\usepackage{subcaption}
\usepackage{mathrsfs}
\usepackage{qtree}

\usepackage{linguex}


   %pminos do not split footnotes
% \interfootnotelinepenalty=10000 %Footnote in Laporte chapters has to be split SN


%\DeclareIndexNameFormat{default}{%
%\nameparts{#1}%
%\usebibmacro{index:name}%
%{\index[names]}%
%{\namepartfamily}%
%{\namepartgiveni}%
% {}% L1
% {}% L2
%{\namepartprefix}% generates spurious space L3
%{\namepartsuffix}% generates spurious space L4
%}

%  {\DeclareIndexNameFormat{default}{%
%     \usebibmacro{index:name}{\index[names]}{#1}{#3}{#5}{#7}}}

%\DeclareIndexNameFormat{default}{%
%  \usebibmacro{index:name}{\sindex[nom]}{#1}{#3}{#5}{#7}}

%\DeclareIndexNameFormat{default}{%
%  \usebibmacro{index:name}{\sindex[person]}{#1}{#3}{#5}{#7}}
%\DeclareIndexNameFormat{default}{%
%\nameparts{#1} \usebibmacro{index:name}{\sindex[person]]}{\namepartfamily}{‌​\namepartgiven}{\nam‌​epartprefix}{\namepa‌​rtsuffix}}

%\newcommand{\smiley}{:)}

%\renewbibmacro*{index:name}[5]{%
%\usebibmacro{index:entry}{#1}%
%{\iffieldundef{usera}{}{\thefield{usera}\actualoperator}\mkbibindexname{#2}{#3}{#4}{#5}}}

% \newcommand{\noop}[1]{}

%remove for final
%\overfullrule=1mm

\newcommand{\tobi}[2]}}
\renewcommand{\S}[1]{\tobi{#1}{\textsc{*}}}

% this volume references
% puts: [this volume]
% already defined: \citetv
%\newcommand{\citepv}[1]{(\citeauthor{#1} \citeyear*{#1} [this volume])}
\newcommand{\citealtv}[1]{\citeauthor{#1} \citeyear*{#1} [this volume]}

%parentheses around example number
\newcommand{\pref}[1]{(\ref{#1})}

% in-text examples

\newcommand{\lnex}[1]{\textit{#1}} %target lang word
\newcommand{\lnlit}[1]{(lit.: `#1')} %literal reading
\newcommand{\lnlat}[1]{(#1)} % latinization
\newcommand{\lntrans}[1]{`#1'} %translation
\newcommand{\lnexl}[2]%
{\lnex{#1}{} \lnlat{#2}} % ex with latinization
\newcommand{\lnexlat}[3]{\lnex{#1}{} \lnlat{#2}{} \lntrans{#3}} % ex with latinization and tranl.

%ch01
\newcommand{\co}[1]{\mbox{\textbf{#1}}}

%ch09

\newcommand{\cyrbulg}[1]{\begin{otherlanguage*}{bulgarian}#1\end{otherlanguage*}}


%ch10
\newcommand{\nlp}{{\small NLP}}
\newcommand{\mwe}{{\small MWE}}
\newcommand{\rae}{{\small RAE}}
\newcommand{\lvc}{{\small LVC}}
\newcommand{\pos}{{\small P}o{\small S}}
%\newcommand{\todo}[1]{ \textcolor{red}{#1} }

%\renewcommand{\labelenumi}{\theenumi}
%\ainamefmt{{vv}{ll}{, ff}{, jj}} % fullname

\newcommand{\biberror}[1]{{\color{red}#1}}

\newcommand{\osenovaitem}{--~}
   %% hyphenation points for line breaks
%% Normally, automatic hyphenation in LaTeX is very good
%% If a word is mis-hyphenated, add it to this file
%%
%% add information to TeX file before \begin{document} with:
%% %% hyphenation points for line breaks
%% Normally, automatic hyphenation in LaTeX is very good
%% If a word is mis-hyphenated, add it to this file
%%
%% add information to TeX file before \begin{document} with:
%% %% hyphenation points for line breaks
%% Normally, automatic hyphenation in LaTeX is very good
%% If a word is mis-hyphenated, add it to this file
%%
%% add information to TeX file before \begin{document} with:
%% \include{localhyphenation}
\hyphenation{
    Beck-man
    Ngu-yen
    back-chan-nel
    back-chan-nels
    mo-not-o-nous
    ste-reo-typ-i-cal
}

\hyphenation{
    Beck-man
    Ngu-yen
    back-chan-nel
    back-chan-nels
    mo-not-o-nous
    ste-reo-typ-i-cal
}

\hyphenation{
    Beck-man
    Ngu-yen
    back-chan-nel
    back-chan-nels
    mo-not-o-nous
    ste-reo-typ-i-cal
}

   \boolfalse{bookcompile}
   \togglepaper[2]%%chapternumber
}{}

\begin{document}
\SetupAffiliations{mark style=none}
\maketitle

\section{Introduction}
 
In this paper, I employ the concept of intertextuality (\citealt{Allen2011, Bakhtin1981, Kristeva1980}) to examine the 18-page open letter written by the former president of Nigeria, Chief Olusegun Obasanjo, to the then-president of Nigeria, Dr. Goodluck Jonathan. The letter titled \textit{Before it is too late} was published in newspapers on December 2, 2013. According to the opening of the letter, it was a follow-up to several letters written to Dr. Jonathan, which were neither acknowledged nor replied to. There are two main reasons for analyzing this letter by Chief Obasanjo. First, it was the first open letter written by Chief Obasanjo (who had had experience as both a military head of state and a democratically elected president of Nigeria) to Dr. Jonathan, the Nigerian president from a minority ethnic group in Nigeria, as stated in the letter. Second, the contents of the letter, whether advertently or inadvertently, had the effect of exposing details about the inadequacies of the Jonathan-led administration and reminded Nigerians about the inconsistencies of Dr. Jonathan, thus suggesting that he could not be trusted for a second term. This may have conceivably led to Dr. Jonathan losing his bid for a second term in office in the 2015 presidential election.

In the letter to Dr. Jonathan, Chief Obasanjo criticized the Jonathan government and advised Dr. Jonathan not to seek re-election in 2015. It is important to note that Chief Obasanjo campaigned for Dr. Jonathan when the latter ran for the office of the president of Nigeria in 2011. Moreover, Chief Obasanjo and Dr. Jonathan were members of the same political party, the People’s Democratic Party (PDP), but Chief Obasanjo left the party a couple of months after writing the open letter. Interestingly, Chief Obasanjo went on to support the main opposition party candidate, Muhammadu Buhari, of the All Progressives Congress (APC), who eventually won the 2015 presidential election. In fact, Chief Obasanjo stated that without his support, Muhammadu Buhari and the APC could not have won the presidential election (see \cite{Erezi2018}). Thus, it is pertinent to unravel the linguistic features of the letter and to explore how and why it may have led to Dr. Jonathan losing the presidential election. 

Chief among the linguistic features employed in the letter to Dr. Jonathan was the use of intertextuality. The function of intertextuality is to present other credible voices within that of Chief Obasanjo. The aim of these voices is to amplify the issues raised in the letter. Thus, if Dr. Jonathan decided not to listen to Chief Obasanjo’s voice at the time, he should at least remember his promise to Nigerians. I classify the use of intertextuality in the letter into five broad categories: (i)~reference to Chief Obasanjo’s speech; (ii)~reference to Dr. Jonathan’s speech; (iii)~reference to quotations from published texts; (iv) reference to English and Yorùbá proverbs; and (v) reference to the Bible. Chief Obasanjo's referred to sources include writings, speeches, sayings, and scriptures by writers that Dr. Jonathan was likely to hold in high honor; that is, sources that were unlikely to be faulted. That Chief Obasanjo made it an open letter shows that he wanted Nigerians and indeed the world to appreciate his views on the state of the nation. 

In the next section, I review previous studies on the letter(s) of Chief Obasanjo and on intertextuality. In the third section, I examine the method of analysis. In the fourth and fifth sections, I discuss an overview of intertextuality and the background of the open letter respectively. In the sixth section, I analyze the data, while in the seventh section, I make concluding remarks.

\section{Review of previous studies}

Several studies have explored the open letter under study by Chief Obasanjo to Dr. Jonathan. For instance, \citet{Ojoetal2022} employed M. A. K. Halliday’s \textit{Systemic Functional Grammar} to analyze what they called ``six discourse features'' in the open letter, while \citet{EkhareafoAmbrose2015} use Critical Discourse Analysis (CDA) to analyze the open letter. In addition, \citet{FawunmiTaiwo2021} analyzed rhetoric, ideology, and power relations in two open letters – \textit{Before it is too late} and \textit{Points for concern and action} – written by Chief Obasanjo. Other studies have explored the open letter under study by Chief Obasanjo and the response from Dr. Jonathan (to the open letter), which was also an open letter. For example, \citet{UnuabonahBoluwaduro2020} examined the pragmatic acts employed in the letters by both Chief Obasanjo and Dr. Jonathan, while \citet{Monehin2015} conducted a stylistic analysis of the open letters by Chief Obasanjo and Dr. Jonathan. Moreover, studies like \citet{IgwebuikeKamalu2015} explored the open letters written to Chief Obasanjo when he was Nigeria’s president between 1999 and 2007. None of these studies have examined the functions of the different voices in the open letters by Chief Obasanjo, which has motivated this study. Intertextuality, or the intertext, is worth studying to unravel the voices evident in the written discourse and their overall implications for political discourse analysis. 

Some studies have examined intertextuality along the lines employed in Obasanjo's open letter, for instance, the use of biblical intertextuality in letters and speeches (\cite{Obeng2011, Obeng2016}). Intertextuality has also been analyzed in how political actors quote or allude to other texts in political discourse (e.g., \citealt{Hodges2008, Obeng2011, Obeng2016, Orwenjo2009}). Moreover, intertextuality has been explored in relation to language planning and policy (e.g., \cite{Johnson2015}), in relation to other theoretical frames (e.g., \cite{Fairclough1992b}), in academic discourse (e.g., \cite{Chandrasomaetal2004}), in literary and media studies (\cite{OttWalter2000}; \cite{Ho2011}), as well as in legal/judicial discourse (e.g., \cite{Matoesian1999}). \citet{Raj2015} examines various perspectives that relate to Kristeva’s notion of intertextuality. The current study applies insights from these existing studies into engaging the use of intertextuality in Chief Obasanjo’s open letter to Dr. Jonathan.

\section{Method of analysis}

Data for this study (the open letter itself) were obtained from Vanguard Newspaper, which was published on December 12, 2013. Though other newspapers around the same period reported the same open letter by Chief Obasanjo, Vanguard Newspaper was randomly selected for this study. The intertext in the letter was analyzed for content and language. \citet[18]{Berelson1957} asserts that content analysis deals with the “objective, systematic and quantitative description” of data. I explored aspects in the letter where other voices were used to amplify the voice of the writer (Chief Obasanjo). Five groups of voices were identified in the letter: Chief Obasanjo’s voice, Dr. Jonathan’s voice, quotations from published texts, English and Yorùbá proverbs, and the Bible. I analyzed these voices to unravel and explore the implications of letter writing in political discourse. 

\section{Intertextuality}

Studies of intertextuality date back to the work of Julia \citet{Kristeva1980}. Kristeva was the first to coin the word \textit{intertextualité} in her analyses of Bakhtin’s works on literary semiotics. Allen \citeyearpar[3]{Allen2011} asserts that Kristeva’s “attempt to combine Saussurean and Bakhtinian theories of language and literature produced the first articulation of intertextual theory”. \citet{Bakhtin1981}, cited in \citet{Obeng2016}, argues that texts do not occur in a vacuum. Rather, they are situated within history and society, and speakers and/or writers insert themselves within them (i.e., the texts) by rewriting (and in some cases by speaking) the texts. The “semiotic notion of intertextuality introduced by the literary theorist Julia Kristeva is associated primarily with poststructuralist theorists” (\cite[197]{Chandler2007}). Kristeva observes that intertextuality deals with “the insertion of history (society) into a text” (\cite[39]{Kristeva1980}, cited in \cite[195]{Fairclough1992a}). She asserts that rather than confining one’s focus to the structure of a text, one should study its ``structuration'', that is, how the structure came into being (\cite[197]{Chandler2007}). Alfaro \citeyearpar[268]{Alfaro1996} argues that the “theory of intertextuality insists that a text cannot exist as a self-sufficient whole.” Johansen \& Larsen \citeyearpar[126]{JohansenLarsen2002} assert that intertextuality “means that the texts refer to each other, quote each other, that there are allusions in the text to other texts. Such an influence can, for example, take the form of adopting the conventions, material, action, or themes of other texts”. 

The intention of intertextuality is to cause us to remember a person, place, or thing in relation to what one is currently discussing, and by so doing, motivate us into action. Language users, for instance, religious leaders and political actors, adopt intertextuality to foreground \is{foregrounding} their position on certain issues. For example, in the Epistle of Jude (see \textit{The English Standard Version Bible}, 2009), Jude refers to texts in the Old Testament about Cain's way, Balaam’s error, and Korah’s rebellion. The intention of Jude was to state that the actors mentioned had done certain things that the readers of Jude’s epistle were encouraged not to do. In addition, politicians employ intertext in their speeches. An example can be found in Barack Obama’s eulogy at the funeral service of Elijah Cummings, a US Congressman, in October 2019. He states at the outset: “The seed on good soil, the parable of the Sower tells us, stands for those with a noble and good heart, who hear the word, retain it, and by persevering produce a crop. The seed on good soil''. From the outset, Obama employs biblical intertextuality to foreground the background of Elijah Cummings.\footnote{\url{https://www.theatlantic.com/politics/archive/2019/10/barack-obamas-eulogy-elijah-cummings/600697/}}

Plett \citeyearpar[5]{Plett1991} observes that an intertext is “not delimited, but de-limited, for its constituents refer to constituents of one or several other texts. Therefore, it has a two-fold coherence: an intratextual one which guarantees the immanent integrity of the text, and an intertextual one which creates structural relations between itself and other texts”. \citet{Chandler2007} argues that Kristeva’s \citeyearpar[69]{Kristeva1980} work refers to “texts in terms of two axes: a horizontal axis connecting the author and reader of a text, and a vertical axis, which connects the text to other texts'' (see also \citealt{Johnson2015, Raj2015}). Fairclough \citeyearpar[271]{Fairclough1992b} identifies \textit{manifest intertextuality}, which involves the verbatim use of texts, for instance, via the use of quotations. Instances of manifest intertextuality employed in the open letter to Dr. Jonathan include direct \is{quotation} quotations and \is{allusion} allusions.

Within the confines of Critical Discourse Analysis (CDA), \citet{Fairclough1992b} proposes that “the theory of intertextuality should be combined with a theory of power since the meaning of a text is not infinitely innovative, but will be limited by conditions of power relations” (\cite[168]{Johnson2015}). Fairclough further states that the use of intertextuality in texts is an instance of discourse practice, as well as social practice. In the data under study, power relations via the use of quotations and allusions are seen not only by virtue of the writer being older than the then-president but also by being more experienced than the then-president. This may be a reason why American presidents, for over thirty years now, have had the tradition of writing letters to their successors. Their position of experience as outgoing presidents makes them qualified to advise their successors.\footnote{\url{https://www.georgewbushlibrary.gov/research/topic-guides/transition-letters}} Chief Obasanjo had been military head of state and president for about eleven years in total while Dr. Jonathan had been president for just three years. Thus, this position also makes it possible to use proverbs to communicate with Dr. Jonathan. In the Yorùbá culture (Chief Obasanjo's ethnic group), most elders use proverbs to advise, warn, or caution younger people. Younger people are not permitted to directly use \isi{proverbs} to advise, warn, or caution an older person. In the following section, I discuss the background of the open letter written by Chief Obasanjo to Dr. Jonathan. 

\section{Background of the open letter}

Chief Obasanjo was the military head of state in Nigeria from 1976–1979. According to an article in the Tribune Newspaper, published on January 25, 2018, Chief Obasanjo had a tradition of writing letters publicly condemning military heads of state and democratically elected presidents after him. For example, he publicly condemned the democratic government of Alhaji Shehu Shagari (1979–1983) and the military rule of both General Ibrahim Babangida (1985–1993) and General Sani Abacha (1993–1998). Unlike Shagari and Babangida, who ignored him, General Abacha implicated Chief Obasanjo in a coup attempt and jailed him. This was after Chief Obasanjo had publicly condemned his military government. It is worthy of note that in the open letter to Dr. Jonathan, Chief Obasanjo referred to what he called the “Abacha era” because, while in prison, Chief Obasanjo had again written letters to General Abacha. The report in the Tribune Newspaper published on January 25, 2018, read (in part) that “usually, Obasanjo’s public condemnation of presidents or military heads of state signals the beginning of the end of such administrations”. Since Chief Obasanjo’s tenure as a democratically elected president from 1999–2007, his successors, the late Alhaji Umaru Yar’Adua, Dr. Goodluck Jonathan, and the current president, Muhammadu Buhari, have all received open letters\slash criticisms from Chief Obasanjo on his concerns over their governance, notwithstanding that he campaigned for each of them. 

In the letter to Dr. Jonathan, Chief Obasanjo referred to the time that he met Dr. Jonathan (where private discussions were not made public). This indicates that meeting Dr. Jonathan to discuss matters was not impossible. Rather, Chief Obasanjo chose to write an open letter, I would propose, to make his concerns known to the public. It is noteworthy that as of December 2013, it was less than one and half years before the March 2015 elections, and thus, there was ample time for Dr. Jonathan to possibly make changes in his governance.

Written texts about political issues both in Africa and the wider world have, over time, been replete with intertextual sources to justify the writer and to give credence to the writing (see \cite{Obeng2016}). Political actors often employ what I call \textit{voices of the past}, as well as voices of other people to add to their plea for change. \citet{Obeng2020} presents an analysis of the letters written by Dr. Danquah to Dr. Nkrumah on the infringement of his (Dr. Danquah’s) liberty. Dr. Danquah presents relevant voices in his letter to give credence to his writing and plea for his liberty.\footnote{Dr. J. B. Danquah and Dr. Kwame Nkrumah were political actors who had worked together for the cause of an independent Ghana. In one letter analyzed in \citet{Obeng2020}, Dr. Danquah employs voices of the past in his letter to Dr. Nkrumah. Dr. Danquah compares how the British treated them both as prisoners and how Dr. Nkrumah was treated him (Dr. Danquah) as a prisoner.} 

In the opening of the letter, Chief Obasanjo gives ten reasons why he chose to write the open letter to Dr. Jonathan. The ten reasons presented in the letter to Dr. Jonathan point to issues such as Dr. Jonathan’s personality (e.g., not responding to four or more letters). Other issues include the implications of his governance within Nigeria (e.g., not “dividing the country along weak seams of North-South and Christian-Muslim”), and outside Nigeria (e.g., international friends getting worried about signs and signals coming out of Nigeria). 

In the “Special Press Statement” to President Buhari in the Punch Newspaper, published January 24, 2018, titled \textit{The Way Out: A Clarion Call for Coalition for Nigeria Movement}, Chief Obasanjo stated,

\begin{quote}
    But my letter to President Jonathan titled: \textit{Before It Is Too Late} was meant for him to act before it was too late. He ignored it and it was too late for him and those who goaded him into ignoring the voice of caution.
\end{quote}

In the case of this Special Press Statement issued about the Buhari-led government, Chief Obasanjo expressed his displeasure at the failures of the Buhari-led government, including the failure of government to end the Boko Haram insurgency, as well as the Fulani herders who had killed Nigerian citizens in states like Benue and Kaduna. In 2018, President Buhari spent over three months in a London hospital, to which Chief Obasanjo remarked that he was unfit for office. Chief Obasanjo again issued a press statement in January 2019, a few weeks before the February 2019 general elections, raising an alarm that President Buhari was planning to rig the elections in his favor. 

The open letter to Dr. Jonathan was over 7,300 words. These words were more than both the January Press Statement and a July open letter\footnote{\url{https://punchng.com/full-text-of-obasanjos-open-letter-to-buhari/}} to President Buhari combined, which were over 3,500 words and over 1,600 words, respectively. Even though intertextuality was employed in Chief Obasanjo’s Press Statement and open letter to President Buhari, I observe that there was a higher frequency of intertextuality in the open letter to Dr. Jonathan. It is thus important to examine the reasons why intertextuality was used to such a high degree and to analyze the functions of intertextuality in the letter to Dr. Jonathan. In the next section, I analyze the functions of intertextuality in Chief Obasanjo’s open letter to Dr. Jonathan.

\section{Analysis of the letter}

A close look at the open letter shows that intertextuality features prominently via the use of direct quotations \is{quotation} and \is{allusion} allusions. The effect of this strategy is that it would be difficult for not only Dr. Jonathan, the recipient of the letter, but also the public, to deny the validity of the voices in the letter. For example, it includes direct quotation from Chief Obasanjo's and Dr. Jonathan’s speeches made before the April 2011 presidential election. As an elder political leader, Chief Obasanjo uses the voices of people who will, to a considerable extent, appeal to Dr. Jonathan, for instance, in the use of published texts, the use of \isi{proverbs}, and the use of Biblical allusions. Next, I examine the five different categories of intertextuality in the open letter to Dr. Jonathan. 

\subsection{Reference to a speech earlier delivered by Chief Obasanjo}

During the campaign before the 2011 election that got Dr. Jonathan elected to office, Chief Obasanjo made a speech to Nigerians on how important it was to vote for Dr. Jonathan. This was significant because it was the first time that someone from a minority group, the Ijaw ethnic group in the South-South region, would contest an election to become president of Nigeria. Obasanjo stated, in the presence of Dr. Jonathan, that Dr. Jonathan had promised to run for a single term. This motivated Nigerians to vote massively for him because political actors in African countries and outside Africa rarely make campaign promises to run for a single term in office (see \cite{PapaioannouandVanZanden2015}). In the letter under analysis, Chief Obasanjo made it clear that Dr. Jonathan should not run for a second term in office. He stated in the letter that there were signs that Dr. Jonathan was attempting to seek re-election. It is interesting to note that, at that time, Dr. Jonathan had not made any public announcements to run for a second term. Chief Obasanjo’s use of intertextuality was to remind Dr. Jonathan of the promise made to the Nigerian people. Reference to Chief Obasanjo’s speech was a clue that the promise to run for a single term was about to fail, which would portray Dr. Jonathan as someone who was not true to his word. I present an excerpt from the letter below in which Obasanjo is quoting himself.

\begin{quote}
You did not hesitate to confirm to me that you are a strong believer in a one-term of six years for the President and that by the time you have used the unexpired time of your predecessor and the four years of your first term, you would have almost used up six years of your first term and you would not need any more term or time. Later, I heard from other sources including sources close to you that you made the same commitment elsewhere, hence, my inclusion of it in my address at the finale of campaign in 2011 as follows: 
    
\textit{PDP should be praised for being the only party that enshrines federal character, zoning and rotation in its Constitution and practices it. PDP has brought stability and sustainability to the polity and to the system. I do not know who will be President of Nigeria after Dr. Goodluck Jonathan.}

\textit{In the present circumstance, let me reiterate what I have said on a number of occasions. Electing Dr. Goodluck Ebele Jonathan, in his own right and on his own merit, as the president of Nigeria will enhance and strengthen our unity, stability and democracy. And it will lead us towards the achievement of our Nigeria dream. }

\textit{There is press report that Dr. Goodluck Jonathan has already taken a unique and unprecedented step of declaring that he would only want to be a one-term President. If so, whether we know it or not, that is a sacrifice and it is statesmanly. Rather than vilify him and pull him down, we, as a party, should applaud and commend him and Nigerians should reward and venerate him. He has taken the first good step.}

\textit{Let us encourage him to take more good steps by voting him in with landslide victory as the fourth elected President of Nigeria on the basis of our common Nigerian identity and for the purpose of actualising the Nigerian dream}. 
\end{quote}

Dr. Jonathan became acting president after the death of Alhaji Umaru Yar’Adua in 2010. Chief Obasanjo, who had worked for Alhaji Umaru Yar’Adua to succeed him, also worked for Dr. Jonathan to become Nigeria's president. Chief Obasanjo was reminding Dr. Jonathan in the letter that Dr. Jonathan had expressed his support for a one-year term of six years for Nigerian presidents. By referring to his speech, which arguably contributed to Dr. Jonathan’s victory in the 2011 presidential election, Chief Obasanjo was reminding him of the premise on which he won the election. Chief Obasanjo campaigned with Dr. Jonathan in Nigerian states for the 2011 elections. Also, since he completed over a year of the remaining term of the late Umaru Yar’Adua, Chief Obasanjo was making it clear that he would have spent more than a term of four years in office.

\subsection{Reference to a speech earlier delivered by Dr. Jonathan}

Chief Obasanjo reminded Dr. Jonathan of the speech the latter gave during the campaign for the 2011 presidential election on another occasion in the open letter. Here, Dr. Jonathan was warned not to attempt to break his promise by Chief Obasanjo, who alluded to the fact that bloodshed might result in this alleged decision. Dr. Jonathan, in the letter in response to Chief Obasanjo, stated that he had not made the intention known to run for a second term. Chief Obasanjo was presenting a case that Dr. Jonathan was not being sincere about not running for a second term in office. Citing Dr. Jonathan’s speech was an attempt to remind him that since he was not willing to take any action that could cause bloodshed in the 2011 elections, he should not be willing to do the same in the 2015 elections. An excerpt from the open letter is presented below.

\begin{quote}
Please Mr. President be mindful of that. You were exemplary in words when during your campaign in the 2011 elections, you said “My election is not worth spilling blood of any Nigerian.” From you it should not be if it has to be, let it be. It should be from you, let peace, security, harmony, good governance, development and progress for Nigeria. That is also your responsibility and mandate. You can do it again and I plead that you do it. We all have to be mindful of not securing pyrrhic victory on the ashes of great values, attributes and issues that matter as it would amount to hollow victory without honour and integrity.
\end{quote}

\begin{sloppypar}
Dr. Jonathan’s language use in 2011, about not wanting any blood spilled, was an \is{indirectness} indirect response to his major opponent, Muhammadu Buhari, who then was the presidential candidate of the Congress for Progressive Change (CPC). Muhammadu Buhari had threatened unrest if the election results were not in his favor. Chief Obasanjo also blamed Dr. Jonathan for the fact that his advisors were encouraging him to run for a second term. \citet[324]{Bulletal2008} noted the importance of political commitment, observing that: “voters may question the extent to which politicians can be trusted to keep their word or to implement their promises”. Thus, a reminder of the speech made by Dr. Jonathan in 2011 was a signal that if Dr. Jonathan decided to run for a second time, it meant that Dr. Jonathan could not be trusted.
\end{sloppypar}

\subsection{Reference to quotations from published texts}

Chief Obasanjo also quoted from renowned West African writers in his open letter. He cited Chinua Achebe, who was not only from Southern Nigeria but also a celebrated literary figure within and outside of West Africa. Chief Obasanjo was using the words of Achebe about integrity to advise Dr. Jonathan not to compromise his stand, even if he was being persuaded to do so by his colleagues to run for a second term in office. In the excerpt below, Chief Obasanjo did not state explicitly that Dr. Jonathan had publicly declared his intention to run for a second term in office, but Chief Obasanjo was depending on hearsay. Chief Obasanjo wrote in the letter that he had observed signs that showed that Dr. Jonathan was planning to run for a second term and advised him against compromising the promise he made in 2011. Chief Obasanjo’s quotation of Achebe in the open letter is presented below.

\begin{quote}
Chinua Achebe said, “one of the truest tests of integrity is its blunt refusal to be compromised.”
\end{quote}

The quotation by Chinua Achebe is employed in Chief Obasanjo’s letter to remind Dr. Jonathan that though he may be advised to run for a second term, thereby compromising his earlier stand not to run for a second term, his integrity was more important than making a compromise. 

One of the charges that Chief Obasanjo leveled against Dr. Jonathan to buttress the point that he had failed as a leader and should not run for a second term was that he was allegedly hiding criminals and using them to his own advantage. Chief Obasanjo cited a publication by the journalist, Lansana Gberie, who had written extensively about politics, conflict, and security in African countries. The 28-page paper, published in 2013, is titled \textit{State Officials and their Involvement in Drug Trafficking in West Africa}. Chief Obasanjo employed Gberie’s work to foreground \is{foregrounding} the failures of the Jonathan government in bringing Kashamu, a drug peddler, to book (i.e., to justice) in addition to the allegation that Kashamu was being used for Dr. Jonathan’s political ambition of running for a second term in office. This allegation about the relationship between Kashamu and the Jonathan administration further presented a negative aspect of his government. I present the excerpt about \citet{Gberie2013} from the open letter below.

\begin{quote}
It may be instructive if I quote fairly extensively from Lansana Gberie’s recent paper titled, \textit{State Officials and Their Involvement in Drug Trafficking in West Africa}:

“The controversial and puzzling case of Buruji Kashamu, a powerful figure in the ruling Peoples Democratic Party (PDP), suggests that a successful and wealthy politician’s association with drug trafficking is hardly disabling. Kashamu was indicted by a grand jury in the Northern District of Illinois in 1998 for conspiracy to import and distribute heroin to the United States. The indictment named him under his own name as well as two suspect aliases: ‘Alhaji’ and ‘Kasmal.’ His whereabouts were unknown at the time, however, and his co-accused were tried and convicted.

Later that year, he was found living comfortably in England, and, on receipt of an extradition request from the US, the UK arrested Kashamu. After a very protracted proceeding lasting until 2003, however, an English judge refused to extradite Kashamu on grounds of uncertainty about his true identity. Kashamu triumphantly returned to Nigeria and soon after became a key political figure. 

He is now believed to be very close to President Goodluck Jonathan, because of his ability to mobilize votes in key states in Western Nigeria. The US government reviewed Kashamu’s case, with the famous Judge Richard Posner presiding. Posner concluded that while Kashamu’s identity remains murky, there is little doubt that the figure now exercising authority in Nigeria’s PDP is the same as Kashamu the ‘Alhaji’ who was indicted for conspiracy to smuggle illicit drugs into the United States.”
\end{quote}

Chief Obasanjo alluded to the writing of Lansana Gberie to present a link between Kashamu and Dr. Jonathan. The example above shows the function of an intertext to the disadvantage of a political actor. Since this was an open letter to Dr. Jonathan, the public was exposed to the writings of Lansana Gberie about the alleged relationship between Kashamu and Dr. Jonathan even if they had not read any of Gberie’s writings in the past. I propose that these quotations about the Jonathan administration contributed significantly to the loss of Dr. Jonathan’s run for a second term in office. This indicates the importance of other voices in addition to one’s voice, which could make or mar a political situation. Chief Obasanjo’s \isi{quotation} of this text was a strategy of letting Dr. Jonathan know that he had failed and could not be trusted as a leader. 

\subsection{Reference to English and Yorùbá proverbs}
\begin{sloppypar}
From the perspective of the Yorùbá culture (Chief Obasanjo's ethnic group), \isi{proverbs} are the horses of communication, and if communication is lost, proverbs are used to find it (\cite[497]{Owomoyela2005}). This means that effective discourse is not possible without resorting to proverbs. Orwenjo \citeyearpar[144--145]{Orwenjo2009}, commenting on the use of proverbs by Africans in political discourse, observes that proverbs have:
\end{sloppypar}

\begin{quote}
    retained certain core discourse functions in the society. They are still the undisputed spices that give the right flavour to any significant dialogue. They still offer one of the most accessible and efficient means of avoiding direct critique by alluding to the criticized matter in an indirect, less aggressive manner. Nowhere is the need to avoid direct critique so urgent and paramount than in political discourse.
\end{quote}

The use of proverbs in Chief Obasanjo’s letter is an instance of \isi{indirectness} in political discourse (\cite{Obeng1997}) because, from the cultural perspective of being a Yorùbá, it is socially unacceptable to directly make a negative statement to someone especially in authority, even if the person in authority is younger in age. Chief Obasanjo states, “The Yorùbá adage says, ‘The man with whose head coconut is broken may not live to savor the taste of the succulent fruit’”. This Yorùbá proverb literally means that “whoever takes foolhardy risks in pursuit of an end seldom lives to enjoy it” (\cite[117]{Owomoyela2005}). Chief Obasanjo essentially tells Dr. Jonathan that the people advising him to train snipers and other armed people for political reasons will eventually implicate him, and in the end, he will suffer the consequences. Chief Obasanjo does not mention his source of information by stating “if it is true”. This means that he was not affirming that Dr. Jonathan was training people against his political rivals but rather implying that his own concerns should be the concern of all Nigerians. 

Chief Obasanjo, to avoid making direct negative statements about the Nigerian economy, which relied heavily on oil, resorted to the proverbial route to inform the president to act on time. He states, “We should make hay while the sun shines” so that Nigeria does not lag in the African continent. He also advised him on looking at ways of improving the oil sector with advanced technology and stated three things that he said were “imperative in the oil and gas sector”. To \textit{make hay} means to act. The \is{implicature} implication of the proverb was that Dr. Jonathan was inactive in a time of opportunity for the prosperity of Nigeria’s economy.

\subsection{Reference to the Bible}

The use of \isi{biblical intertextuality} as well as mentioning \textit{God} 17 times presents Chief Obasanjo as (a) a religious person who believes in God; (b) one standing on the right side with God, while “the guard” who has become “the thief” stands on the wrong side. He adopts a strategy of \isi{delegitimization} (\cite[47]{Chilton2004}), that is, creating a picture of the “negative other presentation” of Dr. Jonathan. This is effective given that Nigerians are religious, and Nigeria is one of the most religious countries in the world (\cite{Nag2018}). Thus, the intent of the letter is for Nigerians to stand on the side of Chief Obasanjo because many Nigerians believe in a supreme being. Chief Obasanjo does not mention the “bible” but only “God”, but the former is implicit in his message. Chief Obasanjo, in stating that “God is watching, waiting, and biding his time to dispense justice” makes it clear that even though he cannot do anything about the situation in the nation, he had resigned to God’s judgment. The biblical text \is{biblical intertextuality} used is not about speaking words of blessings or words of faith with the hope of a better Nigeria, but that God is watching, and at the right time, would pronounce his judgment. In Zechariah 3:5, the Bible states that God dispenses judgment. I present an excerpt from Chief Obasanjo’s letter below. 

\begin{quote}
    Let me repeat that as far as the issue of corruption, security and oil stealing is concerned, it is only apt to say that when the guard becomes the thief, nothing is safe, secure, or protected in the house. We must all remember that corruption, inequity and injustice bred poverty, unemployment, conflict, violence and wittingly or unwittingly create terrorists because the opulence of the governor can only lead to the leanness of the governed. But God never sleeps. He is watching, waiting and bidding his time to dispense justice. If we leave God to do His will and we don’t rely only on our own efforts, plans and wisdom, God will always do his best. As I go round Nigeria and the world, I always come across Nigerians who are first-class citizens of the world and who are doing well where they are and who are passionate to do well for Nigeria. My hope for our country lies in these people. They abound and I hope that all of us will realize that they the jewels of Nigeria wherever they may be and not those who arrogate to themselves eternal for ephemeral.
\end{quote}

His reference to God who never sleeps comes from Psalms 121 in the Bible. God is one who does not support evil (James 1:13). Here, Chief Obasanjo is indirectly \is{indirectness} saying that God is not in support of the actions within the PDP. Earlier in the letter, he stated that Dr. Jonathan was the leader of the party. Thus, he alludes \is{allusion} to the fact that God is not in support of the evil actions within the party,  under the watch and leadership of Dr. Jonathan. This is important because Chief Obasanjo is indirectly stating that there are “destroyers” within the party whom, if not exposed, will not only ruin the party but the country. He was also indirectly calling on those interested in the progress of the country to ``step aside to think''. Chief Obasanjo stated in the letter:

\begin{quote}
    God is not a supporter of evil and will surely save PDP and Nigeria from the hands of destroyers. If everything fails and the Party cannot be retrieved from the hands of criminals and commercial jobbers and discredited touts, men and women of honour, principles, morality, and integrity must step aside to think.
\end{quote}

By choosing to state that God will act, Chief Obasanjo resorted to solace in God who will save the party and the country. He believed that some people were about to destroy the party and the country. In other words, if the president, who is the leader of the party and the country, will not play his part, God, as the last option will act to save the party and the country. 

\section{Conclusion}

In this study, I have examined intertextuality as a discourse strategy in the open letter by Chief Olusegun Obasanjo to the then-president of Nigeria, Dr. Goodluck Jonathan. The use of intertextuality in the open letter revealed the ills in the Jonathan administration and consequently the reasons why Dr. Jonathan should not seek a second term as president. Chief Obasanjo, in the open letter, alluded to various texts to validate his stance on the state of governance in Nigeria. Chandler \citeyearpar[202]{Chandler2007} observes that by “alluding to other texts and other media this practice reminds us that we are in a mediated reality”. Intertextuality is used as a mediator between the writer and the recipient of the writing. Chief Obasanjo’s use of direct quotations and allusions helps to foreground his messages to Dr. Jonathan and indeed to Nigerians.

The open letter exposed to the public the inadequacies of the government. Dr. Jonathan’s reply, dated December 20, 2013, read in part that Chief Obasanjo was writing about some “classified information”, which had a negative impact on the Jonathan-led government. The classified information and the allegations arguably contributed, I would argue significantly, to why Dr. Jonathan lost his bid in the 2015 general elections. The use of intertextuality in the letter helped to remind both the recipient (Dr. Jonathan) and the Nigerian public about the speeches by both Chief Obasanjo and Dr. Jonathan before the 2011 presidential election. The function of this was to call Dr. Jonathan to remember where he began from, as well as the promises he made to Nigerians. The use of proverbs and biblical allusions represented ageless sayings, which are assumed to be credible oftentimes by individuals who belong or (sometimes) do not belong to the culture or religion respectively. Also, references to writings by Chinua Achebe and Lansana Gberie helped to foreground that Dr. Jonathan was on the wrong path because of his alleged association with a questionable individual. The implications of this study in political discourse are that intertextuality via quotations and allusions in oral or written discourse (e.g., an open letter) would appear to have the ability to reshape the minds of a people for or against the recipient of the letter thereby leading to an upturn or downturn of events, especially in the political space.



\section*{Acknowledgments}

The idea to write this paper was conceived in one of my classes, \textit{Political Discourse Analysis}, with Prof. Samuel Obeng. Prof. Obeng is an excellent teacher and researcher. He also served as my doctoral dissertation advisor and his extraordinary mentorship has been key to my academic progress. For these reasons I am grateful to him. I also appreciate the reviewers and editors of this volume for their invaluable and insightful comments on this paper.

%\section*{Contributions}
%John Doe contributed to conceptualization, methodology, and validation.
%Jane Doe contributed to the writing of the original draft, review, and editing.

{\sloppy\printbibliography[heading=subbibliography,notkeyword=this]}
\end{document}
