\documentclass[output=paper,colorlinks,citecolor=brown]{langscibook}
\ChapterDOI{10.5281/zenodo.11091837}

\author{Paul Newman\affiliation{Indiana University}}
\title{Linguistic ``oddities'' explained} 
\abstract{This paper investigates irregular phenomena or ``oddities'' in Hausa, Kanakuru, and Tera, three languages belonging to the Chadic family. These phenomena appear odd in that they seem to be at variance with the patterns and normal grammatical formation rules in these languages. The Hausa anomalies are the plurals of the words \textit{màatáa} `woman’ and \textit{’yáa} `daughter, small’ (\textit{máatáa} `women' and \textit{’yáa} `daughters', respectively). The Kanakuru anomaly, which also involves plurality, is the strange pair \textit{buut} `he-goat’, plural \textit{bukurin} `he-goats'. The anomaly in Tera relates to the form of the Linker \textit{-t(ǝ)}, which normally suffixes to the stem, e.g. \textit{luku} `garment’, \textit{luk-tǝ-ku} `garments’, but in rare cases replaces the final consonant of the noun to which it is attached, e.g., \textit{sǝɗi} `snake’, \textit{sǝ-tǝ-ku} `snakes’. It is shown that with a fuller and richer understanding of these languages, one can explain all of these supposed oddities as manifestations of regular morphological and phonological processes, whether viewed as deep synchronic morphophonology or as historical vestiges.}

\IfFileExists{../localcommands.tex}{
   \addbibresource{../localbibliography.bib}
   % add all extra packages you need to load to this file

\usepackage{tabularx,multicol}
\usepackage{url}
\urlstyle{same}

\usepackage{listings}
\lstset{basicstyle=\ttfamily,tabsize=2,breaklines=true}

\usepackage{langsci-basic}
\usepackage{langsci-optional}
\usepackage{langsci-lgr}
\usepackage{langsci-osl}
% \usepackage{./langsci/styles/langsci-lgr}
% \usepackage{./langsci/styles/langsci-osl}
% \usepackage{langsci-gb4e}

\usepackage{tikz}
\usetikzlibrary{patterns,calc}
\pgfdeclarepatternformonly{south east lines}{\pgfqpoint{-0pt}{-0pt}}{\pgfqpoint{3pt}{3pt}}{\pgfqpoint{3pt}{3pt}}{
    \pgfsetlinewidth{0.6pt}
    \pgfpathmoveto{\pgfqpoint{0pt}{3pt}}
    \pgfpathlineto{\pgfqpoint{3pt}{0pt}}
    \pgfpathmoveto{\pgfqpoint{.2pt}{-.2pt}}
    \pgfpathlineto{\pgfqpoint{-.2pt}{.2pt}}
    \pgfpathmoveto{\pgfqpoint{3.2pt}{2.8pt}}
    \pgfpathlineto{\pgfqpoint{2.8pt}{3.2pt}}
    \pgfusepath{stroke}}
    
\usepackage{stmaryrd}
\usepackage{wasysym}
\usepackage{multirow}
\usepackage{caption}
\usepackage{subcaption}
\usepackage{mathrsfs}
\usepackage{qtree}

\usepackage{linguex}


   %pminos do not split footnotes
% \interfootnotelinepenalty=10000 %Footnote in Laporte chapters has to be split SN


%\DeclareIndexNameFormat{default}{%
%\nameparts{#1}%
%\usebibmacro{index:name}%
%{\index[names]}%
%{\namepartfamily}%
%{\namepartgiveni}%
% {}% L1
% {}% L2
%{\namepartprefix}% generates spurious space L3
%{\namepartsuffix}% generates spurious space L4
%}

%  {\DeclareIndexNameFormat{default}{%
%     \usebibmacro{index:name}{\index[names]}{#1}{#3}{#5}{#7}}}

%\DeclareIndexNameFormat{default}{%
%  \usebibmacro{index:name}{\sindex[nom]}{#1}{#3}{#5}{#7}}

%\DeclareIndexNameFormat{default}{%
%  \usebibmacro{index:name}{\sindex[person]}{#1}{#3}{#5}{#7}}
%\DeclareIndexNameFormat{default}{%
%\nameparts{#1} \usebibmacro{index:name}{\sindex[person]]}{\namepartfamily}{‌​\namepartgiven}{\nam‌​epartprefix}{\namepa‌​rtsuffix}}

%\newcommand{\smiley}{:)}

%\renewbibmacro*{index:name}[5]{%
%\usebibmacro{index:entry}{#1}%
%{\iffieldundef{usera}{}{\thefield{usera}\actualoperator}\mkbibindexname{#2}{#3}{#4}{#5}}}

% \newcommand{\noop}[1]{}

%remove for final
%\overfullrule=1mm

\newcommand{\tobi}[2]}}
\renewcommand{\S}[1]{\tobi{#1}{\textsc{*}}}

% this volume references
% puts: [this volume]
% already defined: \citetv
%\newcommand{\citepv}[1]{(\citeauthor{#1} \citeyear*{#1} [this volume])}
\newcommand{\citealtv}[1]{\citeauthor{#1} \citeyear*{#1} [this volume]}

%parentheses around example number
\newcommand{\pref}[1]{(\ref{#1})}

% in-text examples

\newcommand{\lnex}[1]{\textit{#1}} %target lang word
\newcommand{\lnlit}[1]{(lit.: `#1')} %literal reading
\newcommand{\lnlat}[1]{(#1)} % latinization
\newcommand{\lntrans}[1]{`#1'} %translation
\newcommand{\lnexl}[2]%
{\lnex{#1}{} \lnlat{#2}} % ex with latinization
\newcommand{\lnexlat}[3]{\lnex{#1}{} \lnlat{#2}{} \lntrans{#3}} % ex with latinization and tranl.

%ch01
\newcommand{\co}[1]{\mbox{\textbf{#1}}}

%ch09

\newcommand{\cyrbulg}[1]{\begin{otherlanguage*}{bulgarian}#1\end{otherlanguage*}}


%ch10
\newcommand{\nlp}{{\small NLP}}
\newcommand{\mwe}{{\small MWE}}
\newcommand{\rae}{{\small RAE}}
\newcommand{\lvc}{{\small LVC}}
\newcommand{\pos}{{\small P}o{\small S}}
%\newcommand{\todo}[1]{ \textcolor{red}{#1} }

%\renewcommand{\labelenumi}{\theenumi}
%\ainamefmt{{vv}{ll}{, ff}{, jj}} % fullname

\newcommand{\biberror}[1]{{\color{red}#1}}

\newcommand{\osenovaitem}{--~}
   %% hyphenation points for line breaks
%% Normally, automatic hyphenation in LaTeX is very good
%% If a word is mis-hyphenated, add it to this file
%%
%% add information to TeX file before \begin{document} with:
%% %% hyphenation points for line breaks
%% Normally, automatic hyphenation in LaTeX is very good
%% If a word is mis-hyphenated, add it to this file
%%
%% add information to TeX file before \begin{document} with:
%% %% hyphenation points for line breaks
%% Normally, automatic hyphenation in LaTeX is very good
%% If a word is mis-hyphenated, add it to this file
%%
%% add information to TeX file before \begin{document} with:
%% \include{localhyphenation}
\hyphenation{
    Beck-man
    Ngu-yen
    back-chan-nel
    back-chan-nels
    mo-not-o-nous
    ste-reo-typ-i-cal
}

\hyphenation{
    Beck-man
    Ngu-yen
    back-chan-nel
    back-chan-nels
    mo-not-o-nous
    ste-reo-typ-i-cal
}

\hyphenation{
    Beck-man
    Ngu-yen
    back-chan-nel
    back-chan-nels
    mo-not-o-nous
    ste-reo-typ-i-cal
}

   \boolfalse{bookcompile}
   \togglepaper[6]%%chapternumber
}{}

\begin{document}
\SetupAffiliations{mark style=none}
\maketitle

\section{Introduction}
 The essence of descriptive fieldwork and analysis is not only collecting raw data but at the same time identifying patterns and regularities that make up the structure of a language. Of course, exceptions, irregularities, and oddities – whatever one likes to call them~-- invariably emerge, and in the early stages of one’s work, one has to put these aside to avoid going off on a tangent and being distracted from one’s (hopefully coherent) research plan. Nevertheless, abnormal examples should not be neglected forever, as often happens. With well-described languages such as Hausa, the oddities become so familiar and commonplace that one forgets that they are abnormal, and one fails to see them as examples needing attention. At some point, as researchers get deeper into a language and acquire a greater understanding of it, they should relish the exciting challenge of trying to figure out why these oddities exist, where they fit in, and what they add to our understanding of the language being studied.

 A fundamental question to address is whether a seeming oddity is truly an unsystematic orphan that tells us nothing about the structure of the language, whether it is an unwelcome counterexample that undermines or requires reformulation of some rules or generalities, or whether, in fact, the odd surface form can be shown to derive by application of established rules and thus reinforce our confidence in their validity. In this latter case, the seeming exception not only demonstrates the efficacy of a rule or rules but can lead to further discovery and understanding of related phenomena. In this modest contribution, I discuss examples drawn from three different Chadic languages~-- Hausa (iso 639-3: hau), Kanakuru (iso 639-3: kna), and Tera (iso 639-3: ttr)~-- showing how seemingly odd phenomena result from and fit naturally into the structure and mechanisms of the individual languages.

 \section{Hausa plurals}

Hausa\il{Hausa|(} is well known for its incredible complexity in the area of noun \isi{pluralization}. It has numerous and varied ways of forming noun plurals (some 40 different formatives being evidenced) variously involving suffixation, suffixal reduplication, infixation, internal reduplication, gemination, tonal alternation, and combinations thereof \citep{Newman2000}. The processes typically involve dropping the final vowel and tones of the singular, and sometimes the language's  \textit{-iyaa} and \textit{-uwaa} feminine endings as well, thereby leaving a toneless, consonant-final base for the plural to be built upon. Several examples are given in Table \ref{tab:PluralFormatives}.\footnote{In the transcription system employed here, long vowels are indicated by double letters. Tone is marked only on the first of the two vowels in these instances, with the understanding that the tone extends over the entire syllable. <'y> indicates a glottalized palatal semivowel.}

\begin{table}
\caption{Examples of plural formatives}
\label{tab:PluralFormatives}
 \begin{tabular}{llll}
  \lsptoprule
  Singular & Gloss & Plural & Base  \\
  \midrule
 ràagóo & `ram'  & ráagúnàa  & raag-\\
 gùdúmàa & `mallet' & gúdúmóomíi & gudum- \\
 gúlbíi &`stream' & gúlàabée & gulb- \\
 túdùu&`hill' & tùddái  & tud- \\
 sàlkáa &`hide water bottle' & sálèekáníi  & salk- \\
 jìmínáa &`ostrich' & jìmìnúu  & jimin- \\
 tsúmángìyáa &`cane stick' & tsùmàngúu & tsumang- \\
 gàbàarúwáa& `acacia tree' & gàbàaríi & gabaar- \\
  \lspbottomrule
 \end{tabular}
\end{table}
 
 In two special cases, there are plurals that stand out as strange even by Hausa standards. One is \textit{màatáa} `woman, wife’, pl. \textit{máatáa}, the forms being segmentally identical but tonally divergent. The other is \textit{’yáa} `daughter, small female’, pl. \textit{’yáa}, the forms being both segmentally and tonally identical. The unanswered question~-- often not even asked~-- is why should these peculiar plural forms exist given the multiplicity of regular plural formatives available in the language? As is often the case, once one looks at aberrations carefully and asks oneself what could account for their weird shapes, an explanation emerges from the abyss.
 
The \textit{màatáa/máatáa} pair \is{grammatical tone} is aberrant in that although Hausa does sometimes employ tone change for grammatical purposes, the change normally takes place at the end of the word and is usually accompanied by some other change as well, e.g. \textit{ídòo} `eye’, but \textit{ídó} `in the eye’, and \textit{bàakíi} `mouth’, but \textit{bákà} `in/on the mouth'. The pair is also strange since we expect pluralization to involve some segmental addition to, or modification in, the word, whether a fully specified suffix, suffixal partial reduplication, or at least replacement of the final vowel.

The key to understanding the \textit{màatáa/máatáa} exception lies in the realization that the phonetically identical \textit{aa}’s at the end of the words are morphologically not the same. There are two different \textit{aa}’s! The \textit{aa} at the end of the singular is an integral part of the lexical representation: it is simply the final vowel of the word. It is not preserved in the plural, as it appears, but rather is dropped in creating a toneless, final-vowel-less base in accordance with the general pattern, i.e., \textit{màatáa}, base \textit{maat-}. The \textit{-aa} in the plural form \textit{máatáa} is instead a plural suffix that is found in other basic words such as \textit{míjìi} (< *mázìi) `man, husband’ (base \textit{maz}-), pl. \textit{mázáa}, \textit{kúusùu} `rat’ (base \textit{kuus}-), pl. \textit{kúusáa}, and [WH]\footnote{The term ``Standard Hausa” used here refers to the variety of Hausa found in the greater Kano area. This is the \is{variation} variant typically used in dictionaries (e.g., \citealt{NewmanNewman2020}), newspapers, and other media. WH (= (North)-Western Hausa) is an inexact term for the Hausa dialects spoken in Sokoto and elsewhere in that general geographical region.} \textit{kárèe} `cornstalk’ (base \textit{kar}-), pl. \textit{káráa}, with the last of these having been reinterpreted in Standard Hausa as a singular with the regular reduplicative \is{reduplication} plural \textit{káràarée}.  As seen in these examples, this plural suffix has an associated H(igh) tone melody that extends leftwards across the entire plural form. The reason why the plural \textit{máatáa} has all H tone is not because the L(ow) tone of the \textit{àa} in the first syllable was raised to H in some ad hoc fashion but because it had added what I refer to as \is{tone-integrating suffixes} a ``tone-integrating” suffix \citep{Newman1986}, namely a suffix with an associated tone melody that spreads from right to left and overrides \is{grammatical tone} the underlying lexical tones. In sum, the \textit{màatáa/máatáa} example is a seeming aberration, but, in fact, it turns out to be an ordinary, perfectly regular singular/plural pair.

Viewed historically, \is{diachrony} the story is even more interesting. From a functional point of view, the vowel suffix \textit{-aa} would seem to be a weak, inadequately distinct plural marker as compared, for example, to other overtly well-marked plural suffixes such as \textit{-unaa} (e.g., \textit{ɗáakúnàa} `rooms’), \textit{-annii} (e.g., \textit{wàtànníi} `months’), or \textit{-anii} (e.g., \textit{fárèetáníi} `fingernails’). The bare vowel \textit{-aa} as a plural marker is particularly poor because Hausa has innumerable \textit{aa}-final singular nouns with all H tone, both masculine and feminine, as seen in the examples in Table \ref{tab:AAFinalSG}. A final 
\break \textit{-aa} suffix is perhaps better than a simple tone change, which is what we originally thought was the plural formative, but not by much.   

\begin{table}
\caption{Examples aa-final singular nouns}
\label{tab:AAFinalSG}
 \begin{tabular}{lll}
  \lsptoprule
  Singular & Gloss & Plural  \\
  \midrule
súunáa \textit{m.} &`name' & súnàayée  \\
ráanáa \textit{f.} &`sun, day' & ràanàikúu  \\
bísáa \textit{f.} & `pack animal' &  bísàashée  \\
bóokáa \textit{m.}& `herbalist' & bóokàayée \\
ɓúrmáa \textit{f.} &`rat trap' & ɓúràamée\\
kwálláa \textit{f.}& `large basin'  & kwállàayée\\
gúzúmáa \textit{f.}& `old cow' &  gúzàamée\\
túkúrwáa \textit{f.}& `bamboo pole' & túkúrwóoyíi \\
  \lspbottomrule
 \end{tabular}
\end{table}

As it turns out, there is a simple historical explanation here involving a natural phonological change that had significant morphological consequences. The original suffix was not *\textit{-aa}, as appears synchronically, but *\textit{-an}, with a final /n/, thereby giving singular/plural pairs such as \textit{màatáa/*máatán}. The loss of the /n/ was due to an early historical change   in Hausa, discovered by Schuh \citeyearpar{Schuh1976}, whereby *N > ∅ /\underline{\hspace{.5cm}}\#, i.e., all word-final nasal consonants, both *\textit{n} and *\textit{m}, were deleted. This regular and seemingly exceptionless sound change is well documented and well established. What we have failed to see until now is its relevance to the analysis of \textit{aa}-final plurals of the \textit{màatáa/máatáa}, \textit{míjìi/mázáa} type.

The other Hausa oddity to be discussed, \textit{’yáa} `daughter, small (fem.)’, pl. \textit{’yáa}, is aberrant in that the singular and the plural are identical in form.\footnote{When functioning as a noun with the literal meaning `child,’ rather than as a diminutive or compound formative, the plural normally takes the reduplicative \is{reduplication} shape \textit{’yáa’yáa} rather than \textit{’yáa}. This reduplicated form represents a secondary development, motivated by the need to avoid the identity of the feminine singular and plural forms. For our discussion, we shall focus on the original non-reduplicated variant.} Although the two \textit{’yáa} words are phonologically identical in citation form, the grammatical difference between them shows up on the surface by means of gender/number \isi{agreement} rules and their form with a suffixal genitive linker attached, i.e., \textit{’yár} vs. \textit{’yán},\footnote{The feminine linked form \textit{’yár} along with its masculine counterpart \textit{ɗán}, literally `son of’, are commonly used in compound \is{compounding} formation, both sharing \textit{’yán} as their plural: e.g., \textit{’yár-hàrtûm} `plain, long-sleeve caftan’, pl. \textit{’yán-hàrtûm} (< \textit{hàrtûm} ‘Khartoum’), \textit{’yár-wàasáa/ɗán-wàasáa} `actress/actor’, pl. \textit{’yán-wàasáa} (< \textit{wàasáa} `playing’), \textit{’yár-ƙásáa/ɗán-ƙásáa} `citizen (fem./masc.)’, pl. \textit{’yán-ƙásáa} (< \textit{ƙásáa}`land, country’), and \textit{ɗán-kúnné} `earring’, pl. \textit{’yán-kúnné} (< \textit{kúnné} `in/on the ear’). A study of this rich formation goes beyond the scope of this paper.} e.g., \textit{’yá-r wúƙáa tá ázùrfáa} `a small silver dagger’ (lit. `small-of (fem.) knife of (fem.) silver’), cf. \textit{’yá-n wúƙàaƙée ná ázùrfáa} `small silver daggers’ (lit. `small-of (pl.) knives of (pl.) silver’). 
 
The simple and surprising explanation for the unexpected phonological identity of these singular and plural forms is that this pair actually manifests the same processes described in the \textit{màatáa/máatáa} pair, although one cannot see it when one only looks at current-day Standard Hausa. The explanation is hidden synchronically because of a lexically restricted \is{diachrony} historical sound change that applied in Standard Hausa, but not in \is{variation} northwestern [WH] dialects. The historically original form of the singular word for `daughter, small’ was \textit{ɗìyáa}, with L-H tone, a form still found in WH. As with other basic nouns, including \textit{màatáa/máatáa} `woman, wife’ and \textit{míjìi/mázáa} `man, husband’, it formed its plural by means of the \textit{-aa} suffix with an associated H tone \is{grammatical tone} melody. The result was, thereby, \textit{ɗíyáa} (which, we now know was historically derived from *ɗíyán), a form that was tonally distinct from the singular, its plural being H-H whereas the singular was L-H.

The historical change at play in this case, a seemingly ad hoc phonological change originally limited to one lexeme(!), involves the fusion of the CVC sequence *ɗiy into a single palatalized stop *ɗy, which subsequently was altered further into the glottalized palatal semivowel /’y/, with this new /’y/ being a lexically restricted but high frequency phoneme in the language. Note that when the initial *ɗiy of the disyllabic noun *ɗiyaa changed into *ɗy and thence /’y/, what in origin was a disyllabic noun became monosyllabic. The tone of the resulting monosyllabic plural form \textit{’yáa} remained H, i.e., *ɗíyáa H-H > \textit{’yáa} H. The underlying tones of the singular, on the other hand, underwent an adjustment. Hausa does not have rising tone in its tonal inventory, and so when presented with LH on a single syllable, as sometimes appears in intermediate structure, the tone simplifies to H. This can be seen in such examples as \textit{dòomín} `for (the sake of)’, cf. the apocopated form \textit{dón}, and \textit{nàawá} `mine’, with the WH dialectal variant \textit{náu} (< /náw/). In Standard Hausa, the originally L-H singular noun *ɗìyáa~-- which before monophthongization was tonally distinct from the plural~-- became H via the sequence *ɗìyáa > *ɗ ̀y ̀áa > ’yǎa  > ’yáa, ultimately ending up being phonetically identical to the singular. 

In short, although not evident at first glance, the explanation for the odd \textit{’yáa} sg./\textit{’yáa} pl. pair turns out to be simple and based on the application of morphological and historical phonological rules, all of which are straightforward and perfectly natural.\il{Hausa|)} 

\section{A Kanakuru plural}

Kanakuru,\il{Kanakuru|(} as described in Newman \citeyearpar{Newman1974}, is a West Chadic language, related somewhat distantly to Hausa. Like Hausa, it typically forms noun plurals by use of various suffixes, some reminiscent of, albeit not identical to, plural \is{pluralization} formatives in Hausa, e.g., \textit{yim} `name’, pl. \textit{yimŋgin}; \textit{shal} `monkey’, pl. \textit{shalin}; and \textit{maawo} `stranger’, pl. \textit{maawuyan}.\footnote{Tone is omitted in the Kanankuru examples since the matters at issue are concerned solely with consonant mutation and alternations.} By contrast, the plural for the word \textit{but} `he-goat’ is \textit{bukurin}. Not only does this plural form look strange to me~-- the \is{infixation} infixal /k/ is particularly curious~-- but my native speaker assistant was also puzzled by it, saying: ``Although I told you yesterday that the plural was \textit{bukurin}, it is not what I say. That is what my grandfather told me, so that is what I told you, but I personally say \textit{buutiŋgin}”. So, how do we explain this odd \textit{bukurin} plural that doesn’t appear to make any sense? 

The first step in unraveling the mystery of the relationship between \textit{but} and \textit{bukurin} is the correction of a transcription error. After all, facts count, and little mistakes can throw us off. The singular, which I had transcribed as \textit{but} when first elicited, is hardly a word that would seem to present great phonological difficulty for a half-competent field worker. But, I goofed! The correct representation is \textit{buut} with a long vowel. Hausa, the Chadic language I knew best and which was serving as the contact language between my Kanakuru fieldwork assistant and myself, has long vowels in open syllables, but it does not allow them \is{phonotactics} in closed syllables. A combination of Hausa influence, plus the fact that vowel length in closed syllables in Kanakuru is not terribly common, plus the fact that my close attention in transcription at that stage tended to be on getting tones right, I simply missed the long /uu/ in \textit{buut}. 

This minor error is a critical key in understanding what is going on here because, as later discovered, although Kanakuru does have long vowels in closed syllables, they almost always derive from CVCVC words where the middle C has been lost. Assuming \textit{buut} to have come from a C\textsubscript{1}VC\textsubscript{2}VC\textsubscript{3} word, and paying attention to the \is{phonotactics} shape of the corresponding plural form, it follows that the lost C\textsubscript{2} must have been /k/. Given the new pairing *bukut/\textit{bukurin} (significantly, the initial \textit{u} in \textit{bukurin} being short, rather than long, as in \textit{buut}), the current forms of the singular and the plural lend themselves to a straightforward derivation. The loss of the medial /k/ in \textit{buut} was due to the operation of two rules. First, there is a general (historical? / synchronic?) \isi{lenition} rule affecting underlying stops (p / t / k) in intervocalic position whereby *p → w, *t → r, and *k → x (a voiceless velar fricative). Second, x → ∅ between identical vowels, with the two vowels coalescing into a single long vowel, e.g. *bukut → buxut → \textit{buut}, cf. *dikil → dixil → \textit{diil} `hoe’, pl. \textit{dikilin}.  

The current-day plural form \textit{bukurin} reflects the addition to the singular of a common plural suffix \textit{-in} (as seen in such examples as \textit{gom/gomin} `baboon(s)’) plus the operation of the following morphological and phonological rules: the appearance of /r/, instead of the final /t/ of the singular, is due to the general lenition rule described above. But, having just appealed to the lenition rule, how do we account for the presence of the non-weakened /k/ in the plural? 

As is widespread, but not ubiquitous, in Chadic, plural suffixes are often accompanied by \isi{gemination} of an internal consonant. Assuming that this was also the case in Kanakuru, the medial consonant in a word such as *bukut would have been geminated in the plural, i.e., *bukkurin (cf. via the same process in the example \textit{liwe} (< *lipe) `calabash’, pl. \textit{lipen}, which we can assume came from *lippen with gemination of the medial /p/). The unsupported intervocalic /k/ in the singular would have undergone lenition, but the strong geminate /kk/ would not have. Subsequently, Kanakuru lost gemination entirely whereby */kk/ > /k/. This change did not, however, feed the lenition processes, and so the now intervocalic stop stayed as such. Applying various morphophonological processes, all of which are regular and quite normal, one ends up with \textit{bukurin} as the plural counterpart of \textit{buut}. Kanakuru, of course, manifests its share of unusual phenomena, e.g., the counter-universal presence of the palatal fricative \textit{sh} without a corresponding \textit{s}, and the apparent hardening of word-final *r to /t/; however, as we have pointed out, the seemingly odd plural pairing of \textit{buut/bukurin} is not one of them.\il{Kanakuru|)}

\section{The Tera linker}

Tera,\il{Tera|(} as discussed in Newman \citeyearpar{Newman1964}, belongs to the Biu-Mandara (= Central) branch of the Chadic family and is even more distantly related to Hausa and Kanakuru than Hausa and Kanakuru are to each other. The problematic oddity here concerns the language's linker. When a Tera noun adds a suffix, such as the pluralizer \textit{-ku} or the definite article \textit{-aŋ}, or is modified in some way, e.g., by means of a postnominal possessive (noun or pronoun), the stem obligatorily adds a linker. With some nouns, the linker consists of phonological fronting of the final vowel of the noun (a form of the linker that I refer to as “Y”). This is seen in comparing \textit{nǝcaka} `weaver’ and \textit{nǝcake-ku} `weavers', \textit{ruŋgu} `stranger, guest’ and \textit{ruŋgi-ku} `strangers, guests', \textit{mbola} ‘dove’ and \textit{mbole-aŋ} (pronounced [mboljaŋ]) ‘the dove’, and \textit{teɬa} `roughing stone’ and \textit{teɬe ɓarem} ‘our roughing stone’. 

With other nouns, the linker is a suffix \textit{-t(ǝ)}, with the \textit{t} appearing variously as [t], [d], or [nd], depending upon the preceding abutting consonant, and the schwa being automatically deleted when juxtaposed to another vowel.\footnote{Tera, like most languages in the Biu-Mandara branch of the family, has lost grammatical gender, a reconstructable feature of Proto-Chadic. The two main forms of the linker are undoubtedly historical vestiges of a former masculine/feminine gender distinction.} This suffix is added to the stem-final consonant, with the lexical final vowel, if any, being dropped: cf. \textit{luku} ‘garment', \textit{luk-tǝ-ku} ‘garments’, and \textit{luk-t-aŋ} ‘the garment’, as well as \textit{waxi} ‘rudeness’ and \textit{wax-t-aŋ} `the rudeness’ and \textit{ɬugu} `knife’ and \textit{ɬug-dǝ ɓaŋa} ‘my knife’. Nouns with /ɗ/ as the final consonant, on the other hand, behave differently. Here, one finds /t/ replacing the lexical \textit{ɗ} rather than being added to it, as in \textit{sǝɗi} ‘snake’ vs. \textit{sǝ-t-aŋ} ‘the snake’, \textit{viɗi} `monkey’ vs. \textit{vi-tǝ-ku} `monkeys’, and \textit{xeɗa} `mat’ vs. \textit{xe-tǝ ɓanda} ‘their mat’. Consonants in Tera are normally quite stable, so the question is: what is going on here? Why does the underlying \textit{ɗ} disappear?    

Again, we find that the explanation relates to the role played by \isi{gemination} and \isi{degemination}. Although consonant clusters as such are rare in Chadic~-- and Tera is typical is not allowing them~-- abutting consonants across a syllable boundary are well attested. There is a large range of different C.C’s abutting with one another. Examples of words with such sequences are shown in Table \ref{tab:CCContact}. 

\begin{table}
\caption{Consonant contact across a syllable boundary}
\label{tab:CCContact}
 \begin{tabular}{ll}
  \lsptoprule
nyax.ɬi &  `young man’\\
jax.ɓa &  `termite’\\
lom.ku & `bats’ \\
wan.xa & `maiden’\\
calaŋ.ku & `cheeks’ \\
dàl.gwàŋ &  `drummer’ \\
kwar.cax & `hill’ \\
ŋgar.ɮi & `egg’ \\
pǝr.gus & `rabbit’ \\
yur.vu & `fish’\\
ɓuɓul.ku & `hips’\\
loɣos.ku & `leaves’ \\
rap.tiki  & `friendship’\\
kozop.ku & ‘clouds’\\
  \lspbottomrule
 \end{tabular}
\end{table}

On the other hand, the abutting \is{phonotactics} sequence \textit{ɗ.t}, which should be the output when the linker is added to a \textit{ɗ}-final stem, does not occur. I propose that when such a sequence is created morphologically, the lexical stem-final \textit{ɗ} is not dropped or replaced, but rather assimilates to the following \textit{t}, thereby producing a geminate /tt/. However, with few exceptions, Tera, like Kanakuru, does not have geminates, and thus the geminates occurring in intermediate structure simplify into single consonants, i.e., */tt/ → \textit{t}. The shared \isi{degemination} in Tera and Kanakuru is a wonderful example of independent parallel drift.\footnote{Insight into the role and development of gemination in Chadic, specifically in West Chadic, is found in an excellent paper by Schuh \citeyearpar{Schuh2001}.} With the words \textit{viɗi} `monkey’, and \textit{sǝɗi} `snake’, for example, we get the following regular derivations: *viɗtǝku  → vittǝku → \textit{vitǝku} `monkeys' and *sǝɗtaŋ → sǝttaŋ → \textit{sǝtaŋ} `the snake’. Thus, what might appear to be a totally aberrant replacement of \textit{ɗ} by \textit{t} in the linked form can be seen as regular suffixation plus the application of totally natural rules of assimilation, gemination, and degemination.

The above analysis, in turn, leads to a possible explanation for a problem that previously didn’t stand out. In addition to the “Y” and \textit{-t(ǝ)} linkers, some nouns simply have the linker \textit{-ǝ}, which, as expected, is deleted when followed by a suffix beginning with a vowel. This is the standard form of the linker for nouns with stem-final /t/. This can be seen in comparing \textit{shipit} `a load, goods’ and \textit{shipit-ǝ-ku} `loads, goods', \textit{ɮiɮit} `tsetse fly’ and \textit{ɮiɮit-ǝ-ku} `tsetse flies', \textit{cicet} `broom’ and \textit{cicet-ǝ ɓarem} `our broom’, \textit{pǝjit} `ashes’ and \textit{pǝjit-aŋ} `the ashes’, and \textit{xǝxet} `wind’ and \textit{xǝxet-aŋ} `the wind’. However, maybe what we really have here underlyingly is the common \textit{-t(ǝ)} linker. That is, what appears on the surface as bare \textit{-ǝ} is probably the result of the processes involving assimilatory gemination followed by degemination that we already observed, i.e., *t-tǝ → ttǝ → \textit{tǝ}, where the single \textit{t} morphologically comprises both the \textit{t} of the stem and the \textit{t} of the linker. The derivation for \textit{cicetǝku} `brooms,’ for example, would thus be *cicet + tǝ + ku (noun + Linker + plural) → cicettǝku → \textit{cicetǝku}, and the derivation for \textit{pǝjitaŋ} `the ashes’ would be *pǝjit + tǝ + aŋ (noun + Linker + definite article) → pǝjittaŋ → \textit{pǝjitaŋ}. Of course, this analysis needs to be verified; however, to me, it is a more likely solution than the alternative of postulating bare \textit{-ǝ} as a distinct linker type, especially since \textit{-ǝ} is a weak vowel that is often elided or deleted.\il{Tera|)}

\section{Conclusion}

In basic field research, exceptions and seeming lexical and morphological oddities constitute problems that lie beyond the scope of early data-collection work and often challenge the competence and know-how of the investigator.  What I have shown in this paper is that with curiosity and intellectual courage, and with deeper knowledge to draw on, one can in fact explain troubling idiosyncrasies and, moreover, that such analyses can lead to a fuller and richer understanding of the workings of the language in question. The key is truly to get to know one’s research language (and related languages) well and be willing to go beyond simple observational “what?” and ask the often more difficult question of “why?”. 

\section*{Abbreviations}
\begin{tabular}{@{}ll@{}}
* & reconstructed form \\
f. & feminine grammatical gender\\
H & High tone \\
L & Low tone \\
m. & masculine grammatical gender \\
pl. & plural \\
WH & (North)-Western Hausa \\
\end{tabular}



%\section*{Acknowledgements}
%\section*{Contributions}
%John Doe contributed to conceptualization, methodology, and validation.
%Jane Doe contributed to the writing of the original draft, review, and editing.



{\sloppy\printbibliography[heading=subbibliography,notkeyword=this]}
\end{document}
