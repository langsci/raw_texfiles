\documentclass[output=paper,colorlinks,citecolor=brown]{langscibook}
\ChapterDOI{10.5281/zenodo.11091827}

\author{Kofi Agyekum\affiliation{University of Ghana, Legon}}
\title{Ethnopragmatics of Akan plant metaphor} 
\abstract{The paper examines the ethnopragmatics of plant metaphor (plantosemy) in Akan. It hinges on \textit{cultural linguistics} and its sub-branches \textit{cultural metaphor} and \textit{worldview metaphor} as established by \citet{Sharifian2015,Sharifian2017}, and Lakoff \& Johnson's \citeyearpar{LakoffandJohnson1980} \textit{conceptual metaphor}. The paper postulates that Akans use their familiarity with plant behavior, structure, and functions to represent human behavior in plant metaphors. The areas to be covered are proverbs, panegyric and elegiac poetry, and \textit{adinkra} symbols and fabrics. Other areas include plant metaphors related to youth, marriage, family, and death. The paper subjects the data to semantic, pragmatic, and stylistic analysis by looking at metaphor, simile, personification, and other literary devices. This paper argues that the Akan people use flora metaphors to depict their indigenous knowledge, philosophy, worldview, and environmental knowledge about plants. The study is qualitative and employs library and internet research, as well as interviews with four renowned Akan scholars. The paper argues that even though plant metaphor is universal, the attributes of specific plants are language- and culture-specific based on the natural ecology of the place, and the ways in which people interact with their environment. This paper expands new interdisciplinary research in the Akan language and culture in the areas of cultural linguistics, cognitive linguistics, ethnography of communication, anthropology, and oral literature.}

\IfFileExists{../localcommands.tex}{
   \addbibresource{../localbibliography.bib}
   % add all extra packages you need to load to this file

\usepackage{tabularx,multicol}
\usepackage{url}
\urlstyle{same}

\usepackage{listings}
\lstset{basicstyle=\ttfamily,tabsize=2,breaklines=true}

\usepackage{langsci-basic}
\usepackage{langsci-optional}
\usepackage{langsci-lgr}
\usepackage{langsci-osl}
% \usepackage{./langsci/styles/langsci-lgr}
% \usepackage{./langsci/styles/langsci-osl}
% \usepackage{langsci-gb4e}

\usepackage{tikz}
\usetikzlibrary{patterns,calc}
\pgfdeclarepatternformonly{south east lines}{\pgfqpoint{-0pt}{-0pt}}{\pgfqpoint{3pt}{3pt}}{\pgfqpoint{3pt}{3pt}}{
    \pgfsetlinewidth{0.6pt}
    \pgfpathmoveto{\pgfqpoint{0pt}{3pt}}
    \pgfpathlineto{\pgfqpoint{3pt}{0pt}}
    \pgfpathmoveto{\pgfqpoint{.2pt}{-.2pt}}
    \pgfpathlineto{\pgfqpoint{-.2pt}{.2pt}}
    \pgfpathmoveto{\pgfqpoint{3.2pt}{2.8pt}}
    \pgfpathlineto{\pgfqpoint{2.8pt}{3.2pt}}
    \pgfusepath{stroke}}
    
\usepackage{stmaryrd}
\usepackage{wasysym}
\usepackage{multirow}
\usepackage{caption}
\usepackage{subcaption}
\usepackage{mathrsfs}
\usepackage{qtree}

\usepackage{linguex}


   %pminos do not split footnotes
% \interfootnotelinepenalty=10000 %Footnote in Laporte chapters has to be split SN


%\DeclareIndexNameFormat{default}{%
%\nameparts{#1}%
%\usebibmacro{index:name}%
%{\index[names]}%
%{\namepartfamily}%
%{\namepartgiveni}%
% {}% L1
% {}% L2
%{\namepartprefix}% generates spurious space L3
%{\namepartsuffix}% generates spurious space L4
%}

%  {\DeclareIndexNameFormat{default}{%
%     \usebibmacro{index:name}{\index[names]}{#1}{#3}{#5}{#7}}}

%\DeclareIndexNameFormat{default}{%
%  \usebibmacro{index:name}{\sindex[nom]}{#1}{#3}{#5}{#7}}

%\DeclareIndexNameFormat{default}{%
%  \usebibmacro{index:name}{\sindex[person]}{#1}{#3}{#5}{#7}}
%\DeclareIndexNameFormat{default}{%
%\nameparts{#1} \usebibmacro{index:name}{\sindex[person]]}{\namepartfamily}{‌​\namepartgiven}{\nam‌​epartprefix}{\namepa‌​rtsuffix}}

%\newcommand{\smiley}{:)}

%\renewbibmacro*{index:name}[5]{%
%\usebibmacro{index:entry}{#1}%
%{\iffieldundef{usera}{}{\thefield{usera}\actualoperator}\mkbibindexname{#2}{#3}{#4}{#5}}}

% \newcommand{\noop}[1]{}

%remove for final
%\overfullrule=1mm

\newcommand{\tobi}[2]}}
\renewcommand{\S}[1]{\tobi{#1}{\textsc{*}}}

% this volume references
% puts: [this volume]
% already defined: \citetv
%\newcommand{\citepv}[1]{(\citeauthor{#1} \citeyear*{#1} [this volume])}
\newcommand{\citealtv}[1]{\citeauthor{#1} \citeyear*{#1} [this volume]}

%parentheses around example number
\newcommand{\pref}[1]{(\ref{#1})}

% in-text examples

\newcommand{\lnex}[1]{\textit{#1}} %target lang word
\newcommand{\lnlit}[1]{(lit.: `#1')} %literal reading
\newcommand{\lnlat}[1]{(#1)} % latinization
\newcommand{\lntrans}[1]{`#1'} %translation
\newcommand{\lnexl}[2]%
{\lnex{#1}{} \lnlat{#2}} % ex with latinization
\newcommand{\lnexlat}[3]{\lnex{#1}{} \lnlat{#2}{} \lntrans{#3}} % ex with latinization and tranl.

%ch01
\newcommand{\co}[1]{\mbox{\textbf{#1}}}

%ch09

\newcommand{\cyrbulg}[1]{\begin{otherlanguage*}{bulgarian}#1\end{otherlanguage*}}


%ch10
\newcommand{\nlp}{{\small NLP}}
\newcommand{\mwe}{{\small MWE}}
\newcommand{\rae}{{\small RAE}}
\newcommand{\lvc}{{\small LVC}}
\newcommand{\pos}{{\small P}o{\small S}}
%\newcommand{\todo}[1]{ \textcolor{red}{#1} }

%\renewcommand{\labelenumi}{\theenumi}
%\ainamefmt{{vv}{ll}{, ff}{, jj}} % fullname

\newcommand{\biberror}[1]{{\color{red}#1}}

\newcommand{\osenovaitem}{--~}
   %% hyphenation points for line breaks
%% Normally, automatic hyphenation in LaTeX is very good
%% If a word is mis-hyphenated, add it to this file
%%
%% add information to TeX file before \begin{document} with:
%% %% hyphenation points for line breaks
%% Normally, automatic hyphenation in LaTeX is very good
%% If a word is mis-hyphenated, add it to this file
%%
%% add information to TeX file before \begin{document} with:
%% %% hyphenation points for line breaks
%% Normally, automatic hyphenation in LaTeX is very good
%% If a word is mis-hyphenated, add it to this file
%%
%% add information to TeX file before \begin{document} with:
%% \include{localhyphenation}
\hyphenation{
    Beck-man
    Ngu-yen
    back-chan-nel
    back-chan-nels
    mo-not-o-nous
    ste-reo-typ-i-cal
}

\hyphenation{
    Beck-man
    Ngu-yen
    back-chan-nel
    back-chan-nels
    mo-not-o-nous
    ste-reo-typ-i-cal
}

\hyphenation{
    Beck-man
    Ngu-yen
    back-chan-nel
    back-chan-nels
    mo-not-o-nous
    ste-reo-typ-i-cal
}

   \boolfalse{bookcompile}
   \togglepaper[2]%%chapternumber
}{}

\begin{document}
\SetupAffiliations{mark style=none}
\maketitle

\section{Introduction}

Metaphors\il{Akan|(} are one of the most researched literary devices, and this is so because our human life is full of metaphors, and we use metaphors daily to indicate our relationship with nature, the environment, our body, emotions, society, culture, religion, etc. \citet[55]{LakoffandTurner1989} stated that a ``metaphor is \textit{conventionalized} to the extent that it is automatic, effortless, and generally established as a mode of thought among members of a linguistic community.'' 

In the last two decades, there has been research on Akan cognitive semantics, with emphasis on metaphor, culminating in many publications, including \citet{Agyekum2006From, Agyekum2015a, Agyekum2015b, Agyekum2016, Agyekum2018, Agyekum2020}, \citet{Ansah2014}, \citet{Dzokoto2010}, \citet{DzokotoandOkazaki2006}, and \citet{Gyekye1997}. None of these papers, all of which have focused on metaphor in Akan, has turned attention to plant metaphor (plantosemy), hence the need to work on it. Ghanaian folk song composers and highlife artistes have composed many songs that employ plant metaphors. The current paper covers plant metaphors in proverbs, praise poetry (panegyrics), drum language, and \textit{adinkra} (fabric symbols). This is a companion piece to \citet{Agyekum2023} on Akan animal metaphor, and many of the findings presented in that paper are also embodied in the current paper on plant metaphor.

The structure of the paper is as follows: Section \ref{07_Background} deals with the Akan language and people, as well as the present study’s methodology and a literature review. Section \ref{07_Theoretical} concerns the theoretical frameworks employed here: cultural linguistics and conceptual metaphor. Section \ref{07_WhatIs} discusses the nature of plant metaphor. Section 5 deals with thematic areas of Akan plant metaphors and their analysis, covering plant metaphor pertaining to youth and new marriage, death and dirges, and family, as well as plant metaphor in poetry, panegyric poetry, and proverbs. Section \ref{07_THematic} also covers use of copular verbs and simile as markers of plant metaphors, and plant metaphors in some Akan adinkra textiles. Section \ref{07_Conclusion} is the conclusion.

\section{Background} \label{07_Background}
\subsection{The Akan language and people}

The word ``Akan'' refers to the people as well as their language. Akans are considered from both an ethnographic and linguistic perspective. The ethnographic Akans encompass the linguistic Akans plus the Ahantas, Aowin, Nzema, and Sefwi, who do not speak Akan as a first language, but rather as a second language, and who share cultural similarities with the Akan (\cite{Obeng-Gyasi1987}). The linguistic Akan speak Akan as their native language and are the largest ethnic group in Ghana. In the 2010 national population census, 47.5\% of the Ghanaian population was linguistic Akan and about 44\% of the rest of the country's population speak Akan non-natively (see \cite{Agyekum2020}). The linguistic Akans occupy the greater part of southern Ghana.

Akan is spoken as a native language in nine out of the sixteen regions in Ghana, namely Ahafo, Ashanti, Bono, Bono East, Central, Eastern, Oti, Western, and Western North. The Akan-speaking communities in the Oti Region are sandwiched by the Ewe and Gur language communities. Akan has 13 dialects: Agona, Akyem, Akuapem, Akwamu, Asante, Assin, Bono, Buem, Denkyira, Fante, Kwawu, Twifo, and Wassaw. There are some Bono speakers in Côte d’Ivoire. Akan is used in the first 3 years of primary education as the medium of instruction, after which it is studied as an academic subject up to the University level \citep[2]{Agyekum2020}. The dialect names, such as Bono and Asante, refer first to the people, then to the dialects and the regions they occupy in Ghana.

\subsection{Methodology and literature review}

The methodology for this paper is qualitative and involves secondary data from library and internet sources, as well as interviews. This research paper involved various works on \isi{cultural linguistics}, cognitive semantics, and metaphors. These include metaphors for the expression of anger, patience, distress, well-being, and face concepts (see \cite{Agyekum2006From, Agyekum2015a, Agyekum2015b, Agyekum2016, Agyekum2018, Agyekum2020}). Other scholars who have worked on areas of Akan metaphor include \citet{Ansah2014}, \citet{Dzokoto2010}, \citet{DzokotoandOkazaki2006}, and \citet{Gyekye1997}. \citet{Ansah2014} discussed the metaphoric and metonymic conceptualizations of fear among the Akans. 

Apart from these academic works, the following works were very useful to this study: Nketia’s \citeyearpar{Nketia1978} \textit{Amoma} and Nketia's \citeyearpar{Nketia1974} \textit{Ayan}. Christaller's \citeyearpar{Christaller1933} and Appiah et 
al.'s \citeyearpar{Appiahetal2000} Akan dictionaries were also consulted, which provided English equivalents of some archaic terms. I also benefited from Irvine’s \citeyearpar{Irvine1930} book \textit{Plants of the Gold Coast} and Abbiw’s \citeyearpar{Abbiw1990} book \textit{Useful plants of Ghana}. Four renowned Akan scholars, namely Agya Koo Nimo, Mr. Bosie Amponsah, Nana Ampadu I, and Mr. Baning Peprah were interviewed for this project. 

Mr. Baning Peprah, an MPhil holder in the Akan language, was interviewed in May 2021. Mr. Bosie Amponsah, a seasoned broadcaster in Akan, was interviewed for four hours in August 2020. I spent four hours with the late Nana Ampadu I, who was well versed in Akan language and culture in October 2020. He was a renowned and talented Akan highlife musician who relied greatly on Akan folktales, especially those on animal and plant metaphors. The last expert, interviewed in July 2021, was Mr. Daniel Amponsah, popularly known as Agya Koo Nimo in the Akan folk music circles. Agya Koo is a folklorist who has composed many songs since the 1960’s. He is a well-known expert in Akan language and culture. Some of his folk songs are in the form of folktales imbued with animal and plant metaphors. We spent about four hours looking at the structure and functions of some Akan plants including those used in his compositions.

The interviewees were given a list of familiar plants, and we spent on average four hours finding out the positive and negative characteristics they know about these plants. I further interviewed them on the plant metaphors in proverbs, panegyrics, elegiac poems, adinkra symbols, and plant metaphors about youth, family, marriage, and those used in our daily conversations. All of them were very knowledgeable in the Akan language and culture. I found out from them more about the characters and functions of plants.

In the case of the panegyrics, I consulted some praise singers at the Otumfoɔ, the Asante King’s palace in Kumasi Manhyia, who explained some of the plant metaphors in the panegyrics and drum language to me. I also relied heavily on J. H. Nketia’s books \textit{Amoma} \citeyearpar{Nketia1978} and \textit{Ayan} \citeyearpar{Nketia1974}, \textit{Funeral Dirges} \citeyearpar{Nketia1955}, where I obtained some other panegyrics and drum text language on plant metaphor. The four experts finally cross-checked the authenticity of the attributes of the Akan plants gathered from the secondary data and gave their feedback. Apart from the four renowned Akan scholars, I also interviewed five male and five female Akan Oral literature graduate students. I also employed my own introspection as a native Akan speaker to analyze the expressions.

This methodology also tested any evidence of the current use of plant metaphors. It came out that, since metaphors are a part of life, these plant metaphors are in current use in ordinary conversations, proverbs, folk songs, and in institutionalized genres like panegyrics, drum language, and folksongs, which Akan people are very proud of. Incidentally, the youth consulted for this project were rather familiar with plant metaphors that relate to love because the highlife artistes of the current generation use plant metaphors to talk about love in their songs.

\section{Theoretical framework: Cultural Linguistics} \label{07_Theoretical}

The theoretical framework for the paper is \textit{Cultural Linguistics}, including its sub-branches \textit{Conceptual Metaphor} and \textit{Cultural Metaphor}. 

\subsection{Cultural Linguistics}

\citet[473]{Sharifian2015} states that ``Cultural Linguistics is a multidisciplinary area of research that explores the relationship between language, culture, and conceptualization.'' It is founded on the principles of cognitive linguistics, hence its application to the discussion of Akan plant metaphor. \citet[476]{Sharifian2015} further postulates that Cultural Linguistics has drawn on several other disciplines to yield a framework that may be best described as \textit{Cultural Cognition and Language} because it affords an integrated understanding of the notions of ``cognition” and ``culture'', as they relate to language (see also \citealt{Sharifian2011}). To Sharifian, one important function of cultural cognition is that it embraces the cultural knowledge that emerges from the interactions between members of a cultural group across time and space. The use of flora metaphor among the Akan of Ghana would be based on the participants, the setting, the goals, and the subject matter, as well as compliance to the Akan cultural norms. In effect, it touches on Hymes’ \citeyearpar{Hymes1972} acronym SPEAKING.\footnote{SPEAKING: S (scene and setting), P (participants), E (ends), A (act sequence), K (key), I (instrumentalities), N (norms of interaction), G (genre)} Our discussion of semiotics, proverbs, panegyrics, and elegiac poetry will attest to this connection.

\citet[477]{Sharifian2015} further considers the analytical tools that are used in the workings of Cultural Linguistics and postulates the super-ordinate term \textit{cultural conceptualizations}. Under this, he identified ``cultural schema'', ``cultural category” (including ``cultural prototype”), and ``cultural metaphor''. The current paper will focus on the ``cultural metaphor''. It is the cultural metaphor that relates very well to ``conceptual metaphor” postulated by Lakoff and Johnson, and both will help to discuss and explain Akan plant metaphor.\footnote{\citet[482]{Sharifian2015} refers to Cultural Linguistics such that it is interested in exploring conceptual metaphors that are culturally constructed as cultural metaphors (see also \citealt{Sharifian2011}).} 

In the conceptualization of Akan plant metaphor, we will see that some meta\-phors have been fossilized based on the strong interrelationships between the Akan as an agrarian society and the plants in traditional periods when they were living in the rural areas and farming was the major occupation. With urbanization and the distancing of many local plants from human beings, the current dispensation may see the plant metaphors as abstract. \citet[482]{Sharifian2015} discussed fossilized conceptualizations as follows:

\begin{quote}
    Some cases of conceptual metaphors are simply “fossilized” conceptualizations that represented active insight at some stage in the history of the cultural cognition of a group. Such metaphors do not imply current speakers of the language have any conscious awareness of the cultural roots of the expressions, or are engaged in any conceptual mapping when they use them. In such cases, the conceptual metaphors may serve rather as cultural schemas which [guide] thinking about and helps with understanding certain domains of experience. In some other cases, the expressions that are associated with such cultural conceptualizations may be considered simply as figures of speech.
\end{quote}

\noindent \citet[482]{Sharifian2015} summarizes \isi{cultural linguistics} as follows: 

\begin{quote}
    Cultural Linguistics explores human languages and language varieties to examine features that draw on cultural conceptualizations such as cultural schemas, cultural categories, and cultural conceptual metaphors, from the perspective of the theoretical framework of cultural cognition.
\end{quote}

\citet[2]{Sharifian2017} echoes the tenets of cultural linguistics: ``Cultural linguistics engages with features of human languages that encode or instantiate culturally constructed conceptualizations encompassing the whole range of human experience.'' We will see how this works in the data on Akan proverbs, panegyrics, and poetry. 

\subsection{Conceptual metaphor (CM)}

\citet{Agyekum2020} uses conceptual metaphor to discuss Akan emotions of distress, stress, sorrow, and depression. We are adopting the same \is{conceptual metaphor}conceptual meta\-phor theory that will usher us into the discussion of plant metaphor. \textit{Conceptual Metaphor Theory} was proposed by \citet[177]{LakoffandJohnson1980} who describe it as follows: 

\begin{quote}
    Many aspects of our experience cannot be clearly delineated in terms of the naturally emergent dimensions of our experience. This is typically the case for human emotions, abstract concepts, mental activity{\dots}Though most of these can be experienced directly; none of them can be fully comprehended on their own terms. Instead, we must understand them in terms of other entities and experiences, typically other kinds of entities and experiences.
\end{quote}

\begin{sloppypar}
In conceptual metaphor, concepts expressed in metaphors, idioms, and proverbs correspond to cultural traits, sociocultural interactions, natural experiences, and basic domains of human life, including bodily perception and movement, basic objects, and the environment (see \cite{LakoffandJohnson1980}). \citet[5]{Semino2008} posits that ``conceptual metaphors are defined as systematic sets of correspondence, or `mappings,’ across conceptual domains, whereby a `target' domain is partly structured in terms of a different `source' domain'' (see also \cite{Kövecses2002}). Using this framework, in this paper, we map source domains, plants, to our target, human behavior. 
\end{sloppypar}

\section{What is plant metaphor?} \label{07_WhatIs}

Plants are sometimes used to conceptualize abstract phenomena (\cite{Kövecses2002}). Most plant metaphors have strong positive evaluations. \citet[8--9]{Kövecses2002} finds that plants are one of the most common source domains for metaphorical mapping of organizations and therefore suggests the conceptual metaphor \textsc{social organizations are plants}. The mapping expressions manifested in plant metaphors are as follows: (a) the whole plant, (b) branches, (c)  growing, (d) pruning, (e) roots, (f) blossoms/flowers, and (g) fruit. \citet[8--9]{Kövecses2002} employs knowledge of plants as a schema for organization. These conceptual mappings are transparent because the plants and animals referenced are those we find in our environments, especially in rural areas. 

The grounding of our metaphors is based on our basic experiences in life, and one of the most fundamental human experiences is agriculture (see \cite[126]{UngererandSchmid2009}). It is natural for human beings to find similarities between plants and themselves.\footnote{\citet[39]{Grząśko2017} states categorically that plantosemy may be defined as employing various plant-terms to denote human qualities. Plantosemy is thus a form of metaphoric semantic change (\cite[80]{GathigiaandMaitaria2019}). \citet[112]{Grząśko2017} postulates that within the plantosemic developmental path, we are dealing with a mapping between the conceptual categories \textsc{plant} and \textsc{human being}, reflected in the pattern \textsc{plant → human being}.}  Plants provide humans with basic needs: shelter, food, medicine, clothing; and, the products of plants are used in most industries. The agricultural revolution opened the gate for the industrial revolution, and most industries globally rely on agricultural raw materials.
 
\citet{Esenova2007} notes that some plants, especially roses, bear sharp and woody thorns. The metaphor of thorns normally has strong negative connotations since we experience the sharp physical pain of a thorn prick. This metaphorical conceptualization stems from a more general metaphor \textsc{emotional pain is physical pain}. We thus speak metaphorically of emotional pain in terms of physical pain caused by thorns with a negative connotation. In Akan, the plant \textit{nsansono} `nettle’ is considered negative because it has some bristles that touch a person’s skin and bring irritation that causes people to scratch themselves. We have the Akan expression, \textit{abusua nsansono}, `a nettle family’ to indicate some family conflicts.

\citet{Charteris-Black2004} postulates that negative emotions are not evoked in all plant metaphors, as some plant metaphors are normally associated with strong positive evaluations. In Akan, we have both positive and negative notions depending on the type of plant that we are dealing with and the context of usage. In the view of the Akan, a stronger and bigger tree connotes a stronger relationship, reliability, and dependability. For example, the Akan say \textit{X anya duapa de ne nnwan amantam mu} ‘X has got a dependable tree to tie his sheep to.’ If a sheep is tied to a dependable tree, it cannot free itself and go astray or walk wayward. \citet[83]{GathigiaandMaitaria2019} states that a strong plant may be compared to a relationship that has a strong bond. The ephemerality of human beings is compared to flowers; they blossom fast and nice but wither later. 

The following section focuses on certain areas where plant metaphors are manifested among the Akan of Ghana. These areas include youth, marriage, death, family, elegiac and panegyric poetry, proverbs, and \textit{adinkra} symbols. 

\section{Thematic areas of Akan plant metaphor and their analysis} \label{07_THematic}

The Akan word \textit{ababunu} refers to the youth, and the metaphor is \textsc{unripened fruit is youth}. The unripened fruit (youth) has conceptually not arrived at its maturity state for harvesting. It has not served its purpose of being mature to serve mankind so that it can be harvested as food. It is bitter and of less importance. The youthful era is not the stage to marry or to give birth. The Akan therefore advise their youth to take their time and mature fully and thus get ripened. The youth are warned that if they rush in life, they will die prematurely, and this is \is{conceptual metaphor} conceptualized as \textit{wɔbɛbu abugyen}. The verb \textit{bu} means ‘to break’; \textit{gyen} is an ideophone that connotes the sound of something breaking instantly and unexpectedly. This is compared to a fruit or plant that has been broken due to wind or a heavy storm. We can thus recognize the Akan conceptual metaphor \textsc{maturity is a ripened fruit}. This is similar to the \ili{Polish} conceptual metaphor \textsc{ripening is developing} (see \cite[18]{Filipczuk-Rosińska2016}). The distinction between a blossoming plant and a withering or dead plant in the work of \citet{Filipczuk-Rosińska2016} is also found among the Akan in the following adage in (\ref{Ex.freshtree}) where the `fresh tree' is the youth, and the `wooden tree' refers to the aged adults.

\ea \label{Ex.freshtree}
\gl Dua-mono	 bu 	na 	dua-wuiɛ 	nso 	bu.\\
\glt  ‘The fresh tree breaks and the wooden tree also breaks.’
\z

The death of the two contrastive trees fosters the conceptual metaphor that \textsc{human beings are plants}, so death can befall on any of them.

The term \textit{ahoɔfɛdua} literally means a `beautiful tree’ that blossoms so nicely. This is metaphorically used to describe a beautiful lady in the metaphor \textsc{beautiful tree is pretty lady}. This is, however, not surprising because the human physical structure is metaphorically compared to a tree in the term \textit{onipadua}, made up of \textit{onipa} `person’ and \textit{dua} `tree'. The Akan, like the \ili{Polish} and the \ili{English}, share the same conceptual metaphor \textsc{a human being is a plant} (see also Romans 11: 17--24 where Paul compares religious communities to trees and branches).
	
Among the Akan, a fresh marriage is compared to a cocoa farm that has started bearing fruit, as in adage (\ref{Ex.newmarriage}). 

\ea \label{Ex.newmarriage}
\gl Awarefoforɔ te sɛ kookoo nsiwa.\\
\glt  ‘A new marriage is like a new cocoa farm that has started bearing fruit.’
\z

Like the cocoa farm, an Akan couple knows that care must be taken to let it maintain its fruitful standing. Again, they must be aware the tree will grow old and deteriorate and eventually die. Conceptually, most marriages travel through this cycle. It implies that a new marriage is full of joy that cannot persist forever (see \cite{Agyekum2012}). Here, the conceptual metaphor \textsc{a human being is a plant} is manifested. 

\subsection{The Akan plant metaphor about death and in dirges}

The Akan use plant metaphor in describing death and in their reflections about death. The extension of the metaphor is seen in their elegiac poetry when women sing dirges. The Akan adage states this view in (\ref{Ex.humanflower}).

\ea \label{Ex.humanflower}
\gl Nnipa te sɛ nhwiren, ɛnkyɛ na atwam, ɛnkyɛ na ate atɔ.\\
\glt  ‘Human beings are like flowers, soon they wither and soon they fall.’
\z

In discussing the metaphor of love as a plant among the Gĩkũyũ (also called Kikuyu or Agĩkũyũ) of Kenya, \citet[83]{GathigiaandMaitaria2019} state that, ``just like a flower opens up and then withers after some time, the same case may happen to love. This implies that love can blossom and wither or die with time''. \citet[577]{Basson2006} expatiates that:

\begin{quote}
    through the plant metaphor people can be viewed as plants with respect to the life cycle – more precisely, they are viewed as that part of the plant that burgeons and then withers or declines, such as leaves, flowers, and fruits, though sometimes the whole plant is viewed as burgeoning and then declining as with grass or wheat.
\end{quote}

\citet[581]{Basson2006} concludes his paper on \textit{People are Plants} by noting that Biblical Hebrew authors relied on their deep and abundant knowledge of plant metaphors to describe the relationship between Yahweh and his people, the Israelites. Psalm (1:3) in the Bible reads ``They are like trees planted by streams of water, which yield their fruit in its season, and their leaves do not wither. In all that they do, they prosper.''

Let us now touch upon plant metaphors that relate to death. On the death of a chief, the Akan use the plant metaphor in (\ref{Ex.mightytree}).

\ea \label{Ex.mightytree}
\gl Odupɔn bi 	atutu.\\
\glt  ‘A mighty tree is uprooted.’
\z

The \isi{conceptual metaphor} is \textsc{a king is a mighty tree}. The impact of the uprooting of a mighty tree is that (a) it comes with a big sound \textit{pum}, (b) it creates a vacuum in the forest because it occupied a big area, and one can see the void, and (c) finding an instant and suitable replacement becomes difficult because it will take a longer period for some younger trees to grow to that stage. All these interpretations imply that the death of a king means a big loss to the society. There is a popular Akan maxim about death that is captured by the plant metaphor in (\ref{Ex.climbtree}).

\ea \label{Ex.climbtree}
\gl X nkɔforo dua a, waforo kakapempem.\\
\glt  ‘To climb a tree, X has climbed the Kakapenpen tree.’\footnote{The \textit{kakapenpen} (rauvolfia vomitoria) tree is very short and has many soft branches.}
\z

The \textit{kakapenpen} tree is not strong, and when people climb it, it will easily break and they will fall. If you need to climb a tree to protect yourself from wild animals chasing you, it would be prudent to climb a strong tree for safety. While the others who took refuge on top of strong trees are safe and alive, the one who climbed the \textit{kakapenpen} has been overcome by death. The conceptual metaphor with the climbing of such a tree is \textsc{death is climbing a kakapenpen tree}.

In describing the death of a dignitary and a responsible member of the family, the deceased is captured in \citet[35]{Nketia1955} as follows in (\ref{Ex.bigbranches}). 

\ea \label{Ex.bigbranches}
\gl Woyɛ dua tan a w’abaaso. Woaso wosowo; mmɔfra ba w’ase a, wɔanya bi adi.\\
\glt  ‘You are a mighty tree with big branches laden with fruits; when children come to you, they find something to eat.'
\z

The \isi{conceptual metaphor} in this piece is \textsc{a benevolent person is a mighty tree}. The many branches of a mighty tree represent to many people who are affiliated with this kind individual that the fruits are the gains the individuals get, and that any time people approach him/her they get something needed for their life. 
 	
In the dirge in (\ref{Ex.motherokra}), the deceased woman (mother) is compared to an okra pod that can contain many seeds and yet maintains its size.

\ea \label{Ex.motherokra}
\gl Eno nkrumakɛsɛ a ne yam abaduasa na ɔmmoa.\\
\glt  ‘Mother, the okra, full of seeds but does not shrink.'
\z

The dependable woman is so responsible and resilient that she can care for all her children and even more. Such a description gives a vivid picture of the loss that children have encountered with the demise of their mother. The conceptual metaphor is \textsc{a reliable mother is an okra pod}. 

\subsection{Plant metaphor in the Akan family}

The Akan word for family is \textit{abusua}. There are certain plant metaphors that describe the Akan family system by looking at the structure of the tree, its roots, stem, branches, and leaves. They further consider how the individual trees combine to form the forest, and how as a collective entity, each forest is also an individual. 

\citet[98]{Kövecses2002} formulated a conceptual metaphorical schema on plants: \textsc{complex abstract systems are plants}. This is evident in fields like kinship with its “family trees”, “branches of families”, and so on. Similar metaphorical schemas are important in many cultures across the world. For instance, in the Austronesian world, the growth and propagation of social groups is systematically likened to the structure of plants. Based on these, \citet[218]{Schecter2015} posits that ``Plant metaphors are deeply embedded in ways of thinking about such issues as social organization in recent Western cultures.'' We can see a reflection of \citet{Kövecses2002} schema in the area of the family with its complexities among the Akan of Ghana.

Some derived plant metaphorical expressions within this realm in Akan include, \textit{abusuaban}, made up of \textit{abusua} `family’ and \textit{ban} `branch’. This term is used for each of the eight matrilineal clans: Aduana, Agona, Asakyiri Asenneɛ, Asona, Bretuo, Ɛkoɔna, and Oyokoɔ. Each of these families is considered the family stem~– roots and branches are the individual families scattered all over the Akan geopolitical areas and the thirteen Akan dialects in the different regions of Ghana. 

There is another term, \textit{dua no ara mu} (lit.) `the tree itself’. This refers to the core and inner circle members of the family, especially, \textit{abusua panin} ‘the family head'. From the standpoint that the Akan \textit{abusua} is a complex, collective entity, there are proverbs based on plant metaphors to indicate the unity in diversity, as in (\ref{Ex.elephantgrass}).

\ea \label{Ex.elephantgrass}
\gl Abusua 	hwedeɛ 	gu 	nkuruwa.\\
\glt  ‘The family elephant grasses	are put	 into groups.'
\z

The conceptual metaphor in (\ref{Ex.elephantgrass}) \textsc{a big family is an elephant grass}. According to \citet[20]{Nketia1955}, The lineage, like grass, grows in little tufts that combine to form the group. An additional plant metaphor that relates to the family is in (\ref{Ex.matriclan}). 

\ea \label{Ex.matriclan}
\gl Abusua te sɛ kwaeɛ, wogyina akyire a, ɛbɔ mu koro. Wopini ho a, na wohunu sɛ dua koro biara wɔne siberɛ.\\
\glt  ‘The matriclan is a like a forest, if you are standing afar, it forms a canopy and a single unit, but if you draw closer, you will see that each tree has its position.’ (see \cite{Appiahetal2000}: Proverb 1259)
\z

The \isi{conceptual metaphor} in (\ref{Ex.matriclan}) is \textsc{a family is a forest}. It implies that since the Akan nuclear family is complex and large, made up of hundreds of people, an outsider may think that there is a strong unity without conflicts, discrimination, and strategies of inclusion and exclusion. When it comes to funerals and inheritance, the sibling relationship precedes cousins and strangers who might have adopted the family. They concretize this division or grouping with the adage in (\ref{Ex.splitfamily}).

\ea \label{Ex.splitfamily}
\gl Deɛ ɔwoo mmaa mmienu na ɔde abusuapae baeɛ.\\
\glt  ‘She who gave birth to two daughters is the one who split the family.' (i.e., splitting of the family is done by giving two daughters)
\z

In the matrilineal system, people still get closer to their siblings than to their cousins, even though the term \textit{nua} is used for both. The offspring of two daughters would be cousins rather than siblings thereby splitting the family.

There are certain \isi{proverbs} that portray that, despite the individualism and set groups in the family, the family is considered a unified and fortified group. This is indicated in (\ref{Ex.familytree}).

\ea \label{Ex.familytree}
\gl Abusua dua wɔntwa.\\
\glt  ‘The family tree is not cut.'
\z

The family is considered as strong and dependable as a mighty tree, a unit that should be protected. The implication is that one cannot disparage his/her origin. No matter how bad and mischievous your family is, it will forever remain your family.

\ea \label{Ex.itchingplant}
\gl Wotɔ abusua nsansono mu a, woyɛ biribiara a ɛnyɛ yie.\\
\glt  ‘If you fall into an \textit{itching plant} family, nothing goes well with you.'
\z

In (\ref{Ex.itchingplant}), the family is conceptualized as a farm that is full of itching plants that make you scratch yourself anytime the thistles touch your body. It depicts that when you find yourself in such a family, there are always standing blocks and impediments that will not allow you to move and concentrate on fruitful ventures. Your available precious time is spent on scratching your body, and that will thwart your efforts of moving forward into action on the farm.

In the family lineage, the Akan talk about \textit{X ase firi Y} `X’s root comes from place Y'. The stem is the personality, and the root is where the person migrated from. In the family, the Akan also have \textit{X kɔ X nkyiri}, `X has gone back to X’s roots'. Another plant metaphor that relates to family is \textit{X ase atrɛ} `X’s root has spread’ or \textit{X ase atim} `X’s root has been well grounded'. To say \textit{X ase atrɛ} implies that the women in the family have given birth to more children, and the family has become large. It further implies that the family becomes more established in terms of numbers and especially in riches.
 
There are cases where because of an accident, ``generational curse'', or lack of childbirth in a family, there are concerns about a decrease in numbers. This can happen in the Akan matrilineal family system where only the offspring of the women are part of the maternal family. The Akan use the expression \textit{abusua no ase ahye} `the root of the family is withered (i.e., burnt)'. This is a painful situation where the growth of the family is thwarted, and there is no replacement for the current family members. In all these examples, the conceptual metaphor is \textsc{a lineage is a root}, and it implies \textsc{a family is a plant}.

\subsection{Plant metaphor in panegryic poetry}

Akan plant metaphors feature prominently in their various poetic genres like panegyric poetry, elegiac poetry, libation, drum language, and all forms of folk songs. In this paper, however, we only concentrate on panegyric poetry and refer to the others only in passing. Akan plant metaphor in oral literature is typically found in poetry more than in prose or drama.

Panegyric is a rich Akan oral literature genre employed as praise poetry for chiefs and important dignitaries of the society. As an agrarian society, some of the lines have plant metaphors that recount the achievements, feats, problems, hardships, and maltreatment that the past kings have gone through. In this section, I draw extracts from Nketia’s \citeyearpar{Nketia1978} book entitled \textit{Amoma} on Akan panegyrics. 

\begin{table}
\small
\caption{Poem 28 (\cite{Nketia1978})}
\label{tab:Poem28}
\begin{tabular}{ll}
\lsptoprule
     Ɔno no!		&		Alas he is here!\\
Ɔbrɛdwane ee!		&		Dried palm fronds!\\
Ɔbrɛdwane ee!	&			Dried palm fronds!\\
Ɔsɛe Tutu yɛpem-yɛya wo	&	Ɔsɛe Tutu they are pulling and insulting you\\ 
Ɔsɛe Tutu yɛpem-yɛya wo	&	Ɔsɛe Tutu they are pulling and insulting\\
Yɛaya wo aya wo		&	They have insulted you\\
Yɛaya wo a, wonkɔ	&		When you are insulted, you do not move\\
Wadi asɛmpa na yɛatwiri wo	&	You have done well but you are insulted\\
Ɔsɛe Tutu, woatɔ nkyene akyɛ.&	You have bought salt as a gift\\
Nanso yɛde mako ada wo ase	&	But you have been thanked with pepper\\
Apampammire a ɔsi abura	&	The \textit{apampammire} plant that covers wells\\
\lspbottomrule
\end{tabular}
\end{table}

In the panegyric Poem 28, the ungratefulness of the citizen is measured by the way they have treated the king Osei Tutu, who is now compared with a wooden palm frond. When it was green and blossoming, the people used it for many things, including sweeping and basketry. Now that it is dried, they are pulling it from the palm tree, not recognizing the pain it goes through. Even though it is painful, the target is not perturbed, and it is not moving an inch. The \isi{conceptual metaphor} is \textsc{a benevolent leader is dried palm frond}. The ungratefulness of the citizens is portrayed by thanking the one who bought them salt (lit. sweetness) with pepper, which is hot. If you cannot reciprocate with his benevolence, do not pay him bitterly. The poem ends by comparing the king with the \textit{apampammire} creeping plant that can cover the well and therefore trap the people. The metaphor is \textsc{a hidden danger is apampammire plant}; in other words, his subjects would realize the negative effect of ungratefulness only once they are trapped by it.\largerpage

\begin{table}
\small
\caption{Poem 35 (\cite{Nketia1978})}
\label{tab:Poem35}
\fittable{\begin{tabular}{ll}
\lsptoprule
Ɔno no!			&			Alas, here he is!\\
Ɔsɛe Tutu wo ho asɛm merete merete	&	Ɔsɛe Tutu, I constantly hear about you\\
Wo ho asɛm te sɛ 		&			Issues about you are like\\
Onyinatan mmiɛnsa so ahahan awisi	&	The leaves of three mighty oak trees\\
Wo ho asɛm merete merete.&			I constantly hear about you\\
Owuo na ɛsi aso			&		It is death that can make one dumb\\
Mete a, mete no ko so	&			I hear it about battles\\
Mete no mmarima so		&		I hear it from your manly prowess\\
Mete no akatakyie so	&			I hear it from a warrior’s perspective\\
Ɔkyere fa-nim-ako a wannane ko	&		The victor who did not leave the war\\
Anto mpanin ne mmɔfra so,	&		As a burden to elders and the youth\\
Okontonkurowi a ɔda amansan kɔn mu,	&	The moon that encircles round our necks\\
Na amansan nkɔmmɔ a yɛdie ne ɔsei Tutu!&	All the conversations are on Osei Tutu \\
\lspbottomrule
\end{tabular}}
\end{table}

Poem 35 utilizes the plant metaphor of the oak tree in the \isi{conceptual metaphor} \textsc{a famous person is an oak tree}. Apart from its hugeness in terms of its trunk, the Akan oak tree in the tropical forest has many branches and uncountable small leaves. If issues and conversations about the king are comparable to all the leaves on three mighty oak trees, then the person is very famous. The poet highlights the areas of the king’s popularity in terms of wars, bravery, industriousness, and dexterity, and finally his personal interest in wars, joining the army on the battlefield.

\begin{table}
\small
\caption{Poem 48 and Poem 71 (\cite[19, 27--28]{Nketia1978})}
\label{tab:Poem4871}
\fittable{\begin{tabular}{ll}
\lsptoprule
Ɔno no!				&		Alas, here he is!\\
Ankaadudwane a 		&			The lime full of thorns\\
Mmɔfra kɔ aseɛ a yɛtu nnɛeɛ		&	That children tread cautiously under it\\
Na ɔsei Tutu da aseɛ rebu mfumpaa no oo!	&	And Osei Tutu is wrestling under it \\
Nana na wayɛ ne ho dufɔkyeɛ		&	Nana who is like a wooden log\\
Sɛnkosɛnko a ɔsɛn akwantemfi,	&	 	That hangs in the middle of the road\\
Yɛwura n’ase a, abɔ yɛn;		&	 	When we bend under it, it hits us\\
Amfɔ a yɛkwati a, yɛayera,	&		A big trap when ignored, we are lost\\
Yɛhuri tra a, yɛasi amena mu,	&		When we jump over it, we land in a hole\\
Yɛtena so a, yɛn to afɔ,		&		When we sit on it our buttocks are wet\\
Yɛpagya yɛn ani a, adeɛ atɔ so,		&	If we lift our eyes, something falls on it\\
Yɛde akuma si mu a, wagye amene!	&	When we put an axe in it, it swallows it\\
\lspbottomrule
\end{tabular}}
\end{table}

Poems 35 and 48 start by comparing the bravery, resilience, and endurance of the chief with his agility and wrestling skills under a lime tree with many thorns,
especially when it has been pruned. Children and people who go under it walk carefully and slowly so that they do not get hurt. Ironically, it is such a place where the wonderful king can roll on the ground and not be afraid of the thorns. The conceptual metaphor is \textsc{bravery and resilience is rolling on thorns}.

In the second part of the poem, the king is compared to a rotten log hanging in the middle of a path that has all the abilities to overcome people who use that path. No matter who and what you are, and what your strategies are, you will be conquered. The conceptual metaphor is \textsc{invincibility is a rotten log}. One may assume that it would be easier to deal with a rotten log than a fresh log, but rather the reverse is the case. It implies that when it comes to the Akan chiefs, nothing should be taken for granted, especially regarding their age and environment.

\subsection{Flora metaphor in some Akan proverbs}

Proverbs are terse sayings that embody general truths or principles, as well as ways of life. These general truths are based on the people’s past experiences, mindset, philosophy, perception, ideology, socio-cultural concepts, environment, and worldview (see \cite[11]{Agyekum2012}). \citet[362]{Momoh2000} looked at African proverbs from a philosophical point of view and avers that:

\begin{quote}
For anything to be known it has to be put into proverbs and for anything to be deknown it has to be removed from proverbs. Proverbs represent the last authority on the communal or public aspect of a people’s beliefs or philosophy on any concept or issue. In short and in summary, for the traditional African, to be is to be in proverbs and not to be is not to be in proverbs.    
\end{quote}

\begin{sloppypar}
Oral literature scholars like \citet{Agyekum2005}, \citet{Finnegan2012}, \citet{Okpewho1992}, and \citet{Yankah1989a} have extensively researched proverbs. \citet[9]{Agyekum2005} states that ``proverbs are interpretations of traditional wisdom based on the experiences and socio-cultural life of our elders''. In Akan indigenous communication, the use of proverbs is the acknowledged mark of one’s communicative competence. Speakers' ability to use appropriate proverbs in appropriate socio-cultural contexts depicts how competent and well versed they are in the language (see  \citealt{Agyekum2012, Finnegan2012, Yankah1989a}). In the Akan agrarian community, indigenous knowledge on plant metaphor is found in proverbs such as (\ref{Ex.stormbrake}) and (\ref{Ex.uprooted}).
\end{sloppypar}

\ea \label{Ex.stormbrake}
\gl Dua korɔ gye mframa a ɛbu.\\
\glt  ‘If one tree is used as a storm brake, it will fall.'
\ex \label{Ex.uprooted}
\gl Nnua nyinaa woso a, ɛbɛka abɛ.\\
\glt  ‘If all trees get uprooted, what would be left behind is the palm tree.'
\z

In examples (\ref{Ex.stormbrake}) and (\ref{Ex.uprooted}), trees represent human beings in society. In (\ref{Ex.stormbrake}), the tree represents a single person in a family who shoulders all the responsibilities~-- the load is too much that he\slash she is overburdened. The other side of the proverb is where, in an organization, the director or chief executive wants to do everything and does not delegate. The conceptual metaphor of a single tree implies that the organization will collapse. In (\ref{Ex.uprooted}), there is the metaphor of resilience in a group or an organization, such that when all other employees, organizations, or family members fall out, the resilient and committed person will remain. \textsc{An indefatigable and never dying person is a palm tree}. The metaphor of the palm tree also features prominently in Akan \textit{adinkra} symbols. 

Examples (\ref{Ex.pawpaw}--\ref{Ex.rope}) offer flora proverbs based on the issues of dependency, benevolence, and ungratefulness.

\ea \label{Ex.pawpaw}
\gl Borɔferɛ a ɛyɛ dɛ na nsee kɔ so ayie.\\
\glt  ‘It is the sweet pawpaw that the \textit{nsee} birds flock on.' (benevolence)
\ex \label{Ex.treebearfruit}
\gl Akɔnkɔdeɛ so a, na ɛnsee kɔ so adidie, ɛso po pɛ na woagyae.\\
\glt  ‘When the \textit{akɔnkɔdeɛ} tree bears fruits, the \textit{nsee} birds flock to eat them, but when they are all finished, they stop.' (ungratefulness)
\ex \label{Ex.beatingstick}
\gl Ankaaadudwane,	 m’ayɛyie 	ne 	mmaa.\\
\glt  ‘I am the lime; my reward (thanks) is the beating by sticks.' (ungratefulness)
\ex \label{Ex.rope}
\gl Ɛnam dua so na homa hunu soro.\\
\glt  ‘It is through the tree that the rope (vine) gets to see the top.' (dependency)
\z

In examples (\ref{Ex.pawpaw}) and (\ref{Ex.treebearfruit}), plant metaphors represent people who are interested in booty, who are greedy and always move to places where they can only get juicy things and run away in times of hardship and scarcity. In (\ref{Ex.pawpaw}), the pawpaw tree refers to wealthy people who have many friends; the conceptual metaphor is \textsc{a benevolent person is a sweet pawpaw tree}. Similarly, in (\ref{Ex.treebearfruit}), the \textit{akɔnkɔdeɛ} tree that bears some pink and red fruits represents the wealthy and benevolent people in the Akan society who are always flocked by family and non-family members. However, when they lose their wealth all the people despise them. Example (\ref{Ex.beatingstick}) comments on the use of lime in much Akan traditional herbal medicine; because of the thorny nature of the lime tree, people do not climb it but rather use sticks to pluck the fruits. The Akan people use the vocative to say that the lime is complaining that its gratitude from all its services to the Akan is how people beat it with sticks. The conceptual metaphor is Akan \textsc{ungratefulness is beating with sticks}. In (\ref{Ex.rope}), the real picture of how creeping plants and rope get to the treetop is captured; it is through the trees that the rope gets to the top, and therefore ropes should appreciate the services of the trees. All four plant metaphors in (\ref{Ex.pawpaw}--\ref{Ex.rope}) talk about real experiences in life and involve services, dependency, benevolence, greed, ungratefulness, and disappointment among people in the society. These foreground the conceptual metaphor \textsc{human beings are plants}.

In addition, the two plant metaphors below represent human beings in society who create problems for themselves.

\ea \label{Ex.middlepath}
\gl Akwantimfi sahoma me na mepɛ m’adonko na yɛreto me.\\
\glt  ‘The swinging rope in the middle of the path, I have created the situation for people to swing me here and there.' 
\ex \label{Ex.longgourd}
\gl Kɔntoa na pɛ na homa sa ne kɔn.\\
\glt  ‘It is the long gourd that gives the chance for a rope to be tied around its neck.'
\z

In example (\ref{Ex.middlepath}), it is the swinging rope in the middle of the path that has created problems for itself; others in the forest are relaxing and are out of trouble. In (\ref{Ex.longgourd}), there are many varieties of gourd, with some being very round, and the one in question has a long neck where a rope is easily tied as a handle. According to the \is{proverbs} proverb, the gourd has created a problem for itself by having a long neck. In society, there are problems that a lot of people encounter that are self-afflicted. These proverbs advise braggarts, husbands and wives, politicians, and heads of institutions who elevate themselves beyond their means, and therefore suffer because of their own behavior. In these two proverbs, plant metaphors are sourced from the swinging rope and the long-necked gourd, which represent self-affliction. 

\subsection{Use of the copula verb \textit{yɛ} and the simile, \textit{te sɛ} in plant metaphor}

There are certain plant metaphors in Akan that employ metaphor and simile to indicate that the characteristic of the plant is transferred absolutely to the human being, or that some features of the plant are like the behaviors of human beings.

\ea \label{Ex.gourdside}
\gl X ayɛ te sɛ apakyie, gye sɛ yɛbɔ ha bɔ ha.\\
\glt  ‘X is like a gourd, unless you hit the sides before it opens.'
\ex \label{Ex.neglected}
\gl Mayɛ aworomo, abusua aworo me ama.\\
\glt  ‘I am the \textit{aworomo} leaves; people have neglected me.'
\ex \label{Ex.kagya}
\gl Mayɛ Kagya, yɛakae me agya.\\
\glt  ‘I am \textit{kagya} (Grifonnia simplicifolia), I have been left uncounted.'
\z

In all the above plant metaphoric proverbs, there is the use of simile and metaphor where human beings are represented by plants. In (\ref{Ex.gourdside}), the insubordinate person is compared to the \textit{apakyie}, the small round gourd that can only be opened by hitting its sides. In (\ref{Ex.neglected}), a despised person in the family is represented by the \textit{aworomo} plant, whose name rhymes with the verb \textit{woro}, `to reject or abandon'. In (\ref{Ex.kagya}), the neglected person is metaphorically represented by the \textit{kagya} plant whose name also rhymes with the phrase \textit{yɛakae me agya} `they have counted and left me behind'. It is important to note that examples (\ref{Ex.neglected}--\ref{Ex.kagya}) also involve play on words (i.e., \textit{woro} and \textit{gya}). All these plant metaphors represent selective justice and exclusion in groups and societies.

Let us now look at two plant metaphors: (\ref{Ex.prekese}) portrays the popularity of an individual, and (\ref{Ex.driedokra}) represents childbirth and the safeguarding of children.

\ea \label{Ex.prekese}
\gl Prɛkɛsɛ gyaanaadu a ɔfiti kurotia a, na ne ho agye afieafie mu.\\
\glt  ‘Prɛkɛsɛ gyaanaadu (tetrapleurate traptera) who appears at the outskirts of the town and its fragrance is felt in all houses.'
\ex \label{Ex.driedokra}
\gl Mayɛ nkrumakɛsɛ, mabɔ me mma agu me yam.\\
\glt  ‘I am a dried okra pod; I have conserved my children in my stomach.'
\z

In (\ref{Ex.prekese}), the \textit{prɛkɛsɛ} is a spice that is so fragrant that even from afar one is aware of its scent. The plant metaphor is \textsc{popularity is prɛkɛsɛ}. In the case of (\ref{Ex.driedokra}), the dried pod with many seeds represents multiple births, while keeping seeds in the pod represents their security. The conceptual metaphor is \textsc{tight security is keeping in a dried pod}.

\subsection{Plant metaphor in some Akan semiotics: \textit{adinkra} textiles}\largerpage

\citet[125--126]{Agyekum2006From} worked on the semiotics of \textit{adinkra} textiles, as well as some on plant metaphor. These include the \textit{aya} plant and \textit{bese saka}, a bunch of kola nuts. The \textit{aya}, `fern’ is a small plant (more of a weed) that normally grows under cocoa farms. Even though it is soft, it can withstand any weather, including the dry season. When most plants are withered, it is still green. The Akan have closely examined the \textit{aya}, `fern’ plant and used it as one of their \textit{adinkra} symbols. The \textit{aya} plant is noted for its invincibility and continuous existence, steadfastness, and resilience. It shoots up again and again after many attempts to destroy it; when you weed it, it sprouts up after a few days. According to the Akan, it symbolizes stamina, fortitude, staying in power, defiance, and the ability to endure in the face of problems and animosities. The symbol advises people who face problems and challenges to consider the situation of the fern (see \cite[125]{Agyekum2006From}). The University of Ghana’s crest has three \textit{aya} ferns on it. This symbolizes the strong determination the university has in pursuing its academic mission without being discouraged by the problems and challenges of the time (see \cite[120]{Opoku1997}).

Another \textit{adinkra} symbol that is part of Akan plant metaphor is \textit{bese saka} `bunch of cola nuts.’ It symbolizes affluence, power, abundance, togetherness, and unity. The bunch refers to unity because separate pods, each with a few seeds, have been brought together. Here, one can mention the philosophy of unity in diversity. The cola nut is also a stimulant and a gift item used by most Islamic and ethnic groups of northern Ghana; the cola is a cash crop. The conceptual metaphors of these two plants are \textsc{fern is invincibility and resilience} and \textsc{unity in diversity is a bunch of cola nuts}. We can see in these two metaphors that there is a triadic relation, with a semantic extension from plants to \textit{adinkra} symbols, and further into the abstract concepts of human relations. 

\section{Conclusion} \label{07_Conclusion}

We have seen how metaphors are indispensable aspects of our life. We concentrated on conceptual metaphor where plants from the physical world are used as the source domains to derive targeted entities in human life. There is metaphorical mapping of one entity to another. Metaphors are culturally based on the people’s environment and their epistemology, indigenous knowledge, and worldview. In this paper, we looked at how the agrarian nature of the Akan culture and society influences the mapping of features of Akan plants onto human behavior in the conceptual world. Plant metaphors reference the entire plant either from its roots, stem, branches, seed, or leaves, or from a combination of some of these; this is prominently manifested in the Akan extended family system.

This paper employed Lakoff and Johnson's conceptual metaphor theory and Sharifian’s cultural linguistics with its sub-discipline worldview metaphor to discuss Akan plant metaphor. The scope of the paper covered Akan plant metaphor in proverbs, poetry, panegyrics, family, youth, marriage, death, and \textit{adinkra} symbols. 

Over the years, most of these rich plant metaphors have become so fossilized that many people, especially the youth, use them without knowing that they are metaphorical. The dynamism of Akan culture, coupled with urbanization and westernization, have made it impossible for the current generation to fully appreciate these plant metaphors. This limitation has emerged from the fact that their knowledge of the agricultural sector is normally tied to the food crops they find in their homes and in the market. They have faint ideas about the names of trees and plants and may be ignorant that some of the metaphorical expressions they hear and use are derived from plants. The youth are now often only conversant in plant metaphors that are derived from English flowers. 

This paper is a contribution to cognitive studies in Akan, and it also opens the gates for further studies and research in the areas of cognitive studies, involving embodiment and emotions, language and the mind, body part expressions, and animal metaphors, among others. It is hoped that scholars of other Ghanaian languages will extend their research into these areas, since metaphors are a part of our lives. All these efforts are aimed at the documentation and development of Akan and other Ghanaian languages.\il{Akan|)}
 


\section*{Acknowledgements}

My major acknowledgements go to my resource persons, especially Agya Koo Nimo, Mr. Bosie Amponsah, Nana Ampadu I, and Mr. Baning Peprah, who were interviewed for this project. They devoted much time and commitment to the study.

\printbibliography[heading=subbibliography,notkeyword=this]
\end{document}
