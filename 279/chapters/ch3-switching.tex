\documentclass[output=paper]{langscibook}
\ChapterDOI{10.5281/zenodo.8269230}


\author{Hanna Lantto\affiliation{University of Turku}} 
\title{Code-switching}  

% \lsConditionalSetupForPaper{} % Please use this instead of loading localpackages.tex, localbibliography.bib, etc. manually


\abstract{Code-switching (CS) refers to the use of two or more languages in the same interaction. Studies on CS have been abundant since the 1970s, and CS has been approached from several different perspectives, which has led to a scattered field with diverse methodologies and terminological debates. In spite of this, recuring patterns of CS have been identified, and several linguistic and extralinguistic factors have been found to affect the formation and distribution of these patterns. Unlike many other language contact phenomena, CS has been regarded as a mostly synchronic form of language contact, in which the linguistic systems stay separate. Some scholars, such as \textcite{muysken2000,muysken2013language} and \textcite{matras2009language}, have tried to situate CS in the wider context of language contact phenomena. During the last decade, proponents of the usage-based approach to language contact \parencite[such as][]{backus2013usage,backus2015usage} have argued for a view that connects the diachronic and synchronic aspects of CS. Nontheless, a more comprehensive framework is needed.}

\begin{document}
\maketitle
\label{chap_codeswitching}

\section{Introduction} 

The study of code-switching has become a prominent field within contact linguistics during the last forty-five years. Code-switching refers to the use of two or more languages in the same interaction.  Researchers working in multilingual communities noticed that monolingual norms previously described in the linguistic literature were not accurate with regard to descriptions of multilingual language use. In the late 1960s and early 1970s, Gumperz and his associates published several articles on CS and bilingual communication in different bilingual communities \parencite{gumperz1964punjabi,gumperz1967linguistic,gumperz1971convergence,blom1972directions}. Poplack’s multifaceted description of code-switching patterns in the Puerto Rican community in New York was published in 1980 \parencite{poplack1980sometimes}, and sparked a debate on CS grammar. These works have been followed by thousands of scholars examining structural-linguistic, pragmatic and sociolinguistic aspects of CS. 

Code-switching has many dimensions, and the research traditions of these dimensions have remained largely separate. Penelope Gardner-Chloros \parencite{gardner2009code} describes the field as akin to the famous poem about an elephant and six blind men who approach the animal from different sides. The first part that they touch – the trunk, the side, the leg – makes them all interpret the form of the elephant from their own point of view. CS studies have been, and still are, a scattered field. One important tradition is the pragmatic tradition that addresses the conversational functions of CS \parencite{gumperz1982discourse,auer1998code}. The structural-linguistic tradition discusses the grammatical patterns and constraints in CS \parencite{poplack1980sometimes,myers1997duelling, muysken2000}, the psycholinguistic tradition focuses on the impact of mixed constituents in the bilingual brain \parencite{gullberg2009technique,couto2017chapter}. The important insights of linguistic anthropology in the area of language ideologies and attitudes surrounding code-switching  \parencite{woolard1998introduction, woolard2004codeswitching, jaffe2007discourses} have mostly been neglected thus far by the scholars of other traditions. The patterns and factors found in these different branches of CS will be discussed in the following subsections.

Even though some researchers, such as \textcite{myers1993common,myers1997duelling} and \textcite{auer1988conversation,auer1999codeswitching}, have approached CS from different perspectives, in general CS scholars have kept the different traditions separate in their work. Few researchers have tried to draw a more complete picture of CS phenomena: Muysken’s model \parencite{muysken2000,muysken2013language} gathers the results of several previously published studies and builds a comprehensive model of the grammatical outcomes of CS in different types of communities. \textcite{matras2009language} situates CS in the context of other language contact phenomena, which he addresses both on an individual and on a community level. \textcite{auer1999codeswitching} connects the conversation-oriented approach with some grammatical models of CS to illuminate different types of CS according to their relation to language change. Backus approaches the role of CS in linguistic change from a usage-based perspective \parencite{backus2013usage}, which theoretically addresses CS as a complex, multi-sided phenomenon, yet focuses on its cognitive aspects. Common ground between the different approaches can be found in some instances, such as in the study of discourse markers. (Section \ref{subsubsec-comprehensive}). However, there is a general lack of exchange between the different subfields, which results in difficulties of seeing the research on CS, with its possibilities and limitations, as a whole.

Apart from the scattered nature of the subfields of CS research, the field has been involved in wide terminological debates about the term \textit{code-switching} itself. Within the structural-linguistic tradition, \textcite{muysken2000}, for example, preferred to reserve the term ``code-switching'' for the types of CS where the two linguistic subsystems stay separate. He proposed \textit{code-mixing} as an umbrella term, which is also apt for types of bilingual speech where the morphosyntactic systems of the languages are intertwined. \textcite{poplack1998introduction} have tried to separate “true code-switching" that follows Poplack’s grammatical CS constraints from “nonce borrowing", in which an item becomes temporarily part of the recipient system, and which they believe to be a different process altogether. \textcite{johanson2002contact} has proposed \textit{code-copying} as a wide term that encompasses several language contact phenomena. One solution for the debate, as suggested by \textcite{gardner2009code}, would be to use right-to-left definitions, for example “to call the alternation of two languages in conversation code-switching", as proposed by \textcite{janicki1990toward}. This would make CS a working definition instead of an essentialist definition of what code-switching is and what it cannot be.

\begin{sloppypar}
Those within the pragmatic tradition have criticized the term \textit{code-switching} since the 1990s \parencite{auer1998code}. Researchers focused on conversational approaches to CS have found the term to be misleading and not descriptive of how the speakers actually use their languages in interaction. All back-and-forth mixing between languages of linguistic varieties does not have a conversational function. According to the proponents of the pragmatic tradition, this suggests that the speakers do not treat the varieties as separate codes, for example they do not “switch" a “code", but treat the bilingual variety as one code. A heavily mixed code can also be juxtaposed with a more purist bilingual register \parencite{meeuwis1998monolectal}, and the shared code can be the mixed one \parencite{alvarez1998codeswitching}. These debates reflect a fundamental difference in the way that researchers of different traditions see the nature of language. The structural-linguistic tradition perceives language as an entity, a coherent inner system that reflects the linguistic competence of an individual, whereas the pragmatic-sociolinguistic tradition considers language to be fundamentally about action, rather a verb than a noun, and that people mainly use their linguistic resources to ``language'' in interaction with other people.
\end{sloppypar}

Many proponents of \textit{translanguaging} \parencite{garcia2009education,creese2010translanguaging, garcia2014translanguaging}, \textit{polylanguaging} \parencite{jorgensen2011polylanguaging} or \textit{metrolingualism} \parencite{otsuji2010metrolingualism} approaches want to distance themselves from the term code-switching, arguing either that CS is not a good descriptor of the language use in the communities that they have studied \parencite{jorgensen2011polylanguaging} or that it is far too marginal a phenomenon to describe the multilayered language use that occurs in the interactions between speakers \parencite{garcia2014translanguaging}. However, this terminology was created partly in reaction to CS, so these studies are part of the same debate. The different perspectives – language as interaction, language as a system – can be combined, though, and I believe that bringing them together might help us to get a better, more comprehensive understanding of multilingual speech phenomena and contact-induced language change. Language is a set of constructions entrenched in the cognitive apparatus of an individual, tendencies and associations that come into being through exchange between speakers in repeated instances of interaction \parencite{beckner2009language, backus2013usage}. Individual speakers use their linguistic resources in a way that makes sense in their specific sociolinguistic circumstances \parencite{heller2007bilingualism}. Without speaker agency and language users’ creative power, innovations would not be possible. At the same time, speakers are guided by community norms, and they recreate or reinforce them in each instance of interaction, both in the case of CS and in monolingual interactions.


\begin{table}[b]
\caption{\label{tab:cs:matrascontinuum}Code-switching – borrowing continuum according to \textcite[111]{matras2009language}}
\label{continuum}
 \begin{tabular} {lcl}
  \lsptoprule
  Prototypical CS & $\leftrightarrow$ & Prototypical borrowing \\
  \midrule
complex utterance & $\leftrightarrow$ & single lexical item \\
used by bilinguals & $\leftrightarrow$ & used by monolinguals \\
one-time occurrence & $\leftrightarrow$ & frequent \\
not integrated & $\leftrightarrow$ & structurally integrated \\
conversational effect & $\leftrightarrow$ & default expression \\
core vocabulary & $\leftrightarrow$ & grammatical operations \\
lexical & $\leftrightarrow$ & para-lexical (unique referent) \\
\lspbottomrule
 \end{tabular}
\end{table}


Another terminological issue that has generated wide debates within the field is the distinction between code-switching and borrowing, as there is no basic consensus when an item has become part of the recipient system. For some scholars, the key to the distinction is morphosyntactic integration \parencite{poplack1998introduction}, for others phonological integration \parencite{halmari1997government}, frequency \parencite{myers1997duelling}, or the degree of entrenchment/conventionalization of the element in the speech patterns of the code-switching individual\slash community \parencite{backus2013usage}. \textcite{matras2009language} suggests that CS and borrowing exist on a continuum (\tabref{tab:cs:matrascontinuum}), and that there are several criteria for their differentiation: on the borrowing end of the continuum are single lexical items uttered by a monolingual speaker, regular occurrences of structurally integrated items that have become default expressions in the community. Prototypical code-switches, in contrast, are used by bilinguals in the form of elaborate utterances for specific conversational effects. They are single occurrences and not integrated into the base or matrix language.


The aim of this chapter is to give an overview of the patterns described in earlier code-switching literature and identify the most important factors that lead to these patterns. The research on CS will also be linked to the other subfields of language contact to examine how these phenomena are intertwined. CS is mostly a synchronic, interactional phenomenon, yet it can be linked to historical patterns and to more permanent contact-induced language change. The examples of code-switching in the chapter are mostly from my own Basque-Spanish conversational data, which was recorded between the years 2005 and 2017 in the Basque Autonomous Community. 

\subsection{Code-switching and contact-induced language change}\label{change}

Contact-induced language change has been seen in terms of long-term effects of language contact, whereas code-switching has been studied from a fundamentally synchronic point of view. It seems reasonable to assume that the individual synchronic interactions set the ultimate stage for language change. Innovations occur in individual interactions, and successful, attractive innovations may get repeated and recycled, which can lead to the conventionalization of these innovations on the community level and, ultimately, to linguistic change. 

There is, however, no consensus on the relationship between code-switching and language change.  Code-switching has even been considered the type of language contact phenomenon in which the systems of the languages stay separate \parencite{gardner2009code}, and which does not bring about more permanent change. \textcite{thomason2001contact} considers CS not to be an important mechanism of structural language change, but an important mechanism of borrowing. \textcite{backus2005codeswitching} argues that the issue is complex and studies would need to take into account all variation in the monolingual variety to establish if it was indeed only CS that led to change. \textcite{cacoullos2018bilingualism} argue that even when code-switching, bilinguals keep the grammars of their languages separate, and that cross-linguistic associations do not equate with cross-language convergence. Permanent structural changes in the recipient system are also the point where it becomes difficult to delimit CS from other language contact phenomena, such as borrowing, loan translations and convergence, so these aspects have been excluded from many CS studies. However, the shift from spontaneous to permanent use can be seen in terms of patterns, individual entrenchment and community-wide conventionalization with various steps in the middle \parencite{backus2013usage}. 

\begin{sloppypar}
The linguistic changes caused by CS occur at several different levels. Firstly, CS at the synchronic level can already be seen as contact-induced language change, as it is change in language use. When two languages come into contact, monolingual norms no longer prevail. The bilingual variety, the CS variety, has vitality in its own right as a vehicle of identity marking. The code-switches are an added resource for the speakers to organize and stylize the speech event. Communication Accommodation Theory, discussed in Chapter \ref{chap_accommodation}, has the convergence of speakers as the object of study and seeks to find out how their speech is altered \parencite{niedzielskietal1996linguistic}. Studies of language contact, in turn, have been interested in the convergence of linguistic systems. CS is both at once: when a community or a single speaker becomes bilingual, the monolingual order is altered both within the speaker and in their linguistic variety. CS can be seen as accommodation on the level of vocabulary and language use patterns to those (speakers) of the other language. \textcite[78]{gardner2009code} considers CS to be one of the ways of accommodating one's speech to the interlocutor’s linguistic preferences. This may be achieved by a switch in the language of interaction or by creating a bilingual style as a compromise strategy. Other accommodation phenomena, such as phonological or morphosyntactic patterns, can be transferred at the same time, leading to partially shared systems at the moment of interaction. If a person grows up bilingual either in a bilingual family or in a bilingual community, the possible convergence between the systems is located inside individual cognition. Bilinguals can find interlingual connections between constructions of both or all of their languages. Accommodation in individual interactions or inside an individual repertoire may become community-wide if various individuals share the same sociolinguistic conditions and linguistic resources. 
\end{sloppypar}

Secondly, CS can function as a strategy of language shift (see Chapter \ref{chap_shift}). In multilingual communities, generational changes in CS patterns often lead toward greater adaptation of the host community's language. \textcite[208--228]{myers1997duelling} suggested that shift happens via a matrix-language turnover. In matrix-language turnover, the CS that begins with insertions from the host-community language into the language of the community of origin gradually leads to mixed constituents and relexification. This is followed by a change of the base or matrix language to the host-community language, along with insertions of material from the language of the community of origin. \textcite{kovacs2001code} modeled the shift as a change from the morphology of the community of origin via bare forms to the morphology of the host-community language, or alternatively, to a new composite matrix. In the Australian Finnish and Australian Hungarian communities she observed, first-generation immigrants overwhelmingly used the morphological matrix of their L1, whereas there was a clear shift away from the morphology of the heritage language in the language use of the second generation immigrants.

In a situation where a group of speakers moves to a society where the numerically and institutionally dominant group speaks another language, a language shift in a few generations seems to be a common direction. However, code-switching does not necessarily lead to language shift. In situations of long-term contact where whole communities speaking different languages come into contact, one can also expect extended periods of relatively stable bilingualism and/or diglossia. Bilingual registers can function as a strategy of language maintenance \parencite{lantto2015code,lantto2016conversations}. The new bilingual varieties have important functions as markers of a community identity that is separate from the identities in monolingual communities \parencites[62]{gumperz1982discourse}[197]{thomason2001contact}[10]{bullock2009cambridge}. Code-switching may also become crystallized in mixed languages and fused lects \parencite{auer1999codeswitching,mcconvell2005gurindji,meakins2011case}. As a result, the language has certainly changed, but parts of the old system are maintained in the fused variety. Code-switching co-occurs with pidginization and creolization (see Chapter \ref{chap_contactlanguages}) across the world and may share features with creolization, such as an analytic approach to vocabulary and grammatical convergence in the case of bilingual compound verbs \parencite[33--35]{gardner2009code}.

\textcite{auer1999codeswitching} suggests that changes in language use at the community level in multilingual communities starts with pragmatically meaningful juxtaposition of the two varieties, which he calls \textit{code-switching}, then develops into \textit{language mixing}, a state of back-and-forth switching in which each switch no longer has a conversational function. This pattern of mixing should then turn into fully or partially established \textit{fused lects} where the participating languages already share resources, and the alternation between material from the participating varieties lacks optionality. Auer argues that the process always proceeds in the same direction, and that shifting back to a state where the languages would re-separate is not possible. However, as \textcite{smith2016regression} shows, patterns may change when sociolinguistic circumstances change, for example due to the language revitalization in regional minority communities, where a step back from language mixing to code-switching can happen, or at least several patterns may co-occur within a community. 

The third type of linguistic change that CS may bring about is \textit{structural change} that concerns specific constructions. \textcite{clyne1967transference,clyne2003dynamics} describes the convergence of closely related languages, which, in bilingual speech production, develop shared structures via trigger words and code-switching. CS most certainly brings about structural convergence at the moment of interaction \parencite{frick2013emergent,riionheimo2014emergence}, yet it is not clear how permanent these changes are. In my own CS data, CS clearly affects the word order of predicative constructions in the Basque-Spanish language contact situation – monolingual Basque constructions usually maintain Basque word order, and bilingual constructions with a Spanish predicative exhibit Spanish word order. Yet Basque has managed to co-exist with Spanish and its antecedents for two thousand years without total convergence of word orders, and the syntactic properties of bilingual constructions have not had a serious effect on the word order of the monolingual constructions.  \textcite{backus2015usage} has proposed approaching CS and other synchronic language contact phenomena from a usage-based perspective focusing on structural change. In his own work, he has discussed the role of multimorphemic, fixed constructions or chunks in the use of the bilingual repertoire \parencite{backus2003units}, while \textcite{hakimov2016plural} focuses on the role of frequency in the solidification and entrenchment of these multilingual chunks. 

CS can also function as a vehicle for lexical change. \textcite{thomason2001contact} considers CS to be an important strategy for borrowing. If we adopt the point of view that CS and borrowing exist on a continuum, the single occurrences of inserted material can be considered spontaneous code-switches. If they are repeated, however, they become more conventionalized and eventually fully adopted to the recipient variety. On a larger scale, the process of borrowing may lead to relexification. Some categories are more prone to CS than others \parencite[133]{matras2009language}. CS of these categories may gradually conventionalize in a way that makes them non-optional, which then leads to the replacement of the previous material. If the categories being replaced are not content words, but grammatical material such as conjunctions, CS can also lead to structural change.

\section{Approaches}

As noted in Chapter \ref{chap_intro}, approaches to CS are diverse. The phenomenon has been examined from a very structural-linguistic to a very sociolinguistic point of view. Studies may focus on the cognitive processes regulating CS, on the language ideologies of the bilingual speakers, on the specifics of bilingual morphosyntax, or, in the recent usage-based approach to CS, even on the intersection of all three. Sociolinguists examining CS tend to be relatively averse to models and rely on detailed descriptions of  language embedded in its local social context. Therefore, most models and methods examined in this section belong to the syntactic tradition of CS research.

\subsection{Models} \label{subsec-models}

The scholars who approach CS theory from a structural-linguistic perspective have been the most eager proponents of models for CS. This modelling often focuses on finding constraints for code-switching \parencite[21]{gullberg2009technique}. Based on her work in the Puerto Rican community of New York, \textcite{poplack1980sometimes} developed two constraints that she claimed to be universal: \textit{the bound morpheme constraint}, which predicts that CS cannot occur between a word and a bound morpheme, and \textit{the equivalence constraint}, which predicts that code-switching can only occur at sites where the syntax of the two participating languages is congruent. These constraints have since been tested in numerous multilingual communities. Despite many counterexamples from different language pairs, especially from language pairs where one of the participating languages has rich agglutinative morphology, Poplack has defended her model, naming many of the counterexamples “nonce borrowings”, i.e. sporadic loanwords, a category different from CS \parencite{poplack1998introduction}. In her \textit{Matrix Language Frame Model}, \textcite{myers1997duelling} describes CS in terms of a matrix language that provides the system morphemes, and an embedded language that provides the vocabulary of the mixed stretches. \textcite{macswan1999minimalist} describes CS within the framework of the Chomskyan Minimalist approach, claiming that all rules for code-switching can be derived from the grammars of the two participating languages, and that mixing of grammars is essentially only a union of two lexicons. \textcite{lopez2020bilingual}, however, calls this view ``separationism'' and argues for a unified, integrated I-language for bilinguals. According to López, this I-language is not substantially different from monolinguals; there are no two lexicons or PFs, but two systems of exteriorization.

\textcite{muysken2000,muysken2013language} sums up the findings in earlier research literature by creating a model for different types of CS. His model predicts the outcome of the language contact in a given sociolinguistic environment according to several factors of the languages and communities, such as linguistic typology, language dominance, language attitudes and linguistic competence.

\begin{sloppypar}
The models described above concentrate on the synchronic state of code-switch\-ing, but some of the proposed models deal with the development of code-switch\-ing patterns and language shift within a community. \textcite{myers1997duelling} proposed the \textit{Matrix Language Turnover Hypothesis}, and \textcite{auer1999codeswitching} predicted that community patterns of multilingual language use follow a certain path. Both of these models were discussed in Section \ref{change} in this chapter.
\end{sloppypar}

\subsection{Methods}
The approaches to CS are diverse, so each sub-branch has used different methods to answer their specific research questions. Scholars advocating for a pragmatic tradition in CS research have used conversation analysis as a tool for identifying the conversational functions of bilingual speech \parencite{auer1988conversation,auer1998code}. These functions will be discussed in Section \ref{patterns}. CS studies that have examined CS syntax from a more generativist (Universal Grammar-based) point of view have attempted to find the rules, or constraints on the points where mixing can and cannot occur in order to find out the structure of the speakers' inner grammar. In this, they have sometimes relied on native speaker judgments, or in this case on the opinions of early “balanced” bilinguals. However, as \textcite[18]{gardner2009code} notes, CS challenges the whole notion of  “native speaker”, as the speakers rewrite the expected rules. She suggests that the study of CS should be approached from “outside the box”, as most of the research methods in linguistics were developed with the monolingual frame in mind. CS could, thus, serve as a way of testing these methods to see if they can be applied in a multilingual context, a perspective also advocated by \textcite{lopez2020bilingual} and \textcite{wyngaerd2021bilingual}. 

Controlled and experimental methods in CS studies used to be relatively rare, even though in some early studies intuition data was used to find syntactic constraints for CS \textcite[22]{gullberg2009technique}. Starting from the latter half of the 2010s, however, more experimental designs such elicitation tasks, acceptability judgment tasks and measuring event related potentials (ERPs) have been used to test the grammaticality of different types of CS and the models and constraints formulated in CS theory (for examples, see \citealt{couto2016gender,couto2017chapter,vaughanetal2020switchmate,bellamyetal2022el}). In another line of experimental studies, psycholinguists have  investigated the processing costs and benefits of CS \parencite{tomicetal2022expecting}.

Most sociolinguistic CS studies have been based on recordings from naturally occurring conversations in bilingual communities, usually conducted by individual researchers. This results in the problem that for several reasons, such as for the privacy of the speakers, competitiveness and fragmented transcriptions, the data is not publicly available for other researchers \parencite[23]{gullberg2009technique}. During the last decade, however, some larger corpora have been made available to other researchers, such as the Welsh-English database \citep{deuchar2014building}. Despite the abundance of CS data collected in numerous projects and language contact settings around the world, the field has not yet established clear standards for data transcription, and so far, no central resources exist for the researchers to share their data \parencite{gardner2009code}. This makes comparisons and the development of common criteria for analysis very hard. Even researchers working on similar questions often interpret their data in different ways, which highlights the limitations of observational techniques \parencite{gullberg2009technique}.

Even though the sociolinguistic tradition has combined descriptions of CS at the community level with extensive ethnographic background knowledge \citep[18]{gardner2009code}, and information about linguistic ideologies and metalinguistic commentary have been collected using ethnographic methods and semi-structured interviews, studies of attitudes and ideologies toward CS are surprisingly rare \parencite{gardner2009code, garrett2010attitudes}. Studies of written CS are also relatively uncommon (see, however, the volume edited by \citealt{enghelsetal2021cruzando}, which includes several chapters on multilingualism in both literary and journalistic texts). Yet, with the amount of communication occurring via written means on the Internet, one should expect more studies such as \textcite{treffers2022explaining} in the future. \textcite{sebba2013multilingualism} proposes a framework to study multilingual texts as multimodal entities. Besides linguistic characteristics, the visual and spatial dimensions of the multilingual texts should also be taken into account.


\section{Patterns} \label{patterns}

In this section, I will describe the patterns that have been found in earlier studies of CS, starting from the conversational patterns that reflect the pragmatic functions of CS. After that, I will briefly go through the patterns of what kind of linguistic material is generally subject to CS, and then focus on the patterns that received the major part of attention in CS studies, namely those that are structural-linguistic in nature. To conclude, I will describe the patterns found in multilingual speech phenomena, many of which are closely linked to CS and exist on the continuum with CS patterns, yet are not fully covered by the term CS and the CS literature. 

\subsection{Conversational patterns} 
\textcite{milroy2003sociolinguistics} distinguish between CS based on the indexical value of the varieties, and CS that is based on exploiting the contrast between the codes for pragmatic functions in a conversation. The same distinction was described by \textcite{blom1972directions} as metaphorical vs. situational switching. In metaphorical switching, the variety changes according to the social domain under discussion, which seems to bring about CS patterns of certain types of lexical material with a common denominator. Cross-culturally, common domain-related categories that are subject to CS and borrowing include numerals, taboo words, colloquialisms and other fashionable items, and cultural concepts related to a certain domain, such as agriculture or technology.

The pragmatic tradition of code-switching research has focused on situational switching. In these conversational patterns (pragmatic functions), the contrast of two languages or varieties is used as a conversational resource to provide \textit{contextualization cues} \parencite{gumperz1982discourse} for the other people present in the interaction. CS often marks sequential contrast, shifts in \textit{footing} \parencite{goffman1981forms} or alignment in a conversation, such as openings and closings.  When code-switching is used for opening a turn, it functions as an attention-getter. When finishing a turn, code-switching – often with expressions that signal ending, such as \textit{that’s it} – is used to indicate that the speaker has said all that they meant to say. Speakers may also use CS to signal the difference between reported speech and the general narrative frame \parencite{alfonzetti1998conversational}, or to distance general comments from personal opinions and side remarks \parencite{gumperz1982discourse}. Example (\ref{reported}) shows how the speaker switches from Basque to Spanish in his narrative to report a conversation. The original conversation occurred in German, and his interlocutor even speaks German. The function of CS here is not to preserve the original language, but to mark the change in footing by creating voices in the narrative.

\begin{exe}
\ex\label{reported} Eta berak, bera hasten da irakurtzen justo itzuli diot nire amari zer esaten zuen, ba bueno, gutxi gora-behera eta gero tipoa hasten da alemanez, \textit{joder, y que soy de Offenburg, cerca de Freiburg, Baden-Württemberg.  Y digo a, pues, mira los de Baden-Württemberg son mucho más majos que los de Baviera y dice sí, pero es verdad, claro, que vas a decirle  tú} eta egon gara hizketan.
\glt `And him, he starts reading I just translated to my mother what he said, er well, almost and then the guy starts in German \textit{shit, and I am from Offenburg, that’s close to Freiburg, Baden-Württemberg. And I say, oh, well, see those from Baden-Württemberg are a lot nicer than those from Bavaria and he says yes, but that’s true, sure, what are you going to say} and we were talking.'
\end{exe}


CS is used for interjections, reiterations of what has been said, and for reformulations of the message. CS may be used to topicalize and highlight an element. It can also be used for humor and bonding, and to add expressiveness and language play to the discussion \parencite[85]{gumperz1982discourse,auer1988conversation,gardner2009code}.

\subsection{Patterns in hierarchy of code-switchability} Scholars of code-switching have found that some categories are more affected by CS than others. \textcite[133]{matras2009language} summarizes several earlier studies in relation to the hierarchy of “code-switchability”, based on the relative frequency of categories affected by CS. The internal order varies slightly from one contact setting to another, but nouns and noun phrases come on top of all the hierarchies. Pronouns figure low on the hierarchies, whereas the place of verbs, adverbs and conjunctions varies considerably. In the Basque-Spanish case that I am most familiar with, the hierarchy would be starting with the elements most likely to be code-switched and ending with the elements least likely to be affected by CS:

\begin{quote}
discourse markers > fixed expressions > bare nouns > noun phrases > \mbox{adverbs} > conjunctions > adjectives > verbs > case markers/prepositions.
\end{quote}

\subsection{Structural-linguistic patterns}
When CS patterns are mentioned, structural patterns are those that are most likely to come to mind, as they have been most thoroughly studied. A basic classification, already used by \textcite{poplack1980sometimes}, is to divide the occurrences of CS into \textit{intersentential, extrasentential}, and \textit{intrasentential} types of CS. Intersentential switches are those that occur between sentences: one sentence is uttered in one language, the next one in another language. Extrasentential switching, also called \textit{tag-switching} or \textit{emblematic switching}, is switching that, apart from the established morphosyntax of fixed expressions, does not involve syntactic structures of the participating languages or varieties. Discourse markers, tags, interjections, etc. are examples of extrasentential switching. The most studied grammatical CS patterns are the intrasentential patterns. Intersentential and extrasentential switching have been considered less informative, as the language systems in theses types of switching are not intertwined. Scholars have formulated rules and constraints, such as the bound morpheme constraint discussed in Section \ref{subsec-models}, that are thought to govern CS patterns and to reflect the speaker's inner grammars. 


The most extensive work examining different types of intrasentential CS patterns has been authored by \textcite{muysken2000,muysken2013language}, who developed a model based on the results of earlier CS studies. In \textit{alternational} code-switching, the language systems stay separate. First one is used, then the other, as in the French-Russian case in example \ref{alternation}). In \textit{insertional} code-switching, one of the languages functions as a matrix into which elements of the other language are then inserted. This is shown in the Quechua-Spanish example (\ref{insertion}), which presents a nested structure, in which the words preceding and following the insertions are morphologically linked. \textit{Congruent lexicalization} indicates a high level of convergence between the systems. It can involve both insertions and alternations, and a base language is hard to define. The example of Spanish-English CS (\ref{congruent}) shows a high degree of linear equivalence created at the moment of interaction.  All the examples below are from \textcite{muysken2000}.

\begin{exe}
\ex\label{alternation}
Les femmes et le vin,  \textit{ne ponimayu}.
\glt `The women and the wine, \textit{I don't understand}.'

\ex \label{insertion}
\gll {Chay-ta} \textit{las} \textit{dos} \textit{de} \textit{la} \textit{noche}-ta chaya-mu-yku\\
 That-\textsc{acc} \textsc{art} \textsc{two} \textsc{prep} \textsc{art} night-\textsc{acc} arrive-\textsc{trans}-\textsc{1pl}\\
\trans `There at two o'clock at night we arrived.'


\ex \label{congruent}
Anyway, \textit{yo creo que las personas} who support \textit{todos estos grupos como los} Friends of the Earth \textit{son personas que} are very close to nature.\\
\glt `Anyway, \textit{I think that the people} who support \textit{all these groups like the} Friends of the Earth \textit{are people who} are very close to nature.'
\end{exe}

To these patterns that have been used in CS research for the last almost 25 years, Muysken added the pattern of \textit{backflagging}, in which speakers use elements such as discourse markers from their L1 in their L2. Example (\ref{backflagging}) is from \textcite{muysken2013language} and shows how heritage language (Moroccan Arabic) discourse markers are inserted in L2 (Dutch) discourse. 

\begin{exe}
\ex \label{backflagging}
Ik ben doctor \textit{wella} ik ben ingenieur.
\glt `I am doctor \textit{or} I am engineer.'
\end{exe}

In his \citeyear{muysken2013language} article, Muysken connects these subtypes of CS to other forms of language contact, rephrasing insertion as the outcome of language contact where the grammatical and lexical properties of the L1 function as the matrix language. Congruent lexicalization describes structures and words that share properties of both languages. Alternation is about universal combinatory principles independent of the grammars involved, whereas in backflagging, the grammatical and lexical properties of L2 function as the base.

\subsection{Multilingual practice patterns}
Major terminological debates in the field of CS were briefly discussed in the introduction. In certain types of contact situations, such as relatively stable multilingual communities involving speakers that are fluent in both languages, \textit{code-switching} might be the most adequate term for what speakers do with their languages. The bilinguals in these communities may use the juxtaposition of the material of two different systems as a conversational resource, and/or they may use the bilingual variety as an identity marker. This is the case, for example, in Basque-Spanish CS, where these CS functions co-occur. Yet the creation of new terms for different types of multilingual language use is usually motivated by the feeling that the existing terminology is insufficient to do justice to the type of multilingual practices found in the data. The preferred terminology seems to reflect the differences in the language contact situation and the data examined in each case. New definitions also bring to the surface differences in the ways that various scholars of multilingualism perceive language.

\subsection{Nonce-borrowing or insertional patterns}
Insertional patterns of CS differ from borrowing only in terms of entrenchment and frequency. They are observed in the speech off all bilinguals, both those that do not have a high competence of both languages and those who do. They are common in situations of diglossia and unequal power relations between the languages of the society. Bare nouns are easily integrated into existing constructions of the recipient language. Rich inflectional morphology also seems to favor insertional patterns, as there are more possibilities for nested or embedded structures. \textcite{poplack1998introduction} have called these patterns “nonce borrowing”, a process different from code-switching. These are patterns that exist somewhere on the CS – borrowing continuum.

Researchers of language use in multiethnic youth groups have also rejected the term code-switching. They want to emphasize language use instead of linguistic systems; interaction and social indexicalities of the linguistic resources instead of language structure and boundaries. \textit{Polylinguistic languaging} \parencite{jorgensen2011polylanguaging} seems to exploit social indexicalities of the linguistic resources associated with particular varieties in a way that is not present in situations of stable language contact. In this formulation, language use is highly innovative and linguistic resources of all types – syntactic, morphological, phonetic – are employed to create group language and to distance members from out-group speakers. This type of languaging has been most thoroughly researched in relation to multiethnic youth groups and globalization \parencite{schoonen2005street,lehtonen2015tyylitellen}. Nevertheless, similar examples can be found, for example, in the old Spanish vernacular in the city of Bilbao. The vernacular shows several iconic Basque features, consciously adapted by its speakers to highlight the authentic Bilbao identity of its residents of Basque origin in a situation where large waves of immigrants from other Spanish provinces arrived to the city due to industrialization in the mid-19th century \parencite{lantto20165}.   

\subsection{Translanguaging} 
Translanguaging, a translation of the Welsh term \textit{trawsieithu}, originally coined by Cen Williams and based on the context of Welsh-English bilingual education in the 1980s, has been adopted by some scholars to address the porosity of language boundaries \parencite{garcia2009education,creese2010translanguaging,garcia2014translanguaging} 
Translanguaging proponents generally address conversational functions of language use in immigrant and multilingual communities. Their focus is on the interaction and  trespassing of language boundaries to ensure effective communication. Regularities or tendencies in the use of linguistic matter, or patterns, are rarely addressed. The research on translanguaging has been popular among scholars who have examined multilingual language use in classroom contexts.


\section{Factors}

In this section I will examine what kind of syntactic, lexical and se\-man\-tic-prag\-mat\-ic patterns one can expect in a specific interaction embedded in a specific social context according to earlier CS literature. First I will examine the intralinguistic factors that affect the type or amount of code-switching, then I will move on to the extralinguistic factors that have been examined (although these are not always separate from one another).

\subsection{Linguistic factors} 

\subsubsection{Typological distance}


The typology of the participating languages has probably been the most studied factor in the CS literature. In the beginning, the focus of CS studies was on syntactic competence, deep structures and constraints for CS due to the generativist tradition in which many of the early researchers were trained. What was found is that typological dissimilarity between the languages involved seems to favor alternational patterns, as the structures do not lend themselves to be easily intertwined \parencite{muysken2000,muysken2013language}. Typological dissimilarity may also favor insertional patterns, if one of the languages has rich inflectional morphology. In these cases, insertions are easily formed, as the elements of the donor language are nested within the matrix structure of the recipient language, and the morphosyntactic relationship is asymmetrical. Mixed languages and varieties (see Chapter \ref{chap_contactlanguages}) typically show insertional patterns in which certain syntactic categories, most typically verbs and nouns, can be drawn from one of the languages and integrated into the other via inflectional morphology. However, the patterns that mixed languages tend to show are more regular and consistent than those usually found in insertional CS \parencite[290]{matras2009language}; see also the discussion on contact languages in Chapter \ref{chap_contactlanguages}. 

In situations of typological similarity, for example between closely related languages, patterns of congruent lexicalization tend to come about, as many of the linguistic resources are already shared. Shared resources and structures lead to fused lects. Closely related languages are prone to accommodation Chapter \ref{chap_accommodation} and convergence at the moment of interaction \parencite{muysken2000,muysken2013language,clyne2003dynamics}.

\subsubsection{Processing and activation}

Multilingual speakers are always situated at a certain point on a continuum between a monolingual and bilingual mode of language production depending on the degree of activation of the languages in their repertoire. The point at the continuum is determined by mostly extralinguistic factors such as interlocutors, topic, and the physical space of the conversation. The more bilingual their mode, the more they switch \parencite{grosjean1997processing}. In bilingual mode, the elements that are easily accessed and processed are most susceptible to switching. Discourse markers often become part of a mixed variety or are borrowed entirely, because they are treated as gesture-like devices \parencite[193]{matras2009language}. Multimorphemic chunks are easily transferred from one language to another \parencite{backus2003units}. They are processed as a whole and, therefore, less processing effort is needed. The multimorphemic chunks can be switched as interjections and tags in tag-switching, backflagging, and alternational patterns, but also as noun phrases in insertional patterns. 

\textcite{matras2011explaining} suggests that the systems of different languages can merge in the minds of bilinguals for reasons of economy. The elements that are stored closely are easily accessible, and cognates have been shown to trigger CS \parencite{clyne1967transference}. Words and expressions related to the other-language culture, such as names and concepts, may function as triggers. Sometimes, common words that seem easily translatable on a surface level do not carry all the connotations of their near-equivalents, and are therefore easily code-switched \parencites{backus2001role}[112]{matras2009language}. Psycholinguistic processing studies on CS have shown that it is harder for bilinguals to inhibit their L1 than their L2, yet switching back from a non-dominant language is harder than vice versa (\citealt[141]{gardner2009code}, see also \citealt{tomicetal2022expecting}).

\subsubsection{Conversational level}

There are also factors at the level of individual interactions that may favor CS. If the speakers use codes with juxtapositional value as conversational resources, this may lead to the emergence of pragmatic patterns and to regularities in the way the varieties are used as contextualization cues (as discussed in Section \ref{patterns}). Similar contrasts may also take place between a mixed, bilingual variety and the purist register \parencite{alvarez1998codeswitching,meeuwis1998monolectal}.

The interlocutors' language use and linguistic background are very important factors in the amount of CS that the bilingual uses in conversation. Addressee specification is a common pragmatic function of CS \parencite{gumperz1982discourse}. Accommodation (see Chapter \ref{chap_accommodation}) to the speech of the interlocutor may function in both directions, both encouraging or discouraging CS. Code-switching can function as a bridge between the varieties \parencite[78]{gardner2009code}. The speaker’s and the interlocutor’s degree of linguistic authority determines whose innovations are code-switched, noticed and recycled. 

\subsubsection{Attractiveness}

There are both semantic and structural motivations for the attractiveness of an element \parencite{johanson2002contact}. Content words are borrowed much more easily than function words \parencite{backus2013usage}, and extrasentential material is easy to process and introduce in bilingual conversations. Semantic specificity is a factor for CS, as in many cases there is no direct equivalent for a concept in the recipient-language culture \parencite{backus2001role}. Nouns are often labels for unique referents, whereas pronouns are not very prone to switching, as there is no real semantic motivation to switch them \parencite[133]{matras2009language} (though see \citealt{treffers2022explaining} for counter-evidence). Salience and markedness of the code-switched element may both encourage and discourage CS depending on the speaker’s personality. The semantic motivations for attractiveness are clearly connected to the sociolinguistic circumstances. For example, numerals are easily code-switched if the actions of counting are usually performed in contexts like business, trade and education where only one of the languages is used. Cultural concepts related to the introduction of nascent fields with new vocabulary pertain to semantically specific categories. Colloquialisms and fashionable terms are easily borrowed and code-switched, as are taboo words. Throughout history, speakers have borrowed concepts from each other related to agriculture and technology when specific practices were introduced. For example, medical doctors still use terminology based on Latin, which used to be the common language for medical studies and CS scholars often use English concepts even when discussing the phenomenon in their native languages, since these concepts might not be readily available or be in common use.

\subsection{Extralinguistic factors}

\textit{Linguistic power relations} seem to have a direct effect on the CS patterns and on the directionality of switching. The speakers of institutionally less dominant languages are often bilingual, whereas the institutionally powerful stay monolingual. Under these circumstances, the socially less dominant variety may become the “bilingual” variety, as all of its speakers can use the resources of both languages without problems in communication. At the same time, the monolingual speakers of the majority language do not easily tolerate switches to the minority language \parencite{matras2009language}. Insertions occur mostly from the dominant variety to the less dominant variety \parencite{muysken2000}. Also, alternations to the dominant variety may occur starting from the less dominant base \parencite[14]{gardner2009code}. In the Basque case, for example, \textcite{aurrekoetxea2011perpaus} observed that even though both the main clause and the subordinate clause could be uttered in both Basque and Spanish, the order was always to start in Basque, and end in Spanish.  When most of the processing effort is focused on the beginning of the utterance, the tension may be then released in order to switch to the pragmatically dominant language \parencite{matras2009language}.

In stable sociolinguistic situations of relatively equal power distribution, alternational patterns are favored. Also fused lects may emerge, if the languages are typologically similar. Long-term contacts may lead to shared constructions and linguistic convergence, which, in turn, might lead to increased equivalence, congruent lexicalization and fusion \parencite{auer1999codeswitching,muysken2000,muysken2013language}. 

\textit{Bilingual proficiency} has been noted to be a factor affecting the type of CS produced by bilingual speakers. Both \textcite{poplack1980sometimes} and \textcite{nortier1990dutch} argued that complex back-and-forth CS requires high bilingual competence. According to \textcite{muysken2000}, high proficiency leads to intensive CS of both alternational and congruent lexicalization types. However, proficiency as a factor should be interpreted in the context of community-related questions such as the underlying language ideologies, or who has the \textit{linguistic authority} in a community. Linguistic authority is often granted to the most integrated speakers of a variety, who are seen as the rightful owners of a language. Often these are the ``native speakers'' \parencite{doerr2009native}.  Even though research on CS fundamentally questions the native speaker ideology by focusing on questions of bilingualism and on bilingual individuals, these ideologies often seem to be reproduced in the belief of the superiority of a balanced bilingual who has a similar native-like competence in both or all varieties in which the switching takes place. \textcite[227]{muysken2000} considers the issue of bilingual proficiency to be closely related to network membership or to a generational membership in a migrant community. The social constraints placed upon various types of speakers are different, and those whose language competence goes unquestioned are more free to move between the subsystems in their linguistic repertoire.  Both \textcite{smith2016regression} and \textcite{lantto2018new}, for example, have found very similar differences in CS patterns – intensive intrasentential switching for native speakers, mostly extrasentential for non-native speakers – in communities that are undergoing the process of revitalizing a minority language. The non-native speakers in these cases are subject to more purist constraints, both due to language acquisition in a purist classroom environment and to their limited linguistic authority. 

\textit{Purist ideologies} are also attested in situations of political competition between languages, which may lead to alternational patterns of language separation \parencite{muysken2013language}. The contrast between languages leads to the use of CS as a conversational resource \parencite{poplack1988contrasting,auer1999codeswitching} instead of more morphosyntactically intrusive forms of switching. Non-purist attitudes can lead to intrusive types of mixing \parencite{poplack1988contrasting}, such as language mixing, and eventually even to fused lects \parencite{auer1999codeswitching}. Intrusive CS, for example in patterns of convergence and congruent lexicalization, may be attested in closely-knit communities with relaxed linguistic norms \parencite{muysken2013language}. The need to keep the languages separate might be particularly strong in minority language settings where purity is seen as essential for language survival \parencite{woolard1998introduction,jaffe2007discourses}. The rejection of overt CS and borrowing may lead to more covert patterns of contact-induced language change, such as structural convergence \parencite[267]{aikhenvald2002language}.

All types of linguistic change are often seen as a decay, and mixed forms may be considered particularly decadent \parencite{woolard1998introduction}. Even though reactions to CS in communities are often purist in nature, these attitudes might be learned rather than spontaneous \parencite[81--82]{gardner2009code}. Code-switched varieties may also be seen as the most authentic, natural, unmonitored reflection of a bilingual community \parencite{lantto2016conversations} and awarded covert prestige. Covert prestige, in contrast to overt prestige, is the value attached to non-standard varieties as markers of solidarity and group identity. Covert prestige might affect the speaker's actual language use more than the overt prestige attached to standard varieties. Nevertheless, if a purist register or a variety is perceived as a carrier and transmitter of authentic, traditional community values, the quest for authenticity can similarly lead to the \textit{avoidance} of CS. The existence and presence of a monolingual standard variety in both languages helps to reinforce the separation of the languages. Institutions can create new vocabulary in the minority language, which would make CS and borrowing less necessary. Yet it is up to the speakers to accept these innovations or to prefer the borrowed vocabulary.

\begin{sloppypar}
The relationship between code-switching patterns and \textit{age/generation} has been the subject of several studies. Adolescents have been reported to engage in CS, and then “grow out of it” when they become adults \parencite[22]{muysken2000}. Most of the generation-related patterns have been observed in immigration contexts, often with an emphasis on language shift \parencites[227]{muysken2000}{kovacs2001code}. In the most stereotypical manner, within immigrant communities, the first generation, with the lowest proficiency in the host community language, would use insertional patterns to introduce nouns and noun phrases from the language of the host community \parencite{muysken2013language}, whereas the middle and second generation would favor alternational patterns and congruent lexicalization \parencite{muysken2000}. Middle and second generation immigrants might also develop emblematic (identity marker) patterns, combined with possible heritage language loss, in communication with the members of the out-group. This is the type of CS that \textcite{muysken2013language} calls \textit{backflagging}, and closely resembles the  \textit{polylingual languaging} of multiethnic youth groups \parencite{jorgensen2011polylanguaging}. These multilingual varieties are marked by specific linguistic elements and features instead of longer stretches of language alternation. Multiethnic youth groups are a source of innovations, which might then spread even among the monolingual majority. To name an example, the use of \textit{wallah} from Arabic, `I swear, I promise', with its loan translations as an emblematic identity marker among youth groups of Muslim origin, has now extended to other speaker groups in several countries of Northern Europe. This, in turn, seems to have provoked changes in the use of structures “I swear" and “I promise", in the youth speech of these communities. \parencite{kallmeyer2003linguistic,schoonen2005street,svendsen2008multiethnolectal,lehtonen2015tyylitellen}.
\end{sloppypar}

Another extralinguistic factor that can be directly linked to CS are the \textit{social domains} associated with each language. Vocabulary related to certain domains that are linked with a particular language, such as education and work, can lead to lexical pattern in CS if these domains become the topic of the conversation. The effect of such variables as \textit{gender} is not straightforward. \textcite[83]{gardner2009code} found no significant differences in the amount of CS used by men and women in Greek Cypriot and Punjabi communities in the UK. In \textcite{poplack1980sometimes},  women in the Puerto Rican community in New York favored intrasentential switching more than men, whereas in the Shipibo community in Lima, men use more Spanish and CS to Spanish than women, who are seen as the guardians of ethnic identity \parencite{zavala2008enra}. All in all, the effect of the different extralinguistic variables seems to be very much linked to the particularities of a given context of language contact. When a component in the sociolinguistic situation as a whole changes, the outcomes and the types of CS may change as well.

\section{Discussion and outlook}
In the previous sections, I have summarized the main findings of earlier CS studies with regard to the models proposed, their methods, patterns in the outcomes and some of the factors leading to these patterns. Due to the huge number of studies on CS, much has been left out of the analysis. Nevertheless, I hope to have succeeded in giving the reader a general overview of the field.  In the following, I will describe the challenges that CS studies need to respond to, and the future directions that I see as fundamental for their development in the future.

\subsection{Comprehensive frameworks} \label{subsubsec-comprehensive}
Firstly, what CS studies need in the future are models and theories that attempt to draw a more complete picture of all sides of CS, and to describe the phenomenon as a whole. 
\textcite{tomicetal2022expecting} highlight the need to integrate sociolinguistic and corpus-based observations to the experimental studies on CS, and \textcite{coutoetal2021code-switching} advocate multi-method, comparative approaches to observe the patterns that emerge across communities.
Muysken’s model \parencite{muysken2000,muysken2013language}, discussed in several sections throughout this chapter, is the most ambitious attempt of description (and prediction) of CS thus far. Yet the model is grammar-oriented and barely discusses the impact of conversational or stylistic CS patterns. Muysken does discuss attitudes towards CS, but the importance of language attitudes and ideologies has been neglected in most of the structural CS research. This is one of the lines of research that the CS scholars should focus on in the future. As for now, purism is sometimes mentioned as a factor that limits the speakers’ use of their linguistic repertoire as a whole, but the discussion rarely goes beyond noting that these purist tendencies exist. \textcite{backus2013usage, backus2015usage} has advocated for the creation of a functioning theoretical framework for CS based on the usage-based approaches to language. Yet in these cognitive frameworks, the intralinguistic aspects of CS are often discussed separately from the sociolinguistic make-up of the code-switching individuals and communities. A comprehensive framework should acknowledge all levels – the structural-linguistic, pragmatic, and psycholinguistic aspects of the elements that are being code-switched, the individual and the community with their ideologies – that are present in the interactions where CS occurs.

For instance, as an example of a multi-sided description of a phenomenon, let us consider the case of discourse markers in the contact situation that I know best, the contact between Basque and Spanish in the city of Bilbao, Basque Country. Code-switched discourse markers have been noted to function as attention-getters in bilingual conversations, so discourse markers have a pragmatic value as contextualization cues \parencite{maschler1997discourse}. In the example below, a non-native Basque speaker is using Spanish discourse markers in her speech in Basque, even though she otherwise tends to avoid CS.

\begin{exe}
\ex\label{nonnative}
Eta bestea, Ana, Ana zegoen oso oso erreta eta oso oso txarto: \textit{pues es que} hau ez da ona…Eta \textit{joder}, helduak gara, \textit{o sea}.
\glt `And the other one, Ana, Ana was so so frustrated and and so so unwell: \textit{well it's that} this is not right… And \textit{shit}, we are adults, \textit{I mean}.'
\end{exe}

\noindent In Example (\ref{nonnative}), the speaker is looking back on a course she and her interlocutor attended. One of the students was very frustrated with the teacher of the class, and was always protesting. The speaker goes on to quote the student’s words: she uses the Spanish discourse markers \textit{pues es que} emphatically as a contextualization cue to mark the transition from the narrative frame to the words of the quoted person and to highlight her militant attitude. Then she changes back to the narrative frame to disapprove of the behavior of the student with the mild Spanish swear word \textit{joder}, which also functions as a discourse marker.

Discourse markers are generally extrasentential switches that do not violate the grammatical norms of the languages involved, so they are accessible even for those individuals with a limited bilingual proficiency or for those who are subjected to purist social constraints. In the Basque variety spoken in the area of Bilbao, discourse markers can function as markers of informal authenticity even for the nonnative Basque speakers who do not have the linguistic authority to move freely within their full linguistic repertoire \parencite{lantto2018new}. Discourse markers from the surrounding majority language are easily conventionalized as non-optional in bilingual varieties and speech styles \parencite{goss2000evolution,matras2009language}. They are processed as gesture-like devices, which are prone to selection malfunctions \parencite{matras2009language}. Even though they would have originated as two different words, such as the Spanish discourse markers \textit{o sea} (conjunction + verbal form), \textit{a ver} (preposition + infinitive), and \textit{en plan} (preposition + noun), they are processed as one multimorphemic item and can be transferred and code-switched as a whole. Just like in example (\ref{nonnative}), the direction of the switching is generally from the majority language (here Spanish) to the minority language (here Basque) \parencite{matras2009language}. The majority language speakers are often monolingual and do not easily tolerate switches to the minority language \parencite{matras2009language}. Processes of fusion often start with this type of relatively unbound element of grammar \parencite{auer1999codeswitching}.

The conventionalization of discourse markers from the majority language in the informal style of a minority language is, therefore, a case of multiple causation, where the conversational functions of CS, the language ideologies of the moment, linguistic typology, processability of the items and linguistic power relations come together in this specific instance of interaction. Even age-specific patterns can be found in the use of discourse markers by the Basque speakers – all of them use discourse markers of Spanish origin, yet the older generations do not use discourse markers such as \textit{en plan}, widely considered a feature of youth speech.

\subsection{From synchrony to diachrony} 

Another challenge for CS studies is to further explore the relationship between CS and structural change. The studies mentioned in Section \ref{change} describe linguistic change in small structural details at the synchronic level, but long-term research within and across communities is needed.  The question is far from resolved, and some new studies, such as \textcite{cacoullos2018bilingualism}, show that even widespread CS in a community does not straightforwardly lead to grammatical convergence. In general, the research field of CS has not yet managed to combine synchronic and diachronic views on CS in a very convincing manner. Perhaps because of its late start, the field has also not yet been able to document how language change may start at the conversational level in individual interactions and then become permanent patterns at the community level, even though \textcite{matras2009language} has theorized about this path. As noted in Section \ref{change},  CS is basically accommodation and convergence of the linguistic subsystems in interaction. All potential linguistic changes initially occur in an individual's cognition, in the rearrangement of the linguistic repertoires of the individual speakers. Bilingual constructions created in individual interactions may gradually become entrenched in the idiolect of one or several speakers. The speakers use their linguistic resources to fit the sociolinguistic contexts \parencite{heller2007bilingualism}, and when these speakers share the same sociolinguistic contexts, the patterns also become similar. If similar changes happen in several idiolects, they may lead to community-wide conventionalization of new patterns and features, starting with small groups. Over time, these community patterns may become non-optional, and the original material may be replaced by new items. Yet long-term studies and extensive corpora are needed to document development, and to situate code-switching in the context of the other phenomena that occur simultaneously in situations of language contact.

The classic examples of lexical borrowing are loanword layers, concepts related to new forms of culture. Entire categories can be code-switched: for example, the older generations in the Basque Country tend to use numerals in Spanish, as it was their language of schooling \parencite{lantto2015conventionalized}. This pattern, however, has changed as the young people have access to Basque-medium education. Yet without the sociolinguistic change and revitalization, Basque numerals could have been replaced by their Spanish equivalents. Long-term language contact and extensive borrowing may lead to heavy relexification, such as in the case of English, whereas short-term contact more probably leads to language shift over a few generations. The features shared and created in polylingual languaging may lead to innovations and changes of single features of languages and varieties.

CS might also be related to structural changes, for example in the case of conjunctions. Conjunctions are prone to selection errors \parencite{matras2009language}. They can become entrenched in a person’s idiolect, and later on become conventionalized at the community level. Code-switching conjunctions may bring about far-reaching structural change – for example in Basque the subordinate causative clauses have been traditionally formed adding a causative suffix \textit{-(e)lako} to the finite verb or by adding causative particles to the end of the sentence. In Example \ref{causative}, however, the speaker uses the Spanish causative conjunction \textit{porque} instead of the Basque causative suffix or particles.

\begin{exe}
\ex\label{causative}
\gll {Bizarra} {moztuko} {dut} {orain} \textit{porque} {bihar...} {bihar} {ez} {dut} {gogorik} {izango}\\
beard.\textsc{det} cut.\textsc{fut} \textsc{aux.tr.1sg} now because tomorrow... tomorrow \textsc{neg} \textsc{aux.tr.1sg} want.\textsc{part} be.\textsc{fut}\\
\trans `I’ll shave my beard now, \textit{because} tomorrow I won’t feel like it.'
\end{exe}

Changing the system of forming subordinate clauses is a significant structural change that affects the very core of Basque grammar. The Spanish conjunction \textit{porque} is not entrenched in all Basque speakers' idiolects,  but it is widespread enough in a way that it is no longer a curiosity, a single occurrence. The same process is happening with other subordinate clauses in Basque, as the relativizer suffix \textit{-(e)na} and the complementizer \textit{-(e)la} are often replaced with the Spanish conjunction \textit{que} in bilingual speech. The development is comparable to the large-scale borrowing of conjunctions that has happened in the Finno-Ugric languages throughout the last millennia: Finno-Ugric languages had mostly nonfinite ways of expressing temporal, causal, etc. relations before the conjunctions were borrowed from Russian and the Germanic languages \parencite{hickey2010handbook}. In some of the languages, such as Finnish and Estonian, the process seems to have been completed, while in others it is still ongoing. The parallel mechanisms working in the synchronic Basque case are visible. \textcite[62]{thomason2001contact} uses the syntax of subordination as a common example of a snowball effect, where the initial change may trigger several other grammatical changes on the way. 

How can one link CS, a spontaneous synchronic phenomenon, back to historical language contact? What connects the modern day innovations and modern day bilingual speech to the historical processes of language contact? I think it is relatively uncontroversial to say that human nature and human cognition have not changed much during the last centuries or millennia, and that the same mechanisms of code-switching and borrowing that were functioning in the past are still functioning in the current situations of language contact. Even though CS is often discussed in relation to globalization, it was a phenomenon that occurred – and was complained about! – also in the multilingual societies of the past \parencite{lantto20165}. The historical written documents that contain CS demonstrate that CS was used with similar functions and in similar ways as it is used now. The particularities of the features involved in historical change, even across macro-areas, can tell us a tale about the organization of past societies, their hierarchies, ideologies and patterns of language use. 


Both long-term studies and studies with wide bilingual corpora would be nee\-ded to truly connect CS to patterns of linguistic change. \textcite{cacoullos2018bilingualism} is one of the first real attempts to track contact-induced grammatical change in a carefully constructed bilingual corpus of the long-term language contact situation in New Mexico, where Spanish has been spoken for over 400 years and English for over 150 years. However, they find very little evidence that change in grammatical systems has actually occurred. As for the near future in CS studies, I predict wider use of quantitative methods to address patterns in large bilingual corpora, as the monolingual norms of corpus building are finally being broken. However, many existing corpora are not open and lack of accessibility continues to present problems for the researchers interested in more quantitative methods. Another new dimension might be a growing interest toward written CS in social media, text messages and in other corners of the Internet, a very common and multi-faceted phenomenon that has received surprisingly little attention thus far.

%\section*{Abbreviations}
%\section*{Acknowledgements}

\printbibliography[heading=subbibliography,notkeyword=this]

\end{document}
