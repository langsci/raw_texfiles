\title{Language contact}
\subtitle{Bridging the gap between individual interactions and areal patterns}
\BackBody{Contact linguistics is the overarching term for a highly diversified field with branches that connect to such widely divergent areas as historical linguistics, typology, sociolinguistics, psycholinguistics, and grammatical theory. Because of this diversification, there is a risk of fragmentation and lack of interaction between the different subbranches of contact linguistics. Nevertheless, the different approaches share the general goal of accounting for the results of interacting linguistic systems. This common goal opens up possibilities for active communication, cooperation, and coordination between the different branches of contact linguistics. This book, therefore, explores the extent to which contact linguistics can be viewed as a coherent field, and whether the advances achieved in a particular subfield can be translated to others. In this way our aim is to encourage a boundary-free discussion between different types of specialists of contact linguistics, and to stimulate cross-pollination between them.}

\author{Rik van Gijn and Hanna Ruch and Max Wahlström and Anja Hasse} 

\renewcommand{\lsSeries}{lv} 
\renewcommand{\lsSeriesNumber}{8}
\renewcommand{\lsID}{279}
 
\renewcommand{\lsISBNdigital}{978-3-96110-420-8}
\renewcommand{\lsISBNhardcover}{978-3-98554-078-5}
\BookDOI{10.5281/zenodo.8269092}
\typesetter{Max Wahlström, Felix Kopecky, Sebastian Nordhoff, Hannah Schleupner}
\proofreader{Amir Ghorbanpour,
Andreas Hölzl,
Aviva Shimelman,
Christopher Straughn,
Carmen Jany,
Elliott Pearl,
Fahad Almalki,
George Walkden,
Georgios Vardakis,
Hella Olbertz,
Janina Rado,
Ksenia Shagal,
Lachlan Mackenzie,
Ludger Paschen,
Maria Zielenbach,
Marijana Janjic,
Rebecca Madlener,
Tom Bossuyt,
Yvonne Treis
}
\lsCoverTitleSizes{50pt}{16mm}% Font setting for the title page
