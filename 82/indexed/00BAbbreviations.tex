\addchap{Abbreviations}
\begin{refsection}

\section*{General abbreviations}

\begin{description}[leftmargin=!, font=\normalfont, itemsep=0pt, labelwidth=\widthof{PBUH}]
\item[A]
Ashret (or the dialect of Ashret Valley)
\item[B]
Biori (or the dialect of Biori Valley)
\item[FLI]
Forum for Language Initiatives (formerly Frontier Language Institute)
\item[\iliHKIA]
\iliHindukushIndoAryan
\item[IA]
\iliIndoAryan
\item[IPA]
International Phonetic Alphabet
\item[lit:]
literally
\item[\iliMIA]
\iliMiddleIndoAryan
\item[\iliNIA]
\iliNewIndoAryan
\item[\iliOIA]
\iliOldIndoAryan
\item[PBUH]
Peace Be Upon Him
\item[pc]
personal communication
\item[PCT]
Palula Common Transcription
\item[SIL]
Summer Institute of Linguistics (now SIL International)
\end{description}


\section*{Grammatical abbreviations}

Abbreviations are listed in upper"=case characters. They are used also in various formats: all capitals, small capitals and lower case, for different purposes.

\begin{description}[leftmargin=!, font=\normalfont, itemsep=0pt,  labelwidth=\widthof{CONDH}]
\item[A]
the most agent-like argument in a~{transitive} {clause} (see O and S)
\item[ACC]
{accusative}
\item[ADJ]
{adjective}, adjectiviser
\item[AG]
agentive (participle)
\item[AGR]
agreement (marking)
\item[AP]
{adjective} {phrase}
\item[AMPL]
amplifier
\item[C]
consonant
\item[CAUS]
{causative}
\item[CNJ]
{conjunction}
\item[CNTR]
contrast
\item[COMP]
{complementiser}
\item[CONDH]
conditional with high degree of verisimilitude 
\item[CONDL]
conditional with low degree of verisimilitude
\item[CPL]
{complement}
\item[CPRD]
copredicative
\item[CV]
converb
\item[DEF]
de{finite}
\item[DIST]
{distal}
\item[DO]
{direct object}
\item[DS]
different {subject}
\item[ERG]
{ergative}
\item[EXP]
experiencer
\item[F]
feminine
\item[FPL]
feminine plural
\item[FSG]
feminine singular
\item[GEN]
{genitive}
\item[HON]
honorific
\item[HOST]
{host element}
\item[HSAY]
hearsay
\item[IDEF]
inde{finite}
\item[IMP]
imperative
\item[INCL]
inclusive
\item[INF]
infinitive
\item[INS]
instrumental
\item[INV]
invariant
\item[IO]
{indirect object}
\item[IDPH]
ideophone
\item[ITR]
in{transitive}
\item[LOC]
{locative}
\item[M]
masculine
\item[MANIP]
manipulee
\item[MPL]
masculine plural
\item[MSG]
masculine singular
\item[N]
neuter
\item[NEG]
negative
\item[NN]
{noun}
\item[NOM]
{nominative}
\item[NNOM]
non"={nominative}
\item[NP]
{noun} {phrase}
\item[O]
the most patient-like argument in a~{transitive} {clause} (see A and S)
\item[OBL]
{oblique}
\item[OBLG]
obligative
\item[{PCU}]
perception, cognition, utterance
\item[PFV]
{perfective}
\item[PL]
plural
\item[PP]
{postpositional phrase}
\item[PPTC]
{perfective} participle
\item[PRD]
{predicate}
\item[PROX]
proximal
\item[PRS]
present
\item[PST]
past
\item[Q]
question marker
\item[QUOT]
{quotative}
\item[RECP]
reciprocal
\item[RED]
{reduplication}
\item[REFL]
reflexive
\item[REL]
relativiser
\item[REM]
{remote}
\item[S]
the sole argument in an~in{transitive} {clause} (see A and O)
\item[S-like]
sentence"=like
\item[SBJ]
{subject}
\item[SEP]
separative
\item[SG]
singular
\item[SS]
same {subject}
\item[SUB]
subordinator
\item[TAG]
tag question
\item[TMA]
{tense}, {mood}, aspect
\item[TOP]
switch"=topic
\item[TR]
{transitive}
\item[V]
verb
\item[V]
vowel (only in reference to {syllable structure})
\item[VN]
verbal {noun}
\item[VOC]
{vocative}
\item[1]
first person
\item[2]
second person
\item[3]
third person
\item[ø]
zero marking
\end{description}



\section*{Abbreviations of example sources}

\subsection*{Palula data references (Ashret dialect)}

\begin{description}[leftmargin=!, font=\normalfont, itemsep=0pt, labelwidth=\widthof{A:MMM}]
\item[A:ABO] Written narrative, Sardar Hayat
\item[A:ACR] Oral narrative, Muhammad Hussain
\item[A:ADJ] Paradigm elicitation, Naseem Haider
\item[A:ANC] Oral narrative, Said Rahim
\item[A:ANJ] Oral hortative {discourse}, Mushtaq Ahmad
\item[A:ASC] Oral narrative, Akhund Seyd
\item[A:ASH] Oral narrative, Akhund Seyd
\item[A:AYA] Oral narrative, Akhund Seyd
\item[A:AYB] Oral narrative, Akhund Seyd
\item[A:BEW] Oral narrative, Fazal ur-Rehman
\item[A:BEZ] Oral narrative, Akhund Seyd
\item[A:BRE] Oral narrative, Haji Sami Ullah
\item[A:CAV] Oral narrative, aunt of Naseem Haider
\item[A:CHA] Oral narrative, Fazal ur-Rehman
\item[A:CHE] Direct elicitation, Naseem Haider
\item[A:CHN] Notes of language use (written), Naseem Haider
\item[A:DHE] Direct elicitation, various informants
\item[A:DHN] Notes of language use, various speakers
\item[A:DRA] Oral narrative, Adils Muhammad
\item[A:GHA] Oral narrative, Lal Zaman
\item[A:GHU] Oral narrative, Ghulam Habib
\item[A:HLE] Direct elicitation, various informants
\item[A:HLN] Notes of language use, various speakers
\item[A:HOW] Oral procedural {discourse}, Hazrat Hassan
\item[A:HUA] Oral narrative, Ghulam Habib
\item[A:HUB] Oral narrative, Muhammad Hanif
\item[A:ISM] Oral narrative-descriptive {discourse}, Muhammad Ismail
\item[A:JAN] Oral narrative, Ghulam Habib
\item[A:KAT] Written narrative, Naseem Haider
\item[A:KEE] Oral procedural-descriptive {discourse}, Lal Zaman
\item[A:KIN] Oral narrative, Haji Sami Ullah
\item[A:MAA] Oral narrative, aunt of Ikram ul-Haq
\item[A:MAB] Oral narrative, Nadir Hussain
\item[A:MAH] Oral narrative, Akhund Seyd
\item[A:MAR] Oral procedural {discourse}, Sher Habib
\item[A:MIT] Oral procedural {discourse}, Said Habib
\item[A:NOR] Written narrative (translated), Sher Haider and Naseem Haider
\item[A:OUR] Oral descriptive {discourse}, Muhammad Jalal ud-Din
\item[A:PAS] Oral narrative, Ghulam Habib
\item[A:PHN] Paradigm elicitation, Naseem Haider
\item[A:PHS] Paradigm elicitation, Sardar Hayat
\item[A:PIR] Oral narrative, aunt of Naseem Haider
\item[A:PRA] Collection of proverbs, Naseem Haider
\item[A:QAM] Direct elicitation, Munir Ahmad, Ihsan Ullah
\item[A:Q6.] Questionnaire (6 from Bouquiaux \& Thomas 1992), Naseem Haider
\item[A:Q9.] Questionnaire (9 from Bouquiaux \& Thomas 1992), Sher Haider
\item[A:REQ] Direct elicitation, Naseem Haider
\item[A:ROP] Oral narrative, Fazal ur-Rehman
\item[A:SEA] Oral descriptive {discourse}, Khurshid Ahmad
\item[A:SHA] Oral narrative, Akhund Seyd
\item[A:SHY] Written narrative, Sher Haider
\item[A:SMO] Oral narrative-hortatory {discourse}, Subadar Rehman
\item[A:TAQ] Questionnaire, TMA (from Dahl 1985), Naseem Haider
\item[A:THA] Oral narrative, Fazli Azam
\item[A:UNF] Written narrative, Misbah ud-Din 
\item[A:UXB] Written narrative, Azhar Ahmad
\item[A:UXW] Written narrative, Mushtaq Ahmad 
\item[A:WOM] Oral narrative, Sardar Hayat
\end{description}

\subsection*{Palula data references (Biori dialect)}

\begin{description}[leftmargin=!, font=\normalfont, itemsep=0pt, labelwidth=\widthof{B:MMM}]
\item[B:ANG] Oral narrative, Atah Ullah
\item[B:ATI] Oral narrative, Atiq Ullah
\item[B:AVA] Oral narrative, Haji Abdul Jalil
\item[B:BEL] Oral narrative, Atiq Ullah
\item[B:CLE] Oral narrative, Atiq Ullah
\item[B:DHE] Direct elicitation, various informants
\item[B:DHN] Notes of language use, various speakers
\item[B:DRB] Written narrative, Atiq Ullah
\item[B:FLO] Oral narrative, Qari Ahmad Saeed
\item[B:FLW] Oral narrative, Atiq Ullah
\item[B:FOR] Written narrative, Riaz ur-Rehman
\item[B:FOX] Written narrative, Miftah ud-Din
\item[B:FOY] Written narrative, Hazrat Noor
\item[B:HLN] Notes of language use, various speakers
\item[B:ISH] Oral enactment, Atah Ullah
\item[B:LET] Oral narrative, Muhammad Zahir Shah
\item[B:MOR] Oral descriptive {discourse}, Atah Ullah
\item[B:PRB] Collection of proverbs, Atiq Ullah
\item[B:QAA] Direct elicitation, Atiq Ullah
\item[B:SHB] Oral narrative, Atiq Ullah
\item[B:SHC] Oral procedural-descriptive {discourse}, Atah Ullah
\item[B:SHI] Oral narrative, Atiq Ullah
\item[B:THI] Written narrative, Mir Alim
\item[B:VIS] Oral narrative-descriptive {discourse}, Ghazi ur-Rehman
\end{description}

\printbibliography[heading=subbibliography]
\end{refsection}