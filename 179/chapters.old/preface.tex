\addchap{Preface}
\begin{refsection}

The fundamental aim of this research is two-fold. First, it represents an attempt to pinpoint the precise origin of the early English lexical and phonetic (lexico-phonetic) influences in Sranan; i.e. whether this influence was from a single dialect, as expressed by a mono-dialectal account of origin, or from a composite of dialects from all over England, as expressed by a pan-dialectal account. Second, it introduces a new methodological tool (comprising of a statistics component, an English dialect geography component and a 17\textsuperscript{th} century English migration history component) with which such linguistic reconstructive work can be done. This tool was used to ascertain the potential dialectal origins of forty-five Sranan words of English origin, alongside the dialectal origin(s) of their speakers. This was done via corroboration of the results of the independent analyses done across the three components of the tripartite methodological tool. The work relies heavily on secondary data sources for both the Sranan data and English dialectal data. The reason for this is the need to use the oldest possible lexical and phonetic information for both language varieties since the early 17\textsuperscript{th} century English influence in Suriname, the country in which Sranan is spoken, ended after 1667.
\printbibliography[heading=subbibliography]
\end{refsection}

