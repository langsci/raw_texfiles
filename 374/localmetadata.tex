\author{Michael Cysouw}
\title{Encyclopaedia of German diatheses}

\BackBody{A diathesis is a sentence structure that reshapes the expression of the roles of a verb. This book presents an encyclopaedic survey of diathesis in German. Currently almost 250 different German diatheses are described in this book, some highly productive, some only attested for a handful of verbs. The main goal of this book is to present this wealth of grammatical possibility in a unified manner, while at the same time attempting to classify and organise this diversity. A summary of the about 80 most prominent diatheses is also provided, including many newly minted German names, because most of these diatheses did not have a German name yet.

Except for diathesis this book also aims to completely catalogue its counterpart: epithesis. An epithesis is a derived sentence structure in which the marking of the roles is not changed. Basically, these are the grammaticalised constructions expressing tense-aspect-mood-evidentiality in German. The list of major epitheses is also quite long (about 40 constructions), but it is quite a bit smaller than the list of major diatheses (about 80 constructions). This indicates that from a purely grammatical perspective, diathesis (“grammatical voice”) is about a two-times more elaborate topic than epithesis (“tense-aspect-mood marking”) in German.}

\renewcommand{\lsSeries}{ogl}
\renewcommand{\lsSeriesNumber}{4}

\renewcommand{\lsID}{374}

\renewcommand{\lsISBNdigital}{978-3-96110-407-9}
\renewcommand{\lsISBNhardcover}{978-3-98554-065-5}
\BookDOI{10.5281/zenodo.7602514}

\renewcommand{\lsCoverTitleFont}[1]{%
    \sffamily\addfontfeatures{Scale=MatchUppercase}%
    \fontsize{50pt}{17mm}\selectfont #1}

\proofreader{Agnes Kim, Annemarie Verkerk, Elliot
Pearl, Felix Kopecky, Jeroen van de Weijer, Katja Politt, Lea Schäfer,
Lisa Schäfer, Patricia Cabredo Hofherr, Yvonne Treis}
