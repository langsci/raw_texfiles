\documentclass[output=paper]{langsci/langscibook} 
\ChapterDOI{10.5281/zenodo.1314329}
\author{Larry M. Hyman\affiliation{University of California, Berkeley}}
\title{Third person pronouns in Grassfields Bantu} 
\abstract{In this paper I have two goals. First, I propose a reconstruction of the pronoun system of Grassfields Bantu, direct reflexes of which are found in Eastern Grassfields, with a close look at the pronoun systems, as reflected across this varied group. Second, I document and seek the origin of innovative third person pronouns in Western Grassfields. While EGB languages have basic pronouns in all persons, both the Momo and Ring subgroups of WGB have innovated new third person (non-subject) pronouns from demonstratives or perhaps the noun ‘body’. However, these languages show evidence of the original third person pronouns which have been restricted to a logophoric function. I end with a comparison of the Grassfields pronouns with nearby Bantoid and Northwest Bantu languages as well as Proto-Bantu.}
\epigram{“In linguistic theory, the 3rd person has had bad luck.”  \citep[5]{Pozdniakov2010}}
\maketitle
\begin{document}
 
\label{sec:6}

\section{The problem} 
While \ili{Eastern Grassfields Bantu}, like \ili{Narrow Bantu}, has an old and consistent paradigm of pronouns, \ili{Western Grassfields Bantu} has innovated new third person forms, often keeping the original forms as logophoric pronouns. The major questions I address in this chapter are: (i) Where do these new third person pronouns come from? (ii) Why were they innovated? (iii) What is the relation, if any, to logophoricity? In the following sections I first briefly introduce the subgrouping of Grassfields \ili{Bantu} that I will be assuming, then successively treat third person pronouns in the different subgroups: \ili{Eastern Grassfields}, \ili{Ring Grassfields}, and \ili{Momo Grassfields}. I then consider some examples from outside \ili{Grassfields Bantu}. The last section provides a brief summary and conclusion.(For a broader discussion of East Benue-Congo noun class systems, their morphological behavior, and, in particular, the place of third-person pronouns in those systems, see Good, Chapter 2 of this volume, and in particular §4 on domains of concord.)

\section{Grassfields Bantu} 

In \REF{ex:grassfields:1}, I present two subclassifications of \ili{Grassfields Bantu}, ignoring the possible inclusion of \ili{Ndemli} (cf. \citealt{Stallcup1980geo}, \citealt{WattersLeroy1989}, \citealt{Piron1995}, \citealt{Watters2003}).


\eabox{\label{ex:grassfields:1}
\small
\begin{xlist}
\parbox[t]{.35\textwidth}{
\vspace*{-3mm}
\ex
\vspace{.1mm}
\begin{forest}
 [\ilit{Grassfields  Bantu} [\ilit{WGB} [{\ilit{Ring}}] [\ilit{Momo}]] [\ilit{EGB}] ]
\end{forest}
\hspace*{-1cm}
}
\parbox[t]{.35\textwidth}{
\vspace*{-2mm}
\ex
\vspace{.1mm}
\begin{forest}
 [\ilit{Grassfields  Bantu} [{\ilit{Ring}}] [\ilit{Momo}] [\ilit{EGB}]]
 \end{forest}
}
\end{xlist}
}



The subgrouping of (\ref{ex:grassfields:1}a) shows a split between \ili{Western Grassfields Bantu} (\ili{WGB}) and \ili{Eastern Grassfields Bantu} (\ili{EGB}), where \ili{WGB} consists of two further subgroups, \ili{Ring} and \ili{Momo}. In (\ref{ex:grassfields:1}b) these two subgroups are considered coordinate with \ili{EGB}. Some of the major languages of each subgroup are identified in \REF{ex:grassfields:2}.

\ea%2
\label{ex:grassfields:2} 
  \begin{xlist}
  \exbox[-.5\baselineskip]{\parbox[t]{1.3cm}{{\ilit{Ring}}:}\parbox[t]{9.5cm}{\raggedright
    Aghem\il{Aghem}, \ilit{Isu}, \ilit{Weh}, \ilit{Bum}, \ilit{Bafmeng}, \ilit{Kom}, \ilit{Oku}, \ilit{Babanki}, \ilit{Lamnso’}, \ilit{Babungo}, \ilit{Babessi}}}
  \medskip
    \exbox[-.5\baselineskip]{\parbox[t]{1.3cm}{\ilit{Momo}:}\parbox[t]{9.5cm}{\raggedright Moghamo\il{Moghamo}, \ilit{Metta}, \ilit{Menemo}, \ilit{Ngembu}, \ilit{Ngamambo}, \ilit{Ngie}, \ilit{Oshie}, \ilit{Ngwo}, \ilit{Mundani}, \ilit{Njen}}}
  \medskip
  \exbox[-.5\baselineskip]{\parbox[t]{1.3cm}{\ilit{EGB}:}\parbox[t]{9.5cm}{\raggedright Ngemba\il{Ngemba} (e.g. \ilit{Mankon}, \ilit{Bafut}), \ilit{Bamileke} (e.g. \ilit{Yemba}, \ilit{Ghomala}, \ilit{Medumba}, \ilit{Fe’fe’}), \ilit{Nun} (e.g. \ilit{Bamun}, \ilit{Bali}), \ilit{North} (e.g. \ilit{Limbum}, \ilit{Adere})}}
  \end{xlist}
\z

 
Although the two subgroupings in \REF{ex:grassfields:1} differ in whether a \ili{WGB} unit is recognized, I will assume the classification in (\ref{ex:grassfields:1}a) for the purpose of the present discussion. The following table in \REF{ex:grassfields:3} summarizes the significant differences between \ili{EGB} and \ili{WGB} \citep[55]{Stallcup1980geo}:


\ea
    \label{ex:grassfields:3}

\parbox{5cm}{\ili{Eastern Grassfields Bantu}}~~~~\parbox{5cm}{\ili{Western Grassfields Bantu}}
% [Warning: Draw object ignored]
\ea \parbox[t]{5cm}{nasal prefix in\\class 1 and 3 nouns}~~\parbox[t]{5cm}{ absence of the nasal}\\[1em]
\ex \parbox[t]{5cm}{no distinction between\\class 6 and class 6a}~~\parbox[t]{5cm}{ distinction between class 6 \textit{a-} and class 6a \textit{mə-}}\\[1em]
\ex \parbox[t]{5cm}{nasal prefix on\\all 9/10 nouns}~~\parbox[t]{5cm}{ nasal prefix only on some 9/10 nouns}\\[1em]
\ex \parbox[t]{5cm}{absence of classes\\4 and 13; class 19 rare}~~\parbox[t]{5cm}{ presence of classes 4 and 13; class 19 frequent}\\[1em]
\ex \parbox[t]{5cm}{noun prefixes all\\carry a /L/ tone}~~\parbox[t]{5cm}{ most noun prefixes carry a /H/ tone}\\[1em]
\ex \parbox[t]{5cm}{no noun suffixes}~~\parbox[t]{5cm}{ many noun suffixes, e.g. plural \mbox{\textit{-tí},} \textit{-sí}}\\[1em]
\ex \parbox[t]{5cm}{class 2 or 6a\\generalizes to mark plural}~~\parbox[t]{5cm}{ class 10 or 13 generalizes to mark plural}\\[1em]
\ex \parbox[t]{5cm}{innovation of\\\textit{-síŋə́} ‘bird’, \textit{-kìə́} ‘water’}~~\parbox[t]{5cm}{ maintenance of \textit{*-nɔ̀}\textit{ní} ‘bird’, \textit{*-díbá} ‘water’}\\[1em]
\ex \parbox[t]{5cm}{maintenance of\\*\textit{-úmà} ‘thing’}~~\parbox[t]{5cm}{ *\textit{-úmà} is lost, other roots come in}\\[1em]
\exi{\textit{Plus}:}\parbox[t]{4.6cm}{maintenance of inherited\\ 3\textsuperscript{rd} person pronouns}~~\parbox[t]{5cm}{ introduction of new 3\textsuperscript{rd} person pronouns}\\[1em]
\z
\z

As seen, the differences in (3a-g) all have to do with noun classes. Significant to this chapter is the last difference, which I have added: As we shall see in the following sections, \ili{EGB} languages maintain the inherited \ili{Proto-Grassfields Bantu} (PGB) third person pronouns, while WGB languages have innovated new pronouns.

\clearpage 
\section{ Eastern Grassfields Bantu} 
In this section I begin with \ili{EGB} pronoun systems, since they directly reflect the reconstructions proposed by   \citet{HymanTadadjeu1976} and others subsequently. In each section we need to consider subject, object and possessive pronouns. I will often illustrate the forms with human third person pronouns, i.e. singular class 1 \textit{*(m)u-}, plural class 2 \textit{*ba-}. Thus, unless otherwise noted, “third person” will refer to class 1 (sg) and class 2 (pl).

  In \REF{extab:grassfields:4} I present the human (class 1/2) subject and object pronouns in selected EGB languages:

\begin{table}
\fittable{
\begin{tabular}{l lllllll p{1mm} llllll}
\lsptoprule 
           & \multicolumn{7}{c}{  subject pronouns} & {} & \multicolumn{6}{c}{  object pronouns}\\
    & {  1sg} & {  2sg} & {  3sg} & {  Log} & {  1pl} & {  2pl} & {  3pl} & {} & {  1sg} & {  2sg} & {  3sg} & {  1pl} & {  2pl} & {  3pl}\\
\cmidrule{2-8}\cmidrule{10-15}
{\ili{Mankon}}     & \textit{mà} & \textit{ò} & \textit{à} & \textit{zɯ́} & \textit{tɨ̀} & \textit{nɨ̀} & \textit{bɨ́} & \textit{} & \textit{ɣə̰̂} & \textit{ɣô} & \textit{yɛ́} & \textit{wɯ́ɣə́} & \textit{wɯŋə́} & \textit{wá}\\

{\ili{Dschang}}    & \textit{mə̀ŋ/Ǹ} & \textit{ò} & \textit{à} & \textit{yí} & \textit{pèk} & \textit{pɛ̀} & \textit{pɔ́} & \textit{} & \textit{ga} & \textit{wu} & \textit{yí} & \textit{wek} & \textit{wɛ} & \textit{wɔp}\\

{\ili{Fe’fe’}}    & \textit{Ǹ} & \textit{ò} & \textit{à} & \textit{---} & \textit{pɑ̀h} & \textit{pɛ̀n} & \textit{pō} & \textit{} & \textit{ā} & \textit{ō} & \textit{ī} & \textit{yɔ̄h} & \textit{yēe} & \textit{yɑɑ}\\

{\ili{Bangangte}}  & \textit{mə} & \textit{o} & \textit{a} & \textit{---} & \textit{bag} & \textit{bi} & \textit{bo} & \textit{} & \textit{ɑ́m} & \textit{ó} & \textit{é} & \textit{yág} & \textit{zín} & \textit{yób}\\

\lspbottomrule
\end{tabular}
}
\caption{Class {1/2} subject \& object pronouns in some EGB languages}
\label{extab:grassfields:4}
In the above forms 1pl = first person plural exclusive.
\end{table} 
As seen, the plural subject pronouns generally begin with class 2 \textit{p-} or \textit{b-} while the corresponding object pronouns all begin with class 1 \textit{w-} or \textit{y-} = class 1 (cf. the possessive forms in \tabref{extab:grassfields:5}). As seen in the table, the subject logophoric pronoun is identical to the 3sg object pronoun \textit{yí} in \ili{Dschang} (\ili{Yemba}) \citep[22]{HarroHaynes1991}. In \ili{Mankon}, on the other hand, the subject logophoric pronoun corresponds to the distinct 3sg independent pronoun \textit{zɯ́} \citep[209]{Leroy2007}.

% As seen, the plural subject pronouns generally begin with class 2 \textit{p-} or \textit{b-} while the corresponding object pronouns all begin with class 1 \textit{w-} or \textit{y-} (cf. the possessive forms in \tabref{extab:grassfields:5}). As seen in the table, the subject logophoric pronoun is identical to the 3sg object pronoun \textit{yí} in \ili{Dschang} (Yemba) \citep[22]{HarroHaynes1991}. In \ili{Mankon}, on the other hand, the subject logophoric pronoun corresponds to the distinct 3sg independent pronoun \textit{zɯ́} \citep[209]{Leroy2007}. 

In \tabref{extab:grassfields:5} I present the class 1/2 possessive pronouns in a wide range of \ili{EGB} languages, where~ { ̰  } 
% {\textasciitilde} 
= nasalization and ° = a level L tone (contrasting with a L that downglides from L to a lower L before pause). For the proposed \ili{Proto-Grassfields Bantu} (\ili{PGB}) reconstructions, indicated below these forms, see    \citet[85]{HymanTadadjeu1976}.
\begin{table} 
\fittable{
\begin{tabular}{l@{~}l@{~}l@{~}l@{~}l@{~}l@{~}l@{~}p{1mm}@{~}l@{~}l@{~}l@{~}l@{~}l@{~}l@{~}}
\lsptoprule
 & \multicolumn{6}{c}{  class 1 \textit{*gù-}} & {} & \multicolumn{6}{c}{class 2 \textit{*bə́-}}\\
{Language} & {  1sg} & {  2sg} & {  3sg} & {  1pl} & {  2pl} & {  3pl} & {} & {  1sg} & {  2sg} & {  3sg} & {  1pl} & {  2pl} & {  3pl}\\
\midrule
{\ili{Mankon}} & \textit{ɣʌ̰̀} & \textit{ɣò} & \textit{yìɛ́} & \textit{wə̀ɣə́} & \textit{wə̀ŋə́} & \textit{wàá} & \textit{} & \textit{bʌ̰̂} & \textit{bô} & \textit{byé} & \textit{bə́ɣə́} & \textit{bə́ŋə́} & \textit{báá}\\
\ili{Mbui}   & \textit{wà} & \textit{ɣò} & \textit{wì˚} & \textit{wìì˚} & \textit{wə̀˚} & \textit{wʌ̀˚} & \textit{} & \textit{bá} & \textit{búó} & \textit{bí} & \textit{bíí} & \textit{bə́} & \textit{bʌ́}\\
\ili{Bamenyan} & \textit{wìɛ̀} & \textit{ɣò} & \textit{ɣě} & \textit{wɯ̌} & \textit{wǒ} & \textit{wǒ} & \textit{} & \textit{píɛ̀} & \textit{pô} & \textit{pé} & \textit{pɯ́} & \textit{pó} & \textit{pó}\\
\ili{Babadjou} & \textit{ɣà} & \textit{ɣò} & \textit{yè˚} & \textit{wɔ̀˚} & \textit{wèì˚} & \textit{ɣàp˚} & \textit{} & \textit{pâ} & \textit{pô} & \textit{pé} & \textit{pɔ́} & \textit{péí} & \textit{páp}\\
\ili{Dschang} & \textit{ɣà} & \textit{wù} & \textit{yì˚} & \textit{wə̀k˚} & \textit{wɛ̀˚} & \textit{wòp˚} & \textit{} & \textit{pá} & \textit{pú} & \textit{pí} & \textit{pə́k} & \textit{pɛ́} & \textit{póp}\\
\ili{Ngwe} & \textit{ɣà} & \textit{ɣò} & \textit{gyè˚} & \textit{wə̀˚} & \textit{wʌ̀˚} & \textit{wʌ̀p˚} & \textit{} & \textit{bá} & \textit{bó} & \textit{bé} & \textit{bə́k} & \textit{bʌ́} & \textit{bʌ́p}\\
\ili{Babete} & \textit{à} & \textit{ò} & \textit{è˚} & \textit{wə̀k˚} & \textit{wɯ̀˚} & \textit{wɔ̀p˚} & \textit{} & \textit{pá} & \textit{pú} & \textit{pé} & \textit{pə́k} & \textit{pɯ́} & \textit{pɔ́p}\\
\ili{Bati} & \textit{à} & \textit{ù} & \textit{ì} & \textit{pɔ̀} & \textit{yì} & \textit{yàp} & \textit{} & \textit{pá} & \textit{pú} & \textit{pí} & \textit{pɔ̀} & \textit{yí} & \textit{yáp}\\
\ili{Bagam} & \textit{à} & \textit{ò} & \textit{è˚} & \textit{wíŋì} & \textit{wùŋ˚} & \textit{wɔ̀p˚} & \textit{} & \textit{pá} & \textit{pó} & \textit{pé} & \textit{píŋì} & \textit{púŋ} & \textit{pɔ́p}\\
\ili{Bangang} & \textit{à̰} & \textit{ò} & \textit{ì˚} & \textit{wə̀k˚} & \textit{ɥì˚} & \textit{wɔ̀p˚} & \textit{} & \textit{pá̰} & \textit{pú} & \textit{pé} & \textit{pə́k} & \textit{pí} & \textit{pɔ́p}\\
\ili{Baloum} & \textit{à} & \textit{ò} & \textit{ì˚} & \textit{w\textsuperscript{h}ɯ̀˚} & \textit{wè˚} & \textit{wɔ̀p˚} & \textit{} & \textit{pá} & \textit{pú} & \textit{pí} & \textit{p\textsuperscript{h}ɯ́} & \textit{pé} & \textit{pɔ́p}\\
\ili{Fomopea} & \textit{à} & \textit{ò} & \textit{ì˚} & \textit{wə̀k˚} & \textit{wè˚} & \textit{wɔ̀p˚} & \textit{} & \textit{pá} & \textit{pú} & \textit{pí} & \textit{pə́k} & \textit{pé} & \textit{pɔ́p}\\
\ili{Bamendjou} & \textit{à} & \textit{ò} & \textit{ì˚} & \textit{wə̀k˚} & \textit{wɯ̀˚} & \textit{wòp˚} & \textit{} & \textit{pá} & \textit{pó} & \textit{pí} & \textit{pə́k} & \textit{pɯ́} & \textit{póp}\\
\ili{Baleng} & \textit{à} & \textit{ò} & \textit{è˚} & \textit{wɔ̀k˚} & \textit{wè˚} & \textit{wùp˚} & \textit{} & \textit{pá} & \textit{pú} & \textit{pyɛ́} & \textit{pɔ́k} & \textit{pé} & \textit{púp}\\
\ili{Bandjoun} & \textit{à} & \textit{ò} & \textit{è˚} & \textit{yɔ̀k˚} & \textit{yɔ̀˚} & \textit{yàp˚} & \textit{} & \textit{pǎ} & \textit{pǔ} & \textit{pyə́} & \textit{pɔ́k} & \textit{pɔ́} & \textit{páp}\\
\ili{Batie} & \textit{à} & \textit{ò} & \textit{è} & \textit{yɔ̀k˚} & \textit{yèè˚} & \textit{yàp˚} & \textit{} & \textit{pɛ́} & \textit{pó} & \textit{pé} & \textit{pɔ́k} & \textit{péé} & \textit{páp}\\
\ili{Bangou} & \textit{à} & \textit{ù} & \textit{ì} & \textit{yɔ̀h} & \textit{yɯ̀} & \textit{yòp} & \textit{} & \textit{pɛ̄} & \textit{pō} & \textit{pə́} & \textit{pɔ́h} & \textit{pɯ́} & \textit{póp}\\
\ili{Bangwa} & \textit{ɛ̀{\textasciitilde}à} & \textit{ù{\textasciitilde}ò} & \textit{ì{\textasciitilde}è} & \textit{yɔ̀} & \textit{ʒyə̀} & \textit{ʒùp} & \textit{} & \textit{pɛ́} & \textit{pú} & \textit{pí} & \textit{pɔ́} & \textit{pyə́} & \textit{púp}\\
\ili{Batoufam} & \textit{à} & \textit{ù} & \textit{ì} & \textit{wɔ̀} & \textit{wɯ̀ɣə̀} & \textit{wùp} & \textit{} & \textit{pɛ̄} & \textit{pū} & \textit{pə́} & \textit{pɔ́} & \textit{pɯ́ɣə́} & \textit{púp}\\
\ili{Fotouni} & \textit{à} & \textit{ɔ̀} & \textit{ì} & \textit{yɔ̀˚} & \textit{yè˚} & \textit{yàp˚} & \textit{} & \textit{βá} & \textit{βɔ́} & \textit{βí} & \textit{βɔ́} & \textit{βé} & \textit{βáp}\\
\ili{Fondanti} & \textit{à} & \textit{ò} & \textit{ì} & \textit{yɔ̀} & \textit{yì} & \textit{yàp} & \textit{} & \textit{bá} & \textit{bó} & \textit{bí} & \textit{yɔ́} & \textit{yí} & \textit{yáp}\\
\ili{Fe’fe’} & \textit{à} & \textit{ò} & \textit{ì˚} & \textit{yɔ̀h˚} & \textit{yìì˚} & \textit{yɑ̀ɑ̀˚} & \textit{} & \textit{bǎ} & \textit{bǒ} & \textit{bī} & \textit{bɔ̄h} & \textit{bīī} & \textit{bɑ̄ɑ̄}\\
\ili{Bali} 	 & \textit{à} & \textit{ù} & \textit{ì} & \textit{yɯ̀ʔ} & \textit{yìn} & \textit{yàp} & \textit{} & \textit{bá} & \textit{bú} & \textit{bí} & \textit{bɯ́ʔ} & \textit{bín} & \textit{báp}\\
\ili{Bamun} 	 & \textit{à} & \textit{ù} & \textit{ì} & \textit{ɯ̀} & \textit{ɯ̀n} & \textit{àp} & \textit{} & \textit{pá} & \textit{pú} & \textit{pí} & \textit{pɯ́} & \textit{pɯ́n} & \textit{páp}\\
\ili{Bapi} 	 & \textit{á} & \textit{ú} & \textit{í} & \textit{yúʔ} & \textit{yɯ́n} & \textit{yɔ́p} & \textit{} & \textit{pá} & \textit{pú} & \textit{pí} & \textit{púʔ} & \textit{pɯ́n} & \textit{pɔ́p}\\
\ili{Bangangte} & \textit{ɑ̀m} & \textit{ò} & \textit{è˚} & \textit{yàg˚} & \textit{zìn˚} & \textit{yòb˚} & \textit{} & \textit{cɑ́m} & \textit{có} & \textit{tsə́} & \textit{cɑ́ghə̀˚} & \textit{tsínə̀˚} & \textit{cóbə̀˚}\\
\ili{Limbum} 	 & \textit{yà} & \textit{yò} & \textit{yì} & \textit{yèr} & \textit{yèè} & \textit{yàb} & \textit{} & \textit{wá} & \textit{wó} & \textit{ví} & \textit{wér} & \textit{wéé} & \textit{wáb}\\
{\ili{Adere}}  & \textit{wàm} & \textit{wɔ̀} & \textit{wì˚} & \textit{-wùt˚} & \textit{-wùn˚} & \textit{-wɔ̂} & \textit{} & \textit{bám} & \textit{bɔ́} & \textit{bí} & \textit{-wùt˚} & \textit{-wùn˚} & \textit{-wɔ̂}\\ 
\tablevspace 
{\ili{PGB}}: & \textit{*gù-àmə̀} &   \textit{*gù-ò} &   \textit{*gù-í} &   \textit{*gù-ítə́} &   \textit{*gù-ínə́ } &  \textit{*gù-ábə́} & \textit{} &    \textit{*bə́-àmə̀} &   \textit{*bə́-ò} &   \textit{*bə́-í} &   \textit{*bə́-ítə́} &    \textit{*bə́-ínə́} &   \textit{*bə́- ábə́}\\
\lspbottomrule
\end{tabular}
}
\caption{Class 1/2 possessive pronouns in various EGB languages}
\label{extab:grassfields:5}
\end{table}

Important for our purposes is that the \ili{PGB} possessive pronoun reconstructions directly correspond to the morphologically complex independent pronominal stems proposed for \ili{Proto-Bantu} (PB) by \citet[215]{KambaMuzenga2003}: *\textit{-a-mi-e}, *\textit{-u-bɪ-e}, *\textit{-a-i-}, *\textit{-i-cu-e}, *\textit{\nobreakdash-i\nobreakdash-ɲu-e}, *\textit{-a-ba-o} (cf. \tabref{extab:grassfields:25}). The story is quite different in the \ili{Ring} and \ili{Momo} subgroups, at least in the third person.

\section{Ring Grassfields Bantu}\label{sec:grassfields:4}

While the \ili{EGB} languages provide a “baseline” for \ili{Proto-Grassfields Bantu}, \ili{Ring} and \ili{Momo} have innovated new third person pronouns. Thus, in the \ili{Ring} languages, \ili{Babanki} \textit{wɛ́}\textit{n} and \ili{Aghem} \textit{wɨ́}\textit{n} ‘him/her, his/her’ are quite different from the \ili{PGB}  3sg \textit{*-í} reconstruction. In past literature I have considered two different historical scenarios (to which we will return below):

\begin{quote}
The third person pronoun ‘his/hers’ is derived from the noun /ə̀wɛ́n/ ‘body’ \citep[245]{Hyman:Babanki}
\end{quote}

  
\begin{quote}
... the form ‘his/her’ is related to the demonstrative root -\textit{ɨ́n} ‘this/these’. Historically, a form such as [\ili{Aghem}] \textit{nwɨ́n {\downstep}}\textit{fɨ{\downstep}}\textit{wɨ́n} ‘his/her bird’ meant ‘bird of this one’. \citep[29]{Hyman1979phonology}
\end{quote}


\largerpage[-1]
\noindent
In other words, it is possible that the new third person pronouns came either from a noun such as ‘body’ or from the near speaker demonstrative. In order to observe the phonetic resemblances, compare in \tabref{extab:grassfields:6} the following \ili{Babanki} and \ili{Kom} pronouns and ‘near speaker’ demonstratives with their words for ‘body’: \ili{Babanki} \textit{ə̀-wɛ́n}, \ili{Kom} \textit{ə̄-wúīn} (\citealt{Hyman1980}, \ili{Kom} notes; \citealt{Jones2001}).
                                            

\begin{table}
\caption{Babanki and Kom pronouns and ‘near speaker’ demonstratives}
    \label{extab:grassfields:6} 
\begin{tabularx}{\textwidth}{ll lllX llll}
\lsptoprule
    
  &  & \multicolumn{4}{c}{  \ili{Babanki}} & \multicolumn{4}{c}{  \ili{Kom}}\\
\midrule
 a. &  & \textit{mò} & ‘me’ & \textit{yɛ̀s} & ‘us’ & \textit{mā} & ‘me’ & \textit{ɣʌ̀s} & ‘us’\\
  &  & \textit{wù} & ‘you sg.’ & \textit{ɣʌ̀ŋ} & ‘you pl.’ & \textit{vvà} & ‘you sg.’ & \textit{zɨ̀} & ‘you pl.’\\
\tablevspace
 b. & \itshape cl. & \multicolumn{2}{l}{ ‘him, them, it’} & \multicolumn{2}{l}{ ‘this/these’} & \multicolumn{2}{l}{‘him, them, it’} & \multicolumn{2}{l}{ ‘this/these’}\\
% <<<<<<< HEAD
& 1 &  \multicolumn{2}{l}{{\textit{wɛ́n}}} & \multicolumn{2}{l}{\textit{ə̀-ɣɛ̀n}} & \multicolumn{2}{l}{\textit{ŋwɛ̄n}} & \multicolumn{2}{l}{\textit{ə̄-wɛ̂n}}\\
& 2 &  \multicolumn{2}{l}{{\textit{və̀-wɛ́{\downstep}}n-ə́}} & \multicolumn{2}{l}{\textit{ə̀-vɛ́n-ə́}} & \multicolumn{2}{l}{\textit{à-ŋə̄ná}} & \multicolumn{2}{l}{\textit{ə̄-ɣɛ̂n˚}}\\
& 3 &  \multicolumn{2}{l}{{\textit{ə̀-wé{\downstep}}é-ɣə́}} & \multicolumn{2}{l}{\textit{ə̀-ɣɛ́n-ə́}} & \multicolumn{2}{l}{\textit{ə̀-ŋwɛ̄n}} & \multicolumn{2}{l}{\textit{ə̄-wɛ̂n˚}}\\
& 5 &  \multicolumn{2}{l}{{\textit{ə̀-wé{\downstep}}é-zə́}} & \multicolumn{2}{l}{\textit{ə̀-ʒɛ́n-ə́}} & \multicolumn{2}{l}{\textit{ì-ɲɛ̄n-ɨ̄}} & \multicolumn{2}{l}{\textit{ī-yɛ́n-ì˚}}\\
& 6 &  \multicolumn{2}{l}{{\textit{à-wé{\downstep}}é-ɣə́}} & \multicolumn{2}{l}{\textit{à-ƒɛ́n-ə́}} & \multicolumn{2}{l}{\textit{à-ŋkə̄n-ā}} & \multicolumn{2}{l}{\textit{ā-kə́n-à˚}}\\
& 7 &  \multicolumn{2}{l}{{\textit{{kə̀-wɛ́n(-kə́)}}}} & \multicolumn{2}{l}{\textit{ə̀-kɛ́n-ə́}} & \multicolumn{2}{l}{\textit{à-ŋkə̄n-ā}} & \multicolumn{2}{l}{\textit{ā-kə́n-à˚}}\\
& 8 &  \multicolumn{2}{l}{{\textit{{ ə̀-wé{\downstep}}é-və́}}} & \multicolumn{2}{l}{\textit{ə̀-vɛ́n-ə́}} & \multicolumn{2}{l}{\textit{ə̀-ŋwɛ̄n}} & \multicolumn{2}{l}{\textit{ə̄-wɛ̂n˚}}\\
& 9 &  \multicolumn{2}{l}{{\textit{wɛ́n}}} & \multicolumn{2}{l}{\textit{ə̀-ʒɛ̀n}} & \multicolumn{2}{l}{ \textit{ɲɛ̄n}} & \multicolumn{2}{l}{\textit{ə̄-yɛ̂n}}\\
& 10 & \multicolumn{2}{l}{{\textit{{sə̀-wɛ́n(-sə́)}}}} & \multicolumn{2}{l}{\textit{ə̀-sɛ̄n-sə́}} & \multicolumn{2}{l}{\textit{ǹsɛ̄n-sə̄}} & \multicolumn{2}{l}{\textit{ə̄-sɛ́n-sə̀˚}}\\
& 13 & \multicolumn{2}{l}{{\textit{{tə̀-wɛ́n(-tə́)}}}} & \multicolumn{2}{l}{\textit{ə̀-tɛ̄n-tə́}} & \multicolumn{2}{l}{\textit{ǹɛ̄n-tə̄}} & \multicolumn{2}{l}{\textit{ə̄-tɛ́n-tə̀˚}}\\
& 19 & \multicolumn{2}{l}{{\textit{{fə̀-wɛ́n(-fə́)}}}} & \multicolumn{2}{l}{\textit{ə̀-fɛ̄n-fə́}} & \multicolumn{2}{l}{\textit{ǹfɛ̄nfə̄}} & \multicolumn{2}{l}{\textit{ə̄-fɛ́n-fə̀˚}}\\
& 6a & \multicolumn{2}{l}{{\textit{\`{ŋ}}-wéé-mə̀}} & \multicolumn{2}{l}{\textit{ə̀-mɛ̀n-ə̀}} & \multicolumn{2}{l}{\textit{ə̀mɛ̄ǹ}} & \multicolumn{2}{l}{\textit{ə̄-mɛ̂n}}\\
% =======
% & 1 & \multicolumn{2}{l}{ \textit{wɛ́n} & \multicolumn{2}{l}{ə̀-ɣɛ̀n} & \multicolumn{2}{l}{ ŋwɛ̄n} & \multicolumn{2}{l}{ə̄-wɛ̂n}}\\
% & 2 & \multicolumn{2}{l}{ \textit{və̀-wɛ́{\downstep}n-ə́} & \multicolumn{2}{l}{ə̀-vɛ́n-ə́} & \multicolumn{2}{l}{à-ŋə̄ná} & \multicolumn{2}{l}{ə̄-ɣɛ̂n˚}}\\
%   & 3 & \multicolumn{2}{l}{ ə̀-wé{\downstep}é-ɣə́} & \multicolumn{2}{l}{ə̀-ɣɛ́n-ə́} & \multicolumn{2}{l}{ə̀-ŋwɛ̄n} & \multicolumn{2}{l}{ə̄-wɛ̂n˚}\\
% & 5 & \multicolumn{2}{l}{ \textit{ə̀-wé{\downstep}é-zə́} & \multicolumn{2}{l}{ə̀-ʒɛ́n-ə́} & \multicolumn{2}{l}{ì-ɲɛ̄n-ɨ̄} & \multicolumn{2}{l}{ī-yɛ́n-ì˚}}\\
% & 6 & \multicolumn{2}{l}{ \textit{à-wé{\downstep}é-ɣə́} & \multicolumn{2}{l}{à-ƒɛ́n-ə́} & \multicolumn{2}{l}{à-ŋkə̄n-ā} & \multicolumn{2}{l}{ā-kə́n-à˚}}\\
% & 7 & \multicolumn{2}{l}\textit{{kə̀-wɛ́n(-kə́)} & \multicolumn{2}{l}{ə̀-kɛ́n-ə́} & \multicolumn{2}{l}{à-ŋkə̄n-ā} & \multicolumn{2}{l}{ā-kə́n-à˚}}\\
% & 8 & \multicolumn{2}{l}\textit{{ ə̀-wé{\downstep}é-və́} & \multicolumn{2}{l}{ə̀-vɛ́n-ə́} & \multicolumn{2}{l}{ə̀-ŋwɛ̄n} & \multicolumn{2}{l}{ə̄-wɛ̂n˚}}\\
% & 9 & \multicolumn{2}{l}\textit{{     wɛ́n} & \multicolumn{2}{l}{ə̀-ʒɛ̀n} & \multicolumn{2}{l}{ ɲɛ̄n} & \multicolumn{2}{l}{ə̄-yɛ̂n}}\\
% & 10 & \multicolumn{2}{l}\textit{{sə̀-wɛ́n(-sə́)} & \multicolumn{2}{l}{ə̀-sɛ̄n-sə́} & \multicolumn{2}{l}{ǹsɛ̄n-sə̄} & \multicolumn{2}{l}{ə̄-sɛ́n-sə̀˚}}\\
% & 13 & \multicolumn{2}{l}\textit{{tə̀-wɛ́n(-tə́)} & \multicolumn{2}{l}{ə̀-tɛ̄n-tə́} & \multicolumn{2}{l}{ǹɛ̄n-tə̄} & \multicolumn{2}{l}{ə̄-tɛ́n-tə̀˚}}\\
% & 19 & \multicolumn{2}{l}\textit{{fə̀-wɛ́n(-fə́)} & \multicolumn{2}{l}{ə̀-fɛ̄n-fə́} & \multicolumn{2}{l}{ǹfɛ̄nfə̄} & \multicolumn{2}{l}{ə̄-fɛ́n-fə̀˚}}\\
% & 6a & \multicolumn{2}{l}{ \textit{ŋ\`{} -wéé-mə̀} & \multicolumn{2}{l}{ə̀-mɛ̀n-ə̀} & \multicolumn{2}{l}{ə̀mɛ̄ǹ} & \multicolumn{2}{l}{ə̄-mɛ̂n}}\\
% >>>>>>> c6bb032e1da3fe3d1db9970bff3123d1a38c78d3
\lspbottomrule
\end{tabularx}
\end{table}
% \todo{check version conflict}

While phonological rules obscure some of the forms (e.g. by deleting an intervocalic [n] in some of the pronominal forms in \ili{Babanki}), the phonetic resemblance of the new third person pronouns to both the demonstrative ‘this/these’ and the word for ‘body’ is striking. (The \ili{Kom} form \textit{ə̄-wúīn} ‘body’ shows labialization into the root from the historically prior form *\textit{ú-wín}; cf. \ili{Oku}, to which we now turn.)

In \tabref{extab:grassfields:7} the first and second person possessives are shown for the different noun classes in \ili{Oku} (from 
my notes). Although the second person singular forms have the unrounded diphthong [iɛ] and final *\textit{t} has become [s] in the 
first person plural forms, the above pronominal forms clearly resemble those reconstructed in \ili{PGB} 
\tabref{extab:grassfields:5}. In addition, there is an initial underlying /\textit{ə̀-}/ on the pronoun, corresponding to \ili{PGB} 
*\textit{ə̀}-CV-\textsc{pron,} which however can become obscured by phonology, e.g. \textit{kēkém ə̀k}\textit{ɔ́m} 
\textit{${\rightarrow}$} \textit{kēké}\textit{\`{m}} \textit{kɔ́}\textit{m} ‘my crab’.

\begin{table}
 \caption{Oku 1p and 2p possessives for each Oku noun class}
 \label{extab:grassfields:7}
\begin{tabularx}{\textwidth}{lllXXlll}
\lsptoprule
cl. &  noun &  gloss &  &  ‘my’ &  ‘your sg’ &  ‘our (excl)’ &  ‘your pl’\\
\midrule
% <<<<<<< HEAD
1 & \textit{wān} & ‘child’      & \textit{wāǹ} &  \textit{wɔ̄m} &  \textit{vīɛ̀} &  \textit{wɛ̄s} &  \textit{wɛ̄n}\\
2 & \textit{ɣɔ́n} & ‘children’   & \textit{ɣɔ́n} &  \textit{ə̀ɣɔ́m} &  \textit{ə̀ɣíɛ̀} &  \textit{ə̀ɣɛ́s} &  \textit{ə̀ɣɛ́n}\\
3 & \textit{ɛ̄bléŋ} & ‘bamboo’   & \textit{ɛ̄bléŋ} &  \textit{wɔ́m} &  \textit{víɛ̀} &  \textit{wɛ́s} &  \textit{wɛ́n}\\
4 & \textit{īléŋ} & ‘bamboos’   & \textit{īléŋ} &  \textit{èyɔ́m} &  \textit{èʒíɛ̀} &  \textit{ə̀yɛ́s} &  \textit{ə̀yɛ́n}\\
5 & \textit{īʃɔ́ŋ} & ‘tooth’     & \textit{īʃɔ́ŋ} &  \textit{èyɔ́m} &  \textit{èʒíɛ̀} &  \textit{ə̀yɛ́s} &  \textit{ə̀yɛ́n}\\
6 & \textit{ɛ̄sɔ́ŋ} & ‘teeth’     & \textit{ɛ̄sɔ́ŋ} &  \textit{ə̀ɣɔ́m} &  \textit{ə̀yíɛ̀} &  \textit{ə̀ɣɛ́s} &  \textit{ə̀ɣɛ́n}\\
7 & \textit{kēkém} & ‘crab’     & \textit{kēké\`{m}} &  \textit{kɔ́m} &  \textit{kīɛ̀} &  \textit{kɛ́s} &  \textit{kɛ́n}\\
8 & \textit{ēbkém} & ‘crabs’    & \textit{ēbké\`{m}} &  \textit{wɔ́m} &  \textit{vīɛ̀} &  \textit{wɛ́s} &  \textit{wɛ́n}\\
9 & \textit{ɲâm} & ‘animal’     & \textit{ɲàm} &  \textit{yɔ̄m} &  \textit{ʒīɛ̀} &  \textit{yɛ̄s} &  \textit{yɛ̄n}\\
10 & \textit{ɲámsə̄} & ‘animals’ & \textit{ɲâm} &  \textit{sɔ́m} &  \textit{ʃíɛ̀} &  \textit{sɛ́s} &  \textit{sɛ́n}\\
13 & \textit{tə̄bɨ́ì} & ‘kolanuts’& \textit{tə̄bɨ́ì} &  \textit{tɔ́m} &  \textit{tíɛ̀} &  \textit{tɛ́s} &  \textit{tɛ́n}\\
19 & \textit{fə̄nə̂n} & ‘bird’    & \textit{fə̄nə̂n} &  \textit{fɔ́m} &  \textit{fíɛ̀} &  \textit{fɛ́s} &  \textit{fɛ́n}\\
6a & \textit{\={m}nə̂n} & ‘birds’& \textit{\={m}nə̂n} &  \textit{mɔ̄m} &  \textit{mīɛ̀} &  \textit{mɛ̄s} &  \textit{mɛ̄n}\\
% =======
% 1 & \textit{wān} & ‘child’ & \textit{wāǹ & wɔ̄m & vīɛ̀ & wɛ̄s & wɛ̄n}\\
% 2 & \textit{ɣɔ́n} & ‘children’ & \textit{ɣɔ́n & ə̀ɣɔ́m & ə̀ɣíɛ̀ & ə̀ɣɛ́s & ə̀ɣɛ́n}\\
% 3 & \textit{ɛ̄bléŋ} & ‘bamboo’ & \textit{ɛ̄bléŋ & wɔ́m & víɛ̀ & wɛ́s & wɛ́n}\\
% 4 & \textit{īléŋ} & ‘bamboos’ & \textit{īléŋ & èyɔ́m & èʒíɛ̀ & ə̀yɛ́s & ə̀yɛ́n}\\
% 5 & \textit{īʃɔ́ŋ} & ‘tooth’ & \textit{īʃɔ́ŋ & èyɔ́m & èʒíɛ̀ & ə̀yɛ́s & ə̀yɛ́n}\\
% 6 & \textit{ɛ̄sɔ́ŋ} & ‘teeth’ & \textit{ɛ̄sɔ́ŋ & ə̀ɣɔ́m & ə̀yíɛ̀ & ə̀ɣɛ́s & ə̀ɣɛ́n}\\
% 7 & \textit{kēkém} & ‘crab’ & \textit{kēké\`{m} & kɔ́m & kīɛ̀ & kɛ́s & kɛ́n}\\
% 8 & \textit{ēbkém} & ‘crabs’ & \textit{ēbké\`{m} & wɔ́m & vīɛ̀ & wɛ́s & wɛ́n}\\
% 9 & \textit{ɲâm} & ‘animal’ & \textit{ɲàm & yɔ̄m & ʒīɛ̀ & yɛ̄s & yɛ̄n}\\
% 10 & \textit{ɲámsē} & ‘animals’ & \textit{ɲâm & sɔ́m & ʃíɛ̀ & sɛ́s & sɛ́n}\\
% 13 & \textit{tēbɨ́í} & ‘kolanuts’ & \textit{tə̄bɨ́ì & tɔ́m & tíɛ̀ & tɛ́s & tɛ́n}\\
% 19 & \textit{fēnún} & ‘bird’ & \textit{fə̄nə̂n & fɔ́m & fíɛ̀ & fɛ́s & fɛ́n}\\
% 6a & \=\textit{{m}nún} & ‘birds’ & \=\textit{{m}nə̂n & mɔ̄m & mīɛ̀ & mɛ̄s & mɛ̄n}\\
% >>>>>>> c6bb032e1da3fe3d1db9970bff3123d1a38c78d3
\lspbottomrule
\end{tabularx}
% \todo[inline]{Check version conflict}
\end{table}

  
\ili{Oku} third person possessives are again quite different, as seen in the forms in \tabref{extab:grassfields:8} compared with those from \ili{EGB} in \tabref{extab:grassfields:5}.

\begin{table}  
\caption{Oku 3p possessives for each Oku noun class}
\label{extab:grassfields:8}  
\fittable{
\begin{tabular}{lllllllllll}
\lsptoprule
         {cl.} &  noun &  gloss & \multicolumn{2}{c}{  ‘his/her’} &  & \multicolumn{2}{c}{  ‘their’} &  & \multicolumn{2}{c}{ ‘his/her (log.)'}\\
\midrule          
% <<<<<<< HEAD
1 & \textit{wān} & ‘child’ & \textit{wāǹ} &  \textit{wēǹ} &   &  \textit{wāǹ} &  \textit{ɣēǹ} &   &  \textit{wāǹ} &  \textit{vī}\\
2 & \textit{ɣɔ́n} & ‘children’ & \textit{ɣɔ́n} &  \textit{ə́ wēǹ} &   &  \textit{ɣɔ́n} &  \textit{ə́ ɣēǹ} &   &  \textit{ɣɔ́n} &  \textit{èɣí}\\
3 & \textit{ɛ̄bléŋ} & ‘bamboo’ & \textit{ɛ̄bléŋ} &  \textit{ə́ wēǹ} &   &  \textit{ɛ̄bléŋ} &  \textit{ə́ ɣēǹ} &   &  \textit{ɛ̄bléŋ} &  \textit{èví}\\
4 & \textit{īléŋ} & ‘bamboos’ & \textit{īléŋ} &  \textit{ə́ wēǹ} &   &  \textit{īléŋ} &  \textit{ə́ ɣēǹ} &   &  \textit{īléŋ} &  \textit{èʒí}\\
5 & \textit{īʃɔ́ŋ} & ‘tooth’ & \textit{īʃɔ́ŋ} &  \textit{ə́ wēǹ} &   &  \textit{īʃɔ́ŋ} &  \textit{ə́ ɣēǹ} &   &  \textit{īʃɔ́ŋ} &  \textit{èʒí}\\
6 & \textit{ɛ̄sɔ́ŋ} & ‘teeth’ & \textit{ɛ̄sɔ́ŋ} &  \textit{ə́ wēǹ} &   &  \textit{ɛ̄sɔ́ŋ} &  \textit{ə́ ɣēǹ} &   &  \textit{ɛ̄sɔ́ŋ} &  \textit{èƒí}\\
7 & \textit{kēkém} & ‘crab’ & \textit{kēkém} &  \textit{ə́ wēǹ} &   &  \textit{kēkém} &  \textit{ə́ ɣēǹ} &   &  \textit{kēkém} &  \textit{èkí}\\
8 & \textit{ēbkém} & ‘crabs’ & \textit{ēbkém} &  \textit{ə́ wēǹ} &   &  \textit{ēbkém} &  \textit{ə́ ɣēǹ} &   &  \textit{ēbkém} &  \textit{èví}\\
9 & \textit{ɲâm} & ‘animal’ & \textit{ɲàm} &  \textit{wēǹ} &   &  \textit{ɲàm} &  \textit{ɣēǹ} &   &  \textit{ɲàm} &  \textit{ʒī}\\
10 & \textit{ɲámsē} & ‘animals’ & \textit{ɲâmsē} &   \textit{wēǹ} &   &  \textit{ɲâmsē} &   \textit{ɣēǹ} &   &  \textit{ɲâmsē} &  \textit{èsí}\\
13 & \textit{tēbɨ́í} & ‘kolanuts’ & \textit{tēbɨ́í} &  \textit{ə́ wēǹ} &   &  \textit{tēbɨ́í} &  \textit{ə́ ɣēǹ} &   &  \textit{tēbɨ́í} &  \textit{tí}\\
19 & \textit{fēnún} & ‘bird’ & \textit{fēnún} &  \textit{ə́ wēǹ} &   &  \textit{fēnún} &  \textit{ə́ ɣēǹ} &   &  \textit{fēnún} &  \textit{fí}\\
6a & \textit{\={m}nún} & ‘birds’ & \textit{\={m}nún} &  \textit{mè wēǹ} &   &  \textit{\={m}nún} &  \textit{mè wēǹ} &   &  \textit{\={m}nún} &  \textit{mɛ̀mī}\\
% =======
% 1 & \textit{wān} & ‘child’ & \textit{wāǹ & wēǹ &  & wāǹ & ɣēǹ &  & wāǹ & vī}\\
% 2 & \textit{ɣɔ́n} & ‘children’ & \textit{ɣɔ́n & ə́ wēǹ &  & ɣɔ́n & ə́ ɣēǹ &  & ɣɔ́n & èɣí}\\
% 3 & \textit{ɛ̄bléŋ} & ‘bamboo’ & \textit{ɛ̄bléŋ & ə́ wēǹ &  & ɛ̄bléŋ & ə́ ɣēǹ &  & ɛ̄bléŋ & èví}\\
% 4 & \textit{īléŋ} & ‘bamboos’ & \textit{īléŋ & ə́ wēǹ &  & īléŋ & ə́ ɣēǹ &  & īléŋ & èʒí}\\
% 5 & \textit{īʃɔ́ŋ} & ‘tooth’ & \textit{īʃɔ́ŋ & ə́ wēǹ &  & īʃɔ́ŋ & ə́ ɣēǹ &  & īʃɔ́ŋ & èʒí}\\
% 6 & \textit{ɛ̄sɔ́ŋ} & ‘teeth’ & \textit{ɛ̄sɔ́ŋ & ə́ wēǹ &  & ɛ̄sɔ́ŋ & ə́ ɣēǹ &  & ɛ̄sɔ́ŋ & èƒí}\\
% 7 & \textit{kēkém} & ‘crab’ & \textit{kēkém & ə́ wēǹ &  & kēkém & ə́ ɣēǹ &  & kēkém & èkí}\\
% 8 & \textit{ēbkém} & ‘crabs’ & \textit{ēbkém & ə́ wēǹ &  & ēbkém & ə́ ɣēǹ &  & ēbkém & èví}\\
% 9 & \textit{ɲâm} & ‘animal’ & \textit{ɲàm & wēǹ &  & ɲàm & ɣēǹ &  & ɲàm & ʒī}\\
% 10 & \textit{ɲámsē} & ‘animals’ & \textit{ɲâmsē &  wēǹ &  & ɲâmsē &  ɣēǹ &  & ɲâmsē & ɛ̀sí}\\
% 13 & \textit{tēbɨ́í} & ‘kolanuts’ & \textit{tēbɨ́í & ə́ wēǹ &  & tēbɨ́í & ə́ ɣēǹ &  & tēbɨ́í & tí}\\
% 19 & \textit{fēnún} & ‘bird’ & \textit{fēnún & ə́ wēǹ &  & fēnún & ə́ ɣēǹ &  & fēnún & fí}\\
% 6a & \=\textit{{m}nún} & ‘birds’ & \=\textit{{m}nún & mè wēǹ &  & \={m}nún & mè wēǹ &  & \={m}nún & mɛ̀mī}\\
% >>>>>>> c6bb032e1da3fe3d1db9970bff3123d1a38c78d3
\lspbottomrule
\end{tabular}
}
% \todo[inline]{check version conflict}
\end{table}


As indicated, in the third person singular, \ili{Oku} distinguishes both anaphoric and logophoric possessive pronouns, the latter cognate with the \ili{EGB} pronominal forms seen above in \tabref{extab:grassfields:5}. This \ili{WGB} pattern was already noted by Voorhoeve:

\begin{quote}
Une comparaison entre les deux types de langues met en évidence que le pronom logophorique sg correspond avec le pronom anaphorique sg dans les langues [EGB] sans pronom logophorique. (\citealt[192]{Voorhoeve1980logophorique}, describing \ili{Ngwo}, a \ili{Momo} language—see \sectref{sec:grassfields:5}).
\end{quote}

\noindent
As also noted, instead of a uniform L tone \textit{ə̀-}, an associative marker ‘of’ occurs between the noun and third person “pronoun”: \textit{ə̀-} after class 1, \textit{sé-} after class 10, \textit{mè-} after class 6a, and \textit{ə́\nobreakdash-}. This follows the same pattern as in ‘Noun\textsubscript{1} of Noun\textsubscript{2}’ genitive constructions in \tabref{extab:grassfields:9}, which is greatly simplified compared to other \ili{Ring} languages:

\begin{table}[t]
\caption{Oku ‘Noun\textsubscript{1} of Noun\textsubscript{2}’ genitive constructions}
    \label{extab:grassfields:9} 
\fittable{    
\begin{tabular}{l r@{\,}c@{\,}l l l r@{\,}c@{\,}l l l}
\lsptoprule
        cl. &    \multicolumn{3}{c}{ \textit{noun\textsubscript{1}} \textit{of noun\textsubscript{2}}} &     &  cl. & \multicolumn{3}{c}{ \textit{noun\textsubscript{1}} \textit{of} \textit{noun\textsubscript{2}}} & \\
\midrule
1  &  \textit{wān} &  \textit{ə̀} &  \textit{kèkɔ̀s}  & ‘child of slave’  &  2 & \textit{ɣɔ́n} &  \textit{ə́} &  \textit{kèkɔ̀s} & ‘children of slave’\\
3  &  \textit{ɛ̄bléŋ} &  \textit{ə́} &  \textit{kèkɔ̀s} & ‘bamboo of slave’  &  4 & \textit{īléŋ} &  \textit{ə́} &  \textit{kèkɔ̀s} & ‘bamboos of slave’\\
5  &  \textit{īʃɔ́ŋ} &  \textit{ə́} &  \textit{kèkɔ̀s}  & ‘tooth of slave’  &  6 & \textit{ɛsɔ́ŋ} &  \textit{ə́} &  \textit{kèkɔ̀s} & ‘teeth of slave’\\
7  &  \textit{kēkém} &  \textit{ə́} &  \textit{kèkɔ̀s} & ‘crab of slave’  &  8 & \textit{ēbkém} &  \textit{ə́} &  \textit{kèkɔ̀s} & ‘crabs of slave’\\
9  &  \textit{ɲàm} &  \textit{ə̀} &  \textit{kèkɔ̀s}   & ‘animal of slave’  &  10 & \textit{ɲám} &  \textit{sē} &  \textit{kèkɔ̀s} & ‘animals of slave’\\
13  &  \textit{tēɣúm} &  \textit{ə́} &  \textit{kèkɔ̀s} & ‘eggs of slave’  &   &  &  &  & \\
19  &  \textit{fēnún} &  \textit{ə́} &  \textit{kèkɔ̀s} & ‘bird of slave’  &  6a & \multicolumn{2}{c}{\textit{\={m}nún   mè}} & \textit{kèkɔ̀s}& ‘birds of slave’\\
\lspbottomrule
\end{tabular}
}
   \end{table}
   
  As in other African languages, logophoric pronouns refer back to person(s) reporting indirect discourse (/\textit{yi}/ ${\rightarrow}$ \textit{ʒi}):

  
\let\eachwordtwo=\itshape
\let\eachwordthree=\itshape
\let\eachwordfour=\itshape
\ea%10
    \label{ex:grassfields:10}
\ea  
Subj:\\
\glll   èb   sōí   gē    èb   gwí   yè  {\upshape\small ‘he\textsubscript{i} says that he\textsubscript{j}/she\textsubscript{j} is coming’}\\
       èb   sōí   gē    ʒī    gwí   yè  {\upshape\small  ‘he\textsubscript{i} says that he\textsubscript{i} (\textsc{log}) is coming’}\\ 
      s/he say  that \textsc{pron} come \textsc{prog}\\
  
\ex  
Obj:\\
\glll èb   sōí   gē  mɛ  ne   lɔ̂  yɛ̄n  wīǹ \parbox{1mm}{\mbox{\upshape\small  ‘he\textsubscript{i} says that I saw him\textsubscript{j}/her\textsubscript{j}’}}\\
      èb   sōí   gē  mɛ  ne   lɔ̂  yɛ̄n   ʒi \parbox{1mm}{\mbox{\upshape\small  ‘he\textsubscript{i} says that I saw him\textsubscript{i} (\textsc{log})’}}\\
      s/he say that \textsc{I}  \textsc{past}  \textsc{asp} see \textsc{pron}\\

\ex  
Poss:\\
\gllll èb sōí gē ʒī yɛ́nə́ kēkém ə́~wīǹ \parbox{1mm}{\mbox{\upshape\small ‘he\textsubscript{i} says that he\textsubscript{i} sees his\textsubscript{j} crab (cl. 7)’}}\\
        èb sōí gē ʒī yɛ́nə́ ēbkém ə́~wīǹ \parbox{1mm}{\mbox{\upshape\small ‘he\textsubscript{i} says that he\textsubscript{i} sees his\textsubscript{j} crabs (cl. 8)’}}\\
        èb sōí gē ʒī yɛ́nə́ kēkém ə̀kí \parbox{1mm}{\mbox{\upshape\small ‘he\textsubscript{i} says that he\textsubscript{i} sees his\textsubscript{i} (\textsc{log}) crab (cl. 7)’}}\\
        èb sōí gē ʒī yɛ́nə́ ēbkém ə̀ví \parbox{1mm}{\mbox{\upshape\small ‘he\textsubscript{i} says that he\textsubscript{i} sees his\textsubscript{i} (\textsc{log}) crabs (cl. 8)’}}\\
\z
\z

\let\eachwordtwo=\upshape
\let\eachwordthree=\upshape
\let\eachwordfour=\upshape

\noindent
While the reconstructed 3sg. \textit{*-í} pronoun serves both an anaphoric and logo\-pho\-ric function in \ili{EGB}, the innovated third person anaphoric pronouns in the \ili{Ring} languages have clearly replaced the inherited \textit{*-í} forms (as will be seen again in the \ili{Momo} languages in \sectref{sec:grassfields:5}). But where did the new pronouns come from, and why?

In order to get a fuller picture, relevant comparative data from different \ili{Ring} languages are presented in \tabref{extab:grassfields:11} on the next page (logophors in parentheses are identical to the anaphors). As can be observed, in most of their paradigm, \ili{Ring} languages have replaced the inherited third person anaphoric pronouns seen in \ili{EGB} in \tabref{extab:grassfields:5} above. Class 2 ‘they, them’ is often derived from the singular, at least in some cases, e.g. \ili{Babanki} \textit{və̀-wɛ́n{\textasciigrave}-ə́}; \ili{Babungo} \textit{və̀-ŋw}\textit{ə́}  > \textit{və̌ŋ} (?). In addition we can observe the following:

\largerpage[3]
\begin{enumerate}
\item[(i)]  Neither ‘body’ nor  ‘this/these’ provides a perfect phonetic source for the third person sg. pronoun.

\item[(ii)]  The root for ‘body’ is identical in \ili{Babanki} and \ili{Lamnso’}; however, class 3 ‘body’ would require new forms to be developed in the other classes (its own plural is in class 4 (\ili{Aghem}, \ili{Kom}), 13 (\ili{Aghem}, \ili{Kom}) or 6a (\ili{Mbizinaku}, \ili{Bafmeng}, \ili{Bum}, \ili{Weh})).

\item[(iii)]  Class 1 ‘this’ is identical to the third person singular pronoun in \ili{Aghem}; both it and ‘body’ work for \ili{Lamnso'}.

\item[(iv)]  Neither works for \ili{Oku} \textit{wēǹ} (with ML tone), where the 'word for ‘body’ is \textit{ēbwún} and demonstratives have the vowel /i/ and L tone.
\end{enumerate}

\begin{table}
\caption{Oku demonstratives with vowel /i/ and L tone} 
\label{tab:grassfields:extra1}
\begin{tabularx}{\textwidth}{llllllllllll}
\lsptoprule
1 & \textit{vìn}  &  2 & \textit{yìn}  &  7 & \textit{kìn}  &  8 & \textit{vìn}  &  19 & \textit{fìn}  &  6a & \textit{mìn}\\
3 & \textit{vìn}  &  4 & \textit{ʒìn}  &  9 & \textit{ʒìn}  &  10 & \textit{ʃìn}  &    &    &  \\
5 & \textit{yìn}  &  6 & \textit{kìn}  &  13 & \textit{tìn}  &    &  \multicolumn{5}{l}{      (w, y ${\rightarrow}$ v, ʒ  / {\longrule} i)}\\
\lspbottomrule
\end{tabularx}
\end{table}

\clearpage 


\begin{sidewaystable}[p]
\caption{Comparative data from different Ring languages} 
\label{extab:grassfields:11}
\fittable{
\begin{tabular}{llllllp{1cm}llp{1cm}lll}
\lsptoprule
  & \multicolumn{2}{c}{   cl.1/2 pronouns} & {} & \multicolumn{3}{c}{   cl.1 demonstratives} & \multicolumn{3}{c}{   class 1 3sg.} & \multicolumn{3}{c}{ \textit{sg. logophorics}}\\
{  Language} & {   3sg.} & {   3pl.} & {   ‘body’} & {   ‘n.s.’} & {   ‘n.h.’} & {   ‘far’} & {   subj.} & {   obj.} & {   poss.} & {   subj.} & {   obj.} & {   poss.}\\
\midrule

{\ili{Aghem}} &\textit{wɨ́n} &  \textit{ɣé} &  \textit{ówé} &  \textit{wɨ́n} &  \textit{vʊ̀} &  \textit{òvʊ̂} &  \textit{ò} &  \textit{wɨ́n} &  \textit{wɨ́n} &  \textit{é} &  \textit{ɣé} &  \textit{ɣé}\\

{\ili{Babanki}} & \textit{wɛ́n} & \textit{və̀wɛ́{\downstep}nə́} & \textit{ə̀wɛ́n`{\textasciigrave}} & \textit{ə̀ɣɛ́n} & \textit{ʒíá}  &  \textit{əʒì}  &  \textit{ɣə̀}  &  \textit{wɛ́n}  &  \textit{wɛ́n}  &   &  \textit{yi} & \\

\ili{Babessi} & \textit{yǐ}  &  \textit{ŋwɛ̂}  &  \textit{ŋú/ŋwɛ́nə̀}  &  \textit{ŋwɛ̂}  &  \textit{yí:}  &   &  \textit{yǐ}  &  \textit{ŋə́}  &  \textit{yí:}  &  \textit{yí} &  & \\

{\ili{Babungo}} & \textit{ŋwə́}  &  \textit{və̌ŋ}  &  \textit{ŋwá̰}  &  \textit{ŋwə̏}  &  \textit{ɣɔ́\={}}   &  \textit{wî}  &  \textit{ŋwə́}  &  \textit{ŋwə́}  &  \textit{wí}  &  \textit{yì}  &  \textit{yì}  &  \textit{(wí)}\\

\ili{Bafmeng} & \textit{vɛ̄ŋ}  &  \textit{ɣə̄nə̂}  &  \textit{ēwíŋ}  &  \textit{vìŋ}  &  \textit{vē {\textasciigrave}}  &  \textit{vî}  &  \textit{èvə́}  &  \textit{vɛ̀ŋ}  &  \textit{vɛ̄ŋ}  &  \textit{èzə́}  &  \textit{(vɛ̄ŋ)}  &  \textit{({vɛ̄ŋ})}\\

\ili{Bum} & \textit{wūn}  &  \textit{ɣə̄nV´}  &  \textit{ūwún}  &  \textit{wùnā{\textasciigrave}}  &  \textit{wɛ̄ {\textasciigrave}}  &  \textit{wɔ̄kɔ̂} &  &  &  &  &  & \\

\ili{Isu} & \textit{{\textasciigrave}wé}  &  \textit{ɣú{\textasciitilde}wú}  &  \textit{úwéé}  &  \textit{{\textasciigrave} wɔ́}  &  \textit{{\textasciigrave} wíy}  &  \textit{{\textasciigrave} wíy}  &  \textit{ù}  &  \textit{{\textasciigrave} wé}  &  \textit{{\textasciigrave} wé}  &  \textit{ìɣé}  &  \textit{ìɣé}  &  \textit{ìɣé}\\

{\ili{Kom}} & \textit{ŋwēn}  &  \textit{àŋə̄ná}  &  \textit{ə̄wúīn}  &  \textit{wɛ̄ǹ}  &  \textit{vɨ̄ \`{}}   &  \textit{vɨ̄ɨ̂}  &  \textit{wù}  &  \textit{ŋwēn}  &  \textit{ŋwēn}  &  \textit{yī}  &  \textit{yī}  &  \textit{vɨ}\\

{\ili{Lamnso’}} & \textit{wūn{\textasciigrave}}  &  \textit{áwūnē´}  &  \textit{wūń}  &  \textit{və̀n}  &  \textit{və̀y}  &  \textit{və̀sə̄}  &  \textit{wù}  &  \textit{wūǹ}  &  \textit{və̄ˊ}  &  \textit{wùn} &  & \\

\ili{Mbizinaku} & \textit{wēìn}  &  \textit{āŋéìn}  &  \textit{ə̄wóīn}  &  \textit{və̀ìn}  &  \textit{vʊ̄{\textasciigrave}}  &   &   &   &  \textit{wēìn} &  &  & \\

{\ili{Oku}} & \textit{wēǹ}  &  \textit{ɣēnè}  &  \textit{ēbwún}  &  \textit{vìn}  &  \textit{vì}  &  \textit{víì}  &  \textit{èb}  &  \textit{wēǹ}  &  \textit{wēǹ}  &  \textit{ʒī}  &  \textit{ʒī}  &  \textit{vī}\\

{\ili{Weh}} & \textit{{\textasciigrave}wí}  &  \textit{ɣɨ̂}  &  \textit{úwú{\downstep}´}  &  \textit{wə́n}  &   &  \textit{wɛ́í}  &  \textit{tə́ˋ}  &  \textit{{\textasciigrave}wí}  &  \textit{{\textasciigrave}wí}  &  \textit{í}&  & \\

\ili{Zoa} & \textit{{\textasciigrave}wí}  &  \textit{ɣ\textsuperscript{j}í}  &  \textit{úƒú{\downstep}´}  &   &   &  \textit{wî}  &  \textit{má}  &  \textit{ˋwɒ́ŋ}  &   &   &   &  \\

\lspbottomrule
\end{tabular}
}
\end{sidewaystable}
\clearpage


\noindent
Could this mean that the demonstrative became a pronoun in one language which had the appropriate vowel and tone, and then spread to the other languages? All of this could have diffused areally (cf. the discussion of \ili{Noni} in \REF{ex:grassfields:14} below).


Let us assume the historical derivation Dem > Pron for the present discussion. Why were the new pronouns innovated? The following observations may serve as hints:

\begin{enumerate}
\item[(i)]  Dem > Pron first affects Obj (object, oblique and independent pronouns), then possessive or subject in the following stages:

  Stage 1:  Obj  :  \ili{Lamnso’}

  Stage 2:  Obj + Poss  :  \ili{Aghem}, \ili{Babanki}, \ili{Bafmeng}, \ili{Kom}, \ili{Oku}

  Stage 3:  Obj + Subj  :  \ili{Babungo}

  Stage 4:  Obj + Poss + Subj  :  no \ili{Ring} language yet attested

\item[(ii)]  It is the demonstrative ‘this’ that is involved—vs. ‘that’ (near hearer) or ‘that’ (remote); cf \sectref{sec:grassfields:7}.

\item[(iii)]  The same languages develop logophoric marking—starting first with subject position:

  Stage 1:  Subj  :  \ili{Bafmeng}

  Stage 2:  Subj + Obj  :  \ili{Babungo}

  Stage 3:  Subj + Obj + Poss  :  \ili{Aghem}, \ili{Kom}, \ili{Oku}
\end{enumerate}

The hypothesis that we can therefore advance is that both innovations have to do with marking co- vs. non-co-referential pronouns. As is well-known, ‘this’ is often an introducer of a new referent (non-coferential): \textit{I ran into this guy and he said...} (vs. ‘that’: \textit{I don’t like that guy!}). In addition, non-subject (Obj) pronouns are more likely to be “new” than subjects (hence non-coreferential?). It therefore should be the case that the demonstrative would become a pronoun first in non-subject positions. Contrasting with this, logophoric pronouns are coreferential, systematically opposed to coreferential third persons, and are best suited for subject position (= most “given”, referring back to the speaker).

However, at least two systems do not fit the pattern. The first, seen in \tabref{extab:grassfields:12}, is the curious “reverse” case of \ili{Lamnso’} third person singular subject: \textit{wù} (anaphoric) vs. \textit{wùn} (logophoric).

\begin{table}
\caption{Lamnso’ personal pronouns including logophoric}
 \label{extab:grassfields:12}    
\fittable{
\begin{tabular}{lllllllll}
\lsptoprule
           & {} & {  1sg} & {  2sg} & {  3sg} & {  Log} & {  1pl} & {  2pl} & {  3pl}  \\
\midrule
 subject  &  \textit{} &  \textit{\textit{ḿ, mo-}} &  \textit{à´, wō-} &  \textit{wù} &  \textit{wùn} &  \textit{vèr´} &  \textit{vèn´} &  \textit{vé-, á }\\
 object  &  \textit{} &  \textit{\textit{mō´}} &  \textit{wò} &  \textit{wūn{\textasciigrave}} &  \textit{} &  \textit{vēr´} &  \textit{vēn´} &  \textit{áwūnē´  }\\
 cl. 1 poss.  &  \textit{} &  \textit{\textit{wōm´}} &  \textit{wò} &  \textit{və̄} &  \textit{} &  \textit{wōr´} &  \textit{wōn´} &  \textit{wōvˊ}  \\
 cl. 2. poss.  &  \textit{} &  \textit{\textit{vém}} &  \textit{vé{\textasciigrave}} &  \textit{və́} &  \textit{} &  \textit{vér} &  \textit{vén} &  \textit{vév  }\\
  &  \textit{{}} &  \textit{\textit{-ém}} &   \textit{-é{\textasciigrave}} &  \textit{{ -ə́}} &  \textit{{}} &  \textit{{ -ér}} &  \textit{{ -én}} &  \textit{{ -év}  (e ${\rightarrow}$ o / w {\longrule})}\\
\lspbottomrule
\end{tabular}
} 
\end{table}

\newpage 
\noindent  
About these, \citet[Appendix II]{Grebe1982} says the following:

\begin{quote}
  /\textit{wùn}/ is used in speech quotation referring to original speaker.... (Appendix II, p.23)

  /\textit{vé-}/ is used in contexts where the subject pronoun receives a suffix to mark tense or mood, e.g. /\textit{vé-é}/ ‘they-past-tense’. /\textit{á}/ is used in all other contexts if the referent is impersonal, as well as for personal referents if the pronoun occurs in a relative or various other subordinate clauses. A third form, /\textit{áwūnē}/ ‘they’ is always personal and occurs only in independent clauses (Appendix II, p.7)

\end{quote}


  Even more curious is \ili{Noni}, a \ili{Beboid} (Bantoid) language spoken near \ili{Lamnso’} and \ili{Oku} \citep[15, 20]{Hyman1981}, where the logophoric pronouns resemble the demonstrative forms in the \ili{Ring} languages:

\begin{table}
 \caption{Noni personal pronouns including logophoric}
    \label{extab:grassfields:13}
\fittable{
\begin{tabular}{lllllllllll}
\lsptoprule
         & 1sg & 2sg & 3sg\textsubscript{j} & 3sg\textsubscript{i} & Log & 1pl & 2pl & 3pl & Log\\
         \midrule
 subj/obj     &  \textit{\textit{mē}} &  \textit{wɔ̀} &  \textit{wvù} &  \textit{---} &  \textit{wēn} &  \textit{bèsèn} &  \textit{bèn} &  \textit{bɔ́} &  \textit{bɔ̀wēn}\\
 cl. 1 poss.  &  \textit{\textit{wɛ̀m}} &  \textit{wɔ̀} &  \textit{wè} &  \textit{---} &  \textit{wēn} &  \textit{wèsèn} &  \textit{wènè} &  \textit{(wù)bɔ̌} &  \textit{bɔ̀wēn}\\
 cl. 2 poss.  &  \textit{\textit{bɛ̄m̀}} &  \textit{bōẁ} &  \textit{bêw} &  \textit{bêŋ} &  \textit{bɔ̄-wēn-ɛ́} &  \textit{bɔ̀sɛ́sɛ̀n} &  \textit{bɔ̀nɛ̂n} &  \textit{bɔ̄bɔ́ɔ́lɛ́} &  \textit{bɔ̄-bɔ̀wēn-ɛ́}	\\
 cl. 7 poss.  &  \textit{\textit{kɛ̄m̀}} &  \textit{kōẁ} &  \textit{kêw} &  \textit{kêŋ} &  \textit{ke-wēn-ɛ́} &  \textit{kèsɛ́sɛ̀n} &  \textit{kènɛ̂n} &  \textit{kēbɔ́ɔ́lɛ́} &  \textit{kē-bɔ̀wēn-ɛ́}\\

\lspbottomrule
\end{tabular}
} 
\end{table}


As I have elsewhere speculated \citep[15-16]{Hyman1981}, \ili{Noni} apparently borrowed \textit{wēn}, but got it “wrong”, allowing it in subject position (as elsewhere only in \ili{Babungo}) and developing a plural form \textit{bɔ̀-wēn}, which I have not found in \ili{Ring}:

\newpage
\let\eachwordone=\itshape\small
\let\eachwordtwo=\itshape\small
\let\eachwordthree=\upshape\small
\ea%14
    \label{ex:grassfields:14}
    \ea 
    sg.: \\
    \glll wvù dòó lɛ̄ wvù bɛ́ɛ̀ gɛ̀n fɔ̀wǎy \parbox{1mm}{\mbox{\hspace*{-3mm}\scriptsize ‘he\textsubscript{i} says that he\textsubscript{j}/she\textsubscript{j} went to market’ (today)}}\\
           wvù dòó lɛ̄ wēn bɛ́ɛ̀ gɛ̀n fɔ̀wǎy \parbox{1mm}{\mbox{\hspace*{-3mm}\scriptsize ‘he\textsubscript{i} says that he\textsubscript{i}~(\textsc{log}.) went to market’}}\\
           s/he say that \textsc{pron} \textsc{past.foc} go to.market\\

  \ex
    pl.:\\
    \glll  bɔ́ dóó lɛ̄ bɔ́    bɛ́ɛ̀ gɛ̀n fɔ̀wǎy \parbox{1mm}{\hspace*{-3mm}\mbox{\scriptsize ‘they\textsubscript{i} say that they\textsubscript{j} went to market’ (today)}}\\
           bɔ́ dóó lɛ̄ bɔ̀wēn bɛ́ɛ̀ gɛ̀n fɔ̀wǎy \parbox{1mm}{\hspace*{-3mm}\mbox{\scriptsize ‘they\textsubscript{i} say that they\textsubscript{i} (\textsc{log}.) went to market’}}\\
      they say that \textsc{pron} \textsc{past.foc} go to.market\\
\z
\z

\let\eachwordone=\itshape\normalsize
\let\eachwordtwo=\upshape\normalsize
\let\eachwordthree=\upshape\normalsize

\ili{Noni} does not provide an exact form \textit{wēn} ‘this’ > \textsc{pron}, but related \ili{Naki} comes closer \citep{Good2010}:

\begin{table}
 \caption{Noni and Naki proximate demonstratives  ‘this’}
    \label{ex:grassfields:15}
\begin{tabularx}{\textwidth}{l lll Xl lll Xl lll}
\lsptoprule
  &  \ili{Noni} &  \ili{Naki} &  &  &  &  \ili{Noni} &  \ili{Naki} &  &  &  &  \ili{Noni} &  \ili{Naki}\\
\midrule 
1 &\textit{wvù\={n}} & {wə̀n} &  &  & 6 & \textit{ɛ̄yān} & \textit{nə̂n} &  &  & 10 & \textit{yīn} & \textit{yə̂n}\\
2 & \textit{bān} & bɔ̌n &  &  & 6a & \textit{mān} & \textit{mɔ̂n} &  &  & 13 & \textit{jīn} & {---}\\
3 & \textit{wvūn} & wə̌n &  &  & 7 & \textit{kīn} & \textit{kə̂n} &  &  & 14 & \textit{bvūn} & \textit{wə̌n}\\
4 & \textit{yīn} & --- &  &  & 8 & \textit{bīn} & \textit{byə̂n} &  &  & 19 & \textit{fīn} & \textit{fyə̂n}\\
5 & \textit{jīn} & --- &  &  & 9 & \textit{yì\={n}} & \textit{yə̀n} &  &  & 18 & \textit{mvūn} & \textit{mɔ̂n}\\
\lspbottomrule
\end{tabularx} 
\end{table}

Finally, note that \ili{Weh} has generalized the associative to all possessives except first and second person singular, e.g. \textit{ndɔ́ŋ/tə̀ndɔ́ŋ} ‘horn(s)’ 9/13:

\begin{table}
 \caption{Weh possessive pronouns and generalized associative}
\label{extab:grassfields:16}
\begin{tabularx}{\textwidth}{lXlllll}
\lsptoprule
 ‘my’ &  ‘your~sg.’ &  ‘his/her’ &  ‘our’ &  ‘your pl.’ &  ‘their’\\
\midrule
\textit{ndɔ́ŋ zú{\downstep}\'{ŋ}} & \textit{ndɔ́ŋ zɯ} & \textit{ndɔ̀ŋ à wé} & \textit{ndɔ̀ŋ à sà} & \textit{ndɔ̀ŋ à ɣà} & \textit{ndɔ̀ŋ à ɣɯ́}\\
\textit{ndɔ́ŋ tú{\downstep}\'{ŋ}} & \textit{ndɔ̀ŋ tɯ̂} &  \textit{ndɔ́ŋ tə́ {\downstep}wé} & \textit{ndɔ̀ŋ tə́ sà} &  \textit{ndɔ̀ŋ tə́ ɣà} &  \textit{ndɔ́ŋ tə́  {\downstep}ɣɯ́}\\
\lspbottomrule
\end{tabularx}
\end{table}

\newpage 
\section{Momo Grassfields Bantu}\label{sec:grassfields:5}

This section will be shorter, as less material has been available to me on \ili{Momo} languages than on \ili{Ring}. The important observation to make is that new third person personal pronouns have been introduced (mostly different in form from the \ili{Ring} pronouns), including reflexives. There is considerable variation. We start with \ili{Ngamambo}, whose independent possessive pronouns are shown in \tabref{extab:grassfields:17}.

\begin{table}
 \caption{Ngamambo independent possessive pronouns}
    \label{extab:grassfields:17} 
\fittable{    
\begin{tabular}{l >{\itshape}l>{\itshape}l>{\itshape}l>{\itshape}l>{\itshape}l>{\itshape}l}     
\lsptoprule
{cl.} & \textup{‘mine’} & \textup{‘yours sg.’} & \textup{‘his/hers’} & \textup{‘ours’} & \textup{‘yours pl’} & \textup{‘theirs’}\\
\midrule 
% <<<<<<< HEAD
1& ɨ̀-wūm&     ɨ̀-wē ˋ& wū   mʌ́t& ɨ̀-wā&  ɨ̀-wə̄n& wū   mə̀- mʌ́t\\
2=8& m̀-búm&     m̀-bê& m̀bə́ mʌ́t& m̀-bá& m̀-bə́n& m̀bə́ mə̄- mʌ̌t\\
3& ɨ̀-wúm&     ɨ̀-wê& wú   mʌ́t& ɨ̀-wá& ɨ̀-wə́n&   wú   mə̄- mʌ̌t\\
6=7& ʌ̀-zúm&     ʌ̀-bê& zʌ́    mʌ́t& ʌ̀-zá & ʌ̀-zə́n& zʌ́     mə̄- mʌ̌t\\
9& ɨ̀-zúm&     ɨ̀-zē ˋ& zə̄    mʌ́t& ɨ̀-zā& ɨ̀-zə̄n& zə̄     mə̀- mʌ́t\\
10=13& ɨ̀-túm&     ɨ̀-tê& lə́     mʌ́t& ɨ̀-tá& ɨ̀-tə́n& lə́     mə̄- mʌ̌t\\
19& ɨ̀-fúm&     ɨ̀-fê& fə́     mʌ́t & ɨ̀-fá& ɨ̀-fə́n& fə́     mə̄- mʌ̌t\\
6a& m̀-búm&     m̀-bē ˋ& m̀bə̄ mʌ́t& m̀-bā& m̀-bə̄n& m̀bə̄  mə̀- mʌ́t\\
&    /-úm/&     /-ê/	  & & /-á/& /-ə́n/&  \\
% =======
% 1	& \textit{ɨ̀-wūm	&     ɨ̀-wē ˋ	& wū   mʌ́t	& ɨ̀-wā	&  ɨ̀-wə̄n	& wū   mə̀-mʌ́t}\\
% 2=8	& \textit{m̀-búm	&     m̀-bê	& m̀bə́ mʌ́t	& m̀-bá	& m̀-bə́n	& m̀bə́ mə̄-mʌ̌t}\\
% 3	& \textit{ɨ̀-wúm	&     ɨ̀-wê	& wú   mʌ́t	& ɨ̀-wá	& ɨ̀-wə́n	&   wú   mə̄-mʌ̌t}\\
% 6=7	& \textit{ʌ̀-zúm	&     ʌ̀-bê	& zʌ́    mʌ́t	& ʌ̀-zá 	& ʌ̀-zə́n	& zʌ́     mə̄-mʌ̌t}\\
% 9	& \textit{ɨ̀-zúm	&     ɨ̀-zē ˋ	& zə̄    mʌ́t	& ɨ̀-zā	& ɨ̀-zə̄n	& zə̄     mə̀-mʌ́t}\\
% 10=13	& \textit{ɨ̀-túm	&     ɨ̀-tê	& lə́     mʌ́t	& ɨ̀-tá	& ɨ̀-tə́n	& lə́     mə̄-mʌ̌t}\\
% 19	& \textit{ɨ̀-fúm	&     ɨ̀-fê	& fə́     mʌ́t 	& ɨ̀-fá	& ɨ̀-fə́n	& fə́     mə̄-mʌ̌t}\\
% 6a	& \textit{m̀-búm	&     m̀-bē ˋ	& m̀bə̄ mʌ́t	& m̀-bā	& m̀-bə̄n	& m̀bə̄  mə̀-mʌ́t}\\
% 	&    \textit{/-úm/	&     /-ê/	  	& 	& /-á/	& /-ə́n/	}&  \\
% >>>>>>> c6bb032e1da3fe3d1db9970bff3123d1a38c78d3
\lspbottomrule            
\end{tabular}
}
% \todo[inline]{Check italics and version conflict}
\end{table}


\noindent
Note the third person pronominal root \textit{/mʌ́t/} (< ?), whose plural form \textit{mə̀-mʌ́}\textit{t} has a class 2 prefix (\textit{*bə̀-}).

As seen in \tabref{extab:grassfields:18}, the above possessive pronouns occur after a noun (the noun glosses are given in \tabref{extab:grassfields:19}).

\begin{table}
\caption{Ngamambo possessive pronouns that follow the noun}
    \label{extab:grassfields:18}
\fittable{
\begin{tabular}{l >{\itshape}l>{\itshape}l>{\itshape}l>{\itshape}l>{\itshape}l>{\itshape}l}
\lsptoprule
        { cl.} & \textup{  ‘my’} & \textup{  ‘your sg’} & \textup{  ‘his/her’} & \textup{ ‘our’} & \textup{  ‘your pl’} & \textup{  ‘their’}\\
\midrule
1 & \textit{kánʌ́ {\downstep}wūm} & { kánʌ́ {\downstep}wē{\textasciigrave}} & kánʌ́ mʌ̄t & kánʌ́ {\downstep}wā & kánʌ́ {\downstep}wə̄n & kánʌ́ mə̀mʌ̌t\\
3 & \textit{ɨ̄kón wúm} & ɨ̄kón wē{\textasciigrave} & ɨ̄kón mʌ́t & ɨ̄kón wá & ɨ̄kón wə́n & {ɨ̄kón mə̀mʌ̌t}\\
7 & \textit{ʌ̄tsám ʌ́ zúm} & ʌ̄tsám ʌ́ zē{\textasciigrave} & ʌ̄tsám ʌ́ mʌ́t & ʌ̄tsám ʌ́ zá & ʌ̄tsám ʌ́ zə́n & {ʌ̄tsám ʌ́ mə̀mʌ̌t}\\
9 & \textit{gwí {\downstep}zūm} & gwí {\downstep}zē{\textasciigrave} & gwí mʌ̌t & gwí {\downstep}zā & gwí {\downstep}zə̄n & {gwí mə̀mʌ̌t}\\
10 & \textit{gwí  tūm} & gwí tē{\textasciigrave} & gwí lə́ mʌ́t & gwí  tā & gwí  tə̄n & {gwí lə̄ mə̀mʌ̌t}\\
19 & \textit{fə́kámə́ fúm} & fə́kámə́ fē{\textasciigrave} & fə́kámə́ fə́ mʌ́t & fə́kámə́ fá & fə́kámə́ fə́n & {fə́kámə́ fə́ mə̀mʌ̌t}\\
\lspbottomrule
\end{tabular}
   }
\end{table}

Shorter preposed variants exist in first and second person, but not third person, and are shown in \tabref{extab:grassfields:19}.

\begin{table}
\caption{Ngamambo shorter possessive pronouns for 1sg \& 2sg that precede the noun}
    \label{extab:grassfields:19}
\begin{tabularx}{\textwidth}{lllXXXX}
\lsptoprule
 { cl.} &  noun & {} & {  ‘my’} & {  ‘your~sg’} & {  ‘our’} & {  ‘your~pl’}\\
\midrule
1 & \textit{kánʌ́} & ‘monkey’ & \textit{mə̄ kánʌ́} & \textit{ē kánʌ́} &  \textit{ā kánʌ́} &  \textit{wə̄ kánʌ́}\\
3 & \textit{ɨ̄kón} & ‘hill’ & \textit{mə́ kón} &  \textit{ē kón} &  \textit{á kón} &  \textit{wə́ kón}\\
7 & \textit{ʌ̄tsám} & ‘home’ & \textit{mʌ́ tsám} &  \textit{ē tsám} &  \textit{á tsám} &  \textit{zə́ tsám}\\
9 & \textit{gwí} & ‘goat’ & \textit{mə̄ gwí} & \textit{ē gwí} &  \textit{ā gwí} &  \textit{wə̄ gwí}\\
10 & \textit{gwí} & ‘goats’ & \textit{túm gwí} &  \textit{tē gwí} &  \textit{tá gwí} &  \textit{tə́n gwí}\\
19 & \textit{fə́kámə́} & ‘crab’ & \textit{fúm~fə́kámə́} &  \textit{fē{\textasciigrave}~fə́kámə́} &  \textit{fá~fə́kámə́} &  \textit{fə́n~fə́kámə́}\\
\lspbottomrule
\end{tabularx}
% \todo[inline]{check italics}
\end{table}

Turning to another \ili{Momo} language, \ili{Ngie} has a full set of sg logophoric possessive pronouns, which \citet{Watters1980ngie} shows preposed to the noun in \tabref{extab:grassfields:20}. Class 4 \textit{ìɲí} is not completely certain.

\begin{table}
\caption{Ngie logophoric possessive pronouns}
\label{extab:grassfields:20}
\begin{tabularx}{\textwidth}{lXX ll lXX ll lXX}
\lsptoprule
  & \scshape [-log] & \scshape [+log] &  &  &  & \scshape [-log] & \scshape [+log] &  &  &  & \scshape [-log] & \scshape [+log]\\
\midrule
1 & \textit{ùŋgwɛ̄n} &  \textit{ùŋgwī} &  &  & 5 & \textit{ùŋgwɛ̄n }& \textit{ìnjí} &  &  & 9 & \textit{ùŋgwɛ̄n} &   \textit{ìnjī}\\
2 & \textit{ùŋgwɛ̄n} &  \textit{ùmbí} &  &  & 6 & \textit{ùŋgwɛ̄n  }& \textit{ìnjí} &  &  & 10 & \textit{ùŋgwɛ̄n} &  \textit{ìtí}\\
3 & \textit{ùŋgwɛ̄n} &  \textit{ùŋgwí} &  &  & 7 & \textit{ùŋgwɛ̄n }& \textit{ìnjí} &  &  & 13 & \textit{ùŋgwɛ̄n} &  \textit{ùfí}\\
4 & \textit{ùŋgwɛ̄n} &  \textit{ìɲí} ? &  &  & 8 & \textit{ùŋgwɛ̄n }& \textit{ùmbí} &  &  & 19 & \textit{ùŋgwɛ̄n} &  \textit{ìtí}\\
\lspbottomrule
\end{tabularx}
% \todo[inline]{check italics}
\end{table}

\tabref{extab:grassfields:21} presents a comparison of four Momo pronoun systems: 
\ili{Ngie} \citep{Elimelech1980,Watters1980ngie}, 
\ili{Ngwo} \citep{Voorhoeve1980logophorique}, 
Mundani \citep{Parker1986,Parker1989}, 
\ili{Metta} \citep{Spreda1991,Spreda2000,Mihas2009}. Different third person forms are innovated (\textit{wɛn, mə́t, ta/to}), again affecting non-subject pronouns first, sometimes only the singular (e.g. \ili{Moghamo} \textit{mə́t} ‘his/her’ vs. \textit{-ɔ́p} ‘their’). In \tabref{extab:grassfields:22} I present the class 1/2 demonstratives in six \ili{Momo} languages. (The \ili{Moghamo} and \ili{Oshie} data are due to \citealt{Stallcup1980geo}; in addition, the ‘near hearer’ forms may also/instead mean ‘the one in question, the one referred to’.)

\begin{table} 
\caption{Momo pronoun systems: Ngie, Ngwo, Mundani, and Metta}
    \label{extab:grassfields:21}
\fittable{
\begin{tabular}{l >{\itshape}l>{\itshape}l>{\itshape}l>{\itshape}l>{\itshape}l>{\itshape}l>{\itshape}l>{\itshape}l>{\itshape}l>{\itshape}l}
\lsptoprule
          \ili{Ngie} & \textup{ 1sg} & \textup{ 2sg} & \textup{ 3sg} & \textup{ log} & \textup{ refl} & \textup{ 1pl} & \textup{ 2pl} & \textup{ 3pl} & \textup{ log} & \textup{ refl}\\
\midrule
subj & mə̄ & ŋgwə̄ & wə & yī &  & \`{m}ba & \`{m}bɛ̀na & \={m}bī & \`{m}bì & \\
obj & ŋwū & yə{\textasciigrave} & ùŋwɛ̌n & yī & mā & (ŋ)gwā & (\`{ŋ})gwɛ̄n & ūŋwɛn & ŋgwī & ùmà˚\\
cl. 1 poss & ùŋwū & ùŋgwɛ̂ & ùŋgwɛ̄n & ùŋgwi & umā & ùŋgwā & ù-ŋgwɛ̄n & u\`{ŋ}gwi & u\`{ŋ}gwi & ùmǎ\\
cl. 2 poss & ùmbɨ̂ŋ & ùmbiɛ̀ & ùŋwɛ̄n & ùmbíɛ́ & ūmā & ù-mbā & ù-mbɛ̄n & u\`{m}bí & u\`{m}bí & ùmǎ\\

\tablevspace
{\ili{Ngwo}} & \textup{ 1sg} & \textup{ 2sg} & \textup{ 3sg} & \textup{ log} & \textup{ refl} & \textup{ 1pl} & \textup{ 2pl} & \textup{ 3pl} & \textup{ log} & \textup{ refl}\\
\midrule
subj & \={m}mɛ̀ & \={ŋ}gwɔ̀ & ŋgɔ̄ & \={m}bè &  & \={m}byɛ̀ & \={m}bɔ̀n & áŋgɔ̄ɔ̄ & \={m}bɔ̀ɔ̀ & \\
obj & āŋgú & awɛ̄{\textasciigrave} &  & āŋgwé & āmɔ́ & āŋgwɛ́ & āŋgɔ́n & áŋgɔ̄ɔ̄ & āŋgɔ́ɔ́ &àmɔ̀°\\
cl. 1 poss. & ŋgwā & ŋgwɛ̄{\textasciigrave} & ŋgɔ̄ & ŋgwē & mɔ̄ & ŋgwɛ̄ & ŋgwɔ̄n & áŋgɔ̄ & ŋgwɔ̄ɔ̀ & àmɔ̀\\
cl. 2 poss. & mbá & mbyɛ̂ & ŋgɔ̄ & mbé & mɔ̄ & mbyɛ́ & mbɔ́n & áŋgɔ̄ & mbɔ́ɔ̀ & âmɔ\\

\tablevspace
{\ili{Mundani}} & \textup{ 1sg} & \textup{ 2sg} & \textup{ 3sg} & \textup{ log} & \textup{ refl} & \textup{ 1pl} & \textup{ 2pl} & \textup{ 3pl} & \textup{ log} & \textup{ refl}\\
\midrule
subj & mâ & a & ta, a, e & yé &  & bâ & bɨ̂ & bɔ̂, bé, é &  & \\
obj & m & wē{\textasciigrave} & tò, we & vi & vi & wá & wɨ́ & wɔ́b, be &  & \\
cl. 1 poss & wɔ̰́ & wê & è-tò & vi & vi & wá & wɨ̰́ & wɔ́b &  & \\
cl. 2 poss & bɔ̰́ & bê & é-tò & bi ? & bi ? & bá & bɨ̰ & bɔ́b &  & \\

\tablevspace
{\ili{Metta}} & \textup{ 1sg} & \textup{ 2sg} & \textup{ 3sg} & \textup{ log} & \textup{ refl} & \textup{ 1pl} & \textup{ 2pl} & \textup{ 3pl} & \textup{ log} & \textup{ refl}\\
\midrule
subj & mə & ə̀wɔ̀ & wɨ̀ &  &  & mbǎ (tɨ̀) & mbə̄ & mbɨ̄ &  & \\
emphatic & mɔ̌ & ə̀wɔ̀ & mə́t &  &  & mbǎ & mbə̀nə & mɨ̀mə́t &  & \\
object & ə̀mɨ́ & ə̀wê & ə̀mə́t &  & ə̀wí & ə̀wá & ə̀wə́n & mɨ̀mə́t &  & ə̀wə́n \\
cl. 1 poss & ìwúm & ìwê & -mə́t &  &  & ìwə́ & ìwə́n & ìwɔ́p &  & \\
cl. 2 poss & ìmbúm & ìmbê & -mə́t &  &  & ìmbə & ìmbə́n & ímbɔ́p &  & \\

\tablevspace
{\ili{Moghamo}} & \textup{ 1sg} & \textup{ 2sg} & \textup{ 3sg} & \textup{ log} & \textup{ refl} & \textup{ 1pl} & \textup{ 2pl} & \textup{ 3pl} & \textup{ log} & \textup{ refl}\\
\midrule
cl. 1 poss & ìwúm & ìwê & {\textasciigrave}mə́t &  &  & ìwá & ìwə́n & ìwɔ́p &  & \\
cl. 2 poss & ìmbúm & ìmbê & mə́t &  &  & ìmbá & imbən & ìmbɔ́p &  & \\
\lspbottomrule
\end{tabular}
}
\end{table}


\begin{table} 
\caption{Class 1/2 demonstratives in six Momo languages}
    \label{extab:grassfields:22}
\begin{tabularx}{\textwidth}{l >{\itshape}X>{\itshape}l>{\itshape}X>{\itshape}X>{\itshape}X>{\itshape}X>{\itshape}X}
\lsptoprule
         & \multicolumn{2}{c}{\upshape   ‘near speaker’} & \multicolumn{2}{c}{ \upshape  ‘near hearer’} & \multicolumn{2}{c}{\upshape   ‘remote’} & \textup{ ‘body’}\\
\midrule
\ili{Ngamambo} & ɨ̀wɔ̄ɔ̀ & \`{m}bɔ́ɔ̀ & ɨ̀wē & \`{m}bé & ɨ̀ywīì & \`{m}bíì & ɨɲɔ́t\\
\ili{Ngie} & ù-ŋwû & u-mbɨ̂ŋ & ù-wə & u-biɛ & ù-wî & u-mbî & iɲó\\
\ili{Mundani} & wáā & báā & wū & bū & wiá & biá & əɲót\\
\ili{Metta} & wɔ̂ & mbɔ̂ & wē & mbé & wîn & mbîn & əɲót\\
\ili{Moghamo} & wɔ̂n & \`{m}bɔ̂n &  &  & wîn & \`{m}bîn & iɲə́t\\
\ili{Oshie} & wâŋ & bâŋ &  &  & wî & bî & ɛɲɛ́t\\
\lspbottomrule
\end{tabularx}
\end{table}

It appears that the new forms do not closely resemble the current demonstratives (although \textit{n}-final forms do occur), nor does the word ‘body’ look promising as a source, except that it ends in \textit{-t}, like \textit{mə́t} (\ili{PGB} *-\textit{nód}, \ili{PB} \textit{*-yʊ́tʊ}). While the original pronominal forms show up again as logophoric \textit{-í, -é }in \tabref{extab:grassfields:21}, there also are new reflexive pronouns of the shape \textit{ma} and \textit{mɔ}. This latter development is quite rare in West Africa, where nouns such as ‘body’ or ‘head’ are used as a reflexive (but cf. \ili{PB} \textit{*-méné} ({\textasciitilde} \textit{*-jéné}) ‘self, same’). Thus:

\begin{quote}
  The reflexive pronoun in \ili{Kenyang} is actually a phrase comprised of the word for ‘body’ (\textit{m-mwɛt}) and a possessive. \citep[22]{Ramirez1998} 
\end{quote}

However, note that the similar \ili{Momo} root \textit{-mə́t} is non-reflexive and non-logophoric. To conclude this section, anaphoric, logophoric and reflexive 3pl are exemplified in \ili{Ngie} \citep[48]{Watters1980ngie}; cf. \citet[174]{Voorhoeve1980logophorique} for \ili{Ngwo}:

 
\ea
    \label{ex:grassfields:23}
  \ea
  \gll  \={m}bī   ɛ̀ɣà̰ì   kwī   \={m}bī   ɛ̀kɔ̀mɔ̀   \={ŋ}wɛn  \\
    they[A]   said   that   they[A]   hit   them[A]   \\
  \glt ‘they\textsubscript{1} said that they\textsubscript{2} hit them\textsubscript{3}’\\
  
  \ex 
  \gll \={m}bī   ɛ̀ɣà̰ì   kwī   \={m}bī   ɛ̀kɔ̀mɔ̀   ŋgwī   \\
    they[A]   said   that   they[A]   hit   them[L]   \\
  \glt ‘they\textsubscript{1} said that they\textsubscript{2} hit them\textsubscript{1}’\\
  
  \ex
  \gll \={m}bī   ɛ̀ɣà̰ì   kwī   \={m}bī   ɛ̀kɔ̀mɔ̀   ùmà˚  \\
    they[A]   said   that   they[A]   hit   them[R]   \\
  \glt ‘they\textsubscript{1} said that they\textsubscript{2} hit themselves\textsubscript{2}’ \\
  
  \ex
  \gll  \={m}bī   ɛ̀ɣà̰ì   kwī   \`{m}bì   ɛ̂kɔ̀mɔ̀   \={ŋ}wɛn \\
    they[A]   said   that   they[L]   hit   them[A]   \\
  \glt  ‘they\textsubscript{1} said that they\textsubscript{1} hit them\textsubscript{2}’\\
  
  \ex
  \gll \={m}bī   ɛ̀ɣà̰ì   kwī   \`{m}bì   ɛ̂kɔ̀mɔ̀   ŋgwī \\
    they[A]   said   that   they[L]   hit   them[L]   \\
  \glt ‘they\textsubscript{1} said that they\textsubscript{1} hit themselves\textsubscript{1}’ \\
  \z
\z

Note in (\ref{ex:grassfields:23}e) that the logophoric takes precedence over the reflexive form!

\section{Beyond Grassfields Bantu}

Perhaps if we take a look outside the \ili{Grassfields Bantu} proper, there will be more hints as to where the \ili{WGB} third person pronouns came from. \tabref{extab:grassfields:24} compares pronouns and demonstrative forms from \ili{Wider Bantu} and \ili{Narrow Bantu} zone A: \ili{Basaá} \citep{Hyman2003}, \ili{Tunen} \citep{Mous2003}, \ili{Akɔɔse} \citep{Hedinger1980}, \ili{Mankon} \citep{Leroy2007}, \ili{Ejagham} \citep{Watters1981}, \ili{Tikar} \citep{Stanley1991}, \ili{Bafia} \citep{Guarisma2000}.

\begin{table}  
\caption{Pronouns and demonstratives  from Wider Bantu and Narrow Bantu}
    \label{extab:grassfields:24}
\fittable{
\begin{tabular}{l >{\itshape}l>{\itshape}l>{\itshape}l>{\itshape}l>{\itshape}l>{\itshape}l>{\itshape}l>{\itshape}l>{\itshape}l}
\lsptoprule
         & \textup{  \ili{Basaá}} & \textup{  \ili{Tunen}} & \textup{  \ili{Mankon}} & \textup{\ili{Akɔɔse}} & \textup{ \ili{Kenyang}} & \textup{  \ili{Ejagham}} & \textup{  \ili{Tikar}} & \textup{\ili{Akɔɔse}} & \textup{  \ili{Bafia}}\\
& \textup{  pron} & \textup{  pron} & \textup{  ind.pron} & \textup{  ref.} & \textup{  ‘def.art.’} & \textup{  ind.pron} & \textup{  pron} & \textup{  ‘this’} & \textup{  ref.}\\
\midrule
1sg & mɛ̀ & mìàŋó & /mè/ &  &  & \`{m}mɛ̀ & mùn &  & \\
2sg & wɛ̀ & àŋó & /ɣɔ̀/ & [r] = /d/ &  & wâ & wù &  & \\
1pl & ɓěs & bʷə̀sú & /bɯɣ´/ &  &  & ɛ̂d & ɓwiʔ &  & \\
2pl & ɓee & bʷə̀nú & /bə̀n´/ &  &  & ɛ̂n & ɓyin &  & \\
cl. 1 & ɲɛ́ & wɛ́y & zɯ́, wɛ́rə́ & àwèré & rɛ & yɛ̂ & nun & ànén & ànɛ́ɛ̀n\\
cl. 2 & ɓɔ́ & bʷə̀bú & bó, bɛ́rə́ & á{\downstep}ɓéré & bɛ́rɛ & ábɔ̌ & ɓon & ábén & ɓɛ́ɛ̀n\\
cl. 3 & wɔ́ & múit & wɛ́rə́ & m{\downstep}méré & rɛ & ḿmə́nɛ̀ & son & ḿmén & wíìn\\
cl. 4 & ŋwɔ́ & mít &  & m{\downstep}méré &  &  & yon & ḿmén & mɛ́ɛ̀n\\
cl. 5 & jɔ́ & nɛ́t & nɛ́rə́ & á{\downstep}déré & nɛ́rɛ & ńjə́nɛ̀ & yon & ádén & ɗíìn\\
cl. 6 & mɔ́ & mát & mɛ́rə́ & m{\downstep}méré & mɛ́rɛ & ḿmánɛ̀ & nun & ḿmén & mɛ́ɛ̀n\\
cl. 7 & yɔ́ & yɛ́t & zɛ́rə́ & é{\downstep}céré & rɛ &  &  & écén & kíìn\\
cl. 8 & gwɔ́ & bɛ́t & tsɛ́rə́ & á{\downstep}ɓéré & bɛ́rɛ & ḿbə́nɛ̀ &  & ábén & ɓíìn\\
cl. 9 & yɔ́ & mɛ́t & zɛ́rə́ & ècèré & rɛ & ɲ́ɲə́nɛ̀ &  & ènén & ì-nɛ́ɛ̀n\\
cl. 10 & yɔ́ & mít & tsɛ́rə́ & é{\downstep}céré & rɛ &  &  & écén & yíìn\\
cl. 13 & cɔ́ & túɛ́t &  & á{\downstep}déré & kɛ́rɛ &  &  & ádén & tíìn\\
cl. 14 &  & búɛ́t &  & á{\downstep}ɓéré &  & ḿbə́nɛ̀ &  & ábén & \\
cl. 19 & hyɔ́ & hít & fɛ́rə́ & á{\downstep}ɓéré & sɛ́rɛ & ḿfə́nɛ̀ &  & ábén & fíìn\\
\lspbottomrule
\end{tabular}
}
\end{table}


\noindent
It is striking that \ili{Tunen}, \ili{Mankon}, \ili{Akɔɔse} and \ili{Kenyang} all have a final \textit{-t} or second syllable [r] (< \textit{*d)} plus mid unrounded vowel, which is reminiscent of the \ili{Momo} pronoun \textit{mə́t}. Additional \ili{Tunen} forms from \citet[301]{Mous2003} reveal an [n], including \textit{wə̂n} ‘that one’ (cl.1) which looks more like the \ili{Ring} pronoun seen in \sectref{sec:grassfields:4}. Also to be considered is \ili{Tunen} \textit{mɛ̌l} ‘body’, where the final [l] likely reconstructs as \textit{*d,} hence strikingly similar again to \ili{Momo} \textit{mə́t}.

When comparing all of these forms to \ili{Proto-Bantu} we see just how widespread *d and *n are in these forms. Thus, Guthrie offers the \ili{Common Bantu} forms \textit{{*-nʊ́,}} \textit{*nó} ‘this’, \textit{*dá, *dé, -dɪ́á} ‘that’ and \citet{Meeussen1967} has \textit{*\nobreakdash-nóò} ‘this’, \textit{-ó} ‘that (not here)’, \textit{*-dɪá} ‘that’ (remote). There are, however, other forms (see \citealt{Weier1985}). The \ili{PB} independent and possessive pronouns reconstructed by \citet[215]{KambaMuzenga2003}  are given in \tabref{extab:grassfields:25}. As seen, these are of considerably lesser help in explaining the third person pronominal forms in \ili{WGB}. For proposed reconstructions of \ili{Proto-Bantoid} 1\textsuperscript{st} and 2\textsuperscript{nd} person pronouns, see \citealt[161]{Babaev2008}.

\begin{table}
\caption{Proto-Bantu independent and possessive pronouns \citep[115]{KambaMuzenga2003}}
    \label{extab:grassfields:25}
\begin{tabularx}{\textwidth}{XXXXX}
\lsptoprule
         & \multicolumn{2}{c}{  independent pronouns} & \multicolumn{2}{c}{  possessive pronouns}\\
& {  sg} & {  pl} & {  sg} & {  pl}\\
\midrule
 1st person & \textit{*a-a-mi-e} & \textit{*a-i-cu-e}\newline
			  \textit{*a-i-cʊ-e} 
				  & \textit{*-a-ngu-Ø}\newline
				    \textit{*-a-nga-Ø} & \textit{*-i-tu-Ø\newline *-i-ʊ-Ø}\\
\tablevspace
 2nd person & \textit{*a-u-bɪ-e} & \textit{*a-i-ɲu-e\newline *a-i-ɲʊ-e} & \textit{*-a-ku-o} & \textit{*-i-nu-Ø\newline *-i-nʊ-Ø}\\
 \tablevspace
3rd person & \textit{*a-i-ju-e\newline *a-i-jʊ-e} & \textit{*a-a-ba-o} & \textit{*-i-ndi-e\newline *-ɪ-ndi-e\newline
						*-a-ka-e\newline *-a-ku-e} 
							  & \textit{*-a-ba-o}\\
\lspbottomrule
\end{tabularx}
\end{table}


\section{Summary and conclusion}\label{sec:grassfields:7}
\largerpage
In the previous sections we have seen that \ili{EGB} languages have kept their pronouns largely intact, descending directly from pronouns reconstructed for \ili{Proto-Bantu} and likely \ili{Proto-Bantu-EGB}. On the other hand, \ili{WGB} languages have changed their pronoun systems in several ways:

\begin{enumerate}
\item[(i)]  New third person anaphoric pronouns have been innovated from two different shapes which appear to reconstruct as \textit{*-én} in \ili{Ring} vs. \textit{*-ád} in \ili{Momo}.

\item[(ii)]  Where kept, the original third person pronouns have become restricted as logophorics.

\item[(iii)]  A subset of \ili{Momo} languages have also introduced reflexive third person pronouns.

\item[(iv)]  In some languages the new pronouns resemble the demonstrative ‘this’, in others the noun ‘body’. This is hardly surprising as demonstratives are often used as pronouns in African languages (cf. \citealt[215-220]{Creissels1991}):

\begin{quote}
  \textit{\ili{Mundani}:} “Demonstratives Used as Emphatic Pronouns. Independent pronouns can be formed from certain dependent demonstrative modifiers.... The independent demonstratives are used in a range of grammatical functions: direct object, complement of the verb 'to be’, and as the second element in an associative construction.” \citep[146]{Parker1989}

  \textit{\ili{Ejagham}:} “...the \ili{Eastern Ejagham} dialect has different forms for the 3ps pronoun for the various noun classes. These forms are identical to the ‘distal’ demonstratives used in the dialect.” \citep[355]{Watters1981}
\end{quote}
\end{enumerate}


\noindent
This fits in exactly with what is known about the diachronic development of new third person pronouns elsewhere in the world:

\begin{quote}
  Most languages allow their demonstrative pronouns to be used as anaphoric pronouns. \citep[184]{Bhat2004} 
\end{quote}

\begin{quote}
 ... demonstratives are primarily the source of third-person forms. \citep[249]{Siewierska2004}
\end{quote}

The expected derivation of demonstrative > third person pronoun contrasts with observed diachronic sources of first and second person pronouns:


\begin{quote}
Whereas the known sources of first- and second-person markers tend to be nominals denoting human relationships [e.g. ‘master’, ‘lord’], those of the third person are typically words such as ‘thing’, ‘human’, ‘man’, ‘person’ or ‘body’. \citep[248]{Siewierska2004}
\end{quote}


\noindent
Although ‘body’ is specifically mentioned as a possible nominal source of third person pronouns, it would fit this second pattern if ‘body’ were the source of the new third person pronouns in \ili{WGB}.

  Although we have focused on two likely sources of the new pronouns in \ili{WGB}, demonstratives and the noun ‘body’, \citet[257]{Siewierska2004} mentions a third potential development:

\begin{quote}
  Another not uncommon way in which new person markers may develop is from conjugated auxiliary verbs in periphrastic constructions. \citep[257]{Siewierska2004}
\end{quote}

Consider in this context the \ili{Kom} reduplicative present vs. the “locative present” (cf. \textit{wɛǹ} ‘this (cl.1)’, \textit{ɣɛ̄ǹ} ‘these (cl.2)’) in \REF{ex:grassfields:26}.

\newcommand{\acutemacron}{\hspace*{-2mm}\textsf{ ᷇}\hspace{.5mm}}
\newcommand{\macronacute}{\hspace*{-2mm}\textsf{ ᷄}\hspace{.5mm}}
\newcommand{\ɯ᷄}{ɯ\macronacute}
\ea%26
    \label{ex:grassfields:26}
    \ea  \textit{wù ǹ ʒ{\ɯ᷄}ʒ{\ɯ᷄}}  ‘he is eating’

    \textit{ɣə ń ʒ{\ɯ᷄}ʒ{\ɯ᷄}}  ‘they are eating’

  \ex  \textit{wù wɛ̄n ʒ{\ɯ᷄}} ‘he’s here eating, here he is eating’  (cf. \textit{fɛ̄ǹ }‘here’)

    \textit{ɣə ɣɛ̄n ʒ{\ɯ᷄}}  ‘they’re here eating, here they are eating’

  \ex  \textit{wù vɨ̄ ʒ{\ɯ᷄} } ‘there he is eating’ (\textit{vɨ  \`{}} ‘that [near hearer]’)
\z
\z


\noindent
Basic present progressive is expressed by reduplicating the verb in (\ref{ex:grassfields:26}a). In (\ref{ex:grassfields:26}b) the near-speaker demonstrative root -\textit{ɛn} is used to give a sense of locative proximity of the action. (The initial [f] of the form \textit{fɛ̄ǹ}‘here’ is cognate with the PB locative class 16 prefix *\textit{pa-.}) (\ref{ex:grassfields:26}c) shows that other demonstratives can become involved in this construction. Since \textit{wù} and \textit{ɣə́} are not the independent pronouns in \ili{Kom}, it is unlikely that (\ref{ex:grassfields:26}b) should be interpreted as ‘he this-one eats’ etc., rather ‘he here eats’. If correct, this would mean that in addition to potential multiple sources, multiple functions of the \textsc{same} source may give rise to new third person pronouns. In some \ili{WGB} languages there are other grammatical markers having the shape \textit{Cɛn}, including the above imperfective \textit{wɛ̄n}, \textit{ɣɛ̄n}, (etc.), invariant perfective \textit{mɛ̄n}\`{} , and an invariant definite marker \textit{tɛ̄n} (cf. \ili{Oku} \textit{tēǹ} ‘inanimate third person object pronoun’). While such speculations are non-conclusive, it is hoped that the above survey will aid further research in unraveling the interesting history of third person pronouns in \ili{Grassfields Bantu} and environs.

\section*{Acknowledgements}
This is a revised version of a paper originally presented at the Niger-Congo Personal Pronouns Workshop, St. Petersburg, Sept. 13-15, 2010. I would like to thank the editor and an anonymous reviewer for helpful comments on the original manuscript. 

{\sloppy
\printbibliography[heading=subbibliography,notkeyword=this]
}

 
\end{document}