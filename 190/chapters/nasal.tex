\documentclass[output=paper]{langsci/langscibook} 
\ChapterDOI{10.5281/zenodo.1314331}
\author{Larry M. Hyman\affiliation{University of California, Berkeley}}
\title{More reflections on the nasal classes in Bantu} 
\abstract{ Although long considered to be a Bantu innovation, \citet{Miehe1991} proposed that the nasal consonants present in Bantu noun classes 1, 3, 4, 6, 9 and 10 should be reconstructed in pre-Proto-Bantu, even possibly at the Proto-Niger-Congo stage. Since there has been no comprehensive response to Miehe, the two of us organized a workshop to look at the question in more detail. In this paper I update the problem from \citet{Hyman1980} and \citet{Miehe1991}, expanding the coverage and considering various scenarios that could have led to innovation (or loss). While there have been three hypothetical reconstructions (nasal consonants, nasalized vowels, no nasal consonants), we have not yet arrived at a ``solution'' that answers the relevant questions discussed in this paper.}
\maketitle
\begin{document}
\label{sec:7}
 
  
\noindent 
The purpose of this paper is to update what we know about the distribution of nasal consonants within certain \ili{Bantu} noun class prefixes and their cognates outside of \ili{Bantu} proper. Whereas \ili{Narrow Bantu} languages have nasal consonants in the noun prefixes in classes 1, 3, 4, 6(a), 9 and 10, found also in certain \ili{Wide Bantu/Bantoid} languages, these nasals are either missing or only partially present in other \ili{Bantoid}, \ili{Benue-Congo} and further outlying subbranches of \ili{Niger-Congo}. \tabref{tab:nasal:1} presents the reconstructions which have been proposed for \ili{Proto-Bantu} \citep{Meeussen1967}, \ili{Proto-Benue-Congo} (\citealt{deWolf1971}), \ili{Proto-Eastern and Western Grassfields Bantu} \citep{Hyman1980nasalclasses}, and \ili{Proto-Gur} (Miehe et al. 2007). Where two columns appear, the first represents the shapes of noun prefixes, the second the shapes of concord prefixes on agreeing elements. (For a broader discussion of East Benue-Congo noun class systems and their use of nasal consonants as noun prefixes, see Good, Chapter 2 of this volume, and in particular §1.)

\newpage 
\begin{table}
\fittable{
\begin{tabular}{l ll ll ll ll l}
\lsptoprule
\textup{class} & \multicolumn{2}{l}{\upshape \ilit{Proto-Bantu}} & \multicolumn{2}{l}{\upshape \ilit{Proto-Benue-Congo}} & \multicolumn{2}{l}{\upshape \ilit{Proto-EGB}} & \multicolumn{2}{l}{\upshape \ilit{Proto-WGB}} & \textup{\ilit{Proto-Gur}}\\
\midrule
% \cmidrule{1-1}\cmidrule{2-3}\cmidrule{4-5}\cmidrule{6-7}\cmidrule{8-9}\cmidrule{10-10}
1  (sg.) & *mʊ̀- & *jʊ̀- & *ù-, *ò- & *gwu-, *à- & *\textsc{N}\`{} - & *ʊ̀- & *ʊ̀(\textsc{N})- & *ʊ̀- & \mbox{*ʊ, *a}\\
\tablevspace
3  (sg.) & *mʊ̀- & *gʊ́- & *ú- & *gu-, *u- & *\textsc{N}\`{} - & *ʊ́- & *ʊ́- & *ʊ́- & *ŋʊ\\
\tablevspace
4  (pl.) & *mɪ̀- & *gɪ́- & *í- & *zí- (?), í- & --- & --- & *ɪ́- & *ɪ́- & *i\\
% [Warning: Draw object ignored]
\tablevspace
6  (pl.) & *mà- & *gá- & *à & *ga-, *a- & *mə̀- & mə́- & *á- & *gá- & *ŋa\\
\tablevspace
9  (sg.) & *\textsc{N}\`{} - & *jɪ̀- & *è-, *ì- & *zì- & *\textsc{N}\`{} - & *ɪ̀- & *ɪ̀(\textsc{N})- & *ɪ̀- & \\
\tablevspace
10 (pl.) & *\textsc{N}\`{} - & *jí- & *í- & *zí- (?), í- & *\textsc{N}\`{} - & *í- & *ɪ́(\textsc{N})- & *Cí- & *ni\\
\tablevspace
6a  (-) & *mà- & *gá- & *mà-, *nà- & *ma-, *na- & *mə̀- & *mə́- & *mə- & *mə́- & *ma\\
\tablevspace
6b (pl) & *mʊ̀- & *mʊ̀- & \mbox{(?*mʊ-)} &  &  &  &  &  & *mʊ\\
\tablevspace
7  (sg.) & *kɪ̀- & *kɪ́- & *ki-, *ke- & *ki- & à- & *ɪ́- & kɪ́- & *kɪ́- & ---\\
\lspbottomrule
\end{tabular} 
}
\caption{Reconstructions of Relevant Niger-Congo Noun Class Prefixes}
\label{tab:nasal:1}
\end{table}


\noindent
As seen, only classes 6a and 6b reveal nasal prefixes through all of the above groups. In the last row I have shown the shapes of class 7 prefixes to illustrate one of the noun classes that is oral throughout \ili{Niger-Congo}.\footnote{I have changed Meeussen’s and my transcriptions for \ili{Proto-Bantu} and \ili{Proto-Grassfields Bantu}, respectively. While \ili{Adere} (\ili{EGB}) has a class 7 nominal prefix \textit{e-}, its prevocalic realization \textit{cw-} may suggest \textit{*kɪ-} for \ili{Eastern Grassfields Bantu} as well. \citep{Voorhoeve1980kenyang}.} 

Such forms as in \tabref{tab:nasal:1} immediately raise two questions: (i) Where do the nasals come from? Are they innovated in \ili{Bantu} according to the Crabb-Green\-berg hypothesis (\citealt{Crabb1965}; \citealt{Greenberg1963}) or should they be reconstructed at the level of \ili{Proto-Niger-Congo} \citep{Miehe1991}? (ii) Whichever position one takes, how does one derive the above and other distributions of nasal vs. oral noun class markers? If innovated, why should this occur only on noun markers in \ili{Bantu}? If lost, why should this occur so generally outside of \ili{Bantu}—and perhaps more mysteriously, only on concord markers within \ili{Narrow Bantu}?

It is generally assumed that cognate noun class markers can be reconstructed at the \ili{Proto-Niger-Congo} (\ili{PNC}) level. Thus consider the resemblance in forms in \tabref{tab:nasal:2}, modified from the German Wikipedia entry “Kordofanische Sprachen”, following \citet{Schadeberg1981,Schadeberg2011}. While some of these resemblances are unmistakable, it is sometimes difficult to identify cognate noun classes between the most distant sub-branches, e.g. \ili{North Atlantic} (\ili{Fula}, \ili{Sereer}) vs. \ili{Bantu} \citep[96]{Wilson1989}. While \citet{Schadeberg2011} presents \ili{Kordofanian} classes which are cognate with \ili{Bantu} classes 1, 3, 4, 6, as in \tabref{tab:nasal:3}, there are several \ilit{Kordofanian} pairings that Schadeberg is not able to identify with \ilit{Bantu} genders, e.g. Talodi \textit{ts-/ɲ-}, \textit{ŋ-/s-}, \textit{g-/n-}, \textit{d̯-/r-} etc.

\begin{table}
 \fittable{
%     \small
\begin{tabular}{l >{\itshape}l>{\itshape}l>{\itshape}l>{\itshape}l>{\itshape}l>{\itshape}l}
\lsptoprule
& 
\multicolumn{1}{p{2cm}}{\upshape Class 1\newline  Man,~Woman} & 
\multicolumn{2}{p{2cm}}{\upshape Classes 3/4\newline  Tree(s),~Wood(s)} & 
\multicolumn{2}{p{2cm}}{\upshape Classes 5/6\newline  Head(s),~Name(s)} &
\multicolumn{1}{p{2.5cm}}{\upshape Class 6a\newline Blood,~Water}\\
\midrule
\ilit{Kordofanian} & gu-, w-, b- & gu-, w-, b- & j-, g- & li-, j- & ŋu-, m- & ŋ-\\
\ilit{Atlantic} & gu- & gu- & ci- & de- & ga- & ma-\\
\ilit{Gur} & -a & -bu & -ki & -de & -a & -ma\\
\ilit{Kwa} & o- & o- & i- & li- & a- & n-\\
\ilit{Benue}-Congo & u- & u- & ti- & li- & a- & ma-\\
\ilit{Bantu} nouns & mʊ̀- & mʊ̀- & mɪ̀- & ì- & à- & mà-\\
\ilit{Bantu} agr. & (j)ʊ̀- & gʊ́- & gɪ́- & dɪ́- & gá- & má- {\textasciitilde} gá-\\
\lspbottomrule
\end{tabular}
}
\caption{Comparison of selected noun class marking across NC groups}
\label{tab:nasal:2} 
\end{table}


\begin{table}
\begin{tabularx}{\textwidth}{l >{\itshape}X>{\itshape}X>{\itshape}X >{\itshape}c >{\itshape}l>{\itshape}X>{\itshape}X>{\itshape}X>{\itshape}X}
\lsptoprule
{class} & {\ilit{Heiban}} & {\ilit{Talodi}} & {\ilit{Rashad}} &  & {} & {class} & {\ilit{Heiban}} & {\ilit{Talodi}} & {\ilit{Rashad}}\\
\cmidrule{1-4}\cmidrule{7-10}
1  (sg.) & gw- & b- & w- &  &  & ?  (sg.) & ŋ- & ŋ- & ---\\

3  (sg.) & gw- & b- & w- &  &  & ?  (pl.) & ɲ- & ɲ- & ɲ-\\

4  (pl.) & j- & g- & y- &  &  & ?  (pl.) & n- & n- & ---\\

6  (pl. of 5) & ŋw- & m- & ŋ- &  & {} & {} & {} & {} & {}\\

\lspbottomrule
\end{tabularx}

\caption{Cognate noun classes in three branches of Kordofanian \citep{Schadeberg2011}}
\label{tab:nasal:3}
\end{table}

  
\noindent
As seen in \tabref{tab:nasal:3}, the assumed cognates classes cognate with \ili{Bantu} 1, 3 and 4 do not exhibit nasal prefixes, while class 6, the plural of class 5, does. However, \ili{Kordofanian} has other unidentified nasal classes, as seen to the right in \tabref{tab:nasal:3}. Where would these nasals have come from?

As mentioned, the position of \citet{Greenberg1963} and \citet{Crabb1965} is that \ili{Bantu} innovated nasals in the noun prefixes of classes 1, 3, 4, 6, 9 and 10:

\begin{quote}
  ... \ili{Bantu} has the prefixes \textit{*mu-} and \textit{*mi-} as against \ili{Semi-Bantu} and \ili{West Sudanic} \textit{*u-} and \textit{*i-}. This is certainly a \ili{Bantu} innovation.” \citep[35]{Greenberg1963}
\end{quote}

It is significant, however, that other than the merger of class 6 \textit{*a-} (plural of class 5) with liquid class 6a \textit{*ma-}, no compelling explanation has been provided for how this might have happened. In addition, the actual situation is much more complex (cf. the extensive review in \citealt{Hyman1980nasalclasses} and below).

 
Contrasting with the Greenberg-Crabb hypothesis,   \citegen{Miehe1991} position is that the nasal prefixes should be reconstructed at the \ili{PNC} stage. Two arguments are given: (i) There are reasonable cognate nasal prefixes and frozen relics for several nasal class markers outside of \ili{Bantu}; (ii) The nasals in classes 1, 3, 4, 6, 9 and 10 are claimed to be gradually lost through erosion and possible re-prefixation.

Given the importance of these nasals in the history of \ili{Niger-Congo}, it is surprising how little reaction there has been to Miehe’s evidence, and the issue has been almost ignored. On the one hand there have been some brief reviews, e.g. \citet{Hedinger1993} and \citet{Heath1994}, from which we can assume skepticism, but open-mindedness on the part of the latter:

\begin{quote}
... the heavy preponderance of \textit{*N-} forms in the survey makes direct comparison with \ili{Bantu} \textit{*mu-} and \textit{*mi-} adventurous. Unraveling cognate relationships among noun class prefixes is treacherous because of mergers and splits among noun classes, and analogical interaction between nominal prefixes and verbal agreement markers, in addition to phonological attrition and (in some languages) contraction or elimination of the prefix system. However, M does succeed in making a strong case for an original wide distribution of nasal prefixes in the semantic domains typical of \ili{Bantu} classes, 1, 3 and 4 (among others). \citep[863]{Heath1994}
\end{quote}

One can also cite positive mention by Williamson (1989: 40; 1993: 43-44), who however accepts Stewart’s (\citeyear{Stewart1999nasal}, \citeyear*{Stewart1999explanation,Stewart2002}) \ili{PNC} reconstruction of nasalized V- prefixes instead of VN- (and presumably NV-):

\begin{quote}
Accepting Stewart’s hypothesis that the prefixes of classes 9 and 10 were originally close nasalized vowels rather than homorganic nasals, it is somewhat easier to explain why these old prefixes surface sometimes as close vowels, sometimes as homorganic nasals, and sometimes as both. \citep[44]{Williamson1993}
\end{quote}

If we include Stewart in the mix, we are left with three hypotheses concerning nasality in the indicated noun classes: proto nasal consonants, proto nasalized vowels, no nasality. In my view we have not yet arrived at a solution that answers all of the relevant questions. Those following the Greenberg-Crabb hypothesis have to address the following questions: (i) Where did the \ili{Bantu} nasals come from? This is not a problem for Miehe, who assumes they were present in \ili{PNC}. (ii) How do we account for the nasals that Miehe reports outside \ili{Bantu}? Again, this is not a problem for Miehe, as these represent retentions from \ili{PNC}. However, even if these questions disappear with Miehe’s hypothesis, other questions remain unresolved: (i) Why were the nasals lost in so much of \ili{Niger-Congo}? While we can attribute this to phonetic erosion or replacement, it would seem odd that only nasal consonants were lost in those \ili{Benue-Congo} languages which otherwise maintain CV- prefixes. (ii) Why were nasal consonants preserved in \ili{Bantu}? (iii) Why does \ili{Bantu} have nasal marking on nominals, but reconstructed non-nasal concord marking? E.g. \ili{Luganda} class 3 \textit{ò-mù-tí gù-nó} ‘this tree’; class 4 \textit{è-mì-tí gì-nó} ‘these trees’. (iv) Is the nasal/oral distinction found anywhere in \ili{Niger-Congo} outside \ili{Bantu}? If not, why not? (v) What is the relation of the two sets of marking, e.g. class 3/4 \textit{*mʊ̀-/*mɪ̀-} vs. \textit{*gʊ́-/*gɪ́-}? Why labial nasals vs. voiced oral velars? Why L tone on noun prefixes vs. H concord tone in most noun classes? Significantly, it is the concord forms which generally correspond to noun marking outside of \ili{Bantu}.

To explain the nasal vs. oral marking of classes 1, 3, 4, 6, 9 and 10 in \ili{Bantu} one might adopt one of three strategies: The first would be to reconstruct two sets of \ili{PNC} allomorphs for these classes. While this could work, it simply delays the ultimate question of why there should be two sets of markers? We would want to know how they arose in pre-\ili{PNC}, if that’s the correct historical stage. To respond to this problem we might instead reconstruct two sets of distinct noun classes, which subsequently merged, as everyone assumes in the case of class 6 \textit{*a-} (plural of class 5) and liquid/mass class 6a \textit{*ma-}. There might also have been a plural class \textit{*mʊ-} that merged with class 4 \textit{*mɪ-}. In this view, PNC likely had more noun classes than \ili{Proto-Bantu} (\ili{PB}).

A quite different proposal would be to reconstruct one set of markers which split into two sets of allomorphs in a way as yet unexplained.\footnote{It is generally assumed that the [m] of classes 1, 3, and 4 and the homorganic nasal N- ([n]?) of classes 9 and 10 have similar distributions, although possibly different origins.} In order to consider how a single set of reconstructions might have split into labial nasal vs. velar oral allomorphs, note the partial or complete complementarity between reconstructed V, N, mV and gV markers in \tabref{tab:nasal:4}.

\begin{table}
\fittable{
\small
\begin{tabular}{l llll}
\lsptoprule
{} & {Labial} & {Dental-Alveolar} & {Velar} & {Vowel/Nasal}\\
\midrule
    & \textit{*(pi-)} & \textit{ *ti- } & \textit{ *kà-, *ki-, *ku- } & \textit{ }\\ 
\ilit{PBC} & \textit{*ba-, *bi-, *bù-}  & \textit{ *li, *lu- }& \shadecell  & \\
   & \multicolumn{2}{c}\textit{{ *ma-  {\textasciitilde}  *na- }(6a)} &\shadecell  &\multirow{-3}{2cm} {\textit{*ù- (1), *ì- }(9), \mbox{\textit{*ú- }(3), \textit{*í- }(4,10),} \textit{*a-} (6)}\\

\tablevspace
\midrule
\tablevspace

     & \textit{*pʊ, *fʊ } & \textit{ *sɪ, *tʊ } & \textit{ *ka, *kʊ} & \\
\ilit{PGur} & \textit{*ba, *bi, *bʊ, *wa } & \textit{ *ɖa, *ɖɪ} &\shadecell  & \textit{*ʊ }(1), \textit{*i }(4), \textit{*a}\\
     & \textit{*ma} (6a), \textit{*mʊ }(pl.) & \textit{*nɪ} (9), \textit{*ni} (10), \textit{*na} & \textit{*ŋʊ} (3), \textit{*ŋa} (6) & \\

\tablevspace
\midrule
\tablevspace

        & \textit{*pì-, *pà-}  & \textit{ *tʊ̀-}  & \textit{ *kà-, *kɪ-, *kʊ̀-} & \\
\ilit{PB}      & \textit{*bà-, *bì-, *bʊ̀-}  & \textit{ *dɪ̀-, *dʊ̀-} & \shadecell & \textit{*ì- }(5)\\
(nouns) & \textit{*mʊ̀- }(1,3), \textit{*mɪ̀- }(4), \textit{*mà-} (6a), &\shadecell  &\shadecell  & \textit{*N}\`{} - (9,10)\\
\tablevspace
\lspbottomrule
\end{tabular}
}
\caption{Reconstructed Noun Class markers arranged by place of articulation}
\label{tab:nasal:4}
\end{table}

Among the gaps seen in \tabref{tab:nasal:4}, \ili{PB} clearly lacks voiced velars on noun prefixes.\footnote{I am ignoring cases where certain \ili{Bantu} languages exploit a concord marker in secondary derivations, e.g. \ili{Luganda} augmentative class 3 \textit{gu-}/class 6 \textit{ga-}: \textit{o-gu-tî} ‘a big tree’, pl. \textit{a-ga-tî}.} The concord prefixes, however, fill this gap: \textit{*jʊ̀-} (1), \textit{*gʊ́-} (3), \textit{*gɪ́-} (4), \textit{*gá-} (6), \textit{*jɪ̀-} (9), \textit{*jí-} (10). Perhaps \ili{Gur} \textit{*ŋV }fills in the \textit{*gV} gap, in which case \ili{Proto-Gur} may have nasalized \ili{PNC} \textit{*gʊ-} and \textit{*ga-}, which are of course identical to the \ili{PB} class 3 and 6 concords. A proposal made by the students in my  Spring (2013) Bantu and Niger-Congo seminar, inspired by the correlation between [gw, ŋw] and [b, m] in \ili{Kordofanian} (\tabref{tab:nasal:3}), is the historical derivation  \textit{*g\textsuperscript{w} > ŋ\textsuperscript{w} > m} (in \ili{PB} noun prefixes).\footnote{\tabref{tab:nasal:3} shows that \ili{Kordofanian} likes nasals in its noun prefixes, including palatals and velars, which may represent an innovation of the sort considered here. Cf. Williamson’s (1989: 40) proposal: \textit{*g\textsuperscript{w}}\textit{u-} > \textit{wu-} > \textit{mũ} > \textit{mu-}.} The major question is where the nasality would have come from? Perhaps there was a nasal that  preceded \ili{PB} noun prefixes, thereby producing a derivation such as: \textit{*N-g\textsuperscript{w} > ŋg\textsuperscript{w} > ŋ\textsuperscript{w} > m}.

Although this is speculative, and there are other possibilities (e.g. why not \mbox{\textit{*ɓʊ-},} \textit{*ɓɪ-}, etc.?), \tabref{tab:nasal:5} shows that there are attested shifts between labials and velars in \ili{Niger-Congo} languages \citep[200]{Hyman1980nasalclasses}.

\begin{table}
\begin{tabularx}{\textwidth}{l Xl Xl Xl llXl}
\lsptoprule
a. &\ilit{Gwari} &  &\ilit{Nupe} &  &\ilit{PNupoid} &  &  &cf. &\ilit{PB} & \\
\cmidrule{2-11}
& \textit{ēɓí} &  & \textit{ēgī} &  & \textit{*ɓí} &  & ‘child’ &  & \textit{*-bí-al-} & ‘give birth’\\
& \textit{ēɓwá} &  & \textit{ēgwā} &  & \textit{*ɓɔ́(k)} &  & ‘hand’ &  &\textit{*-bókò} & ‘arm, hand’\\

\tablevspace
b. &\ilit{Mankon} &  &\ilit{Bafmeng} &  &\ilit{PGB} &  &  &  &  & \\
\cmidrule{2-11}
&  \textit{àbô}&  & \textit{āɣó{\textasciigrave}} &  & \textit{*-ɓó{\textasciigrave}} &  & ‘hand’ &  & \textit{*-bókò} & ‘arm, hand’\\
& \textit{nɨ̀bòmə́} &  & \textit{īɣú\={m}} &  & \mbox{\textit{*-ɓùm´}} &  & ‘egg’ &  &  & \\
& \textit{bɨ́ }&  & \textit{ɣə́} &  & \textit{*-ɓá} &  & ‘they’ &  & \textit{*bá-} & SM, class 2\\
\lspbottomrule
\end{tabularx}

\caption{Labial-velar correspondences in Nupoid and Grassfields Bantu}
\label{tab:nasal:5}
\end{table}

However, we still have the issue of determining where the nasality would have come from. Since \citet{Miehe1991} there have been other developments that potentially interface with the problem at hand. First, \citet{Stewart1999nasal}, \citeyear*{Stewart1999explanation,Stewart2002} proposes \ili{PNC} nasalized vowels, which Williamson (1989: 40; 1993: 43-44) extends to noun class prefixes (although they are almost totally lacking in present-day languages). Also of potential importance is the role of the \ili{PB} determiner prefix known as the “augment”:

\largerpage
\begin{quote}
A correct view of the augment as a correspondence in \ili{Bantu} may enable us to bridge a gap between \ili{Bantu} and the other \ili{Benue-Congo} languages, by showing how the system of prefixes with differential m- ... arose. \citep[13]{Meeussen1973}
\end{quote}

\newpage 
As seen in \tabref{tab:nasal:6}, the augment resembles oral noun prefixes with H tone as found outside \ili{Bantu}, but also in the \ili{PB} concord markers (\citealt{deBlois1970}):\footnote{In \tabref{tab:nasal:6} \ili{Bukusu} devoices \textit{*g} > \textit{k} by the \ili{Luyia} Law \citep{HinnebuschEtAl1981}, while \ili{Haya} deletes the augment consonant, as in most \ili{Bantu}. While the classes 1 and 9 augments are reconstructed as *L, I know of no \ili{Bantu} language where they are distinguished tonally from other noun class augments.}

\begin{table}
\fittable{
\begin{tabular}{lllllll}
\lsptoprule
\ilit{PB}: & \textit{*ʊ-mʊ-} (1) & \textit{*gʊ́-mʊ-} (3) & \textit{*gɪ́-mɪ-} (4) & \textit{*gá-mà-} (6(a)) & \textit{*ɪ-Ǹ -} (9) & \textit{*(j)í-Ǹ -} (10)\\
\ilit{Bukusu}: & \textit{ó-mu-} & \textit{kú-mù-} & \textit{kí-mì-} & \textit{ká-mà-} & \textit{é-N-} & \textit{cí-N-}\\
\ilit{Haya}: & \textit{ó-mu-} & \textit{ó-mu-} & \textit{é-mi-} & \textit{á-ma-} & \textit{é-N-} & \textit{é-N-}\\
\lspbottomrule
\end{tabular}
}
\caption{The augment in PB and two daughter languages}
\label{tab:nasal:6}
\end{table}

It has therefore been attractive to relate the non-\ili{Bantu} oral prefixes to the augment. The significance of this move is seen from \citeauthor{GrégoireJanssens1999}' (1999) demonstration that the augment+noun prefix sequence can simplify in one of two ways: (i) loss of the augment: V-CV- > CV-; (ii) loss of the noun class prefix: V-CV > V-. Starting with a \ili{PB} reconstruction such as class 3 \textit{*gʊ́-mʊ̀-}, loss of the augment would leave \textit{mʊ̀-} as the noun prefix, while loss of the prefix would yield \textit{gʊ́-} in concords (and in noun prefixes and suffixes outside of \ili{Bantu}). This still does not explain why the two noun class markers should be different from each other.\footnote{\citet{Williamson1993} relates the class 9/10 split to the augment: \textit{Ǹ-} (or a nasal vowel?) is the class 9/10 prefix, with class 10 often enhanced by an augment, e.g. \ili{Kikongo} \textit{m-bwa} ‘dog’, pl. \textit{zi-m-bwa}.}

\newpage 
Note that \citet{deWolf1971} reconstructs the above noun class prefixes in \ili{PBC} with the shape V-, not CV-. However, \citet{Hyman1980nasalclasses}, \citet{Miehe1991} and especially  \citet{GrégoireJanssens1999} show different ways to derive a V- prefix (variant VN-). In the following potential changes, note the potential differences in tonal outcome (although high tone prefixes, especially V-, can independently become L as a kind of reduction process):

\ea%1
    \label{ex:nasal:1}
  \ea  CV prefix without augment
    \ea  the consonant drops:  *C\`{V}- > \`{V}-, e.g. class 7 *kɪ̀- >ɪ̀-; class 12 *kà- > à-
    \ex  the NV metathesizes:  *m\`{V}- > \`{V}m- > \`{V}N-
    \z
  \ex  CV prefix with vocalic augment
    \ea  the prefix drops:  *\'{V}-C\`{V}- > \'{V} -, e.g. class 7 *ɪ́-kɪ̀- > ɪ́-; class 12 *á-kà- > á-
    \ex  the prefix vowel drops:  *\'{V}-m\`{V}- > \'{V}-N- (>\'{V}-), e.g. class 3 *ʊ́-mʊ̀- > ʊ́N-;          class 4 *ɪ́-mɪ̀- > *ɪ́N-
    \z
  \z
\z

With this in mind, note the different realization of classes 1, 3, 4, 6 vs. 6a and plural “18a” in \ili{Tuki} (\citealt{Hyman1980tuki}; cf. \citet{Musada1995}), which derives \textit{VN-} from /\textit{V-mV-}/):

\ea%2
    \label{ex:nasal:2} 
\ea  class 1:  \textit{òŋ-gìnī}  ‘guest, stranger’  (but cf. \textit{mo-to} ‘person’, \textit{mw-ànā} ‘child’)\\
    class 3:  \textit{òŋ-gòl\={o}}  ‘foot’  \textit{òm-bàβē}  ‘wing’  \textit{ò-tēmā}  ‘heart’\\
    class 4:  \textit{ìŋ-gòl\={o}}  ‘feet’  \textit{ìm-bàβē}  ‘wings’  \textit{ì-tēmā}  ‘hearts’\\
    class 6:  \textit{àŋ-bāné}  ‘breasts’  \textit{àŋ-bīlé}  ‘palmtree’  \textit{à-tānē}  ‘stones’ (\textit{àŋ- > à-})
\ex  class 6a:  \textit{mà-tīá}  ‘water’  \textit{mà-wūtē}  ‘fat’\\
    class 18a:  \textit{mù-nū}  ‘brain’  \textit{mù-nɔ́ɔ̀ní}  ‘birds’  (cf. \ili{PNC} “6b”, \ili{PGur} 22 \textit{*mʊ})
\z
\z

In (2b), the two \textit{mV-} classes (6a, 18a) perhaps lacked an augment by virtue of their semantics. \ili{Tuki} has other \textit{CV-} prefixes, \textit{bà-} (2), \textit{bì-} (8), \textit{nò-} (11), \textit{wù-} (14) without augment, which may have fallen out. \citet[65]{Dugast1971} reports comparable data concerning collectives in \ili{Tunen} (cf. \citealt{Mous2003}: 302--303), e.g. \textit{ò-n-dɔ̀mb} ‘sheep’ (class 3), pl. \textit{è-n-dɔ̀mb} (class 4), \textit{mà-n-dɔ̀mb} ‘types of sheep’ (class 6).

\begin{quote}
  Signalons enfin que nous rencontrerons un collectif dont le préfixe paraît présenter un prépréfixe (ama- > am-, əm-). \citep[65]{Dugast1971}
\end{quote}


The history of noun class marking and ultimately nasality may thus implicate the presence of an augment—or different augments, as the case may be. The differential behavior of 1, 3, 4, 6, 9, 10 marking may also be attributed to a reconstructed (or evolved) \textit{*V} vs. \textit{*CV} shape. One attractive idea (for which, unfortunately, there is no evidence), is that there was a morpheme whose final [m] syllabifies with V-initial prefixes, but otherwise drops out before a consonant-initial prefix:

\ea%3
    \label{ex:nasal:3}
    \ea  \textit{*Vm-V-  >  V-mV-  >  mV- } (1, 3, 4, 6)
\ex      \textit{*Vm-CV-  >  V-CV-  >  CV-}  (2, 7, 8 etc.)
\z
\z

The loss of the initial \textit{*j} or \textit{*g} may also account for the merger of classes 4 (pl.) and 9 (sg.) in a number of \ili{Bantu} languages (Tables \ref{tab:nasal:7} \& \ref{tab:nasal:8}) .

\begin{table}
\begin{tabularx}{\textwidth}{lXlllll}
\lsptoprule
 &\ilit{Haya} &noun &subject &numeral &object &connective\\
\midrule
& class 9 & \textit{Ǹ - }&\textit{ è- }&\textit{ è- }&\textit{ -gi- }&\textit{ ya-}\\
& class 4 & \textit{mì- }&\textit{ è- }&\textit{ è- }&\textit{ -gi- }&\textit{ ya-}\\
cf. & class 8 & \textit{bì- }&\textit{ bí- }&\textit{ bí- }&\textit{ -bi- }&\textit{ bya-}\\
\lspbottomrule
\end{tabularx}
\caption{Merger of classes 9 and 4 in Haya \citep[8]{Byarushengo1977}}
\label{tab:nasal:7}
\end{table}

\begin{table}
\begin{tabularx}{\textwidth}{lXllllll} 
\lsptoprule
&\ilit{Tunen} &noun &subject &numeral &ProPref & \multicolumn{2}{l}{cl.6 collective}\\
\midrule
& class 9 & \textit{mè-, èN- }&\textit{ yè- }&\textit{ é- }&\textit{ yè }&\textit{  }&\textit{} \\
& class 4 & \textit{mè-, èN- }&\textit{ yé- }&\textit{ í- }&\textit{ yí }&\textit{ mà-Ǹ-} &  \\
cf. & class 8 & \textit{bè- }&\textit{ bé- }&\textit{ bé- }&\textit{ -bí-} &  & \\
\lspbottomrule
\end{tabularx}
\caption{Merger of classes 9 and 4 in Tunen (\citealt{Dugast1971}, \citealt{Mous2003}: 300-2)}
\label{tab:nasal:8}
\end{table}

Another factor that should be considered is the sporadic evidence of relic noun class suffixation in \ili{Bantu}, which is more widespread elsewhere in \ili{Niger-Congo}. It is likely that such suffixes never contained a nasal in classes 1, 3, 4, 6, 9 and 10. Again, the nasal classes may have had \textit{-V} (vs. \textit{-CV}) suffixes, as in \ili{Tiv} (\citealt{VoorhoeveDeWolf1969}: 52, based on Arnott).

\begin{table}
\begin{tabularx}{\textwidth}{l >{\itshape}r@{ }>{\itshape}l >{\itshape}r@{ }>{\itshape}l >{\itshape}l>{\itshape}l@{} >{\itshape}r@{}>{\itshape}c@{}>{\itshape}l}
\lsptoprule
\textup{class} & \multicolumn{2}{l}{\upshape noun affixes} & \multicolumn{2}{l}{\upshape adjective} & {\upshape subject (pr.cont.)} & {\upshape subject (past)} & \multicolumn{3}{c}{\upshape ‘my’}\\
\midrule
1 & Ø &  & ù- &  & ŋgù & a &  w- & àḿ & \\
2 & ù-, mbà- & -v & mbà- & -\'{v} & mbá\textsuperscript{↓} & ve &  & á\textsuperscript{↓} & -\'{v}\\
3 & (ú-) &  & ú- &  & ŋgú\textsuperscript{↓} & u &  w- & áḿ & \\
4, 5, 10 & í- &  & í- &  & ŋgí\textsuperscript{↓} & i &  y- & áḿ & \\
6 & á- &  & á- &  & ŋgá\textsuperscript{↓} & a &  & áḿ & \\
6a & ḿ- & -ḿ & mà- & -ḿ & má\textsuperscript{↓} & ma &  & á\textsuperscript{↓} & -ḿ\\
7 & í- & -ɣ́ & kì- & -ɣ́ & kí\textsuperscript{↓} & ki &  y- & á\textsuperscript{↓} & -ɣ́\\
8 & í- & -\'{v} & mbì- & -\'{v} & mbí\textsuperscript{↓} & mbi &  & á\textsuperscript{↓} & -\'{v}\\
9 & ì- &  & ì- &  & ŋgì & ì &  y- & àḿ & \\
14 &  & -\'{v} & mbù- & -\'{v} & mbú\textsuperscript{↓} & mbu &  & á\textsuperscript{↓} & -\'{v}\\
15 &  & -ɣ́ & kù- & -ɣ́ & kú\textsuperscript{↓} & ku &  & á\textsuperscript{↓} & -ɣ́\\
\lspbottomrule
\end{tabularx}

\caption{Tiv Noun Classes}
\label{tab:nasal:9}
\end{table}

If classes 1, 3, 4, 6, 9 and 10 had a -V suffix, then when suffix vowels dropped, the whole suffix was lost. The alternative is that these classes had earlier \textit{wV, yV} and \textit{ɣV} markers, where the glide first drops out, then the vowel. Note also that class 14 and 15 \textit{*Cu-} prefixes drop out entirely (class 3 leaves relics). There is a similar distribution of suffixes in \ili{Noni} \citep[188]{Hyman1980nasalclasses}. Understanding the nasals thus necessarily means understanding that the forms from different parts of a paradigm may originally have been different, may come to be different, and may influence the future of a system, e.g. whether nasals are spreading vs. retracting.

Finally, it should be noted that having a nasal (N) vs. oral (O) concord is not an all or nothing thing \citep[194-5]{Hyman1980nasalclasses}. One of the aforementioned noun classes can have nasal concord, another oral. Thus note the following out of 52 \ili{Bantu} languages (mostly Northwest, Table \ref{tab:nasal:10}).

\begin{table}
\begin{tabularx}{\textwidth}{llllXl}
\lsptoprule
class 3 &class 4 &class 6(a) &total &  &observations\\
\midrule
 N & N & N & 20 &  & 11/20 are in zone C\\
 O & N & N & 18 &  & 12/18 are in zones A-B\\
 O & O & N & 14 &  & 7/20 are in zones D-F \\
\lspbottomrule
\end{tabularx}
\caption{Distribution of nasal concord by Noun Class}
\label{tab:nasal:10}
\end{table}

The class 3, 4, and 6(a) distributions N-O-N, N-N-O, O-N-O and O-O-N are all unattested. We thus can draw the following implicational scale: class 3 N ${\supset}$ class 4 N ${\supset}$ class 6 N. 

Occasionally non-\ili{Bantu} languages have a nasal in their pronoun system which resembles \ili{Bantu}. Thus the \ili{Fula} [\ili{North Atlantic}] third person singular human subject and object pronoun \textit{mo} \citep{Arnott1970} and the \ili{Wawa} [\ili{Bantoid}] third person singular human pronoun \textit{mū} \citep[169]{Martin2012} ought to be cognate with the \ili{Proto-Bantu} class 1 object marker \textit{*-mu-}. Similarly re class 1 \textit{mù} and class 3 \textit{mū} pronouns in \ili{Esimbi} [\ili{Bantoid}]. As seen in \tabref{tab:nasal:11}, from \citeauthor{Stallcup1980esimbi} (S) (1980: 142) and \citet[8--9, 27]{KoenigEtAl2013}, the other pronouns resemble the corresponding noun class prefixes. For some reason the two sources give different oral vs. nasal reflexes on the noun prefixes of classes 2, 6a, 14, and 18a (\tabref{tab:nasal:11}).

\begin{table}
\begin{tabularx}{\textwidth}{l >{\itshape}l>{\itshape}l>{\itshape}l Q}
\lsptoprule
\textup{class} & \textup{noun (S)} & \textup{noun (K et al)} & \textup{ Pronoun} & {/\textit{I, U, A}/ =\upshape archiphonemes}\\
\midrule
1 & (w)Ù- & ((w)U)- & mù\footnote{ Cf. the object marker \textit{-ŋw-}. In symbols /\textit{I, U, A}/ stand for archiphonemes whose vowel height depends on the following stem.} & Koenig et al. exx. have L or M tone\\
2 & bÀ- & mA- & bú & why L tone?\\
3 & Ú & U- & mū & \\
5 & Í &  &  & \\
6 & Á & A- & zú & \\
6a & bÀ-, m- & mA- & bù & note L tone; \textit{m-} is used before /\textit{b}/\\
7 & kI- & kI- & kī & \\
8 & bI- & mI- & bī & \\
9 & Ì- & I- & zù & exx. from Koenig et al. have L tone\\
10 & Í- & I- & zú & exx. from Koenig et al. have non-L tone\\
12 & kA- & kU-, kA- & kū & \\
13 & tA- & tU-, tA- & tí & \\
14 & bÚ- & mU- & bú & \\
18a & bÙ- & mU- & bù & note L tone\\
19 & sÍ- & sI- & sī & \\
\lspbottomrule
\end{tabularx}
\caption{Esimbi Noun Class prefixes and pronouns}
\label{tab:nasal:11}
\end{table}

The fact that one dialect denasalizes class 6 and 18a prefixes and the other nasalizes class 2 and 8 prefixes is something which repeats itself elsewhere in \ili{Bantoid}, e.g. \ili{Ekoid} \citep[133]{Watters1980ejagham}, \ili{Kenyang} \citep{Voorhoeve1980kenyang}, and \ili{Mbe} \citep{Bamgbose1965}—and even \ili{Narrow Bantu}, e.g. zone C denasalization of \textit{*mV- > bV-}. Any proposed scenario such as \textit{*gw > m} must be grounded in what we know about the natural history of nasality.

In conclusion, \citegen{Miehe1991} demonstration of widespread nasals still leaves a lot to interpret: Who had what when? How did everyone get what they have today? What does this say about the evolution of noun class systems: mergers, splits, loss? (cf. \citealt{Good2012}). There is still a lot of work to do before we can arrive at a definitive solution to the issues that I have outlined above.

\section*{Acknowledgements}

I would like to thank Gudrun Miehe, my co-organizer in Paris of the workshop “Nasal Noun Class Prefixes in \ili{Bantu}: Innovated or Inherited?”, at the 5th International Conference on Bantu Languages, June 12, 2013, at which this paper was first presented, and the other presenters: Jeff Good, Jesse Lovegren, Konstantin Pozdniakov, Valentin Vydrin, and John Watters. I also thank the doctoral students in my Spring (2013) Seminar on \ili{Bantu} and \ili{Niger-Congo} who were full of ideas and enthusiasm: Matt Faytak, Florian Lionnet, Jack Merrill, Zachary O’Hagan, Nik Rolle.

{\sloppy
\printbibliography[heading=subbibliography,notkeyword=this]
}
  
\end{document}