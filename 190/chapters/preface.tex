\addchap{Preface} 
This volume, \textit{East Benue-Congo: Nouns, pronouns, and verbs} is the first volume in the \ili{Niger-Congo} Comparative Series of the Language Science Press (\url{langsci-press.org}). The aim of the \ili{Niger-Congo} Comparative Series (NCCS) is to enhance comparative-historical studies and linguistic reconstruction of proto-languages of the groups and families within \ili{Niger-Congo}, and eventually, of \ili{Proto-Niger-Congo} itself. 

The edited volumes and monographs in this series will deal with all aspects of comparative-historical \ili{Niger-Congo} studies, including both segmental and prosodic phonology, morphology and syntax, etymological dictionaries of groups and families, problems of genetic classification, application of statistical methods to the comparative-historical \ili{Niger-Congo} studies, correlation of genetic relationships, contact-induced affinities, and so on. This series provides an academic forum and publishing entity for scholars to present their findings in comparative-historical studies of \ili{Niger-Congo} and its subdivisions. Researchers are encouraged to join in the advancing of the frontiers of our knowledge about the historical development of \ili{Niger-Congo} and its constituents. 

The \ili{Niger-Congo} macro-family (the biggest in the world, comprising more the 20\% of all the world’s languages) was postulated by Joseph Greenberg in his 1948 paper and subsequent publications. It is now widely accepted. However, most of the mid-range language families included in \ili{Niger-Congo} are characterized by an insufficient level of comparative-historical study, and in certain cases, even the validity of groupings has not been adequately demonstrated.

During  {the } 1960-80s, numerous comparative studies were carried out on different \ili{Niger-Congo} subdivisions, and serious amendments to Greenberg's classification were proposed. In  {the }  1990s, there was a lull: the potential of the first assault was more or less exhausted, and, on the other hand, an exponential growth in the amount of descriptive data available on African languages required reconsideration of the approaches that could provide reliable comparative results. A by-product of the lull was a growing skepticism about the reality of the \ili{Niger-Congo} as a genetic unit, a skepticism supported by a general suspicion toward  comparative linguistics — and especially, about long-range comparison - a suspicion which grew very popular at that time and, I dare say, remains popular among linguists, especially those who are not personally involved in comparative studies and protolanguage reconstruction.

A first attempt to curb this trend was related to the Babel Tower project headed by Sergey Starostin and Murray Gell-Mann who made a courageous attempt to survey the state of the art in protolanguage reconstruction of all the language families of the world. They organized, together with Konstantin Pozdniakov, a \ili{Niger-Congo} workshop in Paris in 2004 where leading specialists in the field were invited. Among other things, the workshop discussions made it clear that \ili{Niger-Congo}, which numbers more than 1500 languages, is not a family, but rather a macro-family (or phylum) whose age is at least 12 millennia — most probably, even more than that. Its major subdivisions are \ili{Benue-Congo}, \ili{Kwa}, \ili{Adamawa}, \ili{Gur}, \ili{Kru}, \ili{Dogon}, \ili{Ijoid}, \ili{Atlantic}, \ili{Mande}, \ili{Kordofanian}.  The time depth of these subdivisions lies most often within the range of 5 to 8 millennia.  They should be considered as mid-range families at  the same level as \ili{Indo-European} or \ili{Semitic}.

In 2012, the First International Congress ``Towards \ili{Proto-Niger-Congo}: Comparison and Reconstruction'' took place in Paris. One of the ideas of its organizers was to canalize the energy of the participants into writing a collective volume that would become a major breakthrough toward the reconstruction of the proto-language. The volume was intended to contain chapters on mid-range families written by specialists in these families according to a template meant to cover all relevant topics.

However, the project of a ``\ili{Niger-Congo} volume'' kept changing  from its very beginning. Already at the initial stage, it became  clear that the chapters would target an average size of some 30,000 words, and therefore, there should be at least two volumes, maybe even three. However, the main difficulty was not the presumable size of the volume(s), but the availability of potential authors and, on the other hand, the state of the art in the reconstructions for the mid-range families. In fact, relative to the complexity and size of \ili{Niger-Congo}, there were few comparative-historical studies that could guide summaries for each major, mid-range subdivision.  

Finally, it was decided that the best strategy, given the current state of affairs, was to launch a series specialized in \ili{Niger-Congo} comparative studies in the Language Science Press. The authors originally invited to write chapters for the \ili{Niger-Congo} volume(s) were reoriented toward producing separate books, and a more flexible approach has been taken in relation to the structure of books acceptable for the Series. The hope is that over time the accumulation of multiple studies in coming years will bring increasing clarity to our understanding of the history of \ili{Niger-Congo}.

This first volume of the Series has a long history too. It was originally planned as an extended version of the \ili{East Benue-Congo} (without \ili{Bantu}) chapter of the \ili{Niger-Congo} volume, to be published in one or two years. However, it grew clear little by little that, due to the immensity of the \ili{Benue-Congo} family and the very uneven level of study of its constituent groups, it would be unrealistic to require authors to stick to the template and to cover, at the same time, all the \ili{East Benue-Congo} groups.  It has turned out that instead of one \ili{East Benue-Congo} volume, it would be more expedient to publish three medium-size volumes, and even in this case, all the topics of the original template for the ``\ili{Niger-Congo} volume'' will be very far from being covered. It is not an understatement to say that many more volumes will be needed to cover the topics of the original template relative to \ili{East Benue-Congo}. This first volume provides comparative insights but it also serves as much as setting a foundation on various topics upon which subsequent studies can be pursued.

Publication of this book marks the end of the four-year incubation period of the series \textit{\ili{Niger-Congo} Comparative Studies}, and there are good reasons to believe that the next volume of the series will not  make us wait as long as the first one. Reconstruction of \ili{Proto-Niger-Congo} is an immense task, and the story of the ``\ili{Niger-Congo} volume project'' and its sequels will serve as  a vaccine against naiveté and impatience. At the same time, let it be a warning: if we want to make the \ili{Niger-Congo} reconstruction commensurable with a human lifespan, we need further concentration and strenuous efforts. More scholars are needed in the project of researching and molding our knowledge of the history of \ili{Niger-Congo} and its subdivisions. Such scholars are invited and encouraged to join in the process.

\bigskip 

\hfill {Valentin Vydrin}
\bigskip 

\hfill 
Chief Editor of the series ``{Niger-Congo} Comparative Studies''

  
