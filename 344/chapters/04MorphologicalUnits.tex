\chapter{Morphological units and processes}
\label{ch:word}
This chapter outlines Kalamang morphological units and processes. Morphological units (§\ref{sec:morphun}) are meaning-bearing units that can be divided into types with different phonological and morphosyntactical behaviour. There are four types: bases (which include roots), affixes, clitics and one particle. Bases combine with affixes and clitics to form words. The definitions given here are predictive for the spelling \parencite[see][71]{haspelmath2011}, as well as necessary for the description of the Kalamang grammar. Morphological processes (§\ref{sec:morphproc}) alter the meaning or part of speech of a word, or show grammatical relationships between words. Kalamang makes use of reduplication, compounding, affixation and cliticisation.\is{morpheme} 

\section{Morphological units}
\label{sec:morphun}
\subsection{Word}\is{word}
\label{sec:worddef}
A word is defined as a morpheme or a sequence of morphemes that has one primary stress, and across whose boundaries morphophonological rules such as lenition\is{lenition} do not apply (§\ref{sec:morphphon}).

It has the following syntactic properties:

\begin{enumerate}
	\item It can occur without other units; it can be uttered on its own and still be meaningful.
	\item It has some freedom with respect to its position in the clause. Although independent units fill slots in the clause, these may be shuffled around for reasons of prominence, for example.
	\item It can contain an affix and one or more clitics, in that order.
\end{enumerate}

Words may consist of more than one morpheme, but the order of these morphemes is fixed. Words are printed with space around them. 
%HH Nota bene that compounds may break the order constraint you state, which should probably be between root and affixes only, not root + root).

Words are syntactic units; their distributional properties are determined by the syntax – the topic of Chapter~\ref{ch:wc} and Chapter~\ref{ch:clause}. The occurrence of words as free, independent forms is attested, for example, in answers to questions, or when describing things or actions by means of pointing. An additional indication of the independence of words comes from the way speakers treat them: when translating from Kalamang to Papuan Malay, words are what people immediately recognise as a unit in their language, and for which they can offer a translation (even if a direct or literal translation is not possible). 



\subsection{Bases and roots}\is{root}\is{base}
\label{sec:root}
A root is a synchronically and diachronically unanalysable form. A base is a form to which further morphological processes such as compounding, reduplication, affixation and cliticisation can be applied, and which may or may not be a morphologically complex form.

A good example of complex bases are compounds, which are words consisting of two roots, to which morphological processes such as reduplication or possessive inflection apply. The roots \textit{tan} `arm; hand' and \textit{parok} `limb extremity' can be compounded to form \textit{tanparok} `finger'. This base can be reduplicated as \textit{tanparok∼parok} `fingers' or inflected with a possessive pronoun, e.g. \textit{tanparok-ca} `your finger'.

Roots can be free or bound. An example of free roots are the demonstrative roots \textit{wa} `\textsc{prox}', \textit{me} `\textsc{prox}', \textit{owa} `\textsc{fdist}', \textit{yawe} `\textsc{down}' and \textit{osa} `\textsc{up}' (Chapter~\ref{ch:dems}), which may occur independently, but which are also commonly inflected. Free roots are phonological words, carrying one main stress (§\ref{sec:stress}). Bound roots cannot function as words on their own, but must undergo some morphological process. An example of bound roots are the five inalienable roots described in §\ref{sec:inal}, such as \textit{nam-} `husband', which must occur with possessive marking. The root of the noun \textit{tumun} `child' is \textit{tum-}, as its reduplicated form is \textit{tumtum} `children'.


\subsection{Affix}\is{affix}
\label{sec:affix}
An affix is a unit that is dependent both phonologically and syntactically. It attaches at word level; each affix is restricted to one word class only. Affixes attach directly to words (consisting of one or more roots), and never to other affixes or to clitics. This attachment leads to the application of the phonological rules sketched in §\ref{sec:morphphon}; for example, when [n] occurs adjacent to a velar, it is velarised to [ŋ]. An affix can have primary stress, but then none of the syllables in the root can. Affixes are part of the phonological word when stress rules are applied (§\ref{sec:stress}). A `word+affix' is spelled as one word with space around it, and is glossed with a dash in between.

Kalamang has prefixes and suffixes. All prefixes are numeral classifiers (§\ref{sec:clf}). A list of affixes and their allomorphs, as well as a reference to where they are described, can be found on page~\pageref{sec:boundmorphs}.
 
An example of an affix is the second-person plural possessive suffix \textit{-ce}. It attaches only to nouns, and undergoes voicing assimilation\is{assimilation} (§\ref{sec:assim}) when it is attached to a noun ending in a nasal. Another suffix is the quantifier object marker \textit{-i}, which only attaches to quantifiers in object position (§\ref{sec:nmodqnt}).

\subsection{Clitic}\is{clitic}
\label{sec:clitic}
Clitics are bound morphemes. They differ from affixes in that they either attach to more than one word class but show morphophonological integration with their host, or they attach to one word class only but do not obey morphophonological rules. Kalamang has proclitics and enclitics. Clitics always follow affixes. Clitics are part of the phonological word when stress rules are applied (§\ref{sec:stress}). The combination `host+clitic' is spelled as one word, and is glossed with an equality sign in between.

Kalamang clitics behave differently with respect to the kind of host they attach to and the degree of morphophonological integration they show. In general, the proclitics attach to one word class only (verbs), but do not obey morphophonological rules like lenition. The enclitics generally are promiscuous with respect to the word class they attach to (typically attaching to either the predicate\is{predicate}\is{verb phrase*|see{predicate}} or the NP\is{noun phrase}), but obey morphophonological rules. Some enclitics show less morphophonological integration with their host. Most of the enclitics starting with /k/, for example, do not show deletion when attached to a noun ending in /k/ (§\ref{sec:elis}); volitional \textit{=kin} attracts stress, and attributive \textit{=ten} lenites sometimes, but not always.

Postpositions are clitics because, although they are perfectly integrated phonologically, they do not attach to nouns only, but to the right edge of the NP, irrespective of which host constituent happens to be there. In practice, this means postpositions\is{postposition} are found on nouns and nominal modifiers. This clitic status has not kept postpositions from forming such close ties with some words that they have become part of them. Examples are the distal demonstrative object form \textit{met}  (from \textit{me} `\textsc{dist}' and \textit{=at} `\textsc{obj}') and the morphophonologically problematic locative and lative demonstratives \textit{metko} `there' and \textit{wangga} `to/from here' (described in §\ref{sec:problems}).

\tabref{tab:clitics} is a comprehensive list of Kalamang clitics, with a summary of their behaviour and a reference to a detailed description elsewhere.

\begin{table}
	\caption{Clitics}
	
		\footnotesize
		\begin{tabular}{l l l l l}
			\lsptoprule
			form & function & attaches to & phonologically & reference\\ 
			&&&integrated&\\\midrule 
			\textit{=a} & focus & NP & yes & §\ref{sec:a}\\
			\textit{=at} & object & NP & yes& §\ref{sec:at}\\
			\textit{=bon} & comitative & NP & yes& §\ref{sec:comi}\\
			\textit{=ero} &conditional &Pred&yes&§\ref{sec:condmood}\\
			\textit{=et} &irrealis&Pred&yes&§\ref{sec:et}\\
			\textit{=i} &predicate linker&Pred&yes&§\ref{sec:mvci}\\
			\textit{=in} &prohibitive&Pred&yes&§\ref{sec:proh}\\
			\textit{=ka} &lative&NP&partly&§\ref{sec:lat}\\
			\textit{=kap}&similative&NP&partly&§\ref{sec:simcase}\\
			\textit{=ki} &instrumental&NP&partly&§\ref{sec:ins}\\
			\textit{=ki} &benefactive&NP&partly&§\ref{sec:ben}\\
			\textit{=kin} &volitional&Pred&no&§\ref{sec:kin}\\		
			\textit{=ko} &locative &NP&partly&§\ref{sec:loc}\\
			\textit{=kongga} &animate lative&Pron, N&no&§\ref{sec:animloclat}\\
			\textit{=konggo} &animate locative&Pron, N&no&§\ref{sec:animloclat}\\		
			\textit{=nan} & `also' & any & yes &§\ref{sec:too}\\	
			\textit{=nin} &negation&Pred&yes&§\ref{sec:clauneg}\\
			\textit{=saet} &`exclusively'&any&yes&§\ref{sec:tenunal}\\			
			\textit{=sawe(t)} &excessive&Pred&yes&§\ref{sec:degradv}\\	
			\textit{=ta} &nonfinal&Pred&yes&§\ref{sec:nfin}\\	
			\textit{=taero} & `even if' & Pred&yes&§\ref{sec:condmood}\\
			\textit{=taet} &`more; again'&Pred&yes&§\ref{sec:taet}\\
			\textit{=tak} &`just; only'&any&yes&§\ref{sec:alonepron}, \\
			&&&&§\ref{sec:quantall},\\
			&&&&§\ref{sec:distmann},\\
			&&&&§\ref{sec:mvcuntil}\\
			\textit{=tar} & plural imperative&Pred&yes&§\ref{sec:imp}\\
			\textit{=te} &imperative&Pred&yes&§\ref{sec:imp}\\
			\textit{=te} &nonfinal&Pred&yes&§\ref{sec:nfin}\\	
			\textit{=teba}&progressive&Pred&yes&§\ref{sec:teba}\\						
			\textit{=ten}&attributive&Pred&sometimes&§\ref{sec:attr}\\			
			\textit{=tenden}&`so'&Pred&yes&§\ref{sec:consconj}\\
			\textit{=tun}&intensifier&Pred&no&§\ref{sec:extrnoun},\\
			&&&&§\ref{sec:quantinfl}, §\ref{sec:verbred},\\
			&&&&§\ref{sec:degradv}\\ \midrule
			\textit{di=} & causative & Pred & no& §\ref{sec:di}\\
			\textit{ko=} & applicative & V & no & §\ref{sec:appl}\\
			\textit{nak=} & `just'& V & no & wordlist\\
			\textit{nau=} & reciprocal & V & no &§\ref{sec:recp}\\
			\textit{ma=} & causative & V & no & §\ref{sec:caus}\\
			\lspbottomrule
		\end{tabular}
	
	\label{tab:clitics}
\end{table}


Clitics follow affixes, as exemplified for the third-person possessive suffix \textit{-un} and the object clitic \textit{=at} in~(\ref{exe:pepmang}).

\begin{exe}
	\ex 
	{\gll in pep-mang-\textbf{un}=\textbf{at} paruon\\
		\textsc{1pl.excl} pig-language-\textsc{3poss}=\textsc{obj}	make\\
		\glt `We're making pig's language [= ugly language].' \jambox*{\href{http://hdl.handle.net/10050/00-0000-0000-0004-1BCB-5}{[conv1\_7:15]}}
	}
	\label{exe:pepmang}
\end{exe}


\subsection{Particle}\is{particle}
One unit, \is{iamitive}iamitive \textit{se} (an aspectual marker that can often be translated as `already', see §\ref{sec:setok}), does not fit into any of the categories described above and is analysed as a particle. 

\textit{Se} is phonologically quite dependent: it does not carry stress, and it usually obeys morphophonological rules. In the great majority of instances, when \textit{se} follows a word ending in a consonant it takes the form \textit{se} and when it follows a word ending in a vowel it takes the form \textit{he}. \textit{Se} is also grammatically dependent, in the sense that it has no meaning on its own, it does not occur on its own, and it has a fixed position in the clause: it comes right after the subject NP. In these respects \textit{se} behaves like an enclitic. However, \textit{se} is not bound to its phrasal host. Consider the question-answer pair in~(\ref{exe:anse}). When answering a question with a subject, \textit{se} and a verb, a speaker may elide the verb, but answering with the subject and \textit{se} is ungrammatical, showing that \textit{se} is not bound to the subject NP.

\begin{exe}
	\ex 
	\begin{xlist}
		\exi{A:}
		{\gll naman=a se bot\\
			who=\textsc{foc} {\glse} go\\
			\glt `Who has already gone?'
		}
		\exi{B:}
		{\gll *an se\\
			\textsc{1sg} {\glse}\\
			\glt `Me.' \jambox*{[elic]}
		}
	\end{xlist}
	\label{exe:anse}
\end{exe}

%elicit 2020

We also find \textit{se} in clauses where the subject is elided, as in example \ref{exe:matdani}, which consists of the clauses [after burying him] and [went back up]. The second clause starts with \textit{se}, and the subject `we' is elided.

\begin{exe}
	\ex 
	{\gll mat dan=i koyet \textbf{se} ecien=i sara\\
		\textsc{3sg.obj} bury={\gli} finish {\glse} return={\gli} ascend\\
		\glt `After burying him [we] went back up.' \jambox*{\href{http://hdl.handle.net/10050/00-0000-0000-0004-1BCF-3}{[narr2\_0:34]}}
	}
	\label{exe:matdani}
\end{exe}

Another example showing that the subject and \textit{se} do not form a tight bond is the \is{repetition}repetition of \textit{se}. What is repeated in a clause with hesitation is not the subject + \textit{se}, but \textit{se} only. Both instances of \textit{se} are unstressed.

\begin{exe}\ex {\glll Emun se... se wanir.\\
		 emun se se wanir\\ 
		mother.\textsc{3poss} {\glse} {\glse} twice\\ 
		\glt `Her mother [did it] already... already twice.' \jambox*{\href{http://hdl.handle.net/10050/00-0000-0000-0004-1BDE-7}{[narr25\_1:45]}}} 
	\label{exe:sese}
\end{exe}

But \textit{se} cannot stand alone as an answer, as in~(\ref{exe:kahebot}).
\begin{exe}
	\ex 
	\begin{xlist}
		\exi{A:}
		{\gll ka se bot\\
			\textsc{2sg} {\glse} go\\
			\glt `Have you gone already?'
		}
		\exi{B:}
		{\gll *se\\
			{\glse}\\
			\glt `Yes/already.'
		}
	\end{xlist}
	\label{exe:kahebot}
\end{exe}

Notice the difference compared to the other aspectual unit \textit{tok} `yet', also described in §\ref{sec:setok}, which is a word. It carries its own stress, retains its phonological form no matter which units surround it (nothing can be prefixed or suffixed to \textit{tok}), and, most importantly, it can stand alone as an answer (see also the section on \is{interjection}interjections, \ref{sec:wcyesno}). In~(\ref{exe:kaiatrep}), the speaker acts out a conversation\is{conversation} between two people. (Thus, A and B do not stand for two actual speakers in the recording, but two fictional speakers.)

\begin{exe}
	\ex 
		\begin{xlist}
		\exi{A:}
	{\gll ki se kai=at rep\\
		\textsc{2pl} {\glse} firewood=\textsc{obj} get\\
		\glt `Did you already get firewood?'
	}
		\exi{B:}
{\gll tok\\
	not.yet\\
	\glt `Not yet.' \jambox*{\href{http://hdl.handle.net/10050/00-0000-0000-0004-1B9F-F}{[conv9\_31:24]}}
}
\end{xlist}
	\label{exe:kaiatrep}
\end{exe}

\subsection{Comparison with native speakers' spelling}\is{orthography}\is{spelling|see{orthography}}
Kalamang is hardly ever written by its speakers, but now that the internet is becoming a part of life for some of them, the body of written Kalamang is growing. Besides that, I have a small corpus of written Kalamang provided by some of my informants. The biggest part of this written corpus constitutes the almost 2000 example sentences that Fajaria Yarkuran wrote for the Kalamang dictionary (§\ref{sec:teachers}). Even though she may have been influenced by my spelling of Kalamang units (we have transcribed hours of text together with her looking at the way I spelled things), it is obvious that there is a grey area between word and affix also for the Kalamang native speaker. Most affixes and clitics (such as the possessive suffixes, predicate linker \textit{=i} and volitional \textit{=kin}) are always spelled by Fajaria as one word together with the root. There are also affixes, such as the classifier prefixes, where spelling varies: they are sometimes spelled as one word with the numeral and sometimes separated. Among the clitics we also see variation. Comitative postposition \textit{=bon} is always written with space around it, but lative and locative \textit{=ka} and \textit{=ko}, respectively, are found both as one word with the host and with space on either side. Most nouns, verbs, adverbials, quantifiers and demonstratives are spelled as words, but there is variation in compounds and incorporated nouns. 

In absence of a standard Kalamang spelling, it is thus absolutely not clear in all cases how to segment Kalamang units for a native speaker.\footnote{I started a test where I let different speakers transcribe the same 11 sentences from corpus recordings, to see how they would segment units. I did not do the test with more than two speakers, however, because I realised there was a lack of good speakers who could write clearly enough to determine segments. Although most Kalamang speakers are literate, many of them are not experienced or fluent writers.} This supports the idea that there is a mismatch between the categories for morphemes that we find in languages, and the ones we have names for (affix, clitic, word). Not only cross-linguistically, as emphasised by \textcite{haspelmath2011}, but also within a language, pace \textcite{gil2020introduction}. He gives the example of associative \textit{nya}, spelled once as a word and once as an affix within the same clause of Riau Indonesian; and he shows the analysis of Papuan Malay units \textit{sa} `\textsc{1sg}' and \textit{pu} `\textsc{poss}' as words by \textcite[][377]{kluge2014} and as clitics by \textcite[][260]{donohue2007}.


%196 sontum eret bon a taru
%242 an dodonana masatkin
%243 yuon daruk ta go kerkap
%265 an bo gous-sat rep te kiem
%266 tumtuma me mat sirieret jabulsawe
%687 motor kon lukta anin emun komet
%803 tami emun perlalang kuar teba ba mahe laur.
%827 rajiba esun mu bo lenggon ga sayang sara
%836 binkur esun go dungga karjangi mindi bo lohar.
%901 nin ruslan taraun me liti marau kno tanun go
%902 mu loska marmar
%916 mustafa esun mu maasko go masr lek amdirun
%49 tatka bot


\section{Morphological processes}
\label{sec:morphproc}
Here, I describe the characteristics of four morphological processes: reduplication, compounding, and affixation and cliticisation.

\subsection{Reduplication}\is{reduplication!phonology|(}
\label{ch:redup}
\label{sec:redup}
Reduplication is the systematic repetition of a root or part of a root, which can have either semantic purposes (to create new words) or grammatical purposes (e.g. to change the number, aspect or word class of a word, \citealt{rubino2005}). Kalamang makes use of both full reduplication, which involves the entire root, and partial reduplication, which involves a part of the root. \tabref{tab:redup} on the next page gives an overview of the different word classes where reduplication is found, with the most common functions, and a reference to where these functions are described. In the following, full and partial reduplication in Kalamang are illustrated. The examples given also list their function as found in context in the corpus.\is{reduplication}

\begin{table}
	\caption{Reduplication: word classes and functions}

		\begin{tabularx}{\textwidth}{lQl}
			\lsptoprule
			word class & function & reference	\\ \midrule
			nouns & plural, distributive, in-between, noun derivation, verb derivation & §\ref{sec:nounredup}, \ref{sec:verbder}, \ref{sec:verbred} \\
			numerals & distributive & §\ref{sec:quantinfl}\\
			stative verbs & intensification & §\ref{sec:verbred} \\
			verbs & durative, distributive, habitual & §\ref{sec:verbdistr}\\
			\lspbottomrule
		\end{tabularx}
	
	\label{tab:redup}
\end{table}

\subsubsection{Full reduplication}
Full reduplication is the most common form of reduplication in Kalamang, and it is used for the inflection of verbs (to make duratives\is{durative}, habituals\is{habitual} and distributives\is{distributive}), to derive verbs from nouns, to \is{intensification}intensify stative verbs, to indicate the extreme of a noun referring to a \is{location}location, to make plural nouns, to make distributive nouns, and for indicating in-between states with nouns. Full reduplication usually involves the root only, such that inflectional morphology like the enclitic \textit{=tun} `very' or locative \textit{=ko} is not reduplicated. Non-productive morphology, like the morpheme \textit{na-} that is found on many AN loan verbs, is not reduplicated either. Incorporated nouns are not reduplicated. The longest reduplicated roots found in the corpus have three syllables and are verbs. Only monosyllabic nouns are fully reduplicated to make plural\is{plural!nominal} forms. Longer nouns are partially reduplicated, as illustrated further below. Morphophonological rules like lenition apply to reduplicated roots as well. Consider the examples in \tabref{tab:basered}.

\begin{table}
	\caption{Full reduplication}
		\begin{tabularx}{\textwidth}{QlQl}
		\lsptoprule
			base & other & reduplicated form & function\\ 
			&morphology&&\\\midrule
			/ˈparuo/ \newline `to do' & \textendash & /ˌparuo∼ˈwaruo/ \newline `to do' & habitual, durative\\ %also taruoraruo
			\tablevspace
			/ˈter-na/ \newline `to drink tea' & /ter/   `tea' & /ter-ˈna∼na/ \newline `to drink tea' & durative \\
			\tablevspace
			/ˈkomet/ \newline `to see; to look' & \textendash & /ˌkomet∼ˈkomet/ \newline `to look' & durative\\
			\tablevspace
			/aˈsokmaŋ/ \newline `to be short of breath' & \textendash & /aˈsokmang ∼ aˌsokmang/\newline  `to be short of breath' & durative\\
			\tablevspace
			/na-ˈbaca/ \newline `to read' & \textit{na-} AN loan V & /na-ˈbaca∼ˌbaca/ \newline `to read' & distributive\\ %nau-besbes, narekin-rekin, na-tuka-tuka
			\tablevspace
			/mun/ \newline `flea' & & /ˈmun∼mun/ \newline `to search for fleas' & noun → verb \\ %buok → buokbuok
			\tablevspace
			/bes/ \newline `good' & \textit{=tun}   `very' & /bes∼ˈbes=tun/ \newline `very good' & intensification\\ %monmontun
			\tablevspace
			/ˈsiun/ \newline `edge' & \textit{=tun}   `very' & /siˌun∼siˈun=tun/ \newline `the very edge' & extreme\\
			\tablevspace
			/leŋ/ \newline `village' & \textendash & /ˈleŋ∼leŋ/ \newline `villages' & plural\\ %\textit{ˈkaŋun} `bone' → \textit{kaŋˈkaŋun} `bones', \textit{ˈolun} `leaf' → \textit{olˈolun} `leaves' and \textit{ˈorun} `tail' → \textit{orˈorun}, \textit{saˈmor} `bead' → \textit{saˌmorsaˈmor} `beads', posposun,
			\tablevspace
			/kitˈ=ko/ \newline `on top' & \textit{=ko} locative & /kit∼ˈkit=ko/ \newline `on tops' & distributive\\ %timtim
			\tablevspace
			/paˈsier/ \newline `sea water' & \textendash & /ˌpasier∼ˈwasier/ \newline `brackish water' & in-between\\%also rangrang
			\lspbottomrule 
		\end{tabularx}
	
	\label{tab:basered}
\end{table}
%verbs: v > v?
%dongdong
%verbs other:
%go kaliskalis (rainy)
%marmar-marmar=et
%na-sandar-sandar=ta (many fish?)
%paruak-paruak (many things?)
%marum-marum
%gosomin-gosomin
%kanggeit-kanggeit
%pak-pak
%asokmang-asokmang=et
%kos-kos=te(distrib?)
%yie-yie
%kiet-kiet
%sou-sou

\subsubsection{Partial reduplication}
Partial reduplication can be leftward or rightward, and can involve one or more syllables. As this type of reduplication is less common, there are not enough data to determine how much material gets repeated. Tables~\ref{tab:partrednoun}, \ref{tab:partrediverb} and~\ref{tab:partredverb} shows the found patterns with different word classes: nouns, stative intransitive verbs\is{stative intransitive verb} and other verbs. For most processes, only one or two examples are available.

The identified processes with nouns are listed in \tabref{tab:partrednoun}. There is rightward reduplication with -CV, -CVC and -CVVC reduplicants, and leftward reduplication with CV-, CVV- and CVCV- reduplicants. This type of reduplication with nouns is mostly used for plurals of polysyllabic nouns. No type of reduplicant has more than two different examples.

\begin{table}[t]
	\caption{Reduplication: rightward and leftward}
		\begin{tabularx}{\textwidth}{llQl}
			\lsptoprule 
			base &  reduplicant & reduplicated form & function\\ \midrule 
			/keˈwe/ `house' & -CV & /kewe∼ˈwe/ `houses' & plural\\ %also osep-ep-ko
			/kuˈliep/ `cheek' & -CVVC & /kuˈliep∼ˌliep/ `cheeks' & plural\\ %also lempuang
			/ˈkorpak/ `knee' & -CVC & /korˈpak∼pak/ `knees/ & plural\\ %also naupar-par
			/don/ `thing' & CV- & /ˈdo∼don/ `clothing' & N → N\\ %also nener
			/ˈsaun/ `night' & CVV- & /ˈsau∼saun/ `very dark; darkness' & N → N\\ %also em-emun (or em-em-un), cf. verb lau-laur
			/seˈlet/ `piece' & CVCV- & /seˌle∼seˈlet/ `pieces' & plural\\ %lawa-lawa-t
			\lspbottomrule 
		\end{tabularx}
	
	\label{tab:partrednoun}
\end{table}


\tabref{tab:partrediverb} illustrates all found patterns with \is{stative intransitive verb}stative intransitive verbs (which are glossed without infinitive marker and copula verb to save space). A -CVC reduplicant is common, because several stative intransitive verbs end in \textit{-sik} (e.g. \textit{paransik} `near', \textit{yorsik} `straight') or \textit{-kap} (all colours) and are reduplicated like \textit{tabusik} `short' and \textit{iriskap} `white' in~(\ref{tab:partrediverb}). However, leftward reduplication with different syllable types is also attested. In fact, all examples in~(\ref{tab:partrediverb}) end in -CVC, yet the last three are not reduplicated rightward, but leftward, with varying amounts of reduplicated material.

\begin{table}[b]
	\caption{Reduplication: stative intransitive verbs}
	\begin{tabularx}{\textwidth}{llQl}
			\lsptoprule 
			base & reduplicant  & reduplicated form & function\\ \midrule 
			/tabuˈsik/ `short' & -CVC & /tabuˈsik∼sik/\newline `very short' & intensification\\
			/iˈriskap/ `white' & -CVC & /iˈriskap∼kap/\newline `very white' & intensification\\
			/ˈtemun/ `big' & CVC- & /tem∼ˈtemun/\newline `very big' & intensification\\
			/ˈalus/ `soft' & CV- & /al∼ˈalus/\newline `very soft' & intensification\\
			/gaˈwar/ `fragrant' & CVCV- & /ˌgawa∼ˈgawar/\newline `very fragrant' & intensification\\
			\lsptoprule 
		\end{tabularx}
	\label{tab:partrediverb}
\end{table}

%intrans verb: intensify -CVC
%siktak-tak
%cicaun-tu-tun
%
%intrans verb: intensify CVC-
%tem-temun (or tem-tem-un, cf cicaun, kinkinun, but not *kinun)
%gawa-gawar
%al-alus=ten

\tabref{tab:partredverb} shows all found patterns with other verbs. The reduplicated form of \textit{ecie} `to return' carries distributive marker \textit{-p} (§\ref{sec:verbdistr}). Both rightward and leftward reduplication is attested. The reduplicant -CVCVV is found with one other verb than \textit{konawaruo} `to forget' (illustrated in~\ref{tab:partredverb}), namely the similar-sounding \textit{koraruo} `to bite', which is reduplicated as \textit{koraruoraruo}. One reduplicant, -CVC, is attested four times in the corpus. It is also found with \textit{eiruk} `to squat', \textit{iskap} `to plane' and \textit{korot} `to slice'. There are not enough data to discern other patterns.

\begin{table}
	\caption{Reduplication: other verbs}

\begin{tabularx}{\textwidth}{QlQp{2cm}}
			\lsptoprule 
			base & reduplicant  & reduplicated form & function\\\midrule
			/ˈewa/\newline `to speak' & -CV & /eˈwa∼wa/\newline `to speak' & durative\\ %dorma-ma(m)
			/ecie/\newline `to return' & -CVV & /eˈcie-p∼ˌcie-p/\newline `to return' & distributive\\
			/saŋgaˈra/\newline `to search' & -CVCV & /saŋˌgara∼ˈgara/\newline `to search' & durative,\newline distributive\\
			/koˈnawaˌruo/\newline `to forget' & -CVCVV & /konaˌwaruo∼ˈwaruo/\newline `to forget' & distributive\\ %koraruo-raruo
			/bolˈkoyal/\newline `to eat' & -CVC & /bolkoˈyal∼yal/\newline `to eat' & durative\\ %iruk-ruk, is-kap-kap, ko-rot-rot
			/laˈur/\newline `to rise' & CVV- & /lau∼ˈlaur/\newline `to rise' & durative\\
			/ˈgoŋgin/\newline `to know' & CVC- & /goŋ∼ˈgoŋgin/\newline `to know' & distributive\\ %marmarua
			/saˈrut/\newline `to rip & CVCV- & /saˌru∼saˈrut/\newline `to rip' & distributive\\
			\lspbottomrule 
		\end{tabularx}
	
	\label{tab:partredverb}
\end{table}
%re-rer (not in context, only in word list)
%towari-wari (toari-oari) \textendash problematic w. spelling/V.V analysis

%There is not enough data to determine whether the verbal endings \textit{-n} and \textit{-t} (§\ref{sec:problems}) are typically reduplicated or not. Distributive suffix \textit{-p} is reduplicated.
%verb rightward
%sarakmang-mang (because morpheme mang still recognizable)

Medial reduplication is of two types: simplex and complex \parencite{rubino2005}. In the simplex type, phonological material to the left or to the right of the reduplicant is exactly reduplicated. This is illustrated in \tabref{tab:medialplain}. For most of these examples, it is arbitrary where to put the ∼'s, as reduplication could be either rightward or leftward. There are some irregularities in this type. The plural of \textit{esa} `man', formed with kinship plural marker \textit{-mur}, involves reduplication of \textit{-mu}, but also deletion of the final vowel \textit{-a}. The reduplicated form of \textit{tarauk} `snapped' involves a vowel change from /a/ to /e/. These are exceptions.

\begin{table}
	\caption{Simplex medial reduplication}
	
\begin{tabularx}{\textwidth}{QlQl}
			\lsptoprule 
			base & reduplicant & reduplicated form & function\\ \midrule 
			%		/paˈreir/ `to follow' & -CVCVV- & /ˌparei∼ˈwarei∼r/ `to follow' & durative\\
			/kaˈlomun/ `unripe' & -CVC- & /kalom∼ˈlom∼un/\newline `unripe' & distributive?\\
			/ˈes\textbf{a}/ `man' & -CV- & /es∼ˈmu∼mur/\newline `men' & plural\\
			/ˈusar/ `to erect' & -CV- & /u∼ˈsa∼sar/\newline `to erect' & distributive?\\
			/t\textbf{a}raˈuk/ `snapped' & -CV- & /t\textbf{e}ra∼ra∼ˈuk/\newline `snapped' & distributive? \\
			/naˈwaŋgar/ `to wait' & -CVCCV- & /na∼waŋˈga∼waŋˌgar/\newline `to wait' & durative\\
			\lspbottomrule 
		\end{tabularx}
	
	\label{tab:medialplain}
\end{table}
%pati-wati-n
%ka-wa-war, ka-wa-war-ma
%kowep kowewepkon (colour derivation) \textendash only example, leave out koeoepkon?
%konasuari-suari-k

In the complex type, a vowel (or vowel and consonant, in one case) from the left side of the reduplicant is used and a consonant from the right side (\tabref{tab:medialcompl}). This is found with CV, CVV and CVC reduplicants only, and is not attested with nouns. With verbs, it has the same functions as described for other types of reduplication above. Reduplicated polysyllabic numerals\is{numeral!reduplication} are always of the complex medial type and serve to create distributive numerals. (Monosyllabic distributive numerals are fully reduplicated.)

\begin{table}[t]
	\caption{Complex medial reduplication}
\begin{tabularx}{\textwidth}{llQl}
			\lsptoprule 
			base & reduplicant & reduplicated form & function\\ \midrule 
			/ˈkaɟie/ `to pick' & -CV- & /ka∼ˈɟa∼ɟie/ `to pick' & durative\\
			/ˈkojal/ `to mix' & -CV- & /ko∼ˈjo∼jal/ `to mix' & intensification\\
			/kaˈhen/ `far' & -CV- & /ka∼ˈha∼hen/ `very far' &  intensification\\
			/naˈurar/ `to turn' & -CVV- & /nau∼ˈrau∼rar/ `to turn' & durative\\
			/kaˈruok/ `three' & -CV- & /ka∼ˈra∼ruok/ `three' & distributive\\ %(monosyll: awap, konkon, purawapkin)
			/kanˈsuor/ `four' & -CVC- & /kan∼ˈsan∼suor/ `four' & distributive\\	\lspbottomrule 
		\end{tabularx}
	
	\label{tab:medialcompl}
\end{table}

A small number of words that seemingly involve reduplication have no corresponding root.

\begin{exe}
	\ex
	\begin{xlist}
		\ex /ˌmisilˈmisil/ `cement floor'
		\ex /ˈwuorˌwuor/ `to dream'
		\ex /nokˈnok/ `to whisper' (cf. /noˈkidak/ `to be silent')
		\ex /boukˈbouk/ `to bark'
		\ex /korˈtaptap/ `to cut out'
		\ex /ˈmarmar/ `to walk'
		\ex /ˌsiŋa(t)ˈsiŋat/ `ant'
		\ex /paŋˈgawaŋˌga/ `leech' 
	\end{xlist}
	\label{exe:nobase}
\end{exe}


Kalamang also makes use of repetition\is{repetition}. Repetition is distinguished from reduplication by two diagnostics: repetition applies to the domain of the word (such that each repeated word carries its own stress) and repetition may have two or more copies (whereas reduplication involves two copies only). Repetition is different from tail-head linking (§\ref{sec:tailhead}) in that it occurs within the clause. As an example, consider the repetition of \textit{yal} `to paddle' (repeated three times, each repetition carrying a main stress) and \textit{war} `to fish' in~(\ref{exe:wariw}) (repeated twice and with predicate linker \textit{=i}). Reduplication is predominantly attested with verbs to indicate iteration or duration. %cite gil 2005?
%also marmar 4x, tui 3x, mindai 3x

\begin{exe}
	\ex \gll an se koi \textbf{yal} \textbf{yal} \textbf{yal} tebol-suban \textbf{war=i} \textbf{war=i} eh sor nat=nin\\
	\textsc{1sg} {\glse} again paddle paddle paddle reef\_edge-fish fish={\gli} fish={\gli} \textsc{int.e} fish consume=\textsc{neg}\\
	\glt `I paddled and paddled again, fished at the reef edge, fished and fished, the fish didn't bite.' \jambox*{\href{http://hdl.handle.net/10050/00-0000-0000-0004-1C99-E}{[narr8\_1:02]}}
	\label{exe:wariw}
\end{exe}	 

Repetition and reduplication may be combined. In~(\ref{exe:tiririri}), the reduplicated \textit{tiri} `to sail' is repeated. There are thus two words in~(\ref{exe:tiririri}): [tiˈritiˌri] [tiˈritiˌri].

\begin{exe}
	\ex \gll warkin naman=et pi wandi siktak∼tak=i \textbf{tiri∼tiri} \textbf{tiri∼tiri}\\
	tide deep={\glet} \textsc{1pl.excl} like\_this slow∼\textsc{ints}={\gli} sail∼\textsc{prog} sail∼\textsc{prog}\\
	\glt `When the tide is deep we sail slowly like this.' \jambox*{\href{http://hdl.handle.net/10050/00-0000-0000-0004-1BCB-5}{[conv1\_1:33]}}
	\label{exe:tiririri}
\end{exe}

In a few rare cases, a reduplicated word is reduplicated. In~(\ref{exe:ewawawa}), the reduplicated reduplication is pronounced [ewaˈwaewaˌwa] (note the stress difference with a singly reduplicated \textit{ewa} `to speak': /eˈwawa/).

\begin{exe}
	\ex \gll ma-mun ma neba \textbf{ewa∼wa∼ewawa}=in {\ob}...{\cb} \textbf{ewa∼wa∼ewawa}\\
	\textsc{3sg-proh} \textsc{3sg} \textsc{ph} talk∼\textsc{prog}∼\textsc{prog}=\textsc{proh} {} talk∼\textsc{prog}∼\textsc{prog}\\
	\glt `She shouldn't talk, talk [disturbing].' \jambox*{\href{http://hdl.handle.net/10050/00-0000-0000-0004-1BA3-3}{[conv10\_15:23]}}
	\label{exe:ewawawa}
\end{exe}	

Both repetition and reduplication of reduplicated words are rather rare. It remains unclear what their function is, and whether there is a difference between the two. Perhaps one cannot distinguish between repetition and reduplication for reduplicated words that already have a primary and secondary stress, like [tiˈritiˌri].\is{reduplication!phonology|)}


\subsection{Compounding}
\label{sec:compounding}
Compounds are bases derived from two roots. Kalamang has nominal compounds, described in §\ref{sec:nomcomp}, and noun-verb compounds, described as noun incorporation in §\ref{sec:incorp}. Compounds act as single syntactic units: inflection and other marking is applied to the base.

Nominal compounds typically consist of two nominal roots. They may be further inflected with possessive suffixes (as in~\ref{exe:yapser}, with a third-person possessive) or case clitics (as in~\ref{exe:minggalorat}, with an object postposition).

\begin{exe}
	\ex \gll yap\_seran-un\\
	yam-\textsc{3poss}\\
	\glt `their yam' \jambox*{\href{http://hdl.handle.net/10050/00-0000-0000-0004-1BA3-3}{[conv10\_16:03]}}
	\label{exe:yapser}
	\ex \gll min-kalot=at kasi bersi\\
	sleep-room=\textsc{obj} give clean\\
	\glt `clean the bedroom' \jambox*{\href{http://hdl.handle.net/10050/00-0000-0000-0004-1B9B-9}{[narr41\_0:13]}}
	\label{exe:minggalorat}	
\end{exe}

Although nominal compounds typically consist of two nouns, they may also be a verb and a noun, such as \textit{min-kalot} in~(\ref{exe:minggalorat}).

Compounds may be one or two phonological words (§\ref{sec:stress} and §\ref{sec:root}). Compounds that form one phonological word (such as \textit{min-kalot} /miŋˈgalot/ `bedroom') are quite rare. Compounds with \textit{mang} `language', \textit{sontum} `person', \textit{-ca} `man' and \textit{-pas} `woman' as the second constituent always form one phonological word. In running text, compounds that form one phonological word are spelled as one word, and compounds that are two phonological words are spelled as two words. This is visible in the gloss by use of a dash for one phonological word and a dot or a space for two phonological words, respectively.

Noun-verb compounds are verbs with an incorporated object noun. Noun incorporation is a process whereby a verb is derived from the compounding of a nominal root and a verbal root \parencite{mithun1984}. For Kalamang, there are two diagnostics to determine whether or not a noun is incorporated: prosody and object marking. Incorporated nouns form a prosodic unit together with the verb, such that stress is assigned on the incorporation construction (i.e. the verb with incorporated noun). In a clause with a non-incorporated noun followed by a verb each has its own stress (see §\ref{sec:stress}). The other diagnostic is object marking. Incorporated nouns lack object marking, indicating that they are no longer an argument in the clause. Both diagnostics are illustrated in the minimal example~(\ref{exe:nounincorp}).

\begin{exe}
	\ex
	\begin{xlist}
		\ex 
		{\glll an ˈperat ˈna\\
			an per=at na\\
			\textsc{1sg}	water=\textsc{obj} consume\\
			\glt `I drink water.' 	}
		\ex 
		{\glll an ˈperna\\
			an per-na\\
			\textsc{1sg} water-consume\\
			\glt `I drink water.' \jambox*{\href{http://hdl.handle.net/10050/00-0000-0000-0004-1C60-A}{[elic\_inc\_2]}}}
		
	\end{xlist}
	\label{exe:nounincorp}
\end{exe}

Some incorporated nouns also show lenition\is{lenition}, such as \textit{muawaruo} `to cook', where the noun \textit{muap} `food' is incorporated in the verb \textit{paruo} `to make'. An example is given in~(\ref{exe:juaria}). Lenition cannot be used as a diagnostic for noun incorporation, as it happens only for some very frequent combinations. In~(\ref{exe:kurera}), the noun \textit{kurera} `basket' is incorporated in the verb \textit{paruo} `to make', but there is no lenition of the initial /p/ of \textit{paruo}.

\begin{exe}
	\ex 
	{\glll Juaria se muawaruoni koyet.\\
		Juaria se muap-paruon=i koyet\\
		Juaria {\glse} food-make={\gli} finish\\
		\glt `Juaria has cooked.' \jambox*{\href{http://hdl.handle.net/10050/00-0000-0000-0004-1BB3-0}{[narr7\_12:08]}}
	}
	\label{exe:juaria}
	\ex 
	{\glll Mama Tua kureraparuo.\\
		Mama Tua kurera-paruo\\
		Mama Tua basket-make\\
		\glt `Mama Tua making a basket.' \jambox*{\href{http://hdl.handle.net/10050/00-0000-0000-0004-1BE7-5}{[stim42\_4:26]}}
	}
	\label{exe:kurera}
	
\end{exe}


\subsection{Affixation and cliticisation}
Affixation and cliticisation are important word-formation processes. Affixes are phonologically and syntactically dependent: they attach to one word class only and are morphophonologically integrated (§\ref{sec:affix}). Clitics, on the other hand, either attach at phrase level (thus being able to be hosted by several word classes) or they do not show morphophonological integration (§\ref{sec:clitic}). In Kalamang, it is impossible to distinguish affixes and clitics based on the morphological processes they are involved in. Affixes are mainly derivational but can also be inflectional. Cliticisation is primarily used for inflection, but there are also derivational clitics.

Affixes typically derive new morphemes by placing them in another word class or a different sub-class of the same word class. For example, nominaliser \textit{-un} derives nouns from verbs (§\ref{sec:der}), agent nominaliser \textit{-et} derives agentive nouns from other nouns (§\ref{sec:nounorig}) and the possessive suffixes make possessive pronouns from pronouns (Chapter~\ref{ch:poss}). Some affixes are inflectional: for example, all the classifier prefixes which inflect numerals (§\ref{sec:clf}), plural kinship suffix \textit{-mur} (§\ref{sec:kinterms}) and prohibitive \textit{-mun}, which attaches to pronouns (§\ref{sec:proh}). The only affixes that can co-occur are plural \textit{-mur} and the possessive suffixes. %this goes against what I say above about max one affix!

\begin{exe}
	\ex \gll dudan-mur-un\\
	sibling-\textsc{kin.pl}-\textsc{3poss}\\
	\glt `his/her siblings'
\end{exe}

Clitics are mainly inflectional, and include postpositions as well as aspect and mood markers. Among the derivational clitics are attributive \textit{=ten}, which derives adjectives from verbs but is also attested in non-verbal predicates (§\ref{sec:attr}), and causative \textit{ma=} (§\ref{sec:caus}). Cliticisation always occurs after affixation. A derived noun, for example, can carry a postposition. %The stem 
\textit{Lenget} `villager' from \textit{leng} `village' and agent nominaliser \textit{-et} becomes \textit{lenget=at} when it is the last constituent of the object NP, as in~(\ref{exe:lenget}). \textit{Amkeiret} `birth parent' from \textit{amkeit} `to give birth' and agent nominaliser \textit{-et} can be inflected with animate lative \textit{=kongga}, as in~(\ref{exe:amkeiret}).

\begin{exe}
	\ex \gll ma sontum leng-et=at merengguen\\
	\textsc{3sg} person village-\textsc{nmlz}=\textsc{obj} gather\\
	\glt `He gathered the village people.' \jambox*{\href{http://hdl.handle.net/10050/00-0000-0000-0004-1BDF-0}{[narr27\_3:17]}}
	\label{exe:lenget}
	\ex \gll don wa me se amkeit-et=kongga\\
	thing \textsc{prox} {\glme} {\glse} give\_birth-\textsc{nmlz}=\textsc{an.lat}\\
	\glt `This thing comes from the birth parent.' \jambox*{\href{http://hdl.handle.net/10050/00-0000-0000-0004-1BCA-4}{[conv20\_38:53]}}
	\label{exe:amkeiret}
\end{exe}	

Enclitics are frequently combined. Postpositions are the innermost enclitics, forming the base together with their host NP. They can be followed by focus marker \textit{=a} (§\ref{sec:a}) when in argument function, as in~(\ref{exe:konia}). Several nouns with postpositions can be used in predicate function too (§\ref{sec:case}). When inflected with aspect and mood morphology or for negation, this follows the postpositions, as illustrated in~(\ref{exe:orkoree}).

\begin{exe}
	\ex
	{\gll an 	kewe=\textbf{at=a}		kon-i			paruo\\
		\textsc{1sg}	house=\textsc{obj=foc}	one-\textsc{objqnt}	make \\
		\glt `I made a house.' \jambox*{\href{http://hdl.handle.net/10050/00-0000-0000-0004-1B9B-9}{[narr41\_0:45]}}
	}
	\label{exe:konia}
	\ex \gll ka me or=\textbf{ko=te}\\
    \textsc{2sg} {\glme} back=\textsc{loc=imp}\\
    \glt `You're in the back!' \jambox*{\href{http://hdl.handle.net/10050/00-0000-0000-0004-1BC1-0}{[narr19\_5:00]}}
\label{exe:orkoree}
\end{exe} 

Certain aspect and mood enclitics may combine on the predicate as described in §\ref{sec:claint}. 

The corpus contains exactly one example of combined proclitics: reciprocal \textit{nau=} and causative \textit{ma=} on the root \textit{sem} `to be afraid'. Causative \textit{ma=} derives the bases \textit{masem} `to scare', which turns into reciprocal \textit{naumasem} `to scare each other'.
