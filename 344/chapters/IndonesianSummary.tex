\chapter[head={Ringkasan Bahasa Indonesia},
		 tocentry={Ringkasan Bahasa Indonesia [Indonesian summary]}]
	    {Ringkasan Bahasa Indonesia}
\begin{otherlanguage}{Indonesian}
Buku ini adalah tentang tata bahasa Kalamang, sebuah bahasa kecil yang terancam punah. Dalam tata bahasa ini, saya men\-je\-las\-kan fitur-fitur bahasa Kalamang se\-banyak mungkin.

Bahasa Kalamang digunakan di Indonesia bagian barat, di dua desa di sebuah pulau kecil di lepas pantai Papua. Ada kurang lebih 130 penutur bahasa Kalamang dimana semua berusia diatas 30 tahun. Anak-anak tidak belajar bahasa Kalamang dari orangtua mereka. Karena ini, bahasa Kalamang diperkirakan \mbox{akan} punah dalam 50 tahun ke depan, ketika semua penutur fasih bahasa tersebut sudah meninggal dunia. Informasi lebih lanjut tentang lingkungan dan budaya tempat bahasa Kalamang digunakan dibahas dalam Bab 1, yaitu pendahuluan.

Di Bab 2, ada ikthtisar tata bahasa Kalamang.

Dalam Bab 3, saya men\-je\-las\-kan fonetis bahasa Kalamang. Vokal dan konsonan yang digunakan dan bagaimana pengucapannya \mbox{akan} dikupas lebih lanjut dalam Bab ini. Selain itu, bab ini juga membahas tentang penggabungan suara dan aturan untuk suara ketika terjadi perubahan kata. Sebagai contoh: ketika kita menambahkan sebagian dari sebuah kata yang berawalan huruf vokal, seperti \mbox{\textit{-an}} yang artinya `milikku', menjadi sebuah kata yang berakhiran \textit{p} atau \textit{t}, maka akhiran suara kata tersebut berubah menjadi \textit{w} atau \textit{r}. Dengan begitu, kita mendapat kata \textit{kip} `ular' tetapi \textit{ki\textbf{w}-an} `ularku' dan \textit{kalot} `kamar' tetapi \textit{kalo\textbf{r}-an} `kamarku'.

Di Bab 4, saya mendefinisikan beberapa sifat penting bahasa yang diperlukan untuk menulis tata bahasa. Saya men\-je\-las\-kan apa itu kata, akar kata, imbuhan, klitik, dan partikel. Saya juga membahas empat proses penting dalam membuat sebuah kata: duplikasi ulang (pengulangan sebagian atau seluruh akar kata), gabungan kata (seperti \textit{min-kalot} `tidur-kamar', yang berarti `kamar tidur'), derivasi dan infleksi.

Dalam Bab 5, saya mendefinisikan kategori kata-kata yang ditemukan dalam bahasa Kalamang. Saya mengkategorikannya sebagai berikut, dengan masing-masing satu contoh kata:

\begin{itemize}
	\item kata kerja -- \textit{na} `minum atau makan'
	\item kata benda -- \textit{teya} `lelaki'
	\item kata ganti orang -- \textit{ka} `kamu'
	\item kata pembilang -- \textit{kaninggonie} `sembilan'
	\item kata demonstratif -- \textit{wa} `ini'
	\item kata keterangan -- \textit{koi} `lagi'
	\item kalimat pertanyaan -- \textit{tamatko} `dimana'
	\item kata penghubung -- \textit{eba} `kemudian'
	\item kata seru -- \textit{some} `tentu saja' atau `sudah mu'
\end{itemize} 

Bahasa Kalamang tidak memiliki kategori terpisah untuk kata sifat, tetapi kata sifat dapat dibuat dengan memberi klitik ten di akhir sebuah kata kerja. Misalnya, kata kerja \textit{welenggap}, yang artinya `menjadi warna biru' atau `menjadi warna hijau', dapat diubah menjadi kata sifat \textit{welenggap-ten}, yang berarti `biru' atau `hijau'. Seperti bahasa Indonesia, bahasa Kalamang juga tidak memiliki artikel.

Di Bab 6, saya mem\-beri\-kan penjelasan mendetail tentang kata benda dan men\-je\-las\-kan frasa-frasa dimana mereka biasa muncul: frasa benda. Saya men\-je\-las\-kan bahwa beberapa kata dalam bahasa Kalamang tidak dapat digunakan tanpa adanya kata kepunyaan. Saya juga men\-je\-las\-kan cara pembentukan kata benda, seperti dengan duplikasi ulang atau penggabunagn kata. Dalam frasa benda, kata benda muncul pertama, diikuti dengan kata pembilang, lalu kata kepunyaan dan kata demonstratif. Perhatikan frasa benda bahasa Kalamang berikut, dimana Anda bisa membaca bahasa Kalamang di baris pertama, terjemahan harafiah di baris kedua, dan terjemahan bebas di baris ketiga.

\begin{exe}
	\ex \gll hukat kon anggon yuwa\\
	jaring\_ikan satu milik\_saya ini\\
	\glt `jaring ikan milik saya satu ini'
\end{exe}

Di akhir frasa benda dapat ditemukan postposisi yang mengindikasikan fungsi dari frasa tersebut. Semisal, benda mati ditandai dengan klitik at, alat musik dengan \textit{ki} dan lokasi dengan \textit{ko}.

Bab 7 sampai 10 men\-je\-las\-kan tentang kategori-kategori kata yang terdapat dalam frasa benda.

Kata ganti orang, yang bisa berada di awal frasa benda, bukan kata benda, adalah topik Bab 6. Kata ganti dasar dapat dimodifikasi untuk membuat arti yang berbeda. Sebagai contoh, kata ganti \textit{mu} `mereka' dapat diubah menjadi \textit{muhutak} atau \textit{murain}, yang memiliki arti `hanya mereka', menjadi \textit{munaninggan} `mereka semua' dan \textit{muin} `milik mereka'.

Bab 8 membahas tentang kata pembilang. Beberapa hal lain saya men\-je\-las\-kan bagaimana mereka terkadang memiliki imbuhan pengklasifikasi. Dalam bahasa Kalamang Anda tidak bisa mengatakan `dua burung camar'. Sebaliknya, Anda harus menandai kata `dua' dengan awalan yang menunjukkan sesuatu tentang kata bendanya, dalam hal ini yang menunjukkan bahwa benda itu makhluk hidup (bernyawa). Awalan pengklasifikasi untuk makhluk hidup adalah \textit{et}, dan contoh untuk hal ini dapat dilihat pada contoh nomor~\ref{exe:kaskas} di bawah ini.

\begin{exe}
	\ex \gll kaskas et-eir\\
	burung\_camar \textsc{hidup}-dua\\
	\glt `dua burung camar'
	\label{exe:kaskas}
\end{exe}

Kata-kata demonstratif (Bab 10) juga memiliki bentuk yang berbeda-beda. Bentuk dasar dari kata-kata ini adalah \textit{wa} `ini', \textit{me} `itu', \textit{osa} `itu yang disana', \textit{opa} `itu yang barusan', \textit{osa} `ini/itu yang di atas sini/sana' dan \textit{yawe} `ini/itu yang di bawah sini/sana'. Maka, ketika Anda membicarakan tentang seorang anak yang sudah pernah Anda bicarakan sebelumnya, Anda bisa mengatakan \textit{tumun opa} `anak tadi'. Jika Anda membicarakan sebuah jaring ikan di bawah tangga, Anda bisa mengatakan \textit{hukat yawe} `jaring di bawah'.

Bab 11 men\-je\-las\-kan tentang kata kerja. Penjelasan ini dimulai dengan gambaran tentang kata kerja biasa dan kata kerja tidak beraturan. Kata kerja dapat dibentuk dengan penggunaan kata benda dengan dua cara: dengan duplikasi ulang (\textit{mun} `kutu' menjadi \textit{munmun} `menyelisik') dan dengan penggabungan (\textit{kofir} `kopi' + \textit{na} `minum atau makan' dapat menjadi \textit{kofirna} `kopi-minum'). Kata kerja dapat menjadi kegiatan timbal-balik ketika diberi proklitik \textit{nau} (\textit{tu} `memukul' menjadi \textit{nautu} `saling memukul'), refleksif ketika diberi \textit{un} (\textit{ganggia} `mengangkat' menjadi \textit{unganggie} `mengangkat diri sendiri') dan kausatif (sebab-akibat) ketika diberi, antara yang lain, \textit{di} (\textit{bara} `turun' menjadi \textit{dibara} `menurunkan').

Bab 12 men\-je\-las\-kan bagaimana membuat klausa dalam bahasa Kalamang. Urutan pada umumnya adalah subyek, obyek, kata kerja. Ini berarti, Anda \mbox{akan} mengatakan \textit{temun emun koup}, dengan arti harafiah `anak ibu peluk' bukan `anak peluk ibu'. Beberapa klausa yang memiliki tiga frasa nominal, seperti dalam klausa yang men\-je\-las\-kan tentang `memberi sesuatu', dapat diungkapkan dengan sangat minimalis dalam bahasa Kalamang. \textit{Ma ma} `dia dia' memiliki arti `dia mem\-beri\-kan dia sesuatu'. Bagian dalam klausa dimana biasanya diisi kata kerja disebut predikat. Jadi, dalam klausa \textit{tumun emun koup} `anak peluk ibu’, koup `peluk' adalah predikat. Dalam bahasa Kalamang, seperti dalam bahasa Indonesia, predikat tidak selalu kata kerja. Dapat juga berupa kata benda (contoh~\ref{exe:zelfi}), kata demonstratif (contoh~\ref{exe:demi}) atau kata pembilang (contoh~\ref{exe:numi}).

\begin{exe}		
	\ex \gll ma se guru\\
	dia sudah guru\\
	\glt `Dia sudah (menjadi seorang) guru.'
	\label{exe:zelfi}
	\ex \gll ma me\\
	dia itu\\
	\glt `Itu dia.'
	\label{exe:demi}
	\ex \gll im kansuor\\
	pisang empat\\
	\glt `Ada empat pisang.'
	\label{exe:numi}
\end{exe}

Predikat kompleks dijelaskan dalam Bab 13. Mereka menduduki posisi kata kerja tetapi mengandung lebih banyak bagian dibandingkan kata kerja. Terkadang predikat kompleks mengandung dua kata kerja, seperti di \textit{kuru bara} `membawa turun'. Di lain waktu, predikat kompleks bisa terdiri dari lokasi dan kata kerja, seperti \textit{sara nakalko} `naik di kepala' (yang berarti `naik ke kepala').

Dalam Bab 14, saya men\-je\-las\-kan cara-cara yang berbeda untuk mengubah predikat atau seluruh klausa. Misalnya, bagaimana cara membedakan antara peristiwa nyata (realis) dan tidak nyata (irrealis), konstruksi kalimat perintah dan larangan, men\-des\-krip\-si\-kan peristiwa sudah ataupun belum terjadi, men\-des\-krip\-si\-kan peristiwa yang sedang terjadi, dan sebagainya. Satu fitur menarik adalah bahwa kalimat larangan ditandai dua kali: pertama dengan sebuah akhiran di kata gantinya dan kedua dengan sebuah klitik di predikat.

\begin{exe}
	\ex \gll ka-mun narabir-in\\
	kamu-jangan berteriak-jangan\\
	\glt `Janganlah kamu berteriak!'
\end{exe}

Kata keterangan juga dibahas dalam Bab 14 ini.

Bab 15 membahas klausa kompleks dan penggabungan klausa. Kata sambung yang berbeda-beda (kata-kata seperti `dan' atau `tetapi') dijelaskan di sini. Cara yang biasa digunakan untuk menggabungkan klausa disebut dengan \textit{tail-head-linking}. Dalam konstruksi ini, kata-kata terakhir dalam sebuah klausa diulang di awal klausa berikutnya untuk menciptakan kepaduan dalam, misalnya, sebuah cerita. Pada umumnya dapat terlihat seperti ini: `Saya menuruni bukit. Menuruni bukit dan mengeluarkan perahu saya. Mengeluarkan dan mulai memperbaikinya. Memperbaiki kemudian saya berpikir saya ingin memanggil seorang teman. Memanggil seorang teman...' Ini biasa terlihat dalam bahasa-bahasa Papua. Kepaduan juga dapat diciptakan dengan meletakkan \textit{te} atau \textit{ta} setelah predikat sebelum melanjutkan dengan klausa baru.

Dalam Bab 16 topik dan fokus fenomena juga dijelaskan. Dalam bahasa Kalamang mereka ditandai dengan \textit{me} dan \textit{a} dan membantu pendengar untuk memahami informasi mana yang penting dalam sebuah klausa. Dalam contoh~\ref{exe:themaI}, \textit{me} menunjukkan bahwa \textit{an} `saya' adalah topik dalam klausa tersebut. Dalam contoh~\ref{exe:focusI}, akhiran \textit{a} di \textit{an} `saya' dan \textit{mu} `mereka' meletakkan fokus pada kata ganti orang untuk membuat kontras.

\begin{exe}
	\ex \gll an me watko nawanggar\\
	saya \textsc{topik} disini tunggu\\
	\glt `Untuk saya, saya menunggu di sini.'
	\label{exe:themaI}
	\ex \gll an-a watko mu-a metko\\
	saya-\textsc{fokus} disini mereka-\textsc{fokus} disana\\
	\glt `\textit{Saya} ada di sini, \textit{mereka} ada di sana.'
	\label{exe:focusI}
\end{exe}

Bab 17 men\-je\-las\-kan bagaimana narasi disusun, bagaimana menyapa orang, bagaimana kata seru digunakan, apa yang harus dikatakan ketika Anda kehilangan kata (\textit{fillers}) dan bagaimana mengutuk. Contoh kutukan adalah \textit{yuon kat mintolmaretkon} `semoga matahari mencabut hatimu'.

Motivasi dibalik buku ini datang dari keinginan untuk men\-des\-krip\-si\-kan se\-banyak mungkin bahasa di dunia, selama mereka masih digunakan. Semoga des\-krip\-si seperti ini bermanfaat bagi peneliti bahasa di masa sekarang dan masa depan, dan berkontribusi untuk pemahaman kita mengenai seperti apa bahasa itu. Buku ini disertai sebuah kamus yang berisi 3.800 kata dan sebuah arsip yang berisi rekaman yang telah diterjemahkan dan diberi catatan yang berdurasi lebih dari 15 jam.
\end{otherlanguage}

Translation from English by Dita Anissa Johar.
