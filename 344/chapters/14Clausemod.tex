\chapter{Clausal modification}
\label{ch:clausmod}
Clausal modifiers in Kalamang are words, a single particle, proclitics, enclitics and suffixes that modify the predicate or clause in one of the following ways. They can change the mood, aspect or mode of a predicate or clause, or specify the manner, temporal setting, degree or other characteristics of the state or event expressed by the predicate, such as \is{repetition}repetition or exclusivity.

\section{Overview}
\label{sec:claint}
Table~\ref{tab:modifiersall} provides an overview of the different kinds of clausal modifiers, as well as their slot in the clause and where in the chapter they are described. Negation is described in §\ref{sec:clauneg}. Clausal modifiers are structurally very diverse: they can be words, clitics on the predicate, suffixes on the pronoun, or a particle\is{auxiliary*}, so they are described according to function from §\ref{sec:TAM} onwards.\is{passive*|see{clausal modification}}\is{evidential*|see{clausal modification}}\is{mirative*|see{clausal modification}}\is{clausal modification}

\begin{table}
	\caption{Predicate and clausal modifiers}
	\label{tab:modifiersall}
	
                \fittable{
		\begin{tabular}{lllll}
			\lsptoprule 
			slot & form & gloss/function & kind of modifier& reference\\ \midrule 
			1 & \textit{bisa} & `can' & modal & §\ref{sec:modal}\\
			& \textit{harus} & `must' & modal & §\ref{sec:modal}\\
			&div.&div.& temporal adverbials & §\ref{sec:tempadv}\\ \midrule 
			Subj&&&&\\
			2 & \textit{-mun} & prohibitive & mood & §\ref{sec:proh}\\
			& \textit{-re} & apprehensive & mood & §\ref{sec:apprmood}\\
			& \textit{=taet} & `again' & adverbial & §\ref{sec:taet}\\\midrule 
			3 & \textit{se} & iamitive (`already') & aspect & §\ref{sec:setok}\\\midrule 
			& \textit{tok} & nondum (`not yet') & aspect & §\ref{sec:setok}\\
			& \textit{bisa} & `can' & modal & §\ref{sec:modal}\\
			& \textit{harus} & `must' & modal & §\ref{sec:modal}\\
			& \textit{gen} & `maybe' & modal & §\ref{sec:modal}\\\midrule 
			4 & \textit{suka}-\textsc{poss} & `not like; not want' & modal & §\ref{sec:modal}\\
			& \textit{koi} & `again' & adverbial & §\ref{sec:taet}\\ \midrule 
			%4b toni
			Obj&&&&\\\midrule
			5 &div.& div. &manner adverbials & §\ref{sec:manner}\\ \midrule
			Pred &&&&\\
			6 & \textit{=teba} & progressive & aspect & §\ref{sec:teba}\\
			& \textit{=te} & imperative & mood & §\ref{sec:imp}\\
			& \textit{=in} & prohibitive & mood & §\ref{sec:proh}\\
			& \textit{=kin} & volitional & mood & §\ref{sec:kin}\\
			& \textit{=ero} & conditional & mood & §\ref{sec:condmood}\\
			& \textit{=i koyet} & completive & aspect & §\ref{sec:compl}\\
			& \textit{=nin} & negation & \textendash & §\ref{sec:clauneg}\\
			& div. & div. & degree adverbials & §\ref{sec:degradv}\\
			& \textit{=taet}& `again' & adverbial & §\ref{sec:taet}\\
			7 & \textit{=teba} & progressive & aspect & §\ref{sec:teba}\\
			& \textit{=et} & irrealis & mood & §\ref{sec:et}\\\midrule 
			8  & \textit{bisa} & `can' & modal & §\ref{sec:modal}\\
			& \textit{reon} & `maybe' & modal & §\ref{sec:modal}\\
			& \textit{eranun} & `cannot' & modal & §\ref{sec:modal}\\
			& \textit{weinun} & `too' & adverbial & §\ref{sec:too}\\  			
			\lspbottomrule
		\end{tabular}
		}
	
\end{table}

There is one slot before the subject, three slots between the subject and the object, two between object and predicate, and three after the predicate. Modifiers that occupy the same slot are mutually exclusive, with the exception of slots 6 and 7, where the waters are a bit murky (see comments below). Some slots (2, 6, 7) are not for words, but for dependent morphology only. The template for clausal modification is given in~(\ref{exe:tamplate}).

\begin{exe}
	\ex 1 | subject-2 | 3 | 4 | object | 5 | predicate=6=7 | 8
	\label{exe:tamplate}
\end{exe}

It is impossible to capture all the details of predicate and clausal modification in the table, so please note the following things. First, several modifiers can occur in more than one slot. This is the case for modal markers \textit{bisa} `can' and \textit{harus} `must'. \textit{Bisa} `can', moreover, can also be the predicate itself, and inflect for irrealis \textit{=et}. Second, some modifiers are discontinuous and occur in several slots at the same time. The prohibitive is formed by simultaneously placing suffix \textit{-mun} on the subject and clitic \textit{=in} on the predicate. Modal marker \textit{suka}-\textsc{poss} `not like; not want' must either be combined with a negated verb, or can be the predicate itself if followed by propositional negator \textit{ge}. The position of negator \textit{=nin} is indicated in the table. Negation is described in §\ref{sec:clauneg}. Third, the use of modal marker \textit{eranun} `cannot' turns the preceding predicate into a noun. Fourth, some of the predicate enclitics in 6 and 7 can co-occur, while cannot. Completive \textit{=i koyet} and intensifier \textit{=tun} may be followed by irrealis \textit{=et}. Volitional \textit{=kin} (as well as negator \textit{=nin}) can be followed by progressive \textit{=teba} and irrealis \textit{=et}. Progressive \textit{=teba}, in turn, may also be followed by irrealis \textit{=et} and is therefore listed in both slots 6 and 7. Fifth, the placement of modal markers in slot 3 relative to the aspectual particle \textit{se} and word \textit{tok}, also given in slot 3, is not entirely clear. More data are needed to see if it is appropriate to posit another slot between current 2 and 3.

The following examples show different combinations of modifiers.

\begin{exe}
	\ex \glll an \textbf{koi} cat=\textbf{kin=teba}\\
	Subj 4 Pred=6=7\\
	\textsc{1sg} again paint.\textsc{mly=vol}={\glteba}\\
	\glt `I want to go painting again.' \jambox*{\href{http://hdl.handle.net/10050/00-0000-0000-0004-1BDD-5}{[narr42\_32:56]}}
	\ex \glll \textbf{go\_dung} inier \textbf{se} ter-nan=\textbf{i} \textbf{koyet} {\ob}...{\cb}\\
	1 Subj 3 Pred=6\\
	morning \textsc{1du.ex} {\glse} tea-consume={\gli} finish {\ob}...{\cb}\\
	\glt `In the morning, after drinking tea, {\ob}...{\cb}.' \jambox*{\href{http://hdl.handle.net/10050/00-0000-0000-0004-1BAE-4}{[narr44\_19:43]}}
	\ex \glll kain me ka-\textbf{mun} \textbf{tok} narorar=\textbf{in}\\
	Obj {} Subj-2 3 Pred-6\\
	\textsc{2sg.poss} {\glme} \textsc{2sg-proh} yet drag=\textsc{proh}\\
	\glt `Yours, don't drag [it] yet!' \jambox*{\href{http://hdl.handle.net/10050/00-0000-0000-0004-1BC9-2}{[conv5\_0:50]}}
	\ex \glll pi \textbf{koi} bo Kanastangan=ko=\textbf{teba=et} \textbf{reon}\\
	Subj 4 Pred Pred=\textsc{loc}=6=7 8\\
	\textsc{1pl.incl} again go Kanastangan=\textsc{loc}={\glteba}={\glet} maybe\\
	\glt `Shall we maybe go to Kanastangan again?' \jambox*{\href{http://hdl.handle.net/10050/00-0000-0000-0004-1BAE-4}{[narr44\_18:23]}}
	\ex \glll \textbf{loi} nasambung=\textbf{te} nasambung=\textbf{te} raor=ko=\textbf{te}\\
	5 Pred=6 Pred=6 Pred=6\\
	quickly connect=\textsc{imp} connect=\textsc{imp} middle=\textsc{loc=imp}\\
	\glt `Quickly connect, connect, in the middle!' \jambox*{\href{http://hdl.handle.net/10050/00-0000-0000-0004-1BCB-5}{[conv1\_6:27]}}
\end{exe}	

The next part of this chapter outlines the particularities of all predicate and clausal modifiers, starting with mood, aspect and modality marking (§\ref{sec:TAM}), followed by the different adverbial modifiers in §\ref{sec:adv}.


\section{Mood, aspect and modality marking}
\label{sec:TAM}
Mood, aspect and modality markers are clausal modifiers indicating the speakers' attitudes towards what they are saying, or the internal temporal constituency of a state or event. Most of these are clitics that attach to the predicate, one is a particle, and some (namely modal markers) are independent words. Because they are structurally so diverse, they are described by function. I describe irrealis mood (§\ref{sec:et}), volitional mood (§\ref{sec:kin}), the imperative and prohibitive moods (§\ref{sec:imp}), aspectual markers iamitive \textit{se} and \textit{tok} `still; yet; first' (§\ref{sec:setok}), progressive aspect (§\ref{sec:teba}), completive aspect (§\ref{sec:compl}) and modal markers (§\ref{sec:modal}). Kalamang has no tense\is{tense*} marking. One modal and one mood marker are described elsewhere: the modal apprehensive marker \textit{-re} and other strategies to form apprehensive constructions are described in §\ref{sec:appreh}, and conditional mood marker \textit{=o}/\textit{=ero} and other strategies for making conditional clauses are described in §\ref{sec:condclause}.

\subsection{Mood}\is{mood|(}
Mood markers indicate the speaker's attitude towards the event or condition in their utterance. Kalamang morphologically marks irrealis, volitional, imperative, prohibitive, conditional and apprehensive mood by means of enclitics on the predicate and/or suffixes on the subject. Mood markers are found in slot 2 (on the subject) and slots 6 and 7 (attaching to the predicate).

\subsubsection{Irrealis \textit{=et} `\textsc{irr}'}\is{irrealis}
\label{sec:et}
Irrealis mood is marked by enclitic \textit{=et} on the predicate. This is a very versatile mood, used in all kinds of hypothetical or prospective situations. It has not been investigated exactly which semantic categories are (not) encoded with irrealis \textit{=et}, and whether irrealis marking is obligatory for any or all of these. 

(\ref{exe:seteba}) and~(\ref{exe:diwatkok}) are taken from a recording where two women sit with different kinds of fishing gear and explain how they (would) use it. In~(\ref{exe:sortam}), someone asks what they should do with their newly caught fish. (\ref{exe:rumasa}) expresses a command and a possible result or consequence. Irrealis \textit{=et} comes at the end of a breath group.

\begin{exe}
	\ex \glll Set. Eba pi dibararet. Lot me tagier o.\\
	set eba pi di=barat=\textbf{et} lot me tagier o\\
	bait then \textsc{1pl.incl} \textsc{caus}=descend={\glet} sinker {\glme} heavy \textsc{emph}\\
	\glt `Bait. Then we lower it down... Wow, this sinker is heavy.' \jambox*{\href{http://hdl.handle.net/10050/00-0000-0000-0004-1C75-D}{[stim15\_0:20]}}
	\label{exe:seteba}
	\ex \glll Nika wa ba mena pi diwatko kanieret, watko kanieret, eba pi muet.\\
	nika wa ba mena pi di=watko kaniet=\textbf{et} watko kaniet=\textbf{et} eba pi muk=\textbf{et}\\
	fishing\_line \textsc{prox} then later \textsc{1pl.incl} \textsc{caus=prox.loc} tie={\glet} here tie={\glet} then \textsc{1pl.incl} throw={\glet}\\
	\glt `This fish line, later we'd tie it here, tie it here, then we'd throw [it].' \jambox*{\href{http://hdl.handle.net/10050/00-0000-0000-0004-1C75-D}{[stim15\_2:30]}}
	\label{exe:diwatkok}
	\ex \glll Ma toni: ``Eh, sor wa me tamandi, pi parinet ye, pi parairet, siraet.''\\
	ma toni eh sor wa me tamandi pi parin=\textbf{et} ye pi parair=\textbf{et} sira=\textbf{et}\\
	\textsc{3sg} say hey fish \textsc{prox} {\glme} how \textsc{1pl.incl} sell={\glet} or \textsc{1pl.incl} split={\glet} salt={\glet}\\
	\glt `He said: ``Hey, these fish, how [should we treat them]? Do we sell them, or do we split and salt them?' \jambox*{\href{http://hdl.handle.net/10050/00-0000-0000-0004-1C99-E}{[narr8\_5:34]}}
	\label{exe:sortam}
	\ex \glll Kuru masara in jieret!\\
	kuru masara in jiet=\textbf{et}\\
	bring move\_landwards \textsc{1pl.excl} buy={\glet}\\
	\glt `Bring [it] towards land, [so that] we [can] buy!' \jambox*{\href{http://hdl.handle.net/10050/00-0000-0000-0004-1BC1-0}{[narr19\_4:04]}}
	\label{exe:rumasa}
\end{exe}	

Irrealis \textit{=et} is also common in expository texts, where it is usually alternated with predicates that are not marked for irrealis. Consider~(\ref{exe:ramasii}).

\begin{exe}
	\ex 
	\begin{xlist}	
		\ex \gll wa me kulun=at kaware=\textbf{et} me\\
		\textsc{prox} {\glme} skin=\textsc{obj} grate={\glet} {\glme}\\
		\glt `This, [you] grate the skin,'
		\ex \gll naramas=i koyet\\
		squeeze={\gli} finish\\
		\glt `after squeezing,'	
		\ex \gll eh naramas=i koyet metko di=mu ∅=\textbf{et}\\
		\textsc{fil} 	squeeze={\gli} finish \textsc{dist.loc} \textsc{caus}=\textsc{3pl} give={\glet}\\
		\glt `after squeezing, give [it] to them,'
		\ex \gll eba di=mu ∅=te mu nan=\textbf{et}\\
		then \textsc{caus}=\textsc{3pl} give={\glte} \textsc{3pl} consume={\glet}\\
		\glt `then, [you] give it to them and they drink [it].' \jambox*{\href{http://hdl.handle.net/10050/00-0000-0000-0004-1BB1-3}{[narr34\_2:58]}}
	\end{xlist}
	\label{exe:ramasii}
\end{exe}	

\is{conditional}Conditional clauses also make use of irrealis \textit{=et} on the predicate expressing the condition. They are treated in §\ref{sec:condclause}. Note that volitional \textit{=kin}, treated in §\ref{sec:kin}, combines with irrealis \textit{=et} if it is used in its conditional sense, as in~(\ref{exe:saor}). The volitional marker always precedes the irrealis marker.

\begin{exe}
	\ex \gll et=at saor=kin=\textbf{et} me isetengamati\\
	canoe=\textsc{obj} anchor={\glkin}={\glet} {\glme} big\_effort\\
	\glt `[Before], if you wanted to anchor your boat, it was very difficult.' \jambox*{\href{http://hdl.handle.net/10050/00-0000-0000-0004-1B9F-F}{[conv9\_32:42]}}
	\label{exe:saor}
\end{exe}

The clitic \textit{=et} is also found on the consequence clause in clauses joined by sequential conjunction \textit{(e)ba} (§\ref{sec:seqconj}). The use of \textit{=et} in topic constructions is described in §\ref{sec:discme}.


\subsubsection{Volitional \textit{=kin} `\textsc{vol}'}\is{volitional}
\label{sec:kin}
Volitional mood enclitic \textit{=kin} attaches to the predicate. It is mainly used to express plans and wishes, as well as events that are about to happen. The different uses of \textit{=kin} are exemplified in~(\ref{exe:kin}) to~(\ref{exe:kin2}). (\ref{exe:kin}) expresses a plan or wish. In~(\ref{exe:kintoni}), the common use of \textit{=kin} with \textit{toni} `want' is illustrated.\footnote{It is unclear whether \textit{toni} `want' can be used without volitional \textit{=kin}. \textit{Toni} can also mean `say; think' (§\ref{sec:mvctoni}), so many of the corpus instances of \textit{toni} followed by a verb that is not marked with \textit{=kin} are ambiguous between the `want' and the `say; think' reading.} The phrase in~(\ref{exe:kinabout}) is used to describe the twelfth picture in the \textit{Family problems picture task} \parencite{carroll2009}, where a child is falling out of its mother's arms, and illustrates the prospective use of \textit{=kin}. (\ref{exe:kin2}) is the first utterance in a text where a woman explains how to make a basket. It can be interpreted as an action that is about to happen (at least in the speaker's mind) or as a plan. \is{prospective} \is{irrealis} \is{mood} \is{volitional}

\begin{exe}
	\ex
	\label{exe:kin}
	{\gll	bal napikir an sor=at nat=\textbf{kin}\\
		dog think \textsc{1sg} fish=\textsc{obj} consume=\textsc{vol}\\
		\glt `The dog thinks: ``I want to eat the fish''.' \jambox*{\href{http://hdl.handle.net/10050/00-0000-0000-0004-1BB9-6}{[stim1\_0:34]}}
	}
\end{exe}

\begin{exe}
	\ex
	\label{exe:kintoni}
	{\gll	an se toni min=\textbf{kin}\\
		\textsc{1sg} \textsc{iam} want sleep=\textsc{vol}\\
		\glt `I already wanted to sleep.' \jambox*{\href{http://hdl.handle.net/10050/00-0000-0000-0004-1B8F-4}{[narr32\_0:18]}}
	}
\end{exe}

\begin{exe}
	\ex
	\label{exe:kinabout}
	{\gll tumun tur=\textbf{kin}\\
		child fall={\glkin}\\
		\glt `The child is about to fall.' \jambox*{\href{http://hdl.handle.net/10050/00-0000-0000-0004-1BA9-9}{[stim6\_2:35]}}
	}
\end{exe}

\begin{exe}
	\ex
	\label{exe:kin2}
	{\gll kiem=at paruot=\textbf{kin}\\
		basket=\textsc{obj} make={\glkin}\\
		\glt `[How to] make a basket.' \jambox*{\href{http://hdl.handle.net/10050/00-0000-0000-0004-1BB8-C}{[narr11\_0:04]}}
	}
\end{exe}

Volitional \textit{=kin} can be followed by progressive aspect marker \textit{=teba} (§\ref{sec:teba}), also an enclitic on the predicate. The combination expresses that a plan is in progress.\footnote{In examples like this, I do not know whether the volitional marker refers to the third person plural subject or to the subject of the matrix clause.}

\begin{exe}
	\ex \gll an toni mu bo os-rep=\textbf{kin=teba}\\
	\textsc{1sg} think \textsc{3pl} go sand-get={\glkin}={\glteba}\\
	\glt `I thought they went to get sand.' \jambox*{\href{http://hdl.handle.net/10050/00-0000-0000-0004-1B9F-F}{[conv9\_1:22]}}
	\label{exe:orsep}
\end{exe}

(An example of volitional \textit{=kin} and irrealis \textit{=et} can be found in~(\ref{exe:saor}) in §\ref{sec:et}.)

In \is{predicate!complex}complex predicates, volitional \textit{=kin} comes on the last constituent of the predicate and has scope over the entire action or event, such as the construction \textit{kuru masara} `bring towards land' in~(\ref{exe:botak}).

\begin{exe}
	\ex \gll mu se taruo toni mena Botak kiun=at kuru masarat=\textbf{kin}\\
	\textsc{3pl} {\glse} say say later Botak wife.\textsc{3poss=obj} bring move\_landwards={\glkin}\\
	\glt `They already said that soon Botak will bring his wife. \jambox*{\href{http://hdl.handle.net/10050/00-0000-0000-0004-1BDC-D}{[conv8\_22:22]}}
	\label{exe:botak}
\end{exe}

\subsubsection{Imperative \textit{=te} `\textsc{imp}'}\is{imperative}
\label{sec:imp}
Imperative mood, used for commands, is formed by adding \textit{=te} (singular, though occasionally used for plural) or \textit{=tar} (plural\is{plural!verbal}) to the predicate. The subject may be elided. (\ref{exe:meieba}) shows the singular imperative cliticised to the verb \textit{kome} `look', and~(\ref{exe:metkore}) shows it cliticised to the locative demonstrative \textit{metko} `there', which functions as the predicate in that clause. Of the locative predicates that are marked with the imperative, it only occurs on those locative predicates that express movement towards a goal, not on locatives that express static \is{location}location (§\ref{sec:loc}). The plural imperative\is{person marking}\is{indexing|see{person marking}} is illustrated in~(\ref{exe:borar}). Imperative \textit{=te} follows aspectual marker \textit{=teba}, as in~(\ref{exe:tebare}).

\begin{exe}
	\ex \gll nene mei eba kome=\textbf{te}\\
	grandmother come.\textsc{imp} then look=\textsc{imp}\\
	\glt `Grandmother come and look!' \jambox*{\href{http://hdl.handle.net/10050/00-0000-0000-0004-1BBD-5}{[conv12\_10:44]}}
	\label{exe:meieba}
	\ex \gll mu toni ka metko=\textbf{te}\\
	\textsc{3sg} say \textsc{2sg} \textsc{dist.loc}=\textsc{imp}\\
	\glt `They said: ``You go there!''' \jambox*{\href{http://hdl.handle.net/10050/00-0000-0000-0004-1BC1-0}{[narr19\_16:48]}}
	\label{exe:metkore}
	\ex \gll ki sontum kansuor ma=bon bo=\textbf{tar}\\
	\textsc{2pl} person four \textsc{3sg=com} go=\textsc{pl.imp}\\
	\glt `You four people go with him!' \jambox*{\href{http://hdl.handle.net/10050/00-0000-0000-0004-1BDE-7}{[narr25\_7:45]}}
	\label{exe:borar}
	\ex \gll ka marmar=teba=\textbf{te} ka-mun sara=in\\
	\textsc{2sg} walk={\glteba}=\textsc{sg.imp} \textsc{2sg}-\textsc{proh} ascend=\textsc{proh}\\
	\glt `You just walk, don't you get up [your bike]!' \jambox*{\href{http://hdl.handle.net/10050/00-0000-0000-0004-1BD3-9}{[stim33\_1:27]}}
	\label{exe:tebare}
\end{exe}

Directional verbs (which end in \textit{-a}) and transitive verbs ending in \textit{-ma}, described in §\ref{sec:dir} and~§\ref{sec:macie}, have imperative forms where final \textit{-a} is replaced with \mbox{\textit{-ei}}. For example, \textit{bara} `go down' has the imperative form \textit{barei} (example~\ref{exe:barei}) and \textit{potma} has the imperative form \textit{potmei} (example~\ref{exe:potmei}). These are also both examples of \is{predicate!complex}complex predicates, which show that the imperative is marked on the second verb, and has scope over the whole construction. An exception is the imperative of \textit{mia}, which becomes not \textit{miei} but \textit{mei}, as in~(\ref{exe:meieba}) above.

\begin{exe}
	\ex \gll mena ka nasuat=i \textbf{barei}\\
	then \textsc{2sg}  tuck={\gli} descend.\textsc{imp}\\
	\glt `Then you tuck down.' \jambox*{\href{http://hdl.handle.net/10050/00-0000-0000-0004-1BB5-B}{[conv17\_40:20]}}
	\label{exe:barei}
	\ex \gll ka neba=at era \textbf{potmei}\\
	\textsc{2sg} \textsc{ph=obj} ascend cut.\textsc{imp}\\
	\glt `You go up and cut a whatsitsname!' \jambox*{\href{http://hdl.handle.net/10050/00-0000-0000-0004-1BC1-0}{[narr19\_12:40]}}
	\label{exe:potmei}
\end{exe}

The plural imperative form of all \textit{-n/-t} verbs is likely \textit{=r}. At this point there is evidence from directional verbs like \textit{sara} `go up', illustrated in~(\ref{exe:sarar}), and other \textit{-n/-t} verbs such as \textit{na} `consume' and \textit{gocie} `to stay', illustrated in~(\ref{exe:nar}) and~(\ref{exe:gocier}), respectively.

\begin{exe}
	\ex \gll muap se kalar sara=\textbf{r} o\\
	food {\glse} ready ascend=\textsc{pl.imp} \textsc{emph}\\
	\glt `The food is ready, come up!' \jambox*{\href{http://hdl.handle.net/10050/00-0000-0000-0004-1BB3-0}{[narr7\_12:10]}}
	\label{exe:sarar}
	\ex \gll coba ki wat yuwat na=\textbf{r}\\
	try \textsc{2pl} coconut \textsc{prox.obj} consume=\textsc{pl.imp}\\
	\glt `You guys try eat this coconut!' \jambox*{\href{http://hdl.handle.net/10050/00-0000-0000-0004-1BA2-F}{[conv11\_4:01]}}
	\label{exe:nar}
	\ex \gll kier tumtum karuok=bon gocie=\textbf{r}\\
	\textsc{2du} children four=\textsc{com} stay=\textsc{pl.imp}\\
	\glt `You two and the four children stay!' \jambox*{\href{http://hdl.handle.net/10050/00-0000-0000-0004-1BC3-B}{[conv7\_4.09]}}
	\label{exe:gocier}
\end{exe}

%Reminiscent of plural suffix remnants in Makalero, Makasae and Fataluku (TAP vol 2) with -Vr.

\subsubsection{Prohibitive \textit{=in} `\textsc{proh}'}\is{prohibitive}
\label{sec:proh}
Prohibitive mood is expressed with a dedicated construction involving suffix \mbox{\textit{-mun}} on the pronominal subject and clitic \textit{=in} on the predicate. (\ref{exe:ewain2}) shows a clause in prohibitive mood with \textit{=in} on a verbal predicate, and (\ref{exe:koineh}) shows \textit{=in} on a locative predicate. There is no distinct plural form of the prohibitive. The combination of \textit{-mun} and negator \textit{=nin} is ungrammatical.\footnote{Alternatively, one could say that prohibitive \textit{=in} triggers the use of special prohibitive pronoun forms (since the form \textit{-mun} only occurs on pronouns).}

\ea {\gll ka-\textbf{mun} se reidak-i ewa=\textbf{in}\\
	\textsc{1sg-proh} {\glse} much-\textsc{objqnt} speak=\textsc{proh}\\
	\glt 	`Don't you speak much anymore!' \jambox*{\href{http://hdl.handle.net/10050/00-0000-0000-0004-1BAA-C}{[stim7\_21:00]}}
}  \label{exe:ewain2}
\z 
\ea  {\gll an toni sor-kang me ki-\textbf{mun} gareor=i pasier=ko=\textbf{in} eh\\
	\textsc{1sg} say fish-bone {\glme} \textsc{2pl-proh} dump={\gli} sea=\textsc{loc=proh} \textsc{tag}\\
	\glt `I said those fish bones, don't you guys dump [them] in the sea!' \jambox*{\href{http://hdl.handle.net/10050/00-0000-0000-0004-1BA3-3}{[conv10\_14:05]}}
}	\label{exe:koineh}
\z 
\ea [*]  {\gll ka-mun tiri=\textbf{nin}\\
	\textsc{2sg-proh} run=\textsc{proh}\\
	\glt `Don't you run!' \jambox*{\href{http://hdl.handle.net/10050/00-0000-0000-0004-1C60-A}{[elic\_proh\_11]}}
}	\label{exe:munnin}
\z   

Other referents than second-person singular may be the subject of prohibition, such as `smoke' in~(\ref{exe:sarein2}). Because \textit{-mun} cannot be attached to nouns, the noun is preceded by a pronoun. Like in the imperative mood, prohibitive forms of directional verbs and transitive verbs in \textit{-ma} are different: instead of ending in \textit{-a} + \textit{-in}, they become \textit{-ein}. (\ref{exe:mumun}) illustrates a prohibitive with a third person plural.

\ea \label{exe:sarein2}
{\glll Mamun dugar sarein!\\
	 ma-\textbf{mun} dugar \textbf{sara=in}\\
	\textsc{3sg-proh} smoke ascend=\textsc{proh}\\
	\glt 	`The smoke must not rise!' \jambox*{\href{http://hdl.handle.net/10050/00-0000-0000-0004-1BBB-2}{[narr40\_8:01]}}
}
\z
\ea \label{exe:mumun}
\gll mu-\textbf{mun} narabi=in\\
\textsc{3pl-proh} make\_noise=\textsc{proh}\\
\glt `They shouldn't make noise!' \jambox*{\href{http://hdl.handle.net/10050/00-0000-0000-0004-1BBB-2}{[narr40\_21:42]}}
\z 

As illustrated in~(\ref{exe:sareinn}), \textit{-mun} cannot be suffixed to nouns. When using someone's title or name, the pronoun with \textit{-mun} follows the noun in such a vocative\is{vocative} example, as exemplified in~(\ref{exe:esaproh2}).

\ea 	
\ea [*] 
{\gll \textbf{esa-mun} sara=in\\
	father-\textsc{proh} ascend=\textsc{proh}\\
	\glt `Father, don't go up!' 
} \label{exe:sareinn}
\ex  [] 
{\gll \textbf{esa} ka-\textbf{mun} sara=in\\
	father \textsc{2sg-proh} ascend=\textsc{proh}\\
	\glt `Father, don't (you) go up!' \jambox*{\href{http://hdl.handle.net/10050/00-0000-0000-0004-1C60-A}{[elic\_proh\_17]}}
} \label{exe:esaproh2}
\z \z 

When the pronoun is elided, \textit{-mun} is elided as well, such as in example~(\ref{exe:tin2}).

\ea \label{exe:tin2}
{\gll bo kuet=te tik=\textbf{in}\\
	go bring={\glte} be\_long=\textsc{proh}\\
	\glt `Don't [you] bring it for a long time!' \jambox*{\href{http://hdl.handle.net/10050/00-0000-0000-0004-1BDD-5}{[narr42\_34:19]}}
}
\z 

Most clausal modifiers are incompatible with the prohibitive. The aspectual particle \textit{se} `already' (iamitive) and the aspectual word \textit{tok} `yet; still; first' (\is{nondum}nondum), however, are compatible. The combination of the iamitive \textit{se} and the prohibitive results in the meaning `not anymore' (i.e. the iamitive has scope over the prohibitive). The nondum \textit{tok} plus a prohibitive results in the meaning `not yet', illustrated in~(\ref{exe:muntok2}). This is parallel to the meanings of the iamitive and nondum with regular verbal negation\is{negation}, as illustrated in Table~\ref{tab:setok} in §\ref{sec:setok}.

\ea 	\label{exe:muntok2}
{\gll ki-\textbf{mun} tok na=\textbf{in}\\
	\textsc{2pl-proh} yet eat=\textsc{proh}\\
	\glt `Don't you eat yet!'\jambox*{\href{http://hdl.handle.net/10050/00-0000-0000-0004-1BA2-F}{[conv11\_3:41]}}}
\z 

The prohibitive may follow a cliticised adverbial such as \textit{=sawe(t)} `too', as exemplified in~(\ref{exe:pimun}).

\begin{exe}
	\ex \gll nasuena bolon-i baran pi-\textbf{mun} talalu pen=\textbf{sawet=in} o\\
	sugar little-\textsc{objqnt} descend \textsc{1pl.incl-proh} too sweet=too=\textsc{proh} \textsc{emph}\\
	\glt `[We] put in a little sugar, we shouldn't make it too sweet!' \jambox*{\href{http://hdl.handle.net/10050/00-0000-0000-0004-1BA2-F}{[conv11\_1:55]}}
	\label{exe:pimun}
\end{exe}

In \is{predicate!complex}complex predicates, it is the last constituent that carries prohibitive \textit{=in}. It has scope over the whole construction. (\ref{exe:ewainmelelu2}) is a single clause.

\ea \label{exe:ewainmelelu2}
\gll ka-\textbf{mun} melelu ewa=\textbf{in}\\
\textsc{2sg-proh} sit speak=\textsc{proh}\\
\glt `Don't you sit speaking!' \jambox*{\href{http://hdl.handle.net/10050/00-0000-0000-0004-1C60-A}{[elic\_neg19\_17]}}
\z 

One may combine a prohibitive and an imperative in one message as in~(\ref{exe:melelure2}), which is biclausal as indicated by square brackets.

\ea \label{exe:melelure2}
\gll {\ob}ka-\textbf{mun} ewa=\textbf{in}{\cb} {\ob}melelu{\cb}\\
\textsc{2sg-proh} speak=\textsc{proh} sit.\textsc{imp}\\
\glt `Don't you speak, sit!' \jambox*{\href{http://hdl.handle.net/10050/00-0000-0000-0004-1C60-A}{[elic\_neg19\_16]}}
\z 

General prohibition, not directed at a specific person, is expressed with \textit{ka-mun} or \textit{ki-mun}, second-person singular or plural, respectively. One could use this, for example, on a prohibition sign. In spoken Kalamang, one may also leave out the verb, and just use the second person \textit{ka-mun!} `don't!'.


\largerpage[2]
\subsubsection{Conditional \textit{=o/=ero} `\textsc{cond}'}\is{conditional|(}
\label{sec:condmood}
Conditional mood is made with conditional clitic \textit{=o} or \textit{=ero}, which attaches to the predicate with the condition. The condition is followed by a new clause with the consequence. It occurs about ten times in the corpus, and mostly in contexts with several conditions. The two variants of the clitic are given in~(\ref{exe:warobes}) and~(\ref{exe:geroo}). (\ref{exe:geroo}) also shows a negative condition. (\ref{exe:nanere}) shows a possible third variant, \textit{=ere}, but it occurs only in that utterance. Note that the consequence is expressed as just a predicate in all cases in the corpus. There is not enough data to determine for which types of conditions (e.g. implicative, predictive, indicative, counterfactual) these clitics are used.

\begin{exe}
	\ex \gll jadi tanaman pun demekian wat=\textbf{o} bes im=\textbf{o} bes sayang=\textbf{o} bes\\
	so plant even thus coconut=\textsc{cond} good banana=\textsc{cond} good nutmeg=\textsc{cond} good\\
	\glt `So whichever plant [we grow], whether it's coconut, banana or nutmeg, it's good.' \jambox*{\href{http://hdl.handle.net/10050/00-0000-0000-0004-1BD0-8}{[narr13\_2:50]}}
	\label{exe:warobes}
	\ex \gll ka-tain=a toni ka Tarus-pis=i bo=\textbf{ero} se bot ge=\textbf{ero} ge mu bukan in kat paksa=nin\\
	\textsc{2sg}-alone=\textsc{foc} say \textsc{2sg} Tarus-side={\gli} go=\textsc{cond} {\glse} go not=\textsc{cond} no \textsc{int} not \textsc{1pl.excl} \textsc{2sg.obj} force=\textsc{neg}\\
	\glt `You said yourself if you wanted to go to Tarus, you go, if not, not, we didn't force you.' \jambox*{\href{http://hdl.handle.net/10050/00-0000-0000-0004-1B9F-F}{[conv9\_16:35]}}
	\label{exe:geroo}
	\ex \gll ter nan=\textbf{ere} bisa kai nan=\textbf{ere} bisa\\
	tea consume=\textsc{cond} can medicine consume=\textsc{cond} can\\
	\glt `So [you] can drink it like tea and [you] can drink it like medicine.' [Lit.: `If you drink it like tea, that's possible, if you take it like medicine, that's possible.'] \jambox*{\href{http://hdl.handle.net/10050/00-0000-0000-0004-1BB1-3}{[narr34\_0:31]}}
	\label{exe:nanere}
\end{exe}
%also elicited ka nanero na, ka muawero muap(te). 

A variant on the conditional is \textit{=taero} `even if', which is a concessive that indicates that the condition in the first clause does not prevent the statement in the second clause from being true. Like \textit{=ero}, \textit{=taero} `even if' attaches to the condition, i.e. the predicate of the first clause, which is followed by a predicate-only clause with the consequence. The clitic is frequently combined with Malay loan \textit{biar} `even if', which precedes the conditional clause, without a change in meaning.

\begin{exe}
	\ex \gll bes bes wa me anti pasier=tenden bes pasier=at ma kosaran=\textbf{taero} bes\\
	good good \textsc{prox} {\glme} resistant sea\_water=so good sea\_water=\textsc{obj} \textsc{3sg} touch=even\_if good\\
	\glt `It's okay, it's okay, this is sea water-resistant, it's okay, even if he touches the sea water it's okay.' \jambox*{\href{http://hdl.handle.net/10050/00-0000-0000-0004-1BCB-5}{[conv1\_2:32]}}
	\label{exe:anti}
	\ex \gll biar kon eir=\textbf{taero} panggalat=nin=et\\
	even\_if one two=even\_if swollen=\textsc{neg}={\glet}\\
	\glt `Even if [you use] one, two kilos [the rice] doesn't get swollen.' \jambox*{\href{http://hdl.handle.net/10050/00-0000-0000-0004-1BA6-6}{[conv13\_2:23]}}
	\label{exe:eirtaero}
\end{exe}	

The long form of \textit{=taero} and the fact that \textit{=ero} or \textit{=o} are the regular conditional morphemes suggest that this is a (diachronically) multimorphemic form. One candidate for a source for \textit{=taero} is \textit{=taet} `more; again', which is a good semantic fit for `even if'.\is{conditional|)}

%EXAMPLES
%pasierat ma kosarandaero bes = even if it touches salt water it's fine
%nebaun eladok met usari sara ba biar ma sarandaero = even if he goes up (w. Indo conjunction)
%padiun biar pi muapte pi korarundaero bes = the husks, even if we eat, if we bite them, it's fine
%biar kon eirtaero: even if you have one or two kilos
%biar ka mingtunat dibarandaero ma koi tamba dongdongdaet se bore mesan = even if you add oil it becomes more chewy
%ka perat temundaero ma perat semnin ma pareir = even if you gave it a lot of water, it wasn't afraid of water, it just swelled
%biar an keweneko gosaundaero an pusingnin = even if I'm in the house at night i'm not bothered
%biar bintang kondaero rampaun me se = even if you use one parteng....
%ma toni an se sadar, biar tamanditaero, ma he sadar = A: he said I'm better, in any case? B: he's better. (not entirely sure about -taero there)
%guru lengkina mu boraero = even if the village's teachers go...
%biar ma pouktendaero ma kos mo = even if it floats it grows
%mu metko eruaptaero, mu kiemet tumtum metko eruap taero mu jietnin mu kiem = even if the children cried, they didn't get them.
%biar sontum aptaero bisa = even five people is possible

\subsubsection{Apprehensive \textit{=re} `\textsc{appr}'}\is{apprehensive}
\label{sec:apprmood}
Apprehensive mood, used when the speaker wishes to express fear that something bad will happen, is done with an apprehensive clitic \textit{=re}, which is attached to the subject or object NP of the clause. The predicate is marked with irrealis \textit{=et} (§\ref{sec:et}), and many clauses are preceded by the \is{interjection}interjection \textit{jaga} `watch out' (a Malay loan). In contrast to precautionary constructions (§\ref{sec:appreh}), only the danger is expressed and not necessarily the precaution to be taken. This morpheme is not found in the natural speech corpus, but was easily elicited with the help of pictures designed by Marine Vuillermet (p.c.) depicting dangerous situations, or by simply asking speakers what they would say in a specific dangerous situation. Examples of pictures were that of a snake under a chair (example~\ref{exe:kipkader}), or a crocodile approaching a human being. An example of a dangerous situation I described is walking on the beach under palm trees bearing ripe coconuts. In that situation, someone might utter~(\ref{exe:watre}). Note that in this example, the suggested precaution to take is also expressed in an imperative clause preceding the apprehensive clause.

\begin{exe}
	\ex \gll jaga eh kip kadera elak-un=ko ma=\textbf{re} kat kararuot=et\\
	watch\_out \textsc{int} snake chair bottom-\textsc{3poss}=\textsc{loc} \textsc{3sg=appr} \textsc{2sg.obj} bite={\glet}\\
	\glt `Watch out, there is a snake under the chair, it might bite you.' \jambox*{\href{http://hdl.handle.net/10050/00-0000-0000-0004-1C60-A}{[elic\_app\_5]}}
	\label{exe:kipkader}
	\ex \gll ka kolko=te wat=\textbf{re} kat kosarat=et\\
	\textsc{2sg} move\_out=\textsc{imp} coconut=\textsc{appr} \textsc{2sg.obj} hit={\glet}\\
	\glt `Move aside, or a coconut might hit you!' (or: `Move aside, lest a coconut hit you!') \jambox*{\href{http://hdl.handle.net/10050/00-0000-0000-0004-1C60-A}{[elic\_app\_4]}}
	\label{exe:watre}
\end{exe}

In the previous examples, the apprehensive morpheme \textit{=re} is attached to the noun that refers to the danger. In intransitive clauses, the danger is the addressee themselves, because they are behaving irresponsibly. In~(\ref{exe:kire}), the speaker utters an imaginary warning to kids jumping from the dock. If they do not jump far enough away from a dock, they might hit sharp rocks.

\begin{exe}
	\ex \gll jaga ki=\textbf{re} tur=et eh dalang=i kolko=rar\\
	watch\_out \textsc{2pl=appr} fall={\glet} \textsc{int} jump={\gli} move\_out=\textsc{imp.pl}\\
	\glt `Watch out or you'll fall, jump away [from the dock].' \jambox*{\href{http://hdl.handle.net/10050/00-0000-0000-0004-1C60-A}{[elic\_app\_3]}}
	\label{exe:kire}
\end{exe}

It seems, however, that the apprehensive morpheme may also be attached to the referent in danger in a transitive clause. Consider the following two elicited examples for the same pictured situation.

\begin{exe}
	\ex 
	\begin{xlist}
		\ex \gll paramuang kat=\textbf{re} koraruot=et\\
		crocodile \textsc{2sg.obj=appr} bite={\glet}\\
		\glt `[Watch out,] a crocodile might bite you!' \jambox*{\href{http://hdl.handle.net/10050/00-0000-0000-0004-1C60-A}{[elic\_app\_6]}}
		\ex \gll paramuang=\textbf{re} kat koraruot=et\\
		crocodile=\textsc{appr} \textsc{2sg.obj} bite={\glet}\\
		\glt `[Watch out,] a crocodile might bite you!' \jambox*{\href{http://hdl.handle.net/10050/00-0000-0000-0004-1C60-A}{[elic\_app\_6]}}
	\end{xlist}
\end{exe}	

%Translated with awas, or nanti or jangan sampai. often combined with jaga
\is{mood|)}

\subsection{Aspect}\is{aspect|(}
Aspect is the internal temporal constituency of a situation \parencite[][3]{comrie1976}. Kalamang has five aspect markers: the iamitive and nondum in post-subject slot 3, the completive (a complex predicate construction) and distributive following the predicate in slot 6, and the progressive (an enclitic) in slot 7. Two aspect-like enclitics on the predicate, non-final \textit{=te} and \textit{=ta}, are described in §\ref{sec:nfin}. They do not co-occur with the other aspect markers on the predicate. Of the five aspect markers discussed here, only the iamitive and nondum can co-occur with progressive \textit{=teba} and completive \textit{=i koyet}.\footnote{The iamitive can co-occur with the progressive, the iamitive can co-occur with the completive, the nondum can co-occur with the progressive and the nondum can co-occur with the completive. The iamitive and nondum cannot co-occur with each other. The progressive and completive cannot co-occur with each other either.}

\subsubsection{Iamitive \textit{se} `already' and nondum \textit{tok} `still; yet; first'}\is{iamitive}\is{nondum}
\label{sec:setok}
Kalamang has one aspectual particle, \textit{se} `already', and one aspectual word, \textit{tok} `still; yet; first'. Both follow the subject NP. \textit{Se} has an allomorph \textit{he}, which is usually used after vowels (see §\ref{sec:len}). This is not a watertight rule: one does find \textit{se} after vowels and (less commonly) \textit{he} after consonants. This suggests that \textit{se/he} is developing into a clitic (which attaches to the subject NP). Because the form \textit{se} or \textit{he} is not completely predictable from the phonological context, I give both variants as they are found in the corpus examples. I gloss \textit{se} `already' as \textsc{iam} for iamitive, which are ``more or less grammaticalised markers that have functions shared by `already' and the perfect\is{perfect}'' \parencite[][4]{olsson2013}. (\ref{exe:se1}) and~(\ref{exe:tokstill}) illustrate the syntactic position of \textit{se} and \textit{tok}, respectively.

\begin{exe}
	\ex 
	{\gll pas opa me dudan-mur-un \textbf{se} mat panok∼panok\\
		woman {\glopa} {\glme} cousin-\textsc{kin.pl-3poss} {\glse} \textsc{3sg.obj} order∼\textsc{red}\\
		\glt `That woman, her cousins already ordered him.' \jambox*{\href{http://hdl.handle.net/10050/00-0000-0000-0004-1BBC-4}{[narr24\_2:04]}}}
	\label{exe:se1}
	\ex 
	{\gll sayang-un \textbf{tok} kalom{\textless}lom{\textgreater}un\\
		nutmeg-\textsc{3poss} still young{\textless}\textsc{red}{\textgreater}\\
		\glt `Their nutmeg is still young.' \jambox*{\href{http://hdl.handle.net/10050/00-0000-0000-0004-1BBD-5}{[conv12\_15:45]}}
	}
	\label{exe:tokstill}
\end{exe}

\textit{Tok} can have one of three meanings. The meaning `still' is illustrated in~(\ref{exe:tokstill}) above. The meaning `first' is demonstrated in~(\ref{exe:tokfirst}). When \textit{tok} is combined with a negated verb, it is translated as `yet'. Expressions with the meaning `not yet' are also known as nondums \parencite{vdauwera1998,veselinova2015}. %cf German erst, schon-ernst-noch paper 1989

\begin{exe}
	\ex \gll ma \textbf{tok} ecien=i kewe=ko\\
	\textsc{3sg} first return={\gli} house=\textsc{loc}\\
	\glt `First he went home.' \jambox*{\href{http://hdl.handle.net/10050/00-0000-0000-0004-1BE7-5}{[stim42\_2:40]}}
	\label{exe:tokfirst}
	\ex
	{\gll pi taruot=et pi \textbf{tok} sampi=\textbf{nin}\\
		\textsc{1pl.incl} speak={\glet} \textsc{1pl.incl} yet arrive=\textsc{neg}\\
		\glt `We are talking, we haven't finished yet.' \jambox*{\href{http://hdl.handle.net/10050/00-0000-0000-0004-1B93-C}{[conv14\_8:47]}}
	}
	\label{exe:toksampi}
\end{exe}

\textit{Tok} can also be a free-standing negative answer (an \is{interjection}interjection) meaning either `still' or `not yet', depending on whether the question contained a negative or not. Consider the contrast between~(\ref{exe:stillschool}) and~(\ref{exe:notschool}).

\begin{exe}
	\ex 
	\begin{xlist}
		\exi{A:}
		{\gll ka tok sekola\\
			\textsc{2sg} still go.to.school\\
			\glt `Do you still go to school?' }
		\exi{B:}
		{\gll tok\\
			still\\
			\glt `Yes [I still go to school].' \jambox*{\href{http://hdl.handle.net/10050/00-0000-0000-0004-1C60-A}{[elic\_wc19\_15]}}}
	\end{xlist}
	\label{exe:stillschool}
\end{exe}

\begin{exe}
	\ex 
	\begin{xlist}
		\exi{A:}
		{\gll ka tok sekola=nin\\
			\textsc{2sg} yet go\_to\_school=\textsc{neg}\\
			\glt `Don't you go to school yet?' }
		\exi{B:}
		{\gll tok\\
			not.yet\\
			\glt `Not yet.' \jambox*{\href{http://hdl.handle.net/10050/00-0000-0000-0004-1C60-A}{[elic\_wc19\_16]}}}
	\end{xlist}
	\label{exe:notschool}
\end{exe}

(\ref{exe:nakalcatok}) is taken from a story about a black-haired monkey and a white-haired cuscus. The monkey asks the cuscus how to become white-haired. The cuscus then traps the monkey in a narrow cage and puts him in the rising sea, whereupon the monkey sees first his feet, then his belly, and then his entire body become lighter. The first \textit{tok} in utterance B, an interjection, contrasts with \textit{se} `already' in utterance A, and hence takes the negative `not yet' meaning. The second \textit{tok} in answer B is an instance of the aspectual marker, here meaning `still'.

\begin{exe}
	\ex 
	\begin{xlist}
		\exi{A:} 
		{\gll an=at kahetmei eren-an \textbf{se} iren\\
			\textsc{1sg=obj} open.\textsc{imp} body-\textsc{1sg.poss} {\glse} white\\
			\glt `Release me, my body is white!'
		}
		\exi{B:} 
		{\gll o kusukusu toni \textbf{tok} nakal-ca \textbf{tok} kuskap=ta ime\\
			\textsc{emph} cuscus say not\_yet head-\textsc{2sg.poss} still black={\glte} \textsc{dist}\\
			\glt `The cuscus says: ``Not yet, your head is still black.''' \jambox*{\href{http://hdl.handle.net/10050/00-0000-0000-0004-1BC1-0}{[narr19\_15:04]}}
		}
	\end{xlist}
	\label{exe:nakalcatok}
\end{exe}

Iamitive \textit{se} can often be translated with English `already' or a perfect. In~(\ref{exe:seboet}), the first \textit{se} acts like a perfect, while the second can be translated as `already'.

\begin{exe}
	\ex
	\label{exe:seboet}
	{\gll in \textbf{se} bo watko mu toni wowa kain \textbf{se} bo=et\\
		\textsc{1pl.excl} {\glse} go \textsc{prox.loc} \textsc{3pl} say aunt \textsc{2sg.poss} {\glse} go={\glet}\\
		\glt `We went here, they said your aunt has (already) left.' \jambox*{\href{http://hdl.handle.net/10050/00-0000-0000-0004-1BBD-5}{[conv12\_28:36]}}
	}
\end{exe}

\textit{Se} is also used to make reference to (cultural) expectations (as is the case in Malay usage/culture, see also \citealt{olsson2013}). One cannot ask `Are you married?' or `Do you have children?' without using \textit{se}. In Kalamang, since one is expected to marry and reproduce at some point, the use of \textit{se} reflects this expectation.\footnote{Similar to English `yet'.} Likewise, the answer to these questions cannot be `yes' or `no', but has to be \textit{se} `already' (plus at least a repetition of the predicate and possibly also including a subject) or \textit{tok} `not yet'. An example is given in~(\ref{exe:senamgon}).

\begin{exe}
	\ex 
	\begin{xlist}
		\exi{A:} {\gll ka se namgon\\
			\textsc{2sg} {\glse} married\_female\\
			\glt `Are you married?'}
		\exi{B:} {\gll tok\\
			not\_yet\\
			\glt `Not yet.'} \jambox*{[overheard]}
	\end{xlist}
	\label{exe:senamgon}
\end{exe}

These (cultural) expectations are also expressed through \textit{se} for more everyday situations. In Maas, the village of the speaker in~(\ref{exe:lampurse}), lights are automatically turned on at 5:30PM. By using \textit{se} with \textit{yuol} `shine', the speaker expresses that an expected situation has occurred.

\begin{exe}
	\ex \gll go se ginggir lampur \textbf{se} yuol\\
	condition {\glse} late\_afternoon lamp {\glse} shine\\
	\glt `It was late afternoon, the lamps were already on.' \jambox*{\href{http://hdl.handle.net/10050/00-0000-0000-0004-1BA2-F}{[conv11\_5:53]}}
	\label{exe:lampurse}
\end{exe}

The iamitive can also be used for changes of state, as in~(\ref{exe:sebotemun}), where a child has grown up in the course of the story.

\begin{exe}
	\ex \gll an bo lembaga nerun tumun-an \textbf{se} bo temun\\
	\textsc{1sg} go cell inside child-\textsc{1sg.poss} {\glse} go big\\
	\glt `I went into prison, my child has grown up.' \jambox*{\href{http://hdl.handle.net/10050/00-0000-0000-0004-1BAA-C}{[stim7\_29:09]}}
	\label{exe:sebotemun}
\end{exe}

Finally, iamitive \textit{se} occurs in a fixed expression with \textit{koyet} `to be finished' to indicate the end of a state or event, or of an entire story. (\ref{exe:pakuin}) narrates the building of a house. \textit{Se koyet} is used to indicate that a day's work is finished. (\ref{exe:maraouk}) is the last utterance in a story about a feast. The last events are listed (the serving and consuming of tea and food), and then the story is closed off with \textit{se koyet} `the end'.

\begin{exe}
	\ex \gll toni eh ma se me \textbf{se} \textbf{koyet} kasur=et eba paku=kin\\
	say hey \textsc{3pl} {\glse} \textsc{dist} {\glse} finish tomorrow={\glet} then nail={\glkin}\\
	\glt `[He] said: ``Hey, that's it, [we're] finished, tomorrow we'll nail.' \jambox*{\href{http://hdl.handle.net/10050/00-0000-0000-0004-1BB3-0}{[narr7\_13:01]}}
	\label{exe:pakuin}
	\ex \gll mu ter=at maraouk muaw=at maraouk in ter-nan=i koyet muap=i koyet \textbf{se} \textbf{koyet}\\
	\textsc{3pl} tea=\textsc{obj} serve food=\textsc{obj} serve \textsc{1pl.excl} tea-consume={\gli} finish eat={\gli} finish {\glse} finish\\
	\glt `They served the tea, served the food, we finished drinking tea, finished eating, the end.' \jambox*{\href{http://hdl.handle.net/10050/00-0000-0000-0004-1BDC-D}{[conv8\_5:23]}}
	\label{exe:maraouk}
\end{exe}

The two aspectual markers are compatible with negated or inherently negative verbs (§\ref{sec:lexicalneg}), but change in meaning. The aspectual markers have scope over the negation\is{negation}. The aspectual word \textit{tok} `still; yet; first' can be combined with a negated predicate to form the meaning `not yet', as illustrated in~(\ref{exe:tokki}) (see also~\ref{exe:toksampi}).


\ea \label{exe:tokki}
\ea \label{exe:tokna}
{\gll ma \textbf{tok} nawanggar\\
	\textsc{3sg} still wait\\
	\glt `He still waits.' \jambox*{\href{http://hdl.handle.net/10050/00-0000-0000-0004-1BD4-C}{[stim29\_1:34]}}}

\ex \label{exe:tokbotnin}
\gll Nyong esun \textbf{tok} bot=\textbf{nin}\\
Nyong father.\textsc{3poss} yet go=\textsc{neg}\\
\glt `Nyong's father doesn't go yet.' \jambox*{\href{http://hdl.handle.net/10050/00-0000-0000-0004-1BA3-3}{[conv10\_7:16]}}
\z 
\z 

When iamitive \textit{se} is combined with a negative verb or negated predicate, it forms the meaning `not any more'. This is illustrated in example~(\ref{exe:se}). 

\ea \label{exe:se}
\ea \label{exe:inse}
{\gll in \textbf{se} mia\\
	\textsc{1pl.excl} {\glse} come\\
	\glt `We had already come.' \jambox*{\href{http://hdl.handle.net/10050/00-0000-0000-0004-1BDC-D}{[conv8\_2:34]}}}

\ex 	\label{exe:heparuotnin}
{\gll ma \textbf{se} paruot=\textbf{nin}\\
	\textsc{3sg} {\glse} do=\textsc{neg}\\
	\glt `He won't do it any more.' \jambox*{\href{http://hdl.handle.net/10050/00-0000-0000-0004-1BAA-C}{[stim7\_28:36]}}}
\z \z 

Table~\ref{tab:setok} shows the meanings associated with \textit{se} and \textit{tok} in affirmative and negative clauses.

\begin{table} [ht]
	\caption{Aspectual markers \textit{se} and \textit{tok} in affirmative and negative clauses}
	\label{tab:setok}
	
		\begin{tabularx}{.8\textwidth}{XXl}
			\lsptoprule
			& iamitive \textit{se} & nondum \textit{tok}\\ 
			\midrule
			affirmative  &   perfective; already  &  still; yet; first    \\
			negative  &   not anymore &  not yet  \\
			\lspbottomrule
		\end{tabularx}
	
\end{table}

The aspectual markers have scope over quantifiers and predicates. Consider (\ref{exe:redai}), where negator \textit{=nin} has scope over \textit{reidak} `much' and \textit{na} `eat', and \textit{tok} has scope over both. For a further description of the behaviour of quantifiers with respect to the NP and the predicate, see §\ref{sec:nmodqnt}.  %This could be an argument for treating those ``object numerals'' that are outside the NP as part of the Predicate or as argument for a VP, but see nmodqnt. 

\begin{exe}
	\ex 
	{\gll ma tok {\ob}{\ob}\textbf{reidak}-i nat{\cb}=\textbf{nin}{\cb}\\
		\textsc{3sg} yet much-\textsc{objqnt} consume=\textsc{neg}\\
		\glt `He hasn't eaten much yet.' \jambox*{\href{http://hdl.handle.net/10050/00-0000-0000-0004-1C60-A}{[elic\_neg\_89]}}
	}
	\label{exe:redai}
\end{exe}

\subsubsection{Progressive \textit{=teba} `\textsc{prog}'}\is{progressive}
\label{sec:teba}
The clitic \textit{=teba}, which attaches to the predicate, expresses progressive or continuous aspect, indicating that an action is incomplete or in progress. Two examples are given in~(\ref{exe:garungdeba}) and~(\ref{exe:repteba}).

\begin{exe}
	\ex \gll ki neba=at=a paruo in garung=\textbf{teba}\\
	\textsc{2pl} what=\textsc{obj=foc} do \textsc{1pl.excl} chat={\glteba}\\
	\glt `{``}What are you doing?'' ``We're chatting.''' \jambox*{\href{http://hdl.handle.net/10050/00-0000-0000-0004-1BA4-1}{[conv16\_14:29]}}
	\label{exe:garungdeba}
	\ex \gll in bo Esa Tanggiun kai-rep=\textbf{teba}\\
	\textsc{1pl.excl} go Esa Tanggiun firewood-collect={\glteba}\\
	\glt `We went collecting firewood at Esa Tanggiun.' \jambox*{\href{http://hdl.handle.net/10050/00-0000-0000-0004-1BA2-F}{[conv11\_1:07]}}
	\label{exe:repteba}
\end{exe}

The progressive clitic can also be used with future reference, such as in~(\ref{exe:kanas}).

\begin{exe}
	\ex \gll pi koi bo Kanastangan=ko=\textbf{teba}=et reon\\
	\textsc{1pl.incl} then go Kanastangan=\textsc{loc}={\glteba}={\glet} maybe\\
	\glt `Then we'll go to Kanastangan maybe?' \jambox*{\href{http://hdl.handle.net/10050/00-0000-0000-0004-1BAE-4}{[narr44\_18:23]}}
	\label{exe:kanas}
\end{exe}

The clitic can be in two different positions in the clause: it can attach directly to the predicate, and then it may be followed by, for example, irrealis \textit{=et} (as in~\ref{exe:kanas} above) or imperative \textit{=te} (example~\ref{exe:tebare2}); it can also come in position 8 (Table~\ref{tab:modifiersall}), where it follows volitional \textit{=kin} or negator \textit{=nin} (examples~\ref{exe:kindeba} and~\ref{exe:nindebaa}).

\begin{exe}
	\ex \gll ka marmar=\textbf{teba}=te ka-mun sara=in\\
	\textsc{2sg} walk={\glteba}=\textsc{imp} \textsc{2sg}-\textsc{proh} sara=\textsc{proh}\\
	\glt `You just walk, don't you get up [your bike]!' \jambox*{\href{http://hdl.handle.net/10050/00-0000-0000-0004-1BD3-9}{[stim33\_1:27]}}
	\label{exe:tebare2}
	
	\ex \gll bal se koraru se kuet=te nat=kin=\textbf{teba}\\
	dog {\glse} bite {\glse} bring={\glte} consume={\glkin}={\glteba}\\
	\glt `The dog has bitten [the fish], has brought [it] and wants to eat [it].' \jambox*{\href{http://hdl.handle.net/10050/00-0000-0000-0004-1BBA-8}{[stim2\_3:55]}}
	\label{exe:kindeba}
	\ex \gll an tok bo mat ketemu=nin=\textbf{teba}\\
	\textsc{1sg} yet go \textsc{3sg.obj} meet.\textsc{mly=neg}={\glteba}\\
	\glt `I have never met him.' \jambox*{\href{http://hdl.handle.net/10050/00-0000-0000-0004-1BCA-4}{[conv20\_13:37]}}
	\label{exe:nindebaa}
	
\end{exe}

%maybe \textit{nak} `just' is the same thing. less frequent, maybe 30 in whole corpus. can be combined with se, also with teba: sor se nakna reba, ka he naktiri reba, in se nakomahalte. te weinig data om over te schrijven misschien, dus laat maar even gewoon `just' betekenen.

\subsubsection{Completive \textit{=i koyet} `\textsc{plnk} finish'}\is{completive}
\label{sec:compl}
%cf. Western Pantar (Holton) has something similar to koyet which functions as a perfect marker, i.e. denoting a past event which is of current relevance. perhaps better analysis?
The combination of predicate linker \textit{=i} and the verb \textit{koyet} `to be finished' expresses completive aspect. The construction is part of a complex predicate also described in §\ref{sec:ikoyetsvc}. Completive \textit{=i koyet} is only used to link one action to another, and is therefore often best translated as `after'. See also §\ref{sec:tailhead}. (\ref{exe:lemarat}) is taken from a narrative about building a house, and illustrates a typical string of actions linked by \textit{=i koyet}.

\ea \glll Lemarat paruoni koyet metko komangganggowet. Salat diran, sal rani koyet, eba metko komangganggowet. Komangganggowi koyet, eba pi koi parararunat diraret.\\
	lemat=at paruon=\textbf{i} \textbf{koyet} metko komanggangguop=et sal=at di=ran sal ran=\textbf{i} \textbf{koyet} eba metko komanggangguop=et komanggangguop=\textbf{i} \textbf{koyet} eba pi koi pararar-un=at di=rat=et\\
	bamboo\_string=\textsc{obj} make={\gli} finish \textsc{dist.loc} put\_on\_roof={\glet} roof\_wood=\textsc{obj} \textsc{caus}=move roof\_wood go={\gli} finish then there put\_on\_roof={\glet} put\_on\_roof={\gli} finish then \textsc{1pl.excl} again floor-\textsc{3poss}=\textsc{obj} \textsc{caus}=go={\glet}\\
	\glt `After making string, [we] put on the roof. Installing the roof wood, after installing the roof wood, then we put on the roof. After putting on the roof, then we install the floor again.' \jambox*{\href{http://hdl.handle.net/10050/00-0000-0000-0004-1BB2-1}{[narr6\_4:24]}}
	\label{exe:lemarat}
\z

The construction Verb-\textit{i koyet} can also be a quantifying expression meaning `all; until finished', as described in §\ref{ch:quant}. Completive aspect typically expresses that a totality of referents is affected, which is close in meaning to the quantifying use of \textit{-i koyet}. However, like other languages in East Indonesia\is{East Indonesian languages}, the completive is used also when an actor deliberately ends an event \parencite{unterladstetter2020}. In~(\ref{exe:lemarat}), there are clear agents, but one could argue that a totality of referents is affected (all the roof wood is installed, the whole roof is closed). In~(\ref{exe:neweri}), however, the referent is firewood, and not all the firewood is bought, and neither is it probable that all the money was spent.

\begin{exe}
	\ex \gll me ma kai=at jien=\textbf{i} \textbf{koyet}=ta me newer=\textbf{i} \textbf{koyet} kusukusu toni pier koi bo=et\\
	{\glme} \textsc{3sg} firewood=\textsc{obj} buy={\gli} finish={\glta} {\glme} pay={\gli} finish cuscus say \textsc{2du} again go={\glet}\\
	\glt `After she bought firewood and paid, the cuscus said: ``Shall we go?''' \jambox*{\href{http://hdl.handle.net/10050/00-0000-0000-0004-1BC1-0}{[narr19\_4:44]}}
	\label{exe:neweri}
\end{exe}


\subsubsection{Distributive \textit{-p} `\textsc{distr}'}
\label{sec:distribv}\is{person marking}
A suffix \textit{-p} was attested on reduplicated directional verbs, and is likely a \is{distributive}distributive or pluractional marker, though there are not currently enough data to determine this.\is{reduplication!verbs} Both attested examples are from descriptions of big events: a wedding and a funeral.


\begin{exe}
	\ex	\gll sontum reidak me marua-\textbf{p}∼marua-\textbf{p}=te\\
	person many {\glme} move\_seawards-{\glp}∼\textsc{distr}-{\glp}={\glte}\\
	\glt `Many people came.' \jambox*{\href{http://hdl.handle.net/10050/00-0000-0000-0004-1BC3-B}{[conv7\_9:06]}}
	\ex	\gll sontum se tan-kinkin=te ecie-\textbf{p}∼cie-\textbf{p}\\
	person {\glse} hand-hold={\glte} return-{\glp}∼\textsc{distr}-{\glp}\\
	\glt `People shook hands and returned.' \jambox*{\href{http://hdl.handle.net/10050/00-0000-0000-0004-1B85-F}{[narr5\_5:32]}}
\end{exe}
%maruapmaruap, marapmarap, sarapsarap, barapbarap, raprap, miapmiap, eraprap, yeciep-yeciep, but not  *bopbop

\is{aspect|)}


\subsection{Modal markers}\is{modality|(}
\label{sec:modal}
Modal markers are adverbials that express the speaker's attitude towards a proposition, such as likelihood, certainty or truthfulness. There are six modal markers, listed in Table~\ref{tab:modal} with their position in the clause.\footnote{There is not currently enough data to determine whether the position in the clause has an effect on the scope of the modal markers. Also, the exact word class of these markers remains unclear. Most of them are verb-like (and three of them are borrowed from Malay verbs), but again, data is too scarce to make a coherent proposal, and hence they are just discussed as `modal markers' here.} Two of them (\textit{bisa} `can' and \textit{harus} `must') are loans from Malay. The construction \textit{suka}-\textsc{poss} \textsc{neg} `not like' contains the Malay loan \textit{suka} `like'. Modal markers are found in the pre-subject slot 1, post-subject slots 3 and 4, and clause-final slot 8.

\begin{table}[ht]
	\caption{Modal markers}
	\label{tab:modal}
	
\fittable{
		\begin{tabular}{lll}
			\lsptoprule 
			marker & expresses & position\\
			\midrule 
			\textit{bisa} `can' & possibility, ability & pre/post-subject, clause-final\\
			\textit{harus} `must' & necessity & pre/post-subject\\
			\textit{reon} `maybe' & uncertainty, possibility & clause-final \\
			\textit{gen} `maybe' & uncertainty, possibility & post-subject\\
			\textit{suka}-\textsc{poss} \textsc{neg} `not like' & dislike & post-subject, clause-final\\
			\textit{eranun} `cannot' & impossibility & clause-final\\
			\lspbottomrule
		\end{tabular}
	}
\end{table}

\textit{Bisa} `can' occurs clause-finally following \is{conditional}conditional markers to express general possibility (example~\ref{exe:bisa0}). After the subject, it expresses possibility or ability (example~\ref{exe:bisa3}). In biclausal conditional clauses, as in example~(\ref{exe:bisa2}), \textit{bisa} precedes the subject. \textit{Bisa} also precedes the subject in questions, like~(\ref{exe:bisa2}).

% AH No, this is not clause-final! It is a main clause in its own right, following the conditional clause: [if drink] [ possible] If it can only occur following conditionals, it is the corresponding apodosis main clause.


\begin{exe}
	\ex
	{\gll jadi ter nan-ere \textbf{bisa} kai nan-ere \textbf{bisa}\\
		so tea consume-{\glcond} can medicine drink-{\glcond} can\\	
		\glt `So if [you] drink it as tea that's possible or if [you] drink it as medicine that's possible.' \jambox*{\href{http://hdl.handle.net/10050/00-0000-0000-0004-1BB1-3}{[narr34\_0:32]}}
	}
	\label{exe:bisa0}
	\ex
	{\gll an mat gerket ka \textbf{bisa} nan ye ge\\
		\textsc{1sg} \textsc{3sg.obj} ask \textsc{2sg} can consume or not\\
		\glt `I asked him: ``Can you eat or not?''' \jambox*{\href{http://hdl.handle.net/10050/00-0000-0000-0004-1BA9-9}{[stim6\_14:29]}}
	}
	\label{exe:bisa3}
	\ex
	{\glll Kalau warkin kararaet bisa. Warkin kararaet bisa pi wangga marmar=et.\\
		kalau warkin kararak=et \textbf{bisa} warkin kararak=et \textbf{bisa} pi wangga marmar=et\\
		if tide dry={\glet} can tide dry={\glet} can \textsc{1pl.incl} \textsc{prox.lat} walk={\glet}\\	
		\glt `If the tide is low, it's possible. If the tide is low, we can walk from here.' \jambox*{\href{http://hdl.handle.net/10050/00-0000-0000-0004-1BE5-2}{[narr38\_1:10]}}
	}
	\label{exe:bisa2}
	\ex \gll \textbf{bisa} mu kosom=i koyet ye ge\\
	can \textsc{3pl} smoke={\gli} finish or not\\
	\glt `Can they smoke it all or not?' \jambox*{\href{http://hdl.handle.net/10050/00-0000-0000-0004-1BC5-7}{[narr16\_2:21]}}
\end{exe}
%first ex to be elaborated??

\textit{Harus} `must' expresses necessity, and usually occurs after the subject as in~(\ref{exe:welbon}). Like \textit{bisa} `can', it precedes the subject in conditional clauses, see~(\ref{exe:watjie}). It is also sometimes used without the predicate it is supposed to modify, presumably when the event to which the main verb refers is clear from the context, as in~(\ref{exe:harus1}).

\begin{exe}
	\ex \gll wele \textbf{harus} sor=bon sor=nan \textbf{harus} wele=bon\\
	vegetables must fish=\textsc{com} fish=too must vegetables=\textsc{com}\\
	\glt `Vegetables must be with fish, fish must be with vegetables.' \jambox*{\href{http://hdl.handle.net/10050/00-0000-0000-0004-1BA5-0}{[conv15\_5:42]}}
	\label{exe:welbon}
	
	\ex \gll kalau kabor-un nain ko{\textless}yo{\textgreater}yal=te nain=kap=et me \textbf{harus} mu wat jie=ta\\
	if stomach-\textsc{3poss} like disturbed{\textless}\textsc{atten}{\textgreater}={\glte} like=\textsc{sim}={\glet} {\glme} must \textsc{3pl} \textsc{prox.obj} get={\glta}\\
	\glt `If the stomach is like it's disturbed, they have to get this.' \jambox*{\href{http://hdl.handle.net/10050/00-0000-0000-0004-1BBE-E}{[narr36\_2:06]}}
	\label{exe:watjie}
	\ex
	{\gll kariak sara nakal=ko \textbf{harus} kai\_modar\\
		blood ascend head=\textsc{loc} must marungga\_tree\\	
		\glt `[If] blood goes up to the head, [you] must [use] marungga tree.' \jambox*{\href{http://hdl.handle.net/10050/00-0000-0000-0004-1BC2-8}{[narr33\_3:10]}}
	}
	\label{exe:harus1}
\end{exe}

\textit{Reon} `maybe' expresses uncertainty or possibility and occurs clause-finally with verbal and non-verbal predicates, and modifies the whole clause. An example with a non-verbal predicate is given in~(\ref{exe:reon}).

\begin{exe}
	\ex
	{\gll kon siun wilak=ko yuwa \textbf{reon}\\
		one edge sea=\textsc{loc} \textsc{prox} maybe\\	
		\glt `The one here on the edge on the sea-side maybe?' \jambox*{\href{http://hdl.handle.net/10050/00-0000-0000-0004-1B92-E}{[stim44\_0:36]}}
	}
	\label{exe:reon}
\end{exe}

\textit{Gen} `maybe' also expresses uncertainty or possibility and follows the subject, as in~(\ref{exe:gen}). Apart from its position in the clause, there is no difference in meaning. \textit{Gen}, like \textit{reon}, modifies the entire clause. It is unclear whether \textit{gen} and \textit{reon} can co-occur.

\ea 	\label{exe:gen}
\gll ma \textbf{gen} sara tabai-jie\\
\textsc{3sg} maybe ascend tobacco-buy\\
\glt	`Maybe he came up to buy tobacco?' \jambox*{\href{http://hdl.handle.net/10050/00-0000-0000-0004-1B9F-F}{[conv9\_22:03]}}
\z 

%very few examples before subject, but probably because other stuff also preceding.

There are not enough data to determine the mutual order of these and aspectual markers \textit{se} and \textit{tok}, which are also post-subject. The corpus contains six instances of different modal markers followed by \textit{se} (1x \textit{bisa se}, 5x \textit{gen se}, no examples with \textit{tok}), and two instances of \textit{se} or \textit{tok} followed by different modal markers (1x \textit{tok bisa}, 1x \textit{se gen}).

The two negative modal markers behave slightly differently. \textit{Eranun} `cannot' triggers nominalisation of the verb it modifies, and follows that nominalised verb. In a few cases, \textit{eranun} stands in a clause on its own (following comma intonation\is{intonation}). In that case, the preceding clause has a normal verbal predicate, as in~(\ref{exe:siendaet}).
\begin{exe}
	\ex \gll an ki=at rup-\textbf{un} \textbf{eranun}\\
	\textsc{1sg} \textsc{2pl=obj} help-\textsc{nmlz} cannot\\
	\glt `I cannot help you.' \jambox*{\href{http://hdl.handle.net/10050/00-0000-0000-0004-1BB3-0}{[narr7\_6:27]}}
	\label{exe:ruwune}
	\ex \glll Mu koi bo siendaet, \textbf{eranun}.\\
	 mu koi bo sien=taet eranun\\
	 \textsc{3pl} again go sharpen=again be\_impossible\\
	 \glt `They went to sharpen [their axes] again. It was impossible.' \jambox*{\href{http://hdl.handle.net/10050/00-0000-0000-0004-1BDF-0}{[narr27\_3:55]}}
	 \label{exe:siendaet}
\end{exe}

The construction \textit{suka}-\textsc{poss} \textsc{neg} `not like; not want' usually appears as the only verb-like element in the clause, but must refer to a preceding proposition.\footnote{\textit{Suka} is likely a nominalised verb or a noun, and hence examples with \textit{suka}-\textsc{poss} \textsc{neg} could be paraphrased as `is not of subject's liking'.} If combined with a verb in the same clause, the verb is negated with negator \textit{=nin}, as in~(\ref{exe:amatmu}). The positive counterparts of \textit{suka}-\textsc{poss} \textsc{neg} `not like; not want' are irrealis marker \textit{=kin}, which can also be used to express volition, and \textit{lo} `to want; to consent'. These are described in §\ref{sec:lexicalneg} and~§\ref{sec:kin}. 

\begin{exe}	 
	\ex \gll canam opa me mat narorar ba lek \textbf{suka-un} \textbf{ge}\\
	man {\glopa} {\glme} \textsc{3sg.obj} drag but goat want-\textsc{3poss} not\\
	\glt `That man drags it, but the goat doesn't want [to be dragged].' \jambox*{\href{http://hdl.handle.net/10050/00-0000-0000-0004-1BD1-D}{[stim31\_0:57]}}
	\label{exe:rarba}
	\ex \gll ma se \textbf{suka-un} am=at mu ∅=\textbf{nin}\\
	\textsc{3sg} {\glse} want-\textsc{3poss} breast=\textsc{obj} \textsc{3pl} give=\textsc{neg}\\
	\glt `She doesn't want to give them breast [milk] any more.' \jambox*{\href{http://hdl.handle.net/10050/00-0000-0000-0004-1BA7-D}{[narr21\_1:55]}}
	\label{exe:amatmu}
\end{exe}	\is{modality|)}

\section{Adverbial modifiers}\is{adverbial modifier}
\label{sec:adv}
Adverbials specify the manner, temporal setting, degree or other characteristics (such as \is{repetition}repetition or exclusivity) of the state or event expressed by the verb. They were introduced in §\ref{sec:wcadv}. Modal adverbials were treated in §\ref{sec:modal}. Adverbial modifiers occur at many different slots in the clause: temporal adverbials are clause-initial (slot 1), manner adverbials precede the verb (slot 5), adverbials of degree and some other adverbials are clitics on the predicate (slot 6) and \textit{weinun} `too' is clause-final (slot 8). \textit{Koi} `again' has a variable position in between the subject and the verb (slot 4).

\subsection{Manner adverbials}\is{manner}
\label{sec:manner}
Kalamang has three manner adverbials, the meaning of only one of which is clear: \textit{loi} `quickly'. Manner adverbials are part of the predicate, scoping over it, and come before the verb. They possibly end in predicate linker \textit{=i} (see §\ref{sec:mvci}). They cannot, however, be inflected like regular verbs, and with the exception of \textit{loi}, do not occur independently of another verb.

\textit{Loi} `quickly' is illustrated in~(\ref{exe:loi1}) and~(\ref{exe:loi2}). Though analysed as monomorphemic here, it is likely that the final \textit{-i} is an instance of predicate linker \textit{=i}. It can also stand alone as a command (with the imperative clitic: \textit{lo=te}).

\begin{exe}
	\ex
	{\gll kome=i koyet \textbf{loi} eti\\
		look={\gli} finish quickly return \\
		\glt `[When you're] done looking, return quickly.' \jambox*{\href{http://hdl.handle.net/10050/00-0000-0000-0004-1B91-5}{[narr39\_8:47]}}
	}
	\label{exe:loi1}
	\ex
	{\gll ka se \textbf{loi} gonggin=et\\
		\textsc{2sg} {\glse} quickly know={\glet}\\
		\glt `You quickly learn.' \jambox*{\href{http://hdl.handle.net/10050/00-0000-0000-0004-1BBD-5}{[conv12\_8:00]}}
	}
	\label{exe:loi2}
\end{exe}

The second manner adverbial, \textit{dumuni}, only has eight occurrences in the corpus, and its semantics are not entirely clear. It is only used with events that express movement, and the use of \textit{dumuni} seems to indicate a change of direction. This is illustrated in~(\ref{exe:dumuni}) to~(\ref{exe:gontum}). The question marks in the translation lines indicate that I am not sure whether I used the right verb.

\begin{exe}
	\ex
	\gll ma tumun opa me se mengga \textbf{dumuni} ra\\
	\textsc{3sg} child {\glopa} {\glme} {\glse} \textsc{dist.lat} \textsc{manner} go\\
	\glt `That child already escaped? to there.' \jambox*{\href{http://hdl.handle.net/10050/00-0000-0000-0004-1B9F-F}{[conv9\_11:51]}}
	\label{exe:dumuni}
	\ex \gll sabar se \textbf{dumuni} Nyong emunkongga mengga mara\\
	front {\glse} \textsc{manner} Nyong mother.\textsc{3poss-an.loc} \textsc{dist.lat} move\_landwards\\
	\glt `The front [of the canoe] already turned? towards Nyong's mother on the land-side.' \jambox*{\href{http://hdl.handle.net/10050/00-0000-0000-0004-1B9F-F}{[conv9\_14:08]}}
	\ex \gll an toni pasa {\ob}...{\cb} barsi=ten se koyet eh eba koi pi \textbf{dumuni} goni-tumun kon\\
	\textsc{1sg} say rice {} clean.\textsc{mly=at} {\glse} finish \textsc{tag} then then \textsc{1pl.incl} \textsc{manner} sack-small one\\
	\glt `I said the clean rice is finished, right, then we turn? to the one small sack.' \jambox*{\href{http://hdl.handle.net/10050/00-0000-0000-0004-1BA6-6}{[conv13\_12:04]}}
	\label{exe:gontum}
\end{exe}

The third manner adverbial, \textit{sororoi}, occurs twice in the corpus, in the same story, to modify the verb \textit{bara} `descend' in the context of climbing down a tree. The utterance in~(\ref{exe:sororoi}) follows an order made by a giant for the protagonist of the story to come down. It is unclear which exact meaning \textit{sororoi} adds to the utterance, but speakers have indicated it has to do with manner.

\begin{exe}
	\ex
	\gll ma sororoi bara\\
	\textsc{3sg} \textsc{manner} descend\\
	\glt `She climbed down.' \jambox*{\href{http://hdl.handle.net/10050/00-0000-0000-0004-1BDE-7}{[narr25\_4:37]}}
	\label{exe:sororoi}
\end{exe}

\textit{Dumuni} and \textit{sororoi} are perhaps better analysed as ideophones. This option is entertained in §\ref{sec:idphon}.

Otherwise, manner is usually expressed in \is{predicate!complex}complex predicates (see Chapter~\ref{ch:svc}, especially §\ref{sec:svcmanneri}). Manner demonstratives \textit{wandi} `like this' and \textit{mindi} `like that' can be found in Chapter~\ref{ch:dems}.

\subsection{Adverbials of degree}\is{degree adverbial}
\label{sec:degradv}
Kalamang has two enclitics that I classify as adverbials of degree. These are the clitics \textit{=sawe(t}) `too' and \textit{=tun}, an intensifier which can also mean `too'. Both attach to and scope over the predicate. 

The clitic \textit{=sawe(t)} usually attaches to predicates in the form of stative verbs, as in~(\ref{exe:sawe}), but can be on any predicate such as the incorporation construction \textit{halanganrep} `to look for trouble' in~(\ref{exe:sawe2}) or the transitive verb \textit{kona} `to see' in~(\ref{exe:irulp}). 

\begin{exe}
	\ex 
	{\gll ma ririn=\textbf{sawe}\\
		\textsc{3sg} tall=too\\
		\glt `It's too tall.' \jambox*{\href{http://hdl.handle.net/10050/00-0000-0000-0004-1BB7-9}{[conv19\_34:30]}}}
	\label{exe:sawe}
	\ex 
	{\gll mu me halangan-rep=\textbf{sawe}\\
		\textsc{3pl} {\glme} trouble-get=too\\
		\glt `They're looking for too much trouble.' \jambox*{\href{http://hdl.handle.net/10050/00-0000-0000-0004-1BA3-3}{[conv10\_1:55]}}
	}
	\label{exe:sawe2}
	\ex \gll Irul pi mat konan=\textbf{sawe}\\
	Irul \textsc{1pl.excl} \textsc{3sg.obj} see=too\\
	\glt `Irul, we keep on seeing him.' \jambox*{\href{http://hdl.handle.net/10050/00-0000-0000-0004-1BE7-5}{[stim42\_7:53]}}
	\label{exe:irulp}
\end{exe}
%check last ex in 2020. -n? saet? is there a diff between sawet and sawe and saet (nominal?)?? if there is a verbal sawe(t), can it really be on transitive verbs? e.g. dia beli terlalu banyak gula. dia makan teralu banyak nasi. or maybe sawet is sawe=et.

The uses of \textit{=tun} are illustrated in~(\ref{exe:tuntoo}) and~(\ref{exe:tunvery}). As an intensifier, it attaches only to reduplicated roots. These can be verbs, adverbials, nouns (as in example~\ref{exe:tunvery}) or quantifiers\is{quantifier}.

\begin{exe}
	\ex 
	{\gll nika kahen=\textbf{tun} ge nika taraman-kodak\\
		fishing\_line long=too no fishing\_line fathom-just\_one\\
		\glt `Not a too long fishing line, just one fathom.' \jambox*{\href{http://hdl.handle.net/10050/00-0000-0000-0004-1B9F-F}{[conv9\_2:21]}}
	}
	\label{exe:tuntoo}
	\ex 
	{\gll ma siun∼siun=\textbf{tun} timbang-un=ko\\
		\textsc{3sg} edge∼\textsc{ints-ints} forehead-\textsc{3sg=loc}\\
		\glt `It's at the very edge, at his forehead.' \jambox*{\href{http://hdl.handle.net/10050/00-0000-0000-0004-1BC7-6}{[stim25\_8:41]}}
	}
	\label{exe:tunvery}
\end{exe}

%q adv 19: -sawe is accepted on all kinds of verbs, tranlsated as ``terlalu'', in one case ``terus'' (kalisawe, hujan terus). soort van versterker? pron. seems to be -saweT by the way, which makes it quite similar to saet. test constrastively 2020. -tun is not accepted on verbs, only on some in-between ones like siktaktaktun and yoryortun, and the adj-like ones such as ririn, temun, kahen.

%Things elicited or found with tun: siktaktaktun, yoryortun, monmontun, kleur/color+tun, ririntun, kahentun, temuntun, cicauntun, cicauntutun, wistuntun, besbestun, siunsiuntun, bolonbolontun, kodakdaktun, tebontebontun, tayuontayuontun (not good), (h)epduntun (from the very back), . so it goes on verbs, adverbs, nouns, quantifiers.


\subsection{Temporal adverbials}\is{temporal adverbial}
\label{sec:tempadv}
Temporal adverbials set the temporal scene for a clause and modify the entire clause. Temporal adverbials come in the clause-initial (i.e. pre-subject) slot. Examples of temporal adverbials are listed in~(\ref{exe:tempadv}) below.

\begin{exe}
	\ex 	\begin{tabbing}
		\hspace*{2cm}\=\hspace*{5cm}\=\kill
		\textit{keitar} \> day before yesterday\\
		\textit{wis} \> yesterday; past\\
		\textit{opa yuwa} \> earlier today\\
		\textit{opa} \> earlier; just now\>\\
		\textit{kasur} \> tomorrow\\
		\textit{keirko} \> day after tomorrow
	\end{tabbing}
	\label{exe:tempadv}
\end{exe}

An example with \textit{wis} `yesterday' is given in~(\ref{exe:wis1}), and \textit{opa yuwa} `(earlier) today' and \textit{kasur} `tomorrow' are shown in~(\ref{exe:kasur}).
%EV only one clear example of post-subj wis, so deleted (HH also said ``curious if lexical diff'')
%EV19: but note that wis IS different. it is (not many exs, but clear ones) used parallel to opa me. even one ex: tamandi opa me wis me, as a verbetering. also verbuigingen (note: some with wisE!) like wise-un?, wise-kin (zaman purba punya), wistuntun, wise-nggap=ni. so maybe (also) a noun.

\ea
	{\gll \textbf{wis} sekitar jam satu in wa kaluar\\
		yesterday around hour one \textsc{1pl.excl} \textsc{prox} exit\\
		\glt `Yesterday around one o'clock we left from here.' \jambox*{\href{http://hdl.handle.net/10050/00-0000-0000-0004-1BD8-4}{[narr1\_0:01]}}
	}
	\label{exe:wis1}
	%	\ex
	%	{\gll ma-tain \textbf{wis} nawerarta me reidak\\
	%		\textsc{3sg}-alone yesterday do {\glme} much\\
	%		`Yesterday he worked a lot.' \jambox*{[conv12\_0:30]
	%	}
	%	\label{exe:wis2}
    \z

\ea
	{\gll ma toni \textbf{opa\_yuwa} an dodon waruo=teba \textbf{kasur} mu kolak=ka bot=kin\\
		\textsc{3sg} say today \textsc{1sg} clothes wash={\glteba} tomorrow \textsc{3pl} mountain=\textsc{lat} go={\glkin}\\
		\glt `She said: ``Today I was washing clothes.'' Tomorrow they want to go to the mountains.' \jambox*{\href{http://hdl.handle.net/10050/00-0000-0000-0004-1BA2-F}{[conv11\_6:15]}}
	}
	\label{exe:kasur}
\z

The Kalamang day is divided into four: from the time around sunrise until it starts getting hot (approximately 5 a.m. to 10 a.m.), around the hottest hours of the day (approximately 10 a.m. to 4 p.m.), the late afternoon until sunset (approximately 4 p.m. to 6 p.m.), and the dark hours (approximately 6 p.m. to 5 a.m.). Adverbial phrases referring to the times of the day are formed with \textit{go} `place; (weather) condition'. \textit{Go} and the time of day can be separated by aspectual markers \textit{se} `already' and \textit{tok} `still; yet; first', e.g. \textit{go se saun} `it's already evening'. The times of the day are listed below with their translations and their corresponding nouns, if available. 

\begin{exe}
	\ex 	\begin{tabbing}
		\hspace*{2cm}\=\hspace*{4.5cm}\=\hspace*{1.5cm}\=\hspace*{4cm}\=\kill
		\textit{go dung} \> in the morning \>\>\\
		\textit{go yuol} \> during the day \> \textit{yuol} \> day; light\\
		\textit{go ginggir} \> in the (late) afternoon \> \textit{ginggir} \> (late) afternoon \\
		\textit{go saun} \> in the night/evening \> \textit{saun} \> night; dark
	\end{tabbing}
	\label{exe:partday}
\end{exe}

When used adverbially, the times of day always occur at the beginning of the clause, illustrated in~(\ref{exe:godung}). They can be accompanied by \textit{bo} `go' to create the meaning `when it had turned [part of day]', as in~(\ref{exe:bogosaun}). 

\begin{exe}
	\ex
	{\gll \textbf{go\_dung} inier se koi bot\\
		morning \textsc{1du.ex} {\glse} again go\\
		\glt `In the morning we left again.' \jambox*{\href{http://hdl.handle.net/10050/00-0000-0000-0004-1BAE-4}{[narr44\_7:40]}}
	}
	\label{exe:godung}
	\ex
	{\gll \textbf{bo} \textbf{go.saun} mu se muap=at maraouk\\
		go evening \textsc{3pl} {\glse} food=\textsc{obj} put\_out\\
		\glt `When it turned evening they put out the food.' \jambox*{\href{http://hdl.handle.net/10050/00-0000-0000-0004-1BD8-4}{[narr1\_3:28]}}
	}
	\label{exe:bogosaun}
\end{exe}

To make the construction `last/earlier this + part of day'  \textit{opa} `earlier' is used. \textit{Opa} cannot be combined with \textit{(go) dung} `(in the) morning', but can be combined with another temporal adverbial, \textit{naupar} `morning'. The only time of day that retains \textit{go} in combination with \textit{opa} is \textit{go yuol}, perhaps because \textit{yuol} can also mean `light'. See the overview in~(\ref{exe:partdayopa}) and~(\ref{exe:opasaun}).

\begin{exe}
	\ex 	\begin{tabbing}
		\hspace*{3cm}\=\hspace*{5.5cm}\=\hspace*{3cm}\=\hspace*{4cm}\=\kill
		\textit{opa naupar} \> earlier this morning \> *\textit{opa go dung}\>\\
		\textit{opa go yuol} \> earlier today \> *\textit{opa yuol}\\
		\textit{opa ginggir} \> earlier this (late) afternoon \> \\
		\textit{opa saun} \> last night; earlier tonight; this evening \>
	\end{tabbing}
	\label{exe:partdayopa}
\end{exe}

\begin{exe}
	\ex 
	{\gll \textbf{opa} \textbf{saun} jam tiga an se toni min=kin\\
		last night o'clock three \textsc{1sg} {\glse} want sleep={\glkin}\\
		\glt `Last night at three o'clock I wanted to go sleep.' \jambox*{\href{http://hdl.handle.net/10050/00-0000-0000-0004-1B8F-4}{[narr32\_0:14]}}
	}
	\label{exe:opasaun}
\end{exe}

The days of the week are all derived from the Malay terms, which in turn are loans from Arabic. They are usually preceded by the Malay loan \textit{hari} `day'. The days are listed in~(\ref{exe:weekdays}). The pronunciation \textit{roba} for `Wednesday' seems to be rather marginal; most people use Malay \textit{rabu}. An example with \textit{ariemun} `Friday' is given in~(\ref{exe:ariemun}). \textit{Ariemun} seems to be a contraction of \textit{hari} `day' and \textit{emun} `mother; big' (since Friday is the most important day in Islam) rather than a loan from Malay/Indonesian \textit{jumat}, ultimately Arabic \textit{aljumʕa}.

\begin{exe}
	\ex 
	\begin{tabbing}
		\hspace*{4cm}\=\hspace*{4cm}\=\hspace*{4cm}\= \kill
		\indent weekday term \> loan from\\
		\indent \textit{senen} `Monday' \> < Malay \textit{senin}\\
		\indent \textit{selasa} `Tuesday' \> < Malay \textit{selasa}\\
		\indent \textit{roba} `Wednesday \> < Malay \textit{rabu}\\
		\indent \textit{kamis} `Thursday'\> < Malay \textit{kamis}\\
		\indent \textit{ariemun} `Friday' \> \\
		\indent \textit{saptu} `Saturday' \> < Malay \textit{sabtu}\\
		\indent \textit{ahat} `Sunday' \> < Malay \textit{ahad}
	\end{tabbing}
	\label{exe:weekdays}
\end{exe}

\begin{exe}
	\ex \gll an toni \textbf{ariemun} eba in tok bo=et\\
	\textsc{1sg} say Friday then \textsc{1pl.excl} first go={\glet}\\
	\glt `I said we wait until Friday and then we go.' \jambox*{\href{http://hdl.handle.net/10050/00-0000-0000-0004-1BA3-3}{[conv10\_6:57]}}
	\label{exe:ariemun}
\end{exe}

The moon phases are designated \textit{pak talawak} `new moon' and \textit{pak tubak} `full moon'. Four months from the Arabic calendar with Kalamang names are currently in use in a twelve-month system. They are listed in~(\ref{exe:months}), with the corresponding Malay names and the number of the month in the Arabic calendar. Three of the four month names are derived from nouns that are currently in use in Kalamang. The reason the month known as `safar' in Arabic is derived from Kalamang \textit{roba} `Wednesday' is that people in Malaysia and parts of Indonesia, including the Karas Islands, celebrate the last bath of the prophet Muhammad by bathing in the sea and having a picnic on the beach on the last Wednesday of that month.

\begin{exe}
	\ex 
	\begin{xlist}
	\ex \textit{robaherpak}, from \textit{roba} `Wednesday', Malay \textit{bulan shafar}, 2nd month
	\ex \textit{dilurpak}, from \textit{dilur} `?', Malay \textit{bulan maulud; rabiul akhir}, 6th month
	\ex \textit{tolaspak}, from \textit{tolas} `break the fast', Malay \textit{bulan puasa; bulan ramadhan}, 9th month
	\ex \textit{hajiwak}, from \textit{haji} `hajj', Malay \textit{bulan haji; dzulhijjah}, 12th month
	\end{xlist}
	\label{exe:months}
\end{exe}

The Karas Islands have two main meteorological seasons: \textit{kemanurpak}, lit. `west month', the wet season with winds from the west, and \textit{tagurpak}, lit. `east month', the dry season with winds from the east. These are not encountered in the corpus, and it remains unclear which position in the clause they take.

To make the construction `... ago', the period of time is put in clause-initial position with a high boundary tone. 

\begin{exe}
	\ex 
	{\glll minggu kon in ar=teba\\
		{} {H} {} {L} \\
		week one \textsc{1pl.excl} fish={\glteba}\\
		\glt `One week ago we were fishing.' \jambox*{\href{http://hdl.handle.net/10050/00-0000-0000-0004-1C60-A}{[elic\_adv\_22]}}
	}
\end{exe}

All `... ago' constructions were elicited. In practice, people prefer to use \textit{wis} to indicate that the event took place in the past. Like \textit{kemarin} in Malay, \textit{wis} can be yesterday, but also any other time before today. To indicate a really long time ago (usually when talking about another generation, or when telling fictional stories), one can use \textit{wiseme} `a long time ago'. (\ref{exe:wiseme}) is the beginning of a story about the Second World War.

\begin{exe}
	\ex \gll \textbf{wiseme} Jepang=bon Amerika=bon nau=sair=ten\\
	long\_time\_ago Japan=\textsc{com} America=\textsc{com} \textsc{recp}=shoot={\glten}\\
	\glt `A long time ago, Japan and America were at war.' \jambox*{\href{http://hdl.handle.net/10050/00-0000-0000-0004-1BBB-2}{[narr40\_0:03]}}
	\label{exe:wiseme}
\end{exe}

\subsection{Too/also}
\label{sec:too}
\textit{Weinun} `too; also' is an adverbial that modifies the clause. It is clause-final. There are no examples of \textit{weinun} with a transitive verb.

\begin{exe}
	\ex
	{\gll koi mindi \textbf{weinun} ba kahaman eirgan kit-pis\\
		then like\_that too but bottom both up-side\\
		\glt `Then [a picture] like that too, but both bottoms are up.' \jambox*{\href{http://hdl.handle.net/10050/00-0000-0000-0004-1BD6-8}{[stim38\_7:06]}}
	}
	\label{exe:weinun}
	\ex \gll ma gawar∼gawar \textbf{weinun}\\
	\textsc{3sg} fragrant∼\textsc{ints} too\\
	\glt `It was fragrant too.' \jambox*{\href{http://hdl.handle.net/10050/00-0000-0000-0004-1BA6-6}{[conv13\_7:47]}}
\end{exe}

\textit{Weinun} `too; also' can be combined with the NP enclitic \textit{=nan} `too; also'. It is unclear what the (pragmatic) effect is of combining the two.

\begin{exe}
	\ex
	{\gll wa=\textbf{nan} im karuok \textbf{weinun}\\
		\textsc{prox}=too banana three too\\
		\glt `[In] this [picture there are] three bananas too.' \jambox*{\href{http://hdl.handle.net/10050/00-0000-0000-0004-1BD6-8}{[stim38\_11:03]}}
	}
	\label{exe:nanweinun}
	\ex \gll bal-un=\textbf{nan} tiri mia \textbf{weinun}\\
	dog-\textsc{3poss}=too run come too\\
	\glt `His dog comes running too.' \jambox*{\href{http://hdl.handle.net/10050/00-0000-0000-0004-1B94-F}{[stim21\_2:27]}}
	\label{exe:unan}
\end{exe}	

%test 2020 dia juga beli gula. bapak juga makan nasi. ibu juga minum kopi (susu).

%\textit{Weinun} has a few co-occurrences with other adverbial modifiers. mun eirtaet weinun + teyasaet weinun in pic-match tasks.


\subsection{More, again}
\label{sec:taet}
There are two elements, a clitic and a word, that mark when a proposition is repeated. They can be combined.

The clitic \textit{=taet} `more; again', which is attached to the predicate, indicates that the proposition is repeated, continued or extended. It almost invariably co-occurs with \textit{koi} `again', which is further illustrated below. (\ref{exe:taet1}) and~(\ref{exe:taet2}) show \textit{=taet} together with \textit{koi}. In the latter, the meaning conveyed is slightly more that of \is{repetition}repetition than of continuation. %non-verbal examples: koi bolodaet = sedikit lagi, koi reidaktaet = tamba banyak lagi, koi kalistaet = masih hujan lagi, eirtaet, iniertaet

\begin{exe}
	\ex
	{\gll ma \textbf{koi} baran=\textbf{taet}\\
		\textsc{3sg} then descend=more\\
		\glt `Then he goes further down.' \jambox*{\href{http://hdl.handle.net/10050/00-0000-0000-0004-1C9A-E}{[narr37\_1:57]}}
	}
	\label{exe:taet1}
	\ex
	{\gll ibu nawanggar in \textbf{koi} sanggaran=\textbf{taet}\\
		miss wait \textsc{1pl.excl} again search=again\\
		\glt `Miss waits and we search again.' \jambox*{\href{http://hdl.handle.net/10050/00-0000-0000-0004-1C97-F}{[stim27\_6:05]}}
	}
	\label{exe:taet2}
	\ex \gll ma se kamera nerun=ka keluar=\textbf{taet}\\
	\textsc{3sg} {\glse} camera inside=\textsc{lat} exit-again\\
	\glt `She already went out of (the view of) the camera again.' \jambox*{\href{http://hdl.handle.net/10050/00-0000-0000-0004-1BA6-6}{[conv13\_6:35]}}
\end{exe}	
%2020 what about locative predicatE? saya lagi di kampung. dia lagi di fakfak. bpak lagi di laut. saya lagi pi fakafk (bo pakpakoraet?)

%can also be with weinun: muna eirtaet weinun. perhaps it was tak+et, often used in combo with koi, and is now grammaticalising to mean more, again on its own (viz. in ime ma he kamera nerungga kaluartaet)

In a few cases, \textit{=taet} is also found on pronouns, when the actor is in \is{focus}focus rather than the action. In~(\ref{exe:taet3}), where the speaker and the addressee are taking turns describing objects, the speaker encliticises \textit{=taet} to the pronoun, because the focus is on who is talking. (Cf. English `Now YOU again speak' vs. `Now you SPEAK again'). Similarly, in~(\ref{exe:andaet}), which is take from a story about a monkey and a cuscus who keep switching places in a boat, the focus is on the pronoun \textit{an} `I'.

%A: so -taet comes on what is in focus? anything else with same distribution? EV -saet is similar?
%In example~\ref{exe:taet4} \textit{-taet} is perhaps suffixed to the number because the clause lacks a verb. Deleted because nubmers can be predicative, as explained under numerals above.

\begin{exe}
	\ex
	{\gll ka=\textbf{taet} ewa=te\\
		\textsc{2sg}=again speak=\textsc{imp}\\
		\glt `You speak again.' \jambox*{\href{http://hdl.handle.net/10050/00-0000-0000-0004-1C75-D}{[stim15\_3:27]}}
	}
	\label{exe:taet3}
	\ex \gll ka me or=ko an=\textbf{taet} bo sabar=et\\
	\textsc{2sg} {\glme} stern=\textsc{loc} \textsc{1sg}=again go bow={\glet}\\
	\glt `You are at the stern, I again go to the bow.' \jambox*{\href{http://hdl.handle.net/10050/00-0000-0000-0004-1BC1-0}{[narr19\_5:10]}}
	\label{exe:andaet}
\end{exe}	

\textit{Koi} `again' indicates \is{repetition}repetition of an event. Its placement is variable. In~(\ref{exe:koi1}), \textit{koi} precedes the object and the verb, but in~(\ref{exe:koi2}) it follows the object and precedes the verb. In clauses without object, such as~(\ref{exe:koi3}), \textit{koi} is between the subject and the verb. \textit{Koi} always follows aspectual markers \textit{se} and \textit{tok}.

%count early 2020:
%tok koi = 9
%koi tok = 2
%se koi = 59 
%koi se = 1

\begin{exe}
	\ex
	{\gll ma \textbf{koi} kaluar ma \textbf{koi} kiun=at tu\\
		\textsc{3sg} again exit \textsc{3sg} again wife.\textsc{3poss}=\textsc{obj} hit\\
		\glt `[If] he goes out again he'll hit his wife again.' \jambox*{\href{http://hdl.handle.net/10050/00-0000-0000-0004-1BB0-D}{[stim12\_4:01]}}
	}
	\label{exe:koi1}
	\ex
	{\gll an se mat \textbf{koi} pouk\\
		\textsc{1sg} {\glse} \textsc{3sg.obj} again carry\_on\_back\\
		\glt `I carried him on my back again.' \jambox*{\href{http://hdl.handle.net/10050/00-0000-0000-0004-1BBB-2}{[narr40\_4:07]}}
	}
	\label{exe:koi2}
	\ex
	{\gll go\_dung inier se \textbf{koi} bot\\
		morning \textsc{1du.ex} {\glse} again go\\
		\glt `In the morning we left again.' \jambox*{\href{http://hdl.handle.net/10050/00-0000-0000-0004-1BAE-4}{[narr44\_7:40]}}
	}
	\label{exe:koi3}	
\end{exe}

%cut from complex clauses, want ik heb het niet erg aannemelijk gemaakt dat het een conj is (HH en AH), misschien kan ik er hier nog wat mee:
%\textit{Koi} `then; further' is a conjunction that links sequential actions or events. It follows the subject of the second clause, as illustrated in~\ref{exe:koi0}, which in contrast to~\ref{exe:koi} has an overt subject. It can also link states, as in example~\ref{exe:koi4}, where the speaker summarises the objects he sees in a picture. There, \textit{koi} `then; further' is combined with comitative \textit{=bon}. The comitative, marked on each object besides \textit{somkon} `one person', indicates the three objects are in the same picture as the person. \textit{Koi} `then; further', mentioned after the first object, indicates the speaker is summarising. \textit{Koi} is also an adverb meaning `again', see §\ref{sec:taet}.
%
%\begin{exe}
%	\ex
%	{\gll 	na=pasan=i koyet kayakat \textbf{koi} bo sanong=at rep\\
%		\textsc{lv}=put.up={\gli} finished wicker then go roof=\textsc{obj} get\\
%		`After putting it up, the wicker, then [we] go get the roof.' \jambox*{[narr6\_1:49]
%	}
%	\label{exe:koi}
%	\ex
%	{\gll 	Dajiba mu-kin lewat ma \textbf{koi} ra Binkur mu-kin\\
%		Dajiba \textsc{3pl-poss} pass \textsc{3sg} then go Binkur \textsc{3pl-poss}\\
%		`Past Dajiba's, he then goes [past] Binkur's.' \jambox*{[narr37\_2:31]
%	}
%	\label{exe:koi0}
%	
%	\ex
%	{\gll kon me son-kon mambara ror-un ar-kon=bon \textbf{koi} kuda kon=bon neba-un kon=bon\\
%		one \textsc{dist} person-one stand tree-\textsc{3poss} \textsc{clf\_stem}-on==\textsc{com} further horse one=\textsc{com} what-\textsc{3poss} one=\textsc{com}\\
%		`That one, one person stands with his one tree and further with one horse and with one what's-its-name.' \jambox*{[stim14\_1:03]
%	}
%	\label{exe:koi4}
%\end{exe}


\section{Unresolved uses of \textit{=ten}}
\label{sec:tenunal}
A clitic on the predicate \textit{=ten} was introduced in §\ref{sec:attr} as an attributive marker, which also functions as a relative clause marker (§\ref{sec:relcla}). Besides these occurrences of \textit{=ten}, several dozen examples in the natural spoken corpus cannot be analysed as an attributively used predicate or as a relativiser. With the data currently available I cannot arrive at a unified analysis for them, which is why the clitic remains glossed as \textsc{ten}. I briefly present some data here.

The clitic occurs several times on the manner \is{demonstrative!manner}demonstratives \textit{wandi} `like this' and \textit{mindi} `like that' (the latter is illustrated in example~\ref{exe:keweneko}), following similative marker \textit{=kap} (example~\ref{exe:rombon}), and on the question word \textit{tamandi} `how' (example~\ref{exe:manditen}). Another example, on the verb \textit{pue} `to hit', is given in~(\ref{exe:purenme}). An analysis that might fit for part of the data, represented by these examples, is that of a subordinate clause marker, expressing meanings like `since', `as' or `after'. This potential reading is indicated in brackets in the translation. If this analysis is correct, the relativiser use of \textit{=ten} is a specific example of this more general use.

\begin{exe}
	\ex \gll an kewe neko=et me \textbf{mindi=ten} me eranun\\
	\textsc{1sg} house inside={\glet} {\glme}  like\_that=\textsc{ten} {\glme} cannot\\
	\glt `If I'm in the house (as it is like that) I can't do it. \jambox*{\href{http://hdl.handle.net/10050/00-0000-0000-0004-1BA6-6}{[conv13\_10:40]}}
	\label{exe:keweneko}
	\ex \gll in opa rombongan baran=\textbf{ten=kap} me tengguen=i koyet in se mengga kubirar=ka bot\\
	\textsc{1pl.excl} {\glopa} group descend=\textsc{ten=sim} {\glme} gather={\gli} finish \textsc{1pl.excl} {\glse} \textsc{dist.lat} graveyard=\textsc{lat} go\\
	\glt `(When) we, that group, moved down, had all gathered, we went from there to the graveyard.' \jambox*{\href{http://hdl.handle.net/10050/00-0000-0000-0004-1BD8-4}{[narr1\_1:15]}}
	\label{exe:rombon}
	\ex \gll sayang \textbf{tamandi}=\textbf{ten} ma wandi pi nak=komahal\\
	nutmeg how=\textsc{ten} \textsc{3sg} like\_this \textsc{1pl.incl} just=not\_know\\
	\glt `Why (how come) is the nutmeg like this? We just don't know.' \jambox*{\href{http://hdl.handle.net/10050/00-0000-0000-0004-1BBD-5}{[conv12\_16:54]}}
	\label{exe:manditen}
	\ex \gll kaden-un metko pue=\textbf{ten} me supaya tu=te di=metko=et bisa balama=te mindi naladur=et bisa\\
	body-\textsc{3poss} \textsc{dist.loc} hit=\textsc{ten} {\glme} so\_that hit={\glte} \textsc{caus}=\textsc{dist.loc}={\glet} can heat\_in\_fire={\glte} like\_that massage={\glet} can\\
	\glt `(When?) [you] hit the body there, so that hitting and putting it there is possible, and heating in the fire and massaging is possible.' \jambox*{\href{http://hdl.handle.net/10050/00-0000-0000-0004-1BB1-3}{[narr34\_2:11]}}
	\label{exe:purenme}
\end{exe}

Not all examples can get a subordinate clause reading, however. The following example shows \textit{=ten} at the end of an utterance.

\begin{exe}
	\ex
	\begin{xlist}
	\exi{A:}
	\gll ka nan=et me mesang=\textbf{ten=kap}\\
	\textsc{2sg} consume={\glet} {\glme} dregs=\textsc{ten=sim}\\
	\glt `If you eat, [it tastes] like dregs.'
	\exi{B:}
	\gll nain plastik mindi\\
	like plastic like\_that\\
	\glt `Like plastic.'	
	\exi{A:}
	\gll ema\\
	mother\\
	\glt `Mother!' \jambox*{\href{http://hdl.handle.net/10050/00-0000-0000-0004-1BA6-6}{[conv13\_5:25]}}
	\end{xlist}
\end{exe}


A combination of \textit{=ten} and \textit{=saet} `all; only; exclusively' is found in indefinite pronoun constructions. Both emphatic clitics, like `only' and constructions similar to non-specific free relative clauses, are cross-linguistically common in the expression of concessive \is{conditional}conditionals \parencite{haspelmath2001indef,haspelmath1998}, so it is possible that this is a use of the relative clitic \textit{=ten}. 

\begin{exe}
	\ex \gll pi don kon se jien=\textbf{ten=saet}\\
	\textsc{1pl.excl} thing one {\glse} get=\textsc{ten}=only\\
	\glt `Whatever thing we obtain...' \jambox*{\href{http://hdl.handle.net/10050/00-0000-0000-0004-1BA4-1}{[conv16\_12:45]}}
	\ex \gll set me memang tebonggan bes=\textbf{ten=saet}\\
	bait {\glme} true all good=\textsc{ten}=only\\
	\glt `The bait, to be honest, they are all good.' \jambox*{\href{http://hdl.handle.net/10050/00-0000-0000-0004-1C76-B}{[stim16\_3:12]}}
\end{exe}
%indef prons are not really described. only this here and under -saet in nouns, and konkon in quantifiers. maybe copy strategies from negation paper.


%earlier comments HArald:
% on mindi:
%\textit{(Harald: ``the first two look like you might have the `topic as conditional' construction and since \ref{exe:minditenme} has -ten in the apodosis too maybe you are dealing with a construction for `If X is Z, Y is likewise Z' expressed as `If X being Z like that, Y being also Z'.'' This seems to make sense when I apply this hypothesis to the other corpus examples, too.)}
%%(88-90) look similar to English "as" in combination with English "-ing".
%In languages like Turkish or Uzbek (or many other verb final languages ---
%I just happen to know them) you often get a participle like -gan or -en
%which can make relative clauses as well as subsistute for subordinate
%clauses of many kinds with a suitable postposition (e.g. since, as,
%after, etc.), but English -ing is not a bad analogy either.
% on indef prons
%(92-93) only one example with a translation (does 92 mean 'whatever
%thing we obtain'?) but in many verb-final languages you get relative
%clauses with a particle (only, even, too) with a quantifier or
%question words for various kinds of indefiniteness, i.e., they
%substute for '-ever' in English who-ever (cf. Dutch wie *dan ook*) and
%for 'a certain' vs. 'any' vs. 'a'. Do you know Haspelmath's book on
%indefinite pronouns? It's probably a good place to start, to check
%what readings are often with a particle meaning 'only'.
% on tamandi:
%(97-99) often languages cleft in questions to focus, e.g., qu'est-ce
%que in French and its similar in insular Celtic and Amharic and
%probably many other lgs that I don't know. But that of course
%typically happens with all question words and not in languages which
%already have a focus marker. But if it's just tamandi in Kalamang that
%can take -ten it would not be more bizarre than e.g. English 'how
%come' developed out phrases like 'how is it that' or 'how has is come
%to be that' but we don't have '*why come' or '*what come' even if
%'why is it that' and 'what is it that' are used. Maybe someone who knows
%more about question words can say whether the 'how' Q-word shows this kind
%of development more often than the others.
