\chapter{Demonstratives}\is{demonstrative|(}
\label{ch:dems}
Demonstratives, introduced in §\ref{sec:wcdem}, are a closed class of forms that locate a referent in space, time or discourse in relation to the deictic centre. Kalamang has a rich demonstrative system with six versatile but mutually exclusive forms that can occur as the NP head, as a modifier in the NP (occupying the last slot of the NP) and as (part of) the predicate. In §\ref{sec:demintro}, I introduce the basic forms and how they can be inflected. In §\ref{sec:wa} to~§\ref{sec:elevdem}, I describe the peculiarities of each of the six demonstratives.\is{article*}\is{definiteness*}

\section{Basic forms and inflections}
\label{sec:demintro}
This section starts with a presentation of the six basic forms, followed by a discussion of the demonstrative affixes and enclitics (§\ref{sec:basicdem}). ``Empty'' affixes used to create longer demonstrative forms are described in §\ref{sec:empty}, suffixes to create manner, quality, quantity and degree demonstratives are introduced in §\ref{sec:mqqd} and the postpositions found on demonstratives are given in §\ref{sec:demcase}.

\subsection{The basic forms}
\label{sec:basicdem}
Kalamang has six basic demonstrative forms: proximal \textit{wa}, distal \textit{me}, far distal \textit{owa}, the elevational demonstratives \textit{yawe} `\textsc{down}' and \textit{osa} `\textsc{up}', and an anaphoric demonstrative \textit{opa}. Only the proximal and distal forms can be used pronominally, adnominally and identificationally. The far distal and the elevational demonstratives can be used adnominally and identificationally. The anaphoric demonstrative can only be used adnominally. The syntactic use of these demonstratives is given in Table~\ref{tab:demch}. Pronominal demonstratives occur instead of nouns as NP heads. Adnominal demonstratives modify nouns and pronouns in the NP, following the head and occupying the rightmost slot of the NP. Predicative demonstratives are locative forms (which may form the single predicate of the clause) and lative forms (which combine with other verbs to form complex predicates, see §\ref{sec:mvcgoal}). Identificational demonstratives occur as the predicate in copular\is{copula*!clause} and non-verbal clauses. The syntactic behaviour of the six forms is given in Table~\ref{tab:demch}, repeated from §\ref{sec:wcdem}. All basic forms are mutually exclusive.

\begin{table}[ht]
	\caption{Demonstratives and their syntactic use}
	\label{tab:demch}
		\begin{tabular}{lccc}
			\lsptoprule 
			form&pronominal&adnominal&identificational\\
			\midrule
			\textit{wa} \textsc{prox} & $+$ & $+$ & $+$\\
			\textit{me} \textsc{dist}&  $+$ & $+$ & $+$\\
			\textit{owa} \textsc{fdist}&&$+$&$+$\\
			\textit{yawe} \textsc{down} &&$+$&$+$\\
			\textit{osa} \textsc{up} &&$+$&$+$\\
			\textit{opa} \textsc{ana} &  & $+$&\\ \lspbottomrule
		\end{tabular}
\end{table}

The two most versatile basic forms, proximal \textit{wa} and distal \textit{me}, typically evoke participant-anchored spatial information. An illustration of each is given in~(\ref{exe:kewewa}).

\begin{exe}
	\ex 
	\begin{xlist}
		\exi{A:}
		{\gll berarti kewe-un=a \textbf{wa} ye\\
			that\_means house-\textsc{3poss=foc} \textsc{prox} or\\
			\glt `That means this is his house?' [points at picture lying on the table]}
		\exi{B:} \gll kewe-un=a \textbf{me} kewe main ecien=i kewe-un=a \textbf{wa}\\
		house-\textsc{3poss=foc} \textsc{dist} house \textsc{3poss} return={\gli} house-\textsc{3poss=foc} \textsc{prox}\\
		\glt `That's his house, he returns to his house.' [points at picture] `This is his house.' \jambox*{\href{http://hdl.handle.net/10050/00-0000-0000-0004-1BA9-9}{[stim6\_16:51]}}
	\end{xlist}
	\label{exe:kewewa}
\end{exe}

The far distal is typically used on a bigger scale, across landscape. An example is~(\ref{exe:yonder}). The far distal basic form \textit{owa} is hardly found uninflected. In this example, it carries a demonstrative suffix \textit{-ne}, which I comment upon below.

\begin{exe}
	\ex \gll lempuang temun=a \textbf{owa}-ne tumun-un=a wa\\
	island big=\textsc{foc} \textsc{fdist-dem} child-\textsc{3poss=foc} \textsc{prox}\\
	\glt `The big island is yonder, the small one is here.' \jambox*{\href{http://hdl.handle.net/10050/00-0000-0000-0004-1B6C-1}{[conv27\_0:02]}}
	\label{exe:yonder}
\end{exe}	

The elevational demonstratives refer to referents on a vertical axis: \textit{osa} and \textit{yawe} are used for referents which are located higher and lower, respectively, than the speaker or another point of reference. (\ref{exe:adisor}) and~(\ref{exe:yawee}) are adnominal uses.

\begin{exe}
	\ex \gll adi sor \textbf{osa} {\ob}...{\cb} kabaruap\\
	\textsc{der} fish \textsc{up} {\ob}...{\cb} grouper\\
	\glt `That fish up there {\ob}...{\cb} is a grouper.' \jambox*{\href{http://hdl.handle.net/10050/00-0000-0000-0004-1BA3-3}{[conv10\_5:49]}}
	\label{exe:adisor}
	\ex \gll ma tamatko ka bo minggalot-an \textbf{yawe} kome=te \\
	\textsc{3sg} where \textsc{2sg} go bedroom-\textsc{1sg.poss} \textsc{down} look=\textsc{imp}\\
	\glt `Where is it? You go look in my bedroom down there!' \jambox*{\href{http://hdl.handle.net/10050/00-0000-0000-0004-1BB5-B}{[conv17\_3:42]}}
	\label{exe:yawee}
\end{exe}	

The anaphoric demonstrative \textit{opa} occurs with referents that in some respect represent knowledge already shared by the speaker and the addressee, typically those that are previously mentioned in discourse. The latter is the case with the referent \textit{dodon} `clothing' in~(\ref{exe:dodonop}).

\begin{exe}
	\ex \gll dodon \textbf{opa} an masa\\
	clothing \textsc{ana} \textsc{1sg} dry\_in\_sun\\
	\glt `I dried those clothes [that we talked about] in the sun.' \jambox*{\href{http://hdl.handle.net/10050/00-0000-0000-0004-1B9F-F}{[conv9\_0:20]}}
	\label{exe:dodonop}
\end{exe}	

\subsection{``Empty'' demonstrative affixes}
\label{sec:empty}
Two prefixes and one suffix, \textit{yu-}, \textit{i-} and \textit{-ne}, can be added to some of the basic forms above. They seemingly have no other function than to create a longer form of the demonstratives. The current corpus does not clarify a difference in meaning, distribution or pragmatic use between the forms with and without these affixes. Table~\ref{tab:emptydem} shows the possible combinations. 

\begin{table}[ht]
	\caption{Possible combinations of demonstratives and affixes}
	\label{tab:emptydem}
	
		\begin{tabular}{ll l l}
			\lsptoprule 
			\textsc{prox} & \textit{yu-} & \textit{wa} & \textit{-ne}\\
			&& \textit{wa} & \textit{-ne}\\
			\textsc{dist} &\textit{yu-} & \textit{me} & \textit{-ne}\\
			&& \textit{me} & \textit{-ne}\\
			&\textit{i-} & \textit{me} & \textit{-ne}\\	
			\textsc{fdist} && \textit{owa} & \textit{-ne}\\
			\textsc{up} &	& \textit{osa} & \textit{-ne}\\
			\textsc{down} && \textit{yawe} & \textit{-ne}\\		
			\lspbottomrule
		\end{tabular}
	
\end{table}

The prefix \textit{yu-} is found on proximal \textit{wa} and distal \textit{me} in pronominal, adnominal and identificational use. Based on the fact that all other basic demonstratives are disyllabic, \textit{yuwa} and \textit{yume} could be argued to be the basic proximal and distal forms, and \textit{wa} and \textit{me} their shortened versions. There is currently no other evidence to corroborate this suggestion, such as emphatic use of the forms with \textit{yu-}.

The prefix \textit{i-} is only found on distal \textit{me}, and only in adnominal and predicative uses, next to adnominal and predicative uses without \textit{i-}. 

The suffix \textit{-ne} is found on five of the six demonstratives: proximal \textit{wa}, distal \textit{me}, far distal \textit{owa} and both the elevationals \textit{osa} `\textsc{up}' and \textit{yawe} `\textsc{down}'. As a suggestion for a starting point for future research, it might be worth looking into the fact that \textit{-ne} attaches to all the basic demonstratives involved in spatial reference, but not to the anaphoric \textit{opa}.

The proximal and distal forms can take one or both prefixes (not simultaneously) and the suffix. They are attested with one of the prefixes and the suffix at the same time resulting in the extra long forms \textit{yu-wa-ne}, \textit{yu-me-ne} and \textit{i-me-ne}. Because these forms are not currently analysable, their morphemes are not separated in the rest of this work.

\largerpage
\subsection{Manner, quality, quantity and degree suffixes}\is{demonstrative!manner, quality, quantity, degree}
\label{sec:mqqd}
The proximal and distal forms \textit{wa} and \textit{me} can be inflected with suffixes unique to the class of demonstratives expressing manner, quality, quantity, or degree. Manner and quality demonstratives are made with \textit{-ndi} `like',\footnote{Compare also \textit{tamandi} `how', apparently formed with the question word root \textit{tama} and the suffix \textit{-ndi}.} quantity demonstratives are made with \textit{-bes} `\textsc{qnt}', degree demonstratives expressing size are made with \textit{-rip} `\textsc{dgr}' and degree demonstratives expressing distance are made with \textit{-sen} `\textsc{dgr}'. The latter is also used for duration. All occur adverbially and/or adnominally. The distal forms can be made with the basic form \textit{me}, but have an alternative form with the root \textit{mia-}. The distal manner and quality demonstrative has a root \textit{mi-} instead of \textit{me}. Table~\ref{tab:mqqd} gives an overview. More details and examples of the forms are given in §\ref{sec:mqqdwa} for the proximal forms and §\ref{sec:distmann} for the distal forms.

\begin{table}[ht]
	\caption{Manner, quality, quantity and degree suffixes}
	\label{tab:mqqd}
	 \fittable{
		\begin{tabular}{l l l l l}
			\lsptoprule 
			type& gloss & syntactic function&proximal form&distal form\\
			\midrule
			manner/quality & `like' &adnominal/adverbial & \textit{wa-ndi} & \textit{mi-ndi}\\
			quantity &\textsc{qnt}& adnominal & \textit{wa-bes} & \textit{me-bes/mia-bes}\\
			degree (size)&\textsc{dgr} & adverbial & \textit{wa-rip} & \textit{me-rip/mia-rip}\\
			degree (distance)&\textsc{dgr} & adnominal/adverbial & \textit{wa-sen} & \textit{me-sen/mia-sen}\\	\lspbottomrule
		\end{tabular}
		}
	
\end{table}

The prefixes \textit{yu-} and \textit{i-}, used to create longer forms of the demonstratives, are also found on manner/quality demonstrative forms. \textit{Yu-wa-ndi}, \textit{yu-mi-ndi} and \textit{i-mi-ndi} (\textit{i-} is not attested on \textit{wa}) are attested in the corpus.

\subsection{Postpositions on demonstratives}\is{demonstrative!postposition}
\label{sec:demcase}
Most demonstrative forms can carry several postpositions\is{postposition!on demonstratives} (§\ref{sec:case}). Postpositions attach to the right edge of the NP. On demonstratives, they have slightly different forms, but across the different demonstrative basic forms they behave regularly. Demonstratives inflected with postpositions are treated as fossilised forms, so just their surface forms are given and their morphemes are not separated in the glosses.

Proximal \textit{wa}, distal \textit{me}, \textit{osa} `\textsc{up}' and \textit{yawe} `\textsc{down}' carry a final \textit{-t} when in object position, or when modifying an object noun. The regular object postposition is \textit{=at} (§\ref{sec:at}). Proximal \textit{wa} and \textit{osa} `\textsc{up}' perhaps carry the full suffix (as two identical adjacent vowels are realised as a single vowel, §\ref{sec:hiatus}). These forms are treated as fused forms in the rest of this work.

\begin{table}[ht]
	\caption{Object demonstrative forms}
	\label{tab:objdem}
	
		\begin{tabular}{l l l l}
			\lsptoprule 
			&surface &underlying & gloss\\
			\midrule
			proximal & \textit{wat} & \textit{wa=(a)t} & \textsc{prox=obj}\\
			distal & \textit{met} & \textit{me=t} & \textsc{dist=obj} \\
			\textsc{down}& \textit{yawet} & \textit{yawe=t} & \textsc{down=obj}\\
			\textsc{up} & \textit{osat} & \textit{osa=(a)t} & \textsc{up=obj}\\
			\lspbottomrule
		\end{tabular}
	
\end{table}

Kalamang has a locative postposition \textit{=ko} (§\ref{sec:loc}) and a lative postposition \textit{=ka} (§\ref{sec:lat}), which can turn the NPs to which they attach into predicates (§\ref{sec:locclau}). All demonstratives except the anaphoric \textit{opa} can carry locative and lative postpositions. It is in such locative and lative constructions that the far distal and elevationals are most commonly found. The locative and lative demonstrative forms are slightly different from other NPs. As introduced in §\ref{sec:demprob}, on demonstratives, the phoneme \textit{-t} is inserted before the locative enclitic, and the phoneme \textit{-n} before the lative enclitic. The forms and morphemes are given in Table~\ref{tab:loclatdem}.

\begin{table}[ht]
	\caption{Locative and lative demonstrative forms}
	\label{tab:loclatdem}
        \fittable{
		\begin{tabular}{l r l r l}
			\lsptoprule 
			&locative& gloss&lative&gloss\\
			\midrule
			proximal & \textit{watko}, \textit{wa-\textcolor{lsLightWine}{t}=ko} & \textsc{prox-t=loc} & \textit{wangga}, \textit{wa-\textcolor{lsMidDarkBlue}{n}=ka} & \textsc{prox-n=lat}\\
			distal & \textit{metko}, \textit{me-\textcolor{lsLightWine}{t}=ko} & \textsc{dist-t=loc} & \textit{mengga}, \textit{me-\textcolor{lsMidDarkBlue}{n}=ka} & \textsc{dist-n=lat}\\
			far distal & \textit{owatko}, \textit{owa-\textcolor{lsLightWine}{t}=ko}& \textsc{fdist-t=loc}& \textit{owangga}, \textit{owa-\textcolor{lsMidDarkBlue}{n}=ka}& \textsc{fdist-n=lat}\\
			\textsc{down}& \textit{yawetko}, \textit{yawe-\textcolor{lsLightWine}{t}=ko} & \textsc{down-t=loc}& \textit{yawengga}, \textit{yawe-\textcolor{lsMidDarkBlue}{n}=ka}&\textsc{down-n=lat}\\
			\textsc{up} & \textit{osatko}, \textit{osa-\textcolor{lsLightWine}{t}=ko} & \textsc{up-t=loc} & \textit{osangga}, \textit{osa-\textcolor{lsMidDarkBlue}{n}=ka}&\textsc{up-n=lat}\\ \lspbottomrule
		\end{tabular}
		}
	
\end{table}

The longer forms of the demonstratives with prefixes \textit{yu-} and \textit{i-} (§\ref{sec:empty}) are also found on the object, locative and lative forms of the proximal and distal. The longer forms with \textit{-ne} are found on some forms in the object position. The following forms are attested.

\begin{exe}
	\ex
	\begin{xlist}
	\ex \textit{yu-watko}
	\ex \textit{yu-wangga}
	\ex \textit{yu-metko}
	\ex \textit{i-metko}
	\ex \textit{yu-mengga}
	\ex \textit{i-mengga}
	\ex \textit{wa-ne=t}
	\ex \textit{owa-ne=t}
	\ex \textit{osa-ne=t}
	\end{xlist}
\end{exe}

One demonstrative, the distal \textit{me}, has an instrumental form \textit{minggi}. The root \textit{mi-} is also encountered in the manner/quality form \textit{mindi}. On other NPs, the instrumental enclitic is \textit{=ki}. Perhaps, analogous to the locative and lative forms, the underlying form is \textit{mi-n=ki}. Like the locative and lative forms, the instrumental demonstrative is treated as a fossilised form and is always displayed as \textit{minggi}, glossed as `\textsc{dist.ins}'. It remains to be investigated whether the instrumental attaches to the other demonstratives. No long forms \textit{i-minggi} or \textit{yu-minggi} were attested.

\section{Demonstrative function}
This section describes the use of the six demonstratives. §\ref{sec:wa} discusses the uses of proximal \textit{wa}, §\ref{sec:me} that of distal \textit{me} and §\ref{sec:owa} that of far distal \textit{owa}. §\ref{sec:demopa} describes anaphoric demonstrative \textit{opa}, and §\ref{sec:elevdem} focuses on elevationals \textit{yawe} `\textsc{down}' and \textit{osa} `\textsc{up}'. \is{discourse}Discourse organisational use of demonstrative forms (§\ref{sec:discdems}) is only found with distal forms. 

As an introduction, consider the following example. The two most versatile basic forms, the proximal \textit{wa} and the distal \textit{me}, typically evoke speaker-oriented spatial information. This is illustrated in~(\ref{exe:kewewaa}), repeated from §\ref{sec:basicdem}, where speaker A uses the proximal form for a picture that is on the table in front of him. Speaker B first uses the distal form, and then switches to the proximal form in combination with pointing, because speaker A has now picked up the picture so that B is closer to it.

\begin{exe}
	\ex 
	\begin{xlist}
		\exi{A:}
		{\gll berarti kewe-un=a \textbf{wa} ye\\
			that\_means house-\textsc{3poss=foc} \textsc{prox} or\\
			\glt `That means this is his house?' [points at picture lying on the table]}
		\exi{B:} \gll kewe-un=a \textbf{me} kewe main ecien=i kewe-un=a \textbf{wa}\\
		house-\textsc{3poss=foc} \textsc{dist} house \textsc{3poss} return={\gli} house-\textsc{3poss=foc} \textsc{prox}\\
		\glt `That's his house, he returns to his house.' [points at picture] `This is his house.' \jambox*{\href{http://hdl.handle.net/10050/00-0000-0000-0004-1BA9-9}{[stim6\_16:51]}}
	\end{xlist}
	\label{exe:kewewaa}
\end{exe}

These basic forms, as well as the far distal \textit{owa}, have additional semantic and pragmatic properties other than the spatial ones. In the following sections, temporal, manner, anaphoric and other uses of the demonstratives are described. With such a plethora of forms and functions, it is not surprising that we find utterances like~(\ref{exe:pai}). Taken from an explanation about the use of plant medicine, it shows a nice combination of the pragmatic use of demonstratives (\textit{metko} as a sequential marker and the expression \textit{ma he me}, discussed in §\ref{sec:discdems}) and the spatial semantics of demonstratives (predicative \textit{watko} `here' and pronominal \textit{wane} `this').

\begin{exe}
	\ex \glll Pai koyet metko mindi kouet, mindi kou watko, wane, wane. Metkoet eh ma he me.\\
	pak=i koyet \textbf{metko} \textbf{mindi} kou=et \textbf{mindi} kou \textbf{watko} \textbf{wane} \textbf{wane} \textbf{metko}=et eh ma se \textbf{me}\\
	chew={\gli} finish \textsc{dist.loc} like\_that blow={\glet} like\_that blow \textsc{prox.loc} \textsc{prox} \textsc{prox} there={\glet} \textsc{int} \textsc{3sg} {\glse} \textsc{dist}\\
	\glt `After chewing, then we do like that, blow, blow like this here, this and this. Then, that's it.' \jambox*{\href{http://hdl.handle.net/10050/00-0000-0000-0004-1BE4-8}{[narr31\_4:10]}}
	\label{exe:pai}
\end{exe}

Figure~\ref{fig:wane} shows the video stills from the time of utterance of the three spatial demonstratives in~(\ref{exe:pai}), which are accompanied by pointing.

\begin{figure}[ht]  
	\centering
%	\hspace*{\fill}
\subfigure[]{
	    \includegraphics[width=.3\textwidth]{Images/hairwatko.png}
	    \label{fig:watko}
	    }
	\hfill
\subfigure[]{
	    \includegraphics[width=.3\textwidth]{Images/hairwane0.png}
	    \label{fig:wane0}
	    }
	\hfill
\subfigure[]{
	    \includegraphics[width=.3\textwidth]{Images/hairwane1.png}
	    \label{fig:wane1}
	    }
%	\hspace*{\fill}
	\caption[Demonstratives and pointing]{Hair Yorkuran explaining the application of \textit{langgulanggur}, \textit{watko} `here' (left), \textit{wane} `this' (middle) and \textit{wane} `this' (right)}\label{fig:wane}
\end{figure}



\subsection{Proximal \textit{wa} `\textsc{prox}'}\is{demonstrative!proximal}
\label{sec:wa}

\subsubsection{Spatial use}\is{demonstrative!spatial}
Proximal demonstrative \textit{wa} is prototypically used adnominally, pronominally and identificationally to indicate referents that are close to the speaker. (\ref{exe:waratu}) illustrates an adnominal and~(\ref{exe:dempro}) a pronominal use of the proximal demonstrative, both in object position (hence the object form \textit{wat}). The demonstrative in~(\ref{exe:waratu}) refers to a woman in a picture in front of the speaker, with the speaker pointing at her. The demonstrative in~(\ref{exe:dempro}) stands in for a fishing net the speaker is holding.

\begin{exe}  
	\ex \gll ma enem \textbf{wat}=a tu\\
	\textsc{3sg} woman \textsc{prox.obj=foc} hit\\	
	\glt `He hits this woman.' \jambox*{\href{http://hdl.handle.net/10050/00-0000-0000-0004-1BA9-9}{[stim6\_11:45]}}
	\label{exe:waratu}
\end{exe}

\begin{exe} 
	\ex \gll ki \textbf{wat} napaki=kin ye ge\\
	\textsc{2pl} \textsc{prox.obj} use=\textsc{vol} or not\\	
	\glt `Are you going to use this or not?' \jambox*{\href{http://hdl.handle.net/10050/00-0000-0000-0004-1BCD-C}{[conv3\_1:59]}}
	\label{exe:dempro}
\end{exe}

Two identificational examples are given in~(\ref{exe:namunawa}) and~(\ref{exe:dauanawa}). In~(\ref{exe:namunawa}), the speaker points at a picture with three people and identifies them one by one. (\ref{exe:dauanawa}) is from a story about a wedding, where the names of the bride's and groom's families are called out as a way of introducing the family members to couple. Note that an identificational demonstrative follows a \is{focus}focused noun, but precedes a \is{topic}topicalised noun.

\begin{exe}
	\ex \gll namun=a \textbf{wa} kiun=a \textbf{wa} tumun-un=a \textbf{wa}\\
	husband.\textsc{3poss=foc} \textsc{prox} wife-\textsc{3poss=foc} \textsc{prox} child-\textsc{3poss=foc} \textsc{prox}\\
	\glt `This is the husband, this is the wife, this is their child.' \jambox*{\href{http://hdl.handle.net/10050/00-0000-0000-0004-1BA9-9}{[stim6\_19:23]}}
	\label{exe:namunawa}
	\ex \gll supaya canam gonggin ma toni o \textbf{wa} me dauk-an=a \textbf{wa} ketan-an=a \textbf{wa} esa-an=a \textbf{wa} mama-an=a \textbf{wa}\\
	so\_that man know \textsc{3sg} say \textsc{int} \textsc{prox} {\glme} sibling\_in\_law-\textsc{1sg.poss=foc} \textsc{prox} parent\_in\_law-\textsc{1sg.poss=foc} \textsc{prox} uncle-\textsc{1sg.poss=foc} \textsc{prox} uncle-\textsc{1sg.poss=foc} \textsc{prox}\\
	\glt `So that the man knows, he says: ``O, this is my sibling-in-law, this is my parent-in-law, this is my uncle, this is my uncle.''' \jambox*{\href{http://hdl.handle.net/10050/00-0000-0000-0004-1B70-6}{[narr4\_3:12]}}
	\label{exe:dauanawa}
\end{exe}	

The longer forms \textit{wane} and \textit{yuwane} are often used identificationally, typically when identifying the right-hand picture in a picture-matching task\is{picture-matching task} (example~\ref{exe:mawane}). They may also be pronominal (example~\ref{exe:wanet}) or adnominal (example~\ref{exe:konyuwane}).

\begin{exe}	
	\ex \gll ma \textbf{wane}\\
	\textsc{3sg} \textsc{prox}\\
	\glt `This is it.' \jambox*{\href{http://hdl.handle.net/10050/00-0000-0000-0004-1BE3-6}{[stim39\_2:04]}}
	\label{exe:mawane}
	\ex \gll mu toni ya \textbf{wanet} me rasa\\
	\textsc{3pl} say yes \textsc{prox.obj} {\glme} like\\
	\glt `They said: ``Yes, this is good.''' \jambox*{\href{http://hdl.handle.net/10050/00-0000-0000-0004-1BC5-7}{[narr16\_2:12]}}
	\label{exe:wanet}
	\ex \gll neba kon=a yuwa sor tumun kon \textbf{yuwane} sor sair=ten\\
	what one=\textsc{foc} \textsc{prox} fish small one \textsc{prox} fish bake=\textsc{at}\\
	\glt `What is this one here, this one small fish here, baking fish?' \jambox*{\href{http://hdl.handle.net/10050/00-0000-0000-0004-1BCE-D}{[conv4\_3:53]}}
	\label{exe:konyuwane}
\end{exe}

The locative and lative proximal forms \textit{watko} `here' and \textit{wangga} `to/from here' are used for indicating the \is{location}location of referents close to the speaker. In~(\ref{exe:ramin}), the speaker uses \textit{watko} to refer to her side of the fishing net, while the addressee is holding the other side a few metres away. In~(\ref{exe:hadiwat}), proximal \textit{watko} refers to Mas village, which is where the speaker is when she utters the sentence.

\begin{exe}
	\ex \gll ka-mun mindi rami=in mena ma \textbf{watko}\\
	\textsc{2sg-proh} like\_that pull=\textsc{proh} otherwise \textsc{3sg} \textsc{prox.loc}\\
	\glt `Don't you pull like that, otherwise it will be here.' \jambox*{\href{http://hdl.handle.net/10050/00-0000-0000-0004-1BCE-D}{[conv4\_4:30]}}
	\label{exe:ramin}
	\ex \gll Hadi me \textbf{watko}\\
	Hadi {\glme} \textsc{prox.loc}\\
	\glt `Hadi was here.' \jambox*{\href{http://hdl.handle.net/10050/00-0000-0000-0004-1BBB-2}{[narr40\_15:53]}}
	\label{exe:hadiwat}
	\ex  \gll \textbf{wangga} Tamisen=ka bot-un eranun\\
	\textsc{prox.lat} Tamisen=\textsc{lat} go-\textsc{nmlz} cannot\\
	\glt `One cannot go from here to Tamisen (Antalisa).' \jambox*{\href{http://hdl.handle.net/10050/00-0000-0000-0004-1BE5-2}{[narr38\_0:09]}}
	\label{exe:wata}
\end{exe}

\textit{Watko} can be used in combination with pointing. (\ref{exe:komana}) is uttered while the speaker points at the eye of a fish lure. 

\begin{exe}
	\ex \gll an kona \textbf{watko}=a komain=et kanggir-un=ko to\\
	\textsc{1sg} think \textsc{prox.loc=foc} puncture={\glet} eye-\textsc{3poss=loc} right\\
	\glt `I think you puncture it here, in its eye, right?' \jambox*{\href{http://hdl.handle.net/10050/00-0000-0000-0004-1C75-D}{[stim15\_4:32]}}
	\label{exe:komana} 
\end{exe}

\textit{Watko}, like other NPs carrying a locative postposition, can be used predicatively, essentially meaning `to be here'. \textit{Wangga}, like other NPs carrying the lative postposition, must always be used in combination with other verbs. Complex source, goal and location constructions are described in §\ref{sec:mvcgoal}.

%conceptually close example: tumtum [...] kewe wat paku-in. 

\subsubsection{Temporal use}\is{demonstrative!temporal}
The proximal demonstrative is seldom used temporally. There are two exceptions. The long forms \textit{yuwa} and \textit{yuwane}, but not \textit{wane} and \textit{wa}, are used in combination with the time adverbial \textit{opa} `earlier' to create the meaning `(earlier) today'. There is no monomorphemic word for `today'.

\begin{exe}
	\ex \gll opa \textbf{yuwa} mu libur=et\\
	earlier \textsc{prox} \textsc{3pl} free={\glet}\\
	\glt `Were they free today?' \jambox*{\href{http://hdl.handle.net/10050/00-0000-0000-0004-1BCE-D}{[conv4\_0:47]}}
	\label{exe:opayuwa}
\end{exe}

%\subsection{Other}
%There are two instances of the proximal demonstrative post-predicately, which function I do not quite understand. They could have some kind of emphasising function. Alternatively, \textit{yuwa} in example~\ref{exe:rekamda} is temporal use (`she is recording is at this very moment') and \textit{yuwa} in example~\ref{exe:karayuwa} is spatial (`here are the fish I'm presenting you with').
%
%\begin{exe}
%	\ex
%	{\gll ma tok inier=at rekam=ta \textbf{yuwa}\\
%		\textsc{3sg} still \textsc{1du.ex=obj} record={\glta} \textsc{prox}\\	
%		`She is still recording us.' \jambox*{[conv12\_20:13]}
%	}
%	\label{exe:rekamda}
%	\ex
%	{\gll an sor=at et-eir-i ka ∅=ta \textbf{yuwa}\\
%		\textsc{1sg} fish=\textsc{obj} \textsc{clf}-two-\textsc{objqnt} \textsc{2sg} give={\glta} \textsc{prox}\\	
%		`I give you two fish.' \jambox*{[narr42\_34:46]}
%	}
%	\label{exe:karayuwa}
%\end{exe}
%
%Cf. PM `dia ada rekam ini'. Kluge analyses adnominal demonstratives in Papuan Malay with `psychological' uses: emotional involvement, vividness, contrast. Maybe do something with that?

\subsubsection{Manner, quality, quantity and degree}
\label{sec:mqqdwa}
The proximal demonstrative \textit{wandi} expresses manner or quality, and usually modifies a verb. The verb can be left out when the speaker enacts the action they are referring to. The speaker in~(\ref{exe:wandirami}) explains to the addressee how to pull bait. First, she uses \textit{wandi} `like this' without a verb, while enacting the movement and encouraging the addressee to look at her, and then repeats \textit{wandi} followed by the verb \textit{rami} `to pull'. Proximal \textit{wandi} is typically used when the speaker is simultaneously imitating a movement or a situation with gestures.

\begin{exe}
	\ex
	\label{exe:wandirami}
	{\gll sor=at pi \textbf{wandi} eh pi \textbf{wandi} rami∼rami\\
		fish=\textsc{obj} \textsc{1pl.incl} like\_this \textsc{int} \textsc{1pl.incl} like\_this pull∼\textsc{prog}\\
		\glt `The fish we do like this, eh, we pull like this.' \jambox*{\href{http://hdl.handle.net/10050/00-0000-0000-0004-1C75-D}{[stim15\_1:20]}}
	}
\end{exe}

Proximal \textit{wandi} can also be used to refer to states, such as `being friends' in (\ref{exe:friends}) or a colour (in~\ref{exe:black}, the speaker points at the black microphone stand). Both examples use the long form \textit{yuwandi}.

\begin{exe}
	\ex \gll taman-un kodaet me ka=bon an=bon \textbf{yuwandi}\\
	friend-\textsc{3poss} one\_more {\glme} \textsc{2sg=com} \textsc{1sg=com} like\_this\\
	\glt `He has a friend like you and me.' \jambox*{\href{http://hdl.handle.net/10050/00-0000-0000-0004-1BB0-D}{[stim12\_1:12]}}
	\label{exe:friends}
	\ex	\gll se bo kuskap=ten \textbf{yuwandi}\\
	{\glse} go black=\textsc{ten} like\_this\\
	\glt `[The nutmegs] turn black like this.' \jambox*{\href{http://hdl.handle.net/10050/00-0000-0000-0004-1BF1-6}{[narr12\_8:14]}}
	\label{exe:black}
\end{exe}

Lastly, proximal \textit{wandi} can be used to introduce quoted speech (as in \ref{exe:kadokca}, see also §\ref{sec:speech}) or as a stand-in for quoted speech, as in (\ref{exe:wandiwandi}).

\begin{exe}
	\ex \gll an se \textbf{wandi} eh ema kadok-ca=at=a tama\\
	\textsc{1sg} {\glse} like\_this \textsc{int} aunt cloth-\textsc{2sg.poss=obj=foc} where\\
	\glt `I went like: ``Hey aunt, where is your cloth?''' \jambox*{\href{http://hdl.handle.net/10050/00-0000-0000-0004-1BBB-2}{[narr40\_4:56]}}
	\label{exe:kadokca}
	\ex \gll ma toni \textbf{wandi} \textbf{wandi}\\
	\textsc{3sg} say like\_this like\_this\\
	\glt `He said such-and-such.' \jambox*{\href{http://hdl.handle.net/10050/00-0000-0000-0004-1BAA-C}{[stim7\_16:50]}}
	\label{exe:wandiwandi}
\end{exe}

%similative wanggap ungrammatical

In addition to the manner and quality demonstrative \textit{wandi}, Kalamang has tree other proximal forms: one for quantity (\textit{wa-bes}, adnominal) and two for degree (\textit{wa-rip} for size and \textit{wa-sen} for length of time, both adverbial). The three forms are illustrated below.

\begin{exe}
	\ex \gll ka se bo yuol \textbf{wa-bes}\\
	\textsc{2sg} {\glse} go day \textsc{prox-qnt}\\
	\glt `You went (away) this many days.' \jambox*{\href{http://hdl.handle.net/10050/00-0000-0000-0004-1C98-7}{[narr26\_11:22]}}
	\ex \gll buwar opa temun-un \textbf{wa-rip}\\
	kind\_of\_fruit {\glopa} big-\textsc{nmlz} \textsc{prox-dgr}\\
	\glt `That \textit{buwar} was this big.' \jambox*{\href{http://hdl.handle.net/10050/00-0000-0000-0004-1BBC-4}{[narr24\_4:38]}}
	\ex \gll goras tok maruat=nin tik \textbf{wa-sen}=ta\\
	crow yet move\_seawards=\textsc{neg} be\_long \textsc{prox-dgr}={\glta}\\
	\glt `The crow didn't come back for this long.' \jambox*{\href{http://hdl.handle.net/10050/00-0000-0000-0004-1B91-5}{[narr39\_6:35]}}
\end{exe}


\subsection{Distal \textit{me} `\textsc{dist}'}\is{demonstrative!distal}
\label{sec:me}
\subsubsection{Spatial use}\is{demonstrative!spatial}
Distal demonstrative \textit{me} occurs adnominally, pronominally and identificationally, prototypically to indicate referents that are relatively far from the speaker. There are no adnominal examples in the naturalistic corpus that are clearly spatial, so an elicited example is given in~(\ref{exe:demadme}). It was elicited for a situation where the speaker points at one of the addressee's teeth (scene 2 from \citealt{wilkins1999}).

\ea  \label{exe:demadme}
\gll gier-ca \textbf{me} me ten\\
tooth-\textsc{2sg.poss} \textsc{dist} {\glme} bad\\	
\glt	`That tooth of yours is bad.' \hfill [elic]
\z 

In object position, the distal demonstrative has the form \textit{met}, as is illustrated for pronominal use in~(\ref{exe:demprome}). The distal demonstrative refers to betel nuts, which the referent of \textit{ma} `she' went to look for in another house.

\ea \label{exe:demprome}
	\gll ma \textbf{met}=a sanggara\\
	\textsc{3sg} \textsc{dist.obj=foc} search\\	
	\glt `She searches for that.' \jambox*{\href{http://hdl.handle.net/10050/00-0000-0000-0004-1BBD-5}{[conv12\_20:42]}}
\z

Identificational examples are given in~(\ref{exe:demidme}) and~(\ref{exe:demidmeme}). Like for proximal forms, the distal demonstrative follows a \is{focus}focused noun and precedes a \is{topic}topic marker.

\ea \label{exe:demidme}
\gll naharen-un=a \textbf{me}\\
leftover-\textsc{3poss=foc} \textsc{dist}\\	
\glt	`That is the leftover.' \jambox*{\href{http://hdl.handle.net/10050/00-0000-0000-0004-1BA2-F}{[conv11\_4:25]}}
\z 

\ea \label{exe:demidmeme}
\gll et me \textbf{me}\\
canoe {\glme} \textsc{dist}\\	
\glt	`That is the canoe.' \jambox*{\href{http://hdl.handle.net/10050/00-0000-0000-0004-1BA3-3}{[conv10\_13:37]}}
\z 

%nothing special to say about yume(ne), so need not discuss.

Distal \textit{metko} `there' is used for indicating the \is{location}location of referents away from the speaker. In~(\ref{exe:maulma}), the distal location is first specified (the sea) and then referred to with \textit{metko}. (\ref{exe:kasawari}) is from a story about a dog and a cassowary which chased each other into the sea and became rock formations. 

\begin{exe}
	\ex \gll se kewe=ka kuru di=maruat=kin=ta to me karena {\ob}...{\cb} mu maulma ran \textbf{metko}\\
	{\glse} house=\textsc{lat} bring \textsc{caus}=move\_seawards=\textsc{vol}={\glta} right \textsc{dist} because {} \textsc{3pl} bend move \textsc{dist.loc}\\
	\glt `They wanted to bring [the corpse] from the house to the sea, right, because [...] they go there and bend [it straight].' \jambox*{\href{http://hdl.handle.net/10050/00-0000-0000-0004-1BC3-B}{[conv7\_7:39]}}
	\label{exe:maulma}
\end{exe}	

\begin{exe}
	\ex
	\begin{xlist}
		\exi{A:}
		{\gll bal=nan ma dalang=i pasier=ko yie marua\\
			dog=too \textsc{3sg} jump={\gli} sea=\textsc{loc} swim move\_seawards\\
			\glt `The dog also jumped in the sea and swam away from the shore.'}	
		\exi{B:} {\gll kasawari se \textbf{metko} telin\\
			cassowary {\glse} \textsc{dist.loc} stay\\
			\glt `The cassowary stayed there.' \jambox*{\href{http://hdl.handle.net/10050/00-0000-0000-0004-1B62-6}{[narr20\_2:32]}}}
	\end{xlist}
	\label{exe:kasawari}
\end{exe}

\textit{Metko}, like other NPs carrying a locative postposition, can be used predicatively, essentially meaning `to be there'. \textit{Mengga}, like other NPs carrying the lative postposition, must always be used in combination with other verbs. An example is given in~(\ref{exe:mengga1}). Complex source, goal and location constructions are described in §\ref{sec:mvcgoal}. 

\begin{exe}
	\ex
	{\gll terus \textbf{mengga} koi Ibrahim tanbes=ko \textbf{mengga} koi Arepnengga bara\\
		further \textsc{dist.lat} then Ibrahim right=\textsc{loc} from\_there then Arepner.\textsc{lat} descend\\	
		\glt `Further from there there's Ibrahim on the right, from there down to Arepner.' \jambox*{\href{http://hdl.handle.net/10050/00-0000-0000-0004-1BE6-4}{[stim36\_1:41]}}
	}
	\label{exe:mengga1}	
\end{exe}

\textit{Ime} is occasionally used as short for \textit{(i)metko}.

\begin{exe}
	\ex \gll kanas ep-kon=a marua \textbf{ime}\\
	kind\_of\_fish \textsc{clf\_group}-one=\textsc{foc} move\_seawards \textsc{dist}\\
	\glt `A school of \textit{kanas} swims towards sea (there?).' \jambox*{\href{http://hdl.handle.net/10050/00-0000-0000-0004-1BC9-2}{[conv5\_0:29]}}
	\label{exe:ime}
\end{exe}

%watko/metko direct contrast, owatko different. no examples to illustrate

\subsubsection{Temporal use}\is{demonstrative!temporal}
Temporal use of demonstratives is largely restricted to the distal form modifying the noun \textit{yuol} `day', illustrated in~(\ref{exe:yuolmeme}) and~(\ref{exe:yuolme}).

\begin{exe}
	\ex \gll yuol \textbf{me} me ma masin=at istar\\
	day \textsc{dist} {\glme} \textsc{3sg} machine=\textsc{obj} start\\
	\glt `That day he started the machine.' \jambox*{\href{http://hdl.handle.net/10050/00-0000-0000-0004-1BB3-0}{[narr7\_7:03]}}
	\label{exe:yuolmeme}
	\ex
	{\gll ma se mu=bon taruon ma kasian yuol \textbf{me} ma se taruo\\
	\textsc{3sg} {\glse} \textsc{3pl=com} say \textsc{3sg} poor day \textsc{dist} \textsc{3sg} {\glse} say\\
	\glt `She already told them, poor her, that day she already told.' \jambox*{\href{http://hdl.handle.net/10050/00-0000-0000-0004-1BBD-5}{[conv12\_21:36]}}}
	\label{exe:yuolme}
\end{exe}

\subsubsection{Anaphoric use}\is{demonstrative!anaphoric}
Clear endophoric usage of the distal demonstrative as in~(\ref{exe:canamme}), where \textit{me} refers back to a referent introduced earlier (anaphora), is rather rare. 

\begin{exe}
	\ex
	{\glll Ma canamat koni koluk. Canam \textbf{me}, pusirunbon.\\
	ma canam=at kon-i koluk canam me pusir-un=bon\\
	\textsc{3sg} man=\textsc{obj} one-\textsc{objqnt} meet man \textsc{dist} bow-\textsc{3poss=com}\\
	\glt `She meets a man. That man has an arrow.' \jambox*{\href{http://hdl.handle.net/10050/00-0000-0000-0004-1C9C-7}{[stim24\_1:23]}}}
	\label{exe:canamme}
\end{exe}

The demonstrative that is most commonly used for anaphoric reference is \textit{opa}, see §\ref{sec:demopa}.

\subsubsection{Manner, quality, quantity and degree}
\label{sec:distmann}
The distal form \textit{mindi} `like that' (occasionally pronounced \textit{mendi}, cf. the distal basic form \textit{me}) expresses manner or quality. In~(\ref{exe:mindijaga}), the speaker tries to explain how they waved away the smoke of fires with leaves to keep their hiding place secret during the Japanese bombings in WWII. 

\begin{exe}
	\ex
	\label{exe:mindijaga}
	{\gll in se lolok=at kowaran \textbf{mindi} din=at jaga\\
	\textsc{1pl.excl} {\glse} leaf=\textsc{obj} bend like\_that fire=\textsc{obj} watch\\
	\glt `We bent leaves, like that we watched the fire.' \jambox*{\href{http://hdl.handle.net/10050/00-0000-0000-0004-1BBB-2}{[narr40\_8:04]}}
	}
\end{exe}

Distal \textit{mindi} is also used as `until' in combination with \textit{bo} `to go' (lit. `go like that', see also §\ref{sec:svcmanneri}). (\ref{exe:karuar}) is about the production of pandanus leaf strips for weaving.

\begin{exe}
	\ex \gll karuar=i \textbf{mindi} bo kararak koi masan\\
	smoke\_dry={\gli} like\_that go dry then dry\_in\_sun\\
	\glt `We dry [on a rack above the fire] until it's dry, then we dry in the sun.' \jambox*{\href{http://hdl.handle.net/10050/00-0000-0000-0004-1BB8-C}{[narr11\_2:50]}}
	\label{exe:karuar}
\end{exe}

The form \textit{mendak} `just like that', which seems derived from distal \textit{me} and \textit{=tak} `just', is also used to express manner. Consider~(\ref{exe:mendak}). There is no corresponding proximal form (see also §\ref{sec:mvcuntil}).

\begin{exe}
	\ex \gll pi \textbf{mendak} kuar langsung=et eba bes\\
	\textsc{1pl.excl} just\_like\_that cook directly={\glet} then good\\
	\glt `If we just cook it directly like that, it's good.' \jambox*{\href{http://hdl.handle.net/10050/00-0000-0000-0004-1BA6-6}{[conv13\_3:41]}}
	\label{exe:mendak}
\end{exe}	

In addition to the manner and quality demonstrative \textit{mindi}, Kalamang has three other proximal forms: one adnominal demonstrative for quantity (\textit{mia-bes}) and two adverbial for degree (\textit{mia-rip} for size and \textit{mia-sen} for distance and duration). The variants \textit{me-bes}, \textit{me-rip} and \textit{me-sen} are also acceptable, but hardly found in the corpus. The three forms are illustrated below. 

\begin{exe}
	\ex \gll eba ka=nan pitis \textbf{mia-bes}=at maraouk=te\\
	then \textsc{2sg}=too money \textsc{dist-qnt=obj} store={\glte}\\
	\glt `Why did you store that much money there?!' \jambox*{\href{http://hdl.handle.net/10050/00-0000-0000-0004-1BBD-5}{[conv12\_2:05]}}
	\ex \gll tumun se bo temun \textbf{mia-rip}\\
	child {\glse} go big \textsc{dist-dgr}\\
	\glt `The child has become that big.' \jambox*{\href{http://hdl.handle.net/10050/00-0000-0000-0004-1BB0-D}{[stim12\_5:05]}}
	\ex \gll an ewa=i sampi \textbf{mia-sen}-tak\\
	\textsc{1sg} speak={\gli} until \textsc{dist-qnt}-just\\
	\glt `I just speak that long.' \jambox*{\href{http://hdl.handle.net/10050/00-0000-0000-0004-1BC0-1}{[narr22\_8:39]}}
\end{exe}

\subsubsection{Discourse}\is{demonstrative!in discourse}
\label{sec:discdems}
\label{sec:medisc}
Several distal demonstrative forms help in the organisation of discourse: as a sequential marker, to indicate the start of a new scene, or to end a section of discourse. Topic marker \textit{me}, probably related to the distal \textit{me}, is discussed in §\ref{sec:discme}.

The distal locative \textit{metko} is used as a sequential marker in \is{conditional}conditional clauses, often in combination with \textit{eba} `then', which can be used on its own to express sequentiality. Adding distal \textit{metko} to \textit{eba} `then' focuses on the ending of the first state or event, before the next can be started. This is illustrated in the following two examples, where certain conditions must be met (the tide must be good, Friday must have passed) before the next event can take place.

\begin{exe}
	\ex \gll warkin tok bes=et \textbf{eba} \textbf{metko} pi war=et\\
	tide first good={\glet} then \textsc{dist.loc} \textsc{1pl.incl} fish={\glet}\\
	\glt `When (lit. first) the tide is good, we go fishing.' \jambox*{\href{http://hdl.handle.net/10050/00-0000-0000-0004-1B9F-F}{[conv9\_2:04]}}
	\label{exe:warkon}
	\ex \gll ka-mun tok bo=in ariemun nasal=et \textbf{eba} \textbf{metko} bo=te\\
	\textsc{2sg-proh} yet go=\textsc{proh} friday open={\glet} then \textsc{dist.loc} go=\textsc{imp}\\
	\glt `Don't you go yet, after Friday has passed, go!' \jambox*{\href{http://hdl.handle.net/10050/00-0000-0000-0004-1BC3-B}{[conv7\_2:39]}}
	\label{exe:nasal}
\end{exe}

Distal manner demonstrative \textit{mindi} is used to indicate a new scene in a story. (\ref{exe:mindimu}) is uttered after an intermezzo in Papuan Malay. The story is taken up again starting with \textit{mindi}. In (\ref{exe:matyie}), \textit{mindi} marks the transition between two scenes: that of the speaker going off for a swim, and that of his friend calling him. \textit{Mindi} also indicates that some time has passed between the speaker going off for a swim, and his friend calling him.

\begin{exe}
	\ex \glll {\ob}...{\cb} Mindi mu he mara.\\
	{} \textbf{mindi} mu se mara\\
	{} like\_that \textsc{3pl} {\glse} move\_landwards\\
	\glt `And so they moved towards land.' \jambox*{\href{http://hdl.handle.net/10050/00-0000-0000-0004-1BAD-2}{[narr29\_9:45]}}
	\label{exe:mindimu}
	\ex \glll An se mat jie mamuni kahen. Mindi ma anat gonggung {\ob}...{\cb}.\\
	an se mat yie mamun=i kahen \textbf{mindi} ma an=at gonggung {\ob}...{\cb}\\
	\textsc{1sg} {\glse} \textsc{3sg.obj} swim leave={\gli} far like\_that \textsc{3sg} \textsc{1sg=obj} call {\ob}...{\cb}\\
	\glt `I went swimming, leaving him far behind. Then, he called me.' \jambox*{\href{http://hdl.handle.net/10050/00-0000-0000-0004-1BAE-4}{[narr44\_21:46]}}
	\label{exe:matyie}
\end{exe}

%Diessel p74: manner demonstratives are often used as discourse deictics

The expression \textit{ma he me} `that's it', containing distal \textit{me}, is used to indicate the end of a paragraph, usually one that has a summary of different things or actions, such as the list of ingredients in~(\ref{exe:gelompang}). It can also close off an entire story, in combination with \textit{se koyet} `finished' (as in~\ref{exe:theend}, see also §\ref{sec:closing}).

\begin{exe}
	\ex \gll gelompang eba wat santang=bon terus kokok-nar eba nasuena \textbf{ma} \textbf{se} \textbf{me}\\
	batter then coconut coconut\_milk=\textsc{com} then chicken-egg then sugar \textsc{3sg} {\glse} \textsc{dist}\\
	\glt `The batter. With coconut milk, eggs, sugar, that's it.' \jambox*{\href{http://hdl.handle.net/10050/00-0000-0000-0004-1B9C-A}{[narr9\_0:24]}}
	\label{exe:gelompang}
	\ex \gll \textbf{ma} \textbf{se} \textbf{me} se koyet\\
	\textsc{3sg} {\glse} \textsc{dist} {\glse} finish\\
	\glt `That's it, finished.' \jambox*{\href{http://hdl.handle.net/10050/00-0000-0000-0004-1BC1-0}{[narr19\_16:52]}}
	\label{exe:theend}
\end{exe}

\textit{Ma he me} can also mean `that's enough', as in~(\ref{exe:eih}), where a monkey wants to be released from his cage.

\begin{exe}
	\ex \gll  eih \textbf{ma} \textbf{se} \textbf{me} \textbf{ma} \textbf{se} \textbf{me} an=at kahetmei\\
	hey \textsc{3sg} {\glse} \textsc{dist} \textsc{3sg} {\glse} \textsc{dist} \textsc{1sg=obj} open.\textsc{imp}\\
	\glt `Hey, that's enough, that's enough, release me!' \jambox*{\href{http://hdl.handle.net/10050/00-0000-0000-0004-1BC1-0}{[narr19\_14:58]}}
	\label{exe:eih}
\end{exe}

Finally \textit{mera}, possibly derived from distal demonstrative or topic marker \textit{me} and non-final \textit{=ta} (homonymous with the focused object form of the distal demonstrative), is used as a conjunction for either sequential events as in~(\ref{exe:davite}) or for reason and consequence as in~(\ref{exe:merame}). 

\begin{exe}
	\ex \glll Davit esun tok Pakpao, ah mara nawanggaret. \textbf{Mera} Bilal esun toni oh Nostal Arepneko.\\
	Davit esun tok Pakpak=ko ah ma=at=a nawanggar=et mera Bilal esun toni oh Nostal Arep-neko\\
	Davit father.\textsc{3poss} still Fakfak=\textsc{loc} \textsc{int} \textsc{3sg=obj=foc} wait={\glet} then Bilal father.\textsc{3poss} say \textsc{int} Nostal Arep-inside\\
	\glt `Davit's father is still in Fakfak, we'll wait for him [to do the job]. Then Bilal's father said ``Oh, Nostal in Arep!''' \jambox*{\href{http://hdl.handle.net/10050/00-0000-0000-0004-1BB3-0}{[narr7\_9:58]}}
	\label{exe:davite}
	\ex \glll Ma lalaren. \textbf{Mera} ma he ecua.\\
	ma lalat=ten mera ma se ecua\\
	\textsc{3sg} dead={\glten} so \textsc{3sg} {\glse} cry\\
	\glt `She died. So he cried.' \jambox*{\href{http://hdl.handle.net/10050/00-0000-0000-0004-1BBC-4}{[narr24\_3:28]}}
	\label{exe:merame}
\end{exe} 	
%e.g. ma toni e an bo warkin mera aryani (emun) toni kasur ki mindi jadi boet to 

\subsection{Far distal \textit{owa} `\textsc{fdist}'}\is{demonstrative!far distal}
\label{sec:owa}
\subsubsection{Spatial}\is{demonstrative!spatial}
Far distal \textit{owa} is prototypically used for referents that are relatively very far away from the speaker and the listener. It is mostly used when referring to places across landscape, e.g. the next beach, behind the mountain, another city, the other side of the island, the other side of the country. This results in \textit{owa} typically, but not necessarily, being used for invisible referents. In the recordings from a round trip around the biggest Karas island, speakers tend to use \textit{owa} for singling out landscape features that are not only somewhat distant, but also have another landscape feature in between. In~(\ref{exe:owane}), the speaker points at Yar Poskon, a cape 100 metres away, while sailing past another cape which is followed by a beach and Yar Poskon.

\begin{exe}
	\ex \gll Yar Poskon=a owane\\
	Yar Poskon=\textsc{foc} \textsc{fdist}\\
	\glt `Yar Poskon is over there!' \jambox*{\href{http://hdl.handle.net/10050/00-0000-0000-0004-1BA1-C}{[conv22\_0:41]}}
	\label{exe:owane}
\end{exe}

In about half of the corpus instances (23 out of 47), \textit{owa} carries a lative or locative postposition. This is illustrated in~(\ref{exe:ruomun}) to~(\ref{exe:belada}). The principles are the same: \textit{owatko} and \textit{owangga} are used to refer across landscape. The referent may be close, as in~(\ref{exe:kol}), where the location referred to is right outside the house, or in another country, as in~(\ref{exe:belada}).

\begin{exe}
	\ex \gll ra Pebis Ruomun \textbf{owangga} in=at nawaruok\\
	go Pebis Ruomun \textsc{fdist.lat} \textsc{1pl.excl=obj} unload\\
	\glt `[You want to] go to Pebis Ruomun over there and drop us off?' \jambox*{\href{http://hdl.handle.net/10050/00-0000-0000-0004-1B6D-C}{[conv28\_3:14]}}
	\label{exe:ruomun}
	\ex \gll bo kol \textbf{owatko} war=te\\
	go outside over\_there fish=\textsc{imp}\\
	\glt `Go fish outside over there!' \jambox*{\href{http://hdl.handle.net/10050/00-0000-0000-0004-1BA3-3}{[conv10\_22:31]}}
	\label{exe:kol}
	\ex \gll Beladar-leng \textbf{owatko}\\
	Netherlands-village over\_there\\
	\glt `In the Dutch village over there.' \jambox*{\href{http://hdl.handle.net/10050/00-0000-0000-0004-1BBD-5}{[conv12\_5:01]}}
	\label{exe:belada}
\end{exe}

One corpus example of \textit{owa} (in its variant \textit{owane}) is used on a much smaller scale: a table top in a picture-matching task\is{picture-matching task}. During this task, the director could see the matcher's pictures, and directed him to the correct picture by explaining the position of the card with the picture on the tabletop. The director utters~(\ref{exe:elakdok}). \textit{Owane} is used to indicate that the picture is at the far extreme of the tabletop, far away from the speaker (and the addressee) as compared to the other pictures.

\begin{exe}
	\ex \gll elak-kadok tua elak-kadok siun-kadok \textbf{owane}\\
	bottom-side old\_man bottom-side edge-side \textsc{fdist}\\
	\glt `Down there, Tua, down there, at the edge over there.' \jambox*{\href{http://hdl.handle.net/10050/00-0000-0000-0004-1C97-F}{[stim27\_10:53]}}
	\label{exe:elakdok}
\end{exe}

The video still in Figure~\ref{fig:elakdok} shows the moment the director (on the left) utters \textit{elak-kadok} for the second time. The matcher is still looking for the right picture, hovering his finger. The picture that the director is referring to is marked in the figure with an arrow in the figure.

\begin{figure}[ht]  
	\centering
	\includegraphics[width=0.8\textwidth]{Images/elakdok}
	\caption[Far distal \textit{owa}]{Directing to \textit{owane} \textsc{fdist}}
	\label{fig:elakdok}	
\end{figure}

\subsubsection{Other}
Far distal locative \textit{owatko} `over there', while usually used for invisible \is{location}locations, can also be used when the location is far away from both the speaker and the addressee, and when the speaker wants to create a mental distance to the referent. (\ref{exe:owatkin}) comes from a conversation between two sisters who are fishing with their much younger sister-in-law. They are not satisfied with her skills, and tease her. In the utterance, the speaker and the addressee stand next to each other in the sea, and refer to the sister-in-law who is standing fifty metres away, but is clearly visible.

\begin{exe}
	\ex \gll mena ma se koi \textbf{owatko} kinkin=taet reon\\
	otherwise \textsc{3sg} {\glse} again over\_there hold=again maybe\\
	\glt `Otherwise she will maybe hold [the fishing net] again over there.' \jambox*{\href{http://hdl.handle.net/10050/00-0000-0000-0004-1BCE-D}{[conv4\_2:06]}}
	\label{exe:owatkin}
\end{exe}

%(\textit{There are a few cases of owandi in the corpus, but it isn't clear what their function is. It doesn't seem to be manner.})


\subsection{Anaphoric \textit{opa} `\textsc{ana}'}\is{demonstrative!anaphoric}
\label{sec:demopa}
\textit{Opa} is an adnominal demonstrative. It occurs with referents that represent shared knowledge, typically because they have been previously mentioned in the \is{discourse}discourse. It has mainly tracking and recognitional uses \parencite{himmelmann1996}, and is therefore glossed as \textsc{ana} for anaphoric. A typical tracking example is~(\ref{exe:semen}), where the referent \textit{semen} `concrete', which is mentioned at minute 2:16, is mentioned again at minute 5:27, and marked with \textit{opa} to indicate that it is the same concrete.

\begin{exe}
	\ex 
	\begin{xlist}		
		\ex	\gll mu se semen=at cetak\\
		\textsc{3pl} {\glse} concrete=\textsc{obj} mould\\	
		\glt	`They already mould the concrete.' \jambox*{\href{http://hdl.handle.net/10050/00-0000-0000-0004-1BB3-0}{[narr7\_2:16]}}
		\ex \gll mu se semen \textbf{opa} koyal=te di=ran\\
		\textsc{3pl} {\glse} concrete \textsc{ana} mix={\glte} \textsc{caus}=move\\	
		\glt	`They already mixed that concrete and put it up.' \jambox*{\href{http://hdl.handle.net/10050/00-0000-0000-0004-1BB3-0}{[narr7\_5:27]}}
	\end{xlist}
	\label{exe:semen}
\end{exe}

When narrating a story with help of a stimulus\is{stimulus}, such as a picture book or a video, speakers may start the story by marking the first mention of a referent with \textit{opa}, referring to the picture or video of the referent that they have just seen. (\ref{exe:kewerecg}) is an example of recognitional use.

\begin{exe}
	\ex \gll tumun \textbf{opa} ma kewe-neko\\
	child \textsc{ana} \textsc{3sg} house-inside\\
	\glt `That child is inside a house.' \jambox*{\href{http://hdl.handle.net/10050/00-0000-0000-0004-1BAF-5}{[stim20\_0:06]}}
	\label{exe:kewerecg}
\end{exe}

The demonstrative may also be used when there is no anaphor, but when the referent is just a part of the shared knowledge of two speakers. For example, a speaker may refer to her daughter Desi (a name carried only by her in the village), who is fishing just outside the window, with help of \textit{opa}, even though Desi has not been mentioned in the conversation yet.

\ea
\label{exe:desii}
\gll Desi \textbf{opa} me yal∼yal=te yawe\\
Desi {\glopa} {\glme} paddle∼\textsc{prog}={\glte} \textsc{down}\\
\glt 	`Desi is paddling down there.' \jambox*{\href{http://hdl.handle.net/10050/00-0000-0000-0004-1BA2-F}{[conv11\_6:36]}}
\z 

It can also be used to establish shared knowledge. In~(\ref{exe:hadi}), the speaker uses \textit{opa} to indicate to the listener that the referent is part of their shared knowledge.

\ea
\label{exe:hadi}
\gll inier \textbf{opa} {\ob}...{\cb} Hadi \textbf{opa} to\\
\textsc{2du.ex} {\glopa} [...] Hadi {\glopa} right \\
\glt 	`We two, with Hadi, right.' \jambox*{\href{http://hdl.handle.net/10050/00-0000-0000-0004-1BB4-6}{[narr14\_3:14]}}
\z 

More details about this demonstrative can be found in~\textcite{visser2020}.


\subsection{Elevational demonstratives \textit{yawe} `\textsc{down}' and \textit{osa} `\textsc{up}'}\is{demonstrative!elevational}\is{elevational|see{demonstrative}}
\label{sec:elevdem}
Kalamang has two elevational demonstratives: \textit{yawe} `\textsc{down}' and \textit{osa} `\textsc{up}'. They can be used adnominally, but are often used adverbially or predicatively, inflected with locative \textit{=ko} or lative \textit{=ka}. They index referents and locations. 

%Tested 2019: osa and yawe can be used adnominally as in kewe osa me kewe anggon, mu yawe amdirat paruo. can also be in object position: ana kewe osat paruo. mu amdir yawet paruotkin.
%pronominal use also fine, e.g. osa (me) anggon. forms osanggara and yawenggara also good (i.e. predicative-ish? prob. already treated.).

Elevationals are typically used to describe referents and \is{location}locations inside the village, such as the beach (down) and other places in the village (up).\footnote{Karas is spoken on a limestone island that rises out of the sea. Both villages, Mas and Antalisa, are built on a strip of beach and the adjacent slopes.} (\ref{exe:keweosa}) shows the adnominal object form \textit{osanet} \textsc{up.obj}, and~(\ref{exe:yawetko}) contains the locative form \textit{yawetko} \textsc{down.loc}. Though \textit{yawe} and \textit{osa} (and their longer forms) are typically used adnominally, they may also be used as locative and lative forms, as in~(\ref{exe:desi}), where one would expect the locative form \textit{yawetko}.

\begin{exe}
	\ex \gll mu era kewe \textbf{osanet} nawanona\\
	\textsc{3pl} ascend house \textsc{up.obj} tidy\\
	\glt `They went up to tidy the house up there.' \jambox*{\href{http://hdl.handle.net/10050/00-0000-0000-0004-1BC3-B}{[conv7\_8:06]}}
	\label{exe:keweosa}
	\ex \gll an toni eh ka bo \textbf{yawetko} war=te\\
	\textsc{1sg} say hey \textsc{2sg} go \textsc{down.loc} fish=\textsc{imp}\\
	\glt `I said: ``Hey, you go fishing down there!''' \jambox*{\href{http://hdl.handle.net/10050/00-0000-0000-0004-1BA3-3}{[conv10\_18:14]}}
	\label{exe:yawetko}
	\ex \gll Desi opa me yal∼yal=te \textbf{yawe}\\
	Desi {\glopa} {\glme} paddle∼\textsc{prog}={\glte} \textsc{down}\\
	\glt `Desi is paddling down there.' \jambox*{\href{http://hdl.handle.net/10050/00-0000-0000-0004-1BA2-F}{[conv11\_6:36]}}
	\label{exe:desi}
\end{exe}

Elevationals may be applied on both a smaller and bigger scale than the village. On a smaller scale, a speaker may use elevationals to refer to the immediate environment such as a house or a tree. In~(\ref{exe:tekya}), the speaker is looking for a knife in her house, and refers to her bedroom as being `down'. \textit{Yawe} is not used to single out which bedroom, as the speaker has only one bedroom (and has already singled it out anyway by inflecting \textit{minggalot} `bedroom' with a possessive marker).

\begin{exe}
	\ex \gll ma tamatko ka bo minggalot-an \textbf{yawe} kome=te \\
	\textsc{3sg} where \textsc{2sg} go bedroom-\textsc{1sg.poss} \textsc{down} look=\textsc{imp}\\
	\glt `Where is it? You go look in my bedroom!' \jambox*{\href{http://hdl.handle.net/10050/00-0000-0000-0004-1BB5-B}{[conv17\_3:42]}}
	\label{exe:tekya}
\end{exe}

On a larger scale, elevationals are used to talk about the wider landscape surrounding Mas, the village where all recordings for the Kalamang corpus were made. The direction of movement from the Karas Islands to Fakfak (the regency capital, NNE of the Karas Islands) is described as \textit{bara} `move down', and from Fakfak to the Karas Islands as \textit{sara} `move up'. Consequently, Fakfak is \textit{yawetko} `down there'. This is illustrated in~(\ref{exe:muyawe}), which is about moneylenders in Fakfak. 

\begin{exe}
	\ex \gll mu \textbf{yawetko} in=bon sampaikan=et\\
	\textsc{3pl} \textsc{down.loc} \textsc{1pl.excl=com} let.know={\glet}\\
	\glt `They down there let us know.' \jambox*{\href{http://hdl.handle.net/10050/00-0000-0000-0004-1B98-6}{[narr45\_2:54]}}
	\label{exe:muyawe}
\end{exe}

The direction of movement from the Karas Islands to Malakuli (the district capital, east of the Karas Islands) is described as landwards, because it lies on the mainland of New Guinea. Nevertheless, Malakuli is \textit{osatko}, as illustrated in~(\ref{exe:kiosa}), which refers to a house for schoolchildren from Mas which was being built in Malakuli at the time.

\begin{exe}
	\ex \gll ki \textbf{osatko}=a kewe-paruot=kin\\
	\textsc{2pl} \textsc{up.loc}=\textsc{foc} house-make=\textsc{vol}\\
	\glt `Do you up there want to work on the house?' \jambox*{\href{http://hdl.handle.net/10050/00-0000-0000-0004-1BA3-3}{[conv10\_19:21]}}
	\label{exe:kiosa}
\end{exe}

The same applies within the Karas Islands archipelago. Mas is on the biggest Karas island. To the west, between Mas and the mainland, lie the two smaller Karas Islands, with the villages Tarak, Tuburuasa, Kiaba and Faor. When marriage negotiations between a man from Kiaba and a woman from Mas were held, the Mas community used the fact that they had sent many people \textit{osatko} `up there' as an argument for the man to come and live in Mas.

\begin{exe}
	\ex 	\gll pi reidak bo \textbf{osatko}\\
	\textsc{1pl.incl} many go \textsc{up.loc}\\
	\glt `Many of us went up there.' \jambox*{\href{http://hdl.handle.net/10050/00-0000-0000-0004-1BCF-3}{[narr2\_10:00]}}
\end{exe} 
\begin{figure}[b]
	\includegraphics[width=.9\textwidth]{Images/Karas-Fakfak_directionals.pdf}
	\caption[Directional verbs]{Directional verbs from Mas to Fakfak, Kiaba and Malakuli and vice versa}\label{fig:dir}	
\end{figure}

The directional verbs and villages described here are illustrated in Figure~\ref{fig:dir}. More on directional verbs can be found in §\ref{sec:dir}.


All these places lie at the same elevation as Mas: they are all villages that are directly situated at the beach, slightly above sea level. If anything, Kiaba and Malakuli are lower than (parts of) Mas, because the land rises more steeply from the beach in Mas than in Kiaba and Malakuli. The choice of directional verbs is thus not guided by actual elevation. Other factors such as sea currents or the direction of (former) centres of power remain to be investigated. There are no examples in the corpus where the location of Mas (or another place on the biggest Karas island) is described as \textit{yawetko} or \textit{osatko} when contrasted with Kiaba, Malakuli or Fakfak. \textit{Yawetko} and \textit{osatko} seem reserved, on this bigger scale, to refer only to places outside the biggest Karas island.
\is{demonstrative|)}
