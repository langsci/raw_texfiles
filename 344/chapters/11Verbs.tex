\chapter{Verbs}
\label{ch:verbs}
This chapter describes verbs and verbal morphology. Verbs were defined in §\ref{sec:wcverb} as words that function as predicates\is{predicate} in the clause, may occur in complex predicates (Chapter~\ref{ch:svc}) and can be used as adnominal modifiers with attributive marker \textit{=ten}. This chapter starts by describing the two major verb classes in §\ref{sec:vclss}: regular and irregular verbs. §\ref{sec:vverbder} examines verb derivation, and §\ref{sec:verbred} verb reduplication. Valency-changing morphology such as reciprocal and causative proclitics are described in §\ref{sec:val}. §\ref{sec:verbdistr} treats plural number, which plays a minor role in verbs. Two fossilised morphemes that are found on Kalamang verbs are described in §\ref{sec:foss}. Verb-modifying morphology such as modal and aspect markers generally attach to the predicate, not to the verb, and are described in Chapter~\ref{ch:clausmod}.

\section{Verb classes}
\label{sec:vclss}
There are two major verb classes: regular and irregular verbs. Regular verbs take mood enclitics, negator \textit{=nin} and predicate linker \textit{=i} directly on the invariable root. Irregular verbs have a variable root in a vowel, \textit{-n} or \textit{-t}. The irregular verb class has two subclasses: transitive/intransitive pairs in \textit{-ma} and \textit{-cie}, and directional verbs. (\ref{exe:regnin}) and (\ref{exe:irregnin}) illustrate a regular and irregular verb, respectively. The a-examples show the uninflected verb, and the b-examples show the verb inflected with the negator \textit{=nin}.

\begin{exe}
	\ex
	\begin{xlist}
	\ex \gll mu muap\\
	\textsc{3pl} eat\\
	\glt `They ate.' \jambox*{\href{http://hdl.handle.net/10050/00-0000-0000-0004-1BAD-2}{[narr29\_3:31]}}
	\label{exe:regular}
	\ex \gll pi tok muap=nin\\
		\textsc{1pl.incl} yet eat=\textsc{neg}\\
		\glt `We didn't eat yet.' \jambox*{\href{http://hdl.handle.net/10050/00-0000-0000-0004-1BCA-4}{[conv20\_8:55]}}
	\end{xlist}
	\label{exe:regnin}	
    \ex
    \begin{xlist}
    \ex
	{\gll goras {\ob}...{\cb} sor=at jie\\
		crow {\ob}...{\cb} fish=\textsc{obj} get\\	
		\glt `The crow {\ob}...{\cb} got a fish.' \jambox*{\href{http://hdl.handle.net/10050/00-0000-0000-0004-1C9B-8}{[stim3\_0:05]}}
	}
	\ex
	{\gll mu tok bo jie\textbf{t}=nin\\
		\textsc{3pl} yet go buy=\textsc{neg}\\
		\glt `They didn't go buy yet.' \jambox*{\href{http://hdl.handle.net/10050/00-0000-0000-0004-1BA3-3}{[conv10\_6:18]}}
	}
	\end{xlist}
	    \label{exe:irregnin}
\end{exe}

\subsection{Regular verbs}
\label{sec:regv}
Regular verbs take mood enclitics, negator \textit{=nin} and predicate linker \textit{=i} directly on the root. This class contains verbs of all valencies, and with all kinds of final vowels on the root. Two generalisations can be made. First, static intransitive verbs such as \textit{muawese} `be hungry' and \textit{toari} `be young' are typically regular. Second, recent loan verbs from Malay or other Austronesian\is{Austronesian!loan} languages (see §\ref{sec:naloanverb} below), such as \textit{namusi} `to kiss' and \textit{rasa} `to like', are always regular. The behaviour of regular verbs under inflection is illustrated for \textit{taot} `to chisel', \textit{muap} `to eat' and \textit{ewa} `to speak' in Table~\ref{tab:regv}.

\begin{table}
	\caption{Behaviour of regular verbs under inflection}
	
		\begin{tabularx}{.8\textwidth}{X X X l}
			\lsptoprule
			 & \textit{taot} & \textit{muap} & \textit{ewa}\\
			 & `chisel' & `eat' & `speak'\\ \midrule
			\textit{=i} \textsc{plnk} & \textit{taot=i} & \textit{muap=i} & \textit{ewa=i}\\
			\textit{=et} \glet & \textit{taot=et} & \textit{muap=et} & \textit{ewa=et} \\
			\textit{=kin} \glkin  & \textit{taot=kin} & \textit{muap=kin}& \textit{ewa=kin} \\
			\textit{=nin} \textsc{neg} & \textit{taot=nin} & \textit{muap=nin} & \textit{ewa=nin}\\
			\textit{=in} \textsc{proh} & \textit{taot=in} & \textit{muap=in} & \textit{ewa=in}\\
			imperative & \textit{taot=te} & \textit{muap=te} & \textit{ewa=te}\\ 	
			\lspbottomrule
		\end{tabularx}
	
	\label{tab:regv}
\end{table}

\subsection{Irregular verbs}\is{verb!irregular}
\label{sec:ntverbs}
Irregular verbs, introduced in §\ref{sec:problems} and §\ref{sec:wcverb}, have a variable root in a vowel, \textit{-n} or \textit{-t}. This variation is apparent when the roots are inflected with mood enclitics, negator \textit{=nin} or predicate linker \textit{=i}, and from variation in the uninflected root. Two subgroups of this category can be defined by a combination of formal and semantic criteria: transitive/intransitive verb pairs in \textit{-ma} and \textit{-cie} (§\ref{sec:macie}) and directional verbs (§\ref{sec:dir}).

A sizeable minority of Kalamang verbs, 185 out of 650 in the corpus, are vowel-final. Of these 185, 71 are irregular. Depending on which enclitic the verb is combined with, the root ends in either of the consonants. Uninflected verbs of this class may be vowel-final or carry \mbox{\textit{-n}}, apparently without a difference in meaning (hence my description of them as ``uninflected''). It is unclear why some vowel-final verbs fall in this class, while others do not. In other words, membership of this verb class cannot be predicted.\footnote{Note that the phonemes \textit{-n} and \textit{-t} (\~{} \textit{-d} \~{} \textit{-r}) also occur on demonstratives and question words, and pose a morphophonological problem that is described in §\ref{sec:problems}.}

Within this class of verbs with a variable root ending are a few patterns. All verbs ending in \textit{-ma} transitive and \textit{-cie} intransitive (described in §\ref{sec:macie}) and all directional verbs (described in §\ref{sec:dir}) are irregular. In addition, most verbs ending \mbox{in \textit{-a}}, \textit{-ie} and \textit{-uo} are irregular.

Most irregular verbs behave as follows (as illustrated with \textit{paruo} `do' in Table~\ref{tab:ntshort}). When inflected with irrealis \textit{=kin}, negator \textit{=nin}, \is{prohibitive}prohibitive \textit{=in} or irrealis marker \textit{=et}, the root ends in \textit{-t}. When inflected with predicate linker \textit{=i}, \textit{=taet} `more; again' or attributive \mbox{\textit{=ten}}, the root ends in \textit{-n}.\footnote{Other predicate enclitics, such as \is{non-final}non-final \textit{=te} and \textit{=ta} or progressive \textit{=teba}, do not trigger the insertion of \textit{-t} or \textit{-n}.} An uninflected verb may be vowel-final or carry \textit{-n}. \is{imperative}Imperative forms are vowel-final, and if the root ends in a diphthong, the last vowel is cut off. Depalatalisation of /c/ to /t/ may also occur. Thus, \textit{taruo} `to say' has imperative form \textit{taru}, and \textit{gocie} `to stay' has imperative form \textit{goti}. 

\begin{table}[b]
	\caption{Behaviour of irregular verbs}
	\label{tab:ntshort}

		\begin{tabularx}{\textwidth}{lXXX l}
			\lsptoprule
			 & `do' &  `consume' &  `go' & `see'\\ \midrule
			uninflected & \textit{paruo} & \textit{na} & \textendash & \textit{kome}\\
			``uninflected''& \textit{paruo\textcolor{lsMidDarkBlue}{n}} & \textit{na\textcolor{lsMidDarkBlue}{n}} & \textit{bo\textcolor{lsLightWine}{t}} & \textit{kome\textcolor{lsLightWine}{t}}\\
			\textit{=i} {\gli}& \textit{paruo\textcolor{lsMidDarkBlue}{n}=i} & \textit{na\textcolor{lsMidDarkBlue}{n}=i} & \textit{bo=i} & \textit{kome=i}\\
			\textit{=et} {\glet} & \textit{paruo\textcolor{lsLightWine}{t}=et} & \textit{na\textcolor{lsMidDarkBlue}{n}=et} & \textit{bo=et} & \textit{kome\textcolor{lsLightWine}{t}=et}\\
			\textit{=kin} \textsc{vol} & \textit{paruo\textcolor{lsLightWine}{t}=kin} & \textit{na\textcolor{lsLightWine}{t}=kin} & \textit{bo\textcolor{lsLightWine}{t}=kin} & \textit{kome\textcolor{lsLightWine}{t}=kin}\\
			\textit{=nin} \textsc{neg} & \textit{paruo\textcolor{lsLightWine}{t}=nin} & \textit{na\textcolor{lsLightWine}{t}=nin} & \textit{bo\textcolor{lsLightWine}{t}=nin} & \textit{kome\textcolor{lsLightWine}{t}=nin}\\
			\textit{=in} \textsc{proh}& \textit{paruo\textcolor{lsLightWine}{t}=in!} & \textit{na=in!} & \textit{bo=in!} & \textit{kome\textcolor{lsLightWine}{t}=in!}\\
			imperative & \textit{paru} & \textit{na} & \textit{bo=te} & \textit{kome=te}\\
			\lspbottomrule
		\end{tabularx}

\end{table}

Not all irregular verbs behave exactly the same. Table~\ref{tab:ntshort} gives the four attested patterns. Most irregular verbs behave like \textit{paruo} `to do'. Three verbs within the irregular class, all highly frequent, show deviant behaviour. \textit{Na} `consume' ends in \textit{-n} instead of \textit{-t} when irrealis \textit{=et} is attached. \textit{Bo} `go' never ends in \textit{-n}. When this verb is uninflected, it must carry \textit{-t} and when inflected with predicate linker \textit{=i} or irrealis \textit{=et}, neither \textit{-n} nor \textit{-t} is allowed. \textit{Bo} carries the imperative enclitic \textit{=te} like regular verbs. \textit{Kome} `to see; to look' can be either \textit{kome} or \textit{komet} when uninflected, but not \textit{komen}. When carrying \textit{=i}, neither \textit{-n} nor \textit{-t} is allowed. Its imperative form is with \textit{=te}. For more examples, see Table~\ref{tab:ntverbs}.

% no effect: -ero, -teba, -te, -ta, -sawe, -tar (imp), -p (distr)


There is no clear rule as to when a specific verb occurs with or without the final \textit{-n} when it is otherwise uninflected. The number of occurrences in the corpus for three frequent irregular verbs with and without final \textit{-n} are given in Table~\ref{tab:n}.

\begin{table}
	\caption{Frequencies for common irregular verbs with and without \textit{-n}}
	\label{tab:n}
	
		\begin{tabularx}{\textwidth}{X X l}
			\lsptoprule
			 & without \textit{-n} & with \textit{-n} \\\midrule 
			 \textit{mia} `come' & 129 & 40\\
			\textit{potma} `cut'  & 41 & 14\\
			\textit{yecie} `return' & 100 & 12\\
		\lspbottomrule
	\end{tabularx}

\end{table}

Although there is a higher frequency of all verbs without \textit{-n}, there is no environment where either of the two is inappropriate. Forms with and without \textit{-n} are found with all persons, in simplex and complex predicates, following \is{iamitive}iamitive \textit{se}, \textit{koi} `again', lative \textit{=ka} and comitative \textit{=bon}, and in both same-subject and different-subject contexts. Transitive verbs like \textit{potma} are found with and without final \textit{-n} in both transitive and intransitive constructions. 

(\ref{exe:kieri}) and~(\ref{exe:koinani}) show \textit{ecien} and \textit{yecie}\footnote{The fact that the form with \textit{-n} does not carry \textit{y-} and vice versa may be related to them having different stress patterns: \textit{ˈyecie} vs \textit{eˈcien}. All ``uninflected'' verbs with \textit{-n} carry stress on the last syllable, even if the verb without \textit{-n} carries stress on the first (e.g. \textit{ˈpotma} vs \textit{potˈman} `to cut', \textit{ˈparuo} vs \textit{paˈruon} `to do', \textit{ˈosie} vs \textit{oˈsien}) but \textit{yecie} is the only one with a difference in initial vowel. Syllable weight does not normally play a role in stress assignment in Kalamang (as described in~§\ref{sec:stress}).} in very similar environments: both describe the actions of a third person in a narrative, both follow the iamitive \textit{he}, and both conclude a paragraph in the narrative where different actions were listed.

\begin{exe}
	\ex
	{\gll kiet=i koyet ma se \textbf{ecien}\\
		defecate={\gli} finish \textsc{3sg} {\glse} return\\
		\glt `After defecating, he returned.' \jambox*{\href{http://hdl.handle.net/10050/00-0000-0000-0004-1BDB-C}{[narr28\_2:06]}}
	}
	\label{exe:kieri}
	\ex 
	{\gll Tat owandi koi melelu nan=i koyet ma se \textbf{yecie}\\
		Tat like\_over\_there then sit consume={\gli} finish \textsc{3sg} {\glse} return\\
		\glt `After sitting and eating at a place like Tat over there, he returned.' \jambox*{\href{http://hdl.handle.net/10050/00-0000-0000-0004-1BDE-7}{[narr25\_2:10]}}
	}
	\label{exe:koinani}
\end{exe}

The irregular verb \textit{bo} `go' has only one uninflected form: \textit{bot}. The form \textit{bo} is reserved for complex predicates where \textit{bo} `go' is the first verb. In an intransitive clause without another verb, \textit{bo} cannot be used. The difference between a complex predicate with \textit{bo} in the first position and \textit{bot} in the last position is shown in~(\ref{exe:luisbon}) and~(\ref{exe:tamisenggabot}). (\ref{exe:hebot}) shows \textit{bot} `go' in a clause without another verb.

\begin{exe}
	\ex
	{\gll go yuol=ta me Luis=bon Nabil esun=bon \textbf{bo} sor-sanggara\\
		condition day={\glta} {\glme} Luis=\textsc{com} Nabil father.\textsc{3poss=com} go fish-search\\
		\glt `The next day, Luis and Nabil's father went fishing.' \jambox*{\href{http://hdl.handle.net/10050/00-0000-0000-0004-1BD7-2}{[narr3\_2:50]}}
	}
\label{exe:luisbon}
	\ex 
	{\gll mu ecien=i Tamisen=ka \textbf{bot}\\
		\textsc{3pl} return={\gli} Tamisen=\textsc{lat} go\\
		\glt \glt `They returned to Tamisen (Antalisa).' \jambox*{\href{http://hdl.handle.net/10050/00-0000-0000-0004-1B70-6}{[narr4\_2:16]}}
	}
\label{exe:tamisenggabot}
	\ex 
{\gll tumtum opa me se \textbf{bot}\\
	children {\glopa} {\glme} {\glse} go\\
	\glt `Those children have gone.' \jambox*{\href{http://hdl.handle.net/10050/00-0000-0000-0004-1B9F-F}{[conv9\_0:14]}}
}
\label{exe:hebot}
\end{exe}


%mia: 151 (2sg, 3sg, 1pl, etc), SVC kuru mia, lots with =i mia, single verb, comitative, lative, koi, se)
%mian: 41 (1sg, 3sg, 3pl, SVC, redup, ran-mian, kuru-mian, ecieni-mian, single verb, lative, se)
%
%potma: 49 (transitive, intrans, SVC potma paruak, potma kuru bara, potma ecieni mian, koi, se)
%potman: 15 (trans, intrans, redup)
%
%ecie: 117 (all persons, SVC deiri-yecie, yali-yecie, tiri-yecie, deiri-yecie, he, koi, di, lative)
%ecien: 13 (diff persons, SVC kuru ecien, deiri ecien, se, koi, lative)
%
%bo: 833 - includes SVCs which must be bo, i.e. bo+verb, bo+loc, otherwise 
%bot: 109 (all persons, SVC yortai-bot, pinggorpisi-bot, teli-bot, koi, se, comitative, lative)

%P. 202: one inf says maybe using the -n form is a bit more kasar, brings security to a statement, or you're already in the act of (i.e. also more security).

\subsubsection{Transitive/intransitive verb pairs in \textit{-ma} and \textit{-cie}}\is{transitivity}
\label{sec:macie}
Within the irregular verb class, Kalamang has a limited group of around twenty verbs that have a regular correspondence between transitive and intransitive forms. Transitive forms end in \textit{-ma}, whereas intransitive counterparts end in \textit{-cie}. There is no productive derivation with these suffixes. All verbs are verbs with semantics relating to opening, turning, cutting or breaking. They are listed in Table~\ref{tab:macie}. Some verbs in \textit{-ma} do not have a counterpart in \textit{-cie}, but there are no verbs in \textit{-cie} that do not have a counterpart in \textit{-ma}.\footnote{Exceptions are \textit{gocie} `to live' and \textit{yecie} `to return', but because these have different semantics I have not listed them here. Note, however, that \textit{-cie} in \textit{gocie} may also be a morpheme, as \textit{go} means `place'. Those counterparts in \textit{-cie} marked with an asterisk in the table were rejected in elicitation; the gaps remain to be tested.} Not all verbs with opening{\slash}turning  or cutting/breaking semantics belong to this group: regular verbs are, for example, \textit{kortaptap} `to cut out', \textit{suo} `to cut a coconut', \textit{nasiwik} `to open a can or box', \textit{maorek} `to break down' and \textit{parair} `to split'.

\begin{table}[t]
	\caption{Transitive and intransitive verb pairs}
	\label{tab:macie}
	\fittable{
		\begin{tabular}{l l}
			\lsptoprule
			transitive & intransitive\\ \midrule
			\textit{barotma} `to turn around' & *\textit{barotcie}\\
			\textit{borma} `to open hand' &\\
			\textit{dorma} `to pull out' & \textit{dorcie} `to be pulled out'\\
			\textit{durma} `to skewer' & \textit{durcie} `to have a hole'\\
			\textit{kahetma} `to open & *\textit{kahetcie}\\
			\textit{(ka)sawirma} `to pull' & \textit{kasawircie} `to be open'\\
			\textit{kararma} `to hit and break' & \textit{kararcie} `broken'\\
			\textit{kasotma} `to scrape?' & *\textit{kasotcie}\\
			\textit{kawarma} `to break; to fold' & \textit{kawarcie} `broken' (folded?)\\
			\textit{kawotma} `to peel' & *\textit{kawotcie}\\
			\textit{koramtolma} `to cut ritually' & \\
			\textit{letma} `to cut off a branch'&\\
			\textit{maulma} `to bend' & \textit{maulcie} `to be bent'\\
			\textit{mayilma} `to flip' & \textit{mayilcie} `to be flipped over'\\
			\textit{mintolma} `to cut out liver/adam's apple' &\\
			\textit{paherma} `to open' (eyes/vulgar) & \textit{pahercie} `to be open'\\
			\textit{potma} `to cut' & *\textit{potcie}\\
			\textit{pulma} `to pinch' & *\textit{pulcie} \\
			\textit{sanggotma} `to break off a branch' & \textit{sanggoyie} `to be broken (of a branch)'  \\
			\textit{seletma} `to cut off a piece' & \\
			\textit{suwarma} `to cut diagonally?' & *\textit{suwarcie}\\
			\textit{tadorma} `to break off' & \textit{tadorcie} `to be pulled out' \\
			\textit{tawotma} `to fold' & *\textit{tawotcie}\\
			\textit{tanggorma} `to open (door or window)' & \textit{tanggurcie} `to be opened'\\
			\textit{tolma} `to cut off; to take a shortcut' & \textit{tolcie} `to be cut' \\
			\textit{wierma} `to open (a book)' & \textit{wiercie} `to be unstuck'\\
			\textit{wurma} `to cut down a tree' & *\textit{wurcie} \\
			\lsptoprule
		\end{tabular}
	}
\end{table}

Most of these words do not have meaningful roots: for example \textit{kahet, kawet, mayil} and \textit{dur} were not recognised. Others have roots that occur (in a very similar form) as nouns. Note the similarities between \textit{sanggoup} `branch', \textit{sanggotma} `to break off a branch' and \textit{sanggoyie} `to be broken (of a branch)', and between \textit{selet} `piece' and \textit{seletma} `to cut off a piece'. There might also be a relation between \textit{tolas} `to break one's fast' and \textit{tolma} `to cut off; to take a shortcut' and \textit{tolcie} `to be cut'. This suggests that there was a productive verb formation device for creating opening, turning, cutting and breaking verbs.

\is{imperative}Imperative forms of the transitive verbs end in \textit{-ei} and \is{prohibitive}prohibitive in \textit{-ein}, e.g. \textit{potma} `to cut', \textit{potmei!} `cut!', \textit{potmein!} `don't cut!'. This is the same pattern as the directional verbs in \textit{-a}, which are described in the next section.


\subsubsection{Directional verbs}\is{verb!directional}
\label{sec:dir}
Directional verbs are verbs that express movement in a specific direction, end in \textit{-a} and are irregular. The subclass of directional verbs distinguishes itself from other verbs in several other ways. All verbs belonging to the class are given in Table~\ref{tab:dirall}. When combined with locative \textit{=ko}, as in~(\ref{exe:dirko}), directional verbs precede the locative, whereas other verbs follow it. Of all verbs, only directional verbs can carry the \is{causative}causative proclitic \textit{di=}, illustrated in~(\ref{exe:dirdi}). They also have a special imperative ending in \textit{-ei}, in contrast to other verbs, which carry imperative \textit{=te} (example~\ref{exe:dirimp}). Although they can stand on their own, directional verbs are commonly used in \is{predicate!complex}complex predicates; and of all Kalamang complex predicates the majority involve a directional verb (see Chapter~\ref{ch:svc}). Many of the directional verbs end in \textit{-ra}, including the rather generic \textit{ra} `to move (away)', suggesting these verbs are diachronically related and perhaps have been part of a paradigm involving the directional verb \textit{-ra}. The way speakers use directional verbs when talking about travelling by sea from Karas island is described in §\ref{sec:elevdem}.

\begin{table}
	\caption{Directional verbs}
	\label{tab:dirall}
	
		\begin{tabular}{l l}
			\lsptoprule
			\textit{sara} & `to ascend (vertically)'\\
			\textit{bara} & `to descend'\\
			\textit{masara} & `to move towards land'\\
			\textit{marua} & `to move towards sea'\\
			\textit{mia} & `to come'\\
			\textit{ra} & `to go; to move away'\\
			\textit{era} & `to move uphill; to ascend diagonally'\\
			\lspbottomrule
		\end{tabular}
	
\end{table}

\begin{exe}
	\ex
	{\gll kariak \textbf{sara} nakal=ko\\
		blood ascend head=\textsc{loc}\\
		\glt `Blood goes up to the head.' \jambox*{\href{http://hdl.handle.net/10050/00-0000-0000-0004-1BC2-8}{[narr33\_3:10]}}
	}
	\label{exe:dirko}
	\ex
	{\gll mat bantu karajang=at \textbf{di=saran} alfukat=at payiem\\
		\textsc{3sg.obj} help basket=\textsc{obj} \textsc{caus}=ascend avocado=\textsc{obj} fill\\
		\glt `[They] help him, put up his basket, fill it with avocados.' \jambox*{\href{http://hdl.handle.net/10050/00-0000-0000-0004-1BD2-D}{[stim30\_1:32]}}
	}
	\label{exe:dirdi}
	\ex 
	{\gll nene \textbf{mei}\\
		grandmother come.\textsc{imp}\\
		\glt `Grandmother, come!' \jambox*{\href{http://hdl.handle.net/10050/00-0000-0000-0004-1BBD-5}{[conv12\_10:40]}}
	}
	\label{exe:dirimp}
\end{exe}

There are two other observations about this subclass: first, their division into opposite pairs, and second, their co-occurrence with causative \textit{di=}.

Six of the seven verbs can be divided into opposite pairs: \textit{sara} `to ascend' and \textit{bara} `to descend', \textit{masara} `to move towards land' and \textit{marua} `to move towards sea', and \textit{mia} `to come' and \textit{ra} `to move away'. These pairs may occur together to indicate a back-and-forth movement. The verb pairs \textit{saran-baran}, \textit{maran-maruan} and \textit{ran-mian} (note that they all carry final \textit{-n}, although it is unclear why) are illustrated in~(\ref{exe:saranbaran}) to (\ref{exe:ranmian}). There is no dedicated opposite of \textit{era} `to move uphill; to ascend diagonally'. Instead, \textit{bara} `to descend' is the opposite of both \textit{sara} `to ascend (vertically)' and \textit{era}. However, \textit{era} and \textit{bara} are never used as an opposite pair in the same clause, whereas \textit{sara} and \textit{bara} are. 

\begin{exe}
			\ex
	{\gll an sara dodon-kawet∼kawet {\ob}...{\cb} \textbf{saran} \textbf{baran} an bara kome=ta me {\ob}...{\cb}\\
		\textsc{1sg} ascend clothing-fold∼\textsc{iter} {\ob}...{\cb} ascend descend \textsc{1sg} descend see={\glta} {\glme}\\
		\glt \glt `I went up to fold clothes, [I] went up and down, I went down to look...' \jambox*{\href{http://hdl.handle.net/10050/00-0000-0000-0004-1BA3-3}{[conv10\_16:05]}}
	}
	\label{exe:saranbaran}
		\ex
	{\gll ma-mun nain eh \textbf{maran} \textbf{maruan}\\
		\textsc{3sg-proh} like \textsc{hes} move\_landwards move\_seawards\\
		\glt `Not like eh, going towards land, going towards sea.' \jambox*{\href{http://hdl.handle.net/10050/00-0000-0000-0004-1BA3-3}{[conv10\_20:54]}}
	}
	\label{exe:maranmaruan}
	\ex
	{\gll mu sontum=at kome \textbf{ran} \textbf{mian}\\
		\textsc{3pl} person=\textsc{obj} look go come\\
		\glt \glt `They looked at the people coming and going.' \jambox*{\href{http://hdl.handle.net/10050/00-0000-0000-0004-1BC5-7}{[narr16\_2:45]}}
	}
	\label{exe:ranmian}
\end{exe}

The causative proclitic \textit{di=} (§\ref{sec:di}), which indicates movement towards a goal, often occurs on directional verbs. It can usually be translated with `put' in combination with a directional verb, such as \textit{di=sara} `put up' in~(\ref{exe:dirdi}). \textit{Di=} is not found with \textit{era} `to go up diagonally' or \textit{mia} `to come'. With \textit{ra} `to move (away), \textit{di=} creates the specific meaning `to put up; to install'. This is illustrated in~(\ref{exe:salat}).

\begin{exe}
	\ex
{\gll lemat=at paruon=i koyet komangganggoup=et sal=at \textbf{di=ran}\\
	bamboo\_string=\textsc{obj} make={\gli} finish put\_on\_roof={\glet} roof\_beam=\textsc{obj} \textsc{caus}=move\\
	\glt `After making the bamboo string [we] put on the roof, install the roof beams.' \jambox*{\href{http://hdl.handle.net/10050/00-0000-0000-0004-1BB2-1}{[narr6\_4:24]}}
}
\label{exe:salat}
\end{exe}

However, \textit{di=ra} can also more predictably mean `to put (away from the deictic centre)', as illustrated in~(\ref{exe:dira}), which also contains \textit{di=bara} `to put down', in this case, put inside a container.

\begin{exe}
	\ex
	{\gll an kaling=at \textbf{di=ran} per=at di=baran\\
		\textsc{1sg} frying\_pan=\textsc{obj} \textsc{caus}=go\_away frying\_pan=\textsc{obj} \textsc{caus}=descend\\
		\glt `I put the frying pan [on the fire], put in the water.' \jambox*{\href{http://hdl.handle.net/10050/00-0000-0000-0004-1C99-E}{[narr8\_1:48]}}
	}
	\label{exe:dira}
\end{exe}

\section{Verb derivation}\is{verb derivation|(}
\label{sec:vverbder}
Verbs can be derived in two ways: by compounding a noun and a verb (noun incorporation §\ref{sec:incorp}) and from nouns, by reduplicating them (§\ref{sec:verbder}).

%note ``nominalizations'' of verbs like mauun, simarun, kararaun. meaning unclar.


\subsection{Noun incorporation}\is{compound}\is{noun incorporation}
\label{sec:incorp}
Nouns can, and frequently are, incorporated in the verb in Kalamang. Noun incorporation is a process whereby a verb is derived from the compounding of a noun and a verb \parencite{mithun1984}, as introduced in §\ref{sec:compounding}. Only objects can be incorporated in Kalamang, and incorporated objects are recognised by their lack of object marking and the fact that the compound is treated as one phonological word. It is a common but optional parallel strategy to having a full (case-marked, stress-carrying) object NP.

In 2018, a count of all incorporations in the Kalamang corpus was conducted for a larger typological study (published as~\textcite{olthof2020}, which also includes more details on frequencies and verb semantics). In 11 hours of transcribed recordings, 155 incorporations with 60 different nouns and 53 different verbs were found. The most commonly incorporating verbs were \textit{na} `to consume', \textit{jie} `to get; to buy', \textit{paruo} `to make', \textit{rep} `to get; to collect' and \textit{suban} `to fish'. The most commonly incorporated nouns, excluding \textit{don} `thing', which is always incorporated, were \textit{sor} `fish', \textit{ter} `tea', \textit{kai} `firewood', \textit{muap} `food' and \textit{per} `water'. Some examples are given in Table~\ref{tab:incorp}. Proportions of incorporations per verb were calculated. The top four were \textit{sair} `to shoot' (4 out of 4 occurrences of this verb are incorporations), \textit{suban} `to fish' (10/11), \textit{na} in the meaning `to drink'\footnote{A distinction between \textit{na} in the meaning `to drink' and in the meaning `to eat' was made to make the dataset comparable with other languages. \textit{Na} `to eat' was very unlikely to incorporate. For details, see \textcite{olthof2020}.} (24/29) and \textit{jie} `to get; to collect' (12/24) \parencite{visseretalappl}. 

\begin{table}[ht]
	\caption{Most commonly incorporating verbs and incorporated nouns.}
	\label{tab:incorp}
	

		\begin{tabular}{l l l}
			\lsptoprule
            incorporating \textsc{v} & examples\\
            \midrule 
            \textit{na} `consume' & \textit{per-na} `water-drink', \textit{wat-na} `coconut-eat'\\
            \textit{jie} `get; buy' & \textit{kabun-jie} `innards-get', \textit{tabai-jie} `tobacco-buy'\\
            \textit{paruo} `make' & \textit{kurera-paruo} `basket-make', \textit{amdir-paruo} `garden-make'\\
            \textit{rep} `get; collect' & \textit{kai-rep} `firewood-collect', \textit{alangan-rep} `trouble-get'\\
            \textit{suban} `fish' & \textit{tebol-suban} `reef\_edge-fish'\\
            \midrule
            incorporated \textsc{n} & examples\\
            \midrule 
            \textit{sor} `fish' & \textit{sor-rur} `fish-skewer', \textit{sor-pasor} `fish-fry'\\
            \textit{ter} `tea' &\textit{ter-na} `tea-consume', \textit{ter-garewor} `tea-pour'\\
            \textit{kai} `firewood' & \textit{kai-rep} `firewood-collect', \textit{kai-narorar} `firewood-drag'\\
            \textit{muap} `food' & \textit{muap-ruon} `food-cook', \textit{muap-koya} `food-plant'\\
            \textit{per} `water' & \textit{per-na} `water-consume', \textit{per-jie} `water-collect'\\	
			\lspbottomrule
		\end{tabular}
	
	\end{table}

While there is one noun that cannot be the object of a verb but instead must be incorporated, \textit{don} `thing', there are also restrictions on which elements can be incorporated. Pronouns, \is{demonstrative}demonstratives and modified nouns cannot be incorporated. As for the verbs, intransitive verbs cannot incorporate. Incorporation can take place on verbs that are part of a complex predicate\is{predicate!complex}. Both the first and the second verb in such a construction may incorporate, as (\ref{exe:jiena}) and (\ref{exe:bomuapruo}) illustrate.

\begin{exe}
    \ex \gll in=a per-jie na\\
    \textsc{1pl.excl=foc} water-get drink\\
    \glt `We got and drank the water.' \jambox*{\href{http://hdl.handle.net/10050/00-0000-0000-0004-1BBB-2}{[narr40\_0:03]}}
    \label{exe:jiena}
	\ex \gll ma {\ob}...{\cb} bo amdir=ka bo \textbf{muap}-\textbf{ruo}\\
	\textsc{3sg} {} go garden=\textsc{lat} go food-dig\\
	\glt `She goes to the garden to dig up food.' \jambox*{\href{http://hdl.handle.net/10050/00-0000-0000-0004-1BA7-D}{[narr21\_0:49]}}
	\label{exe:bomuapruo}
\end{exe}

Many pairs of incorporated noun and verb are also found in the corpus as normal pairs of object + verb. For those incorporations for which an object + verb counterpart was not found in the corpus, it was checked whether not incorporating that noun in that verb was allowed. This was the case for all, except combinations with \textit{don} `thing', and the combination of \textit{min} `adam's apple; liver' and \textit{tolma} `to cut'. The incorporation \textit{mintolma} `to cut [the] throat' is a fixed expression used as a \is{swearing}\is{cursing}curse, and must be incorporated (see §\ref{sec:curse} on cursing for examples).

It is unclear what the semantic or pragmatic difference is between incorporated and non-incorporated noun-verb pairs, and thus it remains for further research what guides the choice between incorporating or not incorporating. Some incorporations are more abstract expressions. \textit{Ter-na} `tea-consume', for example, does not necessarily mean `drinking tea', but can also mean `having breakfast'. Incorporations with tree names and \textit{sara} `to ascend' mean `to climb that tree in order to harvest'. However, we also find object + verb pairs with the same nouns and verbs which also have the more abstract meanings, suggesting that incorporation is not a requirement for the abstract reading. This is illustrated in~(\ref{exe:terincorp}). In~(\ref{exe:teratna}), the speaker first uses \textit{ter} `tea' as the object and \textit{na} `consume' as the verb. Two turns later, in~(\ref{exe:terna}), he incorporates \textit{ter} into \textit{na} with the same meaning, `to have breakfast'. (\ref{exe:ternatnin}) shows that negation does not play a role in the difference between~(\ref{exe:teratna}) and~(\ref{exe:terna}), and that \textit{ter} may be incorporated in \textit{na} also when the verb is negated.

\begin{exe}
	\ex 
	\begin{xlist} \ex \gll an tok \textbf{ter=at} \textbf{nat}=nin an se sontum=at gonggung\\
	\textsc{1sg} yet tea=\textsc{obj} consume=\textsc{neg} \textsc{1sg} {\glse} person=\textsc{obj} call\\
	\glt `I haven't had breakfast yet, I'm already calling people.' \jambox*{\href{http://hdl.handle.net/10050/00-0000-0000-0004-1B9B-9}{[narr41\_1:55]}}
	\label{exe:teratna}
	\ex \gll	Nur an=at gonggung=te tok \textbf{ter-na}\\
	Nur \textsc{1sg=obj} call={\glte} first tea-consume\\
	\glt `Nur calls me to have breakfast first.' \jambox*{\href{http://hdl.handle.net/10050/00-0000-0000-0004-1B9B-9}{[narr41\_2:11]}}
	\label{exe:terna}
	\end{xlist}
	\label{exe:terincorp}
	\ex \gll mu muap ba mu \textbf{ter-nat=nin}\\
	\textsc{3sg} eat but \textsc{3sg} tea-consume=\textsc{neg}\\
	\glt `They ate but they didn't drink tea.' \jambox*{\href{http://hdl.handle.net/10050/00-0000-0000-0004-1BAD-2}{[narr29\_3:31]}}
	\label{exe:ternatnin}
%	\ex \gll wat=at saran=i koyet wat-un=at ma di=bara kiem-neko\\
%	coconut=\textsc{obj} ascend={\gli} finish coconut-\textsc{3poss=obj} \textsc{3sg} \textsc{caus}=descend basket-inside\\
%	After picking coconuts he put the coconuts in the basket. \jambox*{[narr30\_0:04]}
%	\label{exe:waratsara}
\end{exe}

This is not a question of referentiality (singling out a new referent by means of a non-incorporated noun), as on several occasions in the corpus the first mention of a referent is already incorporated. Other possible factors, such as whether or not the verb was inflected for mood or aspect, or the animacy or plurality of the referent, do not seem to have an effect on the choice of incorporation vs. no incorporation. From looking at the corpus examples, the only factor that was found to play a role in the choice between incorporation or no incorporation is \is{repetition}repetition in the same or adjacent turns. Whenever a speaker repeats themselves or the words of another speaker, there seems to be a heightened chance of repeating the previously used construction. But even after three repetitions of \textit{bir-na} `beer-consume' in~(\ref{exe:birnana}), speaker A suddenly opts for a non-incorporated \textit{bir} `beer'.

\begin{exe}
	\ex 
	\begin{xlist}
		\exi{A:}
		\gll \textbf{bir-na}=teba eh\\
		beer-consume={\glteba} right\\
		\glt `[They are] drinking beer, right?' 
		\exi{B:}
		\gll \textbf{bir-na}=teba\\
		beer-consume={\glteba}\\
		\glt `[They are] drinking beer.'
		\exi{A:}
		\gll mier=a \textbf{bir-na} kona \textbf{bir=at} \textbf{na}\\
		\textsc{3du=foc} beer-consume look beer=\textsc{obj} consume\\
		\glt `They are drinking beer, look, drinking beer.'
		\exi{B:}
		\gll mier \textbf{bir-na}=teba\\
		\textsc{3du} beer-consume={\glteba}\\
		\glt `They are drinking beer.' \jambox*{\href{http://hdl.handle.net/10050/00-0000-0000-0004-1BA8-B}{[stim4\_1:42]}}
	\end{xlist}
\label{exe:birnana}
\zlast


\subsection{Noun-to-verb derivation}
\label{sec:verbder}
Verbs can be derived from nouns by means of reduplication (introduced in §\ref{sec:redup}). This strategy is not very common or productive, as illustrated by the examples in~(\ref{exe:dokad}), some of which are rather idiosyncratic. \is{reduplication!verbs}

\begin{exe}
	\ex 
	\begin{xlist}
		\ex \textit{buok} `betel' → \textit{buokbuok} `to chew betel'
		\ex \textit{mun} `flea' → \textit{munmun} `to search for fleas'
		\ex \textit{doka} `heron' → \textit{dokadoka} `to sit and do nothing' (like a heron searching for fish)
		\ex \textit{yuol} `day' → \textit{yuolyuol} `to be light or bright; to shine'
		\ex \textit{kiet} `feces' → \textit{kietkiet} `to defecate'
	\end{xlist}
	\label{exe:dokad}
\end{exe}

The functions of reduplication of verbs are described in §\ref{sec:verbred}.

\largerpage
\section{Reduplication of verbs}\is{reduplication!verbs|(}
\label{sec:verbred}
Verbs can be reduplicated with several functions. \is{stative intransitive verb}Stative intransitive verbs are reduplicated to \is{intensification}intensify their meaning. Other verbs are reduplicated to indicate \is{habitual}habitual aspect, durativity (or maybe \is{repetition}repetition) or distribution. The formal aspects of reduplication are described in §\ref{sec:redup}. \is{reduplication!verbs}

Most intransitive verbs may be intensified through a combination of reduplication and the use of enclitic \textit{=tun} `very'. It is not attested with transitive verbs.

\begin{exe}
	\ex 
	\begin{xlist}
		\ex \textit{mon} `quick' → \textit{monmontun} `very quick'
		\ex \textit{yor} `true' → \textit{yoryortun} `very true'
	\end{xlist}
	\label{exe:tun}
\end{exe}	

Stative intransitive verbs ending in \textit{-kap} (colour terms) and \textit{-sik} are partially reduplicated. An example of each is given in~(\ref{exe:sikkap}). This (rightward) reduplication may be combined with \textit{=tun} `very'.

\begin{exe}
	\ex 
	\begin{xlist}
		\ex \textit{tabusik} `short' → \textit{tabusiksik} `very short'
		\ex \textit{kerkap} `red' → \textit{kerkapkap} `very red'
	\end{xlist}
	\label{exe:sikkap}
\end{exe}	

Some intransitive verbs are reduplicated leftward.

\begin{exe}
		\ex 
		\begin{xlist}
			\ex \textit{temun} `big' → \textit{temtemun} `very big'
			\ex \textit{alus} `soft' → \textit{alalus} `very soft'
			\ex \textit{gawar} `fragrant' → \textit{gawagawar} `very fragrant'
		\end{xlist}
		\label{exe:alalus}
\end{exe}	
		
\textit{Cicaun} `small' cannot be reduplicated in this way, but notice the other word for small: \textit{kinkinun}, which looks to be analogous in form to \textit{temtemun} `very big'. \textit{Kinun}, however, has no meaning in the contemporary language.

Both transitive and intransitive verbs may be reduplicated to denote duration, repetition or progressive aspect (examples~\ref{exe:winyy} and~\ref{exe:paruopp}), or distribution (examples~\ref{exe:dalangd} and~\ref{exe:ciep} with \is{distributive}distributive marker \textit{-p}). At least one verb, \textit{paruo} `to do; to make' can be used in a \is{habitual}habitual sense (example~\ref{exe:paruop}).

\begin{exe}
	\ex \gll an se yal=i mengga bo karimun=at kuangi bo Mas=ko, winyal, metko \textbf{winyal∼winyal}\\
	\textsc{1sg} {\glse} paddle={\gli} \textsc{dist.lat} go cape=\textsc{obj} pass go Mas=\textsc{loc} fish \textsc{dist.loc} fish∼\textsc{prog}\\
	\glt `I paddled from there, passed the cape to go to Mas, and fished, there [I] was fishing.' \jambox*{\href{http://hdl.handle.net/10050/00-0000-0000-0004-1C99-E}{[narr8\_0:24]}}
	\label{exe:winyy}
	\ex \gll mu yalyal ba menyanyi-un=at \textbf{paruo∼paruo}\\
	\textsc{3pl} paddle and song.\textsc{mly-3poss=obj} make∼\textsc{prog}\\
	\glt `They paddled making their song.' \jambox*{\href{http://hdl.handle.net/10050/00-0000-0000-0004-1BC1-0}{[narr19\_8:58]}}
	\label{exe:paruopp}
	\ex \gll kamamual tumun opa \textbf{dalang∼dalang}=ta opa me\\
	needlefish small {\glopa} jump∼\textsc{distr}={\glta} {\glopa} {\glme}\\
	\glt `That needlefish just jumped around there.' \jambox*{\href{http://hdl.handle.net/10050/00-0000-0000-0004-1BCE-D}{[conv4\_1:55]}}
	\label{exe:dalangd}
	\ex \gll sontum se \textbf{ecie-p∼cie-p}\\
	person {\glse} return-\textsc{distr∼distr-distr}\\
	\glt `People were returning (one by one/in small groups).' \jambox*{\href{http://hdl.handle.net/10050/00-0000-0000-0004-1BC3-B}{[conv7\_14:34]}}
	\label{exe:ciep}
	\ex \gll  amdir=at=a \textbf{paruo∼paruo}\\
	garden=\textsc{obj=foc} make∼\textsc{hab}\\
	\glt `gardening; work in the garden' \jambox*{\href{http://hdl.handle.net/10050/00-0000-0000-0004-1B83-E}{[narr43\_4:35]}}
	\label{exe:paruop}
\end{exe}
%durative also: garunggarung

%more durative
%\begin{exe}
%	\ex
%	{\gll kiun=a mat komet∼komet\\
%		wife.\textsc{3poss}=\textsc{foc} \textsc{3sg.obj} look∼\textsc{red}\\
%		`His wife looks at him.' \jambox*{[stim7\_25:26]}
%	}
%	\label{exe:kometkomet}
%	\ex
%	{\gll go kalis∼kalis\\
%		place rain∼\textsc{red}\\
%		`It was rainy.' \jambox*{[conv11\_2:16]}
%	}
%	\label{exe:kaliskalis}
%\end{exe}

%both noun and verb redup:
%korpakpak usasar

While duration and distribution or a combination of both are the most common functions of reduplicated verbs, there are a number of examples in the corpus that suggest other readings are possible.\footnote{This is why some reduplications are glossed as `\textsc{red}', instead of forcing a specific reading on them.} \textit{Istiraharearet} in~(\ref{exe:harew}) could be interpreted as `rest for a little' (diminutive) or `rest for a while' (\is{durative}durative, progressive). \textit{Narasnaras} in~(\ref{exe:narasn}) could mean `fight often' (repetitive) or `fight \is{habitual}habitually' (habitual), or `fighting' (progressive). \textit{Marmarmarmar} in~(\ref{exe:mmmm}) could mean `walk a little' or `walk slowly' (attenuative) or `walking' (progressive). \textit{Konawaruowaruo} in~(\ref{exe:knowa}), where the conversation partners list traditional medicines that they know, could refer to forgetting some things (\is{distributive}distributive?), or to slowly forgetting them (progressive? attenuative?). \textit{Kalomlomun} in~(\ref{exe:lomlom}), derived from a \is{stative intransitive verb}stative intransitive verb, may mean `very young' (intensification), `a little young' (attenuative) or `some are young' (distributive). \textit{Gosomingosomin} in~(\ref{exe:gosgos}) could be an attenuative `disappear a little'. 

\begin{exe}
	\ex \gll an toni Nyong emun pier tok \textbf{istirahare∼are∼t} warkin tok bes=et eba metko pi war=et\\
	\textsc{1sg} say Nyong mother.\textsc{3poss} \textsc{1du.in} first rest∼\textsc{red} tide first good={\glet} then \textsc{dist.loc} \textsc{1pl.excl} fish={\glet}\\
	\glt `I said: ``Nyong's mother, we two rest first. First when the tide is good, we go fishing.' \jambox*{\href{http://hdl.handle.net/10050/00-0000-0000-0004-1B9F-F}{[conv9\_2:01]}}
	\label{exe:harew}
	\ex \gll doa supaya mier sanang=et mier hidup-un bes mu-mun \textbf{naras∼naras}=in\\
	prayer so\_that \textsc{3du} happy={\glet} \textsc{3du} life-\textsc{3poss} good \textsc{3pl}-\textsc{proh} fight∼\textsc{red=proh}\\
	\glt `A prayer so that they will be happy, they have a good life, [so that] they don't fight.''' \jambox*{\href{http://hdl.handle.net/10050/00-0000-0000-0004-1BDC-D}{[conv8\_5:14]}}
	\label{exe:narasn}
	\ex \gll pi wilak yuwatko \textbf{marmar∼marmar}=et Nyong esun=at nawanggar=et\\
	\textsc{1pl.incl} sea \textsc{prox.loc} walk∼\textsc{red}={\glet} Nyong father.\textsc{3poss=obj} wait={\glet}\\
	\glt `We walk to the sea over there [while we] wait for Nyong's father.' \jambox*{\href{http://hdl.handle.net/10050/00-0000-0000-0004-1BCB-5}{[conv1\_3:44]}}
	\label{exe:mmmm}

	\ex \gll ikon se \textbf{konawaruo∼waruo} ge\\
	some {\glse} forget∼\textsc{red} no\\
	\glt `Some [we] already forgot, didn't we?' \jambox*{\href{http://hdl.handle.net/10050/00-0000-0000-0004-1BCA-4}{[conv20\_9:48]}}
	\label{exe:knowa}

	\ex \gll Tima emun mu mu=nan muin teun reidak ba tok \textbf{kalom∼lom∼un}\\
	Tima mother.\textsc{3poss} \textsc{3pl} \textsc{3pl}=too \textsc{3poss} fruit many but still unripe∼\textsc{red}\\
	\glt `Tima's mother's family have a lot of fruits, but [they] are still unripe.' \jambox*{\href{http://hdl.handle.net/10050/00-0000-0000-0004-1BBD-5}{[conv12\_16:45]}}
	\label{exe:lomlom}
	
	\ex \gll nasuena bolon baran pi-mun talalu pen=sawet=in o pen koi \textbf{neba∼neba} \textbf{gosomin∼gosomin}\\
	sugar little descend \textsc{1pl.incl-proh} too sweet=too=\textsc{proh} \textsc{emph} tasty then \textsc{ph∼red} disappear∼\textsc{red}\\
	\glt `Put in a little sugar, we shouldn't make it too sweet, the tastiness [could] disappear.' \jambox*{\href{http://hdl.handle.net/10050/00-0000-0000-0004-1BA2-F}{[conv11\_1:55]}}
	\label{exe:gosgos}
\end{exe}


There is one example of a pair composed of a transitive and an intransitive verb, where the latter is formed through reduplication. The transitive verb carries \textit{ma-}, which is found in some other derived transitives (like \textit{masa} `to dry in the sun' from \textit{sa} `to be dried' and \textit{maraouk} `to put' from \textit{taouk} `to lie'). 

\begin{exe}
	\ex \textit{masarut} `to tear' → \textit{sarusarut} `to be torn'
	\label{exe:sarus}
\end{exe}

\is{reduplication!verbs|)}\is{verb derivation|)}


\section{Valency changing}\is{valency}\is{voice}
\label{sec:val}
Kalamang has four derivational constructions and operations that result in a change in valency of the verb: reflexives (§\ref{sec:rflx}), reciprocals (§\ref{sec:recp}), applicatives (§\ref{sec:appl}) and causatives (§\ref{sec:caus}), all of which are predominantly made with verbal prefixes or proclitics. For the non-productive transitive verbs in \textit{-cie} (from intransitive verbs in \textit{-ma}) see §\ref{sec:macie}.

\subsection{Reflexive constructions}\is{reflexive}
\label{sec:rflx}
Reflexive constructions are verbal clauses where both arguments have the same referent, and where the subject carries out an action upon itself, such as `he shaves himself'. The valency of transitive verbs that are used in a reflexive construction is reduced by one: instead of two core arguments, there is only one, the subject.

The corpus contains only eight examples of constructions with the reflexive prefix \textit{un-}, four of which are in combination with the restricting pronoun marker \textit{-tain} (which may or may not be focused). These constructions occur with five different verbs: \textit{balaok} `to show' (three occurrences), \textit{deir} `to bring' (two occurrences), \textit{ganggie} `to lift', \textit{kajie} `to pick' and \textit{rua} `to kill' (each one occurrence). Three of eight occurrences have the reflexive verb as the first verb in a \is{predicate!complex}complex predicate linked with predicate linker \textit{=i}. (\ref{exe:undeiri}) shows \textit{un-deir} `to bring oneself'. (\ref{exe:unruani}), with \textit{un-rua} `to kill oneself', is from a recording made during net fishing, where one of the speakers describes the movements of a needlefish in or close to the net.\footnote{Four words in the lexicon might contain a fossilised prefix \textit{un-}. These are \textit{unmasir} `to give birth'\is{birth} (cf, \textit{masir} `to weed'), \textit{unkoryap} `to divide' (cf. \textit{koryap} `to divide', \textit{yap} `to divide'), \textit{unkawer} `body fat' (\textit{kawer} not attested) and \textit{unsor} `orange-spotted trevally' (cf. \textit{sor} `fish').}

\begin{exe}
	\ex \gll ma-\textbf{tain} se \textbf{un}-deir=i luk\\
	\textsc{3sg}-alone {\glse} \textsc{refl}-bring={\gli} come\\
	\glt `She came herself.' (Lit. `She brought herself coming.') \jambox*{\href{http://hdl.handle.net/10050/00-0000-0000-0004-1BBC-4}{[narr24\_5:33]}}
	\label{exe:undeiri}
	\ex \gll ma-\textbf{tain} se metko \textbf{un}-ruan=i mia o ma se koi kiem\\
	\textsc{3sg}-alone {\glse} \textsc{dist.loc} \textsc{refl}-kill={\gli} come \textsc{surpr} \textsc{3sg} {\glse} again flee\\
	\glt `He comes there killing himself, oh, he fled again.' \jambox*{\href{http://hdl.handle.net/10050/00-0000-0000-0004-1BCE-D}{[conv4\_1:58]}}
	\label{exe:unruani}
	\ex \gll mindi an se parar=te \textbf{un}-ganggie mat pareir\\
	like\_that \textsc{1sg}  {\glse} get\_up={\glte} \textsc{refl}-lift \textsc{3sg} follow\\
	\glt `Like that I lifted myself up and followed her.' \jambox*{\href{http://hdl.handle.net/10050/00-0000-0000-0004-1B8F-4}{[narr32\_1:06]}}
\end{exe}	

This is not a well-established construction, and the three other reflexive corpus examples rely on Malay loans. The Malay reflexive pronoun \textit{diri} `self' is used in those constructions, which are combined with an Austronesian\is{Austronesian!loan} loan verb \textit{natobat} `repent' (example~\ref{exe:diritobat}) or Malay loan verb \textit{sadar} `to be aware; to have self-awareness' (example~\ref{exe:ansadar}). It is unclear why the reflexive pronoun is marked with the object postposition in the latter but not in the former, but it suggests that there is not necessarily a valency reduction in reflexive constructions with Malay \textit{diri}.

\begin{exe}
	\ex 
	{\gll ma toni ma \textbf{diri}-un natobat\\
		\textsc{3sg} say \textsc{3sg} self-\textsc{3poss} repent\\
		\glt `He said he repents [himself].' \jambox*{\href{http://hdl.handle.net/10050/00-0000-0000-0004-1BAA-C}{[stim7\_24:33]}}
	}
	\label{exe:diritobat}
	\ex 
	{\gll an-\textbf{tain}=a se \textbf{diri}-an=at sadar\\
		\textsc{1sg}-alone=\textsc{foc} {\glse} self-\textsc{1sg.poss=obj} aware\\
		\glt `I already have self-awareness.' \jambox*{\href{http://hdl.handle.net/10050/00-0000-0000-0004-1BB0-D}{[stim12\_7:13]}}
	}
	\label{exe:ansadar}
\end{exe}	

Note that in the latter example, the subject pronoun \textit{an} `I' is also inflected with the restricting pronoun marker \textit{-tain} (§\ref{sec:alonepron}), like in~(\ref{exe:undeiri}) and~(\ref{exe:unruani}). In elicited reflexive examples, one encounters this strategy again, without the use of \textit{diri} `self', suggesting that these restricting pronouns can also have a reflexive interpretation. Consider~(\ref{exe:andainawaruo}), where \textit{an-tain} `I alone' is used to express `myself'. This example is ambiguous between a reflexive and a restrictive \is{focus!pronominal}focus meaning. It could also mean `It is I who washes'.\footnote{This construction was elicited with a picture of child washing its own hair, which was contrasted with a picture of a mother bathing a child. Showing the picture was accompanied by a request for a translation of Malay \textit{anak mandi diri} `the child bathes itself' and \textit{saya mandi diri} `I bathe myself'.}

\begin{exe}
	\ex 
	{\gll an-\textbf{tain}=a waruo\\
		\textsc{1sg}-alone=\textsc{foc} wash\\
		\glt `I wash myself.' \jambox*{\href{http://hdl.handle.net/10050/00-0000-0000-0004-1C60-A}{[elic\_refl\_4]}}
	}
	\label{exe:andainawaruo}
\end{exe}

Quantifying pronoun marker \textit{-tain} can also be used when a subject carries out an action upon a part of itself, and the referent is thus not exactly the same, as in~(\ref{exe:westalan}). In the corpus, this kind of construction without use of \textit{-tain} occurs, illustrated in~(\ref{exe:nakalan}). This suggests that \textit{-tain} is neither sufficient nor necessary to make a reflexive construction.

\begin{exe}
	\ex
	\gll an-\textbf{tain}=a westal-an=at sikat\\
	\textsc{1sg}-alone=\textsc{foc} hair-\textsc{1sg.poss=obj} brush\\
	\glt `I brush my own hair.' \jambox*{\href{http://hdl.handle.net/10050/00-0000-0000-0004-1C60-A}{[elic\_refl\_6]}}
	\label{exe:westalan}
	\ex 
	\gll an tok nakal-\textbf{an}=at sisir=et\\
	\textsc{1sg} first head-\textsc{1sg.poss=obj} comb={\glet}\\
	\glt `I comb my head first.' \jambox*{\href{http://hdl.handle.net/10050/00-0000-0000-0004-1BB5-B}{[conv17\_3:51]}}
	\label{exe:nakalan}
\end{exe}

\subsection{Reciprocal constructions}\is{reciprocal}
\label{sec:recp}
In a reciprocal clause, two (groups of) participants have a symmetrical relation to each other. That is, what counts for the one group or individual counts for the other group or individual, as in `they push each other' \parencite{haspelmath2007}. In Kalamang, reciprocal constructions are made with a verbal proclitic \textit{nau=}. In these constructions, the valency of the verb is typically reduced by one such that there is only a subject argument. The reciprocal proclitic is also, but less commonly, used in distributive constructions, as described at the end of this section.

The following examples illustrate simultaneous reciprocal actions.

\begin{exe}
	\ex
	\label{exe:nautabarak}
	{\gll mier \textbf{nau}=tabarak to\\
		\textsc{2du} \textsc{recp}=crash right\\
		\glt `They crash into each other, right?' \jambox*{\href{http://hdl.handle.net/10050/00-0000-0000-0004-1BD4-C}{[stim29\_:41]}}
	}
	\ex \gll in se tan \textbf{nau}=kinkin\\
	\textsc{1pl.excl} {\glse} hand \textsc{recp}=hold\\
	\glt `We already shake hands.' \jambox*{\href{http://hdl.handle.net/10050/00-0000-0000-0004-1BD8-4}{[narr1\_4:18]}}
	\label{exe:naukinkin}
\end{exe}

In the \textit{Reciprocal constructions and situation type field stimulus kit}\is{stimulus} \parencite{evans2004}, which consists of short video clips that speakers are asked to describe, \textit{nau=} was applied to events of hugging, chasing, hitting and exchanging where more than two people were acting upon each other in a unidirectional chain, pairwise, radially or mixed. The reciprocal proclitic is also used for asymmetrical actions, i.e. actions that one group or individual do(es) to another, but not vice versa (for example one person shaking others' hands, one person giving things to several others, a person chasing another who is being chased by a third). It can also be used for sequentially reciprocal events (like people hitting each other in turn). (\ref{exe:naukosalir}) is the description of a clip (no. 21) of four people sitting at a table, giving each other different items of food and drink. Each person gives and receives, but not necessarily to and from the same person. The verb \textit{kosalir} `to exchange' retains its valency, but changes in meaning. In non-reciprocal constructions, its object is replaced with another version of itself (another set of clothes, a part of the house when renovating, fresh food on the table). In reciprocal constructions, the object changes owner but is not replaced.

\begin{exe}
	\ex \gll mu muap=at \textbf{nau}=kosalir\\
	\textsc{3pl} food=\textsc{obj} \textsc{recp}=exchange\\
	\glt `They exchange food with each other.' \jambox*{\href{http://hdl.handle.net/10050/00-0000-0000-0004-1C60-A}{[elic\_rec\_21]}}
	\label{exe:naukosalir}
\end{exe}

(\ref{exe:naukoup}) describes a clip (no. 16) where three people hug a fourth person sequentially.

\begin{exe}
	\ex \gll mu \textbf{nau}=koup\\
	\textsc{3pl} \textsc{recp}=hug\\
	\glt `They hug each other.' \jambox*{\href{http://hdl.handle.net/10050/00-0000-0000-0004-1C60-A}{[elic\_rec\_23]}}
	\label{exe:naukoup}
\end{exe}

Some verbs, when used in reciprocal constructions, take a more abstract meaning. The verbs \textit{tu} `to hit' and \textit{sair} `to shoot (with a gun)' are used for expressing fighting or even war. (\ref{exe:nausair}) is about the Second World War, and~(\ref{exe:nautu}) describes picture 4 from the \textit{Family problems picture task}\is{picture-matching task} \parencite{carroll2009} with angry-looking people drinking beer. One person is touching another on the shoulder and another is raising his fist but they are not actually hitting each other. While \textit{nausair} is only attested in war contexts, \textit{nautu} can also simply mean `hit each other'.

\begin{exe}
	\ex \gll wiseme Jepang=bon Amerika=bon \textbf{nau}=sair=ten\\
	long\_time\_ago Japan=\textsc{com} America=\textsc{com} \textsc{recp}=shoot={\glten}\\
	\glt `A long time ago, Japan and America were at war.' \jambox*{\href{http://hdl.handle.net/10050/00-0000-0000-0004-1BBB-2}{[narr40\_0:03]}}
	\label{exe:nausair}
	\ex \gll mu bir namabuk=teba \textbf{nau}=tu\\
	\textsc{3pl} beer drunk={\glteba} \textsc{recp}=hit\\
	\glt `They're drunk on beer, fighting.' \jambox*{\href{http://hdl.handle.net/10050/00-0000-0000-0004-1BA9-9}{[stim6\_12:23]}}
	\label{exe:nautu}
\end{exe}

Other verbs that get a more abstract meaning when used reciprocally are \textit{tan nau=kinkin} `hand \textsc{recp}=hold', which means `to shake hands', and \textit{nau=bes} `to make up', from \textit{bes} `to be good'. The latter is illustrated in~(\ref{exe:naubes}), which describes people making up after a fight. The intransitive verb \textit{bes} is transitivised with help of \textit{paruo} `to make' (§\ref{sec:caus}), and then reduced in valency again with help of reciprocal \textit{nau=}.

\begin{exe}
	\ex \gll mu paruo \textbf{nau}=bes\\
	\textsc{3pl} make \textsc{recp}=good\\
	\glt `They are making up.' \jambox*{\href{http://hdl.handle.net/10050/00-0000-0000-0004-1BA9-9}{[stim6\_8:32]}}
	\label{exe:naubes}
\end{exe}

A verb that gets a more specific reading with \textit{nau=} is \textit{komahal} `to not know', which means `to not have encountered' when inflected with the reciprocal. This is also valid where one of the parts is inanimate, such as the forest to which the speaker describes fleeing in~(\ref{exe:naukomahal}).

\begin{exe} 
	\ex \gll ka tamandi=a kiem=et, ka tamangga=ta, in se \textbf{nau}=komahal=te\\
	\textsc{2sg} how=\textsc{foc} flee={\glet} \textsc{2sg} where.\textsc{lat}={\glta} \textsc{1pl.excl} \textsc{iam} \textsc{recp}=not\_know={\glte}\\
	\glt `How could you flee, where? We had never encountered [that part of the forest] before.' \jambox*{\href{http://hdl.handle.net/10050/00-0000-0000-0004-1BBB-2}{[narr40\_3:42]}}
	\label{exe:naukomahal}
\end{exe}

At least one verb, \textit{munmun} `to delouse' (derived from \textit{mun} `louse') cannot be used with the reciprocal marker, regardless of whether the action is symmetrical or asymmetrical, simultaneous or sequential.

Reciprocal \textit{nau=} is found on two nouns in the corpus. It is unclear whether this is a productive process, but attaching reciprocal \textit{nau=} appears to change these nouns into verbs with a meaning related to the noun. The root \textit{kia-} `same-sex sibling' turns into \textit{nau=kia(ka)} `to be/have siblings' (example~\ref{exe:naukia}), and the root \textit{kahaman} `bottom' turns into \textit{nau=kahaman} `to be close to or touch each other's bottoms' (example~\ref{exe:naukahaman}). (The example is from a picture-matching task\is{picture-matching task} with pictures of two building blocks.)

\begin{exe}
	\ex
	{\gll ma koi \textbf{nau}=kiaka mambon=ta (...)\\
		\textsc{3sg} then \textsc{recp}=be\_sibling \textsc{exist}={\glta}\\
		\glt `He, then, has siblings [who] are there (...).' \jambox*{\href{http://hdl.handle.net/10050/00-0000-0000-0004-1BC3-B}{[conv7\_8:27]}}
	}
	\label{exe:naukia}
	\ex {\gll \textbf{nau}=kahaman\\
		\textsc{recp}=bottom\\
		\glt `Touching each other's bottoms.' \jambox*{\href{http://hdl.handle.net/10050/00-0000-0000-0004-1BD6-8}{[stim38\_6:08]}}
	}
	\label{exe:naukahaman}
\end{exe}

\textit{Nau=} is in a few cases used in a \is{distributive}distributive sense. In~(\ref{exe:naunggang}), it is prefixed to \textit{gang} `to hang' to talk about items hanging in a tree. It is distributive in the sense that the items hang in different places. In~(\ref{exe:us}), it is used to describe a naked man. That example is distributive in the sense that the man's penis is seen in movement, dangling from the body.

\begin{exe}
	\ex \gll se \textbf{nau}=gang\\
	\textsc{iam} \textsc{recp}=hang\\
	\glt `[They] hung everywhere.' \jambox*{\href{http://hdl.handle.net/10050/00-0000-0000-0004-1BDF-0}{[narr27\_1:54]}}
	\label{exe:naunggang}
	\ex \gll ma handuk=at jien=i koyet paruak=i kor kerunggo us \textbf{nau}=gang\\
	\textsc{3sg} towel=\textsc{obj} get={\gli} finish throw={\gli} leg on\_top penis \textsc{recp}=hang\\
	\glt `After getting a towel he threw it over his legs. His penis was dangling.' \jambox*{\href{http://hdl.handle.net/10050/00-0000-0000-0004-1BC6-C}{[narr17\_1:13]}}
	\label{exe:us}
\end{exe}

One corpus example suggests that reciprocal constructions may also be used to express the omnipresence or perhaps the duration of a feeling, similar to a distributive reading. The reciprocal proclitic is not attested with other bodily sensations, and a clear example illustrating the meaning of \textit{layier} `to itch' without the reciprocal is lacking.

\begin{exe}
	\ex
	\label{exe:naulaier}
	{\gll kaden-un \textbf{nau}=layier=te\\
		body-\textsc{3poss} \textsc{recp}=itch=\glte\\
		\glt `His body is itchy.' \jambox*{\href{http://hdl.handle.net/10050/00-0000-0000-0004-1B8F-4}{[narr32\_0:59]}}
	}
\end{exe}


%also: naukia, naukiaka. `to be siblings'. in naukiaka reidak, an naukiaka reidak. *naukiaka anggon reidak.
%naunamakin (make each other feel uncomfortalbe), naukenal, nauketemu, naukasuo (sex), naukinkin (w. hand), naukanggirar/naunamanghadap (face), naukoraruo (bite each other), naugerket, naukomere, naukoup, naukosaranden (that are a fit, baku kenal), nautu (hit ech other), naumalawan, naunawatana
%grammaticalised: nauaona is tidy, but also balance a boat, i.e. to do with symmetry, naukaia? not tested kaia, naulanggos `sit with arms or legs crossed', nauleluk ` to meet', maybe naurar.

%One verb in the natural spoken corpus has a lenited first consonant when it is used reciprocally: \textit{pareir} `to follow'. In that instance, \textit{pareir} itself is reduplicated which causes the reduplication to have a lenited first consonant, which perhaps triggered the lenition\is{lenition} after \textit{nau=}. Leave out? I included it because I have similar examples for noun incorporation and the applicative, which also usually don't show lenition but in a few cases do.
%
%\begin{exe}
%	\ex \glll Eir nauwarirwarir.\\
%	eir nau=pareir∼pareir\\
%	two \textsc{recp}-follow∼\textsc{red}\\
%	`Two follow each other.' \jambox*{[stim40\_1:32]}
%\end{exe}	

Lastly, the form \textit{nauleluk} `to meet each other' might be a lexicalised form of \textit{nau=} and \textit{koluk} `to meet'. [k] → [l] is not a synchronic morphophonological process.

\subsection{Applicative constructions}\is{applicative}
\label{sec:appl}
In an applicative construction, an underlying peripheral argument is promoted to become an object argument \parencite[][335]{dixon2012}, i.e. there is an increase in valency. This is done with a verbal proclitic \textit{ko=}. Table~\ref{tab:appl} gives some examples of verbs and one noun with the applicative proclitic, together with the semantic role of the promoted argument.\footnote{The translational equivalent of two applicatives and three promoted arguments is not entirely clear and therefore given with a question mark.} 


\begin{table}[ht]
	\caption{Applicatives}
	\label{tab:appl}
	\fittable{
		\begin{tabular}{l l l}
			\lsptoprule
			original	& applicative & promoted argument\\\midrule 
			\textit{gareor} `pour' tr & \textit{ko=gareor} `pour onto' & location\\
			\textit{sara} `move up' intr & \textit{ko=sara} `move onto' & location\\
			\textit{palom} `to spit' intr & \textit{ko=palom} `to spit on' & location\\
			\textit{yuon} `to rub' tr & \textit{ko=yuon} `to rub with' & instrument\\
			\textit{ewa} `to speak' intr & \textit{ko=ewa} `to be mad at' & goal\\
			\textit{kademor} `be mad' intr & \textit{ko=kademor} `to be mad at' & goal\\
			\textit{naurar} `to circle' intr & \textit{ko=naurar} `to circle somebody?' & patient?\\
			\textit{nasuarik} `to scatter' tr & \textit{ko=nasuarik} `to scatter something' & patient\\
			\textit{garung} `to chat' intr & \textit{ko=garung} `talk about?' & theme?\\
			\textit{kanggirar} `face' noun  & \textit{ko=kanggirar} `to face' & goal?\\
			\lspbottomrule
		\end{tabular}
		}
\end{table}

Most applicatives promote a \is{location}location (which typically remains unexpressed in the original) to object. The original object of \textit{yuon} `to rub' in~(\ref{exe:yuon}) is the location of the action (the body part), which changes to the instrument in an applicative construction (ointment, water, a rag), or coconut oil as in~(\ref{exe:koyuon}). 

\begin{exe}
	\ex \gll kaden-un=at mu \textbf{yuon}\\
	body-\textsc{3poss=obj} \textsc{3pl} rub\\
	\glt \glt `His body they rub.' \jambox*{\href{http://hdl.handle.net/10050/00-0000-0000-0004-1BD4-C}{[stim29\_1:05]}}
	\label{exe:yuon}
	\ex \gll mingtun=at bolon-i \textbf{ko=yuon}\\
	coconut\_oil=\textsc{obj} little-\textsc{objqnt} \textsc{appl}=rub\\
	\glt `Rub [the sore body part] with a little coconut oil.' \jambox*{\href{http://hdl.handle.net/10050/00-0000-0000-0004-1BE4-8}{[narr31\_1:32]}}
	\label{exe:koyuon}
\end{exe}

Two other verbs \textit{ewa} `to speak' and \textit{kademor} `to be mad' introduce an object to the scene when the applicative is used. The speaker in~(\ref{exe:koewa}) describes how she finished a task quickly, so that her husband would not get mad at her. The meaning of \textit{ewa} `to speak' thus changes to `to be mad at' with the applicative.

\begin{exe}
	\ex \gll mena ma ecie ma an=at \textbf{ko}=ewa\\
	otherwise \textsc{3sg} return \textsc{3sg} \textsc{1sg=obj} \textsc{appl}=speak\\
	\glt `Otherwise when he comes back, he is mad at me.' \jambox*{\href{http://hdl.handle.net/10050/00-0000-0000-0004-1BA3-3}{[conv10\_0:15]}}
	\label{exe:koewa}
\end{exe}

The applicatives \textit{ko=naurar} `to circle somebody?' and \textit{ko=garung} (/koŋgaruŋ/) `to talk about?' have only two occurrences each in the corpus, in which a patient and a theme are promoted to object, respectively. The only occurrences of \textit{ko=naurar} are the ones in~(\ref{exe:konaurar}), from a story that describes a ritual\is{ritual} for the welcoming of a new spouse to Karas Island, whereby the spouse sits in a canoe while a plate with betel leaves is circled above her head. While in this example a person is circled, more data might show that \textit{ko=naurar} takes any location as its object. \textit{Ko=garung} `to talk about?' has one example (see \ref{exe:kadanat}) where the theme is the object argument. The other example, given in~(\ref{exe:lembaec}), from the same text, lacks an object. The understood object may be either the prison (the theme), mentioned in the clause before or the people the ex-prisoner is talking to (the goal), who are visible in the picture that is being described in this utterance.

\begin{exe}
	\ex \gll buok\_sarung=bon sara et kit=ko mat wan-karuok-i \textbf{ko}=naurar \textbf{ko}=naurar=i koyet\\
    betel\_plate=\textsc{com} ascend canoe above=\textsc{loc} \textsc{3sg.obj} time-three-\textsc{objqnt} \textsc{appl}-circle \textsc{appl}-circle={\gli} finish\\
    \glt `[They] move the betel plate above the canoe and circle her three times. After circling [her]...' \jambox*{\href{http://hdl.handle.net/10050/00-0000-0000-0004-1BDC-D}{[conv8\_2:40]}}
\label{exe:konaurar}
	\ex \gll ma ecie ma kadan=at \textbf{ko}=garung\\
	\textsc{3sg} return \textsc{3sg} situation.\textsc{mly=obj} \textsc{appl}=chat\\
	\glt `He returns, he talks about the situation.' \jambox*{\href{http://hdl.handle.net/10050/00-0000-0000-0004-1BAA-C}{[stim7\_24:24]}}
	\label{exe:kadanat}
	\ex \gll ma lembaga=at=a ecie ma se \textbf{ko}=garung\\
	\textsc{3sg} prison.\textsc{mly=obj=foc} return \textsc{3sg} {\glse} \textsc{appl}=chat\\
	\glt `He returns from prison, he talks [about it? to them?].' \jambox*{\href{http://hdl.handle.net/10050/00-0000-0000-0004-1BAA-C}{[stim7\_24:31]}}
	\label{exe:lembaec}
\end{exe}

Lastly, there is one recorded case of a noun-to-verb conversion with help of applicative \textit{ko=}. The noun \textit{kanggirar} `face' changes to \textit{kokanggirar} `to face'. Though this is not a prototypical applicative, as no argument is promoted to become object, the semantics are very similar (\textit{kokanggirar} must be used with an object). \textit{Kanggirar} cannot be used as a verb, either with or without an object.

\begin{exe}
	\ex \gll ma ror=at \textbf{ko}=kanggirar\\
	\textsc{3sg} tree=\textsc{obj} \textsc{appl}=face\\
	\glt `He faces the tree.' \href{http://hdl.handle.net/10050/00-0000-0000-0004-1BC8-8}{[stim26\_8:35]}
	\label{exe:kokang}
\end{exe} 	

There is one applicative that shows lenition\is{lenition} on the first consonant: \textit{koalom} `to spit on' from \textit{palom} `to spit'.

The productivity of \textit{ko=} remains for further research.


\subsection{Causative constructions}\is{causative|(}	
\label{sec:caus}
Causative constructions increase the valency of intransitive verbs by one, introducing a subject argument and making the subject of the intransitive verb into an object argument, so that they become transitive. The main strategy is with causative \textit{di=}, a proclitic on the predicate (§\ref{sec:di}). This causative occurs only with predicates that express movement and location\is{location}. Other causative constructions, with causative \textit{ma=} and with complex predicates with \textit{paruo}, are described in §\ref{sec:causother}.


\subsubsection{Causatives with \textit{di=} `\textsc{caus}'}
\label{sec:di}
The causative proclitic \textit{di=} attaches to the left edge of the predicate and occurs with directional verbs (except \textit{era} `to move up diagonally') and with locative constructions. It is optional in give-constructions (see §\ref{sec:give}). The three uses are illustrated below. 

(\ref{exe:kiesiri}) illustrates causative \textit{di=} on \textit{marua} `to move seawards'. The subject of the intransitive verb (`two poles') becomes the object and a subject (`they') is introduced into the clause. In~(\ref{exe:dibint}), the predicate is a location.

\begin{exe}
	\ex \gll ror=at mu kis-eir-i \textbf{di}={\ob}maruan{\cb}\textsubscript{\upshape Pred}\\
	wood=\textsc{obj} \textsc{3pl} \textsc{clf\_long}-two-\textsc{objqnt} \textsc{caus}=move\_seawards\\
	\glt `Of wood, they moved two poles to the sea-side.' \jambox*{\href{http://hdl.handle.net/10050/00-0000-0000-0004-1BC3-B}{[conv7\_13:56]}}
	\label{exe:kiesiri}
	\ex \gll ma se per=at \textbf{di}={\ob}bintang neko{\cb}\textsubscript{\upshape Pred}\\
	\textsc{3sg} {\glse} water=\textsc{obj} \textsc{caus}=tub inside\\
	\glt `He had put water inside the tub.' \jambox*{\href{http://hdl.handle.net/10050/00-0000-0000-0004-1BC3-B}{[conv7\_7:08]}}
	\label{exe:dibint}
\end{exe}

Locative predicates may be quite long, even including a demonstrative, as in~(\ref{exe:dikar}). As this example also illustrates, the object may be elided, as is common in Kalamang (§\ref{sec:syntel}).

\begin{exe}
	\ex
	\gll mu=a kansuor mia kajie {\ob}...{\cb} \textbf{di}={\ob}karanjang opa nerunggo{\cb}\textsubscript{\upshape Pred} to\\
		\textsc{3pl=foc} four come pick {} \textsc{caus}=basket \textsc{ana} inside right\\
		\glt `They four come and pick {\ob}...{\cb}, put [the avocados] in that basket, right.' \jambox*{\href{http://hdl.handle.net/10050/00-0000-0000-0004-1BD5-6}{[stim34\_0:41]}}
		\label{exe:dikar}	
\end{exe}

Lastly, causative \textit{di=} is employed in give-constructions (§\ref{sec:give}). It is not obligatory, and it remains unclear what guides the choice for using (example~\ref{exe:sungdi}) or omitting (example~\ref{exe:pitin}) \is{causative}causative \textit{di=} in a give-construction, and whether it is a valency-increasing device in these constructions or not. In any case, the zero morpheme `give' may be used with the theme (the indirect object argument that is most commonly omitted from the construction) whether \textit{di=} is used or not, as the two examples in~(\ref{exe:sungdi}) and~(\ref{exe:pitin}) illustrate.%also unclear if recipient is an argument or is predicate.

\begin{exe}
	\ex \gll an sungsung=at \textbf{di}=tumun-an=ki ∅\\
	\textsc{1sg} pants=\textsc{obj} \textsc{di}=child-\textsc{1sg.poss=ben} give\\
	\glt `I give pants to my child.' \jambox*{\href{http://hdl.handle.net/10050/00-0000-0000-0004-1C60-A}{[elic\_give\_12]}}
	\label{exe:sungdi}
	\ex \gll ka pitis=at in ∅=kin\\
	\textsc{2sg} money=\textsc{obj} \textsc{1pl.excl} give=\textsc{vol}\\
	\glt `Do you want to give us money? \jambox*{\href{http://hdl.handle.net/10050/00-0000-0000-0004-1B9F-F}{[conv9\_10:59]}}
	\label{exe:pitin}
\end{exe}

In fact, in the give-constructions, as in many other constructions with causative \textit{di=}, the proclitic seems to indicate movement towards an endpoint besides being a valency increaser. The use of \textit{di=} seems to put the focus on the endpoint or the goal of a movement rather than on the movement itself. When used with directional verbs, \textit{di=} indicates that the movement ends somewhere. In~(\ref{exe:disaran}), the endpoint is the edge of the canoe, on which a plank (the omitted object noun) is attached after drilling holes. In~(\ref{exe:kiesirii}), repeated from~(\ref{exe:kiesiri}) above, the terminus is the sea-side, and the object is two poles of wood. When used with directional verbs, \textit{di=} can often be translated as `put'.

	\begin{exe}
	\ex
{\gll in er=at bor=i koyet to eba taikon-i \textbf{di}=\textbf{saran}\\
	\textsc{1pl.excl} canoe=\textsc{obj} drill={\gli} finish right then one\_side-\textsc{objqnt} \textsc{caus}=ascend\\
	\glt `After finishing drilling the canoe, right, [we] put up one side.' \jambox*{\href{http://hdl.handle.net/10050/00-0000-0000-0004-1BB4-6}{[narr14\_5:57]}}
}
\label{exe:disaran}
	\ex \gll ror=at mu kis-eir-i \textbf{di}=\textbf{maruan}\\
	wood=\textsc{obj} \textsc{3pl} \textsc{clf\_long}-two-\textsc{objqnt} \textsc{caus}=move\_seawards\\
	\glt `Of wood, they moved two poles to the sea-side.' \jambox*{\href{http://hdl.handle.net/10050/00-0000-0000-0004-1BC3-B}{[conv7\_13:56]}}
	\label{exe:kiesirii}
	\end{exe}

Directional verbs may also be used with a subject and object argument without use of \textit{di=}. In these cases, the verbs indicate ongoing movement, or have less focus on the goal of the movement and more on the movement itself. (\ref{exe:waratsara}) is about climbing a coconut palm in order to pick coconuts. The focus is on the entire harvesting action, which does not finish at the top of the tree.  Also, directional verbs without causative \textit{di=} take a subject that is the moving referent, and an object that is a location. Directional verbs with causative \textit{di=}, on the other hand, take a subject that is the causer, and an object that is the moving referent. See Table~\ref{tab:disubob}.

\begin{exe}
	\ex
	{\gll an bo wat=at \textbf{sarat}=et\\
		\textsc{1sg} go coconut=\textsc{obj} ascend={\glet}\\
		\glt `I climbed the coconut.' \jambox*{\href{http://hdl.handle.net/10050/00-0000-0000-0004-1BBB-2}{[narr40\_15:46]}}
	}
	\label{exe:waratsara}
\end{exe}

\begin{table}
	\caption{Directional verbs with and without causative \textit{di=}.}
	\label{tab:disubob}
	
		%
		\begin{tabular}{lll} 
			\lsptoprule
			&subject & object\\\midrule 
			with \textit{di=} `\textsc{caus}' & causer & moving referent\\
			without \textit{di=} `\textsc{caus}'  & moving referent& location\\\lspbottomrule 
		\end{tabular}
	
\end{table}

Causative \textit{di=} is not productive; it is ungrammatical on non-directional verbs.

\begin{exe}
	\ex
	\label{exe:difail}
	\begin{xlist}
		\ex [*]
		{\gll ema tumun=at di=waruon\\
			mother child=\textsc{obj} \textsc{di}-wash\\
			\glt Intended: `Mother washes the child.' \jambox*{[elic]}
		}
		\ex [*]
		{\gll ma an kasamin=at di=komet\\
			\textsc{3sg} \textsc{1sg} bird=\textsc{obj} \textsc{di}-see\\
		\glt 	Intended: `He shows me the bird.' \jambox*{[elic]}
		}
		\ex [*]
		{\gll ki an=at di=mian tamisen=ka\\
			\textsc{2pl} \textsc{1sg=obj} \textsc{di}=come Antalisa=\textsc{lat}\\
			\glt Intended: `You made me come to Antalisa.' \jambox*{[elic]}
		}
	\end{xlist}
\end{exe}

\textit{Di=} is not a verb: it cannot stand on its own and must be accompanied by a predicate, and it is never inflected for anything. Diachronically, however, it could derive from the verb \textit{jie} `to get' (palatalised, see §\ref{sec:assib}), which could have taken on a more grammatical role in \is{predicate!complex}complex predicates. Compare the Papuan language Eastern Timor, where `to take' was part of a serial verb construction for give-constructions \parencite{klamer2012}.

\subsubsection{Other causative strategies}
\label{sec:causother}
Kalamang employs several other causative strategies. One is causative proclitic \textit{ma=} (\textit{na=} before verbs in \textit{m-}), as in~(\ref{exe:namin}) and~(\ref{exe:masala}). Another strategy is to use a \is{predicate!complex}complex predicate with \textit{paruo} `to do; to make' (example~\ref{exe:eiganpa}, see also §\ref{sec:svccaus}). Both strategies have 28 corpus occurrences, and neither is productive.

\begin{exe}
	\ex \gll mat \textbf{na}=min ye \textbf{na}=melelu ge\\
	\textsc{3sg.obj} \textsc{caus}-sleep or \textsc{caus}-sit no\\
	\glt `Put him to sleep, wake him up, no.' \jambox*{\href{http://hdl.handle.net/10050/00-0000-0000-0004-1BC3-B}{[conv7\_10:58]}}
	\label{exe:namin}
	\ex \gll in koi wewar=ki \textbf{ma}=salaboung\\
	\textsc{1pl.excl} then axe=\textsc{ins} \textsc{caus}=broken\\
	\glt `Then we break [it] off with the axe. \jambox*{\href{http://hdl.handle.net/10050/00-0000-0000-0004-1BB4-6}{[narr14\_1:05]}}
	\label{exe:masala}
	\ex \gll ma tan-un eir-gan \textbf{paruo} yorsik\\
	\textsc{3sg} arm-\textsc{3poss} two-all make straight\\
	\glt `He straightens both his arms.' \jambox*{\href{http://hdl.handle.net/10050/00-0000-0000-0004-1B92-E}{[stim44\_2:18]}}
	\label{exe:eiganpa}
\end{exe}
%other nice exe: masem

Complex predicate causative constructions may also be made with the Malay loan \textit{kasi} `to give' (11 occurrences, not productive). In the elicitation of causative constructions (with translations from Malay), if none of the strategies above is employed, people opt for biclausal constructions. In~(\ref{exe:mamia}), for example, I tried to elicit `I made the child come' and got a biclausal construction in return. It is unclear how common the biclausal strategy is in more naturalistic settings.

\begin{exe}
	\ex \gll an tumun=at gonggung=te ma mia\\
	\textsc{1sg} child=\textsc{obj} call={\glte} \textsc{3sg} come\\
	\glt `I called the child, it came.' \jambox*{\href{http://hdl.handle.net/10050/00-0000-0000-0004-1C60-A}{[elic\_cau19\_9]}}
	\label{exe:mamia}
\end{exe} 

Three transitive verbs seem to contain causative \textit{ma=}, but are now merged with the intransitive root they are derived from. Compare the pairs in~(\ref{exe:mararak}) to~(\ref{exe:manyor}).

\begin{exe}
	\ex 
	\begin{xlist}
		\ex \textit{mararak} `to dry'
		\ex \textit{pararak} `to be dry'
	\end{xlist}
	\label{exe:mararak}
	\ex 
	\begin{xlist}
		\ex \textit{manggang} `to hang up'
		\ex \textit{ganggang} `to be hanging'
	\end{xlist}
	\ex 
	\begin{xlist}
		\ex \textit{manyor} `to adjust'
		\ex \textit{yor} `to be right'
	\end{xlist}	
	\label{exe:manyor}
\end{exe}

One verb shows lenition\is{lenition} of its initial consonant: \textit{maoyet} `to finish' from \textit{koyet} `to be finished'.

Something similar is the case with three verbs that start with \textit{me-}. Strikingly, these all have intransitive counterparts starting with /t/.

\begin{exe}
	\ex 
	\begin{xlist}
		\ex \textit{merengguen} `to heap up'
		\ex \textit{tengguen} `to gather'
	\end{xlist}
	\label{exe:tengguen}
	\ex 
	\begin{xlist}
		\ex \textit{melebor} `to get rid of; to move aside'
		\ex \textit{telebor} `to fall off; to fall off and move aside'
	\end{xlist}
	\ex 
	\begin{xlist}
		\ex \textit{meraraouk} `to break'
		\ex \textit{taraouk} `to be broken'
	\end{xlist}	
	\label{exe:taraouk}
\end{exe}

Compare also some of the verbs in \textit{-uk} in §\ref{sec:uk}, which contain \textit{ma-} or \textit{na-} and an element \textit{-uk} (roughly `out') and could be causativised verbs. This suggests that \textit{ma=} (and perhaps \textit{na=}) are old Kalamang elements that were productive transitivisers or causativisers, but which have lost their productivity.

%ma- can be combined with nau-: naumasemdem = baku kasih takut orang. but weird form: nau-ma-sem-den-gap, so don't discuss (no other ex's)
%cf TAP -(na)ng, V-, a-.
%lalat/nalat too?
\is{causative|)}

\section{Plural number}\is{plural!verbal}
\label{sec:verbdistr}
Number is not normally inflected on verbs. There is one exception: the plural imperative. Plural \is{imperative}imperative forms \textit{=tar} and \textit{=r} are described in §\ref{sec:imp}. 

The suffix \textit{-p} was attested on reduplicated directional verbs, and is possibly a \is{distributive}distributive or pluractional marker.\is{reduplication!verbs} For a further description, see §\ref{sec:distribv}.

\section{Fossilised morphology}
\label{sec:foss}
Kalamang has two remnants of what has been either productive or borrowed verbal morphology: a prefix \textit{na-} on loan verbs from Austronesian\is{Austronesian!loan} languages, and a morpheme \textit{-uk}, which is synchronically found on verbs denoting movement along an axis, the meeting of entities, pulling, and snapping.

\subsection{Austronesian loan verbs}\is{Austronesian!loan}
\label{sec:naloanverb}
Loan verbs often have a first syllable \textit{na-}. It is likely that \textit{na-} is an Austronesian\is{Austronesian!loan} morpheme that was borrowed together with the verbs, probably a third-person singular marker. The source language for the borrowings is unclear. Sometimes they appear to be Austronesian languages of the region, a few examples of which are given in Table~\ref{tab:naverbs}. This is not to suggest that e.g. \textit{naloli} `to mince' is a direct borrowing from \ili{Yamdena} or \ili{Fordata}, which are spoken relatively far away from Karas, in the south Moluccas, but for those languages large vocabularies are available. Other loans are more likely to be borrowed from \ili{Papuan Malay} or Indonesian, such as \textit{namenyasal} `to be sorry; to regret' (in the lower part of Table~\ref{tab:naverbs}), because it carries \ili{Indonesian} prefix \textit{meN-}.\il{Uruangnirin}\il{Geser-Gorom}\is{Moluccas}

\begin{table}[ht]
	\caption[Loan verbs]{Loan verbs with \textit{na-} and comparable verbs in other languages. U = Uruangnirin [urn], \textcite{uruangnirinLR}; G = Geser-Gorom [ges], \textcite{goromLR}; Y = Yamdena [jmd], \textcite{yamdena}; F = Fordata [frd], \textcite{fordata}.}
	\label{tab:naverbs}
	

		\begin{tabular}{ll} 
			\lsptoprule
			Kalamang & compare\\\midrule 
			\textit{nafafat} `to slap'& U \textit{afafat} `to slap'\\
			\textit{nafikir, napikir} `to think'& G \textit{hiʔir}, U \textit{pikpikir}\\
			\textit{naloli} `to mince'& Y and F \textit{n-loli} `fijnstampen' [`to mash']\\
			\textit{namot} `to block' & F \textit{n-motak} `verstopt zitten' [`to be clogged']\\
			\textit{natewa} `to hit'& Y \textit{n-tebak} `steken, prikken' [`to stab']\\
			\textit{narekin} `to count'& U \textit{mirekiŋ}, G \textit{reʔin}, ultimately from\\
			& Dutch \textit{rekenen} `to count'\\ %\arrayrulecolor{gray}	\hline
			\textit{namenyasal} `to regret' & Indonesian \textit{menyesal} `to regret'\\
			\lspbottomrule 
		\end{tabular}
	
\end{table}

As appears from the examples above, \textit{na-} is not a productive morpheme: newer \ili{Malay} loans do not carry it. When asked to translate sentences with very new loans such as \textit{cas} `to charge', \textit{WA} `to WhatsApp', \textit{nonton} `to watch (television)' and \textit{SMS} `to SMS', \textit{na-} was not used to mark these. The natural spoken corpus is also full of unmarked Malay loans, such as \textit{cat} `to paint', \textit{campur} `to mix', \textit{pariksa} `to check', \textit{rekam} `to record' and \textit{tulun} `to help' (<\textit{tolong}). An example of a Malay borrowed verb with \textit{na-}, \textit{nafikir} `to think', is given in~(\ref{exe:naverb1}), while an example with \textit{telpon} `to telephone', which is borrowed as is, is given in~(\ref{exe:ompet}).

\begin{exe}
	\ex
	\label{exe:naverb1}
	{\gll jadi waktu kon an-autak melelu me \textbf{nafikir}\\
		so time one \textsc{1sg}-alone sit {\glme} think\\
		\glt `So when I was sitting alone I was thinking.' \jambox*{\href{http://hdl.handle.net/10050/00-0000-0000-0004-1BB0-D}{[stim12\_7:08]}}}
	
	\ex
	\label{exe:ompet}
	{\gll Om Pet=a gen se \textbf{telpon}=i Unyil emun mu=konggo\\
		uncle Pet=\textsc{foc} maybe {\glse} telephone={\gli} Unyil mother.\textsc{3poss} \textsc{3pl=an.loc}\\
		\glt `Uncle Pet maybe already telephoned Unyil's mother's.' \jambox*{\href{http://hdl.handle.net/10050/00-0000-0000-0004-1BC3-B}{[conv7\_1:06]}}}
\end{exe}


\subsection{Verbs in \textit{-uk}}
\label{sec:uk}
A small group of verbs, around twenty, end in \textit{-uk} (sometimes \textit{-ouk}) and have semantics related to movement on an axis (typically in/out or from/towards the deictic centre), the meeting of two entities or forces, or pulling and breaking or snapping (usually causative). These verbs do not behave differently from other verbs, and \textit{-uk} is not a productive morpheme. An exhaustive list is given in Table~\ref{tab:uk}.

\begin{table}[ht]
	\caption{Verbs in \textit{-uk} and their semantic categorisation}
	\label{tab:uk}
\fittable{
		\begin{tabular}{lll} 
			\lsptoprule	
			\textit{duk} & `walk into; be hit' & meeting of forces\\
			\textit{eiruk} & `squat, bow, bend down' & movement down\\
			\textit{emguk} & `vomit' & movement out\\
			\textit{komasasuk} & `close w. lid or tap' & meeting of entities?\\
			\textit{koluk} & `find' & meeting of entities\\
			\textit{lauk} & `exit; protrude; appear' & movement out\\
			\textit{luk} & `come' & movement towards deictic centre\\
			\textit{malaouk} & `turn over' & movement around axis\\
			\textit{maouk} & `spit out' & movement out\\
			\textit{meraraouk} & `cause to snap' & pull and snap/break\\
			\textit{muk} & `throw fishing line' & movement away from deictic centre\\
			\textit{mukmuk} & `rock tree to harvest' & shake and snap/break\\
			\textit{nadeduk} & `pull' & pulling\\
			\textit{namasuk} & `give back' & movement towards deictic centre \\
			\textit{nauleluk} & `meet' & meeting of entities\\
			\textit{nasuk} & `go backwards' & movement backwards\\
			\textit{rouk} & `fall over (of tree)' & snapping/breaking, movement down\\
			\textit{saouk} & `emerge from water' & movement out\\
			\textit{taluk} & `exit' & movement out\\
			\textit{taouk} & `lie down' & movement down\\
			\textit{taraouk} & `snap' & snapping/breaking\\
			\lspbottomrule
		\end{tabular}
		}
\end{table}

Some of these words likely have valency-changing morphology, such as applicative \textit{ko=} (\textit{koluk}, \textit{komasasuk}), \is{reciprocal}reciprocal \textit{nau=} (\textit{nauleluk}) or causative \textit{ma=} or \textit{na=} (and in one case \textit{me-}, compare \textit{meraraouk} `cause to snap' and \textit{taraouk} `snap'). In the case of \textit{rouk}, which is used to describe the falling over of a tree, there could be the remnant of \textit{ror} `tree', followed by \textit{-uk}.
