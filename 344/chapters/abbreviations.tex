\addchap{\lsAbbreviationsTitle}
% \addchap{Abbreviations and symbols}


\section*{Interlinear glossing}
\label{sec:gloss}
%\todo{can't refer to this section for some reason}
Throughout this work, I adhere to the Leipzig Glossing Rules \citep{comrie2008}. Additional abbreviations used are:

\begin{tabbing}
	\hspace*{2cm}\=\hspace*{8cm}\= \kill
	\textsc{an.lat} \> animate lative\\
	\textsc{an.loc} \> animate locative\\
	\textsc{ana} \> anaphoric demonstrative\\
	%appl
	\textsc{apprh} \> apprehensive\\
	\textsc{at} \> attributive\\
	\textsc{atten} \> attenuative\\	
	%ben
	%caus
	%clf
	%com
	%dem
	%dist
	%distr
	%du
	\textsc{down} \> elevational `down'\\
	\textsc{emph} \> emphatic\\
	\textsc{enc} \> interjection of (annoyed) encouragement\\
	\textsc{exist} \> existential\\
	\textsc{fdist} \> far distal\\
	\textsc{fil} \> filler\\
	%\textsc{foc} = focus \\
	\textsc{hes} \> hesitation\\
	\textsc{iam} \> iamitive (`already')\\
	%\textsc{imp} = imperative\\
	%\textsc{ins} = instrumental\\
	\textsc{int} \> interjection\\
	\textsc{int.e} \> interjection of the form \textit{e}\\
	\textsc{int.pej} \> interjection expressing contempt or dissatisfaction\\		
	%\textsc{vol} = irrealis\\
	\textsc{ints} \> intensive \\
	\textsc{iter} \> iterative \\
	\textsc{lat} \> lative (combined ablative and allative)\\
	%		\textsc{loc} = locative \\
	%		\textsc{neg} = negative/negation \\
	\textsc{n} \> unanalysed phoneme \textit{-n} (§\ref{sec:problems})\\
	\textsc{neg\_exist} \> negative existential\\
	\textsc{nfin} \> non-final\\
	%		\textsc{nmlz} = nominaliser\\
	%obj
	\textsc{objqnt} \> object quantifier (quantifier modifying an object)\\
	\textsc{pain} \> interjection expressing pain\\
	\textsc{ph} \> placeholder\\
	\textsc{plnk} \> predicate linker \\			
	%		\textsc{pl} = plural \\
	%		\textsc{poss} = possessive \\
	%		\textsc{proh} = prohibitive\\
	% prog
	%		\textsc{prox} = proximal \\
	\textsc{q} \> question word root \textit{tama}\\
	% quot
	%	\textsc{recp} = reciprocal \\
	\textsc{red} \> reduplication\\
	% refl		
	%		\textsc{rel} = relativiser \\
	%		\textsc{sg} = singular \\
	\textsc{sim} \> similative \\
	\textsc{surpr}	\> interjection of surprise\\
	\textsc{t} \> unanalysed phoneme \textit{-t} (§\ref{sec:problems})\\
	\textsc{tag} \> confirmation-seeking interjection\\
	\textsc{ten} \> unanalysed morpheme \textit{=ten} (§\ref{sec:tenunal})\\
	%\textsc{top} = topic\\
	\textsc{vol} \> volitional \\
	\textsc{up} \> elevational `up'
\end{tabbing}

\section*{Languages}
\begin{tabbing}
	\hspace*{1.5cm}\=\hspace*{8cm}\= \kill
	AN \> Austronesian\\
	PM \> Papuan Malay\\
	PMP \> Proto-Malayo-Polynesian
\end{tabbing}

\section*{Intonation}
\begin{tabbing}
	\hspace*{1.5cm}\=\hspace*{8cm}\= \kill
	H \> high tonal target\\
	L \> low tonal target\\
	{*} \> pitch accent\\
	| \> division between breath groups\\
	\% \> boundary tone
\end{tabbing}

\section*{Kinship}
\begin{tabbing}
	\hspace*{1.5cm}\=\hspace*{8cm}\= \kill
	B \> brother\\
	D \> daughter\\
	e \> elder\\
	F \> father\\
	H \> husband\\
	M \> mother\\
	S \> son\\
	W \> wife\\
	y \> younger\\
	Z \> sister
\end{tabbing}

\section*{Other abbreviations}
\begin{tabbing}
	\hspace*{1.5cm}\=\hspace*{8cm}\= \kill
	A \> agent-like argument of a transitive predicate\\
	k.o. \> kind of\\
	NP \> noun phrase\\
	O/Obj \> object\\
	P \> patient-like argument of a transitive predicate\\
	Pred \> Predicate\\
	R/Recp \> recipient\\
	S/Subj \> subject\\
	T \> theme\\
	V\> verb
\end{tabbing}