\chapter{Complex predicates}\is{predicate!complex|(}
\label{ch:svc}

In Kalamang, complex predicates are monoclausal predicates with more than one verb or verb-like element with a shared argument. Verbs in complex predicates need not be contiguous: the verbs may be separated by an object\is{object} and an indirect object. They are of four morphosyntactic types, which are described in order of frequency: complex predicates connected by predicate linker \textit{=i} (§\ref{sec:mvci}), complex predicates with one dependent verb (§\ref{sec:mvcdep}), complex source, goal and location predicates (§\ref{sec:mvcgoal}) and serial verb constructions (§\ref{sec:svc}). Serial verb constructions (SVCs) differ from the other types in that they are not linked by a predicate linker and that the verbs are independent That is, they can be the single predicate in a clause, and they can be inflected by all mood, aspect and modality markers available in Kalamang. This is a relatively minor type in Kalamang. Complex predicates are distinct from the clause-combining strategies described in §\ref{sec:clacomb}.

All elements in complex predicates, being part of the same clause, have the same mood\is{mood}, aspect\is{aspect} and polarity\is{polarity}, which is only marked on the last verb. In other words, there can only be one instance of the same mood, aspect or modality marking per clause. An example of shared mood with the plural imperative is given in~(\ref{exe:jieborar}), contrasted with a biclausal example in~(\ref{exe:jierarku}). An example with a complex predicate with predicate linker \textit{=i} is given in~(\ref{exe:cimei}).

\begin{exe}
	\ex 
	\begin{xlist}
	\ex \gll jie bo=\textbf{tar}\\
	get go=\textsc{pl.imp}\\
	\glt `Go get!' \jambox*{\href{http://hdl.handle.net/10050/00-0000-0000-0004-1B9F-F}{[conv9\_22:57]}}
	\label{exe:jieborar}
	\ex \gll jie=\textbf{tar} kuet=\textbf{tar}\\
	get=\textsc{pl.imp} bring=\textsc{pl.imp}\\
	\glt `Get and bring!' \jambox*{\href{http://hdl.handle.net/10050/00-0000-0000-0004-1B9F-F}{[conv9\_22:58]}}
	\label{exe:jierarku}
	\end{xlist}
	\ex \gll mena ka nasuarik=i \textbf{barei}\\
	later \textsc{2sg} tuck={\gli} descend.\textsc{imp}\\
	\glt `Then you tuck [it] down!' \jambox*{\href{http://hdl.handle.net/10050/00-0000-0000-0004-1BB5-B}{[conv17\_40:20]}}
	\label{exe:cimei}
\end{exe}
%exe:cimei is shared patient, not shared subject or object

%evt corps ex parair paruak, contrast with
% 	\ex 
%{ror me parair=te paruak=te\\
%	tree \textsc{top} split=\textsc{imp} throw.away=\textsc{imp}\\
%	`Split the wood and throw it away!' \jambox*{[elic\_svc]}}

Also in non-contiguous predicates, mood, aspect and negation\is{negation} can only be expressed once. The \is{imperative}imperative marker in~(\ref{exe:yuatnar}) has scope over the entire predicate.

\begin{exe}
	\ex \gll melelu wele yuwa=at narari=\textbf{te}\\
	sit vegetables \textsc{prox=obj} slice=\textsc{imp}\\
	\glt `Sit and slice these vegetables!' \jambox*{\href{http://hdl.handle.net/10050/00-0000-0000-0004-1C60-A}{[elic\_svc\_33]}}
	\label{exe:yuatnar}
\end{exe}	

Non-verbal predicates like NPs marked with a lative or locative postposition can also be part of complex predicates, as is shown by the \is{prohibitive}prohibitive enclitic \textit{=in} on the locative \textit{pasierko} `in the sea', which has scope over a predicate which also contains the verb \textit{gareor} `to dump'.

\ea \label{exe:koin} {\gll an toni sor-kang me ki-mun \textbf{gareor=i} \textbf{pasier=ko=in} eh\\
	\textsc{1sg} say fish-bone {\glme} \textsc{2pl-proh} dump={\gli} sea=\textsc{loc=proh} \textsc{tag}\\
	\glt `I said: ``Those fish bones, hey, don't you guys dump [them] in the sea!''' \jambox*{\href{http://hdl.handle.net/10050/00-0000-0000-0004-1BA3-3}{[conv10\_14:05]}}
}	
\z  

Shared aspect is illustrated here with progressive \textit{=teba}. (\ref{exe:mggt}) is a complex predicate; (\ref{exe:mtggt}) contains two clauses with two predicates.

\begin{exe}
	\ex
	\begin{xlist}
		\ex \gll emumur mambaran garung∼garung=\textbf{teba}\\
		woman.\textsc{pl} stand chat∼\textsc{prog}={\glteba}\\
		\glt `The women stand chatting.' \jambox*{\href{http://hdl.handle.net/10050/00-0000-0000-0004-1C60-A}{[elic\_svc\_37]}}
		\label{exe:mggt}
		\ex \gll emumur mambara=\textbf{teba} garung∼garung=\textbf{teba}\\
		woman.\textsc{pl} stand={\glteba} chat∼\textsc{prog}={\glteba}\\
		\glt `The women are standing and [they] are chatting.' \jambox*{\href{http://hdl.handle.net/10050/00-0000-0000-0004-1C60-A}{[elic\_svc\_39]}}
		\label{exe:mtggt}
	\end{xlist}
\end{exe}

\hspace*{-5pt}Finally, consider an example of a negated predicate, where the negator is marked on the second verb \textit{masara} `move towards land' in~(\ref{exe:kurumaratnin}).

\begin{exe}
	\ex \gll kalau mat \textbf{kuru} \textbf{masarat=nin=et} me pi barat=nin\\
	if \textsc{3sg.obj} bring move\_landwards=\textsc{neg}={\glet} {\glme} \textsc{1pl.incl} descend=\textsc{neg}\\
	\glt `If [they] don't bring him towards land, we don't go down.' \jambox*{\href{http://hdl.handle.net/10050/00-0000-0000-0004-1BC3-B}{[conv7\_4:22]}}
	\label{exe:kurumaratnin}
\end{exe}


\section{Complex predicates with predicate linker \textit{=i}}
\label{sec:mvci}
The most common type of complex predicate is made with predicate linker \textit{=i} on all but the last word in the construction. They are monoclausal constructions with more than one independent verb or verb-like element and at least one shared argument. With the exception of `until'-constructions with \textit{bo} `to go' (§\ref{sec:mvcuntil}), no arguments can come between the elements in these predicates. I present different semantic types roughly in order of frequency of occurrence.

The use of predicate linker \textit{=i} is ungrammatical with complex predicates with dependent verbs (§\ref{sec:mvcdep}), directional verbs (§\ref{sec:mvcgoal}) and \textit{bo} `to go' (§\ref{sec:svcbo}) as the first verb.

% barani kasur opa yuwabon
%irargawi kajien

\subsection{Aspectual serialisation with \textit{koyet} `to be finished'}\is{aspect}
\label{sec:ikoyetsvc}
Completive aspect, discussed in greater detail in~§\ref{sec:compl}, is made with help of the verb \textit{koyet} `to be finished'. This is an independent verb, as illustrated in~(\ref{exe:kaitok}).

\begin{exe}
    \ex \gll kai tok \textbf{koyet}=nin\\
    firewood yet finish=\textsc{neg}\\
    \glt `The firewood isn't finished yet.' \jambox*{\href{http://hdl.handle.net/10050/00-0000-0000-0004-1BC1-0}{[narr19\_6:38]}}
    \label{exe:kaitok}
\end{exe}

Completive aspect is expressed with the construction Verb\textit{=i} \textit{koyet}, where \textit{=i} is attached to a matrix verb of any kind. At the same time as being a complex predicate, this construction sequentially links the state, event or action in the first clause (which has to be completed) to a state, event or action in the next clause (which was started after completion of the first). Multiple monoclausal Verb\textit{=i koyet} constructions may be strung together in this way. There are no restrictions on the first verb in the predicate. Both verbs in the predicate share the same arguments. (\ref{exe:ewunat}) shows the construction with a transitive verb in a narrative about making a canoe from a tree trunk. (\ref{exe:yoriko}) is taken from the end of that narrative, and shows the construction on an intransitive verb. Although it is uncommon, even the verb \textit{koyet} `to be finished' may be used as the matrix verb in this construction, as in~(\ref{exe:koyeri}).

\begin{exe}
	\ex \gll ewun=at potman=\textbf{i} \textbf{koyet} koi tim-un=at potma\\
	stem=\textsc{obj} cut={\gli} finish then tip-\textsc{3poss=obj} cut\\
	\glt `After cutting the stem, [I] cut its tips.' \jambox*{\href{http://hdl.handle.net/10050/00-0000-0000-0004-1BDD-5}{[narr42\_0:17]}}
	\label{exe:ewunat}
	\ex \gll ma yor=\textbf{i} \textbf{koyet} {\ob}...{\cb}\\
	\textsc{3sg} right={\gli} finish\\
	\glt `After it is right, {\ob}...{\cb}.' \jambox*{\href{http://hdl.handle.net/10050/00-0000-0000-0004-1BDD-5}{[narr42\_15:00]}}
	\label{exe:yoriko}
	\ex \gll koyet=\textbf{i} \textbf{koyet} kawarman\\
	finish={\gli} finish fold\\
	\glt `After finishing, fold.' \jambox*{\href{http://hdl.handle.net/10050/00-0000-0000-0004-1BB8-C}{[narr11\_2:29]}}
	\label{exe:koyeri}
\end{exe} 	

It is also possible, though not often employed, to make aspectual serialisation with two verbs before \textit{koyet} `to be finished', as in~(\ref{exe:tena}). There is not enough data to determine the relationship between the two verbs marked with \textit{=i}.

\begin{exe}
	\ex \gll tena-un=at \textbf{tawaran=i} \textbf{manyor=i} \textbf{koyet} {\ob}...{\cb}\\
	keel-\textsc{3poss=obj} chop={\gli} adjust={\gli} finish {\ob}...{\cb}\\
	\glt `After chopping the keel straight, {\ob}...{\cb}.' \jambox*{\href{http://hdl.handle.net/10050/00-0000-0000-0004-1BDD-5}{[narr42\_3:01]}}
	\label{exe:tena}
\end{exe}	
%otherdouble: paruoni potmani, kanggarani kowarani, lolotuni jieni koyet, ramini sarani

Lastly, aspectual serialisation is also used with locatives, which are NPs carrying locative postposition\is{postposition!locative} \textit{=ko} functioning as predicates (§\ref{sec:loc}).

\begin{exe}
	\ex \gll os=at \textbf{di=timbang-un=ko=i} \textbf{koyet} {\ob}...{\cb}\\
	sand=\textsc{obj} \textsc{caus}=forehead-\textsc{3poss=loc}={\gli} finish {\ob}...{\cb}\\
	\glt `After putting sand on her forehead, {\ob}...{\cb}.' \jambox*{\href{http://hdl.handle.net/10050/00-0000-0000-0004-1BDC-D}{[conv8\_3:02]}}
	\label{exe:osatd}
\end{exe}

Because this construction has a clause-linking function, it cannot be modified for other moods, aspects or modes. It cannot be negated. The construction also has related properties as a \is{quantifier}quantifier, meaning `all; until finished' (§\ref{sec:quantall}).\footnote{The relation between aspect and universal quantification is discussed for Timor-Alor languages in \textcite{huber2014}. See also \citet[][341]{unterladstetter2020}: ``Completives differ from finish semantics in that the endpoint of the event is not reached by some actor wilfully ending it, but because a totality of referents is affected''. In §\ref{sec:compl}, I argue that Kalamang \textit{=i koyet} does both.}

%Seems to be parallel to Bunaq haqal, see Schapper's thesis pdf 492 e.v.

%\textcite{unterladstetter2020} type: MOD tense-aspect. V1 matrix verb + V2 modifier verb or vv. (Kalamang: vv, which is iconic, cf. \cite[][340]{unterladstetter2020}, who shows that in EI languages with `finish' complex predicates the modifier verb follows the matrix verb, whereas with `begin' complex predicates the modifier verb precedes the matrix verb.)


\subsection{Motion}\is{motion}
\label{sec:svcmanneri}
Complex motion predicates have a motion verb as the second verb, and a manner or other verb as the first verb. The second verb is very commonly a directional verb (§\ref{sec:dir}). The verbs share all arguments.

(\ref{exe:tumkar}) illustrates a manner and directional verb, (\ref{exe:yalipa}) illustrates a manner and other motion verb, and~(\ref{exe:domanis}) shows the verb \textit{dorma} `to pull out' with a directional verb.

\begin{exe} 
	\ex \gll tumtum karuok \textbf{marmar=i} \textbf{mia}\\
	children three walk={\gli} come\\
	\glt `Three children come walking.' \jambox*{\href{http://hdl.handle.net/10050/00-0000-0000-0004-1BD1-D}{[stim31\_1:48]}}
	\label{exe:tumkar}
	\ex \gll setela ma yie=te an=a mat \textbf{yal=i} \textbf{parei∼pareir}\\
	after \textsc{3sg} swim={\glte} \textsc{1sg=foc} \textsc{3sg.obj} paddle={\gli} \textsc{distr}∼follow\\
	\glt `After he [started] swimming I followed him paddling.' \jambox*{\href{http://hdl.handle.net/10050/00-0000-0000-0004-1BAE-4}{[narr44\_1:28]}}
	\label{exe:yalipa}
	\ex \gll ar-un wa-rip ye \textbf{dorman=i} \textbf{sara}\\
	stem-\textsc{3poss} \textsc{prox-qlt} or pull\_out={\gli} ascend\\
	\glt `Pull up [from the soil] a stem about as big as this.' \jambox*{\href{http://hdl.handle.net/10050/00-0000-0000-0004-1BE4-8}{[narr31\_3:22]}}
	\label{exe:domanis}
\end{exe}	

These constructions are also very common with \textit{(y)ecie} `to return', making complex centrifugal motion constructions, as illustrated in~(\ref{exe:danik}).

\begin{exe}
	\ex \gll mat dan=i koyet se \textbf{ecien=i} \textbf{sara}\\
	\textsc{3sg.obj} bury={\gli} finish {\glse} return={\gli} ascend\\
	\glt `After burying him, [we] went back up.' \jambox*{\href{http://hdl.handle.net/10050/00-0000-0000-0004-1BCF-3}{[narr2\_0:34]}}
	\label{exe:danik}
\end{exe}	

As with aspectual serialisation (§\ref{sec:ikoyetsvc}), other complex predicates with predicate linker \textit{=i} may also contain more than one verb. For example, in the manner + direction construction in~(\ref{exe:sirnag}), both manner verbs are marked with \textit{=i}. Again, there is not enough data in the corpus to determine patterns in the (semantic) relationship between the verbs in the construction.

\begin{exe}
	\ex \gll nasirang=i \textbf{mon∼mon=tun=i} \textbf{ran}\\
	pour={\gli} quick∼\textsc{ints=ints}={\gli} go\\
	\glt `Go pour quickly.' \jambox*{\href{http://hdl.handle.net/10050/00-0000-0000-0004-1BA3-3}{[conv10\_14:37]}}
	\label{exe:sirnag}
\end{exe}
%also double: marmari talewisi bot, deiri terusi, per se nabestai talui sara o

Nouns carrying \textit{-pis} `side' (§\ref{sec:inal}) are also inflected with \textit{=i} when they modify a verb.

\begin{exe}
	\ex \gll kon se marmar=i \textbf{talep-pis=i} \textbf{bot}\\
	one {\glse} walk={\gli} outside-side={\gli} go\\
	\glt `One is walking outside.' \jambox*{\href{http://hdl.handle.net/10050/00-0000-0000-0004-1BAB-6}{[stim14\_2:33]}}
	\label{exe:talewis}
\end{exe}

This is also the case for NPs carrying similative postposition\is{postposition!similative} \textit{=kap}, like \textit{ododa} `gado-gado' (a dish) in~(\ref{exe:ododang}). Most colour terms end in \textit{-gap} or \textit{-kap} and are probably derived from nouns marked with similative \textit{=kap} (§\ref{sec:sim}), but are treated as monomorphemic. They can also be inflected with \textit{=i} and become part of a complex predicate, like \textit{baranggap} `yellow' in~(\ref{exe:gawi}).

\begin{exe}
	\ex \gll ododa=kap=i paruot=et\\
	gado\_gado=\textsc{sim}={\gli} make={\glet}\\
	\glt `Make [it] like gado-gado.' \jambox*{\href{http://hdl.handle.net/10050/00-0000-0000-0004-1BA5-0}{[conv15\_5:20]}}
	\label{exe:ododang}
	\ex \gll wa me paden taikon \textbf{baranggap=i} \textbf{saran}\\
	\textsc{prox} {\glme} pole one\_side yellow={\gli} ascend\\
	\glt `This one, a yellow pole goes down on one side.' \jambox*{\href{http://hdl.handle.net/10050/00-0000-0000-0004-1BE3-6}{[stim39\_1:27]}}
	\label{exe:gawi}
\end{exe}	

%also =tun, =tak

Three manner adverbials, described in §\ref{sec:manner}, end in /i/, but it is not clear at this point whether this is predicate linker \textit{=i}.


\subsection{Action and result}
Complex action and result predicates express an action in the first verb and a result in the second verb. Like resultative SVCs, §\ref{sec:ressvc}, they are rare. One natural spoken corpus example is~(\ref{exe:kakat}) with the verbs \textit{komain} `to stab' and \textit{rua} `to kill'. The object of the first verb is the subject of the second, such that the verbs in the construction do not share any arguments. In~(\ref{exe:dakcamp}), it is unclear exactly whether \textit{campur} `to mix' is the result obtained by chopping two ingredients and the same time, or whether this should be read as two sequential events. The same conversation also contains the construction \textit{kawareni campur} `grate mix'.

\begin{exe}
	\ex \gll o mu se kaka=at \textbf{komain=i} \textbf{rua}\\
	\textsc{int} \textsc{3pl} {\glse} older\_sibling=\textsc{obj} stab={\gli} kill\\
	\glt `O, they killed the older brother (by stabbing).' \jambox*{\href{http://hdl.handle.net/10050/00-0000-0000-0004-1BBC-4}{[narr24\_3:51]}}
	\label{exe:kakat}
	\ex \gll mier-gan=at paruo \textbf{dakdak=i} \textbf{campur}\\
	\textsc{3du}-all=\textsc{obj} make chop={\gli} mix\\
	\glt `Mix both of them (by chopping).' \jambox*{\href{http://hdl.handle.net/10050/00-0000-0000-0004-1BCA-4}{[conv20\_13:24]}}
	\label{exe:dakcamp}
\end{exe}
%also kowareni campur

A few other examples of action-result constructions with predicate linker \textit{=i} were elicited with help of the cut and break clips \parencite{cutbreak2001}. Here, the resultative function of the constructions is clear. Consider~(\ref{exe:donsl}) and~(\ref{exe:pueip}).

\begin{exe}
	\ex \gll ma karek=at \textbf{ramien=i} \textbf{meraraouk}\\
	\textsc{3sg} string=\textsc{obj} pull={\gli} cause\_to\_snap\\
	\glt `He causes the string to snap (by pulling it).' \jambox*{\href{http://hdl.handle.net/10050/00-0000-0000-0004-1C60-A}{[elic\_cut\_28]}}
	\label{exe:donsl}
	\ex \gll pue=i parair\\
	hit={\gli} break\\
	\glt `[Of a clip where a pot is smashed with a hammer:] break (by hitting).' \jambox*{\href{http://hdl.handle.net/10050/00-0000-0000-0004-1C60-A}{[elic\_cut\_39]}}
	\label{exe:pueip}
\end{exe}
% also puei parair, nasedui/ramini meraraouk.
%\textcite{unterladstetter2020} type: SREL resultative. V1 causing verb + V2 stative resultant verb. \ref{exe:donsl} does not conform.

\subsection{Durative and `until'-constructions}\is{durative}\is{until}
\label{sec:mvcuntil}
Reduplicated verbs marked with predicate linker \textit{=i} occur in two contexts. The first is with reduplicated verbs that indicate durativity, followed by a construction with `until' which indicates the result of the action. The `until'-construction (with help of \textit{bo} `to go') may (example~\ref{exe:kadok}) or may not (example~\ref{exe:kuarkuar}) be made in a separate clause. The second is in constructions with verbs that indicate durativity, but without an explicitly linked action (example~\ref{exe:rawiraw}), or with an implied `until'-construction (example~\ref{exe:komainik}). The verbs are typically lengthened (§\ref{sec:lenght}), as indicated in~(\ref{exe:rawiraw}).

\begin{exe}
	\ex \gll kit-kadok=at \textbf{paruon=i} \textbf{paruon=i} bo koyet\\
	top-side=\textsc{obj} make={\gli} make={\gli} go finish\\
	\glt `[We] made the top-side until [it was] finished.' \jambox*{\href{http://hdl.handle.net/10050/00-0000-0000-0004-1BB3-0}{[narr7\_2:45]}}
	\label{exe:kadok}
	\ex \gll \textbf{kuar=i} \textbf{kuar=i} ma se bo ruon\\
	cook={\gli} cook={\gli} \textsc{3sg} {\glse} go cooked\\
	\glt `Cooking, cooking, until it's cooked.' \jambox*{\href{http://hdl.handle.net/10050/00-0000-0000-0004-1C99-E}{[narr8\_2:10]}}
	\label{exe:kuarkuar}
	\ex \gll mier se mu=at \textbf{komain=i} \textbf{komain=i} kon se tur\\
	\textsc{3du} {\glse} \textsc{3pl=obj} stab={\gli} stab={\gli} one {\glse} fall\\
	\glt `They stabbed and stabbed [until] one fell.' \jambox*{\href{http://hdl.handle.net/10050/00-0000-0000-0004-1BDB-C}{[narr28\_9:10]}}
	\label{exe:komainik}
	\ex \glll inier se melalu \textbf{raːwi} \textbf{raːwi}\\
	inier se melelu rap=i rap=i\\
	\textsc{1du.ex} {\glse} sit laugh={\gli} laugh={\gli}\\
	\glt `We sat laughing, laughing.' \jambox*{\href{http://hdl.handle.net/10050/00-0000-0000-0004-1BAE-4}{[narr44\_22:46]}}
	\label{exe:rawiraw}
\end{exe}	
%also tui tui tui ma bo palat. ma he wuongi wuongi eh. narawi narawi. paruoni paruoni bo koyet. iskawi iskawi iskawi se bo.. koyet. an se koi yal yal yal, tebolsuban, wari wari eh sor natnin. 

The `until'-constructions with \textit{bo} `to go' illustrated in~(\ref{exe:kadok}) and~(\ref{exe:kuarkuar}) are also often found with distal manner \is{demonstrative!manner}demonstrative \textit{mindi} and another verb, marked with predicate linker \textit{=i}. The order is Verb=\textsc{plnk} + `like that' + `go' (example~\ref{exe:karuarr}). The construction can be elaborated with \textit{mendak}, a demonstrative form probably derived from distal demonstrative \textit{me} and clitic \textit{=tak} `just', meaning `just like that' (§\ref{sec:distmann}). The `until'-construction then takes the form \textit{mendak=i} + \textit{mindi} + \textit{bo} (example~\ref{exe:nasuis}). See §\ref{sec:distmann} for a discussion of the uses of the distal manner demonstrative \textit{mindi}.

\begin{exe}
	\ex \gll \textbf{karuar=i} \textbf{mindi} \textbf{bo} kararak koi masan\\
	smoke\_dry={\gli} like\_that go dry then dry\_in\_sun\\
	\glt `We dry [on a rack above the fire] until it's dry, then we dry in the sun.' \jambox*{\href{http://hdl.handle.net/10050/00-0000-0000-0004-1BB8-C}{[narr11\_2:50]}}
	\label{exe:karuarr}
	\ex \gll mu ko=melelu \textbf{mendak=i} \textbf{mindi} \textbf{bo} ma se nasuk=i saran\\
	\textsc{3pl} \textsc{appl}=sit just\_like\_that={\gli} like\_that go \textsc{3sg} {\glse} go\_backwards={\gli} ascend\\
	\glt `They sit on it until it [the haemorrhoids] has gone back up.' \jambox*{\href{http://hdl.handle.net/10050/00-0000-0000-0004-1BBE-E}{[narr36\_0:53]}}
	\label{exe:nasuis}
\end{exe}

The corpus also includes `until'-constructions with Austronesian loans like \textit{sampe} `until' or \textit{selama} `as long as' preceded by a verb marked with predicate linker \textit{=i}.

%\textcite{unterladstetter2020} types: reduplicated: unsure. until-constructions: SREL resultative or MOD `until' (sort of tense-aspect?) or MOD adverbial (manner)?? Cf. resultative SVC: \textit{bara pol}.

%also kuari mindi bo nani koyet, etuani mindi bo

%fixed order:
%- marmari sara
%- turi bara
%- tiri ra
%- mu yal=imara narurik
%- na-min=i baran = neck that is bent down on wooden man
%- marmari ra
%- dumuni ra
%- ramini mara
%- kosiauri mara wilao (pull towards shore)
%- arari maran (dive towards land)
%manggangi baran -- really manner?
%dalangi baran
%pararuoni mia, bara, bo ror kitko
%also w non-directional verbs:
%- undeiri luk
%- siktaktai tiri
%- borani tiri
%- paruai boet
%- borani bot
%- kaheni botnin
%- marmari botkin
%- narorari bot
% tiri bo war
%tiri bo monak-ko
%deiri bara/sara/etc is associated motion: cause + path (e.g. push while going down), same subject, transitive OF resultative = cause + path (e.g. push down). object = subject. latter is no real svc, maar was dit toch al niet.

\subsection{With give-constructions}\is{give-construction}
\label{sec:givemvc}
Give-constructions (§\ref{sec:give}) are made with a zero morpheme `give'. They may and frequently do occur without any other verb in the clause. However, they also occur in complex predicates with predicate linker \textit{=i}. The verb marked with \textit{=i} precedes the recipient. The zero morpheme `give' comes after the recipient, which makes these discontinuous complex predicates. The verbs only share their subject, and the recipient comes between the two verbs. The theme (pandanus leaf in the first example and fish in the second) is the direct object of both verbs. 

\begin{exe}
	\ex \gll naman=a padanual=at \textbf{rep=i} \textbf{ka} ∅\\
	who=\textsc{foc} pandanus=\textsc{obj} get={\gli} \textsc{2sg} give\\
	\glt `Who got pandanus [leaf] and gave it to you?' \jambox*{\href{http://hdl.handle.net/10050/00-0000-0000-0004-1BB5-B}{[conv17\_23:37]}}
	\label{exe:padarep}
	\ex \gll an toni kuru ma \textbf{yap=i} \textbf{sontum=ki} ∅\\
	\textsc{1sg} say bring move\_landwards divide={\gli} person=\textsc{ben} give\\
	\glt `I said bring it here and divide it among people.' \jambox*{\href{http://hdl.handle.net/10050/00-0000-0000-0004-1BB7-9}{[conv19\_8:42]}}
	\label{exe:yawison}
\end{exe}
%also jieni kiet

Like with other types of complex predicates with predicate linker \textit{=i}, we also find multiple verbs marked with \textit{=i} in give-constructions. Consider~(\ref{exe:ningetmet}). As for the other types, there is not enough data to systematically analyse the relationship among the \textit{=i}-marked verbs.

\begin{exe}
	\ex \gll kalau sontum nakal-un ning=et, met me kulun=at, \textbf{kawaren=i} \textbf{naramas=i} \textbf{mu} ∅=\textbf{ta} mu nan=et\\
	if person head-\textsc{3poss} ill={\glet} \textsc{dist} {\glme} skin=\textsc{obj} grate={\gli} squeeze={\gli} \textsc{3pl} give={\glta} \textsc{3pl} consume={\glet}\\
	\glt `If a person has a headache, the skin of that [kawalawalan], grate, squeeze and give [it] to them and they drink [it].' \jambox*{\href{http://hdl.handle.net/10050/00-0000-0000-0004-1BBE-E}{[narr36\_1:18]}}
	\label{exe:ningetmet}
\end{exe}

%\textcite{unterladstetter2020} types: the first two may be handling-to-placement (V1 handling verb + V2 placement verb, cf. take-give constructions in \textcite{klamer2012}) - that's why I treated them separately here. The last is JUXT sequential, just like some symmetrical SVCs?

\subsection{With become-constructions}
\label{sec:become}
The corpus contains three examples of complex predicates consisting of a noun inflected with predicate linker \textit{=i}, followed by \textit{ra} `to become'. An example with \textit{mun} `lime' is given in~(\ref{exe:munira}). The other examples are with \textit{lempuang} `island' and \textit{yar} `stone'.

\begin{exe}
	\ex \gll ma se mun=i ra\\
	\textsc{3sg} {\glse} lime={\gli} become\\
	\glt `He became a lime.' \jambox*{\href{http://hdl.handle.net/10050/00-0000-0000-0004-1BE2-C}{[narr23\_3:40]}}
	\label{exe:munira}
\end{exe}	


\section{Complex predicates with dependent verbs}\is{dependent verb}
\label{sec:mvcdep}
All four Kalamang dependent verbs occur in complex predicates. These verbs cannot be negated or inflected for e.g. aspectual and modal categories. Two of the dependent verbs (\textit{kuru} `bring' and \textit{bon} `bring') occur as the first verb in the construction, and one (\textit{eranun} `cannot; not be possible') occurs as the second verb in the construction. \textit{Toni} can occur in both positions. As the first verb, it means `want', and as the second verb, it means `say' or `think'.

\subsection{With \textit{kuru} `bring'}
\label{sec:mvckuru}
The independent verb \textit{kuet} `to bring', illustrated in~(\ref{exe:mingkuet}), is only rarely combined with other verbs into a complex predicate, and is then always marked with predicate linker \textit{=i}, as in~(\ref{exe:kueri}). It is also used in give-constructions, which have the zero morpheme `give', as in~(\ref{exe:kuerima}).

\begin{exe}
	\ex \gll ka nene ming-un yuwa=at \textbf{kuet}=et\\
	\textsc{2sg} grandmother oil-\textsc{3poss} \textsc{prox=obj} bring={\glet}\\
	\glt `You bring this oil of granny.' \jambox*{\href{http://hdl.handle.net/10050/00-0000-0000-0004-1BBD-5}{[conv12\_2:21]}}
	\label{exe:mingkuet}
	\ex \gll bolon opa me tok \textbf{kuet=i} ran\\
	little {\glopa} {\glme} first bring={\gli} move\\
	\glt `First bring over that little bit.' \jambox*{\href{http://hdl.handle.net/10050/00-0000-0000-0004-1B9F-F}{[conv9\_9:41]}}
	\label{exe:kueri}
	\ex \gll \textbf{kuet=i} ma ∅\\
	bring={\gli} \textsc{3sg} give\\
	\glt `Bring him.' \jambox*{\href{http://hdl.handle.net/10050/00-0000-0000-0004-1C60-A}{[elic\_i19\_4]}}
	\label{exe:kuerima}
\end{exe}

However, there is a dependent verb \textit{kuru} `bring' which occurs as the first verb in complex predicates expressing transfer and motion. \textit{Kuru} `bring' cannot be used independently, and never carries any morphology. The most common construction is with a directional verb (§\ref{sec:dir}), illustrated in~(\ref{exe:adiun}), or another verb expressing motion, such as \textit{luk} `come' in~(\ref{exe:kurluk}).

\begin{exe}
	\ex \gll ma se mara adik-un=at \textbf{kuru} \textbf{marua}\\
	\textsc{3sg} {\glse} move\_landwards younger\_sibling-\textsc{3poss=obj} bring move\_seawards\\
	\glt `He came towards land and [he] brought his brother towards sea.' \jambox*{\href{http://hdl.handle.net/10050/00-0000-0000-0004-1BDB-C}{[narr28\_12:46]}}
	\label{exe:adiun}
	\ex \gll in se \textbf{kuru} \textbf{luk} et=at\\
	\textsc{1pl.excl} {\glse} bring come canoe=\textsc{obj}\\
	\glt `We brought [it] back, the canoe.' \jambox*{\href{http://hdl.handle.net/10050/00-0000-0000-0004-1BB4-6}{[narr14\_4:42]}}
	\label{exe:kurluk}
\end{exe}

(\ref{exe:kwarin}) has three verbs: dependent verb \textit{kuru} `bring', a directional verb \textit{ra} `go' and the zero morpheme `give'. 
%There is likely some nesting in this complex predicate, where \textit{kuru ra} `go bring' is a complex predicate which is part of the complex predicate \textit{kuru ra ∅} `go bring give'. 
(\ref{exe:sorg}) is similar, except that the recipient is a NP marked with benefactive \textit{=ki}.

\begin{exe}	
	\ex \gll kawir-un=at\textsubscript{\upshape T} \textbf{kuru}\textsubscript{\upshape V1} \textbf{ra}\textsubscript{\upshape V2} ma\textsubscript{\upshape R} ∅\textsubscript{\upshape V3}\\
	hat-\textsc{3poss=obj} bring go \textsc{3sg} give\\
	\glt `[They] bring him his hat.' \jambox*{\href{http://hdl.handle.net/10050/00-0000-0000-0004-1BD2-D}{[stim30\_1:48]}}
	\label{exe:kwarin}
	\ex \gll kaling=at=a\textsubscript{\upshape T} ka \textbf{kuru}\textsubscript{\upshape V1} \textbf{marua}\textsubscript{\upshape V2} sor\textsubscript{\upshape R} ∅\textsubscript{\upshape V3}=ki\\
	fish.hook=\textsc{obj=foc} \textsc{2sg} bring move\_seawards fish give=\textsc{ben}\\
	\glt `Fish hooks, you bring them to sea to give them to the fish!' \jambox*{\href{http://hdl.handle.net/10050/00-0000-0000-0004-1BA3-3}{[conv10\_10:50]}}
	\label{exe:sorg}
\end{exe}

Otherwise, \textit{kuru} `bring' is frequently combined with \textit{bo} `go' and a location marked with the locative postposition \textit{=ko} (indicating goal, example~\ref{exe:kuruboko}), with just a location marked with the locative (also indicating goal, as in~\ref{exe:kurulaut}), or with a lative and a motion verb (indicating source, example~\ref{exe:rumasak}). See §\ref{sec:svcloclat} for more complex predicates expressing source, goal or location.

\begin{exe}
	\ex \gll ma \textbf{kuru} \textbf{bo} ror keit=\textbf{ko}\\
	\textsc{3sg} bring go tree top=\textsc{loc}\\
	\glt `He brought [it] up to the tree.' \jambox*{\href{http://hdl.handle.net/10050/00-0000-0000-0004-1C9B-8}{[stim3\_0:19]}}
	\label{exe:kuruboko}
	\ex \gll ma se an=at \textbf{kuru} laut=\textbf{ko}\\
	\textsc{3sg} {\glse} \textsc{1sg=obj} bring sea=\textsc{loc}\\
	\glt `She brought me to the sea.' \jambox*{\href{http://hdl.handle.net/10050/00-0000-0000-0004-1C98-7}{[narr26\_7:42]}}
	\label{exe:kurulaut}
	\ex \gll mu \textbf{kuru} rumasakit=\textbf{ka} \textbf{sara}\\
	\textsc{3pl} bring hospital=\textsc{lat} ascend\\
	\glt `They brought [him] up to the hospital.' \jambox*{\href{http://hdl.handle.net/10050/00-0000-0000-0004-1BC3-B}{[conv7\_3:22]}}
	\label{exe:rumasak}
\end{exe}
%probably a mistake: an di koi kuru ror kitko
%probably two clauses of zo: kuru yecieni kewengga bot
%No examples with kuru=ifound

%\textit{kuru} can be compared to TAP \textit{me}-like forms meaning `to take', which are ubiquitous in directional manner SVCs (?) in those languages \textcite{schapper2014}. Usually, the take-verb is used to realise themes of transfer and locomotion \parencite{klamer2012}. Kalamang is comparable to Sawila, where the take-verb describes the transfer/motion itself. E.g. TAP-volume p430 has an example that is parallel to kuru bara.

%\textcite{unterladstetter2020} type: CREL transport complex. V1 transport + V2 motion.


\subsection{With \textit{bon} `bring'}
\label{sec:mvcbon}
The dependent verb \textit{bon} `bring' (possibly related to comitative postposition\is{postposition!comitative} \textit{=bon}, §\ref{sec:comi}) occurs as the first verb in a complex predicate in combination with a motion verb like \textit{tiri} `to run' (example~\ref{exe:borus}), \textit{rep} `to get', \textit{marmar} `to walk', \textit{bo} `to go', \textit{bara} `to descend' (example~\ref{exe:bonbara}) or \textit{sara} `to ascend'.

\begin{exe}
	\ex \gll tumun opa me sara bo rusa suor-un keit-un=ko ma mat \textbf{bon} \textbf{tiri}\\
    child {\glopa} {\glme} ascend go deer antler-\textsc{3poss} top-\textsc{3poss=loc} \textsc{3sg} \textsc{3sg.obj} bring run\\
    \glt `That child goes up the deer's antlers, he brings him running.' \jambox*{\href{http://hdl.handle.net/10050/00-0000-0000-0004-1BAF-5}{[stim20\_4:43]}}
    \label{exe:borus}
	\ex \gll Desi se nawas=te mengga \textbf{bon} \textbf{bara}\\
	Desi {\glse} carry={\glte} \textsc{dist.lat} bring descend\\
	\glt `Desi came down carrying [the child], bringing it down from there.' \jambox*{\href{http://hdl.handle.net/10050/00-0000-0000-0004-1BA2-F}{[conv11\_5:50]}}
	\label{exe:bonbara}
\end{exe}

While \textit{kuru} `bring' is a generic bring-verb, \textit{bon} `bring' can only be used for things that are carried by the subject (and not, for example, escorted).

%\textcite{unterladstetter2020} type: CREL transport complex. V1 transport + V2 motion.

\subsection{With \textit{toni} `say; think; want'}
\label{sec:mvctoni}
The dependent verb \textit{toni} `say; think; want' is used with the meaning `say; think' as the second verb in complex predicates in combination with verbs expressing speech, thought, or sensation, such as \textit{taruo} `to say', as illustrated in~(\ref{exe:ttoni}). Other verbs accompanied by \textit{toni} are \textit{gerket} `to ask', \textit{narasa} `to feel' and \textit{konawaruo} `to forget'. More examples are given in §\ref{sec:speech}.

\begin{exe}
	\ex \gll kiun=a \textbf{taruo} \textbf{toni} mu=nan se ma go-un=at ruon\\
	wife-\textsc{3poss=foc} say say \textsc{3pl}=too {\glse} move\_seawards place-\textsc{3poss=obj} dig\\
	\label{exe:ttoni}
\end{exe}

With the meaning `want' (discussed in detail in §\ref{sec:kin}), \textit{toni} can be the first verb in a complex predicate, without any restriction on the semantics of the other verb. An example with \textit{bara} `to descend' is given in~(\ref{exe:tonibara}).\footnote{It is unclear whether \textit{toni} `want' can be used without volitional \textit{=kin} (§\ref{sec:kin}). The corpus instances of \textit{toni} followed by a verb that is not marked with \textit{=kin} are ambiguous between the `want' and the `say; think' reading.}

\begin{exe}
	\ex \gll lusi \textbf{toni} bara mat konggelem=kin\\
	eagle want descend \textsc{3sg.obj} grab=\textsc{vol}\\
	\glt `The eagle wants to descend and grab him.' \jambox*{\href{http://hdl.handle.net/10050/00-0000-0000-0004-1BAF-5}{[stim20\_4:21]}}
	\label{exe:tonibara}
\end{exe}

%\textcite{unterladstetter2020} type `say': CREL speech act complex. V1 speech act verb + V2 say.
%\textcite{unterladstetter2020} type `want': MOD modal. V1 matrix verb + V2 modifier verb or vv (here: vv).

\subsection{With \textit{eranun} `cannot'}
The dedicated negative verb \textit{eranun} `cannot; to not be possible' is not a full verb: it cannot be inflected for mood and aspect. Moreover, it must always be preceded by another verb. This verb is nominalised with \textit{-un}, such as \textit{karajang} `to work' in~(\ref{exe:noobody}). For a further discussion of this construction, see §\ref{sec:lexicalneg}.

\ea \label{exe:noobody}{
	\gll som kon-\textbf{barak} karajang-un \textbf{eranun}\\
	person one-any work-\textsc{nmlz} cannot\\
	\glt `No-one can do the work.' \jambox*{\href{http://hdl.handle.net/10050/00-0000-0000-0004-1C60-A}{[elic\_ind]}}}
\z 

%\textcite{unterladstetter2020} type `want': MOD modal. V1 matrix verb + V2 modifier verb or vv.

\section{Complex source, goal and location predicates}\is{location}
\label{sec:mvcgoal}
Complex source, goal and location constructions can be made with locative and lative postpositions, §\ref{sec:svcloclat}, or with \is{causative}causative \textit{di=}, §\ref{sec:svcdi}. They are made by means of the different complex predicates presented in this section, and show a variety of order constraints.

\subsection{Complex locative and lative predicates}\is{postposition!locative}\is{postposition!lative}
\label{sec:svcloclat}
Source, goal and location are commonly expressed with help of four postpositions: locative \textit{=ko}, animate locative \textit{=konggo}, lative \textit{=ka} and animate lative \textit{=kongga} (§\ref{sec:loc}, §\ref{sec:lat} and~§\ref{sec:animloclat}). While a NP marked with a locative postposition may and frequently does occur as the predicate of the clause, without any other verb, a NP marked with a lative postposition must be followed by a verb. In both cases, these locative and lative constructions may combine with other verbs to create even more complex goal, source and location predicates. The options for these complex predicates are given in Table~\ref{tab:loclatsvc}. The building blocks of these predicates are three groups of verbs: manner verbs, verbs expressing motion and other verbs. These combine in a limited number of ways with NPs carrying a locative or lative clitic. Six slots in these complex predicates can be distinguished: three before and two after the noun with the postposition. The first slot is reserved for manner verbs (marked with predicate linker \textit{=i}, §\ref{sec:mvci}) and \textit{kuru} `to bring' (§\ref{sec:mvckuru}). Motion verbs occur in either the second or the fifth slot. Motion verbs include all verbs expressing motion: the directional verbs (§\ref{sec:dir}), \textit{bo} `to go' (§\ref{sec:svcbo}) and other verbs like \textit{taluk} `to come out'. In some constructions, only \textit{bo} `to go' is allowed in the motion slot; in others, \textit{bo} `to go' is specifically not allowed in the motion slot. In one case, only motion verbs that are not \textit{bo} or a directional are allowed. \textit{Bo} `to go' has its own slot before the NP with the postposition. The verb slot (V), finally, is normally used for any verb, including the aforementioned. In one construction, this slot is used for a directional verb. In all cases, if \textit{bo} `to go' precedes the locative or lative, it takes the form \textit{bo}. If it follows it, it takes the form \textit{bot} (see also §\ref{sec:svcbo}).

\begin{table}[ht]
	\caption{Complex source, goal and location predicates}
	\label{tab:loclatsvc}
	\fittable{
		\begin{tabular}{l l l c l l l}
			\lsptoprule 
			manner & motion & `go' &N=\textsc{loc}/N=\textsc{lat}&motion&V&example\\
			\midrule
			&&& N=\textsc{loc}&&&§\ref{sec:loc}\\ %standard LOC.
			& $+$ && N=\textsc{loc}&&&\ref{exe:karsa}\\ %mov+loc
			$+$ &&& N=\textsc{loc}&&&\ref{exe:dalangip}\\%manner+loc
			$+$ & & $+$ & N=\textsc{loc}&&&\ref{exe:deirib}\\	
			%	&& N=\textsc{loc}& $+$\footnotemark &&\\ %loc+dir. deleted. all examples may be be biclausal, also doesn't make sense with dir+loc being very common
			& dir. only &$+$& N=\textsc{loc}&&&\ref{exe:karou}\\%dir+bo+loc
			& dir. only &$+$& N=\textsc{loc}&&$+$&\ref{exe:bararor}\\%dir+bo+loc+posture V		
			&&& N=\textsc{loc}&& $+$&\ref{exe:weleop}\\ %loc+v. posture or other, inner reference?
			%add structure for metko unruani mia
			\midrule 
			&&& N=\textsc{lat} & $+$, *\textit{bo} &&§\ref{sec:lat}\\%not with bo. this is the standard LAT.
			&&& N=\textsc{lat} &$+$, *\textit{bo}, *dir.& must be dir.&\ref{exe:koijem}\\%lat+mov+dir. infl w i, must be dir.	
			&&& N=\textsc{lat} & \textit{bo} only &$+$&\ref{exe:ambomuap}\\%lat + bo + V.
			$+$&&& N=\textsc{lat} &$+$&&\ref{exe:nitam}\\%manner+lat+ mov. boT
			&&$+$& N=\textsc{lat} &&$+$&\ref{exe:bosuo}\\%manner+bo+lat+V. only one good ex. 
			$+$&&$+$& N=\textsc{lat} &&$+$&\ref{exe:kiatbok}\\	%bo+lat+V. 				
			\lspbottomrule 
		\end{tabular}
	}
\end{table}

While all constructions have several examples in the natural spoken corpus, not all are found (or tested) with the animate forms \textit{=konggo} and \textit{=kongga}, which are much less frequent in the corpus. There is, however, no reason to assume they would behave differently. More constructions may be found if more data become available. Each type will be exemplified in turn below. The standard locative construction is described in §\ref{sec:loc} and the standard lative construction is described in §\ref{sec:lat}.

\begin{exe}
	\ex
motion + locative (goal):\\
	\gll kariak \textbf{sara} nakal=\textbf{ko}\\
	blood ascend head=\textsc{loc}\\
	\glt `Blood rises to the head.' \jambox*{\href{http://hdl.handle.net/10050/00-0000-0000-0004-1BC2-8}{[narr33\_3:10]}}
	\label{exe:karsa}
\end{exe}

\begin{exe}
	\ex
 manner + locative (goal):\\
	\gll ma \textbf{dalang=i} pasier=\textbf{ko}\\
	\textsc{3sg} jump={\gli} sea=\textsc{loc}\\
	\glt `He jumps in the sea.' \jambox*{\href{http://hdl.handle.net/10050/00-0000-0000-0004-1BAE-4}{[narr44\_4:11]}}
	\label{exe:dalangip}
\end{exe}

\begin{exe}
	\ex manner + \textit{bo} + locative (goal):\\
	\gll mat \textbf{deir=i} \textbf{bo} tompat-un=\textbf{ko}\\
	\textsc{3sg.obj} accompany={\gli} go place-\textsc{3poss=loc}\\
	\glt `I accompanied him to his place.' \jambox*{\href{http://hdl.handle.net/10050/00-0000-0000-0004-1BE8-0}{[stim43\_0:37]}}
	\label{exe:deirib}
\end{exe} 
	
\begin{exe}
\ex directional + \textit{bo} + locative (goal):\\
	\gll ma \textbf{sara} \textbf{bo} karop-un osa-t=\textbf{ko}\\
	\textsc{3sg} ascend go branch-\textsc{3poss} \textsc{up}-{\gltt}=\textsc{loc}\\
	\glt `He climbed up to the branch up there.' \jambox*{\href{http://hdl.handle.net/10050/00-0000-0000-0004-1BBD-5}{[conv12\_16:39]}}
	\label{exe:karou}
\end{exe}

\begin{exe}
\ex directional + locative + posture V (goal + location):\\
	  \gll ma se \textbf{bara} ror ewun=\textbf{ko} \textbf{taouk}\\
	\textsc{3sg} {\glse} descend tree trunk=\textsc{loc} lie\\
	\glt `He went down to the tree trunk and lay down.' \jambox*{\href{http://hdl.handle.net/10050/00-0000-0000-0004-1BAF-5}{[stim20\_2:59]}}
	\label{exe:bararor}
\end{exe}

\begin{exe}
\ex directional + locative + verb (location):\\
	  \gll nene \textbf{marua} wele opa me pasier=\textbf{ko} \textbf{waruo}\\
	grandmother move\_seawards vegetable {\glopa} {\glme} sea=\textsc{loc} wash\\
	\glt `Granny went to sea to wash those vegetables in the sea.' \jambox*{\href{http://hdl.handle.net/10050/00-0000-0000-0004-1BE2-C}{[narr23\_3:50]}}
	\label{exe:weleop}
\end{exe}

\begin{exe}
\ex lative + motion + directional (source/goal):\\
	 \gll koi jembatan=\textbf{ka} \textbf{marmar=i} \textbf{masara}\\
	then dock=\textsc{lat} walk={\gli} move\_landwards\\
	\glt `Then walking from the dock inland.' \jambox*{\href{http://hdl.handle.net/10050/00-0000-0000-0004-1BE8-0}{[stim43\_20:21]}}
	\label{exe:koijem}
\end{exe}
	
\begin{exe}
\ex	lative + \textit{bo} + V (source/goal):\\%may be purposive
	  \gll ma amdir=\textbf{ka} \textbf{bo} \textbf{muap-ruo}\\
	\textsc{3sg} garden=\textsc{lat} go food-dig\\
	\glt `She goes to the garden to dig for food.' \jambox*{\href{http://hdl.handle.net/10050/00-0000-0000-0004-1BA7-D}{[narr21\_0:49]}}
	\label{exe:ambomuap}
\end{exe}
	
\begin{exe}
\ex manner + lative + motion (source/goal):\\
	 	\gll mu \textbf{ecien=i} tamisen=\textbf{ka} \textbf{bot}\\
	\textsc{3pl} return={\gli} Antalisa=\textsc{lat} go\\
	\glt `They returned to Antalisa.' \jambox*{\href{http://hdl.handle.net/10050/00-0000-0000-0004-1B70-6}{[narr4\_2:21]}}
	\label{exe:nitam}
\end{exe}
	
\begin{exe}
\ex manner + \textit{bo} + lative + V (goal):\\ %only this one is purposive the other (marmari bo lengungga bara) is not
	  \gll mu \textbf{kiem=i} \textbf{bo} Suo=\textbf{ka} \textbf{nung}\\
	\textsc{3pl} flee={\gli} go Suo=\textsc{lat} hide\\
	\glt `They ran to (until) Suo to hide.' \jambox*{\href{http://hdl.handle.net/10050/00-0000-0000-0004-1BBB-2}{[narr40\_2:33]}}
	\label{exe:bosuo}
\end{exe}
	
\begin{exe}
\ex \textit{bo} + lative + V (source/goal):\\
	 	\gll tiri ki=at \textbf{bo} kibis=\textbf{ka} \textbf{rep=et}\\
	sail \textsc{2pl=obj} go shore=\textsc{lat} get={\glet}\\
	\glt `[We] sail and go pick you up from the shore.' \jambox*{\href{http://hdl.handle.net/10050/00-0000-0000-0004-1B6D-C}{[conv28\_3:07]}}
	\label{exe:kiatbok}
\end{exe}

The four complex predicates where one or more verbs precede the locative are all goal constructions, i.e. expressing movement towards a goal. This order obeys the principle of iconicity: the goal, the endpoint of the movement, follows the motion verb \parencite[cf.][]{schapper2011}. The fifth complex predicate in the table, which has verbs preceding and following the locative, expresses both movement towards a goal and the posture taken at the location of the goal. %EV{(This may be nesting of complex predicates or even a biclausal construction, but I wouldn't know where to cut it.)} 
The last complex locative predicate in the table, where a verb follows the locative, is a static locative construction. In~(\ref{exe:weleop}), the locative is combined with \textit{waruo} `to wash', but the verb is also often a posture verb like \textit{maraouk} `to put', \textit{mambara} `to stand' or \textit{melelu} `to sit' (cf. posture SVCs in~§\ref{sec:post}). %untst: position-to-position, no real complex predicate?

All latives must be followed by a verb, regardless of the construction. Most constructions can express both movement from a source and movement towards a goal. These constructions are not iconic, so the correct reading must always be inferred from the context. The construction manner + \textit{bo} + lative + Verb is rare: there are only two examples in the corpus, both of which express movement towards a goal.

%\textcite{unterladstetter2020} types  in order of presentation: SREL action-to-position (rather: motion-to-pos), MOD adverbial?, MOD adverbial/CREL tansport complex?, SREL action-to-position?, SREL action-to-position?, SREL position-to-action, MOD adverbial or CREL motion complex, SREL motion-to-action or JXT purpose?, CREL complex motion, CREL complex motion or JUXT purpose, SREL motion-to action. last one a combi of several.

\subsection{Complex causative predicates}\is{causative}
\label{sec:svcdi}
Causative \textit{di=} (§\ref{sec:di}) occurs in combination with directional verbs and locations in complex causative predicates. It attaches to the left edge of the complex predicate. It is also optionally employed in give-constructions (§\ref{sec:give}). (\ref{exe:bint}) is an example of a complex causative predicate with \textit{ra} `to move (away); to become' (§\ref{sec:svcra}).

\begin{exe}
	\ex \gll mendak embir-ne=ka sasul \textbf{di=ra} bintang ne=\textbf{ko}\\
	like\_that bucket-inside=\textsc{lat} spoon \textsc{caus}=move tub inside=\textsc{loc}\\
	\glt `So [I] spooned from the bucket into a tub.' \jambox*{\href{http://hdl.handle.net/10050/00-0000-0000-0004-1BA6-6}{[conv13\_3:11]}}
	\label{exe:bint}
\end{exe} 

%\textcite{unterladstetter2020} type: SREL handling-to-placement, but more like (V1 handling) V2 placement. 

\section{Serial verb constructions}\is{serial verb construction}
\label{sec:svc}
Following \textcite{lovestrand2018} and \textcite{haspelmath2016}, I define serial verb constructions (SVCs) as monoclausal constructions with more than one independent verb, no linking element between the verbs, and with at least one shared argument.\footnote{This is a morphosyntactic definition for a morphosyntactic phenomenon. Often-used criteria such as intonation and event structure (in e.g. \citealt{aikhenvald2006}) are argued to be hard to falsify and to follow from a morphosyntactic definition (e.g. \citealt{haspelmath2016,defina2016}).} This type of complex predicate is not very common in Kalamang. SVCs can be divided into symmetrical SVCs (with verbs from open verb classes, §\ref{sec:symsvc}) and asymmetrical SVCs (with at least one verb from a restricted class, §\ref{sec:asymsvc}).

\subsection{Symmetrical SVCs}\is{serial verb construction!symmetrical}
\label{sec:symsvc}
Symmetrical SVCs consist of verbs from open verb classes, with no restrictions on the semantics of the verb. All components in a symmetrical SVC contribute equally to its meaning, so there is no `head' in the construction \parencite[][22]{aikhenvald2006}. Symmetrical SVCs express events consisting of two or more active verbs, given in sequential order (obeying the iconicity of order, \cites[288]{unterladstetter2020}[29]{staden2008}). They share their arguments. (\ref{exe:jieborar}) above is a symmetrical SVC. Additional examples are provided in~(\ref{exe:potmapar}) to~(\ref{exe:kajkow}). Note that in~(\ref{exe:perjiena}), the noun is incorporated into the SVC (see §\ref{sec:incorp} on noun incorporation)\is{noun incorporation}.
%Palmer & Krause: compound SVC, activity + result

\begin{exe}
	\ex \gll pawan=at worman=i koyet in koi potman=i koyet timun=at \textbf{potma} \textbf{paruak}\\
	plank=\textsc{obj} cut\_down={\gli} finish \textsc{1pl.excl} then cut={\gli} finish tip=\textsc{obj} cut drop\\
	\glt `After cutting down [trees for] planks, after cutting, [we] cut off the tips.' \jambox*{\href{http://hdl.handle.net/10050/00-0000-0000-0004-1BB4-6}{[narr14\_5:00]}}
	\label{exe:potmapar}
	\ex \gll in=a \textbf{per-jie} \textbf{na}\\
	\textsc{1pl.excl=foc} water-get consume\\
	\glt `We fetched water and drank.' \jambox*{\href{http://hdl.handle.net/10050/00-0000-0000-0004-1BBB-2}{[narr40\_1:45]}}
	\label{exe:perjiena}
	\ex \gll usar=et mul-un=ka \textbf{kajien} \textbf{kowarman}\\
	erect={\glet} side-\textsc{3poss}=\textsc{lat} pick fold\\
	\glt `Erect [the basket], pick and fold [strips] from the side.' \jambox*{\href{http://hdl.handle.net/10050/00-0000-0000-0004-1BB8-C}{[narr11\_0:38]}}
	\label{exe:kajkow}
\end{exe}
%also tawie nananana
%ko=yuon paruak
%in se bara pera(t?) rep (go down get water)\\
%...koi mia komet konawaruo. konowaruo, bara minde, bo godungda me...\\

When an unmarked directional verb (§\ref{sec:dir}) is in the first position of a complex predicate, the construction can be a SVC. Like the examples above, they are sequential actions with a shared subject. They can also be purposive, such as \textit{bara komet} `come down to look' in~(\ref{exe:barakom}). If a directional verb in a complex predicate is preceded by a non-directional verb, this verb is always marked with \textit{=i} (described in §\ref{sec:mvci}), and this construction is thus not a SVC. (\ref{exe:marabara}) and~(\ref{exe:ranmina}) show SVCs of two directional verbs. (\ref{exe:ragerket}), finally, shows a SVC with shared subject, separated by the object of the second verb.

\begin{exe}
	\ex \gll tim-un=at potma \textbf{bara} \textbf{melalu}\\
	tip-\textsc{3poss=obj} cut descend sit\\
	\glt `[We] cut off the tips, and lower down [the canoe].' \jambox*{\href{http://hdl.handle.net/10050/00-0000-0000-0004-1BB4-6}{[narr14\_1:37]}}
	\label{exe:baramel}
	\ex \gll ma se \textbf{ra} \textbf{min}\\ %motion+activity=purposive: same subject, intransitive.
	\textsc{3sg} {\glse} go sleep\\
	\glt `He goes to sleep.' \jambox*{\href{http://hdl.handle.net/10050/00-0000-0000-0004-1BB9-6}{[stim1\_1:05]}}
	\label{exe:ramina}
	\ex \gll ma \textbf{bara} \textbf{komet}\\
	\textsc{3sg} descend look\\
	\glt `He came down to look.' \jambox*{\href{http://hdl.handle.net/10050/00-0000-0000-0004-1BA3-3}{[conv10\_4:39]}}
	\label{exe:barakom}
	\ex \gll mengga koi mara \textbf{masara} \textbf{bara}\\
	\textsc{dist.lat} then move\_landwards move\_landwards descend\\
	\glt `Then go inland from there, descend inland.' \jambox*{\href{http://hdl.handle.net/10050/00-0000-0000-0004-1BE6-4}{[stim36\_0:46]}}
	\label{exe:marabara}
	\ex \gll pi konenen=i koi \textbf{ran} \textbf{mia=nin}\\
	\textsc{1pl.excl} remember={\gli} then go come=\textsc{neg}\\
	\glt `We don't [have to] remember to come and go anymore.' \jambox*{\href{http://hdl.handle.net/10050/00-0000-0000-0004-1BCD-C}{[conv3\_3:55]}}
	\label{exe:ranmina}
	%other sor opa tiri ran-mian-den, mu sontumat kome ran mian.
	\ex \gll ma \textbf{ra} ulan=at \textbf{gerket} eh\\
	\textsc{3sg} go aunt=\textsc{obj} ask \textsc{tag}\\
	\glt `He asks my aunt, right?' \jambox*{\href{http://hdl.handle.net/10050/00-0000-0000-0004-1BBD-5}{[conv12\_2:51]}}
	\label{exe:ragerket}
\end{exe}

%\textcite{unterladstetter2020} types for the examples in this section: SREL cause-result?, SREL motion-to-action/JUXT purposive?, JUXT sequential, SREL action-postion, SREL motion-action, SREL motion-action (optionally purposive, may be sequential), 2x CREL motion complex or JUXT, SREL motion action. I.e. most SREL and JUXT, latter is logical because these resemble asyndetic coordination, with the difference that they share (an) argument(s).

\subsection{Asymmetrical SVCs}\is{serial verb construction!asymmetrical}
\label{sec:asymsvc}
Asymmetrical SVCs have one or more verbs from a restricted class of verbs \parencite[][21]{aikhenvald2006}. In most cases described here, this is not a syntactic subclass of verbs; rather, one of the verbs is from a particular semantic class. Some asymmetrical verb constructions may only occur with one particular verb. All asymmetrical SVCs have a fixed order, where the verb from the (most) restricted class usually is the first verb.

\subsubsection{With \textit{bo} `go'}
\label{sec:svcbo}
The single most common asymmetrical SVC is with \textit{bo} `to go', an independent verb that can be modified with all mood, aspect and modality markers available in Kalamang. It is, however, also an irregular verb, which always takes the form \textit{bot} when it is the only verb in the construction, but \textit{bo} when it occurs in a SVC (see §\ref{sec:stems} for an introduction to verb roots in \textit{-t}, and §\ref{sec:ntverbs} for a discussion of these as a separate verb class).\footnote{Alternatively, it could be argued that \textit{bo} is an independent form of \textit{bot}. In that case, complex predicates with \textit{bo} are not SVCs.} It is multifunctional, and always occurs as the first verb in the SVC.

First, it occurs in purposive motion serialisation, where \textit{bo} `to go' indicates the movement of the subject, and the second verb indicates the purpose. The second verb can be any dynamic verb, transitive or intransitive. Consider the following examples with an intransitive verb, a transitive verb with incorporated object, and a transitive verb with object, respectively.

\begin{exe}
	\ex \gll ma toni e an \textbf{bo} \textbf{war=kin}\\
	\textsc{3sg} say \textsc{int} \textsc{1sg} go fish=\textsc{vol}\\
	\glt `He said: ``Eh, I go fishing.''' \jambox*{\href{http://hdl.handle.net/10050/00-0000-0000-0004-1BA3-3}{[conv10\_13:26]}}
	\label{exe:bowark}
	\ex \gll tumun kon se \textbf{bo} \textbf{kai-rep}\\
	child one {\glse} go firewood-get\\
	\glt `One of the children went to collect firewood.' \jambox*{\href{http://hdl.handle.net/10050/00-0000-0000-0004-1BE2-C}{[narr23\_5:21]}}
	\label{exe:tumkon}
	\ex \gll mu tok \textbf{bo} walor=at \textbf{saran}\\
	\textsc{3pl} first go coconut\_leaf=\textsc{obj} ascend\\
	\glt `They first went to harvest [lit. ascend] coconut leaves.' \jambox*{\href{http://hdl.handle.net/10050/00-0000-0000-0004-1BD7-2}{[narr3\_10:54]}}
	\label{exe:walsar}
\end{exe} 

%also bo rorpotma, bo wolnelebor, bo sorsanggara, bo taluk (weird)

Second, \textit{bo} `to go' occurs in SVCs with stative intransitive verbs in change-of-state serialisation. Like in purposive motion serialisation, \textit{bo} `to go' needs to be the first verb in the construction. Consider~(\ref{exe:borei}) and~(\ref{exe:eiririr}).

\begin{exe} 
	\ex \gll sontum \textbf{bo} \textbf{reidak} mindi\\
	person go many like\_that\\
	\glt `We became many people, like that.' \jambox*{\href{http://hdl.handle.net/10050/00-0000-0000-0004-1BC3-B}{[conv7\_12:38]}}
	\label{exe:borei}
	\ex \gll bungkus eir∼eir-i kahetma gelas \textbf{bo} \textbf{mikon}\\
	sachet two∼\textsc{distr-objqnt} open glass go full\\
	\glt `Open two sachets each, until the glass is full.' \jambox*{\href{http://hdl.handle.net/10050/00-0000-0000-0004-1BD7-2}{[narr3\_11:56]}}
	\label{exe:eiririr}
\end{exe}	
%also bo temun, bo cukup. temun bo is ungrammatical.
%mu tok pakuret sampi bo selesai

A fixed expression that is a variant of change-of-state serialisation with \textit{bo} `to go' and \textit{tik} `to take a long time' is \textit{bo tik} `before long'. Note that change-of-state can also be expressed with nominals referring to times of the day in predicate function, e.g. \textit{bo go saun} `until the evening; after it had turned evening' (with \textit{go saun} `evening' ).

Third, \textit{bo} `to go' occurs in constructions with locatives and latives to indicate motion towards a goal (see §\ref{sec:svcloclat}).

%\textcite{unterladstetter2020} types: SREL motion-action + some kind of MOD (`become', maybe tense-aspect?).

\subsubsection{With \textit{ra} `move'}
\label{sec:svcra}
Purposive motion serialisation can also be achieved with the directional verb \textit{ra} `to move (along a path); to become' as V1. It is less generic than \textit{bo} `to go', described above, in that it specifies the path of motion in combination with the other verb. This is also evident from its use as the antonym of \textit{mia} `to come' (as in~\ref{exe:ranmina} above). In~(\ref{exe:mast}), the use of \textit{ra} refers to the path from a floating fish cage to a boat. In~(\ref{exe:kulp}), the path from the speaker to the fire is indicated by \textit{ra}. (\ref{exe:pasaromat}) is made with \textit{ra} and the zero morpheme `give', to indicate the path between the giver and the recipient.

\begin{exe}
	\ex \gll Mas toni eh pi tiri \textbf{ra} \textbf{komet=et}\\
	Mas say hey \textsc{1pl.incl} sail move look={\glet}\\
	\glt `Mas said: ``Hey, let's sail out to look.''' \jambox*{\href{http://hdl.handle.net/10050/00-0000-0000-0004-1BC6-C}{[narr17\_0:51]}}
	\label{exe:mast}
	\ex \gll im=at {\ob}...{\cb} walawala=i din-neko kulpanggat=bon \textbf{ra} \textbf{sair}\\
	banana=\textsc{obj} {} throw={\gli} fire-inside triggerfish=\textsc{com} move bake\\
	\glt `[I] threw the bananas in the fire, baked [them] with the triggerfish.' \jambox*{\href{http://hdl.handle.net/10050/00-0000-0000-0004-1B9F-F}{[conv9\_19:42]}}
	\label{exe:kulp}
	\ex \gll an tok \textbf{ra}\textsubscript{\upshape V1} pasarom=at\textsubscript{\upshape T} ma\textsubscript{\upshape R} ∅\textsubscript{\upshape V2}=et\\
	\textsc{1sg} first move ambarella=\textsc{obj} \textsc{3sg} give={\glet}\\
	\glt `I go give him an ambarella first.' \jambox*{\href{http://hdl.handle.net/10050/00-0000-0000-0004-1B9F-F}{[conv9\_16:17]}}
	\label{exe:pasaromat}
\end{exe} 

Other examples are~(\ref{exe:ramin}) and~(\ref{exe:ragerket}) above. \textit{Ra} `to move' is also used as the second verb in constructions with dependent verb \textit{kuru} `to bring' (§\ref{sec:mvckuru}, example~\ref{exe:kwarin}), as the second verb in complex predicates linked by predicate linker \textit{=i} (§\ref{sec:mvci}, example~\ref{exe:sirnag}), and in causative constructions (§\ref{sec:svcdi}, example~\ref{exe:bint}).

%\textcite{unterladstetter2020} type: SREL motion-action. V1 motion + V2 action.

\subsubsection{With \textit{melelu} `sit' and \textit{mambara} `stand'}\is{posture}
\label{sec:post}
SVCs with \textit{melelu} `to sit' and \textit{mambara} `to stand' (introduced in examples~\ref{exe:mggt} and~\ref{exe:mtggt} above) can tentatively be grouped as posture SVCs. The first verb in the construction is the posture verb, and the second is a transitive or intransitive verb expressing some kind of activity. The verbs share the same subject.

\begin{exe}
	\ex \gll kaman-neko \textbf{mambara} \textbf{komet∼komet}\\
	grass-inside stand look∼\textsc{prog}\\
	\glt `[He] stands looking in the grass.' \jambox*{\href{http://hdl.handle.net/10050/00-0000-0000-0004-1BBA-8}{[stim2\_2:23]}}
	\label{exe:mambakom}
	\ex \gll Afin \textbf{mambara} pi=at=a \textbf{komet=et}\\
	Afin stand \textsc{1pl.excl=obj=foc} look={\glet}\\
	\glt `Afin stands looking at us.' \jambox*{\href{http://hdl.handle.net/10050/00-0000-0000-0004-1BA2-F}{[conv11\_9:20]}}
	\label{exe:piarakom}
	\ex \gll in-naninggan \textbf{melelu} \textbf{ewa}\\
	\textsc{1pl.excl}-all sit talk\\
	\glt `We all sat talking.' \jambox*{\href{http://hdl.handle.net/10050/00-0000-0000-0004-1B70-6}{[narr4\_0:26]}}
	\label{exe:inmelel}
	\ex \gll Mustafa emun \textbf{melelu} \textbf{wele-narari}\\
	Mustafa mother.\textsc{3poss} sit vegetables-slice\\
	\glt `Mustafa's mother sits slicing vegetables.' \jambox*{\href{http://hdl.handle.net/10050/00-0000-0000-0004-1BE7-5}{[stim42\_8:29]}}
	\label{exe:welen}
	\ex \gll kon \textbf{melelu} main=at \textbf{na}\\
	one sit \textsc{3poss=obj} consume\\
	\glt `One sits eating his.' \jambox*{\href{http://hdl.handle.net/10050/00-0000-0000-0004-1BA8-B}{[stim4\_2:05]}}
	\label{exe:mainatna}	
\end{exe}	

%\textcite{unterladstetter2020} type: position-action. V1 posture verb, V2 action verb.


\subsubsection{With \textit{bara} `descend'}\is{resultative}
\label{sec:ressvc}
There is one example of an asymmetrical SVC that is composed of an activity and a result, and that may tentatively be termed resultative. It consists of the directional verb \textit{bara} `to descend' and the \is{stative intransitive verb}stative intransitive verb \textit{pol} `to be compact'. They share the same subject: soil. The context is digging soil for making the foundation of a house. The speaker says that one shouldn't use soil without small stones, because otherwise, if it rains, the soil will become too compact.

\begin{exe}
	\ex \gll mena ma \textbf{bara} \textbf{pol}\\
	otherwise \textsc{3sg} descend compact\\
	\glt `Otherwise it becomes compact.' \jambox*{\href{http://hdl.handle.net/10050/00-0000-0000-0004-1BA3-3}{[conv10\_5:03]}}
	\label{exe:barapol}
\end{exe}

%\textcite{unterladstetter2020} type: SREL resultative. V1 causing verb + V2 stative resultant verb.
%Krause: resultative = activity + result (e.g. shout awake). no shared arguments? so no SVC?
%resultative = cause + path (e.g. push down). object = subject.

\subsubsection{With \textit{paruo} `make, do'}\is{causative}
\label{sec:svccaus}
Causative constructions are usually made with one of the proclitics \textit{di=}, \textit{na=} or \textit{ma=} (§\ref{sec:caus}), but may also be made help of \textit{paruo} `to make; to do' as the first verb in a complex predicate. It is often combined with stative intransitive verbs like \textit{samsik} `thin' (example~\ref{exe:caussam}), but can also be combined with the \is{reciprocal}reciprocal verb \textit{naubes(bes)} `to have a good relationship' (from reciprocal \textit{nau=} and \textit{bes} `good') to form the meaning `to make up' (after a fight).

\begin{exe}
	\ex \gll
	manyor=i koyet ma yorsik an koi desil=i \textbf{paruo} \textbf{samsik}\\
	adjust={\gli} finish \textsc{3sg} straight \textsc{1sg} then plane={\gli} make thin\\
	\glt `After adjusting it's straight and then I plane it to make it thin.' \jambox*{\href{http://hdl.handle.net/10050/00-0000-0000-0004-1BDD-5}{[narr42\_5:43]}}
	\label{exe:caussam}
	\ex \gll mu \textbf{paruo} \textbf{nau=bes∼bes}\\
	\textsc{3pl} make \textsc{recp}=good∼\textsc{prog}\\
	\glt `They are making up.' \jambox*{\href{http://hdl.handle.net/10050/00-0000-0000-0004-1BA9-9}{[stim6\_16:31]}}
\end{exe}
\is{predicate!complex|)}
%also paruo kinkin
% ma he paruo rua, but unclear rec.

%\textcite{unterladstetter2020} type: SREL causative. V1 generic causative verb + V2 resultant verb.
