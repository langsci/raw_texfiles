\addchap{Abstract}
Simultaneous interpreting (SI) is regarded as a highly complex cognitive task. To provide support for its underlying subprocesses, bespoke computer-assisted interpreting (CAI) tools were developed over the past few years. These solutions are expected to positively impact the product of interpreting, while minimising the additional cognitive effort resulting from human-machine interaction, particularly thanks to the integration of automatic speech recognition (ASR).

Despite the increasing relevance of these solutions for the interpreting profession, the focus of CAI research has thus far remained rather narrow, concerning mainly the product of SI. However, the cognitive implications of CAI tool use during SI have remained largely unexplored. Moreover, no empirically validated methodology for the combined collection of product- and process-related data during computer-assisted simultaneous interpreting (CASI) has yet been developed. This represents an important limitation in current CAI research, which limits the comparability of the studies conducted and our understanding of the impact of human-machine interaction on the SI process.

With the aim to address said limitations, this dissertation developed and tested a methodology deriving its approach from Translation Process Research (TPR) to explore the CASI process. The results from the combined analysis of terminological accuracy, errors and omissions, EVS, inter-cluster pause duration, eye movement data collected with eyetracking and qualitative questionnaires suggest that ASR tools may indeed offer an advantage in terms of both cognitive effort and terminological accuracy. However, these remain initial observations which must be corroborated by future studies. The study also highlighted the potential of eyetracking for the study of the CASI process.

As technology increasingly permeates the interpreting profession, conducting further research on the impact of support tools on the interpreting process appears essential to ensure that such solutions are truly targeted to the needs of interpreters. It appears equally important to develop a validated methodology to promote the comparability of studies conducted in this area of interpreting research and thus gain a deeper understanding of the impact of CAI tools on the interpreting process and product. This is the contribution which the present work ultimately strives to provide.
