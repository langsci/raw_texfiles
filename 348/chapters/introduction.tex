\addchap{Introduction}

When conference interpreters interpret a speech simultaneously, they are often faced with the need to quickly and precisely render specialised terminology in the target language. Hence, pre-assignment preparation is fundamental to acquire specialised terminology to aptly express domain-specific knowledge in the target language. In the preparation phase, interpreters usually compile glossaries containing specialised terms likely to be used by the speaker (e.g. \citealt{rutten_informations-_2007,fantinuoli_computer-assisted_2017,will_terminology_2007,gile_basic_2009}). Despite learning the terminological equivalents ahead of the event, interpreters may not always be able to retrieve the target-language equivalent from memory during interpreting. To cope with this difficulty, among other tactics \citep[14]{gile_basic_2009}, they may choose to look up the required term in their glossaries. Traditionally, specialised glossaries have been compiled on paper, or prepared in digital format and printed out for the booth (e.g. \citealt{jiang_interpreters_2013,jiang2015survey}).

With the increasing permeation of the profession by technology over the past couple of decades, the booth is now increasingly paper-less \citep{rutten2017terminology}. On a laptop or a tablet, interpreters can now conduct glossary queries in the digital medium.

As for the software employed to create digital glossaries for interpreting assignments, traditionally this has consisted in text processing programmes or database applications aimed at the general public. For lack of dedicated tools, some interpreters have resorted to computer-assisted translation (CAT) tools. The use of tools for corpus-based terminology work, terminology extraction, and of speech recognition, to name but a few examples, is therefore not the exclusive preserve of translators and terminologists but can also be found in interpreters' terminology work.

Despite the potential usefulness of these technologies for interpreters, scholarship remarked that the nature of interpreting imposes specific demands, both cognitive and related to interpreters' workflow (e.g. \citealt{rutten_why_2004,will_bemerkungen_2000,will_terminology_2007}). Thus, around the same time when CAT tools started to appear, applications geared towards the specific needs of interpreters were created and later increasingly refined which fall under the name of computer-assisted interpreting (CAI) tools.\footnote{For a terminological clarification of the term ``CAI tool'' and its use in the present work, see \sectref{CAIdef}.} Their aim is to support interpreters along several phases of their workflow, especially during preparation, but also for terminology retrieval during the interpreting task. The recent advances in automatic speech recognition (ASR) technology have motivated its integration into CAI tools, which may now offer live support for terminology and other units of information without physical interaction between the interpreter and the machine. The first prototypes of ASR-enhanced CAI tools are already emerging (e.g. \citealt{fantinuoli_computer-assisted_2017}).

In light of these developments, several studies have been conducted on the topic of CAI tools over the past few years. They have mainly explored the tools' potential to improve terminological accuracy during simultaneous interpreting (e.g. \citealt{prandi_uso_2015,prandi_use_2015}), the extent to which ASR improves the rendition of number words and specialized terms in the target language (e.g. \citealt{defrancq_automatic_2020}), or their potential for offering support to interpreters in the consecutive mode (e.g. \citealt{wang2019can}). With a few exceptions \citep{biagini_glossario_2015,frittella_cai-supported_2021}, the focus of such studies has been rather narrow, using specific performance indicators such as the accuracy of interpreted terms and numerals to assess the tools' impact on the overall quality of the interpretation (e.g. \citealt{pisani_measuring_2021}), without taking stock of the interpretation beyond these individual items. Many findings have emerged from small-scale experiments conducted on students in the framework of master's theses (e.g. \citealt{canali_utilizzo_2018,van_cauwenberghe_etude_2020}). Despite the emphasis on the postulated difficulty of integrating CAI tools into the interpreting process and, on the other hand, the widespread enthusiasm for the potential of ASR to alleviate cognitive load during SI, the impact of such solutions on the cognitive subprocesses underlying SI has so far remained largely unexplored. This represents an evident lacuna compared to the large body of research conducted on cognition in the translator-machine interaction. Such interaction has been addressed by numerous empirical studies in the area of Translation Process Research (TPR) from multiple perspectives and with a variety of methods \citep[xvii]{carl_measuring_2021}. Indeed, as ``a research tradition within cognitive translation studies (CTS) [...] exploring factors that determine human translation behavior'' (ibid.), TPR may constitute a valuable reference point and provide useful tools to the analysis of computer-assisted simultaneous interpreting (CASI), especially from a methodological standpoint. However, no empirically-validated methodology for the combined collection of product- and process-oriented data with a markedly cognitive focus has yet been developed to explore the phenomenon of technology-supported SI.

The present doctoral thesis seeks to address this limitation by developing and testing an empirical methodology for a cognitive exploration of CASI. In particular, the present work derives its methods from TPR to analyse the impact on cognitive load of different forms of digital terminological support for interpreters through a within-subject experimental study.

To the best of my knowledge, this is the first study to address com\-put\-er-as\-sist\-ed SI from a cognitive perspective. As such, it presents an exploratory character which aims to provide first findings and, at the same time, to identify open questions and formulate hypotheses for further investigations of the phenomenon.

The following section establishes the paradigm for the present research work and provides a conceptual framework for the cognitive inquiry into the phenomenon of CASI. The rest of the present chapter illustrates how the present work is organised and briefly describes the content of each chapter.


\section*{Choice of paradigm for the present study}\label{paradigm}

Especially in the simultaneous mode, interpreting has often been described as a complex cognitive activity, involving concurrent information processing and temporary storage tasks competing for attentional resources. This view of interpreting\footnote{Here intended in the broader sense of the term and referring not only to the conference setting.} as cognitive information processing represents one of the ``supermemes'' of interpreting, as observed by \citet[51]{pochhacker_introducing_2004}\footnote{Pöchhacker derives the notion of memes and supermemes from \citet{chesterman_memes_2016}: as in translation, these ``socio-biological concepts'' have arisen as metaphors to illustrate particular views of interpreting as an object of study. From this perspective, the interpreter is seen as a ``human processor'' performing several ``cognitive skills […] the combination of which would account for the complex task of interpreting'' \citep[53]{pochhacker_introducing_2004}.}.

Yet, even though this ``internal'' perspective has been particularly prolific in the academic inquiry into the phenomenon, interpreting may also be viewed as a socially embedded human activity, situated in a real communicative context. The supermeme of communicative activity elucidates interpreting as a combination of listening and speaking aimed at facilitating communication beyond the linguistic barrier.

For the scope of the present study, I chose to conduct my inquiry into interpreting from an explicitly cognitive perspective. Selecting one perspective does not, however, mean discarding or denying the other, but rather focusing the spotlight on one aspect of this multi-faceted activity, and is necessary to establish the theoretical framework guiding investigation. The next step lies into the definition of my research paradigm\footnote{See \citet{pochhacker_introducing_2004,pochhacker_introducing_2016} for a detailed account of research paradigms in Interpreting Studies.}. The two supermemes of interpreting as cognitive information processing and as communicative activity are at the core of a number of paradigms that can be divided into social, psycholinguistic, and cognitive approaches following Setton's classification \citep{setton_models_2003}.

The social approach reflects a tradition which looks at interpreting within a broader framework including social and behavioural factors. Prominent issues deriving from this view of interpreting are the interaction between the actors involved in the communicative event, the role of the interpreter and the long-standing issue of neutrality, the view of quality assessment as a question of pragmatics (and not only of identity between the source text and the interpreted message), but also the issue of interpreting strategies. To this category may be ascribe the "target-text oriented, translation-theoretical" \citep[77]{pochhacker_introducing_2004} paradigm exemplified by \citet{salevsky_probleme_1987}, \citet{schjoldager_an_1995/2002}, \citet{pochhacker_simultandolmetschen_1994} and \citet{kalina_strategische_1998}, as well as the ``dialogic discourse-based interaction'' paradigm \citep[79]{pochhacker_introducing_2004} of which \citet{roy_interactional_1996, roy_interpreting_2000} and \citet{wadensjo_double_1993, wadensjo_interpreting_1998} are the most prominent representatives.

Psycholinguistic approaches are grounded in theories of communication and focus on features of discourse rather than on the cognitive processing of interpreting. This perspective is at the core of the interpretive paradigm pioneered by \citet{seleskovitch_interpretation_1976} and \citet{lederer_traduction_1981} with their \textit{théorie du sens}. The models by \citet{chernov_semantic_1979, lambert_message_1994}, \citet{dejean_le_feal_satzsegmentierung_1980, dejean_le_feal_lectures_1981}, \citet{donovan_fidelite_1990}, \citet{laplace_theorie_1994} and \citet{setton_simultaneous_1999} himself may also be included in this category.

Finally, interpreting has been studied from a cognitive perspective and viewed as a matter of information processing. The scholars belonging to this tradition focus on the exploration of the cognitive underpinnings of interpreting. It is not by chance that they derive their methods of investigation from the cognitive sciences. This third approach has generated the cognitive processing paradigm \citep[73]{pochhacker_introducing_2004} initiated by \citet{gerver_empirical_1976} and further exemplified by \citet{lambert_information_1988}, \citet{gerver_information-processing_1978}, \citet{moser_simultaneous_1978} and \citet{hauenschild_process_1997}, \citet{kurz_simultandolmetschen_1996}, \citet{shlesinger_strategic_2000} and \citet{gile_partage_1988, gile_conference_1997, gile_testing_1999} and \citet{seeber_thinking_2007,seeber_cognitive_2011,seeber_multimodal_2017}. To this perspective one can also ascribe the ``neurophysiological/neurolinguistic paradigm'' \citep[75]{pochhacker_introducing_2004} exemplified by \citet{lambert_neurological_1994}, \citet{snell-hornby_look_1994,kurz_simultandolmetschen_1996}, \citet{lambert_non-linguistic_1994,daro_experimental_1997}, \citet{petsche_brain_1993}, \citet{rinne_translating_2000}, and \citet{tommola_mental_1990}.

The focus of the present doctoral thesis lies on the way the use of digital terminological support tools during simultaneous interpreting affects the cognitive processes involved in simultaneous interpreting. Hence, I situate my inquiry in the cognitive processing paradigm, while adopting an interdisciplinary approach to the exploration of simultaneous interpreting with digital terminology support. Methodologically, the study draws heavily on research methods developed and validated in the framework of empirical Translation Process Research (TPR). As translation and interpreting are rather similar activities in terms of their underlying cognitive processes and cognitive control functions, the approaches developed in TPR for the exploration of the translation process are expected to provide a valuable methodological reference point.\label{paradigmEnd}


\section*{Organisation}

The present thesis is organised as follows. \chapref{chapter1} introduces the topic of terminology in interpreting. Specifically, it addresses terminology work in conference interpreting highlighting its commonalities and differences with terminology work in translation. Additionally, it underlines the role of terminology as an important quality factor in translation and interpreting, which motivates the focus on terminological support in the present study. The last section closes \chapref{chapter1} with a discussion of the requirements for CAI tools and of the potential and limitations of non-bespoke terminology tools for interpreters. Against this background, \chapref{chapter2} frames the technology object of inquiry, i.e. CAI tools, within the larger framework of technology applied to interpreting. After an overview of the available technologies for interpreting (\sectref{tech_overview}), CAI tools are discussed in detail (\sectref{CAI_overview}), and InterpretBank, the CAI tool chosen for the experiment, in particular (\sectref{IB}). The chapter offers a review of interpreters' practices in compiling terminological resources ahead of and during the interpreting assignment, specifically in terms of their level of computerisation and choice of tools. This section closes with a review of how CAI tools have been studied in interpreting research thus far (\sectref{CAI_evaluation}) and illustrates current attitudes towards CAI tools (\sectref{CAI_attitudes}), which motivates the present work. Hence, in \chapref{chapter3} I proceed to discuss simultaneous interpreting as a complex cognitive activity, specifically as a question of attention allocation and resource sharing between co-occurring subtasks (\sectref{workingmemory}), a key issue in the inquiry into technology-supported (simultaneous) interpreting. \sectref{CL} illustrates cognitive load as a fundamental construct often encountered in academic discourse around CAI tools, but not yet explored experimentally. These two sections pave the way for the discussion of interpreting as an issue of multi-tasking within the area of Interpreting Studies (\sectref{interpreting_cognitive}). Here, I illustrate in detail two models of SI which address this activity from the perspective of the concurrent performance and coordination of cognitive sub-tasks. On this basis, I motivate the choice of framework to operationalise hypotheses on SI with digital terminological support (\sectref{choiceofmodel}). \chapref{chapter4} describes the methods adopted in TPR and neighbouring disciplines (e.g. cognitive psychology) to measure cognitive load. \chapref{chapter5} presents and discusses the research approach (\sectref{approach}) and the methodology deployed to test the hypotheses formulated for the present study (\sectref{hypotheses}). \sectref{pilot_study} presents the methods and results of the pilot study conducted to test the research methodology and validate the stimuli to be used for data collection. \sectref{main_study} goes into the details of the experimental design adopted in the main study, describing the adaptations conducted in light of the results of the pilot test. In \chapref{chapter6}, the results of the experiment are presented and discussed against the background of the hypotheses formulated in \sectref{hypotheses} and of relevant publications in the area of TPR and CAI research. \sectref{validation} discusses and validates the application of \citet{seeber_thinking_2007,seeber_cognitive_2011,seeber_multimodal_2017} CLM of SI to illustrate task interference and cognitive load in SI with the support of traditional digital glossaries, CAI tools with manual look-up, and ASR-enhanced CAI tools. The limitations of the present study are addressed in \sectref{limitations}. \chapref{conclusions} presents the methodological, didactic and practical implications of the present study and concludes this work with final remarks on potential avenues for future CAI research.
