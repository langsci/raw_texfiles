\appendix

\chapter{Speech A}

\ea Commissioner, thank you once again for inviting me to today's conference.

I would like to start my speech by providing some information about \textit{biocrude}.

At present, the efforts to produce it are still limited, but things are changing.

\ex Before talking about its production, let me address some of its chemical properties.

Perhaps the most interesting one is the fact that this kind of fuel has a low \textit{flash point}.

This is very important because it means that less energy is required to produce it.

\ex I believe we should support research in this field, because the advantages don't end here.

One should also mention that its \textit{pour point} is also much lower than for other fuels.

But this conference is not about chemistry, so let's move on to the next point.

\ex Now, of course the production of this fuel is quite complex, as is often the case.

It first undergoes what is known as \textit{hydrotreating}, which is quite common in this industry.

After that, the fuel is ready to be used in transport instead of more harmful ones.

\ex To make the production of this fuel possible, various devices have been developed.

A good example is the \textit{bubbling fluidized bed}, which has a pretty long history.

It is quite clear, however, that we need to pursue many options in terms of technology.

\ex These kinds of devices I have just talked about come with some disadvantages.

The biggest one is that they require an \textit{inert carrier gas} to operate, unlike other devices.

This might lead to a series of issues in terms of quality of the end product.

\ex There are other options that make our transport sector less harmful to the environment.\label{ex:speechA:Tag2}

By focusing, for instance, on \textit{woody biomass fuels}, we can truly make a difference.

They have the potential to help us respond to the challenges we're facing.

\ex These fuels can be used in road, railroad and also aviation transportation.

A real breakthrough in the transport sector could be represented by \textit{water splitting}.

Luckily, there are already many different types of clean fuels that can be used in transport.

\ex As you can imagine, the production of these fuels involves very advanced technologies.

New technologies are replacing \textit{plug flow reactors}, which present some issues.

And I think that producing clean fuel shouldn't harm the environment. 

\ex I don't want to bore you too much, so I'm not going to name all kinds of clean fuel.

However, I would like to provide you with a good example: \textit{rapeseed methyl ester}.

I know this name may sound intimidating, but it's actually just fuel.

\ex Clean fuels are obtained through different processes and with different methods.

Of the two main ways to obtain them, the most innovative one is \textit{transesterification}.

It's however a rather complicated chemical process, so I won't go into too much detail.

\ex I would like to mention, however, that this process presents a lot of advantages.

For instance, it makes it possible to reuse materials such as \textit{spent bleaching earths}.

This shows that simple materials can be used to produce clean fuel.

\ex Fuel of this kind can be obtained, for example, through a process that extracts oil from seeds.

At the end of this process, we have a \textit{residue cake}, which is what is left after seeds are crushed.

Depending on its characteristics, this residue can then be used as a fertiliser or to feed animals.

\ex But what are the resources we can use to produce these kinds of fuels?

The first category involves, for instance, \textit{short rotation coppice}.

Let me give you some more information before moving on to the other two categories.

\ex These resources are used to produce more efficient and less harmful fuels.

Fuels of this sort are still quite new and can be derived, for example, from \textit{switchgrass}.

The new technologies used in this field are expected to help expand production as well.

\ex The second category involves resources that are derived from the resources in the first category.

These resources can be obtained chemically, as is the case for \textit{black liquor}.

They can, however, also be produced using physical or biological processes.

\ex This means that we have a lot of different methods at our disposal.

Let me just briefly mention \textit{pulping}, which will be further discussed this afternoon.

I am sure my colleagues will be able to tell you something more about it.

\ex Unfortunately, we don't have much time to address this topic in particular.

I would like, however, to mention \textit{liquid rosin}, which I know a bit more about.

Unlike the resource I mentioned before, it is the result of a natural process. 

\ex The third category is nothing special as it involves all kinds of waste from human activities.

Another good example I forgot to give you for the first category is \textit{corn stover}.

As you can see, it is possible to use very common materials to produce fuel.

\ex There are, however, also other sources from which clean fuels can be derived, such as wheat.

To this aim, wheat must undergo two different processes, the first of which is called \textit{milling}.

This simply means that the cereal is cut into much smaller pieces.

\ex After undergoing this first process, the second phase begins.

An important stage of this process is \textit{kilning}, which occurs under controlled conditions.

So far, I've been talking about clean fuels mainly used in the transport system.

\ex There is however another field I would like to describe in more detail.

A good alternative to coal, which can be used for heating purposes, is \textit{lignocellulosic solid biomass}.

The gas obtained from it can also be used to generate electricity.

\ex There are two main processes used to produce the necessary resources for clean heating.

The first one is called \textit{gasification} and what we obtain through it can also be used for cooking.

The gas can, however, also be converted to electricity or used for other applications.

\ex To perform this process, various devices can be used, although some are preferable.

Among these systems, the \textit{entrained flow gasifier} is an interesting device.

It is preferred because it has a lower impact on the environment.

\ex After the process, the gas is filtered and can be used to generate power more efficiently.

Sometimes, however, the efficiency of the process can be lower due to \textit{elutriation}.

This is why it is very important to focus on devices that limit this issue.

\ex Many African countries are already trying to find clean alternatives to traditional coal.

One of these is the production of \textit{green charcoal}, which looks to be very promising.

It is an environmentally friendly alternative to traditional charcoal, derived from vegetation.

\ex There has been a lot of media coverage about the negative consequences on the environment.

One of the first things that can be done to prevent them is investing in \textit{soil amendment}.

The negative impact of uncontrolled production is, unfortunately, quite wide-ranging.

\ex Luckily, there are various methods and tools that we can use to counteract such effects.

Let me briefly talk about \textit{biochar} – not a new invention, but a very useful one.

It can endure in soil for many years and has many potential benefits.

\ex On the one hand, it can be used to improve the fertility of our soils.

On the other hand, however, it may also play an important part in \textit{carbon sequestration}.

This is true especially when coupled with other processes and measures.

\ex We have witnessed improvements in many sectors and many countries.

In the transportation sector there has been an increase in the use of \textit{carbon sequestration}.

This technology makes trains more efficient and is good for the environment.

\ex To produce gas, research and development is exploring new possibilities.

In order to maximise production, researchers are working on \textit{steam methane reforming}.

This process comes with a lot of advantages, even though it is quite expensive.

\ex Despite the advantages, the process I just mentioned comes with some challenges.

One of these is something called \textit{coking}, which might have negative effects on production.

Additionally, the whole process is still considered quite costly and impractical.

\ex We all know that we must also start looking elsewhere if we want to protect the environment.

We need to keep investing in \textit{tidal barrages}, for example, which are proving to be a valid alternative.

By investing in this new sector, my country had an annual export of several billion euros a year.

\ex Let me conclude my speech by describing how we can harvest the energy of ocean waters.

The most famous system used to obtain this kind of energy is probably \textit{ebb generation}.

I guess all of us have heard about it before, so I don't need to explain this further.

\ex We can, however, also capture the energy derived from ocean waves.

For instance, we can use \textit{point absorber buoys} in order to generate electricity.

This is, however, only one of the four most common approaches and technologies.

\ex Over the past few years, new technological solutions have entered the market.

Experts have already been working to develop \textit{oscillating water columns}.

This is, however, still not enough, so Europe must keep investing to bring about real change.
Thank you.
\z

\chapter{Speech B}

\ea
Ladies and gentlemen, today I will talk about a topic that is very dear to me.

I will be talking about \textit{transmutation}, a process used to generate nuclear energy.

It is rather complicated, so I promise not to go into too much detail.

\ex Some of you may have already heard this term before, although in a different context.

It described the transformation of \textit{base metals} into gold attempted by alchemists.

I find the history of this word very fascinating, but let's not digress.

\ex I'm sure you will agree with me that nothing in life is without consequences.

Let's think about what happens when we build a \textit{storage dam}.

It is for sure useful to generate electricity, but also negatively impacts the environment.

\ex This is true also for nuclear energy, which generates a lot of waste of different types.

To start, I'd like to mention \textit{depletalloy}, which was long considered unusable.

It is only mildly radioactive, but is not, however, the only type of waste we get. 

\ex After its extraction, the raw material containing uranium is crushed into sand.

The useful material is then removed and we are left with \textit{mill tailings}.

They can be carried by the wind and enter our waters, which is not a desirable outcome.

\ex This is a complicated topic, because even apparently safe materials can be radioactive.

Even though we're talking about small amounts, this is true also for \textit{shale}.

But this is something that will be addressed by other speakers later today.

\ex Luckily, however, in this industry there are also less problematic types of waste.

You might have guessed that I'm referring to \textit{spent nuclear fuel}.

Even though it can no longer be used directly, it can be repurposed.

\ex So as you can see, even in this controversial sector there's a positive side of the coin.

Industry is focusing on \textit{mixed oxide fuel}, which is already widely used.

Who would have thought that recycling was possible in this sector, too?

\ex One of the main advantages of this fuel is that it's pretty easy to manufacture.

It is not used all over the world, but in some countries it powers \textit{breeder reactors}.

This certainly doesn't solve the problem, but at least reduces the amount of waste.

\ex Not all waste is equally dangerous and can be classified into three groups.

After initial treatment, some waste can be mixed with \textit{blast furnace slag}.

This is, however, not a solution feasible for the most dangerous material.

\ex Let's take a look at the other two categories that I haven't discussed yet.

The first one isn't very interesting, while an example for the second is \\ \textit{cladding}.

A problem, however, still remains: at some point you have to get rid of the waste. 

\ex And this is where things get tricky; this is what causes concern.

What makes people think twice is especially the issue of \textit{nuclear waste disposal}.

The waste can indeed remain radioactive for thousands of years.

\ex There are, however, solutions to this problem, but not all of them are good.

The first one is burying the waste in so-called \textit{deep geological repositories}.

But who would want to have these near where they live?

\ex I am quite sure no one would, and that's why so many people are against this type of energy.

And the waste has to go through \textit{vitrification} beforehand, so the process is actually not so easy.

So as you can see, things are actually not quite as easy as they might seem at first.

\ex What I just mentioned isn't, however, the only solution we could adopt.

One additional potential solution is storing the waste in \textit{boreholes}.

But because of its costs and impact on the environment, this is not implemented.

\ex Some scientists have already come up with very creative solutions.

It may actually be possible to transport the waste into space using \textit{mass drivers}.

The necessary technology is not there yet, but it might be available in the future.

\ex As you all know, the waste cannot be eliminated completely.

This is the reason why \textit{dry cask storage} is often used in this field.

This system comes in various designs, some of which can also be used for transportation.

\ex Research is constantly looking for new solutions to the issues I'm discussing today.

One such solution is simply storing the waste products in \textit{salt domes}.

Some, however, criticise this solution because they consider it dangerous.

\ex Another problem is what to do with the plants once they are no longer active.

Among the solutions to this problem, \textit{entombment} is just one of the many possibilities.

It is the least used option because it is very complex and requires continuous surveillance.

\ex After a plant has been shut down, some issues must still be solved.

Even with no reactions, some \textit{decay heat} remains, which is very dangerous.

For this reason, many people tend to discard the subject very quickly.

\ex The answer to all these issues lies in the process I'm talking about today.

It should not be confused with the process called \textit{fission}, which produces the waste.

That process is the one used to create the atomic bomb, such as that of Hiroshima.

\ex And then of course there's another process that we all know and learnt about at school.

The reaction I'm talking about naturally occurs in stars and is called \textit{fusion}.

It is the process that takes place in the Sun, at extremely high temperatures.

\ex This process can produce energy with a virtually unlimited supply of fuel.

For this and other reasons, \textit{tritium breeding} is already being tested and demonstrated.

As you can see, the process I'm discussing is feasible and can help get rid of the waste.

\ex It cannot, however, solve the problem completely, so we must keep looking for alternatives.

Let me just mention, for instance, \textit{airborne wind turbines}, which have proven to be very useful.

It may not sound like it, but I am in favour of this kind of technology, too.

\ex And I particularly appreciate the investments that have been made into it.

This has made possible to develop \textit{nacelles} made from more sustainable materials.

This would also reduce costs and make the technology accessible to more countries.

\ex I was reading about this the other day and I have to say things look very promising.

The reduction in the costs for \textit{onshore wind} is just one factor in favour of this source of energy.

Nonetheless, huge progress has been made in the sector we're discussing today.

\ex Part of this progress is linked with the possibility of using it for other applications.

Many countries today are discussing the need to promote \textit{cogeneration}.

Though not very widespread, this concept has already been applied in various industry sectors.

\ex It has significant potential and could open up many new markets in the mid and long term.

Partnerships between the public and private sector are underway in the field of \textit{desalination}.

Through this process, the same plant can be used to produce fresh water as well as energy.

\ex This is not only good for our communities, but especially for the environment.

In fact, we can make full use of \textit{brackish water} thanks to this process.

This is relevant for the Middle East, which will probably invest in the sector we're discussing today. 

\ex I believe that if we truly want to protect the environment, we must be flexible and open.

We cannot get rid of \textit{peaker plants} because they are still needed.

But I think focusing on new approaches can really help us take the next step.

\ex And this is true not only for clean energy, but also for traditional solutions.

Governments should start setting aside funds for \textit{fast burst reactors}.

The technology is already there, but developing it further would help cut costs too.

\ex On the other hand, technology is also needed to make sure the plants can last years.

That is why research is essentially focused on long-lived components, such as the \textit{reactor vessel}.

This component is not replaced regularly, so it's essential to ensure it can stand the test of time.

\ex I really believe that in the energy sector technology is the key and can help solve many current issues.

The industry is working on \textit{organically moderated reactors} to reach isolated markets.

The first prototypes are already available, but there's still much to be done in this sector.

\ex Technology will really be essential to shape the future of the sector that we're talking about today.

At the moment, for example, \textit{pressurised water reactors} are the most widespread kind.

There is only a handful of countries where this is not the case.

\ex They are very stable because they tend to produce less power as temperatures increase.

The second most widespread kind is the \textit{boiling water reactor}.

These reactors tend to have uniform designs and are very similar to one another.

\ex In order to improve sustainability and safety, research is trying to offer new solutions as well.

Among the designs currently available, the \textit{molten salt reactor} is one of the most promising.

I won't go into detail about this, but I hope my speech helped clarify any doubts.
Thank you.
\z

\chapter{Speech C}\largerpage

\ea Ladies and gentlemen, let me give you some information about the coal industry.

The vast majority of the world's electricity is generated using the \textit{vapour power cycle}.

The main source of heat used in this process is coal, used to generate steam.

\ex Its use is likely to continue in certain regions of our planet, even if it's very polluting.

Organic material derived, for instance, from \textit{peatlands}, can also be used as fuel.

We should, however, be very careful when we use natural resources.

\ex Mother nature is truly very generous, even though we often forget it.

When we think of the existence of \textit{carbon sinks}, this becomes immediately clear.

But let's move on to explaining how the cycle I'm talking about works.

\ex I guess part of the audience will already be familiar with this, but I'll just describe it briefly.

First of all, a large \textit{grinder} turns the coal that has been transported to the plant into a very fine powder.

This step is necessary to make sure that all the coal is burnt to maximize the production of heat.

\ex Unfortunately, this cycle requires a lot of water, so it's not good for the environment.

Before moving on to the next step, the water can be transported to a \textit{feedwater heater}.

This additional step in the cycle increases the efficiency of the system.

\ex It can, however, also lead to a pretty big issue, which shouldn't be forgotten.

Serious damage can be caused to the plant if the water doesn't go through a \textit{deaerator}.

It comes in very different types, but they all serve the same purpose.

\ex Let's now move on to describing the other steps and elements of the cycle.

After the first step, the coal ends up in a \textit{boiler}, where it is burnt.

This way, the coal provides heat to the power plant, which is used to produce steam.

\ex I know the audience isn't made up of engineers, so I won't be too technical.

However, I think I should also mention the importance played by the \textit{boiler steam drum}.

It's exciting to see so much technical progress being made in this respect.

\ex In the cycle I'm describing, there are also optional steps the steam can go through.

For instance, the steam can enter a \textit{superheater}, which can be pretty small or very big.

The main advantage of using it is that it reduces the consumption of water.

\ex There are, of course, also clean ways of generating energy using steam.

This requires using different technologies, such as the one involving \textit{flash steam}.

Nonetheless, coal is still widely used and this comes with huge environmental issues.

\ex So focusing on coal to generate energy is not a feasible solution, and let me tell you why.

Coal has the most substantial \textit{carbon footprint} of all fuels.

For this reason, this source of energy will have to be used less and less.

\ex As you can see, what this industry is promoting is not a real solution.

I cannot deny that there are ways to reduce its impact, like using \textit{baghouses}.

But it still remains a messy business and a pretty complex one.

\ex So before discussing the desirable alternatives, let's try to understand it better.

Among the different methods used to extract coal, we can mention \textit{longwall mining}.

It is the one that makes best use of the natural resources present underground.

\ex As we all know, coal is not readily available on the surface.

In order to reach it, the \textit{overburden} must first be removed.

This is done mechanically nowadays, but this wasn't the case in the past.

\ex Not all of the coal is used to produce heat and, ultimately, electricity.

Part of it can be used in other industries, for example in the case of \textit{middlings}.

This is, however, a totally different topic, so let's move on.

\ex The coal doesn't reach the plant immediately; its journey is a bit longer.

In order to remove all impurities, the coal must first reach a \textit{coal handling plant}.

This increases its value and at the same time lowers its transportation costs.

\ex To achieve better quality, the coal must undergo a series of processes to be purified.

One such process requires \textit{dense medium separation} to discard useless material.

There are many different types of it, but we don't need to name them.

\ex I'm really just trying to give you an idea of the complexity of this industry.

Coal contains water, so a \textit{screen bowl centrifuge} can be used to remove it.

The good thing about this process is that water can also be recycled.

\ex When it comes to this industry, however, there are really a lot of negative aspects.

Let's think, for instance, of the many lethal accidents caused by \textit{firedamp}.

This industry is extremely dangerous for the workers, as I'm sure you all know.

\ex Using coal to generate electricity also has a devastating impact on the environment.

Let me just mention the amount of \textit{flue ash} that contaminates our waters.

And we all know how precious water is, especially in today's world.

\ex Plants using coal as fuel are also the biggest water polluters, at least in my country.

Every day, these plants dump millions of litres of \textit{sullage} into rivers and lakes.

So it's quite clear that we need to use clean sources of energy, like that of the sun.

\ex Solar energy is within the reach of individuals and small businesses.

It is expected to play a major role in the future \textit{power generation mix}.

There have been a lot of developments in this field, so let's discuss them.

\ex First of all, research is focusing on improving some components of the system.

For instance, industry seems to be increasingly in favour of \textit{negatively doped wafers}.

This is by far the best choice available on the market, because it's the most efficient.

\ex They are, however, still quite expensive, because there are only a few manufacturers.

The recent introduction of \textit{interdigitated back contacts} has made things easier.

They make the cells more efficient and easier to place closer together.

\ex It is also very important that storage is efficient and nothing goes to waste.

This has been made possible with the creation of batteries with better \textit{float life}.

There is a wide range of devices used to harvest the energy of the sun.

\ex They are usually classified into three generations, depending on the basic materials used.

The first generation uses \textit{crystalline silicon}, the material most commonly used in this industry.

This is a mature technology that currently dominates the market and is in mass production.

\ex Despite being an advanced technology, its costs can be reduced through improvements in materials.

Manufacturing also plays a role, for example when it comes to the production of \textit{boules}.

This is however not the only technology available on this market at the moment.

\ex After many years of research and development, a new technology is beginning to be deployed.

The second generation focuses on another kind of technology called \textit{thin film}.

It could potentially provide lower-cost electricity than first generation cells.

\ex Three primary types of this kind of cells have been commercially developed, using different materials.

The most widely known type is based on \textit{amorphous silicon}, which has pretty good efficiency.

Then there are other technologies, some of which are still in the early stage of development.

\ex I'm going to talk about three of the four technologies of the third generation.

The first of these new technologies are the so-called \textit{concentrating photovoltaic systems}.

They offer a pretty high level of efficiency, which is even higher in experimental settings.

\ex Let's now move on to another solution that is part of third generation technology.

I'm talking about \textit{evacuated tube collectors}, which use low-cost materials.

They are also simple to manufacture, but their performance can degrade over time.

\ex Then there are a number of innovative technologies, all very promising.

Good examples of those technologies are the ones that rely on \textit{superlattices}.

Now I'm not going to into detail about this, because it's rather complicated.

\ex It is beside the point I want to make, but I think it's always interesting to know what's out there.

Another type of these new, third generation technologies is based on \textit{quantum dots}.

These are very small particles that can conduct electricity and are used in other fields too.

\ex Finally, there are some ideas that haven't yet been developed fully.

Some of those have a huge potential, like \textit{solar updraft towers}.

They have been criticised in the past, but could become increasingly relevant.

\ex A very positive thing about this technology is that it is very versatile.

Let's think for example of \textit{floating solar arrays}, which can be installed on water.

They represent a new development and make this technology more accessible.

\ex There are many positive aspects when it comes to these technologies.

The most evident one is that they allow us to easily reach \textit{socket parity}.

So as you can see this is a good alternative to coal, which we should focus on in the future.
Thank you.
\z
