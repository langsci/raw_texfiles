\chapter{Terminology work in interpreting} \label{chapter1}

The present chapter discusses the role of terminology in simultaneous interpreting at specialised conferences, highlighting the commonalities with translation while at the same time foregrounding the specific requirements of terminology work in interpreting. Against this background, the ideal features of a bespoke tool for terminology work in the different phases of interpreting are discussed. To conclude, the chapter highlights the limitations of terminology tools available to translators, thus motivating the need for interpreter-specific tools, which are discussed in the following chapter.

\section{Translating and interpreting for LSP} \label{LSP}
Translation and interpreting share a large set of common features, as both activities concern the interlingual and intercultural transmission of a message. The most apparent element of distinction is perhaps the channel through which the target text is produced: written, in the case of translation, and oral or visual, in the case of interpreting and signed language interpreting, respectively. This distinction is usually adopted for ease of explanation, for instance in communication with laypeople. However, translation and interpreting are in numerous respects quite similar. For instance, hybrid forms of translation and interpreting are possible: one such instance is sight translation (ST), which consists in the spoken translation of a written source text. With the advent of new possibilities offered by technology, additional hybrid forms of translation and interpretation are emerging, such as live subtitling (or interlingual respeaking), in which the interpreter produces a live translation of the speaker's words that is, however, rendered in written form. Second, TPR has elucidated that cognitive control functions are very similar in translation and interpreting. Both translation and interpreting involve the performance of parallel sub-tasks (e.g. \citealt{vardaro_translation_2019}) and the first draft of a translation does not require significantly more time to be produced than a target text (TT) in interpreting.

Translation and interpreting also share another characteristic: a large amount of texts translated or interpreted are specialised texts. The profile of translators is thus increasingly becoming that of a specialised translator \citep{campo2005politics}; specialised conferences, technical meetings, seminars or workshops represent the most common type of assignment for professional conference interpreters, both for those working in the private market and those employed by public institutions (e.g. \citealt[22]{rutten2008zielgerichtet}).

When translating or interpreting in specialised fields, translators and interpreters face a common challenge: they are called upon to enable expert communication across languages despite not being subject-matter experts themselves. This is particularly true for interpreters, who often work ``in very different thematic scenarios, usually for a public of experts, and are given the task of transmitting highly specialised knowledge'' \citep[21]{rodriguez_look_2009}. Despite the increasing expertise acquired by translators and interpreters during their careers, specialised translation and interpretation in technical settings remains a complex task for both groups of professionals. This complexity is related to the demands imposed on translators and interpreters by specialised language (see \sectref{featuresLSP}) and to the expectations of expert readers or listeners (see \sectref{termquality}).

\subsection{Features of LSP} \label{featuresLSP}
As specialised communication is considered primarily informative \citep[427]{olohan2013scientific}, the focus is on the content, which must be conveyed with the highest accuracy possible. A key characteristic of specialised communication is thus the use of languages for special purposes (LSP, \citealt{desblache2001aspects,scarpa2010traduction}), the in-house jargon \citep{kalina2006dokumentation} adopted by experts in a specific subject matter to convey specialised knowledge to other experts or laypeople. LSP thus ``has a precise purpose and function, that is, to specifically communicate concrete information clearly to a target set of users so that it can be productively used'' \citep[207]{folaron2019technology}. LSP can therefore be seen as the way specialised knowledge manifests itself and is constituted in discourse.
As compared to ordinary language, LSP presents several specific features concerning its lexicon as well as its phraseology and syntax. LSPs may therefore be described as ``contextual-functional varieties of the ordinary language \citep{garzone2006perspectives}, […] characterized by specific morpho-syntactic forms and by some discursive and pragmatic features'' \citep[134]{pignataro2012terminology}.

Especially in scientific contexts, one such characteristic is the widespread use of English as a lingua franca \citep{ammon2001dominance}. As remarked for instance by \citet{hansen2017predicting}, this might result in shining-through effects in translation, especially if the language is particularly susceptible to Anglicism (as is the case for German). Additionally, \citet{braun2014} evidence how academic texts present a high number of Latinisms and Grecisms in addition to Anglicisms.

This type of influence combined with similar etymology, especially for certain language pairs, may favour the use of cognates, i.e. ``those translation words that have similar orthographic-phonological forms'' \citep[1285]{costa2000cognate}, in the translation of specialised discourse. On a cognitive level, cognates have been found to be easier to process. This is referred to as the cognate facilitation effect \citep{costa2000cognate}. The use of terms with etymological roots shared by different languages may therefore to some extent facilitate access to scientific discourse.

Another feature of LSP is a certain preference for nominalisation and especially for complex noun phrases or compounds \citep[247]{olohan2008scientific}, especially in English and other Germanic languages. These linguistic devices are functional to the economy of expression which is essential to specialised language \citep[135]{pignataro2012terminology}. With reference to \citet{halliday2003writing}, \citet[428]{olohan2013scientific} remarks in this respect that:

\begin{quote}
This shift from verb to noun, they argue, is significant because the meaning construed by the nominalization is a new one – an abstract theoretical entity which forms part of a scientific theory. These and other features, like expanded nominal groups, privilege experts and exclude others from accessing scientific discourse.
\end{quote}

In translation and interpreting, these linguistic features of LSP pose an additional layer of difficulty as elliptical, multi-word expressions must be re-coded in the target language. This may prove particularly problematic if the target-language equivalent is not readily available (see \sectref{featurestermwint}).

Finally, LSP is characterised by specific phraseology and syntax \citep[426]{olohan2013scientific}, as well as specialised terminology, which translators and interpreters must use ``as a means to achieve the interlinguistic transfer of specialized knowledge units'' \citep[447]{velasquez2002traduccion}.

\subsection{Terminology as a parameter of quality} \label{termquality}
In light of the status of terminology for effective specialised communication, it is not surprising that terminology is acknowledged as an important parameter of quality in translation and interpreting. It may be affirmed that ``the adequacy of the terminology in a text as well as its suitability for the level of specialization determines to a great extent the quality of a translation'' \citep[91]{martinez2009terminological}.

While terminology is not the only benchmark against which the quality of a translation may be assessed, its relevance as an important element of quality emerges for instance through its inclusion in many evaluation frameworks, both for human and for machine translation (MT). For instance, \citet{mertin2006prozessorientiertes} developed a typology of translation errors which includes terminology as one of the categories of the evaluation matrix. The Multidimensional Quality Metrics framework (MQM, see \citealt{lommeletal2014MQMdef,lommeletal2014using}) for the assessment of translation quality includes terminology as one of the higher-level issue types. Terminology is one of the 20 ``core'' issue types included in MQM Core, a simplified version of MQM, and it is included as a key issue type in the TAUS DQF Error Typology \citep{gorogQualityEvaluationToday2014,gorogQuantifyingBenchmarkingQuality2014}, a subset of MQM. The SAE J2450 Translation Quality Metric for Language Translation of Service Information \citep{SAEJ24502001} also includes the issue type ``wrong term'', which can be mapped to the ``terminology'' issue type in MQM. Similarly, Hjerson \citep[59]{popovic2011hjerson}, a framework for automatic classification of errors in machine translation output, includes ``incorrect lexical choice'' as one of the error classes. These are but a few examples, but they highlight the role attributed to terminology for the assessment of translation quality.\largerpage

By the same token, especially in the context of interpreting settings where LSP is used, e.g. specialised conferences and technical meetings or workshops, terminology emerges as an important element of quality, although it is but one parameter \citep[231]{garcia2011terminologie}. The weight attributed to terminology for the evaluation of interpreting quality may also be said to reflect the overall approach to interpreting and its evaluation. As discussed by \citet{pochhacker_quality_2001, pochhacker_introducing_2004}, interpreting has been addressed essentially from two perspectives: on the one hand, it has been viewed as a primarily linguistic task of text reception and production; on the other, as a chiefly communicative, socially-embedded task. Terminology as a quality parameter tends to play an important role in evaluation frameworks belonging to the first approach, as here the focus is on the product. This approach reflects the expectation of equivalence between the interpreted and the source speech, and is reflected in the notions of accuracy and fidelity \citep{larson_communicative-oriented_1991}, resulting in what \citet[155]{bowen_thoughts_1990} defined as ``equivalent effect''. This approach is exemplified by \citegen{barik_description_1971} taxonomy of error and omission types in simultaneous interpreting, seen primarily as linguistic production.
The second approach focuses on interpreting as interaction. Here, the evaluation of quality focuses on the communicative aim of interpreting as situated in a real communicative event. Rather than equivalence, what is important for scholars choosing this approach is the efficacy in achieving a pragmatic communication goal. In this sense, important contributions stem from research on community interpreting, which by its nature promotes the involvement of all players in the communication triad, i.e. the interpreter and the communication parties. An example of this approach to the evaluation of interpretation is \citet{wadensjo_interpreting_1998,wadensjo_interpreting_2005} evaluation framework.

In the context of specialised conferences, the adequate use of specialised terminology is particularly relevant in terms of the clients' and end-users' expectations, because precision, economy of expression, and accuracy are considered key aspects of specialised discourse.
It is not surprising, then, that several surveys on quality in simultaneous interpreting have highlighted the use of correct terminology as one of the most important benchmarks of perceived quality (e.g. \citealt{gile1990evaluation,pochhacker_simultandolmetschen_1994}), both in terms of the expectations on and of the evaluation of the interpreting service, and both among end-users and interpreters themselves.
Many such user surveys have been conducted over the years, predominantly through questionnaires sometimes combined with other methods, such as interviews (\citealt{mack_user_1995,vuorikoski1993simultaneous,vuorikoski1998user}, see \citealt{garciabecerra2016researchquality} for a discussion on administration methods).

For instance, a survey by \citet{kurz_conference_1989,kurz_conference_1993} regarding user expectations on interpreting quality found that correct terminology was the third out of eight factors in order of importance, after ``sense consistency with the original message'' and ``logical cohesion of the utterance''.

\citet{meak_interpretation_1990} conducted a small-scale survey on 10 Italian doctors. The results on the role of correct terminology are less conclusive in this case, but this is to be expected considering the limited sample. Nonetheless, even the most lenient respondents stressed that too frequent imprecise use of vocabulary may prove distracting.

\citegen{marrone_quality_1993} survey on a sample of 87 conference attendees combined expectations and evaluation and found that ``inaccurate terminological usage'' (p. 37) was considered as a shortcoming by most respondents, more serious than unpleasant delivery. Quality of style and correct terminology were ranked second in order of importance after information completeness and before intonation and delivery.

\citet{kopczynski1994quality} explored the expectations of 57 Polish speakers and receptors of interpreting and found that although conference attendees may in some cases show a certain degree of leniency towards the use of incorrect terminology, this may be perceived as one of the main irritants. Terminological precision was identified as the second most important parameter of quality, independent of the respondents' role and professions. By the same token, in a survey by \citet{pochhacker_simultandolmetschen_1994} on quality evaluation, ``mastery of technical language'' was ranked second in order of importance after the quality of verbal expression.

\citet{mack_user_1995} surveyed user expectations in Italy. In their research, correct terminology was identified as the most important quality factor. The findings of \citegen{weller1998audience} user survey are also in line with Mack and Cattaruzza's.

An AIIC survey presented in \citet{moser_survey_1995,moser_expectations_1996} involved 94 users of interpreting who were interviewed based on a structured questionnaire. Terminological accuracy was deemed important especially by experienced attendees of technical meetings. It was even considered more important than completeness. For general meetings, the opposite was true.

In sum, end-users attending technical conferences tend to rank the use of correct terminology among the top quality factors for interpreting. Completeness of rendition, knowledge of the subject matter and precise terminology tend to be assigned higher values than delivery-related aspects such as native accent or pleasant voice.

Interpreters also recognise the role of terminology as a key quality parameter. The first survey on interpreters' expectations was conducted by \citet{buhler_linguistic_1986} on 41 AIIC members and 6 members of the association's Admissions Committee. Both linguistic and extra-linguistic criteria were included. Among linguistic criteria, ``use of correct terminology'' (p. 232) was ranked particularly high (83\% highly important) by the members of the Admissions Committee, similarly to the ranking for ``sense consistency with the original message'', which received the highest relative and absolute ranking by all participants. As Bühler suggests, this might indicate that using accurate terminology can promote a faithful rendition of the message. Indeed, the use of adequate terminology may contribute to improve performance as it favours cohesion and the correct transmission of the message \citep[219]{garcia2011terminologie}.

A larger number of respondents were reached through online questionnaires in the 2000s. In a survey by \citet{chiaro_interpreters_2004}, involving 286 respondents, interpreters ranked correct terminology usage as one of the second-most important factors in addition to fluency of delivery and correct grammatical usage.

\citet{pochhacker2010survey} collected survey data on interpreters' expectations and self-perceptions, which the authors see as ``inherently linked with the issue of quality'' (p. 11). They partly replicated Bühler's survey by adopting the same quality criteria, although their questionnaire also involved an evaluation of recorded interpretations. As for the role of correct terminology, the results were similar to Bühler's. Zwischenberger and Pöchhacker, however, also explored how the relative weight assigned to the individual criteria is affected by the type of interpreted event. They found that correct terminology was listed as a top priority by 38\% of the respondents who cited seminars and workshops as a setting with specific quality requirements (ibid., p. 16).

In light of these survey results, it appears clear that both interpreting service providers and users acknowledge terminology as an important quality criterion. Therefore, interpreters may more easily meet the quality expectations of their audience by using adequate specialised terminology. This goal can be achieved through effective domain knowledge acquisition and proper terminological preparation. Against this background, the following section will discuss the role of terminology work, with a focus on interpreting as compared to translation.

\section{Terminology work in translation and interpreting} \label{terminology}
The contribution of the terminological discipline to the field of specialised translation has been fundamental. For this reason, translation scholars and practitioners tend to consider terminology as an integral part of the professional translation practice, as testified by its inclusion in training programmes. However, terminology work in translation and interpreting presents specific features which are related to the nature of translation and interpreting and to the purpose served by specialised terminology in these professions. For these reasons, while the terminological discipline has laid the foundations for terminology work in translation and interpreting, the approach to terminology in these fields has evolved to adapt to their specific requirements, giving rise to dedicated models of translators' and interpreters' terminology work.


\subsection{Terminology and translation} \label{terminology_translation}
\citet{wuster_internationale_1931,wuster_einfuhrung_1979} may be considered as the nestor of terminology, for which he and the ``Viennese school'' \citep{drewer_terminologie_2019} claimed the status of independent discipline. His normative intent, formulated in his ``General Theory of Terminology'', pursued the goal of ensuring unequivocal communication in technical fields. He saw terminology ``as a tool for disambiguating scientific and technical documentation and communication'' \citep[17]{cabre_castellvi_we_1998}. Specialised communication requires brevity and clarity, which is only achieved through adequate linguistic tools, i.e. terms.

Terminology thus emerges as essential both for intralingual and interlingual specialised communication \citep{arntz_einfuhrung_1982}, be it in the form of translation or interpreting. In Wüster's model, which expands \citegen{saussure_course_1959} triangle, the term is described as an entity made up of a denomination, the ``symbol'' in \citegen{ogden_meaning_1923} semiotic triangle, or \citegen{saussure_course_1959} ``signifier'', and a concept, the signified. Wüster includes the distinction between \textit{langue} and \textit{parole} in the model itself \citep[163]{mikkelsen_arntz_1991}, a distinction which appears both in the signified (content) and in the signifier (expression).

While Wüster's model lays the ground for a structured terminological discipline, it presents several limitations which determined its later expansion by loyalists or its firm critique and rejection by competing schools of thought \citep{cabre_castellvi_theories_2003,drewer_terminologie_2019}.\footnote{Although, as \citet{drewer_terminologie_2019} rightly observe, ``the differences between these schools [are] too small to be able to talk about different schools at all'' [my translation from the German original].}

One such limitation is the exclusion of the individual contexts in which terminology is inevitably embedded, and which determines a certain degree of variability \citep{will_terminology_2007}. Therefore, translators and interpreters should not rely on parallel texts as reliable sources for the mining of terminology and the creation of ontologies, since in Wüster's theory terminologies are seen as universal, super-ordinate structures, unaffected by cultural differences. To overcome the limitations of Wüster's approach, \citet{gerzymisch-arbogast_termini_1996} proposed a context-specific term model which allows for a comparison of term meaning across texts, as used by the individual authors. The establishment of knowledge systems of this type requires static textual environments, as in translation.

\subsection{Terminology work in translation} \label{termworktrans}
At the beginning of the 1980s, terminology positioned itself as a necessity for translation \citep[84]{drewer_terminologiearbeit_2019}. Perhaps the most influential contribution to terminology theory and practice applied to translation can be identified in \citegen{arntz_einfuhrung_1982} ``Einführung in die übersetzungsbezogene Terminologiearbeit''. The authors offer a detailed discussion of Wüster's model and provide indications for terminology work in translation. Unlike for terminologists, translators' terminology work is always descriptive rather than prescriptive, thus aiming to pin down the knowledge system of a particular field and its expression through terminology (intended both as terms and as syntagma and collocations). While ideal terminology work proceeds onomasiologically, i.e. from the signified to the signifier, translators (and especially interpreters, as discussed in \sectref{terminology_interpreting}) often proceed semasiologically, i.e. from the expression to the meaning. As \citet[103]{drewer_terminologiearbeit_2019} observes, the onomasiological approach is only rarely used in terminology work by translators because of the considerable effort it entails.

As remarked for interpreters (\sectref{LSP}), it is not required of translators to be expert in a specific specialised field. Rather, like ad-hoc terminologists and terminographers \citep{wright1997terminology}, they must be able to quickly acquire the elements of expert knowledge necessary to fill their knowledge gaps and deliver a high-quality translation. Thus, terminology work is conducted as part of the translation process \citep[104]{martinez2009terminological} and is strictly intertwined with the individual context and co-text in which specialised terms are embedded. This further denotes how a prescriptive, onomasiological approach to terminology and terminography is not reflective of the way translators work on specialised texts.
In their terminology work, translators can make use of a variety of tools and resources that ``can contribute to facilitating and accelerating the identification, description, consultation and reuse of terminology in a translation context'' \citep[74]{kageura2019terminology}. These tools and resources comprise terminology databases, terminology extraction tools (TET) from monolingual and bilingual comparable and parallel corpora (especially previous translations), concordancers, tools for terminology research, text-alignment software, and terminology management systems (TMS). A translator's workbench typically integrates all or most of these technologies \citep[211]{hansen2012nutzbarkeit}. While translating a text, translators can consult multilingual glossaries and terminology databases to look up target-language equivalents, find definitions, etc. \citep[4]{blancafort2011user}.

A translation project often starts with monolingual term extraction from the source text or other documents. The source text can be pre-processed to identify and extract candidate terms; previously translated and aligned texts can be used to create terminology databases as parallel corpora. Terminology extraction can be performed both by project managers and in-house terminologists and made available to translators, either for entire domains or for individual translation projects and documents. In principle, terminology can be extracted manually, for instance by reading and annotating the text. However, due to the increasing volumes of texts to be translated and the increasingly shorter time to production, manual terminology extraction is often excessively time-consuming. Here, automatic terminology extraction (ATE) can provide valuable support in combination with subsequent refinement to exclude pseudo-terminological units \citep[88]{pavel2001handbook}. Bilingual ATE can be performed on parallel corpora \citep{vintar2001using}, often in the form of translation memories (TM), to create termbases containing terminological pairs. Especially for non-standardised and emerging domains, parallel corpora \citep[263]{blancafort2010ttc} or authoritative databases or handbooks may not be available \citep[586]{heid2012term}. This lack can be addressed by the use of comparable corpora, i.e. of ``texts of the same domain (and possibly genre) in different languages which need not be translations of each other'' \citep[1]{blancafort2011user}. Incidentally, the results of terminology extraction are not only directly useful to translators but can be fed to further tools which translators can use to support their translation workflow, such as computer-assisted translation (CAT) tools or MT systems \citep[588]{heid2012term}. Most CAT tools include TETs, such as MultiTerm Extract in SDL Trados, which require parallel texts.

Corpora can be further analysed through concordancers to explore how terms are used in context to derive the most relevant phraseologisms \citep[89]{pavel2001handbook}.
Some freely available examples are Linguee\footnote{\url{https://www.linguee.com} (Accessed: 10.09.2021)} or Reverso Context\footnote{\url{https://context.reverso.net/translation/} (Accessed: 10.09.2021)} for the exploration of parallel corpora. Sketch Engine\footnote{\url{http://www.sketchengine.eu/} (Accessed: 06.09.2021)} \citep{kilgarriff2014sketch} is also a popular concordancer which offers monolingual, bilingual, or multilingual corpora analysis.

Terminological resources can be used for the immediate translation task, but they can also be further stored in dedicated terminology management tools to be consulted and reused in future translation tasks. Although dedicated tools are available to this aim (e.g. termbases in TMSs), the use of spreadsheet software or other general-purpose solutions to store and exchange terminology is also rather widespread (e.g. \citealt{SDL2008survey,blancafort2011user}).


\subsection{Terminology work in interpreting} \label{terminology_interpreting}
Even though interpreting shares many similarities with translation, it presents some distinctive features that affect the way terminology work is conducted to ensure effective interlingual oral communication.

\subsubsection{Distinctive features of terminology work in interpreting} \label{featurestermwint}
Translators work with a source text fixed in writing, which they can revisit as often as needed during the task. They can scan the text to clarify doubts by analysing the co-text, conduct terminological and content-related queries to fill their knowledge gaps during the task, and in theory pace the translation process as they require. The final product is therefore seldom the first draft produced, but rather the result of a process of increasing refinement. In interpreting, the source text is seldom available. Even when interpreters have access to the script of the speech before the assignment, they encounter the final speech as pronounced by the speaker only in the moment in which it is actually delivered, i.e. during the assignment. Interpreters are therefore often faced with the task of foreseeing which topics will be addressed in the speech and, consequently, which terminology may be used. Therefore, they must strive to fill their domain-knowledge and terminological gaps ahead of the task \citep[22]{rutten2008zielgerichtet}. Additionally, interpreting is performed under severe time constraints and cannot be interrupted. It would be unthinkable for interpreters to stop their rendition of the speaker's message because they need to acquire additional knowledge to correctly understand and transfer the meaning. The possibilities of revision and monitoring are also more limited in interpreting, as the target text produced by interpreters cannot be further refined or modified \citep[38]{will2020computer}. Indeed, self-corrections or reformulations are often regarded as undesirable in interpreting. For interpreting assignments, the process of knowledge and terminological acquisition is thus mainly relegated to the pre-process phase and can only be integrated to some extent into interpreting either peri- or in-process \citep[778]{kalina_quality_2005}, i.e. during the assignment.

Terminology work in interpreting thus aims to avoid knowledge gaps \citep[25]{rutten2008zielgerichtet}. Such knowledge gaps are, however, not only related to an interpreter's overall general knowledge or knowledge of the subject field, but are also situation-related. While conferences and their interpretations can be recorded and made available even after the event has ended, the communication taking place during the conference is mostly meant to serve an immediate purpose and aimed at a specific audience. The situatedness of speeches and presentations also determines the situatedness of their interpretations. What matters in an interpreter's preparation is its suitability for the assignment at hand; terms are considered adequate if they help communication. Thus, terminological preparation is also more pragmatic and situation-oriented than the terminology work of translators \citep{rutten2012pragmatik}, because the product is usually meant to remain available for a longer period of time, and especially of terminologists, whose work must remain valid also beyond the individual text \citep[49]{rutten2013TM}. The pragmatism and situatedness of interpreters' terminology work emerge in the type of information contained in interpreters' glossaries: the terms chosen may be even very simple and general if the interpreter deems them more suitable to ensure effective multilingual communication. As remarked by \citet[43]{rutten2012TM}, ``translators and terminologists are far less free to generalize (and interpreters, in a way, are meant to generalize if the situation calls for it)''.

Another difference between terminology work in translation and interpreting is due to the interpreting mode used in specialised conferences. In this setting, interpreters mostly work in the simultaneous mode, while consecutive interpreting is reserved to short presentations or other settings characterised by a higher degree of interaction between the speaker and the audience, e.g. technical workshops. Unlike in the consecutive mode, in the simultaneous mode the interpreter does not deliver the speech after the delegate, but rather almost at the same time as the speaker (for a discussion of interpreting modes, see e.g. \citealt{pochhacker2015routledge}).

The immediacy which characterises both modes of interpreting \citep[10]{pochhacker2011simultaneous} is particularly apparent in simultaneous interpreting; the external pacing of the task profoundly affects the cognitive and strategic processes underlying interpretation. Due to the limited amount of time and cognitive resources available during SI \citep[49]{rutten2013TM}, being able to quickly retrieve tar\-get-lan\-guage equivalents from long-term memory to properly render specialised discourse is paramount. This foregrounds the key role of terminology work pre-, peri- and in-process in interpretation for specialised conferences.
Conducting preparatory work ahead of the interpreting process allows for the internalisation of domain knowledge and specialised terminology, which can later be more easily recalled by interpreters, thus facilitating both their understanding of the source speech and their rendition in the target language. \citet{gile_basic_2009} describes the role of language availability in his gravitational model. When units of linguistic knowledge are readily available because they have been activated, for example through preparation, or because they are often used, they can be easily accessed by the interpreter, as they gravitate to the core of available lexicon. When this does not happen, i.e. when the units of linguistic knowledge (terms) ``drift outwards'', a series of issues can occur both at the comprehension and the production level. Activating terminology before the event is therefore paramount to prevent such problems. More specifically, shifting cognition upstream from the simultaneous phase \citep[3]{stoll_heidelberg_2010} allows for easier anticipation and faster understanding of the speaker and, in turn, for faster processing and a more accurate rendition. Finally, favouring the use of specific terminology over the adoption of alternative strategies such as paraphrasing, abstracting, or using hypernyms also helps interpreters avoid cognitive overload by freeing up cognitive resources thanks to a faster rendition. However, not all and not only domain-relevant terms are included in interpreters' glossaries: only those terms are noted which the interpreter fears not being able to retrieve from long-term memory, even though they are trivial words \citep[42]{rutten2011TMS}. This shows that terminology work in interpreting is also highly personal, corresponding to the individual interpreter's preferences, working and memory structures (\citealt[48]{rutten2012TM}, \citealt[9]{wagener2012vorbereitende}, \citealt[147]{rutten2018boothnotes}).

\subsubsection{Modelling terminology work in interpreting} \label{termwmodels}
For the reasons outlined thus far, interpreting does not allow for a direct application of \citegen{gerzymisch-arbogast_termini_1996} model. In interpreting, the context is in fieri and co-created by the communication parties, albeit to different extents according to the individual interpreting modes. Additionally, the time constraints inherent to interpreting, especially in the simultaneous mode, determine an even clearer preference for the semasiological approach, which saves time when compared to the ideal onomasiological approach \citep[168]{mikkelsen_arntz_1991}. It should be noted that expeditious terminology work is essential not only in terms of consultation, but increasingly also during the preparation phase, which has become progressively shorter \citep{rutten2008zielgerichtet}. Speeding up preparation also seems convenient in terms of the profitability of interpreting, as preparing for an assignment seems to require a considerable amount of time.\footnote{One or more working days according to \citegen{AIICworkload2002} workload study.}

At the same time, interpreters need to constitute knowledge systems by relying on ``external textual structures'' \citep[67]{will_terminology_2007} to be able to understand the texts they are interpreting. \citegen{will_terminology_2007} model of terminology work in SI closed the gap left open by previous models by considering the particular nature of interpretation and the conditions under which texts are produced and translated during SI.

Will's model represents an expansion and integration of his own model of knowledge management during a simultaneous interpreting assignment \citep{will_bemerkungen_2000}. In this first model, Will separated the interpreting assignment into three phases of knowledge acquisition: the preparation phase (Stage I), the conference itself (Stage II) and the de-briefing phase post-assignment (Stage III). Or, in \citegen[778]{kalina_quality_2005} terms, into the pre-process, peri- and in-process, and post-process stages, respectively. As Will argues, new knowledge in the form of single terminological units can only be constituted during Stage I and III and during the non-interpreting phases of Stage II (i.e. right before and right after the interpreting session). Additionally, Will draws on Gerzymisch-Arbogast's model as well as on \citegen{Mudersbach1999DieHB} and \citegen{floros_kulturelle_2003} models. Mudersbach's model integrates terminological units into their relative knowledge systems necessary for text comprehension, while Floros describes how relevant knowledge is selected.

As \citet[65]{will_terminology_2007} observes, ``it would be impossible [for conference interpreters] to acquire the same amount of knowledge as their specialised public''. They ``have thus to be able to constitute and to use relevant information in a very effective and specific way''. With some rare exceptions, conference interpreters working with LSP do not possess the same level of domain knowledge as the speakers and audiences they are called to interpret for. They are therefore presented with an arduous task: that of compiling terminological resources and acquiring sufficient information on the subject matter at hand to be able to precisely convey the speaker's message despite not being experts themselves. Quite often, this is further complicated by the limited time available for preparation. How, then, can they solve this conundrum?

According to Will's model of terminology work in SI, when preparing for a specialised conference, the interpreter must first of all attribute an individual term (``Texterm'') to a specific knowledge system, by comparing it with a reference definition (``Systerm''). This process leads to the creation of a ``Terminological Knowledge Entity'' (TKE), ``the smallest complete knowledge unit for understanding and producing technical texts'' \citep[69]{will_terminology_2007}. To obtain a picture of the general knowledge structures in a speech, TKEs are grouped to form what Will defines as ``Terminological Knowledge Constellations'' (TKC). The most relevant knowledge systems for the assignment in question should be further investigated, while the less relevant ones can be treated in a more ``economic'' way, which allows for a more effective approach to terminology work. According to Will, conference interpreters' terminology work can thus be equated to a detective's investigative work: rather than following an onomasiological approach, which would involve a top-down approach, i.e. starting from acquiring the knowledge and then identifying the terminology used to express the concepts, conference interpreters tend to follow a semasiological approach. They explore the corpus of preparation material looking for the relevant terminology, the ``explicit units'' that refer to a knowledge system, using terms as evidence to identify the units of knowledge they convey. Semantic links between terms reflect logical connections between the individual knowledge items. This process allows interpreters to acquire the necessary terminological knowledge and at the same time to link such knowledge to conceptual reference systems.

This ``detective work'' presupposes the availability of the ``evidence'', of sufficient preparation material to ensure effective terminology extraction and knowledge constitution. In order to face these limitations, \citet{fantinuoli_specialized_2006} proposes the adoption of a corpus-driven preparation methodology. After creating corpora of specialised texts with corpora creation tools, conference interpreters can explore the topic with a concordancer starting from an automatically-extracted terminology list. This also facilitates the identification of ``phraseological knowledge'' \citep[29]{fantinuoli_computer-assisted_2017}, primarily in the form of collocations, which in addition to the individual terms also characterise specialised terminology and the client's jargon. Thus, following a Corpus-Driven Interpreter Preparation (CDIP), ``interpreters preparing for a conference can obtain a list of relevant terms and texts within minutes, even when targeted preparatory materials have not been made available by the conference organisers (as is often the case in professional settings)'' \citep[188]{fantinuoli_specialized_2006}.

\section{Terminology tools for conference interpreters} \label{termtoolsint}
The need for supporting terminology tools capable of optimising and automating the interpreting workflow clearly emerges against the background of what was discussed in \sectref{terminology_interpreting}. Will's model of the terminology workflow in interpreting highlights that dedicated tools should address several requirements specific to interpretation: they should lighten interpreters' preparation load by promoting targeted preparation and easy access to glossaries; be as little obtrusive and complex as possible, especially during the in-process phase, in order not to interfere with the concurrent cognitive processes involved in interpreting; easily adapt to the individual preparation styles of conference interpreters; facilitate the goal of moving cognition upstream of the interpreting phase \citep{stoll_jenseits_2009}.

\subsection{Requirements for a support tool for conference interpreters} \label{requirements_CAI}
To define the ideal architecture and features of a bespoke tool for interpreters, \citet{rutten_terminologieprogramme_2000, rutten_why_2004, rutten_informations-_2007} laid down a model of the ideal interpreter-specific tool which mirrors the workflow defined by \citet{will_bemerkungen_2000,will_terminology_2007}.

Rütten envisaged a five-tier structure organised around a ``central starting page'' (\citealt[173]{rutten_why_2004}, see Figure \ref{fig:ruetten}) from which the different modules can be accessed. Each module serves a specific purpose and represents one phase of an interpreter's workflow. According to Rütten, interpreters would benefit from:

\begin{itemize}
    \sloppy
    \item a modality allowing for online and offline research of documents and information;
    \item a module for document management, connected to
    \item a function for terminology extraction, also from parallel texts when available, for the compiling of multilingual dictionaries;
    \item a terminology management module synchronised to the previous one which would allow for the identification of duplicates or similar entries;
    \item a ``trainer'' to systematise vocabulary memorisation.
\end{itemize}

\begin{figure}
%\includegraphics[width=0.8\linewidth]{images/Rutten_model.png}
\begin{tikzpicture}[
    >=Triangle,
    node distance = 10 pt and 20 pt,
    rnode/.style={rectangle, draw, text width=5cm, minimum height = 20pt},
    ]
    \node[rnode](one){1 – Online\,+\,Offline research};
    \node[rnode](two)[below=of one]{2 – Document management};
    \node[rnode](three)[below=of two]{3 – Terminology extraction};
    \node[rnode](four)[below=of three]{4 – Terminology management};
    \node[rnode](five)[below=of four]{5 – Trainer};
    \node(csp)[left=3cm of three,draw]{Central starting page};
    
    \draw[->](csp) to[out=0,in=180](one.180);
    \draw[->](csp) to[out=0,in=180](two.180);
    \draw[->](csp) to[out=0,in=180](three.180);
    \draw[->](csp) to[out=0,in=180](four.180);
    \draw[->](csp) to[out=0,in=180](five.180);
    
    \draw[->](one) to [out=360, in=0, looseness=2] (two.5);
    \draw[->](two) to [out=360, in=0, looseness=2] (three.5);
    \draw[->](three) to [out=360, in=0, looseness=2] (four.5);
    \draw[->](four) to [out=360, in=0, looseness=2] (five.5);
    \draw[->](five.175) to [out=180, in=180, looseness=2] (four.185);
    \draw[->](four.175) to [out=180, in=180, looseness=2] (three.185);
    \draw[->](three.175) to [out=180, in=180, looseness=2] (two.185);
\end{tikzpicture}
\caption[Rütten's software model]{Rütten's software model \protect\citep[173]{rutten_why_2004}}
\label{fig:ruetten}
\end{figure}

In addition to these interdependent modules, the author argues in favour of an ``overall quick-search key that can be used blindly and independently of the module or function'' \citep[175]{rutten_why_2004} for querying the whole database and all the modules. For students, a dedicated training function with exercises to fine-tune the sub-skills involved in interpreting could also prove beneficial. Rütten's suggestions map the circularity of interpreters' terminology work, from preparation material to glossaries before and during interpreting, and back to glossaries after the assignment. The document management and terminology extraction modules facilitate the collection of textual material to constitute the necessary reference knowledge systems. The vocabulary trainer addresses another key necessity for interpreters: the acquisition and activation of relevant specialised vocabulary ahead of the interpreting task, as terminological resources, unlike for translators, can also be consulted to some extent during interpreting.

\subsection{Usefulness and limitations of non-interpreter-specific tools} \label{proconstools}
\begin{sloppypar}
Against the background of Will's and Rütten's models, the question arises whether and to what degree the tools already available to translators may already satisfy the specific requirements of interpreters' terminology work.
\end{sloppypar}

As remarked by \citet[43]{rutten2011TMS}, CAT tools and other technologies such as TETs, TMSs, concordancers, etc., already cover several subphases of an interpreter's workflow. These tools have the potential to provide useful support to interpreters, although they have not been explicitly targeted to these professionals.
Useful functions offered by translation memory tools and terminology management tools are for instance sorting and filtering functions or differentiated search functions (e.g. fuzzy search, see \citealt[43]{rutten2011TMS}). It should be noted that complex searches or filtering are mainly feasible in the pre-, peri-, or post-process phase (\citealt[50]{rutten2013TM}, \citealt[8]{wagener2012vorbereitende}). Translation memories can represent useful tools especially for those interpreters also working as translators, who may reuse their terminological databases and save precious preparation time \citep[49]{rutten2013TM}. TMs can also be useful to show aligned parallel texts on screen. This function may be useful also during interpreting, for instance if texts are read out loud by speakers and the translation is already available. Translation memory systems which save all terminological entries in a single database can promote managing the data as a whole \citep[44]{rutten2012TM}. Additionally, some TM tools and TMSs usually allow embedding a picture to illustrate a term, which can be useful to quickly grasp the term's meaning (ibid.) and support memorisation.

Tools for terminology extraction, either stand-alone or integrated into CAT tools, can save interpreters time by automatically identifying candidate terms and promote knowledge acquisition by allowing interpreters to focus on the meaning of the text. These tools can also represent useful emergency solutions when long documents are made available last-minute during the assignment \citep[48]{rutten2012TM}.

Finally, tools for terminology search on the web or in online databases and dictionaries can prove useful to identify or check the validity of a target-language equivalent during glossary compilation.
The application of TMs, TMSs, CAT tools, etc., to interpreting presents, however, some limitations, which emerge especially in relation to their in-process use. To start, while non-bespoke tools do address some operations and subprocesses, they do not cover the entire interpreting workflow satisfactorily \citep[49]{rutten2013TM}.

Non-interpreter-specific tools are often criticised by interpreters because they are perceived as too complex and not intuitive. TMSs, for instance, usually offer multiple data levels, which promote a systematisation of entries, but are often seen as too cumbersome or excessively complex (\citealt[50]{rutten2012TM}, \citealt[141]{stoll_jenseits_2009}).
The host of functions offered by non-bespoke tools does not necessarily have to be exploited completely. However, interpreters not familiar with these tools may perceive them as overwhelming and unnecessarily complex.
Additionally, an interface offering too many functions or unoptimised entry structures may be difficult to process, especially for in-process terminology look-up when attention is already allocated to several subprocesses. None of the available tools offers ``a mouse-free, blind and intuitive search function'' \citep[46]{rutten2012TM}.

The limitations of non-bespoke tools outlined above have led to the development of increasingly sophisticated tools dedicated to conference interpreters, known as CAI tools. The following chapter discusses them in detail, highlighting the postulated advantages they offer, as well as their unique features as compared to CAT tools and the other resources discussed thus far (see also \sectref{termworktrans}). The chapter further explores whether and to what degree interpreters prefer CAI tools to other solutions for their terminology work and reviews the research conducted thus far on the topic of CAI.
