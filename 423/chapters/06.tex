\chapter{“Languages'' index belonging}\label{chap:5}
\label{bkm:Ref113005745}\hypertarget{Toc125444663}{}\section{Languages as optional social indices}
\label{bkm:Ref121491384}\hypertarget{Toc125444664}{}
The elements co-occurring in particular com-sits form linguistic systems; they are integrated with each other through grammatical patterns. This is all that is needed for grammatical systems to work, and accordingly, this is all that we get in some settings, as witnessed, for instance, at the Maybachufermarkt. However, in other settings, the system that comes out of such co-occurrences takes on a language index, and this index can then become a salient feature of the individual linguistic elements themselves.

Different elements are now not just associated with each other through such co-occurrence in com-sits, but as members of the same “language". As such, they can retain this index if they are used in new com-sits. For instance, as mentioned in \ref{bkm:Ref115420422} above, for my daughter, \textit{tea} developed from one of the words she used in com-sits with Daddy to an “English" word, and as such it would be appropriate with anyone with whom she was expected to speak “English". And when \textit{bye bye} takes on the “English" index, this does not, in principle, prevent it from being used in the context of “German" (in the same settings as, e.g., \textit{Wauwau} and \textit{Tee}) or “Turkish" (together with, e.g., \textit{kuçu kuçu} and \textit{çay}), but that will then be a marked usage: an English word used in German or Turkish.

Typically, “language" indices will be based on general, more abstract com-sits. The same “language" index usually holds across systems that are associated with a range of more specific {{com-sits}}, for instance encompassing both informal and formal settings and/or different named “dialects". In general, we will find a stronger separation and fewer shared elements in (macro) contexts with a monolingual habitus and standard language ideology, that is, with stronger ideologies of linguistic purism.

The development of a “language" index is an optional one, but one that has powerful social implications. “Languages", including language borders and named languages, can play an important role for negotiating and marking social groups and affiliations. The link between group affiliation and language was also at the bottom of European nation-state building, with its nexus of “one country – one nation – one language", which is what got us settled with our present idea of bound languages in the first place.

This is by no means a natural or automatic development, but rather one that depends on a societal macro context that supports such constructions, often backed by purist language ideologies that reinforce linguistic borders.\footnote{Cf. also \citet{KrämerEtAl} on ‘language making’; \citet{HellerDuchêne2016} on language as a commodity in late capitalism.} “Languages" then emerge as social indices. This involves a second-order indexicality in the sense of \citet{Silverstein2003}, or a marker in \citegen{Labov1972} distinction of indicators vs. markers.\footnote{Cf. \citet{JohnstoneEtAl2006} for an integration of the two systems.} A first-order index or Labovian indicator emerges when elements are used together in the same com-sits. Once this co-occurrence takes on a meaning of itself, this can establish a second-order index.

In a further development, we can also find Labovian stereotypes or higher-order indices in the sense of Silverstein, which can emerge when such varieties are then associated with specific social groups as their speech communities. Linguistic elements carrying a certain language index can then signal associated stereotypes of group prestige or belonging. This can also involve \textit{iconisation} in the sense of \citet{IrvineGal2000}, where the language is perceived as having such stereotypical features as well, thus iconically reflecting the social group it is associated with. The association of a language or dialect with certain groups can also be exploited, for instance, in patterns of “crossing" in the sense of \citet{Rampton1995}.

 {In 1750}, when Voltaire was in Prussia, he wrote in a letter to the Marquis de Thibouville:

\begin{quote}
Je me trouve ici en France. On ne parle que notre langue. L’allemand est pour les soldats et les chevaux. \citep{Voltaire1750}
\end{quote}

\begin{quote}
[‘I find myself in France here. One speaks only our language. German is for soldiers and horses.’]
\end{quote}

This quote, which is pretty well known in Germany, neatly brings together some of the aspects of languages as social indices. First of all, language use is separated along the borders of named languages, and these are associated with different nations or countries. So, when French is dominant somewhere, one feels oneself `in France'. Second, as somebody from France, one belongs to the social group that owns this language, it is `our language'. Finally, the com-sits where one encounters a language might reflect well or badly on it. So, restricting German to com-sits with `soldiers and horses' is a bit of a put-down, since soldiers are associated with stereotypes of being subordinate and rough and horses are non-human. Taken together, this makes French as a language, and by association the French as a nation, come away as superior and more cultivated than German (which is also why this quote is usually cited somewhat self-deprecatingly in Germany).

19\textsuperscript{th} century European nation-state building also threw in standard language ideology as a further restriction to what the “language" associated with a nation should be about.\footnote{Cf. \citet{HüningEtAl2012}.} This can be understood as a case of ideological recursion of monoglossic ideologies:\footnote{\citet{IrvineGal2000}; \citet{Fuller2018}.} monolingual ideologies restrict language use to one “language" at the first level, and standard language ideology restricts this language to a specific variety at the next level.

\largerpage[-1]
Note, though, that this further restriction is not a necessary ingredient for languages as social indices. The use of a language as a means to build unity through external borders can also be observed in the absence of standard language ideology and prescriptivism. A striking example is the free-range language setting of Yiddish in Israel. As Assouline (\citeyear[Ch.1.4.1]{Assouline2017}; \citeyear{Assouline2021}) describes, the ultra-orthodox (Haredi) community in Israel who speaks Yiddish as the main family language does not enforce any linguistic standardisation, but rather accepts a wide range of variability: “Speakers do not usually attribute any socio-cultural value to ‘correct’ language” \citep[18]{Assouline2017}. Yet, this does not prevent the use of language as a marker of belonging and solidarity. Yiddish textbooks for girls do not prescribe a specific dialect, leaving it, e.g., open which grammatical gender to use for certain nouns, and girls are told that it does not matter how they speak, as long as they do it the way their family does, using Yiddish in contrast to the larger Israel Jewish society, which is dominantly Hebrew-speaking (Assouline, p.c.). It is the use of Yiddish, that is, of elements and patterns indexed as “Yiddish", that signals belonging, not necessarily the observance of a specific kind of standardisation.

In our research on the Maybachufermarkt we found that the use of different “languages" served as a linguistic marker of alignment and solidarity with different ethnic groups. In her ethnographic study, İrem Duman Çakır observed that ethnicity-language ties played a key role in the language that sellers chose to approach passers-by. Based on people’s outer appearance and their perceived accent, they made hypotheses about group membership and ethnic belonging and then aimed to use a customer’s “own language" in order to engage them in a sales interaction.\footnote{Cf. \citet{Duman2019}, \citet{SchulteDuman2019}.} In such practices, languages are used to index different ethnicities.

This can also trickle down to individual expressions. As \citet{Duman2019} shows, an important distinction at the market is that between speakers constructed as Turkish or Arabic Muslims, and others. Sellers who perceive themselves as belonging to the first group use different forms of address towards female customers from this in-group than to outsiders. Female passers-by perceived as in-group members will be addressed as \textit{abla}, which is associated with Turkish (lit. ‘older-sister’). Using \textit{abla} signals respect (‘older’) and mutual belonging (‘sister’), and it is the second aspect that is reinforced through its sociolinguistic association with Turkish. Towards passers-by perceived as outsiders, a seller will use \textit{Madame}. This is a term of address that is used in Turkey as well and also signals respect. Unlike \textit{abla}, though, it is a loanword that is not closely linked to Turkish, but rather has a more neutral association. Duman Çakır describes that it became clear how important these markers are when she once inadvertently transgressed this rule during her ethnographic research. While working as a seller at one of the market stalls, she addressed a passer-by as \textit{Madame}, and got the indignant response “Ne Madamı? Türküm ben!” ‘Why “Madame”? I am a Turk!’ \citep{Duman2019}.

Such observations point to the use of languages as social indices even in free-range settings that offer speakers a large degree of freedom to transcend culturally defined language borders, a kind of language use that has been captured by the concept of “translanguaging”.\footnote{See, for instance, \citet{García2009}, \citet{GarcíaTupas2018}.} According to  \citet{LiWei2021}, humans might have a “translanguaging instinct”, that is, an instinct to use language as a fluid repertoire that does not obey the boundaries of named nation-state languages. However, in the “languagised” societal macro contexts that are dominant today, languages as bound entities can be real in speakers’ minds and as part of speakers’ competences.\footnote{Cf. \citet{JaspersMadsen2019}.} Our socialisation into languagised societies means that language indices become a familiar means for us to signal group membership, belonging and social identities.\footnote{Cf. \citet{Møller2019}.} Language borders can then take on a life of their own, segregating linguistic resources. As a consequence, in certain settings, crossing such borders might not feel necessarily more natural, and such practices as translanguaging might have to be (re\nobreakdash-)learned.\footnote{\citet{Ruuska2019}.}

The naming of languages and dialects can lead to further establishing them. In the case of urban contact dialects, this has sometimes been criticised as deliminating and essentialising fluid practices,\footnote{Cf. \citet{Jaspers2008}, \citet{Androutsopoulos2011}, \citet{CornipsEtAl2015}.} and this is certainly a danger that any labelling brings with it (including such emic labels as, e.g., \textit{Kiez\-deutsch} or \textit{Sheng}, and their subsequent use by linguists). However, from a perspective of languages as indices, it can also contribute to empowering speakers:\footnote{See also \citet{Wiese2015}.} a label can help establish a new linguistic index that gives prestige to a dialect as a legitimate variety, counteracting perceptions of “broken language". I am following my colleague Philip Krämer from Free University Brussels on Twitter, and among his many insightful postings I noticed one where he draws a parallel between languages and food that highlights this point:\footnote{Cf. also \citet{Krämer2017} on patterns of delegitimising multiethnolects and creoles in public debates.}


\begin{figure}
\fbox{\includegraphics[width=.8\textwidth]{figures/kraemertweet.png}}
\caption{\label{fig:15} Labelling effects on language and food (\url{https://twitter.com/ph_kraemer/status/1430209710393856001?} Aug 24th, 2021)}
\end{figure}

\section{Language mixing can index multilingual identities}
\hypertarget{Toc125444665}{}
While elements associated with specific bound languages can signal belonging to a certain group, mixing such elements can in turn signal belonging to a multilingual and/or multiethnic group. This is something we observed at the Maybachufermarkt, too. As I mentioned earlier, the market has gained a reputation for being linguistically and ethnically diverse. This is a source of pride among sellers, and it is also good for business, since it attracts customers from all over Berlin as well as tourists who come to enjoy the vibrant market atmosphere. In this setting, language mixing can be used for a higher-order indexicality to the highly diverse speech community that is associated with the market.

In their market cries, sellers exploit this with a practice that Yüksel \& Duman Çakır (\textit{to appear}) analysed as “commercial code-switching”. In this practice, sellers express the same meaning with elements from different named languages, for instance going “taze Brot, frische Brot”. Semantically, this just means `fresh bread, fresh bread', which would be kind of redundant. However, from the point of view of language indices, it is more informative: while \textit{Brot} (German for ‘bread’) is repeated verbatim, ‘fresh’ is expressed by two different words, \textit{taze} and \textit{frisch}, which are associated with Turkish and German, respectively. This can serve a double purpose. Individually, each of these elements indexes a different language and by so doing, signals a different ethnic group; by using both, the seller can thus cover a larger group of potential customers passing by. Over and above this, though, juxtaposing these elements in one market cry indexes the multilingual and multiethnic market community, drawing on local pride and supporting an atmosphere that is good for commerce.


%%[Warning: Draw object ignored]
Such a pattern is also visible in some shop signs in the multilingual area surrounding the market. Here is an example from a pharmacy:


\begin{figure}
\fbox{\includegraphics[width=.6\textwidth]{figures/pharmacy.jpg}}
\caption{`Pharmacy' signs in Turkish and German in Berlin-Kreuzberg}
\end{figure}

Again, we get two words with the same meaning: \textit{Eczane} and \textit{Apotheke} both mean `pharmacy'. Since in addition, the symbol for a pharmacy is clearly visible on the door (the red “A” with the aesculapian snake), we probably would not need a translation for people to understand what the store is about. The main point is the social indexing: \textit{Eczane} and \textit{Apotheke} are indexed for Turkish and German, respectively, and having them side-by-side signals a validation of multilingual repertoires and the different ethnolinguistic groups associated with Turkish and German. Accordingly, this is a common occurrence in this area, with pharmacies, shops, and offices putting a range of multilingual signs in their windows to attract customers.

An interesting case of this kind of indexing outside commercial settings can be found in Namibian German. In an investigation of a Facebook group of German-Namibians in Germany, \citet[468]{Radke2021} quotes the following post:

\ea\label{ex:key:1}
{  Miskien.Perhaps,vielleicht, may bee… its raining?}\\
\z
In this case, the same meaning, ‘perhaps’, is expressed several times, through elements indexed for three different languages, Afrikaans (\textit{miskien}), English (\textit{perhaps}, \textit{may bee}), and German (\textit{vielleicht}). We can now analyse this practice as a group marker that works similar to the commercial code-switching at the Maybachufermarkt. Just like at the market, from a purely semantic point of view such “repetition" would be pointless, but it makes sense from the perspective of languages as indices. Indexing, in the same utterance, three languages – and in particular these three languages – points to the linguistic repertoire characteristic for German Namibians, and thus can serve as a higher-order index to this group. That the pattern is semantically redundant is then not pointless, but instead reinforces the sociolinguistic point.


In Namibia, heritage speakers of German form a close-knit group that finds itself sociolinguistically somewhat drawn in two opposite directions:\footnote{Cf. \citet{WieseBracke2021}, \citet{Leugner2022}.} the ethnic German identification supports linguistic purism and the exclusive use of German in demarcation to other ethnic groups in Namibia, while the local Namibian identification supports language mixing and borrowing. The latter can also be in demarcation to the “Jerries", that is, us folks from Germany. Hence, when German Namibians find themselves in diaspora in Germany with its strong monolingual habitus, the second pattern becomes more relevant, and such language mixing can then become a strong in-group marker.

Translinguistic integration can also signal belonging to a new group that transcends ethnic boundaries. An example are linguistic practices among multiethnic adolescent peer groups. In such settings, drawing on linguistic resources associated with different heritage languages can signal belonging to a new, multiethnic group.\footnote{\citealt{Wiese2022}; cf. also \textrm{\citet{Jungbluth2016} on bilingual acts of identity expressing biculturality.}} Accordingly, a general feature of the urban contact dialects that emerge in such settings is the integration and mixing of elements from different languages.

If we compare such dialects in different societal macro contexts, we find an interesting difference related to the strength of monolingual hegemonies there.\footnote{\citet{Wiese2022}, \citet{KerswillWiese2022}.} In societies with a general multilingual orientation, for instance Kenya, Cameroon, or 19\textsuperscript{th} century Finland, the integration of elements across language borders can be more elaborate, so that the urban contact dialect can constitute a new mixed language. In societies more dominated by a monolingual habitus, such as Germany, the UK, or Tanzania, such developments are held back, and urban contact dialects typically constitute vernaculars of the majority language. This indicates that the com-sit impact is mediated by the societal macro context – or, if you remember our “pond" metaphor from Chapter \ref{bkm:Ref114133435}: the local weather is influenced by the region’s overall climate. \REF{ex:4} gives two examples, from Camfranglais and Kiezdeutsch, to illustrate this:

\ea\label{ex:4}
{\label{bkm:Ref82166863}  Language mixing in urban contact dialects}\\
\ea\label{bkm:Ref82166863a}
 Camfranglais (\citealt{KießlingMous2004})\\
 \begin{quote}
\gll \fra{On} \fra{a}     \eng{kick}    \fra{mon} \hau{agogo}.\\
     one has stolen my   watch\\
\glt `My watch got stolen.'
\end{quote}
\ex\label{ex:4b}
Kiezdeutsch (\citealt{WiesePolat2016})\\
 \begin{quote}
\gll \deu{Du}  \deu{bringst} \deu{Teller} \deu{Meller} … \tur{Vallah}.\\
     you bring    plates  \\
\glt `You bring plates and stuff.'
\end{quote}
\z
\z

Again, I have used colour-coding to mark elements associated with different languages: \eng{English} is purple, \deu{German} green, and \tur{Turkish} red again; in addition, \fra{French} is marked in blue and \hau{Hausa} in orange. If you look at the two examples, you will notice that \REF{ex:4b} is much less mixed. In the Camfranglais example, a French grammatical frame is integrated with lexical elements from English and Hausa. In the Kiezdeutsch example, we find mostly German, with only one borrowed element, (Arabic-)Turkish \textit{vallah}, which not accidentally is a discourse marker and stands in peripheral position.\footnote{See, e.g., \citet{Fuller2001}, \citet{Matras2009} on the easier borrowability of discourse markers.}

The monolingual habitus fosters a majority language that is so dominant that it holds a tight rein on linguistic developments here, and this is typical for Europe. Urban contact dialects in such societal settings constitute variations on their majority language that are a far cry from the integrated mixing that we find in most African countries. \citet{Mair2022} reported from sociolinguistic interviews with Nigerian immigrants to Freiburg in South Germany that they missed a proper “street language” in Germany, and this explicitly did not refer to Kiezdeutsch, but to a kind of informal language use that involved a more liberal mixing and integration of linguistic resources.

However, these are gradual rather than categorical differences, and in both cases the power of the multilingual com-sit is such that it can also support new grammatical patterns. These are much less frequent in cases like Kiezdeutsch, but we do find them, and we can even observe some transfer of syntactic patterns from heritage languages, although that is exceedingly rare.

In order to illustrate such mixing even under conditions of a societal monolingual habitus, I chose an example in \REF{ex:4b} that does include such a rare kind of syntactic transfer. You will have noticed that I omitted \textit{Meller} in my interlinear translation, but had an additional “and stuff” in the idiomatic paraphrase. This is because \textit{Meller} is actually not a word, but derived from \textit{Teller} through \textit{m}{}-reduplication. This is not a productive pattern in German outside Kiezdeutsch, but it is common in Turkish, which presumably is the source of this. We can account for the Kiezdeutsch pattern as shown in \figref{fig:wiese:6-1}.\footnote{Cf. \citet{WiesePolat2016}.}

\begin{figure}
\begin{tikzpicture}
  	\matrix[matrix of math nodes,right delimiter=\},cells={anchor=base west,inner ysep=0pt,font=\strut},inner xsep=0pt,outer xsep=4pt]
  	(matrix) 
 	{	
 		\text{PHON:} & \textnormal{O} ~~~ \textnormal{O}\textsuperscript{[}\textsc{\textsuperscript{onset} := /\textnormal{m}/]}  \\
 		\text{SYN:} & \textnormal{N} ~~~ \textnormal{N}    \\
 		\text{SEM:} & \textit{e}  \\
 		\text{PRAG:} & \textnormal{p} ~~ . ~~ \textnormal{p} \in \{\textit{e}^{+}, < \text{speaker I}^{\textnormal{n}} \textit{e} >, \textnormal{C}(\text{speaker})\}  \\
 		\text{COM-SIT:~} & \textit{kiez\textsuperscript{D}} \\
    };
    \node [right=2mm of matrix,font=\strut] (mredupD) {[\textit{m}-reduplication]\textsuperscript{\textbf{D}}};
    \node [above right=3mm and 5mm of mredupD.north,font=\strut,anchor=south] (mredupT) {[\textit{m}-reduplication]\textsuperscript{\textbf{T}}};
    \draw (mredupD) -- (mredupT);
    \draw (matrix.south west) rectangle (mredupT.north east);
  \end{tikzpicture}
\caption{Entry for \textit{m}-reduplication in Kiezdeutsch\label{fig:wiese:6-1}}
\end{figure}

At the phonological level, we have a placeholder for a representation (\textbf{O}) that gets copied and modified by replacing its onset with /m/. In syntax, this is paralleled by two nouns. Only the first one has a semantic representation (\textbf{\textit{e}}), since the second is a nonsense word (for instance \textit{Meller} in the example above). Pragmatically, this can be associated with three possible patterns, represented as elements of a set (p ε \{ …\}). In this set, \textbf{\textit{e}}\textbf{\textsuperscript{+}} indicates elaboration: not just plates, but plates ‘and stuff’. \textbf{I\textsuperscript{n}} stands for the negative expressive interval mentioned in \ref{bkm:Ref114129291} above; it captures the fact that \textit{m}{}-reduplication can also express pejoration – something I called the ‘whatever’ effect, relating to the interaction of elaboration and pejoration here (think of your teenage daughter responding to your gentle criticism of the state of her room). Finally, \textbf{C} is a function we invented, based on a study with Kiezdeutsch speakers: it captures that by using this pattern, speakers can present themselves as “chilled” or “cool” (\citealt{WiesePolat2016}).

Finally, the com-sit is characterised as a multiethnic urban youth setting in Germany, which I referred to as \textit{kiez}. \textit{Kiez} is part of informal com-sits. Since we have a strongly monolingually oriented society in Germany, com-sits are indexed with the majority language, German, by default, and \textit{kiez} inherits this index: \textit{kiez} ${\in}$ \textit{informal\textsuperscript{D}}, hence \textit{kiez\textsuperscript{D}}. At the same time, \textit{kiez} is a com-sit that is characterised by multiethnicity and linguistic diversity involving a range of heritage languages. As a result, it supports the transfer of \textit{m}{}-reduplication by encouraging language mixing in general (to mark a multilingual group), and the integration of Turkish elements in particular (given that Turkish is a salient heritage language in this setting).

This entry captures \textit{m}{}-reduplication in Kiezdeutsch as a pattern that is associated with its Turkish source (represented by the line to \textit{m}{}-reduplication indexed for Turkish, “T”). Kiezdeutsch \textit{m}{}-reduplication has a lot of parallels with its Turkish source pattern, but at the same time, in this new \textit{kiez} com-sit, the pattern has also developed its own characteristics, reflecting its integration into a German youth setting. The first one is phonological: in Turkish \textit{m}{}-reduplication, only the first consonant is replaced by /m/, but in Kiezdeutsch, it is the whole onset, in line with German phonology. For instance, when the Turkish president Erdoğan once got annoyed by some critical Twitter threads, he threatened to ‘eradicate Twitter Mwitter’.\footnote{“Twitter Mwitter hepsinin kökünü kazıyacağız”; see \citet[247]{WiesePolat2016}.} In German, one would have said “Twitter Mitter”, replacing the whole onset /tw/ by /m/, not just the first consonant /t/.\footnote{Cf. also \citet{Stamer2014}.} The second difference comes in at the pragmatic level: the “chilled” component in Kiezdeutsch is an additional feature that is absent in Turkish \textit{m}{}-reduplication. This is linked to the urban youth aspect of the com-sit, rather than to German in general: the \textit{kiez} setting is the basis for higher-order indices to a speech community stereotyped as “cool".

An interesting example of social indexing through language mixing comes from the phonetic de-integration of loan words. At the beginning, in Chapter \ref{sec:1.1}, I discussed the integration of \textit{Computer} as a loan word into German to highlight the reality of grammatical systems. Using a metaphor from Star Trek, I compared linguistic systems to Borg-like collectives that integrate loan words along the lines of “You will get grammatically assimilated. We will add your linguistic distinctiveness to our own.”. However, resistance is not always futile, and we can actually have words de-integrating again. When I grew up, French loanwords like \textit{Orange}, \textit{Restaurant}, or \textit{Parfüm} in German were fully integrated phonologically and we pronounced them with German vowels ([{ʔ}oʁa{ŋ}ʒə], [ʁɛstoʁa{ŋ}], [paʁfyːm]). However, this has been changing, and I hear my daughters pronouncing these words with nasalised vowels that make them much closer to the French version and less integrated into the German phonological system ([{ʔ}oʁ\~ɑːʒə], [ʁɛstoʁ\~ɑ], [paʁf\~{œ}]). At first, this felt a bit stylised to me and even slightly pretentious, but with more and more exposure, I increasingly find myself doing the same.

Such de-integration seems odd from the point of view of grammar, given that the way to go for loanwords is to blend in rather than to stick out, so we should expect them to integrate more and more, rather than make a U turn. However, this makes sense if we understand languages as social indices: flagging the original “Frenchness" of loans can signal multilingual competences in times of globalisation, so that it now sounds unknowledgeable and slightly provincial to pronounce them too German. Using pronunciation for such signalling works well because phonetic deviations are salient and thus useful for social indexing, but they do not overly affect the grammatical system, and hence the linguistic integration is preserved. This, then, explains the exceptional behaviour that has been observed for the phonological/phonetic domain compared to morphosyntax when it comes to borrowing.\footnote{For instance, \citet{Poplack2018}.}

\section{Language separation can index formality}
\hypertarget{Toc125444666}{}
The fact that languages function as indices can also work the other way round and support language separation, rather than mixing. As mentioned, European nation-states build on ideological ties between nation and language. This also means that they are often dominated by a linguistic purism that delegitimises the combination of elements indexed for different languages, and standard language ideologies discourage mixing. As a result, com-sits with higher formality, which are associated with such “standard language", tend to restrict themselves to a single language index. In heritage language settings where language mixing is otherwise common, language separation can then still be preserved as a high-formality marker.

In an open-guise study on register perception, we asked German Namibians to listen to recordings in German that differed with respect to language separation:\footnote{\citet{WieseEtAl2021}} one involved lexical borrowings from English and Afrikaans, while the other did not. We asked participants whether the different recordings sounded more like talking to a friend, i.e., pointing to an informal setting, or to a teacher, i.e., a formal setting. Responses indicated that recordings with lexical borrowing were more strongly associated with the informal situation, while those without were associated with the formal one. This suggests that separating elements along language indices can mark com-sits with a higher formality, even in groups where mixing signals belonging.

\citet{Møller2019} describes something similar for heritage-Turkish in Denmark. In his paper, he quotes a young man who reports speaking only Turkish, without any Danish mixed in, to his girlfriend’s parents, his prospective in-laws, even though “they speak fine Danish”, explaining “it’s just the respect […]. I have to present myself from the best side” \citep[45]{Møller2019}. Again, separating elements along language indices can mark formality.

Looking at this from the opposite side, that means that the combined use of elements indexed for different languages can in turn serve as a marker of informal registers. In other words, code switching and borrowing can take on informal register associations. This puts elements from different languages on the same plane as informally marked elements from the same language, something which is very much in line with our approach to com-sits: as discussed in \ref{bkm:Ref114129291} above, when we understand com-sits as the basis of linguistic systems, we can integrate “language" differences and differences in formality under a unified perspective of registers as linguistic choices that are associated with different com-sits.

What counts as language mixing vs. “pure" Turkish, German, etc., depends on what kinds of cross-linguistic transfer are perceived as such: what do speakers notice as coming from another linguistic system? To test this, in our perception study on Namibian German, we included a third stimulus: a recording that involved grammatical, rather than lexical transfers. Responses to this stimulus took a middle place between talking to a friend and to a teacher. We interpret this as evidence for a lesser social salience of grammatical compared to lexical variables. This is in line with approaches assuming that elements that involve more overt material are more consciously accessible and thus easier to borrow.\footnote{E.g., \citet{ThomasonKaufman1988,Labov2001,Matras2020,LevonFox2014}, cf. also \citet{WieseBracke2021}, \citet{SauermannEtal} for Namibian German.}

\largerpage
An example that shows how this can pan out in the formation of new varieties comes from our corpus of Namibian German.\footnote{DNam corpus; \citet{ZimmerEtAl2020}; see \url{https://hu.berlin/DNam}.} As part of this corpus, we have recordings where speakers were asked to play-act talking about the same accident in an informal setting with a friend and in a formal setting with a teacher.\footnote{This was done using the ‘LangSit’ method of eliciting naturalistic, register-differentiated language use (\url{https://hu.berlin/LangSit}; \citealt{Wiese2020b}), which also provided the basis for the RUEG corpus.} In their descriptions, they often mention that the person who had the accident did not seem to have been injured. In German, you can express this using the phrase “weh tun”, lit. ‘do/cause painful’. This is kind of an awkward construction because it is atelic. The person who gets injured, the \textsc{recipient}, is expressed by a dative phrase, while the \textsc{source} or \textsc{agent} gets to be the subject. If you just want to express that you injured yourself, without including the source, you have to put it like ‘I did me painful’.

This makes the standard German pattern inconvenient to use if the important player is the injured party, and the source is not relevant. In contrast to this, Afrikaans offers a pattern much more suitable for this: \textit{seer kry}, lit. ‘get/receive painful’ expresses the \textsc{recipient} as a subject and does not need a dative complement. Not surprisingly, then, German Namibians, who have this pattern at their disposal as part of their multilingual resources, make good use of it. They integrate the pattern into their German by using a close German counterpart of the verb \textit{kry}, namely \textit{kriegen}, which basically means the same (‘to get/receive’) and is also phonologically somewhat similar (diachronically, they go back to the same root).

Now, what is used as the complement of \textit{kriegen} depends on the com-sit. In informal settings, speakers tend to use the original Afrikaans element, \textit{seer}, resulting in \textit{seer kriegen} as a Namibian-German pattern that combines the language indices of German and Afrikaans, thus highlighting the speech community’s multilingual character. In formal settings, though, this kind of language mixing should be avoided. Speakers solve this by replacing \textit{seer} with its German counterpart, \textit{weh}. This results in \textit{weh kriegen}, a Namibian-German pattern that, unlike \textit{weh tun}, has all the advantages of the Afrikaans model, but does not involve any overt lexical material from a language other than German. There is nothing in it that is socially indexed as “Afrikaans”, making it suitable for com-sits that favour language separation, rather than mixing.

\figref{fig:18} summarises this development. Elements indexed for Afrikaans (A) are rendered with dotted lines, those indexed for German (D) or, more specifically, for Namibian German (ND), with dashed lines. \textit{kry} and \textit{kriegen} are associated as crosslinguistic counterparts within the linguistic resources of Namibian Germans, as are \textit{seer} and \textit{weh}. The pattern of \textit{seer kry} supports \textit{seer} \textit{kriegen} in informal com-sits, while \textit{weh kriegen} is specified for formal ones.


\begin{figure}
%%[Warning: \textsuperscript{D}raw object ignolsMidWine]
\fittable{
\begin{tikzpicture}
%   \node(seerkriegen)[rectangle,draw,]{
%   \nambox{seer\textsuperscript{A} kriegen\textsuperscript{D}}{receive(pain)(x)}{comm-sit $\in$ \textbf{informal}\textsuperscript{N\textsuperscript{D}}}
%   };
  \node(seerkriegen)[rectangle,draw,lsMidGreen]{\nambox{lsMidGreen}{seer\textsuperscript{A} kriegen\textsuperscript{D}}{receive(pain)(x)}{com-sit $\in$ \textbf{informal}\textsuperscript{ND}}};
  \node(seerkry)[rectangle,draw,lsMidWine,left=of seerkriegen]{\nambox{lsMidWine}{seer\textsuperscript{A} kry\textsuperscript{A}}{receive(pain)(x)}{com-sit $\in$ A}};
  \node(kry)[rectangle,draw,lsMidWine,above=2cm of seerkry]{\nambox{lsMidWine}{kry\textsuperscript{A}}{receive(y)(x)}{com-sit $\in$ A}};
  \node(wehkriegen)[rectangle,draw,lsMidGreen,right=of seerkriegen]{\nambox{lsMidGreen}{weh\textsuperscript{D} kriegen\textsuperscript{D}}{receive(pain)(x)}{com-sit $\in$ \textbf{formal}\textsuperscript{ND}}};
  \node(weh)[rectangle,draw,lsMidGreen,above=2mm of wehkriegen,xshift=6mm]{\nambox{lsMidGreen}{weh\textsuperscript{D}}{painful}{com-sit $\in$ D}};
  \node(kriegen)[rectangle,draw,lsMidGreen,right=4cm of kry]{\nambox{lsMidGreen}{kriegen\textsuperscript{D}}{receive(y)(x)}{com-sit $\in$ D}};
  \node(seer)[rectangle,lsMidWine,draw,left=6.1cm of weh]{\nambox{lsMidWine}{seer\textsuperscript{A}}{painful}{com-sit $\in$ A}};
%
  \draw[dashed,thick](kry)--(kriegen);
  \draw[dashed,thick](seer)--(weh);
  \draw[dashed,thick](seerkry)--(seerkriegen);
  \draw[dashed,thick](seerkriegen)--(wehkriegen);
%
  \draw[thick,lsMidWine](kry)--(seerkry);
  \draw[thick,lsMidWine](seer)--(seerkry);
  \draw[thick,lsMidWine](seer)--(seerkriegen);
  \draw[thick,lsMidGreen](kriegen)--(seerkriegen);
  \draw[thick,lsMidGreen](kriegen)--(wehkriegen);
  \draw[thick,lsMidGreen](weh)--(wehkriegen);
\end{tikzpicture}
}
\caption{\label{fig:18}Com-sit differentiation for language mixing vs. separation in Namibian German \textit{seer/weh kriegen}}
\end{figure}

What I particularly like about \textit{weh kriegen} – despite its less pleasant semantics – is not only the rich translinguistic network of resources it builds on, but also what it shows us about linguistic dynamics: Namibian German does not only boast characteristic features in informal com-sits, but has also developed some new patterns in formal ones, supporting a new variety of Namibian standard German.\footnote{This variety also includes some lexical borrowings, but since these are more salient, they are generally restricted to words referring to Namibian-specific phenomena, e.g., \textit{braai} for a certain kind of wood-based barbecue. See \citet{Kellermeier-Rehbein2016} on such loans indicating a Namibian-German standard variety.}

The fact that abstract grammatical patterns are less socially salient as loans than overt lexical elements makes them suited for such a variety (as long as they are not highlighted and sanctioned as “wrong grammar" in metalinguistic discussions, e.g., at school). Since all their components are indexed as “German", they are perfectly fine in com-sits concerned with linguistic purism. This, then, illustrates the differential impact that language indices can have on developments in different com-sits, even though they are secondary to com-sits as the foundation of grammatical systems.


