\chapter{Conclusions: Com-sits, grammatical systems, and languages}\label{chap:6}
\hypertarget{Toc125444667}{}
In this book, I developed an architecture that allowed for grammatical systems without requiring languages and language borders. In so doing, I reconciled two strands of linguistic research, namely, on the one hand, sociolinguistic approaches to linguistic fluidity, (super\nobreakdash-)diversity, and multi-competence and, on the other hand, structural approaches to linguistic coherence and grammatical systems. Researchers from these strands are usually not in the habit of talking to each other a lot. This would not need to bother us too much (after all, not everyone needs to talk to everyone else all the time), if it wasn’t for the fact that the two strands provide us with two core insights that seem to be irreconcilable at first sight:\medskip

\vspace{.3cm}

\begin{tabular}{l p{1\linewidth}}
Sociolinguistic findings reveal  & Findings on linguistic structure\\
bound ‘languages’ to be social  & indicate internal organisation\\
constructs that cannot capture & and coherence and the\\
the diversity and fluidity of & workings and interactions of\\
actual language use. & distinct grammatical systems.\\
\end{tabular}

\vspace{.4cm}


I showed that we can actually have it both ways: we can acknowledge that languages are social constructs that can impose boundaries that do not reflect speakers’ linguistic realities, and at the same time we can recognise the workings of different grammatical systems that are also evident in the way speakers use language. As a basis for this, I introduced a specific concept of communicative situations, or short “com-sits”. I understood com-sits as the setting of communication, understood as a social activity that is typically centred around language production and perception, through which meaning is (co\nobreakdash-)constructed. As such, they provided the social and functional characteristics associated with different choices of linguistic elements. I modelled such com-sits as part of the information represented in lexical entries (in a broad sense of lexical entries that also encompasses more abstract patterns).

I demonstrated that, with this in hand, we can accommodate both sociolinguistic and structural findings, showing that languages are not real, but grammatical systems are, and they are anchored in com-sits. Once a grammatical system is in place, it can optionally be socially indexed as a named language or dialect. This primacy of com-sits means that we do not need languages for grammatical structure. It put the traditional picture on the head, making languages not the basis for linguistic differentiation, but a peripheral, optional addition.

Note that this also means that we do not need to assume multiple grammars for different varieties, but can instead do with an inventory of patterns. The question of one grammar vs. multiple grammars does not occur here because we do not keep within language borders with the grammatical patterns we posit: grammatical patterns are captured through lexical entries that are associated with different com-sits, and they can combine into larger grammatical systems by virtue of shared com-sits.

I fleshed out this architecture by looking at examples of “free-range" language, a metaphor I introduced for language in settings that are less confined by normative ideologies of monolingualism and linguistic “purity". Such settings might still feel some effects of monoglossic ideologies dominant in their societal macro context, but they will be less impacted by them. I discussed findings from four kinds of settings that have also been targeted in approaches to linguistic (super\nobreakdash-)diversity and fluidity: urban markets, heritage language settings, multiethnic adolescent peer groups, and digital social media. I argued that such settings allow us to shed further light on the role of com-sits and languages. Specifically, I showed that there are three lessons we can learn from them:

(I) Com-sits support linguistic differentiation. In speakers’ repertoires, linguistic elements can be overall connected as part of (generic) language. However, they are also organised into different domains according to their use in different com-sits. Speakers choose different linguistic resources from their repertoire according to the com-sit they are in. This is something children learn early on in first language acquisition, and it can also be observed in the emergence of new varieties, for instance in the case of urban contact dialects.

(II) Grammar is grounded in com-sits. The co-occurrence of elements in a com-sit supports selective strengthening of the connections between them. This way, com-sits provide the grounds for linguistic systematicity; they support the organisation of linguistic resources from an unstructured “feature pool" into systematic “feature ponds", that is, linguistic ecologies that support characteristic grammatical patterns. This implies that there can be grammatical patterns that are associated with specific com-sits, but do not involve “language" boundaries. I discussed evidence for this from both spoken and written language, including language-agnostic constructions at an urban market, and emoji as translinguistic discourse markers in digital social media.

(III) “Languages" index belonging. Linguistic systems can support the emergence of languages as sociolinguistic indices. This is an optional development that draws on a specific societal macro context, namely one that constructs elements as belonging to a certain “language" and reinforces “language" borders. Such macro contexts are typically found as a legacy of European nation-state ideologies. Languages are hence social constructs imposed upon linguistic systems, but that does not make them less real from a sociolinguistic point of view. Quite the opposite: as social indices, they play an important role for negotiating patterns of belonging. The choice of elements with different “language" indices can call up different ethnic identities, for instance to exploit associations with different groups of customers at a market. In addition to such different “language" choices, I showed that mixing or separating elements along “language" indices can also fulfil differential functions in multilingual settings. Transcending the borders of socially constructed languages can be used to indicate multilingual group identities in informal settings, while keeping within such borders can index formality. Once “language" borders are established, this can then also support dynamics in the other direction, that is, elements that are identified as belonging to a certain language can, as a result, be used in a certain com-sit. However, this is a secondary development that builds on the emergence of systems that are initially based on com-sits.

So, languages and their boundaries turn out to be anything but superfluous; rather, they have a range of important uses as social indices. However, that does not mean that they are needed for grammar. For grammatical systems, we can happily do without named languages and language borders, since all we need is com-sits. In a way, this leads us to an unexpected outcome: it is the grammar folks who should do away with “languages" in their work, while sociolinguists should find them most relevant.

I hope I have not managed to alienate both sides with this conclusion now, but rather that this might stimulate a further integration of grammatical and sociolinguistic approaches. I believe that this is important and fruitful for our understanding of language, not least of all for settings outside monolingual and purist hegemonies. These free-range settings are crucial for a complete picture of what language is actually like. So far, this picture has been infelicitously skewed towards an historic and geographic peculiarity introduced by European nation-state building: the imagined linguistically homogeneous setting. In contact-lin\-guis\-tics, there has lately been an increased interest in small-scale multilingualism in traditional societies that have not been impacted by European colonialism. I hope to have shown that we can find interesting examples of free-range language even within industrialised societies, as counter-sites to a monoglossic macro context

The dynamics of free-range language settings with their high diversity at levels of both linguistic structure and social meaning makes them a particularly interesting domain for linguistic analysis and not least of all for integrative theory building. For future work, I believe it will be promising to see what further insights such free-range language settings can afford us into linguistic architecture, grammar, and the social and linguistic reality of com-sits.
