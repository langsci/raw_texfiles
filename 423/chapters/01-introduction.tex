\chapter{By way of introduction}
\label{chap:intro}

I have always understood myself as coming from a monolingual background, and I used to believe I grew up in a linguistically more or less homogeneous environment. – Northeim, the place where I spent most of my childhood and adolescence, is a small town in a rural area of Lower Saxony, South of Hanover, with not much industry and accordingly very little labour immigration. However, I later came to realise that in fact I grew up in the midst of linguistic diversity. For one, there was intense language contact between High German and Low German. At the time I grew up in Northeim, standard German, which is a High German variety, already dominated most of public life and was also the language of school. However, Low German, the earlier regional language, was still spoken regularly by the older generation (although it was rapidly receding, and my locally born friends had mostly only a passive knowledge of it from listening to their grandparents).

In addition to this contact-linguistic dynamics between High and Low German, this was a time of intense dialect contact and dialect levelling in the area through an intake of German dialects from the East. I was born in 1966, just a bit over 20 years after the end of WW2, and Northeim was close to the inner-German border. As a result, a substantial proportion of the population were refugees from such areas as East Prussia, Pomerania and Silesia. This even included an entire village whose inhabitants had come from Silesia together. In addition, there were refugees from the GDR who had just made it across the border before the ‘iron curtain’ came up, including both my mother and my father, who had each come to West Germany in the 1950s as teenagers with their parents, from Thuringia and East Berlin, respectively.

This made for a dynamic linguistic mixture, grounded in a population with diverse linguistic backgrounds and repertoires. However, this was not something that was openly acknowledged, and we all behaved as if we spoke just one and the same variety of one language, German, reflecting the strong monoglossic ideologies dominating the society.

Today I live in Berlin, in a neighbourhood that is known for its linguistic diversity, Kreuzberg, and I have a multilingual family with a British husband and with two daughters who grew up with English and German, plus a smattering of Turkish from their babysitter and their friends (I will use a few examples of this in this book). While I am well aware of the rich linguistic diversity of my current life in Berlin, it is only recently that I realised that my environment in small town Lower Saxony was linguistically diverse as well.

This change of perspective in my own biography somehow parallels that in our field, in particular when it comes to structural linguistics and grammatical analysis, which is where I come from – an area where we have typically been targeting homogeneous speech communities and monolingual speakers. The roots of this can already be found in such earlier structuralist idealisations as Saussure’s focus on one-to-one correlations of language and place as the “forme idéale” (\citealt[Part 4, Ch. 2, §1]{Saussure1916}) or the Chomskian “ideal speaker-listener” \citep{Chomsky1965}. Today, this is changing, with more and more empirical approaches in structural linguistics that also take into account linguistic diversity and variability.

However, language in multilingual contexts is often still confined to  specialised domains of contact linguistics. Furthermore, there is hardly any meaningful interaction of structural approaches with current sociolinguistic models of language that take linguistic diversity and multilingual settings as their point of departure.\footnote{Earlier examples for the benefits of integrating sociolinguistic and structural perspectives come from approaches to syntactic variation as sociolinguistic markers that paid attention to vernaculars and microvariation (see, e.g., contributions in Cornips \& Corrigan eds. \citeyear{CornipsCorrigan2005}).} Findings from this area have little impact on structural models, and vice versa.

This separation of the two research lines has led to results that look irreconcilable on first sight. Core insights from grammatical and sociolinguistic analysis support two perspectives on language that seem to be fundamentally opposed to each other: structural findings point to linguistic coherence and grammatical systems, while sociolinguistic findings indicate linguistic fluidity and reveal languages and their boundaries as ideological constructions. How can they both be right? We seem to be faced here with some kind of ‘quantum linguistics’ paradox (to borrow some STEM prestige from another discipline) that calls for a closer look, and, if possible, a resolution within an account that can capture both insights. In this book, I hope to convince you that this can and should be done, and that both sides can benefit from closing this gap.

This gap is something that has been bothering me for a while, since I kind of have a foot in both fields, sociolinguistics and grammatical analysis. My initial background is in formal linguistic architecture and the syntax-semantics interface. Living in Berlin-Kreuzberg, I then got interested in a way of speaking I heard young people use in the street that seemed to display some interesting grammatical patterns. I published some articles on this and summarised central findings in a book \citep{Wiese2012_kiezdeutsch}. To account for the grammatical characteristics I had observed, I described this way of speaking as a new German dialect that I called “Kiezdeutsch” ‘((neighbour-)hood German’), using a term that some of the adolescents I worked with had suggested in an interview. The book was written in German and aimed at a broader, non-specialist audience. As such, it quickly gained a lot of interest and was taken up in national (and later also international) media, sparkling a public debate that lasted over months. In the context of this debate, I experienced something many researchers whose work went public know, namely not only positive feedback, but also personal attacks and – as is a common experience for women who are in the public eye – a barrage of verbal insults and even some violent threats (see Rampton ed. \citeyear{Rampton2014}).

As disconcerting as this was at first, it had some positive consequences for me, since it opened my eyes for sociolinguistics: on closer examination, the emails and online comments on Kiezdeutsch made for interesting data on language attitudes and ideologies,\footnote{The data is available through an open-access corpus, KiDKo/E, a subcorpus of the Kiezdeutsch Corpus, \href{http://www.kiezdeutschkorpus.de}{{www.kiezdeutschkorpus.de}}.} and something I analysed as “proxy racism”, a projection of racist marginalisation onto the linguistic plane \citep{Wiese2015}.\footnote{Cf also Dirim's (\citeyear{Dirim2010}) analysis of “(neo-)linguicism“ in the public debate on multingualism in Germany.}

The interesting grammatical and sociolinguistic patterns I found through my work on Kiezdeutsch led to my current research focus, which is on language in urban diversity, and it broadened my disciplinary outlook, with my work today targeting both grammatical and sociolinguistic domains. With the approach I develop in this book, I hope to further the integration of these fields, as mentioned above, by reconciling core insights on linguistic structure on the one hand and on linguistic fluidity on the other hand. 

In the following, first chapter (\chapref{chap:1}), I discuss the paradox we seem to be facing: named languages and their boundaries aren’t real (they are just ideological constructions), but at the same time they can be associated with grammatical systematicity and reflect actual linguistic structure. In order to solve this, I suggest a linguistic architecture that allows us to acknowledge grammatical systems without committing ourselves to language borders, based on a concept of communicative situations, short “com-sits” (I promise that this will be the only abbreviation you will need in this book!). The second chapter (\chapref{chap:2}) introduces the concept of “free-range language” for settings of linguistic diversity that are particularly challenging for assumptions of bound grammatical systems, and describes four central examples for such settings. The next three chapters (\chapref{chap:3}--\chapref{chap:5}) discuss three relevant lessons we can learn from such settings:
(I) communicative situations support linguistic differentiation;
(II) such differentiation provides the basis for grammar, hence grammatical systematicity is grounded in communicative situations, rather than determined by the boundaries of named “languages"; and
(III) languages come in as optional social indices that can signal belonging. The final chapter (\chapref{chap:6}) summarises the results and integrates the findings on communicative situations, grammatical systems, and languages.
