\chapter{Com-sits support linguistic differentiation}\label{chap:3}
\label{bkm:Ref121466446}\hypertarget{Toc125444657}{}\section{Com-sits and language development}
\label{bkm:Ref115420422}\hypertarget{Toc125444658}{}
I lately came across an Instagram post on “How does a new language start?”,\footnote{post by shaney.duffy; \url{https://www.instagram.com/reel/Cec2seDPqUD/?igshid=YmMyMTA2M2Y=} (last accessed: Sept 1\textsuperscript{st}, 2022). The original post is a video, for which I provide a transcript here.} which went like this:

\begin{quote}
You know what makes no sense to me? How does a new language just start? Do two guys just get together one day and just be like:
\end{quote}

\begin{quote}
Guy 1: ‘You know what dude? Fuck this!’
\end{quote}

\begin{quote}
Guy 2: ‘You’re right! Fuck this!’
\end{quote}

\begin{quote}
Guy 1: ‘Tre wa de hana bl rbl wawa guna?’
\end{quote}

\begin{quote}
Guy 2: ‘A bana hibs lr bla rbleya.’
\end{quote}

This is not meant seriously, of course. – We do not expect people to suddenly realise that they don’t like their present language and then to start speaking a completely new one out of nowhere. If we look at how new ways of speaking actually emerge, free-range language settings can give us a good idea. What we find here is that people start using some novel words and grammatical patterns that might eventually establish a new variety, but they do not do so across the board. Rather, these novel elements are used in certain communicative situations, they are associated with characteristic com-sits.

Take urban contact dialects, like Kiezdeutsch, as an example. In these cases, we have a new generation of speakers that grow up in a multiethnic and multilingual setting, and they express or perform this aspect of their identity when speaking in peer-group situations. The monolingual habitus dominating German society renders German the main source for Kiezdeutsch, but there are some characteristic ways in which it deviates from spoken standard-close German.

This is evident in the lexicon, for instance. First, we find markers of youth language, e.g., \textit{Alter} (lit. ‘old.one’), used as a term of address. Second and partly overlapping with this, there are elements from global English, such as \textit{lol} as a discourse marker (which we also saw in one of the WhatsApp messages I quoted in (\ref{bkm:Ref113003596}) above), \textit{cringe} as an evaluative term, or \textit{sus} (‘suspect’, from the online game ‘Among Us’). And finally, highlighting the multilingual setting of Kiezdeutsch, speakers integrate elements from a range of different heritage languages, e.g., Turkish \textit{lan} (lit. ‘guy’) as a term of address, or \textit{canım} (lit. ‘my soul’) as a term of endearment.

This specific mixture is not used in just any random setting; for instance, it does not turn up in conversations with parents or towards teachers (\cite{Wiese2013}; \citeyear{Wiese2022}). It is reserved for peer-group situations and is characteristic for this specific com-sit.\footnote{See also \citet{WiesePohle2016} for an example of \textit{grammatical} patterns specialised like this.} The following quotes highlight this, one from a speaker in Germany on his use of Kiezdeutsch, and a strikingly similar one from a young woman in Tanzania on her use of the urban contact dialect there (translations by me and Reuster-Jahn \& Kießling 2006, respectively):

\ea
Speakers on com-sit specialisation of urban contact dialects\\
\ea
{Germany \citep[213]{Wiese2012_kiezdeutsch}}\\
  \begin{addmargin}[15pt]{0pt}``At home with my parents, I speak more respectfully. After all, they are the ones who created me, whereas my friends are the ones whom I met, so I speak differently
     to them."  \end{addmargin}
    
\ex
{Tanzania \citep[16]{ReusterJahnKießling2006}}\\
   \begin{addmargin}[15pt]{0pt}  ``I can’t speak in that way to my father, but in the back-yard we use this language." \end{addmargin} 
\z
\z

Such linguistic differentiation can then lay the foundation for the emergence of new varieties in a specific com-sit. Within speakers’ general repertoires, linguistic elements are overall connected as part of (generic) language. Through their distribution over different com-sits, though, they get differentiated into different linguistic ecologies. As the example of Kiezdeutsch shows, this can lead to the formation of new varieties, and I will get back to this point in more detail in Chapter \ref{bkm:Ref112935602}. What is relevant for us at present is that com-sits support linguistic differentiation, and this is particularly obvious in multilingual contexts of language development.

We can see this also in first language acquisition. This is a bit different from the formation of new varieties, since in individual development, elements already bring their different com-sit associations with them. A young child’s task is then – among others – to figure out this com-sit distribution. Children manage to achieve this through clues they get in social interaction, and they are pretty good at it, differentiating their linguistic resources according to com-sits from early on. For an illustration, let me spell this out for the case of my daughters, when they started out linguistically.\footnote{Given the history of the field of first language acquisition, I should emphasise that I am describing this to illustrate, using a concrete example, the point of com-sits (vs. socially constructed languages) from a developmental perspective, not as a case study on multilingual acquisition (see \citealt{Clark2019} for an overview of research on first language acquisition, including early diary studies on researchers’ own children).}

Before our first daughter, Carlin, was born, my husband and I – like many parents in the Global North – planned to look after her just the two of us, which we felt confident we could do, having read everything on baby care that we could lay our hands on. Once she was born, though, it very quickly became clear that we were pretty much clueless and urgently needed help. This is where Kadriye came in. A mother of a friend’s colleague and already several times a grandmother, immigrated to Germany from Turkey in her thirties, she was looking for a part-time job and was willing to rescue us. From Carlin’s second month of life, Kadriye became her third main caretaker several times a week.

Hence, Carlin had three adults in her life to provide her main first linguistic input: Kadriye, my husband and myself. Kadriye usually spoke Turkish to Carlin, my husband (who is British) used English towards her, and I spoke mostly German. We hence formed a multilingual household with a range of different “groups” of interlocutors in the sense of \citet{LepageTabouretkeller1985}, including three different groups of an adult with Carlin, plus different ones consisting of several adults with Carlin and of adults only.

We can now understand these “groups” as aspects of com-sits. As such, they reinforce systematicity. In our case, com-sits also differentiate along language borders. However, this is, of course, not what the picture looked like for Carlin, who did not have a concept of such social constructs as “German", “English", and “Turkish" yet. This is something that would come in later (I will talk about this possibility in Chapter \ref{bkm:Ref113005745}).

The foundations for the linguistic differentiation were laid by the mostly monolingual parenting strategy we employed (we did not know better at that time), reflecting the monolingual societal habitus that is generally dominant in the Global North (see \citealt{Fuller2018}). This can be quite different in the Global South (de \citealt{DeHouwer2021}), but also in vital heritage communities even within strongly monolingually oriented societies. For instance, Carlin’s childhood friends from heritage-Turkish families experienced much more language mixing in their families, and as a result were much more familiar with translanguaging practices from early on.

If, for the sake of simplicity, we look at single words again, to take just the expressions for ‘dog’ and ‘tea’ as an example, the kind of linguistic input Carlin would get would consist of words like \textit{tea}, \textit{çay}, \textit{doggie}, \textit{Wauwau}, etc., as part of (generic) language. From the point of view of com-sits, though, these elements were distributed over settings with different people. She will have encountered the words \textit{doggie} and \textit{tea} in com-sits with her father; \textit{Wauwau} and \textit{Tee} with me, and \textit{kuçu kuçu}, and \textit{çay} with Kadriye. And she might have noticed that her father and I also used \textit{tea} when we talked to each other, but that we said \textit{dog}, rather than \textit{doggie} in those com-sits; that Kadriye and I used \textit{Hund} and \textit{Tee} with each other, and that this was also what I used in com-sits with most others; and that when Kadriye talked to our next-door neighbour, they used \textit{köpek} and \textit{çay}.

%%[Warning: Draw object ignored]
So, Carlin encountered elements from English, German, and Turkish with different adults, and in each case, this could include some “baby-talk" elements specific for com-sits with her, different from what the adults used in conversations among themselves. Hence, the organisation into com-sits can support differentiation along societally constructed language borders, but also along the lines of such registers as, for instance, informal vs. more formal language, or baby- vs- adult-directed speech. \figref{fig:11} illustrates what this might have looked like from Carlin’s point of view. In this figure, different interlocutors identify different com-sits (note that this is a simplification for illustration: these are not the only relevant aspects of com-sits, of course).

Hence, different com-sits pick out different linguistic elements, supporting their differentiation. As a result of this differentiation, the connections between elements co-occurring in the same com-sits are strengthened: \textit{tea}, \textit{doggie} etc. become elements of systems, represented by the networks of lines connecting them (we will have a closer look at this in Chapter \ref{bkm:Ref112935602}). This development of a more systematic state through selective strengthening of connections mirrors the typical characteristics of a learning process.

\begin{figure}
\fbox{\includegraphics[height=6cm]{figures/Wiese_Cloud.jpg}}
\caption{Development of linguistic systems in com-sits with different interlocutors}
\label{fig:11}
\end{figure}

In principle, some elements can also remain shared even among systems that might later become associated with different “language" indices. For instance, for Carlin the phrase \textit{bye bye} will not have been restricted to com-sits with her father, since this is something used (as a loan from English) with small children in Turkish and German as well. Accordingly, she encountered \textit{bye bye} across the board, including in com-sits with her babysitter (i.e., in the same contexts as \textit{kuçu kuçu} and \textit{çay}) and with me and others (i.e., in the same contexts as \textit{Wauwau} and \textit{Tee}). Such a phrase can then later develop into an element of baby-directed registers shared between systems indexed as “English", “Turkish", and “German".

In addition, there can be also some unconventional mixing of elements across com-sits. When Carlin was 16 months old, we moved to New Haven in the US for a year. While we lived there, at the age of about 1 ½ years, she started using a construction that seemed to be underspecified with respect to com-sits involving me vs. my husband (and other English-speaking adults): a demonstrative pattern “ohtsa [noun]”. My guess is that she got this construction from kindergarten, where teachers would look at picture books with the children, point to a picture, and then often go “Oh! It’s a …” and then name the animal etc. on the picture.

What is relevant for our discussion here is that Carlin would use her \textit{ohtsa}{}-construction with nouns from both English and German, for instance going “oh\-tsa woogie” to a bird (\textit{woogie} > German \textit{Vogel} ‘bird’), “ohtsa ameis” for animals around a tree, such as ants or squirrels (\textit{ameis} > German \textit{Ameise} ‘ant’), but also “ohtsa beeps” when she saw grapes (\textit{beeps} > English \textit{grapes}), or “ohtsa doggie” for a dog. Hence, she used this pattern across the board. Presumably at this stage, the com-sit distinctions were not yet strong enough for her in this case to lead to complete differentiation, and allowed some mixing.

This is in line with what we generally know from research with young children growing up in bi- and multilingual environments. Children start distinguishing different linguistic systems early on,\footnote{See, for instance, \citet{Meisel1989}, \citet{DeHouwer1990}; (\citeyear{DeHouwer2021}), \citet{Genesee2019}.} and in speech perception, bilinguals seem to distinguish their languages from the first year of life.\footnote{\citet{Serratrice2018}.} This does not necessarily mean that the linguistic systems are kept totally separate, though, and, e.g., priming studies suggest that there might also be some crosslinguistic overlap.\footnote{For recent findings and overviews see, for instance, \citet{Serratrice2022}, \citet{GámezEtal2022} on morphosyntax; \citet{Engemann2022} on semantics. See \citet{Genesee2019} for an overview of evidence for differentiated grammatical systems from early stages of development.}

Note that our approach to com-sits allows for both shared representations and separations along socially constructed “language" borders: linguistic elements and patterns can participate in one as well as in more than one system that emerged in a specific com-sit. What is important from the point of view of com-sits is that young children show evidence for systematic linguistic choices according to interlocutor and other situational factors early on in the acquisition process.\footnote{For an overview of findings on language choice in bilingual interaction see de \citet{DeHouwer2018}, who shows that such selective choices are evident as early as in children’s 2\textsuperscript{nd} to 3\textsuperscript{rd} year.}

This highlights that linguistic elements are learned within com-sits and with their com-sit association, as part of what \citet{Gumperz1981} described as communicative competence.\footnote{See also \citet{Møller2019}.} This does not mean that children’s choices will be adult-like right from the start, since some early deviations and idiosyncrasies are a natural part of the acquisition path. Recognising com-sits as the basis for linguistic differentiation allows for both: the early emergence of separate systems out of a pool of linguistic resources as well as some mixing and overgeneralisation along the way.

Apart from unconventional mixing, we might also see idiosyncratic deviations when children pick up on characteristics of com-sits other than the conventionally associated ones. An example of this is the common observation that young children in multilingual families sometimes avoid using their heritage language in certain com-sits that they perceive to be wrong for it, reflecting their limited experience with the heritage language so far. For instance, a child growing up in Germany with a Croatian-speaking mother and aunt might refuse to speak Croatian to male interlocutors, since they understand it as a register that is specialised for com-sits with women.

A somewhat extreme, but not uncommon, case was when at the age of three our younger daughter, Inya, seemed to consider English a linguistic quirk of her father, restricted to com-sits with him alone. Her experience had been that Daddy spoke that way, and most people also used this way of speaking when interacting with him, but not in other situations. So, when our British in-laws came visiting, Inya refused to speak English to them because she found it awkward to use the “Daddy register" with anyone else – even when it became clear they did not understand German. While this was an idiosyncratic identification of relevant com-sit characteristics, it in fact underlines the conventional and hence community-oriented nature of com-sits: such deviations put a spotlight on learner hypotheses that will be revised in later development, with more exposure to language use reflecting the conventional patterns, similarly to grammatical hypotheses from earlier acquisitional stages.

In the case of our daughter, this kind of development was evident in a transition from her idiosyncratic understanding of relevant com-sits to a more conventional one. While first, she understood the relevant com-sits as those that are “with Daddy", later she learned to use this register in all com-sits that are socially constructed as appropriate for “English". This kind of transition can be supported through explicit labelling by adults. An example of this comes from a conversation transcribed in \citet[133--134]{StavansPorat2019}, between a young multilingual child (“R”, age 3;8) and her grandmother (“GM”); the girl speaks Hebrew to her grandmother, who answers in Spanish (translation by \citealt{StavansPorat2019}):

\ea\label{ex:constructingENG} Constructing English as a named language
\begin{quote}
   \begin{enumerate}
    \item [R: ] \textit{Safta, at yodaat, ani yodaat ledaber basafa shel hagdolim.}\\
    “Grandma, do you know, I know (how to speak) the language of the grownups.”
    \item[GM:] \textit{¿De veras? A ver, ¿qué sabes decir?}\\
    “Really? Let’s see, what can you say?”
    \item[R: ] \textit{Ani yodaat lehagid shalosh milim.}\\
    “I know how to say three words.”
    \item[GM:] \textit{A ver, ¿cuáles palabras sabes?}\\
    “Let’s see which words you know?”
    \item[R: ] \textit{“Yes“, “no”, “goodbye”}\\
    “‘Yes’, ‘no’, ‘goodbye’”
    \item[GM:] \textit{Aha, sí ese idioma se llama inglés.}\\
    “Oh, yes that language is called English.”
\end{enumerate} 
\end{quote}
\z

So, for this girl, words like \textit{yes}, \textit{no}, and \textit{goodbye} are elements of language use between grown-ups, they are associated with com-sits among adults for her. The grandmother then constructs this as a named language, “English”, thus setting the ground for a broader com-sit association that is conventional in the girl’s larger society and involves this language index.

This being said, if an unconventional com-sit identification occurs widely and systematically, it might also become conventionalised, rather than undergo revision. If we assumed children to be the drivers of this, this would be a scenario parallel to what, in some generative approaches, has been hypothesised for “transmission failures” as a source of grammatical change.\footnote{Cf. \citet{Lightfoot1991}, \citet{Kroch2001}.} The primary location of such changes in child acquisition has subsequently come under criticism, though,\footnote{Cf. \citet{Meisel2011} for a discussion.} and sociolinguistic studies point to adolescents as a central group for language change, with innovation rather than non-target hypotheses playing a key role.\footnote{\citet{Eckert2000}, \citet{Tagliamonte2016}.} This is plausible for the level of com-sits as well.

In young people, unconventional com-sit identifications can be seen, for instance, when lexical elements associated with informal com-sits are also used in more formal ones, or when elements from written com-sits in social media are also used in spoken interaction. In both cases, this points to a generalisation of the associated com-sits, and this generalisation can be taken up in the broader society and become conventionalised. This has just happened, for instance, for \textit{lol}, which has crossed the boundary between written and spoken com-sits and is now increasingly used in speech as well.

An example of another kind of com-sit change comes from the “language mixing" sometimes observed in multilingual adolescent peer groups. We can now characterise this as a development where elements associated with different named languages (e.g., “Turkish" and “German") and initially different com-sits (e.g., talking to parents at home vs. talking to a teacher at school) can be used within the same com-sit (hanging out with friends). Such scenarios can support diachronic change at the level of com-sits, and we can now understand this com-sit development as the basis for grammatical change in language contact, for instance for the emergence of new mixed languages (see the discussion of urban contact dialects in the Global South in Chapter \ref{bkm:Ref113005745} below).

\section{How do we distinguish com-sits? A look at registers}
\label{bkm:Ref114129291}\hypertarget{Toc125444659}{}
Free-range language settings hence highlight that com-sits support a differentiation of linguistic resources, and in addition, they also underline the dynamics of this process. What speakers pick up as the relevant characteristics of a com-sit can be variable over time and across different social groups.

When we model com-sits as part of lexical entries, we want to capture only those that have a differential linguistic impact, then.\footnote{Cf. already \citet[Ch. 1.5]{Halliday1978}, who emphasises that we need to describe a “relevant” situation, that is, concentrate on “those features which are relevant to the speech that is taking place” (p.29).} But what does this mean exactly, how can we identify a com-sit and distinguish it from another, how can we pinpoint what is relevant in a given com-sit? To answer this question, a look at register studies can be informative, since we can regard register variation as the linguistic reflection of com-sit differences. In their influential approach to linguistic registers, \citet[6]{BiberConrad2009} define a register as “a variety associated with a particular situation of use”, and propose that to investigate registers, one needs to look for associations of linguistic differences with situational characteristics.

While these linguistic features constitute registers, the situational characteristics can now be understood to identify com-sits, making com-sits and registers two sides of a coin: registers represent the linguistic side, com-sits the situational one; as illustrated in \figref{fig:12}.

\begin{figure}
\includegraphics[height=3.2cm]{figures/Wiese_RegisterComsit.jpg}
\caption{Com-sits and registers}
\label{fig:12}
\end{figure}

As we saw in \figref{fig:11} above, com-sits can differentiate such registers as baby- vs- adult-directed speech, but they also differentiate what is ordinarily understood as different languages, e.g., English, German, or Turkish. Furthermore, different com-sits can also differentiate between ways of speaking that are commonly regarded as different dialects of one language. An example is discussed in \citet{Sharma2018}, who analyses the bidialectism of an Indian-American actor and shows how he uses American English with a US audience, but Indian English with an Indian one.

In our approach, both named languages and dialects can now also be understood as registers. The key is the choice of linguistic resources for different com-sits. What form these resources take does not need to be restricted any further: we do not need to make categorical differences between, say, “languages", “dialects", and informal vs. formal “registers". Instead, we can capture them under a unified perspective as registers, understood as systematic linguistic choices associated with different com-sits.

If we look at the sociolinguistic literature on multilingual settings, we see that this is in line with, e.g., Pennycook’s (\citeyear{Pennycook2018}) call to allow for languages as registers. This sets different languages on a par with different language varieties and styles. Accordingly, it has been suggested not only in language acquisition and contact linguistics,\footnote{For instance, \citet{Tracy2014}, \citet{McSwann2017}.} but also in generative grammar,\footnote{\citet{Roeper1999}.} to regard everyone as multilingual in the sense that they choose from a broader linguistic repertoire. Along these lines, \citet{MontanariQuay2019} apply the concept of translanguaging to supposedly monolingual speakers as well, given that they can integrate different varieties of a language in their linguistic practice. In the same vein, \citet{PageTabouret-Keller2006} draw a parallel between those linguistic choices that signal acts of identity and belonging in multilingual communities and those in monolingual ones, with the only difference being that the varieties in multilingual communities (viz. different socially constructed “languages") might be more distinct than those in monolingual ones.

This is how Michelle Obama put it in her autobiography when describing her linguistic repertoire at a time when she had her first job after university in a Chicago law firm:

\begin{quote}
I thought of myself basically as trilingual. I knew the relaxed patois of the South Side and the high-minded diction of the Ivy League, and now on top of that I spoke Lawyer, too. \citep[94]{Obama2018}
\end{quote}

Under our unified approach, such linguistic variation can now consistently be understood as different registers that are chosen according to different relevant com-sits.

An important methodological question is then: How can we pinpoint what characteristics are relevant; what are the situational characteristics that distinguish one com-sit from another? The answer is that there is in fact no definite answer to this, and intentionally so: the situational characteristics that distinguish one com-sit from another are those that speakers pick out as relevant, and what counts as relevant can turn on a range of different aspects, including cultural, social, and psychological factors. Hence, there is no extensive list as a definition of com-sits, but rather a principally open set, and what is relevant from this set is an empirical question.

Methodologically, this means that if we observe speakers using different ways of speaking, we need to check whether these differences are routinely associated with different situations. If so, then these are indeed different com-sits that support register distinctions. The next step is then to find out which characteristics of these situations are the ones that support different ways of speaking, and this will give us the relevant com-sit characteristics. Not coincidentally, this is the same challenge that young children face in language acquisition: figuring out what the key characteristics are that distinguish one com-sit from another (Is it the interlocutor, e.g., Daddy? Or the location, e.g., at home vs. in the shop?).

So, com-sits identify the relevant situational contexts for register distinctions, and it is an empirical question what is relevant. For this empirical investigation, we can draw on findings from register studies that can give us a first idea as to what might be promising candidates for this.

In a way, this is parallel to what we do in grammatical analysis, and this parallelism might help to further elucidate the issue of com-sit characteristics, so let me spell this out for an example. When we want to identify a syntactic context, for instance when we want to determine what is the relevant context for bare NPs, this is likewise an empirical question, and we have to look at the data: In what contexts are determiners optional? Based on what we have learned from similar investigations, we will already have some ideas of what might be promising candidates, for instance characteristics of semantic class (e.g., animacy), information structure (e.g., topichood), or definiteness. So, when we are looking for the relevant contextual characteristics for bare NPs, we would probably start checking such grammatical and pragmatic features. Note, though, that this does not mean that we would categorically restrict our investigation to these features; rather we would remain open to unexpected ones, and it would actually be particularly interesting to find a novel domain.

Similarly, when investigating situational characteristics that might be relevant for com-sits, we should be principally open, since it is an empirical question what is relevant for a certain way of speaking. At the same time, just like in syntax, we already know based on previous observations that there are some general situational characteristics that will often pop up as relevant.

Key candidates for such parameters that we know from register research are (in-)formality and the relationship between interlocutors on the one hand, and spoken vs. written mode on the other hand. In addition, register research has identified mode and tenor of discourse,\footnote{See, e.g., \citet{Neumann2014}, \citet{Halliday1978}.} its narrative vs. non-narrative nature,\footnote{\citet{Biber2014}.} or speaker constellation and social distance.\footnote{\citet{Maas2010}.} The latter can be seen as possible specifications of the (in\nobreakdash-)formality parameter, which illustrates another important point: just as there is no closed set of relevant situational characteristics, there is also no fixed level of categorisation. Depending on what we want to capture, we might relate to broader categories or to finer-grained distinctions between com-sits. This also means that differences between com-sits are not rigid, and that they have permeable “borders". Hence, when we speak of a specific com-sit, this is an abstraction for analysis, parallel to what we do to capture register variation.

Another aspect of registers that can shed light on com-sits is that registers are not idiosyncratic, but shared in their essentials across language users in a community. This is emphasised, e.g., by \citet[3]{LüdelingEtAl2019}, who define register as “those aspects of socially recurring intra-individual variation that are influenced by situational and functional settings”. In keeping with our discussion above, the “situational and functional settings” in this definition correspond to the situational characteristics (including functional aspects) relevant for com-sits, and the intra-individual variation of registers captures the fact that the elements co-occurring in a particular com-sit are part of larger speaker repertoires. What is interesting now is that this variation is required to be socially recurring, which is in line with Agha’s (\citeyear{Agha2004}) notion of “enregisterment”. This relates to our understanding of communication as a social activity. As such, communication is an interaction between speakers that is guided by social conventions. This means that speakers refer to a social community as the frame for what the relevant characteristics are.\footnote{Note that this does not mean that there cannot be any individual differences within a community. For instance, as \citet{Adli2017} shows for Parisian French, speakers’ lifestyle (toward “orthodoxy” vs. “heterodoxy”) can influence the way they act linguistically in formal vs. informal settings.} Accordingly, we can understand a sociolinguistic community of practice as one that is based on shared patterns in com-sits.

What speakers pick up as the relevant characteristics of a com-sit affects how they see the way of speaking that they associate with that com-sit, it has an effect on the social meaning they attribute to that register. This can guide their choice of linguistic resources and the way they integrate them into their own style in different contexts. Such relevant characteristics can also change over time. An example of this are the com-sits associated with urban contact dialects that emerge among adolescents in multiethnic and multilingual neighbourhoods.

Typically, the com-sits associated with such dialects are initially restricted to these specific settings, that is, relevant characteristics are urbanity, youth, and ethnic and linguistic diversity. Further on, such dialects can loosen their association with a specific community and setting and spread to broader contexts, for instance, generally to com-sits among adolescents or to informal urban settings. From the point of view of com-sits, we can capture this as a broadening of the com-sit base when less specific situational characteristics become relevant. As a result of such broadening, such ways of speaking can take on new social meanings. At first, an urban contact dialect might have been associated with multiethnicity in urban youth culture, optionally with such additional speaker stereotypes thrown in as masculinity or street toughness. Once their com-sits broaden, the social meaning of such dialects changes as well, and they might then indicate “urbanity" in general, or, linked to this, “modernity" or “coolness", as has been described for examples from Africa as well as Europe.\footnote{Cf., for instance, \citet{Kießling2005} on \textit{Camfranglais} in Cameroon; \citet{Kerswill2014} on \textit{Multicultural London English} in the UK; \citet{Wiese2022} for an overview.}

Com-sit deviations can also carry social meaning. For instance, if we use elements linked to child-directed com-sits with adults, this is associated with some kind of infantilisation. Accordingly, baby talk can be used to make someone look insignificant and ridicule them, but it can also be used to signal intimacy among lovers.

As another example, expressions that are associated with highly informal com-sits might be neutral there, but can take on a pejorative meaning if they are used in more formal ones. For instance, in German, if you ask a friend for a cigarette, you could call it a \textit{Zigarette}, but also colloquial \textit{Kippe}, but the latter would be impolite when you talked to a shop owner. Similarly, in English you might suggest having a little \textit{nosh} when you are out with friends, but when you then go to a restaurant, you would normally not use this term for food with the staff there. This is not restricted to words, but can also be observed for grammatical distinctions, for instance for the difference between formal and informal forms of 2\textsuperscript{nd} person pronouns (as in French \textit{vous} vs. \textit{tu}, or German \textit{Sie} vs. \textit{du}). In 2003, a politician of the German Greens found himself facing a 2000€ charge for addressing a police officer with the informal “du” rather than the formal “Sie” form, which the court ruled was an insult since the officer was not a friend of his.

Conversely, expressions that initially have a pejorative meaning might get neutralised in specific informal com-sits. This often happens in youth language, and examples for such allegedly “rude" words are often quoted in the public discussion of urban contact dialects (sometimes involving righteous indignation about some perceived decline of manners).

We can now capture such patterns with reference to com-sits:

\ea
{\label{bkm:Ref121480625}Social meaning of com-sit shifts}\\
\ea
From baby talk to adult talk\\
\begin{quote}
  CS\textsubscript{LE} ${\in}$ babytalk, LE $\subset$ \{…, addressee=adult\} {\textbar}

  P. P ${\in}$ \{speaker I\textsuperscript{n} addressee, speaker I\textsuperscript{p} addressee[lover], …\}
\end{quote}
  \ex
From informal com-sits to formal com-sits\\
\begin{quote}
  CS\textsubscript{LE} ${\in}$ informal, LE $\subset$ formal {\textbar}

  P. P ${\in}$ \{speaker I\textsuperscript{n} SEM\textsubscript{LE}, …\}
\end{quote}
  \ex
From unspecific com-sits to particular informal com-sits\\
\begin{quote}
  CS\textsubscript{LE} ${\in}$ \{MUY, …\} {\textbar}

  P. P ${\in}$ \{speaker I\textsuperscript{n} SEM\textsubscript{LE}\} ${\blacktriangleright}$ P ${\in}$ \{speaker I SEM\textsubscript{LE}, …\}
  \end{quote}
\z
\z

In these patterns, LE stands for a lexical item (in the general sense we are using this here, i.e., including more abstract patterns), CS for its com-sit, SEM for its semantics, and P for its pragmatic contribution. I is an expressive interval as defined by \citet{Potts2007}; I\textsuperscript{n} is a negative interval, I\textsuperscript{p} is a positive one, and I without a superscript is neutral, that is, not marked as either negative or positive. MUY identifies multiethnic urban youth as an example for a peer-group setting among adolescents, illustrating the possible neutralisation of pejorative terms in this kind of com-sit, where ${\blacktriangleright}$ indicates that the common, pejorative pragmatics are replaced by a neutral one.

Hence, the formalisations in (\ref{bkm:Ref121480625}) can be read as something like (a) “For a lexical item whose characteristic com-sit is part of babytalk and which occurs with an addressee who is an adult: the speaker expresses a negative evaluation of the addressee, or a positive one of an addressee who is their lover.”  (b) “For a lexical item whose characteristic com-sit is part of informal ones and which occurs in a formal setting: the speaker expresses a negative evaluation of its referent.” (c) “For a lexical item whose characteristic com-sit is part of multiethnic urban youth settings (and some others): the speaker’s evaluation of its referent can change from negative to neutral.”

Another aspect of com-sits that urban contact dialects highlight is that com-sits support the emergence of grammar. These urban contact dialects are not just characterised by a bunch of words, of course: their elements are integrated grammatically. Hence, linguistic elements can organise into different systems through their association with different com-sits. Let us have a closer look at this in the following chapter now.
