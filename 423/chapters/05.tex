\chapter{Grammar is grounded in com-sits}\label{chap:4}
\label{bkm:Ref112935602}\hypertarget{Toc125444660}{}\section{Becoming birds of a feather}
\label{bkm:Ref114133435}\hypertarget{Toc125444661}{}
Com-sits distinguish linguistic elements, and these elements can then form separate systems. Crucially, they do so through their co-oc\-cur\-rence in specific com-sits, not because they are associated with different ``languages''. Frequent co-occurrence supports the emergence of coherence within a linguistic ecology. If we look at elements like \textit{doggie}, \textit{tea}, or \textit{çay} as the linguistic resources that a young child might access, we can understand them as part of a general “feature pool”. This is a metaphor that has been used to describe the diverse linguistic resources that speakers can access in contact situations: a pool of features coming from different sources.\footnote{Cf. \citet{Mufwene2001}, \citet{CheshireEtAl2011}.}

When the elements of such a feature pool form different systems through their differentiation into com-sits, they support something more like a “feature \textit{pond}”.\footnote{\citet{Wiese2013}; (\citeyear{Wiese2022}).} The “pond" metaphor emphasises that what we see here is a network of interdependent features, a linguistic ecology that brings forth interconnected linguistic patterns at different levels.

Urban contact dialects like Kiezdeutsch are a good example for this. As we have seen, Kiezdeutsch is a language use that developed in specific com-sits, namely in peer-group situations of multiethnic urban youth settings. The linguistic characteristics of Kiezdeutsch are not just lexical, but also include grammatical patterns. For instance, as described in Chapter \ref{bkm:Ref82075181}, we find additional word order options that are not part of standard German. Some other features that have been described for Kiezdeutsch are bare local NPs and new light verb constructions. A closer look at such patterns shows that they are not just co-occurring in the relevant com-sits, but that they are also interrelated and integrated into a new, coherent system.\footnote{\citet{WieseRehbein2015}.} Hence, the association with a characteristic com-sit supports the emergence of a grammatical system.

This is in accordance with approaches such as Pennycook's (\citeyear{Pennycook2010}) who takes coherence to emerge from sedimented linguistic practices. In our approach, the basis for such coherence is the systematic association of linguistic practices with different com-sits. The ‘pond’ metaphor captures the way linguistic elements form systems based on their interaction in such different com-sits. Within this metaphor, we can think of different com-sits as contributing different environmental conditions that favour certain elements over others, and speakers as something like the gardeners in such settings, with an active role in the creation of such ponds.

This means that the “pond" metaphor allows us to capture speakers’ selective choices from a broader range of linguistic resources in different com-sits and, crucially, it allows us to do this without neglecting systematic relations within linguistic systems. This way, we can acknowledge grammatical patterns and systematicity at levels of linguistic form, rather than speaking of “errors" whenever something is not part of standard language.\footnote{Cf. \citet{Wiese2013} for a critique of such a deficit perspective in earlier sociolinguistic accounts of multiethnolects.} The “pond" metaphor captures that something systematic is going on, with patterns that are not just reinvented every time someone says something: while language, as a social practice, is variable and, in this sense, fluid, it is at the same time also restricted by social constraints.

Taking the metaphor a bit further, the impact of local weather and overall climate on a pond can stand for sociolinguistic influences at the meso level of the local setting (for instance, multiethnic urban youth) and at the macro level of society, respectively. I will discuss this in Chapter \ref{bkm:Ref113005745} below for the example of local pride of ethnic and linguistic diversity ($\rightarrow$ ‘weather’) and a monolingual habitus dominant in the country as a whole ($\rightarrow$ ‘climate’). Both meso and macro levels can influence what part of linguistic resources are used in a certain com-sit. Vice versa, the feature pond in this com-sit, that is, the linguistic system emerging here, can also influence the sociolinguistic context. To remain in our metaphor, think of the way a pond might impact the surrounding meteorological conditions.

As Ben Rampton (p.c.) pointed out, the lines of influence are multi-directional: linguistic form can shape situated use, and situated practice can affect cultural ideology. Accordingly, com-sits can also impact how language ideologies play out, or, as \citet[120]{Fuller2018} puts it, “there is situational variation in how ideologies are made manifest."

Note that the “pond" perspective does not imply that ``languages'' are the source of grammatical systematicity. An important point in the scenario I developed here is the primacy of com-sits over languages. Under this view, linguistic elements are initially not used in one com-sit rather than in another because they belong to different languages. Instead, they might be represented as belonging to different languages because they are used in different com-sits. This puts, so to speak, the saying of “Birds of a feather flock together” on its head. Elements can obtain the same “language" or “dialect" feathers because they occur together, not the other way round; in other words: Those who flock together become birds of a feather.

Remember that the way com-sits support different language use is not idiosyncratic, but plays out within a speech community.\footnote{See our discussion of \citet{LüdelingEtAl2019} in Chapter \ref{bkm:Ref114129291} above on “socially recurring” patterns.} In this sense, grammar, as \citet{Höder2018} points out, is bound to a community. However, the grammatical systems that emerge from such usage are not globally linked to a community, but to specific com-sits. For instance, as discussed in \ref{bkm:Ref115420422} above, the grammatical characteristics of Kiezdeutsch will not pop up in any language use by adolescents in urban multiethnic areas, but only when they are chatting informally in peer groups. That is, Kiezdeutsch grammar does not emerge at the overall level of the speech community per se, but in specific com-sits within that community.

In settings of small-scale multilingualism, the relevance of com-sits can be highlighted by geographical links that some traditional cultures draw for different kinds of language use. For instance, for Australia, \citet{PakendorfEtAl2021} describe some examples with conventions about what language use is suitable for a specific region which also included the requirement to know the expressions for the flora and fauna in the language use associated with the respective territory; \citet[146]{Merlan1981} discusses the “use of language varying with locale” and illustrates this with an example from story-telling where two totemic figures who were travelling together changed into another language when entering another geographical region. She argues that such associations between place and language can remain stable over time, even if the “personnel” changes, that is, independently of the (dis-)continuity of speaker groups.\footnote{See also \citet{Khanina2021} for linguistic associations with geographic or territorial social groups (rather than ethnicities) for the nomadic people of the Lower Yenisei in northern Siberia.}

\hspace*{-1.2pt}Taken together, such phenomena show that grammatical systems are grounded in com-sits and do not require bound languages and linguistic borders (and possibly not even specific communities). Some free-range language settings take this even further, with language-agnostic grammatical patterns, that is, patterns that do not involve any “language" specification at all. Examples for this “grammar without language" phenomenon come from two very different com-sits: urban markets and digital social media.

\section[Grammar without ``languages'': market cries and emojis]{Grammar without “languages'': What market cries and emoji have in common}
\label{bkm:Ref119923192}\hypertarget{Toc125444662}{}
\largerpage
I have always been interested in number assignments (e.g., \citealt{Wiese2003}), and a market is, of course, a great place to investigate these, since numbers play an important role in sales interactions. In our project on the Berlin Maybachufermarkt, we found a large range of variation in the way sellers offered their products. I illustrate this with some examples in (\ref{bkm:Ref121487920}) below. I transcribed all number words as Arabic numerals to make it easier to read. The other words are nouns referring to fruit or vegetable that you will probably recognise (possibly apart from Turkish \textit{roka} ‘rocket’), numeral classifiers (German \textit{Stück} lit. ‘piece’, and Turkish \textit{tane} lit. ‘grain’) and container nouns (German \textit{Schale} ‘bowl’, \textit{Kiste} ‘box’, and \textit{Packung} ‘package’, and Turkish \textit{kasa} ‘box’). To make it more reader-friendly, I marked all vegetable nouns by italics, and classifiers and container nouns by bold script. As in the market example in Chapter \ref{bkm:Ref82075181}, I distinguished elements by colour according to different ``languages'' here: \eng{English} is marked in purple, \deu{German} in green, and \tur{Turkish} in red (in case you overlook it: the “\eng{1}” in the last example is English).

 \ea
{\label{bkm:Ref121487920}Offering produce at the Maybachufermarkt: some examples}\\
\begin{quote}
    \deu{2 \textbf{Stück} 1,50 \textit{Brokkoli}}\\
    \deu{2 \textbf{Schale} 3 \textit{Cherimoya}}\\
    \deu{\textbf{Kiste} 3 Euro \textit{Rucola}}\\
    \deu{\textbf{Kiste} 3 \textit{Mango}}\\
    \deu{\textit{Avocado} \textbf{Kiste} 3 Euro}\\
    \tur{\textit{roka} \textbf{kasa} 4 Euro}\\
    \deu{\textit{Brokkoli} 3 \textbf{Stück} 1,50}\\
    \deu{\textit{Cherimoya} 2 \textbf{Schale} 3}\\
    \deu{\textit{Mango} \textbf{Schale} 1 Euro}\\
    \deu{\textit{Mango} 3 \textbf{Schale} 2}\\
    \deu{2 Euro 4 \textbf{Stück}}\\
    \deu{1 Euro \textbf{Stück}}\\
    \tur{1  2 \textbf{tane}}\\
    \deu{5 \textbf{Stück} 23 Euro}\\
    \tur{2 \textbf{tane} 10 Euro}\\
    \deu{4 \textbf{Schale} 2 Euro}\\
    \deu{2 \textbf{Stück} 15}\\
    \deu{\textbf{Stück} 1 Euro}\\
    \deu{\textbf{Packung} 2 Euro}\\
    \tur{2 \textbf{tane} 16}\\
    \eng{1} \deu{\textbf{Stück} 50} Cent
\end{quote} 
\z\clearpage

At first glance, this kind of variability might give the impression that “anything goes" here, but a closer look reveals a recurring pattern in this linguistic diversity. This pattern involves three main components: an expression for the kind of product, one for its quantity, and one for the price. The way these elements are combined is not random, but organised by a number of rules restricting their syntactic categories, their positions, and their presence or optionality:

\begin{itemize}
\item The expressions for the product kind are nouns; the product quantity is expressed by a cardinal numeral followed by either a classifier or a container noun; and the price, by a cardinal numeral followed by the currency.
\item The expressions for product quantity and price are adjacent, while the one for the product kind goes in the periphery. Their linear order with respect to each other is variable: we can have first the quantity and then the price, or the other way round, and the expression for the product can either be in the left or in the right periphery.
\item The expressions for the product kind and for the currency are optional. The numeral in the expressions for the quantity and for the price can be left out if it refers to ‘one’.
\end{itemize}

This, then, indicates a systematic linguistic pattern that emerged in the com-sit of sales interactions at the market. Interestingly, this pattern can be used for elements across “language" boundaries. As illustrated by the examples above, there might be some defaults: in our data, elements associated with German are dominant, followed by Turkish, and then English. However, in principle elements from any language are possible here, and, as we saw in the passage in (\ref{bkm:Ref125444555}) above, the range of possible resources can be continuously broadened as speakers add new elements to their repertoire (e.g., Hebrew numerals in (\ref{bkm:Ref125444555}) on page \pageref{bkm:Ref125444555}). The last example in (\ref{bkm:Ref121487920}) illustrates that speakers can also combine elements associated with different languages, e.g., English and German in \textit{one Stück fünfzig Cent}.

We can capture this with the following lexical entry for this pattern (\figref{fig:13}).

\begin{figure}[H]
\fbox{
\parbox{.8\textwidth}{
\TabPositions{1.8cm}
    SYN:\tab \begin{forest}[Coord$_1$ [Coord [QP [Q$^0_2$] [N$^0_3$]] [QP [Q$^0_4$] [N$^0_5$]]] [N$_6$]]\end{forest}

    \vspace{0.2cm}
    SEM:\tab        TRADE\textsubscript{1} (\#\textsubscript{2}(UNIT\textsubscript{3}(KIND\textsubscript{6})), \#\textsubscript{4}(CURR\textsubscript{5}))
    \vspace{0.1cm}

    COM-SIT\tab        ${\in}$ \textit{market}
    }
    }
\caption{Entry for offering-produce pattern on the market}
\label{fig:13}
\end{figure}

Lower indices in this entry indicate links between syntactic and semantic components. The syntactic structure represents a coordination, which accounts for the flexible order of elements with respect to each other and their adjacency within the pattern. We have two quantifier phrases with cardinal numerals (Q\textsuperscript{0}) and simple nouns (N\textsuperscript{0}) as head adjuncts, which are linked to the semantic representation of numbers (\#) and either a unit of quantity (\textsc{unit}) or a currency (\textsc{curr}). Units of quantity are individuals identified by classifiers, or they are containers identified by container nouns.\footnote{For a detailed discussion of the semantics of numeral constructions, see \citet{Wiese2003}.} These quantifier phrases form a coordination that is then, at the next level, coordinated with another noun (N) that semantically refers to the kind of product (\textsc{kind}).

This grammatical pattern is associated with the market, and it has some characteristics that distinguishes it from what we are used to in, for instance, German and Turkish, two dominant market languages with typologically different numeral constructions. It involves a number of simple nouns that do not receive number marking. This does not only hold for the classifiers involved, which wouldn’t get number marking in German or Turkish either (as is typical for classifiers in general). It also holds for container nouns and for vegetable nouns, which would receive plural in standard German, although not in Turkish. This behaviour might hence be influenced by Turkish, which is one of the dominant market languages, and it might be further supported by general cross-linguistic tendencies for nouns to be transnumeral.\footnote{Cf. \citet{Wiese2019}.} A feature that sets the \textit{offering-produce} pattern apart from both German and Turkish numeral constructions, as well as from the other languages frequently heard at the market, is its status as a coordination that supports adjacency but no deeper syntactic dependencies. This might be something underlying other characteristic grammatical patterns at the market as well,\footnote{\citet{WieseSchumann2020}. See \citet{SchumannEtAl2021} for a discussion of other grammatical phenomena on the market.} suggesting a converging market grammar.

This “market grammar" was confirmed by speakers’ judgements that we elic\-ited in our focus group discussions with cooperating sellers. Using examples of German product nouns and container nouns that are often used at the market, we asked them whether one could also say “zwei Mango\textbf{s}” (‘two mango\textbf{s}’) or “zwei Kiste\textbf{n}” (‘two box\textbf{es}’). This use of standard German plural forms got roundly rejected for the market setting, with negative answers as the following (my translation):

\ea Can one also say something like “zwei Mangos” or “zwei Kisten”?\\
{}-  Not “Mangos”. “Mango”!\\
{}-  Nobody says “Mangos” here!\\
{}-  “Kiste”! That’s how one talks on the market.\\
{}-  No plural on the market!
\z

Hence, what we find here is a systematic grammatical pattern associated with a particular com-sit, the market. This confirms our view of com-sits as the basis for linguistic systematicity: the com-sit of the market supports the emergence of characteristic patterns, a “market grammar" with its own rules, options, and restrictions.

In the case of the \textit{offering-produce} pattern, the grammar is agnostic with respect to the “language" of the elements that follow these rules, and accordingly I did not include a language index for the entry in \figref{fig:13} above. This is, then, the place where the “anything goes" bit comes in: the choice of elements with respect to their linguistic affiliation is principally open – there might be some defaults, but in general you use whatever works in communication. Hence, what we have here is grammatical structure without a language specification. This underlines the primacy of com-sits I discussed above, highlighting an important point: we don’t need languages for linguistic systematicity, it is com-sits that engender grammatical systems. As a result, grammatical systems can transcend “language" borders.

This lack of “language" restrictions can also be found at the level of individual lexical items. In settings of contact between closely related languages, the linguistic integration can be such that not all elements can be associated with one specific source language. \citet{Pecht2021} described this for \textit{Cité Duits}, a Dutch-Maaslands-German contact variety that is a bit like an historic example of Kiezdeutsch: Cité Duits emerged among the children of immigrants in the coal mining district of Eisden in Belgian Limburg in the 1930s and served as a marker of identity in a linguistically diverse community. As \citet{Pecht2021} shows, Cité Duits developed its own grammatical characteristics and, at the lexical level, included a number of elements that are not discernibly either Belgian, German, or Maaslands, but cross such boundaries. This further emphasises that languages are optional and that linguistic elements can do without a language index.

Another case in point comes from digital social media. In this type of free-range language use, we find a number of new graphic elements, including emoticons and emoji. These elements can be used referentially – an example would be if I texted you about using my
\includegraphics[height=0.4cm]{figures/Wiese_computerEmoji.jpg}
  for writing this book, or about my
\includegraphics[height=0.4cm]{figures/Wiese_dogEmoji.jpg}
  who is lying under my desk while I do this. In a more interesting, and much more frequent use, though, emoji have pragmatic, non-referential functions, as illustrated by the following example from the data we collected in the RUEG group.

% \ea
% WhatsApp message to a friend (German original on the left; idiomatic translation on the right):
% \begin{tabularx}{.8\textwidth}{QQ}
% \ttfamily
%   Dikka Brat du weißt nicht grad was passiert ist ja...einfach eine Frau ist über Rot gefahren und ein auto hat sie erwischt \includegraphics[height=1em]{figures/emoji_cry} ...einfach so traurig ja...Autofahrer hilft ihr zwar aber er könnte auch bremsen\includegraphics[height=1em]{figures/emoji_grin}. Andererseits ist die Frau dumm &
%   Man, bro, you don’t know just what
% happened yes … simply a woman
% crossed the red light and a car
% got her \includegraphics[height=1em]{figures/emoji_cry} … simply so sad
% yes … Car driver is helping her all
% right but he could also brake \includegraphics[height=1em]{figures/emoji_grin}. On
% the other hand, the woman is dumb.’
% \end{tabularx}
% \z

\ea\label{ex:Whatsapp} WhatsApp message to a friend (German original on the left; idiomatic translation on the right):

 \begin{minipage}{0.4\textwidth}
    \hspace*{-5mm}\includegraphics[height=.2\textheight]{figures/Whatsapp_screenshot.png}
    \end{minipage}
    \hfill
\begin{minipage}{0.45\textwidth}
        ‘Man, bro, you don’t know just what\\
        happened yes\dots simply a woman\\
        crossed the red light and a car\\
        got her {\includegraphics[height=.025\textheight]{figures/emoji_cry}} \dots simply so sad\\
        yes\dots Car driver is helping her all\\
        right but he could also brake {\includegraphics[height=.025\textheight]{figures/emoji_grin}} . On\\
        the other hand, the woman is dumb.’
    \end{minipage}
\z

In examples like this, emoji can be understood as graphic discourse markers:\footnote{\citet{WieseLabrenz2021}.} they do not contribute to the truth value of an utterance and their contribution is not at the referential level, but rather at the level of discourse. In this usage, emoji can fulfil different kinds of discourse functions that can be characterised as intersubjective (e.g., conveying a positive social persona), subjective (e.g., sadness or expressing a sympathetic stance towards the contents), and textual ones (e.g., marking narrative boundaries).

When you look at my translation into English, you will notice that I did not translate the emoji. This reflects what I called their \textit{translinguistic} status: they are elements with no specific “language" affiliation, transcending linguistic borders. Interestingly, their grammar is the same across German and English. They appear dominantly after sentences or, more generally, communicative units (since such messages do not require full syntactic sentences), and this is also the position of the two emoji in our example. Emoji can also appear as lone items, that is, people might send just an emoji, with no additional text. This is often in reaction to a message, so in a broad approach, we can subsume this under the right-peripheral pattern. In addition, they can be used in the left periphery of a communicative unit, but this is less frequent.

We can account for these options with an ordered set < (CU)\_\_ ,  \_\_ CU > where the first element represents the more dominant choice. In this representation, “\_\_” marks the position of the emoji, and “CU” stands for a communicative unit. Hence, this tuple captures that emoji appear typically after a communicative unit (“CU \_\_”), but can also, less frequently, appear in front of it (“\_\_CU”). Emoji as lone items can be captured by marking the first “CU” as optional (indicated through round bracketing). An entry for these translinguistic graphic discourse markers could hence look like in \figref{fig:14}.

\begin{figure}
\fbox{
\begin{tabular}{ll}
% \lsptoprule
PHON: & [unicode], e.g., \includegraphics[height=1em]{figures/emoji_happy.png}
  or
  \includegraphics[height=1em]{figures/emoji_happy2.png}
 \\
SYN: & < (CU) {\longrule},  {\longrule} CU >\\
PRAG: & intersubjective, subjective, textual discourse functions\\
COM-SIT & %%[Warning: Draw object ignored]
${\in}$  \textit{digital-social-media}\\
% \lspbottomrule
\end{tabular}
}
\caption{\label{fig:14} Entry for emoji}
\end{figure}

This gives us a lexical entry for emoji in general. The phonological representation stands for the different unicode definitions that identify individual emoji. The choice of individual elements and their specific pragmatic range can differ, for instance for age groups: I gave two variants of the ‘smiling face’ emoji as an example, where the first variant seems to be more typical for the Boomers among us, while the second is currently preferred by the younger crowd (… at least by those of them that are still using emoji and have not switched to emoticons in order to distance themselves from the emoji-enthusiastic older generation).

Different usage patterns can also occur in different cultural contexts, and in principle also for different language contexts, of course. The important point is that a specific language affiliation is not necessary for the linguistic systematicity we observe here. Just like the market cries above, this free-range language use shows us that it is the com-sit that calls the shots: the com-sit (in this case digital social media) is the basis for the emergence of linguistic systematicity, in written as much as in spoken language, and the fact that patterns can also remain agnostic with respect to languages makes this particularly salient.

So, what is the point of languages and dialects, then? Do we do away with them altogether, just sticking to registers and be done? The next chapter shows that free-range language also points to an important function of languages, namely as social indices.
