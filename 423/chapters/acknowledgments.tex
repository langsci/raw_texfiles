\chapter{Acknowledgements}
% \todo{in Heike's version acknowledgements comes after "by way of intro"}
In developing this approach, I was fortunate to be able to build on a range of projects over the last years, which involved many collaborators whose contributions are gratefully acknowledged, as is the funding by the German Research Foundation (DFG) that made those projects possible:

\textit{Kiezdeutsch: Grammar and information structure} (2008–11; INST 336/53-1,2; project B6 of CRC 632 “Information Structure”; PI: Heike Wiese; research associates: Ulrike Freywald, Ines Rehbein, Sören Schalowski); \textit{The dynamics of German in the multilingual context of Namibia} (2016–20; WI 2155/9-1; PIs: Heike Wiese, Horst Simon; research associates: Christian Zimmer, Janosch Leugner, Laura Perlitz, Yannic Bracke; collaborators: Marianne Zappen-Thomson, Hans Boas); \textit{Integration of linguistic resources in highly diverse urban settings} (2017–21; INST 336/131-1; project A01 of
CRC 1278 “Limits of Variability in Language”: PIs: Heike Wiese, Ulrike Freywald; research associates: Kathleen Schumann, Serkan Yüksel, İrem Duman Çakır, Britta Schulte); \textit{Emerging grammars in language contact situations: A comparative approach} (2018–24;
FOR 2537/1,2; speaker: Heike Wiese; PIs (15 projects): Artemis Alexiadou, Shanley Allen, Oliver Bunk, Natalia Gagarina, Mareike Keller, Anke Lüdeling, Judith Purkarthofer, Christoph Schroeder, Anna Shadrova, Luka Szucsich, Rosemarie Tracy, Heike Wiese, Sabine Zerbian; Mercator fellows: Shana Poplack, Maria Polinsky; Cristina Flores, Jeannine Treffers-Daller); \textit{Register perception in a multilingual context of German} (2020–23; INST 276/830-1; project C07 of
CRC 1412 “Register”; PIs: Heike Wiese, Antje Sauermann; research associate: Britta Schulte).

Key aspects of the approach developed here were presented at
UKLVC 2019, the Sociolinguistics Series at the University of
Leiden 2019,
DGfS 2020 and 2022,
IPrA 2021, German Abroad 2021, Abralin ao
Vivo 2021,
SLI 2022, the Cognitive Science Series at the CEU
Vienna 2022, Heritage Language Syntax 3, and SALT 33. I thank the organisers and participants for constructive discussions and suggestions.

Special thanks go to Ben Rampton, who was the discussant for my paper at the Abralin series and provided insightful questions and comments that helped me to further develop and sharpen my approach. The results of this went into a contribution to the WPULL series (\textit{Working Papers in Urban Languages and Literatures}, ed. Ben Rampton) that provided the foundation for this book, and accordingly the book can be understood as a revised and extended version of that working paper \citep{Wiese2021}.

My account has benefited from numerous discussions with colleagues over the past years, from their helpful and always constructive feedback, suggestions, and insights. I want to thank Aria Adli, Artemis Alexiadou, Shanley Allen, Jannis Androutsopoulos, Dalit Assouline, Hans C. Boas, Inke Du Bois, Oliver Bunk, Jenny Cheshire, İnci Dirim, İrem Duman Çakır, Cristina Flores, Ulrike Freywald, Janet Fuller, Natalia Gagarina, Annick de Houwer, Matthias Hüning, Ray Jackendoff, Paul Kerswill, Philipp Krämer, Annika Labrenz, Nils Langer, Anke Lüdeling, Yaron Matras, Christine Mooshammer, Mehmet Öncü, Maria Piñango, Heike Pichler, Judith Purkarthofer, Jason Rothman, Joe Salmons, Christoph Schroeder, Nicole Schumacher, Sheena Shah, Devyani Sharma, Horst Simon, Bente Ailin Svendsen, Marina Terkourafi, Rosemarie Tracy, Lisa Verhoeven, Eva Wittenberg, Marianne Zappen-Thomson, and Sabine Zerbian.

I am grateful to the series editor, Mark Dingemanse, for his valuable input on this book, and to two anonymous reviewers for their insightful, constructive, and helpful suggestions.

Special thanks go to Johanna Pott, Marvin Brink and Jonas Paetsch for their invaluable help with the final editing and conversion of the text and to Sebastian Nordhoff and his wonderful team at Language Science Press.

And finally a big “Thank you” to my family, Charles Stewart and Carlin and Inya Stewart-Wiese, whose interest in this topic was a great encouragement for me, who helped me get my thoughts in order, and also agreed to be used in some examples here!

