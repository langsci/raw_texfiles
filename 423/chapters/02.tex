\chapter{A quantum-linguistics paradox – and its solution}
\label{chap:1}


\section{Fluid registers and grammatical systems}
\label{sec:1.1}
There is a small café in my neighbourhood, Berlin-Kreuzberg, where I sometimes have a glass of tea in the morning. The other day, I saw the following sign there:

\begin{figure}[htbp]
 \begin{minipage}{0.4\textwidth}
        \centering
        \fbox{\includegraphics[height=.3\textheight]{figures/fig1.jpg}}
        \caption{Breakfast offer in a small Kreuzberg caf{é}}
        \label{fig:1}
    \end{minipage}
    \hfill
\begin{minipage}{0.5\textwidth}
        \textit{English translation:}
        \\
        Breakfast
        \\
        \\
        \\
        Toast\\
        Salad\\
        Simit
    \end{minipage}
\end{figure}

At first sight, this sign is unremarkable. It lists, in German, the food on offer for breakfast, and the spelling is in accordance with standard orthography – except, that is, for <Tost>: In standard German orthography, this would be <Toast>, a spelling that reflects the English origin of this word. In German, a “Toast” is a specific kind of bread, namely white bread of usually rectangular shape that is eaten toasted. The German pronunciation is [t\textsuperscript{h}oːst], with a monophthong [o], which means that the <oa> spelling is unusual, marking it as a loanword: following the general rules of phoneme-grapheme correspondence in German, the spelling should be <Tost> (or, alternatively, <Tohst>, with an <h> to mark vowel length).

So, the spelling on the sign could indicate a further integration into German; this would then reflect an internal motivation that also produced, e.g., the spelling <Keks> from initial English <cakes>.

Note, though, that \textit{toast} has also been borrowed into Turkish, and in Turkish the spelling is <tost> already, given the even stronger phonemic orthography of Turkish. This provides us with an alternative explanation. Kreuzberg is a multilingual neighbourhood where Turkish is a salient heritage language, and this café serves a lot of Turkish-German customers and offers a range of Turkish specialities, including the \textit{Simit} (a Turkish sesame ring) mentioned at the bottom of the sign. So, using the Turkish spelling could be flagging a Turkish-German identity, e.g., for marketing reasons.

In this case, the sign could be interpreted as integrating elements from German, namely \textit{Frühstück} and \textit{Salat}, and from Turkish, namely \textit{Tost}, plus one that works in both, namely \textit{Simit}, which has the same spelling in Turkish and (as a Turkish loanword) in German.

So, with <Tost>, we see a noncanonical spelling that could be internally motivated, or it could reflect crosslinguistic mixing. Or it could be both, with two sources of support reinforcing each other. This is what makes a setting like that of the café interesting from the point of view of language variation and linguistic architecture. In such everyday settings, there is less normative pressure from “standard" language and orthography, which means that speakers\footnote{For the sake of brevity, I use the term “speaker” here to refer to language users in general, including sign language users, but also writers.} can feel freer to follow their own linguistic style and their sense of what fits best in a certain communicative situation, and this also includes making use of linguistic resources in a way that transcends conventional language borders. This suggests integrated speaker repertoires based on “multi-competence” \citep{Cook2016}, that is, a compound system encompassing and blending elements from different sources.

The kind of language use we observe in urban cafés, markets, inner-city neighbourhoods and similar settings challenges assumptions of homogeneous speech communities and monolingualism, and along those lines, the idea of separate fixed codes. In this spirit, influential sociolinguistic approaches reject the notion of distinct “languages" and linguistic borders, as illustrated by the following quotes [my emphasis, H.W.], characteristic for what \citet{Pennycook2016} called “the trans-super-poly-metro movement”:

\begin{quote}
A serious consideration of the ways in which ideas about language have been constructed and invented forces us to consider anew not only emergent language mixes but the terms in which we think about them. […] neologisms such as \textit{translanguaging}, \textit{polylanguaging} and \textit{metrolingualism} have been used to \textbf{take us beyond the assumed frameworks of bounded languages} \citep[3]{Pennycook2018}
\end{quote}

\begin{quote}
we have to \textbf{abandon the traditional notion of separately structured languages} \citep[34]{Canagarajah2018}
\end{quote}

\begin{quote}
we \textbf{challenge} one of the most widely held views of language as a social, human phenomenon, namely \textbf{that ‘language’ can be separated into different ‘languages’}, such as ‘Russian’, ‘Latin’, and ‘Greenlandic’ (\citealt[22]{JorgensenEtAl2011})
\end{quote}

\begin{quote}
Translanguaging […] \textbf{challenges the conventional understanding of language boundaries between the culturally and politically labelled languages} (e.g. English, Chinese). (\citealt[3f]{LiWei2016})
\end{quote}

\begin{quote}
There is now a substantial body of work on ideologies of language that \textbf{denaturalizes the idea that there are distinct languages}, and that a proper language is bounded, pure and composed of structured sounds, grammar and vocabulary […]. \textbf{Named languages – ‘English’, ‘German’, ‘Bengali’ – are ideological constructions} (\citealt[3--4]{BlommaertRampton2011})
\end{quote}

Note that the last statement goes a step further than rejecting traditional linguistic boundaries: it also challenges the assumption that a language is “composed of structured sounds, grammar and vocabulary”. As someone who is interested in grammatical analysis, I found this somewhat disconcerting, since it seems to threaten the foundations of analysing language structure and linguistic systems. In fact, \citet[114]{Pennycook2010}, in line with such challenges, commends us to “move away from the attempts to capture language as a system”.\footnote{An ontological perspective on this difference is developed in \citet{DemuroGurney2021} who argue that different linguistic approaches (re-)create different realities, involving languages as objects (in line with traditional structural approaches) vs. practices and assemblages (in line with current sociolinguistic approaches to languaging and linguistic repertoires).}

Meanwhile, in a different part of the discipline it is business as usual. The field that would be most affected by this, namely grammatical analysis including theoretical syntax, morphology, semantics, and phonology, has carried on seemingly untouched by such challenges. The field has certainly moved on from assumptions of monolingualism and homogeneity implicit in such earlier structuralist idealisations as the “ideal speaker-listener” that I mentioned in the introduction. Yet, this does not mean that bound languages as a basis of analysis have been reconsidered in any way. The general attitude seems to be to ignore such contestations and to happily continue targeting structural patterns within distinct linguistic systems.

This holds not only for grammatical research on individual languages, but can also be found in language contact research. In that domain, structural linguistic findings support assumptions of separate varieties and languages, for instance when \citet[2]{PagePutnam2020} conclude that “Indian English is a distinct variety of English”, or when \citet[8]{MacWhinney2019} characterises some linguistic effects in heritage speakers as an “intrusion of an L2 form when speaking L1”.

So, why have accounts of linguistic fluidity and multi-competence and contestations of bound languages had little or no impact beyond what in social media would be called “our own bubble”? I think there are two major reasons for this. One is the sociology of linguistics as a discipline –  people tend not to talk enough to each other across subfields, and if results in another field challenge the very foundations of your work, it might be easier to just dismiss them than to engage with them seriously.

The second reason, though, is probably grounded in linguistic evidence, and is something we should actually take into acount in sociolinguistics. This is the fact that grammatical results do, after all, point to internal organisation and coherence, and to the workings and interaction of distinct linguistic systems. Hence, from this perspective, there seems to be little to compel us to do away with linguistic systems. This suggests that such an engagement across subdisciplines could be fruitful, and that it should actually cut both ways.

As a simple example, take the grammatical integration of the word \textit{Computer} into German (spelled with a capital <C> since all nouns are capitalised in standard German orthography). Traditionally, we would say that this is a loanword in German that has been borrowed from English. However, in order to avoid committing ourselves to bound languages and linguistic systems at present, let us phrase this a bit more neutrally. In this vein, we can say that this is a word that is commonly used in the context of elements societally marked as “German" but is comparably new in this context. To mark its source, we can add that the word was already used in the context of elements societally marked as “English" before that, and that its uptake in the new, “German" context is based on that earlier, “English" use.

Now, what has happened in this new context is that the noun’s grammatical behaviour has changed:

\begin{itemize}
\item The pronunciation is something like [k\textbf{ɔ}mˈpjuːt\textbf{ɐ}], with a different vowel in the first syllable and a vocalised [r] in the last coda (I marked those in bold).
\item It has gained gender: it is “\textbf{der} Computer” (masculine) in German.
\item It uses a zero allomorph to mark plural: “the computers” translates as “die Computer”, with no overt ending on the noun, a common choice for nouns ending in \textit{{}-er} in German (let us not get into German plural marking, though – a domain notorious for its abundance of irregular forms and exceptional patterns …).
\item It inflects for case. For instance, its dative plural form is “den Computer\textbf{n}”.
\end{itemize}

It will come as no big surprise when I tell you that this kind of behaviour is in accordance with what we see in other elements societally marked as “German". So, what happens when we take an element from one context to another is that its grammar changes to fit the new context.

This, then, suggests two different grammatical systems corresponding to what is commonly known as “English" and “German": when \textit{computer} enters German, it changes in a way that points to the workings of a system that absorbs new elements and brings them in line. – For the Star Trek fans among us, a collective comes to mind that integrates newcomers along the lines of “You will get grammatically assimilated. We will add your linguistic distinctiveness to our own.”

\hspace*{-3.8pt}This “Borg"-like tendency for assimilation is not restricted to German, of course, but happens generally in such cases, and as \citet{Poplack2018} argues, this even holds for nonce borrowings, that is, instances of spontaneous lexical transfer by individual speakers.

Such integration, then, illustrates the workings of grammatical systems. It provides evidence for the linguistic reality of systematicity and coherence within traditional “language" boundaries. At the same time, though, as discussed above, sociolinguistic findings provide evidence for linguistic practices that systematically transcend such boundaries.

This looks like we are left with two competing and equally justified perspectives. In fact, \citet{BlommaertRampton2011}, while reminding us that “[n]amed languages – ‘English’, ‘German’, ‘Bengali’ – are ideological constructions”, speak, in the very same paper, of a “mix of Chinese, Korean and English” and of “translations from Chinese to English”. The first point challenges named languages, but the next two quotes then seem to reintroduce them, since a mix of elements from different named languages, and translations from one to the other do, after all, imply distinct named languages. While this seems contradictory at first, I think that it actually makes a lot of sense, because named languages and their boundaries are both: ideological constructions and a reflection of actual structure.

This is what I characterised as something like a “quantum-linguistics” paradox above. We are presented with two perspectives that seem to be impossible to reconcile, yet are equally supported by linguistic evidence.

\section{Reconciliation via “com-sits”}
\label{sec:1.2}
How can we solve this paradox, then? How do we bridge the gap between sociolinguistic and grammar-theoretical insights into language? In what follows, I am going to reconcile the two perspectives by integrating findings on linguistic multi-competence with those on grammatical structure and coherence. As mentioned in the Introduction, an important element of my account is the notion of communicative situations, which I abbreviate as “com-sits”, as an intuitive shorthand for the kind of concept spelled out here.

As the foundation for this, I understand communication as a social activity through which meaning is (co-)constructed and which typically centers  around language production and perception. This definition is broadly compatible with Gumperz’ (\citeyear{Gumperz1981}) definition of “communicating as the outcome of exchanges involving more than one participant” and of communicative competence as “the knowledge of linguistic and related communicative conventions that speakers must have to initiate and sustain conversational involvement”. Like Gumperz’, our definition captures the interactive aspect of communication, by marking it as a “social activity”. It goes beyond Gumperz’ definition by further characterising this activity as meaningful, highlighting the construction and exchange of meaning that is central to communication. The construction of meaning in communication is a shared activity between interlocutors, and such “meaning” covers not only propositional meaning (what contents do interlocutors exchange?), but also social meaning (what do they communicate about themselves and their relationship?).

The specification that this is centered around language production and perception is not really necessary, since communication can also be nonlinguistic. However, for our purposes, this specification is useful, since we intend to target language use. Note, though, that the competence necessary for linguistic communication does not only relate to grammatical knowledge, but crucially also to knowing which linguistic options are appropriate in an encounter (see also \citealt{Ruuska2019}). This is an aspect that is central to capture the different choices speakers make in different com-sits.



Based on our definition of communication as a specific kind of social activity, we can now define com-sits as the setting of this activity:

\tblsfi{Definition}{
A \textit{com-sit} (``communicative situation'') is the setting of communication, understood as a social activity, typically centered around language production and perception, through which meaning is (co-)constructed.
}

Following \citet{Pinango2019}, we can understand a situation as a conceptual representation of a state or event that is organised algebraically. This means that a com-sit is always about speakers’ representations of what is going on in a communication, that is, about how they perceive it and make sense of it (see also \citealt{Malinowski1923}, \citealt{Firth1957}, \citealt{Halliday1978} on the “context of situation”). Different com-sits are then distinguished by their different characteristics as perceived by speakers. This can be broadly understood as involving everything that is socially relevant for a given communication, including such aspects as audience or topic (see \citealt{LepageTabouretkeller1985}). Hence, a com-sit is dynamic in conversation, rather than fixed. This is not primarily about the physical aspects of a setting, but about their social and cultural relevance: a situation as it is socially perceived and evaluated. If we think of \citegen[135]{Goffman1964} definition of a social situation as an “environment of mutual monitoring possibilities” that support encounters with “mutual openness to all manner of communication”, we can understand com-sits as a subset of such encounters or as a specific perspective on them: com-sits are special in that they provide a view on the actual communication in a social situation.

Com-sits will be an important element of our account. In \chapref{chap:3}, I will discuss them in more detail, when I show that they serve as a basis for the differentiation of linguistic systems, as the first lesson to be learned from free-range language. At this point, I will also discuss the relation of com-sits to registers (and I will show how they allow for a unified view on linguistic resources that are traditionally regarded as languages, dialects, and registers). The second lesson, in \chapref{chap:4}, will be that com-sits can serve as an anchor of grammatical systems, drawing on the co-occurrence of elements in such com-sits. Taken together, this means that the notion of com-sits will enable us to develop a linguistic architecture that does not need bound languages as a point of departure, but can still account for grammatical structure. As the third lesson, in \chapref{chap:5}, will show, this does not mean that we do away with languages altogether: they can come in as social indices. Crucially, though, this makes them an optional add-on, rather than the foundation of grammar.

To give you an idea of where we are going, \figref{fig:2} outlines the key features of the approach I am going to develop. The important point is the primacy of com-sits; we start from communicative situations as the primary component.\footnote{As a shorthand, I rendered com-sits as a box in \figref{fig:2}, but note that they are to be understood as dynamic and processual in communication, in line with our definition above.} In com-sits, speakers make use of different linguistic resources, with certain linguistic elements co-occurring in certain com-sits.

\begin{figure}[H]
% \includegraphics[width=\textwidth]{figures/fig2.jpg}
\fbox{
\begin{tikzpicture}
  \node(comsit)[rectangle,fill=blue!20!white]{Communicative situations};
  \node(cooccurence)[above=5mm of comsit]{co-occurence of elements};
  \node(coherence)[above=5mm of cooccurence]{coherence, systematicity};
  \node(socially)[above=5mm of coherence,xshift=-5cm,text width=3.5cm]{socially constructed\newline ``languages'', ``dialects''};
  \node[fill=blue!20!white,below=0mm of socially]{social indices};
  \draw[->,thick](comsit)--(cooccurence);
  \draw[->,thick](cooccurence)--(coherence);
  \draw[dashed](coherence)--(socially);
\end{tikzpicture}}

\caption{Outline of the approach}
\label{fig:2}
\end{figure}

We can think of such elements as lexical entries in the sense of a Tripartite Parallel Architecture (\cite{Jackendoff1997}; \citeyear{Jackendoff2002}), that is, tuples of information from different grammatical and pragmatic levels that can be more or less complex and abstract. A simple example would be a lexical entry for a word like \textit{tomato} that identifies, at the grammatical level, its phonological representation, its syntactic category, and its semantic contribution. A more complex and abstract example would be the representation of mass/count coercions that takes a count noun and yields a mass noun, for instance getting us from \textit{tomato} in “There is a tomato on the kitchen counter” to \textit{tomato} in “There is tomato in the soup.”. In this case, our lexical entry would not contain a specific phonological representation, since it represents a more abstract rule and can be applied to count nouns in general (I will spell this out in more detail for the example of \textit{chicken} in \ref{sec:1.3} below). Hence, in the Tripartite Parallel Architecture, lexical entries do not just represent lexical words, but also more abstract grammatical patterns.

\hspace*{-2.4pt}The co-occurrence of linguistic elements supports coherence and internal structure: elements that frequently co-occur form part of a system. Depending on the societal macro context, such systems can then be socially indexed as named languages or dialects.




As emphasised above, the last feature is optional, not compulsory: named languages, dialects, etc. are social constructions that might or might not emerge, depending on the speech community and/or the societal macro context in question. Named languages can have a social reality and impact speakers’ linguistic experiences and practices,\footnote{I will discuss this in more detail in \sectref{bkm:Ref121491384} below.} but it is important to keep in mind that this is not necessarily the case in all settings. Such languages are, after all, a relatively recent invention. As \citet[1]{MakoniPennycook2006} remind us, “languages were, in the most literal sense, invented, particularly as part of the Christian/colonial and nationalistic projects in different parts of the globe”. A striking example is the way some Sub-Saharan African dialects and languages were manufactured by missionaries and colonial administrations (e.g., \citealt{MakoniEtAl2007} on Shona). If we understand com-sits as the basis for linguistic patterns, we can capture this by leaving named languages optional and recognising them as social constructions.

\section{An illustration}
\label{sec:1.3}

Now that I have sketched the general outline of where we are heading, let me illustrate this by spelling out some characteristics for an example that also includes cross-linguistic interactions. Let us consider the English word \textit{chicken}. In the approach I sketched, we can account for it with the following lexical entry:

\begin{figure} [H]
% \includegraphics[width=\textwidth]{figures/fig3.jpg}
\caption{Entry for English \textit{chicken}}
\label{fig:3}
\fbox{
$
\left.
\begin{array}{ll}
\text{PHON:} & \text{/ˈtʃɪkɪn/}\\
\text{SYN:} &  \text{N}\textsubscript{count}\\
\text{SEM:} &  \textsc{chicken}\\
\text{COM-SIT} & \in E\\
\end{array}
\right\rbrace \textit{chicken}\textsuperscript{E}
$}
\end{figure}

The first three lines look more or less like what we would ordinarily see, representing the phonological, syntactic, and semantic representation of the word. PHON gives us the IPA representation, SYN the syntactic category, and SEM the meaning, with \textsc{chicken} standing for the concept of the bird in question.

This, then, is all pretty much standard. The new bits come in with the last line, the COM-SIT representation. That line supplies an additional type of information as part of the lexical entry: an identification of the relevant com-sits for this element. In our case, the com-sit is not restricted by any specific situational features, but is characterised as part of all com-sits indexed as “English" (that is what the “E” stands for). \textit{chicken} is generally used in situations that are associated with “English" as a socially constructed named language.

This illustrates that com-sit representations can involve language indices, and for \textit{chicken}, this is all we need, given that there is no further specialisation for this word. The language index is inherited by the lexical element as a whole, as shown by \textit{chicken}\textbf{\textsuperscript{E}} on the right, representing the lexical item as part of those elements that are indexed for “English". This might look a little bit like the language tags used in the code-switching literature (cf. the discussion in \citealt{MacSwan2017}). Note, though, that in our model, these are not tags that refer to a language as an independent object. Rather, languages themselves are understood as social indices, that is, they are socially constructed in the sense of, e.g., \citet{Silverstein2003}. We will discuss the implications of languages as social indices in more detail in \chapref{chap:5}, in connection with the third lesson to be learned from free-range language.

Com-sit specifications can also involve other kinds of social meaning, such as pointing to particular interlocutors. For instance, English \textit{beddy bye-bye} or German \textit{Wauwau} (lit. “bow-wow”, ‘doggie’) would both be characterised as part of com-sits involving small children. We will encounter such examples in Chapter \chapref{chap:3}, as a basis for the first lesson to be learned from free-range language.

The way I have modelled the com-sit contribution in my example fits into a Tripartite Parallel Architecture, which is close to Construction Grammar. However, the com-sit feature is not wedded to one specific kind of approach to linguistic architecture and grammatical theory, but can be implemented in others as well. For instance in HPSG, com-sits could be captured as part of the social meaning specification suggested by \citet{AsadpourEtAl2022}, who integrate such specifications within a conventional implicature (CI) feature.

Note that in the representation above, at the syntactic level, \textit{chicken} is characterised as a count noun, as in “There is a chicken / there are chickens in the yard.” When you saw this, it might have occurred to you that \textit{chicken} can also be used as a mass noun in English, as in “There is chicken in the soup.” This second usage does not need to be listed in the lexical entry for \textit{chicken}, though, since it reflects a general option, namely the possibility of mass/count coercions that I mentioned above, that is, general patterns that turn count nouns into mass nouns (or vice versa). When used as a mass noun, \textit{chicken} undergoes a ``grinder'' coercion that effects both the syntactic and the semantic level. Semantically, the concept \textsc{chicken} gets enriched by a \textsc{grinder} function that takes an object as its input and yields a substance as its output, namely the edible parts of this object – in our case the chicken meat.\footnote{What counts as edible parts – and, more generally, whether it is ethically defensible to perceive of animals in terms of foodstuff – is a cultural aspect, and the interpretation and availability of such mass-coerced nouns can accordingly differ in different sociocultural groups and settings. For a more detailed discussion and an overview of mass/count coercions and the “Universal Grinder”, cf. \textrm{\citet{PelletierSchubert1989},} \citet{Wiese2012_mass}.} Since this is a general pattern, it should not be listed in the specific entry for \textit{chicken}. Instead, we capture it by its own entry, which is more abstract than that for lexical words:


\begin{figure} [H]
\fbox{
$
\left.
\begin{array}{ll}
\text{SYN:} &  \textnormal{N}\textsubscript{count} \to \textnormal{N}\textsubscript{mass}\\
\text{SEM:} & \textsc{s} \to \textsc{gr(s)}\\
\end{array}
\right\rbrace[\textit{grinder coercion}]
$}
\caption{Entry for grinder coercions}
\label{fig:4}
\end{figure}

This entry captures a general pattern that is not restricted to English,\footnote{This is a slight simplification since the syntactic representation is restricted to languages with a nominal count/mass distinction like that in English: in languages with predominantly transnumeral nouns that do not require this distinction at the syntactic level, the SYN representation would be ‘N → N’ (see \citealt{Wiese2012_mass} for a detailed discussion, including such typologically different languages as Persian, Turkish, or Mandarin).} and involves the \textsc{grinder} concept described above, represented by the function \textsc{gr} that takes the initial semantic representation (\textsc{s}) as its argument. As such, this derivation can be applied to anything that could be conceived as edible, including such unconventional examples as (\ref{ex:1.1}).

\ea{}
\label{ex:1.1}
[…] a mother termite concerned over her child: Johnny is very choosey about his food. He will eat book, but he won’t touch shelf. \citep[136]{Gleason1965}\\
\z

What should go into lexical entries is a restriction when this pattern can \textit{not} be applied because it is blocked lexically, as is the case for such English nouns as \textit{cow} or \textit{pig}, which have ``grinder'' counterparts (\textit{beef}, \textit{pork}) that are different lexical items, based on old French loanwords.

This is not all there is to \textit{chicken}, though. What I especially like about \textit{chicken}, and the reason why I chose this particular item, is that it can also be used in German, as a comparably recent loanword with some interesting features. Here is a photo of the menu at a diner in my neighbourhood in Berlin, where they are offering “Chicken im Brot”, that is, chicken in a sandwich (lit. ‘Chicken in-the bread’):

\begin{figure}[H]
\fbox{\includegraphics[height=2.75cm]{figures/fig5.jpg}}
\caption{\textit{Chicken} as a new loanword in German}
\label{fig:5}
\end{figure}

As a loanword, \textit{Chicken} doesn’t actually get “loaned” or “borrowed”, of course, but more like replicated.\footnote{That the metaphor of “borrowing” is absurd here was already pointed out by \citet{Haugen1950}. \citet[Ch.6.1.]{Matras2020} summarises the terminological discussion and suggests the term “replication”.} This replication involves modification: \textit{Chicken} gets integrated into the new system, picking up local habits, so to speak. This is what we discussed for \textit{Computer} as another English-based loanword in German, and we can visualise such an integration as in \figref{fig:6} (with the \eng{English} context represented at the top in purple, and the \deu{German} at the bottom in green).

\begin{figure}[H]
\fbox{\includegraphics[height=4.5cm]{figures/Wiese_ChickenNetwork.jpg}}
\caption{Loanword integration: \textit{Chicken} gets copied from
\eng{English} to \deu{German}}
\label{fig:6}
\end{figure}

As can be seen in the bottom right image, German has now gained an additional element, and this element gets integrated into the new system, it becomes part of the network in this new domain. We can account for this through a lexical entry for German \textit{Chicken} as in \figref{fig:7}.

\begin{figure}
  \begin{tikzpicture}
  \matrix[matrix of nodes,right delimiter=\},cells={anchor=base west,inner ysep=0pt,font=\strut},inner xsep=0pt, outer xsep=4pt]
  (matrix) 
 	{
      PHON: & \text{/ˈʃɪkən/}\\
      SYN: & \text{N}\textsubscript{mass}\\
      SEM: & \textsc{gr(chicken)}\\
      COM-SIT & $\in$ \textit{diner}\textbf{\textsuperscript{D}}\\
    };
  \node [right=2mm of matrix,font=\strut] (ChickenD) {\textit{Chicken}\textsuperscript{D}};
  \node [above right=2mm and 5mm of ChickenD,font=\strut] (ChickenE) {\textit{chicken}\textsuperscript{E}};
  \draw (ChickenD) -- (ChickenE);
  \draw  (matrix.south west) rectangle (ChickenE.north east);
  \end{tikzpicture}
\caption{Entry for German \textit{Chicken}}
\label{fig:7}
\end{figure}

This entry is characterised as part of elements that are socially constructed as “German” (index “D”), but it still includes a link to the English source, which accounts for the fact that \textit{Chicken} is still recognisable as a loanword from English (which distinguishes it from older English-based loans in German, for instance, \textit{Keks}, which has lost its link to \textit{cakes}). As you will have noticed, there are quite a few differences to the entry for English \textit{chicken}. At the phonological level, we get a representation streamlined to the German phoneme system and syllable structure (avoidance of /tʃ/ in the onset, and vowel reduction to schwa in unstressed secondary syllables). This is what you would expect when an element enters a new system, and that was, after all, why we discussed such patterns as evidence for the reality of grammatical systematicity.

The next lines, though, cannot be motivated by grammatical restrictions of German: nothing forces nouns to be mass here, as evidenced by the large number of nouns that are not. Among them are many that regularly undergo grinder coercions similar to \textit{chicken}, notably including also its German counterparts, namely \textit{Huhn} or (male) \textit{Hähnchen}.\footnote{\textit{Hähnchen} is a (more or less lexicalised) diminutive form of \textit{Hahn} ‘cock’, ‘rooster’. Since nominal gender is determined by the morphological head, that is, in this case the suffix -\textit{chen}, which is neuter, \textit{Hähnchen} is grammatically neuter despite referring to males.} This parallelism is, in fact, the relevant point here – and this is also why I gave you that longish spiel about grinder coercions earlier: we already have a lexical item in German that covers the count noun usage, and with \textit{Chicken}, we now have a new item that is specialised for the corresponding mass usage. Hence, this is a development that is kind of similar to the borrowings from French into Middle English that resulted in English \textit{pork}, \textit{beef}, \textit{mutton}, etc.

\textit{Huhn} or \textit{Hähnchen} have lexical entries that closely correspond to that for English \textit{chicken}, where they are characterised as count nouns that refer to countable entities (viz. chickens). Like English \textit{chicken} (and unlike present-day English \textit{pig}, \textit{cow}, etc.), they can also undergo grinder coercions through the derivation sketched in \figref{fig:4}.

In contrast, German \textit{Chicken} already includes this derivation as part of its lexical entry: it refers to the result of this grinder coercion right away, and it accordingly is a mass noun in its lexical entry. Hence, it can be used for a substance like chicken meat offered in a sandwich, as illustrated above, but not for a countable object, i.e., not for a chicken. For the latter, we must still use the older German terms, \textit{Huhn} or \textit{Hähnchen}. A good illustration for this is the diner menu again: if we look at the broader picture, we see that they use \textit{Chicken} for chicken meat in all kinds of dishes, in a sandwich, as kebab, dürüm, etc., but they refer to half and whole chickens as \textit{Grillhähnchen} ‘roasted / grilled chicken’ (\figref{fig:8}).

\begin{figure}[H]
\fbox{\includegraphics[height=.25\textheight]{figures/fig8.jpg}}
\caption{\textit{Chicken} as a ``grinder''-specialised mass noun in German}
\label{fig:8}
\end{figure}

The obligatory, lexicalised grinder enrichment of German \textit{Chicken} goes together with a specification at the level of com-sits: the use of \textit{Chicken} is restricted to diners, and this is captured in \figref{fig:7} by characterising the COM-SIT representation as part of diner settings that are indexed for German. Thus, you will find the word used in the little diner where I took the photo, but not in fancy restaurants. (I did once spot it in a non-diner context, namely on the menu signboard in Potsdam University’s mensa, but I think this rather underlines the point, given that mensa food is, regrettably, rather close to the food in diners, and not to that in fancy restaurants.\footnote{This said, note that such COM-SIT entries are not set in stone, but can change, reflecting the dynamic character of com-sits (see the discussion of com-sits in \ref{sec:1.2} above).})

The reason for this specification might be an association with US-American fast food chains in Germany, which probably represent the com-sits that served as a source for the borrowing of \textit{Chicken} in the first place (think \textit{chicken burger}, \textit{chicken nuggets}, etc.). Hence, we might want to further specify the link to the English source that is included as part of the lexical entry for \textit{Chicken} in \figref{fig:7}: in order to be more precise, we could add the grinder specialisation to \textit{chicken}\textbf{\textsuperscript{E}}, plus a COM-SIT characterisation as “US-American fast food chain".

Such a cross-linguistic link is not only a pointer to a source of borrowing that is still active in current use. It can also be a basis for higher-order indexicality involving stereotypes of certain settings, speech communities, languages, etc. For instance, in the case of American English, these higher-order indices can associate lexical elements with such concepts as “globalisation", “urbanity", or “coolness".

What these associations are, depends on the cultural context, and this can be influenced by historical and political contingencies. A good example for this comes from a recent lexical development in Albanian. As \citet{Jusufi2022} shows, there are a number of new loanwords from German, for instance \textit{luft} from German \textit{Luft} ‘air’, or \textit{blicbllank} from German \textit{blitzblank} ‘spick and span’, which are used systematically, but have not made it into dictionaries yet and do not underly norming or standardisation.

Now, if you were surprised by these loans, asking yourself whether Albanian didn’t have words for air or cleanliness already and wondering why they would feel the need to import this from German, you would be perfectly justified, of course. The explanation is that these loan words are more specialised in Albanian than their German sources are, similar to the specialisation of German \textit{Chicken} compared to its English source. Albanian \textit{luft} does not refer to just any old air, but only to that in car tyres, and \textit{blicbllank} is what you call a car polish that is shiny clean, but not, say, your kitchen table – even though in German, \textit{Luft} is just air, and a kitchen table can be described as \textit{blitzblank} (I wish ours could …). Hence, while in Albanian the core semantics of these elements is still \textsc{air} and \textsc{shiny} \textsc{clean}, they have gained additional meaning components. This is based on the specific com-sits in which they occur: the setting of automobile repairs.

The background for this development is that there has lately been substantial migration to Germany and back to Albania by speakers who work with automobile technology, bringing German technical terms with them. This has made German what \citet{Jusufi2022} calls a “Lingua Tecnica”, a modern lingua franca for technology (except digital technology, which is associated with English), and in particular for car-repair settings. The specific com-sits for new loans from German have supported a higher-order indexicality, with associations of “(non-digital) technology” for the language.

\largerpage
Such examples show that linguistically diverse settings can throw  a spotlight onto the dynamics of com-sits and language indices. These  kinds of settings are less subject to monoglossic constraints and will be our prime source for the three lessons on grammatical systems without language borders to be discussed in \chapref{chap:3} through \chapref{chap:5}. Before we go there, let me briefly explain why I understand language use in such contexts as “free-range” language, and what kinds of contexts support this.
