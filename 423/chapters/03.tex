\chapter{What is the point of “free-range” language?}\label{chap:2}

\section{What is free-range language?}
\label{sec:2.1}

Sociolinguistic perspectives on linguistic fluidity and multi-competence are drawing on linguistically and socially diverse settings. Such linguistic fluidity is particularly evident in settings that are less restricted by monolingual hegemonies and standard language ideologies. These kinds of monoglossic ideologies might dominate the societal macro context, but they will exert less of their power in some settings. As a result, what counts as appropriate language use will be more open to variation. Speakers will feel more at ease using nonstandard forms and accessing a broader range of options from their linguistic repertoires.

Take Germany as an example. At the societal macro context, we find a strong monolingual habitus \citep{Gogolin1994} and a monolingual bias (\citealt{Kachru1994}; \citealt{Cook1997}) towards German, making this a highly dominant majority language (\citealt{Gogolin1994}; \citealt{Fuller2012}; \citealt{Wiese2015}). In public life, be it when you go shopping, talk to your kid’s teacher, ask a stranger for the way, ring up a plumber, apply for a new passport, or go to your local bank, it is totally acceptable, and in fact strongly expected, to behave as if German was the only language there is. This can go so far that at a university where I used to work, the administration insisted on sending German-language emails to two colleagues in the US who had kindly agreed to serve as external examiners for a PhD defense, causing me no small embarrassment, since these colleagues did not speak German and were doing the committee work as a favour, without any reimbursement – two issues that the administration was well aware of.

This might be an extreme case, but it illustrates how strong this monolingual habitus is: people routinely act as if everyone is monolingually German, all other languages getting basically erased. What is more, this monolingual habitus is complemented by a strong standard language ideology (\citealt{Durrell1999}; \citealt{DaviesLanger2006}; \citealt{Davies2012}) which constructs only a certain hegemonic variant (oriented at middle-class language use) as correct and proper German, with the effect that linguistic choices become even more limited, especially in more formal settings.

\hspace*{-1.2pt}However, despite this overwhelming monoglossic dominance, at the meso level of local settings we find a number of instances where these ideologies exert less of their power. For instance, at many street markets, you will find yourself in the middle of flamboyant linguistic diversity, with sellers and customers using a lively mixture of languages and dialects. Young people in inner-city neighbourhoods can draw on a large range of different heritage languages and make good use of this in their peer-group interactions, which has, among other things, led to an interesting new dialect, Kiezdeutsch. The urban hipster scene supports cafés where staff might approach you in English rather than German (to the chagrin and outspoken indignation of some conservative politicians, who complain about this in angry letters to local newspapers). And if you could see the messages that my daughters exchange via digital platforms with their friends, you would certainly not associate this with monolingual standard German.

In these settings, speakers obviously feel less constrained in their choices than the societal macro context would lead us to expect. Challenging societal mono\-glossic restrictions, such settings constitute linguistic \textit{hétérotopies} in the sense of \citet{Foucault1967}, that is,

\begin{quote}
counter-sites, a kind of effectively enacted utopia in which the real sites, all the other real sites that can be found within the culture, are simultaneously represented, contested, and inverted. (\cite{Foucault1967} / 1986: 24)
\end{quote}

As a result, these settings permit more (socio-)linguistic variation, supporting a special dynamics both at levels of linguistic practices and choices and of linguistic structure. This makes them particularly suited to contribute to our understanding of fluid registers and their relation to grammatical systems.

I understand language in such settings as “free-range”, using a metaphor from organic farming: just as chickens are not meant to be cooped up in tiny cages, I believe that the monolingual and standard-language confines going back to 19\textsuperscript{th} century European nation-state building are not a suitable setting for language. If we want to learn about characteristic patterns of chicken behaviour, we could do better than looking at factory hens.\footnote{Note that “free-range” chickens aren’t truly free, but still part of animal husbandry. I use the metaphor here to signal fewer restrictions (not necessarily no restrictions, see below), similar to other popular uses of “free-range”, for instance in the “free-range children” / “free-range parenting” movement.} Along the same lines, if we want to study language and its dynamics, it would be more promising to look at settings that are less restricted by policing and censoring along “purist" lines oriented towards an imagined monolingual standard variety.

Note that the metaphor of “free-range language” does not mean to imply a view of language as an object: just as it is not the eggs, but the chickens who are actually free-range, it is not language, but the speakers who can express themselves freer. “Free-range language” in this sense is to be understood a bit like “open air sports”, that is, as a rule-based activity, not a fixed object.

We might recognise free-range language settings as more natural, and this is what is often assumed for linguistic practices understood as trans- or polylanguaging that we see there (e.g., \citealt{AgJørgensen2013} on polylanguaging as a more natural state). However, whether such practices actually feel more natural for speakers depends on their linguistic biography and the society they grew up in. And as \citet{Jaspers2019} points out, it can also be reasonable and positive to restrict this free range, depending on a community's goals, for instance, if speakers want to preserve a minority language.

So “free-range” does not necessarily mean more natural in any given cultural context. What it does mean is that there is less power of the kinds of monoglossic hegemonies found in much of Europe and the countries impacted by European colonialism (including former European settler colonies, such as the US or Australia). Since these monoglossic ideologies are a very specific and historically recent phenomenon,\footnote{As \citet{Pavlenko2023} points out, even in European nation-states, the ideological link between one language and one nation for a long time contrasted with the multilingual societal reality, and it was not before the 20\textsuperscript{th} century with its large-scale “linguistic and ethnic unmixing” \citep[34]{Pavlenko2023} that the linguistic reality on the ground became more monolingual – thus making the multilingual urban diversity we presently see in Europe seem special, when it really only brings back some normalcy.} it might be misleading to restrict ourselves to language use dominated by them. In contrast, free-range settings might give us a better idea of how language ordinarily works, outside such historical idiosyncrasies.

This is why I focus on free-range language in this book and examine what insights might be gained from this. I am going to look at such settings from the point of view of com-sits, exploring what this perspective might contribute to our understanding of language use and linguistic systems, and the integration of grammatical structure and linguistic fluidity.

In the remainder of this chapter, I describe four characteristic kinds of settings for free-range language that will inform our account: urban markets, heritage language settings, multiethnic adolescent peer-groups, and digital social media. As my examples above illustrated, language use in these kinds of settings is more free-range in the sense of being less dominated by monolingual and standard language ideologies. At the same time, the settings complement each other in illustrating different aspects of linguistic diversity and fluidity and thus give us different handles on free-range language: markets constitute linguistically diverse spaces characterised by trade encounters that often take place between strangers; heritage languages draw on the intimate settings in bilingual families; multiethnic adolescent peer-groups are part of urban youth culture; and language use in social media is, unlike the other examples, mostly written rather than spoken, but still part of informal communication outside standard language confines.

Note that these settings are not isolated from their societal macro context and its language-ideological hegemonies, and the linguistic diversity and openness that characterises them will be in contrast to – and affected by – widespread monoglossic ideologies. This is what makes them counter-sites in the sense discussed above: they are special in allowing more freedom from monoglossic restrictions, but they do so within societies that can still be very much governed by these. Again, the “free-range” metaphor can shed light on this: free-range chickens are not living in a natural state outside animal husbandry, but they are living a more natural life than factory chickens. Along the same lines, free-range language settings are not altogether free of the language-ideological restrictions dominating their societal macro context, but the impact of such hegemonies will be lessened in these multilingual contexts.

This distinguishes the free-range language settings that I focus on in this book from settings of multilingualism in pre-industrial societies, so-called “small-scale multilingualism”: multilingual settings mostly in the Global South, “in areas of the globe that have been spared from Western settlement colonies” \citep[35]{Lüpke2016} and are not impacted by hierachical relationships between named languages. These settings are interesting in showing patterns of egalitarian and balanced multilingualism. Treating such setting in depth would go outside the scope of this book, but I will give pointers to multilingual patterns similar to our examples of free-range in some places.\footnote{For an overview of small-scale multilingualism settings see \citet{Lüpke2016}, \citet{PakendorfEtAl2021} and contributions to \citet{DobrushinaEtal2021} (eds.).}

\section{Free-range language settings}
\label{bkm:Ref82075181}\hypertarget{Toc125444652}{}
\subsection{Urban markets}
\hypertarget{Toc125444653}{}
Markets are magnets for diversity; they have always been places where people from different social and linguistic backgrounds come together, transcending the boundaries of socially constructed “languages" and “ethnicities". This has made them a particularly suitable setting for research into linguistic fluidity, multi-competence and metrolinguism: urban markets are a hotbed of linguistic and social mixing and integration, and this is particularly true for urban street markets, with their more informal character (\citealt{HiebertEtal2015}; \citealt{PennycookOtsuji2015,PennycookOtsuji2019}; \citealt{Adami2018}).

When I first learned about the exciting findings coming from such research, I immediately thought of the Maybachufermarkt, a street market where we often go for grocery shopping. The Maybachufermarkt is an open-air market at the border of Berlin-Neukölln and -Kreuzberg, two inner-city neighbourhoods where people routinely engage with a large variety of linguistic resources in their daily life. This includes German as the societally dominant majority language, but also a diverse range of heritage languages, with Turkish, Arabic, and Kurdish as the most salient but by no means the only ones, plus English as the language of globalisation and tourism.

Set in this context, the Maybachufermarkt is characterised by a great linguistic openness: people liberally mix and match linguistic elements in communication, and the market’s linguistic diversity has become something of a selling point (cf. also \citealt{Heller2010} on the commodification of language). Customers include people from the neighbourhood who do their daily shopping there, but also Berliners from other parts of the city and a fair amount of tourists, and people who work at the stalls express a multilingual pride that somewhat counteracts the strong monolingual habitus that is dominant at the macro level of German society \citep{Wiese2020_contact}. This is also evident in the market’s linguistic landscape (cf. \cite{DumanÇakır2023}):\ while German, as the societal majority language, is still highly visible, it is complemented by a range of other languages, as illustrated by this sign that offers “men’s shirts” in German, Turkish, and Arabic (\figref{fig:9}).\footnote{\textit{Stück} etc. (`piece') is used as a classifier here; I will analyse this in \sectref{bkm:Ref119923192}.}

\begin{figure}[H]
\fbox{\includegraphics[height=3.75cm]{figures/fig9.jpg}}~~\parbox[b]{3cm}{\textit{English translation:}\\
                                                   \footnotesize Men's shirts\\[1.5em]
                                                   fine/heavy rib\\
                                                   100\% cotton\\[.33em]
                                                   3 piece\\
                                                   5,-€/Euro\\[.33em]
                                                   4 piece\\
                                                   6,-€/Euro} 
\caption{\label{fig:9} Multilingual sign at a market stall}
\end{figure}

Situated at the waterways of the Landwehrkanal, the market initially started, in the late 19\textsuperscript{th} century, as a farmers’ market for produce from the Spreewald area in Brandenburg, South of Berlin. After the division of Germany in the mid-20\textsuperscript{th} century, West Berlin including Neukölln and Kreuzberg was cut off from the surrounding countryside, which put an end to the market. However, it managed to reinvent itself in the 1970s thanks to new Berliners who had immigrated from Turkey as part of the so-called “Gastarbeiter" generation and started what soon became known as the “Türkenmarkt” (‘Turks’ market’), offering produce from Turkey and South Europe. Today, this is still a dominant part of the market, but by no means the only one, and there are also stalls selling, for instance, organic farmers’ produce from Brandenburg and Poland, Greek delicatessen, Ghanaian street food, incense sticks and New Age candles, haberdashery, or children’s clothing.

This makes the market an exciting place to investigate linguistic diversity, and this is what we did in a project that I ran together with several collaborators as part of a larger research cluster.\footnote{\textit{Integration of linguistic resources in highly diverse urban setting}s; project A01 of {CRC 1278}; see acknowledgements.} In particular, we set out to see whether despite the large linguistic variability we observe at the market, there might still be restrictions pointing to linguistic systems. As a basis for our investigation, we recorded sales interactions at different market stalls over the course of several months, conducted interviews with sellers and customers, ran focus group discussions with cooperating sellers on different (socio\nobreakdash-)linguistic patterns we had observed in the spontaneous data, collected pictures of all stall signs to capture the linguistic landscape of the market, and one of us, İrem Duman Çakır, conducted an ethnographic study where she worked at one of the stalls for several months.

Here is a short transcript from a sales interaction at a mokka stall that illustrates the kind of integration of diverse linguistic resources that we found to be characteristic for the market (cf. \cite{YükselDuman2021}). The interaction involves a customer (C), the seller at the mokka stall (S1), and her colleague from a neighbouring stall (S2). The customer is a tourist from Israel visiting Berlin; seller S1 was born in Turkey and came to Germany in 1993, at the age of 14; her colleague S2 was also born in Turkey, where he grew up in the Southeast with many Arabic-speaking friends; he came to Germany in 1998 at the age of 34.

\renewcommand{\exfont}{\itshape}
\resizebox{.95\textwidth}{!}{
\parbox{\textwidth}{
\ea
\examplesitalics
\label{bkm:Ref125444555}Sales interaction at the Maybachufermarkt\\
\textup{C:}   \eng{How} \eng{much} \eng{is} \eng{it}?\\

\textup{S1:} \gll Ehm, \deu{sechs} \deu{fünfzig}. \eng{Six} \eng{Euro} \eng{forty} cent.\\
         ahm  six      fifty\\

\textup{C:}  \eng{Six Euro, six Euro} …?\\

\textup{S1:} \gll \deu{Sechs} Euro   \eng{fifty} cent, \deu{sechs} \deu{fünfzig}. \ita{Italiano}?\\
     six    ~      ~    ~     six     fifty       Italian?\\

\textup{C:} \gll   \deu{Nein}, \deu{Israel}.\\
          no    Israel\\

          \textup{S1:} \gll \deu{Was}?\\
       what\\

\textup{C:}  \deu{Israel}.\\

\textup{S1:} \gll \tur{İsrail}   ehm [name] \tur{abi}      \\
         Israel  ahm  S2      brother \\

\hphantom{\textup{S1:}} \gll \tur{Arapçada} \tur{altı}  \tur{elli}  \tur{neydi}?\\
    in\_Arabic six   fifty what\_is?\\

\hphantom{\textup{S1:}} \gll  \tur{Arapçada}   \tur{altı} \tur{elli}   \tur{ne}?\\
     in\_Arabic   six   fifty what?\\

\textup{S2:} \glll \ara{{\db}\textarab{ستة}}   Euro  \ara{{\db}\textarab{نص}}\\
              [sitːe]     ~      [nu​sˁ]\\
           six    Euro          a\_half\\

\textup{S1:} \gll \ara{\textarab{ستة}}   Euro  \ara{\textarab{نص}}\\
             six    Euro  a\_half\\

\textup{C:}   [\textit{laughs}]\\

\textup{S1:} \gll \deu{Auch} \deu{nicht}?\\
          also  not?\\

\textup{C:} \gll  \deu{Wir} \deu{haben} \deu{nicht} \deu{Arabisch}.\\
         we  have   not    Arabic\\

\textup{S1:} \gll \deu{Nicht} \deu{Arabisch}?\\
        not    Arabic?\\

\textup{C:} \gll  \deu{Hebräisch}.\\
         Hebrew\\

\textup{S1:} \gll \deu{Hebräisch}, \deu{ah}, \deu{noch} \deu{schlimmer}. \deu{Das} \deu{könnwa} \deu{nicht}.\\
         Hebrew     ah   even  worse       that  can.we    not\\

\textup{C:} \glll \deu{Hebräisch} \deu{ist} eh    \heb{\texthebrew{וחצי}} \heb{\texthebrew{שש}} \\
       ~         ~   ~      {[waχet͡si} {ʃeʃ]}\\
        Hebrew    is   ah  {(six a\_half)}\\

\textup{S1:} \gll \heb{\texthebrew{וחצי}} \heb{\texthebrew{שש}} \\
     {(six a\_half)}\\
\z
}
}

\renewcommand{\exfont}{\upshape}
\examplesroman

I am sure this short segment will already have shown you why I love this market. It illustrates how sellers and customers put all their linguistic resources into service in order to make communication happen. In this endeavour, the language hierarchies of the larger society are suspended: this is not about everyone trying to speak German as the majority language, but aligning with each other and learning from each other (note how S1 repeats the Hebrew phrase, presumably committing it to memory for possible future use). This is hence in contrast to the linguistic hegemonies of German society, and more like the egalitarian multilingualism known from settings of small-scale multilingualism.

In the transcript, I have colour-coded elements according to different “languages" here, with purple and green used for \eng{English} and \deu{German} again, plus pink for \ita{Italian}, red for \tur{Turkish}, brown for \ara{Arabic}, and blue for \heb{Hebrew}. Note, though, that speakers show little or no concern about separating and isolating elements along the lines of such “languages". In this setting, languages are not used to construct borders, but to probe into each others’ linguistic multi-competences. Speakers use language labels in order to find out what commonalities the communication can build on, and these commonalities are expanded as the communication unfolds and speakers learn from each other, adding to their linguistic resources as they go.

This kind of language use, I think, is free-range at its best, and one could write a whole paper on this segment alone. In the current book, I am going to look at such language use from the point of view of com-sits and see what the patterns we observe can tell us about grammatical systems and named languages in the face of linguistic fluidity that transcends language borders.

\subsection{Heritage language settings}
\hypertarget{Toc125444654}{}
Heritage language settings are characterised by multilingualism in the family  and eth\-nic\-ity--language ties that distinguish them from the larger societal context. Heritage speakers grow up with an additional family language that is typically associated with immigration experiences in an earlier generation and that is not the majority language of the larger society.\footnote{See, e.g., \citet{MontrulPolinsky2019}. Note that since there is no clear linguistic cut between a “dialect" and a “language", this can in principle also include speakers of different dialects brought into a new setting in the course of immigration. For our present purpose, I will concentrate on heritage speakers of societally constructed “languages".} Since languages in heritage settings face less overt policing than when they serve as national, majority languages, heritage language settings participate in the dynamics of free-range language and will thus inform our approach on grammatical systems and com-sits.

Two examples that I will use in this book are Turkish as a heritage language in Germany, and German as a heritage language in Namibia. In the first case, the heritage language is spoken in a society characterised by a monolingual habitus and, accordingly, a dominant majority language (German), as described above. This is the kind of macro setting that has usually been targeted in heritage language research so far, reflecting a research bias in our discipline that renders such monolingual-habitus societies as the US or Germany as the norm.\footnote{For a critical discussion see \citet{WieseEtal2022_continuum, KerswillWiese2022}.}

Given this macro context, Turkish is mostly restricted to informal communication, especially within the family. However, Turkish is one of the largest heritage languages in Germany, and there is a vital heritage community supporting it, and as a result in some neighbourhoods one can also hear Turkish in less intimate settings, e.g., in shops, in the schoolyard, and – as illustrated in the previous section – at some urban markets. Under conditions of a monolingual societal habitus, this deviation from an expected norm is then particularly salient – so much so that one sometimes finds claims (in the public discussion, but also in some linguistic publications) to the effect that children growing up in such neighbourhoods do not encounter any German before they attend school.

This seems to be a case of perceptual erasure, though \citep{WieseEtal2022_multilinguals}: a closer look at the actual linguistic practices shows that German is, in fact, widespread. For instance, in my neighbourhood in Berlin-Kreuzberg, which is nicknamed “Little Istanbul”, referring to its large Turkish-heritage community, German is a common language not only in shops, in cafés and in the street, but also on playgrounds and in heritage-Turkish families, where it typically becomes dominant among children as soon as they attend kindergarten (which most do by the age of 3). The societal monolingual habitus is so strong that you could not raise a young child without any exposure to German even if you tried (and why would you?). At the same time, it means that multilingual practices are highly salient (and get perceptually overrepresented) when they occur, and the widespread use of German gets perceptually erased.

The second example I will focus on differs from this in an interesting way: Namibian German is a heritage language that is integrated in a setting of societal multilingualism. In Namibia, English is defined as the only “official language", but in addition there are 13 recognised “national languages", and many more that are used in Namibian society (\citealt{WieseEtal2017}; \citealt{ShahZappen-Thomson2018}; \citealt{Zimmer2021}). Other than, e.g., Germany and the US, and similar to most African countries, Namibia embraces its societal multilingualism and is much more open to linguistic diversity. Multilingual practices, including code-switching and language mixing, are accepted as a normal part of everyday life.

German was introduced into Namibia in the course of colonialism when what was then called “German South-West Africa” (1884-1915) was intended as a settlement colony of the German Empire. German was the language of colonial administration and German settlers. As such, it was associated with the colonial seizing of land and colonial crimes including a genocide against the local populations of Herero and Nama.\footnote{It took over 100 years, until 2021, before the German state formally acknowledged this genocide, see \url{https://www.auswaertiges-amt.de/de/newsroom/-/2463396} for the Foreign Office’s statement [last accessed June 20\textsuperscript{th}, 2023].} When Germany lost its colonies after WW1, a lot of the German settler population remained in Namibia, providing the basis for a German-speaking community.

Today, German is the main household language for 1\% of Namibia’s population, with a heritage language community of about 20,000 speakers. Given the multilingual societal habitus, for one, German is not restricted to informal contexts, but is also used in German-language schools, media, churches, and clubs. Secondly, German heritage speakers are generally at least trilingual and regularly use Afrikaans and English besides German in their daily lives. Afrikaans used to be the official language during the South-African Mandate over Namibia until independence in 1990, and is still a common lingua franca in interethnic communication. English, as the official language since independence, is considered ethnically neutral and somewhat associated with education. In addition to these two widespread languages, some heritage-German speakers also have some competences in other Namibian languages such as Herero, Nama/Damara, or Oshivambo.\footnote{Although this is much less common given the societal division along “racial" lines, which is still strong even three decades after the end of Apartheid. On the positive side, the interest in learning these languages seems to be greater in the younger generation, which might indicate a positive future trend.}

Heritage languages, in particular in monoglossic societal macro contexts, will often be unfavourably compared to their national counterparts in the sending countries of the first, immigrant generation. Monolingual bias and standard language ideologies can make characteristics of heritage language use seem deficient, e.g., Turkish in Germany compared to that in Turkey. In the past, heritage language research has often approached them from a deficit perspective, with assumptions of “errors”, “attrition”, and “incomplete acquisition” compared to a native speaker model that was restricted to monolinguals and often implied standard language norms (see criticism in, e.g., \citealt{RothmanTreffers-Daller2014}; \citealt{Flores2017}; \citealt{WieseEtal2022_continuum}).

This is changing, though, and recent findings emphasise that heritage languages are especially suited to contribute to our understanding of language variation and change (e.g., \citealt{Wiese2013}; \citealt{YagerEtal2015}; \citealt{Boas2016}; \citealt{KupischPolinsky2022}; \citealt{WieseEtal2022_continuum}). In line with this, in this book I will approach them as an example of free-range language that can shed a light on the dynamics of grammatical systems and sociolinguistic alignments.

\subsection{Multiethnic adolescent peer-groups}
\hypertarget{Toc125444655}{}
One of the things I like so much about my neighbourhood in Berlin is its great linguistic and social diversity. Kreuzberg is an old inner-city working-class neighbourhood and a part of West Berlin that found itself blocked off on three sides by the Berlin wall in 1961. This made it less attractive economically, but the low rents attracted an influx of people both from West Germany and from abroad: artists and political activists who were looking for an alternative lifestyle as well as immigrants of the so-called “Gastarbeiter” generation who came to work in Germany. This has made for a lively mix of people and their cultural and linguistic resources, and today Kreuzberg is renowned for its vibrant atmosphere of diversity (which, at the time of writing, it has still managed to sustain – despite the challenges of gentrification it has been facing ever since the wall came down and it suddenly found itself in the centre of a reunited Berlin rather than at the outer fringes of West Berlin).

Urban neighbourhoods like this bring together a range of dialects and heritage languages whose roots go back to the immigration of earlier generations. Young people who grow up in such neighbourhoods find themselves as part of a new generation for whom such diversity is a normal part of daily life. Kreuzberg adolescents, whether they acquired an additional heritage language in their family or not, routinely access a much broader range of linguistic resources than most of their parents or grandparents will have done when growing up, and there is a fair amount of local pride in this. Here is a photo I took of a playground wall that illustrates this, with the phrase “our playground” in German, Turkish, and English, above a large graffiti %%[Warning: Draw object ignored]
saying “Kreuzberg”:

\begin{figure}[H]
\fbox{\includegraphics[height=.17\textheight]{figures/fig10.jpg}}
\caption{\label{fig:10} Multilingual wall at a Kreuzberg playground}
\end{figure}

The new generation of young people who grow up in such neighbourhoods systematically transcends linguistic and ethnic boundaries.\footnote{Note that ethnicity is to be understood as a social category (\citealt{Wiese2022}).} In such multiethnic adolescent groups, we find new ways of speaking: new urban vernaculars (see \citealt{Rampton2010}) that are used in peer-group situations. I have characterised these vernaculars as “urban contact dialects” (\cite{Wiese2013,Wiese2022}), bringing together perspectives of variety and style that target structural patterns and sociolinguistic choices, respectively (see also \citealt{Quist2008} for such an integration).

I first encountered such a way of speaking when sitting on a bus in my neighbourhood and overhearing young people talking to each other across rows. They spoke mostly German, but integrated a number of new loanwords, and I also noticed some unusual grammatical patterns, and I immediately became hooked. This led to a number of research projects, where we described this new way of speaking as “Kiezdeutsch”, a term that some speakers had used in an interview (as mentioned in the Introduction).

Several of the examples of free-range language in multiethnic adolescent peer groups will come from Kiezdeutsch, but this is, of course, not the only such urban contact dialect. Well-known other examples are \textit{Multicultural London English} in the UK, \textit{Sheng} in Kenya, or \textit{Camfranglais} in Cameroon.

The earliest accounts of such varieties in urban Europe came from Scandinavia, through the pioneering work of Ulla-Britt \citet{Kotsinas1988} in Stockholm and, based on this, Pia \citet{Quist2000} in Copenhagen. This was followed by a range of research projects, in particular in North-West Europe, including the UK, Norway, the Netherlands, France, Germany, and others.\footnote{For overviews see \citet{CheshireEtal2015}, contributions in Kern \& Selting (eds.) (\citeyear{KernSelting2011}), Quist \& Svendsen (eds.) (\citeyear{QuistSvendsen2010}) , Nortier \& Svendsen (eds.) (\citeyear{NortierSvendsen2015}),
Kerswill \& Wiese (eds.) (\citeyear{KerswillWieseEds2022}).} Comparable urban contact dialects have also been described for a range of countries in Sub-Saharan Africa, including Senegal, South Africa, Cameroon, DR Congo, Ghana, Kenya, and others.\footnote{For overviews see \citet{KießlingMous2004}, contributions in Nassenstein \& Hollington (eds.) (\citeyear{NassensteinHollington2015}), Mensah (ed.) (\citeyear{Mensah2016}), Mesthrie et al. (eds.) (\citeyear{MesthrieEtal2021}), Kerswill \& Wiese (eds.) (\citeyear{KerswillWieseEds2022}).}

Early accounts of such vernaculars in Europe described their grammatical characteristics in terms of errors, simplification and reduction compared to standard language,\footnote{See \citet{Wiese2009} for a critique.} but this picture has changed, and it is now evident that these urban dialects can contribute to our understanding of linguistic variation and speakers’ options.\footnote{For an overview see contributions in Kerswill \& Wiese (eds.) (\citeyear{KerswillWieseEds2022}).} Let me illustrate this for one of my favourite structural phenomena, namely word order variation in the left periphery of Kiezdeutsch (and similarly in urban contact dialects based on some other Germanic languages).

In addition to the conventional German verb-second order in main declaratives, Kiezdeutsch allows patterns as in (\ref{bkm:Ref112749104}), where the finite verb is preceded by an adverbial and a subject:


\ea\label{ex:key:3}\label{bkm:Ref112749104}\label{bkm:Ref112750051}Kiezdeutsch (KiDKo, MuH9WT)\footnote{Corpus data from KiDKo, see \url{www.kiezdeutschkorpus.de}. Capitalisation
indicates main stress.}
\begin{quote}
\gll \textit{danach} \textit{sie} \textit{hat} \textit{misch} \textit{AUCH} \textit{geblockt} \\
afterwards she has me also blocked  \\
\trans “After that, she blocked me, too [in a social network].”
\end{quote}
\z



In earlier accounts, such patterns were described as a replacement of the German XV\textsubscript{fin} order with SVO (e.g. \citealt{Auer2003}), and \citet[37]{Auer2013} claimed that this “intervenes deeply in the structures of autochthonous German in its standard and nonstandard forms" [German original, my translation, H.W.].

Closer analysis showed, though, that what we see here is not so much a somewhat “allochthonous" restructuring to SVO, but rather a variation on verb-second that fits well into German: a verb-third pattern that follows the general outline of German sentences and keeps the characteristic German verb bracket intact (hence, it is `has me blocked' rather than `has blocked me' in (\ref{bkm:Ref112750051}) above), motivated by information-structural preferences.\footnote{\citet{Wiese2009,Wiese2011,Wiese2013}.} Not surprisingly, then, this pattern has subsequently also been found in language use outside Kiezdeutsch.\footnote{\citet{WieseMüller2018}, \citet{Bunk2020}.}

Rather than bringing an alien element into German, such patterns put a spotlight on the actual range of variation within presumed “strict verb-second" languages, and can thus inform syntactic theory\footnote{ \citet{teVelde2017}, \citet{Walkden2017}, \citet{Bunk2020}.} and our understanding of the interface between syntax and information structure.\footnote{\citet{Wiese2011}, \citet{FreywaldEtal2015}, \citet{WieseEtal2017,WieseEtal2020}.}

This, then, underlines what makes free-range language so useful for us: it highlights the range of possibilities – at structural as well as sociolinguistic levels – and prevents us from mistaking a specific condition of language, namely language use under monoglossic constraints, as the normal case and/or an exhaustive picture.

\subsection{Digital social media}
\hypertarget{Toc125444656}{}
Digital social media are a locus of informal writing outside codified norms (e.g., \citealt{AndroutsopoulosBusch2020}). The internet has become an important site for maintaining and managing social relationships \citep{McCulloch2019}, and this has made digital media a rapidly evolving and particularly fertile ground for the development of new communicative patterns. These patterns integrate the dynamics of informal language into the written mode, including such new graphic elements as emoji (e.g., \citealt{DainasHerring2021}).

Unlike the other three settings for free-range language, digital social media does not necessarily involve speakers with a multilingual family background. However, the fact that constraints of standard language and codified orthographic rules are substantially loosened means that speakers also use a broader range of linguistic options and resources. In today’s globalised and interconnected world, this typically does not only mean more linguistic variation per se, but it also includes crossing traditional language borders. Have a look at the following examples, taken from WhatsApp messages by young people in Germany:\footnote{From the RUEG Corpus, see \url{https://hu.berlin/rueg-corpus}. “\textsc{mp}” = modal particle}

\ea\label{ex:key:4}\label{bkm:Ref113003596}Adolescents’ messenger communications in Germany\\
\ea

\gll da     ist  ja  dieser parkplatz,   you know? \textup{[DEmo53FD]}\\
     there is  \textsc{mp} this    parking\_lot  you know\\
\glt `There’s this parking lot, you know?'
\ex
\gll lol  bis     gleich   {\db}\includegraphics[height=1em]{figures/cowboy.png}\includegraphics[height=1em]{figures/cowboy.png}       \textup{[DEmo84FD]}\\
     lol  until  soon    {[\textsc{cowboy} \textsc{hat} \textsc{face} \textsc{emoji,} 2x]}\\

\glt `lol see you soon.'
\z % you might need an extra \z if this is the last of several subexamples
\z

The speakers – or rather writers – of these messages come from monolingual family backgrounds, that is, they have not grown up with an additional heritage language, but generally used only German at home. Nevertheless, they routinely integrate English elements into their messages, and we often find loans from other languages as well. Some of these are used internationally, transcending linguistic borders, and I will characterise them as “translinguistic” elements below (\chapref{chap:4}). This also includes such graphic markers as the emoji in the second example, which are not associated with a specific language to begin with. In the example above, they can be analysed as graphic discourse markers, and such occurrences can broaden our understanding of the expression of textual and (inter-)subjective discourse relations (cf. \citealt{WieseLabrenz2021}).

My initial interest in such digital messaging was much less ambitious: as part of a larger research unit (the RUEG group), we wanted to elicit register-differ\-en\-tiated data in formal and informal situations and cover language use in both the spoken and the written modality, and we needed something to fill the “informal-written" slot. Once we looked at the data we got there, though, it quickly became obvious that such language use is fascinating in its own right, and in particular as an example of free-range language that involves graphic as well as verbal patterns.

In the following three chapters (\chapref{chap:3} -- \chapref{chap:5}), I will now discuss three main lessons that can be learned from investigating such free-range settings, for our understanding of grammatical systems and their foundation in view of linguistic fluidity and the status of “languages" as social constructs.
