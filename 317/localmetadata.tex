\title{Subatomic quantification}
% \subtitle{Add subtitle here if exists}
\author{Marcin Wągiel}
\renewcommand{\lsSeries}{osl}%use series acronym in lower case
\renewcommand{\lsSeriesNumber}{6}

\renewcommand{\lsID}{317}
\dedication{Dla Mamy i Taty}
\BackBody{The goal of this book is to explore the relationship between the cognitive notion of \textsc{parthood} and various grammatical devices expressing this concept in natural language. The monograph aims to investigate syntactic constructions and lexical categories, e.g., partitives, whole-adjectives, and multipliers, encoding different kinds of part-whole structures both in Slavic and non-Slavic languages. It is envisioned to inspire radical rethinking of the ontology of models accounting for nominal semantics. Specifically, it provides novel evidence for a mereotopological approach to meaning, i.e., a theory of wholes that captures not only parthood but also topological relations holding between parts. This evidence comes from the phenomenon of \textsc{subatomic quantification}, i.e., quantification over parts of referents of concrete count nouns.}

\typesetter{Marcin Wągiel}
\proofreader{Berit Gehrke, Nina Haslinger, Adam Przepiórkowski, Viola Schmitt, Peter Sutton}
\renewcommand{\lsImpressumExtra}{GAČR grant number GA17-16111S\\Reviewers: Adam Przepiórkowski, Peter Sutton}

\ISBNdigital{978-3-96110-315-7}
\ISBNhardcover{978-3-98554-011-2}
\BookDOI{10.5281/zenodo.5106382}
