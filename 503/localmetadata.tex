\title{Discourse structure and narration}
\subtitle{A diachronic view from Germanic}
% \subtitle{Add subtitle here if it exists}
\BackBody{The volume Discourse structure and narration: A diachronic view from Germanic deals with questions of information structuring at discourse level, focusing on narrative discourses. More precisely, it is about the contribution of grammatical devices to the organization of texts as well as their diagnostic potential for the narrative text type. Although it is well-known that information packaging had a much greater impact on the distribution of grammatical patterns in historical stages of a language than it does today, so far many studies on the relationship between information structure and grammatical patterns do not go beyond the sentence level, in other words, they do not take into account the possible influence of the text type on the manifestation of certain grammatical patterns. How and to which degree changes in grammatical patterns correlate or are affected by changes in either discourse and/or narrative structure, how the two layers interact with each other and affect each other, and how such issues can be operationalized are still understudied. This volume aims to shed more light on these issues by presenting eight papers, which address these questions more or less explicitly. As the research questions imply, the papers all take a historical or diachronic perspective. Another commonality between the studies is that they all focus on data from Germanic languages, as we assume that by comparing closely related languages, the relationships in question become more pronounced. Specifically, the languages in question are German, Dutch, English and Icelandic. Understandably, the contributions in this volume can only highlight some aspects of the complex relationship between grammar and narration(s). Addressing among others questions of narrative progression, temporal structure, reference tracking and discourse functions, the contributions discuss phenomena such as temporal adverbials at the left periphery as well as later in the clause, left dislocation structures, fronting of the finite verb in dependent and independent clauses, linguistic means to express aspectual and tense information, and the distribution of nominalization patterns across text types.}
\author{Ulrike Demske and Barthe Bloom} 
 

\renewcommand{\lsISBNdigital}{978-3-96110-529-8}
\renewcommand{\lsISBNhardcover}{978-3-98554-149-2}
\BookDOI{10.5281/zenodo.15657267}
\typesetter{Hannah Schleupner}
\proofreader{Amy Amoakuh}
% \lsCoverTitleSizes{51.5pt}{17pt}% Font setting for the title page


\renewcommand{\lsSeries}{ogl} 
\renewcommand{\lsSeriesNumber}{13}
\renewcommand{\lsID}{503}
