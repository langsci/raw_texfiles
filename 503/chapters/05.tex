\documentclass[output=paper,colorlinks,citecolor=brown]{langscibook}
\ChapterDOI{10.5281/zenodo.15689127}
\author{Hannah Booth\orcid{}\affiliation{Ghent University}}

\title{Narration and clause-internal adverbs: An information-structural watershed in Old Icelandic}
\shorttitlerunninghead{Narration and clause-internal adverbs} 
\abstract{Various discourse functions have been attributed to temporal/spatial adverbs in Early Germanic narrative texts, especially in Early West Germanic. 
By contrast, the role of such adverbs in Early North Germanic has been neglected, despite there being a rich attestation of saga narratives, ripe for studying such phenomena. 
In this paper, I investigate the behaviour of clause-internal \textit{þá} (lit.~`then'), \textit{þar} (lit.~`there') and \textit{nú} (lit.~`now') in relation to discourse in Old Icelandic saga narratives.  
I show that these adverbs serve as an ``information-structural watershed''\is{information-structural watershed} (\citealp{krivonosov1977deutsche,grosz2016information}), segmenting the clause into discourse domains, and show how this function can be modelled within Lexical Functional Grammar (\citealp{bresnan-kaplan82,Bresnan2015lexical}). I also show that, diachronically, these adverbs are virtually in complementary distribution with the clause-initial expletive \textit{það}, a later development which takes over some of the discourse function previously served by the clause-internal adverbs.
}

%\DeclareRobustCommand{\disambiguate}[3]{#2~#3}
% Uncomment if only this chapter is to be compiled
%\IfFileExists{localcommands.tex}{
%   \addbibresource{localbibliography.bib}
%   \usepackage{langsci-optional}
\usepackage{langsci-gb4e}
\usepackage{langsci-lgr}

\usepackage{listings}
\lstset{basicstyle=\ttfamily,tabsize=2,breaklines=true}

%added by author
% \usepackage{tipa}
\usepackage{multirow}
\graphicspath{{figures/}}
\usepackage{langsci-branding}

%   
\newcommand{\sent}{\enumsentence}
\newcommand{\sents}{\eenumsentence}
\let\citeasnoun\citet

\renewcommand{\lsCoverTitleFont}[1]{\sffamily\addfontfeatures{Scale=MatchUppercase}\fontsize{44pt}{16mm}\selectfont #1}
  
%   %% hyphenation points for line breaks
%% Normally, automatic hyphenation in LaTeX is very good
%% If a word is mis-hyphenated, add it to this file
%%
%% add information to TeX file before \begin{document} with:
%% %% hyphenation points for line breaks
%% Normally, automatic hyphenation in LaTeX is very good
%% If a word is mis-hyphenated, add it to this file
%%
%% add information to TeX file before \begin{document} with:
%% %% hyphenation points for line breaks
%% Normally, automatic hyphenation in LaTeX is very good
%% If a word is mis-hyphenated, add it to this file
%%
%% add information to TeX file before \begin{document} with:
%% \include{localhyphenation}
\hyphenation{
affri-ca-te
affri-ca-tes
an-no-tated
com-ple-ments
com-po-si-tio-na-li-ty
non-com-po-si-tio-na-li-ty
Gon-zá-lez
out-side
Ri-chárd
se-man-tics
STREU-SLE
Tie-de-mann
}
\hyphenation{
affri-ca-te
affri-ca-tes
an-no-tated
com-ple-ments
com-po-si-tio-na-li-ty
non-com-po-si-tio-na-li-ty
Gon-zá-lez
out-side
Ri-chárd
se-man-tics
STREU-SLE
Tie-de-mann
}
\hyphenation{
affri-ca-te
affri-ca-tes
an-no-tated
com-ple-ments
com-po-si-tio-na-li-ty
non-com-po-si-tio-na-li-ty
Gon-zá-lez
out-side
Ri-chárd
se-man-tics
STREU-SLE
Tie-de-mann
}
%   \boolfalse{bookcompile}
%\togglepaper[23]%%chapternumber
%}{}

%%%%%%%%%%%%%%%%%%%%%%%%%%%%%%%%%%
\begin{document}
\maketitle

%%%%%%%%%%%%%%%%%%%%%%%%%%%%%%%%%%
\section{Introduction}
%%%%%%%%%%%%%%%%%%%%%%%%%%%%%%%%%%

\noindent Narratives as a \isi{genre} bring their own specific stylistic characteristics which have been shown to interact with the expression of information structure in various ways (e.g.~\citealp{carroll2003information,dimroth2010given,riester2015analyzing}). 
As one type of narrative, the Old Icelandic sagas\is{Icelandic saga} have long been recognised by literary historians for their unique \isi{style} and special rhetorical devices (e.g.~\citealp{clover1974scene,olason2004family,pehlpstead2017time}), but so far their discourse-related properties have only been scarcely investigated within theoretical linguistics (e.g.~\citealp{hroarsdottir2009ov,booth-beck20200jhs}). As such, their great potential to shed light on phenomena at the syntax-discourse interface remains largely untapped.

In this paper, I contribute to this area via an investigation of a small set of light adverbs\is{adverb} (\textit{þá}, lit.~`then'; \textit{þar}, lit.~`there'; \textit{nú}, lit.~`now') which regularly appear clause-internally in Old Icelandic, especially in verb-initial\is{verb-first} clauses, e.g.~(\ref{top-com-thet}).\footnote{Throughout the paper, I gloss the respective adverbs\is{adverb} simply as \textsc{þá}, \textsc{þar}, and \textsc{nú} to reflect the fact that, in clause-internal position, they often undergo some semantic weakening.}

\ea \label{top-com-thet}
\ea
\gll Sótti Ormur \textbf{þá} Sturlu að þessu máli og fóru \textbf{þá} margar orðsendingar millum þeirra Orms og Sturlu... \\
pursued Ormur.\textsc{nom} \textsc{þá} Sturla.\textsc{acc} at this.\textsc{dat} matter.\textsc{dat} and went \textsc{þá} many.\textsc{nom} messages.\textsc{nom} between they.\textsc{gen} Ormur.\textsc{gen} and Sturla.\textsc{gen}\\
\glt `Ormur pursued Sturla in this matter and there went many messages between them\dots' \hfill [IcePaHC: 1250, Sturlunga.396.269--270]
        
\ex
\gll Sátu þeir \textbf{þar} við er lengra fóru til. Var \textbf{þar} skipað saman þeim sem jafnsterkastir voru\dots\\
sat they.\textsc{nom} \textsc{þar} \textsc{ptcl} \textsc{rel} further went to was \textsc{þar} matched together they.\textsc{dat} \textsc{rel} equally.strongest.\textsc{nom} were \\
\glt `Those who had come from further off would stay near by. They were matched together according to their strength\dots' \\ \hfill [IcePaHC: 1310, Grettir.162--163]

\ex
\gll  Söðull var undir kviði niðri á hesti Grettis en {í burt} malurinn. Fer hann \textbf{nú} til leitar og fann eigi. Sér hann \textbf{nú} hvar maður gengur.\\
saddle.\textsc{nom} was under belly.\textsc{dat} lower.\textsc{dat} on horse.\textsc{dat} Grettir.\textsc{gen} and away knapsack.\textsc{nom} goes he.\textsc{nom} \textsc{nú} to search.\textsc{gen} and found \textsc{neg} sees  he.\textsc{nom} \textsc{nú} where man.\textsc{nom} goes\\
\glt  `Grettir's horse had got its saddle twisted round under its belly and the knapsack had gone, so he went to look for it, but couldn't find it. He sees a man walking by.' \hfill [IcePaHC: 1310, Grettir.237--240]
\z 
\z 

\noindent The characteristic presence of these elements in the Old Icelandic sagas\is{Icelandic saga} has been previously noted (cf.~\citealp[71]{clover1974scene}; \citealp[71]{Faarlund1990}), but the precise discourse properties of such clauses have not been previously investigated.  
On the basis of corpus data from \isi{IcePaHC} \citep{IcePaHC} and \isi{MIcePaHC} \citep{MIcePaHC}, 
I claim that the clause-internal adverbs\is{adverb} serve a specific \isi{discourse function}, 
segmenting the clause into discourse domains, thus acting as an ``\isi{information-structural watershed}'' (cf.~\citealp{krivonosov1977deutsche,grosz2016information} on Modern \ili{German} and \citealp{vanKem-Los2006} on \ili{Old English}). I show how this specific function of clause-internal adverbs\is{adverb} can be modelled within the parallel architecture of Lexical Functional Grammar \citep{bresnan-kaplan82,Bresnan2015lexical,dalrymple2019oxford}, in which different types of linguistic information are represented at independent, interacting dimensions. I also examine the diachronic status of these clause-internal adverbs\is{adverb}, showing that, across the history of Icelandic, they are virtually in complementary distribution with the clause-initial \isi{expletive} \textit{það}, which represents a later development (cf.~\citealp{Hroarsdottir1998,Rognvaldsson2002,booth2018,booth2019cataphora,booth2020expletives}). I relate this to a broader change whereby Icelandic gradually shifts its structural specification of discourse functions\is{discourse function} from the postfinite domain to the clause-initial position over time.

The paper proceeds as follows. In Section \ref{sect:emgc}, I review previous work on adverbs\is{adverb} and discourse in Early Germanic, especially in Early West Germanic where the bulk of previous research has been conducted, which serves as a backdrop to my investigation of Old Icelandic. Section \ref{sect:data} presents the corpus findings with respect to the frequency and distribution of clause-internal \textit{þá}/\textit{þar}/\textit{nú}, and how this relates to the broader mapping correspondences between position and information structure in the postfinite domain. Section \ref{sect:lfg} outlines how the findings can be theoretically modelled within Lexical Functional Grammar and Section \ref{sect:dia} presents some findings with respect to the diachrony of clause-internal adverbs\is{adverb} in relation to broader changes in the grammar. Section \ref{sect:conc} concludes the paper.

%%%%%%%%%%%%%%%%%%%%%%%%%%%%%%%%%%
\section{Adverbs, position and discourse in Early Germanic}\label{sect:emgc}
%%%%%%%%%%%%%%%%%%%%%%%%%%%%%%%%%%

Previous work on Early Germanic, in particular West Germanic (\ili{Old English}, \ili{Old High German}, Old Saxon), has highlighted the fact that certain temporal/spatial adverbs\is{temporal/spatial adverb} often serve particular discourse functions,\is{discourse function} especially in narratives (e.g.~\citealp{foster1975use,enkvist1986more,enkvist1987old,vanKem-Los2006,trips2009syntax,waarvik2013participant,Cichosz2022}). Crucially, some studies have highlighted the fact that the precise discourse functions\is{discourse function} of such adverbs\is{adverb} vary by position (e.g.~\citealp{vanKem-Los2006,Axel2007,donhauser2009rolle,vanKem2011syntax,petrova2011modeling,vanKem2020discourse,catasso2021he}), with important differences between clause-initial adverbs\is{adverb} and clause-internal adverbs\is{adverb}, as I now discuss.


%Clause-initial
%%%%%%
\subsection{Clause-initial adverbs}
%%%%%%
Various authors have claimed for \ili{Old English} that \textit{þa/þonne} (lit.~`then'), when occurring in the clause-initial position, marks a sequence of actions/events which represent the main line of the narrative (``foregrounding'') \citep{foster1975use,enkvist1986more,enkvist1987old,waarvik2013participant,Cichosz2022}. 
An example of a continuous narrative whereby the main storyline is carried forward by repeated use of clause-initial \textit{þa} is shown in (\ref{oe-cont}).

\ea \label{oe-cont} Old English: \\
{\itshape He sæde þæt he æt sumum cirre wolde fandian hu longe þæt land norþryhte læge, oþþe hwæðer ænig mon be norðan þæm westenne bude. \textbf{Þa} for he norþryhte be þæm lande; let him ealne weg þæt weste land on ðæt steorbord \& þa widsæ on ðæt bæcbord þrie dagas. \textbf{Þa} wæs he swa feor norþ swa þa hwælhuntan firrest faraþ. \textbf{Þa} for he þa giet norþryhte swa feor swa he meahte on þæm oþrum þrim dagum gesiglan. \textbf{Þa} beag þæt land þær eastryhte, oþþe seo sæ in on ðæt lond, he nysse wæðer buton he wisse ðæt he ðær bad westanwindes \& hwon norþan \& siglde ða east be lande swa swa he meahte on feower dagum gesiglan. \textbf{Þa} sceolde he ðær bidan ryhtnorþanwindes, for ðæm þæt land beag þær suþryhte, oþþe seo sæ in on ðæt land, he nysse hwæþer. \textbf{Þa} siglde he þonan suðryhte be lande swa swa he mehte on fif dagum gesiglan. \textbf{Ða} læg þær an micel ea up in on þæt land.}\\
`He said that at one occasion he wanted to find out how far that land extended northwards, or whether any man lived north of the wilderness. \textbf{Then} he travelled northwards along the coast; keeping all the way the waste land on the starboard and the open sea on the portside for three days. \textbf{Then} he was as far north as the whalehunters go furthest. \textbf{Then} he travelled still northwards as far as he could sail in another three days. \textbf{Then} the land turned east, or the sea into the land, he didn't know which, but he knew that he there waited for a wind from the west and somewhat from the north and sailed then east along the coast as far as he could sail in four days. \textbf{Then} he had to wait for a due north wind, because that land turned there directly to south, or the sea into the land, he didn't know which. \textbf{Then} he sailed from there southwards along the coast as far as he could sail in five days. \textbf{Then} there was a large river reaching up into the land.' \hfill [Or\_1:1.14.5.226--235];\\ \hfill (\citealp[234]{enkvist1987old}, as cited in \citealp[179--180]{trips2009syntax})
\z 

\noindent Others have focussed on the discourse continuity/\isi{discourse-linking} effect of clause-initial \textit{þa/þonne} (e.g.~\citealp{Los2000,vanKem-Los2006,trips2009syntax}).
\citet{trips2009syntax}, for instance, argue that clause-initial \textit{þa/þonne} function as discourse-anchoring temporal anaphors, which they relate to the fact that \textit{þa/ þonne} derive from former demonstratives\is{demonstrative} (cf.~Proto-Germanic *\textit{TO-}), and thus were always anaphoric/deictic\is{deictic markers} in relation to something previously mentioned.

Similar observations have been made for certain clause-initial elements in Gothic \citep{klein1994gothic} and Old Saxon \citep{linde2009aspects}. \citet{klein1994gothic} claims that Gothic \textit{þar-uh} (lit.~`there'+ \textit{uh}) and \textit{þan-uh} (lit.~`then' + \textit{uh}) function as ``discourse articulators'' which indicate narrative continuity, albeit associated with a change of grammatical subject. For Old Saxon, \citet{linde2009aspects} observes that clause-initial \textit{tho} (lit.~`then') and other adverbs\is{adverb} such as \textit{so} (lit.~`so') and \textit{nu} (lit.~`now') have a similar function, signalling the continuation of the \isi{narration} and guaranteeing progress in discourse. 

%%%%%%
\subsection{Clause-internal adverbs}
%%%%%%
At the same time, a series of studies have shown that, when occurring clause-internally, such adverbs\is{adverb} have a rather different function. \citet{donhauser2009rolle}, for instance, examine the properties of clause-initial and clause-internal \textit{tho} (lit.~`then') in \ili{Old High German} and observe a clear difference: While clause-initial \textit{tho} refers to a time span already established in the preceding section, as shown in (\ref{don-1}), clause-internal \textit{tho} refers to a novel, indefinite time interval introduced as the topic time (cf.~\citealp{klein1994time}) of a new episode, e.g.~(\ref{don-2}) (see also the discussion in \citealp{petrova2011modeling}).

\ea Old High German:
\ea\label{don-1} 
\gll \textbf{tho} uuas man in hierusalem\\
then was man in Jerusalem\\
\glt `A man lived in Jerusalem at that time'\\
Latin: \textit{homo erat in hierusalem} \hfill [T 37, 23]; \citep[219]{petrova2011modeling}\\

\ex\label{don-2}
\gll uaas \textbf{thó} giuuortan in anderemo sambaztag\\
was then become in another Sabbath\\
\glt `It happened on another Sabbath'\\
Latin: \textit{Factum est in alio sabbatum autem} \hfill [T 37, 23]; \citep[219]{petrova2011modeling}
\z 
\z

\noindent Tying this in with the position of the finite verb, \citet[224]{petrova2011modeling} argues that, in cases of clause-internal \textit{tho}, a verb-initial\is{verb-first} (V1) structure is triggered by the novelty of the time interval referred to by \textit{tho}, in line with the fact that V1 declaratives\is{verb-initial declaratives}\is{verb-first} more generally are used to introduce a new situation (cf.~\citealp{hinterholzl2010v1}).\is{discourse-new}

The special status of clause-internal \textit{tho} and its close relation to V1\is{verb-first} in \ili{Old High German} is also observed by \citet{Axel2007,axel2009verb}. \citet[156, 167]{Axel2007} states that ``most'' \ili{Old High German} V1 declaratives\is{verb-initial declaratives}\is{verb-first} contain a clause-internal \textit{tho} as ``a very characteristic lexical feature''. In such contexts, she observes, \textit{tho} often has very weak semantics, as evidenced by the fact that it occurs with an additional temporal adverbial in the same clause, e.g.~(\ref{axel-tho}). Axel suggests that, in such contexts, \textit{tho} serves as a narrative-emphatic particle, belonging to the broader, residual system of sentence particles exhibited in early \ili{Old High German} texts, which is a vestige from an earlier period before the generalisation of \isi{verb-second} (\citealp[169--170]{Axel2007}, \citealp[35--36]{axel2009verb}). 

\ea\label{axel-tho} Old High German:\\
\gll inti uuas \textbf{tho} giheilit/ ira tohter \textbf{fon} \textbf{dero} \textbf{ziti}/\\
and was \textsc{tho} healed her daughter from that hour\\
\glt `and her daughter was healed from that hour'\\
Latin: \textit{\& sanata est/filia illus ex illa hora/} \hfill  [T 273,31]; \citep[156]{Axel2007}
\z 

The \isi{discourse function} of clause-internal adverbs\is{adverb} has also been highlighted for \ili{Old English} in work by van Kemenade and colleagues \citep{vanKem-Los2006,vanKem2008balance,vanKemenade2009discourse,vanKem2011syntax,vanKem2020discourse}. The central claim is that, in subordinate clauses, clause-internal \textit{þa/þonne} (`then') functions as a ``\isi{discourse partitioner}'', separating presupposed,\is{presupposition} given information from new,\is{discourse-new} focussed\is{focus} information, cf.~(\ref{vankem-schem}) (adapted from \citealp[10]{vanKem2008balance}).

\ea \label{vankem-schem} [\textsubscript{\textit{utterance}} \textsc{presupposition} -- \textit{þa/þonne} -- \textsc{focus}]
\z 

\noindent The evidence for this claim comes from a series of quantitative corpus findings. For instance, \citet[232]{vanKem-Los2006} observe that pronominal subjects in the postfinite domain occur overwhelmingly to the left of \textit{þa/þonne}, e.g.~(\ref{vankem-pro}); additionally, pronominal objects also regularly appear in this position, as in the example in (\ref{vankem-pro}).

\ea  \label{vankem-pro} Old English:\\
\gll forþæm he wenð þæt [he] [hi] \textbf{þonne} ealle hæbbe\\
because he knows that he them then all have\\
\glt `because he know that he then had them all' \\ \hfill [coboeth, Bo: 24.56.16.1031]; \citep[12]{vanKem2008balance}
\z 


By contrast, indefinite nominal subjects are preferred to the right of \textit{þa/þonne} \citep[237]{vanKem-Los2006}, e.g.~(\ref{vankem-indef}); \citet[13]{vanKem2008balance} find that indefinite NPs introducing new\is{discourse-new} discourse entities do not occur to the left of \textit{þa/þonne} in their data.

\ea \label{vankem-indef} Old English: \\
\gll Gif \textbf{ðonne} [hwelc mon] forbireð his synna\\
if then any many forebears his sins\\
\glt `If anyone then refrains from his sins' \\ \hfill [cocura, CP: 37.265.1.1719]; \citep[13]{vanKem2008balance}
\z 

\noindent Definite nominal subjects can occur to the left or right of \textit{þa/þonne} but, as \citet[13]{vanKem2008balance} observe, definite nominals occurring in the left position are +anaphoric. Via a statistical analysis, \citet[19]{vanKem2008balance} find that the most important trigger for a nominal to occur in the position to the left of \textit{þa/þonne} is the presence of an antecedent in the previous discourse. The overall findings in these studies are related in \citet[233]{vanKem-Los2006} to a broader claim that \ili{Old English} is discourse-configurational\is{discourse configurationality} (cf.~\citealp{Kiss1995discourse}), in other words a language with designated structural positions for discourse functions\is{discourse function} such as topic and \isi{focus}. This is reminiscent of work on modern \ili{German}, where it has been argued that clause-internal discourse particles segment the postfinite domain into (aboutness) topic\is{aboutness topic} and \isi{focus} fields (\citealp{krivonosov1977deutsche}, revisited by \citealp{grosz2016information}), e.g.~(\ref{mod-germ}).


\ea 
\label{mod-germ} Modern German: \\
\gll  Dann hat [Riko]\textsubscript{\textsc{topic}} \textbf{ja} [eine Frau]\textsubscript{\textsc{focus}} geküsst\\
then has Riko \textsc{ptcl} a woman kissed\\
\glt `Then Riko has kissed a woman' \hfill \citep[338]{grosz2016information}
\z 

Elsewhere in Early West Germanic, \ili{Old High German}/Old Saxon \textit{tho} has also been claimed to serve an information-structural function, namely marking topicality\is{topicalization} in a specific \isi{verb-third} construction of the form XP-\textit{tho}-V  (\citealp[224--225]{Axel2007}, \citealp{catasso2021he}), e.g.~(\ref{wgmc-tho}).

\ea \label{wgmc-tho}
\ea Old High German:\\
\gll sie \textbf{tho} antalengitun imo. neín\\
they \textsc{do} answered him.\textsc{dat} no\\
\glt `They said to him: ``No''.'\\
Latin ~\textit{Responderunt ei: non} \hfill [T.~337]; \citep[2]{catasso2021he}
    
\ex Old Saxon:\\
\gll Petrus th{\^o} gimahalde [\dots]\\
Peter \textsc{tho} said \\
\glt `Peter said [\dots].' \hfill [Hel.~XXXVIII, 3136]; \citep[2]{catasso2021he}   
\z 
\z 

\noindent \citet{catasso2021he} observe that the clause-initial constituent in constructions like (\ref{wgmc-tho}) is usually a pronominal \isi{shift topic}, and they thus argue that \textit{tho} is a topic marker in such contexts. \citet[225]{Axel2007} similarly states that \textit{tho} signals a change in topic in such contexts.

%%%%%%
\subsection{Summary}
%%%%%%

In sum, there is a good deal of work on Early West Germanic which has highlighted the special status of clause-internal adverbs\is{adverb} in terms of their discourse functions.\is{discourse function} By comparison, the behaviour of clause-internal adverbs\is{adverb} in Early North Germanic has generally been neglected, with only a few sparse references to their frequent occurrence in Old Norse/Icelandic texts. 
For instance, \citet[71]{clover1974scene} observes that there is ``repeated interposition'' of temporal/spatial adverbs\is{temporal/spatial adverb} in Icelandic translations of \textit{chansons de geste} where the Old French original has none. Similarly, \citet[71]{Faarlund1990}, in relation to Old Norse texts more broadly, states that the \isi{adverb} \textit{þar} ``often'' occurs immediately after the finite verb, providing the examples in (\ref{faarlund}).

\ea \label{faarlund} Old Norse/Icelandic:
\ea Konungs skuggsjá\\
\gll Þá kemr \textbf{þar} elldur af því jarni\\
then comes \textsc{þar} fire.\textsc{nom} from that.\textsc{dat} iron.\textsc{dat}\\
\glt `Then fire comes from that iron.' 
\ex Óláfs saga Tryggvasonar\\
\gll Ok varð \textbf{þar} mikit mannfall\\
and was \textsc{þar} great.\textsc{nom} man.fall.\textsc{nom}\\
\glt `And there was a great loss of men' 
\ex Óláfs saga Tryggvasonar\\
\gll Kemr \textbf{þar} mikill fj\c{o}ldi manna\\
comes \textsc{þar} great.\textsc{nom} multitude.\textsc{nom} men.\textsc{gen}\\
\glt `There are many men coming' \hfill \citep[71]{Faarlund1990}
\z 
\z 

Nevertheless, the precise behaviour of clause-internal adverbs\is{adverb} in Old Norse/Ice\-landic, and the broader discourse-related properties of such clauses, has not been closely examined. In this paper, I address this gap in the existing literature on the morpho-syntax-discourse interface in Early Germanic.


%%%%%%%%%%%%%%%%%%%%%%%%%%%%%%%%%%
\section{Old Icelandic \textit{þá}, \textit{þar} and \textit{nú}}\label{sect:data}
%%%%%%%%%%%%%%%%%%%%%%%%%%%%%%%%%%

In this section, I examine the distribution and properties of three temporal/spatial adverbs\is{temporal/spatial adverb} in Old \isi{Icelandic saga} narratives, namely \textit{þá} (lit.~`then'), \textit{þar} (lit.~`there') and \textit{nú} (lit.~`now'). It is possible that there are other adverbs\is{adverb} which fall into a class with these, but these three appear to be the most representative and most frequent clause-internally. The claims are made on the basis of a series of corpus investigations using \isi{IcePaHC} \citep{IcePaHC} and \isi{MIcePaHC} \citep{MIcePaHC}. For \isi{IcePaHC}, I restrict myself specifically to sagas\is{Icelandic saga} dated to the period 1150--1450.\footnote{From the IcePaHC saga texts I removed 1350.BandamennM and 1350.Finnbogi since these are duplicated in MIcePaHC.} For \isi{MIcePaHC}, I examine all texts dated to 1450 (all texts in the corpus represent saga\is{Icelandic saga} narratives). Combined, this yields a total of 46 texts and 134,239 sentence tokens. I will refer to both corpora together as (M)\isi{IcePaHC}\is{MIcePaHC}.

%%%%%%%
\subsection{Distribution of \textit{þá}, \textit{þar} and \textit{nú}}
%%%%%%%

All three adverbs\is{adverb} occur both clause-initially, e.g. (\ref{ice-initial}), and clause-internally, e.g.~(\ref{ice-internal}).

\ea \label{ice-initial}
\ea
\gll \textbf{Þá} gekk út kerling ein og hafði ullkamb í hendi.\\
then went out woman.\textsc{nom} one.\textsc{nom} and had wool.comb.\textsc{acc} in hand.\textsc{dat}\\
\glt `Then a certain woman went out and (she) had a wool comb in her hand.' \hfill [MIcePaHC: 1390, Graenlendingath.55--56]
\ex 
\gll \textbf{Þar} sökk og niður lík Þorvalds\dots \\
there sank also down corpse.\textsc{nom} Þorvaldur.\textsc{gen}\\
\glt `There also the corpse of Þorvaldur sank\dots' \\ \hfill [MIcePaHC: 1300, Njals.742]
\ex 
\gll \textbf{Nú} komu þeir til Grænlands og eru með Eiríki rauða um veturinn.\\
now came they.\textsc{nom} to Greenland.\textsc{gen} and are with Eiríkur.\textsc{dat} red.\textsc{dat} in winter.\textsc{acc.def}\\
\glt `Now they came to Greenland and (they) are with Eiríkur the Red for the winter.' \hfill [MIcePaHC: 1305, Eiriks.845--846]
\z 
\z 

\ea \label{ice-internal}
\ea
\gll Segir Ólafur \textbf{þá} ætlan sína\dots \\
says Ólafur.\textsc{nom} \textsc{þá} opinion.\textsc{acc} his.\textsc{acc}\\
\glt `Ólafur says his opinion\dots' \hfill [MIcePaHC: 1275, Laxdaela.1179]
\ex
\gll Vésteinn gengur \textbf{þar} til húss\dots\\
Vésteinn.\textsc{nom} goes \textsc{þar} to house.\textsc{gen}\\
\glt `Vésteinn goes to the house\dots' \hfill (MIcePaHC: 1400, Gisla.511]
\ex 
\gll Fóru þeir \textbf{nú} sína leið.\\
went they.\textsc{nom} \textsc{nú} their.\textsc{acc} way.\textsc{acc}\\
\glt `They went on their way.' \hfill [MIcePaHC: 1250, Egils.4036]
\z 
\z 

\noindent In the (M)\isi{IcePaHC}\is{MIcePaHC} data, clause-initial \textit{þá}/\textit{þar}/\textit{nú} occurs at a frequency of 12.4\% in matrix declarative V2\is{verb-second} clauses and 4.1\% in subordinate declarative V2\is{verb-second} clauses, see Table \ref{tab:initial_da}. Assuming that clause-initial \textit{þá}/\textit{þar}/\textit{nú} serve a \isi{discourse-linking} function and mark a \isi{sequence of events} on the main line of the narrative just as they do elsewhere in Early Germanic (see Section \ref{sect:emgc}), it is not surprising that the phenomenon is more common in matrix clauses, since subordinate clauses typically present presupposed,\is{presupposition} backgrounded information (e.g.~\citealp{HooperHooper1973, quirk1985comprehensive,matsuda1998conservatism,matic2014information}) and as such are less likely to express events on the main storyline. 

\begin{table}
\caption{Frequency of clause-initial \textit{þá}/\textit{þar}/\textit{nú} in V2 matrix and subordinate declaratives in (M)IcePaHC saga narratives, 1210--1450}
\label{tab:initial_da}
 \begin{tabularx}{.6\textwidth}{X rr}
  \lsptoprule
  & matrix & subordinate\\
  \midrule
initial \textit{þá}/\textit{þar}/\textit{nú} & 10,276  &  1,738  \\
all V2                                       &  82,687 &  42,256  \\
\% initial \textit{þá}/\textit{þar}/\textit{nú} & \textbf{12.4\%} & \textbf{4.1\%} \\
  \lspbottomrule
 \end{tabularx}
\end{table}

\begin{table}[b]
\caption{Frequency of clause-initial \textit{þá}/\textit{þar}/\textit{nú} in V2 non-conjunct/first conjunct and non-initial conjunct subordinate declaratives in (M)IcePaHC saga narratives, 1210--1450}
\label{tab:initial_da_sub_conj}
 \begin{tabularx}{.85\textwidth}{X rr}
  \lsptoprule
  & non-/first conjunct & non-initial conjunct\\
  \midrule
initial \textit{þá}/\textit{þar}/\textit{nú} & 1,588 &  132  \\
all V2                                       & 38,483  &  1,466  \\
\% initial \textit{þá}/\textit{þar}/\textit{nú} & \textbf{4.1\%} & \textbf{9.0\%} \\
  \lspbottomrule
 \end{tabularx}
\end{table}

Interestingly, if one isolates separately (i) subordinate clauses which are either non-conjunct clauses or initial conjunct clauses and (ii) subordinate clauses which are non-initial conjunct clauses, clause-initial \textit{þá}/\textit{þar}/\textit{nú} are more frequent in non-initial conjunct clauses than in their non-conjunct/first conjunct counterparts, see Table \ref{tab:initial_da_sub_conj}. Some examples of the relevant non-initial conjunct clauses with clause-initial \textit{þá}/\textit{þar}/\textit{nú}  are provided in (\ref{non-initial-conj}).



\ea \label{non-initial-conj} 
\ea 
 \gll En svo var djúpið mikið að það var jafnskjótt er hann kom til grunna og [\textbf{þá} var þrotið örendi hans].\\
 and so was deep.water.\textsc{nom.def} great.\textsc{nom} that it.\textsc{nom} was immediately that he.\textsc{nom} came to bottom.\textsc{gen} and then was failed breath.\textsc{nom} he.\textsc{gen}\\
 \glt `And the deep water was so great that immediately he came to the bottom and then his breath failed.' \hfill [IcePaHC: 1210, Jartein.789]
 \ex 
 \gll Svo er sagt að Háls Fjörleifarson gerði bú að Tjörnum í Ljósavatnsskarði og [\textbf{þar} gerðist vinátta mikil {í millum} og Eysteins í Rauðaskriðu].\\
 so is said that Háls.\textsc{nom} Fjörleifarson.\textsc{nom} made home.\textsc{acc} at Tjarnir.\textsc{dat} in Ljósavatnsskarð.\textsc{dat} and there became friendship.\textsc{nom} great.\textsc{nom} between also Eysteinn.\textsc{gen} in Rauðaskriða.\textsc{dat}\\
 \glt `So it is said that Háls Fjörleifarson set up home at Tjarnir in Ljósavatnsskarð there became a great friendship between (him and) Eysteinn in Rauðaskriða.' \hfill [MIcePaHC: 1400, Reykdaela.74]
 \ex 
\gll  Gerðist nú rómur mikill {að því að} eytt væri vígsmálinu og [\textbf{nú} væri vörn framar en sókn].\\
became now applause.\textsc{nom} great.\textsc{nom} because come.to.nothing was.\textsc{sbjv} manslaughter.case.\textsc{dat.def} and now was.\textsc{sbjv} defence.\textsc{nom} more than prosecution.\textsc{nom}\\
\glt `There was now great applause because the manslaughter case had come to nothing and now there would be more deference than prosecution.'  \hfill [MIcePaHC: 1300, Njals.9017]
\z 
\z 

This pattern, whereby a syntactic phenomenon which is overall more frequent in matrix than subordinate clauses is unusually frequent in subordinate clauses (which are non-initial conjuncts), has been observed for another discourse-related phenomenon in Old Icelandic  \citep{booth-beck20200jhs}. \citet[23]{booth-beck20200jhs} find that \isi{narrative inversion}, a special type of V1\is{verb-first} which features a topical subject in the postfinite position, is robustly attested in subordinate clauses which are non-initial conjuncts, despite the fact that it is virtually unattested in subordinate clauses which are either not coordinated or first conjuncts. The examples which \citet{booth-beck20200jhs} provide are given here in (\ref{ni-conj}).

\ea \label{ni-conj}
\ea
\gll ``Eigi veit eg,'' segir Urðarköttur, ``{því að} eg er ungur og [\textbf{kann} eg á fá skyn]'' \\
\textsc{neg} know I.\textsc{nom} says Urðarköttur.\textsc{nom} because I.\textsc{nom} am young.\textsc{nom} and know I.\textsc{nom} on few.\textsc{acc} understanding.\textsc{acc}\\
\glt ‘``I do not know'', says Urðarköttur, ``because I am young and I do not have much knowledge''' \\ \hfill [IcePaHC: 1350, Finnbogi.631.351]; (as cited in \citealp[23]{booth-beck20200jhs})
\ex 
\gll Þú skalt... og mæla síðan  þessum orðum við konunginn, að eg leiði hér eftir mér einn svein, [og \textbf{kalla} eg þar öngan mann...] \\
you.\textsc{nom} shall  and say then these.\textsc{dat} words.\textsc{dat} with king.\textsc{acc.def} that I.\textsc{nom} lead here after I.\textsc{dat} one.\textsc{acc} boy.\textsc{acc} and call I.\textsc{nom} \textsc{þar} no.\textsc{acc} man.\textsc{acc}\\
\glt `You should\dots and then say these words with the king that I lead here after me a certain boy and I call there no man\dots' \\ 
$[$IcePaHC: 1260, Jomsvikingar.1091]; (as cited in \citealp[23]{booth-beck20200jhs})
\z 
\z 

\noindent \citet[25]{booth-beck20200jhs} offer an explanation for the occurrence of \isi{narrative inversion} in subordinate non-initial conjuncts on the basis that \isi{narrative inversion} is a marker of topic anaphoricity, such that it is only motivated in contexts where there is a directly preceding subordinate clause with its own topic as a potential antecedent. On the assumption that clause-initial \textit{þá}/\textit{þar}/\textit{nú} serve a \isi{discourse-linking} function and mark a \isi{sequence of events}, a similar line of reasoning is a likely explanation for their higher frequency in subordinate non-initial conjuncts compared to their non-conjunct/first conjunct counterparts.

With respect to clause-internal \textit{þá}/\textit{þar}/\textit{nú}, I examine their frequency across four environments, (i) V2\is{verb-second} clauses with an overt subject, (ii) V2\is{verb-second} clauses with an unexpressed subject, (iii) V1\is{verb-first} clauses with an overt subject, and (iv) V1\is{verb-first} clauses with an unexpressed subject.\footnote{
I operationalise ``clause-internal'' in this context to mean either directly after the finite verb or one constituent later than the finite verb, and not the last constituent in the clause.}
The results for the two V2\is{verb-second} environments are shown in Table \ref{tab:internal_v2}, and for the two V1\is{verb-first} environments in Table \ref{tab:internal_v1}. In all four environments, clause-internal \textit{þá}/\textit{þar}/\textit{nú} occur more frequently in matrix compared to subordinate clauses. Strikingly there are no examples of V2\is{verb-second} subordinate clauses which have an unexpressed subject and also a clause-internal \textit{þá}/\textit{þar}/\textit{nú}, see Table \ref{tab:internal_v2}. Comparing V2\is{verb-second} and V1\is{verb-first} environments, clause-internal \textit{þá}/\textit{þar}/\textit{nú} are a lot more frequent in V1\is{verb-first} clauses, and are particularly frequent in matrix V1\is{verb-first} clauses with an overt subject (28.5\%, see Table \ref{tab:internal_v1}) and even more frequent in those with an unexpressed subject (34.4\%, see Table \ref{tab:internal_v1}). This is quantitative evidence which confirms the claim in \citet[71]{Faarlund1990}, that these adverbs\is{adverb} ``often'' occur immediately after the finite verb in such contexts (see Section \ref{sect:emgc}). More broadly, the relative commonness of the phenomenon in V1\is{verb-first} declaratives\is{verb-initial declaratives} specifically is in line with Axel's observations for \ili{Old High German}, where she claims that the presence of clause-internal  \textit{tho} is a ``very characteristic lexical feature'' \citep[167]{Axel2007} (see discussion in Section \ref{sect:emgc}). 

\begin{table}
\caption{Frequency of clause-internal \textit{þá}/\textit{þar}/\textit{nú} in V2 declaratives in (M)IcePaHC saga narratives, 1210--1450}
\label{tab:internal_v2}
 \begin{tabularx}{.9\textwidth}{X rrrr}
  \lsptoprule
  &   \multicolumn{2}{c}{overt subject} &   \multicolumn{2}{c}{unexpressed subject}\\
   \cmidrule(lr){2-3}   \cmidrule(lr){4-5}
            & matrix & subordinate  & matrix & subordinate\\
  \midrule
  internal \textit{þá}/\textit{þar}/\textit{nú}  &  5,385   & 941     &   29       & 0 \\
  all V2                                         &  82,687 &  42,256   &   245    & 371 \\
 \% internal \textit{þá}/\textit{þar}/\textit{nú}  &  \textbf{6.5\%}  &  \textbf{2.2\%} & \textbf{11.8\%}  & \textbf{0.0\%}\\
  \lspbottomrule
 \end{tabularx}
\end{table}



\begin{table}
\caption{Frequency of clause-internal \textit{þá}/\textit{þar}/\textit{nú} in V1 declaratives in (M)IcePaHC saga narratives, 1210--1450}
\label{tab:internal_v1}
 \begin{tabularx}{.9\textwidth}{X rrrr}
  \lsptoprule
  &   \multicolumn{2}{c}{overt subject} &   \multicolumn{2}{c}{unexpressed subject}\\
   \cmidrule(lr){2-3}   \cmidrule(lr){4-5}
            & matrix & subordinate  & matrix & subordinate\\
  \midrule
  internal \textit{þá}/\textit{þar}/\textit{nú}  &  4,311   &   38   &   145     &  12\\
  all V1                                         & 15,135  &  827   &   422   & 341 \\
 \% internal \textit{þá}/\textit{þar}/\textit{nú}  &  \textbf{28.5\%}  &  \textbf{4.6\%} & \textbf{34.4\%}  & \textbf{3.5\%}\\
  \lspbottomrule
 \end{tabularx}
\end{table}

Strikingly, there are cases where more than one of the adverbs\is{adverb} (\textit{þá/þar/nú}) occur clause-internally and adjacent to one another, e.g.~(\ref{multiple-advs}).

\ea \label{multiple-advs}
\ea \label{multiple-clash1}
\gll Og um daginn gerðist mæði mikil á þeim. Þorgils var þó miklu hraustastur um allt. Tekur \textbf{þá} \textbf{nú} að þyrsta mjög. \\
and in day.\textsc{acc.def} became  exhaustion.\textsc{nom} great.\textsc{nom} to they.\textsc{dat} Þorgils.\textsc{nom} was though much.\textsc{dat} strongest.\textsc{nom} in everything.\textsc{acc} begins \textsc{þá} \textsc{nú} to get.thirsty very\\
\glt `And in the day a great exhaustion came upon them. Þorgils, though, was much the strongest. He begins to get very thirsty.'\\ \hfill [MIcePAHC: 1400, Floamanna.1313--1315]
\ex \label{multiple-clash2} 
\gll En er tíðindi þessi spyrjast fer Þorkell á fund Þórarins og leitar þangað ráða og meðferðar. Hann segir: ``Vera má \textbf{nú} \textbf{þá} að hún segi Ástríður að hann hafi eigi til engis risið á legginn.''\\
but when news.\textsc{nom} this.\textsc{nom} is.reported goes Þorkell.\textsc{nom} to meeting.\textsc{acc} Þórarinn.\textsc{gen} and seeks thither counsel.\textsc{gen} and intercession.\textsc{gen} he.\textsc{nom} says be may \textsc{nú} \textsc{þá} that she.\textsc{nom} says.\textsc{sbjv}  Ástríður.\textsc{nom} that he.\textsc{nom} had.\textsc{sbjv}  \textsc{neg} to nothing.\textsc{gen} risen to leg.\textsc{acc.def}\\
\glt `But when this news is reported, Þorkell goes to meet Þórarinn and seeks there counsel and intercession. He says: ``It may be that she says, Ástríður, that he didn't get up for nothing.'''\\ \hfill [MIcePaHC: 1350, Viga.538--539]
\ex
\gll Og þegar minnkar dirfð þeirra við þetta. Og nema þeir að þjóna felmtinum. Snerist \textbf{þar} \textbf{nú} skjótt staðfestin í hræslu.\\
and soon diminishes courage.\textsc{nom} they.\textsc{gen} with this.\textsc{acc} and take they.\textsc{nom} to serve fear.\textsc{dat.def} turns \textsc{þar} \textsc{nú} quickly steadfastness.\textsc{nom.def} in dread.\textsc{dat}\\
\glt `And soon their courage diminishes with this. And they take to serve their fear. The steadfastness quickly turns to dread.'\\ \hfill [IcePaHC: 1300, Alexander.901--903]
\ex \label{aux-4} 
\gll Nú skal=tu þangað fara og reyna vini þína {fyrir því að} nú vættir mig að menn séu forkunnar margir er lengi hefir dvalist.
Mun \textbf{nú} \textbf{þar} vera fjölmennt.\\
now shall=you.\textsc{nom} thither go and try friends.\textsc{acc} your.\textsc{acc} because now expect I.\textsc{acc} that men.\textsc{nom} are.\textsc{sbjv} remarkably many.\textsc{nom} \textsc{rel} long have stayed may \textsc{nú} \textsc{þar} be crowded.\textsc{nom}\\
\glt `Now you shall go there and try your friends because now I expect that there are remarkably many men who have stayed there for long. It may be crowded.' \hfill [MIcePaHC: 1300, Heidarviga.1166--1167]
\ex 
\gll Þorgils var forstjóri fyrir búi í Torgum, {þá er} Þórolfur var eigi heima; hafði Þorgils \textbf{þá} \textbf{þar} ráð\\
Þorgils.\textsc{nom} was leader.\textsc{nom} for farmstead.\textsc{dat} at Torgar.\textsc{dat} when Þórolfur.\textsc{nom} was \textsc{neg} home had Þorgils.\textsc{nom} \textsc{þá} \textsc{þar} authority.\textsc{acc}\\
\glt `Thorgils was the leader of the farmstead at Torgar when Þórolfur was not at home; Þorgils  had the authority there.' \hfill [MIcePaHC: 1250, Egils.515]
\ex
\gll Snorri góði kom þá að með flokk sinn. Var \textbf{þar} \textbf{þá} Skafti í liði með honom\dots\\
Snorri.\textsc{nom} chief.\textsc{nom} came \textsc{þá} \textsc{ptcl} with company.\textsc{acc} his.\textsc{acc} was \textsc{þar} \textsc{þá}  Skafti.\textsc{nom} in troop.\textsc{dat} with he.\textsc{dat}\\
\glt `Chief Snorri came with his company. Also in the troop with him was Skafti\dots' \hfill [MIcePaHC: 1300, Njals.9384--9385]
\z 
\z 

\noindent The examples in (\ref{multiple-advs}) illustrate that all combinations and orders of two adverbs\is{adverb} are attested in the (M)\isi{IcePaHC}\is{MIcePaHC} data (\textit{þá nú, nú þá, þar nú, nú þar, þá þar, þar þá}). The examples which feature \textit{þá nú} (lit.~`then now') and \textit{nú þá} (lit.~`now then'), in (\ref{multiple-clash1}) and (\ref{multiple-clash2}) respectively, are particularly striking, as the literal meanings of the two adverbs\is{adverb} present a temporal contradiction. I take the attestation of sequences of clause-internal adverbs\is{adverb} such as in (\ref{multiple-clash1}) and (\ref{multiple-clash2}) to indicate that such adverbs\is{adverb} have grammaticalised\is{grammaticalization} to some extent and undergone some semantic weakening/bleaching. Next, I investigate the information-structural properties of clauses with clause-internal \textit{þá}/\textit{þar}/\textit{nú}.

%%%%%%%
\subsection{Mapping syntax to information structure in the postfinite domain}
%%%%%%%

Any study of the syntax-information structure interface must first outline one's terminology and understanding of key information-structural concepts. 
In this paper, ``topic'' will be understood as roughly equivalent to ``\isi{aboutness topic}'', i.e.~the entity about which information is expressed (cf.~``sentence topic'', \citealp{reinhart1981pragmatics}). 
In this context, the diagnostic tests provided by \citet[165]{gotze2007information} can be used to identify the \isi{aboutness topic} of an utterance, see (\ref{aboutness-test}).

\ea \label{aboutness-test}  An NP X is the aboutness topic of a sentence S containing X if:\\
 \ea[]{ S would be a natural continuation to the announcement\\
         \textit{Let me tell you something about X}
}
\ex[]{ S would be a good answer to the question\\
         \textit{What about X?}
}
\ex[]{ S could be naturally transformed into the sentence\\
         \textit{Concerning X, S'}\\
         where S' differs from S only insofar as X has been replaced by a suitable pronoun
}
\z 
\z
 
Within the category of (aboutness) topic\is{aboutness topic}, I recognise different types of topic. One way of distinguishing between different subtypes of topic is to consider specifically the relation between a current topic and the topic of the immediately preceding utterance, i.e.~the topic transition (see e.g.~\citealp{danes1974functional} and ``Centering Theory'' in \citealp{groszetal95}). In this paper, I recognise three subtypes of topic in this way, as defined in (\ref{topic-types}).

\ea \label{topic-types}
\begin{enumerate}
\item \textsc{continuing topic}: current topic is co-referential with the topic of immediately preceding utterance
    \begin{itemize}
        \item \textit{Helen arrived in Utrecht. \textbf{She} went directly to her favourite café in the city centre.}
    \end{itemize} 
\item \textsc{shift topic}: current topic is not co-referential with the topic of previous utterance
    \begin{itemize}
        \item \textit{Helen gave a book to Julie for her birthday. \textbf{The book} was a new collection of recipes by Mary Berry.}
    \end{itemize}
\item \textsc{subsectional topic}: current topic is an element of a previously introduced set of entities
        \begin{itemize}
            \item \textit{Helen and Julie were discussing their plans for the weekend. \textbf{Julie} wanted to go for a bike ride.}
        \end{itemize}
\end{enumerate}
\z 

\noindent Type 1 in (\ref{topic-types}), \textsc{continuing topic},\is{continuing topic} equates to what is elsewhere defined as ``familiar topics'' (e.g.~\citealp{frascarelli2007types}). 
In Type 3, \textsc{subsectional topic},\is{subsectional topic} the current topic constitutes a subset of a set of entities introduced in the previous discourse (cf.~\citealp[86]{hendriks1996information}); in this sense, it is roughly equatable with the notion of ``contrastive topic'', i.e.~a topic which creates oppositional pairs with respect to other topics 
(cf.~\citealp{frascarelli2007types,krifka07}). 

Another information-structural notion which will be relevant for this paper is \textsc{information focus},\is{information focus} as defined in (\ref{inf-foc}) (cf.~\citealp{reinhart1981pragmatics,vallduvi92,gotze2007information}). 

\ea \label{inf-foc}
\textsc{information focus}: the part of the utterance which provides the new and missing information which is most relevant to the current discourse and serves to develop the discourse
\begin{itemize}
    \item Q:\textit{Where did you go in the Easter holidays?}\\
    A: \textit{We went \textbf{to Italy}.}
\end{itemize}
\z 

To investigate the correspondence between syntax and information structure with respect to \textit{þá}/\textit{þar}/\textit{nú}, I first extract matrix clauses with clause-internal \textit{þá}/\textit{þar}/ \textit{nú} and different types of nominative-marked subject in the postfinite domain (pronominal subject, proper name subject, nominal subject with definiteness marking, nominal subject without definiteness marking, quantified subject). I then examine the positional distribution of the different subject types with respect to \textit{þá}/\textit{þar}/\textit{nú}.\footnote{(M)IcePaHC also tags some non-nominative-marked arguments (typically experiencer arguments) as subjects, following the standard analysis of such arguments in Modern Icelandic as grammatical subjects (e.g.~\citealp{zmt1985}). I exclude clauses annotated with such non-nominative subjects from this investigation of Old Icelandic, as it remains an ongoing debate to what extent non-nominative experiencer arguments already qualify as grammatical subjects at this early stage of the language (e.g.~\citealp{barddal-eythorsson03,barthdal-tholly09,barddaletal2012,schaetzleetal15,schaetzle2018}).}
An important detail to note is that, in Old Icelandic, the distinction between nominals with definiteness marking and nominals without definiteness marking does not map directly onto the distinction between semantically definite and semantically indefinite nominals, since definiteness marking was not fully obligatory with common nouns in Old Icelandic \citep{leiss2000artikel,leiss2007covert,Borjarsetal16}.
Thus, while all of the definite-marked nominals are expected to be semantically definite, the nominals without definite marking will not all be semantically indefinite, but may also include some semantically definite nominals. Since semantic definiteness is not annotated in the (M)\isi{IcePaHC}\is{MIcePaHC} data, it is not possible to extract semantically definite versus indefinite nominals and so I rely instead on the presence/absence of definiteness marking as a proxy, combined with manual examination of the examples in context. 

The (M)\isi{IcePaHC}\is{MIcePaHC} data indicate that the five different types of subject show distinct and rather strong preferences in terms of their position with respect to clause-internal \textit{þá}/\textit{þar}/\textit{nú}, see Table \ref{tab:subjs}. Pronominal subjects, proper name subjects and definite-marked nominal subjects all favour the position to the left of \textit{þá}/\textit{þar}/\textit{nú}, while nominal subjects without definiteness marking and quantified subjects favour the position to the right of the \isi{adverb}. Next, I examine the relation between position and information structure for each of the five subject types.

\begin{table}
\caption{Position of different types of (nominative-marked) subject relative to clause-internal \textit{þá}/\textit{þar}/\textit{nú} in matrix declaratives in (M)IcePaHC saga narratives, 1210--1450}
\label{tab:subjs}
 \begin{tabularx}{.85\textwidth}{lrrrr}
  \lsptoprule
  &   \multicolumn{2}{c}{V--\textsc{subj}--\textit{þá}/\textit{þar}/\textit{nú}} &   \multicolumn{2}{c}{V--\textit{þá}/\textit{þar}/\textit{nú}--\textsc{subj}}\\
   \cmidrule(lr){2-3}   \cmidrule(lr){4-5}
            & \textit{n} & \%  & \textit{n} & \%\\
  \midrule
  pronominal subject  &  3,297  &  99.3\% &    22    &  0.7\%\\
  proper name subject &  859 &   75.6\%  &  277    & 24.4\% \\
  nominal subject\textsubscript{+\textsc{def}}  &  78 &  94.0\%   &   5  & 6.0 \% \\
  nominal subject\textsubscript{--\textsc{def}} &  275 & 21.9\%  &  979    &  78.1\% \\
  quantified subject & 10 & 2.5\% & 395 & 97.5\% \\
  \lspbottomrule
 \end{tabularx}
\end{table}


%%%%
\subsubsection{Pronominal subjects}
%%%%

With respect to pronominal subjects, those that occur to the left of \textit{þá}/\textit{þar}/\textit{nú} (the vast majority, see Table \ref{tab:subjs}) are generally continuing topics,\is{continuing topic} e.g.~(\ref{pro-cont-top}).

\ea \label{pro-cont-top}
\ea % CONT TOPIC
\gll Svo eignaðist Haraldur konungur Naumdælafylki og Hálogaland; setti [hann] \textbf{þar} menn yfir ríki sitt.\\
so acquired Haraldur.\textsc{nom} king.\textsc{nom} Naumdælir.county.\textsc{acc} and Hálogaland set he.\textsc{nom} \textsc{þar} men.\textsc{acc} over kingdom.\textsc{acc} his.\textsc{acc}\\
\glt `So King Haraldur acquired Naumdælir county and Hálogaland; he installed men over his kingdom.' \hfill [MIcePaHC: 1250, Egils.63]
\ex %CONT TOPIC
\gll Maður heitir Ólafur og er Höskuldsson og er [hann] \textbf{nú} frægstur maður einnhver.\\
man.\textsc{nom} is.called Ólafur.\textsc{nom} and is Höskuldsson.\textsc{nom} and is he.\textsc{nom} \textsc{nú} most.famous.\textsc{nom} man.\textsc{nom} anywhere\\
\glt `There is a man called Ólafur, son of Höskuldur and he is the most famous man around.' \hfill [MIcePaHC: 1275, Laxdaela.1467--1468]
\z 
\z 
 
Of the 22 pronominal subjects which occur to the right of \textit{þá}/\textit{þar}/\textit{nú}, 12 are instances of a \isi{dummy pronoun} which is co-referential with a clausal argument later in the sentence, e.g.~(\ref{pro-subj-clause}).\footnote{The dummy \textit{það} is annotated in the (M)\isi{IcePaHC}\is{MIcePaHC} data as a subject, and so is included here, even though it is widely assumed that it does not qualify as a grammatical subject in Icelandic (e.g. \citealp[480--481]{Thrainsson1979}; \citealp{Platzack1983existential}; \citealp{Maling1988}). Whatever the status of dummy \textit{það} in Old Icelandic, I still include it here as an instructive example of an element which is neither a topic nor a focus.}

\ea \label{pro-subj-clause}
\ea %ch61
    \gll Er \textbf{nú} [það]\textsubscript{\textit{i}} ætlan þeirra bræðra [að venda til hefnda við Helga Harðbeinsson {því að} hann veitti Bolla banasár]\textsubscript{\textit{i}}.\\
    is \textsc{nú} it.\textsc{nom} intention.\textsc{nom} they.\textsc{gen} brothers.\textsc{gen} to turn to avenge with Helgi.\textsc{dat}  Harðbeinsson.\textsc{dat} because he.\textsc{dat} dealt Bolli.\textsc{dat} death.blow.\textsc{acc}\\
    \glt `It is the intention of the brothers to turn to attack in vengeance Helgi Harðbeinsson because he dealt Bolli his fatal blow.'\\ \hfill [MIcePaHC: 1275, Laxdaela.4401]
    \ex \label{aux-3} %ch68
    \gll Verður \textbf{nú} [það]\textsubscript{\textit{i}} {af ráðið} [að brullaup skal vera að Helgafelli að sex vikum sumars]\textsubscript{\textit{i}}.\\
    becomes \textsc{nú} it.\textsc{nom} decided that wedding.\textsc{nom} shall be at Helgafell.\textsc{dat} at six weeks.\textsc{dat} summer.\textsc{gen}\\
    \glt `It is decided that the wedding shall be held at Helgafell in the sixth week of summer.' \hfill [MIcePaHC: 1275, Laxdaela.4913]
\z
\z

\noindent At the same time, dummy pronouns\is{dummy pronoun} which co-occur with a clausal argument can also occur to the left of \textit{þá}/\textit{þar}/\textit{nú}, e.g.~(\ref{dummy-pre}).

\ea \label{dummy-pre} 
 \gll Fór [það]\textsubscript{\textit{i}} \textbf{þá} víða um héruð [hversu ólíkir þeir þóttu flestum mönnum vera í sínum framferðum]\textsubscript{\textit{i}}. Var [það]\textsubscript{\textit{i}} \textbf{þá} þegar vitra manna mál [að hvergi mundi vera vænna til að leita en þar]\textsubscript{\textit{i}}\dots\\
 went it.\textsc{nom} \textsc{þá} widely around district.\textsc{acc} how unlike.\textsc{nom} they.\textsc{nom} seemed most.\textsc{dat} men.\textsc{dat} be in their.\textsc{dat} procedures.\textsc{dat} was it.\textsc{nom} \textsc{þá} soon wise.\textsc{gen} men.\textsc{gen} talk.\textsc{nom} that nowhere would be more.likely.\textsc{nom} to to search than there\\
 \glt `It went widely around the district how unlike most men they seemed to be in their procedures. It soon became the talk of wise men that nowhere would be more likely to search than there\dots'\\ \hfill [IcePaHC: 1210, Thorlakur.114--115]
\z 

\noindent The properties of \textit{það} in contexts where it co-occurs with a clausal argument in Old Icelandic were examined by \citet{booth2018,booth2019cataphora}. Booth found that, unlike in certain present-day Germanic languages like \ili{German}, where the parallel pronoun \textit{es} is often assumed to mark the clausal argument as a topic (see \citealp{Berman1998argument} and further references there),
in early Icelandic the clausal argument in such contexts expresses discourse-new,\is{discourse-new} non-topical information (\citealp[chapter 7]{booth2018}, \citealp{booth2019cataphora}).\footnote{In contrast to the traditional view on \ili{German} \textit{es}, \citet[44--45]{fuss2023historical} point out that, while it can co-occur with a clausal argument which denotes a given situation, it can also occur with a clausal argument which introduces a new situation.\is{discourse-new} This suggests that the relationship between such pronouns and information structure in Germanic is more complex than standardly assumed.} Thus, the \isi{dummy pronoun} in Old Icelandic does not merit a topical analysis.

Of the remaining 10 examples of pronominal subjects which occur to the right of \textit{þá}/\textit{þar}/\textit{nú}, 9 involve misannotations and one involves Stylistic Fronting 
 (cf.~\citealp{Maling1990}) of a nonfinite verb form to the prefield, shown here in (\ref{pro-subj-sf}).
    
\ea \label{pro-subj-sf}
    \gll Þorkell mælti: ``Liðið er \textbf{nú} það. Gangi konur út úr búðinni og viljum vér leita mannsins.'' \\
    Þorkell.\textsc{nom} said passed is \textsc{nú} that.\textsc{nom} goes.\textsc{sbjv} women.\textsc{nom} out of booth.\textsc{dat.def} and wish we.\textsc{nom} search man.\textsc{gen.def}\\
    \glt `Þorkell said: ``That is passed. Let the women go out of the booth and we want to search for the man.'''\\ \hfill [MIcePaHC: 1350, Hallfredarmodruvallabok.1079--1081]
\z 

\noindent It remains an open debate as to whether Stylistic Fronting in Modern Icelandic is purely syntactic, without pragmatic effects (cf.~\citealp{Maling1990,holmberg2000scandinavian}), or whether it contributes information-structural effects such as contrastive \isi{focus} (cf.~\citealp{hrafnbjargarson2004stylistic}) or backgrounding (cf.~\citealp{egerland2013fronting}). With respect to Old Icelandic, the picture is even more unclear, given that the phenomenon has not been studied in detail; in a preliminary survey, \citet[chapter 4]{booth2018} showed that the syntactic constraints on Stylistic Fronting were much less restricted in early stages of Icelandic, and that Stylistic Fronting today may represent a remnant of a more general fronting process. As such, it is hard to say anything specific about this example with respect to Stylistic Fronting and the position of the pronominal subject. I simply note that it is another exception to the general preference for pronominal subjects to occur to the left of \textit{þá}/\textit{þar}/\textit{nú}.

%%%%
\subsubsection{Proper name subjects}
%%%%

In the (M)\isi{IcePaHC}\is{MIcePaHC} data, proper name subjects show an overall preference for the position to the left of \textit{þá}/\textit{þar}/\textit{nú}, but nevertheless occur to the right of \textit{þá}/\textit{þar}/\textit{nú} in roughly a quarter of cases (see Table \ref{tab:subjs}). 
Those proper name subjects which occur to the left of \textit{þá}/\textit{þar}/\textit{nú} are generally either continuing topics,\is{continuing topic} e.g.~(\ref{name-cont-top}), or shift topics\is{shift topic}, e.g.~(\ref{name-shift-top}).

\ea 
\ea  \label{name-cont-top}
    Preceding context: \textit{Ólafur asked Gunnar to take care of himself and said that he had many ill-wishers ``because you are now the most famous man in all the land''. Gunnar thanked him for the gifts and counsel and rode home.}\\
    \gll Situr [Gunnar] \textbf{nú} heima nokkura hríð og er kyrrt.\\
    sits Gunnar.\textsc{nom} \textsc{nú} home some.\textsc{acc} time.\textsc{acc} and is quiet.\textsc{nom}\\
   \glt  `Gunnar sits at home for some time and things are quiet.' \\ \hfill [MIcePaHC: 1300, Njals.14809]
\ex \label{name-shift-top}
    Preceding context: \textit{The King replies and says that he will never again accept hospitality from Þórólfur.}\\
    \gll Gekk [Þórólfur] \textbf{þá} {í brott} og bjóst síðan til heimferðar.\\
    went Þórólfur.\textsc{nom} \textsc{þá} away and prepared then to home.journey.\textsc{gen}\\
    \glt `Þórólfur went away and prepared to travel home.'\\ \hfill [MIcePaHC: 1250, Egils.624--625]
\z 
\z 

By contrast, proper name subjects which occur to the right of \textit{þá}/\textit{þar}/\textit{nú} show different information-structural properties and fall into two broad types. Firstly, many of the examples involve a proper name subject which is in \isi{information focus},\is{focus} e.g.~(\ref{name-pres-foc}).

\ea \label{name-pres-foc}
\ea
    Preceding context: \textit{Reginbald and Hercules go into battle against one another.}\\
    \gll Samson hans merkismaður bar djarflega fram merkið og hjó á báðar hendur og stökk allt undan honum en {í mót} honum kom \textbf{nú} [Balabán] með merki Erkiles\dots \\ 
    Samson.\textsc{nom} he.\textsc{gen} standardbearer.\textsc{gen} carried boldly forth standard.\textsc{acc.def} and struck at both.\textsc{acc} hands.\textsc{acc} and sprang everything.\textsc{nom} under he.\textsc{dat} and towards he.\textsc{dat} came \textsc{nú} Balabán.\textsc{nom} with standard.\textsc{acc} Hercules.\textsc{gen}\\
    \glt `Samson, his standardbearer, carried the standard boldly forth and struck at both hands and everything sprang from under him and towards him came Balabán with Hercules' standard\dots' \\ \hfill [IcePaHC: 1450, Vilhjalmur.81.1629--1632]
\ex 
    Preceding context: \textit{And after that Gunnar, Karl, Svarthöfði and Ögmundur set off for home. There were two Eastmen besides Gunnar, and Karl was with the sixth man.}\\
   \gll  Og er þeir komu ofan á hólana fyrir sunnan Brimnessá og til dælar þeirrar er ofan er og suður er frá ánni þá spretta þar upp fyrir þeim þrír tigir manna og var \textbf{þar} [Ljótólfur].\\
    and when they.\textsc{nom} came up on hills.\textsc{acc.def} before south Brimnessá.\textsc{acc} and to valley.\textsc{gen} that.\textsc{gen} \textsc{rel} up is and south is from river.\textsc{dat.def} then spring there up before they.\textsc{def} three.\textsc{nom} tens.\textsc{nom} men.\textsc{gen} and was \textsc{þar} Ljótólfur.\textsc{nom}\\
    \glt `And when they came to the top of the hills to south of Brimnessá and to the valley that is above and south of the river, there sprang up before them thirty of man and there was Ljótólfur.'\\ \hfill [MIcePaHC: 1450, Svarfdaela.3915--3916]
\ex 
    Preceding context: \textit{Vémundur, Þorkell, Þórodður and  Hánefur engage in battle with Hrói and his entourage.}\\
    \gll Og nú kemur \textbf{þar} [Áskell] við þrjá tigu manna og vildi sætta þá.\\
    and now comes \textsc{þar} Áskell.\textsc{nom} with three.\textsc{acc} tens.\textsc{acc} men.\textsc{gen} and wished reconcile they.\textsc{acc}\\
   \glt  `And now comes Áskell with thirty men and wanted to reconcile them.' \hfill [MIcePaHC: 1400, Reykdaela.2478--2479]
\z 
\z 

Secondly, proper name subjects which occur to the right of \textit{þá}/\textit{þar}/\textit{nú} can be subsectional topics\is{subsectional topic}.
This is illustrated in the continuous passage in (\ref{name-contr-top}), which relays the activities of Hallur and Þórólfur, and three proper name subjects which are subsectional topics\is{subsectional topic} occur to the right of \textit{þá}/\textit{þar}/\textit{nú}.

\ea \label{name-contr-top}  
Preceding context: \textit{One evening Hallur and Þórólfur came ashore and were to divide their catch. Hallur wanted to both divide and choose, because he thought himself to be the better man. Þórólfur didn't want to lose his share and did not hold back in his words. They exchanged some words and each insisted on their way.}\\
    \gll Þrífur \textbf{þá} [Hallur] upp höggjárn er lá hjá honum og vill færa í höfuð Þórólfi. Nú hlaupa menn í milli þeirra og stöðva Hall en hann var hinn óðasti og gat þó engu á leið komið að því sinni og ekki varð fengi þeirra skipt. Réðst \textbf{nú} [Þórólfur] {á brott} um kveldið en Hallur tók einn upp fang það er þeir áttu báðir {því að} þá kenndi að ríkismunar. Fær \textbf{nú} [Hallur] sér mann í stað Þórólfs á skipið.\\
    grasps \textsc{þá} Hallur.\textsc{nom} up hewing.iron.\textsc{acc} \textsc{rel} lay by he.\textsc{dat} and wishes strike to head.\textsc{acc} Þórólfur.\textsc{dat} now run men.\textsc{nom} in between they.\textsc{gen} and restrain Hallur.\textsc{acc} and he.\textsc{nom} was \textsc{def.nom} most.stubborn.\textsc{nom} and could though nothing.\textsc{dat} on way.\textsc{acc} come at this.\textsc{dat} time.\textsc{dat} and \textsc{neg} became catch.\textsc{nom} they.\textsc{gen} divided went \textsc{nú} Þórólfur.\textsc{nom} away in evening.\textsc{acc.def} but Hallur.\textsc{nom} took one.\textsc{acc} up catch.\textsc{acc} \textsc{that.acc} \textsc{rel} they.\textsc{nom} had both.\textsc{nom} because then recognised to power.difference.\textsc{gen} gets \textsc{nú} Hallur.\textsc{nom} himself.\textsc{dat} man.\textsc{acc} in place.\textsc{acc} Þórólfur.\textsc{gen} on ship.\textsc{acc.def}\\
    \glt `Hallur (not Þórólfur) picks up the hewing iron which lay by him and wishes to strike at Þórólfur's head. Now men run between them and restrain Hallur and he was the most stubborn man but could not get his way this time and their catch was not divided. Þórólfur (not Hallur) left in the evening but Hallur himself took up the catch which they had both had because he was recognised to be more powerful. Hallur (not Þórólfur) gets himself another man to replace Þórólfur on the ship.' \\ \hfill [MIcePaHC: 1275, Laxdaela.6721--6727]
\z 

\noindent Further examples of proper name subjects which are subsectional topics\is{subsectional topic} occurring to the right of \textit{þá}/\textit{þar}/\textit{nú} are shown in (\ref{name-contr-top-2}).

\ea \label{name-contr-top-2}
\ea
    \gll Eftir það skiljast þeir konungur og Kjartan með miklum kærleik. Gengur \textbf{þá} [Kjartan] út á skip.\\
     after that.\textsc{acc} part they.\textsc{nom} king.\textsc{nom} and Kjartan.\textsc{nom} with great.\textsc{dat} love.\textsc{dat} goes \textsc{þá} Kjartan.\textsc{nom} out to ship.\textsc{acc}\\
    \glt `After that, the King and Kjartan part with great love. Kjartan (not the King) goes out to the ship.' \hfill [MIcePaHC: 1275, Laxdaela.3167--3168]
\ex 
    \gll Síðan gerðist Þórólfur handgenginn konungi ok gekk þar í hirðlög, en Berðlu-Kári ok Eyvindur lambi, sonr hans, fóru suðr með skip þat, er Þórólfr hafði norðr haft. Fór \textbf{þá} [Kári] heim til búa sinna ok þeir Eyvindur báðir.\\
    then became Þórólfr.\textsc{nom} retained.\textsc{nom} king.\textsc{dat} and went \textsc{þar} in company.laws.\textsc{acc} but Berðlu-Kári.\textsc{nom} and Eyvindr.\textsc{nom} Lamb.\textsc{nom} son.\textsc{nom} he.\textsc{gen} went south with ship.\textsc{acc} that.\textsc{acc} \textsc{rel} Þórólfur.\textsc{nom} had north had  went \textsc{þá} Kári.\textsc{nom} home to homestead.\textsc{gen} his.\textsc{gen} and they.\textsc{nom} Eyvindur.\textsc{nom} both.\textsc{nom}\\
    \glt `Then Þórólfur became a retainer to the king and joined  his followers by law but Berðlu-Kári and Eyvindur Lamb, his son, went south with the ship which Þórólfur had had in the north. Kári went back to his homestead, and Eyvindur too.' \hfill [MIcePaHC: 1250, Egils.5432--5434]
\z 
\z 

%%%%
\subsubsection{Nominal subjects with definite marking}
%%%%

Definite-marked nominal subjects occur overwhelmingly to the left of \textit{þá}/\textit{þar}/\textit{nú} in the (M)\isi{IcePaHC}\is{MIcePaHC} data, see Table \ref{tab:subjs}, and are generally shift topics\is{shift topic}, e.g.~(\ref{def-subj-fam-top}).

\ea \label{def-subj-fam-top}
\ea
Previous context: \textit{But Ásgerðr, who was married to Þórólfr Skalla-Grímsson, was then with Arinbirn, his cousin. She and Þórólf had one young daughter, whose name was Þórdís,\dots}\\
\gll {\dots}ok var [mærin] \textbf{þar} með móður sinni.\\
and was maiden.\textsc{nom.def} \textsc{þar} with mother.\textsc{dat} her.\textsc{dat}\\
\glt `{\dots}and the maiden was with her mother.'\\ \hfill [IcePaHC: 1250, Thetubrot.29]
\ex
Previous context: \textit{Þorgils awakes and thinks what he will have there, and now he remembers that he gave Þór a certain calf a long time ago. Þorgils tells this to Þóreyja\dots}\\
\gll  {\dots}og var [þetta] \textbf{þá} gamall uxi\dots \\
and was this.\textsc{nom} \textsc{þá} old.\textsc{nom} ox.\textsc{nom}\\
\glt `{\dots}and this (calf) was an old ox\dots' \hfill [MIcePaHC: 1400, Floamanna.1025]
\z 
\z

The 5 examples of definite-marked nominal subjects which occur to the right of \textit{þá}/\textit{þar}/\textit{nú} 
involve proximal demonstratives\is{demonstrative} which occur as the subject of ``happen'' predicates, e.g.~(\ref{happen-dem}), or proximal demonstratives\is{demonstrative} which co-occur with and are co-referential with a clausal argument later in the sentence, e.g.~(\ref{dem-clause1}) and (\ref{dem-clause2}). 

\ea \label{this-dummy-pre}
\ea  \label{happen-dem}
        \gll Og nú fyrir sakir vinsælda hans og hann var sínum mönnum ástfólginn, þá játtu þeir þessu, að konungur skyldi ráða, og fór \textbf{nú} [þetta] fram.\\
        and now for sakes.\textsc{acc} popularity.\textsc{gen} he.\textsc{gen} and he.\textsc{nom} was his.\textsc{dat} men.\textsc{dat} beloved.\textsc{nom} then agreed they.\textsc{nom} this.\textsc{dat} that king.\textsc{nom} should decide and went \textsc{nú} this.\textsc{nom} forth\\
        \glt `And now because of his popularity and because he was beloved of his men, they agreed that the king should decide, and this was done.'\\ \hfill [IcePaHC: 1260, Jomsvikingar.61--62]
\ex \label{dem-clause1}
        \gll Réðst \textbf{nú} [þetta]\textsubscript{\textit{i}} [að þeir Þorsteinn og Lambi skulu ráðast með Þorgísli til ferðar]\textsubscript{\textit{i}}\dots\\
        decided \textsc{nú} this.\textsc{nom} that they.\textsc{nom} Þorsteinn.\textsc{nom} and Lambi.\textsc{nom} should set-forth with Þorgísl.\textsc{dat} to journey.\textsc{gen}\\
        \glt `It is decided that Þorsteinn and Lambi should set off with Þorgísl on the journey\dots' \hfill [MIcePaHC: 1275, Laxdaela.4435]
\ex \label{dem-clause2} 
        \gll Fór \textbf{nú} [þetta]\textsubscript{\textit{i}} fram [að Ólöf var gefin Þórði]\textsubscript{\textit{i}}\dots \\ 
        went \textsc{nú} this.\textsc{nom} forth that Ólöf.\textsc{nom} was given Þórður.\textsc{dat}\\
        \glt `Things proceeded such that Ólöf was given to Þórður\dots'\\ \hfill [MIcePaHC: 1400, Thordar.1882]
\z 
\z 

\noindent In these contexts, the proximal \isi{demonstrative} appears to be an alternative \isi{dummy pronoun}, alongside \textit{það}, cf.~(\ref{pro-subj-clause}) and (\ref{dummy-pre}) above. In addition to examples like (\ref{this-dummy-pre}), there are also examples in the (M)\isi{IcePaHC}\is{MIcePaHC} data where \textit{þetta} as a \isi{dummy pronoun} precedes \textit{þá}/\textit{þar}/\textit{nú}, e.g. (\ref{thetta-da}). This mirrors the behaviour of the \isi{dummy pronoun} \textit{það}, which can also appear to the left or right of \textit{þá}/\textit{þar}/\textit{nú}, cf.~(\ref{pro-subj-clause}) and (\ref{dummy-pre}) above. 

\ea \label{thetta-da}
\gll Fór [þetta]\textsubscript{\textit{i}} \textbf{þá} fram [að grið voru sett með mönnum þar til að hver kæmi til síns heima]\textsubscript{\textit{i}}.\\
went this \textsc{þá} forth that settlements.\textsc{nom} were established with men.\textsc{dat} there to that each.\textsc{nom} came.\textsc{sbjv} to his.\textsc{gen} home.\textsc{gen}\\
\glt `Things proceeded such that settlements were established with the men so that each would go to his home.' \hfill [MIcePaHC: 1300, Eyrbyggja.2596]
\z 


%%%%
\subsubsection{Nominal subjects without definite marking}
%%%%

Nominal subjects without definite marking generally prefer the position to the right of \textit{þá}/\textit{þar}/\textit{nú}, see Table \ref{tab:subjs}, where they are typically in \isi{information focus},\is{focus} e.g.~(\ref{indef-subj-right}).

\ea \label{indef-subj-right}
\ea 
    Previous context: \textit{But as soon as she saw the church at Skálaholt, she felt herself become more comfortable than she had previously ever been in her pain. She came there some nights before the church day\dots}\\
    \gll En þar var \textbf{þá} [fjölmenni mikið]\dots\\
    but there was \textsc{þá} crowd.\textsc{nom} great.\textsc{nom}\\
    \glt `But there was a great crowd there\dots' \hfill [IcePaHC: 1210, Jartein.680]
\ex
    Previous context: \textit{Þorleifur shot from a bow and he was very likely to wound but nothing came of Jósteinn's shots and Þorleifur was able to hurt him. Deacon Þórður Símonarson held the shield for Þorleifur.}\\
    \gll Var \textbf{þar} [harður bardagi]\dots\\
    was \textsc{þar} hard.\textsc{nom} battle.\textsc{nom}\\
    \glt `There was a hard battle\dots' \hfill [IcePaHC: 1250, Sturlunga.391.123]
\z 
\z

\largerpage[2]
However, there are also examples where such a nominal subject without definite marking occurs to the left of the \isi{adverb}, see Table \ref{tab:subjs}. Such examples fall into two broad types. Firstly, there are contexts where the subject is an occupation noun which is semantically definite though not morphologically marked for definiteness. Such subjects can be continuing topics,\is{continuing topic} e.g.~(\ref{occup-cont}) or shift topics\is{shift topic}, e.g.~(\ref{occup-shift}), but they can also be subsectional topics\is{subsectional topic}, e.g.~(\ref{occup-subsect}). In particular, the examples in  (\ref{occup-subsect}) show that subsectional topics\is{subsectional topic} can appear to the left of  \textit{þá}/\textit{þar}/\textit{nú} as well as to the right, cf.~(\ref{name-contr-top}) and (\ref{name-contr-top-2}) above.

\ea \label{subj-occup}
\ea \label{occup-cont} %Ch9 %cont topic
    Preceding context: \textit{The king went north to Trondheim in the autumn. Then Þórólfur asks for leave to travel north to Halogaland to visit the gifts he had received in the summer from Bárður, his kinsman. The king grants that and gives him a message and token that Þórólfur should have all the property that Bárður gave him, letting it be known that the gift was made on the King's advice and that he wants it to be left so.}
    \gll Gerir [konungur] \textbf{þá} Þórólf lendan mann\dots\\
    makes king.\textsc{nom} \textsc{þá} Þórólfur.\textsc{acc} landed.\textsc{acc} man.\textsc{acc}\\
    \glt `The king makes Þórólfur a landholder\dots' \hfill [MIcePaHC: 1250, Egils.353]
\ex \label{occup-shift}
    Preceding context: \textit{Grettir fights with Skeggi, and Grettir strikes his head with an axe.}\\
    \gll Féll [húskarl] \textbf{þá} dauður til jarðar.\\
    fell man.servant.\textsc{nom} \textsc{þá} dead.\textsc{nom} to ground.\textsc{gen}\\
    \glt `The man servant (= Skeggi) fell dead to the ground.'\\ \hfill [IcePaHC: 1310, Grettir.273]
\z 
\z

\ea \label{occup-subsect}
\ea  
    \gll Og nú ganga þeir {í braut} af stefnunni, konungur og jarl, og nú eftir það var jarl \textbf{þar} með honum nokkora hríð í mikilli sæmd.\\
    and now go they.\textsc{nom} away from meeting.\textsc{dat.def} king.\textsc{nom} and earl.\textsc{nom} and now after that.\textsc{acc} was earl.\textsc{nom} \textsc{þar} with he.\textsc{dat} some.\textsc{acc} time.\textsc{acc} in great.\textsc{dat} honour.\textsc{dat}\\
    \glt `And now they depart from the meeting, the King and the Earl, and now after that the Earl was with him for some time in great honour.'\\ \hfill [IcePaHC: 1260, Jomsvikingar.299--300]
\clearpage
\ex  
    Preceding context: \textit{A farmer and Þormóður are in conversation with one another.}\\
    \gll Snýr [bóndi] \textbf{þá} utar eftir hlöðunni\dots\\
    turns farmer.\textsc{nom} \textsc{þá} outside after storehouse.\textsc{dat.def}\\
    \glt `The farmer turns back towards the storehouse\dots \\ \hfill [MIcePaHC: 1305, Fostbraedra.3301]
\z 
\z 

The second broad group of subjects which are annotated as nominals which lack definiteness marking in the (M)\isi{IcePaHC}\is{MIcePaHC} data and which can occur to the left of \textit{þá}/\textit{þar}/\textit{nú} are impersonal pronouns\is{impersonal pronoun}, e.g.~(\ref{imp-pre}). 


\ea \label{imp-pre}
\ea 
\gll Riðu [menn] \textbf{þá} heim af þinginu\dots \\
       rode men.\textsc{nom} \textsc{þá} home from assembly.\textsc{dat.def}\\
\glt `People rode home from the assembly\dots' \hfill [IcePaHC: 1310, Grettir.1411]
\ex  
Previous context: \textit{One autumn, Urðarköttur went out each evening and wouldn't come in before the night was late.}\\
\gll Vita [menn] \textbf{nú} eigi hvað hann gerir.\\
knew men.\textsc{nom} \textsc{nú} \textsc{neg} what he.\textsc{nom} does\\
\glt `People didn't know what he was doing.'\\ \hfill [MIcePaHC: 1350, Finnboga.310]
\z 
\z

\noindent However, this is not the only position available to impersonal pronouns\is{impersonal pronoun}; they can also occur to the right of \textit{þá}/\textit{þar}/\textit{nú}, e.g.~(\ref{imp-post}). 

\ea \label{imp-post}
\ea  
\gll {\dots}og fóru \textbf{þá} [menn] meðal þeirra og voru þá grið sett.\\
and went \textsc{þá} men.\textsc{nom} among they.\textsc{gen} and was \textsc{þá} truces.\textsc{nom} set\\
\glt `and people went among them and there were set truces.' \\ \hfill [IcePaHC: 1250, Sturlunga.436.1665--1666]
\ex 
\gll Gjöra \textbf{nú} [menn] mikinn róm að máli Barða\dots\\
make \textsc{nú} men.\textsc{nom} great.\textsc{acc} noise.\textsc{acc} at case.\textsc{dat} Barði.\textsc{gen}\\
\glt `People make a lot of noise about Barði's case\dots' \\ \hfill [MIcePaHC: 1300, Heidarviga.1080]
\z 
\z 

%%%%
\subsubsection{Quantified subjects}
%%%%

Quantified subjects, which cannot generally be topical (e.g.~\citealp{frey2004medial}), overwhelmingly favour the position to the right of \textit{þá}/\textit{þar}/\textit{nú}, see Table \ref{tab:subjs} and the examples in (\ref{quant-post}).

\ea \label{quant-post}
\ea \label{aux-2} 
\gll ``Það munu \textbf{þá} [sumir menn] mæla,'' segir Höskuldur, ``að eg flýi þaðan fyrir
hræðslu sakir og vil eg það eigi.''\\ 
that.\textsc{acc} will \textsc{þá} some.\textsc{nom} men.\textsc{nom} say says Höskuldur.\textsc{nom} that I.\textsc{nom}  flee.\textsc{sbjv} thence for fear.\textsc{gen} sakes.\textsc{acc} and will I.\textsc{nom} that.\textsc{acc} \textsc{neg}\\
\glt ‘``Some men will say,'' says Höskuldur, ``that I am fleeing from there for fear's sake, and I don't want that.'''  \hfill [MIcePaHC: 1300, Njals.6602]
\ex 
\gll Eru \textbf{nú} [fáir slíkir menn] í yðvarri ætt sem Bolli er.\\
are \textsc{nú} few.\textsc{nom} such.\textsc{nom} men.\textsc{nom} in your.\textsc{dat} generation as Bolli.\textsc{nom} is.\\
\glt `There are few men of your generation who are such as Bolli is.' \\ \hfill [MIcePaHC: 1275, Laxdaela.10072]
\z 
\z

\largerpage
\noindent Nevertheless, there are 10 examples in the (M)\isi{IcePaHC}\is{MIcePaHC} data where a quantified subject precedes \textit{þá}/\textit{þar}/\textit{nú} in the postfinite domain, see Table \ref{tab:subjs}. On closer inspection, these either involve misannotations or a quantified subject which broadly translates as `people' (\isi{impersonal pronoun}), e.g.~(\ref{quant-pre}).\footnote{I do not rule out the possibility that quantified subjects in the different positions relative to \textit{þá}/\textit{þar}/\textit{nú} have potentially distinct information-structural characteristics, since quantified noun phrases can have partitive or existential readings (see e.g.~\citealp[79]{jonsson2002} on Modern Icelandic). However, the number of examples with a quantified subject to the left of \textit{þá}/\textit{þar}/\textit{nú} (which are not misannotated) are too few to draw any meaningful conclusions here.}


\ea \label{quant-pre}
\ea \label{aux-1}
Preceding context: \textit{The King calls a meeting at Eyrar and urges the local people to convert. Local farmers prepare to do battle with the King rather than convert. The King scares the farmers and they surrender to him.}\\
\gll {\dots}og var [margt fólk] \textbf{þá} skírt.\\
and was much.\textsc{nom} people.\textsc{nom} \textsc{þá} baptised\\
\glt `{\dots}and many people were baptised' \hfill [MIcePaHC: 1275, Laxdaela.2890]
\ex 
Preceding context: \textit{There was a man called Þórólfur who lived under Spákonufell. He had a fortuneteller, Þórdís, who was mentioned earlier. }
\gll Þóttust margir \textbf{þar} traust mikið eiga er hún var.\\
seemed many.\textsc{nom} \textsc{þar} confidence.\textsc{nom} much.\textsc{nom} have where she.\textsc{nom} was\\
\glt `Many seemed to have much confidence wherever she was.' \\ \hfill [MIcePaHC: 1350, Kormaks.1636]
\z 
\z 


%%%%
\subsubsection{Summary}
%%%%

In sum, on the basis of the (M)\isi{IcePaHC}\is{MIcePaHC} data, the correspondence between syntax and information structure in the postfinite domain can be schematised as in (\ref{is-syntax-schema}). 
(\ref{is-syntax-schema}) shows a mixed picture, whereby certain information-structure roles are associated with a single position (e.g.~information foci,\is{information focus}\is{focus} right of \textit{þá/þar/nú}; continuing/shift topics,\is{continuing topic}\is{shift topic} left of \textit{þá/þar/nú}) whereas others, such as subsectional topics\is{subsectional topic}, can appear freely in either position. 
The finding for subsectional topics\is{subsectional topic} is particularly striking, as they can occur both in the position associated with continuing/shift topics,\is{continuing topic}\is{shift topic} and also in the position associated with \isi{information focus}.\is{focus} One explanation for this behaviour could be that subsectional topics\is{subsectional topic}, which roughly equate to the notion of contrastive topics, are often assumed to involve a combination of topic and \isi{focus} \citep[267]{krifka07}. The finding that both dummy pronouns\is{dummy pronoun} and impersonal pronouns\is{impersonal pronoun} are not restricted in terms of their position with respect to \textit{þá/þar/nú} is also interesting, since these elements are neither straightforwardly analysable as topic nor \isi{focus} in information-structural terms.

\ea \label{is-syntax-schema}
\begin{footnotesize}
\begin{tabular}{cccccccc}
& \rdelim\{{7}{0pt}  &    &   \ldelim\}{7}{0pt}  & & \rdelim\{{7}{0pt}  & information focus & \ldelim\}{7}{0pt} \\
 & &   continuing topic    &   &  \\
&   &   shift topic  &   &  \\
%& &   promoted topic  &   &  \\
        V\textsubscript{finite} --     &        &    &      &   -- \textit{þá/þar/nú} --   &     \\
                             %\hline 
     % V\textsubscript{finite} -- & &        -- \textit{þá/þar/nú} -- & \\   
                            % \hline 
                       &      & subsectional topic & &  & & subsectional topic \\
& & dummy pronoun & & & & dummy pronoun\\
                    &         & impersonal pronoun & &  & & impersonal pronoun\\
\end{tabular}
\end{footnotesize}
\z 

Overall, clause-internal  \textit{þá/þar/nú} serve an \isi{information-structural watershed} function in the postfinite domain in Old Icelandic (cf.~\citealp{krivonosov1977deutsche,grosz2016information} on modern \ili{German}; \citealp{vanKem-Los2006,vanKem2008balance,vanKemenade2009discourse,vanKem2011syntax,vanKem2020discourse} on \ili{Old English}). However, this effect appears to be rather subtle in Old Icelandic. Whereas in modern \ili{German} clause-internal discourse particles have been argued to partition\is{discourse partitioner} topic and \isi{focus} (cf. \REF{mod-germ} above), and in \ili{Old English} clause-internal adverbs\is{adverb} have been argued to separate \isi{presupposition} from \isi{focus} (cf. \REF{vankem-schem} above), in Old Icelandic the watershed\is{information-structural watershed} function appears to apply specifically to \isi{information focus}\is{focus} and particular subtypes of topic. In the next section, I explore how these subtle correspondences between position and information structure can be modelled within Lexical Functional Grammar. 


%%%%%%%%%%%%%%%%%%%%%%%%%%%%%%%%%%
\section{An LFG approach}\label{sect:lfg}
%%%%%%%%%%%%%%%%%%%%%%%%%%%%%%%%%%
%%%%%%%%%
\subsection{The LFG architecture}\label{subsect:lfg-basics}
%%%%%%%%%

%GENERAL
\isi{Lexical Functional Grammar} (LFG\is{Lexical Functional Grammar}, \citealp{bresnan-kaplan82,Bresnan2015lexical,dalrymple2019oxford}) is a ``declarative'' approach to grammar (cf.~\citealp{levine2006declarative,sells2021view}). This means that LFG\is{Lexical Functional Grammar} does not commit to any procedural mechanism for deriving linguistic representations, and that all information in the model is  simultaneously present in parallel (``model-theoretic'' approach, cf.~\citealp{pullum2001distinction}). Different types of linguistic information are represented at independent dimensions which are related to each other within an overall projection architecture, see Figure \ref{fig:proj-arch}. Each dimension differs in terms of its formal representation and must satisfy certain constraints. The core components of syntactic representation are c(onstituent)-structure, which captures information about category and constituency, and f(unctional)-structure, which captures abstract functional information. A third dimension which is relevant to this paper is i(nformation)-structure.

\begin{figure}
     \includegraphics[width=\textwidth]{figures/proj.png}
    \caption{Parallel projection architecture of LFG  \citep[369]{asudeh2006direct}}
    \label{fig:proj-arch}
\end{figure}

LFG\is{Lexical Functional Grammar}'s c-structure is represented as a tree diagram, while f-structure and i-structure are represented as attribute-value matrices. An example c-structure/f-structure/i-structure triad for Modern English is shown in (\ref{eng-aux}). The abstract functional information represented at f-structure includes grammatical functions (\textsc{gf}s), e.g.~\textsc{subj}(ect) and \textsc{obj}(ect),
as well as grammatical features, e.g.~\textsc{tense}\is{tense} and \textsc{def}(initeness). A special type of functional feature is \textsc{pred}, which is a pointer into the semantics of a predicate, takes a semantic form as its value, and captures the argument(s) (if any) a predicate requires in terms of grammatical function. i-structure represents information-structural information such as the discourse functions\is{discourse function} \textsc{topic} and \textsc{focus}.\is{focus}\footnote{The \textsc{pred-fn} notation in the i-structure in (\ref{eng-aux}), as opposed to the standard \textsc{pred} feature from f-structure, indicates that instead of projecting the entire \textsc{pred} feature, including its argument structure, only the basic meaning of the \textsc{pred} is projected to i-structure (cf.~\citealp{kaplan1996grammar,king1997focus,buttjabeen16}).} Since i-structure is an underdeveloped part of the LFG\is{Lexical Functional Grammar} architecture, it remains an open question as to what precise discourse function\is{discourse function} should be captured here (see e.g.~\citealp[chapter 10]{dalrymple2019oxford}, \citealp{zaenenforthcoming-handbook-IS}). In this paper, I will assume that specific subtypes of topic (e.g.~continuing topics/\isi{shift topics}/\isi{subsectional topics}) and \isi{focus} (e.g.~information foci)\is{information focus}\is{focus} are represented as distinct discourse functions\is{discourse function} at i-structure.

\ea 
\label{eng-aux} 
\begin{small}
Q: What will Maria eat?\\
A: Maria will eat the focaccia. 
\end{small}

\vspace{2ex}

\begin{minipage}{0.49\textwidth}
\begin{small}
c-structure:\\
\qtreecenterfalse
        \Tree [.IP  \qroof{Maria}.($\uparrow$\textsc{subj})=$\downarrow$\\DP 
           [.$\uparrow$=$\downarrow$\\I$'$ 
                   [.$\uparrow$=$\downarrow$\\I will ]
              [.$\uparrow$=$\downarrow$\\{VP } [.$\uparrow$=$\downarrow$\\{V} eat ] 
              \qroof{the focaccia}.($\uparrow$\textsc{obj})=$\downarrow$\\DP 
             ]
            ] ]


\end{small}
\end{minipage}
\begin{minipage}{0.4\textwidth}
\begin{small}
f-structure: \\
\avm{ [ pred & `eat {\textless}subj, obj{\textgreater}' \\
tense & fut \\
subj & [ pred & `maria' ] \\
obj &  [ pred & `focaccia' \\
         def & + ]          ]
}
\end{small}

\vspace{4ex}

\begin{small}
i-structure:\\
\avm{ [ topic & [ pred-fn & `maria' ]\\
        focus & [ pred-fn & `focaccia'] ]
}
\end{small}
\end{minipage}

\z 

Within LFG's\is{Lexical Functional Grammar} projection architecture, see Figure \ref{fig:proj-arch}, all linguistic dimensions are present in parallel and a number of functions map between the various dimensions. With respect to the correspondence between c-structure and f-structure, 
I uncontroversially assume that this is formally handled via the correspondence function $\phi$, whereby c-structure nodes are related to f-structures \citep{Bresnan2015lexical,dalrymple2019oxford}.
In a language like English where grammatical functions (e.g.~subject, object) are structurally specified, functional annotation on specific c-structure positions exclusively associate a position with a specific grammatical function at f-structure, see (\ref{eng-aux}). 
In the c-structure tree in (\ref{eng-aux}), $\downarrow$ and $\uparrow$ are metavariables over f-structure variables and serve to relate every node in the c-structure to its corresponding f-structure. $\downarrow$ denotes the f-structure corresponding to that node itself, and $\uparrow$ denotes the f-structure corresponding to that node's mother node. As such, SpecIP bears an annotation which relates that node to the \textsc{subj} function of the maximal f-structure; the annotation on the complement of V relates that node to the \textsc{obj} function.\footnote{Additionally, two c-structure nodes in a mother-daughter relation may correspond to the same f-structure, in which case they are annotated as $\uparrow$=$\downarrow$: 
this indicates that the functional information associated with a given node is the same as the functional information associated with that node's mother node.} 

In a language like English, discourse functions\is{discourse function} such as \textsc{topic} and \textsc{focus}\is{focus} are not generally associated with fixed positions but are expressed via e.g.~prosody. However, many languages exhibit ``\isi{discourse configurationality}'' (cf.~\citealp{vilkuna1989free,Kiss1995discourse}), where discourse functions\is{discourse function} are associated with specific positions. In LFG\is{Lexical Functional Grammar}, this can be modelled by annotations which exclusively associate certain c-structure positions with discourse functions\is{discourse function} at i-structure, e.g.~(\ref{is-tree-top}); the arrows annotated with $\iota$ indicate projection to i-structure, and \textsc{gf} stands for any grammatical function (e.g.~\textsc{subj, obj}).\footnote{In this paper, I follow \citet{butt-king97} in assuming a model where i-structure projects from c-structure as defined by the function $\iota$, which can be considered as a parallel to the $\phi$ function which relates c-structure nodes to f-structures. This is in line with \citet{asudeh2006direct}, see again Figure \ref{fig:proj-arch}.
For a different proposal, see \citet{dalrymple2019oxford} who present a model where i-structure instead projects from s-structure.}

\begin{small}
\ea  
  \label{is-tree-top} \Tree 
        [.CP ($\uparrow$\textsubscript{$\iota$}\textsc{topic})=$\downarrow$\textsubscript{$\iota$}\\($\uparrow$\textsc{gf})=$\downarrow$\\XP
        [.IP ($\uparrow$\textsubscript{$\iota$}\textsc{focus})=$\downarrow$\textsubscript{$\iota$}\\($\uparrow$\textsc{gf})=$\downarrow$\\XP
[.I\1 {...} {...} ] ] ]
\z 
\end{small}

%%%%%%%%%
\subsection{Modelling the information-structural watershed}
%%%%%%%%%
\largerpage

%DC
The relevant question in the context of this paper is how the \isi{information-structural watershed} function of  clause-internal \textit{þar/þá/nú} in Old Icelandic can be modelled within LFG\is{Lexical Functional Grammar}. For Old Icelandic, I model the postfinite domain in terms of a flat structure within I', see the c-structure in (\ref{dc-c-structure}), following previous accounts of Icelandic clause structure within LFG\is{Lexical Functional Grammar}, including Old Icelandic (\citealp{sells2001structure,Sells2005,booth-schaetzle:2019-cr,booth_revisiting_2021}).\footnote{In (\ref{dc-c-structure}) the adverb position bears the annotation $\downarrow$$\in$($\uparrow$\textsc{adj}), which specifies that the f-structure contributed by the \isi{adverb} is a member of a set of values for the grammatical function \textsc{adj}(unct) in the maximal f-structure; this allows for multiple adjuncts in a clause.} 
One possibility for capturing the information-structural partitioning \is{discourse partitioner} of the postfinite domain, and the watershed\is{information-structural watershed} function of clause-internal \textit{þar/þá/nú}, is to assume \isi{discourse configurationality} (see Section \ref{subsect:lfg-basics}), whereby the position to the left of \textit{þar/þá/nú} is uniquely associated with \textsc{topic} at i-structure, and the position to the right of \textit{þar/þá/nú} with \textsc{focus}.\is{focus} This possibility is shown in (\ref{dc-c-structure}).\footnote{A similar discourse-configurational approach is employed by \citet{mahowald2011lfg} in LFG\is{Lexical Functional Grammar} in relation to the \ili{Old English} postfinite domain and \textit{þa/þonne}, based on the findings by \citet{vanKem-Los2006} (see Section \ref{sect:emgc}).}

\begin{small}
\ea 
\label{dc-c-structure}
 
    % \qtreecenterfalse
    \Tree [.IP  
    [.{...} ]
    [.$\uparrow$=$\downarrow$\\I' $\uparrow$=$\downarrow$\\I 
($\uparrow$\textsubscript{$\iota$}\textsc{topic})=$\downarrow$\textsubscript{$\iota$}\\($\uparrow$\textsc{gf})=$\downarrow$\\XP
  \qroof{\textit{þar/þá/nú}}.$\downarrow$$\in$($\uparrow$\textsc{adj})\\\textbf{AdvP}  
($\uparrow$\textsubscript{$\iota$}\textsc{focus})=$\downarrow$\textsubscript{$\iota$}\\($\uparrow$\textsc{gf})=$\downarrow$\\XP
    [.$\uparrow$=$\downarrow$\\V ]  ] ] 
\z
\end{small}


However, capturing the syntax-information structure correspondence in the postfinite domain in terms of dedicated topic and \isi{focus} positions on either side of \textit{þar/þá/nú} is unsatisfactory in many ways, given the findings from the (M)\isi{IcePaHC}\is{MIcePaHC} data outlined in Section \ref{sect:data}. Firstly, the position to the left of \textit{þar/þá/nú} is not an exclusive topic position, since it can also host elements which cannot be construed as topics, such as dummy pronouns\is{dummy pronoun} and impersonal pronouns\is{impersonal pronoun}, see (\ref{is-syntax-schema}) above. Secondly, the position to the right of \textit{þar/þá/nú} is not an exclusive \isi{focus} domain, but can also host subsectional topics\is{subsectional topic}, and again dummy pronouns\is{dummy pronoun} and impersonal pronouns\is{impersonal pronoun} which cannot either be construed as foci.\is{focus} As such, an alternative to straightforward \isi{discourse configurationality} as in (\ref{dc-c-structure}) appears to be motivated in this context.


The alternative I suggest in this paper takes advantage of the fact that LFG\is{Lexical Functional Grammar} provides for alternative ways of expressing ordering dependencies, besides strictly c-structure relations. One such alternative which has been developed and applied in various contexts is f(unctional)-precedence, which essentially constrains the ordering of c-structure constituents in terms of the f-structures they map onto, (cf.~\citealp{bresnan1984bound,kaplan1987three,kaplan1989long}). f-precedence is formally defined as in (\ref{f-pred}).

\ea
\label{f-pred}
         f-precedence \\
    \textit{f}\textsubscript{1} f-precedes \textit{f}\textsubscript{2} if and only if all c-structure nodes corresponding to \textit{f}\textsubscript{1} precede all c-structure notes corresponding to \textit{f}\textsubscript{2}\\
    \hfill \citep[ex.~(11)]{dalrymple2001weak}
\z 

\noindent For example, \citet[229--230]{zaenen1995formal} employ f-precedence  in \isi{verb-final} West Germanic varieties (e.g.~Zürich \ili{German}), where verbs in infinitival constructions cannot precede their nominal arguments, but where the verb and its nominal arguments do not necessarily have to be adjacent. In such contexts, \citet[230]{zaenen1995formal} employ the c-structure rule in (\ref{f-pred-zuri}),  where the V node is associated with an f-precedence constraint which states that the f-structure associated with the V node cannot f-precede its mother node's nominal grammatical functions (\textsc{ngf}), i.e.~the nominal arguments of the particular verb.

\ea \label{f-pred-zuri}
\begin{tabular}{cccc}
  V'  &  $\longrightarrow$ & V\\
    &  & $\downarrow$$\nless$\textsubscript{\textit{f}}($\uparrow$ \textsc{ngf}) \\
\end{tabular}
\hfill  \citep[230]{zaenen1995formal}
\z 

In order to model the mapping between syntax and information structure in the Old Icelandic postfinite domain, and the specific watershed\is{information-structural watershed} function of clause-internal \textit{þar/þá/nú}, I suggest a potential i-structure parallel to f-precedence, which I call i-precedence. i-precedence essentially allows one to constrain the ordering of c-structure constituents in terms of the i-structures they map onto, and can be defined as in (\ref{i-pred}) (cf.~the definition for f-precedence in (\ref{f-pred}) above).

\ea 
\label{i-pred} i-precedence\\
    \textit{$\iota$}\textsubscript{1} i-precedes \textit{$\iota$}\textsubscript{2} if and only if all c-structure nodes corresponding to \textit{$\iota$}\textsubscript{1} precede all c-structure notes corresponding to \textit{$\iota$}\textsubscript{2}
\z 


\noindent I specify three i-precedence constraints on the position which hosts  \textit{þar/þá/nú}, see (\ref{my-tree}). The first i-precedence constraint -- $\downarrow$\textsubscript{$\iota$}$\nless$\textsubscript{$\iota$}($\uparrow$\textsubscript{$\iota$}\textsc{cont-topic}) -- states that the i-structure associated with the daughter node (i.e.~associated with \textit{þar/þá/nú}) does not i-precede the mother node's \textsc{cont}(inuing)-\textsc{topic} at i-structure, i.e.~the clause's \textsc{cont}(inuing)-\textsc{topic}. This prevents continuing topics\is{continuing topic} from occurring in the postfinite domain to the right of \textit{þar/þá/nú}, as is the generalisation borne out from the (M)\isi{IcePaHC}\is{MIcePaHC} data, see (\ref{is-syntax-schema}) above.
Likewise, the second i-precedence constraint --  
$\downarrow$\textsubscript{$\iota$}$\nless$\textsubscript{$\iota$}($\uparrow$\textsubscript{$\iota$}\textsc{shift-topic}) -- prevents shift topics\is{shift topic} from occurring to the right of \textit{þar/þá/nú}, another generalisation from the (M)\isi{IcePaHC}\is{MIcePaHC} data, see again (\ref{is-syntax-schema}).
The third i-precedence constraint on the clause-internal adverb position -- ($\uparrow$\textsubscript{$\iota$}\textsc{inf-focus})$\nless$\textsubscript{$\iota$}$\downarrow$\textsubscript{$\iota$} -- 
states that the mother node's \textsc{inf}(ormation)-\textsc{focus}\is{focus} at i-structure, i.e.~the \textsc{inf}(ormation)-\textsc{focus}\is{focus} of the overall clause, cannot i-precede the daughter node (i.e.~the position of \textit{þar/þá/nú}). This thus prevents information-foci\is{information focus}\is{focus} from occurring in the postfinite domain to the left of \textit{þar/þá/nú}, as indicated by the (M)\isi{IcePaHC}\is{MIcePaHC} data, see (\ref{is-syntax-schema}).


\begin{footnotesize}
\ea \label{my-tree}
    \qtreecenterfalse
    \Tree [.IP  
    [.{...} ]
    [.$\uparrow$=$\downarrow$\\I' $\uparrow$=$\downarrow$\\I 
    [.{($\uparrow$\textsc{gf})= $\downarrow$\\XP} ]
\qroof{\textit{þar/þá/nú}}.$\downarrow$\textsubscript{$\iota$}$\nless$\textsubscript{$\iota$}($\uparrow$\textsubscript{$\iota$}\textsc{cont-topic})\\$\downarrow$\textsubscript{$\iota$}$\nless$\textsubscript{$\iota$}($\uparrow$\textsubscript{$\iota$}\textsc{shift-topic})\\($\uparrow$\textsubscript{$\iota$}\textsc{inf-focus})$\nless$\textsubscript{$\iota$}$\downarrow$\textsubscript{$\iota$}\\$\downarrow$$\in$($\uparrow$\textsc{adj})\\\textbf{AdvP}   
    [.{($\uparrow$\textsc{gf})= $\downarrow$\\XP}   ]
    [.{\dots} ]  ] ] 
\z 
\end{footnotesize}

i-precedence, as a parallel to \isi{LFG}'s\is{Lexical Functional Grammar} f-precedence, thus allows one to model the subtle \isi{information-structural watershed} function of \textit{þar/þá/nú}, which is more nuanced than standard \isi{discourse configurationality}. The three i-structure constraints specify ordering restrictions specifically for continuing topics,\is{continuing topic} shift topics\is{shift topic} and information foci\is{information focus}\is{focus} relative to \textit{þar/þá/nú}, but leave the position of other types of topic (e.g.~\isi{subsectional}/contrastive topics), and constituents which are neither topics nor foci\is{focus} (e.g.~dummy pronouns\is{dummy pronoun} and impersonal pronouns\is{impersonal pronoun}) underspecified with respect to \textit{þar/þá/nú}, as borne out in the (M)\isi{IcePaHC}\is{MIcePaHC} data, see (\ref{is-syntax-schema}).
In addition, these three i-precedence constraints are all relative to the I' node in terms of their domain of application (cf.~\citealp{zaenen1995formal}, ``relativized f-precedence''). The constraints thus specify orderings in the postfinite domain, but crucially still allow for various types of topic or foci\is{focus} to appear in the clause-initial prefinite position (prefield), as is attested in Old Icelandic (cf.~the findings in \citealp{booth_revisiting_2021}).

%%%%%%%%%%%%%%%%%%%%%%%%%%%%%%%%%%
\section{Diachronic developments}\label{sect:dia}
%%%%%%%%%%%%%%%%%%%%%%%%%%%%%%%%%%

Although clause-internal \textit{þar/þá/nú} have been acknowledged to be a dominant feature of Old Icelandic in the previous literature (e.g.~\citealp[71]{Faarlund1990}), their status in later stages of Icelandic has not been addressed in specific detail.
For Modern Icelandic, \citet[38--39]{Thrainsson2007} states that place and time adverbs,\is{temporal/spatial adverb}\is{temporal adverb} including \textit{þar/þá/nú}, typically occur either in the prefield or in a position after the verb phrase, but cannot occur in a clause-internal position immediately after the finite verb, providing the data in (\ref{mod-ice-advs}).

\ea \label{mod-ice-advs} Modern Icelandic:
\ea
\gll \textbf{Þar/þá/nú} hefur Jón lokið þessu.\\
there/then/now has Jón.\textsc{nom} finished this.\textsc{dat}\\
\glt `There/then/now Jón has finished this.'
\ex
\gll Jón hefur lokið þessu \textbf{þar/þá/nú}.\\
Jón.\textsc{nom} has finished this.\textsc{dat} there/then/now\\
\glt `Jón has finished this there/then/now.'
\ex[*]
{\label{mod-out} 
\gll Jón hefur \textbf{þar/þá/nú} lokið þessu.\\
Jón.\textsc{nom} has there/then/now finished this.\textsc{dat} \\
\glt Intended: `Jón has there/then/now finished this.'} \hfill \citep[39]{Thrainsson2007}
\z 
\z 

However, \citet[40]{Thrainsson2007} also states that, when used as a ``discourse particle'', presumably bleached of its temporal semantics, \textit{nú} typically occurs clause-internally and cannot occur in the prefield, providing the data in (\ref{nu-ptcl}). \citet{Thrainsson2007} does not, however, discuss the status of \textit{þar} or \textit{þá} in this context.

\ea \label{nu-ptcl} Modern Icelandic:
\ea
\gll Jón hefur \textbf{nú} lokið þessu.\\
Jón.\textsc{nom} has \textsc{ptcl} finished this.\textsc{dat}\\
\glt `Well, Jón has finished this.'
\ex[*]
{\gll \textbf{Nú} hefur Jón lokið þessu.\\
\textsc{ptcl} has Jón.\textsc{nom} finished this.\textsc{dat}\\
\glt Intended: `Well, Jón has finished this.' \hfill \citep[40]{Thrainsson2007}}
\z
\z

\noindent At the same time, the dedicated study of \textit{nú} as a particle in Modern Icelandic spontaneous conversation by \citet{hilmisdottir2010present} presents a rather different picture. \citet[276--277]{hilmisdottir2010present} observes that, while \textit{nú} as a particle can occur clause-internally (see \REF{nu-ptcl}), it in fact shows a very strong tendency to occur in the prefield (93\% of all instances, compared to just 6\% clause-internally), apparently going against the data from \citet[40]{Thrainsson2007}.

When one examines the full diachronic sweep of the data in \isi{IcePaHC}, which spans 1150--2008 (all text types), it is clear that the relative frequency of clause-internal \textit{þar/þá/nú} decreases over the course of the history of Icelandic, though it remains a possibility in the latest period (1901--2008), see Table \ref{tab:rel_freq_dia}. Note that the period 1751--1900 contains the text \textit{Hellismanna saga},\is{Icelandic saga} which alone provides 153 examples with clause-internal \textit{þar/þá/nú}. This text is a ``newly written'' 19th century saga\is{Icelandic saga} which aims to reproduce an earlier saga \isi{style} and is a known outlier in this subsection of the corpus with respect to syntactic properties (cf.~\citealp{schaetzle-booth2019}). The data for the period 1551--1749 also comes with an important caveat, since this subsection of \isi{IcePaHC} contains a lot of translation texts and for the most part text types other than narratives, which dominate in the other periods (cf.~\citealp[chapter 1]{booth2018}).
These caveats aside, the key finding in Table \ref{tab:rel_freq_dia} is that, by the latest period (1901--2008), the frequency of clause-internal \textit{þar/þá/nú} has dramatically declined compared to the previous centuries.

    \begin{table}
         \setlength{\tabcolsep}{1ex}
        \centering
       
        \begin{tabular}{lrrr}
                \lsptoprule
        & \multicolumn{2}{c}{clause-internal \textit{þar/þá/nú}} & all matrix declaratives  \\
          \cmidrule(lr){2-3} 
      &  \textit{n} & \% &   \\
      \midrule
      1150--1350 & 1,385 & 10.5\% & 13,321 \\
      1351--1550 & 1,272 & 11.4\% &  11,205 \\
      1551--1750 & 561 & 6.5\% & 8,590  \\
      1751--1900 & 747 & 8.6\% &  8,645 \\
      1901--2008 & \textbf{249} & \textbf{2.9\%} & 8522 \\
              \lspbottomrule
        \end{tabular}
        \caption{Frequency of clause-internal \textit{þar/þá/nú} across all matrix declaratives in IcePaHC (all text types, 1150--2008)}\
         \label{tab:rel_freq_dia}
    \end{table}{}    



Taking a narrower view, \citet[260]{booth2018} observed the high frequency of clause-internal \textit{þar/þá/nú} specifically in V1\is{verb-first} \isi{presentational constructions} in early stages of Icelandic, citing the examples in (\ref{v1-pres-thesis}) (cf.~also the corpus study in \citealp{booth-schaetzle:2019-cr}).


\ea Early Icelandic (\citealp[260]{booth2018}):
\label{v1-pres-thesis}
\ea 
\gll Voru \textbf{þar} tvö skip í búnaði.\\
were \textsc{þar} two.\textsc{nom} ships.\textsc{nom} in preparation.\textsc{dat}\\
\glt `There were two ships in the preparations.' \\ \hfill [IcePaHC: 1250, Sturlunga.408.710]
\ex 
\gll Verður \textbf{nú} mannfall ógurligt.\\
becomes \textsc{nú} man.loss.\textsc{nom} terrible.\textsc{nom}\\
\glt `There comes to be a terrible loss of men.' \\ \hfill [IcePaHC: 1480, Jarlmann.381]
\ex 
\gll Kom \textbf{þá} veður {á móti} þeim.\\
came \textsc{þá} wind.\textsc{nom} towards they.\textsc{dat}\\
\glt `There came wind towards them.' \\ \hfill (Supplementary manual data: [1250, Eirik.9.2])
\z 
\z 

\noindent Presentational constructions\is{presentational constructions} are a well known example of a clause which lacks a topic-comment articulation (``thetic'', cf.~\citealp{sasse2013thetic}). In such clauses, which lack a topic, the clause-internal \isi{adverb} occurs immediately after the finite verb, and before the subject which is in \isi{information focus}.\is{focus} In this sense, \textit{þar/þá/nú} serve to ``close off'' the topic domain in such contexts, and thus can be viewed as markers of a topicless clause.

Strikingly, this very function of signalling a topicless construction in \isi{presentational constructions} has also been attributed to the prefield \isi{expletive} \textit{það} (e.g. \citealp[29]{Rognvaldsson1990}, \citealp{Sells2005}, \citealp[145]{Sigurdsson2007}), which represents a later development in the language (\citealp{Hroarsdottir1998,Rognvaldsson2002,booth2018,booth2019cataphora,booth2020expletives}), e.g.~(\ref{expl-pres}).

\ea \label{expl-pres}
\ea 19th Century Icelandic:\\
\gll \textbf{Það} rísu upp tveir nýir kaupmenn.\\
\textsc{expl} stoop up two.\textsc{nom} new.\textsc{nom} merchants.\textsc{nom}\\
\glt `There stood up two new merchants.' \hfill [IcePaHC: 1888, Grimur.126]
\ex Modern Icelandic: \\
\gll \textbf{Það} komu nokkrir vopnaðir menn af næstu bæjum\dots\\
\textsc{expl} came some.\textsc{nom} armed.\textsc{nom} men.\textsc{nom} from next.\textsc{dat} farms.\textsc{dat}\\
\glt `There came some armed men from the nearby farms\dots' \\ \hfill [IcePaHC: 2008, Ofsi.634]
\z 
\z 

\noindent Another investigation of the full diachronic sweep of \isi{IcePaHC} reveals that, in \isi{presentational constructions}, clause-internal  \textit{þar/þá/nú} are virtually in complementary distribution with the prefield \isi{expletive} \textit{það}, see Table \ref{tab-dia}. Once the \isi{expletive} becomes established in presentationals\is{presentational constructions}, clause-internal \textit{þar/þá/nú} are scarcely attested (\textit{n}=3). I interpret these results to suggest that, in presentationals\is{presentational constructions}, the role of marking a thetic, topicless construction shifts diachronically from clause-internal \textit{þar/þá/nú} to clause-initial (prefield) \textit{það}.

  \begin{table}
         \setlength{\tabcolsep}{1ex}
        \centering
        \begin{tabular}{lrrrrrr}
        \lsptoprule
        & \multicolumn{2}{c}{V-\textit{þá}/\textit{þar}/\textit{nú}-\textsc{subj}} & \multicolumn{2}{c}{\textsc{expl}-V-\textsc{subj}}  & \multicolumn{2}{c}{\textsc{expl}-V-\textit{þá}/\textit{þar}/\textit{nú}-\textsc{subj}}  \\
          \cmidrule(lr){2-3} \cmidrule(lr){4-5}  \cmidrule(lr){6-7}   
      &  \textit{n} & \% & \textit{n} &  \%  & \textit{n} &  \%   \\
        \midrule
      1150--1350 & 22 & 100.0\% & 0 & 0.0\% & 0 & 0.0\% \\
      1351--1550 & 22 & 81.5\% & 5 & 18.5\% & 0 & 0.0\% \\
      1551--1749 & 24 & 85.7\% & 4 & 14.3\% & 0 & 0.0\% \\
      1750--1900 & 17 & 32.1\% & 35 & 66.0\% & 1 & 1.9\% \\
      1901--2008 & 1 & 1.1\% & 86 & 96.6\% & 2 & 2.2\% \\
       \midrule
     all &  86 & 39.3\% & 130 &  59.4\%  & \textbf{3} &  \textbf{1.4\%} \\
          \lspbottomrule
        \end{tabular}
        \caption{Distribution of presentational constructions over three configurations in IcePaHC (all text types, 1150--2008)}
        \label{tab-dia}
    \end{table}{}

This diachronic finding is further supported by Modern Icelandic data provided and discussed by \citet[78--79]{rognvaldsson1982}, shown here in (\ref{rogn}). For presentationals\is{presentational constructions}, a V2\is{verb-second} clause with \isi{expletive} \textit{það} in the prefield is possible (\ref{rogn-expl}), as is a V1\is{verb-first} clause with clause-internal \textit{þá} (\ref{rogn-v1-adv}), provided other pragmatic conditions for V1 declaratives\is{verb-initial declaratives}\is{verb-first} are fulfilled. However, a V1\is{verb-first} clause without the clause-internal \isi{adverb} is ruled out (\ref{rogn-v1}), indicating that either a clause-internal \isi{adverb} (e.g.~\textit{þar/þá/nú}) or the prefield \isi{expletive} is required to render a thetic clause well-formed.  

\ea \label{rogn} Modern Icelandic:
\ea[]{\label{rogn-expl} \gll \textbf{Það} kóm einhver maður til mín.\\
\textsc{expl} came some.\textsc{nom} man.\textsc{nom} to I.\textsc{gen}\\
\glt `Some man came to me.'
}
\ex[]{\label{rogn-v1-adv}\gll Kóm \textbf{þá} einhver maður til mín.\\
came \textsc{þá} some.\textsc{nom} man.\textsc{nom} to I.\textsc{gen}\\
\glt `Some man came to me.'
}
\ex[*]{\label{rogn-v1}\gll Kóm einhver maður til mín.\\
came some.\textsc{nom} man.\textsc{nom} to I.\textsc{gen}\\
\glt Intended: `Some man came to me.' \hfill \citep[79]{rognvaldsson1982}
}
\z 
\z 

I suggest that the rise of the prefield \isi{expletive} \textit{það} in presentationals\is{presentational constructions} and attendant decrease in V1\is{verb-first} presentationals\is{presentational constructions} with clause-internal \textit{þar/þá/nú}, as indicated in Table \ref{tab-dia}, reflects a broader change whereby Icelandic gradually shifts its structural specification of discourse functions\is{discourse function} from the postfinite domain to the prefield over time. 
Indeed, the fact that clause-internal \textit{þar/þá/nú} has decreased in frequency in matrix declaratives in general, beyond specifically presentationals\is{presentational constructions} (see Table \ref{tab:rel_freq_dia}), indicates a wider change in the mapping between clause structure and information structure, likely connected with recent observations that topics are becoming increasingly firmly associated with the prefield in Icelandic (see previous work by \citealt{bsbb:2017} and \citealt{booth-beck20200jhs}).


%%%%%%%%%%%%%%%%%%%%%%%%%%%%%%%%%%
\section{Conclusion}\label{sect:conc}
%%%%%%%%%%%%%%%%%%%%%%%%%%%%%%%%%%


In this paper I have shown that, in Old \isi{Icelandic saga} narratives, clause-internal adverbs\is{adverb} (specifically \textit{þar/þá/nú}) act as an \isi{information-structural watershed} in the postfinite domain, contributing to the structural encoding of different types of information-structure roles in a highly specialised way. This phenomenon is in line with other Germanic varieties (e.g.~\ili{Old English}, Modern \ili{German}), where similar effects have been observed in the postfinite domain, and suggests that the role of adverbs\is{adverb} in the structural management of information structure is a relatively common phenomenon, at least within Germanic, and one which deserves more attention both empirically and theoretically. The need for further crosslinguistic research in this area is especially strong given the findings for Old Icelandic presented here, 
which indicate a complex system whereby the watershed\is{information-structural watershed} adverbs\is{adverb} are sensitive to different subtypes of topics and foci,\is{focus} and at the same time irrelevant for content which is neither topical or in \isi{focus}. I argued that this nuanced and complex system calls for an alternative model other than straightforward \isi{discourse configurationality} with designated topic and \isi{focus} positions, and showed that the LFG\is{Lexical Functional Grammar} architecture and in particular the notion of i-precedence can offer such an alternative.
% SAGA narratives as great
More broadly, the paper serves as a showcase for the rich empirical and theoretical insights which can be gained from the investigation of the morphosyntax-discourse interface in historical narrative texts, and in particular Old \isi{Icelandic saga} narratives, which offer ample opportunities for future research in this area.




%%%%%%%%%%%%%%%%%%%%%%%%%%%%%%%%%%
\section*{Abbreviations}
%%%%%%%%%%%%%%%%%%%%%%%%%%%%%%%%%%
\begin{tabularx}{.5\textwidth}{@{}lQ@{}}
\textsc{acc} & accusative \\
\textsc{adj} & adjunct\\
\textsc{cont-topic}  & continuing topic\\
\textsc{dat} & dative \\
\textsc{def} & definite\\
\textsc{expl} & expletive\\
\textsc{gen} & genitive \\
\textsc{gf} & grammatical function\\
\textsc{inf-focus} & information focus\\
%IcePaHC & Icelandic Parsed Historical Corpus\\
\textsc{neg} & negation \\
\end{tabularx}%
\begin{tabularx}{.5\textwidth}{@{}lQ@{}}
%MIcePaHC & Machine-Parsed Icelandic Parsed Historical Corpus\\
\textsc{ngf} & nominal grammatical function\\
\textsc{nom} & nominative \\
\textsc{ptcl} & particle \\
\textsc{rel} & relativiser\\
\textsc{sbjv} & subjunctive\\
\textsc{subj} & subject\\
V & verb\\
V1 & verb-initial\\
V2 & verb-second\\
\\
\end{tabularx}

%%%%%%%%%%%%%%%%%%%%%%%%%%%%%%%%%%
\DeclareRobustCommand{\disambiguate}[3]{#1}
\sloppy
\printbibliography[heading=subbibliography,notkeyword=this]
\end{document}
