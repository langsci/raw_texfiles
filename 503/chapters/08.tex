\documentclass[output=paper,colorlinks,citecolor=brown]{langscibook}
\ChapterDOI{10.5281/zenodo.15689133}
\author{Anna Cichosz\affiliation{University of Łódź}}
\title{Verb-initial declaratives in Old English prose: A functional analysis}

\abstract{The aim of this study is to conduct a comprehensive, corpus-based analysis of verb-initial main declarative clauses in Old English prose records in order to determine their discourse functions and to establish what linguistic and/or extralinguistic factors may be responsible for the highly uneven distribution of the structure. The analysis shows that Old English verb-initial declaratives perform two main functions: narrative and descriptive, with the latter restricted to just a handful of early compositions, and the former attested as a low-frequency item in the vast majority of texts. Both patterns are analysed in context and it turns out that verb-initial declaratives with and without an overt subject share many properties and functions, and that verb-initial declaratives with personal pronoun subjects are rare, which has some important implications for the interpretation of the results.}

\begin{document}
\maketitle 
%\shorttitlerunninghead{}%%use this for an abridged title in the page headers


\section{Introduction}\label{sec:cichosz:1}

Verb-initial (V1\is{verb-first}) main declarative clauses\is{verb-initial declaratives}\is{verb-first} with an overt subject are a relatively well-attested structure present in numerous Germanic languages (\citealt{Walkden2014}: 94; \citealt{RingeTaylor2014}: 408). In the Old Germanic context, V1\is{verb-first} was quite frequent in negated declaratives \citep{Hopper1975}, whereas non-negated V1\is{verb-first} declaratives\is{verb-initial declaratives} were a marked pattern \citep{Lehmann2005}, usually associated with some sort of emphasis or “dramatic force” \citep[96]{Smith1971}. The special function of V1\is{verb-first}\is{verb-first} is directly reflected in the term “\isi{narrative inversion}”, which is usually equated with a non-negated V1\is{verb-first} declarative\is{verb-initial declaratives} containing an overt subject, with the underlying assumption that the motivation behind the pattern was always to push the \isi{narration} forward. This study confirms the initial findings of \citet{Cichosz2022}, proving that in \ili{Old English} (OE) this assumption is not fully confirmed by the data and V1\is{verb-first}\is{verb-first} cannot be exclusively associated with a narrative function because there is an alternative structure, relatively frequent in early prose texts, where V1\is{verb-first} declaratives\is{verb-initial declaratives} are used to provide additional information about a known referent.

As far as other Old Germanic languages are concerned, V1\is{verb-first} declaratives\is{verb-initial declaratives} are for example attested in \ili{Old High German} (OHG), where they are reported to appear in presentational\is{presentational constructions} and existential structures, with mutative verbs (i.e. verbs of change, including verbs of movement), passive or passive-like constructions, verbs of saying, and negated clauses \citep{Axel2007}. Their frequency in OHG\il{Old High German} is quite robust (with the maximum at 14\% in the OHG\il{Old High German} Tatian Gospel Harmony, cf. \citealt{CichoszEtAl2016}: 127), while in Old Icelandic (OI) they constitute 18\% of all main declaratives \citep{booth2018}. Most of these are classified as narrative \citep[107]{booth2018}, and \isi{narrative inversion}, in the context of OI, is defined as “a particularly common type of V1\is{verb-first}” (\citealt{booth-beck20200jhs}: 41) alongside presentational\is{presentational constructions} structures, impersonal constructions and clauses without an overt subject. By contrast, in \ili{Old English} (OE) the frequency of V1\is{verb-first} declaratives\is{verb-initial declaratives} is generally on the low side (ca. 2\% in non-negated main clauses, cf. \citealt{Cichosz2020}: 365), and the structure, if used, is reported to introduce a new storyline or mark some sort of transition in the narrative structure of the text (\citealt{Mitchell1985}: $§$3933; \citealt{Los2000}; \citealt{Ohkado2004}; \citealt{Calle-MartínMiranda-García2010}). Its use is quite regular in OE\il{Old English} poetry, but the pattern is very unevenly distributed in the prose records, with the frequency of V1\is{verb-first} declaratives\is{verb-initial declaratives} relatively high in just a handful of OE\il{Old English} prose texts, for example, Bede's \textit{Historia Ecclesiastica}\is{Bede’s Historia Ecclesiastica}\is{Bede} (\citealt{Ohkado2000}, \citealt{Calle-MartínMiranda-García2010}, \citealt{Cichosz2017}).

Thanks to the availability of the syntactically annotated York-Toronto-Helsinki Parsed Corpus of \ili{Old English} Prose (\isi{YCOE}, \citealt{TaylorTaylor2003}), it is possible to conduct detailed quantitative and qualitative studies of syntactic structures, including \isi{narrative inversion}. The quantitative approach, however, faces a serious problem in terms of the adequate operationalisation of the structure. As signalled in \citet{Cichosz2022}, there are two parallel constructions taking exactly the same form in the OE\il{Old English} data, illustrated by \REF{ex:cichosz:1} and \REF{ex:cichosz:2}:

\ea%1
\label{ex:cichosz:1}
\gll Wæs  þa  sum   Godes   þegen    binnon  þære byri \\
was then some.\textsc{nom}  God.\textsc{gen}  nobleman.\textsc{nom}  in  the.\textsc{dat} city.\textsc{dat} \\
\glt ‘There was a man of God in the city.' \hfill [ÆCHom I, 27: 401.23--25]
\z 

\ea%2
\label{ex:cichosz:2}
\gll Hæfde  þæt   deor   þrie   hornas   on foran heafde \\
had the.\textsc{nom}  animal.\textsc{nom} three.\textsc{acc}  horns.\textsc{acc} on forehead.\textsc{dat} \\
\glt ‘The animal had three horns on its forehead.' \hfill [coalex, Alex: 20.2.229]
\z 

\REF{ex:cichosz:1} is a classic example of \isi{narrative inversion}: The clause introduces a new\is{discourse-new} participant and it is used at the beginning of a story. \REF{ex:cichosz:2}, however, is quite different, providing additional information about a known referent without pushing the \isi{narration} forward in any way. In \citet{Cichosz2022}, where samples from 4 different prose texts with the highest frequency of V1\is{verb-first} declaratives\is{verb-initial declaratives} are analysed, it turns out that \isi{narrative inversion} is used by \isi{Ælfric} (though not consistently in all of his works), while other prose texts (\isi{Bede}, \isi{Vercelli Homilies} and Alexander's Letter\is{Alexander's Letter to Aristotle}) opt for the construction shown in \REF{ex:cichosz:2}. Thus, it seems that in OE\il{Old English} V1\is{verb-first} declaratives\is{verb-initial declaratives} cannot be equated with \isi{narrative inversion} as at least some of them perform a drastically different \isi{discourse function}. It is not uncommon in language for the same form to have distinct meanings or to perform different functions, but such a situation is easily missed if only aggregate numbers are taken into account. In this case, both structures involve exactly the same corpus annotation: The clause is main, non-conjunct and non-negated, the verb is indicative, and there is an overt subject in the clause. It is technically impossible to separate one from the other in any automated way: Manual analysis in context is an indispensable component of the study.

The aim of this investigation is to conduct a comprehensive quantitative and qualitative analysis of all V1\is{verb-first} declaratives\is{verb-initial declaratives} in the \isi{YCOE} corpus in order to establish the actual frequency of both narrative and descriptive V1\is{verb-first} clauses in OE\il{Old English} prose (and, if possible, to identify other functions of V1\is{verb-first} declaratives\is{verb-initial declaratives}), seeking the main reasons for the uneven distribution of the patterns in \isi{YCOE}. On a more general level, the study shows the importance of qualitative, context-based analyses, which can and should complement any data-driven corpus studies.


\section{V1 declaratives in Old English}\label{sec:cichosz:2}

\noindent The functions of V1\is{verb-first} declaratives\is{verb-initial declaratives} in OE\il{Old English} are usually quite broadly defined as marking some sort of transition in the narrative structure of the text (\citealt{Mitchell1985}: $§$3933; \citealt{Calle-MartínMiranda-García2010}), which is not necessarily as exceptional as the reported markedness of the pattern would suggest. OE\il{Old English} had a large set of clause-initial discourse-sequencing adverbs,\is{adverb} mostly \textit{þa} ‘then' and \textit{þonne} ‘then', which could and did mark transition, with \textit{þa} being the most frequent and neutral device used for this function in OE\il{Old English} prose (\citealt{vanKem-Los2006}). It has been noted that \isi{narrative inversion} tends to co{}-occur with particularly vivid or dramatic fragments of the \isi{narration} (\citealt[291]{Stockwell1977}; \citealt[163]{Kiparsky1995}; \citealt{Los2000}), though \citet[168]{Petrova2006} reports that in Beowulf V1\is{verb-first} clauses “occur exactly in cases where the plot enters a new stage of development or the \isi{narration} needs to be pushed forward” and according to her study, V1\is{verb-first} and \textit{þa}-VS clauses (at least in OE\il{Old English} poetry) are functionally interchangeable.

Nonetheless, existing research on \isi{narrative inversion} in prose still treats V1\is{verb-first} declaratives\is{verb-initial declaratives} as a special discourse device and \citet{Ohkado2004} is most explicit in his classification of all the functions fulfilled by the structure. His analysis of V1\is{verb-first} declaratives\is{verb-initial declaratives} used in \isi{Ælfric}'s Catholic Homilies I (30 clauses in total) reveals that they mark the transition to a new action, summarise a discussion, introduce a new\is{discourse-new} or contrasting \isi{character} or type and open a new story or paragraph. \citet[$§$3930]{Mitchell1985} mentions that V1\is{verb-first} declaratives\is{verb-initial declaratives} are often presentational\is{presentational constructions} constructions (which would probably correspond to \citeauthor{Ohkado2004}'s (\citeyear{Ohkado2004}) “new\is{discourse-new} \isi{character}” function). \citet[34]{Allen1995} suggests that V1\is{verb-first} serves to “emphasise the (new\is{discourse-new} information) subject”, which is quite problematic since V1\is{verb-first} clauses are known to employ personal pronoun subjects (\citealt{RingeTaylor2014}: 408), and such constituents by definition cannot introduce new\is{discourse-new} discourse referents, but \citet[171]{Petrova2006} describes the function of V1\is{verb-first} as “focusing\is{focus} not only the subject but the entire proposition”. \citet{Los2000} finds a difference between the two on the basis of her analysis of \isi{Ælfric}'s works: In her study, both V1\is{verb-first} and \textit{þa}{}-VS clauses occur at episode boundaries but the former indicate “a thematic discontinuity”, while the latter mark “the smooth flow of narrative” \citep[263]{Los2000}.

As mentioned in Section \ref{sec:cichosz:1}, more recently, it has been shown that V1\is{verb-first} declaratives\is{verb-initial declaratives} attested in OE\il{Old English} prose texts have more functions, and not all of them may reasonably be covered by the umbrella term “\isi{narrative inversion}” \citep{Cichosz2022}. Attested patterns include both \isi{narration}-progressing clauses as in \REF{ex:cichosz:3} and descriptive structures giving additional information about previously introduced discourse referents shown in \REF{ex:cichosz:4}. Interestingly, the patterns are reported to show drastically different lexical preferences: While the former tend to co-occur with dynamic and punctual verbs (as in \textit{wæs cumen} ‘was come' in \REF{ex:cichosz:3}, the latter is associated with verbs of state ({\textit{wesan}} ‘to be' or \textit{habban} ‘to have' as in \REF{ex:cichosz:4}.

\ea%3
\label{ex:cichosz:3}
\gll Þa  foron  forð  oþ  hi   comon  to  Lundenbyrig  \& ða mid  þam   burhwarum  \& mid  þam   fultume  ðe him    westan     com   foran   east   to   Beamfleote; \textbf{Wæs}   \textbf{ða}   \textbf{Hæsten}   \textbf{þær}   \textbf{cumen}   \textbf{mid}   \textbf{his}     \textbf{herge} \textbf{þe}   \textbf{ær}   \textbf{æt}   \textbf{Middeltune}     \textbf{sæt}\\
then  went  forth  until  they.\textsc{nom}  came  to  London.\textsc{dat}  and  then  with  the.\textsc{dat}  citizens.\textsc{dat}  and  with  the.\textsc{dat}  help.\textsc{dat}  that   they.\textsc{dat}  from.west  came  went  east  to  Benfleet was  then  Hasten.\textsc{nom}  there  come  with  his.\textsc{dat}  army.\textsc{dat} which  before  at  Middletown.\textsc{dat}  sat \\
\glt ‘Then they went forth until they reached London and then they went east to Benfleet with the citizens and with the reinforcements that came to them from the west. \textbf{Then} \textbf{Hasten} \textbf{came} \textbf{there} \textbf{with} \textbf{his} \textbf{army,} \textbf{which} \textbf{had} \textbf{previously} \textbf{stayed} \textbf{at} \textbf{Middletown}'.\\ \hfill [ChronC (O'Brien\_O'Keeffe): [038400 (894.35)]]
\z 
        
\ea%4
\label{ex:cichosz:4}
\gll Þa   eode   he,     ure     dryhten   Crist,     ut beforan   þa     Iudeas. \textbf{Hæfde}     \textbf{he} \textbf{þa}\textbf{þyrnenne}   \textbf{coronan}   \textbf{on}  \textbf{his}     \textbf{heafde,} \& mid þam     readan     hrægle     gegyred   wæs.   Þa   cwæð he,     Pilatus,     to   Iudeum:\textit{Ecce homo}\\
then  went  he.\textsc{nom}  our.\textsc{nom}  lord.\textsc{nom}  Christ.\textsc{nom}  out before    the.\textsc{acc}  Jews.\textsc{acc}   had    he.\textsc{nom}   the.\textsc{acc}thorny.\textsc{acc}  crown.\textsc{acc}  on  his.\textsc{dat}  head.\textsc{dat}  and  with the.\textsc{dat}  red.\textsc{dat}  gown.\textsc{dat}  clothed    was  then  said he.\textsc{nom}  Pilate.\textsc{nom}  to  Jews.\textsc{dat}  [Latin]\\
\glt‘Then he, our Lord Christ, went out before the Jews. \textbf{He} \textbf{had} \textbf{a} \textbf{crown} \textbf{of} \textbf{thorns} \textbf{on} \textbf{his} \textbf{head} and he was wearing a red robe. Then he, Pontius Pilate, said to the Jews: \textit{Ecce homo}.'\\ \hfill [HomS\_24\_ (ScraggVerc\_1): [009500 (159)]--[009700 (161)]]
\z
      
This means that the same form (a non-negated V1\is{verb-first} declarative clause\is{verb-initial declaratives} with an overt subject) may represent both coordinating\is{coordinating discourse relations} and \isi{subordinating discourse relations},\is{discourse relation} as defined in the Segmented Discourse Representation Theory. In this theory, clauses are classified as coordinating\is{coordinating discourse relations}\is{discourse relation} if they belong to the main line of narrative, introducing main events and pushing the story forward, and subordinating\is{subordinating discourse relations} when they are placed in a substructure of the text, presenting background, providing additional information, comments or explanations (\citealt{AsherLascarides2003}; \citealt{AsherVieu2005}). \REF{ex:cichosz:4} is a clear example of elaboration or background, while \REF{ex:cichosz:3} operates on the highest level of text hierarchy, presenting one of subsequent events in the story. In the OHG\il{Old High German} context, \citet[73]{HinterhölzlPetrova2005} determine that “verb-initial\is{verb-first} structures establish coordinative discourse relations\is{coordinating discourse relations}\is{discourse relation} whereas \isi{verb-second} clauses signal subordinating\is{subordinating discourse relations} linkage to previous discourse part”, which is in line with the conclusions reached by \citet{Los2000} for OE\il{Old English}. However, examples such as \REF{ex:cichosz:4} clearly diverge from this pattern as V1\is{verb-first} declaratives\is{verb-initial declaratives} are not consistently found on the main level of the \isi{narration}.\footnote{An example of a study applying the Segmented Discourse Representation Theory to the analysis of OE\il{Old English} is \citet{Bech2012}, who concludes that there is no straightforward relation between a specific constituent order pattern (with the focus of the study on V-final and V-late clauses) and discourse function in OE\il{Old English}. Thus, at least in OE\il{Old English}, the situation seems quite complex, not only in the context of V1 declaratives.}

One of the main limitations of various functional studies on OE\il{Old English} \isi{narrative inversion} is that they are predominantly based only on the output of \isi{Ælfric} of Eynsham, a famous OE\il{Old English} writer and translator, who is single-handedly responsible for ca. 25\% of OE\il{Old English} prose records \citep{CichoszPęzik2021}. What is more, \isi{Ælfric} (c. 955--1010) was active at the very end of the OE\il{Old English} period, so his output, probably full of stylistic\is{style} idiosyncrasies, represents only the later part of period. As a result, such studies do not cover early OE\il{Old English}, which is the most probably reason why the high frequency of the structure represented by \REF{ex:cichosz:4} has gone largely unnoticed. Moreover, the analyses are usually based on relatively small datasets, for example \citet{Ohkado2004} analyses 30 V1\is{verb-first} clauses from \isi{Ælfric}'s Catholic Homilies, \citet{Los2000} analyses variation between V1\is{verb-first} and \textit{þa}-VS in clauses with the verb \textit{onginnan/beginnan} ‘to begin' in all works by \isi{Ælfric} (11 V1\is{verb-first} clauses vs. 35 {\textit{þa}}-VS clauses), and \citet{Cichosz2022} takes samples of V1\is{verb-first} clauses with the three most frequent verbal collocates from four different \isi{YCOE} texts (57 clauses in the qualitative part of the analysis). As a result, the current understanding of the functional load of V1\is{verb-first} declaratives\is{verb-initial declaratives} and the division of labour between narrative V1\is{verb-first} and other OE\il{Old English} discourse devices is fragmentary and calls for a comprehensive analysis.

Finally, it should be noted that there are considerable differences in the distribution of V1\is{verb-first} declaratives\is{verb-initial declaratives} in the OE\il{Old English} textual records. The structure is relatively rare in prose but frequent in poetry (while \textit{þa}-VS is frequent in prose and rare in poetry), although there is one prose text where the frequency of V1\is{verb-first} declaratives\is{verb-initial declaratives} is exceptionally high, namely the OE\il{Old English} translation of Bede's \textit{Historia Ecclesiastica}\is{Bede’s Historia Ecclesiastica}\is{Bede} (\citealt{Mitchell1985}: $§$3932; \citealt{Calle-MartínMiranda-García2010}). However, \citet{Afros2022} shows that in later manuscripts the frequency of V1\is{verb-first} declaratives\is{verb-initial declaratives} is slightly lower (15\% in T as opposed to 20\% in earlier witnesses), with changes into the less marked \isi{verb-second}, V-late or \textit{þa}-VS. Since the OE\il{Old English} \isi{Bede} is based on a Latin source, \citet{Ohkado2000} concluded that it is a foreign structure, but more detailed studies by \citet{Cichosz2017} and \citet{CichoszEtAl2016} show the pattern is mostly independent from the Latin original. This study investigates whether the same level of independence\is{transfer from Latin} is visible for all the functions of V1\is{verb-first} declaratives\is{verb-initial declaratives} attested in the OE\il{Old English} \isi{Bede}.


\section{Study design}
\label{sec:cichosz:3}

The aim of the study is to conduct a comprehensive quantitative and qualitative analysis of all V1\is{verb-first} declaratives\is{verb-initial declaratives} found in the \isi{YCOE} corpus \citep{TaylorTaylor2003} and find answers to the following research questions:

\begin{itemize}
\item[a.] 
What are the functions of V1 declaratives in OE prose, and is the narrative function dominant?
\item[b.] 
What is the actual frequency of narrative V1 and how is it distributed in the OE prose corpus?
\item[c.]
What is the relation between narrative V1 and descriptive V1 declaratives?
\item[d.] 
Which OE prose texts use the identified patterns and do these compositions form a homogeneous group?
\item[e.] 
Does Latin influence any particular function of V1 declaratives in the OE Bede?
\item[f.]
What is the main reason for the uneven distribution of V1 declaratives in OE prose?
\end{itemize}

Since \isi{YCOE} is a syntactically annotated corpus, it was possible to extract the clauses automatically by means of the following CorpusSearch 2 (\citealt{Randall2005}--13) query:

\begin{quote}
node: IP-MAT*

query:   ((IP-MAT* iDomsNumber 1 finite\_verb)

    AND (IP-MAT* iDoms NP-\textsc{nom}*)

    AND (NP-\textsc{nom}* iDoms ! {\textbackslash}*con{\textbackslash}*{\textbar}{\textbackslash}*pro{\textbackslash}*{\textbar}{\textbackslash}*exp{\textbackslash}*))    
\end{quote}

The query searched for verb-initial main clauses,\is{verb-initial declaratives} which automatically excludes conjunct clauses introduced by coordinating conjunctions,\is{conjunction} since these are treated separately in studies of OE\il{Old English} syntax because of their specific constituent order patterns, often overlapping with subordinate clauses (\citealt{Mitchell1985}; \citealt{vanKemenade1987}; \citealt{Pintzuk1999}; \citealt{FischerEtAl2000}; \citealt{RingeTaylor2014}; \citealt{Cichosz2022}). An obligatory element of the clause included in the query was an overt subject, though an additional mirror query searching specifically for clauses without overt subjects was used for the purposes of the case study described in Section \ref{sec:cichosz:4.3}. The definition file of the finite verb used in the query neither covers negated forms (negation is a known factor associated with V1\is{verb-first} order in OE\il{Old English} and research shows that it is not related to \isi{narrative inversion}, cf. \citealt{Cichosz2020}) nor subjunctive forms\footnote{ {The file includes the following verb labels: \\ RP+VBPI{\textbar}RP+VBDI{\textbar}RP+VBP{\textbar}RP+VBD{\textbar}RP+BEPI{\textbar}RP+BEDI{\textbar}RP+BED{\textbar}RP+BEP{\textbar}RP+HVPI{\textbar}RP+HV\\
DI{\textbar}RP+HVP{\textbar}RP+HVD{\textbar}RP+AXPI{\textbar}RP+AXDI{\textbar}RP+AXP{\textbar}RP+AXD{\textbar}RP+MDPI{\textbar}RP+MDDI{\textbar}RP+MDP{\textbar}\\
RP+MDD{\textbar}VBPI{\textbar}VBDI{\textbar}VBP{\textbar}VBD{\textbar}BEPI{\textbar}BEDI{\textbar}BED{\textbar}BEP{\textbar}HVPI{\textbar}HVDI{\textbar}HVP{\textbar}HVD{\textbar}AXPI{\textbar}AXDI{\textbar}AXP\\{\textbar}AXD{\textbar}MDPI{\textbar}MDDI{\textbar}MDP{\textbar}MDD}} since the latter were predominantly directive, representing the so-called jussive subjunctive structures. Forms that were ambiguous between indicative and subjunctive were included in order to not narrow down the sample excessively; any examples of directives were manually excluded, and so were the examples with modal verbs since a closer analysis in context revealed that all of them could be interpreted as directive. Since imperative clauses in OE\il{Old English} (as in any Germanic language) are usually V1\is{verb-first}, morphologically indicative verbs in clauses with similar illocutionary force could follow the same pattern. Such clauses were not taken into account since their function is clearly different from narrative and descriptive V1\is{verb-first} declaratives\is{verb-initial declaratives}, which this study focuses on. \REF{ex:cichosz:5} and \REF{ex:cichosz:6} are examples of clauses returned by the query which were discarded at this stage.

\ea%5
\label{ex:cichosz:5}
\gll Weaxe   ge   nu \\
grow  you  now \\
\glt‘Grow now.' \hfill [cootest, Gen:9.7.381]
\z 

\ea%6
\label{ex:cichosz:6}
\gll  Mæg   gehyran   se     ðe   wyle   be   þam     halgan mædene   Eugenian … \\
may  hear    this.\textsc{nom}  who  will  by  the.\textsc{dat}  holy.\textsc{dat} maiden.\textsc{dat}  Eugenia.\textsc{dat} \\
\glt‘Let everyone interested hear about the holy maiden Eugenia….'\\ \hfill [coaelive,+ALS\_[Eugenia]:1.197]
\z 

The query returned 918 examples but the manual cleanup narrowed the final number of V1\is{verb-first} clauses included in the study down to 840.

The qualitative analysis of V1\is{verb-first} clauses and the division into particular functions required looking at the examples in broader context, and the Dictionary of \ili{Old English} Web Corpus (DOEC) was used for this purpose. As a result, when V1\is{verb-first} clauses are presented in isolation, they are accompanied by \isi{YCOE} identifiers, but when broader context is given, the examples are followed by DOEC identifiers. The functional classification was based on a bottom-to-top approach, in other words, the only pre-set category was narrative and the analysis was supposed to identify all the distinct patterns visible in the data.  

The analysis of extralinguistic factors (\isi{genre}, translation status, dialect) was based on \isi{YCOE} metadata; texts were divided into early (o1-o2) and late OE\il{Old English} (o3-o4) on the basis of their presumed date of composition.
\largerpage
The analysis of the relation between source Latin clauses and their OE\il{Old English} equivalents in the case study of Bede's \textit{Historia Ecclesiastica}\is{Bede’s Historia Ecclesiastica}\is{Bede} (Section \ref{sec:cichosz:4.2}) was based on manual alignment of all 459 V1\is{verb-first} declaratives\is{verb-initial declaratives} from \isi{Bede} to their most probable Latin sources. Since it is not certain which Latin manuscript was the direct source of the OE\il{Old English} translation \citep[43]{Lemke2015} and it is very probable that this particular version of Latin \isi{Bede} did not survive \citep{Wallis2013}, the choice of the Latin edition was based on its availability in electronic form \citep{Plummer1896}. Clauses were classified as  modelled\is{transfer from Latin} on Latin if the source clause had a clause-initial verb form (finite or non-finite); presence of the subject was not considered necessary since pronominal subjects are rarely overt in Latin. If the verb was placed later in the Latin clause, or if the OE\il{Old English} clause was a serious paraphrase of the original fragment, the clause was classified as \isi{independent}. Examples provided in Section \ref{sec:cichosz:4.2} hopefully make this methodology transparent.

\largerpage
\section{Results}
\label{sec:cichosz:4}

\subsection{General results}
\label{sec:cichosz:4.1}
\tabref{tab:cichosz:1} presents \isi{YCOE} texts with the highest absolute number of V1\is{verb-first} declaratives\is{verb-initial declaratives}, using 10 clauses as a cut-off point. Rather unsurprisingly, and confirming the results of numerous studies reported in Section \ref{sec:cichosz:2}, \isi{Bede} takes the first position in this classification, with the highest number of clauses covering more than a half of all the OE prose examples of the structure under investigation.

\begin{table}
\begin{tabularx}{0.6\textwidth}{Xr}
\lsptoprule
Bede's \textit{History of the English Church}  & {459}\\
Vercelli Homilies & {67}\\
Blickling Homilies & {45}\\
Alexander's \textit{Letter to Aristotle} & { 42}\\
Ælfric's Catholic Homilies I  & {35}\\
Ælfric's Lives of Saints  & {21}\\
Ælfric's Catholic Homilies II  & {20}\\
Anglo-Saxon Chronicle D & {13}\\
Boethius' \textit{Consolation of Philosophy}  & {11}\\
Heptateuch & {10}\\
Anglo-Saxon Chronicle C & {10}\\
YCOE in total & {840}\\
\lspbottomrule
\end{tabularx}
\caption{Absolute frequency of V1 declaratives in different OE prose texts}
\label{tab:cichosz:1}
\end{table}

What is more, it should be noted that 73\% of the clauses in question come from 4 non-Ælfrician texts (\isi{Bede}, \isi{Vercelli Homilies}, \isi{Blickling Homilies} and Alexander's Letter\is{Alexander's Letter to Aristotle}), while (as shown in Section \ref{sec:cichosz:2}) the only detailed and context-based functional studies of the structure in OE\il{Old English} prose are primarily based on the output of \isi{Ælfric}. 

\begin{figure}[t]
    \centering
    \includegraphics[width=\textwidth]{figures/norm_freq_v1_decl-cropped.png}
\caption{Normalised frequencies of V1 declaratives per 1,000,000 words in the longest YCOE texts}
\label{fig:cichosz:1}
\end{figure}

When normalised frequencies are taken into account, the difference between individual texts becomes even more conspicuous, and it is actually Alexander's Letter,\is{Alexander's Letter to Aristotle} not \isi{Bede}, which turns out to have the highest frequency of V1\is{verb-first} declaratives\is{verb-initial declaratives} in \isi{YCOE}. Nevertheless, the top four remains the same (\isi{Bede}, Alexander's Letter\is{Alexander's Letter to Aristotle}, \isi{Vercelli Homilies} and \isi{Blickling Homilies}) and \isi{Ælfric}'s compositions show very low results, which indicates that functional studies of non-Ælfrician texts are needed to understand the function of V1\is{verb-first} declaratives\is{verb-initial declaratives} in OE\il{Old English} prose.

Functionally speaking, it is necessary to analyse the clauses in a larger context to get a clear idea of what a V1\is{verb-first} clause does in a particular place in discourse. As explained in Section \ref{sec:cichosz:3}, this study followed a bottom-to-top approach, which made it possible to identify four clear patterns in the data, with some interrelations between them. A typical context for a V1\is{verb-first} declarative\is{verb-initial declaratives} (in bold) is shown in \REF{ex:cichosz:7}.

\ea
\label{ex:cichosz:7} 
\gll Ða wæs sum þegen annanias gehaten. \& his wif saphira hi cwædon him betweonan þæt hi woldon bugan to þæra apostola geferrædene namon þa to ræde þæt him wærlicor wære þæt hi sumne dæl heora landes wurþes æthæfdon weald hu him. getimode \textbf{Com} \textbf{þa} \textbf{se} \textbf{þegen} \textbf{mid} \textbf{feo} \textbf{to} \textbf{þam} \textbf{apostolum}. Ða cwæð petrus: annania: deofol bepæhte þine heortan. \& þu hæfst alogen þam halgan gast.\\
then was some.\textsc{nom} nobleman.\textsc{nom} Ananias called and his.\textsc{nom} wife.\textsc{nom} Saphira.\textsc{nom} they.\textsc{nom} said them.\textsc{dat} between that they.\textsc{nom} would turn to the.\textsc{gen} apostles.\textsc{gen} fellowship.\textsc{dat} took then to counsel.\textsc{dat} that them.\textsc{dat} wiser were that they.\textsc{nom} some.\textsc{acc} part.\textsc{acc} their.\textsc{gen} land.\textsc{gen} worth.\textsc{gen} withhold in case them.\textsc{dat} befell came then the.\textsc{nom} nobleman.\textsc{nom} with riches.\textsc{dat} to the.\textsc{dat} apostles.\textsc{dat} then said Peter.\textsc{nom} Ananias.\textsc{nom} devil.\textsc{nom} deceived your.\textsc{acc} heart.\textsc{acc} and you.\textsc{nom} have lied the.\textsc{dat} holy.\textsc{dat} ghost.\textsc{dat}\\

\glt ‘There was some nobleman called Ananias and his wife Saphira. They said to each other that they wanted to turn to the apostles' fellowship. (They) agreed that it would be better for them to withhold some part of their land's worth in case something happened to them. \textbf{Then} \textbf{the} \textbf{nobleman} \textbf{came} \textbf{with} \textbf{riches} \textbf{to} \textbf{the} \textbf{apostles}. Then Peter said: Ananias, devil deceived your heart and you have lied to the Holy Ghost.'\\ \hfill [ÆCHom I, 22, [004000 (357.88--92)]]
\z

As can be seen, the V1\is{verb-first} declarative\is{verb-initial declaratives} is surrounded by \textit{þa}-VS clauses (and there is also a V1\is{verb-first} clause without an overt subject, \textit{namon þa} …, which may be an important part of the whole picture, cf. \ref{sec:cichosz:4.3}), so narrative V1\is{verb-first} seems to be one of many tools OE\il{Old English} writers had at their disposal when presenting a story. The clause \textit{Com þa se þegen} undoubtedly introduces a subsequent event in the \isi{narration}.\footnote{An important factor here is also the subject shift (from the plural, indicating Ananias and his wife, to the nobleman himself), observed regularly in narrative V1\is{verb-first} clauses with a nominal subject, see Section \ref{sec:cichosz:4.3}.} Whether this clause is more “dramatic” than the following \textit{Ða cwæð petrus} is perhaps less clear, but the narrative function of the clause in bold is certain.

The next pattern, related to the general narrative function and potentially a subtype of it, is the presentational\is{presentational constructions} structure, illustrated by \REF{ex:cichosz:8}.

\ea%8
\label{ex:cichosz:8}
\gll Mid þy   ic   ða   wolde   geornlicor   þa     þing     geseon   \& furðor   eode   þa   geseah   ic   gyldenne   wingeard   trumlicne   \& fæstlicne,   \& þa     twigo     his     hongodon geond   þa     columnan   Ða   wundrode   ic   þæs swiðe. \textbf{Wæron}   \textbf{in}   \textbf{þæm}     \textbf{wingearde}   \textbf{gyldenu}   \textbf{leaf} \& his     hon     \& his     wæstmas   wæron cristallum   \& smaragdus,   eac   þæt     gimcyn     mid þæm     cristallum   ingemong   hongode.\\
when  I.\textsc{nom}  then  wanted  more carefully  the.\textsc{acc}  things.\textsc{acc}  see  and further   went  then  saw  I.\textsc{nom}  golden.\textsc{acc}  vine.\textsc{acc}  strong.\textsc{acc}  and firm.\textsc{acc}  and  the.\textsc{nom}  twigs.\textsc{nom}  his.\textsc{nom}  hang over  the.\textsc{acc}   columns.\textsc{acc}   then  wondered  I.\textsc{nom}  this.\textsc{gen} very   were    in   the.\textsc{dat}  vinebush.\textsc{dat}   golden.\textsc{nom}  leaves.\textsc{nom} and   his.\textsc{nom}   tendrils.\textsc{nom}  and  his.\textsc{nom}  fruit.\textsc{nom}  were crystal.\textsc{nom}  and  emeralds.\textsc{nom}  also  the.\textsc{nom}  gem.\textsc{nom}   with the.\textsc{dat}  cristals.\textsc{dat}   among    hang\\
\glt ‘When I wanted to see these things more carefully and went further, then I saw a golden vine, strong and firm, and the twigs were hanging over the columns. Then I wondered at this very much. \textbf{There} \textbf{were} \textbf{golden} \textbf{leaves} \textbf{in} \textbf{the} \textbf{vine} \textbf{bush} and the vine's tendrils and fruit were crystal and emerald, and also a gem was hanging there among the crystals.' \\ \hfill [Alex, 003700 (8.14--8.16)]
\z 

Such clauses would usually be translated into Modern English with existential \textit{there}. In this case, the golden leaves of the vine are an important element in the story of all the riches that Alexander's army encountered in the city of Porres, so the clause represents \isi{coordinating discourse relations}, functioning on the main level of the storyline. Since this seems to be the case with all the presentational\is{presentational constructions} structures identified in this study, unless signalled otherwise, presentational\is{presentational constructions} clauses (93 in total) will be included in the general statistics considering narrative V1\is{verb-first}.

In addition, there is a big group of V1\is{verb-first} declaratives\is{verb-initial declaratives} used in reporting structures\is{reporting clause} to quote (mainly direct) speech, as in \REF{ex:cichosz:9}. While such reported dialogues\is{dialogue} undoubtedly push the story forward and could also be treated as examples of \isi{narrative inversion}, I will keep them separate in the statistics since the level of attraction between V1\is{verb-first} and verbs of saying is exceptionally high \citep{Cichosz2017b}.

\ea%9
\label{ex:cichosz:9}
\gll Eft   he     frægn,   hwæt   seo     þeod     nemned wære,   þe   heo     of   cwomon. \textbf{Ondswarede}   \textbf{him} \textbf{mon} þæt   heo     Ongle     nemde   wæron. \textbf{Cwæð}   \textbf{he}:  Wel   þæt     swa   mæg:   forðon   heo     ænlice     onsyne habbað,   \& eac   swylce   gedafonað,   þæt   heo     engla æfenerfeweardas   in heofonum     sy.\\
Again   he.\textsc{nom}   asked   what   the.\textsc{nom}  people.\textsc{nom}   called were   which   they.\textsc{nom}   of   came     answered   him.\textsc{dat} one.\textsc{nom}   that  they.\textsc{nom}   Angles.\textsc{nom}   called   were   Said   he.\textsc{nom} well   that.\textsc{nom}   so   may   because they.\textsc{nom}   angelic.\textsc{acc}   faces.\textsc{acc} have     and   also   so   becomes   that   they.\textsc{nom}   angels.\textsc{nom} afterwards     in heavens.\textsc{dat}  be\\
\glt ‘He asked again about the name of the tribe from which they came. \textbf{Someone} \textbf{answered} \textbf{him} {that they were called Angles.} \textbf{He} \textbf{said}: It may well be so because they have angelic faces and it is only becoming that they will later be angels in heaven.' \hfill [Bede 2, [002000 (1.96.21--22)]]
\z

Last but not least, the descriptive type shown in \REF{ex:cichosz:10} clearly stands out from \REF{ex:cichosz:7}--\REF{ex:cichosz:9}. In this case, the text presents the life and heroic death of Saint Martin. Before the story unfolds, St. Martin's background is presented and some additional information about his place of birth and family is presented. Then, the story moves back to the main line of \isi{narration} and the most important events of the saint's life are described. Apart from the fact that the V1\is{verb-first} declaratives\is{verb-initial declaratives} shown in \REF{ex:cichosz:10} play a different role in discourse, appearing in a sub-structure of the text, what is quite striking is the great accumulation of such clauses in one fragment, which is not an exceptional situation as such clusters of descriptive V1\is{verb-first} declaratives\is{verb-initial declaratives} appear in all of the “top four texts” identified in this study (cf. \tabref{tab:cichosz:1}).

\ea%10
\label{ex:cichosz:10}
\gll ... þe   Martinus   wæs   haten. \textbf{Wæs}   \textbf{he}     \textbf{Gode}     \textbf{swiðe} \textbf{gecoren}   \textbf{on}   \textbf{his}     \textbf{þeawum}.   \textbf{Wæs}   \textbf{he}     \textbf{in} \textbf{Pannana}   \textbf{þære}     \textbf{mægðe}     \textbf{in}   \textbf{woruld}   \textbf{cumen}, \textbf{in}   \textbf{Arrea}     \textbf{ðam}     \textbf{tune}.     \textbf{Wæs}   \textbf{he}     \textbf{hwæðre} \textbf{in}   \textbf{Italia}     \textbf{afeded},   \textbf{in}   <\textbf{Ticinan}>   \textbf{þære}     \textbf{byrig}. \textbf{Wæs}   \textbf{he}     \textbf{for}   \textbf{worulde}   \textbf{swiðe}   \textbf{godre}     \textbf{gebyrde}. \textbf{Wæron}   \textbf{his}     \textbf{fæder}     \textbf{\&} \textbf{his}     \textbf{modor} \textbf{buta}     \textbf{hæðen}.   \textbf{Wæs}   \textbf{his}     \textbf{fæder}     \textbf{ærest} \textbf{cyninges}   \textbf{þegn,}     \& þa   æt   nehstan   geþah þæt   he     wæs   tribunus,   þæt   is    ealdorman cyninges   þegna.     Þa   sceolde   he,     sanctus    Martinus,   nyde     beon   sona   on   his     giogoðhade    on   geferræddenne    cyninges   þegna \\
… who   Martin.\textsc{nom}  was   called   was   he.\textsc{nom}   God.\textsc{dat}   very dear     in   his.\textsc{dat}  customs.\textsc{dat}  was   he.\textsc{nom}   in Pannonia.\textsc{dat}  the.\textsc{dat}   diocese.\textsc{dat}  in   world.\textsc{acc}  come in   Arrea.\textsc{dat}   the.\textsc{dat}   city.\textsc{dat}  was   he.\textsc{nom}   nevertheless in   Italy.\textsc{dat}   raised    in  Ticine.\textsc{dat}  the.\textsc{dat}  city.\textsc{dat} was   he.\textsc{nom}   for   world.\textsc{dat}  very   good.\textsc{gen}   birth.\textsc{gen} were     his.\textsc{nom}   father.\textsc{nom}   and   his.\textsc{nom}   mother.\textsc{nom} both.\textsc{nom}   heathen.\textsc{nom}  was   his.\textsc{nom}   father.\textsc{nom}   first king.\textsc{gen}  thane.\textsc{nom}   and   then  at   next     happened that  he.\textsc{nom}  was   tribune.\textsc{nom}   that   is   alderman.\textsc{nom} king.\textsc{gen}   thanes.\textsc{gen}   then  should  he.\textsc{nom}  saint.\textsc{nom} Martin.\textsc{nom}   necessary  be  soon  on  his.\textsc{dat}  youth.\textsc{dat} on  fellowship.\textsc{dat} king.\textsc{gen}  thanes.\textsc{gen}\\
\glt ‘… who was called Martin. {\textbf{He} \textbf{was} \textbf{very} \textbf{dear} \textbf{to} \textbf{God} \textbf{in} \textbf{his} \textbf{customs.} \textbf{He} \textbf{has} \textbf{come} \textbf{to} \textbf{the} \textbf{world} \textbf{in} \textbf{the} \textbf{diocese} \textbf{of} \textbf{Pannonia,} \textbf{in} \textbf{the} \textbf{city} \textbf{of} \textbf{Arrea.} \textbf{He} \textbf{was} \textbf{nevertheless} \textbf{brought} \textbf{up} \textbf{in} \textbf{Italy,} \textbf{in} \textbf{the} \textbf{city} \textbf{of} \textbf{Ticinum} \textbf{(Pavia).} \textbf{He} \textbf{was} \textbf{of} \textbf{very} \textbf{good} \textbf{birth} \textbf{to} \textbf{the} \textbf{world.} \textbf{His} \textbf{father} \textbf{and} \textbf{his} \textbf{mother} \textbf{were} \textbf{both} \textbf{heathen.} \textbf{His} \textbf{father} \textbf{was} \textbf{first} \textbf{the} \textbf{king's} \textbf{thane}}, and then (he) was a tribune, that is the alderman of the king's thanes. Then he, St. Martin, in his youth had to belong to the fellowship of the king's thanes.' \\ \hfill [LS17.2, [000200 (1))]--[000800 (10))]]
\z

After the aforementioned manual analysis of all the V1\is{verb-first} clauses in context, it turned out that almost one third of all examples retrieved from \isi{YCOE} corresponds to the descriptive function represented by \REF{ex:cichosz:10}, which is quite a high proportion, considering the generally low frequency of the structure. \tabref{tab:cichosz:2} illustrates the functional division in the texts with the highest number of V1\is{verb-first} clauses (the presentational\is{presentational constructions} function, being closest to the \isi{narration}-pushing function, is included under the label “narrative”).

\begin{table}
\begin{tabularx}{\textwidth}{Xrrrr}
\lsptoprule
{text} & \multicolumn{1}{Q}{narrative (incl. presentational)} & {quotative} & {descriptive} & {all}\\
\midrule
{Bede's History}  & {213 (46\%)} & {93 (21\%)} & {153 (33\%)} & {459}\\
{Vercelli Homilies} & {27 (40\%)} & {11 (17\%)} & {29 (43\%)} & {67}\\
{Blickling Homilies} & {14 (31\%)} & {13 (29\%)} & {18 (40\%)} & {45}\\
{Alexander's Letter to Aristotle} & {12 (29\%)} & {2 (4\%)} & {28 (67\%)} & {42}\\
{Ælfric's Catholic Homilies I}  & {20 (95\%)} & {1 (5\%)} & {0 (0\%)} & {21}\\
{Ælfric's Lives of Saints}  & {28 (80\%)} & {4 (20\%)} & {0 (0\%)} & {35}\\
{Ælfric's Catholic Homilies II}  & {17 (85\%)} & {2 (5\%)} & {0 (0\%)} & {20}\\
{Anglo-Saxon Chronicle D} & {11 (85\%)} & {0 (0\%)} & {2 (15\%)} & {13}\\
\midrule
{YCOE (total)} & {448 (53\%)} & {136 (16\%)} & {256 (31\%)} & {840}\\
\lspbottomrule
\end{tabularx}
\caption{Functions of V1 declaratives in YCOE texts}
\label{tab:cichosz:2}
\end{table}

First of all, it should be noted that narrative V1\is{verb-first} is very well represented in all the texts, though the proportions differ from 95\% in Catholic Homilies I to 29\% in Alexander's Letter\is{Alexander's Letter to Aristotle}. In three of the four texts with the highest number of V1\is{verb-first} clauses, the proportions between narrative, quotative and descriptive uses are quite balanced. Alexander's Letter\is{Alexander's Letter to Aristotle} (a fictitious account of Alexander the Great's military campaign in India written to his master Aristotle, cf. \citealt{Khalaf2013}) stands out from the other texts because of its rare use of quotative V1\is{verb-first}, but this must be related to the narrative structure of the travelogue since there are hardly any dialogues\is{dialogue} reported in the text. Most importantly, however, \tabref{tab:cichosz:2} shows that descriptive V1\is{verb-first} is literally never used by \isi{Ælfric}, even though he consistently uses V1\is{verb-first} declaratives\is{verb-initial declaratives} as a narrative device. This is a full confirmation of the tentative observation made in \citet{Cichosz2022} on the basis of text samples. Interestingly, though, other texts are not that consistent, that is, non-Ælfrician compositions, if they use V1\is{verb-first} declaratives\is{verb-initial declaratives} at all, would usually make use of both narrative and descriptive V1\is{verb-first}. 

\largerpage
When particular functions are presented as normalised frequencies (\figref{fig:cichosz:2}) it becomes evident that the “top four” texts are exceptional not only in their particularly frequent use of V1\is{verb-first} declaratives\is{verb-initial declaratives}, but also in the employment of V1\is{verb-first} clauses for the descriptive and quotative functions. Other texts, if V1\is{verb-first} declaratives\is{verb-initial declaratives} are attested in them at all, treat them as a narrative device.

\begin{figure}
    \centering
    \includegraphics[width=\textwidth]{figures/norm_freq_v1_decl_w_part_funct-cropped.png}
    \caption{Normalised frequencies of V1 declaratives with particular functions per 1,000,000 words}
\label{fig:cichosz:2}
\end{figure}

Zooming in on the longest \isi{YCOE} files, where the probability of finding a reasonable number of V1\is{verb-first} declaratives\is{verb-initial declaratives} is the highest (see \tabref{tab:cichosz:3}), it can be found that many texts clearly avoid the structure entirely, and there is no clear pattern connecting texts where V1\is{verb-first} is in use. For example, \isi{Gregory's Dialogues} C, Orosius and Cura Pastoralis, all being early translations just like Bede's History,\is{Bede’s Historia Ecclesiastica}\is{Bede} show no V1\is{verb-first} of any type. The only regularity visible in the data is that descriptive V1\is{verb-first} is only used in early texts (though this is a one-way relation only, i.e. virtually all descriptive V1\is{verb-first} clauses are early OE\il{Old English}\footnote{The only exception is the Peterborough Chronicle (Anglo-Saxon Chronicle E), which is classified as a late text in \isi{YCOE}, but it is partly based on earlier manuscripts, so the possibility of retaining obsolete structures in this particular text is quite high.}but not all early OE\il{Old English} texts have descriptive V1\is{verb-first} clauses), while narrative V1\is{verb-first} is scattered around the corpus and appears (mostly with very low frequencies) in various texts. Its use must have been stylistically conditioned since even \isi{Ælfric} does not use it in all of his compositions (the highest frequency of V1\is{verb-first} is visible in Catholic Homilies I, and the structure is virtually absent from his Supplemental Homilies).

\begin{sidewaystable}
\begin{tabularx}{\textwidth}{lrQQQQrrr}
\lsptoprule
{text} & \multicolumn{1}{l}{words} & {sub-period} & {genre} & {trans-lated} & {dialect} & \multicolumn{1}{l}{V1} & \multicolumn{1}{Q}{narrative} & \multicolumn{1}{Q}{descrip-tive}\\
\midrule
{Ælfric's CH I}  & {106,173} & {late} & {homilies} & {no} & {WS} & {330} & {292} & {0}\\
{Ælfric's LoS}  & {100,193} & {late} & {biography} & {no} & {WS} & {210} & {200} & {0}\\
{Ælfric's CH II} & {98,583} & {late} & {homilies} & {no} & {WS} & {203} & {183} & {0}\\
{Gregory's Dialogues C} & {91,553} & {early} & {biography} & {yes} & {mixed} & {156} & {156} & {0}\\
{Bede's History} & {80,767} & {early} & {history} & {yes} & {mixed} & {5683} & {2637} & {1894}\\
{West-Saxon Gospels} & {71,104} & {late} & {bible} & {yes} & {WS} & {28} & {15} & {0}\\
{Cura Pastoralis} & {68,556} & {early} & {treatise}  & {yes} & {WS} & {15} & {15} & {0}\\
{Ælfric's Sup. Homilies}  & {62,669} & {late} & {homilies} & {no} & {WS} & {32} & {32} & {0}\\
{Heptateuch} & {59,524} & {late} & {bible} & {yes} & {WS} & {168} & {168} & {0}\\
{Orosius} & {51,020} & {early} & {history} & {yes} & {WS} & {78} & {78} & {0}\\
{Boethius} & {48,443} & {early} & {philosophy} & {yes} & {WS} & {227} & {227} & {0}\\
{Vercelli Homilies} & {45,674} & {early} & {homilies} & {no} & {WS(?)} & {1467} & {591} & {635}\\
{Blickling Homilies} & {42,506} & {early} & {homilies} & {no} & {mixed} & {1059} & {329} & {423}\\
{ASC E} & {40,641} & {late} & {history} & {no} & {mixed} & {221} & {197} & {25}\\
\lspbottomrule
\end{tabularx}
\caption{The use of V1 declaratives in the longest YCOE texts (normalised frequencies)}
\label{tab:cichosz:3}
\end{sidewaystable}

Alexander's Letter\is{Alexander's Letter to Aristotle} is not included in \tabref{tab:cichosz:3} because of its length (only 7,271 words), but it is also an early text, so the analysis shows that descriptive V1\is{verb-first} clauses seem diachronically limited. Before any final conclusions are drawn, I will check whether the structure may safely be treated as native to OE\il{Old English}.


\subsection{Bede: Influence of Latin}
\label{sec:cichosz:4.2}

The discussion about the Latinate character of the syntax of OE\il{Old English} \isi{Bede} is long-lasting and so far inconclusive. On the one hand, \isi{Bede} is not a gloss and it shows quite a lot of independence\is{transfer from Latin} from the Latin source (according to \citealt{CichoszEtAl2016}: 351, only 5\% of all clauses in the text are translated phrase by phrase; for the West-Saxon Gospels the corresponding result is 15\%). In addition, OE\il{Old English} textual records, rich as they are, have their limitations, so it would not be practical to exclude a long text such as \isi{Bede} from syntactic investigations only because sometimes “[t]he choices involved in translating \isi{Bede}'s Latin into OE\il{Old English} […] manifest themselves in somewhat artificial structures” \citep[9]{Rowley2011}. It is a fact that the OE\il{Old English} \isi{Bede} contains unexpectedly high amounts of constructions which are considered foreign or at least inspired by Latin, for example, the dative absolute (\citealt{Stanton2002}: 58; \citealt{Scheler1961}). On the other hand, the \isi{Bede} translator did introduce numerous and sometimes substantial changes to the original composition, for example, extensively modifying the preface and omitting certain events from the \isi{narration}, letters relating to the English Church and some information concerning the physical world \citep[86]{St-Jacques1983}. What is more, the overall dependence\is{transfer from Latin} of V1\is{verb-first} declaratives\is{verb-initial declaratives} in \isi{Bede} on the Latin source text suggested in earlier studies \citep{Ohkado2000} was shown to be only partial in more recent corpus-based investigations \citep{Cichosz2017}. What has not been checked so far, however, is the impact of Latin on V1\is{verb-first} declaratives\is{verb-initial declaratives} performing particular functions in the text. 

\tabref{tab:cichosz:4} shows that the proportion of clauses inspired by the Latin source is not the same between the functions, but none of the functions is completely \isi{independent} from the original.

\begin{table}
\begin{tabularx}{0.8\textwidth}{Xrr}
\lsptoprule
& \multicolumn{1}{l}{modelled} & \multicolumn{1}{l}{}{independent} \\
\midrule
narrative (with presentational) & {133 (62\%)} & {81 (38\%)}\\
quotative & {48 (52\%)} & {45 (48\%)}\\
descriptive  & {64(42\%)} & {89 (58\%)}\\
\lspbottomrule
\end{tabularx}
\caption{Relation to Latin in V1 declaratives from Bede depending on function}
\label{tab:cichosz:4}
\end{table}


\REF{ex:cichosz:11} is an example of a V1\is{verb-first} declarative\is{verb-initial declaratives} with a narrative function which was classified as  modelled\is{transfer from Latin} on the Latin since both languages have the verb in the clause-initial position, and the subject is expressed later in the clause. It turns out that as many as 62\% of narrative V1\is{verb-first} clauses in \isi{Bede} may be treated as inspired by Latin, though the degree of closeness is not always the same (e.g. in many cases there is no subject in the Latin source).

\ea \label{ex:cichosz:11}%11
\ea \label{ex:cichosz:11a}
\gll Forðferde   se     arwyrða     fæder Cuðbyrht   in   Farne     ðæm     ealonde;\\
forth-went  the.\textsc{nom}  honorable.\textsc{nom}  father.\textsc{nom} Cuthbert.\textsc{nom}  in   Farne.\textsc{dat}  the.\textsc{dat}   island.\textsc{dat}\\
\glt
\ex \label{ex:cichosz:11b}
\gll Obiit   autem   pater     reuerentissimus     in   insula Farne \\
died  also  father.\textsc{nom} {most honourable.\textsc{nom}}  in  island.\textsc{abl} Farne.\textsc{abl}\\
\glt ‘The honourable father Cuthbert died on the Isle of Farne.'\\ \hfill [cobede, Bede\_4:30.374.1.3734]
\z 
\z

Nonetheless, \isi{independent} uses such as \REF{ex:cichosz:12}, which shows a substantially paraphrased fragment, are well attested in the text as well. This, together with the fact that narrative V1\is{verb-first} declaratives\is{verb-initial declaratives} appear in many original OE\il{Old English} prose works and in poetry, may be treated as a clear indication of the native origin of the structure, but it seems that the actual frequency of the structure in some (translated) texts may be inflated by Latin influence\is{transfer from Latin}.

\ea    \label{ex:cichosz:12}%12
\ea    \label{ex:cichosz:12a}
\gll Com   se     foresprecena   hungur     eac   swylce hider   on   Bryttas \\
came  the.\textsc{nom}  aforesaid.\textsc{nom} famine.\textsc{nom}  also   such hither  on  Britons.\textsc{acc}\\
\glt ‘Likewise the aforesaid famine reached the Britons.' 
\ex    \label{ex:cichosz:12b}
\gll INTEREA   Brettones   fames     sua     praefata        magis    magis-que   adficiens,   ac   famam       suae malitiae     posteris   diuturnam   relinquens \\
    meanwhile  Britons.\textsc{nom}  famine.\textsc{acc}  their.\textsc{acc} aforesaid.\textsc{acc} greatly greatly-and  suffering  and  reputation.\textsc{acc}    their.\textsc{acc} wickedness.\textsc{gen}   posterity.\textsc{dat}  lasting.\textsc{acc}  leaving\\
\glt ‘Meanwhile the Britons, suffering more and more from the aforesaid famine, and leaving the long-lasting reputation of their malice to the posterity….' \hfill [cobede, Bede\_1:11.48.19.434]
\z 
\z
           
In the case of quotative V1\is{verb-first}, the ratio of \isi{independent} uses to  modelled\is{transfer from Latin} clauses is roughly 1 to 1, while in the case of descriptive V1\is{verb-first}, which has the most limited distribution in the corpus, examples which clearly do not follow the Latin source text are a visible majority (58\%). This is a rather surprising result since the easiest explanation for the use of this untypical structure in OE\il{Old English} prose records would be to treat it as a foreign structure transferred\is{transfer from Latin} from Latin into two translations (\isi{Bede} and Alexander's Letter\is{Alexander's Letter to Aristotle}) and two collections of homilies, which – even though not translated directly from a particular Latin source – were most probably based on a number of different religious texts written in Latin, like most OE\il{Old English} prose texts \citep{Stanton2002}. This hypothesis, however, is not corroborated by the data, since most examples of descriptive V1\is{verb-first} clause resemble \REF{ex:cichosz:13}, where no inspiration for the clause-initial placement of the verb may be identified in the Latin text.

\ea \label{ex:cichosz:13}%13

\ea \label{ex:cichosz:13a}
\gll Wæron   hi     begen     on   ciriclicum   wisum ge   on wisdome     haligra     gewrita     genihtsumlice   gelæred\\
were  they.\textsc{nom}  both.\textsc{nom}  on  church.\textsc{dat}  ways.\textsc{dat} and  on wisdom.\textsc{dat}  holy.\textsc{gen}  scriptures.\textsc{gen}  enough educated\\
\glt
\ex \label{ex:cichosz:13b}
\gll ambo     et   in   rebus     ecclesiasticis,     et in   scientia   scripturarum   sufficienter   instructi.\\
both.\textsc{nom}  and  in  matters.\textsc{abl}  ecclesiastic-\textsc{abl}  and in  science.\textsc{abl}   scriptures.\textsc{gen}  enough    educated\\
\glt ‘Both were adequately educated in matters related to the Church and in the science of the scripture.' \hfill [cobede, Bede\_5:16.446.20.4489]
\z 
\z

Naturally, the possibility of copying is unquestionable in cases such as \REF{ex:cichosz:14}, but the number of such instances in the text is too low to give proper justification to the hypothesis of foreign transfer\is{transfer from Latin}.

\ea \label{ex:cichosz:14}%14
\ea \label{ex:cichosz:14a}
\gll Wæs   þes     ilca     Æðelbehrt   Eormanrices sunu;\\
was  this.\textsc{nom}  same.\textsc{nom}  Ethelbert.\textsc{nom}  Eormenric.\textsc{gen} son.\textsc{nom}\\
\glt
\ex \label{ex:cichosz:14b}
\gll Erat   autem   idem     Aedilberct   filius     Irminrici, \\
was  also  this.\textsc{nom}  Ethelbert.\textsc{nom}  son.\textsc{nom}  Eormenric.\textsc{gen}\\
\glt ‘This Ethelbert was the son of Eormenric.'\\ \hfill [cobede, Bede\_2:5.110.16.1037]
\z 
\z

In short, whatever the origin and motivation for the use of descriptive V1\is{verb-first} is, Latin cannot be treated as its source and it should rather be interpreted as an additional factor, possibly increasing the frequency of the structure in the translated texts which use it.


\subsection{Does the subject matter?}\label{sec:cichosz:4.3}
So far, the analysis has not identified a clear motivation for the extremely skewed distribution of V1\is{verb-first} declaratives\is{verb-initial declaratives} (especially those with the descriptive function) in OE\il{Old English} prose records. The context-based analysis of V1\is{verb-first} clauses, however, makes it possible to notice a certain regularity in the data, namely that V1\is{verb-first} declaratives\is{verb-initial declaratives} with an overt subject tend to be followed by V1\is{verb-first} clauses without a subject, just like in \REF{ex:cichosz:15}. Here, a V1\is{verb-first} declarative\is{verb-initial declaratives} with a narrative function is immediately followed by a subjectless V1\is{verb-first} declarative\is{verb-initial declaratives} with a descriptive function, providing additional information about the referent mentioned in the preceding clause.

\ea \label{ex:cichosz:15}
\gll Feng   to   his     rice     Osred     his     sunu; \textbf{wæs}   \textbf{eahtawintre} \textbf{cniht}\\
took   to   his.\textsc{dat}  kingdom.\textsc{dat}   Osred.\textsc{nom}   his.\textsc{nom}   son.\textsc{nom} was   eight-winter.\textsc{nom}   boy.\textsc{nom}\\
\glt ‘Osred, his son, took over the throne; (he) \textbf{was} \textbf{an} \textbf{eight-year} \textbf{boy}.' \\ \hfill [Bede 5, [037800 (16.446.4)]]
\z 

In \isi{Bede}, where V1\is{verb-first} declaratives\is{verb-initial declaratives} are so frequent, it is possible to come across several examples of such V1\is{verb-first} clusters, though the subjectless V1\is{verb-first} is not necessarily descriptive. In \REF{ex:cichosz:16} a V1\is{verb-first} narrative clause with a subject (\textit{seo cwen} {‘the queen') is followed by two subjectless V1\is{verb-first} clauses which clearly operate on the highest level of the narrative structure, presenting subsequent events in the story.}

\ea \label{ex:cichosz:16}
\gll Wæs   heo     seo     cwen     sona   lustfulliende þære     godan     foresetenesse   \& willan     þæs iungan. \textbf{Sende}   \textbf{him}     \textbf{þa}   \textbf{to}   \textbf{Cent}     \textbf{to} \textbf{Ercenbyrhte} \textbf{þam}     \textbf{cyninge},   se   wæs   hyre     eames     sunu; \textbf{bæd}   \textbf{þæt}   \textbf{he}     \textbf{hine} \textbf{arwurðlice}   \textbf{to}   \textbf{Rome} \textbf{onsende}\\
was   she.\textsc{nom}   the.\textsc{nom}   queen.\textsc{nom}   soon   glad the.\textsc{dat}   good.\textsc{dat}    purpose.\textsc{dat}   and   will.\textsc{dat}   the.\textsc{gen} young.\textsc{gen}   sent   him.\textsc{acc}   then   to   Kent.\textsc{acc}   to Ercenbyrht.\textsc{dat} the.\textsc{dat}   king.\textsc{dat}  this.\textsc{dat} was   her.\textsc{gen}  uncle.\textsc{gen}  son.\textsc{nom} asked   that   he.\textsc{nom}   him.\textsc{acc}   kindly     to   Rome.\textsc{dat} sent\\
\glt ‘She, the queen, was soon glad about the good purpose and will of the young man. \textbf{(She)} \textbf{sent} \textbf{him} \textbf{then} \textbf{to} \textbf{Kent} \textbf{to} \textbf{Ercenbyrht} \textbf{the} \textbf{king}, he was her uncle's son; \textbf{(she)} \textbf{asked} \textbf{that} \textbf{he} \textbf{kindly} \textbf{sent} \textbf{him} \textbf{to} \textbf{Rome}.' \\ \hfill [Bede 5, [042100 (17.452.19--21)]]
\z 

Thus, it seems that V1\is{verb-first} declaratives\is{verb-initial declaratives} with a subject are not that specific in their functional range because, at least in \isi{Bede}, the same functions are performed by subjectless V1\is{verb-first} clauses. A similar phenomenon seems to operate in Lives of Saints\is{Ælfric's Lives of Saints}. Here, subjectless V1\is{verb-first} clauses are narrative in function as in \REF{ex:cichosz:17}, and they are rarely preceded by a V1\is{verb-first} clause with a subject, because the latter are not that frequent in this text.

\ea%15
\label{ex:cichosz:17}
\gll And   he     leop   sona   cunnigende   his     feðes,
hwæðer   he      cuðe   gan. \textbf{Eode}   \textbf{þa}   \textbf{mid}   \textbf{blisse}
\textbf{binnan}     \textbf{þam}     \textbf{temple}     \textbf{mid}   \textbf{þam}     \textbf{halgum} \textbf{apostolum,}   \textbf{þone}     \textbf{hælend}   \textbf{herigende}.   Þa   oncneowan hine     ealle     þe   hine     cuðon   ær,   and
micclum   wundrodon   þæs     wædlan   hæle.\\
and   he.\textsc{nom}   ran   soon  testing     his.\textsc{acc}   feet.\textsc{acc} whether   he.\textsc{nom}   could   walk   went   then   with glory.\textsc{dat}  into     the.\textsc{dat}   temple.\textsc{dat}   with   the.\textsc{dat}   holy.\textsc{dat} apostles.\textsc{dat}   the.\textsc{acc}  Saviour.\textsc{acc}   praising   then   recognised him.\textsc{acc}   all.\textsc{nom}   that   him.\textsc{acc}   knew   before   and greatly     wondered   the.\textsc{gen}   poor.man.\textsc{gen}   health.\textsc{dat}\\
\glt ‘And he soon ran, testing his feet, whether he could walk. \textbf{(He)} \textbf{went} \textbf{then} \textbf{with} \textbf{glory} \textbf{to} \textbf{the} \textbf{temple} \textbf{with} \textbf{the} \textbf{holy} \textbf{apostles,} \textbf{praising} \textbf{Jesus}. Then everyone who had known him recognised him and wondered greatly at the health of the poor man.' \hfill [ÆLS, Peter's Chair, 001400 (32--36)]
\z 

The number of non-conjunct main declarative clauses without a subject is not particularly high in OE\il{Old English} prose. \tabref{tab:cichosz:5} shows that the whole \isi{YCOE} contains only around a thousand examples of the structure, but the interesting thing is that in most texts V1\is{verb-first} clauses without a subject clearly outnumber those with a subject. The only exceptions to this rule are (again) the “top four” texts: \isi{Bede}, \isi{Vercelli Homilies}, Blickling Homiles and Alexander's Letter,\is{Alexander's Letter to Aristotle} where the frequency of V1\is{verb-first} declaratives\is{verb-initial declaratives} with an overt subject is not only exceptionally high but higher than the frequency of subjectless V1\is{verb-first} clauses (it should also be noted that in one of \isi{Ælfric}'s texts, Catholic Homilies I, the two patterns are quite balanced, whereas in his other works the tendency is the same as in most \isi{YCOE} texts).

\begin{table}
\begin{tabularx}{0.8\textwidth}{Xrrr}
\lsptoprule
{text} & \multicolumn{1}{l}{with S} & \multicolumn{1}{l}{without S}\\
\midrule
{Bede's History of the English Church}  & {459} & {195}\\
{Vercelli Homilies} & {67} & {28}\\
{Blickling Homilies} & {45} & {26}\\
{Alexander's Letter to Aristotle} & {42} & {9}\\
{Ælfric's Catholic Homilies I}  & {35} & {36}\\
{Ælfric's Lives of Saints}  & {21} & {236}\\
{Ælfric's Catholic Homilies II}  & {20} & {97}\\
{Anglo-Saxon Chronicle D} & {13} & {35}\\
{Boethius Consolation of Philosophy}  & {11} & {38}\\
{Heptateuch} & {10} & {30}\\
{Anglo-Saxon Chronicle C} & {10} & {32}\\
\midrule
{YCOE (total)} & {840} & {1,069}\\
\lspbottomrule
\end{tabularx}
\caption{V1 declaratives with and without an overt subject in YCOE texts}
\label{tab:cichosz:5}
\end{table}

In order to investigate the possible interrelations between V1\is{verb-first} declaratives\is{verb-initial declaratives} with and without a subject, I conducted a case study of two texts where both patterns are attested, but they seem to be on two extremes regarding the analysed structure: \isi{Bede} (the whole text was taken into account) and Lives of Saints\is{Ælfric's Lives of Saints} (only the books where V1\is{verb-first} clauses with a subject are used at least once, i.e. Book of Kings (28 V1\is{verb-first} clauses without a subject), Martin (25), Basil (25, Peter's Chair (18) and Maccabees (16), 112 (47\%) clauses in total). The classification was based on the labels determined for V1\is{verb-first} declaratives\is{verb-initial declaratives} with a subject (narrative, with presentational\is{presentational constructions} as a subtype, quotative and descriptive). \tabref{tab:cichosz:6} presents the outcomes of this analysis (based on raw numbers).

\begin{table}
\begin{tabularx}{0.8\textwidth}{Xrrrr}
\lsptoprule
& \multicolumn{2}{c}{{LoS}} & \multicolumn{2}{c}{{Bede}}\\
& \multicolumn{1}{l}{V1 with S} & \multicolumn{1}{l}{V1 no S} & \multicolumn{1}{l}{V1 with S} & \multicolumn{1}{l}{V1 no S}\\
\midrule
{narrative} & {20 (95\%)} & {89 (79\%)} & {213 (46\%)} & {123 (64\%)}\\
{quotative} & {1 (5\%)} & {23 (21\%)} & {93 (20\%)} & {35 (18\%)}\\
{descriptive} & {0 (0\%)} & {0 (0\%)} & {153 (33\%)} & {35 (18\%)}\\
{all} & {21} & {112} & {459} & {193}\\
\lspbottomrule
\end{tabularx}
\caption{A case study of the function of V1 declaratives with and without S in Bede and Lives of Saints}
\label{tab:cichosz:6}
\end{table}

It turns out that there may be some connection between the two patterns since \isi{Ælfric} consistently avoids the descriptive function in both cases, employing V1\is{verb-first} declaratives\is{verb-initial declaratives} (with or without a subject, mostly the latter) to push the \isi{narration} forward or introduce a quote. In \isi{Bede}, on the other hand, all the functions are attested and even though the proportions are not identical, it cannot be said that the presence of the subject makes a drastic difference. This result makes it possible to draw a tentative conclusion: Perhaps it is the clause-initial position of the finite verb which is responsible for the pragmatic function of the clause, and the subject is not a crucial element of the pattern at all. 

\begin{table}[t]
\begin{tabularx}{0.8\textwidth}{Xrrr}
\lsptoprule
& \multicolumn{1}{l}{personal pronoun} & \multicolumn{1}{l}{noun} & \multicolumn{1}{l}{other pronoun} \\
\midrule
narrative & 140 (31\%) & 253 (56\%) & 58 (13\%)\\
quotative & 78 (57\%) & 44 (32\%) &  14 (10\%)\\
descriptive & 125 (49\%) & 109 (43\%) & 19 (7\%)\\
\lspbottomrule
\end{tabularx}
\caption{Subject types in V1 declaratives with an overt S in YCOE}
\label{tab:cichosz:7}
\end{table}

Comparing subject types between all the functions of V1\is{verb-first} declaratives\is{verb-initial declaratives} identified in this study in \isi{YCOE} (\tabref{tab:cichosz:7}), it turns out that while the proportion of other (non-personal) pronouns is similar, personal pronouns are well-represented in the quotative and descriptive function, and less frequent in narrative V1\is{verb-first}, which is quite logical, considering the fact that \isi{narrative inversion} often introduces new characters\is{character} to the story.\is{discourse-new} The tendency becomes completely clear when nominal subjects are divided into proper nouns, nouns with overt marking of definiteness (modification by a demonstrative\is{demonstrative} pronoun, a possessive pronouns or a genitive) and without it (bare nouns, optionally modified by adjectives and quantifiers such as \textit{sum} “some”). \tabref{tab:cichosz:8} shows that in the case of descriptive V1\is{verb-first} clauses, the subject is rarely discourse-new,\is{discourse-new} even if it is nominal, while narrative V1\is{verb-first} clauses co-occur with indefinite noun phrases in 25\% of the cases (112 out of 451).

\begin{table}
\begin{tabularx}{0.8\textwidth}{Xrrr}
\lsptoprule
& \multicolumn{1}{l}{definite marking} & \multicolumn{1}{l}{no marking} & \multicolumn{1}{l}{proper noun}\\
\midrule
narrative & 105 (41\%) & 112 (44\%) & 36 (14\%) \\
quotative & 21 (48\%) & 17 (39\%) & 6 (14\%)\\
descriptive & 90 (83\%) & 2 (2\%) & 17 (16\%)\\
\lspbottomrule
\end{tabularx}
\caption{Subtypes of nominal subjects in V1 declaratives in YCOE}
\label{tab:cichosz:8}
\end{table}

In addition, it should be noted that 117 out of 140 cases of narrative V1\is{verb-first} with personal pronoun subjects (84\%) come from the “top four” texts, another 9 from one of the manuscripts of the Anglo-Saxon Chronicle and another 4 from \isi{Vercelli Homilies} E (130 out of 140, 93\% altogether), so the distribution of the structure is very limited.

Moreover, when adding clauses without an overt subject to the equation, an interesting pattern emerges. \tabref{tab:cichosz:9} shows the proportions for \isi{Bede}, \tabref{tab:cichosz:10} for the sample of Lives of Saints.\is{Ælfric's Lives of Saints}

\begin{table}
\begin{tabularx}{\textwidth}{Xrrrrrr}
\lsptoprule
& \multicolumn{1}{Q}{personal pronoun} & \multicolumn{1}{Q}{definite NP} & \multicolumn{1}{Q}{indefi-nite NP} & \multicolumn{1}{Q}{proper N} & \multicolumn{1}{Q}{other pronoun} & \multicolumn{1}{Q}{no S}\\
\midrule
narrative &   {98 (29\%)} & {42 (12\%)} & { 39 (11\%)} & { 21 (6\%)} & {13 (4\%)} & {123 (37\%)}\\
quotative &  {69 (54\%)} & {4 (3\%)} & { 8 (6\%)} & { 2 (2\%)} & {10 (8\%)} & {35 (27\%)}\\
descriptive &  {75 (40\%)} & {53 (28\%)} & { 1 (1\%)} & { 14 (7\%)} & {10 (5\%)} & {35 (19\%)}\\
\lspbottomrule
\end{tabularx}
\caption{Subject types in V1 declaratives in Bede}
\label{tab:cichosz:9}
\end{table}


\begin{table}
\begin{tabularx}{\textwidth}{Xrrrrrr}
\lsptoprule
& \multicolumn{1}{Q}{personal pronoun} & \multicolumn{1}{Q}{definite NP} & \multicolumn{1}{Q}{indefi-nite NP} & \multicolumn{1}{Q}{proper N} & \multicolumn{1}{Q}{other pronoun} & \multicolumn{1}{Q}{no S}\\
\midrule
narrative & {0 (0\%)} & {9 (8\%)} & { 8 (7\%)} & { 0 (0\%)} & {3 (3\%)} & {87 (81\%)}\\
{quotative} &  {0 (0\%)} & {0 (0\%)} & { 1 (4\%)} & { 0 (0\%)} & {0 (0\%)} & {23 (96\%)}\\
\lspbottomrule
\end{tabularx}
\caption{Subject types in V1 declaratives in LoS sample}
\label{tab:cichosz:10}
\end{table}

{It turns out that in Lives of Saints\is{Ælfric's Lives of Saints} (LoS) the proportion of V1\is{verb-first} declaratives\is{verb-initial declaratives} without an overt subject is much higher in all of the attested functions than in \isi{Bede} (81\% vs. 37\% for the narrative function, 96\% vs. 27\% for the quotative function; the descriptive function is not used in the former text), while personal pronouns are never used. Thus, it turns out that there are two important differences between the analysed texts: a complete lack of V1\is{verb-first} declaratives\is{verb-initial declaratives} with a descriptive function in LoS\is{Ælfric's Lives of Saints} and consistent avoidance of V1\is{verb-first} declaratives\is{verb-initial declaratives} with personal pronoun subjects by \isi{Ælfric}. Interestingly, when the number of clauses with personal pronoun subjects and no overt subjects in \isi{Bede} are added, the resulting proportions are much closer to the ones in LoS\is{Ælfric's Lives of Saints} (66\% vs. 81\% for the narrative function, 81\% vs. 96\% for the quotative function). The difference between \isi{Bede} and LoS\is{Ælfric's Lives of Saints} in subject types in the narrative function proves significant when personal pronouns and clauses without overt subjects are counted separately (X}{\textsuperscript{2}}{~(3, N = 437) = 77.9917, p < 0.00001)}\footnote{ {Subtypes of nominal subjects are merged for the purposes of the statistical analysis since some of them are not represented in LoS.}}{, and even when the two categories are merged, the difference remains significant (X}{\textsuperscript{2}}{~(2, N = 437) = 16.1207, p = 0.000316). It seems that narrative V1\is{verb-first} declaratives\is{verb-initial declaratives} with no overt subject such as \REF{ex:cichosz:18} and those with a personal pronoun subject such as \REF{ex:cichosz:19} are variants of the same structure whose purpose is to introduce subsequent events in the story when the main protagonist remains the same, and there is no need to change the subject.}\footnote{ {A similar observation was made with regard to Old Icelandic (\citealt{booth-beck20200jhs}).}}

\ea%18
\label{ex:cichosz:18}
\gll Þa   gelæhte   Petrus     hire     liþian     hand,
\textbf{arærde}   \textbf{hi}     \textbf{upp}   \textbf{hale}
\textbf{of}   \textbf{þam}     \textbf{bedde.}\\
then  took    Peter.\textsc{nom}  her.\textsc{acc}  supple.\textsc{acc}  hand.\textsc{acc}
raised    her.\textsc{acc}  up  healthy.\textsc{acc}   of  the.\textsc{dat}  bed.\textsc{dat}\\
\glt ‘Then Peter took her supple hand, (he) \textbf{raised} \textbf{her} \textbf{up} \textbf{healthy} \textbf{from} \textbf{the} \textbf{bed}.' \hfill [ÆLS (Peter's Chair) [002800 (73)]]
\z 

\ea%19
\label{ex:cichosz:19}
\gll Þa   geseah   he     swa   þeostre     dene     ane
under   him     in   niþernesse   gesette.   \textbf{Geseah}   \textbf{he}
\textbf{eac}   \textbf{feower}     \textbf{fyr}     \textbf{onæled}   \textbf{on}   \textbf{þære}     \textbf{lyfte}
\textbf{noht}   \textbf{micle}     \textbf{fæce}     \textbf{betweoh}   \textbf{him}     \textbf{tosceaden.} \\
then  saw  he.\textsc{nom}  such  dark.\textsc{acc}  valley.\textsc{acc}  one.\textsc{acc}
under  him.\textsc{dat}  in  bottom.\textsc{dat}  placed    saw    he.\textsc{nom}
also  four.\textsc{acc}  fires.\textsc{acc}  lit    on   the.\textsc{dat}  air.\textsc{dat}
not   great.\textsc{dat}  space.\textsc{dat}  between  them.\textsc{dat}  separated\\
\glt ‘Then he saw a deep valley beneath. \textbf{He} \textbf{saw} \textbf{also} \textbf{four} \textbf{fires} \textbf{lit} \textbf{in} \textbf{the} \textbf{air} \textbf{not} \textbf{far} \textbf{away} \textbf{from} \textbf{one} \textbf{another}.' \\ \hfill [Bede 3, [040400 (14.212.20)]{}--[040500 (14.212.22)]]
\z 

When the subject is a noun, the narrative V1\is{verb-first} clause often uses a new\is{discourse-new} subject as in the clause with \textit{se cyning} ‘the king' shown \REF{ex:cichosz:20}, or there is a subject shift. Descriptive V1\is{verb-first} clauses may also use a nominal subject, but in such a case, the subject is never new but rather old\is{discourse-old} or accessible (as in \textit{his baan} ‘his bones' in the same example).

\ea%20
\label{ex:cichosz:20}

\gll Forðon     þa   he     wæs   mid   wæpnum   \& mid
feondum   eall   utan   beheped,  \& he     seolfa  
onget   þæt   hine     mon      ofslean    scolde,   þa   gebæd
he     for   þam     sawlum   his     weorodes. Cwædon   heo     bi   ðon   þus   in   gydde:     Drihten
God     miltsa   þu     sawlum   ussa     leoda, 
cwæð   se     halga     Oswald,    þa   he     on
eorðan     saag.   \textbf{Wæron}   \textbf{his}    \textbf{baan}     \textbf{gelæded}   \textbf{\&}
\textbf{gehealden}   \textbf{in}   \textbf{þæm}     \textbf{mynstre},     þe   we
ær   cwædon   æt   Beardan   ea.     \textbf{Heht}  \textbf{se}
\textbf{cyning},    \textbf{se}     \textbf{ðe}   \textbf{hine}     \textbf{slog},   \textbf{his}   \textbf{heafod}     \textbf{on}
\textbf{steng}     \textbf{asetton};\\
because    then  he.\textsc{nom}  was  with  weapons.\textsc{dat}   and  with
enemies.\textsc{dat}  all  outside  cut.off    and  he.\textsc{nom}  self
knew  that  him.\textsc{acc}  one.\textsc{nom}   kill  shall     then  prayed
he.\textsc{nom}  for  the.\textsc{dat}  souls.\textsc{dat}  his.\textsc{gen}   army.\textsc{gen} said    they.\textsc{nom}  by  this  thus  in  song.\textsc{dat}  lord.\textsc{nom}
God.\textsc{nom}  pity  you.\textsc{nom}  souls.\textsc{dat}  our.\textsc{gen}  people.\textsc{gen}
said  the.\textsc{nom}  holy.\textsc{nom}  Oswald.\textsc{nom}   when  he.\textsc{nom}  on
earth.\textsc{acc}  sank  were    his.\textsc{nom}  bones.\textsc{nom}  taken    and
held    in  the.\textsc{dat}  monastery.\textsc{dat}    which  we.\textsc{nom}
before  said    at   Bardney.\textsc{dat}  river.\textsc{dat}   ordered  the.\textsc{nom}
king.\textsc{nom}  this.\textsc{nom}  who  him.\textsc{acc}  killed  his.\textsc{acc} head.\textsc{acc}  on
pole.\textsc{acc}  put\\
\glt ‘Because he was surrounded on all sides by armed enemies, and he understood that he would be killed, he prayed for the souls of his army. Thus they said it in a song: ‘Lord God, have mercy on the souls of our people, said the holy Oswald, when he fell to the ground.' \textbf{His} \textbf{bones} \textbf{were} \textbf{taken} \textbf{and} \textbf{preserved} \textbf{in} \textbf{the} \textbf{monastery} \textbf{at} \textbf{Bardney} \textbf{river}, which we mentioned before. \textbf{The} \textbf{king} \textbf{who} \textbf{slew} \textbf{him} \textbf{ordered} \textbf{that} \textbf{his} \textbf{head} \textbf{should} \textbf{be} \textbf{set} \textbf{on} \textbf{a} \textbf{pole}.' \hfill [Bede 3 [024700 (10.188.13)]--[025000 (10.188.20)]]
\z 

In LoS,\is{Ælfric's Lives of Saints} nominal subjects used in \isi{narrative inversion} (infrequent as they are) behave in a similar way, if they are not discourse-new,\is{discourse-new} there is a change of protagonist in the story as in \REF{ex:cichosz:21}, when Avitianus is introduced, but then the story switches to the actions of St. Martin in order to get back to Avitianus a few clauses later.

\ea%21
\label{ex:cichosz:21}
\gll Auitianus     hatte     sum     hetol     ealdorman,
wælhreow   on   his     weorcum,   se     gewrað
fela     manna,     and   on   racenteagum   gebrohte   to þære     byrig     Turonia,   wolde   hi     þæs     on 
mergen     mislice     acwellan   ætforan   þære 
burhware,   þa   wearð   hit   þam     bisceope   cuð.
Þa   smeade   se     halga     wer     hu   he   
heora     gehelpan   mihte,   and   eode   to   middre
nihte     ana     to   his     gatum,     and   þa
þa   he     inn   ne   mihte,   he     anbidode   þærute. 
\textbf{Wearð}   \textbf{þa}   \textbf{se}     \textbf{ealdorman}     \textbf{awreht}     \textbf{færlice} 
\textbf{þurh}   \textbf{Godes}     \textbf{engel},     and   he     him   gramlice   to 
cwæð,   List   ðu     and   rest   þe,   and   Godes
þeowa      lið   æt   þinum     gatum?\\
Avitianus.\textsc{nom}    was.called  some.\textsc{nom}  cruel.\textsc{nom}  commander.\textsc{nom}
savage.\textsc{nom}  on  his.\textsc{dat}   deeds.\textsc{dat}  who.\textsc{nom}  bound
many.\textsc{acc}  men.\textsc{gen}  and  on  chains.\textsc{dat}  brought    to  the.\textsc{dat}  city.\textsc{dat}  Tour.\textsc{dat}  would  they.\textsc{acc}  this.\textsc{gen}  on
morning.\textsc{dat}  cruelly    kill    before    the.\textsc{dat} 
citizens.\textsc{dat}   then  became  it.\textsc{nom}  the.\textsc{dat}  bishop.\textsc{dat}    known
then  thought  the.\textsc{nom}  holy.\textsc{nom}  man.\textsc{nom}  how  he.\textsc{nom}
they.\textsc{gen}  help    might  and  went  to  middle.\textsc{dat}
night.\textsc{dat}  alone.\textsc{nom}  to  his.\textsc{dat}  gates.\textsc{dat}  and  then
when  he.\textsc{nom}  in  not   might  he.\textsc{nom}  waited    outside
became  then  the.\textsc{nom}  commander.\textsc{nom}  awoken    suddenly
through  God.\textsc{gen}  angel.\textsc{acc}  and  he.\textsc{nom}  he.\textsc{dat}  sternly    to
said  lie  you.\textsc{nom}  and  rest  REFL  and  God.\textsc{gen}
servant.\textsc{nom}  lies  at   your.\textsc{dat}   gates.\textsc{dat}\\
\glt ‘There was a cruel commander called Avitianus who bound many men and brought them in chains to the city of Tour, (he) wanted to kill them cruelly in front of the citizens the next morning. Then the bishop heard about it and considered how he could help them. He went in the middle of the night alone to the city gates and when he could not get in, he waited outside. \textbf{Then} \textbf{the} \textbf{commander} \textbf{was} \textbf{suddenly} \textbf{awoken} \textbf{by} \textbf{God's} \textbf{angel} and he sternly told him “You are lying and resting and God's servant is lying at your gates.' \hfill [ÆLS (Martin), [028200 (1143)]--[028400 (1151)]]
\z 

In short, it seems that in the fragments with a new\is{discourse-new} topic or \isi{topic shift} \isi{narrative inversion} with a nominal subject is used, and in the cases with topic continuation, there is a text-dependent variation between V1\is{verb-first} declaratives\is{verb-initial declaratives} without an overt subject (attested in \isi{Ælfric} and many other, mostly late, texts) and with a personal pronoun subject (attested in \isi{Bede}, \isi{Vercelli Homilies}, Blickling Homiles and Alexander's Letter,\is{Alexander's Letter to Aristotle} which are all early texts). Thus, the most probable variable responsible for this variation is diachrony.


\section{Summary and conclusions}
\label{sec:cichosz:5}

The study confirms that V1\is{verb-first} declaratives\is{verb-initial declaratives} are a rare structure with a skewed distribution, but they are present in most OE\il{Old English} prose texts, both early and late, translated and non-translated, written in West Saxon or showing mixed dialect features. While the frequency of V1\is{verb-first} declaratives\is{verb-initial declaratives} in \isi{Bede} is inflated by Latin, uses \isi{independent} of the Latin source correspond to all the functions of V1\is{verb-first} identified in OE\il{Old English} prose, so one cannot treat any of them exclusively as examples of foreign transfer\is{transfer from Latin}. Since no clear variable connects the compositions where the pattern is used, it seems to have been a stylistic choice of the (mostly) anonymous Anglo-Saxon writers, perhaps inspired by OE\il{Old English} poetry where this discourse device is used most regularly (\citealt{Mitchell1985}: $§$3932).

The basic function of V1\is{verb-first} declaratives\is{verb-initial declaratives} is narrative, that is, they are used to introduce subsequent events in the story and represent clearly \isi{coordinating discourse relations}. This function has the highest corpus frequency (53\% of all examples) and the most balanced textual distribution (despite generally low frequency, single examples are attested in numerous texts). Next, there are descriptive V1\is{verb-first} clauses, operating in the sub-structure of the text, that is, providing additional information about known discourse referents and functioning as examples of elaboration and explanation. Such clauses, frequent as they are if \isi{YCOE} is analysed as a whole (31\% of the whole dataset), are largely restricted to 4 texts regularly mentioned in this study: Bede's Historia,\is{Bede’s Historia Ecclesiastica}\is{Bede} \isi{Vercelli Homilies}, \isi{Blickling Homilies} and Alexander's Letter.\is{Alexander's Letter to Aristotle} Finally, the last function attested in the data is the V1\is{verb-first} reporting construction,\is{reporting clause} which constitutes only 16\% of all examples, and its distribution is also rather restricted.

{The analysis suggests that V1\is{verb-first} descriptive clauses are an early native structure (they are absent from all late texts, including \isi{Ælfric}'s works), whose function must have been taken over by other, probably more hypotactic\is{hypotaxis} structures, relative clauses\is{relative clause} for example. Narrative V1\is{verb-first} was considerably more conventionalised in OE\il{Old English}, since it appears in numerous texts, covering both early and late records. It must be noted that narrative V1\is{verb-first} declaratives\is{verb-initial declaratives} with pronominal subjects are used mostly by the “top four” early OE\il{Old English} texts; they alternate with V1\is{verb-first} without an overt subject in other \isi{YCOE} texts. This has some important implications for the syntactic interpretation of the structure: OE\il{Old English} is a V2\is{verb-second} language but its specificity is related to the behaviour of personal pronoun subjects: They usually fail to invert and there is a very specific group of closed{}-class elements which may trigger \isi{inversion} of a pronominal subject (this includes} {\textit{þa}} {‘then' and} {\textit{þonne}} {‘then',} {\textit{ne}} {‘not' and wh-words, cf. \citealt{RingeTaylor2014}: 399--400). The fact that V1\is{verb-first} declaratives\is{verb-initial declaratives} feature personal pronoun subjects has been a topic of major discussion among OE\il{Old English} syntacticians since it means that \isi{narrative inversion} must be structurally equalled with other pronoun-inverting cases representing V-to-C movement (\citealt{Pintzuk1999}; \citealt{Ohkado2004}), but it is unclear what actually triggers the pattern (\citealt{RingeTaylor2014}: 408). Considering the extremely limited distribution of \isi{narrative inversion} with pronominal subjects, the question is whether the structure should be included in the same category, on a par with other, more well attested patterns. Perhaps the stronger presence of V1\is{verb-first} declaratives\is{verb-initial declaratives} in early OE\il{Old English} data means that they are a remnant of an older, possibly Proto-Germanic, structure, which was already becoming obsolete. It is a fact that V1\is{verb-first} declaratives\is{verb-initial declaratives} disappeared from English soon after the end of the OE\il{Old English} period, and the constraint on \isi{personal pronoun inversion} was a factor which could have contributed to its growing limitation and final loss.}

All in all, the study reveals that V1\is{verb-first} (with or without an overt S) is an OE\il{Old English} discourse device creating dynamics in the story, and it is quite certain that it was one of many forms with such a function available to Anglo-Saxon writers. This suggests a number of potential directions for further study. First of all, in order to have a full picture of the situation, it is necessary to take a closer look at both prose and poetry records. We know that \isi{narrative inversion} is used quite regularly in OE\il{Old English} verse, but there are no studies that show whether descriptive V1\is{verb-first} declaratives\is{verb-initial declaratives} are attested in poetry and discussing the \isi{discourse function} of V1\is{verb-first} clauses without an overt subject. What is more, a promising direction for further study would be a function-to-form mapping of various discourse devices used for specific functions in OE\il{Old English} (opening a new story, summarising the discussion, transitioning to a new action,\is{discourse-new} introducing a new\is{discourse-new} \isi{character}, quoting speech, describing a known referent, etc.). Such a study would be difficult to design and would require quite a lot of close reading and contextual analysis, but this paper, among many other analyses of OE\il{Old English}, shows quite clearly that it is impossible to understand OE\il{Old English} discourse structure without a more qualitative approach to textual data.

\sloppy\printbibliography[heading=subbibliography,notkeyword=this]
\end{document} 

