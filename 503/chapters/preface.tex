\addchap{Preface}
\begin{refsection}
The current volume is part of a research project looking into the relationship between grammatical means and discourse structure in German from a diachronic perspective.  Of particular interest are word order patterns at the left edge of sentences to better understand the impact of verb placement as well as the shaping of the preverbal position on the composition of discourse and narrative structure. These relationships are being investigated using a sample of early prose novels in German written in the fifteenth and sixteenth centuries (\textit{Wortstellung und Diskursstruktur in der Frühen Neuzeit} `Word order and discourse structure in the Early Modern Period', funded by the German Research Foundation, project number 456973946). 

To enable an even deeper understanding of the relationship between grammar and narrative discourse, we invited colleagues working on similar issues to a workshop at the University of Potsdam. The present volume is the product of this workshop. It constitutes a significant extension of the research project in that it not only contains contributions on German but also addresses the relationship between grammar and narration in other Germanic languages. The spectrum of phenomena examined is furthermore significantly broadened by the volume: In addition to the influence of word order patterns, it deals with the discourse functions of adverbs, the verbalization of time, and the distribution of nominalizations in historical narrative texts, among other things. We see this volume as a stimulus for hopefully many more (diachronic) studies of narrative discourses in other (Germanic) languages, which will lead to a more profound understanding of the relationship between grammar and narration(s) in general.

\bigskip 

\noindent
Potsdam, December 2024 \hfill Ulrike Demske \& Barthe Bloom


\end{refsection}

