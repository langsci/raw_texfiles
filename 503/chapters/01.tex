\documentclass[output=paper,colorlinks,citecolor=brown]{langscibook}
\ChapterDOI{10.5281/zenodo.15689119}

\author{Ulrike Demske\affiliation{Universität Potsdam} and Barthe Bloom\orcid{0000-0003-3378-9856}\affiliation{Friedrich-Alexander-Universität Erlangen-Nürnberg}}
\title{Grammar in narration(s)} 
\abstract{}

% \IfFileExists{../localcommands.tex}{
%   \addbibresource{../localbibliography.bib}
%  \usepackage{langsci-optional}
\usepackage{langsci-gb4e}
\usepackage{langsci-lgr}

\usepackage{listings}
\lstset{basicstyle=\ttfamily,tabsize=2,breaklines=true}

%added by author
% \usepackage{tipa}
\usepackage{multirow}
\graphicspath{{figures/}}
\usepackage{langsci-branding}

% 
\newcommand{\sent}{\enumsentence}
\newcommand{\sents}{\eenumsentence}
\let\citeasnoun\citet

\renewcommand{\lsCoverTitleFont}[1]{\sffamily\addfontfeatures{Scale=MatchUppercase}\fontsize{44pt}{16mm}\selectfont #1}
  
% %% hyphenation points for line breaks
%% Normally, automatic hyphenation in LaTeX is very good
%% If a word is mis-hyphenated, add it to this file
%%
%% add information to TeX file before \begin{document} with:
%% %% hyphenation points for line breaks
%% Normally, automatic hyphenation in LaTeX is very good
%% If a word is mis-hyphenated, add it to this file
%%
%% add information to TeX file before \begin{document} with:
%% %% hyphenation points for line breaks
%% Normally, automatic hyphenation in LaTeX is very good
%% If a word is mis-hyphenated, add it to this file
%%
%% add information to TeX file before \begin{document} with:
%% \include{localhyphenation}
\hyphenation{
affri-ca-te
affri-ca-tes
an-no-tated
com-ple-ments
com-po-si-tio-na-li-ty
non-com-po-si-tio-na-li-ty
Gon-zá-lez
out-side
Ri-chárd
se-man-tics
STREU-SLE
Tie-de-mann
}
\hyphenation{
affri-ca-te
affri-ca-tes
an-no-tated
com-ple-ments
com-po-si-tio-na-li-ty
non-com-po-si-tio-na-li-ty
Gon-zá-lez
out-side
Ri-chárd
se-man-tics
STREU-SLE
Tie-de-mann
}
\hyphenation{
affri-ca-te
affri-ca-tes
an-no-tated
com-ple-ments
com-po-si-tio-na-li-ty
non-com-po-si-tio-na-li-ty
Gon-zá-lez
out-side
Ri-chárd
se-man-tics
STREU-SLE
Tie-de-mann
}
% \boolfalse{bookcompile}
% \togglepaper[23]%%chapternumber
% }{}
%%%%%%%%%%%%%%%%%%%%%%%%%%%%%%%%%%%%%%%%%%%%%%%%%%%%%%%%%%%%%%%%%%%%%%%%%%%%%%%%%%%%%
\begin{document}
\maketitle

%%%%%%%%%%%%%%%%%%%%%%%%%%%%%%%%%%%%%%%%%%%%%%%%%%%%%%%%%%%%%%%%%%%%%%%%%%%%%%%%
\section{Introduction}
The current volume deals with questions of information structuring at the discourse level, focusing on narrative discourses. More precisely, it is about the contribution of grammatical devices to the organization of texts, as well as their diagnostic potential for the narrative text type. Although it is well-known that information packaging had a much greater impact on the distribution of grammatical patterns in historical stages of a language than it does today \citep{herring2001,traugott2007,KeMi2012,BeSa2014}, so far many studies on the relationship between information structure and grammatical patterns do not go beyond the sentence level, in other words, they do not take into account the possible influence of the text type on the manifestation of certain grammatical patterns. How and to which degree changes in grammatical patterns correlate with or are affected by changes in discourse and narrative structure, how the two layers interact with each other and affect each other, and how such issues can be operationalized are still understudied. A noteworthy exception is \citet{Somers2024}, who presents a new perspective on the first literization of German. She takes into account, and thoroughly describes, the different needs of language of immediacy vs. distance and of oral vs. written communication \citep{koch1985sprache}, as well as the sociolinguistic and cognitive context in which the early German texts were created. She considers the relationship of the use of oral language in context of distance to the linguistic strategies of e.g., integration, nominalizations, and deictic expressions and argues, amongst other things, that it is the literization of the language that drives the shift from more paratactic to hypotactic structures \citep[196]{Somers2024}.

This volume aims to shed more light on these issues by presenting eight papers, which address these questions more or less explicitly. As the research questions imply, all contributions take a historical or diachronic perspective. Another commonality between the studies is that they all focus on data from Germanic languages, as we assume that by comparing closely related languages, the relations in question become more pronounced. Specifically, the languages in question are German, Dutch, English, and Icelandic. The introduction to this volume first presents some key concepts of narrative structure, focusing on narration as a text type or genre. This is followed by a discussion of narration in terms of discourse modes and discourse relations. In the final section of this chapter, the eight contributions are briefly presented. 

%%%%%%%%%%%%%%%%%%%%%%%%%%%%%%%%%%%%%%%%%%%%%%%%%%%%%%%%%%%%%%%%%%%%%%
\section{Discourse structure and narration}
From a linguistic point of view, narration can be viewed in three different ways: narration as a text type \citep{Labov1972}, narration as a discourse mode \citep{Smith2003}, and narration as a discourse relation \citep{AsherLascarides2003}. The latter two will be introduced in Section \ref{sec:1:3}, but we will first zoom in on the conception of narration as a text type, which surfaces, often implicitly, in all contributions in this volume.

%%%%%%%%%%%%%%%%%%%%%%%%%%%%%%%%%%%%%%%%%%%%%%%%%%%%%%%%%%%%%%%%%%%%%%%%%%
\subsection{Narration as text type}\label{sec:1:2}
What makes a narrative? The definition and delimitation of the narrative as a (overarching) text type, or as a genre, is not straightforward. Not only do narratives form a ``fuzzy set" \citep[193]{Ryan2006}, but even the archetype is heavily debated \citep[e.g.,][]{Ryan2017}. What sets narratives apart from other text types is their purpose of telling stories \citep[e.g.,][]{Labov2013}. It contrasts in this way with other genres, for example to argumentative and instructive genres, used for persuasion and teaching \citep{Fludernik2000}. Two crucial components of narratives are put forward in the literature: (i) events, which are changes of states, and (ii) actors, or characters that engage in or are affected by these events (cf. \citealt{Bal1985}, \citealt[56]{virtanen2011}, among others).

Events in a story are not random, but are temporally and/or causally related to each other (\citealt[16--18]{RimmonKenan1983}, \citealt[903]{Sternberg1990}). Whether and how these relations are presented as such varies and is a matter of \textit{order} \citep{Genette1972, Genette1980}. Consider classical Icelandic sagas as an example, like \textit{Brennu-Njáls saga}, which are typically organized by causally linked events: Although the settings and the characters that are in focus shift and are regularly abandoned, the narrative follows the linear order from cause to action to consequent (\citealt[23]{Loennroth1976}, \citealt{Korecka2019}). Often, this goes together nicely with a chronological representation of the story events. Anticipatory segments can be realized, but whenever they are, they tend to be anchored in the story-now, for example, in the form of prophecies and dreams \citep[31--32]{Loennroth1976}. This is illustrated by the segment from \textit{Njála} in \REF{ex:ch1.1}, in which the death of Þorvaldr is anticipated.

\ea \label{ex:ch1.1}
{\itshape Yngvildr, móðir þeira, var hjá stǫdd; hon grét, er hon heyrði, er þeir hétu alþingisferðinni. Þorkell mælti: “Hví grætr þú, móðir?” Hon svarar: “Mik dreymði, at Þorvaldr, bróðir þinn, væri í rauðum kyrtli, ok þótti mér svá þrǫngr vera sem saumaðr væri at honum; mér þótti hann ok vera í rauðum hosum undir ok vafit at vándum dreglum. Mér þótti illt á at sjá at honum var svá óhœgt, en ek mátta ekki at gera.”}\\
\glt `Yngvildr, their mother, was present, and when she heard them promise to ride to the Allthing, she started weeping. “Why do you weep mother?” asked Þorkell. “I dreamed,” she replied “that your brother Þorvaldr was wearing a red tunic and this tunic was so tight that it looked as if he had been sewn into it. He was also wearing [sic, read \textit{a}] red hose bound with shabby tapes. I was distressed to see him looking so uncomfortable, but I could not do anything to help him.”' \hfill \citep[51--52]{Loennroth1976}.
\z 

In Early New High German narrative prose, it is similarly typical to follow the temporal-causal order that is representative of the order of the story events. On occasion, extra-diegetic segments may be inserted, which present retroflective or anticipatory comments. The latter is illustrated in example \REF{ex:ch1.2}

\ea \label{ex:ch1.2} {\itshape Nu laßen wir von den reden vnd komen zu Bernhart vom steynrotsche wie der zu hoffe qwame.}\\
\glt `Now we leave that conversation and turn to Bernhart vom Steynrotsche, how he came to court.'\\
 \hfill \citep[Ponthus, 42vb, translation my own]{RokoUP}
\z

\noindent Such anachronisms are, however, quite short and do not normally intervene with the order of events. What does repeatedly re-occur is that multiple events happen simultaneously. While Middle High German narrations tended to be single-stranded narratives, i.e., they concentrated on one main narrative thread, with the “now” located in the narrative act, the prose novel with multiple strands becomes more dominant in early modern times \citep{Zeman2023c}. Such a change in narrative structure is expected to correlate with changes in the use and productivity of grammatical patterns and the freezing of expressions, like the one illustrated above, marking the transition from one strand to another.

The temporal organization of narratives may additionally exhibit variation in terms of \textit{duration} and \textit{frequency} \citep{Genette1972, Genette1980}. This is also captured by notions like \textit{rhythm} \citep{Bal1985} and \textit{narrative speed} \citep{Hume2005}. By duration, we mean the rate at which the story progresses in relation to the \isi{narration}, and frequency  concerns the repetition of events \citep{Genette1972}. Narrative speed is a technique that can be used to create tension or focus attention on a particular aspect of the story \citep{Hume2005, Kukkonen2020}. It is by definition related to the linguistic representation: The more linguistic material is used to express a single event, the slower the story time progresses in comparison to the narrative time. Concerning frequency, an implicit relation exists to the grammatical patterns associated with the contrast between given and new information.

Besides events, the actors, or characters are a crucial component of the story. Typically, three roles are distinguished: major characters, i.e., the main protagonist(s), minor characters, and non-protagonists \citep{Redeker1987}. Major characters re-occur regularly throughout the entire story. This tends to be the case for the Germanic literature, although Old Icelandic sagas systematically diverge from this pattern, as the main protagonist is typically introduced relatively late. \textit{Njál} is for example only introduced in the twentieth chapter of the Brennu-Njáls saga, and \textit{Gunnar} (a second main protagonist) does not occur until the ninetieth chapter. Main protagonists are typically introduced with a lot of linguistic material and by a proper name \citep{Sanford1988}, because they have to remain salient throughout the narration. Because main protagonists are repeatedly referred to and therefore remain highly accessible, they are often expressed by pronouns after their introduction \citep[93--96]{Ariel2004}, at least within episodes. Minor characters interact with the main protagonist(s) and are typically only there for a part of the story, and non-protagonists do not contribute to the plot \citep{Redeker1987}. Minor characters, therefore, tend to be introduced by less linguistic material, and their relation to the main protagonist(s) is often communicated. Moreover, they generally do not need to remain accessible for later and may thus be simply introduced by their role within the story \citep{Sanford1988, Ariel2004}. This is illustrated in \REF{ex:ch1.3}, an excerpt from the life of Saint Æthelthryth, in which both Æthelthryth and her father are introduced by a proper name in combination with a relatively long description (marked in bold). In contrast, Æthelthryth's two husbands (underlined) are only referred to by a broad gendered term (`men') and their role (`alderman') but remain unnamed.

\ea \label{ex:ch1.3}
\textit{Wē wyllað nū āwrītan, þēah ðe hit wundorliċ sȳ, be \textbf{ðǣre hālgan sancte Æðeldrȳðe þām Engliscan mǣdene, þe wæs mid} \underline{\textbf{twām werum}} \textbf{and swā ðēah wunode mǣden,} swā swā þā wundra ġeswuteliað þe hēo wyrċð ġelōme. \textbf{Anna} hātte \textbf{hyre fæder}, \textbf{Ēastengla cynincg, swȳðe Cristen man,} swā swā hē cȳdde mid weorcum, and eall his tēam wearð ġewurðod þurh God. \textbf{Æðeldrȳð} wearð þā forġifen \underline{ānum ealdormenn} tō wīfe.} \\
\glt `We now wish to write, although it may be surprising, about \textbf{the holy saint Æthelthryth, the English maiden, who was with} \underline{\textbf{two men}} \textbf{and still remained a virgin,} as the miracles, which she often performed, prove. \textbf{Anna} was \textbf{her father's name}, \textbf{the king of the East-Angles, a very christian man,} as he showed with his deeds, and his entire line was honored by God. Æthelthryth was then married to \underline{an alderman}.'\\
\hfill \citep[255, translation my own]{Baker2012}
\z 

The importance of characters in a narration does not only affect the choice of referring expression but can be connected to other types of grammatical structures as well. In her study on the Old English \textit{þa} `then', \citet{Warvik2013} argues that \textit{þa}-clauses are primarily associated with major characters, and as such, they serve to foreground and maintain the main story line, despite their frequent co-occurrence with linguistic patterns that indicate temporal shifts.

What makes a story? A story is made up by actors and events, as we have seen above, but it is also the product of the recounting of events by an agent \citep[58]{Prince1987}. The recounted events are filtered by, to quote \citet[70]{RimmonKenan1983}, ``the mediation of some `prism', `perspective', `angle of vision'". This lens is what is often referred to as the focalization. The focalizor (`the one who sees') differs from the narrator (`who speaks'), although they may coincide \citep{Genette1972, RimmonKenan1983}. 

Narrative texts are thus multi-layered, containing both the descriptive content and the focalization and verbalization of that content. A wide variety of terminology has been employed to refer to the ``real or fictitious events", to what is being told. Here, we will use the term (main) \textit{story} \citep{Genette1980, Jong2004}\footnote{Note that this may include a layer of focalization, that is, it may have been decided which events are considered relevant and which ones not.}. Two modes of presenting the story --  which may not be confused with discourse modes -- have been traditionally distinguished, mimesis and diegesis. In essence, the idea posits that there is a difference between imitation (\textit{mimesis}) and narration (\textit{diegesis}). In the French structural tradition, this opposition has been viewed as one between showing versus telling \citep{Chatman1978} and is primarily thought to be represented in the distinction between the narrative summary, expressed by the narrator, and the scene, in which characters embody and directly reflect the presupposed reality. However, in texts of the narrative genre, the entire story, including segments of direct speech, is mediated and achieved by means of diegesis (e.g., \citealt[108]{RimmonKenan1983}, \citealt[28]{Fludernik1993}).

The presence of a narrator and focalizor makes it so that narratives lend themselves well for shifting viewpoints \citep[see e.g., the contributions in][]{Dancygier2012}. This is particularly interesting as perspective-taking interacts with the use of tense and aspect \citep{Fleischmann1988}. \citet{Schiffrin1981} illustrates this for English, where the present tense can be used to report on past events in narrative segments, i.e., the historical present is used to represent speaker experience. 
German data are discussed by \citet{Zeman2018}, who argues that the preterite and the present tense contrasted in Middle High German regarding their potential of reporting the temporal sequence of events as well as their co-occurrence with \textit{do} `then'. Moreover, verbal tense has been used for many narrative structuring functions, such as distancing \citep{Declerck2003} and emphasizing the experience of story progression \citep[30]{Hoffman2023}. Furthermore, the concept of focalization may prove useful as well in relation to understanding  languages of the bounded and unbounded type \citep{carroll2002typology}, because a focalizor may or may not have a particular endpoint in sight, motivating the selection of a particular verb form. 

The discourse structure of a narrative is thus determined to a large extent by its temporal organization at the text level, as expressed in the rendering of the narrated events as well as in the relationship between narrated time and narrative time (i.e., duration and frequency). Likewise, linguistic means to express time and aspect are used to convey the narrative perspective.

%%%%%%%%%%%%%%%%%%%%%%%%%%%%%%%%%%%%%%%%%%%%%%%%%%%%%%%%%%%%%%%%%%%%%%%%%%%%%%%%%%%%%%%
\subsection{Discourse modes and discourse relations}\label{sec:1:3}
Temporal structuring also plays a decisive role at the level of text passages that consist of more than one discourse segment. Such sequences of discourse segments introduce situations of different kinds into the universe of the discourse.  \citet{Smith2003} distinguishes four situation types:\footnote{The key factor in the classification of situation types is their aspectual features, which result from the interaction of the verb, its arguments and any tense adverbial that occurs.} Events, states, general statives and abstract entities, giving rise to temporal discourse modes such as \textsc{Narrative, Description} and \textsc{Report} as well as atemporal discourse modes such as \textsc{Information} and \textsc{Argument}, that are characteristic of texts in other genres (cf. above). \citet{Smith2003} considers the discourse mode as an intermediate level of discourse structure between text level and discourse segment, which comprises discourse segments exhibiting linguistic features typical of the particular mode. In her view, a narrative text consists primarily of text passages that belong to the \textsc{narrative} discourse mode, but will probably -- due to their aspectual characteristics -- also contain passages that belong to the \textsc{descriptive} discourse mode. The following excerpts from the novel \textit{Effi Briest} by Theodor Fontane illustrate both the \textsc{narrative} and the \textsc{descriptive} mode of discourse: The text passage in \REF{ex:event} exemplifies the \textsc{narrative} discourse mode, describing three situations that express bounded events (getting dressed, going there, sending a card), i.e., events having reached the end of their runtime. The second text passage, \REF{ex:state}, on the other hand, represents the \textsc{descriptive} discourse mode and describes Effi Briest's and her companion's lodgings in Ems. The occasional piano playing is part of the description of their sojourn in Ems, insofar as it is not a description of a particular but a recurring event. According to \citet{Smith2003}, we are dealing here with a generalizing sentence. 

\eal \label{ex:dm}
\ex \label{ex:event}
\gll Gleich am andern Vormittag kleidete sie sich sorgfältig in ein dezentes Schwarz und ging auf die Linden zu, sich hier bei der Ministerin melden zu lassen. Sie schickte ihre Karte herein, auf der nur stand: Effi von Innstetten geb. von Briest. \\ Right at.the next morning dressed she herself carefully in a discreet black and went towards the Linden \textsc{ptcl} herself here with the minister report to let She sent her card in on which only appeared: Effi von Innstetten née von Briest   \\
\glt `The very next morning she dressed carefully in a discreet black dress and went to the Linden to be put in touch with the minister. She sent in her card, which simply read: Effi von Innstetten née von Briest.' 
\ex \label{ex:state}
\gll Effi und die Geheimrätin Zwicker waren seit fast drei Wochen in Ems und bewohnten daselbst das Erdgescho\ss{} einer reizenden kleinen Villa. In ihrem zwischen ihren zwei Wohnzimmern gelegenen gemeinschaftlichen Salon mit Blick auf den Garten stand ein Palisanderflügel, auf dem Effi dann und wann eine Sonate, die Zwicker dann und wann einen Walzer spielte;\\  Effi and the Privy.Counsellor Zwicker were since almost three weeks in Ems and occupied there the ground.floor a charming little villa. In their between their two living.rooms situated common drawing.room with view on the garden stood a rosewood.piano on which Effi now and then a sonata the Zwicker now and then a waltz played \\
\glt `Effi and Privy Counsellor Zwicker had been in Ems for almost three weeks and lived there on the ground floor of a charming little villa. In their communal drawing room, situated between their two living rooms and overlooking the garden, stood a rosewood grand piano on which Effi occasionally played a sonata and Zwicker sporadically played a waltz;' 
\zl

Only eventive descriptions in a discourse advance narrative time by updating reference time\footnote{Using the distinction of event time, speech time and reference time, suggested by \citet{Reichenbach1947}.}, stative descriptions fail to do so  \citep{hinrichs1986,partee1984,dry1983}. Instead of narrative progress, they signal a narrative halt. According to \textcite[35]{Smith2003}, the distinction between foreground and background in a narrative is established on the basis of the discourse modes that occur: The \textsc{narrative} discourse mode represents foreground information, the \textsc{descriptive} mode background information in a narrative text. 

Discourse segments in narrative passages are typically held together by a discourse relation called \textsc{narration}, a coordinating discourse relation which is temporal in nature, linking two eventive descriptions \citep{Smith2003, altshuler2021}. Both arguments of the discourse relation must refer to particular events, thus excluding generalizing sentences which express eventive patterns. \textsc{Background} is another characteristic coordinating discourse relation in narratives, which, as illustrated above, attaches stative descriptions to previous discourse segments. A hierarchical discourse structure arises when a sequence of coordinating discourse relations is interspersed with subordinating discourse relations such as the relation \textsc{elaboration}. 

Temporal relations between discourse segments in a narrative manifest themselves in the use of temporal anaphora such as tense morphemes, temporal adverbials or temporal conjunctions: Regardless of the situation type involved, the newly introduced situation is always linked to the reference time of the previous discourse, establishing a temporal anaphoric relation between temporal expressions \citep{hinrichs1986,partee1984,Smith2003}. Only the way in which the situation is related to reference time depends on its situation type and gives rise to either narrative progression or a narrative halt. As regards the narratives in \REF{ex:dm}, the use of temporal anaphora means that the individual past tense forms of the finite verbs establish an anaphoric relation between the respective situations, thus signaling continuity across discourse segments on the temporal level. The temporal adverbial \textit{gleich am n\"achsten Vormittag} `the very next morning', on the other hand, is a frame adverbial, indicating a temporal shift and functioning as a temporal anchor for the eventive descriptions to follow. Frame adverbials therefore signal a boundary between text fragments. The temporal adverbial \textit{schon eine Viertelstunde vorher} `a quarter of an hour earlier' in the example below, on the other hand, shows how temporal adverbials can establish anaphoric relationships to a temporal expression in the preceding discourse. The example hence indicates temporal continuity between two discourse segments.
\eal
\gll Das Schiff, ein leichtes Segelschiff (die Dampfboote gingen nur sommers), fuhr um zwölf. Schon eine Viertelstunde vorher waren Effi und Innstetten an Bord; auch Roswitha und Annie. \\ the ship a light sailing.vessel (the steamboats went only in.summer) left at twelve. Already a quarter.of.an.hour earlier were Effi and Instetten on board likewise Roswitha and Annie  \\  
\glt `The ship, a light sailing vessel (the steamboats only went in summer) left at twelve. Effi and Innstetten were already on board a quarter of an hour earlier, as were Roswitha and Annie.' 
\zl

\noindent
Despite the prominent role of temporal relations and their rendering in narratives, anaphoric relations between entities are an equally important means of indicating coherence between discourse segments in a narrative passage \citep{vonStutt2005subjektwahl,virtanen2011}. This can be demonstrated by the passages under \REF{ex:dm}, where the personal pronoun \textit{sie} `she' in \REF{ex:event} and the proper name \textit{Effi} in  \REF{ex:state} are used to express topic continuity. In terms of discourse structure, the left edge of discourse segments is of particular importance here: In a V2 language like German, the domain of entities and the temporal domain compete for the preverbal position, as \citet{vonStutt2005subjektwahl} observe in a series of experimental studies involving L1 and L2 speakers of German. They state that native speakers of German prefer participant-orientation over temporal orientation in the narrative discourse mode. A pertinent corpus example, taken from Theodor Fontane's novel, which illustrates this linking strategy is given below:
\eal
\gll \textbf{Innstetten}, unbefangen und heiter, schien sich seines häuslichen Glücks zu freuen und beschäftigte sich viel mit dem Kinde. \textbf{Roswitha} war erstaunt, den gnädigen Herrn so zärtlich und zugleich so aufgeräumt zu sehen. \textbf{Auch} \textbf{Effi} sprach viel und lachte viel, es kam ihr aber nicht aus innerster Seele. \textbf{Sie} fühlte sich bedrückt und wu\ss{}te nur nicht, wen sie dafür verantwortlich machen sollte, Innstetten oder {sich selber}. \\ Innstetten light-hearted and cheerful seemed \textsc{refl} his domestic happiness to enjoy and engaged \textsc{refl} much with the child. Roswitha was astonished the honorable master so tender and {at the same time} so tidy to see Also Effi talked {a lot} and laughed {a lot} it came her however not from innermost soul. She felt \textsc{refl} depressed and knew only not whom she {for it} responsible hold should Instetten or herself   \\
\glt `Innstetten, light-hearted and cheerful, seemed to be enjoying his domestic happiness and was very engaged with the child. Roswitha was astonished to see the honorable master so tender and at the same time so tidy. Effi also talked and laughed a lot, but it did not come from her innermost soul. She felt depressed and did not know who to blame for it, Innstetten or herself.'
\zl

\noindent
According to \citet{vonStutt2005subjektwahl}, languages such as present-day English use other cohesive devices to signal continuity across discourse segments in narrative fragments. This is because, as \citet{los2012} shows, differences between V2 languages such as German and Dutch on the one hand and English on the other have emerged over the course of language history: In Old English, with its at least partial V2 syntax, the preverbal position could be used for temporal adverbials of an anaphoric nature as well as for temporal frame adverbials, just as in modern Dutch or German. In contrast, modern English favors frame adverbials at the left edge of the sentence, if temporal information appears at all at the left edge of a sentence, cf. \posscitet{virtanen2011} example in \REF{ex:Virtanen}. The changes that affect linking strategies in English across the temporal domain and the domain of entities are captured under the concept of boundedness: English is seen as a language that has changed from a bounded to an unbounded language \citep{los2012}.
\ea \label{ex:Virtanen}
\textbf{In} \textbf{1483}, Leonardo offered his services to Lodovico Sforza, Duke of Milan, as painter, sculptor, military engineer and architect. He remained at the Milanese court until it fell to the French in 1499.
\z

\noindent
Very briefly summarized, narratives, as related to an entire text, are the result of telling a story, in which events and characters are central. How a narrative is structured appears to interact directly with the linguistic structures, as indicated by both the temporal dimension and the tracking of characters. How and to which degree (changes in) grammatical patterns correlate with or are affected by (changes in) the narrative structure remains an open question to which this volume hopes to give more insight.

%%%%%%%%%%%%%%%%%%%%%%%%%%%%%%%%%%%%%%%%%%%%%%%%%%%%%%%%%%%%%%%%%%%%%%
\section{Outline of contents}
Understandably, the contributions in this volume can only highlight some aspects of the complex relationship between grammar and narration(s). Adressing, among other things, questions of narrative progression, temporal structure, reference tracking and discourse functions, the contributions discuss phenomena such as temporal adverbials at the left periphery as well as later in the clause, left dislocation structures, fronting of the finite verb in dependent and independent clauses, linguistic means to express aspectual and tense information, and the distribution of nominalizations across text types.

\bigskip
\noindent
\textbf{Sonja Zeman} considers the implications of narration on the use of grammatical patterns, focusing on the history of German and paying empirical attention to left dislocation and the pluperfect. In her contribution, she addresses all three dimensions of narration: Narration as text type, as discourse mode, and as discourse relation. In her analysis of left dislocation in Middle High German narrative texts, she shows that left dislocation is not simply an oral feature, used to track important characters, but is used differently in the various text types, modes, and discourse relations. The interaction of the latter dimensions highlights a different function of left dislocation in different environments. Focusing on the interaction of temporality and narration, Sonja Zeman presents a study of the pluperfect along the same three dimensions and shows that discourse mode was the main motivation for using present or past perfect in Middle High German. The grammaticalization of the pluperfect was triggered by the non-narrative mode, as reference time plays a less prominent role in its meaning within that mode. With the two studies, Sonja Zeman showcases the role narration can play in the use and development of grammatical patterns.

\bigskip
\noindent
\textbf{Barthe Bloom} addresses the role of narration on the choice between two types of clause-combining strategies within one narrative text, translated to German and Dutch in the fifteenth and sixteenth centuries. She finds a correlation between the use of a temporal \textit{da} `then' or a general resumptive \textit{so} `so' and discourse mode and duration, i.e., narrative speed. German adverbial clauses with \textit{so} are associated with scenes and isochrony, whereas those that occur with \textit{da} are found in narrative summaries, continuing the story. Dutch shows a similar tendency, but not as pronounced, due to a wider spread of a third clause-combining pattern of integration. Barthe Bloom addresses the interconnection between temporality and grammatical expressions and finds an important role of narration in the selection of clause-combining strategies, which may have, because of the contrastive comparison to the later translation in Dutch, diachronic implications.

\bigskip
\noindent
\textbf{Jordan Chark} discusses the rise and spread of two aspectual patterns in the history of Icelandic, the \textit{búinn}-perfect and the \textit{að vera}-progressive. She argues that this can be better understood if one considers the changing narrative style, which moves away from an iconic, chronological reporting of events to more overt grounding by means of temporal adverbial clauses. This is related to the shift from a bounded system to an unbounded system, which she argues took place in Icelandic in the early modern period in a similar way as it did in English. This means that the kind of narrating shifts from being organized around a sequence of events to the reporting of events related to a topic. With corpus data, Jordan Chark shows that the number of backgrounded temporal clauses increased, while the use of local anchoring adverbial, in particular the indicator of topic-time \textit{þá} `then' decreased. For the two aspectual patterns, the reanalysis is proposed to have happened in the early stages of the boundedness shift, but the circumstances of reanalysis are independent of this. The subsequent spread is, in contrast, highly advanced by it, since the focalized usage of the progressive allows setting up a new temporal frame, in other words, a new topic time through which events are viewed, and the \textit{búinn}-perfect serves as a frame-setter strategy.

\bigskip
\noindent
\textbf{Hannah Booth} examines a number of light adverbs in Old Icelandic saga narratives that have so far been well studied in particular for Old and Middle English. Depending on their position in the sentence, different discourse functions are ascribed to these adverbs in English. In a corpus study of Old Icelandic spanning two collections of saga narratives covering a period from 1150 to 1450 -- she examines whether temporal and local adverbs such as \textit{þá} `then', \textit{þar} `there' and \textit{nú} `now' have the same discourse functions as their counterparts in older stages of the English language. By focusing on their clause-internal position, Hannah Booth can show that the adverbs in question actually demarcate two discourse domains in the sentence: While the position to the left of the adverb is favored by various types of topics, the domain to the right of the adverb is reserved for the assignment of information focus. Considering the decrease of clause-internal adverbs over time, Hannah Booth suggests that this development is linked to broader changes affecting the Icelandic language system and establishing the left edge of the clause as a preferred topic position.   
 
\bigskip
\noindent
\textbf{Fabian Fleissner} discusses the spread of complex prepositions in the history of German and verbonominal constructions  with \textit{in} in light of a general development towards a more nominal style. In his paper, we evidence the effect of text type on the distribution of grammatical structures, in particular with regard to the complex preposition construction. As narrative text types tend to focus on the telling of events, at least in German and Dutch \citep{carroll2004language}, the expression of events as nominals is much more restricted in narratives than in other text types, especially scientific texts. Furthermore, the backgrounding of the agent, which is a characteristic trait of the construction, is consistent with the central role of characters in narration. As such, the construction lends itself well to condensing information without losing clarity and is suitable for the new scientific text type. For verbonominals, the relation to text type is less clear and the construction is characterized by multiple sub-schemas with different functions, showing partial overlap with complex prepositions. Besides these discourse-pragmatic considerations, Fleissner furthermore evaluates the age of nouns that occur within the constructions, and despite fluctuations, the data indicate that the increase in the use of both patterns originates in adding nouns to either construction as they are adopted into the language.

\bigskip
\noindent
Based on the observation that argument-realizing V2 clauses occur much more frequently in Early New High German (1350--1650) than in present-day German, \textbf{Malika Reetz} investigates the question of which pragmatic factors license the occurrence of this non-canonical form of dependent clauses in Early New High German. Her corpus study is based on narrative texts from the fifteenth and sixteenth centuries which are particularly well suited to address her research question: Categories such as givenness, newness, at-issueness are easy to determine in a self-contained narrative. With the tool of conditional inference trees, she can establish that V2-clauses are preferred over V-final clauses when they are relevant for the progression of the narrative. At the same time, V2 is also used to indicate that an assertion is mediated. Only the V-final clause pattern can be used when the proposition is discourse-old and does not contribute to the main story line. 

\bigskip
Verb-initial declaratives are associated with lively storytelling across older Germanic languages (Old English \citealt {Los2000}, Old High German \citealt{hinterholzl2010v1} and Old Icelandic \citealt{booth-beck20200jhs}). In particular, they are supposed to signal the coordinating discourse relation of \textsc{narration} holding between two subsequent events. In a comprehensive corpus study for Old English, \textbf{Anna Cichosz} shows that the use of verb-initial declaratives is in fact not limited to a unique discourse function. As expected, the function of event-linking is most frequently attested in the corpus and seems to be conventionalized to a certain extent in Old English, as it is attested in a number of texts across different stages of the period. Verb-initial declaratives are nevertheless also used in narratives to provide additional descriptions of given entities and, hence, to establish subordinate discourse relations such as the relation \textsc{elaboration} between discourse segments. The distribution of these verb-initial declaratives across the corpus suggests that they are an early native structure in Old English. According to Anna Cichosz, a deeper understanding of discourse functions of the syntactic pattern in question requires at least partial manual analysis of large amounts of data.   

\bigskip
\noindent
Old English also features the asymmetry of verb placement between main and subordinate clauses known from present-day German and present-day Dutch, albeit not as pronounced as in the two modern West Germanic languages \parencite{Pintzuk1999}. It is therefore not all that surprising that Old English also features relative clauses with the finite verb in final as well as in second position. In their contribution, \textbf{Bettelou Los and Stefano Coretta} focus on Old English \textit{se}-relatives exhibiting V2 word order. Compared to relative clauses with other connectors such as \textit{þe} or \textit{se þe} with the finite verb in final position, V2-relatives are less integrated into their parent clause because they are always non-restrictive in Old English. In their corpus study, Bettelou Los and Stefano Coretta establish that the fronting of the finite verb in V2-relatives is triggered by the fact that they take a discourse referent that is new to the discourse as their antecedent, i.e. V2-relatives primarily have a presentational function. Essentially, it remains an open question how \textit{se}-relatives with the finite verb in second position can be distinguished from main clauses that are introduced by the demonstrative \textit{se}. As a trial, the authors propose to decide this on the basis of the distance between the relative clause and the antecedent.   
%%%%%%%%%%%%%%%%%%%%%%%%%%%%%%%%%%%%%%%%%%%%%%%%%%%%%%%%%%%
\section*{Acknowledgements}
First of all, we would like to thank all the colleagues who contributed to this volume for not only engaging with the topic of the workshop, i.e., discussing discourse structures in Germanic from a diachronic perspective, but also for subsequently writing up their presentations for the present volume. Secondly, this volume would not have come about in its present form if a number of other colleagues had not supported us with their external expertise. We would like to thank these colleagues for the time they have invested. Last but not least, we would like to thank Luisa Boeckmann, who assisted us in preparing the print template for the volume.  

{\sloppy\printbibliography[heading=subbibliography,notkeyword=this]}

%%%%%%%%%%%%%%%%%%%%%%%%%%%%%%%%%%%%%%%%%%%%%%%%%%%%%%%%%%%%%%%%%%%%%%%%%%%%%%
\end{document}

