\documentclass[output=paper,colorlinks,citecolor=brown]{langscibook}
\ChapterDOI{10.5281/zenodo.15689129}

\author{Fabian Fleissner\affiliation{Université de Neuchâtel}}
\title[In consideration of Age and Genre]{In consideration of age and genre: A diachronic corpus analysis of deverbal nouns in German verbonominal constructions and complex prepositions}

\abstract{This paper deals with the formation and the diachronic development of German complex prepositions (CPs) following the general syntactic pattern [\textit{in} N\textsubscript{dev} P/\textsc{gen}] and verbonominal constructions (VNCs) following the pattern [\textit{in} N\textsubscript{dev} V]. It sets out to answer two general questions: What are the discursive functions fulfilled by CPs\is{complex prepositions} and VNCs across different genres, and how does the age of their nominal core noun influence their genesis and persistence? It will be shown that both constructions exhibit different distributional and preferential patterns in these respects. [\textit{in} N\textsubscript{dev} P/\textsc{gen}] shows clear genre-specific characteristics of 19th century scientific texts which are reflected in the emergence of a \textsc{reference} discourse schema, stimulated by the coinage of new nouns with this semantics in New High German. [\textit{in} N\textsubscript{dev} V] displays a broader, more open structure with diverse semantics and therefore shows a less genre-specific distribution. This construction primarily contains nouns from Early New High German, but repel those from New High German tailored to their functional niche.}

% Uncomment if only this chapter is to be compiled
%\IfFileExists{localcommands.tex}{
%   \addbibresource{localbibliography.bib}
 %  \usepackage{langsci-optional}
\usepackage{langsci-gb4e}
\usepackage{langsci-lgr}

\usepackage{listings}
\lstset{basicstyle=\ttfamily,tabsize=2,breaklines=true}

%added by author
% \usepackage{tipa}
\usepackage{multirow}
\graphicspath{{figures/}}
\usepackage{langsci-branding}

  % 
\newcommand{\sent}{\enumsentence}
\newcommand{\sents}{\eenumsentence}
\let\citeasnoun\citet

\renewcommand{\lsCoverTitleFont}[1]{\sffamily\addfontfeatures{Scale=MatchUppercase}\fontsize{44pt}{16mm}\selectfont #1}
  
   %%% hyphenation points for line breaks
%% Normally, automatic hyphenation in LaTeX is very good
%% If a word is mis-hyphenated, add it to this file
%%
%% add information to TeX file before \begin{document} with:
%% %% hyphenation points for line breaks
%% Normally, automatic hyphenation in LaTeX is very good
%% If a word is mis-hyphenated, add it to this file
%%
%% add information to TeX file before \begin{document} with:
%% %% hyphenation points for line breaks
%% Normally, automatic hyphenation in LaTeX is very good
%% If a word is mis-hyphenated, add it to this file
%%
%% add information to TeX file before \begin{document} with:
%% \include{localhyphenation}
\hyphenation{
affri-ca-te
affri-ca-tes
an-no-tated
com-ple-ments
com-po-si-tio-na-li-ty
non-com-po-si-tio-na-li-ty
Gon-zá-lez
out-side
Ri-chárd
se-man-tics
STREU-SLE
Tie-de-mann
}
\hyphenation{
affri-ca-te
affri-ca-tes
an-no-tated
com-ple-ments
com-po-si-tio-na-li-ty
non-com-po-si-tio-na-li-ty
Gon-zá-lez
out-side
Ri-chárd
se-man-tics
STREU-SLE
Tie-de-mann
}
\hyphenation{
affri-ca-te
affri-ca-tes
an-no-tated
com-ple-ments
com-po-si-tio-na-li-ty
non-com-po-si-tio-na-li-ty
Gon-zá-lez
out-side
Ri-chárd
se-man-tics
STREU-SLE
Tie-de-mann
}
   %\boolfalse{bookcompile}
   %\togglepaper[23]%%chapternumber
%}{}

\shorttitlerunninghead{In consideration of age and genre}
\begin{document}
\maketitle

\shorttitlerunninghead{In consideration of age and genre}

\section{Introduction}\label{sec:fleissner:1}

A prevailing observation in the contemporary history of \ili{German} is its notable inclination towards a nominal \isi{style}\is{German Nominalstil} which is characterized by the frequent use of nouns derived from verbs, adjectives, or other nouns.\footnote{This study is part of the project \textit{Zusammenspiel von Wortbildung und Syntax: }\textbf{}\textit{Nominalisierungsstrategien in verbonominalen Konstruktionen und sekundären Präpositionen} ‘The interaction of \isi{word formation} and syntax: Nominalization strategies in \isi{verbonominal constructions} and secondary prepositions' funded by the Swiss National Science Foundation (GNr. 197124).} This \isi{style} is commonly found in academic writing, technical texts, and bureaucratic language (cf. \citealt{Eggers1973}, \citealt{Moslein1981}, \citealt{Admoni1973, Admoni1990}, \citealt{Polenz1987, Polenz2013}). This paper sheds light on two constructions associated with this tendency and, in some cases, held responsible for it: \isi{verbonominal constructions} (henceforth VNC) of the syntactic pattern [P (Det) (Mod) N\textsubscript{dev} (P/\textsc{gen}) V]\footnote{In research literature, constructions without prepositions are often considered part of the same class of VNCs\is{verbonominal constructions}. In this study, I exclusively discuss VNCs\is{verbonominal constructions} with prepositions to facilitate a comparison with complex prepositions.}  as in \xxref{ex:fleissner:1}{ex:fleissner:2} and \isi{complex prepositions} (henceforth CP) of the syntactic pattern [P (Det) N\textsubscript{dev} P/\textsc{gen}] as in \xxref{ex:fleissner:3}{ex:fleissner:4} (cf. e.g. \citealt{SeppänenTrotta1994} and \citealt{HüningEtAl2020}). 

\ea%1
\label{ex:fleissner:1}
%\gll 
{\itshape Es kommt aber noch ein zweiter Gegenſtand in Betrachtung.}\\
%{}  come-\textsc{pres.3.s} {} {} {} {} {} in  consideration-\textsc{acc.sg}   \\
\glt ‘However, a second issue \textbf{comes into consideration}.' \hfill (Clausewitz, 1834)\footnote{Throughout the paper, examples from the DTA-corpus are cited according to the following pattern: [author, year].}
\z 

\ea%2
\label{ex:fleissner:2}
%\gll  
{\itshape Diese Hütte \textbf{stand} bis 1866 [...] \textbf{im} \textbf{Betrieb}.}\\
%{} {}  stand-\textsc{pret.3.s} {} {} {}  in.the\textsc{-dat.sg}  operation- \textsc{dat.sg} \\
\glt ‘This hut \textbf{was in operation} until 1866.' \hfill (Beck, 1899)
\z 

\ea%3
\label{ex:fleissner:3}
%\gll  
{\itshape [...] der [wolle] ihn \textbf{in} \textbf{Betrachtung} \textbf{solchen} \textbf{Gehorsams} der gnädigen Rücksicht des Himmelssohnes empfehlen.}\\
%{} {} {} {} {in} consideration-\textsc{dat.sg} such-\textsc{gen.sg} obedience-\textsc{gen.sg} {} {} {} {} {} {}\\
\glt ‘[...] He wishes to commend him, \textbf{in consideration of such obedience,} to the gracious care of the Son of Heaven.' \hfill (Berg, 1873)
\z 

\ea%4
\label{ex:fleissner:4}
%\gll 
{\itshape \textbf{In} \textbf{Bezug} \textbf{auf} \textbf{den} \textbf{Monotheismus} der Indoeuropäer bemerke ich kurz Folgendes.}\\
% in relation-\textsc{dat.sg} on the-\textsc{acc.sg}  monotheism-\textsc{acc.sg} {} {} {} {}  \\
\glt ‘\textbf{With regard to the monotheism} of the Indo-Europeans, I briefly note the following.' \hfill (Chamberlain, 1899)
\z 
          
Both VNCs\is{verbonominal constructions} and CPs\is{complex prepositions} usually contain a deverbal noun\is{deverbal nouns} and are assumed to be used in similar discursive contexts. Nevertheless, it has not yet been possible to trace functional similarities or even an interdependent development of both constructions. One obvious reason for this is that the usually assumed ``information condensation'' of \isi{deverbal nouns} only comes into play with CPs\is{complex prepositions} that potentially shorten subordinate clause structures. VNCs\is{verbonominal constructions}, on the other hand, are characterized by a complex predicative structure which, at first sight, does not offer any advantages over the use of verbs of the derivational base. Much attention has been paid to both constructions, especially from a diachronic and morphosyntactic perspective: Longstanding interest in the study of \ili{German} VNCs\is{verbonominal constructions} has led to various insights in this respect (cf. e.g. \citealt{Pottelberge2001}, \citealt{Seifert2004}, \citealt{Kamber2008}, \citealt{Kabatnik2020}, \citealt{Harm2021}, \citealt{FleischhauerHartmann2021}, \citealt{Fleischhauer2022}). It is generally accepted that \ili{German} \textit{Funktionsverbgefüge}\is{Funktionsverbgefüge} (lit. ``function verb constructions'') are characterized by a ``desemanticized'' verb and an eventive noun that constitutes the core of the predication. There is also a certain consensus that, from a semantic point of view, individual constructions display different degrees of compositionality and idiomaticity. Whereas older studies usually treat \ili{German} VNCs\is{verbonominal constructions} as one homogenous category (\citealt{Polenz1987}, \citealt{Helbig1979}, \citealt{Kamber2008}), more recent studies focus on individual (families of) constructions and their specific properties (\citealt{Zeschel2008}, \citealt{Kabatnik2020}, \citealt{Fleischhauer2022}, \citealt{Smirnova2022}). There has been growing acceptance that they do not uniformly exhibit identical characteristics in terms of their morphosyntactic structure and function. However, both of these perspectives are yet to offer a comprehensive and unifying definition of the phenomenon itself or a clear distinction of its subclasses. The term \textit{Funktionsverbgefüge}\is{Funktionsverbgefüge} implies that these constructions are primarily described in relation to the status of the verb, which, in contrast to its ``heavy'' use, is considered restricted or devoid of its ``full'' semantic weight. Consequently, VNCs\is{verbonominal constructions} have been repeatedly examined and analyzed from the perspective of grammaticalization\is{grammaticalization} theory. 

Complex prepositions\is{complex prepositions} in \ili{German} are often described as an open class of constructions (cf. e.g. \citealt{Meola2000}, \citealt{Szczepaniak2009}, \citealt{HüningEtAl2020}). Defining this class precisely has proven to be challenging due to the blurred boundaries between simple prepositions and free syntactic combinations (cf. e.g. \citealt[17]{LehmannStolz1992}, \citealt{Meibauer1995}). The majority of CPs\is{complex prepositions} in present-day \ili{German} are formed according to the pattern [P (Det) N P/\textsc{gen}] as shown in \REF{ex:fleissner:5}.

\ea%5
\label{ex:fleissner:5}
\gll in\textsubscript{P} {Bezug}\textsubscript{N} auf\textsubscript{P}, 
in/im\textsubscript{P} {Zusammenhang}\textsubscript{N} {mit}\textsubscript{P}, im\textsubscript{P} {Vergleich}\textsubscript{N} {zu}\textsubscript{P}\\
‘in relation to, in connection with, in comparison with'\\
\z 

The principal characteristics of CPs\is{complex prepositions} of the syntactic pattern [P (Det) N P/\textsc{gen}] in \ili{German} have been researched in numerous ways (for the complete list of criteria see \citealt[34--35]{Beneš}, \citealt{LehmannStolz1992}, \citealt{Lindqvist1994}, \citealt{Meola2000}, \citealt{Szczepaniak2009}, \citealt{HüningEtAl2020}, \citet{Ruffthc}. In CPs\is{complex prepositions}, the first preposition is usually assumed to be invariable, and the second, postnominal preposition or the genitive case of the noun phrase is also widely taken as being fixed. Regarding the nominal core [(Det) N] of the pattern, as highlighted in the referenced literature, it is generally observed that the noun cannot undergo syntactic modifications such as adjectives and determiners, or morphological changes like plural marking. Generally, the characteristic properties of CPs\is{complex prepositions} are associated with the formal fixedness of these multi-word expressions. From a historical perspective, this formal rigidity is often interpreted as an outcome of gradual conventionalization, wherein regular syntactic combinations evolve into fixed expressions over time. This process involves the reanalysis of postnominal attributive phrases as complements of a new CP, as outlined by \citet{Lehmann1998}. Throughout this diachronic evolution, the constituent elements of the previously freely combined syntactic structure progressively lose their autonomy, eventually merging into a unified and distinctive structure. But none of the attributes examined in the literature can be considered a defining feature of a CP, as individual instances typically exhibit a combination of these properties, sometimes to varying degrees. According to an alternative hypothesis, CPs\is{complex prepositions} are formed directly after a general schema by filling its open slots with lexical material (see \citealt{Hoffmann2005}). The mechanism at hand in the traditional scenario, along with reanalysis, is \textit{routinization} and conventionalization of a syntagmatic string that results from the frequent and repeated use in the linguistic community. However, these characteristics remain crucial for delineating the category of CPs\is{complex prepositions} from syntactic combinations formed in a regular manner. 

Apart from these individual morphosyntactic observations of particular patterns, \ili{German} nominal \isi{style}\is{German Nominalstil} has been associated with the development of modern scientific language\is{language of science} in discourse-pragmatic approaches. These approaches generally assume a shift of information from the verbal to the nominal domain in recent \isi{scientific texts} of \ili{German} (see \citealt{Kretzenbacher1994}: 31). Whereas in narrative texts verbs with great semantic intensity play a major role as action carriers, the shift of information from the verbal to the nominal domain in non-narrative\is{non-narrativity} texts leads not only to a higher frequency of nouns in general, but also to a higher frequency of semantically ``weak" verbs such as auxiliary verbs, copulative verbs and VNCs\is{verbonominal constructions}. 

Modern scientific linguistic \isi{style},\is{language of science} like modern scientific methodology, has its roots in the 17th century. The emancipation of the empirical natural sciences, once despised as \textit{Realwissenschaften}, from an outsider position within the scholastic understanding of science to their modern role as leading sciences, was promoted during this period, especially in England. This century also saw the beginnings of the transition from Latin to \ili{German} in academic writing. Scholars like Gottfried Wilhelm Leibniz, Christian Thomasius and Christian Wolff are considered the forerunners of this protracted and laborious transition to \ili{German} as the \isi{language of science}. Leibniz is the one who points to a language crisis throughout his writings. However, he does not see the improvement of the state of language in poetry, as the language societies of that time did, but in the development of \textit{Sachprosaschriften} (cf. \citealt{Pörksen1984}: 90). These formed the basis for the \ili{German} \isi{language of science}. One of the essential factors of \ili{German} history in the absolutist era was the extraordinary extent of foreign language use, especially French. While Latin-\ili{German} bilingualism existed in the state system, in the field of science, the transition to \ili{German} was delayed for a long time due to internationally renowned French or adherence to Latin. In Italy, the transition from Latin to the ``vernacular'' as a \isi{language of science} was already gradually taking place in the 16th century, and in France and England in the 17th century, academic writers in Germany predominantly retained Latin until the late 18th century \citep[cf.][53--58]{Polenz2013}. It is precisely during this period that the most quantitatively striking diachronic developments in CPs\is{complex prepositions} and VNCs\is{verbonominal constructions} occur, as the data show. The following two figures illustrate this with the development curves of CPs\is{complex prepositions} (\figref{fig:fleissner:1}) and VNCs\is{verbonominal constructions} (\figref{fig:fleissner:2}) with certain selected individual constructions.

\begin{figure}
\caption{CPs of the pattern [in \textup{N P/}\textsc{gen}]; N = \textit{Hilfe, Beihilfe, Bezug, Beziehung, Hinsicht, Rücksicht}}\label{fig:fleissner:1}
\includegraphics[height=0.43\textheight]{figures/FleissnerFig1.png}
\end{figure}

\begin{figure}
\caption{VNCs of the pattern [\textit{in N kommen}]}
\includegraphics[height=0.43\textheight]{figures/FleissnerFig2.png}
\label{fig:fleissner:2}
\end{figure}

In the present study, I will show that the history of the two structures under investigation is related to the addressed extralinguistic developments. I argue that both the rise of nominalizations in \ili{German} and the rise of CPs\is{complex prepositions} and VNCs\is{verbonominal constructions} can be explained discourse-pragmatically. The constructions feed on nominal material available at diachronically different points in time. The gateways are assumed to be certain texts that make corresponding stylistic and discursive demands, which can be fulfilled through the constructions under investigation. 

The paper is structured as follows. In \sectref{sec:fleissner:2}, I will introduce my research objects and theoretical background. \sectref{sec:fleissner:3} presents data and methods. \sectref{sec:fleissner:4} presents the results of the corpus analysis which are discussed in \sectref{sec:fleissner:5}. \sectref{sec:fleissner:6} summarizes and consolidates the findings of the study and provides an outlook on future points of connection.


\section{Research objects and theoretical background}\label{sec:fleissner:2}

The present study investigates multi-word expressions which are formed following the pattern [\textit{in} N \_]. This pattern gave rise to different constructions at various points in the history of \ili{German}, with CPs\is{complex prepositions} of the pattern [\textit{in} N\textsubscript{dev} P/\textsc{gen}] and VNCs\is{verbonominal constructions} of the pattern [\textit{in} N\textsubscript{dev} V] being the most striking constructions. In contrast to previous approaches, I will not examine the development of individual morphosyntactic patterns. Instead, I will approach it functionally from semantic clusters, which need to be identified for the entire construction. The most promising starting point in this case is the nominal core of the structures, i.e. \isi{deverbal nouns}. The study starts out with two simple questions: Why are some nominalizations more prone to being used in the relevant constructions than others? And how do these preferences change over time? I will focus on 2 parameters:

\begin{enumerate}[label=(\roman*)]
\item
\textsc{age}. Following the tendency for nominalizations to become ``nounier'' over time, younger and ``verbier'' nominalizations are believed to be more often used in both types of constructions. Although the data show that this is true in principle, there is a non-negligible number of important exceptions that require explanation.
\item 
\textsc{genre}. The explosive spread of CPs\is{complex prepositions} and VNCs\is{verbonominal constructions} does not take place before the late 18th century, and it is assumed to initially occur in certain non-narrative\is{non-narrativity} texts, especially in \isi{scientific texts}.
\end{enumerate}

From the diachronic point of view, the category of CPs\is{complex prepositions} and VNCs in \ili{German} are usually conceived of as an expanding open class. As mentioned in \sectref{sec:fleissner:1} above, it is generally assumed that both stem from regular syntactic combinations with compositional semantics. However, this perspective faces a challenge, as highlighted by \citet{Hoffmann2005} with respect to English CPs\is{complex prepositions}, that for numerous CPs\is{complex prepositions}, there is insufficient empirical support for a transition from a freely combined structure to a fixed expression.

In English, numerous constructions of this nature emerge suddenly in written records, typically in the format [P (Det) N P/GEN]. This hints at the possibility that certain CPs\is{complex prepositions} might have arisen through analogy to pre-existing units that likely evolved previously in a manner akin to that outlined in \REF{ex:fleissner:5}. \citet{Hoffmann2005} refers to this process as ``grammaticalization\is{grammaticalization} by analogy'' and he suggests the presence of an ``abstract construct'':

\begin{quote}
    […] the sequence ‘preposition–noun–preposition' appears to be available as a grammaticalized\is{grammaticalization} yet abstract construct which under certain circumstances can be filled by new lexical entities to form a new complex preposition. \citep[171]{Hoffmann2005}
\end{quote}

Hoffmann proposes what, in constructionist terms, could be termed as a schema-tic construction which is used to generate new constructions spontaneously by filling the variable slots [P] and [N] with lexical content. This appears to be the case for numerous CPs\is{complex prepositions} in English. In \citet{Ruffthc}, a first sketch has been offered of how the category of CPs\is{complex prepositions} in \ili{German} may be modeled in terms of a constructional network with several levels of abstraction and with different sub-schemas within the general schema [P (Det) N P] (see \figref{fig:fleissner:3}). 


\begin{figure}
\centering
\resizebox{\textwidth}{!}{
\begin{tikzpicture}[node distance=2.5cm]
\node (start) [process] {[P N P \textsc{gen}]};
\node (in1) [process, below of=start] {[\textit{mit} N \textit{auf}/\textsc{gen}]};
\node (pro1) [process, below of=in1] {[\textit{mit Bezug auf}] [\textit{mit Rücksicht auf}] \linebreak[4] [\textit{mit Hinsicht auf}]\linebreak[4] [\textit{mit Beziehung auf}] \linebreak[4][\textit{mit Ber\"ucksichtigung \textsc{gen}}]};
\node (dec1) [process, right of=pro1, xshift=2.5cm] {[\textit{mit Hilfe \textsc{gen}}]\linebreak[4] instrumental semantics};

\node (pro2b) [startstop, right of=dec1, xshift=2cm] {individual construction(s)};
\node (out1) [startstop, above of=pro2b] {sub-schema};
\node (stop) [startstop, above of=out1] {general schema};

\draw [-] (start) -- (in1);
\draw [->] (in1) -- (pro1);
\draw [dashed] (start) -- (dec1);

\end{tikzpicture}
}
\caption{Partial network of German complex prepositions \citep{Ruffthc}}
\label{fig:fleissner:3}
\end{figure}

The examined data indicate that the structure [\textit{mit} N\textsubscript{dev} \textit{auf}/GEN], involving nouns such as \textit{Bezug} `relation', \textit{Berücksichtigung} `consideration', \textit{Rücksicht} `consideration', \textit{Hinsicht} `regard', and \textit{Beziehung} ‘relation', qualifies as an abstract and productive constructional schema. This schema exhibits the capacity to attract various nouns sharing similar semantics of \textsc{reference}, and it maintains consistent across most instances. Of course, the class of \isi{complex prepositions} in \ili{German} is much more diverse and consists of many more members following different structural patterns, as for example [\textit{in} N \textit{mit}] as in \textit{in Zusammenhang mit} ‘in connection with' or [\textit{im} N \textit{von/}GEN] as in \textit{im Zuge} ‘in the course of'. In the case of [\textit{mit Hilfe} GEN] ‘with the help of', there is a conventionalized lexical item that emerged by a gradual process of routinization, whereby the internal structure became more and more fixed, and the meaning became less compositional and shifted towards the instrumental semantics. Some other synonymous nouns which, given their semantic content, had the potential to follow the analogical model of [\textit{mit Hilfe} GEN], such as \textit{Beistand} ‘assistance' or \textit{Unterstützung} ‘support', did not succeed in this way. Whether the constructional network above applies to other \isi{complex prepositions}, remains an open question. Unlike the preposition \textit{mit}, the preposition \textit{in} appears in both CPs\is{complex prepositions} and VNCs\is{verbonominal constructions}, and is therefore best suited to investigate any common developments between the two construction types. 


\section{Data}\label{sec:fleissner:3} 

The present study focuses on the two aforementioned patterns with the preposition \textit{in} ‘in' [\textit{in} N\textsubscript{dev} P/\textsc{gen}] and [\textit{in} N\textsubscript{dev} V], see for example \REF{ex:fleissner:6} and \REF{ex:fleissner:7} below, as this syntagmatic pattern is recurrent in both CPs\is{complex prepositions} and VNCs\is{verbonominal constructions}, while other prepositions like \textit{mit} or \textit{zu} usually occur in just one structure. The study is based on data from the DTA (Deutsches Textarchiv, version 2018). For the purposes of this research project, I selected a smaller data sample from the whole DTA that covers nine periods and amounts to about 10\% of the entire corpus. \tabref{tab:fleissner:1} represents the composition of the data with respect to the individual periods. 

\begin{table}
\begin{tabularx}{0.5\textwidth}{lrr}
\lsptoprule
 \multicolumn{1}{l}{period} &  \multicolumn{1}{l}{years} &  \multicolumn{1}{l}{corpus size}\\
 \midrule
 {1} & 1605–1607 & 544,752\\
 {2} & 1643–1646 & 286,361\\
 {3} & 1683–1685 & 367,164\\
 {4} & 1719–1722 & 1,282,914\\
 {5} & 1758–1760 & 1,154,723\\
 {6} & 1797–1800 & 2,172,629\\
 {7} & 1834–1836 & 4,366,178\\
 {8} & 1871–1874 & 2,256,498\\
 {9} & 1896–1899 & 4,411,414\\
 \midrule
 &{{\textbf{Total}}} & \textbf{16,842,633}\\
\lspbottomrule
\end{tabularx}
\caption{Investigated periods}
\label{tab:fleissner:1}
\end{table}


From this corpus, I created a data set containing all occurrences of \isi{deverbal nouns}, based on a list that was compiled using the studies by \citet{Kamber2008}, \citet{Hartmann2016} and \citet{Smirnova2022}.\footnote{I restrict myself to deverbal nominalizations because of their high affinity to the structures under investigation. Though one can find CPs\is{complex prepositions} and VNCs\is{verbonominal constructions} with nouns of non-verbal origin, \isi{deverbal nouns} are much more frequent in the structures [\textit{in} N\textsubscript{dev} P/\textsc{gen}] and [\textit{in} N\textsubscript{dev} V].} The data set from the last period not only makes up the largest proportion of the corpus but also contains the most occurrences of CPs\is{complex prepositions} and VNCs\is{verbonominal constructions}.\footnote{A complete list of the nouns as well as the data used for this study can be found at \url{https://osf.io/yancd}.} Each individual nominal token was examined with regard to these syntactic structures. For this study, Period 9 was selected as the main research area because, in this segment, all diachronic developments concerning the parameters \textsc{age} and \textsc{genre}\is{genre} are concluded and discernible in the data. A comprehensive overview across all time periods would obscure the focus on the discursive functional characteristics of individual constructions, as written language patterns also undergo change. Furthermore, lexical diversity is greatest at the end of the 19th century, as only a few nouns fell out of use during the observed periods. \tabref{tab:fleissner:2} provides an overview of the absolute numerical ratios of all \isi{deverbal nouns} in Period 9.

\begin{table}
\centering
\begin{tabularx}{0.5\textwidth}{lrrr}
\lsptoprule
 &  \multicolumn{1}{l}{CP} &  \multicolumn{1}{l}{VNC} &  \multicolumn{1}{l}{ZERO}\\
\midrule
{n (tokens)} & 2148 & 3553 & 68393\\
{n (types)} & 191 & 231 & 346\\
\lspbottomrule
\end{tabularx}
\caption{Overview of the token and type frequency of [in N\textsubscript{dev} \_]}
\label{tab:fleissner:2}
\end{table}

The number of structures with a verb exceeds that of structures without a verb significantly, which is not surprising, since VNCs\is{verbonominal constructions}, with their additional slot, allow for higher syntactic variability and are more strongly integrated\is{integration} into the general argument structure of \ili{German} as more or less ordinary predicate-forming units. As is immediately evident, the vast majority of nouns are used outside of the two structures (ZERO). At least, this further suggests that the increase in the use of nominal structures in \ili{German} is not necessarily attributable to the increasing use of the two most conspicuous morphosyntactic patterns. Rather, it seems that both structures draw from a comprehensive lexical pool that can be utilized for specific discourse functions.\is{discourse function} A synchronic point at the end of the 19th century serves as a suitable culmination of all tendencies in preceding centuries. However, this does not alter the fact that the data must be interpreted in terms of their diachronic dynamics. Therefore, each noun was annotated based on the non-morphosyntactic parameter of \textsc{age}, as explained in \sectref{sec:fleissner:2}. The assignment of \textsc{age} had to be done manually, since this parameter has not played a role in previous research and, accordingly, no corpus provides relevant information. The productivity of different \isi{word formation} patterns at the same time and numerous analogical formations complicated the access via the mere form of individual lexemes for diachronic sorting. For this reason, a conscious decision was made to omit this simpler, objectively measurable parameter of \isi{word formation} type in favor of \textsc{age}. Various historical dictionaries and reference corpora served as the basis, occasionally requiring searches for reconstructed forms. For the majority of nouns, this method was sufficient to assign a specific age to an individual lexeme. The categorization of different stages was done according to traditional conventions of historical linguistics: \ili{Proto Germanic} (PROTOG), \ili{Old High German} (OHG), \ili{Middle High German} (MHG), \ili{Early New High German} (ENHG) and \ili{New High German} (NHG). \tabref{tab:fleissner:3} provides an overview of the token frequency of each noun and the time of their emergence.

\begin{table}
\centering
\begin{tabularx}{0.55\textwidth}{lrrrr}
\lsptoprule
{age} &  \multicolumn{1}{l}{CP} &  \multicolumn{1}{l}{{VNC}} &  \multicolumn{1}{l}{{ZERO}} &  \multicolumn{1}{l}{total}\\
\midrule
\textsc{protog} & {249} & {436} & {9653} & {10338}\\
\textsc{ohg} & {659} & {710} & {15159} & {16528}\\
\textsc{mhg} & {335} & {934} & {17832} & {19101}\\
\textsc{enhg} & {490} & {1245} & {16421} & {18156}\\
\textsc{nhg} & {412} & {225} & {8664} & {9301}\\
\textsc{NA}\footnote{These are mainly nominalized infinitives whose age are difficult to determine.}  & {3} & {3} & {664} & {670}\\
\lspbottomrule
\end{tabularx}
\caption{Overview of the token frequency of [in \textup{N}\textup{\textsubscript{dev}} \textup{\_] by} \textsc{age}}
\label{tab:fleissner:3}
\end{table}

The mere comparison of absolute frequencies may not be very informative, but it suggests a closer examination of the respective ratios and the underlying lexical types. Both analyzed structures not only behave differently in relation to each other but also each in relation to the lexical pool outside of the specific constructions. 

The underlying corpus is balanced in terms of the \isi{genre} affiliation of its individual texts. As in the DTA, four different text genres\is{genre} are distinguished: \textit{Belletristik} ‘fiction', \textit{Zeitung} ‘newspaper', \textit{Gebrauchsliteratur} ‘reference literature'\footnote{Reference literature refers to publications designed to provide concise and specific information on a wide range of topics. Examples include dictionaries, encyclopedias, atlases, almanacs, handbooks, cookbooks, decency literature, travel literature and directories. Reference literature is structured for easy access to specific pieces of information. Readers typically consult reference materials for quick answers, definitions, or explanations rather than engaging in a continuous narrative. Reference literature occupies a position between more narrative genres such as fiction and newspapers on one side and clearly non-narrative scientific texts on the other.}, and \textit{Wissenschaft} ‘science'.\is{scientific texts}

\begin{table}
\centering
\begin{tabularx}{0.6\textwidth}{lrrrr}
\lsptoprule
{genre} &  \multicolumn{1}{l}{CP} &  \multicolumn{1}{l}{VNC} &  \multicolumn{1}{l}{ZERO} &  \multicolumn{1}{l}{total}\\
\midrule
fiction & {390} & {697} & {4583} & {5670}\\
newspaper & {99} & {156} & {1227} & {1482}\\
reference & {400} & {784} & {4212} & {5396}\\
science & {1259} & {1916} & {11812} & {14987}\\
\lspbottomrule
\end{tabularx}
\caption{Overview of the token frequency of [in \textup{N}\textup{\textsubscript{dev}} \textup{\_] by} \textsc{genre}}
\label{tab:fleissner:4}
\end{table} 

The absolute distribution already suggests that, as anticipated, both constructions predominantly occur in \isi{scientific texts}. But in comparison to the usage of the nominal core elements outside the respective constructions, it also becomes apparent that these are at best tendencies, and no construction-specific patterns are readily discernible at first glance. The analyses conducted in the following subchapter aim to explain the emergence of the distributional patterns outlined here. 


\section{Analysis}\label{sec:fleissner:4}

In this section, I will present and discuss the results of the corpus study based on the data outlined in \sectref{sec:fleissner:3}. I will initially provide the findings for the parameter \textsc{age} (\sectref{sec:fleissner:4.1}) and subsequently for \textsc{genre}\is{genre} (\sectref{sec:fleissner:4.2}), before proceeding to the synthesis in \sectref{sec:fleissner:4.3}. Initially, the focus lies on conducting statistical tests for the respective sub data sets to elucidate the distributional patterns hinted at. For this purpose, Pearson Residuals\footnote{Pearson Residuals are included in the results obtained from the chisq.test() function in R (\citealt{R}).}  are calculated. These residuals help identify deviations between observed and expected values in the data, aiding in the interpretation of distributional trends. Once specific and statistically valid distribution patterns have been identified, conclusions can be drawn regarding the discourse-pragmatic potential of the analyzed structures.     


\subsection{\textsc{age} \textit{–} N\textsubscript{dev}}\label{sec:fleissner:4.1}
The deployment of Pearson residuals facilitates the identification of cells where the observed data significantly diverge from expected outcomes, assuming independence between the categorical variables. In essence, Pearson Residuals normalize a table's distribution, placing less emphasis on raw frequencies. The numerical values, presented in \tabref{tab:fleissner:5}, serve as a measure for the relative importance of cells.

\begin{table}
\begin{tabularx}{0.45\textwidth}{lrrr}
\lsptoprule
\textsc{age} &  \multicolumn{1}{l}{CP} &  \multicolumn{1}{l}{VNC} &  \multicolumn{1}{l}{ZERO}\\
\midrule
\textsc{protog} & {{}-3.05} & {{}-2.86} & {1.20}\\
\textsc{ohg} & {8.02} & {{}-3.15} & {-0.70}\\
\textsc{mhg} & {{}-9.44} & {0.34} & {1.60}\\
\textsc{enhg} & {{}-1.75} & {12.39} & {-2.52}\\
\textsc{nhg} & {8.51} & {{}-10.60} & {0.91}\\
\lspbottomrule
\end{tabularx}
\caption{Pearson Residuals, \textsc{cxn} \textup{by} \textsc{age}}
\label{tab:fleissner:5}
\end{table}

A positive residual indicates that the observed value surpasses expectations (and vice versa for negative values). Residuals within the range of 0 to ±2 suggest that the difference between observed and expected is small enough to be interpreted as random variation. Examining \tabref{tab:fleissner:5} for CPs\is{complex prepositions}, OHG\il{Old High German} nouns and NHG\il{New High German} nouns significantly exceed their expected occurrence, while the reverse holds true for MHG\il{Middle High German} nouns and PROTOG\il{Proto Germanic} nouns. For VNCs\is{verbonominal constructions}, different distributional patterns are observed. There is a strong tendency towards ENHG\il{Early New High German} nouns, while all other periods, especially NHG\il{New High German}, are disfavored. For ZERO, only ENHG\il{Early New High German} nouns reach statistical significance, as they slightly but significantly fall below the expected value. 

Employing residuals as a proportion-based method for data analysis is more advantageous than relying solely on raw frequencies. However, interpreting patterns of association between variables can be challenging within the confines of a table format. To address this, complex tabular data can be visualized through extended association plots (\citealt{Cohen1980}, \citealt{Meyer2003}). Grounded in the principle of residuals, these plots provide an intuitive interpretation of multiple pieces of information in a single visual representation. Furthermore, they enhance the identification of trends across a table. \figref{fig:fleissner:4} shows the association plot for the data set shown in \tabref{tab:fleissner:5}, plotting \textsc{age} by construction. These plots are interpreted as follows: In the graphical representation of contingency tables, the presence of red and blue tiles serves a pivotal role. These colors are indicative of the direction and magnitude of the residuals. Red tiles signify positive residuals, denoting instances where observed frequencies surpass expectation. Conversely, blue tiles represent negative residuals, signaling scenarios where observed frequencies fall below expectation. Light grey shades indicate the relative absence of a trend. The height of a tile corresponds to the absolute value of the residual, with tiles positioned above the line for positive residuals and below the line for negative residuals. Additionally, the width of a tile reflects the observed frequency, being wider for higher frequencies.

\begin{figure}
    \includegraphics[height=.46\textheight]{figures/Fig_6_Fleissner-cropped.png}
    \caption{\textsc{cxn} \textup{by} \textsc{age} \textup{in an extended association plot}}
    \label{fig:fleissner:4}
\end{figure}

It became clear that in this data set, as indicated by the examination of the tables, CPs\is{complex prepositions} show a significant positive association with NHG\il{New High German} nouns, depicted as high, dark blue tiles above the line, whereas VNCs\is{verbonominal constructions} show a significant positive association with ENHG\il{Early New High German} nouns. This is noteworthy because it suggests that both constructions may be subject to different diachronic dynamics. While CPs\is{complex prepositions} exhibit a particular affinity for the most recent available lexical units, VNCs\is{verbonominal constructions} seem to integrate them to a lesser extent. Given this background, the comparison of the developmental curves in \figref{fig:fleissner:1} and \figref{fig:fleissner:2} should also be reinterpreted: The significant increase in CPs\is{complex prepositions} in the 19th century correlates with the rise of modern lexemes in the structure. In VNCs\is{verbonominal constructions}, the increase is more moderate and diachronically preceding. The striking association of CP with OHG\il{Old High German} nouns, on the other hand, will need to be explained differently, as it contradicts the diachronic trend. The question arises whether it is possible to harmonize the trends of lexical favoring in specific periods with discourse-functional characteristics. A first step in this direction is taken with the second parameter.


\subsection{\textsc{genre} \textit{–} N\textsubscript{dev}}\label{sec:fleissner:4.2}
The examination of the parameter \textsc{genre} is based on the same principles as for \textsc{age}.  \tabref{tab:fleissner:6} shows the residuals.

\begin{table}[t]
\begin{tabularx}{0.5\textwidth}{lrrr}
\lsptoprule
{{genre}} &  \multicolumn{1}{l}{CP} &  \multicolumn{1}{l}{{VNC}} &  \multicolumn{1}{l}{{ZERO}}\\
\midrule
{{fiction}} & {{}-2.49} & {{}-1.28} & {1.30}\\
{{newspaper}} & {{}-1.54} & {{}-2.55} & {{}-1.51}\\
{{reference}} & {{}-1.02} & {3.32} & {{}-1.02}\\
{{science}} & {2.63} & {{}-0.41} & {{}-0.66}\\
\lspbottomrule
\end{tabularx}
\caption{Pearson Residuals, \textsc{cxn} \textup{by} \textsc{genre}}
\label{tab:fleissner:6}
\end{table}

The data reveal some interesting patterns. Firstly, it is notable that, for \textsc{genre},\is{genre} there are generally no statistical fluctuations to the extent observed with \textsc{age}. Regarding CPs\is{complex prepositions}, there is an expected distribution pattern: They are particularly present in \isi{scientific texts}, while there is a significant negative correlation in fiction. On the other hand, VNCs\is{verbonominal constructions} exhibit a stronger tendency towards reference literature, with other genres\is{genre} showing no positive values. In the case of ZERO, all residuals fall within the range of 0 to ±2, indicating that the disparity between observed and expected values is minimal and can be interpreted as random. This finding is far from trivial, as it indicates that the general use of \isi{deverbal nouns} is not \isi{genre}-driven, as often assumed. Even with lower expectations for significance values, it must be noted that nouns outside of the constructions CP and VNC exhibit a slight tendency towards narrative texts. \figref{fig:fleissner:5} shows the association plot for \textsc{genre}.\is{genre}

\begin{figure}
     \includegraphics[height=.46\textheight]{figures/Fig_7_Fleissner.png}
     \caption{\textsc{cxn} \textup{by} \textsc{genre} \textup{in an extended association plot}}
     \label{fig:fleissner:5}
 \end{figure}
\newpage
In summary, specific patterns emerge, but they are confined to the individual construction level. Both CPs\is{complex prepositions} and VNCs\is{verbonominal constructions} exhibit a preference for nouns from specific historical periods. CPs\is{complex prepositions} show an affinity for NHG\il{New High German} nouns, while VNCs\is{verbonominal constructions} favor ENHG\il{Early New High German} nouns. Interestingly, there is a reciprocal disfavoring, where CPs\is{complex prepositions} tend to disfavor ENHG\il{Early New High German} nouns, and VNCs\is{verbonominal constructions} disfavor NHG\il{New High German} nouns. Additionally, both constructions display a distinct genre-specific\is{genre} distribution, favoring non-narrative\is{non-narrativity} genres. The reciprocal preference extends to \textsc{genre}\is{genre} as well, since each construction is more prominent in a specific \isi{genre} within the non-narrative\is{non-narrativity} category. This reciprocal pattern in both historical periods and different texts adds an intriguing layer to the analysis. The general difficulty in studying non-association lies in the challenge of providing clear evidence for absence, particularly when dealing with linguistic data. Therefore, the first step is to shed light on the role of ENHG\il{Early New High German} and NHG\il{New High German} nouns in their respective constructions.


\subsection{The emergence of [\textit{in} N\textsubscript{dev}\_]}\label{sec:fleissner:4.3}
To obtain a more precise understanding of the function of nouns of different ages within the structure [\textit{in} N\textsubscript{dev}\_], I applied a Correspondence Analysis \citep{Glynn2014} as an explorative technique to identify patterns and dependencies among the categories and represent them graphically. This method is particularly useful when dealing with large data sets with multiple categorical variables, providing insights into the structure and associations within the data – especially when dealing with two variables, where one exhibits a higher level of complexity compared to the other. For this analysis, the two variables \textsc{genre}\is{genre} and \textsc{nouns} have been chosen. \textsc{Genre}\is{genre} involves a relatively limited set, only four distinct \isi{genre} types, while the high number of individual lexical types complicates a clear presentation in a contingency table: CPs\is{complex prepositions} are distributed across 191 different nouns in Period 9 (69 of them being of NHG\il{New High German} origin), VNCs\is{verbonominal constructions} across 231 (of which 74 are ENHG\il{Early New High German}). 

For CPs\is{complex prepositions}, all the nouns that appeared only once in the data set where removed, leaving a set of 20 lexemes. The exclusion of hapaxes from the investigation is imperative in techniques such as CA, where the objective is to unveil relationships and associations between variables by exploring co-occurrence patterns. Hapaxes contribute disproportionately to the sparsity of the data matrix without providing substantial information regarding their associations with other variables. Their singular occurrences lack the consistency and repetition required to discern statistically significant patterns. \figref{ex:fleissner:6} displays the reduced data set and the corresponding CA plot.    

 \begin{figure}
     \centering
     \includegraphics[width=0.95\linewidth]{figures/Fig_8_Fleissner.png}
     \caption{NHG nouns by \textsc{genre} \textup{in a CA plot (CP)}, see \tabref{tab:fleissner:tofig8}}
     \label{fig:fleissner:6}
 \end{figure}

The observable patterns for CPs\is{complex prepositions} reveal a predominant clustering of the majority of nouns and non-narrative\is{non-narrativity} genres\is{genre} like Reference literature and Science\is{scientific texts} on the left side of the plot. The axis weighting within the graph indicates that over 83\% of the data can be explained through the \isi{genre} dichotomy ``narrative – non-narrative\is{non-narrativity}''. The size of the dots represents the frequency or weight of the categories or variables they represent. Larger dots indicate higher frequency, while smaller dots indicate lower frequency in the analyzed data.

When looking at the most frequent lemmas, certain discourse-functional patterns become apparent. The constructions [\textit{im Verhältnis} P] ‘in relation to', [\textit{im Vergleich mit}] ‘in comparison to' and [\textit{im Zusammenhang mit}] ‘in connection with' are used to establish relationships, make comparisons, and highlight connections between different elements or concepts. They are employed to provide context, clarify relationships between ideas under discussion, or emphasize their relevance. These constructions contribute to precision and clarity, helping readers understand the context or the comparative nature of the statements being made. Instead of explicitly emphasizing who is engaging in the discussion of the topics, there is a backgrounding of the agent, placing less focus on the individuals involved in the discourse. In addition, sentences can be formulated that do not have an agent at all. This strategy can contribute to a more objective or formal tone, making the text more impersonal or general in nature, see \xxref{ex:fleissner:6}{ex:fleissner:8}.

\ea%6
\label{ex:fleissner:6}
%\gll 
{\itshape Ich bezweifle, ob man die Vergangenheit \textbf{in} \textbf{ihrem} \textbf{Verhältnis} \textbf{zur} \textbf{Gegenwart} überhaupt richtig zu beurteilen vermag.}\\
%{} {} {} {} {} {} in \textsc{poss}{}-\textsc{dat.sg} relation-\textsc{dat.sg} to.the\textsc{{}-dat.sg} present\textsc{{}-dat.sg} {} {} {} {} {} \\
\glt ‘I doubt whether one is able to judge the past \textbf{in its relation to the present }correctly at all.' \hfill (Chamberlain, 1899)
\z 

\ea%7
\label{ex:fleissner:7}
%\gll 
{\itshape Wir wissen, welche kleine Zahl regierender Königinnen die Geschichte \textbf{im} \textbf{Vergleich} \textbf{zu} \textbf{den} \textbf{Königen} aufzuweisen hat.}\\
%{} {} {} {} {} {} {} {} {} in.the\textsc{{}-dat.sg} comparison-\textsc{dat.sg} to the\textsc{{}-dat.pl} king\textsc{{}-dat.pl} \\
\glt ‘We know what a small number of reigning queens history has \textbf{compared to kings}.' \hfill (Ichenhaeuser, 1898)
\z 

\ea%8
\label{ex:fleissner:8}
%\gll 
{\itshape Umgekehrt können \textbf{im} \textbf{Zusammenhang} \textbf{mit} \textbf{dem} \textbf{Rechte} des Gemeingebrauchs dem Einzelnen auch besondere Nachteile erwachsen.}\\
%{} {} in.the\textsc{{}-dat.sg} connection{}-\textsc{dat.sg} with the\textsc{{}-dat.sg} right\textsc{{}-dat.sg} \\
\glt ‘Conversely, \textbf{in connection with the right} of public use, the individual may also suffer special disadvantages.' \hfill (Mayer, 1896)
\z 

This subgroup of nouns behaves similarly with regard to the concept of \textsc{relation} and suggest a thematic connection associated with analyzing relationships of different subjects. 

A second major subgroup can be distinguished from \textsc{relation}, falling under the broad category of \textsc{consideration} including constructions like [\textit{in Rücksicht} P/\textsc{gen}] ‘in consideration of', [\textit{in Hinsicht} P/\textsc{gen}] ‘in view of' and [\textit{in Betreff} \textsc{gen}]\footnote{The noun \textit{Betreff} is somewhat misleading in the plot. Essentially, it is equivalent to \textit{Hinsicht} but seems to exhibit a tendency toward fictional texts. However, this inclination is solely due to the extensive use of the construction [\textit{in Betreff} \textsc{gen}] by Otto von Bismarck in his memoirs. Outside of this context, \textit{Betreff} appears only in \isi{scientific texts}. The less frequent constructions [\textit{in Rückblick auf}] and [\textit{in Rücksicht auf}] are also used by Bismarck, which accounts for their position in the middle of the plot. This illustrates the limitations of exploratory statistical methods, as they cannot account for the individual habits of specific authors. Unsurprisingly, Otto von Bismarck leans towards a formal \isi{style} even in his personal writings, reflecting his accustomed political demeanor.} ‘regarding' and others. The nouns used in these constructions suggest a thematic connection related to cognitive processes like consideration, understanding, agreement, and acknowledgment, see \xxref{ex:fleissner:9}{ex:fleissner:11}.

\ea%9
\label{ex:fleissner:9}
%\gll
{\itshape Dadurch war seiner Untersuchung \textbf{in} \textbf{Hinsicht} \textbf{der} \textbf{individuellen} \textbf{Tugenden} der Weg vorgezeichnet}\\
%{} {} {} {} \textit{in} regard-\textsc{dat.sg}  the-\textsc{gen.pl}  individual-\textsc{gen.pl}  virtue-\textsc{gen.pl} \\
\glt ‘Thus the way was marked out for his investigation \textbf{with regard to the individual virtues}.' \hfill (Natorp, 1899)
\z 
         
\ea%10
\label{ex:fleissner:10}
%\gll  
{\itshape Nichtbefolgung der Anweisungen wird \textbf{in} \textbf{Rücksicht} \textbf{auf} \textbf{Strafen}  \textbf{und} \textbf{Zwangsmittel} ebenso behandelt wie die Weigerung der Erfüllung überhaupt.}\\
%{} {} {} {} in regard-\textsc{dat.sg}  on  penalty{}-\textsc{acc.pl}  and  coercive{}-\textsc{acc.pl} \\
\glt ‘Non-compliance with the instructions shall be treated in the same way as refusal to comply at all, \textbf{with regard to penalties and coercives}.'\\ \hfill (Mayer, 1896)
\z 
        
\ea%11
\label{ex:fleissner:11}
%\gll
{\itshape Beobachtungen \textbf{in} \textbf{Betreff} \textbf{der} \textbf{umgebenden} \textbf{Natur} (z.B. des gestirnten Himmels) können schon weit gediehen und ein mannigfaltiger Götter- und Dämonenkultus entstanden sein.}\\
%{}  in regard-\textsc{dat.sg}  the-\textsc{gen.sg}  sourrounding{}-\textsc{gen.sg}  nature{}-\textsc{gen.sg} \\
\glt ‘Observations \textbf{concerning the surrounding nature} (e.g. the starry sky) may already have developed far and a manifold cult of gods and demons may have arisen.' \hfill (Chamberlain, 1899)
\z 

As the constructions within the group of \textsc{relation}, the constructions in the given examples serve to shift the focus away from the specific agent or doer of the action, emphasizing instead the perspective, context, or considerations related to the action. In this case, the attention is directed towards the investigation and the path that was outlined in terms of individual virtues, rather than explicitly highlighting who conducted the investigation. 

It can be concluded that NHG\il{New High German} nouns, which are preferred in the structure [\textit{in} N\textsubscript{dev} P/\textsc{gen}], are used to form a \textsc{reference} construction to meet the requirements of non-narrative\is{non-narrativity}, especially scientific formal texts.\is{scientific texts} Both \textsc{relation} and \textsc{consideration} can be understood as discursive means of \textsc{reference}: Mental content is introduced into the discourse, whereby, in the case of \textsc{relation}, the aim is to compare different subjects, while, in the case of \textsc{consideration}, individual subjects are established as a frame of \textsc{reference} within which various independent subjects are perspectivized. \tabref{tab:fleissner:7} summarizes these findings.

\begin{table}
\begin{tabularx}{\textwidth}{lQQ}
\lsptoprule
{group} & \multicolumn{2}{c}{{{reference} }}\\
{subgroup} & {relation} & {consideration}\\
\midrule
\textsc{nouns} & \textit{Verhältnis, Vergleich, Entwicklung, Zusammenhang, Abschluss, Entfaltung} & \textit{Betreff, Rücksicht, Hinsicht, Auffassung, Einverständnis, Rückblick, Berücksichtigung, Anknüpfung}\\
\textsc{characteristics} & connections, similarities, or associations between different subjects & cognitive aspects related to the author's perspective and consideration of the subject\\
\textsc{construction} & \multicolumn{2}{c}{{[\textit{in} N\textsubscript{reference} P/\textsc{gen}]}}\\
\lspbottomrule
\end{tabularx}
\caption{Semantic groups within the general pattern [in \textup{N}\textup{\textsubscript{dev}} \textup{P/}\textsc{gen}]}
\label{tab:fleissner:7}
\end{table}

The subgroup \textsc{relation} is characterized by constructions that involve comparing or establishing relationships between different subjects or concepts, expressing connections, similarities, or associations between entities, without explicitly highlighting the agent or source of the comparison. On the other hand, the subgroup \textsc{consideration} appears to involve cognitive aspects related to the author's perspective and consideration of the subject matter. These constructions suggest a focus on the author's examination, understanding, or acknowledgment of various aspects, and they may indicate a form of consideration or agreement with certain ideas or points. This subgroup reflects a discourse strategy where the author engages in a more reflective and contemplative stance towards the subject without explicitly foregrounding themselves as the agent of the action.

The emergence of a \textsc{reference} construction explains the increase in use of the structure in the late 18th and 19th century, as it is mainly used in modern-like \isi{scientific texts}. This assumption is supported by the picture that emerges when contrasting the distribution of NHG\il{New High German} nouns with those of other periods. \figref{fig:fleissner:7} shows the distribution of the 20 most frequent OHG\il{Old High German}, MHG\il{Middle High German}, and ENHG\il{Early New High German} nouns in relation to their \isi{genre} affinity.

\begin{figure}
    \includegraphics[width=0.95\linewidth]{figures/Fig_9_Fleissner-cropped.png}
    \caption{PROTOG, OHG, MHG and ENHG nouns by \textsc{genre} in a CA plot (CP), see \tabref{tab:fleissner:tofig9}}
    \label{fig:fleissner:7}
\end{figure}

This distribution pattern presents a less structured image compared to NHG\il{New High German} nouns. A clear distinction between narrative and non-narrative\is{non-narrativity} genres,\is{genre} as seen in \figref{ex:fleissner:6}, does not emerge to the same extent. While individual lexical representatives of the reference group like \textit{Bezug} ‘relation', \textit{Verbindung} ‘connection' or \textit{Übereinstimmung} ‘agreement' can be found here as well, they coexist with a variety of nouns with very heterogeneous semantics. \textit{Bezug} shows most clearly that the existence of a productive \textsc{reference} construction is to be assumed for the period under examination. As an OHG\il{Old High German} word, it is the prototypical and most used noun with \textsc{reference} semantics in \ili{German}. Nevertheless, it appears in the structure [\textit{in} N\textsubscript{dev} \_] only from the 18th century on, see \figref{ex:fleissner:8}. It seems to be a general trend in \ili{German} that \textit{ung}{}-derivations are increasingly replaced by products of older word formations\is{word formation} (see \citealt{Demske2000}: 373, \citealt{Hartmann2016}: 259). \citet[153]{Schmidt2007} sees a tendency towards shortening from the 18th century onwards, which could be linked to the development of technical language. This would not only explain why \textit{Bezug} possibly prevails over the younger word \textit{Beziehung} ‘relation', but also why \textit{Lauf} ‘course’ prevails over \textit{Verlauf}  ‘course’. These two frequent lexemes are responsible for the striking association of CPs\is{complex prepositions} with OHG\il{Old High German} nouns, which is shown in \figref{ex:fleissner:4}. The appearance of \textit{Bezug} further supports the assumption of a NHG\il{New High German} \textsc{reference} construction. While \textit{Rücksicht}, \textit{Hinsicht}, and \textit{Berücksichtigung} originated relatively recently, all these nouns are attested almost simultaneously along \textit{Bezug} within the pattern [\textit{in} N\textsubscript{dev} \_]. This simultaneous emergence, despite the varied historical periods of their coinage, lends credence to the likelihood of a formation schema guiding their usage.

\begin{figure}
     \includegraphics[width=\textwidth]{figures/Figure 8 new.pdf}
     \caption{Token frequency of [in Bezug\_] in the DTA Kernkorpus}
     \label{fig:fleissner:8}
\end{figure}

When examining VNCs\is{verbonominal constructions} and their favored ENHG\il{Early New High German} nouns with the highest frequency, two observations can be made. Firstly, the token count significantly exceeds that of CPs\is{complex prepositions}, as shown in \figref{fig:fleissner:4}. Secondly, the organization of the semantic space with regard to \textsc{genre}\is{genre} appears less rigid. While CPs\is{complex prepositions} exhibited a two-parted system with a clear distinction between narrative and non-narrative\is{non-narrativity} genres,\is{genre} VNCs\is{verbonominal constructions} present a more triangular configuration in which the non-narrative\is{non-narrativity} genres\is{genre} in particular form different clusters, see \figref{fig:fleissner:9}.

\begin{figure}
    \centering
    \includegraphics[width=0.95\linewidth]{figures/Fig_11_Fleissner-cropped.png}
    \caption{ENHG nouns by \textsc{genre} in a CA plot (VNC), see \tabref{tab:fstofig11}}
    \label{fig:fleissner:9}
\end{figure}

The most significant difference to CPs\is{complex prepositions} is that non-narrative\is{non-narrativity} genres\is{genre} exhibit distinct behaviors concerning the structure [\textit{in} N\textsubscript{dev} V]. The sorting of individual functional clusters proves to be significantly more challenging, as no semantic group dominates in absolute frequency. Just as for CPs\is{complex prepositions}, a certain tendency towards \textsc{reference} can also be observed for VNCs\is{verbonominal constructions}. Both the subgroup \textsc{relation} and the subgroup \textsc{consideration} are represented, see \xxref{ex:fleissner:12}{ex:fleissner:14}.

\ea
\label{ex:fleissner:12}
%\gll 
{\itshape Die vielfachen, namentlich in England gemachten Verſuche, denſelben mit der modernen Entwickelungslehre \textbf{in} \textbf{Einklang} \textbf{zu} \textbf{bringen}, ſind völlig fehlgeſchlagen.}\\
%{} {} {} {} {} {} {} {} {} {} {} {} in harmony-\textsc{acc.sg}  to  bring{}-\textsc{inf}  \\
\glt ‘The many attempts, especially in England, to \textbf{bring it in line} with the modern theory of development have completely failed.' \hfill (Haeckel, 1899)
\z 

\ea%13
\label{ex:fleissner:13}
%\gll 
{\itshape Form und Jnhalt der Poesie \textbf{treten} \textbf{in} \textbf{Uebereinstimmung}.}\\
%{} {} {} {} {}  step{}-\textsc{pres.3.pl}  in  accordance{}-\textsc{acc.sg} \\
\glt ‘The form and content of the poetry \textbf{coincide}.' \hfill (Wolff, 1899)
\z

\ea%14
\label{ex:fleissner:14}
%\gll
{\itshape Er \textbf{hat} nämlich den Ort \textbf{in} \textbf{Erwägung} \textbf{gezogen}, von wo dieſe Wetterfahne ſtammt.}\\
%{}  have{}-\textsc{aux.perf.3.sg} {} {} {}  in consideration-\textsc{acc.sg}  pull-\textsc{ptcp.perf}  \\
\glt ‘In fact, \textbf{he has considered} the place from where this weathervane comes.'\\ \hfill (Fontane, 1899)
\z 

As can be seen in example \REF{ex:fleissner:14}, the agent-backgrounding function of VNC is not as distinct as it was in the case of CPs\is{complex prepositions}. This is due to the fact that the argument structure of VNCs\is{verbonominal constructions} requires the explicit mention of a subject, unless a passive construction is employed. It is therefore not coincidental that VNCs\is{verbonominal constructions} are used more frequently in narrative genres.\is{genre} Non-narrative\is{non-narrativity} \isi{scientific texts} necessarily have fewer agentive subjects due to their focus on presenting objective information and analysis rather than on describing actions performed by specific agents. The emphasis typically lies on conveying findings, observations, and interpretations rather than on detailing the actions of individuals or entities. As a result, the use of agentive subjects may be minimized in favor of passive constructions or nominalizations, which allow for a more impersonal and objective presentation of information. VNCs\is{verbonominal constructions} exhibit a greater openness towards different genres,\is{genre} reflected in expressions of emotional or psychological states with nouns like \textit{Stimmung} ‘mood' or \textit{Erregung} ‘excitement', which mainly require animate and agentive subjects, see \REF{ex:fleissner:15}.

\ea%15
\label{ex:fleissner:15}
%\gll
{\itshape Übrigens fange ich an, \textbf{in}  \textbf{Stimmung} \textbf{zu}  \textbf{kommen}.}\\
%{} {} {} {}    in mood-\textsc{acc.sg}  to  come-\textsc{inf}  \\
\glt ‘By the way, I'm starting\textbf{ to get in the mood}.' \hfill (Bierbaum, 1897)
\z 

Another noticeable subgroup comprises nouns that can be associated, in the broadest sense, with the concept \textsc{usage}. It includes nouns related to aspects of processes and activities. These nouns suggest a focus on how things operate, are used, controlled or function in a practical sense, see \xxref{ex:fleissner:16}{ex:fleissner:17}.

\ea%16
\label{ex:fleissner:16}
%\gll 
{\itshape Der Union-Ofen bei Connelsville \textbf{kam} 1791 \textbf{in} \textbf{Betrieb}.}\\
%{} {} {} {} come{}-\textsc{pret.3.sg}  {}  in  operation{}-\textsc{acc.sg} \\
\glt ‘The Union Furnace at Connelsville \textbf{came into operation }in 1791.'\\ \hfill (Beck, 1897)
\z 

\ea%17
\label{ex:fleissner:17}
%\gll 
{\itshape Das Verfahren \textbf{wurde} 1810 in dem damaligen Königreich Westfalen bekannt gemacht und auch \textbf{in} \textbf{Anwendung} \textbf{gebracht}.}\\
%{} {} `become-\textsc{aux.pass.3.sg} {} {} {} {} {} {} {} {} {} {} in usage{}-\textsc{acc.sg}  bring{}-\textsc{ptcp.perf} \\
\glt ‘The method \textbf{was} introduced and \textbf{applied} in the Kingdom of Westphalia in 1810.' \hfill (Beck, 1899)
\z 

\tabref{tab:fleissner:8} provides an overview of the rough classification of \ili{Early New High German} (ENHG) nouns in the structure [\textit{in} N\textsubscript{dev}V]. 

\begin{sidewaystable}[ph!]
\begin{tabularx}{\textwidth}{lQQQQQ}
\lsptoprule
{group} & \multicolumn{2}{c}{{{reference} }} & \multicolumn{2}{c}{{usage}} &  {(emotional) state}\\
\textsc{subgroup} & {\textsc{relation}} & \textsc{considera-tion} & \textsc{operation} & \textsc{(technical)}
\textsc{state} & \\
\midrule
\textsc{nouns} & \textit{Verbindung, Beziehung, Übereinstimmung} & \textit{Aussicht, Einklang, Erinnerung, Vergessenheit, Verdacht, Erwägung} & \textit{Betrieb, Anwendung, Drehung, Verkehr, Umlauf, Verwendung,} \textit{Verbindung} & \textit{Stand, Besitz, Aufnahme,} \textit{Entfernung,} \textit{Zustand} & \textit{Stimmung, Zustand, Erregung,} \textit{Beziehung,} \textit{Verbindung,}
\textit{Stand}\\
\textsc{characteristics} & connections, similarities, or associations between different subjects & cognitive aspects related to the author's perspective and consideration of the subject & {nouns related to the practical aspects of processes and activities} & {nouns related to a controlled state by an agent} & {feelings or emotions experienced by an animated entity}\\
\textsc{construction} & \multicolumn{5}{c}{{[\textit{in} N\textsubscript{dev}V]}}\\
\lspbottomrule
\end{tabularx}
\caption{Semantic groups within the general pattern [in \textup{N}\textup{\textsubscript{dev}} \textup{V]}}
\label{tab:fleissner:8}
\end{sidewaystable}

The variety of usage contexts of ENHG\il{Early New High German} nouns in the structure [\textit{in} N\textsubscript{dev}V] makes it clear that hardly any construction type is able to prevail over others. It is therefore not surprising that the semantic space differs little compared to that of all other nouns. \figref{ex:fleissner:10} shows the distribution of the 20 most frequent OHG\il{Old High German}, MHG\il{Middle High German} and NHG\il{New High German} nouns in relation to their \isi{genre} affinity.

\begin{figure}
    \centering
    \includegraphics[width=0.95\linewidth]{figures/Fig_12_Fleissner-cropped.png}
    \caption{PROTOG, OHG, MHG and NHG nouns by \textsc{genre} \textup{in a CA plot (VNC)}, see \tabref{tab:fleissner:tofig12}}
    \label{fig:fleissner:10}
\end{figure}
 
It must be noted that such an approach, which has proven fruitful for CPs\is{complex prepositions}, is misguided for VNCs\is{verbonominal constructions}. While \textit{Betracht} ‘consideration', \textit{Gebrauch} ‘usage', \textit{Bewegung} ‘motion', \textit{Frage} ‘question', and \textit{Gang} ‘motion' can be assigned to the identified subgroups, the semantic heterogeneity of the remaining lexical items is difficult to meaningfully categorize. At first, older nouns like \textit{Betracht} seem to indicate that VNCs\is{verbonominal constructions} are also drawn into the pull of a \textsc{reference} construction, as from the 18th century, there is an increased use of [\textit{in Betracht} \_], see \figref{ex:fleissner:11}.

 \begin{figure}
     \includegraphics[width=\textwidth]{figures/Figure 11 new.pdf}
     \caption{Token frequency of [in Betracht\_\textup{] in the DTA Kernkorpus}}
     \label{fig:fleissner:11}
 \end{figure}
\largerpage
\textit{Betracht} is actually a MHG\il{Middle High German} adjective that was rarely used as a noun before the 17th century. In the 18th century, it established itself against the lexical competitor \textit{Betrachtung}. It is conceivable that the shorter \textit{Betracht} was preferred in more fixed and specified constructions, as it does not exist as an autonomous noun outside of [\textit{in Betracht} V]. Here, again, the aforementioned tendency to shorten is found. With the increasing use of \textit{Betracht}, the decline of \textit{Betrachtung} was initiated, see \figref{fig:fleissner:12}.

 \begin{figure}
     \includegraphics[width=\textwidth]{figures/Figure 12 new.pdf}
     \caption{Token frequency of [in Betrachtung\_\textup{] in the DTA Kernkorpus}}
     \label{fig:fleissner:12}
 \end{figure}
\clearpage
While \textit{Betrachtung} no longer appears in [\textit{in} N\textsubscript{dev} V] in contemporary \ili{German}, it is still in use outside of the construction. [\textit{in Betrachtung} \_]\footnote{In this case, [\textit{in Betrachtung} \_] and [\textit{in Betrachtung} V] can be considered almost identical, as the structure is formed almost exclusively with a verb.} was already an established construction in earlier periods. In summary, it can be said that VNCs\is{verbonominal constructions} exhibit a high degree of diversity in terms of \textsc{Genre}\is{genre} and are challenging to cluster effectively with respect to their discourse-pragmatic functions. The increase in formations of this type can be explained from this perspective only insofar as there was already an established structure that was additionally fueled by the increased availability of nominal material in \ili{Early New High German}. 


\section{Discussion}\label{sec:fleissner:5}

CPs\is{complex prepositions} are strongly associated with the function \textsc{reference}, reflecting a specialized syntactic and discourse-pragmatic development as required by evolving communicative needs and changes in textual conventions. This includes, first and foremost, agent backgrounding. Non-narrative texts,\is{non-narrativity} especially scientific papers\is{scientific texts} or technical documents, prioritize the communication of information. Agent backgrounding helps to keep the focus on the actions, processes, or concepts being discussed rather than on the individuals involved. \textsc{reference} constructions can be employed to establish a clear and precise connection between concepts or entities without foregrounding the agent. CPs\is{complex prepositions} also play a crucial role in condensing information in a concise and efficient manner. Unlike whole subordinated clauses, \isi{complex prepositions} allow for the expression of complex relationships and ideas within a compact structure. This conciseness is particularly valuable in \isi{scientific texts}, where clarity and precision are paramount. In contrast to CPs\is{complex prepositions}, which excel at expressing relationships in a compact form, VNCs\is{verbonominal constructions} operate differently in terms of structure, since they form syntactic patterns that include a subject position. Therefore, VNCs\is{verbonominal constructions} lack the obligatory agent backgrounding, even though numerous constructions of the pattern [\textit{in} N\textsubscript{dev} V] allow for an agent-shifted perspective. It is not surprising that some constructions of this type can be found, which are also associated with a realm of \textsc{relation} and \textsc{consideration} and thus contribute to the discourse-functional strengthening of the structure [\textit{in} N\textsubscript{dev} \_]. In contrast to CPs\is{complex prepositions}, multi-word predicates like VNCs\is{verbonominal constructions} are not suitable for information summarization, as they require significantly more material compared to simplex verbs to verbalize similar semantic content. Because of these differences, it would therefore not be justified to speak of a separate semi-schematic construction [\textit{in} N\textsubscript{dev} \_] with the meaning \textsc{reference}. If one wishes to continue operating within this constructional framework, one would need to assume a more abstract schema at the discourse level, which is constituted by the CPs\is{complex prepositions}, but also by some VNCs\is{verbonominal constructions}. In recent years, many linguists have recognized that constructions encompass not only conventional patterns within individual syntactic patterns but also extend to broader linguistic units (for an overview, see \citealt{EnghelsSansiñena2021}). But the integration of these approaches into constructional frameworks is still up for debate, as they would necessarily entail loosening specific form-meaning dichotomies. Considering the existence of such discursive macrostructures, CPs\is{complex prepositions} and VNCs\is{verbonominal constructions} would at least be suitable functional bridging contexts, as they also exhibit a formal overlap establishing a potential associative relationship. 
 \figref{fig:fleissner:13} shows the partial network of such a \isi{\textsc{reference} discourse schema}.

\begin{figure}
\centering
\resizebox{\textwidth}{!}{
\begin{tikzpicture}[node distance=2.5cm]
\node (start) [process] {[\textit{in} N\textsubscript{reference}]};
\node (in1) [process, below of=start] {[\textit{in} N\textsubscript{reference} P/\textsc{gen}]};
\node (pro1) [process, below of=in1] {[\textit{in Bezug auf}] \linebreak[4] [\textit{in Ber\"ucksichtigung des}]};
\node (dec1) [processlight, left of=pro1, xshift=-2.5cm] {[\textit{in Beziehung setzen}]\linebreak[4] [\textit{in Betracht kommen}]};
\node (dec2) [processlight, above of=dec1] {[\textit{in}] N\textsubscript{reference} V};

\node (pro2b) [startstop, right of=pro1, xshift=2cm] {individual construction(s)};
\node (out1) [startstop, above of=pro2b] {schema};
\node (stop) [startstop, above of=out1] {discourse schema};

\draw [->] (in1) -- (start);
\draw [->] (dec2) -- (start);
\draw [->] (pro1) -- (in1);
\draw [->] (dec1) -- (dec2);
\end{tikzpicture}
}
\caption{Partial network of [in \textup{N}\textup{\textsubscript{dev}} \textup{\_]}}
\label{fig:fleissner:13}
\end{figure}

CPs\is{complex prepositions} contribute significantly to the existence of the schema. In contrast, the contribution of VNCs\is{verbonominal constructions} is considerably less strong, since the semantics of the nouns within the structure [\textit{in} N\textsubscript{dev} V] are too heterogeneous for a prototypical meaning to emerge to the same extent. There is, therefore, much to suggest that much more diverse and freer syntactic patterns are at play in VNCs\is{verbonominal constructions}, which were already established earlier. An abstract \isi{discourse schema} cannot be derived for this structure. This is also indicated by the development curves of individual nouns such as \textit{Betrachtung}, which, as the oldest reference noun, shows a historically stable frequency. From \ili{New High German} onwards, an increase in the frequency of VNCs\is{verbonominal constructions} can be observed, as an enlarged lexical pool is available for the established structure to draw from. \figref{fig:fleissner:14} shows the partial network for VNCs\is{verbonominal constructions}.

\begin{figure}
\centering
\resizebox{\textwidth}{!}{
\begin{tikzpicture}[node distance=2.5cm]
\node (start) [process] {[\textit{in} N\textsubscript{dev} V]};
\node (in1) [process, below of=start] {[\textit{in} N\textsubscript{reference} V]};
\node (pro1) [process, below of=in1] {[\textit{in Beziehung setzen}] \linebreak[4] [\textit{in Betracht kommen}]};
\node (dec1) [process, left of=pro1, xshift=-2.5cm] {[\textit{in Betrieb setzen}]\linebreak[4] [\textit{in Gebrauch kommen}]};
\node (dec2) [process, above of=dec1] {[\textit{in}] N\textsubscript{usage} V};
\node (in2) [process, right of=pro1, xshift=2.5cm] {[\textit{in Stimmung komnen}]\linebreak[4] [\textit{in Erregung geraten}]};
\node (in3) [processlight, above of=in2] {[\textit{in}] N\textsubscript{state} V};
\node (pro2b) [startstop, right of=in2, xshift=2cm] {individual \linebreak[4] construction(s)};
\node (out1) [startstop, above of=pro2b] {sub-schemas};
\node (stop) [startstop, above of=out1] {schema};
\draw [->] (in1) -- (start);
\draw [->] (dec2) -- (start);
\draw [->, dashed] (in2) -- (start);
\draw [->] (pro1) -- (in1);
\draw [->] (dec1) -- (dec2);
\draw [->, dashed] (in2) -- (in3);
\end{tikzpicture}
}
\caption{Partial network of German verbonominal constructions [in \textup{N}\textup{\textsubscript{dev}} \textup{V]}}
\label{fig:fleissner:14}
\end{figure}

It is evident that VNCs\is{verbonominal constructions} do not exhibit uniform discourse-specific behavior. The two identified and frequency dominant sub-schemas \textsc{usage} and \textsc{reference} reflect the functionality of the texts in which they predominantly appear. They differ primarily in the deictic origo shift with which they are prototypically associated. While the reference construction often fulfills the function of establishing an author-distant discourse perspective, this is not necessary for the usage construction. The agent involved, whether explicitly named or not, is usually not the author himself. It is not surprising that, unlike CPs\is{complex prepositions}, VNCs\is{verbonominal constructions} can be seen to have a more even distribution of the most frequent nouns across different genres,\is{genre} although there is some tendency towards non-narrative texts.\is{non-narrativity} In this respect, however, the nominal cores of VNCs\is{verbonominal constructions} differ little from autonomous \isi{deverbal nouns} occurring outside of the structures under study. This is shown by a final look at the distribution of the 20 highest frequency \textsc{ZERO} nouns, see \figref{fig:fleissner:15}.

 
\begin{figure}[t]
    \centering
    \includegraphics[width=1\linewidth]{figures/Fig_17_Fleissner-cropped.png}
    \caption{Most frequent nouns by \textsc{genre} \textup{in a CA plot (}\textsc{ZERO}), see \tabref{tab:fleissner:tofig17}}
    \label{fig:fleissner:15}
\end{figure}

The triangulated system, already encountered with the VNCs\is{verbonominal constructions} in \figref{ex:fleissner:9}, appears again. It most closely reflects the general distribution of nouns in the 19th century. It seems that the assumption of a general nominal \isi{style}\is{German Nominalstil} in non-narrative\is{non-narrativity} text types seems to be confirmed, even independently of the two structures studied. VNCs\is{verbonominal constructions} of the pattern [\textit{in} N\textsubscript{dev} V] only deviate slightly from the typical null model in this genre, but CPs\is{complex prepositions} of the pattern [\textit{in} N\textsubscript{dev} P/GEN] deviate significantly.


\section{Conclusion}\label{sec:fleissner:6}

This paper has offered a diachronic take on the formation of \isi{complex prepositions} and \isi{verbonominal constructions} in \ili{German}. Two distinct syntactic patterns were analysed, [\textit{in} N\textsubscript{dev} P/GEN] exhibiting clear characteristics and [\textit{in} N\textsubscript{dev} V] displaying a broader, more open structure with diverse semantics. This leads me to some conclusions, which return to the initial observation mentioned that \ili{German} has developed a strong tendency towards nominal \isi{style}.\is{German Nominalstil} This phenomenon seems to be mainly related to the development of modern non-narrative\is{non-narrativity} text structures in \ili{German}, as often assumed. Embedded in this diachronic trend is the development of a \isi{\textsc{reference} discourse schema}, which is mainly supported by CPs\is{complex prepositions}. Interestingly, this construction shows a very strong tendency towards the most modern nouns available to 19th century \ili{German}. The availability of these nouns thereby stimulates the discursive schematization of the pattern [\textit{in} N\textsubscript{dev} \_]. Looking across constructional boundaries also shows that focusing on purely formal or structural characteristics of individual constructions in a discourse-functional context should not be overemphasized. CPs\is{complex prepositions} with other prepositions than \textit{in} show similar behavior. This is exemplified by the schema [\textit{mit} N\textsubscript{dev} \textit{auf}/GEN] with the nouns \textit{Bezug} ‘regard', \textit{Beziehung} ‘relation', \textit{Berücksichtigung} ‘consideration', \textit{Hinsicht} ‘regard', and \textit{Rücksicht} ‘consideration' in the data. \citet{Ruffthc} observe a rapid emergence of these five individual CPs\is{complex prepositions} following the same formation pattern and displaying \textsc{reference} semantics. This diachronic scenario does not necessarily depend on the gradual routinization and conventionalization of a particular syntagmatic string. Instead, it involves the creative utilization of resources available to the speakers of a speech community. In this process, new lexical items in a language are created almost instantly through the creative application of a \textsc{reference} schema.\is{\textsc{reference} discourse schema} Competing structures with the same function should prove fruitful as future fields of research, since it cannot be ruled out that different prepositions are also used to convey nuanced discourse functions.\is{discourse function} However, such functions are not foreseeable at this point. In addition to the traditional consideration of genre-specific\is{genre} distribution patterns, it also became clear that the parameter \textsc{Age} in relation to different constituents of constructions should be considered in future studies. From a speaker's point of view, it must be assumed that established schemata are usually insensitive to the age of their nominal elements. After all, language users are not usually aware of the age of a lexeme, and nouns of all ages can be found in both constructions. Across the multitude of observable contexts, however, it should be noted that modern structures are also preferentially filled with modern material during the initial phase of their emergence. Since VNCs\is{verbonominal constructions} can draw on established argument structures, they are available in an earlier phase of the expanding nominal pool. Despite their continued increase throughout the following centuries, they are particularly prone to reject those nouns that emerge during this time. CPs\is{complex prepositions}, on the other hand, are primarily a fashion of the 19th century and reflect the strongly genre-specific\is{genre} characteristics of \isi{scientific texts}. They are therefore representatives of the delayed emancipation of \ili{German} academic language mentioned initially. However, this cannot be extended to VNCs\is{verbonominal constructions}, and consequently, not to deverbal nominalizations as a whole.


\section*{Acknowledgements}

The research reported in this paper stems from the research project “Zusammenspiel von Wortbildung und Syntax: Nominalisierungsstrategien in verbonominalen Konstruktionen und sekundären Präpositionen” (2021–2025) supported by the Swiss National Science Foundation, Grant number 197124. 


{\sloppy\printbibliography[heading=subbibliography,notkeyword=this]}
\newpage
\section*{Appendix}


\begin{table}
\begin{tabularx}{\textwidth}{lr X lr}
\lsptoprule
{noun} &  n && noun & n\\
\midrule
\textit{Verhältnis} ‘relation'  & 125  &&      \textit{Gefühl} ‘feeling'  & 6\\
\textit{Vergleich} ‘comparison'  & 81  &&      \textit{Berücksichtigung} ‘consideration'  & 3\\
\textit{Betreff} ‘regard'  & 49  &&            \textit{Herstellung} ‘production'  & 3\\
\textit{Entwicklung} ‘development'  & 36  &&   \textit{Abschluss} ‘conclusion'  & 2\\
\textit{Zusammenhang} ‘connection'  & 20  &&   \textit{Anknüpfung} ‘connection'  & 2\\
\textit{Rücksicht} ‘consideration'  & 19  &&   \textit{Entfaltung} ‘unfolding'  & 2\\
\textit{Hinsicht} ‘regard'  & 18  &&           \textit{Anwandlung} ‘mood'  & 2\\
\textit{Auffassung} ‘conception'  & 8  &&      \textit{Annahme} ‘assumption'  & 2\\
\textit{Einverständnis} ‘agreement'  & 8  &&   \textit{Anpassung} ‘adaption'  & 2\\
\textit{Rückblick} ‘regard'  & 6  &&           \textit{Abgrenzung} ‘limitation' & 2\\
\lspbottomrule
\end{tabularx}
\caption{NHG nouns by \textsc{genre} \textup{in a CA plot (CP)}, see \figref{fig:fleissner:8}}
\label{tab:fleissner:tofig8}
\end{table}

\begin{table}
\begin{tabularx}{\textwidth}{lr X lr}
\lsptoprule
{noun} &  n && noun & n\\
\midrule
\textit{Lauf} ‘course' & 253  &&           \textit{Übereinstimmung} ‘agreement' & 28\\
\textit{Bezug} ‘relation' & 235  &&        \textit{Dienst} ‘service' & 28\\
\textit{Folge} ‘consequence' & 200  &&     \textit{Bund} ‘tie' & 27\\
\textit{Fall} ‘case' & 94  &&              \textit{Besitz} ‘property' & 23\\
\textit{Richtung} ‘direction' & 87  &&     \textit{Berührung} ‘contact' & 22\\
\textit{Verbindung} ‘connection' & 79  &&  \textit{Verkehr} ‘traffic' & 21\\
\textit{Anfang} ‘begin' & 75  &&           \textit{Beginn} ‘begin' & 19\\
\textit{Verlauf} ‘course' & 57  &&         \textit{Ermangelung} ‘lack' & 15\\
\textit{Auftrag} ‘order' & 42  &&          \textit{Anlehnung} ‘backing' & 14\\
\textit{Entfernung} ‘distance' & 30  &&    \textit{Erinnerung} ‘memory' & 13\\
\lspbottomrule
\end{tabularx}
\caption{PROTOG, OHG, MHG and ENHG nouns by \textsc{genre} in a CA plot (CP), see \figref{fig:fleissner:9}}
\label{tab:fleissner:tofig9}
\end{table}

\begin{table}
\begin{tabularx}{\textwidth}{lr X lr}
\lsptoprule
{noun} &  n && noun & n\\
\midrule
\textit{Betrieb} ‘usage' & 277  &&           \textit{Verkehr} ‘traffic' & 23\\
\textit{Stand} ‘condition' & 149  &&         \textit{Entfernung} ‘distance' & 22\\
\textit{Verbindung} ‘connection' & 131  &&   \textit{Erinnerung} ‘memory' & 19\\
\textit{Anwendung} ‘usage' & 110  &&         \textit{Vergessenheit} ‘oblivion' & 19\\
\textit{Besitz} ‘property' & 87  &&          \textit{Verdacht} ‘suspicion' & 11\\
\textit{Beziehung} ‘relation' & 69  &&       \textit{Zustand} ‘state' & 10\\
\textit{Aussicht} ‘prospect' & 47  &&        \textit{Erwägung} ‘contemplation' & 9\\
\textit{Aufnahme} ‘uptake' & 40  &&          \textit{Übereinstimmung} ‘agreement' & 9\\
\textit{Einklang} ‘harmony' & 31  &&         \textit{Erregung} ‘excitement' & 8\\
\textit{Drehung} ‘turning' & 25  &&          \textit{Umlauf} ‘circulation' & 7\\
\textit{Stimmung} ‘mood' & 25  &&            \textit{Verwendung} ‘usage' & 7\\
\lspbottomrule
\end{tabularx}
  \caption{ENHG nouns by \textsc{genre} in a CA plot (VNC), see \figref{fig:fleissner:11}}
  \label{tab:fstofig11}
\end{table}

\begin{table}
\begin{tabularx}{\textwidth}{lr X lr}
\lsptoprule
{noun} &  n && noun & n\\
\midrule
\textit{Anspruch} ‘claim' & 222  &&              \textit{Ordnung} ‘order' & 56\\
\textit{Betracht} ‘consideration' & 203  &&      \textit{Handel} ‘trade' & 39\\
\textit{Lage} ‘situation' & 144  &&              \textit{Begriff} ‘concept' & 35\\
\textit{Bewegung} ‘motion' & 139  &&             \textit{Vorschlag} ‘suggestion' & 27\\
\textit{Gebrauch} ‘usage' & 115  &&              \textit{Spiel} ‘game' & 24\\
\textit{Frage} ‘question' & 98  &&               \textit{Stich} ‘stab' & 23\\
\textit{Dienst} ‘service' & 76  &&               \textit{Ruhe} ‘calm' & 23\\
\textit{Gang} ‘motion' & 69  &&                  \textit{Verlegenheit} ‘embarrassment' & 23\\
\textit{Leben} ‘life' & 67 &&                     \textit{Erfahrung} ‘experience' & 18\\
\textit{Berührung} ‘touch' & 65  &&              \textit{Richtung} ‘direction' & 17\\
\lspbottomrule
\end{tabularx}
\caption{PROTOG, OHG, MHG and NHG nouns by \textsc{genre} \textup{in a CA plot (VNC)}, see \figref{fig:fleissner:12}}
 \label{tab:fleissner:tofig12}
\end{table}

\begin{table}[H]
\begin{tabularx}{\textwidth}{lr X lr}
\lsptoprule
{noun} &  n && noun & n\\
\midrule
\textit{Leben} ‘life' & 2146  &&               \textit{Zustand} ‘state' & 860\\
\textit{Gesetz} ‘law' & 1404  &&               \textit{Regierung} ‘government' & 808\\
\textit{Frage} ‘question' & 1350  &&           \textit{Begriff} ‘concept' & 802\\
\textit{Fall} ‘case' & 1282  &&                \textit{Bewegung} ‘motion' & 796\\
\textit{Verhältnis} ‘relation' & 1230  &&      \textit{Richtung} ‘direction' & 788\\
\textit{Versuch} ‘attempt' & 1164  &&          \textit{Erfindung} ‘invention' & 753\\
\textit{Entwicklung} ‘development' & 944  &&   \textit{Beziehung} ‘relation' & 732\\
\textit{Wirkung} ‘effect' & 927  &&            \textit{Vorstellung} ‘concept' & 721\\
\textit{Verfahren} ‘procedure' & 906  &&       \textit{Bedeutung} ‘meaning' & 706\\
\textit{Gedanke} ‘thought' & 869  &&           \textit{Vertrag} ‘contract' & 655\\
\lspbottomrule
\end{tabularx}
 \caption{Most frequent nouns by \textsc{genre} \textup{in a CA plot (}\textsc{ZERO}), see \figref{fig:fleissner:15}}
 \label{tab:fleissner:tofig17}
\end{table}


\cleardoublepage
\end{document}
