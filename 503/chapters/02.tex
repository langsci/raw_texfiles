\documentclass[output=paper,colorlinks,citecolor=brown]{langscibook}
\ChapterDOI{10.5281/zenodo.15689121}

\author{Sonja Zeman\orcid{0000-0002-5923-670X}\affiliation{Universität Augsburg}}
\title{The impact of narration -- A diachronic view on German} 
\shorttitlerunninghead{The impact of narration} 
\abstract{With respect to the question of how grammatical patterns correlate with changes in discourse structure, the paper investigates the influence of narration on the usage of grammatical means in older stages of German and their diachronic development. The first part of the paper provides an overview on previous literature on the different dimensions of narration and their interaction with different grammatical means. Against this background, two empirical case studies show that these different dimensions of narration interact differently with the distribution and development of grammatical elements. Narration has thus a relevant impact and is an important aspect to consider when describing the diachronic relation between grammar and discourse structure.}

\begin{document}
\maketitle 
\shorttitlerunninghead{The impact of narration}%%use this for an abridged title in the page headers

\section{Narration and discourse structure} \label{sec:zeman:1}

It is a general consensus that the use of grammatical forms is strongly influenced by discourse structure. It is therefore undisputed that grammatical elements should not be investigated in isolation but rather in their respective communicative contexts. This naturally also applies to the study of grammar in historical stages of language (see \citealt{Ziegler2010} for a plea for a historical discourse grammar that goes beyond the sentence level). In the historical linguistics of \ili{German}, it has been shown by numerous studies that the use of grammatical means depends on various contextual factors such as text types (e.g. \citealt{Gloning2010}), the oral vs. written distinction \citep{Elspaß2005,ÁgelHennig2006}, the prose vs. verse distinction (e.g. \citealt{Prell2005}, \citealt{Speyer2013}), information structure (e.g. \citealt{PetrovaSolf2005}, \citealt{Speyer2013}, \citealt{Demske2018}), and regional variation (e.g. \citealt{Elspaß2010}, \citealt{Werth2020}). In addition, the relationship between grammatical means and discourse structure has been studied across languages, particularly in historical \isi{register} analysis (see for an overview \citealt{Kytö2019} and \citealt{GoulartEtAl2020}).

Recent investigations have emphasised that text and discourse parameters are also key factors in \isi{language change} (see e.g. \citealt{Hennig2009} for \ili{German}; \citealt{BiberGray2013}, \citealt{BiberEtAl2021}, for \ili{English}). Consequently, it has been claimed that registers\is{register} are a mediating factor and “act as powerful vehicles promoting or retarding \isi{language change}, or even contributing to stability” \citep[155]{Kytö2019}. 

One specific aspect of discourse that needs to be considered in terms of its influence on grammatical devices and their diachronic development is \isi{narration}. There are two reasons for this. First, many of the texts that have survived from the earlier stages are narrative texts. Second, many texts contain narrative passages. Narration is therefore an important parameter in many studies and corpus annotations. In Biber's multimodal discourse analysis, for example, the narrative vs. non-narrative distinction is one of the listed parameters \citep[e.g.][]{BiberConrad2019}. “Narrative” is (next to expository, argumentative, and instructive) also one of the categories for defining the prototypical text classes in the Helsinki Corpus of \ili{English} texts \citep[138]{Kytö2019}. In addition, various studies have shown that the distribution of grammatical features varies within the intra-textual registers\is{register} of \isi{dialogue} and \isi{narration} (see, for example, \citealt{Padučeva2011}, \citealt{MazziottaGlikman2019}, \citealt{EgbertMahlberg2020}, and \citealt[2]{Larrivée2022} for an overview with further references), and it has been discussed whether \isi{dialogue} or \isi{narration} are more often the initiating context for innovation in \isi{language change} (see \citealt{BrownTagliamonte2012}, \citealt{MazziottaGlikman2019}). 

Although the influence of \isi{narration} has been proven across languages and for quite specific aspects of \isi{narration} (e.g. on intensifiers in storytelling in \citealt{BrownTagliamonte2012}), we still do not know exactly which aspects are responsible for the different distribution of grammatical elements in narrative vs. \isi{dialogue}, and the influence of \isi{narration} on \isi{language change} remains vague (see also \citealt[2]{Larrivée2022}). 

This applies particularly to \ili{German} linguistics, in which \isi{narration} itself has not often been a topic in its own right. Rather, the focus of diachronic discourse studies has been on the development from \isi{orality} to writing and the influence of different text types and communicative situations (see, for an example, the various papers in \citet{Ziegler2010}'s volume on historical text grammar in \ili{German}). 

The aim of this article is to take another step towards a systematic investigation of the influence of \isi{narration} in the history of \ili{German} and in \isi{language change} from both a theoretical and empirical point of view by (i) clarifying the theoretical concept of \isi{narration} and its interaction with discourse structure, \isi{orality}, and grammar, and (ii) presenting two exemplary case studies on the development of the \isi{pluperfect}\is{tense} and left dislocations\is{left dislocation}.

In this respect, \sectref{sec:zeman:2} provides the theoretical background focusing on the interaction between \isi{narration}, discourse structure, and grammar. This is based on the assumption that three dimensions of \isi{narration} have to be distinguished, namely (i) narrative \textit{texts} that present a story by following a narrative macrostructure (in the sense of \citealt{LabovWaletzky1967}) and prototypically integrate the viewpoints of different characters\is{character} and narrators\is{narrator} \citep{Zeman2020}, (ii) narrative \textit{discourse mode}, i.e. narrative segments that can occur in (narrative and non-narrative) texts and are characterised by a specific pattern of grammatical use \citep{Smith2003}, and narrative \textit{discourse relation} that holds between two utterances in discourse (in the sense of \citealt{AsherLascarides2003}). Against this background, two empirical case studies will show that these three dimensions of \isi{narration} have an impact on the distribution and grammaticalisation\is{grammaticalization} of linguistic elements (\sectref{sec:zeman:3}). The case studies address left dislocations\is{left dislocation} in \ili{Middle High German} (\sectref{sec:zeman:3.1}) and the grammaticalisation\is{grammaticalization} of the \isi{pluperfect}\is{tense} (\sectref{sec:zeman:3.2}). In sum (\sectref{sec:zeman:4}), the paper argues that the different dimensions of \isi{narration} interact differently with the distribution and development of grammatical elements. Narration is therefore an important aspect to consider when describing grammatical elements in their context in the discourse structure and should be recognised as an essential factor in \isi{language change}.


\section{Dimensions of narration} \label{sec:zeman:2}
\subsection{Macrostructure of narration}\label{sec:zeman:2.1}

To address the question of which properties of \isi{narration} influence the distribution and development of grammatical elements, it is first necessary to specify what a \isi{narration} is and how it can affect the use of grammatical devices. In this respect, \isi{narration} has often been seen as a property of narrative \textit{texts}. Fairytales such as “Little Red Riding Hood” or novels are seen as prototypical examples of narrative texts, as they tell a story. Although there is no generally accepted definition of \isi{narration} in either narratology or linguistics (see \citealt{Zeman2018} for an overview), there is broad agreement in linguistics that narrative texts are characterised by sequences of specific events in temporal and causal order and follow an underlying macrostructure in the sense of \citet{LabovWaletzky1967}, which contains a setting, a complication and a resolution as main components. According to this definition, many text sources from older stages of language are prototypical narrative texts, such as \ili{Middle High German} epic poems (e.g. \textit{Nibelungenlied}, \textit{Tristan}, \textit{Parzival}) and \ili{Early New High German} novels (e.g. \textit{Fortunatus}, \textit{Melusine}). 

This is an important fact as the specific discourse structure of \isi{narration} can significantly influence both the distribution and the development of grammatical elements. This can be seen, for example, in the discourse organisation of a narrative text through scene shifts,\is{scene shift} as shown for \ili{Old English}. \citet{Fludernik2003} has shown that in the history of \ili{English} there has been a development of “chapterification” and an “attendant emphasis on the chapter beginning as a salient point for temporal or locative shifts” \citep[337]{Fludernik2003}, as it is common in the 19th century novel. In contrast, episodic narratives in \ili{Old English} were organised by means of macroepisodic markers, that is, scene settings were merely indicated by the change of protagonists or the next stage of the hero's adventure \citep[337]{Fludernik2003}. This development had a considerable influence on the use of metanarrative and metafictional comments and formulae. 

For the history of \ili{German}, systematic investigations on scene shift\is{scene shift} from a linguistic perspective are missing. However, it can be assumed that there has been a similar development, as the increase in explicit and conventionalised \isi{scene shift} markers (e.g. “Now let us leave X and Y (in location A) and turn to O and P” \citep[334]{Fludernik2003}) seems to correlate with the emergence of stories with multiple subplots such as the \textit{Prosalancelot} from Late \ili{Middle High German} onwards (see \citealt[699f.]{Zeman2023b}). In contrast, the discourse organisation of earlier \ili{Middle High German} verse epics is characterised by the frequent use of the particle \textit{dô} (‘then'), which indicates the next step in the \isi{sequence of events}. In this regard, the organisation of the narrative structure is not very explicit or elaborate but comparable to patterns that are characteristic of everyday oral storytelling (e.g. “and then she said, and then I said [...]”).

In addition to such discourse-organizing patterns that are specific to the narrative macrostructure, another important difference between narrative and non-narrative \textit{texts} is that they are not objective representations of a \isi{sequence of events} but present a story from a particular point of view. The narrator's\is{narrator} perspective is therefore a crucial aspect that distinguishes narrative from non-narrative texts with corresponding consequences for the discourse structure (see \citealt{Zeman2018}). Since utterances in narrative discourse can contain both the narrator's\is{narrator} and the protagonists' perspectives, it allows for complex viewpoint constructions such as Free Indirect Discourse (\textit{Tomorrow was Monday}.) and the Future-of-Fate construction (\textit{He was never to see her again.}), which for this reason are restricted to narrative discourse only (see in detail \citealt{Zeman2018}). 

Both Free Indirect Discourse and Future-of-Fate are specific constructions of narrative discourse and do not occur in all stages of language. There are, however, other more common grammatical elements that interact with the narrator's\is{narrator} \isi{voice}. One example is interjections.\is{interjection} \ili{Middle High German} verse epics are characterised by the fact that interjections\is{interjection} are not only used in the characters'\is{character} dialogues\is{dialogue} but are also frequently linked to the narrator\is{narrator} who is “an active participant in the act of \isi{narration}” \citep[338]{Fludernik2003}. In this regard, the epics reflect the fact that storytellers in medieval times recited the stories in front of an audience. In this shared communicative situation between narrator\is{narrator} and audience, the narrator\is{narrator} addresses the audience and comments on the story. The story told in the distant world is often presented as if the audience could see the events unfolding before their eyes. Consequently, medieval storytelling is characterised by a combination of deictic markers\is{deictic markers} of distance and proximity. In addition to the use of past tenses\is{past tense}\is{tense} that mark the distance of the story world, the narrator's\is{narrator} discourse is often characterised by deictic adverbials\is{deictic markers} that indicate the “here” and “now” of the story, verbs of perception and expressive interjections\is{interjection} that “intrude” into the plot level (e.g. “alas how many chainmails the bold Dancwart broke there!”; \textit{Nibelungenlied} B, 212.4). 

Such a conflation of proximity and distance is seen as a general feature of oral storytelling. With the transition from metrically bound verse epics in \ili{Middle High German} to the increase of written stories and the development of prose novels with multiple narrative strands (such as the \textit{Prosalancelot}), the role of the narrator\is{narrator} and the deictic conceptualisation of place and time have changed (see for an overview \citealt{Zeman2023b}). The changes in narrative macrostructure thus correlate with a different use of grammatical devices such as interjections,\is{interjection} personal pronouns, exclamative sentences.

In conclusion, narrative macrostructure is an important factor to consider when investigating the interaction between discourse structure and grammatical means in the history of \ili{German}. However, it is not the only dimension of \isi{narration} that has an impact on the distribution and development of grammatical means. In addition to the macrostructure of narrative \textit{texts}, it is also important to distinguish between two narrative dimensions \textit{within} narrative texts, i.e. narrative \textit{discourse modes}\is{discourse mode} and narrative \textit{discourse relations}.\is{discourse relation} 


\subsection{Microstructure of narration: Discourse mode}\label{sec:zeman:2.2}
As explained before, \isi{narration} can refer to the macrostructure of discourse and, as such, to whole texts. A text telling the story of Little Red Riding Hood would thus be considered “narrative”. In the case of an excerpt such as \REF{ex:zeman:1}, the whole excerpt would be narrative because it belongs to a narrative text. In the tradition of distinguishing between discourse modes,\is{discourse mode} however, narrative does not refer to entire texts, but to \textit{segments} within a – narrative or non-narrative – text. In \REF{ex:zeman:1}, only those utterances would count as narrative that refer to the events of the plot (marked in bold) as opposed to the dialogic passages that represent the speech and thought of the protagonists.

\ea%1 
\label{ex:zeman:1}
{\itshape \textbf{But} \textbf{the} \textbf{wolf} \textbf{ran} \textbf{straight} \textbf{to} \textbf{the} \textbf{grandmother's} \textbf{house} \textbf{and} \textbf{knocked} \textbf{on} \textbf{the} \textbf{door.} “Who's there'' – “Little Red Riding Hood. I'm bringing you some cake and wine. Open the door for me." – “Just press the latch," \textbf{called} \textbf{out} \textbf{the} \textbf{grandmother.} “I'm too weak to get up." \textbf{The} \textbf{wolf} \textbf{pressed} \textbf{the} \textbf{latch,} \textbf{and} \textbf{the} \textbf{door} \textbf{opened.} \textbf{He} \textbf{stepped} \textbf{inside,} \textbf{went} \textbf{straight} \textbf{to} \textbf{the} \textbf{grandmother's} \textbf{bed,} \textbf{and} \textbf{ate} \textbf{her} \textbf{up.}}\\
\hfill (\url{https://www.grimmstories.com/language.php?grimm=026\& l=en\& r=de}; \\
\hfill access 01-11-2023)\\
\z
 
Regarding the interaction between grammar and discourse, the distinction between narrative and dialogic segments is crucial, as it is associated with different behaviours of grammatical devices such as pronouns, aspect, and \isi{tense} (see e.g. \citealt{MazziottaGlikman2019}, \citealt{EgbertMahlberg2020}, and \citealt[2]{Larrivée2022} with further references). The grammatical differences are particularly evident in the textual distribution of tenses.\is{tense} \citet[264]{benviste1972}, for example, has shown that the French tenses\is{tense} are distributed differently in the two text modes “histoire” vs. “discourse”. In the “histoire” only the aorist, imperfect, pluperfect,\is{pluperfect}\is{tense} and prospective tenses\is{tense} are used, whereas the prototypical tenses\is{tense} of “discours” – i.e. present,\is{present tense}\is{tense} \isi{present perfect},\is{tense} and future – are excluded. Benveniste sees this distribution as an argument for the fact that the two levels of enunciation
“discours” and “histoire” are “two different and complementary systems” \citep[266]{benviste1972}. Similarly, \citet{Weinrich1964} and \citet{Hamburger1957} assume a complementary distribution of tenses\is{tense} in narrative and non-narrative contexts.

To date, the close interdependence between verbal categories and narrative context has been confirmed in various studies (e.g. \citealt{Dahl1985}, \citealt{Fleischman1990}, \citealt{Smith2003}, \citealt{Zeman2010}, \citealt{Padučeva2011}, \citealt{Fischer2020}), albeit sometimes using different terminology (cf. for an overview \citealt[461--466]{Zeman2020}). As a result, narrative discourse is characterised by “a special subset of rules, conditions and restrictions” (van \citealt[289]{Dijk2015}) and by “its own special features” \citep[9]{Padučeva2011}. According to \citet{Smith2003}, the narrative \isi{discourse mode} has the status of a “covert” category, i.e. a category characterised by a specific pattern in terms of the distribution of grammatical devices. Its basic structure is characterised by the fact that narrative events are anaphorically related to each other and are not directly related to speech time but oriented to a displaced reference point \citep[13]{Smith2003}.

Regarding the discourse structure in the historical stages of \ili{German}, the differences between narrative and non-narrative discourse modes\is{discourse mode} are also evident in the distribution of \isi{tenses}. In Old\il{Old High German} and \ili{Middle High German}, the \isi{present perfect}\is{tense} cannot denote events in the narrative past \citep{Zeman2010} and is therefore primarily used in dialogic passages. This preference is also reflected in the grammaticalisation\is{grammaticalization} process (see \citealt{Fischer2020}). As can be seen for other examples of perfect expansion in the Romance languages \citep[9]{Drinka2017}, the \isi{present perfect}\is{tense} increasingly occurs in non-narrative texts, while the main domain of the past tense\is{past tense}\is{tense} remains the narrative \isi{discourse mode}.

In addition to tense,\is{tense} other grammatical devices are also sensitive to the distinction between narrative vs. non-narrative. This can be seen in a comparison between text samples from the \textit{Nibelungenlied} and \textit{Tristan}. \tabref{tab:zeman:1} shows the results of a study comparing two different historical texts based on the method by \citet{ÁgelHennig2006}. On the basis of a sample of 2000 words, different linguistic features are annotated, which are considered to be markers of the language of proximity and parameters of syntactic complexity (sentence length, relationship between main clauses and subordinate clauses). In addition, a distinction is made between narrative and dialogical passages. All statements within \isi{direct speech} (but without any embedded narrative passages) as well as narrator\is{narrator} comments in the text, which are oriented towards the time of speaking, were considered “dialogical”. All passages that represent the story of the narrated world were considered narrative.

\begin{table}
 \caption{Comparison of selected features between \textit{Nibelungenlied} B and \textit{Tristan} \citep[188]{Zeman2022}}
\label{tab:zeman:1}
\begin{tabularx}{\textwidth}{L{4.3cm}rrrrr}
\lsptoprule
Parameter & \multicolumn{2}{c}{Nibelungenlied B} & \multicolumn{2}{c}{{Tristan}} \\
(Ágel/Hennig 2006) & \multicolumn{1}{X}{narrative} & \multicolumn{1}{X}{dialogical} & \multicolumn{1}{X}{narrative} & \multicolumn{1}{X}{dialogical} \\
\midrule
\textsc{microstructure} &&&&\\
code (interjections, emotive) & 4 & 4& 2 & 58\\
role (imperative, vocative) & 0 & 52 & 0 & 55\\
situation (personal deixis; & 5 & 268&8 & 264 \\
temporal \& local deixis) & 4 & 24 & 9 & 24\\
time (left-placement) &17 & 4 & 11 & 15\\
\midrule
\textsc{macrostructure} &&&&\\
matrix sentences & 180 & 200 & 140 & 163\\
subordinate sentences & 65 & 97 & 117 & 125\\
length on average & 11.1& 10 & 14.3&12.3\\
(in words) & & & & \\
\midrule
(Microstructural) features in total & 30 & 352 & 30 & 416\\
words in total & 2000 & 2000 & 2003 & 2004 \\
& 1.5\% & 17.6\% & 1.5\% & 20.8\% \\
\lspbottomrule
\end{tabularx}
\end{table}

\tabref{tab:zeman:1} shows a significant difference between narrative and dialogical passages. This is particularly true for address-oriented features such as interjections,\is{interjection} imperatives, vocatives and deictic elements\is{deictic markers} which occur significantly more often in dialogical passages than in narrative passages. Due to the sensitivity of several features to the narrative vs. non-narrative distinction, discourse modes\is{discourse mode} must be taken into account with regard to the distribution and development of grammatical features in the history of \ili{German}.



\subsection{Microstructure: Discourse relations} \label{sec:zeman:2.3}
In addition to narrative macrostructure and narrative \isi{discourse mode}, there is a third dimension of \isi{narration}, namely the dimension of discourse \textit{relations}. Discourse relations\is{discourse relation} are semantic connections that link two utterances within a text (cf. \citealt{Hobbs1979}, \citealt{AsherLascarides2003}). For example, a sentence may be an explanation or an elaboration of the previous utterance. Since in a coherent discourse each utterance be semantically and structurally connected in some way to the preceding and the following utterances, it can be assumed that the entire discourse is structured by underlying discourse relations\is{discourse relation} that either remain implicit or are explicitly marked by connectors such as conjunctions or conjunctional adverbs\is{adverb}.

According to \citet{AsherLascarides2003}, a narrative \isi{discourse relation} links two events by a temporal sequence that corresponds to their natural order. Narrative connections can be overtly marked by instances of \textit{and then} \citep[200]{AsherLascarides2003}. Narrative texts and discourse modes\is{discourse mode} often contain narrative connections, as indicated by [then] in example \REF{ex:zeman:2}. However, not every connection between two propositions in a narrative text or in narrative \isi{discourse mode} is necessarily a narrative connection. The first sentence in \REF{ex:zeman:2}, for example, stands in a contrastive relation to the preceding text segment, as indicated by the connector \textit{but}. 

\ea%2
\label{ex:zeman:2}
{\itshape \textbf{But} the wolf \textbf{[in} \textbf{contrast} \textbf{to} \textbf{Little} \textbf{Red} \textbf{Riding} \textbf{Hood]} ran straight to the grandmother's house and \textbf{[then]} knocked on the door. “Who's there?" - “Little Red Riding Hood. I'm bringing you some cake and wine. Open the door for me." - “Just press the latch," called \textbf{[then]} out the grandmother. “I'm too weak to get up." The wolf pressed the latch \textbf{[then]}, and \textbf{[then]} the door opened. He \textbf{[then]} stepped inside, \textbf{[then]} went straight to the grandmother's bed, and \textbf{[then]} ate her up.}\\
\hfill (\url{https://www.grimmstories.com/language.php?grimm=026\& l=en\& r=de}; \\
\hfill access 05-11-2023)\\
\z

In terms of the interplay between grammar and discourse structure, discourse relations\is{discourse relation} have also been relevant in the history of \ili{German}. A recent study by \citet{Speyer2022}, for example, investigates the dependency between syntactic serialisation and the semantic connections between propositions. It shows that the ranking of informational constituents in the prefield differs depending on the \isi{discourse relation} \citep[225]{Speyer2022}. Furthermore, it is shown that this ranking is affected by \isi{language change}. In \ili{Early New High German}, the preferences of the information structure in the prefield differ from those in Modern High \ili{German} \citep[227]{Speyer2022}. Studies of discourse relations\is{discourse relation} therefore promise to shed more light on the diachronic development of connective markers and word serialisation.


\section{Case studies}\label{sec:zeman:3}

Against the background of Section 2, it becomes evident that all three different dimensions of \isi{narration} – (i) narrative \textit{texts} that exhibit a narrative macrostructure, (ii) narrative \textit{discourse mode}, i.e. narrative segments that can occur in (narrative and non-narrative) texts, and (iii) narrative \textit{discourse relations}\is{discourse relation} between propositions – can have effects on linguistic changes. The distinction is therefore crucial to systematically trace the development of grammatical means in relation to discourse structure. This is illustrated in the following by two case studies that will take a closer look at the effects of \isi{narration} on the distribution and development of left dislocations\is{left dislocation} and the \isi{pluperfect}\is{tense} in the history of \ili{German}.


\subsection{Left dislocations (LDs)}\label{sec:zeman:3.1}
“Left dislocations” (LDs),\is{left dislocation} such as for example \textit{My aunt, she used to play the guitar}, are traditionally described as constructions in which a syntactic constituent is “displaced” to the front (i.e. “to the left”) of a sentence and is subsequently taken up by an anaphoric pronoun. As such, the constituent is seen as located “outside” of the actual sentence. Consequently, LDs\is{left dislocation} have a highlighting effect because they divide the sentence into a two-part structure (cf. \citealt{Bally1932}), with the first part containing the referencing element, the second part containing the statement about the respective referenced object.

The two-part structures of LDs\is{left dislocation} are considered to be typical features of spoken language cross-linguistically \citep[67]{Chafe1994}. It has been claimed that the division of a sentence into reference and statement provides a cognitive processing advantage, because it breaks the sentence down into smaller information or processing units, thus facilitating production and comprehension in spoken language (cf. \citealt[39]{Lötscher1994}, \citealt[§ 2015]{Fiehler2009}). They are therefore also seen as evidence of a more “oral syntax”\is{oral syntax} in historical stages of language, i.e. a syntax that is structured in information units and characterised by a stronger “aggregative” (as opposed to “integrative”) organisation (e.g. \citealt[48]{Lötscher1994}, \citealt{ÁgelHennig2006}). One might therefore assume that LDs\is{left dislocation} are mainly used in dialogic passages. However, this is not the case. As \tabref{tab:zeman:1} above shows, the LDs\is{left dislocation} in the \textit{Nibelungenlied} show a preference for narrative passages. This leads to the questions of whether syntactic structures depend on the difference between discourse modes\is{discourse mode} and whether LDs\is{left dislocation} display different functions in narrative vs. non-narrative passages – and how the interaction between syntactic structure and \isi{discourse mode} affects the development of LDs\is{left dislocation} in \ili{German}. 


\subsubsection{LD in the history of German}\label{sec:zeman:3.1.1}
To pursue these questions, it is first necessary to take a closer look at LDs\is{left dislocation} in the history of \ili{German}. With regard to their general diachronic development, \citet[42]{Lötscher1994} has shown that LDs\is{left dislocation} were particularly common in the 13th and 14th centuries, but that their frequency decreased significantly in the period from 1200 to 1600. This development is usually seen in parallel with the spread of literacy. It is assumed that as literacy increases, the characteristics of spoken language in written texts diminish and the linguistic norm becomes more determined by the laws of writing. This assumption is supported by the fact that LDs\is{left dislocation} occur particularly frequently in texts that are close to oral language\is{oral language} \citep[48]{Lötscher1994}.

With regard to the use of LDs\is{left dislocation} in \ili{Middle High German}, two distinctions have to be made. First, it should be noted that LDs\is{left dislocation} are not a homogeneous group but can occur in different constellations, cf. \REF{ex:zeman:ld}. 

\ea \label{ex:zeman:ld}
\ea \label{ex:zeman:lda}
\textit{Den Schatz, den kennt jetzt niemand.}\\
\glt `The treasure that knows now nobody'
\ex \label{ex:zeman:ldb}
\textit{Der Schatz – den / ihn kennt niemand / niemand kennt ihn.}\\
\glt `The treasure – that / it knows nobody /  nobody knows it'
\z
\z

In \REF{ex:zeman:lda} the \isi{resumptive pronoun} is a \textit{d}{}-pronoun, and pronoun and NP are coreferential and case-congruent. In \REF{ex:zeman:ldb}, on the other hand, there is no case congruence, the \isi{resumptive pronoun} does not necessarily occur at the beginning of the sentence, and the NP can be resumed by either a \textit{d}- or \textit{p}-pronoun. In addition, there is often a pause in speech between the preceding NP and the referential statement. In cases like \REF{ex:zeman:ldb}, the first constituent forms a syntactically more independent unit and is less integrated\is{integration} into the sentence than in \REF{ex:zeman:lda}. 

These two syntactic variants are also associated with different discourse-pragmatic functions (cf. \citealt{Cinque1977}; \citealt{Selting1993}; \citealt{Averintseva2007}; \citealt{Dewald2012}). Although both \REF{ex:zeman:lda} and \REF{ex:zeman:ldb} are used to mark topical referents, they differ in the way they mark them. While LDs\is{left dislocation} like \REF{ex:zeman:lda} are locally linked to the immediately preceding context, LDs\is{left dislocation} such as \REF{ex:zeman:ldb} are assumed to be a global resumption strategy by introducing a topic for the entire following discourse section \citep[309]{Selting1993}.

There are different terms for these two variants used in the literature. In the \ili{German}-language literature, a distinction is usually made between “Linksversetzung” (‘left dislocation')\is{left dislocation} in the narrow sense for examples such as \REF{ex:zeman:lda} and “Freies Thema” (‘free topic') for \REF{ex:zeman:ldb} \citep{Altmann1981}. The latter have also been called ‘hanging topics'. \citet{Frey2004}, for example, distinguishes between “\ili{German} left dislocation”\is{left dislocation} and “hanging topic left dislocation”\is{left dislocation} \citep{Frey2004}. Older publications have used terms like “nominative pendens” (\citealt{Behaghel1932}, \citealt{Zäch1931}, \citealt{PaulEtAl2007}) and “isolated nominative” (\citealt[§ S 56]{PaulEtAl2007}).

For historical stages of the language, it is difficult to distinguish clearly between the two variants of LDs\is{left dislocation}, as one of the key criteria for evaluating the degree of integration is the phonetic level. In the works of \citet{Selting1993} and \citet[§ 2015]{Fiehler2009}, intonation and prosody are identified as the most reliable criteria for distinguishing whether the referential expression and the subsequent statement are integrated\is{integration} in a common structure or represent two independent phrases. With regard to the written \ili{Middle High German} data, a distinction can only be made if the syntactic features such as the absence of case congruence, distance position, type of pronoun deviate from the common pattern set out in \REF{ex:zeman:lda}. Consequently, all occurrences of LDs\is{left dislocation} have been analysed with respect to these parameters within the case study. This procedure has shown that only a few instances deviate from pattern \REF{ex:zeman:lda}. This result corroborates \citeauthor{Lötscher1994}'s (\citeyear{Lötscher1994}) observation that ‘free topics' do not play a significant role in \ili{Middle High German} literature. Consequently, the subsequent analysis will focus on LDs\is{left dislocation} in the narrow sense (i.e. the pattern in \ref{ex:zeman:lda}).

A second distinction that must be considered is the difference between “simple” and “complex” LDs\is{left dislocation}, as illustrated in examples \REF{ex:zeman:3a} and \REF{ex:zeman:3b}. In comparison to example \REF{ex:zeman:3a}, where a simple NP (\textit{den hort}) is resumed by a \textit{d}-pronoun right after the referential object, example \REF{ex:zeman:3b} is more complex since the referential object in the prefield is further attributed by relative clauses.\is{relative clause}

\ea \label{ex:zeman:3s}
\ea \label{ex:zeman:3a} 
“simple” LDs\\
\gll den hort den weiz nu niemen wan got unde mîn\\
the treasure that knows now nobody except god and myself\\
\glt \hfill [\textit{Nibelungenlied} C, 2431, 3--4]

\ex \label{ex:zeman:3b}
“complex” LDs\\
\gll Strît den {aller hôhsten} der {inder dâ geschach}, {ze jungest und zem êrsten,} den ieman gesach den tet vil willeclîche {diu Sîfride hant}\\
 fight the biggest that {happened there to them} {last and first,} that somebody saw \textbf{that} did very intentionally {the hand of Siegfried}\\
\glt \hfill [\textit{Nibelungenlied} C, 237, 1–3]
\z 
\z

This distinction is of particular importance in the context of \ili{Middle High German}, given that the \isi{resumptive pronoun} is obligatory for complex LDs\is{left dislocation} in the 13th and 14th century. This implies that in the \textit{Nibelungenlied}, complex NPs are always resumed by a pronoun and thus obligatorily display left dislocation\is{left dislocation}. In contrast, the LDs\is{left dislocation} of simple NPs as exemplified in \REF{ex:zeman:3s}, are never obligatory \citep[44]{Lötscher1994}. The following analysis will therefore be restricted to the examination of ‘simple' LDs\is{left dislocation}. Only in these cases will it be possible to examine the contexts in which they occur and their functions in comparison to non-LD constructions.


\subsubsection{LDs in narrative and non-narrative contexts} \label{sec:zeman:3.1.2}
To clarify whether the preference for narrative passages exhibited by the LDs\is{left dislocation} in the \textit{Nibelungenlied} (as evidenced in \tabref{tab:zeman:1}) can be regarded as representative of the entire text or whether it is merely a random occurrence in the sample, all LD\is{left dislocation} occurrences of simple noun phrases in the \textit{Nibelungenlied} were analysed. To facilitate a comparison, an equal number of LDs\is{left dislocation} were analysed (verses 1 to 5469) for the \textit{Tristan} (i.e. 28\% of the entire text of 19545 verses). As “dialogical” I consider all passages in \isi{direct speech} (without embedded narrations)\is{narration} and the narrator's\is{narrator} speech, which are both oriented to speech time. In contrast, “narrative” will apply to text passages that represent the story in the narrated world. Since the distribution of LDs\is{left dislocation} in the \isi{dialogue} and narrative passages can only be compared if the proportion of \isi{dialogue} passages in the epic is taken into account, the speech density (i.e. the proportion of \isi{direct speech} in relation to the entire text) was also determined for both texts,\footnote{The speech density was counted automatically. Therefore, the number of narrator's comments and embedded narrations could not be tracked for the entire text. Since narrator comments and embedded narrations\is{narration} are very rare for the Nibelungenlied, this should not have an impact on the ratios.} see \tabref{tab:zeman:2}. 

\begin{table}
\caption{Comparison of LDs in \textit{Nibelungenlied} C and \textit{Tristan} \citep[77]{Zeman2023a}}
\label{tab:zeman:2}
\begin{tabularx}{\textwidth}{l rrrr Yrrr}
\lsptoprule
\multicolumn{5}{c}{{Nibelungenlied C}} & \multicolumn{4}{c}{{Tristan} (verses 1--5469)} \\ 
 & \multicolumn{1}{l}{narr} & \multicolumn{1}{l}{dia} & \multicolumn{1}{l}{narrator} & \multicolumn{1}{l}{sum} & \multicolumn{1}{l}{narr} & \multicolumn{1}{l}{dia}& \multicolumn{1}{l}{narrator}& \multicolumn{1}{l}{sum} \\
 \midrule
LD & 136 & 33& 1& 170 & 120 & 37 & 13 & 170 \\
 & 80\% & 19.4\% & 0.6\% & & 70.6\% & 21.8\% & 7.6\% & \\
\midrule
total & 60.2\% & 39.8\% & & & 71.9\% & 28.1\% & & \\
\lspbottomrule
\end{tabularx}
\end{table}

If LDs\is{left dislocation} were evenly distributed in both discourse modes,\is{discourse mode} one would expect that the percentage in the dialogic passages would correspond to the percentage of \isi{direct speech} in the whole text. In the \textit{Nibelungenlied}, however, the frequency of LDs\is{left dislocation} is higher in the narrative passages than in the dialogic passages. These quantitative findings correspond to qualitative differences.

In the narrative \isi{discourse mode}, LDs\is{left dislocation} usually serve to actualise or re-actualise a protagonist as a prominent agent in the discourse. It is particularly often used to select a referent from an established set of protagonists, cf. \REF{ex:zeman:3}.

\ea \label{ex:zeman:3} 
\gll Der künic von Tenemarke der sprach sâ zestunt. \\
The king of Denmark, the spoke there suddenly\\
\glt \hfill [\textit{Nibelungenlied} \textit{C}, 300, 1]\\
\z 

In the previous context of \REF{ex:zeman:3}, the first meeting between the hero Siegfried and his future wife Kriemhild, princess of Burgundy, is recounted. The king of Denmark is not mentioned throughout the chapter (Aventiure 5). However, it is mentioned in the immediately preceding section (\textit{Nibelungenlied} \textit{C}, 299) that guests from all over the world attend the court's celebration of its victory over Denmark and the Saxons. Although the king of Denmark is not mentioned since verse 238,2, he is part of the group of guests and, as such, an (implicitly) available discourse referent.

This functional pattern is well known for the use of LDs\is{left dislocation} in spoken language. However, it is also characteristic of oral storytelling. According to \citet[67]{Chafe1994}, oral narratives are characterised by the fact that the intonation unit often verbalises only a single isolated referent who is (re-)actualised or (re-)focused as the protagonist. The importance of this function in the \textit{Nibelungenlied} is reflected in the high proportion of animated (70.5\%) and definite subjects (93.3\%) and the frequent use of proper names (61.2\%). When unanimated objects are displaced, they are often objects of particular relevance\is{informative relevance} to the plot, such as the stone that Brünhild throws in competition with Gunther (verses 473.2; 474.1; 475.2) and the treasure of the Nibelungs (89.1; 2431.3).

\ea \label{ex:zeman:4}
\gll Hort {der Nibelunges} der was gar getragen \\
treasure {of the Nibelungs} the was wholly carried\\
\glt \hfill [\textit{Nibelungenlied} C, 89, 1]
\z

Furthermore, LDs\is{left dislocation} often occur at the beginning of an episode. In the manuscript, these are often marked by coloured initials. In these contexts, LDs\is{left dislocation} contribute to the organisation of the text. This function can also be observed in the \textit{Tristan}, where the LDs\is{left dislocation} are often combined with the discourse marker \textit{nu}, which sets a new temporal reference point for the story, cf. \REF{ex:zeman:5} (similar also \textit{Tristan} 3379, 3576).

\ea \label{ex:zeman:5}
\gll Nû Tristan der was aber bereit. \\
now Tristan the was again willing\\ 
\glt \hfill [\textit{Tristan}, 2970]
\z

In comparison with the \textit{Nibelungenlied}, the \isi{scene shift} is thus marked more explicitly. This observation would be in line with the diachronic development of “chapterification” \citep{Fludernik2003} and clearer marking of textual organisation, as described in \sectref{sec:zeman:2}. 

LDs \is{left dislocation} thus serve several functions in narratives, which can be described more systematically when considering the distinction between the three dimensions of \isi{narration}. First, LDs\is{left dislocation} contribute to the macrostructure of \isi{narration} by (re-)ac-tualizing and highlighting the story's protagonists. Second, they also contribute to the organisation of the narrative by marking episode shifts. As such, it is understandable why they are frequently used in the narrative \isi{discourse mode}. In terms of discourse relations,\is{discourse relation} however, it is interesting to note that the LDs\is{left dislocation} do not only (and not primarily) occur in narrative discourse relations.\is{discourse relation} While 60/136 LDs\is{left dislocation} in narrative \isi{discourse mode} are used in utterances that are linked to the previous discourse by a narrative discourse relation,\is{discourse relation} 58/136 LDs\is{left dislocation} are used in utterances that provide a description or background to the previous discourse segments, cf. \REF{ex:zeman:5} above and \REF{ex:zeman:6}.

\ea%6
\label{ex:zeman:6}
\gll ir schilde wol getân, \textbf{die} lûhten von den handen den vil waetlîchen man.\\
their beautiful shields, {} they shone from their hands {for the} very splendid men \\
\glt \hfill [\textit{Nibelungenlied} C, 408, 3f.]
\z 

\REF{ex:zeman:5} above sets a new reference point for the story now. However, it does not indicate the next step of action, but gives the background (\textit{being ready}) for Tristan's subsequent actions, which are recounted in the sentence that immediately follows. In \REF{ex:zeman:6}, there is no \isi{narration} of an event. Rather, the story comes to a pause and the Nibelungs' weapons are described. This shows that the LDs\is{left dislocation} are not restricted to a single function within a narrative but can be used in very different ways.

In the dialogic \isi{discourse mode}, LDs\is{left dislocation} only rarely (1/33) occur in the context of narrative discourse relations.\is{discourse relation} Most often they are used to evaluate a situation or to indicate a result. An example is \REF{ex:zeman:3} above, repeated here as \REF{ex:zeman:7}.

\ea%7
\label{ex:zeman:7}
\gll den hort den weiz nu niemen wan got unde mîn \\
the treasure that knows now nobody except god and myself\\
\z 
 
Again, \REF{ex:zeman:7} does not refer to the next step of an action or a specific event. Hagen as one of the heroes of the story comments on the treasure of the Nibelungs as an object that is relevant within the whole story and emphasises the result of the preceding actions. Since all the others are dead, Hagen is the only one how who knows where he has sunk the treasure in the Rhine. The analysis of discourse relations\is{discourse relation} thus shows that narrative and non-narrative discourse modes\is{discourse mode} correlate with different functional patterns of the LDs\is{left dislocation}. 

This assumption is supported by a look at the use of LDs\is{left dislocation} in sermons. Sermons are known for their high frequency of LDs\is{left dislocation} occurrences \citep{Lötscher1994}. So far, this has been explained by the fact that sermons are a text type that is relatively close to spoken language. Since sermons are also characterised by narrative passages, it could have been assumed that the LDs\is{left dislocation} would also be found primarily in narrative functions. However, this is not the case in Berthold von Regensburg's sermon. LDs\is{left dislocation} are mainly used in explanatory descriptions, cf. \REF{ex:zeman:8}.

\ea%8
\label{ex:zeman:8}
{\itshape Swaz daz firmament begriffen hât – daz ist der himel, 
den wir dâ sehen, dâ die sternen ane stênt –, swaz der umbe sich begriffen hât, daz ist geschaffen als ein ei. Diu ûzer schale \textbf{daz} ist der himel den wir dâ sehen. Daz wîze al umbe den tottern \textbf{daz} sint die lüfte. Sô ist der totter enmitten drinne, daz ist diu erde.}\\
\glt ‘What the firmament encompasses – that is the sky that we see, where the stars are –, what it encompasses, that is built like an egg. The outer shell, \textbf{that} is the sky that we see there. The white all around the yolk, \textbf{that} is the air. So is the yolk in the middle, that is the earth.' \\ \hfill [Berthold von Regensburg, sermo XXV: \textit{Sælic sint die reines herzen sint}]\\
\z 

The passage is not narrative but instructional as it explains the world according to the knowledge of the time. In this context, there are several sentences beginning with \textit{that} in resumptive function.\is{resumptive pronoun} The two LDs\is{left dislocation} in the text (marked in bold) reflect the same pattern as the surrounding sentences. The sentence following \textit{that} offers a description of the previous statement or referent.

Considering the narrative \isi{discourse mode} and the discourse relations,\is{discourse relation} it seems that the LDs\is{left dislocation} in Berthold's sermon are mostly used in the non-narrative \isi{discourse mode}.

In conclusion, LDs\is{left dislocation} show a great variety both in terms of their frequency and their range of functions in different text types. Further research is needed in order to gain a more detailed insight into the discourse functions.\is{discourse function} This is also relevant with regard to the diachronic line of development. As the analysis has shown, the development of LDs\is{left dislocation} from 1200 to 1600 is not linear. While there is a general decrease of LDs\is{left dislocation} in the history of \ili{German}, there are also micro-changes in dependence of discourse modes\is{discourse mode} and discourse relations.\is{discourse relation}\footnote{Nor is there a simple dependence to the degree of \isi{orality}: LDs are more frequent in the courtly epic \textit{Tristan} than in the \textit{Nibelungenlied}, which is considered closer to oral poetry (see in detail \citealt{Zeman2023a}).} The combination of LDs\is{left dislocation} with other markers of discourse organisation suggests that the development of the LD is connected to a more explicit strategy of marking discourse structure in narrative texts. Further research would be needed to test this hypothesis. It is clear from the observations in this section that the three dimensions of \isi{narration} serve to adequately describe the distribution and functions of LDs\is{left dislocation} and their development.


\subsection{The grammaticalisation of the pluperfect}\label{sec:zeman:3.2}
It has already been mentioned in \sectref{sec:zeman:2.2} that the interaction between \isi{narration} and grammar is particularly evident in relation to the category of \isi{tense}. Again, this interaction involves all three dimensions of \isi{narration}. In terms of narrative macrostructure, it has been noted that \isi{narration} “displays some specifically narrative uses of tense”\is{tense} \citep[94]{Fludernik2012}, such as the Epic Preterite\is{preterite}\is{tense} and the Historical Present.\is{present tense}\is{tense} Both uses of \isi{tense} – the use of the \isi{present tense}\is{tense} to denote past events and the use of the \isi{preterite}\is{tense} in combination with speaker-oriented adverbials such as \textit{now} or \textit{tomorrow} (e.g. \textit{Tomorrow was Christmas day}) are restricted to narratives only, since they presuppose the distinction between the narrator's\is{narrator} and the character's\is{character} \isi{voice} and hence the double-layered structure of \isi{narration} (see in detail \citealt{Zeman2010} and \citealt{Zeman2018}). 

The distribution of tenses\is{tense} has also been the main argument for distinguishing between different discourse modes.\is{discourse mode} \citet{benviste1972}, \citet{Weinrich1964} and \citet{Hamburger1957} have argued that there are two different systems of tenses\is{tense} which are distributed complementarily within the text. Although their approaches are conceptually different (see \citealt{Fludernik2012} for a comparison), more recent literature has confirmed that the different modes of discourse show different patterns of \isi{tense} use (e.g. \citealt{Smith2003}). 

This is also evident for the \isi{tense} distribution in \ili{Middle High German} \citep{Zeman2010}. \tabref{tab:zeman:3} shows that the present\is{present tense}\is{tense} and the \isi{present perfect}\is{tense} have a strong preference for the speaker-oriented \isi{discourse mode} (SR), while the \isi{preterite}\is{tense} and the \isi{pluperfect}\is{tense} are mainly attested in the narrative \isi{discourse mode} (NR). As such, the tenses\is{tense} in \ili{Middle High German} are in complementary distribution.

\begin{table}
\caption{Tense distribution in the verse epic \textit{Herzog Ernst} / Middle High German (according to \citealt[117]{Zeman2010})}
\label{tab:zeman:3}
\begin{tabularx}{0.9\textwidth}{Xrrrrrrr}
\lsptoprule
& \multicolumn{2}{l}{narrative} & \multicolumn{2}{l}{speaker-oriented} & \multicolumn{2}{l}{general} & sum\\
& n & \% & n & \% & n & \% n\\
\midrule
PRES & 2 & 0.24\% & 820 & 97.74\% & 17 & 2.03\% & 839\\
PRET & 2586& 97.47\% & 66 & 2.49\% & 1 & 0.04\% & 2653\\
PERF (\textit{hân}) & 1& 0.89\% & 108 & 96.43\% & 3 & 2.68\% & 112\\
PERF (\textit{sîn})& 0& 0\% & 18 & 100\% & 0 & 0\% & 18\\
PLUP (\textit{hân})& 80& 100\% & 0 & 0\% & 0 & 0\% & 80\\
PLUP (\textit{sîn})& 43 & 97.73\% & 1 & 2.27\% & 0 & 0\%& 44\\
\midrule
sum & 2712 && 1013&& 21 &&3746\\
\lspbottomrule
\end{tabularx}
\end{table}


While there have been many studies on the functional difference between the \isi{present tense},\is{tense} the \isi{preterite}\is{tense} and the \isi{present perfect},\is{tense} the \isi{pluperfect}\is{tense} has not yet been the focus of attention. This may be because the development of the \isi{pluperfect}\is{tense} has been seen as parallel to that of the \isi{present perfect}.\is{tense} It is assumed that both perfect constructions follow the same grammaticalisation\is{grammaticalization} path from a resultative construction to a \isi{tense} form with eventive meaning. Furthermore, its semantics is assumed to be straightforward, signalling “that there is a reference point in the past, and that the situation in question is prior to that reference point, i.e. the \isi{pluperfect}\is{tense} can be thought of as ‘past in the past'” \citep[65]{Comrie1985}. In terms of \citet{Reichenbach1947}, its event time (E) is set before a reference time (R), which in turn is set before the speech time (S), i.e.: E < R < S. As such, its prototypical meaning is seen in the designation of the temporal order of events\is{sequence of events} within the past, such as in \REF{ex:zeman:9}. 

\ea%9
\label{ex:zeman:9}
{\itshape Dô sî ditz hâten vernomen, dô sprach der rîter mittem leun\\}
\glt ‘When they had heard that, then the knight with the lion spoke.'\\
\hfill [\textit{Iwein} 6108]; (example taken from \citealt[§ S10]{PaulEtAl2007})
\z 

The \isi{pluperfect}\is{tense} seems to have the same semantics as the perfect, with the only difference being the past meaning of the auxiliary. 

In terms of their relations to discourse mode,\is{discourse mode} however, \tabref{tab:zeman:3} evidently shows that the \isi{present perfect}\is{tense} and the \isi{pluperfect}\is{tense} differ in their preferences for narrative vs. non-narrative discourse mode.\is{discourse mode} While the \isi{present perfect}\is{tense} in Old\il{Old High German} and \ili{Middle High German} cannot be used to denote a sequence of past events\is{sequence of events} and is therefore restricted to non-narrative discourse modes,\is{discourse mode} the \isi{pluperfect}\is{tense} is used almost exclusively in the narrative discourse mode.\is{discourse mode} This preference runs through the entire history of language. To this day, the \isi{pluperfect}\is{tense} is seen as a \isi{tense} of written language and a form of “distance” (e.g. \citealt[24]{Luscher2011}). As will be shown below, this impression is related to the preference of the \isi{pluperfect}\is{tense} for the narrative discourse mode,\is{discourse mode} which in turn leads to a different semantic development of the \isi{past perfect}\is{tense} in comparison to the \isi{present perfect}.\is{tense} 

A major methodological problem in examining the use of the \isi{pluperfect}\is{tense} in narrative vs. non-narrative discourse modes\is{discourse mode} is the infrequent use of the \isi{pluperfect}. This is particularly true for non-narrative modes of discourse. In \tabref{tab:zeman:3}, there is only one instance of the \isi{pluperfect}\is{tense} that is not documented in the narrative discourse mode.\is{discourse mode} Looking at such rare instances, however, it is striking that the early uses in the non-narrative \isi{discourse mode} do not seem to have the prototypical meaning of the \isi{pluperfect}\is{tense} as described in \REF{ex:zeman:9}, see \REF{ex:zeman:10} for comparison. 

\ea%10
\label{ex:zeman:10}
{\itshape Was hettet ir vernomen, gut öheim, das ir des wondent?}\\
\glt ‘What had you heard, good uncle, that you could believe that?' \\
\hfill [\textit{Prosalancelot}, 2, 25]
\z 
 
In \REF{ex:zeman:10}, the speaker is referring to an event that happened in the past as seen from the time of speech. The example shows the semantics of a \isi{pluperfect}\is{tense} by indicating that the hearing of the news was already completed at a certain time in the past, i.e. E < R < S. However, due to the \isi{dialogue} situation, there is a strong reference to the actual speech time. This has the effect that the reference point R is less dominant than in cases such as \REF{ex:zeman:9}, where the past reference point is given by the context. Consequently, one event serves as the reference time for the other one, while both events are located in a past narrative frame before the speech time. 

The weaker status of R in \REF{ex:zeman:10} also depends on the interaction between tenses\is{tense} and narrative discourse mode.\is{discourse mode} As already mentioned, the narrative \isi{discourse mode} is characterised by a different deictic pattern\is{deictic markers} than non-narrative discourse modes,\is{discourse mode} cf. \citet{Smith2003}: 

\begin{quote}
In Narrative, situations are related to each other and dynamic Events advance narrative time. In Reports, situations are related to Speech Time and time progresses forward and backward from that time. \citep[13]{Smith2003}
\end{quote}

The narrative \isi{discourse mode} is characterised by the temporal \isi{sequence of events} in a deictically displaced story world. As such, the (dynamic) \textit{story now} provides a reference point for anchoring the events. In non-narrative discourse modes,\is{discourse mode} such as reports, the deictic anchor is not localised within a story world but is set by the deictic origo of speech time. If a third reference point is not contextually given within the text, the impression arises that R – and thus the semantics of the \isi{pluperfect}\is{tense} – seems “weaker”, since there is no reference point that distances the time of the event and the time of speech. 

This observation is particularly noteworthy since \citet{Hennig2000} and \citet{Bertinetto2010} have described similar cases of “r-weakening” (term by \citealt{Bertinetto2010}) for the \isi{pluperfect}\is{tense} in Modern \ili{German}. Based on their investigations, they suggest that the semantics of the \isi{pluperfect}\is{tense} has changed from \isi{past perfect}\is{tense} (E < R < S) to a general past tense\is{past tense}\is{tense} (E < S) – which would be in line with the cross-linguistic grammaticalisation\is{grammaticalization} path of the \isi{pluperfect}\is{tense} as described by \citet{Squartini1999}. Strikingly, the examples discussed by Hennig are attested in instances of spoken language \citep[67]{Hennig2000}. Against the background of the above, one might assume that the development was triggered by the non-narrative discourse mode.\is{discourse mode}

To test this hypothesis, a pilot study was conducted. The data sample consists of 982 instances of the \isi{pluperfect}\is{tense} from \ili{Middle High German} texts. The analysis was based on \ili{Middle High German} epic texts (\textit{Herzog Ernst}, Konrad von Würzburg: \textit{Trojanerkrieg}), a verse epic with a first-person narrator\is{narrator} (Rudolf von Ems: \textit{Der Guote Gerhart} (1220)) and verse narratives by \textit{Der Stricker} which are considered to be close to \isi{orality}. All occurrences were analysed with regard to discourse mode,\is{discourse mode} text type, 1st, 2nd and 3rd person, matrix clause vs. subordinate clause as well as further contextual features such as adverbials and negation. \tabref{tab:zeman:4} shows the distribution of the \isi{pluperfect}\is{tense} (i) in the entire text, (ii) in first and second person, and (iii) in non-narrative contexts (as laid out above, passages of narrative passages within \isi{direct speech} were counted as “narrative”). Furthermore, the occurrences were semantically analysed with respect to a possible r-weakening. Although the absolute number of the \isi{pluperfect}\is{tense} in the non-narrative \isi{discourse mode} is low (n = 15), it is obvious that the pluperfect\is{pluperfect}\is{tense} in non-narrative contexts tend to carry the meaning E < (R) < S. It is also important to note that the non-narrative context has more influence than the use of the \isi{pluperfect}\is{tense} in the 1st or 2nd person. 

\begin{table}
\begin{tabularx}{0.8\textwidth}{lQQQ}
\lsptoprule
& {in total} & {1st/2nd person} & {non-narrative}\\
\midrule
r-weakening & (12/982) & (7/28) & (11/15) \\
& 1.22\% & 25.00\% & 73.33\%\\
\lspbottomrule
\end{tabularx}
\caption{R-weakening of the pluperfect in Middle High German}
\label{tab:zeman:4}
\end{table}

\tabref{tab:zeman:4} supports the hypothesis above that the semantic change occurs primarily in non-narrative contexts. Although the instances in the \ili{Middle High German} corpus do not show cases of general past semantics (as assumed for Modern \ili{German} by \citealt{Hennig2000}), the high numbers of r-weakening suggest that narrative and non-narrative contexts have an impact on the meaning of grammatical forms. This can be said for the synchronic state of \ili{Middle High German}. In comparison with the more recent development in Modern \ili{German}, it can be assumed that the motivation for this development lies in the distinction between discourse modes.\is{discourse mode} 

This hypothesis should be tested with more diachronic data. If confirmed, it would suggest that narrative and non-narrative discourse modes\is{discourse mode} are not only relevant in terms of the synchronic distribution of \isi{tense} forms. They also seem to be able to trigger different grammaticalisation\is{grammaticalization} pathways. In contrast to the \isi{present perfect}\is{tense}, which is initially restricted to non-narrative contexts and then expands into the narrative \isi{discourse mode} in the course of linguistic history, the \isi{pluperfect}\is{tense} initially appears only in the narrative \isi{discourse mode}. But it is not restricted to this context. When the \isi{pluperfect}\is{tense} is used in non-narrative contexts, there is a shift in meaning that triggers a development that is different from the development of the \isi{pluperfect}\is{tense} forms used in narrative discourse. Because of the rare use of the \isi{pluperfect}\is{tense} in non-narrative discourse contexts, one might assume that this shift is a recent development related to the qualitative spread of the \isi{pluperfect}\is{tense} into non-canonical contexts of use. The observations on \ili{Middle High German}, however, suggest that the development is already mapped out in the history of \ili{German}. More precisely, in the distinction between narrative and non-narrative discourse modes.\is{discourse mode} 


\section{The impact of narration} \label{sec:zeman:4}
The aim of this article was to examine \isi{narration} as an important factor in the interaction between grammar and discourse structure in relation to \isi{language change}. In order to allow for a systematic investigation of the effects of \isi{narration}, it has been argued that there are three different dimensions of \isi{narration} that need to be distinguished – (i) narrative \textit{texts} that exhibit a narrative macrostructure, (ii) narrative \textit{discourse mode}, i.e. narrative segments that can occur in (narrative and non-narrative) texts, and (iii) narrative \textit{discourse relations}\is{discourse relation} between propositions. The effects of these different dimensions of \isi{narration} have been illustrated by means of two empirical case studies, leading to the following conclusions:

\begin{enumerate}
\item 
The analysis of left dislocations\is{left dislocation} in \ili{Middle High German} has shown that (i) they have functions specific to narrative texts, (ii) that their distribution and textual functions differ in narrative and non-narrative discourse modes\is{discourse mode} and (iii) that this is linked to different preferences for discourse relations.\is{discourse relation} Against this background, it has also become clear that the line of development of LDs\is{left dislocation} from 1200 to 1600 is not straight. While there is a general decline of LD\is{left dislocation} in the history of \ili{German}, there are also micro-changes in the dependence of discourse modes\is{discourse mode} and discourse relations.\is{discourse relation} This supports the claim that all three dimensions of \isi{narration} must be taken into account in order to adequately investigate the development of LD\is{left dislocation} in the history of \ili{German}. 
\item 
The analysis of the \isi{pluperfect}\is{tense} in \ili{Middle High German} has shown that present\is{present perfect}\is{tense} and \isi{past perfect}\is{past perfect}\is{tense} constructions are distributed complementarily in non-narrative and narrative discourse modes.\is{discourse mode} With regard to the development of the pluperfect,\is{pluperfect}\is{tense} it has been shown that the occurrences in the non-proto-typical context of non-narrative discourse modes\is{discourse mode} tend to show “r-weakening” and thus a different semantics than the \isi{pluperfect}\is{tense} forms used in narrative contexts. This suggests that the \isi{pluperfect}\is{tense} developed differently in narrative and non-narrative discourse modes\is{discourse mode} and that the underlying \isi{discourse mode} may trigger different grammaticalisation\is{grammaticalization} pathways. 
\end{enumerate}

In sum, it can be said that \isi{narration} – in its various dimensions – is an important factor in the interaction between grammar and discourse structure. In this context, it is also important to note that both the distinction between narrative and non-narrative discourse modes\is{discourse mode} and the distinction between narrative and non-narrative discourse relations\is{discourse relation} allow for a finer and more systematic distinction than the distinction between text types and text genres,\is{genre} which has so far been one of the central parameters of investigation in many studies of the structure of discourse in historical languages. The distinction between text types and text genres,\is{genre} however, faces considerable difficulties. Apart from the problem of inconsistent classifications, diachronic comparisons are difficult because not every text type or \isi{genre} occurs in every period. The differentiation of discourse modes\is{discourse mode} and discourse relations,\is{discourse relation} on the other hand, allows not only for a finer distinction within the text, but also comparisons across centuries. In this respect, further research into discourse modes\is{discourse mode} and discourse relations\is{discourse relation} could help to answer the question of whether the effects of text type observed in many studies are a causal effect or rather an epiphenomenon of the underlying discourse structure (see also \citealt[2224]{Speyer2022}). 

%%%%%%%%%%%%%%%%%%%%%%%%%%%%%%%%%%%%%%%%%%%%%%%%%%%%%%%%%%%%%%%%%%%%%%%%%%%
\section*{Primary sources}
\begin{hangparas}{.25in}{1}
Das \textit{Nibelungenlied} nach der Handschrift C. Edited by Ursula Hennig. Tübingen: Niemeyer 1977.

Gottfried von Straßburg: \textit{Tristan}. Edited by Karl Marold. Berlin: De Gruyter 2016.

Berthold von Regensburg. Sælic sint die reines herzen sint. In \textit{Deutsche Neudrucke}. Edited by Franz Pfeiffer. Berlin: De Gruyter 2021.

\textit{Herzog Ernst}. Ein mittelalterliches Abenteuerbuch. In der mittelhochdeutschen Fassung B nach der Ausgabe von Karl Bartsch mit den Bruchstücken der Fassung A. Edited and translated by Bernhard Sowinski. Stuttgart: Reclam 1979.

\textit{Prosalancelot}. Heidelberger Handschrift Cod. Pal. Germ. 147. Edited by Reinhold Kluge. Berlin: Deutscher Klassiker Verlag 1948. 

Der Stricker. \textit{Verserzählungen I und II.} Edited by Hanns Fischer \& Johannes Janota. Berlin: De Gruyter 1997/2000.
\end{hangparas}


\bigskip



{\sloppy\printbibliography[heading=subbibliography,notkeyword=this]}

\end{document} 



