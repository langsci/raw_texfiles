\documentclass[output=paper,colorlinks,citecolor=brown]{langscibook}
\ChapterDOI{10.5281/zenodo.15689123}

\author{Barthe Bloom\orcid{0000-0003-3378-9856}\affiliation{Universität Potsdam}}
\title[Structuring the narrative with preposed adverbial clauses]{Structuring the narrative with preposed adverbial clauses: A study of the German and Dutch Ponthus adaptations}

\abstract{This study investigates the use of preposed adverbial clauses in the Early New High German and Dutch adaptations of \textit{Ponthus et la belle Sidoine}. The focus lies on so-called resumptive constructions, that is, on adverbial clauses followed by \textit{so/soo} or by a temporal adverb (e.g., \textit{da} `then'). The study shows that the choice between these resumptive constructions correlates with a distinction between mimesis and diegesis and that they are associated with different narrative speed.

For German, adverbial clauses that are followed by \textit{so} tend to occur in scenes and are associated with isochrony, whereas those followed by the temporal adverb \textit{da}\is{resumptive} `then' occur in summaries and are used to continue the story. This is correlated with the type of adverbial clauses that occur in the constructions: The prototypical protasis-apodosis construction – V1 clause + \textit{so} – is more dominant in dialogues,\is{dialogue} whereas a temporal adverbial clause with \textit{da} tends to occur in summary.

The same tendency is visible for Dutch, albeit weaker. This is explained, as the temporal resumptive construction, which was dominant in German, is primarily replaced by non-resumed adverbial clauses. These constructions are similarly preferred in diegetic parts of the discourse that progress the story.}

% Uncomment if only this chapter is to be compiled
%\IfFileExists{localcommands.tex}{
 % \addbibresource{localbibliography.bib}
 % \usepackage{langsci-optional}
\usepackage{langsci-gb4e}
\usepackage{langsci-lgr}

\usepackage{listings}
\lstset{basicstyle=\ttfamily,tabsize=2,breaklines=true}

%added by author
% \usepackage{tipa}
\usepackage{multirow}
\graphicspath{{figures/}}
\usepackage{langsci-branding}

 % 
\newcommand{\sent}{\enumsentence}
\newcommand{\sents}{\eenumsentence}
\let\citeasnoun\citet

\renewcommand{\lsCoverTitleFont}[1]{\sffamily\addfontfeatures{Scale=MatchUppercase}\fontsize{44pt}{16mm}\selectfont #1}
  
 % %% hyphenation points for line breaks
%% Normally, automatic hyphenation in LaTeX is very good
%% If a word is mis-hyphenated, add it to this file
%%
%% add information to TeX file before \begin{document} with:
%% %% hyphenation points for line breaks
%% Normally, automatic hyphenation in LaTeX is very good
%% If a word is mis-hyphenated, add it to this file
%%
%% add information to TeX file before \begin{document} with:
%% %% hyphenation points for line breaks
%% Normally, automatic hyphenation in LaTeX is very good
%% If a word is mis-hyphenated, add it to this file
%%
%% add information to TeX file before \begin{document} with:
%% \include{localhyphenation}
\hyphenation{
affri-ca-te
affri-ca-tes
an-no-tated
com-ple-ments
com-po-si-tio-na-li-ty
non-com-po-si-tio-na-li-ty
Gon-zá-lez
out-side
Ri-chárd
se-man-tics
STREU-SLE
Tie-de-mann
}
\hyphenation{
affri-ca-te
affri-ca-tes
an-no-tated
com-ple-ments
com-po-si-tio-na-li-ty
non-com-po-si-tio-na-li-ty
Gon-zá-lez
out-side
Ri-chárd
se-man-tics
STREU-SLE
Tie-de-mann
}
\hyphenation{
affri-ca-te
affri-ca-tes
an-no-tated
com-ple-ments
com-po-si-tio-na-li-ty
non-com-po-si-tio-na-li-ty
Gon-zá-lez
out-side
Ri-chárd
se-man-tics
STREU-SLE
Tie-de-mann
}
 % \boolfalse{bookcompile}
 % \togglepaper[23]%%chapternumber
%}{}

\begin{document}
\maketitle

\section{Introduction}
This study investigates the use of adverbial V3\is{verb-third} patterns in the \ili{Early New High German} ({GE\il{German}}) and \ili{Dutch} ({DU}) adaptations of the Ponthus. The primary focus lies on preposed adverbial clauses\is{adverbial clause} (henceforth \textsc{pac}) followed by \textit{so/soo} \REF{ex:bloom:1} or by a \isi{temporal adverb}\is{adverb}  (e.g., \textit{da}\is{resumptive} `then') \REF{ex:bloom:2}, known as \isi{resumptive} or \isi{correlative} constructions \citep[e.g., ][]{Pittner1999, Meklenborg2020, Axel2023}. 

\ea
\ea \label{ex:bloom:1}
\gll ob sie noch kein man haitt \textbf{so} wil ich einen man mittbringen 
\\
if she not.yet no man has.\textsc{sbjv} \textsc{so} want I a man with.bring\\
\glt `If she does not yet have a man, I will bring a man with me.'\\ \hfill [GE, 119va]
\ex \label{ex:bloom:2}
\gll Da ir glenen waren zu Brochen \textbf{da} zugen sie die swert\\
then.\textsc{cnj} their lances were to broken \textsc{temp} drew they the swords\\
\glt `When their lances had broken, they drew the swords.' \hfill [GE, 66va]
\z
\z

\noindent A terminological note: With \textit{so}\is{resumptive} and \textit{da}\is{resumptive} I will refer to their preverbal use as illustrated above, namely following adverbial clauses\is{adverbial clause} and preceding the finite verb of the host sentence, unless otherwise specified. With \textsc{temp}\is{resumptive}, I refer to the use of temporal adverbs\is{temporal adverb}\is{adverb} in the prefield following \textsc{pac}s\is{adverbial clause}, including \textit{da}\is{resumptive}.

In Present-Day \ili{German}, \textit{so}\is{resumptive} and \textit{da}\is{resumptive} are associated with different registers\is{register}: \textit{so}\is{resumptive} after adverbial clauses\is{adverbial clause} is found in formal, written \isi{register}, whereas \textit{da}\is{resumptive} is found after adverbial clauses\is{adverbial clause} in spoken, colloquial language \citep{Condoyannis1944, Weinert2007, Catasso2021}. If this distribution was a continuation of a historical situation, one would expect to find \textit{da}\is{resumptive} primarily in parts of the text that imitate spoken language in earlier stages of \ili{German}, whereas \textit{so}\is{resumptive} would be the more prevalent construction in more formal segments. What we find is that the constructions, in fact, statistically correlate with a distinction between \isi{mimesis} (\textit{μίμησις} from \textit{μιμεῖσθαι} `to imitate'), where characters'\is{character} speech is imitated, and \isi{diegesis} (\textit{διήγησις} from \textit{διηγεῖσθαι} `to narrate'), narrated segments. However, the results point in the opposite direction: \textsc{Pac}s\is{adverbial clause} that are followed by \textit{so}\is{resumptive} tend to occur in mimetic segments, whereas constructions with \textit{da}\is{resumptive} occur primarily in diegetic parts of the text. To understand this and to explain the distribution of the patterns in the text, the study evaluates the role of \isi{narrative speed} \citep[e.g.,][]{Genette1980, Packard2008}. Specifically, it is argued that, although constructions statistically correlate with a distinction between \isi{mimesis} and \isi{diegesis}, this is better accounted for in terms of \isi{narrative speed} \citep[e.g.,][]{Genette1980, Packard2008}.

Mimesis\is{mimesis} in narrative prose is found in segments of \isi{direct speech}. These sections are associated with \isi{isochrony}, namely the simultaneous progression of story time (i.e., the events that are reported) and narration time (i.e., the reporting of the events). Differently, in diegetic segments, the events of the story are often presented by the \isi{narrator} as narrative summaries. The events of the story unravel more rapidly than the \isi{narration} event in these sections. As such, narrative summaries continue the story at a higher \isi{pace}.

The results show that \textit{da}\is{resumptive} is associated with \isi{narrative summary}, whereas \textit{so}\is{resumptive} occurs in segments in which the story progresses more slowly. This is correlated with the type of \textsc{pac}s\is{adverbial clause} that occur in the constructions: The V1-conditional clause, introducing alternative events, is more dominant in dialogues\is{dialogue} and strongly associated with \textit{so}\is{resumptive}, whereas temporal clauses -- normally combined with \textit{da}\is{resumptive} -- tend to occur in \isi{narrative summary}.

A comparison with the \ili{Dutch} adaptation of the Ponthus finds a similar tendency in \ili{Dutch}, although the effect is weaker: \textit{Soo} occurs with sentences that introduce alternative events and is associated with \isi{isochrony}, whereas temporal adverbs\is{temporal adverb}\is{adverb} are found in narrative summaries. This is illustrated in \REF{ex:bloom:3} and \REF{ex:bloom:4}.

\ea
\ea \label{ex:bloom:3}
\gll kinderen is dat sake dat ghi niet steruen en wilt \textbf{soo} blijft hier binnen\\
children is that sake that you not die \textsc{neg} want \textsc{so} stay here inside\\ 
\glt `Children, if you do not want to die, stay here inside.' \hfill [DU, A3ra]
\ex \label{ex:bloom:4}
\gll Als hi dat hoorde \textbf{terstont} dede hy hem brenghen eten ende drincken.\\
when he that heard \textsc{temp} did he them bring food and drink\\
\glt `When he heard that, he immediately let food and drink be brought for them.' \hfill [DU, A3vb]
\z
\z

\noindent The weaker effect in Dutch is primarily due to a decreased use of the \isi{correlative} structure with a \isi{temporal adverb}\is{adverb} that is highly frequent in \ili{German}, however. This becomes apparent once other configurations with \textsc{pac}s\is{adverbial clause} are taken into consideration. Particularly relevant are \textsc{pac}s\is{adverbial clause} that are followed by a subject (viz. the \isi{juxtaposition} of the \textsc{pac})\is{adverbial clause} and those that are directly adjacent to finite verb of the host (typically the \isi{integration} of the \textsc{pac}\is{adverbial clause} into the host). These patterns are illustrated for \ili{Dutch} in \REF{ex:DUjux} and \REF{ex:DUint} respectively.

\ea
\ea \label{ex:DUjux}
\gll ist dattet pontus verneemt \textbf{hi} \textbf{salt} wel beletten \\
is.it that.it Ponthus perceived he will.it \textsc{disc} prevent\\
\glt `If Ponthus learns from it, he will prevent it.' \hfill [DU, j2ra]
\ex \label{ex:DUint}
\gll Als landri dat vernam \textbf{hief} hy op sijn swaert\\
when Landry that perceived heaved he up his sword\\
\glt `When Landry learned of this, he raised his sword.' \hfill [DU, f1vb]
\z
\z

\noindent What we will see is that the dominant realizations of \textsc{pac}s\is{adverbial clause} in the \ili{German} Ponthus are not the same as in the \ili{Dutch} text, where \isi{juxtaposition} and \isi{integration} are more prevalent. In particular, where \ili{German} uses \textit{da}\is{resumptive} in narrative summaries, \isi{integration} is the dominant strategy to realize \textsc{pac}s\is{adverbial clause} in narrative summaries in \ili{Dutch}. 

This paper is structured as follows. \sectref{sec:2} broadly discusses the role of \isi{diegesis} and \isi{mimesis} in narrative prose and the connection to \isi{narrative speed}. In \sectref{sec:3}, I introduce the texts from which the data is extracted, the data collection process, the annotation, and the statistical methods used in this paper. \sectref{sec:4} presents the analysis of the two \isi{correlative} constructions in \ili{German} and in \ili{Dutch}, discussing first the relation to \isi{diegesis} and \isi{mimesis} and subsequently the relation to \isi{narrative speed}. \sectref{sec:5} elaborates on other constructions with \textsc{pac}s\is{adverbial clause}. \sectref{sec:6} concludes this contribution.

\section{Style, voice, and speed in narrative prose}\label{sec:2}
This section first introduces \isi{mimesis}, \isi{diegesis}, and extradiegesis as different narrative styles\is{narrative style}\is{style} in \sectref{sec:2.1} and relates this to the concept of \isi{voice}. Thereafter, \isi{narrative style}\is{style} is related to \isi{narrative speed} in \sectref{sec:2.2}.

\subsection{Mimesis, diegesis, and extradiegesis}\label{sec:2.1}
The distinction between \isi{mimesis} and \isi{diegesis}, originally introduced by Plato, has been subject to a long tradition in which the concepts themselves and the relation between them have been interpreted and analyzed in many ways. In essence, the idea is that there is a difference between imitation (\isi{mimesis}) and \isi{narration} (\isi{diegesis}). This distinction in \isi{style} holds beyond prose text across other media, as can, for example, be seen in film \citep{Chatman1978}. I will focus on narrative prose, focusing on and adopting \citeauthor{Jong2004}'s (\citeyear{Jong2004}) interpretation of Plato's classification.

First, I deem it useful to define what is meant with \textit{story}, \textit{narration}\is{narration} and \textit{text} and to introduce these layers, as these terms have been used with (partially) overlapping definitions. I here follow \citet{Bal1985} and \citet{Jong2004} in much of the narratological definitions and methodologies. The \textit{story} denotes the events that are being told. These events do not necessarily have to be temporally ordered, but they have been organized and interpreted in some way.\footnote{The story is the result of the focalization of the \textit{fabula}, which is the logically or chronologically connected \isi{sequence of events} \citep{Bal1985, Jong2004}. The layer of the \textit{fabula} will not be central in the later analysis and will therefore not be introduced here.} The \textit{narration}\is{narration} is the telling of the events, and the \textit{text} is the result of this telling event.

Stories told by literary means are always mediated by a \isi{narrator} \citep[28]{Fludernik1993} and thus literary texts are entirely achieved by \isi{diegesis}, i.e., they are the product of \isi{narration}, of telling \citep[2--3]{Jong2004}. When talking about \isi{mimesis} as a component of literary narrative texts, this additional layer of the \isi{narrator} -- which may be absent in other media -- consequently complicates its definition. Plato and Aristotle analyzed speeches in a different way than the rest of the text. Plato views this as a distinction between \isi{mimesis} and non-mimesis. He identifies \textit{diegesis haple}\is{diegesis} (single \isi{narration}), in which the poet speaks as himself, and \textit{mimesis},\is{mimesis} in which the poet impersonates a \isi{character}. Aristotle, in contrast, seems to make a distinction between the poet speaking in first person as the poet (in for example the preface of a text) and the poet speaking as a \isi{narrator}. The latter is considered to be mimetic by Aristotle, only the former is non-mimetic according to him. In Plato's division, both would fall under \textit{\isi{diegesis} haple} \citep[6--8, 38]{Jong2004}.

Plato has the more restricted application of the concept of \isi{mimesis}, which I will follow here. However, since he did not distinguish between \isi{narrator} and poet, a slight modification is in order. Therefore, I here assume that the \isi{narrator} speaking as the \isi{narrator}, i.e., as itself, is diegetic, whereas the \isi{narrator} speaking as a \isi{character} is mimetic.

Both \isi{mimesis} and \isi{diegesis} can be illustrated by means of the example in \REF{ex:embNF}. The first sentence of the example presents a diegetic segment. The external \isi{narrator}, who is not a \isi{character} in the story, recounts an event. In contrast, \isi{character} text -- as presented in quotation marks in \REF{ex:embNF} -- is considered to be mimetic. 

\ea\label{ex:embNF}

Loki went to their underground forge. ``Hello, sons of Ivaldi. I have asked around, and people here tell me that Brokk and Eitri, his brother, are the greatest dwarf craftsmen there are or have been,'' said Loki.\\ \hfill \citep[54]{Gaiman2017}\\
\z

\noindent Technically, the quoted sentence is uttered by the same \isi{narrator} as in the previous sentence. He imitates the \isi{voice} of one of the characters,\is{character} i.e., of Loki, and the events are told as if was the narrator. As such, the answer to the question “who speaks?” is dual: Both the \isi{narrator} and the \isi{character} can be heard. Without going into much detail, this is possible due to the different layers a \isi{narration} is build of \citep{Bal1985} and the \isi{character}'s \isi{voice} is embedded within that of the \isi{narrator}. For simplicity's sake, I will speak of a \isi{character}'s \isi{voice} in such cases.

Cases of \isi{indirect speech} are, following \citet{Jong2004} but contrary to \citet{Genette1972}, taken to be diegetic, because it is the \isi{narrator} speaking as \isi{narrator}. The crucial distinction between \isi{indirect speech} and other segments of \isi{diegesis} lies in a difference in the embedded focalization \citep[255]{Jong2004}, namely a difference in who sees, orders and interprets. In those cases, it is the perspective of one of the characters\is{character} being filtered and voiced\is{voice} by the \isi{narrator} without imitating a \isi{character}.

I recognize that prefaces are different in some way. In the \il{German} prose, they do not typically present the events of the story but comment on it and give context to the creation of the manuscript. Such instances in which the \isi{narrator} speaks but does not narrate the story may also be found in parts other than the preface. An example is given in \REF{ex:edieg}.

\ea \label{ex:edieg}
There. That is the story of the mead of poetry and how it was given to the world. It is a story filled with dishonor and deceit, with murder and trickery. But it is not quite the whole story. There is one more thing to tell you. The delicate among you should stop your ears, or read no further.\\ \hfill \citep[151]{Gaiman2017}
\z

\noindent Here, the events of the story do not come into play. Instead, the \isi{narrator} evaluates the developments in the story that were just presented and comments on the structure of the \isi{narration}. As such the ``narrative level (...) is external to that of the world represented in the primary narrative" \citep[211]{Pier1986}. In such cases, we speak of an extradiegetic segment.

In sum, the distinguishing factor between \isi{mimesis} and \isi{diegesis} haple -- in the rest of the paper referred to as \isi{diegesis} -- is who speaks: Technically, if there is embedded \isi{character} \isi{narration} within the external \isi{narrator} that we find in the Ponthus, we speak of \isi{mimesis}. In the other cases, the \isi{narrator} speaks in his own \isi{voice}. These are mostly segments of \isi{diegesis} but may also be cases of extradiegesis, in which not the story itself but something else is presented.

\subsection{Narrative speed}\label{sec:2.2}
The different styles\is{style} of \isi{narration} are associated with differences in \isi{narrative speed} \citep[e.g][]{Genette1972, Genette1980, Packard2008}. I will here first discuss the definition of \isi{narrative speed} adopted in this paper and relate it to the notions of \isi{mimesis}, \isi{diegesis}, and extradiegesis.
 
The speed at which the story progresses in relation to the \isi{narration}  can vary, resulting in different velocities. This is what is presently understood by \isi{narrative speed}. \citet[95]{Genette1972} distinguishes primarily between four relations between story time (ST) and narration time (NT): 

\begin{enumerate}[label=(\roman*), noitemsep]
	\item Pause, in which NT continues, but ST halts (NT $\infty$ $>$ ST); 
	\item Scene, in which NT and ST coincide (NT = ST); 
	\item Summary, in which NT passes more quickly than ST (NT $<$ ST); and 
	\item Ellipsis, in which ST is not represented in NT (NT $<$ $\infty$ ST).
\end{enumerate}

\noindent The theoretically possible stretch (NT $>$ ST) is not canonically used, as slowing down of the \isi{narration} is typically done by means of pauses \citep[93--112]{Genette1972}.

Of course, this is a rather simplified representation of \isi{narrative speed} \citep{Packard2008, Kukkonen2020}, but assuming that the base-line of \isi{isochrony} is the dialogue,\is{dialogue} this categorical division suffices for the purposes of this paper. In fact, ellipsis will not be relevant, as this results in an absence of linguistic structure. Thus, we are left with a three-part division: pause, scene, and summary.\is{narrative summary}

\begin{figure}
\caption{Narrative style and narrative speed in the Ponthus adaptations}\label{fig:mosp}
	\begin{tikzpicture}[node distance=2cm]
		\node (4) [io] {character};
		\node (5) [io, right of = 4, xshift=2.5cm] {narrator};
		
		\node (1) [decision, below of = 4] {mimetic};
		\node (2) [decision, below of = 5, xshift = -2.25cm] {diegetic};
		\node (3) [decision, below of = 5, xshift=2.25cm] {extradiegetic};
		
		\node (6) [pro, below of=1] {isochrony};
		\node (7) [pro, below of=2, ] {summary};
		\node (8) [pro, below of=2, xshift = 2.25cm] {pause};
		
		\draw [line] (4)-- (1);
		\draw [line] (1)-- (6);
		\draw [line] (5)-- (2);
		\draw [line] (5)-- (3);
		
		\draw [line] (2)-- (7);
		\draw [line] (2)-- (8);
		\draw[line] (2)--(6);
		\draw [line] (3)-- (8);
	\end{tikzpicture}
	\end{figure}

The connection between the different narrative styles\is{narrative style}\is{style} and the narrative speed is visualized in \figref{fig:mosp}. In the Ponthus adaptations that will be discussed, a text segment is either the product of a \isi{narration} by an external but present \isi{narrator} (i.e., homodiegetic), or by a \isi{character} (and \isi{narrator}). If the \isi{character} is speaking, i.e., in case of \isi{mimesis}, narration and story time coincide, and we speak of \isi{isochrony}. This is illustrated by the \isi{direct speech} in \REF{ex:embNF}.

Whenever the \isi{narrator} is speaking, the situation becomes more complex, as what is being narrated can either be the story or something outside of the story; the \isi{style} can then be either diegetic or extradiegetic. Segments of extradiegesis pause the story, as was illustrated in \REF{ex:edieg}, but segments of \isi{diegesis} may constitute a pause or a summary.\is{narrative summary} Example \REF{ex:dieg} illustrates this alternation nicely. The segment starts with a description, as a state of affairs, which does not temporally progress the story but still concerns it -- a diegetic pause. With the introduction of the explicit temporal adverbial \textit{over time}, the story temporally progresses, and a part of the story that must have taken centuries, if not millennia, is narrated in one sentence -- a summary.\is{narrative summary}

\ea \label{ex:dieg} 
Between Muspell and Niflheim was a void, an empty place of nothingness, without form. The rivers of the mist world flowed into the void, which was called Ginnungagap, the ``yawning gap''. \textbf{Over time} beyond measure, these posed rivers, in the region between fire and mist, slowly solidified into huge glaciers. \hfill \citep[10]{Gaiman2017}
\z

Finally, diegetic segments may be isochronous,\is{isochrony} or very close to the temporal progression of the story in \isi{mimesis}. Segments of \isi{indirect speech} and free indirect speech may very closely mimic the words of a \isi{character}, while the deictic origo and the \isi{voice} do not lie with the \isi{character} but the \isi{narrator}. This is illustrated in \REF{ex:isoNA}, where the question closely imitates the presumed words of the man in the audience, yet the personal pronouns signal that this discourse segment is here spoken by the \isi{narrator}.

\ea \label{ex:isoNA}
A man in the audience with a misshapen leg stood up and challenged her: was she seriously suggesting that the restoration of her legs was comparable to the loss of his wife? \hfill \citep[130]{Chiang2002}
\z

In sum, \isi{mimesis} is isochronous,\is{isochrony} \isi{diegesis} is found with a variety of velocities, and extradiegesis pauses the story. The particular application of these categories to the current data will be discussed in \sectref{sec:3.3}.


\section{Methodology}\label{sec:3}

Prior to the analysis, this section concerns the methodology used in the present study. The texts under investigation are presented in \sectref{sec:3.1}. The collection of the data is explained in \sectref{sec:3.2}. \sectref{sec:3.3} discusses the annotation of the data, and \sectref{sec:3.4} mentions the statistical methods used.


\subsection{The texts}\label{sec:3.1}
The data used in the current study originate from a \ili{German} and a \ili{Dutch} reworking of the originally \ili{French} prose text \textit{Ponthus et la belle Sidonie}. The \ili{German} rendering, specifically the Rhine Franconian Fassung B manuscript, originates in the second half of the 15th century, before 1474 \citep[25]{BertelsmeierKierst2019}. \citet[28]{Schneider1961} suggests that the translation dates from somewhere between 1455 and 1470. The author is unknown. The edition that is used here is a diplomatic transcription of \textit{Pontus und Sidonia Tandareis und Flordibel} -- BSB Cgm 577 made within the DFG project ``Wortstellung und Diskursstruktur in der Frühen Neuzeit" \citep{RokoUP}.\is{Romankorpus Frühneuhochdeutsch} This is the same handwriting that underlies the modernized transcription by \citet{Schneider1961}. The \ili{Dutch} version dates from 1564 and was printed in Antwerp by Niclaes vanden Wouwere and is signed by Philips de Lens. The used edition is the diplomatic transcription by \citet{Kuiper2010}. Sporadically, I will also refer to the first printed edition of the \ili{French} version, which dates from 1479 in Geneva \citep[75]{BesamuscaWillaert2019} and has been made available online by Gallica, the online library of the Bibliothèque nationale de France.

Very briefly, the tale is about Ponthus, a very virtuous son of a king. We follow his adventures and the war of the Christians against the heathen kingdoms. He falls in love with Sidoine, but he faces many obstacles before he can finally be with her. The story is highly moralistic and focuses on the virtues a good Christian should have, which are embodied by Ponthus and Sidoine. For a full summary of the plot see \citet[549--551]{Classen2016}. Note that his paper concerns a different adaptation of the text than the ones discussed in the current paper. Although the names of the protagonists are quite different, it concerns the same main events.

Importantly, the narrative focuses on Ponthus but includes different, smaller strands focusing on others. These strands do not serve immediately to progress the main story, i.e., the adventures of Ponthus, but happen parallel to it and aid the development of other characters.\is{character} As such, the story is multi-stranded and shifts perspectives with explicit transitions like \textit{Nv laten wy te spreken vanden .xiiij. kinderen ende comen tot den ridder diese versonden hadde metten schepe ter zee waert} [DU, A4ra] and \textit{NV lassen wir die xiiij kynder vff dem mere farn vnd sprechen vorbaß von dem Ritter der sie ernert vnd dar gefertiget hette} [GE, 5ra].\footnote{A similar pattern to `now we leave X and come to Y' is often found in transitions and is attested in the \ili{French} print as well, e.g., \textit{Si laisseray icy a parler des quatorze enfans / et retourneray au cheualier qui les auoit mis dedansla nef.} [FR, 11l].} This can be seen in a diachronic perspective, as the shift from a single to multiple plot-lines is likely related to the development from the oral narrative tradition to written prose \citep{Fludernik2003, Zeman2023a}.

Though the \ili{German} and \ili{Dutch} Ponthus both portray the same story, they are of different lengths and vary in the speed in which the story progresses. In particular, the present \ili{German} adaptation has a drastically increased amount of direct discourse in comparison to other versions \citep[959]{Phinney1962}.\footnote{Though, since it is uncertain which French manuscript the translator has used, it is possible that the additions were made initially by a French scribe.} Additionally, narrative sequences are embellished as well but to a lesser degree than the dialogues\is{dialogue} \citep[19--26]{Schneider1961}. Therefore, the text is quite long and the story progresses rather slowly at times.
In the \ili{Dutch} adaptation, the extradiegetic comments of the \isi{narrator} are manifold. They halt the progression of the story, often in order to explicitly address issues of morality. 


\subsection{Data collection}\label{sec:3.2}
The first data set contains all \textsc{pac}s\is{adverbial clause} that are followed by \textit{so(o)} or \textsc{temp}\is{resumptive} in preverbal position in the \ili{German} and \ili{Dutch} Ponthus, as for example \REF{ex:bloom:1} and \REF{ex:bloom:2} for \ili{German} and in \REF{ex:bloom:3} and \REF{ex:bloom:4} for Dutch. They were manually extracted from the text. This data set contains 327 observations, 276 from \ili{German} and 51 of \ili{Dutch}. The second data set is an expansion of the former and contains all \textsc{pac}s\is{adverbial clause} that are preposed to clauses that are not introduced by a complementizer. That is, I do not consider constructions as in \REF{ex:compvl}, where the \textsc{pac}\is{adverbial clause} is preposed to the complement clause to \textit{globen} `promise', because this complement clause is introduced by the complementizer \textit{das}. Such clauses have their finite verb at the end of the clause.
 
\ea
\ea \label{ex:compvl}
\gll Da must er ir globen wan yme got gehulffe Das er sins vatter konigrich gewonne \textbf{das} er dan wider zu ir kommen \textbf{solt}\\
then must he her promise if/when him God help.\textsc{sbjv} that he his father's kingdom win.\textsc{sbjv} \textbf{that} he then again to her come \textbf{should}\\
\glt `Then he had to promise her that he would come back to her when God would have helped him to win his father's kingdom.' \hfill [GE, 85rb]

\ex \label{ex:compv2}
\gll vnd beschiet sie wan sie qwemen zu der statt so \textbf{solten} sie guden kauff geben\\
and instructed them if/when they come.\textsc{sbjv} to the city \textsc{so} \textbf{should} they goods sell give\\
\glt `And instructed them that they should present goods to sell when they would come to the city.' \hfill [GE, 2rb]
\z
\z

I include cases in which a dependent clause does not contain a complementizer, such as the complement clause to \textit{beschiet} in \REF{ex:compv2}, where the verb occurs early in the sentence/clause. In cases where it was unclear whether the initial clause was an adverbial or an independent main clause after consideration of the wider context, the sentence was not included in the data set. An example is presented in \REF{ex:amb}.

\ea \label{ex:amb}
\gll hebben wi goet betrouwen op hem hi sal ons wel helpen\\
have we good faith on him he will us \textsc{disc} help\\
\glt `Let us have good faith in him. He will help us.' \textit{Alternatively,} `If we have good faith in him, he will help us.' \hfill [DU, A3vb]
\z

In total, the texts contained 784 \textsc{pac}s,\is{adverbial clause} 427 in the \ili{German} and 357 in the \ili{Dutch} Ponthus. The more restricted data set, with only the \isi{correlative} patterns, contains 327 clauses, 276 \ili{German} and 51 \ili{Dutch} ones.


\subsection{Annotation}\label{sec:3.3}
The full data set was annotated for a number of variables, which will be presented here.

The response variable is the element (besides the adverbial clause) in the prefield (\textsc{LP}), which could be empty (\textit{--}), the element \textit{so}\is{resumptive} (\textit{so}), a subject (\textit{subj}), a temporal adverb/\textit{da}\is{resumptive}\is{temporal adverb} (\textit{temp}), or another constituent (\textit{other}) in the full data set. Note that an empty prefield does not necessarily mean that the \textsc{pac}\is{adverbial clause} is integrated\is{integration}. This also depends on the structure assumed for the host. For example, the prefield of imperatives is typically empty. Hence, if the \textsc{pac}\is{adverbial clause} combines with one, it is not likely to be syntactically integrated\is{integration}. This is exemplified in \REF{ex:IMP}. Nevertheless, it is possible to have a preverbal element with imperatives (\citealt[e.g.,][432]{Ebert1993}; \citealt[1450]{Muller2014}); seeing that there is an optional slot, the \textsc{pac}\is{adverbial clause} in \REF{ex:IMP} \textit{could} be argued to be integrated\is{integration}. Furthermore, a prefield can also not be generally assumed for \ili{Early New High German} independent main clauses, since V1 declaratives\is{verb-initial declaratives}\is{verb-first} were widespread (\citealt[431f.]{Ebert1993}; \citealt[145]{Demske2018}). Thus, one can also not be certain that the \textsc{pac}\is{adverbial clause} in \REF{ex:V1?} is syntactically integrated\is{integration}. For this reason, I have simply encoded the adjacency of the \textsc{pac}\is{adverbial clause} and finite verb instead of \isi{integration}.

\ea
\ea \label{ex:IMP}
 \gll Jst das ir sie zornig sehent \textbf{machent} sie zu freden mit gutten dogenden \\
is that you her spiteful see make her to peace with good deeds\\
\glt `If you see her spiteful, make her content through good deeds.'\\ \hfill [GE, 127vb]
\ex \label{ex:V1?}
\gll Da sie zwo is lange wyle angedreben druckten sie ir augen\\
then.\textsc{cnj} they two it long while continued pressed they their eyes\\
\glt `When the two of them had continued it for a long time, they dried their tears.' \hfill [GE, 59vb]
\z
\z

The situation is different in 16th century \ili{Dutch} which did not have V1 main clauses, and the Ponthus only contains one case in which the \textsc{pac}\is{adverbial clause} is directly adjacent to a finite verb in the imperative. This is exemplified in \REF{ex:NLimp}. Note that this is represented with a similar construction in the \ili{German} text \REF{ex:IMP} and in the \ili{French} print \REF{ex:FRimp}. As such, the adjacency of the \textsc{pac}\is{adverbial clause} and the finite verb may be due to transfer\is{transfer from French}.\il{French} Therefore, there is no direct evidence in the \ili{Dutch} Ponthus of potentially unintegrated \textsc{pac}s\is{adverbial clause} when \textsc{pac}\is{adverbial clause} and finite verb are adjacent.

\ea
\ea \label{ex:NLimp}
\gll Jst dat ghise te onureden siet \textbf{steltse} te vreden met duechdelike woorden\\
is.it that you.her to unhappiness sees make.her to peace with virtuous words\\
\glt `In case you see her unhappy, make her content through virtuous words.' \hfill [DU, n4ra]
\ex \label{ex:FRimp}
\gll et se vous la voyez courroucee \textbf{appaisez} la par courtoysie \\
and so you her see furious appease her through courtesy \\
\glt `And if you see her furious, appease her through courtesy.' \hfill [FR, 119r]
\z
\z

In sum, the direct adjacency of the \textsc{pac}\is{adverbial clause} and the finite verb of the host is thought to be reflective of the \isi{integration} of the \textsc{pac}\is{adverbial clause} into the host sentence in the \ili{Dutch} Ponthus but not in the \ili{German} text.


\subsubsection{Who speaks?}\label{sec:3.3.1}
The operationalization of \isi{mimesis} and \isi{diegesis} is encoded by annotating whether the utterance was spoken by a \isi{character} (\textit{char}) or by the \isi{narrator} (\textit{narr}). In the cases where the \isi{narrator} comments are extradiegetic, there were labelled (\textit{narr\_ex}). Since this is infrequent, this falls under the \isi{narrator} speaking for the statistical analysis.\footnote{In total, only six sentences were extradiegetic, five of which were found in the \ili{Dutch} text with a subject following. Since they were few, these cases will not be further discussed separately in the paper.} (See \sectref{sec:2.1}.)

There are a couple of sentences in the data set for which the categorization was not as clear-cut. Especially the boundary between \isi{indirect speech}, \isi{direct speech} and extradiegesis is not always easily distinguishable. In ambiguous case, I have chosen the most likely interpretation given the context. In one case, there seems to be a shift from \isi{character} to \isi{narrator}. This sentence is presented below \REF{ex:IS?} and full glosses and a translation are given in the \href{bloom:app}{Appendix}. The segment starts with \isi{indirect speech}. This is identifiable by third person pronoun \textit{sie} with the plural verb \textit{wolten}, which indicates that \textit{sie} refers to the speakers. Yet, the verb mood shifts from subjunctive to indicative in the last sentence. This suggests that the independent clause at the end of this segment is direct\is{direct speech} or free indirect speech \citep{Demske2019}. The \isi{present tense}\is{tense} of this final sentence (\textit{ist}) is also reflected in the \textsc{pac}\is{adverbial clause} by \textit{kompt}, but the main clause shows a \isi{tense} shift to \isi{preterite}\is{tense} (\textit{wart}). This is indicative of at least a shift of the deictic origo from the story-now to the \isi{narration}-now. This probably coincides with a shift in \isi{voice} from the \isi{characters}' to the \isi{narrator}'s.

\ea \label{ex:IS?}
{\itshape Sie sprachen da were vil abe zu sagen Dan siner schonheit weydelicheit vnd sins gutten gela\ss{} were nit me in der werlt
 als \uline{sie} wol gleuben wolten williche frauwe oder jungffrauwe yne zu bolen mochte haben da ist kein rede in Sie mocht wol berummen das sie den schonsten den besten den wol kommesten vnd den aller lustigesten frunt hette der da lebt vnd schon ist sin jugent \textbf{\uline{kompt} er also inn das alter so \uline{wart} sin gliche nie me ersehen} das \uline{ist} vngezwiuelt die rechte warheit }\hfill [GE, 16rb]
\z

\noindent This particular instance is removed from further analysis.

In the vast majority of cases, the context and the verbal morphology indicated clearly whether the \isi{narrator}'s or a \isi{character}'s \isi{voice} is heard.


\subsubsection{Narrative speed}\label{sec:3.3.2}
The data was additionally coded for \isi{narrative speed} (\textsc{Narr.Sp}). In dialogues,\is{dialogue} story time and narration time are taken to coincide, so that one may speak of \isi{isochrony} (\textit{iso}). Segments of \isi{indirect speech} where it is likely that the \isi{character}'s speech is presented as elaborate as if it were \isi{direct speech} are annotated as isochronous\is{isochrony} as well. When narration time is slower than story time, the sentence is classified as a summary\is{narrative summary} (\textit{sum}), and pauses in story time while the discourse is continued, both diegetically and extradiegetically, are annotated as \textit{pause} (see \sectref{sec:2.2}.).

In some cases, one sentence pertains to more than one level of \isi{narration}. If the \textsc{pac}\is{adverbial clause} and the host differ in this regard, the sentence is labeled \textit{shift}, as in \REF{ex:shift}. The \textsc{pac}\is{adverbial clause} summarizes the events of the story, but the host sentence is an extradiegetic comment by the \isi{narrator}, thus pausing the \isi{narration} of the story.

\ea
\ea \label{ex:shift}
\gll Als sidonie dit hoorde men derf niet vragen of si ontstelt wert van herten\\
when Sidoine this heard one dares not ask if she distraught was of heart\\
\glt `When Sidoine heared this, one dares not ask if she was sick at heart.'\\ \hfill [DU, e1va]
\ex \label{narrsp}
\gll Doen pontus den sarazijn versleghen hadde ende dat hooft den Eedelingen gegeuen hadde om dat sijt haren conick presenteren souden \textbf{als} \textbf{ghy} \textbf{nu} \textbf{ghehoort} \textbf{hebt} sloech hi sijn paert met sporen\\
then.\textsc{cnj} Ponthus the Saracen defeated had and the head the noblemen given had to that they.it their king present should as you now heard have hit he his horse with spurs\\
\glt `When Ponthus had defeated the Saracen and had given the head to the noblemen to present it to their king, \textbf{as you have now heard}, he spurred his horse on.' \hfill [DU, c3rb-c3va]
\z
\z

\noindent Whenever another embedded clause reflects a similar shift, this is ignored. For example, the \textit{als}-clause in \REF{narrsp}, pauses the narrative as the \isi{narrator} addresses his audience directly. The remainder of the utterance is a \isi{narrative summary}, continuing the telling of the story. As the \textit{als}-clause is a dependent clause within the preposed adverbial, the change is not annotated.

\ea \label{ex:iso}
\gll gebe er yme alle die werlt er neme ir nit vor sin dochter\\
gives.\textsc{sbjv} he him al the world he take.\textsc{sbjv} he not for his daughter\\
\glt `Even if he would give him the entire world, he wouldn't take it over his daughter.' \hfill [GE, 114vb]
\z

As mentioned in \sectref{sec:2.2}, segments of \isi{indirect speech} can be classified as (nearly) isochronous\is{isochrony} when the \isi{indirect speech} is a close reflection of the words of a \isi{character}, as for example in \REF{ex:iso}. Whenever the \isi{narrator} summarizes what would have been a more elaborate monologue or dialogue,\is{dialogue} the segment is considered to be a summary.\is{narrative summary}


\subsubsection{Type of \textsc{pac}}\label{sec:3.3.3}
Finally, the type of \textsc{pac}\is{adverbial clause} was coded. This is approximated by the element that introduces the \textsc{pac}\is{adverbial clause}, \isi{conjunction} or adverb, or the verb-initial\is{verb-first} position in the case of unintroduced \textsc{pac}s.\is{adverbial clause} This variable is encoded in order to see whether the above discussed narrative functions are reflected in or can be reduced to more local relations between \textsc{pac}\is{adverbial clause} and host-sentence. 


\subsection{Statistical methods}\label{sec:3.4}
The \textit{p-}values and residuals resulting from Pearson's $\chi$\textsuperscript{2}-test will be reported for simple frequency table. When the amount of data was insufficient, the \textit{p-}values from Fisher's exact test will be reported instead.

To rank the importance of the three annotated variables, “who speaks”, \isi{narrative speed}, and the type of \textsc{pac}\is{adverbial clause}, I use random forests and variable importance measures. For illustrative purposes, I present \isi{conditional inference tree}s. The “party-package” and the functions \textsc{cforest, varimp} and \textsc{ctree} are used to model these \citep{Hothorn2006, Strobl2007, Strobl2008}. All modeling was done in R \citep{R}. %I can make the script and data available if so desired


\section{Adverbial V3 with \textit{so(o)} and \textsc{temp}}\label{sec:4}

Why would \textit{da}\is{resumptive} and \textit{so}\is{resumptive} be associated with \isi{mimesis} vs. \isi{diegesis}? In Present-Day \ili{German}, temporal correlatives\is{correlative} (\textit{dann, da}) are associated with the spoken domain, and resumption with \textit{so}\is{resumptive} is restricted to the written, formal \isi{register} \citep{Catasso2021, Axel2023}. Working with the assumption that segments of direct discourse are more reflective of spoken language, since the language of (fictional) persons is imitated, and that segments of \isi{diegesis} are representative of more formal language \citep{Ebeling2020}, one would expect \textit{so}\is{resumptive} to be more dominant in diegetic segments of the \isi{narration} and \textsc{temp}\is{resumptive} in mimetic parts. If this Present-Day \ili{German} distinction has historical foundation, we would expect this to be the case for earlier stages of \ili{German} as well.

\begin{table}
\begin{tabularx}{0.8\textwidth}{Xrrrrr}
\lsptoprule
& \multicolumn{2}{c}{character} & \multicolumn{2}{c}{narrator} & \\
& {raw} & {($\chi$\textsuperscript{2}-residuals)} & {raw} & {($\chi$\textsuperscript{2}-residuals)} & {total}\\
\midrule
\textit{so} & 78 &(7.69) & 33 &(-5.05) & 111 \\
\textsc{temp} & 5 &(-6.32) & 159 &(4.16) & 164\\
\midrule
total & 83 & & 192& & 275 \\
\lspbottomrule
\end{tabularx}
\caption{Who speaks and \textit{so} \& \textsc{temp} in the German Ponthus}\label{tab-RE:mim}

\end{table}

It turns out that the opposite is true, as can be seen in Table \ref{tab-RE:mim}. The data indicate that there is a distinction between \textit{so}\is{resumptive} and \textit{da}\is{resumptive} in the prefield, which is significant with \textit{p} $<$ 0.001. The residuals in Table \ref{tab-RE:mim} show that each cell has an effect, as the residual is outside of the range -1.96 to 1.96, meaning that the specific combination either occurs less or more than expected. Thus, it can be concluded that \textit{so}\is{resumptive} occurs considerably more often than expected in \isi{character} direct speech segments and less often when the \isi{narrator} speaks in the \ili{German} Ponthus. The results indicate that the opposite is true for \textsc{temp}\is{resumptive}, which occurs more often than expected with the \isi{narrator}'s \isi{voice}.
What should be additionally noted is that all except one of the \textsc{temp}s\is{resumptive} is \textit{da/do}, as exemplified in \REF{ex:zulest}.

\ea \label{ex:zulest}
\gll Do er lange in dem walde geging sich zu bedencken in diesen sachen die yme wieder faren weren \textbf{zu lest} fant er einen schonen anslag\\
then.\textsc{cnj} he long in the forest went \textsc{refl} to ponder in these things that him again happened were \textbf{to latest} found he a beautiful placard\\
\glt `When he had went into the forest for a long time to ponder about these things that happened to him, finally he found a beautiful placard.'\\ \hfill [GE, 38va--38vb]
\z

\noindent There is thus not much variation with regard to the types of temporal elements that can occur preverbally after \textsc{pac}s.\is{adverbial clause} In other words, it is not a productive schema but a lexically specific construction.

The findings for \ili{German} thus point in the opposite direction than what was expected: Contrary to the hypothesis, \textit{so}\is{resumptive} is associated with mimetic segments and \textit{da}\is{resumptive} with non-mimetic ones in German. Dutch, however, has not recruited one specific adverb as a \isi{temporal resumptive}. Instead, a variety of temporal adverbs\is{temporal adverb}\is{adverb} is attested following an \textsc{pac}\is{adverbial clause} in the \ili{Dutch} Ponthus: \textit{doen} `then', \textit{nemmermeer} `nevermore', \textit{rechtevoort} `forthwidth', and \textit{terstont} `immediately'. Second, the overall frequency of \textsc{temp}s\is{resumptive} following a \textsc{pac}\is{adverbial clause} is considerably lower in comparison to \textit{so}\is{resumptive} than in \ili{German} (3.08\% vs. 38.41\% of all \textsc{pac}s).\is{adverbial clause}

\begin{table}[ht]
\begin{tabularx}{0.6\textwidth}{Xrrr}
\lsptoprule
& {\textit{{so(o)}}} & {\textsc{{temp}}} & {total}	\\
\midrule
German: character& 93.98\% & 6.02\% & 83\\
German: narrator & 17.19\% & 82.81\% & 192 \\
\midrule
Dutch: character & 93.55\% & 6.45\% & 31\\
Dutch: narrator & 55.00\% & 45.00\% & 20\\
\lspbottomrule
\end{tabularx}
\caption{The distribution of \textit{so(o)} and \textsc{temp} with respect to voice}\label{tab-RE:WSP}
\end{table}

\noindent As the frequencies for the \ili{Dutch} data are low, a Fisher's Exact test verifies that the \isi{voice} distinction is significant (\ili{Dutch}: \textit{p} = 0.002). That an overwhelming number of the \textsc{pac}s\is{adverbial clause} are followed by \textit{so(o)} and not by \textsc{temp}\is{resumptive} in \isi{direct speech} is also true for \ili{Dutch}, as can be seen in Table \ref{tab-RE:WSP}. \ili{Dutch} diverges from \ili{German}, however, in that it does not show a strong preference for \textsc{temp}\is{resumptive} in \isi{narrator} text. 

The results point in one of two directions: Either correlatives\is{correlative} were not in the same way associated with different registers\is{register} as in Present-Day \ili{German}, or the dialogues\is{dialogue} in the Ponthus are not reflective of spoken language (see for a similar point on \isi{direct speech} in \ili{Old English} \citealt{Louviot2016}). Investigating this is left for other studies. In any case, the data suggest that the two segments of discourse are distinctive so that the realization of the prefield shows different preferences. The unexpectedness of the results require an explanation. As mentioned earlier, direct discourse has been traditionally considered to be the temporal baseline, in which narrative and story time are taken to be isochronous\is{isochrony}. As such, when a \isi{character} speaks, the story develops steadily but rather slowly. Differently, when the \isi{narrator} speaks, the progression of the events in the story may be halted, go faster than the discourse, or there may be a shift between the \textsc{pac}\is{adverbial clause} and the following clause. Due to this connection between `who speaks' and \isi{narrative speed}, it can be hypothesized that the observed difference in Table \ref{tab-RE:WSP} is in fact an artifact of \isi{narrative speed}. \textit{so}\is{resumptive} would then be associated with \isi{isochrony} or an overall slower \isi{pace} of \isi{narration}, whereas temporal adverbs\is{temporal adverb}\is{adverb} might be associated with segments in which the evens of the story unfold comparatively rapidly in \ili{German}, or with no particular \isi{narrative speed} in \ili{Dutch}.

If this is the case, the occurrence of \textit{so}\is{resumptive} in segments of the story spoken by the \isi{narrator} should be of a slower \isi{pace} (i.e., pauses) than temporal adverbs\is{temporal adverb}\is{adverb}, which should be found within narrative summaries. This seems indeed to be the case.
\REF{ex:GEdaC} illustrates one of the examples of \textit{so}\is{resumptive} in the \isi{narrator} \isi{voice} and \REF{ex:GEsoN} of \textit{da}\is{resumptive} in a segment of \isi{direct speech}.

\ea 
\ea \label{ex:GEdaC} 
\uline{Context:} No-one ever hears him vow more than ``It is like that" in good faith.\\
\gll vnd so er dure sweren solt so sprach er (...)\\
and so he dearly swear should \textsc{so} sprach er (...)\\
\glt `And whenever he had to vow dearly, he said (...).' \hfill [GE, 9va]
\ex \label{ex:GEsoN}
\gll 	 Vnd do ich gewar wart das er erslagen was Da det ich myner jungffrauwen rock einen an\\
and then.\textsc{cnj} I aware became that he defeated was \textsc{temp} did I my lady's tunic one on\\
\glt `And when I became aware that he had been defeated, I put on one of my lady's tunics.' \hfill [GE, 106rb]
\z
\z

\noindent The segment in \REF{ex:GEdaC} is a rather straightforward case of description: The \isi{narrator} describes Ponthus and all his positive traits. In such segments, the temporal progression of the story is halted.

In \REF{ex:GEsoN}, the queen is speaking to Ponthus and the sequence \textsc{pac}\is{adverbial clause} + \textit{da}\is{resumptive} is used. In this segment, which comes relatively late in the \isi{narration} (106rb), both Ponthus and the reader learn about what has happened to the queen when the city was invaded and she was separated from Ponthus, which happens at the beginning of the text (2vb). 
The segment has a retrospective function and summarizes the events from a different perspective, thereby progressing the story rapidly. 

The question is now whether this observation can be statistically supported. Table \ref{tab-RE:SPEEDY} presents the proportions of the two patterns with the different velocities. 

\begin{table}[ht]
\caption{\textit{so} and \textsc{temp} and narrative speed in narrator speech} \label{tab-RE:SPEEDY}
\begin{tabularx}{0.8\textwidth}{lYYY}
\lsptoprule
& {\textit{{so(o)}}}& {\textsc{{temp}}}& {{total}}	\\
\midrule
German: pause	& 100\% & 0\% & 	6\\
	German: isochrony & 94.74\% & 5.26\% &19 \\
	German: summary & 5.06\% & 94.94\% & 158 \\
	German: shift & 11.11\% & 88.89\% & 9\\
	\midrule
	Dutch: pause 	&0\% & 100\% & 1	\\
	Dutch: isochrony & 100\% & 0\% & 2\\
	Dutch: shift & 100\% & 0\% & 2\\
	Dutch: summary & 46.67\% & 53.33\% & 15 \\
	\lspbottomrule
	\end{tabularx}
\end{table}

\noindent 
For \ili{German}, the prediction is indeed borne out, albeit with a low frequency of pauses and shifts (\textit{p} $<$ 0.001). \textit{So}\is{resumptive} occurs more frequently with a slower \isi{pace} (\isi{isochrony} \& pause) than \textsc{temp}\is{resumptive}. Since the \ili{Dutch} \textsc{pac}s\is{adverbial clause} with \textit{so}\is{resumptive} or a \isi{temporal adverb}\is{adverb} are only represented by one instance of a pause and two of a shift and \isi{isochrony}, no conclusion can be drawn. 

These results indicate that, at least for the \ili{German} data, \isi{narrative speed} is a better explanation for the distinction between \textit{so}\is{resumptive} and \textsc{temp}\is{resumptive} than \isi{voice}: Parts of the narrative that are voiced\is{voice} by the \isi{narrator} occur with \textit{so}\is{resumptive} when they progress the story slowly (\isi{isochrony} and pause).

Finally, the type of \textsc{pac}\is{adverbial clause} is correlated with the element that resumes it. That is, certain \textsc{pac}s\is{adverbial clause} seem to be associated with \textit{so}\is{resumptive}, whereas others have a tendency to be followed by \textit{da}\is{resumptive}. \textsc{Pac}s\is{adverbial clause} that are introduced by the typically temporal but also possibly causal \isi{conjunction} \textit{da} are also combined with \textit{da}\is{resumptive} and not by \textit{so}\is{resumptive}, see \REF{ex:da1}. 
Differently, V1-clauses,\is{verb-first} which typically express conditionality, are almost always followed by \textit{so}\is{resumptive}, as in \REF{ex:so}.

\ea
\ea \label{ex:da1}
\gll Da messe gesongen was Da ging man in den sale\\
then.\textsc{cnj} mass sung was \textsc{temp} went one in the hall\\
\glt `When mass had been sung, then they went into the hall.' \hfill [GE, 51rb]
\ex \label{ex:so}
\gll wolt ir yne sehen so wil ich yne uch bringen\\
wanted you him see \textsc{so} want I him you bring \\
\glt `If you wanted to see him, I will bring him to you.' \hfill [GE, 11ra]
\z
\z

For \ili{Dutch}, there are a few types of \textsc{pac}s\is{adverbial clause} that occur more than once in the smaller data set: \textit{als, doen, nadien (dat), indien} or V1\is{verb-first}. Each of those occurs \textit{so}\is{resumptive}, that is, there is no clause type that is restricted to a \isi{correlative} structure with \textsc{temp}\is{resumptive}.\footnote{The conjunction \textit{ten} occurs once with \textit{nimmermeer} `nevermore' but not with \textit{so(o)}: \textit{ten sy dat ghi op dese tijt uwen sin inden mijnen set nimmermeer en sal ic v lief hebben} `if it is that you now your desires in mine put, I will never love you.' [DU, j1vb].} \textit{Als-}clauses occur more often than others with a following \isi{temporal adverb}\is{adverb} (39.13\%, n = 23), all other \textsc{pac}s\is{adverbial clause} have an even higher proportion of \textit{so}\is{resumptive} as compared to temporal adverbs\is{temporal adverb}\is{adverb} (88.89\% for V1-clauses,\is{verb-first} and 100\% for \textit{nadien (dat), indien} and \textit{doen}).

To evaluate which variable has the most impact on the realization of the \isi{correlative} in \ili{German}, a random forest model was fitted to the data. Due to their low frequency, the categories \textit{pause} and \textit{shift} of \isi{narrative speed} are left out of the quantitative analysis. The random forest calculates 2000 trees (ntree = 2000) and selects two variables for each tree (mtry = 2). The random forest has an accuracy of 98.46\%, a C-value of 0.995 and Somers' D 0.990, which means that the model performs outstandingly well.

\begin{figure}
\caption{Conditional variable importance measures for German}\label{fig:1}
	\includegraphics[width=0.9\linewidth]{figures/BloomFig2.png}
\end{figure}

\largerpage
Presented in \figref{fig:1} are the results of the conditional variable importance measures for \ili{German}. They show that \textsc{type.pac}, i.e., the type of \textsc{pac}\is{adverbial clause} is the most important factor in the choice between a \isi{temporal adverb}\is{adverb} or \textit{so}\is{resumptive} in the prefield \textsc{who.speaks} which has not been found to be an important predictor. Narrative speed,\is{narrative speed} \textsc{narr.sp} is important as well. In other words, this confirms what was noted earlier in this section, that \isi{narrative speed} has more explanatory value than \isi{voice}. Importantly, the effect of \isi{narrative speed} does not disappear when accounting for the type of \textsc{pac},\is{adverbial clause} suggesting that the local \isi{discourse function} and the wider narrative function of the construction are both contributing factors.

\begin{figure}
\caption{Conditional inference tree: \textsc{temp} versus \textit{so} in the German Ponthus}\label{fig:2}
	\includegraphics[width=\linewidth]{figures/BloomFig3.png}
\end{figure}

The data for \ili{German} is further visualized as a \isi{conditional inference tree} in \figref{fig:2}. The first split in the data concerns the variable \textsc{type.s}, with \textsc{pac}s\is{adverbial clause} introduced by \textit{als}\footnote{It should be noted that \textit{als} is, unexpectedly, rather infrequent in the Ponthus for a thus-far unknown reason.} `when, while, as', \textit{als bald} `as soon', \textit{da} `then', \textit{nyt lange} `not long' and \textit{so bald} `as soon as' being more often temporally resumed than the other types of clauses, which occur more frequently with \textit{so}\is{resumptive}. What each of these clauses have in common is that they prototypically express a temporal relation, in particular one of simultaneity or rapid succession. This is illustrated in \REF{ex:darsu}
 
\ea\label{ex:darsu}
\gll Vnd so balde das geschach da kanten sie yne von stont wol\\ 
and as soon that happened \textsc{temp} knew they him of stand well\\
\glt `And as soon has that happened, they recognized him immediately.'\\ \hfill [GE, 87va]
\z

Within these temporal adverbial clauses\is{adverbial clause} of \textsc{pac}s,\is{adverbial clause} there is also a significant difference between those introduced by \textit{als} and \textit{so bald} versus those introduced by \textit{als bald, da} and \textit{nyt lange}. The first two are occasionally followed by \textit{so}\is{resumptive}, whereas this is not attested in the Ponthus for the latter group. Note that one should be rather careful with generalizing these findings: They are based on one text and thus cannot be taken as evidence that \textit{so}\is{resumptive} was categorically excluded from this group in other texts as well. Moreover, while the \isi{conditional inference tree} nicely presents the optimal binning for the current data set, whether this is a robust result that can be extrapolated remains to be seen. Nevertheless, the marginal occurrence with \textit{so}\is{resumptive} with conjunctions\is{conjunction} with a temporal meaning falls in line with what is known about these conjunctions\is{conjunction} in \ili{Dutch} and in \ili{English}.

Within the group of conjunctions\is{conjunction} that are associated with \textit{so}\is{resumptive}, a significant distinction is made regarding \isi{narrative speed}: Isochrony\is{isochrony}, as compared to summary,\is{narrative summary} has a significant higher proportion of \textit{so}\is{resumptive} and less temporal adverbs\is{temporal adverb}\is{adverb}. In other words, the types of \textsc{pac}s\is{adverbial clause} that are typically followed by \textit{so}\is{resumptive} can be taken up by a \isi{temporal adverb}\is{adverb} instead when they progress the story rapidly.

A reviewer suggested that verbal \isi{tense} might play a role in this relation between \textit{da}\is{resumptive} and \isi{narrative speed}. I acknowledge that this is very probable, but the relation between the two is not as straightforward as one might suspect. The examples thus far have suggested that a \isi{pluperfect}\is{tense} in the \isi{adverbial clause} is typically combined with \textit{da}\is{resumptive} and occurs in \isi{narrative summary}. This is not exclusively the case, as the examples in \REF{ex:PQP} illustrate: In \REF{ex:PQP1} it combines with \textit{so}\is{resumptive} and in \REF{ex:PQP2} as a \isi{juxtaposition}.

\ea\label{ex:PQP}
\ea \label{ex:PQP1}
\gll Aber dwile is \textbf{begonnen} \textbf{were} \textbf{so} wolt er is vß herten was da von qweme bose oder gut \\
but while it started was.\textsc{sbjv} \textsc{so} wanted he it out harden what there of came.\textsc{sbjv} evil or good\\
\glt `But since he had started it, he wanted to tough it out, whatever would have come of it, evil or good.' \hfill [GE, 112va]
\ex \label{ex:PQP2}
\gll Da sie nu \textbf{gescheyden} \textbf{waren} \textbf{Sydonie} begunde mit irn jungffern zu reden wie yne pontus gefiel\\
then.\textsc{cnj} they now separated were Sidoine started with her lady to talk how her Ponthus pleased\\
\glt `When they had separated, Sidoine started to talk with her lady about how she liked Ponthus.' \hfill [GE, 16rb]
\z
\z

\noindent Yet, all adverbial clauses\is{adverbial clause} with a \isi{pluperfect}\is{tense} verb \isi{tense} that combine with \textit{so}\is{resumptive} have subjunctive verbal mood, as in \REF{ex:PQP1}. This is also the case for pluperfects\is{pluperfect}\is{tense} with \isi{juxtaposition}. In the few cases in which the verb is indicative (e.g., \ref{ex:PQP2}), the sentence presents a \isi{narrative summary}. Thus, the indicative \isi{pluperfect}\is{tense} indeed appears to be associated with \textit{da}\is{resumptive} and segments of \isi{narrative summary}. Due to the inconsistency in spelling, which results in a conflation of the third person singular present\is{present tense}\is{tense} and \isi{past tense}\is{tense} of \textit{haben} `have' in combination with the interaction of \isi{tense} and mood, a more systematic analysis of the impact of \isi{tense} goes beyond the scope of this paper and will be left for further research.

\begin{figure}
\caption{Conditional inference tree: \textsc{temp} versus \textit{so} in the Dutch Ponthus}\label{fig:3}
	\includegraphics[width=\linewidth]{figures/BloomFig4.png}
\end{figure}

Turning now to \ili{Dutch}, the \isi{conditional inference tree} presented in \figref{fig:3} shows, curiously, no identifiable distinction regarding the type of \textsc{pac}.\is{adverbial clause} Like \ili{German}, the model identifies a significant split in the data by \isi{narrative speed}, with \isi{isochrony} showing a higher proportion of \textit{so}\is{resumptive} than summaries, which have a relatively high amount of \textsc{temp}\is{resumptive}. 

This section has illustrated that there is a systematic distinction between \textsc{pac}s\is{adverbial clause} that are followed by \textit{so}\is{resumptive} versus those that are followed by \textsc{temp}\is{resumptive}: The former are associated with narrative segments in the \isi{character}'s \isi{voice} and the latter with the \isi{voice} of the \isi{narrator}. This was argued to be a reflection not of \isi{narrative style}\is{style} but of velocity, with \textsc{pac}\is{adverbial clause} + \textsc{temp}\is{resumptive} being indicative of a more rapid story progression. More specifically, while much of the variation between the two constructions in \ili{German} can be explained by the type of \textsc{pac},\is{adverbial clause} \isi{narrative speed} is shown to be an important factor as well. In \ili{Dutch}, the type of \isi{adverbial clause} is not significant, and only \isi{narrative speed} is found to be a significant variable. It should be noted that this may be due to the low overall frequency of the two constructions, in particular of \textsc{pac}s\is{adverbial clause} with \textsc{temp}\is{resumptive} as well as the relatively low frequency of narrative summaries. Two things should now be explained: 
Seeing that the \ili{Dutch} Ponthus is not significantly longer than the \ili{German} one and presents the same story, it is safe to assume that the amount of \isi{narrative summary} overall should not be lower in the \ili{Dutch} text than in the \ili{German} one. This raises the question whether this narrative structuring function is associated with other realizations of \textsc{pac}s,\is{adverbial clause} and whether this could simultaneously explain why \textsc{pac}s\is{adverbial clause} with \textsc{temp}\is{resumptive} are so rare in \ili{Dutch}.

In addition, it must be checked whether the type of \textsc{pac}\is{adverbial clause} has really no effect on the realization of the prefield in the \ili{Dutch} Ponthus, whether this was an effect of low overall frequency of \textsc{pac}\is{adverbial clause} + \textit{so}\is{resumptive} and \textsc{pac}\is{adverbial clause} + \textsc{temp}\is{resumptive}, or whether it plays a role after all but with different constructions. The following section will address these issues.


\section{\textsc{Pac}s in the German and Dutch Ponthus}\label{sec:5}
The previous section identified considerable differences between the \ili{German} and \ili{Dutch} Ponthus. In particular, the \ili{Dutch} data set contained few cases of \isi{narrative summary} with \textsc{temp}\is{resumptive} in comparison to the \ili{German} one and did not present as many correlatives overall.\is{correlative} The question rises then whether the \isi{narrative summary} is expressed by a different construction in \ili{Dutch}. In addition, the noteworthy absence of the effect of type of \textsc{pac}\is{adverbial clause} on the realization in \ili{Dutch} requires an explanation. I will therefore address these issues here, making use of the larger data set containing all \textsc{pac}s.\is{adverbial clause}

To recap, following the \textsc{pac},\is{adverbial clause} the options are relatively limited. In \ili{German}, the majority of the constructions belong to one of the following four types: 1) direct adjacency between \textsc{pac}\is{adverbial clause} and finite verb \REF{ex:int}, 2) \textsc{pac}\is{adverbial clause} + \textsc{temp}\is{resumptive}, predominantly \textit{da}\is{resumptive} \REF{ex:da}, 3) \textsc{pac}\is{adverbial clause} + \textit{so}\is{resumptive} \REF{ex:so2}, or 4) a subject intervenes between \textsc{pac}\is{adverbial clause} and the finite verb \REF{ex:subj}. On occasion, other elements -- mainly adverbs and objects -- stand in between \textsc{pac}\is{adverbial clause} and the finite verb \REF{ex:oth}.

\ea
\ea \label{ex:int}
\gll Do is tag wart stont er vff \\
then.\textsc{cnj} it day became stood he on\\
\glt `When the day came, he stood up.' \hfill [GE, 107ra]
\ex \label{ex:da}
\gll Da er das hortt \textbf{da} erschracke ere[sic!] sere \\
then.\textsc{cnj} he that heard \textsc{temp} shocked he much\\
\glt `When he heard that, he was very frightened.' \hfill [GE, 111va]
\ex \label{ex:so2}
\gll Aber dwile is begonnen were \textbf{so} wolt er is vß herten \\
but while it started was.\textsc{sbjv} \textsc{so} wanted he it out hard\\
\glt `But because it had been started, he wanted to see it through.'\\ \hfill [GE, 112va]
\ex \label{ex:subj}
\gll lebet der lange \textbf{er} dott mir alle myn folk \\
lives.\textsc{sbjv} he long he kills me all my folk\\
\glt `If he lives longer, he will kill all my people.' \hfill [GE, 72vb]
\ex \label{ex:oth}
\gll Ob ir yne dan sehent \textbf{was} wolten ir thun\\
if you him then see what would you do\\
\glt `If you see him then, what would you do?' \hfill [GE, 87va]
\z
\z

Table \ref{tab-PVbig} presents the distribution of the types of elements that occur in between the \textsc{pac}\is{adverbial clause} and the finite verb: \textit{so}\is{resumptive} and \textsc{temp}\is{resumptive} are known from the previous section; the hyphen (\textit{-}) stands for direct adjacency of the two; \textit{subject} stands, obviously, for the subject; and \textit{other} refers to all other elements that may occupy the position between the \textsc{pac}\is{adverbial clause} and the finite verb, which predominantly are other adverbs (e.g., \textit{darzu/daer bi} `therewith') and object pronouns (e.g., \textit{das/dat} `that').
 
\begin{table}[ht]
\caption{Preverbal elements that co-occur with \textsc{pac}s}\label{tab-PVbig}
\begin{tabularx}{0.75\textwidth}{Xrrrrrr}
\lsptoprule
& {-} & {other}& {\textit{so}} & {subject} & {\textsc{temp}} & {total}\\
\midrule
German &31&18&112&102&164& 427 \\
Dutch &211&25&40&70&11& 357\\
\midrule
total &242&43&152&172&175& 784 \\
\lspbottomrule
\end{tabularx}
\end{table}

The languages are significantly different (\textit{p} $<$ 0.001). Residuals show that adjacency of \textsc{pac}\is{adverbial clause} and finite verb has a higher than expected frequency in \ili{Dutch} and is lower than expected in \ili{German} (9.60 and -8.78 respectively). The opposite is true for \textit{so}\is{resumptive} in the prefield (-3.51 vs. 3.21) and especially for \textsc{temp}\is{resumptive} (-7.69 and 7.04): These patterns are infrequent following \ili{Dutch} \textsc{pac}s\is{adverbial clause} but frequent in \ili{German}. It thus seems that where \ili{German} uses \textsc{pac}s\is{adverbial clause} with \textsc{temp}\is{resumptive} or \textit{so}\is{resumptive}, we find \isi{integration} of the \textsc{pac}\is{adverbial clause} in \ili{Dutch}.

In the following section, this frequency difference is related to the two factors that were discussed in the previous section. In \sectref{sec:5.1}, \isi{narrative speed} is considered and I will also take into account the type of \textsc{pac}\is{adverbial clause} in \sectref{sec:5.2}.


\subsection{Narrative speed}\label{sec:5.1}
Figures \ref{fig:GE4} and \ref{fig:NL4} present the distribution of the different prefield fillings in relation to the \isi{narrative speed} in the \ili{German} and \ili{Dutch} Ponthus respectively. What should be kept in mind while reading these figures is that the \ili{Dutch} data contains less \isi{isochrony} (29.69\%) and more \isi{narrative summary} (62.75\%) in comparison to the \ili{German} data (43.09\% and 50.12\%, respectively). This is in line with the expectation, as both texts present the same story although the \ili{German} Ponthus is ca. 20000 words longer.


\begin{figure}
\caption{Narrative speed and the prefield in the German Ponthus}\label{fig:GE4}
	\includegraphics[width=\linewidth]{figures/BloomFig5.png}
\end{figure}

\begin{figure}
\caption{Narrative speed and the prefield in the Dutch Ponthus}\label{fig:NL4}
 	\includegraphics[width=\linewidth]{figures/BloomFig6.png}

\end{figure}

\figref{fig:GE4} shows that adjacency of \textsc{pac}\is{adverbial clause} and finite verb is overall infrequent but occurs most frequently in narrative summaries. Narrative summaries are particularly frequent with \textsc{temp}\is{resumptive}, as \isi{isochrony} is with \textit{so}\is{resumptive}. Subjects in the prefield occur mainly with \isi{isochrony} (62.75\%), though \isi{narrative summary} is also frequent (33.33\%). This suggests that \textsc{pac}s\is{adverbial clause} that occur with a preverbal subject are not strongly associated with a particular \isi{narrative speed}. Other elements are overall infrequent and occur most often in isochronous\is{isochrony} segments (61.11\%).

In the \ili{Dutch} Ponthus, the situation is different. \figref{fig:NL4} makes it clear that the function of \isi{narrative summary} is dominant with the adjacency of the \textsc{pac}\is{adverbial clause} and the finite verb (91.94\%). This explains why so few occurences of \textsc{temp}\is{resumptive} in the prefield in \ili{Dutch} could be seen: The narrative function is expressed by integrating \textsc{pac}\is{adverbial clause}. Moreover, the frequency of \textit{so}\is{resumptive} is overall lower than in \ili{German}; nevertheless, the strong preference for \isi{isochrony} is similar there (77.50\%). A relatively high frequency of \isi{isochrony} is found with subjects in the prefield (64.29\%). In \ili{German}, this preference could also be seen, but the construction occurred also relatively frequently with narrative summaries. In \ili{Dutch}, subjects in the prefield are rarer in summaries (18.57\% versus 33.33\% in \ili{German}). Other elements occur overall very infrequently, but they are attested with all velocities.

In sum, the major difference between the two languages lies in narrative summaries and the construction that is associated with it: In the \ili{German} Ponthus, this function is expressed by \textsc{pac}s\is{adverbial clause} with \textsc{temp}\is{resumptive} -- specifically \textit{da}\is{resumptive} -- in the prefield; in the \ili{Dutch} text, the \textsc{pac}\is{adverbial clause} is directly adjacent to the finite verb in these contexts, viz. integrated\is{integration} in the host sentence.


\subsection{Type of \textsc{pac}}\label{sec:5.2}
To consider the types of \textsc{pac}s.\is{adverbial clause} In this section, I will first discuss the situation in the \ili{German} Ponthus and subsequently that in the \ili{Dutch} text. Thereafter, I will compare the two and discuss some implications.

Not all types of \textsc{pac}s\is{adverbial clause} show variation regarding the type of construction they occur in. Of course, those that only occur once in the data set exhibit no variation. These are \textit{als liep} `as dear', \textit{sint das} `since', \textit{sonst} `otherwise if', \textit{warumb} `why', and \textit{wol} `although', which occur once with \textit{so}\is{resumptive}; \textit{nyt lange} `not long', which is followed \textsc{temp}\is{resumptive}; and \textit{wie stark} `how strong', which has a following subject. \textit{Auf das} `so that' (2) and \textit{wie wol} `although' (5) are furthermore only attested with \textit{so}\is{resumptive} in the \ili{German} Ponthus. The other types of \textsc{pac}s\is{adverbial clause} are attested in more than one of the constructions.

The most frequent types of \textsc{pac}s\is{adverbial clause} that occur ten times or more in the \ili{German} Ponthus are \textit{da}\is{resumptive} `then', V1,\is{verb-first} \textit{wan} `if, when', \textit{so bald} `as soon as', \textit{ob} `if', \textit{wo} `where, when, if' and \textit{ee} `before'. \textit{Da, so bald} and \textit{ee} typically introduce temporal adverbial clauses,\is{adverbial clause} whereas \textit{ob-, wo-} and V1-clauses\is{verb-first} mainly express conditionality. \textit{Wan}-clauses are often ambiguous between the two readings. They all occur in more than one construction. Considering both the type of \textsc{pac}\is{adverbial clause} and \isi{narrative speed}, I will first illustrate by means of \isi{conditional inference tree}s the trends that are shown in the data. Then I will use random forests to predict the realization of the prefield. For this model, the input data only contains the most frequent types of \textsc{pac}s,\is{adverbial clause} which are \textit{da,} V1,\is{verb-first} \textit{wan, so bald, ob, ee,} and \textit{wo} for \ili{German}.

\figref{fig:GE5} presents a \isi{conditional inference tree} for the \ili{German} Ponthus. The model's first split is between \textit{da}\is{resumptive} and \textit{so bald} on the one hand, and \textit{ee, ob,} V1,\is{verb-first} \textit{wan,} and \textit{wo} on the other (node 1). In the latter group (node 7), \textit{wo-}clauses are separated from the other, because they occur less often followed by \textsc{subj} and more often by \textsc{temp}\is{resumptive} than the others. The group with \textit{da}\is{resumptive} and \textit{so bald} has a comparatively high proportion of \textsc{temp}\is{resumptive} and {--}. Within this group, the model identifies a separation in the data regarding \isi{narrative speed} (node 2): Clauses that are introduced by \textit{da} or \textit{so bald} that encode isochronous\is{isochrony} narrative progression are found with \textit{so}\is{resumptive}; when they encode pauses, shifts and summaries, they occur more often with \textsc{temp}\is{resumptive} and --. Specifically, \textit{da}- and \textit{so bald}-clauses that encode summaries may occur with a following subject and more often occur directly adjacent to the finite verb than the pauses and shifts (node 4). Still, \textsc{temp}\is{resumptive} is the preferred pattern for both.

\begin{sidewaysfigure}[p]
\caption{Conditional inference tree: Prefield filling in the German Ponthus}\label{fig:GE5}
	\includegraphics[width=\linewidth]{figures/BloomFig7.png}
\end{sidewaysfigure}

\begin{sidewaysfigure}[p]
\caption{Conditional inference tree: Prefield filling in the Dutch Ponthus}\label{fig:NL5}
	\includegraphics[width=\linewidth]{figures/BloomFig8.png}
\end{sidewaysfigure}

In \ili{Dutch}, the included types of \textsc{pac}s\is{adverbial clause} in the model were only \textit{als,} V1,\is{verb-first} \textit{doen} and \textit{nadien (dat)}, because they were the only ones that occurred with a frequency of ten or more (and showed variation). The \isi{conditional inference tree} can be found in \figref{fig:NL5}. The first split (node 1) in the data that the model identifies is between V1-clauses\is{verb-first} and those that are introduced by \textit{als, doen} and \textit{nadien (dat)}. The V1-clauses\is{verb-first} show a relatively low proportion of \isi{integration}, a high amount of \textsc{subj}, and a slightly higher proportion of \textit{so}\is{resumptive} following than the other clauses. Different from what we have seen in \figref{fig:3}, \isi{narrative speed} is relevant here for the relation of the prefield: In \isi{isochrony}, \textit{so}\is{resumptive} is more frequent than in pauses and shifts (node 2). Finally, summaries are realized differently than pauses and shifts as they have a higher proportion of \isi{integration} and fewer occurrences of \textit{so}\is{resumptive}, \textsc{subj} and \textsc{temp}\is{resumptive}.

The findings from these models show that \isi{narrative speed} affects the realization of the prefield in both languages within the subset of temporal adverbial clauses.\is{adverbial clause} 
Where narrative summaries are encoded by \textsc{pac}\is{adverbial clause} + \textsc{temp}\is{resumptive} in \ili{German}, \textsc{pac}s\is{adverbial clause} occur directly adjacent to the finite verb in \ili{Dutch}. To evaluate whether these findings are stable, I ran two random forest models; one for each language. 2000 trees are calculated (ntree = 2000), each selecting one variable (mtry = 1).

Table \ref{tab-acc} presents the accuracy matrices resulting of the models. They present the observed distribution and the distribution as predicted by the model.

\begin{table}
\caption{Random forest accuracy matrices}\label{tab-acc}
\begin{subtable}[c]{0.65\linewidth}
\subcaption{German}
\centering
\begin{tabularx}{\textwidth}{Xrrrrr}
\lsptoprule
& {--} & {\textit{so}}& {subject} & {\textsc{temp}}& {observed}\\
\midrule
-- & \textbf{0} & 10 & 0 & 18 & 28\\
\textsc{so} & 0 & \textbf{77} & 0 & 0& 77\\
subject & 0 & 60 & \textbf{0} & {31}& 91 \\
\textsc{temp} & 0 & 2 & 0 & \textbf{154}& 156\\
predicted & 0 & 150 & 0 & 202&\\
\lspbottomrule
\end{tabularx}
\end{subtable}

\begin{subtable}[c]{0.65\linewidth}
\subcaption{Dutch}
\centering
\begin{tabularx}{\textwidth}{Xrrrrr}
\lsptoprule
& {--} & {\textit{so}}& {subject} & {\textsc{temp}}& {observed}\\
\midrule
-- & \textbf{193} & 8 & 3 & 0 &204 \\
\textsc{so} & 7 & \textbf{15}& 9 & 0& 31\\
subject & 12 & 3 & \textbf{38} & 0&53 \\
\textsc{temp} & 9 & 0 &1 & \textbf{0}& 10\\
predicted & 221 & 26 & 51 & 0&\\
\lspbottomrule
\end{tabularx}
\end{subtable}
\end{table}

Overall, the \ili{German} model erroneously predicts that there are no cases in which the \textsc{pac}\is{adverbial clause} and the finite verb are directly adjacent (--), nor any cases of subjects in the prefield. However, it accurately predicts all instances of \textit{so}, and it only erroneously classifies two cases of \textsc{temp}\is{resumptive} as \textit{so}. The model thus overestimates the use of the \isi{correlative} patterns. Interestingly, 64.29\% of the \textsc{pac}s\is{adverbial clause} adjacent to the finite verb are classified as \textsc{temp}\is{resumptive}, whereas 65.93\% of \textsc{subj} are classified as \textit{so}\is{resumptive}. 

The \ili{Dutch} model does not predict any \textsc{temp}\is{resumptive} and slightly over-predicts \isi{integration}. It accurately predicts 94.61\% of \isi{integration}, which drops to 71.70\% for \isi{juxtaposition} with a subject in the prefield, and to a meager 48.39\% for \textit{so}\is{resumptive}. \textsc{temp}\is{resumptive} is mainly predicted to be integrated\is{integration} (90.00\%). This is the case for the wrongly classified \textsc{subj} as well (22.64\% in total, 80.00\% of wrong predictions). \textit{so}\is{resumptive} is more often wrongly predicted to be \textsc{subj} rather than integrated\is{integration}, but this difference is small (16 vs. 15 wrong predictions).

These matrices provide insight as to how accurately the random forests predict the element following the \textsc{pac}\is{adverbial clause} based on two factors: \isi{narrative speed} and the type of \textsc{pac}.\is{adverbial clause} For \ili{German}, the model reliably identifies \textit{so}\is{resumptive} and \textsc{temp}\is{resumptive}, but it does not identify -- and \textsc{subj}. In contrast, in \ili{Dutch} \textsc{temp}\is{resumptive} is not predicted at all. This suggests that the factors that motivated the use of \textsc{temp}\is{resumptive} in \ili{German} do not do the same in \ili{Dutch}. Instead, \isi{integration} is predicted with a high accuracy, which is fully in line with the hypothesis that where the \ili{German} Ponthus uses \textsc{pac}\is{adverbial clause} + \textsc{temp}\is{resumptive}, \ili{Dutch} prefers integrated\is{integration} \textsc{pac}s.\is{adverbial clause} Moreover, \isi{juxtaposition} seems to be associated with particular settings in \ili{Dutch} (the tree in \figref{fig:NL5} suggests V1\is{verb-first} and pauses/shifts if not V1\is{verb-first}) but not in \ili{German}. This can reflect that \isi{juxtaposition} has developed its own niche now that \isi{integration} has become the default in \ili{Dutch} but that \ili{German} has not (yet) fully undergone this process.


\subsection{Returning to Dutch: Integration versus \textit{so}}
In the previous section, it was argued that where the \ili{German} Ponthus uses \textit{da}-resumption\is{resumptive}\is{resumptive}, \ili{Dutch} tends to use the \isi{integration} of \textsc{pac}s.\is{adverbial clause} This means that the findings for \ili{Dutch} in \sectref{sec:4} need to be revised for a complete picture, as it is not \textsc{temp}\is{resumptive} that contrasts with \textit{so}\is{resumptive} but \isi{integration}. For this reason, I present one final \isi{conditional inference tree}, which parallels the \ili{German} one presented in \figref{fig:2}. Here, all types of \textsc{pac}s\is{adverbial clause} are included.

\begin{figure}
\caption{Conditional inference tree: \textit{so} versus integration in the Dutch Ponthus}\label{fig:NL52}
	\includegraphics[width=\linewidth]{figures/BloomFig9.png}
\end{figure}

\figref{fig:NL52} illustrates that the two constructions contrast regarding \isi{narrative speed}, with \textit{so}\is{resumptive} being considerably more frequent with \isi{isochrony}. With \linebreak[4]summaries, the preference for \isi{integration} is strong.\footnote{Since this data only contains one pause and five shifts, I do not consider this strong evidence.} This motivates the first split in the tree. Second, within the group of clauses that encode pauses or summaries, most of the types of adverbial clauses\is{adverbial clause} do not, or only rarely, occur with \textit{so}\is{resumptive}. This preference for \isi{integration} is significantly stronger with \textit{als, mettien dat, nadat, nadien (dat)} and \textit{niet lang} than with V1\is{verb-first} and \textit{doen}. This split reflects partly what we saw for \ili{German} in \sectref{sec:4}: Both the type of \textsc{pac}\is{adverbial clause} and the \isi{narrative speed} are important factors. Yet, different constructions are associated with narrative summaries and \isi{isochrony}. Moreover, the variation in the types of \textsc{pac}s\is{adverbial clause} was greater in \ili{German} than in \ili{Dutch} and \textsc{pac}\is{adverbial clause} + \textit{so}\is{resumptive} was overall more frequent.

\section {Conclusion}\label{sec:6}
This paper investigated how preposed adverbial clauses\is{adverbial clause} (\textsc{pac}s)\is{adverbial clause} are used to structure the narrative in the \ili{Early New High German} Ponthus and compares the results to what is found in the \ili{Dutch} adaptation.

The first part of the paper focuses the two \isi{correlative} constructions, as \textsc{pac}s\is{adverbial clause} were very often followed by \textit{da}\is{resumptive} or \textit{so}\is{resumptive} in \ili{Early New High German}. Their contrasting uses regarding narrative structure were evaluated. Although \textit{so}\is{resumptive} is predominantly found in \isi{mimesis}, viz. dialogic segments, and \textsc{temp}\is{resumptive} in diegetic segments, which are voiced\is{voice} by the \isi{narrator}, it is argued that this is an artifact of a difference in \isi{narrative speed}. Specifically, \textit{so}\is{resumptive} is associated with \isi{isochrony}, which slowly progresses the story, and \textsc{temp}\is{resumptive} with narrative summaries, which progress it rapidly. This effect is observed even when including the type of \textsc{pac}\is{adverbial clause} as a predictor variable. 
The \ili{Dutch} Ponthus had, differently than \ili{German}, a remarkably low amount of \textsc{temp}\is{resumptive}. The results were consequently not very convincing. The second part of the paper therefore considers \textsc{pac}s\is{adverbial clause} in all constructions in the two Ponthus-texts. That is, it includes subjects following the \textsc{pac},\is{adverbial clause} direct adjacency of the \textsc{pac}\is{adverbial clause} and the finite verb, and ``other'' elements that stand between the \textsc{pac}\is{adverbial clause} and the finite verb of the host. The main result is that where \ili{German} uses temporal resumption in narrative summaries, \ili{Dutch} uses \isi{integration}. Moreover, the sequence \textsc{pac}\is{adverbial clause} + \textsc{subj} is more easily predictable in \ili{Dutch} than in \ili{German} on the basis of the type of \textsc{pac}\is{adverbial clause} and \isi{narrative speed}, which suggests that it has developed its own niche in \ili{Dutch}. Seeing that the \ili{Dutch} Ponthus was published about a century later than the \ili{German} manuscript, the differences between the languages may be a reflection of a larger diachronic development, but to draw any reliable conclusions other types of data will need to be considered.

It should be noted, however, that the current study investigates the use of the \textsc{pac}s\is{adverbial clause} in two adaptations of one and the same text. As such, the findings may be an artifact of different translation strategies or idiosyncrasies of the individual translators rather than a reflection of conventionalized patterns. Further research based on a broader base is needed to verify whether the relation between \isi{narrative speed} and the realization of the prefield as well as the contrast between \ili{German} and \ili{Dutch} is represented in narrative prose or the languages more generally. In addition, the verbal \isi{tense} and mood seemingly interact with both \isi{narrative speed} and the type of adverbial clause. Further research in this interplay would likely be fruitful and provide deeper insight into the issue at hand. In any case, the paper forms a first basis to the interaction between the realization of the prefield following preposed adverbial clauses\is{adverbial clause} and narrative structure. It illustrates how the different constructions in which preposed adverbial clauses\is{adverbial clause} may occur can be used to temporally progress the story, and how the \ili{Dutch} and \ili{German} Ponthus use similar strategies but different realizations to do so.



\section*{Abbreviations}
Glosses follow the Leipzig Glossing rules.\\

\begin{tabularx}{.45\textwidth}{lQ}
\textsc{cnj} & conjunction \\
\textsc{disc} & discourse marker \\
DU & Dutch \\
FR & French \\
GE & German\\
iso & isochrony \\
\textsc{neg} & negation\\
\end{tabularx}
\begin{tabularx}{.5\textwidth}{lQ}
\textsc{pac} & preposed adverbial clause \\
\textsc{sbjv} & subjunctive\\
\textsc{subj} & subject\\
sum & summary\\
\textsc{temp} & temporal adverb\\
V1 & verb-first \\
V3 & verb-third \\
\end{tabularx}

\section*{Acknowledgements}
This research was funded by the Deutsche Forschungsgemeinschaft (ProjectID 456973946, ``Wortstellung und Diskursstruktur in der Frühen Neuzeit'').

\section*{Primary sources}
\begin{hangparas}{.25in}{1}
\textbf{German Pontus und Sidonia.} (3rd quartile 15th c.). \textit{Pontus und Sidonia. Tandareis und Flordibel - BSB Cgm 577.} München, Bayerische Staatsbibliothek. \url{https://www.digitale-sammlungen.de/en/view/bsb00014877}. Transcribed as part of the Romankorpus Früneuhochdeutsch.

\textbf{Dutch Ponthus ende Sidonie.} 1564. \textit{Een schoone ende amoruese historie van Ponthus ende die schoone Sydonie.} Transcribed by Willem Kuiper \& Paul Jacob Brüggeman. 2010. Diplomatic edition. Amsterdam: Bibliotheek van Middelnederlandse Letterkunde.


\textbf{French Ponthus.} 1479. \textit{Le livre de Ponthus filz du roy de Galice et de la belle Sydoine fille du roy de Bretaigne.} Gallica, Bibliothèque nationale de France, département Réserve des livres rares, RES-Y2-567. \url{https://gallica.bnf.fr/ark:/12148/btv1b8600082d/}

\end{hangparas}

\bigskip

 {\sloppy\printbibliography[heading=subbibliography,notkeyword=this]}

\newpage
\section*{Appendix}\label{bloom:app}

\ea
\gll Sie sprachen 
da were vil abe zu sagen Dan siner schonheit weydelicheit \sout{$were$} vnd sins gutten gela\ss{} were nit me in der werlt
als \uline{sie} wol gleuben wolten williche frauwe oder jungffrauwe yne zu bolen mochte haben da ist kein rede in Sie mocht wol berummen das sie den schonsten den besten den wolkommesten vnd den aller lustigesten frunt hette der da lebt vnd schon ist sin jugent
\textbf{kompt} er also inn das alter so \uline{wart} sin gliche nie me ersehen das \uline{ist} vngezwiuelt die rechte warheit\\
 they said 
there was.\textsc{sbjv} much about to say because his beauty proficiency \sout{$was.\textsc{sbjv}$} and his good behavior was.\textsc{sbjv} not more in the world 
as they \textsc{disc} believe wanted which woman or lady him to court might have there is no reason in she might \textsc{disc} state that she the most.beautiful the best the most.considerate and the of.all amusing friend have.\textsc{sbjv} who there lives and beautiful is his youth 
comes he thus in the age \textsc{so} became his equal never more seen that is undoubtedly the right truth\\
 \glt `They said that there was much to say about that, because his beauty, proficiency, and good manners were unequaled in this world. As \uline{they} would believe whichever woman or lady might court him -- there is no reason in that -- she might proudly profess that she would have the most beautiful, the best, the most considerate, and the uttermost amusing friend in the world. And beautiful is his youth. \textbf{Comes} he of age, his equal was never to be seen again. That \uline{is} undoubtedly the truthful truth.'\\ \hfill[GE, 16rb]\footnote{The crossed-out \textit{were} is written in the handwriting. This is an accidental doubling.}
\z

\end{document}
