\documentclass[output=paper,colorlinks,citecolor=brown]{langscibook}
\ChapterDOI{10.5281/zenodo.15689131}
\author{Malika Reetz\orcid{0009-0004-7923-0606}\affiliation{Universität Potsdam}}
\title[Argument realizing V2-clauses in Early New High German narratives]{Argument realizing verb-second clauses in Early New High German narratives: A discourse-pragmatic account}
\abstract{This paper presents a corpus study of argument realizing verb-second clauses in Early New High German narratives from a discourse pragmatic point of view. By means of a conditional inference tree, it will be shown that there are two pragmatic types of argument realizing verb-second clauses at this point in time: One is licensed by mediated assertivity, the other by discourse progressiveness. Furthermore, the mapping between these discourse pragmatic properties and the syntactic form of a clause is -- depending on the respective text -- considerably systematic, which will be proposed to be the relevant factor behind the assumed higher frequency of dependent V2 clauses in Early New High German, compared to present-day German.}

\begin{document} 
\maketitle

\section{Introduction}\label{sec:reetz:1}
In present-day German (PDG), there is a formal distinction between asyndetic verb-second structure (henceforth: V2\is{verb-second}) and syndetic verb-final\is{verb-final} structure (henceforth: VF\is{verb-final}) in declarative clauses. V2\is{verb-second} is broadly used in main clauses, but generally considered to be the non-canonical form of dependent clauses, especially in written language \citep{Auer1998, Axel2012}. However, regarding argument realizing clauses, there is variation between these two syntactic forms for some semantic classes of matrix verbs. Representatives of these are verbs that denote an act of \isi{assertion} \REF{ex:reetz:1}, like \textit{sagen} ‘saying' or \textit{denken} ‘thinking'. In contrast, V2\is{verb-second} is generally blocked under factive-emotive verbs \REF{ex:reetz:2}, which inherently presuppose\is{presupposition} the proposition in the argument clause.


\ea\label{ex:reetz:1}
\ea \label{ex:reetz:1a}
\gll Amanda sagte, \textbf{dass} sie Paul nicht gesehen \textbf{habe}\\
     Amanda said \textsc{cop} she Paul \textsc{neg} seen has.\textsc{aux}.\textsc{sbjv}\\
\ex \label{ex:reetz:1b}
\gll Amanda sagte, sie \textbf{habe} Paul nicht gesehen\\
     Amanda said she has.\textsc{aux}.\textsc{sbjv} Paul \textsc{neg} seen\\
\glt `Amanda said that she hadn't seen Paul.'
\z
\z

\ea \label{ex:reetz:2}
\ea \label{ex:key:2.1}
\gll Amanda bedauerte, \textbf{dass} sie Paul nicht gesehen \textbf{hatte}.\\
     Amanda regretted \textsc{comp} she Paul \textsc{neg} seen had.\textsc{aux}\\
\ex \label{ex:key:2.2}
\gll *Amanda bedauerte, sie \textbf{hatte} Paul nicht gesehen.\\
     Amanda regretted she had.\textsc{aux} Paul \textsc{neg} seen\\
\glt `Amanda regretted that she hadn't seen Paul.'
\z
\z

The semantics of the respective matrix verb is generally assumed to be indicative of the pragmatic features controlling the use of argument realizing verb-second clauses (arg.V2\is{verb-second}). Numerous people espouse the notion of \textit{assertion}\is{assertion} as the relevant notion to account for arg.V2\is{verb-second} in PDG (\citealt{Reis1997}, \citealt{Auer1998}, \citealt{Gärtner2002}, \citealt{Meinunger2006} amongst others). Yet, it has also been observed that in PDG, there is no iconism between the syntax and pragmatics of a dependent clause: The arg.V2\is{verb-second} is always an optional variant of the  (canonical) VF\is{verb-final}-clause \citep{Reis1997}.

Regarding earlier stages of \ili{German}, especially the periods preceding \ili{Early New High German} (ENHG), it has been shown that syntactic variation used to be affected by discourse pragmatic categories to a higher degree than in PDG \citep{hinterholzl2010v1, speyer2010}. However, the question on how argument realizing clauses fit into the picture has not been taken into account in these discussions.

In ENHG\il{Early New High German}, a period which is assumed to constitute a transitional phase in the standardization of \ili{German} written language \citep{Betten2000, polenz2000}, a similar syntactic variation in argument clauses in prose texts can be found: 

\ea
\ea \label{ex:reetz:3}
\gll die diener sagten zu dem graffen das er im ain cronen zů zerung g\oldae{}b  \\
     The servants said to the.\textsc{dat} count \textsc{cop} he him a krona to consumption gives.\textsc{sbjv}\\
\glt `The servants told the count that he had given him a krona for consumption.' \hfill [FORT, 436]

\ex \label{ex:reetz:4}
\gll sagt er hab mir ainen grossen sold geben\\
     said he has.\textsc{sujv} me.\textsc{dat} a big pay given\\
\glt `[He] said he had given me a big pay.' \hfill [FORT, 410]
\z
\z

As operationalized by means of the arg.V2\is{verb-second}-selecting matrix verbs, it has been argued that the arg.V2\is{verb-second} is historically old \citep{Axel2012}, and its distribution in Old-\il{Old High German} and \ili{Middle High German} “does not differ from that of PDG in a radical way” \citep[556]{Petrova2020a}. Against the background that syntactic variation is more strongly affected by discourse pragmatics in earlier stages of \ili{German} than in PDG, the question arises whether arg.V2\is{verb-second} is used much more systematically in ENHG\il{Early New High German} written text than in PDG written text. Since arg.V2\is{verb-second} can be assumed to be mainly a phenomenon of spoken language \citep{Auer1998}, this question is especially relevant from a language modality perspective.

This present study aims at adding to the theoretical field that is concerned with the mapping of pragmatic categories and syntactic (clause) form in earlier stages of \ili{German}. By means of a \isi{conditional inference tree}, argument clauses of ENHG\il{Early New High German} narratives will be arranged according to the pragmatic features that are considered to affect their form in previous literature. Whereas \isi{assertion} is often assumed to be the relevant notion to define the licensing requirements, theories on where the locus of the pragmatic \isi{assertion} lies offer different perspectives to account for the V2\is{verb-second} syntax in the argument clause: Seeing that it is generally matrix verbs denoting an act of \isi{assertion} that embed arg.V2\is{verb-second}, mediated assertivity\is{assertion} has been deemed the relevant pragmatic requirement for the V2\is{verb-second} syntax \citep{Reis1997, Gärtner2002}. This term refers to the fact that the embedded proposition is mediated by a speaker who is not the originator of the \isi{assertion}. In line with this, the possibility to mark a dependent clause with V2\is{verb-second} in Germanic is often directly associated with the semantics of the matrix verb. Primarily, this goes back to \citet{HooperHooper1973}. More recent approaches consider the reference level of the speaker/\isi{narrator} (the mediator of the embedded \isi{assertion}) as the relevant level on which – depending on the respective theory – the \isi{assertion} is made \citep{Meinunger2006} or moved to in order to be marked as informatively relevant  \citep{Djärv2022} to account for the V2\is{verb-second} syntax. Note that from this point of view, the assessment of the distribution of arg.V2\is{verb-second} on the basis of the matrix verb is only indirectly taking into account the pragmatic requirements of the dependent syntax.

Since both the “assertive levels” mentioned above come into question as being crucial for the licensing of arg.V2\is{verb-second}, they are both included into the analysis. Therefore, the analysis is not merely based on the matrix verb, but on the pragmatic features mentioned here.

The paper is organized as follows: In Section \ref{sec:reetz:2}, the theoretical approaches are outlined in more detail and under consideration of some representative concepts that will be fundamental for the further investigation. Section \ref{sec:reetz:3} moots the value of the historical approach to the topic and summarizes the state of research on the diachrony of the arg.V2\is{verb-second}. Section \ref{sec:reetz:4} outlines the methodology, and in Section \ref{sec:reetz:5}, the results of the corpus study are presented. In Sections \ref{sec:reetz:6} and \ref{sec:reetz:7}, the results are summarized and discussed. 

\section{Theoretical approaches}\label{sec:reetz:2}

As has been mentioned in the introduction, \textit{assertion}\is{assertion} is commonly considered the relevant notion to account for arg.V2\is{verb-second}. In speech act theory, “[t]he point or purpose of the members of the assertive class is to commit the speaker (in varying degrees) to something's being the case, to the truth of the expressed proposition” \citep[12]{Searle1975}. However, there is more than one possible level of an \isi{assertion} that has to be taken into account with regard to argument realizing clauses. Consider \REF{ex:reetz:2} (repeated here as \ref{ex:reetz:5})

\ea\label{ex:reetz:5}
\gll Amanda sagte, sie \textbf{habe} Paul nicht gesehen\\
     Amanda said she has.\textsc{aux}.\textsc{sbjv} Paul \textsc{neg} seen\\
\glt `Amanda said that she hadn't seen Paul.'
\z

First, there is the level of the matrix subject \textit{Amanda} that is the originator of the \isi{assertion}, and second, there is the reference level of the speaker/\isi{narrator} that functions, in any case, as the mediator of the \isi{assertion}. Seeing that it is mainly verbs denoting an act of \isi{assertion} that are able to select arg.V2\is{verb-second}, some researchers have drawn the conclusion that a \textit{mediated} \isi{assertion} yields the relevant driving force behind the V2\is{verb-second} syntax \citep{Reis1997, Gärtner2002}. This notion refers to the fact that the asserted truth value of the dependent proposition is originally anchored at a matrix subject and is (only) mediated by the speaker. According to \citet[142]{Reis1997}, who coined the term \textit{mediated assertion}\is{assertion} (“vermittelte Assertion”), the declarative mode in arg.V2\is{verb-second} complies with an “assertive restriction” that the matrix verb projects on its propositional complement. Correspondingly, the matrix verb invokes an alternative mental world of the matrix subject \citep[122]{Reis1997}. Reis concludes that arg.V2\is{verb-second} are cases of non-structural theta-marking. Similarly, \citet[40]{Gärtner2002} states that arg.V2\is{verb-second} have “embedded assertional proto-force” that “can be ‘absorbed' on arguments of predicates that denote acts of assertion”.\is{assertion} In line with these approaches, arg.V2\is{verb-second} are commonly analyzed with regard to their selecting matrix verbs.

These approaches have frequently been adapted and further developed in the literature. From a more discourse-pragmatic perspective, it has been recognized that arg.V2\is{verb-second} do not appear as flexibly under verbs of saying as the matrix-verb-approach would predict. For instance, \citet{Meinunger2006} shows that V2\is{verb-second} syntax is unacceptable under direct repetition of a proposition. This holds true even if the proposition is selected by a verb of saying, which is illustrated in \REF{ex:reetz:6} (slightly adapted from \citealt[7]{Meinunger2006}).

\ea \label{ex:reetz:6}
\glt A: {\itshape Bernd ist endlich gekommen}! \\
    `Bernd has finally arrived!'\\
    B:
\ea \label{ex:reetz:7}
\gll Ja, mir wurde schon gesagt, dass Bernd endlich gekommen ist.  \\
     Yes me.\textsc{dat} was already told \textsc{comp} Bernd finally come.\textsc{pst}.\textsc{ptcp} is.\textsc{aux}\\
\ex \label{ex:reetz:8}
\gll *Ja, mir wurde schon gesagt, Bernd ist endlich gekommen\\
     Yes me.\textsc{dat} was already told Bernd is.\textsc{aux} finally come.\textsc{pst}.\textsc{ptcp}\\
\glt `Yes, I have already been told that Bernd has finally arrived.'
\z
\z
It is restrictions like this that have led researchers to take on a more discourse-pragmatic approach as an explanatory framework of arg.V2\is{verb-second} licensing. \citet[2]{Meinunger2006} suggests, for instance, that V2\is{verb-second} can optionally mark a dependent clause that the speaker “intends to assert”, and furthermore, “these dependent verb-second clauses have illocutionary force and mark new information” (ibid: 15). In a reverse conclusion, Meinunger assumes \isi{discourse-old} propositions to be obligatorily realized as syndetic VF\is{verb-final}-clauses (canonical dependent clauses), because “[t]hese belong to the background and are hence infelicitous candidates to make an \isi{assertion} with” (ibid.).

\citet{LohnsteinStaratschek2020} present a somewhat weaker version of the speaker-assertion-hypothesis,\is{assertion} claiming that “[t]he process of fronting finiteness has (…) exactly the same effect as in root clauses, with the exception that the speaker is not liable, but gives the information in the name of the matrix subject's referent” \citep[138]{LohnsteinStaratschek2020}. In that vein, the illocutionary force is not on the side of the speaker, but the mediated proposition is still relevant with respect to the common ground between the speaker and the addressee. Similarly, \citet{Djärv2022} hypothesizes that arg.V2\is{verb-second} “represent discourse moves\is{discourse move} whereby the speaker adds the issue represented by the embedded clause to the conversational table as an issue for discussion”.\is{at-issueness}

Other researchers (to some degree in line with the speaker--assertion--\linebreak[4]hypothesis) \is{assertion} consider the pragmatic distinction between the background and the foreground of a discourse to be crucial for the syntactic variation of dependent clauses. \citet{Antomo2016} argues that in \ili{German}, V2\is{verb-second} can only be applied to clauses that have so-called \textit{at-issue} content,\is{at-issueness} and that V2\is{verb-second} is, reversely, excluded from all clauses that do not. Antomo aligns her conception of \textit{at-issueness} with \citet{KleinStutterheim1992}, \citet{roberts1996}, and \citet{Simons2010} who share the opinion that the primary goal of each discourse is to answer a \textit{Question under Discussion}\is{Question under Discussion} (in Roberts' words), which is typically implicit in a discourse. In line with this, \citet{Simons2010} state that all content that is relevant with regard to the current \isi{Question under Discussion} is at-issue-content.\is{at-issueness} Corresponding to this notion of at-issueness, \citet{Antomo2016} states that all at-issue-content\is{at-issueness} might possibly be marked by V2\is{verb-second} in \ili{German}.

Whatever pragmatic factor is taken to be the relevant requirement to account for arg.V2\is{verb-second} does at the same time imply which factor is taken to be the ``blocking factor'' of arg.V2\is{verb-second}. \citet{Meinunger2006}, who considers discourse newness\is{discourse-new} a relevant notion to account for arg.V2\is{verb-second}, makes it explicit by stating that \isi{discourse-old} propositions are necessarily realized as VF\is{verb-final}-clauses because they are backgrounded and cannot be used for assertion\is{assertion} \citep[15]{Meinunger2006}. Discourse-old\is{discourse-old} information is sometimes defined as the epitome of the notion of \textit{presupposition}\is{presupposition} \citep{lambrecht1996}. In \citeauthor{Antomo2016}'s (\citeyear{Antomo2016}) account, on the contrary, it is non-at-issue content\is{at-issueness} that is diametrically opposed to at-issueness\is{at-issueness} and that therefore blocks arg.V2\is{verb-second}. Once again, she refers to \citet{Simons2010}, who state that non-at-issue content\is{at-issueness} is information that is peripheral to the conversational goal. These different notions of “V2\is{verb-second}-blocking contexts” are not precisely congruent, but this does not entail that they have to be mutually exclusive. As \citet[33]{Antomo2016} claims, “presuppositions\is{presupposition} form a subset of not-at-issue material”.\is{at-issueness} Since the aim of this study is to find out how different factors interact, it is important to keep the different conceptions of “V2\is{verb-second} blockers” in mind.

\section{A historical perspective}\label{sec:reetz:3}

Some researchers assume that in historical \ili{German}, even still in ENHG\il{Early New High German}, syntactic variation is affected by discourse pragmatics to a higher degree than in PDG \citep{fleischmann1973,hinterholzl2010v1,speyer2010}. \citet{hinterholzl2010v1} model their theory within the frame of Segmented Discourse Representation Theory (SDRT) \citep{AsherLascarides2003}. They argue that in OHG\il{Old High German}, an older system in which the verb functions as a means “to separate the \isi{aboutness topic} plus other background elements from the \isi{focus} domain of the clause” \citep[5]{hinterholzl2010v1} is still clearly recognizable. According to them, V1-clauses\is{verb-first} served to introduce new\is{discourse-new} discourse referents, whereas V2\is{verb-second}-clauses “serve to provide additional information about an already established discourse referent” \citep[2]{hinterholzl2010v1}.\footnote{This latter \isi{discourse relation} is called \textit{elaboration} in SDRT.} \citet{speyer2010} adds to this and states that in the ENHG\il{Early New High German} period, a shift occurred in which V2\is{verb-second}-clauses began to be used for purely “rhematic” sentences as well. In both these accounts, it is not explicitly considered how argument realizing clauses fit into the picture, but it implies that the VF\is{verb-final}-variant was historically reserved for clauses that are informationally completely backgrounded, and hence more restricted than their PDG structural equivalents. A relevant account for the situation in ENHG\il{Early New High German} comes from \citet{fleischmann1973}: He argues that the asymmetry of main and dependent clauses is primarily reflective of pragmatic categories.

A hypothesis that is frequently discussed in the literature on the historical development of dependent verb second is that the syntactic variation between V2\is{verb-second} and VF\is{verb-final} is attributable to a universal, hence diachronically robust, principle of the association between a pragmatic feature and a syntactic feature (amongst others, \citealt{Auer1998}, \citeauthor{Petrova2020a} \citeyear{Petrova2020b}, \citeyear{Petrova2020a}). With regard to the “original” type of the PDG arg.V2\is{verb-second}, \citet[187]{Axel2012} argues that already the earliest attestations of the arg.V2\is{verb-second} in \ili{Old High German} seem to be restricted to matrix verbs denoting an \isi{assertion}, which is indicative of a clause type that is historically old and diachronically unrelated to the verb-final\is{verb-final} dependent clause (ibid).

As has been mentioned earlier, previous historical studies on the distribution of arg.V2\is{verb-second} have primarily concentrated on the selecting matrix verbs. The arg.V2\is{verb-second} is documented by \citet{Axel2012} mainly with assertive verbs  in the MHD period, which leads her to argue that the pragmatic licensing of arg.V2\is{verb-second} might, in fact, be diachronically constant (ibid: 173). \citeauthor{Petrova2020a}'s (\citeyear{Petrova2020b, Petrova2020a}) data from Old\il{Old High German} and \ili{Middle High German} support Axel-Tober's hypothesis, as the verbs that Petrova finds to predominantly select arg.V2\is{verb-second} are verbs of saying and thinking, whereas factive-emotive verbs are only found with the VF\is{verb-final}-variant.

In fact, there is more than one possible origin scenario of the of the arg.V2\is{verb-second}. One would be that the original pattern was a clause pragmatically relatively independent of and only loosely connected to the matrix clause. This scenario has been discussed in detail in previous literature. For example, \citet{Demske2019} states that up until the 16th century, clauses of \isi{indirect speech} that are only loosely connected to the speech markers in the preceding clause frequently occur in prose texts. She sees this type of “autonomous” \isi{indirect speech} as a sign for a strong \isi{character} perspective. The possibility that there has been a pattern in which the dependent V2\is{verb-second}-clause was pragmatically more integrated\is{integration} can, however, not be excluded. The PDG V2\is{verb-second}-variant(s) of argument clauses might even go back to two diachronically different types of (more or less) dependent verb second: one that is indisputably integrated\is{integration}, and might historically be marked as integrated\is{integration} by non-syntactic means such as verbal mood \citep{GärtnerEythórsson2020, ConiglioWeskott2021}, and one that is less integrated\is{integration}. From a syntactic perspective on PDG, \citet{Freywald2016} suggests that there is a truly embedded type, on the one hand, and a (syntactically) unembedded type, on the other hand. According to her, this bipartition accounts for structural differences like a more fine-grained structure of the left periphery of the unembedded type and its confinement to the right of the superordinate clause. Pragmatically, the coexistence of two types of arg.V2\is{verb-second} could be reflected in the coexistence of arg.V2\is{verb-second} that are attributed informative weight by the speaker/\isi{narrator}, on the one hand, and assertions\is{assertion} by a matrix subject that are only mediated but are left unspecified in the course of the \isi{narration}, on the other hand. Seeing that narratives “are multiperspectival since they prototypically integrate viewpoints of different characters\is{character} and narrators\is{narrator} and thus offer a set of possible alternate perspectives that allow for viewpoint switches" \citep[2]{Zeman2020a}, this assumed coexistence of arg.V2\is{verb-second}, which are driven by different pragmatic features, becomes even more plausible. I will come back to this scenario in more detail in Sections \ref{sec:reetz:4} to \ref{sec:reetz:6}.

Summing up, it has to be noted critically that a focus on matrix verbs of arg.V2\is{verb-second} provides a useful basis to analyze arg.V2\is{verb-second} historically but will only provide rather indirect insights into the pragmatic requirements of V2\is{verb-second} (and, by extension, VF\is{verb-final}). Therefore, I am going to operationalize the discussed pragmatic features of arg.V2\is{verb-second} to analyze the distribution of arg.V2\is{verb-second} in ENHG\il{Early New High German} narratives. How this is done is further explained in the following sections.

\section{Methodological approach}\label{sec:reetz:4}

\subsection{Corpus}\label{sec:reetz:1.1}
The data are collected from six narrative texts, five of which  are part of the \isi{Romankorpus Frühneuhochdeutsch} (Roko.UP) \citep{BloomEtAl2023} within the DFG-funded project \textit{Wortstellung und Diskursstruktur in der Frühen Neuzeit} at the University of Potsdam. For the sixth text, the Huge Schapler, the Roko.UP\is{Romankorpus Frühneuhochdeutsch} provides a different edition than the one used here. The edition used here is an (incomplete) handwriting, which has been chosen for this study because it has a considerably higher incidence of \isi{indirect speech} than the print which is provided by the Roko.UP.\is{Romankorpus Frühneuhochdeutsch} The utilized texts are given in Table \ref{tab:myname:texts}. As can be seen, the texts come from different narrative traditions, the Fortunatus being the only text that is an ENHG\il{Early New High German} original. Schöne Magelone, Melusine, and Huge Schapler have a French template text, and Wigalois and Tristrant und Isalde are prose versions of \ili{Middle High German} lyrical texts.

\begin{table}
\caption{Early New High German narratives}
\label{tab:myname:texts}
 \begin{tabularx}{0.95\textwidth}{lll}
  \lsptoprule
    text  & model text & dialect area\\
  \midrule
  Huge Schapler (HUG) & French  & Rhine Franconian\\
  Schöne Magelone (MAG) & French & Upper German\\
  Melusine (MEL) & French & Upper German\\
  Tristrant u. Isalde (TRIST) & Middle High German & Upper German\\
  Wigalois (WIG) & Middle High German & Upper German \\
  Fortunatus (FORT) & no model text & Upper German \\
  \lspbottomrule
 \end{tabularx}
\end{table}

Prose texts lend themselves well to the investigation of the discussed pragmatic properties because one \isi{narrator} tells a self-contained story, which makes the features discourse newness\is{discourse-new} and at-issueness\is{at-issueness} relatively easy to assess. The exact method is outlined in the section on the operationalization.

\subsection{Preliminaries}\label{sec:reetz:4.2}

To recapitulate, the study aims at shedding light on the pragmatic licensing conditions of arg.V2\is{verb-second} in ENHG\il{Early New High German} narratives. To extract the data, all argument realizing clauses have been subcategorized as either V2\is{verb-second} (when they are unintroduced with the verb in the left periphery) or VF\is{verb-final} (when they are introduced by the complementizer \textit{dass} `that'). The implication of this is, of course, that in the \ili{Early New High German} period, there is a bipartite pragmatic distinction between these two categories. This is based on the presumption that the complementizer \textit{dass} was originally restricted to a specific (pragmatic) clause type (for a discussion of origin scenarios of this complementizer, see \citealt[91--126]{Axel2012}). This procedure entails that syndetic clauses that have a post-verbal element will also fall into the VF\is{verb-final}-class.

My working hypothesis is that the distinction between VF\is{verb-final} and V2\is{verb-second}-clauses is pragmatically driven in ENHG\il{Early New High German}. As has been mentioned, to test the hypothesis, I am choosing pragmatic categories as predictor variables. This is further explained in the next subsection. I am mainly focusing the study on verbs of saying as a semantic class where there is variation in the pragmatic function of the argument clause, in other words, they are characterized by mediated assertivity\is{assertion} by nature but might be further influenced by informative weight. This has been explained in Section \ref{sec:reetz:2}.

To ensure that the main focus of the study is well-founded, I have conducted a preliminary step: I have collected argument clauses of verbs of saying as well as argument clauses of factive-emotive verbs. This was to ensure that the “general distribution” of arg.V2\is{verb-second} in ENHG\il{Early New High German} is indeed similar to the distribution in PDG, meaning that arg.V2\is{verb-second} are blocked under the same matrix verbs. If this is not the case, this would already imply a high flexibility of arg.V2\is{verb-second} in ENHG\il{Early New High German} that is not necessarily pragmatically restricted. Note that it has already been shown by \citeauthor{Axel2012}'s (\citeyear{Axel2012}) and \citeauthor{Petrova2020a}'s (\citeyear{Petrova2020a}) data that arg.V2\is{verb-second} do not occur under factive-emotive matrix verbs in OHG\il{Old High German} and MHG\il{Middle High German}, so it is highly unlikely that this drastically changes towards ENHG\il{Early New High German} before changing  back towards PDG.

After the extraction of argument clauses and the classification of their syntactic form, the clauses were annotated according to their matrix verbs, which I then grouped into different semantic classes (in line with, for example, \citealt{Petrova2020a}). As the two semantic classes I am focusing on are factive-emotive verbs and verbs of saying, the grouping of their matrix verbs is illustrated below.

\begin{enumerate}
\item Factive-emotive verbs, as they represent verbs that inherently presuppose\is{presupposition} the truth value of their dependent propositions and render them as discursively backgrounded. Verbs that fall under this class are for instance \textit{bereuen} `regret', \textit{genießen} ‘enjoy', or \textit{wehtun} `hurt'.
\item Verbs of saying, as representatives of verbs that denote an act of \isi{assertion}, whereby the proposition can either be discursively foregrounded or backgrounded, and there might be varying degrees of speaker commitment. Verbs that fall under this class are \textit{sagen} ‘saying', \textit{geloben} `swearing', or \textit{antworten} ‘respond'.
\end{enumerate}

The results of this survey show that the expectation holds true: There is striking syntactic variation under verbs of saying. Under factive verbs, on the contrary, only the VF\is{verb-final} variant can be accounted for, as \tabref{tab:myname:frequencies} shows.

\begin{table}
\caption{Distribution of arg.V2}
\label{tab:myname:frequencies}
 \begin{tabularx}{.65\textwidth}{lrrr}
  \lsptoprule
& \multicolumn{1}{l}{total} & \multicolumn{1}{l}{VF}  & \multicolumn{1}{l}{V2} \\
  \midrule
factive-emotive verbs   &   146  &    146  &    0 (0\%) \\
 verbs of saying & 362 &   149  &   213 (58.9 \%) \\
  \lspbottomrule
 \end{tabularx}
\end{table}

\subsection{Operationalization}\label{sec:reetz:4.3}
The second phase of the study focusses on argument clauses of verbs of saying in order to investigate their formal variation regarding their pragmatic properties. For this, the argument clauses have been annotated according to the concepts of pragmatic licensing that I have referred to in Section \ref{sec:reetz:2}. Argument clauses within sections of \isi{direct speech} are not taken into account, as \isi{direct speech} represents a different discourse situation.

In Section \ref{sec:reetz:2}, I have referenced different hypotheses on arg.V2\is{verb-second}. I am categorizing these hypotheses on the basis of the (implicit) assumptions about the locus of the \isi{assertion}. The group of hypotheses that considers the \isi{assertion} on matrix subject level crucial to account for arg.V2\is{verb-second}, and that stands in distant operational relation to the matrix-verb-approach, I label mediated-assertivity-approach.\is{assertion} The group of hypotheses that considers the level of the speaker/\linebreak[4]\isi{narrator} as the relevant assertive level, I label relevance\is{informative relevance}-approach. I have presented some concepts that I consider representative for the relevance\is{informative relevance}-approach, within which discourse newness\is{discourse-new} \citep{Meinunger2006} and at-issueness\is{at-issueness} \citep{Antomo2016} are two ideas with contrary assumptions: Whereas both refer to the \isi{informative relevance} on discourse level, newness\is{discourse-new} does not necessarily imply at-issueness,\is{at-issueness} and vice versa. As the assumptions are not necessarily mutually exclusive, but might interact as influencing variables, the pragmatic licensing conditions can be best illustrated by means of a \isi{conditional inference tree}, as this is a statistic model in which several effects can be considered simultaneously. The categories that are taken into account as being associated with arg.V2\is{verb-second} are mediated assertivity\is{assertion}, on the one hand, and \isi{informative relevance}, on the other hand, with their different specifications. In the following, I will explain how I define and operationalize these specifications. The statistical model is further explained in Section \ref{sec:reetz:4.4}.

Corresponding to the different informative-relevance-hypotheses, I consider two variables: Newness status\is{discourse-new} (new vs old\is{discourse-old}) and narrative relevance\is{informative relevance} (progressive vs non-progressive). Furthermore, I consider mediated assertivity\is{assertion} as another variable. For newness status,\is{discourse-new} I am appealing to a simple concept: If the truth value of the proposition in question can either be derived from the preceding storyline, or has an explicit antecedent in the preceding text (with the same or a similar wording), it is labelled “old"\is{discourse-old} accordingly. Regarding the discourse antecedent, I do not differentiate between \isi{narrator}-assertions\is{assertion} and protagonist-assertions.

For the category narrative relevance\is{informative relevance}, I am appealing to the conception that propositions are characterized by varying degrees of \isi{informative relevance} on the side of the speaker/\isi{narrator}. What has been labelled \textit{speaker assertion},\is{assertion} \textit{at-issue content},\is{at-issueness} and \textit{discourse move}\is{discourse move} respectively -- mostly for communication-based discourse models -- must be operationalized for narrative texts. Therefore, I base my operationalization of this category on general principles of narrative coherence. I follow \citet{KleinStutterheim1992}, who describe narrative texts as presentations of events that stand in a temporal relation to one another but can also be connected by spatial or logical concepts. In \citeauthor{KleinStutterheim1992}'s (\citeyear{KleinStutterheim1992}) sense, the superordinate question under discussion,\is{Question under Discussion} that may be something like \textit{What happened?} determines the main structure of the text. In conclusion, the \isi{juxtaposition} of propositions that form a (logical) progression of states and events can be understood as “main structure”. Therefore, I am going to focus precisely on the \isi{discourse relation} between the proposition expressed by the argument clause and the following proposition of the narrative segment: If the sequence in question can be interpreted in a way that the proposition in the arg.V2\is{verb-second} is logically directly followed by the proposition in the next clause (i.e., they are connected by a consecutive relation), the label \textit{progressive} is given. In such cases, the two propositions progressively contribute to answering the same question under discussion.\is{Question under Discussion} By contrast, if the following clause connects to a matrix clause or an earlier main clause, the arg.V2\is{verb-second} is not understood as contributing to the same question under discussion\is{Question under Discussion} and is labelled “non-progressive”.

Note that this procedure resembles the analytical criteria of \citet{hinterholzl2010v1} and \citet{speyer2010}. However, I do not explicitly follow the discourse model of \citet{AsherLascarides2003}. This is because I do not agree with the assumption that a coherent storyline unfolds on only one superordinate discourse level. In line with \citet[177]{langacker2001}, I assume that “a discourse is not limited to the building of a single connected structure. Structures can be assembled and developed in distinct channels, worlds, or mental spaces, either successively or by shifting back and forth between them." Concretely, I assume that in narrative texts, the storyline is not built on just one narrative level, but can yield new points of departure. This idea will become more clear in the discussion of the results.

The last pragmatic category is mediated assertivity\is{assertion}. This receives no further label, as all propositions under verbs of saying are mediated assertions\is{assertion} by nature. Note that this does not mean that it cannot be operationalized: If none of the named predictors come into question as licensing factors of arg.V2\is{verb-second}, mediated assertivity\is{assertion} will be the last predictor variable that remains and will therefore be assumed to have explanatory power. Eventually, the category \textit{text} will also be taken into account as a predictor variable, in order to determine whether there is intertextual variation with regard to the distribution of arg.V2\is{verb-second}.

\subsection{Modelling the data}\label{sec:reetz:4.4}
I will illustrate the data on verbs of saying by means of a \isi{conditional inference tree}, using the ctree function in R. The \isi{conditional inference tree} algorithm produces a tree that can be seen as a visualization of how predictors combine in their effect size, for a specific data set.  
The tree uses recursive partitioning to estimate the effect of a predictor variable while the other variables are simultaneously taken into account. The significance of the effect of each variable is calculated based on permutation. The statement that each of the branches makes can be formulated as “if…then…”-statements.

\section{Pragmatic distribution of arg.V2 under verbs of saying}\label{sec:reetz:5}

\figref{fsaying} shows the distribution of arg.V2\is{verb-second} under verbs of saying by means of a \isi{conditional inference tree}.

\begin{sidewaysfigure}
    \centering
    \includegraphics[width=1\linewidth]{figures/ReetzFig1.png}
    \caption{arg.V2 under verbs of saying}
    \label{fsaying}
\end{sidewaysfigure}

The tree finds the variable “text” as the variable to have the highest association with the response variable (the form of the argument clause). The texts are split into two subsets, one of which consists of the Melusine and the Wigalois, which have a relatively high proportion of \isi{verb-final} argument clauses. In fact, all clauses that are both labelled ``old''\is{discourse-old} and ``non-progressive'' (11 out of 42 cases) are realized with VF\is{verb-final}. But as the proportion of VF\is{verb-final}-clauses is relatively high in general, these hypothesized influence variables have no significant effect and hence, this node is not bisected any further.

For the second set of texts, consisting of the Fortunatus, Huge Schapler, Schöne Magelone, and Tristrant und Isalde, the tree returns newness\is{discourse-new} status as the category where the next best association is found. The data are sectioned into new\is{discourse-new} and old\is{discourse-old} propositions, in accordance with the fact that in the group of old\is{discourse-old} propositions, the frequency of arg.V2\is{verb-second} is lower than in the set of new\is{discourse-new} propositions. However, \isi{discourse-old} status alone does not uniformly block V2\is{verb-second}. This can be derived from node 7 in the tree, where the response is once again sectioned according to discourse progression. This means that discourse progression plays a decisive role in predicting the form of an argument clause in the set of old\is{discourse-old} propositions. If the proposition is progressive, it is very likely to be realized with V2\is{verb-second}, even though it is disourse-old\is{discourse-old} at the same time. This is exemplified in (\ref{Trist}).

\newpage 

\ea \label{Trist}
\glt Context: After a battle, Tristrant is critically wounded. His ship drifts towards the coast of Ireland. There it is discovered by the king, who asks his servants to find out what is in the ship.\\
\gll dye diener kamen vnd sagten. da w\oldae{}r ein man wundt in den tod. 
der küng gieng selbs dar vnd fand als jm gesagt was\\
The servants came and said there was a man wounded in the death. The king went himself there and found as him told was\\
\glt `The servants came and said that there was a man who was critically wounded. The king went there himself and discovered what he had been told.' \hfill [TRIST, 19r]
\z

As the proposition has a discourse antecedent that positively decides its truth value, the proposition under question in (\ref{Trist}) is old. At the same time, the proposition progresses the storyline, in that it contributes in causing the event expressed in the following clause. Therefore, it is progressive. The perspectivation of the expressed state of affairs seems to play a decisive role for the “main clause character” of the arg.V2\is{verb-second}: In this case, the reader is following the stream of consciousness of the king, who inquires about a matter. For him, the state of affairs (that there is a wounded man in the ship) is new,\is{discourse-new} informative and decisive for his further action. In this case, the V2\is{verb-second} syntax might be attributable to the fact that the \isi{narrator} “takes on the perspective of the protagonist” in order to progress the storyline coherently. In contrast, (\ref{Rupert}) shows a sentence in which the proposition in question is \isi{discourse-old} and non-progressive (node 8 in \figref{fsaying}).

\ea \label{Rupert}
Context: Rupert, a servant of the earl of Flanders, has deliberately convinced Fortunatus to leave the earl's court without his knowledge. The earl is now devastated about Fortunatus' disappearance. Rupert notices the earl's frustration and is frightened that one of his companions might reveal that he is the causer of Fortunatus' disappearance.\\
\gll Und gieng zu yn allen/ und yedem in sonderhait/ bat sy/ das sy nyendert meldeten/ daz er ain ursacher w\oldae{}r/ sines hinwegschaidens/ das gelobten sy im gar trüwlich \\
and went to them all and everyone in particular asked them that they nobody told that he a causer was his disappearance this promised they him very faithfully\\
\glt `and (Rupert) went to all of them individually and asked them not to tell anybody that he was a causer of his (Fortunatus') disappearance. This they promised him faithfully.' \hfill [FORT, 403]
\z

First note that in (\ref{Rupert}), the matrix verb of the proposition in question is negated. The truth value of a proposition is not affected by the scope of the negation in the matrix clause, which is commonly assumed to be a sign that the proposition is presupposed.\is{presupposition} This can be explained in discourse-pragmatic terms with regard to the two criteria under consideration: The proposition is old\is{discourse-old} and non-progressive. The proposition is already incorporated in the reader's mental system, so for the reader, its truth value is decided. A privotal factor here is that the proposition is also already incorporated in the mental state of the protagonists of the following proposition (the companions). Their action does not consecutively follow from the proposition in question. Instead, the consecutive relation lies between a preceding proposition (Rupert asking his companions to do him a favor) and the clause following the arg.V2\is{verb-second} (his companions agreeing). Note that when the proposition is old\is{discourse-old} and non-progressive at the same time, it is realized as a VF\is{verb-final}-clause in nine out of nine cases, which is highly significant with regard to the group of discourse-old\is{discourse-old}, progressive propositions, according to the model.

The effect of the discourse progression status on the form of the argument clause gets lost in the subset of \isi{discourse-new} propositions. This corroborates the hypothesis that there is another licensing factor for arg.V2\is{verb-second} besides progressiveness, which I assume to be mediated assertivity\is{assertion}. In other words, mediated assertivity\is{assertion} does yield V2\is{verb-second} under the premise that the proposition is new in the discourse.\is{discourse-new} This becomes especially clear if one looks at a clause where the proposition in the argument clause is non-progressive, as it is the case in \REF{Mag}.

\ea \label{Mag}
 Context: Magelone, who works in a hospital, is visited by a count and countess.\\
\gll vnd baten die sie solt mit jnen zů morgen essen/ aber jr hertz
vermocht jhn solchs nit zů zesagen/ darumb sie sprach sie hette daheim zů thůn (...) Nach dem kam die Magelona wider zů dem Peter\\
and asked her she shall with them to morning dine but her heart could them this \textsc{neg} \textsc{prtcl} to.promise therefore she said she had at.home to do (...) After this came the Magelona again to the Peter\\
\glt `And (they) asked her to eat with them in the morning, but her heart could not promise that, therefore she told them she had something to do at home (...). After this, Magelone returned to Peter.' \hfill [MAG, 674]
\z

In (\ref{Mag}), the clause under question is obviously non-progressive, as it does not have any kind of consecutive \isi{discourse relation} with the proposition following it. The mediated proposition might be new\is{discourse-new} and informative for the characters\is{character} involved in the described situation (the count and countess), but the reader does not follow their stream of consciousness. Instead, the storyline continues to build around Magelone (the originator of the proposition in the arg.V2\is{verb-second}) and for the stream of the narrated events, the proposition in question does not play a decisive role. In other words, there is no consecutive relation between what Magelone makes up (the proposition in the argument clause), and the proposition following it. Furthermore, it is important to note that the data do not imply that the correlation between the newness\is{discourse-new} status and the clause form is a spurious correlation. Instead, it rather suggests that newness\is{discourse-new} status individually affects the form of the argument clause, to a somewhat lesser degree than it does in combination with (non)-progressiveness. In the Fortunatus, for example, all instances of \isi{discourse-new} propositions (28 in total) are marked by V2\is{verb-second}, but \isi{discourse-old} propositions that are progressive have VF\is{verb-final} syntax in 3 out of 16 cases. Consider \REF{Fort}.

\ea \label{Fort}
Context: The count introduces Fortunatus to other people.\\
\gll und sagt yn wie er so ain gůter j\oldae{}ger w\oldae{}re/ die vogel in dem lufft und die thyer in den w\oldae{}lden w\oldae{}r kaines sicher vor ym/ tzu dem das er sunst wol dienen kund\\
and told them how he such a good hunter was.\textsc{sujv} the birds in the air
and the animals in the woods were.\textsc{sbjv} none safe in.front him
to this COMP he else well serve could\\
\glt `And told them how he was such a good hunter, none of the birds in the air and the animals in the forest were safe from him, furthermore, that he could serve well.' \hfill (FORT, 394)
\z

In \REF{Fort}, most interestingly, there is a formal variation between the coordinated dependent clauses, although they all (as a configuration) bear the same \isi{discourse relation} to the following event (consecutive). Nevertheless, only the discourse-old\is{discourse-old} (and positively decided) clauses have VF\is{verb-final} syntax, whereas the one that is new\is{discourse-new} (in that a state of affairs is mediated in an “embellished” way by using the subject referent's words) has V2\is{verb-second} syntax.


\section{Discussion}\label{sec:reetz:6}
The findings of the present study can be summarized as follows: The mapping between pragmatics and syntax is not equally systematic in all of the investigated texts. The Melusine and the Wigalois both have a relatively high frequency of syndetic VF\is{verb-final} argument clauses, so that neither the newness status\is{discourse-new} nor the discourse progression status of a proposition have a significant effect on the form of the clause. VF\is{verb-final} seems to be the default form of argument clauses here. I suspect this tendency to become more dominant diachronically towards PDG. This, however, needs further investigation.

In a larger group of texts, \isi{discourse-old} status of a proposition (“old"\is{discourse-old} within the mental state of the reader) reduces the likelihood for the proposition to be realized with arg.V2\is{verb-second}, but does not exclude the possibility. The discourse progression status seems decisive here: If a \isi{discourse-old} proposition is at the same time progressive, argument clauses are still frequently realized with V2\is{verb-second}. In contrast, if an old\is{discourse-old} proposition is non-progressive, that is, not decisive for the stream of events, VF\is{verb-final} is the only formal variant that is used. Hence, it can be seen in the category of old\is{discourse-old} propositions that the discourse progression status is a decisive predictor variable with respect to the form of an argument clause. 

VF\is{verb-final} seems to be rather restricted. Only the category of propositions that are discourse-old\is{discourse-old} and non-progressive seems to be rather (if not exclusively) restricted to VF\is{verb-final} syntax. Therefore, I propose the interaction of discourse-old\is{discourse-old} status and non-progressive status to be relevant for defining the blocking context of arg.V2\is{verb-second} in ENHG narratives. Furthermore, it has been shown that within the subset of \isi{discourse-new} propositions, the progression status of a proposition has no significant effect on the form of the argument clause. In how far can this be linked to the different levels of \isi{assertion} that have been discussed earlier? If in a \isi{narration}, a proposition is consecutively followed by another, it is informatively heavier than when this is not the case, because it becomes emphasized on narrative level. It can be said that the embedded proposition is “shifted” to the narrative level. This does not necessarily entail that it is an \isi{assertion} on the \isi{narrator}'s side, but is assigned narrative relevance\is{informative relevance} by the \isi{narrator}. However, I have shown that propositions that are mediated, but left unspecified in the progression of events (i.e., have no or weaker narrative relevance\is{informative relevance}), are also frequently realized with V2\is{verb-second} syntax in ENHG\il{Early New High German} narratives. As these clauses lie on a deeper narrative level, they can be classified as pragmatically embedded. How concordant this pragmatically embedded class is to the class of “true V2\is{verb-second}-complements” \citep{Freywald2016} cannot be derived with certainty. However, there are signs that V2\is{verb-second}-clauses can be embedded. Consider the argument clause in (\ref{Rupert}) again: Here, the finite verb is used in subjunctive form, although the proposition's truth value is clearly decided. This would be marked in PDG, where subjunctive mood is used to relativize truth claims \citep{Fabricius-Hansen2019}. This is a sign that the usage of subjunctive mood in ENHG\il{Early New High German} still differs from that in PDG: In ENHG\il{Early New High German}, subjunctive verbal mood can function as a means to mark syntactic dependence, as it is often proposed to be the case in older stages of \ili{German} \citep{Schrodt2004}. While the finite verbs of arg.V2\is{verb-second} are frequently used in subjunctive form in the ENHG\il{Early New High German} narratives, there is also a considerable amount of subjunctive verbs in arg.V2\is{verb-second}. As subjunctive verbal mood can function as a marker of clausal dependency, it can be derived that the varying use of subjunctive is (at least occasionally) attributable to the syntactic (in)dependence of the respective clause. Consider (\ref{ex:reetz:10}) as a minimal pair.

\ea \label{ex:reetz:10}
\ea \label{ex:reetz:11}
\gll sagten sy er w\oldae{}r der gr\oldoe{}st b\oldoe{}swicht der vff ertrich lebt  \\
     said they he was.\textsc{sbjv} the biggest villain that on earth lives\\
\glt `They said he was the greatest villain to walk the earth.' \hfill [FORT, 499]
\ex \label{ex:reetz:12}
\gll Zů dem Anndolosia sagt ia er wolt ym geren dienen mit leib vnd mit gůt\\
     To which Andolosia said yes he wanted him willingly serve with body and with good\\
\glt `To which Andolosia responded, yes, he wanted to serve him willingly, with body and good' \hfill [FORT, 515]
\z
\z

At this point, it is also important to note that in some -- if not many -- cases of arg.V2\is{verb-second}, there can be ambiguities with regard to the hypothesized progressive status. Hence, in written text, there might be a misconception of narrative relevance\is{informative relevance}/irrelevance\is{informative relevance}. As I said in the beginning of the discussion, I suspect the tendency of VF\is{verb-final} being the default form of argument clauses to become more pronounced diachronically. The hypothesized ambiguity between narrative relevance\is{informative relevance} and narrative irrelevance as well as the change in the use of subjunctive mood (which is still ongoing in ENHG\il{Early New High German}) might be a crucial factor for this change. Therefore, what will be highly relevant for upcoming studies is the question how exactly PDG proportions have prevailed, and which role the interaction of sentence mood and verb mood plays diachronically, from ENHG\il{Early New High German} onwards. Whether \isi{discourse-old} status becomes a stronger differentiating factor for the syntactic form of argument clauses (as \cite{Meinunger2006} proposes for PDG) is an open question as well. These questions on the motivations for changes are especially relevant from a language modality perspective: \citet{Auer1998} finds 60 percent “dependent main clauses” in spoken language and 35 percent in written language. These numbers suggest that it is precisely the written variety that has “drifted away” from the original proportions. Although spoken language has often been argued to have a stronger progressive force than the written variety, literacy might drive forward linguistic innovations in the field of syntax through its inherently low degree of situational embedding, as \citet{koch1996} propose. The relevance of ENHG\il{Early New High German} narrative prose for the assessment of linguistic changes within the written language is especially justified against the background that the importance of this \isi{genre} increased at that time, which is reflected in the rapid growth of the number of prose novels as prints and manuscripts. Hence, the scientific focus on the narrative \isi{genre} can provide more insight not only into linguistic circumstances at this point in time but also into what factors are crucial for the development of linguistic phenomena in the written language.


\section{Summary and conclusion} \label{sec:reetz:7}
In previous literature, different pragmatic factors have been discussed with respect to their relevance\is{informative relevance} as licensing requirements for arg.V2\is{verb-second}. The main finding of the present study is that the inventory of arg.V2\is{verb-second} in ENHG\il{Early New High German} is best understood as the outcome of an interaction of factors in ENHG\il{Early New High German} prose texts. Hence, the theoretical approaches on the licensing requirements do not necessarily contradict one another. It has been shown that V2\is{verb-second} comes into effect not only if the \isi{assertion} is assigned narrative relevance\is{informative relevance} by progressing the storyline, but also if an \isi{assertion} is just mediated without being further specified on the narrative level. In some texts, V2\is{verb-second} is used considerably frequently in both these cases. Only propositions that are both \isi{discourse-old} and non-progressive seem to clearly block V2\is{verb-second} syntax.\\

\section{Abbreviations}
\begin{tabularx}{.5\textwidth}{lQ}
V2 & asyndetic verb second  \\
VF & syndetic verb-final  \\
Arg.V2 & argument realizing verb-second clause   \\
\end{tabularx}%
\begin{tabularx}{.5\textwidth}{lQ}
ENHG & Early New High German  \\
PDG & present-day German  \\
\end{tabularx}

\section*{Acknowledgements}
This research was funded by the Deutsche Forschungsgemeinschaft (Project\linebreak[4] 456973946, \glqq Wortstellung und Diskursstruktur in der Fr\"u{}hen Neuzeit\grqq).

\section*{Primary sources}
\begin{hangparas}{.25in}{1}
\textbf{Die Schöne Magelone} (1535). Müller, Jan-Dirk (ed.). 1990. Romane des 15. und 16. Jahrhunderts. Nach den Erstdrucken mit sämtlichen Holzschnitten, 587–678. Frankfurt am Main: Deutscher Klassiker Verlag.

\textbf{Fortunatus} (1509). Schmitt, Ludwig Erich and Renate Noll-Wiemann (eds.) 1974. Fortunatus: Von Fortunato und seynem Seckel auch Wünschhütlein. Hildesheim: Olms.

\textbf{Huge Scheppel} (1500). Huge Scheppel; Königin Sibille Saarbrücken. 1455--1456. 76 pages. Hamburg: Staats-und Universtitätsbibliotek. Codex 12 in scrinio. pp. 1ra--57vb. Fascimile available online: \url{https://resolver.sub.uni-hamburg.de/kitodo/HANSh495}

\textbf{Melusine} (1474). Müller, Jan-Dirk (ed.). 1990. Romane des 15. und 16. Jahrhunderts. Nach den Erstdrucken mit sämtlichen Holzschnitten, 9–176. Frankfurt am Main: Deutscher Klassiker Verlag.

\textbf{Tristrant und Isalde} (1484). Elsner, Helga (ed.). 1989. Tristan und Isolde (Augsburg bei Antonius Sorg, 1484). Hildesheim: Olms.

\textbf{Wigalois vom Rade} (1519). Melzer, Helmut (ed.). 1973. Wigalois. Hildesheim: Olms.
\end{hangparas}

\bigskip 

\sloppy
\printbibliography[heading=subbibliography,notkeyword=this]
\end{document}
