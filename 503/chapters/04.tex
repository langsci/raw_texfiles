\documentclass[output=paper,colorlinks,citecolor=brown]{langscibook}
\ChapterDOI{10.5281/zenodo.15689125}

\author{Jordan Chark\affiliation{Humboldt-Universität zu Berlin \&  Leibniz-Zentrum Allgemeine Sprachwissenschaft}}
\title{Discourse structure and the reorganisation of the Icelandic aspectual system} 
\shorttitlerunninghead{The Icelandic aspectual system} 
\abstract{Icelandic is commonly considered a conservative language. While this claim holds true for its morphology, especially in the broader Germanic perspective, one area that has undergone considerable changes since the 16th century is the mapping between syntax and discourse structure. In this paper, I will argue that otherwise puzzling facts from the history of the language can be better understood if one takes this shift in discourse structuring as fundamental. While most research to date has focused on a limited set of syntactic phenomena, I bring new data to bear on this question: the emergence of two aspectual markers, the \textit{búinn}-perfect and the \textit{vera að}-progressive. I trace the shift in discourse structure alongside the emergence of innovative aspectual markers in two corpora of historical Icelandic (\textbf{IcePaHC},\is{IcePaHC} \citealt{Wallenberg2011}, and \textbf{Íslenskt textasafn}, \citealt{IslensktTextasafn2019}). I show that what seem to be abrupt changes in the corpus record no longer seem so abrupt when considered in the larger context of this shift from a bounded to an unbounded system. Rather, the loss of topic-time anchors like \textit{þá} `then' paved the way for the emergence of new ways of construing events \citep{los2012} during the 16th-18th centuries, a period of variability reflected in the corpus dataset by travelogues and memoirs that combine more archaic and innovative characteristics.}

% Uncomment if only this chapter is to be compiled
%IfFileExists{localcommands.tex}{
%   \addbibresource{localbibliography.bib}
%   \usepackage{langsci-optional}
\usepackage{langsci-gb4e}
\usepackage{langsci-lgr}

\usepackage{listings}
\lstset{basicstyle=\ttfamily,tabsize=2,breaklines=true}

%added by author
% \usepackage{tipa}
\usepackage{multirow}
\graphicspath{{figures/}}
\usepackage{langsci-branding}

%   
\newcommand{\sent}{\enumsentence}
\newcommand{\sents}{\eenumsentence}
\let\citeasnoun\citet

\renewcommand{\lsCoverTitleFont}[1]{\sffamily\addfontfeatures{Scale=MatchUppercase}\fontsize{44pt}{16mm}\selectfont #1}
  
%   %% hyphenation points for line breaks
%% Normally, automatic hyphenation in LaTeX is very good
%% If a word is mis-hyphenated, add it to this file
%%
%% add information to TeX file before \begin{document} with:
%% %% hyphenation points for line breaks
%% Normally, automatic hyphenation in LaTeX is very good
%% If a word is mis-hyphenated, add it to this file
%%
%% add information to TeX file before \begin{document} with:
%% %% hyphenation points for line breaks
%% Normally, automatic hyphenation in LaTeX is very good
%% If a word is mis-hyphenated, add it to this file
%%
%% add information to TeX file before \begin{document} with:
%% \include{localhyphenation}
\hyphenation{
affri-ca-te
affri-ca-tes
an-no-tated
com-ple-ments
com-po-si-tio-na-li-ty
non-com-po-si-tio-na-li-ty
Gon-zá-lez
out-side
Ri-chárd
se-man-tics
STREU-SLE
Tie-de-mann
}
\hyphenation{
affri-ca-te
affri-ca-tes
an-no-tated
com-ple-ments
com-po-si-tio-na-li-ty
non-com-po-si-tio-na-li-ty
Gon-zá-lez
out-side
Ri-chárd
se-man-tics
STREU-SLE
Tie-de-mann
}
\hyphenation{
affri-ca-te
affri-ca-tes
an-no-tated
com-ple-ments
com-po-si-tio-na-li-ty
non-com-po-si-tio-na-li-ty
Gon-zá-lez
out-side
Ri-chárd
se-man-tics
STREU-SLE
Tie-de-mann
}
%   \boolfalse{bookcompile}
%   \togglepaper[23]%%chapternumber
%}{}

%%%%%%%%%%%%%%%%%%%%%%%%%%%%%%%%%%%%%%%%%%%%%%%%%%%%%%%%%%%%%%%%%%%%%
\begin{document}
\maketitle

\section{Introduction}\label{sec:Chark1}

Icelandic, despite its well-known morphosyntactic conservatism (cf. \citeauthor{fridhriksson2011modern}  \citeyear{fridhriksson2011modern}; \citeauthor{rognvaldsson2011morphosyntactic} \citeyear{rognvaldsson2011morphosyntactic}), has undergone considerable changes in the mapping between clause and information structure since the early modern period (1540 onwards).\footnote{This period is demarcated by the publication of Oddur Gottskálksson's translation of the New Testament into Icelandic \citep{ottosson1990islensk}.} To date, scholarly work in this area has focused on the syntactic reflexes of these changes \citep{rognvaldsson1995old, rognvaldsson2002thadh, booth-beck20200jhs}. In this paper, I focus on a topic that has received little attention in the diachronic literature on Icelandic: the emergence of periphrastic\is{periphrasis} aspectual\is{aspect} constructions, the \textit{búinn}-\isi{perfect}  and the \textit{vera að}-\isi{progressive}. I will argue that the trajectories of these periphrastic\is{periphrasis} constructions can be understood within the context of this fundamental shift in mapping between form and discourse structure. The diachrony of Icelandic exhibits a shift from a predominantly `bounded' system for expressing narrative to a predominantly `unbounded' one (see \citeauthor{petre2010functions} \citeyear{petre2010functions}, \citeyear{petre2014constructions}, \citeauthor{los2012} \citeyear{los2012}, \citeauthor{bech2014} \citeyear{bech2014}, \citeauthor{fanego2024english} \citeyear{fanego2024english} on a similar shift in the history of \ili{English}).

Drawing on corpora of historical Icelandic, I show that between 1540--1850 (what I define as the early modern period)\footnote{Note that terminology surrounding the periodisation of Icelandic is used inconsistently in the literature, with some authors including early 20th century \citep{svavars2014}, while others such as \citet{brendel2023translation} refer roughly to the period spanning the 16th to 19th centuries.} the \isi{perfect}  construction \textit{búinn} and the dedicated \isi{progressive}  construction formed with \textit{vera að} `be to' went from once marginal constructions to integral parts of the aspectual\is{aspect} oppositions encoded in Icelandic grammar. Both are attested in earlier Icelandic, not having yet acquired their modern aspectual\is{aspect} meanings. More broadly, I argue that these changes can be subsumed under a shift from a bounded\is{boundedness} to unbounded\is{boundedness} system.

I argue that this shift in \isi{boundedness} is reflected in a change in grounding strategies in the construing of narrative sequences. This includes the marking of episodic boundaries in narratives and the way in which information-structural notions are tracked overtly in the syntax. In this paper, I focus on the early modern period of Icelandic, which I take to begin with the New Testament translation by Oddur Gottskálksson (completed in 1540) and to end with the publication of what may be considered the first modern Icelandic novel, \textit{Piltur og stúlka} `Boy and Girl', published in 1850 \citep{bernhardhsson2017jon, bernhardhsson2018spreading}. %Early modern Icelandic (appx. 1550-1850) is a language in flux on a number of fronts. 

This paper is structured as follows. First, this introductory section continues with background information on the corpora used. Next, in Section \ref{sec:Chark2}, I discuss the grammaticalisation\is{grammaticalization} of the \textit{búinn}-\isi{perfect}  and the dedicated \textit{vera að}-\isi{progressive}: Section \ref{sec:Chark2.1} begins with the schema I assume for grammaticalisation\is{grammaticalization} in context, Section \ref{sec:Chark2.2} discusses the emergence of the \textit{búinn}-\isi{perfect}, Section \ref{sec:Chark2.3} discusses the emergence of the \textit{vera að}-\isi{progressive}. Section \ref{sec:Chark3} continues with a discussion of discourse traditions\is{discourse tradition} (historical registers\is{register}): \ref{sec:Chark3.1} introduces the term and \ref{sec:Chark3.2} situates the texts used in this study within their respective discourse traditions.\is{discourse tradition} Section \ref{sec:Chark4} then proceeds with a discussion of the shift from \isi{boundedness} to unboundedness, which is the primary claim of this paper: \ref{sec:Chark4.1} provides the motivation behind this approach and \ref{sec:Chark4.2} discusses the literature in this area in more detail. I then argue that Old(er) Icelandic (prior to 1540) represents a predominantly \isi{bounded} system \ref{sec:Chark4.2.1}), that 17th and 18th century Icelandic has an intermediate system (\ref{sec:Chark4.2.2}) and that modern Icelandic (post 1850) ought to be considered unbounded (\ref{sec:Chark4.2.3}). In Section \ref{sec:Chark5}, I delve into this decrease in \isi{bounded} strategies: \ref{sec:Chark5.1} discusses work on \ili{Old English} in this regard, in order to set the stage for the Icelandic data in comparison. Section \ref{sec:Chark5.2} then provides statistical corpus evidence to support my claim that Icelandic underwent a shift in \isi{boundedness} during this period: this evidence includes a general increase in temporal subordinate clauses (\ref{sec:Chark5.2.1}), that the rise of the subordinator \textit{þegar} is negatively correlated with the decline of the topic-time adverbial \textit{þá} (\ref{sec:Chark5.2.2}), that the requirement for overt temporal adverbials in the left periphery becomes less strict (\ref{sec:5.2.3}) and finally that the adverbials associated with \textit{búinn} and \textit{PROG} (\ref{sec:Chark5.2.4}) as well as their relative timing support this view as well (\ref{sec:Chark5.2.5}). Section \ref{sec:Chark6} concludes with a summary of my findings and a reflection on implications for future work on the diachrony of Icelandic as well as \isi{diachronic typology} more generally.

%\subsection{Background on corpora}\label{sec:Chark1.1}
Throughout the paper, I draw my data\footnote{The data and scripts are available via \url{https://github.com/jordanchark/discourse_narration}.} primarily from two corpora: 
\begin{enumerate}
    \item \textbf{IcePaHC}  \citep{Wallenberg2011} $\approx$ 1 million words. Timeframe: 1150-2008, with diverse \isi{genre} distribution. Parsed in Penn treebank format.\footnote{Version 0.9. More information on IcePaHC can be found at \url{https://github.com/antonkarl/icecorpus}.}
    \item \textbf{Íslenskt textasafn} \citep{IslensktTextasafn2019}. I utilise two components of this corpus: (i) the 16th-18th century subcorpus which consists of more substantial material than that available in \isi{IcePaHC} for this time period  $\approx$ 421.000 words, outlined in Table \ref{fig:textasafn},\footnote{The texts \textit{Nikulás Klím}, \textit{Nýja testamentið }and \textit{Ævisaga síra Jóns}\is{Icelandic saga} are present in \isi{IcePaHC} in much smaller excerpts.} and (ii) I make use of \textit{Ritmálssafn Orðabókar Háskólans}, the archive of written language for the dictionary of Icelandic compiled at the University of Iceland, which contains 2.5 million example sentences of 700.000 header words.\footnote{This corpus can also be accessed at \url{https://ritmalssafn.arnastofnun.is/}}
\end{enumerate}  


\begin{table}
\centering
\begin{tabularx}{\textwidth}{lQrr}
\lsptoprule
Title & Author & Year & Words \\
\midrule
Einfalt matreiðsluvasakver & Marta María Stephensen & 1800 & 16,592 \\
Nikulás Klím & Holberg, Ludvig (Jón Ólafsson) & 1745 & 23,332 \\
Nýja testamentið & (Oddur Gottskálksson) & 1540 & 189,692 \\
Stuttur siðalærdómur & Campe, J.H (Guðlaugur Sveinsson) & 1799 & 36,773 \\
Uppkast til forsagna & Eggert Ólafsson & 1767 & 35,242 \\
Vídalínspostilla & Jón Þorkelsson Vídalín & 1718-20 & 156,497 \\
Ævisaga síra Jóns & Jón Steingrímsson & 1791 & 107,926 \\
\lspbottomrule
\end{tabularx}
\caption{Íslenskt textasafn, 16-18th century subcorpus. Translator in parentheses. Count is total words present in documents provided by the corpus editors. }
\label{fig:textasafn}
\end{table}


\section{Emergence of a new perfect and progressive}\label{sec:Chark2}

Modern Icelandic exhibits a tripartite system for aspectual\is{aspect} distinctions: inchoative, \isi{progressive}  and completive. This system is comprised of periphrastic\is{periphrasis} constructions which embed an infinitival phrase. By the end of the early modern period, this system is well established. Evidence for this comes from grammatical descriptions of this period, as for example by Jón Magnússon under the heading \textit{De Verbis Auxiliaribus} in his \textit{Grammatica Islandica} (written 1737-1738) \citep{jonsson1933islandske}. The headings below are taken from this text, translated from Latin:\footnote{Spelling is adapted to modern orthography. The list is not complete as Magnússon's account also includes modal expressions, which are not relevant for the purposes of this paper.}

\ea These all distinguish the present
  \ea[]{
    \gll Ég tek nú til að gjöra \\
    I take now to to do \\
    \glt  `I now begin to do.'
  }
  \ex[]{
    \gll Ég er farinn að gjöra. \\
    I am gone to do \\
    \glt `I began doing.'
  }
  \ex[]{
    \gll Ég er að gjöra. \\
    I am to do \\
    \glt `I am doing.' \hfill \citep[130]{jonsson1933islandske}
  }
\z 
  \z

\ea These signify the imperfect past
  \ea[]{
    \gll Ég fór að gjöra. \\
    I went to do \\
    \glt `I began doing.'
  }
  \ex[]{
    \gll Ég var að gjöra. \\
    I was to do \\
    \glt `I was doing.' \hfill \citep[130]{jonsson1933islandske}
  }
  \z 
\z

\ea These indicate the perfect
  \ea[]{
    \gll Ég er búinn að gjöra. \\
    I am búinn to do \\
    \glt `I have done.'
  }
  \ex[]{
    \gll Ég hef gjört. \\
    I have done \\
    \glt `I have done.' \hfill \citep[130]{jonsson1933islandske}
  }
  \z 
\z

\ea These indicate the pluperfect
  \ea[]{
    \gll Ég hafði gjört. \\
    I had done \\
    \glt `I had done.'
  }
  \ex[]{
    \gll Ég var buinn að gjöra. \\
    I was \textsc{búinn} to do \\
    \glt `I had done.' \hfill \citep[130]{jonsson1933islandske}
  }
  \z 
\z 

As we see here, by the mid-18th century the primary tripartite oppositions are considered to be integral enough to the language for Jón to include all of their forms in the paradigm.\footnote{His description also makes clear that Icelandic of this period had competing forms for the inchoative: one formed by \textit{taka} `take' along with an infinitive and the other formed by the participle \textit{farinn} `gone' in \isi{conjunction} with an infinitival phrase.}

Prior to this, the paradigm of aspectual\is{aspect} distinctions made available by the grammar is arguably incomplete (cf. \citeauthor{arnason1977buinn}  \citeyear{arnason1977buinn}). By this I mean that the distinctions made overtly are asymmetrical. First, prior to 1550, marking of inchoative \isi{aspect} is already established, marked by an infinitival phrase embedded under the verb \textit{taka} `take'. However, a completive \isi{aspect} and \isi{progressive}  \isi{aspect} do not emerge until later. These will be the focus of this paper. I begin with the diachrony of the \isi{progressive}  construction with \textit{vera að}. 


\subsection{Incidental variation}\label{sec:Chark2.1}

In what follows, I argue for the role of what I term incidental variation. Incidental privileging contexts in diachrony have also been termed \textit{beachheads} \citep{givon1979understanding, aldai2002grammaticalization}.  Such an approach is pursued by \citeauthor{petre2016grammaticalization} (\citeyear{petre2016grammaticalization}) in his account of the diachrony of the \ili{English} \isi{progressive}.  I would like to suggest that changing frequencies of environments that themselves privilege usages in question were responsible for the seemingly rapid rise in frequency in the diachrony of Icelandic for both \textit{búinn} and the \isi{progressive}. The focalised \isi{progressive}  and anterior \textit{búinn} serve similar text-structuring purposes\ \textendash{}\  the prevalence of privileging contexts should correlate with dimensions of textual organisation.

It has long been recognised by scholars that grammaticalisation\is{grammaticalization} happens in context \citep{diewald2002model, heine2002role}. That is, ambiguities arise in particular contextual configurations, leading to the possibility of an alternation between two expressions at this juncture, which eventually can give rise to the cementing of new semantic meaning emerging from contextually-dependent inferences. Schematically, this can be represented as follows \citep{heine1991grammaticalization}:\footnote{In the system proposed by \citet{heine1991grammaticalization}, bridging contexts allow ambiguity between conservative and innovative meanings, whereas switch contexts only permit an innovative reading.}

\ea
Increased frequency $\rightarrow$ Context-dependent implicature becomes more salient → Ambiguity (Bridging contexts) $\rightarrow$ Reanalysis (Switch contexts) $\rightarrow$ Generalisation
\z

In the following subsections, I argue for the importance of temporal subordinate contexts in the conventionalisation of the two aspectual\is{aspect} constructions which are the focus of this paper: \textit{búinn} and \textit{PROG}. I begin by examining the emergence of \textit{búinn} in \ref{sec:Chark2.2}, followed by \textit{PROG} in \ref{sec:Chark2.3}.

\subsection{Emergence of BÚINN}\label{sec:Chark2.2}

Modern Icelandic has two constructions that have a number of properties typically ascribed to perfects\is{perfect} \citep{jonsson1992, thrainsson2017developing}: \textit{búinn} `finish' and \textit{hafa} `have'. All primary perfect readings contain the \textit{hafa} \isi{perfect}  (experiential, resultative, universal, cf. \cite{mccawley1971tense, comrie1976aspect})  and are already found in the earliest written Icelandic sources \citep{nygaard1905norron, pollak1930}. The \isi{past participle}  \textit{búinn} is derived from the verb \textit{búa} `reside, prepare, adorn' and does not acquire an anterior, \isi{perfect}-like meaning until the 17th century \citep{thrainsson2017developing}. Prior to the 17th century, \textit{búinn} predominantly has a `prepared' reading. The example below, for instance, cannot have anterior reference (i.e. receive a \isi{perfect}-type reading).\footnote{The adjectival meaning of the participle outside of this construction also shifts from predominantly having a `prepared' reading to meaning `complete' after the grammaticalisation\is{grammaticalization} of the infinitival construction \citep{arnason1977buinn}.} 

\ea\label{ex: 1250} \gll En er Egill var búinn og byr gaf þá siglir hann í haf.\\
and when Egill was \textsc{búinn} and wind gave then sails he to sea\\
\glt  `And when Egill was prepared and there was wind, he sailed to sea.'\\
 \hfill [\textbf{1250}.THETUBROT.NAR-SAG.74]\footnote{Examples sourced from IcePaHC are given with metadata information in the format: Year.Text.Genre-Subgenre.TokenID.}
\z

My claim is that the later anterior reading is due to the underspecification of \textit{búinn} (cf. \citeauthor{rosemeyer2017road}, \citeyear{rosemeyer2017road} on a similar diachronic path in Spanish). Preparedness is a context-dependent notion which bleeds together with completeness (an example of ``type coercion''; \cite{pustejovsky1998generative}). This is illustrated for historical Icelandic in (\ref{ex: netit}) (14th century) where it is apparent that a net which is prepared for fishing implies a complete or finished net.

\ea\label{ex: netit} 
\gll Ok er búið var netit, þá fara Æsir til árinnar ok kasta neti í forsinn\\
and when \textsc{búinn} was the:net, then go Æsir to the:river and throw net into the:waterfall\\
\glt `And when the net was finished, the Æsir went to the river and threw the net into the waterfall.' \\
\hfill (\citeauthor{onp} (ONP): \citeauthor{jonsson1931edda}, \citeyear[69]{jonsson1931edda}, \\
\hfill English from \citeauthor{faulkes1987snorri} \citeyear{faulkes1987snorri})\footnote{ONP \citep{onp} is a digital dictionary of Old Norse prose and can be accessed at \url{https://onp.ku.dk/onp/onp.php}.}
\z
\largerpage[-1]
In early stages of Icelandic \textit{búinn} is often used to add emphasis to a temporal sequence, adding some punch to the narrative by highlighting the sequential relationship between two eventualities (for a diachronic trajectory with similar characteristics, see \cite{rosemeyer2017road}). Clauses involving \textit{búinn} are typically complement-less, though not exclusively so. In my investigation of the \textit{Fornrit} corpus \citep{IslensktTextasafn2019}, out of 1591 tokens, 46 of them are infinitival.\footnote{The \textit{Fornrit} corpus is part of the \textit{Íslenskt textasafn} corpus, additional available for search at \url{https://malheildir.arnastofnun.is/}. The corpus consists of approximately 1.5 million words \citep{IslensktTextasafn2019, rognvaldsson2011morphosyntactic}.} These have a future-oriented prepared-type reading.  By contrast, in my extended dataset ranging from the 16th to 19th centuries, when this conventionalisation was ongoing, upwards of 80\% of \textit{búinn} tokens have infinitival complements. 

\ea
\gll En er Haraldur konungur var búinn að stíga á hest sinn þá bað hann kalla til sín Áka búanda.\\
and when Haraldur king was \textsc{búinn} to mount on horse his then asked he call to him Áki farmer\\
\glt `When King Harald was ready to (about to) mount his horse, he asked for Áki the farmer to be called to him.' \\
\hfill (Heimskringla, cited in \textit{Fornrit} corpus:\\
\hfill \citeauthor{IslensktTextasafn2019} \citeyear{IslensktTextasafn2019})
\z

In temporal subordinate contexts expressing a sequence between  ``event time" (E) and ``reference time" (R) (E-R-S, as per \cite{Reichenbach1947}), it would typically be under-informative to overtly express the event description for which one is preparing, assuming that the following main clause asserts such an event. In other words, overtly expressing that an event that was an imminent occurrence (cf. \citeauthor{eckardt2006meaning} \citeyear[91--127]{eckardt2006meaning} on \textit{be going to}) at some past interval did indeed become instantiated is under-informative (on Gricean assumptions; \cite{holvoet2014phasal}).

Below in (\ref{ex: tempant1}), we see a prototypical early anterior usage of \textit{búinn}, highlighting a relationship between two events, $e_1$ strictly preceding $e_2$.

\ea \label{ex: tempant1} \gll Þegar ég var nú búinn að tala við Jón yngra úti kom ég inn aftur í kirkjuna\\
when I was now \textsc{búinn} to speak with Jón younger outside came I in again into the:church\\
\glt `When I was finished speaking/had spoken with John junior outside I came into the church again.'  \hfill [\textbf{1659}.PISLARSAGA.BIO-AUT.56]
\z
\largerpage[-1]
I assume that innovative speakers took advantage of the underspecification of \textit{búinn} as well as analogy with related expressions (relating to the present\is{present tense}\is{tense}) and began to combine \textit{búinn} with infinitival complements. Hearers, in turn, were left to make sense of the utterance, interpreting \isi{pluperfect} \textit{búinn} constructions as expressing that the event description associated with this infinitival complement is informative relative to the discourse context (cf. \citeauthor{rosemeyer2017road} \citeyear{rosemeyer2017road} for an analysis of a Spanish construction along similar lines). To be more precise, I am assuming a mechanism along the lines of \citet{eckardt2009apo}'s principle of Avoid Pragmatic Overload.\footnote{\citet[14]{eckardt2009apo} defines this principle as follows: ``The utterance under a conservative interpretation will trigger presuppositions\is{presupposition} that the hearer can not easily accommodate or refute. The hearer has three options: S/he can (a) be uncharitable and refuse to interpret the utterance at all, or (b) face the pragmatic overload and attempt to reconceptualize the world such that the \isi{presupposition} makes sense and is consistent, or (c) hypothesize a new meaning for parts of the utterance, notably the item that gave rise to the problematic \isi{presupposition}. Option (c) will allow the hearer to Avoid the Pragmatic Overload."}

One way of resolving informativity with regard to the discourse context here is to interpret the event description of the infinitival complement not as expressing a prospective, yet uninstantiated event but rather, highlighting the relationship between two instantiated eventualities. This results in the construction involving the participle \textit{búinn} alongside the infinitive being reanalysed. The hearer can reason that what is relevant is the temporal interval associated with the result state of the event described by the infinitive, which brings about a new state of affairs (cf. \citeauthor{rosemeyer2017road} \citeyear{rosemeyer2017road}). In other words, asserting the existence of this state has immediate consequences for what can happen afterwards (cf. \citeauthor{wide2002perfect} \citeyear{wide2002perfect} on \textit{búinn} in modern Icelandic discourse).

The question remains, however, of what constituted the system-internal motivation for the emergence of this marker, if Icelandic already had a canonical means of marking the \isi{perfect}  (i.e. with \textit{hafa} `have'). It seems to me that a key component of such a system-internal motivation is the considerable overlap between the \textit{hafa}-\isi{perfect}  and the simple past that could be observed in Old Icelandic. In narrative texts, Old Icelandic is also known for considerable \isi{tense} alternations between present\is{present tense}\is{tense} and past\is{past tense}\is{tense} for stylistic effect \citep{nygaard1905norron, heusler1921}. I take it that these facts led speakers of Icelandic to innovate a new marker for the purposes of highlighting the relationship between two events, i.e. the ``perfect-in-the-past" reading \citep{comrie1976aspect}. The canonical \textit{hafa}-\isi{perfect}, compatible with adverbs\is{adverb} that specify Reichenbachian event time (i.e. ``past-in-the-past" reading: \citeauthor{comrie1976aspect} \citeyear{comrie1976aspect}) and overlapping in use with the simple past, would have constituted a functional motivation for innovation.\footnote{The canonical \textit{hafa} perfect has an inferential flavour in older as well as modern stages of the language, which pertains to the ``source of information" \citep{liberman90}. This differentiates it from the simple past. However, it could also pave the way for a form to emerge that has a \textit{current relevance}\is{informative relevance} function \citep[52]{comrie1976aspect}. A ``past-in-the-past" reading can be seen as closer to a past perfective\is{past perfect} or remote past \citep{salkie1989perfect}. That pluperfects\is{pluperfect} can themselves incorporate a degree of current relevance\is{informative relevance} is discussed in \citet[102--103]{wada1995english} and \citet[72]{lee2017pluperfect}.} This fact should be considered in light of the emerging system for overtly marking aspectual\is{aspect} distinctions; \ili{Old Icelandic} already used \textit{taka} `take' as an inchoative marker \citep{nordling1928islandskt, holm1967epoker}. As I have discussed, at this stage we also see early focalised, aspectual\is{aspect} usages of the \isi{progressive}. I would like to propose that this constituted system-internal analogical motivation to complete this paradigm with a completive, an option that the chance reanalysis of the \textit{búinn} construction made possible.

These facts ought to, in turn, be considered in the context of inter-clausal relations being somewhat in flux during the transitional period of \ili{Early Modern Icelandic}, which I draw attention to later on in this paper (Section \ref{sec:Chark5.2}). I take it that the tighter integration of adverbial clauses\is{adverbial clause} at this point in time had its part, too, in encouraging the explicit marking of temporal overlap, a hallmark of unbounded\is{boundedness} narrative.

While \textit{búinn} is still young and/or not yet established, one finds a number of strategies in Icelandic for the expression of completive aspect. Constructions involving the \isi{past participle}  are often used to designate time in the place of subordinate clauses \citep[399]{hauksson1994islensk}. The \isi{past participle}  can also be found as part of a \textit{dativus ablativus}, which is a feature of \isi{learned style}\is{style} in Icelandic of the medieval period (ibid.). The latter is illustrated below in (\ref{ex:ablativus}).

\begin{exe}
\ex\label{ex:ablativus} 
\gll Og strax honum uppgengnum, þremur stykkjum affýrðum og kveðjum orðnum, afdró hann sinn ysta kyrtil og gekk svo til káhyttu. \\
and immediately he.\textsc{dat} up:go:\textsc{ptcp.pst.dat} three.\textsc{dat} pieces.\textsc{dat} off:fire.\textsc{ptcp.pst.dat} and farewells:\textsc{dat} become.\textsc{ptcp.pst.dat} off:threw he his outermost cloak and went so to cabin \\
\trans `When he was come on board and three pieces of cannon had been fired, and the greetings were ended, he threw off his outermost tunic and went down into the cabin.' \hfill \citep[111]{olafsson1908aefisaga}\footnote{Translation from \citet[203--204]{phillpotts2017life}.}
\end{exe}

These alternative strategies all reflect the situation at this point in the diachrony of the language in which no formal distinction is made between a stative, adjectival, reading of past participles\is{past participle} and that of an eventive passive, as shown below in (\ref{ex:eventivestative}). \citet{kress1982islandische} argues that pressure to fill in this paradigm was a factor in the grammaticalisation\is{grammaticalization} of \textit{búinn}, as it makes this distinction overt, as shown in (\ref{ex:eventivestativebuinn}).

\ea
  \ea\label{ex:eventivestative}{
\gll Þeir voru ráðnir. \\
they were hired \\
\glt `They were hired.' (eventive or stative possible, cf. \textit{Sie waren/wurden angestellt}) \hfill \citep{kress1982islandische}
  }
  \ex\label{ex:eventivestativebuinn}{
    \gll Það er búið að ráða þá. \\
    it is \textsc{búinn} to hire them\\
    \glt `They were/have been hired.' (eventive only) \hfill \citep{kress1982islandische}
  }
  \z
\z

Crucial for the purposes of this paper is the role of context-dependent inferences for the rise of the modern, completive meaning of \textit{búinn}. This modern usage arises in the language of those born shortly after 1550 (established by an investigation of writings by authors in and around this period by \cite{arnason1977buinn}). Early anterior readings of \textit{búinn} are found predominantly in temporal subordinate contexts: these act as \textit{switch} contexts \citep{heine2002role}, as the older meaning is typically ruled out given the main clause that follows. The importance of past-marked, temporal subordinate contexts for the establishment of the construction is clear when one investigates the proportion of \textit{búinn} tokens which appear in a subordinate clause expressing temporal sequentiality (which I term \textit{SuB} in the plot below). This is shown in Tables \ref{tab:subafter2} and \ref{tab:subafter1} for the 17th and 18th century respectively.\footnote{Here, 0 indicates the absence of the property, whereas 1 indicates its presence. More information on the texts can be found at \url{https://corpus.arnastofnun.is/leit.pl?info=2}. Rms-sitot designates examples that originate from \textit{Ritmálssafn Orðabókar Háskólans}, the archive of written language for the dictionary of Icelandic compiled at the University of Iceland. This corpus can be queried independently at \url{https://ritmalssafn.arnastofnun.is/}.}

\begin{table}
\small
\begin{tabularx}{.8\textwidth}{Xrrrrr}
\lsptoprule
Source & Century & SuB & Count & Total & Proportion \\ 
\midrule
(pislarsaga) & 17 & 0 & 14 & 16 & 0.88 \\ 
(pislarsaga) & 17 & 1 & 2 & 16 & 0.12 \\ 
(Rms-sitot-17) & 17 & 0 & 59 & 70 & 0.84 \\ 
(Rms-sitot-17) & 17 & 1 & 11 & 70 & 0.16 \\ 
\lspbottomrule
\end{tabularx}
\caption{Frequency of \textit{búinn} perfect in subordinate clauses expressing temporal sequentiality (18th century, \textit{Íslenskt textasafn})}
\label{tab:subafter2}
\end{table}

\begin{table}
\small
\begin{tabularx}{.8\textwidth}{Xrrrrr}
\lsptoprule
Source & Century & SuB & Count & Total & Proportion \\ 
\midrule
(ÆVISAGAN) & 18 & 0 & 74 & 91 & 0.81 \\ 
(ÆVISAGAN) & 18 & 1 & 17 & 91 & 0.19 \\ 
(campsid) & 18 & 0 & 6 & 11 & 0.55 \\ 
(campsid) & 18 & 1 & 5 & 11 & 0.45 \\ 
(eggbrud) & 18 & 0 & 19 & 20 & 0.95 \\ 
(eggbrud) & 18 & 1 & 1 & 20 & 0.05 \\ 
(klim) & 18 & 0 & 13 & 28 & 0.46 \\ 
(klim) & 18 & 1 & 15 & 28 & 0.54 \\ 
(martam) & 18 & 0 & 11 & 18 & 0.61 \\ 
(martam) & 18 & 1 & 7 & 18 & 0.39 \\ 
(Rms-sitot-18) & 18 & 0 & 148 & 166 & 0.89 \\ 
(Rms-sitot-18) & 18 & 1 & 18 & 166 & 0.11 \\ 
\lspbottomrule
\end{tabularx}
\caption{Frequency of \textit{búinn} perfect in subordinate clauses expressing temporal sequentiality (18th century, \textit{Íslenskt textasafn})}
\label{tab:subafter1}
\end{table}

This point is further illustrated in \figref{fig:image1}, which displays the distribution in \isi{IcePaHC} by clause type and century. We see that, while the innovative construction emerges in past-marked subordinate clauses, it subsequently gains ground in present\is{present tense}\is{tense}-marked main clauses in the 19th century. Finally, \figref{fig:image2} illustrates the time course of propagation for the \textit{búinn} construction. Relative frequency is given in percentage of total words. It remains somewhat marginal until the early 19th century, never exceeding 0.5\% relative frequency except for in one instance (\textit{klim}), after which point many texts in \isi{IcePaHC} either closely approach or surpass this threshold.

\begin{figure}
        \includegraphics[width=\textwidth]{figures/Fig2Buinndescr.png}
        \caption{Clause type/Tense by Century in IcePaHC}
        \label{fig:image1}
\end{figure}

 \begin{figure}
        \includegraphics[width=\textwidth]{figures/Fig3BuinnTimeCourse.png}
        \caption{Time course of relative frequency of \textit{búinn}+infinitive by text in IcePaHC, per 100 words. THe anterior reanalysis occurs ca. 1600.}
        \label{fig:image2}
\end{figure}

In \figref{fig:pastcom} and \figref{fig:prescom}, we see the proportion of \textit{búinn} constructions (\textit{búinn} followed by an infinitive) in the \isi{IcePaHC} corpus relative to all \isi{perfect}  environments, that is in competition with \textit{hafa}. The tokens are sorted by clause type: IP-MAT designates matrix clauses and IP-SUB designates subordinate clauses. The importance of past-marked subordinate clauses early on in the diffusion of the construction is apparent from these plots, as well.

\begin{figure}
  \caption{Pluperfect environments: \textit{búinn} and \textit{hafa} in IcePaHC (past-marked)}
\includegraphics[width=.95\textwidth]{figures/Fig6past.png}
\centering
\label{fig:pastcom}
\end{figure}

\begin{figure}
  \caption{Perfect environments: \textit{búinn} and \textit{hafa} in IcePaHC (present-marked)}
\includegraphics[width=.95\textwidth]{figures/Fig5present.png}
\centering
\label{fig:prescom}
\end{figure}


%TD: Sub_After dataset

\subsection{The progressive in Icelandic: Diachrony}\label{sec:Chark2.3}

The Icelandic \isi{progressive} conforms to a cross-linguistically established trajectory, emerging from a locative \citep{bybee1994evolution} e.g. \textit{vera að kirkju} `be at church', \textit{vera að smíðum} `be at building' \citep{benediktsson1976isl}. \citet[129]{bybee1994evolution} observe within their typological dataset a tendency for \isi{progressive}  markers to originate from source constructions that are locative expressions of some kind. The authors write that such locatives are ``probably ‘be in the place of verbing' or ‘be at verbing''' (ibid. 130).\footnote{That such a construction involves a copula is also cross-linguistically common, which is perhaps not surprising given the semantic import of copular predication.} This function then seems to have undergone metaphorical or metonymic extension, whereby the spatial concept of location within an activity is extended to the temporal domain.

The Icelandic \isi{progressive}  construction combines the copula \textit{vera} `to be' accompanied by double \textit{að}, which functions as a preposition and infinitival marker. These fall together starting in the 14th century (ibid.).\footnote{An anonymous reviewer asks whether this is due to phonological or, alternatively, phonetic fast-speech effects. As discussed by \citet{benediktsson1976isl}, it is not entirely clear what the cause is. In any case, the doubled form is older (ibid.). } Progressive\is{progressive} readings provide an \textit{internal perspective} of an event \citep{landman1992progressive}. This corresponds to ongoingness, since the bounds of an event's inception or culmination are not taken into account.\footnote{An additional resource regarding the development of the \isi{progressive}  is \citet{nordling1928islandskt}. Its status in modern Icelandic is discussed comprehensively in \citet{johannsdottir2011aspects}.} 

\citet{bertinetto2000progressive} propose two fundamental types of readings for \isi{progressive}  constructions across Germanic and Romance: focalised and durative. The focalised type is characterised predominantly by the presence of a focalisation instant, which provides a vantage point (\textit{As X occurred, I was Y-ing}): the event's ongoingness is viewed through a singular focal point. Durative progressives,\is{progressive} by contrast, are evaluated relative to a larger temporal interval and are therefore often associated with adverbial phrases such as \textit{during}, \textit{until}, and \textit{since} \citep[76]{Killie2008}.

An early example of an Icelandic \isi{progressive}  which is plausibly, though not definitively, focalised is shown below in (\ref{ex:focal}). The wider context provides a singular focalisation point through which this event may be viewed, namely the man picking up a bundle of cloth. However, the focal point for this could also be the more specific timeframe given by `when they were setting up the choir area' (\textit{er þeir tjölduðu sönghúsið}).\footnote{The verb \textit{tjölduðu} is \isi{past tense}\is{tense}. It refers to the practice of decorating the walls of a church with cloth \citep[27]{kern2019oh}.}

\ea\label{ex:focal} 
\gll Menn voru að tjalda kirkju á einhverjum bæ.  \\
people were to decorate church at some farm\\ 
\glt
`People were decorating a church (with cloth) at some farm...' (But when they were setting up the choir area, a man picked up a bundle of cloth from the floor near the altar...) \hfill [\textbf{1210}, JARTEIN-REL.SAG.2553]
\z

Literature on the diachronic development of progressives\is{progressive}  cross-linguistically often assumes a progression from durative to focalised readings \citep{bertinetto2000progressive, bertinetto2000progressive2}. Looking at Germanic in particular, \citet{killie2014development} (see also \cite{Killie2008}) provides a quantitative investigation of the \isi{progressive}  in the diachrony of \ili{English} through the Helsinki Corpus. The results in \citet{killie2014development} demonstrate that, despite the fact that focalised progressives\is{progressive} are more frequent than duratives in all time periods (Old, Middle and Early Modern \ili{English}), one can observe an increase in the prevalence of focalised readings from one period to the next. This dataset is thus somewhat inconclusive: it is amenable to the proposal that duratives constitute an earlier stage, but this stage would have preceded the evidence in the corpus record. On the other hand, it remains possible that focalised and durative usages existed alongside one another from the earliest stages.\footnote{A further complication is presented by the fact that in the \ili{Old English} period include other usages, which appear to be non-aspectual. These include what \citet{Killie2008} terms \textit{narrative} progressives,\is{progressive} which mark points of narrative climax for emphatic effect, as well as stative progressives,\is{progressive} which denote ``feelings, eternal truths, habits". Habitual progressives\is{progressive} resemble the latter and are classified separately by \citet{Killie2008}.}

As for the progression in the history of Icelandic, I carried out a preliminary corpus investigation focusing especially on the transitional stage of reanalysis. To do so, I investigated the portion of \textit{Íslenskt textasafn} (16-18th century subcorpus) which includes the 16th century New Testament translation of Oddur Gottskálksson. This investigation was further supplemented by an assessment of the Old Icelandic examples discussed by \citet{benediktsson1976isl} in his article on the development of the \isi{periphrasis}. We begin with the latter. 

As we see below in Table \ref{tab:classification}, Old Icelandic differs from \ili{Old English} in that durative examples are most common, with focalised usages coming in second. There are also a handful of indeterminate usages, which are somewhat ambiguous between durative and focalised. They do not clearly provide either a singular or multiple focal frame. Habitual usages are also represented in this corpus, as in \ili{Old English} (cf. \citeauthor{Killie2008} \citeyear{Killie2008}).

\begin{table}
\begin{tabular}{lr}
\lsptoprule
Classification & Number of examples \\ 
\midrule
Durative                 & 15  \\
Focal                    & 10  \\
Indeterminate (Indet)    & 8   \\
Stative habitual         & 3   \\
\lspbottomrule
\end{tabular}
\caption{Classification of progressive reading types in Old Icelandic examples listed by \citet{benediktsson1976isl}}
\label{tab:classification}
\end{table}

Moving now to the New Testament translation examples, I could find seven focalised usages, as opposed to one durative usage and two stative habitual usages. In (\ref{ex:focal_oddur}), a focalised example is provided. The \isi{progressive}  construction \textit{en á meðan þeir voru að búa til} `but while it was being prepared' sets up a focal frame for what follows.

\ea\label{ex:focal_oddur} 
\gll Og er hann hungraði, vildi hann matar neyta. En á meðan þeir voru að búa til, leið yfir hann brjósthöfgi og sá himininn opinn og ofan fara að sér disk nokkurn mikinn svo sem línlak í fjórum hyrningum upp bundinn.  \\
and as he hungered wanted he food consume but when {} they were to prepare {} came over him breast:drowsiness and saw the:sky open and from.above go to himself disc some great so as linen:sheet in four corners up bound\\
\glt `He became hungry and wanted something to eat; but while it was being prepared, he fell into a trance, and saw that the heavens were open, and that something like a great sail was descending, let down by its four corners towards the earth.' \\ \hfill (Acts 10; Nýja testamenti Odds, Íslenskt textasafn: \citeauthor{IslensktTextasafn2019} \citeyear{IslensktTextasafn2019})\footnote{Translation from the Open English Bible \citep{oeb2022}.} \z

Below in (\ref{ex:durative_oddur}), a durative example is shown. As opposed to the focalised example above, there is no singular focal point through which the progressive-marked\is{progressive} eventuality is viewed. Instead, the event in question (\textit{þeir voru að tala um alla þá hluti sem við höfðu borið} `they were talking about all those things that had taken place') is viewed as ``relative to a larger interval" \citep[527]{bertinetto2000progressive}. In this instance, the larger interval in question is the journey from Jerusalem to Emmaus.

\ea\label{ex:durative_oddur} 
\gll Og sjá, að tveir af þeim gengu á þann sama dag til nokkurs kauptúns það er var frá Jerúsalem rúms sextigi skeiða, hvert eð Emmahus var að nafni. Og þeir voru að tala um alla þá hluti sem við höfðu borið. Og það skeði þá þeir ræddust við og spurðust á sín í millum að sjálfur Jesús nálægist þá og gekk jafnframt þeim.\\
and behold, that two of them went on that same day to some village \textsc{dem.n} \textsc{rel} was from Jerusalem about sixty measures which \textsc{comp} Emmahus was by name and they were to talk about all those things which {} had occurred and it happened when they conversed {} and asked {} themselves between {} that himself Jesus approached them and went alongside them\\
\glt `It happened that very day that two of the disciples were going to a village called Emmaus, which was about seven miles from Jerusalem, talking together, as they went, about all that had just taken place. While they were talking about these things and discussing them, Jesus himself came up and went on their way with them.' \\ \hfill (Luke 24; Nýja testamenti Odds, Íslenskt textasafn: \citeauthor{IslensktTextasafn2019} \citeyear{IslensktTextasafn2019})\footnote{Translation from the Open English Bible \citep{oeb2022}.}
\z

In conclusion, the corpus investigation at hand reveals that the development of the \isi{progressive}  \isi{periphrasis} in Icelandic does not neatly fit the path described by Bertinetto. Rather, it more closely resembles what Killie describes for the diachrony of \ili{English}, in the sense that durative, focalised, as well as stative, non-aspectual usages are represented in the earliest stage of the corpus. Moving forward a couple of centuries to the 1540 New Testament translation, it can be seen that focalised usages predominate, in contrast to the earlier texts discussed by \citet{benediktsson1976isl}. Further investigation is needed into the details of this trajectory, but we can nevertheless establish that focalised usages became more entrenched in the language in the 16th century and beyond. 

The \isi{IcePaHC} corpus was queried for the \isi{progressive}  construction. Given the form of the construction, there is potential for false positives: combinations of the copula and infinitival marker show up in other environments (e.g. \textit{Það er að segja} `that is to say'). It was therefore necessary to manually sort out valid instances: I manually tagged tokens which could receive a focalised interpretation. The results are displayed in \figref{fig:image3} (clause type and \isi{tense} by century) and \figref{fig:image4} showing the timecourse of propagation.

\begin{figure}
        \includegraphics[width=.95\textwidth]{figures/progprop.png}
        \caption{Clause type/Tense by Century in IcePaHC}
        \label{fig:image3}
\end{figure}

\largerpage
As in the case of \textit{búinn}, \textit{PROG} remains relatively infrequent until the early 19th century. Focusing on our period of interest, what stands out is the higher frequency of subordinate instances in the 16th and 17th centuries. Though numerically the difference is not large, I would like to argue that these subordinate instances are especially important for the establishment of the focalised usage. This is due to the fact that temporal subordinate clauses provide the appropriate frame for focalisation. In summary, we see that, despite the presence of modern-like focalised usages of the \textit{PROG} construction as early as the 13th century, it does not become established in the language until much later. 

\newpage
\largerpage
To better understand the status of the \isi{progressive}  in the early modern period, I also examined attested examples of the construction in the New Testament translation by Oddur Gottskálksson (1514--1556) in \textit{Íslenskt textasafn}. Oddur, while a user of the focalised construction, does not use it in some instances where it would be obligatory in modern Icelandic. \citet[135--136]{helgason1929malid}, for instance, writes that Oddur uses the \isi{progressive}  ``not quite as much as would be done today''. Later authors represented in this corpus, such as Jón Ólafsson úr Grunnavík (\textit{Nikulás Klím}) and Jón Steingrímsson (\textit{Ævisaga Jóns prófasts Steingrímssonar}) in the 18th century, both have a \isi{progressive}  in their grammar with a division of labour which resembles the modern usage. Future research is necessary in order to map out the diffusion of the \isi{progressive}  construction in the language, on the path towards obligatoriness in focalised instances. A classification of \isi{progressive}  instances in \isi{IcePaHC} (manually tagged) broadly supports this view, as shown in \figref{fig:progicepahcclass}.\footnote{Note that only unambiguously focalised cases were counted, so the 13th century example discussed in Figure \ref{ex:focal} is not classified as such.}
\clearpage


\begin{figure}[t]
        \includegraphics[width=\textwidth]{figures/prog_timecourse.png}
        \caption{Time course of \textit{búinn}. Relative frequency over time in IcePaHC (by text)}
        \label{fig:image4}
\end{figure}

 \begin{figure}
        \includegraphics[width=\textwidth]{figures/Figure9focal.png}
        \caption{Progressives in IcePaHC prior to 1900 classified according to \textit{focalisation} (n=68)}
        \label{fig:progicepahcclass}
\end{figure}


\section{Tracing discourse traditions in the Icelandic context}\label{sec:Chark3}

In the following section, I argue that a shift in conventionalised strategies for structuring events and their overlap in narrative can be traced through the prominent discourse traditions\is{discourse tradition} of the early modern Icelandic period and that this shift is represented in our corpus evidence of the grammar of the language. Section \ref{sec:Chark3.1} introduces the term \textit{discourse tradition}.\is{discourse tradition} Section \ref{sec:Chark3.2} discusses the outset for our investigation, the saga \isi{style} narrative prominent in the corpus of Old Icelandic prior to 1540. Section \ref{sec:Chark3.3} discusses the 17th and 18th century memoirs and travelogues which are an intermediate stage, I argue, with regard to the question of \isi{boundedness}. Finally, Section \ref{sec:Chark3.4} discusses the relationship between narrative traditions and \isi{orality}.

\subsection{On discourse traditions}\label{sec:Chark3.1}
\largerpage
Linguistic form has conventionalised associations with situations of language use, associations which are themselves always subject to change\ \textendash{}\  these are what \citet{koch1987distanz} calls \textit{discourse traditions}\is{discourse tradition} (henceforth: DT).\footnote{This discourse tradition model effectively adds a diachronic dimension to the notion of ``\isi{register}", which can be defined as ``those aspects of socially recurring intra-individual variation that are influenced by situational and functional settings" \citep[3]{Luedeling202frametext}.} As \citet{kabatek2024} spells out, changes can occur from within a certain \isi{discourse tradition}  and spread to others from there. The notion of DT\is{discourse tradition}  as proposed by \citet{kabatek2024} is somewhat more expansive than the view espoused by \citet[123]{labov1972sociolinguistic} who 
conceives of changes as either originating \textit{from below} (common vernacular) or \textit{from above} (prestige varieties). In the DT\is{discourse tradition}  framework, language diachrony is modeled as consisting of developments parallel to one another temporally but differing in the degree to which they are conventionally tied to situations of language use.\footnote{\citet{kabatek2024} assumes a continuum of situations from communicative distance to communicative immediacy \citep{koch1985sprache}.}

In what follows, I will advance the claim that discourse traditions\is{discourse tradition} surrounding Icelandic narratives, specifically how speakers construe events and their overlap in written language, underwent a shift between the 16th century and today.

A prominent \isi{discourse tradition}  of the Old Icelandic period is termed saga-\isi{style}, as it is exemplified by \isi{style} found in the \textit{Íslendingasögur} (`Sagas of the Icelanders').\is{Icelandic saga} This \isi{style} has a number of characteristics pertaining to the mapping between grammar and discourse structure: it is described by \citet{hauksson1994islensk} as being `common' (\textit{alþýðustíll}); `learned' (\textit{lærður}) characteristics are not prevalent.\footnote{`Common' refers to the language of the common people.} The representation of events in saga \isi{style} narrative is iconic: events are described in the order in which they occur. Relations between events are left largely underspecified. Thus, on the whole, texts of this sort are dominated by relatively brief matrix clauses, strung together paratactically\is{parataxis} (e.g. with the coordinating conjunctions\is{conjunction} \textit{ok} `and', \textit{en} `but', or non-coordinating, e.g. \textit{þá} `then', \textit{nú} `now', \textit{síðan} `since'). %Subordinate clauses are uncommon at this stage (cf. Walkden 2021…). 

The way in which word order transformations map onto pragmatic functions resembles modern Icelandic, but there are some differences \citep{holm1967epoker, hauksson1994islensk}. Two ways in which the older system differs from the modern one are with regard to the frequency of V1\is{verb-first} and with regard to stylistic fronting (movement of constituents to the left periphery which otherwise are base generated in the post-finite domain). \citet{booth-beck20200jhs} link both of these changes to the emerging status of a dedicated subject position. The authors (ibid.) also observe, in the \isi{IcePaHC} corpus, a temporally concomitant decrease of non-subject initial V2\is{verb-second} clauses\ \textendash{}\  on the whole, it becomes increasingly dispreferred for subject topics to appear post-finitely: their corpus evidence (from \isi{IcePaHC}) points to a period of considerable intertextual variation around 1600, in which religious and biographical texts, as opposed to narratives, pattern in the opposite direction, which then gives way to a constant decrease in frequency.

For concreteness, I assume that both the \isi{progressive}  and the \textit{búinn} \isi{perfect}  are first reanalysed prior to or at the early stages of the shift in \isi{boundedness}. The circumstances that allowed for reanalysis (or supported the subsequent early entrenchment of the novel constructions) are independent of the shift in \isi{boundedness}. 

What is the relationship between changes in the language system on the one hand, and changes in norms surrounding narrative on the other?\footnote{Thanks are due to an anonymous reviewer for prompting me to elaborate on this point.} Language-specific properties have been shown to influence the way events are conceptualised and subsequently expressed linguistically: there exists experimental evidence for this in production/comprehension from monolinguals \citep{vonstutterheim2003} as well as in acquisition \citep{carroll2003information, flecken2013}. That is to say, if there exists a conventionally coded (grammaticalised\is{grammaticalization} option) distinction in a given language, it is more likely to be  used than if only alternative, less conventionalised means, are available. Such work is couched within \citeauthor{slobin1994talking}'s (\citeyear{slobin1994talking}) ``thinking for speaking'' hypothesis, which adopts the perspective that linguistic variation in terms of what is conventionally coded and that which is not serves as a kind of interface between the level of abstract conceptualisation and linguistic production \citep{levelt1993speaking}. While ``macro-planning" (ibid.) refers to the process where speakers determine what it is they want to say, ``micro-planning" involves, among other things, the decisions a speaker makes in terms of intra- as well as inter-clausal relations, such as the use of referring expressions and whether to use coordinating\is{coordinating discourse relations}\is{discourse relation} or \isi{subordinating discourse relations} \citep{asher2005}.


There is, on the whole, considerable converging evidence regarding language-specific properties and their consequences for information packaging from cross-linguistic comparison in monolinguals (e.g. \citeauthor{starren2017} \citeyear{starren2017}), acquisition (e.g.  \citeauthor{schmiedtova2011} \citeyear{schmiedtova2011}), bilingualism (e.g. \citeauthor{daller2011} \citeyear{daller2011}) and in historical-typological research \citep{starren2017}. To summarise: the available evidence in the literature appears to support a view according to which language-specific properties, defined by which conceptual distinctions receive conventional encoding, result in a given language's propensity for bounded\is{boundedness} or unbounded\is{boundedness} construals. 

As touched on above, I assume that certain discourse traditions\is{discourse tradition} can become conventionally associated with certain linguistic co-occurrence patterns \citep{kabatek2024} (lexical and grammatical), which I assume occurs via a process of entrenchment (as conceived of by e.g. \citeauthor{schmid2015} \citeyear{schmid2015}). Saga \isi{style} carries with it conventionalised associations to co-occurrence patterns of lexical, grammatical and structural characteristics.\footnote{As pointed out by an anonymous reviewer, saga style is typically conceived of as a written norm, which is partially what enabled the revival of some of its characteristics in modern Icelandic narrative style.} Discussing the structure of the Icelandic family saga,\is{Icelandic saga} \citet[33]{andersson1967} writes: ``There is no such thing as a digression [...] All the episodes are linked in a sequence leading up to the climax or leading down from it. This is a fundamental rule and is the key to saga economy''. The highly paratactic\is{parataxis} nature of saga style\is{style} is characterised by relatively minimal temporal backtracking or divergence from the main narrative storyline: ``even with the most artless story-telling in the paractic style,\is{style} however, there will always be greater or lesser degrees of syntactic divagation from the straight line of narrative events'' \citep[392]{amory1980narrative}. Such divergences, in narrative generally, serve evaluative functions (cf. \citeauthor{LabovWaletzky1967} \citeyear[26--35]{LabovWaletzky1967}) and as argued by \citet{amory1980narrative}, saga writers used a number of stylistic devices to insert their evaluation into the narrative despite being ``notoriously non-committal about their opinions" \citep[394]{amory1980narrative}. The paratactic\is{parataxis} \isi{style} is well-suited for such non-committal, minimally evaluated narrative.

``Saga \isi{style}" refers here to what is termed ``popular \isi{style}" and typically contrasted with ``\isi{learned style}",\is{style} the latter abounding with Latinate characteristics\il{Latin} \citep{nygaard1905norron, Kristjansson1985, hauksson1994islensk, smaari1920}. Latinate characteristics, as touched upon earlier on in this paper, include the appositive use of present participles and absolutive constructions (\textit{dativus ablativus}), both of which represent devices for communicating temporal sequentiality and overlap, with these specific grammatical devices presumably being favoured due to translation influence\is{transfer from Latin} (cf. \citeauthor{brendel2023translation} \citeyear{brendel2023translation}). This distinction between learned characteristics on the one hand and popular ones on the other thus provides an avenue for characterising the predominant discourse traditions\is{discourse tradition} in Icelandic prior to the 16th century. Already at this stage, we observe divergences between these traditions in the way in which events are structured. This seems to be shaped by the available grammatical means; translation influences\is{transfer from Latin} led to the use of constructions which were presumably less frequent/conventionalised in the spoken vernacular but rendered more salient due to the influence of Latin,\is{transfer from Latin} which in turn could impact the subsequent development of the language (within this particular \isi{discourse tradition}  and beyond, cf. \citeauthor{wagner2013scribes} \citeyear{wagner2013scribes} on scribes as agents of \isi{language change}).

Our picture of the diachronic developments of a language are inevitably skewed by available corpus evidence. For Icelandic, the evidence furnished for different periods is shaped by dominant discourse traditions\is{discourse tradition} including the influence of conceptual \isi{orality}/writtenness (cf. \citeauthor{koch1985sprache} \citeyear{koch1985sprache}). The prevalence of aspectual\is{aspect} expressions is, for instance, often tied to narratives in contrast with expository texts \citep{ragnarsdottir2002verbal}. Overall, I think the corpus evidence has to be carefully considered in tandem with the socio-historical processes that shape discourse traditions.\is{discourse tradition} 

The diachrony of Icelandic provides a powerful case study for the role of emerging norms and cultural processes in shaping discourse traditions.\is{discourse tradition} The modern literary standard begins to take shape in the 19th century with a concentrated effort to bring the language closer to the \isi{narrative style}\is{style} of the sagas \citep{bernhardhsson2017jon, bernhardhsson2018spreading}. Having now given some background on the norms surrounding written narratives for Icelandic, I now proceed to discuss narrative structure in more detail. This lays the groundwork for the discussion in Section \ref{sec:Chark4}, where I argue that Icelandic displays a shift towards unboundedness\is{boundedness} between the 16th and 19th centuries.

\subsection{On narrative structure}\label{sec:Chark3.2}

Narratives can be segmented into episodes, which are a level of textual organisation situated between the smaller sentence-level and larger discourse-level \citep{van1976philosophy, enkvist1987old, Fleischman1990}. These episodes are characterised by thematic or temporal coherence and episode transitions thus correspond to transitions of location, participants, perspective, among others. Episode transitions are commonly marked linguistically \citep{Brinton1996, warvik1995a}: for instance by spatial or temporal adverbials or \isi{tense} alternations.

Narratives in the saga \isi{style} have a formulaic structure, often linked to their oral origins \citep{hauksson1994islensk}. Narrative episodes often begin, for instance, with formulaic introductions and scene setting, which then transitions into the description of a series of events \citep{clover1974scene}. Tense\is{tense} alternations are very common: \isi{past tense}\is{tense} alternates with historical present\is{present tense}\is{tense}, the latter giving the sense of the primary \isi{focus} of the narrative episode coming more clearly into \isi{focus} \citep{Fleischman1985}. 

%TD:Clover?

%These alternations have also been linked to expressive and metalinguistic functions of other kinds (see Fleischman 1990 on the textual and expressive role of tense and aspectual alternations more broadly).

The way in which episode boundaries are marked overtly relates to the notion of grounding \citep{hopper1979some}. The foreground is, most basically speaking, comprised of those events which comprise the main storyline\ \textendash{}\  everything else is background. On the whole, one can distinguish between the linguistic expressions most associated with backgrounding and foregrounding, though these are non-deterministic correlations \citep[49]{Brinton1996}. Adverbial clauses,\is{adverbial clause} the primary environment of interest, are frame setters \citep{talmy1978figure}\ \textendash{}\  they signal continuity in a ``temporal line" \citep{Fleischman1985}. 

The two expressions in question, \textit{búinn} and \textit{PROG}, pattern with other imperfective elements in that they communicate temporal overlap and serve to reorient the reader to the narrative timeline \citep{Fludernik2012}. They are backgrounded in the sense that they do not advance this timeline but rather have the function of providing ``guideposts" for the reader \citep{chafe1984people}. However, they differ from prototypically backgrounded elements in that they can provide new information\is{discourse-new} \citep{reinhart1984principles, Fleischman1985} and are therefore well-suited to beginning a novel narrative episode. In the case of progressives,\is{progressive} their focalised usage most clearly has this function, setting up a new temporal frame, which establishes a new topic time (in the sense of \citeauthor{klein1994time} \citeyear{klein1994time}) as a vantage point through which subsequent events are viewed (see e.g. the example in \ref{ex:ps}).\footnote{This is in accordance with the definition of frame setters given in \citet[270]{krifka2008basic}:  ``the frame setter indicates that the information actually provided is restricted to the particular dimension specified." Focalised progressives\is{progressive} (e.g. in \textit{when}-clauses) restrict the following clause on the temporal dimension.} Likewise, the \textit{búinn}-\isi{perfect}  often functions as a frame-setting strategy where it appears in temporal adverbial clauses,\is{adverbial clause} which typically serve to bring the reader up to date, setting the stage for what can subsequently occur (see e.g.  \ref{ex:klim1}).

\subsection{17th and 18th century memoirs and travelogues}\label{sec:Chark3.3}
Travelogues and memoirs of the 17th-18th century combine so-called \textit{lærður stíll} `learned \isi{style}' and \textit{alþýðustíll} `common/popular \isi{style}' \citep{hauksson1994islensk}.\footnote{Texts in \isi{IcePaHC} with such characteristics include: \textit{Nikulás Klím}, \textit{Ævisaga Jóns Ólafssonar Indíafara}, \textit{Reisubók séra Ólafs Egilssonar} and \textit{Píslarsaga Jóns Magnússonar}.} These texts contain dialogues\is{dialogue} and \isi{indirect speech} reminiscent of earlier styles,\is{style} specifically saga writing. Direct event descriptions are often paratactic\is{parataxis} and iconic. Characteristics associated with the \isi{learned style} are also present, however, including expository passages (usually interpreting events described) with high levels of \isi{subordination} (ibid.). These travelogues and memoirs additionally exhibit influence from Low German\is{transfer from Low German} and Danish\is{transfer from Danish} due to language contact. This influence has been under-researched and is not well understood, though it is undeniably strongly present in texts of this \isi{genre} at all linguistic levels (e.g. syntax and lexicon) \citep{ottosson1990islensk, hauksson1994islensk, sigmundsson1998}. 

It is precisely these texts, as far as our corpus dataset is concerned, that constitute the bridge between saga-\isi{style} narratives and the emerging modern literary standard, as, for instance, reflected in Jón Thoroddsen's novels \citep{bernhardhsson2018spreading}. These texts are also the first in the corpus to exhibit frequencies of \textit{PROG} and \textit{búinn} which exceed 0.25\% (overall relative frequency).

\subsection{Adverbial clauses and orality}\label{sec:Chark3.4}

%TD: Citation for Haiman, Slobin, Chafe, Walkden

As we move forward, we will see that as the predominant \isi{narrative style}\is{style} changes, there is a consequent shift from a mainly iconic, strictly chronological representation of events to other new strategies for construing event sequentiality as reflected in clause structure, namely overt grounding by means of (predominantly preposed) temporal adverbial clauses.\is{adverbial clause} Contrasting the two poles, \isi{hypotaxis} is less dependent on the inference of causal and temporal relations, as these are now expressed. Saga \isi{style} is quite sparse in these terms. \citet[39--71]{haiman1985natural} claims that there can be a kind of trade-off between linear order and other mechanisms which convey the way in which clauses relate to one another. This is roughly the line of analysis which I pursue here. In a predominantly unbounded\is{boundedness} system, the functional domain previously occupied by local anchoring adverbials (by means of \isi{tail-linking}: \cite{los2012}) gives way to new strategies: backgrounded, temporal clauses situate events within a global topic time.\footnote{The bounded/unbounded distinction as conceived of by \citet{carroll2003information} must be conceived of as a gradient of preferred ways in which to structure narrative. It is not absolute nor is it assumed that all situations of language use behave identically in terms of preference for boundedness.}

\citet{warvik1995a} and \citet{Brinton1996} posit that an analogous shift in  OE\il{Old English}\ \textendash{}\  from main clause constructions to subordinate \textit{whan} clauses\ \textendash{}\  constitutes such a shift.\footnote{Middle English \textit{whan} corresponds to PDE \textit{when}.} \citet[179]{Brinton1996} argues that an increase in \isi{subordination} reflects the higher degree of ``integration" reflected in (conceptually) written language, thus correlating with a degree of plannedness (cf. ``thinking for writing" in the terms of \cite{slobin2003language}).

Corpus work on \isi{register} variation broadly supports this general view but the picture is not straightforward: adverbial \isi{subordination} is characteristic of written expository registers\is{register} in PDE, however, it also correlates with the ``interactive'' dimension of oral communication \citep{biber2005register}. \citet{walkden2021parataxis} investigates the ratio of \isi{hypotaxis} to \isi{parataxis} in \isi{IcePaHC}. He finds that saga texts\is{Icelandic saga} exhibit most \isi{parataxis} on average. Religious and non-fiction texts exhibit the highest degree of \isi{hypotaxis}, especially in the 17th-19th centuries.

\section{From boundedness to unboundedness}\label{sec:Chark4}

\subsection{Motivation behind approach}\label{sec:Chark4.1}
The main idea behind the approach pursued in this paper is as follows: A fundamental shift in mapping between clause and discourse/information structure occurred in the diachrony of Icelandic, with consequences for the expression of aspect. To date, scholarly work in this area has focused on the syntactic reflexes of these changes \citep{holmberg2000scandinavian, sigurdsson2018icelandic, hroarsdottir2000word, booth2018, booth-beck20200jhs}. In what follows, I will show that the emergence of \textit{búinn} and \textit{PROG} was a consequence this broader shift in mapping to clause structure. 

The syntactic reflexes of this shift are often discussed together in the literature, especially in reference to the relationship between the pre-finite position and subjecthood \citep{jonsson1992, barddal-eythorsson03, booth-beck20200jhs}. These include a decrease in \textit{Stylistic Fronting} (when, in clauses with a gap, non-subject XPs are fronted to the pre-finite position) \citep{holmberg2000scandinavian}, a decrease in V1 declaratives\is{verb-initial declaratives}\is{verb-first} \citep{sigurdsson2018icelandic,booth-beck20200jhs}, as well a decrease in argument drop, accompanied by an increase in the \isi{expletive} element \textit{það} \citep{rognvaldsson1995old, rognvaldsson2002thadh, booth2018}. \citet{booth_revisiting_2021} argues that this shift can be understood at least partially in terms of the emergence of SpecIP as a dedicated subject position; the inflection point for this change is determined to be between 1500-1700 (based on \isi{IcePaHC}).

What is novel in the present approach is its ability to explain otherwise puzzling facts, such as the time course of propogation of these aspectual\is{aspect} constructions. I argue that these can be better understood if one takes the shift in discourse structuring as fundamental. This shift led to fertile ground for emerging aspectual\is{aspect} constructions to conventionalise. 

How can the shift be characterised more broadly, moving beyond the syntactic phenomena more often discussed? My claim is that the shift in mapping between form and discourse structure in the history of Icelandic represents, more broadly, a shift from a predominantly ``bounded" system for expressing narrative to a predominantly ``unbounded" one (for related work on the diachrony of \ili{English} \citep{petre2010functions, petre2014constructions, los2012, fanego2024english}.

\subsection{(Un)Boundedness}\label{sec:Chark4.2}
%TD:Citations: Kupersmitt, Hickmann, Berman, Berman and Slobin
What is the notion of ``\isi{boundedness}" that I use here? \citet{von2012grammaticized, von2002cross} propose, on the basis of psycholinguistic experiments, an account of how the grammatical system can constrain how events are expressed in narratives: Bounded\is{boundedness}  language invokes a series of completed events strung together one after the other in a sequence, while unbounded\is{boundedness} language views events in terms of their co-articulation within an overarching topic-time. Bounded\is{boundedness}  construals thus broadly correspond to perfective/telic\is{perfect} eventualities and unbounded\is{boundedness} ones to imperfective/atelic \citep{kupersmitt2015moving}. 

Contrasting modern day \ili{English} and \ili{German}, the former seems to prefer unbounded\is{boundedness} construals, and the latter bounded\is{boundedness} ones, for constructing narrative sequences \citep{vonstutterheim2005crosslinguistic, berman2013relating}. There are other linguistic reflexes of \isi{boundedness} which are discussed in this literature \citep{vonstutterheim2005crosslinguistic, kupersmitt2015moving}: \ili{German}, for instance, makes heavy use of topic-time adverbials and scene-setting on the left periphery which allows for anaphoric continuity. On the other hand, \ili{English} makes little use of such topic-time adverbials, instead the determination of topic-time is often context-dependent and there is more focus on overtly marking the co-articulation of events within the interval provided by topic-time.

Older Germanic more closely resembles modern \ili{German} than \ili{English}. The linguistic reflexes present in these languages include heavy use of topic-time adverbials and word order transformations for anaphoricity/discontinuity \citep{los2012}. The bounded-unbounded transition\is{boundedness} in the history of \ili{English} is reflected in the loss of these topic-time adverbials, and the loss of V2\is{verb-second} (ibid.).\footnote{Note, however, that there are alternative perspectives on the loss of V2,\is{verb-second} e.g. \citet{haeberli2002inflectional} who reduces the change to a consequence of a loss of inflectional morphology or \citet{speyer2010topicalization} who proposes a primary role for prosodic factors. I thank an anonymous reviewer for making this point.} An increase in the frequency of aspectual\is{aspect} constructions could be observed: the \isi{progressive}  \textit{BE -ende}, the ingressive aspectual\is{aspect} verb \textit{beginnan} \citep{petre2010functions, Los2000}. Alongside these latter changes one also observes an increase in preposed \textit{whan}-clauses \citep[169--172]{Brinton1996}. The relevance of work on \ili{English} for the Icelandic data is discussed later in this paper (Section \ref{sec:Chark5.1.1}). We turn now to the data from successive stages of Icelandic, demonstrating the language's shift from a bounded\is{boundedness} to unbounded system.

\subsubsection{Old Icelandic as a bounded system}\label{sec:Chark4.2.1}

In this section, I argue that Old(er) Icelandic (prior to 1540) is a language which exhibits a strong preference for bounded\is{boundedness} construals, in the sense of \citet{von2002cross}. In order to discuss the ways in which this is reflected in the grammar, I begin by describing the basic characteristics of Old Icelandic grammar. Icelandic is at all stages a strictly V2\is{verb-second} language. The question of what may be considered the basic or unmarked word order in Old(er) Icelandic is not entirely straightforward, as SVO and SOV co-exist for many centuries \citep{sigurdsson1988from, rognvaldsson1994breytileg, rognvaldsson1995old, rognvaldsson1996word}\ \textendash{}\  the latter only disappears entirely in the 19th century \citep{hroarsdottir2000word}. Icelandic at all stages makes heavy use of transformations for pragmatic purposes, including operations such as topicalisation,\is{topicalization} extraposition, and verb-initial orders \citep{Faarlund1990, booth_revisiting_2021}. Old(er) Icelandic can, in fact, be considered to have Topic-Verb-Subject word order \citep{nielsen2017remark}.

Verb-first\is{verb-first} orders are used for discourse cohesion in order to express anaphoric continuity \citep{booth-beck20200jhs}, often termed ``Narrative Inversion" \citep{Platzack1985}. In narrative sequences, this results in continuity along the main line of \isi{narration}.\footnote{On the relationship between narration and topic anaphoricity in the diachrony of Icelandic, see \citet[20]{booth-beck20200jhs}. Relatedly, \citet{Kinn2016NullSubjects} discuss the connection between V1\is{verb-first} and null subjects.} Additionally, there is an argument to be made that the pragmatic effect of V1\is{verb-first} is preserved even when the surface word exhibits V2;\is{verb-second} this is the case in the presence of the coordinating \isi{conjunction} \textit{ok}\footnote{Old Icelandic \textit{ok} corresponds to modern Icelandic \textit{og}.} (which may also resemble an adverbial, see \cite{nielsen2017remark}) as well as the adverbial \textit{þá}. For the purposes of this paper, I treat V1\is{verb-first} clauses as parallel to those introduced with \textit{ok} and \textit{þá} as far as their relationship to the main narrative line is concerned: they are foregrounded and move the narrative timeline forward temporally \citep{endersverbets, donhauser2006informationsstruktur}.\footnote{For work on discourse adverbs with similar function in the diachrony of \ili{English}, see \citet{vanKem-Los2006}. For Old High German, see \citet{trips2009syntax}.}

Further evidence for the status of Old(er) Icelandic as having been predominantly bounded\is{boundedness} comes from \citet{leiss2000artikel} who discusses V1\is{verb-first} as covertly signaling aspectual\is{aspect} value: perfectivity.\is{perfect} Leiss builds on previous literature (e.g. \cite{rieger1966spitzenstellung, kossuth1980linguistic}) which develops the view that V1\is{verb-first} is used to convey successive events, such as in battle descriptions found so commonly in the saga literature.\is{Icelandic saga} \citet[95]{leiss2000artikel} argues that certain syntactic contexts can coerce a perfective\is{perfect} interpretation where the underlying verb is ambiguous (``aspectually polysemous'' in her terms). These contexts include: V1\is{verb-first}, historical present\is{present tense}\is{tense}, proximity to definite articles\is{definite article} and the presence of perfective\is{perfect} adverbials (cf. \cite{richardson1995tense}).

The following passages are taken from \textit{Finnboga saga ramma} (mid-14th century, present in \isi{IcePaHC}) and serve to illustrate the core characteristics of saga- \isi{style} narrative. The passage below sets the scene for what is to come: a tail-linked topic-time adverbial \textit{um sumarið} `during the summer' provides a local anchor for what follows. The second sentence is V1\is{verb-first}, maintaining that the event directly succeeds the one in the first sentence. Here, a perfective\is{perfect} interpretation is most natural for the verb phrases in both sentences, though one is in historical present\is{present tense}\is{tense} and the other is past-marked V1\is{verb-first}, in accordance with Leiss' claims.

\ea
\gll Um sumarið býr Bergur skip sitt og ætluðu þau að sigla. Lætur Finnbogi flytja varnað þeirra til skips.\\
     during summer prepare Bergur ship his and intended they to sail  lets Finnbogi move goods their to ship \\
\glt `During the summer, Bergur prepared his ship and they intended to sail. Finnbogi had their goods moved to the ship.' \\
\hfill $[$\textbf{1350}, FINNBOGI.NAR-SAG, 1851-4]
\z

The passage which follows is similar in structure; the adverbial phrase \textit{og er Bergur er búinn} `and when Bergur is ready' establishes a local topic-time. This is followed by a V2/V1\is{verb-first}\is{verb-second} alternation in the next two sentences, indicating a shift in subject-topics. The concluding V1\is{verb-first} sentence is most naturally construed perfectively:\is{perfect} their riding has a concrete endpoint.

\ea
\gll Og er Bergur er búinn ríða þeir Finnbogi og Þorkell með honum. Hrafn hinn litli fór og með nogkura klyfjahesta. Ríða þeir þar til er þeir koma mjög svo vestur af hálsinum Hrútafjarðarhálsi.\\
     and when Bergur is \textsc{búinn} ride they Finnbogi and Þorkell with him Hrafn the little went and with some pack:horses Ride they there until when they come very so west off the:ridge Hrútafjarðarháls\\
\glt `And when Bergur is ready, they, Finnbogi and Þorkell, ride with him. Hrafn the little went and [took] with him some pack horses. They ride there until they come very far west off the ridge of Hrútafjarðarháls.' \\
\hfill $[$\textbf{1350}, FINNBOGI.NAR-SAG, 1855-7]
\z

A few passages later, following an exchange in \isi{direct speech}, the storyline continues in a typical manner with the local topic-time anchor \textit{þá} `then' followed by a chain of perfective\is{perfect} events marked in \isi{past tense},\is{tense} expressed paratactically\is{parataxis} with the coordinating \isi{conjunction} \textit{og} `and'.

\ea
\gll Þá hljópu fram tveir menn með vopnum og tóku hestana og leiddu upp undir brekkuna. \\
     then jumped out two men with weapons and took the:horses and led up under the:slope \\
\glt `Then two men jumped out with weapons, took the horses, and led them up under the slope.' \hfill  $[$\textbf{1350}, FINNBOGI.NAR-SAG.1863]
\z

In sum, we have seen that narrative in the saga \isi{style} exhibits the characteristics of a bounded\is{boundedness} system: \isi{Tail-linking}  (use of local topic-time adverbials at each discourse turn, cf. \citealt{los2012}) is prominent; sequences of events are construed perfectively\is{perfect} as though in a continuous chain of self-contained events which move the timeline forward.\footnote{``Prominent" is a relative term; the decrease in \isi{tail-linking}  is discussed in quantitative terms in \ref{sec:5.2.3} below.}

\subsubsection{17th--18th century Icelandic as an intermediate system}\label{sec:Chark4.2.2}

In Section \ref{sec:Chark3.3}, I introduced the linguistic characteristics of 17th and 18th century memoirs and travelogues, which I argue constitute an intermediate stage in Icelandic's shift from a predominantly bounded\is{boundedness} to a predominantly unbounded\is{boundedness} system. To illustrate this, I draw on some passages from the memoir of Jón Ólafsson \textit{Indíafari} (1661) \citep{olafsson1908aefisaga} and the \textit{Nikulás Klím} translation by Jón Ólafsson of Grunnavík (1745) \citep{holberg1948nikulas}.\footnote{A large excerpt of the Klím text is present in the \textit{Textasafn} corpus, including pages 1--251.} We begin with an excerpt from Jón Ólafsson Indíafari: a scene-setting passage which establishes a global topic time (a Saturday in 1617 in Copenhagen). Notable in this passage is the use of extensive \isi{hypotaxis} as well as the use of the \isi{progressive}  in the final sentence \textit{og voru að spássera þar um grundirnar} `and were strolling around the fields...', which does not move the global topic-time forward.

\ea 
\gll Einn tíma bar svo við 1617 um sumarið, á einum torgddegi sem var laugardag, að einn maður, Jens að nafni, sá eð var nýgiftur einnri frómri dándiskvinnu og bjó í Litla Ferjustræti í Kaupinhafn, var út genginn fyrir borgina um Vesturport með einum frómum borgara og voru að spássera þar um grundirnar, sem bændurnir skyldu fram um fara, þeir eð áttu heima uppá landsbyggðinni\\
one time happened so in 1617 during the:summer, on one market:day which was Saturday, that one man, Jens by name, who rel was newly:married a.\textsc{dat} pious lady and lived in Little Ferry:street in Copenhagen, was out walked outside the.city through West:gate with a distinguished citizen and were to stroll there around the:fields, which the:farmers should forward around go, they who lived {} up:in the:countryside\\
\glt `One market day, a Saturday, in the summer of 1617, it happened that a man, by name Jens, who had recently married an excellent lady and lived in Lille Færgestræde in Copenhagen, had left the town by the Western Gate with a worthy citizen. They were taking a walk on the plain crossed twice a week by the farmers who live in the neighbourhood, on their way to the market-place in the town.' \hfill \citep[38]{olafsson1908aefisaga}\footnote{Translation from \citet[95]{phillpotts2017life}}
\z

The above passage continues as shown in (\ref{ex:indiacont}), exhibiting both unbounded and bounded\is{boundedness} characteristics. The passage begins with an inchoative construction \textit{upphóf að} `began to', situating the event of Jens pelting the farmers with horse dung within the global topic-time of their stroll around the fields. A sequence of bounded\is{boundedness} events begins once it is specified that a certain farmer is struck by Jens' insults: \textit{En með því að hann mjög svo kenndi undan hans áköstum} `Now as he was hurt by the pelting...'. This is the moment of inception for the farmer to react, establishing a new topic-time at which the farmer jumps forth to attack Jens.

\ea\label{ex:indiacont} 
\gll Þessi ungi borgari, Jens að nafni, upphóf eftir gamalli venju að grýta bændurna með hestaperum, þar til þessi drukkni bóndi varð fyrir hans kasti, hrópi og spottyrðum. En með því að hann mjög svo kenndi undan hans áköstum og þar með illa sveið hans hæðni, og var í þessu óaflátsamur og með engu móti hirti um hins annars borgarans umtölur né afletjan, hljóp úr vagninum með ryðugan korða, sem bændur plaga þar án bals liggja láta, og strax hann með honum í gegn nísti.\\
\textsc{dem}.\textsc{nom.m} young citizen, Jens by name, began according\_to old custom to mock the:farmers with horse:dung until {} this drunk farmer became for his throws shouts and insults but with \textsc{dem}.\textsc{dat.n} that he very so felt from his attacks and with that badly stung his scorn and was in this unyielding and with no way heeded of \textsc{dem}:\textsc{gen} other:\textsc{gen} citizen:\textsc{gen} words nor distraction jumped out.of the:cart with rusty sword.\textsc{m} which farmers tend there without sheath lie let and immediately he with he.\textsc{dat} {} against struck\\
\glt `The young citizen, Jens, according to his time-honoured custom, set to pelting the farmers with horse-dung, and finally the drunken farmer became the object of his aim and taunts. Now as he was hurt by the pelting, and at the same time smarted sorely under the taunts, with which the other moreover continued to ply him, paying no heed to the advice and dissuasion of his fellow-townsman, the farmer leapt down from his cart with a rusty sword such as farmers use to have lying in their carts without a sheath; ran him through with it...' \hfill \citep[39]{olafsson1908aefisaga}\footnote{Translation due to \citet[95--96]{phillpotts2017life}}
\z

The narrative continues with a series of bounded\is{boundedness} events reminiscent of saga style.\is{style} A series of perfective events\is{perfect} are construed in sequence, moving forward the narrative timeline: The farmer leapt, Jens fell dead, the news is carried into the city, Jens' body is collected and buried at St. Nicholas cemetery. The alternation between V1\is{verb-first} and V2\is{verb-second} in the passage in (\ref{ex:indiacont2}) resembles saga texts:\is{Icelandic saga} An anaphoric subject topic is maintained across the V1\is{verb-first} clauses and V2\is{verb-second} is used to signal a shift in this topic (\textit{Hljóp svo \dots {} Hinn féll \dots}). These events take place in a bounded\is{boundedness} sequence each moving forward the local topic time.

\ea\label{ex:indiacont2}
\gll Hljóp svo á sinn vagn og burt keyrði.  Hinn féll þar strax dauður niður til jarðar.  Borgarinn hinn annar bar þessi tíðindi í borgina. Var hann svo sóttur og inn fluttur og að morgni greftraður uppá Sankti Nikulás kirkjugarði.\\
jumped then on his cart and away drove \textsc{dem.m} fell there immediately dead down to ground the:citizen \textsc{dem.m} other brought these news into the:city was he then fetched and in brought and at morning buried up:on Saint Nicholas cemetery\\
\glt `He leapt on to his cart again and drove away. Jens at once fell dead. The other citizen carried this news into the city, and the body was fetched and conveyed within, and buried the next morning in St Nicholas' churchyard.' \hfill \citep[39]{olafsson1908aefisaga}\footnote{Translation from \citet[96]{phillpotts2017life}}
\z

Next, in (\ref{ex:klim1}) below, we turn to Nikulás Klím and examine a passage which describes the protagonist being medically investigated by a tree\ \textendash{}\  the story is science-fiction and takes place in an underworld of sorts. 

\ea\label{ex:klim1}
\gll Meðan eg á þessa lund talaðe við sjálfann mig í einrúmi, geck ein eik inn til mín, sem hafðe einn triangel (þríhyrníng) í hendinne. Eik þesse beraðe minn handlegg og brjóst; sló hún rjett lagliga miðæðina með verkfære þessu, og þegar hún hafðe látið renna nockut blóð, so mikið sem henni sýndest þurfa, þá batt hún til arminn aptur með eigi minni lagvirkne. Þá hún var þannin búin að útrjetta sitt erinde, og hafðe skoðat blóðit þegjande og forundrande, geck hún aptr sinn veg í burt.\\
while I in this manner spoke to self me in alone, walked \textsc{art}.\textsc{indf} oak.\textsc{f} in to me, who had a triangle (tri:corner) in the.hand oak.\textsc{f} this touched my forearm and chest; struck she precisely properly the:middle:vein with tool this, and when she had let flow some blood as much as her seemed to.need then tied she up the:arm again with not less skill when she was thus \textsc{búinn} to accomplish her task and had examined the:blood silently and wondering went she back her way out {}\\
\trans `While I mused on the strange things I had witnessed, a tree came into my cell, with an instrument resembling a lancet in his hand. It stripped one of my arms, and made a puncture in the median vein. When he had taken from me as much blood as he deemed sufficient, he bound up the wound with great dexterity. He then examined my blood with much attention, and departed silently, with an expression of wonder.'\\
\hfill \citep[29--30]{holberg1948nikulas}\footnote{Translation by John Gierlow sourced at Project Gutenberg \citep{holberg1845niels}. Page numbers are absent since reference is made to the digital edition.}\footnote{It should be mentioned that the noun \textit{eik} `oak' is feminine and thus glossed as such. In the subsequent discussion, I refer to the tree as  ``it'', following \ili{English} convention.}
\z

\largerpage
From the perspective of \isi{boundedness}, this passage begins with the establishment of a global topic-time, namely that which occurs while the \isi{narrator} is busy musing on what they had witnessed thus far. The entrance of the tree and its subsequent actions constitute events on the main timeline, situated within this global topic-time. Temporal subordinate clauses beginning with \textit{þegar} and \textit{þá} are used to situate events within the topic-time, without moving forward the global narrative timeline. This \isi{sequence of events} is then summarised with a \textit{búinn} construction, the material embedded under which is backgrounded in relation to the main timeline. Its function is to recapitulate the relation between this \isi{sequence of events} and what comes after: having done his job, the tree is now in a position to leave. The passage above in (\ref{ex:klim1}) is followed by a sequence of expository passages detailing the circumstances which led to him being investigated in this strange new world, reflecting on his understanding of the workings of this society. In the passage below, the \textit{búinn} construction is used twice, again in order to summarise the \isi{sequence of events} up until this point: the protagonist emphasises that had he not learned the subterranean language, his opinion would be quite different.

\ea
\gll Þetta styrkte enn fremur meiníngu mína, sem eg hafðe feingið um
fávitsku fólks þessa. Enn epter þat eg var búinn at læra
underjarðarmálit, og allt var búit at segja mier, þa snerest
allt mitt forackt í forundran.\\
\textsc{dem}.\textsc{nom.n} strengthened even further belief my which I had received about foolishness people.\textsc{gen} \textsc{dem}.\textsc{gen} but after that I was \textsc{búinn} to learn the:underground:language and everything was \textsc{búinn} to tell me, then turned all my contempt into wonder\\
\glt  `This circumstance by no means weakened the opinion which I had for some time entertained, that these people were shallow and foolish. But my judgment proved to be too hasty. When I was better enabled to judge of what passed about me, by acquaintance with the subterranean languages, my contempt was changed to admiration.' \\
\hfill \citep[30]{holberg1948nikulas}\footnote{Translation from \citet{holberg1845niels}}
\z

In summary, travelogues and memoirs such as \textit{Nikulás Klím} and \textit{Æfisaga Jóns Indíafara}\is{Icelandic saga} clearly exhibit both bounded\is{boundedness} and unbounded characteristics. The authors make use of unbounded\is{boundedness} strategies such as \textit{búinn} and \textit{PROG} alongside adopting stylistic characteristics favouring bounded\is{boundedness} construals from the earlier saga \isi{style}.


\subsubsection{Modern Icelandic as unbounded}\label{sec:Chark4.2.3}

I this subsection, I examine an excerpt from the novel \textit{Piltur og stúlka}, written by Jón Thoroddsen and published in 1850 \citep{Thoroddsen_PilturOgStulka}. The novel was highly influential at its time of publication; \citet{bernhardhsson2018spreading} argues that it was an important ingredient in the establishment of a new linguistic standard in the 19th century. I therefore consider it to be at the dividing line between early modern and modern Icelandic.\footnote{The novel exhibits some archaicisms which go back to older narrative styles,\is{narrative style}\is{style} e.g. OV word order \citep{bernhardhsson2018spreading}. Nonetheless, it is in my view a representative example of narrative at this stage in the language.}


\ea\label{ex:ps}
\gll Það var eitt kvöld nokkru eftir miðjan vetur, að þær Sigríður og Guðrún sátu tvær einar í stofu. Veður var fagurt, sólin var að setjast, og kvöldroðanum kastaði á gluggana og inn um stofuna.\\ it was one evening sometime after mid winter that they Sigríður and Guðrún sat two alone in living\_room. Weather was beautiful the:sun was to set and the:evening:glow cast onto windows and in through the:living\_room\\
\glt `One evening, not long after mid-winter, Sigrid and Gudrun were sitting alone together in the living-room. The weather was fine, the sun was just setting, and the evening-red fell upon the windows and the floor of the room.' \hfill \citep{Thoroddsen_PilturOgStulka}\footnote{Translation due to Arthur Reeves \citep[164]{thoroddsen1890lad}}
\z

In passage (\ref{ex:ps}), we observe little in the way of \isi{tail-linking}  adverbials. Instead, the hallmarks of an unbounded\is{boundedness} narrative are present. A topic-time is first established via the scene-setting construction at the inception of the paragraph, during which other events are ongoing. The narrative continues as follows:


\ea\label{ex:ps1}
\gll Guðrún sat á stóli út við gluggann og var að sauma og þagði; sá, sem þá hefði séð hana og tekið eftir brosunum, sem voru að smáflögra um munnvikin á henni, og séð hvernig spékopparnir á kinnunum á henni ýmist voru að myndast eða hverfa.\\ Guðrún sat on chair out by the:window and was to sew and was\_quiet; \textsc{dem}, which then have.\textsc{sbj} seen her and notice {} the:smiles which were to flutter around mouth:corners on her and seen how the:dimples on the:cheeks on her either were to form or disappear\\
\glt `Gudrun sat silently sewing in a chair by the window. Whoever could have seen her then and observed the smile which played (\textit{was fluttering)} upon her lips, and the dimples in her cheeks, as they came and vanished (\textit{were coming and vanishing}).' \hfill \citep{Thoroddsen_PilturOgStulka}\footnote{Translation from \citep[164]{thoroddsen1890lad}}\footnote{My additions to highlight where the \isi{progressive} is used; the translator did not always reflect it in the \ili{English}.}
\z

In example (\ref{ex:ps1}) above, \isi{progressive}  constructions are prominent \textit{voru að smáflögra} `were fluttering' and \textit{voru að myndast eða hverfa} `were forming or disappearing'. These describe unbounded\is{boundedness} events within the global topic-time, as established by the \isi{expletive} construction at the onset of the narrative episode (this one evening after mid-winter). The narrative continues with Sigríður singing lullabies to baby Sigrún, giving way to the following passages:

\ea\label{ex:ps2}
\gll En er Sigrún litla var sofnuð, lagði hún hana hægt í legubekkinn og lítinn kodda undir höfuðið og breiddi svuntuna sína ofan á hana, en settist sjálf út við gluggann allskammt frá Guðrúnu og horfði um hríð út. \\
but when Sigrún little was asleep laid she her slowly into the:crib and little pillow under the:head and spread the:bib hers on onto her, and sit\_down herself out by the:window all:close from Guðrún and watched for.a while out\\
\glt `And when little Sigrún was fast asleep, Sigríður laid her gently on the couch with a small pillow beneath her head, spread her apron over her, and then seating herself a short distance from Guðrún, looked out the window for a time.' \hfill \citet{Thoroddsen_PilturOgStulka}\footnote{Translation due to \citet[166]{thoroddsen1890lad}}
\z

The passage above begins with a backgrounded \isi{adverbial clause} resituating the following events on the main narrative timeline. The text continues:

\ea\label{ex:ps3}
\gll Allt var kyrrt á strætum bæjarins; dálítill snjófölvi var yfir jörðunni; veðrið var hreint og heiðríkt, og sólin var þegar sigin, og sló blóðrauðum geislum um allan vestursjóinn; jökulinn {hillti upp;} fiskibátarnir voru að koma að, sumir að lenda, en sumir voru komnir inn fyrir eyjarnar og skriðu fagurlega í logninu. \\
all was quiet on streets the.\textsc{gen}:town a\_little snow:paleness was over the:ground; the.weather was clear and fair and the:sun was already set and cast blood.red rays about all the:west:sea; the:glacier visible; the:fishing:boats were to arrive {} some to land but some were arrived further in.front.of the:islands and glided beautifully in the:calm\\
\glt `Everything was quiet on the town's streets; a little snow covered the earth; the weather was clear and fair, the sun had already set, casting blood-red rays across the entire western sea; the glacier stood visibly tall; the fishing boats were coming in, some approaching to anchor, and others navigated the calm waters near the islands gliding beautifully in the calm winds.' \hfill \citep{Thoroddsen_PilturOgStulka}\footnote{Translation due to \citet[166]{thoroddsen1890lad}}
\z

The above excerpt situates a number of atelically construed events within a global topic-time: the sun's rays, the sight of the glacier, the arrival and anchoring of fishing boats. We again see the \isi{progressive}  used to describe such events: (the fishing boats) \textit{voru að koma að, sumir voru að lenda, sumir voru komnir inn fyrir eyjarnar} `were arriving, some anchoring, others had made it past the islands'.\footnote{An anonymous reviewer points out that this sort of descriptive passage, outside of the main narrative, may represent one of the innovations of the emerging \isi{genre} of \textit{novel} \citep{Fludernik2003}. The reviewer points out that Dutch, a language with an emergent \isi{progressive}  construction, does not use it in passages such as this one.}

This section has served to illustrate the predominance of unbounded\is{boundedness} strategies found in the novel \textit{Piltur og stúlka}, which I take to be representative of narrative structure for Icelandic of this period. In the following section, I demonstrate that a decrease in bounded\is{boundedness} strategies can be observed in corpora for the time period in question.

\section{Decrease in bounded strategies}\label{sec:Chark5}

This section is structured as follows: in Section \ref{sec:Chark5.1}, I discuss the clear parallels between the shift in \isi{boundedness} in the diachrony of \ili{English} to the Icelandic case. Section \ref{sec:Chark5.2} then draws on corpus evidence to substantiate similar claims for Icelandic.

\subsection{Comparison to English}\label{sec:Chark5.1}

In this section, I draw on evidence regarding a shift in the frequency of temporal adverbials as a proxy for the shift in \isi{boundedness}. This analytical move has precedent in the literature on \ili{Old English}, a language which is not only closely related to Old Icelandic but also parallels it closely in this regard. In the following two subsections, I discuss the parallels with \ili{Old English} in more detail, focusing first on the topic-time adverbials \textit{þa} (OE\il{Old English}) and \textit{þá} (OI) in \ref{sec:Chark5.1.1} and on the parallels between the rise of a dedicated \isi{progressive} marker in both languages in \ref{sec:Chark5.1.2}. 

\subsubsection{Contrasting OE \textit{þa} and OI \textit{þá}}\label{sec:Chark5.1.1}

Just as Old Icelandic has the multi-functional adverbial item\textit{ þá}, so too did \ili{Old English} have the item \textit{þa}: both could be used as local discourse anchors, as subordinating conjunctions\is{conjunction} or as \isi{resumptive} items \citep{Brinton1996, enkvist1987old}.

In Old\il{Old English} and \ili{Middle English}, \textit{þa} had a specific role in narrative episodes: the demarcation of incipient narrative episodes \citep{enkvist1987old, warvik1995a}. Interestingly for the present purposes, this functional domain was later overtaken by other expressions in the \ili{Middle English} period\ \textendash{}\  I contend that analyses of this development are highly relevant for understanding what happened in the transitional period that Early Modern Icelandic finds itself in.

As mentioned above in Section \ref{sec:Chark4.2}, a number of changes pertaining to the mapping between clause and information structure in historical Icelandic have been tied to the emergence of a dedicated subject position \citep{booth-beck20200jhs}. This is the case for historical \ili{English} as well, as discussed by \citet{fludernik1995middle} who shows that the functional domain previously occupied by \textit{þa} gives way to a number of competing expressions, among them \textit{And whan NP+VP} constructions.

\citet{warvik1995a}, too, investigates the role of \textit{þa} and \textit{when} in \ili{Middle English}. As is the case in Old Icelandic, which has both \textit{þá} and \textit{þegar}, the two may co-occur. Wårvik shows that there is individual variation among others with regard to their preference for \isi{parataxis} as opposed to temporal subordinate clauses: A decrease in the former corresponds to an increase in the latter.

\citet{Brinton1996} looks specifically at the development of the \isi{discourse function} of the inception of narrative episodes in \ili{Middle English} in texts by Malory and Chaucer. She shows that, in the \ili{Middle English} period, preposed \textit{when}-clauses become the predominant means of doing so. \citet{Brinton1996} also discusses the status of these clauses in terms of grounding: She shows that in earlier Chaucer as well as later Malory, events described in these clauses are more backgrounded. They express given information, which is off the main narrative timeline.

Finally, \citet{enkvist1987old} discusses the status of \textit{when}-clauses in \ili{Middle English}. The authors conclude that their status as foregrounded or backgrounded depends on whether or not they can be interpreted as advancing the narrative timeline. This is the case when \textit{when} is to be interpreted as `after' rather than `while' and the event it embeds is new,\is{discourse-new} focused\is{focus} information that receives a perfective\is{perfect} construal \citep[876]{Fleischman1985}. In contrast, the embedded event can receive an imperfective construal which is backgrounded on account of containing given information, conveying temporal simultaneity. 

In sum, \citet{Brinton1996}, \citet{warvik1995a} and \citet{enkvist1987old} describe the linguistic system of \ili{Middle English} as one with a tendency for increased backgrounding. Concretely, this is reflected in the gradual demise of \textit{þa} as a foregrounding topic-time adverbial. The functional domain occupied by \textit{þa} in earlier stages of the language\ \textendash{}\  the marking of the inception of narrative episodes\ \textendash{}\  becomes overtaken by backgrounded adverbial clauses\is{adverbial clause} with the subordinator \textit{when} by late \ili{Middle English}. We will see later on, in Section \ref{sec:Chark5.2}, that Icelandic \textit{þá} underwent a similar fate, and like shown in \citet{warvik1995a}, its demise correlates with the rise of preposed clauses containing \textit{þegar} `when'.

\subsubsection{The rise of the progressive}\label{sec:Chark5.1.2}

\citet{petre2016grammaticalization} discusses the rise of the dedicated \isi{progressive}  construction \textit{be V-ing} in the history of \ili{English}. His line of argumentation will broadly be adopted here and it can be summarised as follows: A change in frequency of associated environments (``co-texts") can be critical in the conventionalisation (semanticisation) of an emerging meaning. \citeauthor{petre2016grammaticalization} argues that the \isi{progressive}  construction became semanticized in its focalised usage (where it is largely obligatory in PDE) as a result of a more general increase of backgrounded adverbial clauses.\is{adverbial clause} This increased frequency is, in turn, a consequence of a broader shift in grounding strategies. Past-marked, temporal subordinate contexts embedded under the subordinating \isi{conjunction} \textit{when} are of special importance to the conventionalisation of this focalised usage. These contexts allow a shift in meaning from the present participle form merely expressing an ongoing state to a state which is ongoing at a particular focal point. Subordinate clauses are crucial here as they provide the time frame at which this focalisation can occur\ \textendash{}\  this then becomes part of the expression's conventional meaning that spreads to main clause contexts.

\subsubsection{Summary: Decline of bounded strategies in the history of English}\label{sec:Chark5.1.3}

\citet{petre2010functions} and \citet{los2012} discuss the decline of bounded\is{boundedness} construals in the history of \ili{English}. Both authors argue that \ili{Old English} had a more transparent correspondence between syntactic transformations and information-structural functions. This argumentation is based upon a number of empirical observations:

\begin{enumerate}
\item Topic-time adverbials, used to re-establish the topic time with each discourse transition, lose prominence (see also \cite{los2023decline}). Clause-initial \textit{þa} as a topic-time anchor decreases from 36.1\% to 15.4\% in early ME and 11.3\% in late ME \citep[93]{westergaard2009word}.
\item Environments which favour unbounded construals show a concomitant rise in frequency: Dedicated \isi{progressive}  marking \citep{Killie2008, killie2014development},  \textit{ginnan}-class (\textit{onginnan}, \textit{beginnan}) with infinitival complements \citep[468]{petre2010functions} begin as perfectivisers but acquire an inchoative usage late in the  OE\il{Old English} period \citep[269--271]{Los2000}, are subsequently subject to semantic bleaching and further conventionalisation into the \ili{Middle English} period \citep{brinton1988, funke1922, mosse1938}.
\end{enumerate}

The relation of the latter two environments to unboundedness\is{boundedness} has to do with their meaning: Inchoative constructions focus\is{focus} on the onset of a new event,\is{discourse-new} which has close pragmatic ties to the ongoingness of said event. \citet{petre2010functions} argues that \textit{ginnan} expressions replaced the functional domain of \textit{weorðan} `become' in ME\ \textendash{}\  the latter focusing instead on the state transition. This, Petre argues, reflects a greater flexibility in ME and afterwards with regard to framing events in an unbounded\is{boundedness} manner. Similarly, progressives\is{progressive} directly \isi{focus} event ongoingness and are thus fundamentally imperfective, they can be considered the “present counterpart” of inchoatives \citep{carroll2004language}. 

\subsection{Statistical corpus evidence}\label{sec:Chark5.2}

As outlined in the previous section, the trajectory for the diachrony of \ili{English}, for which a shift from a bounded\is{boundedness} to unbounded\is{boundedness} system has been proposed, resembles Icelandic very strongly. Here, I propose that an independent case can be made for Icelandic on the basis of similar argumentation and empirical evidence. Concretely, I argue for the following on the basis of corpus evidence:
\begin{enumerate}
    \item There is an overall trend towards a higher frequency of backgrounded temporal subordinate clauses, as discourse traditions\is{discourse tradition} shift.
    \item There is a strong correlation between a decrease in foregrounded, iconic narration strategies (including verb-initial and \textit{þá}-verb-subject structures) and those which rely more heavily on temporal backshifting and backgrounding.
    \item The decrease in bounded\is{boundedness} strategies is further reflected in the \isi{tail-linking}  requirement: global establishment of topic-time no longer necessitates strict \isi{tail-linking}. This can be seen in the decrease of adjunct-initial main clauses relative to other possible configurations.
    \item Both \textit{búinn} and \textit{PROG} rise rapidly in frequency during the early modern period, going from marginal constructions to integral parts of the grammar. I argue that this rise in frequency should be tied to the increased preference for unbounded\is{boundedness} construals in the language. For both constructions, temporal subordinate contexts are of special importance, as argued in Section \ref{sec:Chark1}. This gives rise to two main predictions concerning both constructions: (a) they should show an association with unbounded adverbials and (b) their increased frequency should be correlated by-text.
\end{enumerate}

In the following subsections, I show the empirical evidence that backs up the above claims. (1) is substantiated in \ref{sec:Chark5.2.1}, (2) is shown in \ref{sec:Chark5.2.2}, (3) is shown in \ref{sec:5.2.3} and the two claims in (4) are argued for with data in \ref{sec:Chark5.2.4} and \ref{sec:Chark5.2.5} respectively. Finally, \ref{sec:Chark5.2.6} discusses an additional construction associated with the loss of \isi{boundedness}, the decline of the present participle.

\subsubsection{Increase in temporal subordinate clauses}\label{sec:Chark5.2.1}

Temporal subordinate clauses in \isi{IcePaHC} were extracted via two separate queries. The first query targets prepositional phrases headed by \textit{þegar} `when' or \textit{nær} `when' which in turn immediately dominate an \isi{adverbial clause}, itself immediately dominating a subordinate clause. The second query targeted temporal subordinate clauses introduced by \textit{er} (an all-purpose relativiser), \textit{þá er} (lit. `at the time that'; typical way of forming a temporal subordinate clause in Old Icelandic), as well as \textit{þá} or \textit{nær} (both meaning `when'). This is done by targeting what are coded as relative clauses\is{relative clause} in the corpus, which immediately dominate a subordinate clause, then embedding an adverbial phrase. The results were collated in order to yield a total relative frequency per 1000 clauses. The results were collated in order to yield a total relative frequency per 1000 clauses. These results are plotted in Figure \ref{ex:tempsub}. I looked at narrative texts specifically for the time period from 1540 to 1900 and compared a statistical model including \textsc{Year} as a predictor to one without (an F-test using the \texttt{anova} function in \texttt{R}) which revealed that this increase is statistically significant (\textit{p} = 0.017).

 \begin{figure}
    \centering
    \includegraphics[width=\textwidth,
      keepaspectratio]{figures/fig9tempsub.png}
    \caption{Relative frequency of temporal subordinate clauses in IcePaHC
}\label{ex:tempsub}
\end{figure}


\subsubsection{Tracing the availability of unbounded strategies: the temporal subordinator \textit{þegar}}\label{sec:Chark5.2.2}

The temporal subordinator \textit{þegar} becomes the predominant one in the language, taking over from \textit{þá (er)} sometime between 1500--1700. In \figref{ex:thaandthegar}, I show that the rise of the temporal subordinator \textit{þegar} is negatively correlated with the decline of \textit{þá} as a foregrounding topic-time adverbial, where it appears pre-verbally (cf. \citet{warvik1995a}'s finding for \ili{Middle English}).

 \begin{figure}
    \centering
    \includegraphics[width=\textwidth,
      keepaspectratio]{figures/adjusted_corr_thegar_tha.png}
    \caption{Correlation of foregrounding \textit{þá} and subordinating \textit{þegar} in IcePaHC. Pearson's coefficient: -0.4, $p < 0.002$
}\label{ex:thaandthegar}
\end{figure}

In the 16th to 19th centuries, written Icelandic prose \isi{style} is under the influence of language contact via translation, for example due to both Latin\is{transfer from Latin} and Low German\is{transfer from Low German} in Oddur Gottskálksson's New Testament translation from 1540; this influence extends from the lexical to the syntactic \citep{helgason1929malid}. There are a couple areas where the language system is in flux during this period. Especially relevant for this paper is a shift in relativisation strategies (from \textit{er} to \textit{sem} with additional variation due to the interrogative strategy \textit{hverr}) \citep{brendel2023translation, sapp2019arrested}, which is intertwined with a change in subordinating strategies more generally. 

Temporal subordinate clauses, which prior to the 16th century involved complementiser doubling, e.g. \textit{þá er} `then \textsc{rel}' and \textit{þegar er} `when \textsc{rel}', emerge both in new variants, e.g. \textit{þegar að} `when \textsc{compl}' as well as without a complementiser or relativiser \citep{helgason1929malid}. In modern Icelandic written language, temporal subordinate clauses involve bare \textit{þegar}. I take it that this change involves a tighter integration of the subordinate clause, as these changes are parallel to a decline in the sort of \isi{correlative} constructions one finds in older texts to express the same sorts of temporal relations.

\citet{kiparsky1995indo} argues for the view that in Old Indo-European languages, what appear to be subordinate clauses should be analysed as instances of adjunction rather than true \isi{subordination}. This explains, among other facts, why \isi{resumptive} adverbials seem to show up in environments without there being, strictly speaking, an antecedent clause, such as clause-initially. Such an environment is illustrated below in (\ref{ex:atlaga}). On \citeauthor{kiparsky1995indo}'s analysis, this development is then effectively a transition from broadly paratactic\is{parataxis} to hypotactic\is{hypotaxis} systems.

\ea\label{ex:atlaga}
\gll Síðan varð hörð atlaga með þeim. Þa mællte Ólafr at hans menn skyldu hlífa sér undir skjöldum, meðan þeir skyti spjótum ok örum... \\
then became hard attack with them. Then said Ólafur that his men should protect themselves under shields, while they shot spears and arrows\\
\glt `Later they were attacked hard. Olaf said that his men should seek cover under the shields while the enemy shot spears and arrows at them...' \hfill $[$\textit{Ólafs saga helga} 24:44 cited in \citeauthor{nilsen2020adverbial} \citeyear[28]{nilsen2020adverbial}]
\z

It seems reasonable to me to tie these temporal \isi{resumptive} uses to \isi{correlative} relative clauses,\is{relative clause} such as the following cited in \citet[229]{Wagener2017}. What they both have in common is their exploitation of the left periphery as an anaphoric position.

\ea\label{ex:raedu}
    \gll Þa ræðu er næst heyrða ec yður sægia um kaupmanna iðrott þa var hon með glæggara froðleic fram flutt isvorum en í spurning \\
    that speech \textsc{rel} next heard I you say about merchant:\textsc{gen.pl} profession then was she with more:intelligent wisdom performed {} in\_answers than in question\\
\glt `The statement that I heard you make about the mercantile profession was uttered with more wisdom in the answer than in the question.' \hfill $[$\textit{Konungs skuggsjá} 38.18, cited in/translation due to \citeauthor{Wagener2017}]
\z

\citet[29]{nilsen2020adverbial} discusses the difference between Older Indo-European languages and their modern counterparts in this respect: in the modern languages, only resumptives\is{resumptive} can occur between fronted adverbials and the finite verb, whereas in the older languages a number of other elements could appear in place of the \isi{resumptive}. Clause-external adverbials in the older languages required that the \isi{resumptive} be a full phrase in order to satisfy V2,\is{verb-second} but this is no longer the case in the modern languages, where \isi{resumptive}s are integrated\is{integration} and can fulfill V2\is{verb-second} as they originate in the left periphery. \citet{nilsen2020adverbial} takes this as evidence for the increased integration of the \isi{adverbial clause}, which is accompanied by syntactic reanalysis of the \isi{resumptive}. 

Early Modern Icelandic thus represents an intermediate period where there is potential for ambiguity between \isi{hypotaxis} and \isi{parataxis}, embedding and adjunction. More conclusive evidence of this increased integration having played a role in Icelandic of the 16th to 19th centuries comes from a shift from double complementation strategies for temporal \isi{subordination}, often involving extraposition \citep{sapp2019arrested, wallenberg2016extraposition}, to the predominance of bare \textit{þegar} for this purpose in the modern language. 

My corpus investigation in \isi{IcePaHC} shows that \textit{þegar} wins out over \textit{þá er/þá} by 1900, with the preceding period reflecting considerable \isi{genre} variation. This is shown in \figref{ex:thaandthegarsubord} below: The overall proportion of \textit{þegar} rarely exceeds 50\% prior to 1600, while by 1850 alternative temporal \isi{subordination} strategies have all but disappeared.


\begin{figure}
  \centering
 \includegraphics[
  width=\textwidth,
  keepaspectratio
]{figures/fig10temporalconj.png}
    \caption{Subordinating \textit{þegar} as opposed to subordinating \textit{þá er/þá} in IcePaHC}\label{ex:thaandthegarsubord}
\end{figure}

The frequencies of the expressions relative to one another show that the conjunctionalisation\is{conjunction} is less advanced with \textit{þá} `then' than \textit{síðan} `since' and \textit{þegar} `when' \citep{Bjerre1935}. This is tied to the fact that \textit{þá} is a determinative \isi{temporal adverb} in contrast with the others: “The \isi{demonstrative} force in \textit{þá} is then a hindrance for its development into a \isi{conjunction}” \citep[156]{Bjerre1935}. According to Bjerre, \textit{er} has a tendency to disappear after all three, but this happens most clearly with \textit{þegar} and least with \textit{þá er}. The \textit{er} in \textit{þegar er} is perceived as pleonastic and is thus ellided easily, whereas the relativiser function of \textit{er} in \textit{þá er} is supported by the determinative component of \textit{þá} \citep{Bjerre1935}. This shows that the conjunctionalisation\is{conjunction} there is less developed.

Why is \textit{þá} poorly suited for unbounded\is{boundedness} construals then? \textit{Þá} is rare in temporal clauses following a determinative correlate. With few exceptions, \textit{þá er} does not occur in temporal clauses after definite temporal expressions \citep{Bjerre1935}. This is not solely due to the fact that \textit{þá} in combination with a previous temporal expression would be pleonastic, since \textit{þá er} does indeed occur in connection with temporal expressions in the main clause. However, this happens nearly exclusively in those cases where the connection between the temporal expression and the \isi{adverbial clause} is relatively loose or the latter can be considered as somewhat freestanding (ibid.). On the whole, it appears that \textit{þegar} emerged as the best available candidate for an all-purpose subordinating \isi{conjunction}.


\subsubsection{Decrease in tail-linking}\label{sec:5.2.3}

To investigate the role of discourse anchoring, I use adjunct-initial V2\is{verb-second} clauses as a proxy\ \textendash{}\ the vast majority of these involve temporal anchors \citep{booth-beck20200jhs}. Here I report on a regression analysis carried out to investigate the proportion of adjunct-initial clauses over time; I build on the discussion in \citet[35]{booth-beck20200jhs}, but I make use of a different periodisation than discussed there. A binomial mixed-effects model was fitted in \texttt{R} \citep{rcoreteam} using the \texttt{lme4} package \citep{lme4} with a term for \textsc{Year} (centred and scaled) and \textsc{Genre},\is{genre} as well as a random effect for \textsc{Text}. The model is specified as follows, including data from all genres\is{genre} for the time period 1540--1900:\footnote{This is a bit broader than the 1850 mark so as to not exclude otherwise valuable data.} 

\ea
\begin{verbatim}
    cbind(successes, failures) ~ scaled_Year + Genre + (1 | Text) \end{verbatim} 
    \z

The model specified above reveals that the overall decrease of non-subject-initial V2\is{verb-second} clauses (as a proportion of possible matrix clause configurations), while descriptively present, is not statistically significant (\(\beta = -0.3640\), p = 0.1192). As observed by \citet{booth-beck20200jhs}, biographies, religious and \isi{scientific texts} pattern differently from narrative ones. However, for the given periodisation there is a decrease in all genres\is{genre} except for \isi{scientific texts}. Of all genres,\is{genre} only narrative texts exhibit a statistically significant decrease for this period (\(\beta = -0.4921\), \textit{p} = 0.00448).

\subsubsection{Frequency of associated adverbials}\label{sec:Chark5.2.4}

The patterning of the phenomena in question with unbounded\is{boundedness} construals can be shown by examining the relative frequency of adverbials which pattern alongside them (cf. \citeauthor{petre2010functions} \citeyear{petre2010functions}). Both phenomena collocate most commonly with no adverbial, which suggests that they pattern with unbounded\is{boundedness} construals. As far as overt adverbials go, \textit{búinn} is predominantly associated with temporal subordinate clauses introduced either by \textit{þegar} or \textit{þá (er)}, both compatible with imperfective construals. Perhaps better evidence for the lack of \isi{boundedness} is that the two constructions do not collocate with bounding adverbials (e.g. \textit{after}, \textit{at X time}, cf. \cite[474]{petre2010functions}). \textit{PROG} also collocates with  \textit{þegar} and \textit{þá (er)}, however, the most dominant adverbial is \textit{meðan} `while', clearly associated with ongoingness. In Tables \ref{tab:collocating1}, \ref{tab:collocating2} and \ref{tab:collocating3} below, \textit{þá-s} designates subordinating \textit{þá} (often, but not always accompanied by the relativiser \textit{er}), while \textit{þá-t} represents \textit{þá} in its use as a topic-time anchor (clause-initial or clause-medial).

\begin{table}
    \centering
    \caption{Collocating adverbials for \textit{búinn} IcePAHC until 1900 (n=90)}
    \label{tab:collocating1}
    \begin{tabularx}{0.48\textwidth}{>{\raggedright\arraybackslash}X r}
            \lsptoprule
            Adv & Percentage \\ 
            \midrule
            none & 69.10 \\ 
           \textit{þegar} `when' & 11.16 \\ 
            \textit{þá-s} `when' & 10.73 \\ 
            \textit{eftir} `after' & 3.43 \\ 
            \textit{þá-t} `then' & 2.15 \\ 
            \textit{fyrr} `before' & 1.29 \\ 
            \textit{nær} `when' & 0.86 \\ 
            \textit{er} `when' & 0.43 \\ 
            \textit{sem} `as/when' & 0.43 \\ 
            \textit{þangað til} `until' & 0.43 \\ 
            \lspbottomrule
    \end{tabularx}
\end{table}

\begin{table}
    \centering
    \caption{Collocating adverbials for \textit{búinn} in \textit{Textasafn} 16th-18th centuries (n=215)}
    \label{tab:collocating2}
    \begin{tabularx}{0.48\textwidth}{>{\raggedright\arraybackslash}X r}
            \lsptoprule
            Adv & Percentage \\ 
            \midrule
            none & 36.67 \\ 
            \textit{þá-s} `when' & 12.22 \\ 
            \textit{nú} `now' & 11.11 \\ 
            \textit{þegar} `when' & 11.11 \\ 
            \textit{eftir} `after' & 7.78 \\ 
            \textit{er} `when' & 3.33 \\ 
            \textit{nær} `when' & 3.33 \\ 
            \textit{fyrr} `before' & 2.22 \\ 
            \textit{svo} `then' & 2.22 \\ 
            \textit{þá-t} `then' & 2.22 \\ 
            \textit{allareiðu} `already' & 1.11 \\ 
            \textit{eftir} `after' & 1.11 \\ 
            \textit{fyrst} `first' & 1.11 \\ 
            \textit{inn til þess} `until' & 1.11 \\ 
            \textit{jafnskjótt og} `as soon as' & 1.11 \\ 
            \textit{nú} `now' & 1.11 \\ 
            \textit{oft} `often' & 1.11 \\ 
            \lspbottomrule
       \end{tabularx}
\end{table}

\begin{table}
    \centering
    \begin{tabularx}{0.5\textwidth}{>{\raggedright\arraybackslash}p{3cm} r}
        \lsptoprule
        Adv & Percentage \\ 
        \midrule
        none & 74.56 \\ 
        \textit{meðan} `while' & 7.89 \\ 
        \textit{þegar} `when' & 6.14 \\ 
        \textit{þá-s} `when' & 3.51 \\ 
        \textit{þá-t} `then' & 3.51 \\ 
        \textit{er} `when' & 2.63 \\ 
        \textit{í þessu} `at this/that moment' & 0.88 \\ 
        \textit{nú} `now' & 0.88 \\ 
        \lspbottomrule
    \end{tabularx}
    \caption{Collocating adverbials for \textit{PROG} in IcePAHC until 1920 (n=114)}
      \label{tab:collocating3}
\end{table}


\subsubsection{Timing of changes}\label{sec:Chark5.2.5}

As displayed in \figref{fig:corrprogb}, the correlation between the relative frequency of \textit{búinn} and \textit{PROG} over time is moderate (0.61) and statistically significant. This supports the argument that the two changes were bolstered by the same underlying factor, the shift in \isi{boundedness}. On the whole, this perspective provides insight into the time course of propagation for both innovations, which is otherwise puzzling: Why did the constructions stay relatively marginal before seeing a jump in frequency from the early 19th century onwards? My proposal is that the establishment of a predominantly unbounded\is{boundedness} system had to occur first. The chain of causality I would like to propose is as follows. The expressions in question could become available via chance reanalysis.\footnote{I thank an anonymous reviewer for requesting I expand on this point.} Their subsequent conventionalisation, leading to the well-developed aspectual\is{aspect} system exhibited by modern Icelandic, was favoured by other ongoing structural developments in the language, i.e. \isi{tense} indeterminacy, inter-clausal relations in flux. Regarding the latter, alongside the increased integration of adverbial subclauses, speakers had expressions available for explicitly marking event phase transitions.


 \begin{figure}
    \centering
    \includegraphics[
width = \textwidth,
  keepaspectratio
]{figures/fig11corrbuinnprog.png}
    \caption{Correlation of relative frequency of \textit{búinn} and \textit{PROG} by text/year. Pearson's coefficient: 0.586, $p < 0.003$
}\label{fig:corrprogb}
\end{figure}

\subsubsection{Additional construction associated with the loss of boundedness: Present participle}\label{sec:Chark5.2.6}

Constructions involving the present participle often used to designate time in the place of subordinate clauses \citep{hauksson1994islensk}. This is accompanied by a decline in dative absolutive constructions. Both are considered features of the \textit{learned style}\is{style} of medieval Icelandic, but are less prominent in modern Icelandic \isi{narrative style}\is{style} (ibid.). The example below in (\ref{ex:andi}) exemplifies both of these: \textit{að morgni} `at morning.\textsc{dat}' is an absolutive, while \textit{komandi} `coming' is a present participle. The example is from \textit{Indíafari}, which, as previously mentioned, exhibits features of bounded\is{boundedness} as well as unbounded\is{boundedness} narrative. The phrase as a whole \textit{að morgni komandi} `at (the) morning coming' (in other words, when morning arrives) is used in place of a subordinate clause indicating temporal sequentiality.


\ea\label{ex:andi}
\gll Þessi lofaði að morgni komandi þar upp á
honum andsvar að gefa. \\ \textsc{dem.m} promised at morning.\textsc{dat} come.\textsc{ptcp.prs} there up to him answer to give\\
\glt `The steward promised to give him an answer the following morning.' \\ \hfill \citep[262]{olafsson1908aefisaga}\footnote{Translation sourced from \citet[83]{phillpotts2017life}}
\z

The loss of prominence of the present participle in this function occurs alongside the increase of \textit{búinn} especially. I would like to propose that this is due to the novel expression overtaking this functional domain across discourse traditions.\is{discourse tradition} Texts with a relative frequency of 2\% or above are not represented in the corpus after 1900. This is shown in \figref{fig:andi}. The texts with the highest relative frequency of the construction are predominantly translations with considerable Latinate influence,\is{transfer from Latin} e.g. \textit{Marta} and \textit{Ectorssaga}.\footnote{See \url{https://github.com/antonkarl/icecorpus/blob/master/info} for more information on the source texts in the corpus.}

 \begin{figure}
    \centering
    \includegraphics[
  width = \textwidth,
  keepaspectratio
]{figures/fig12andi.png}
    \caption{Present participle (\textit{-andi}) in IcePaHC over the centuries}\label{fig:andi}
\end{figure}

\section{Conclusion}\label{sec:Chark6}

%TD:Expand this

The seemingly abrupt rise in frequency of both \textit{búinn} and \textit{PROG}, starting in the 18th century, is difficult to explain without recourse to a shift in mapping between clause and discourse structure. In this article, I argue that this shift paved the way for these unbounded\is{boundedness} aspectual\is{aspect} constructions to gain sufficient frequency in order to be propelled to conventionalisation.

To summarise, I have argued for the following:
\begin{enumerate}
    \item Icelandic exhibits a shift in preference for predominantly bounded to predominantly unbounded\is{boundedness} strategies for structuring narrative during the early modern period (1540--1850).
    \item This shift in \isi{boundedness} is reflected in both a higher frequency of backgrounded temporal clauses as well as a concomitant decrease in foregrounding strategies, most notably clauses starting with the topic-time adverbial \textit{þá} followed by the finite verb.
    \item Two periphrastic\is{periphrasis} aspectual\is{aspect} constructions, \textit{búinn} and \textit{PROG} went from being relatively marginal to being integral parts of the aspectual\is{aspect} oppositions made by Icelandic grammar. I show that an increased availability in unbounded\is{boundedness} construals favoured this change, as backgrounded temporal clauses are important for the conventionalisation of their meaning. This can moreover be seen in the time course of the changes in question and the fact that they are highly correlated.
\end{enumerate}

Alongside the aspectual\is{aspect} constructions which are the focus of this paper, a number of other changes occurred in the history of Icelandic pertaining to the mapping between syntax/semantics and discourse structure, which I take to be compatible with this perspective \citep{booth_revisiting_2021}. Most prominently discussed among these
changes in the literature on Icelandic diachrony is the rise of a dedicated subject position, which has furthermore been tied to the shift from optionally SOV to obligatorily SVO \citep{rognvaldsson1996word, hroarsdottir2000word} and the decline of verb-first declaratives\is{verb-initial declaratives}\is{verb-first} \citep{booth-beck20200jhs}. I leave it to future research to spell out the relationship between these changes in more detail.


Before concluding, a few words are in order about the implications of this paper for \isi{diachronic typology}. The notion of \isi{boundedness} is conceived of as a typological one in the literature \citep{vonstutterheim2005crosslinguistic, vonstutterheim2003} and there is as of yet little research which explicitly discusses the diachronic component, i.e. the switching of systems from predominantly bounded\is{boundedness} to unbounded\is{boundedness} or vice-versa.\footnote{I thank an anonymous reviewer for bringing up this point.} My discussion in the paper can be related to the work of \citet{bhat1999prominence}, who proposes that languages can be classified as \isi{tense}-, \isi{aspect}- or mood-prominent. In my understanding of the literature on \isi{boundedness} \citep{carroll2003information, carroll2004language, vonstutterheim2003, vonstutterheim2005crosslinguistic, von2002cross}, bounded\is{boundedness} structures correspond, broadly speaking, to \isi{tense} prominence and unbounded\is{boundedness} to \isi{aspect} prominence. \citet[179]{bhat1999prominence} discusses his typological findings in the context of \citet{hopper1979some} on grounding, writing that ``languages appear to select a verbal form that belongs to their most prominent verbal category for encoding the foregrounded material". In a \isi{tense}-prominent language like Kannada, this would be the simple (narrative) past, whereas in \isi{aspect}-prominent languages, like Russian, it is the perfective form\is{perfect}. A shift in prominence would thus be expected to correspond to a change in preferred strategies for narrative grounding. This is also why I take it that \isi{boundedness} and prominence are tightly related.

A number of diachronic studies relate their findings to \citeauthor{bhat1999prominence}'s (\citeyear{bhat1999prominence}) notion of prominence. The diachrony of Hebrew, for instance, is widely considered to have undergone a shift from \isi{aspect} prominence to \isi{tense} prominence between the Biblical and Tannaitic periods \citep[74--81]{penner2015verbal}. An entirely unrelated language, Basque, has been claimed to have undergone a shift from \isi{tense}-prominence to \isi{aspect} prominence, including the emergence of a dedicated \isi{progressive}  form \citep{martinez2022latin}. A further point of comparison are the notions of satellite- and verb-framed languages \citep{talmy1985lexicalization}; see \citet{fanego2024english} for a recent discussion that ties a \isi{boundedness} shift in historical \ili{English} to that literature.

On the whole, I hope that this paper can provide a lasting contribution to this emerging empirical literature regarding the relationship between linguistic structure and the encoding of discourse structure.

\section*{Acknowledgements}
Thanks to Ulrike Demske and Barthe Bloom for their feedback and editorial work, without which this paper would not exist. For their thought-provoking comments on the talk that led to this paper, I am grateful for conversations with workshop participants:  Malika Reetz, Sonja Zeman, Anna Cichosz and Bettelou Los. This paper was improved greatly thanks to comments from two anonymous reviewers; their contribution is hereby acknowledged. Finally, thanks are due to Uli Sauerland, Stephanie Solt and Artemis Alexiadou for their academic and personal support throughout the final phase of my PhD project, a part of which is reflected in this paper.

Funded by the Deutsche Forschungsgemeinschaft (DFG, German Research Foundation) -- SFB 1412, 416591334.




\section*{Abbreviations}
\begin{tabularx}{.55\textwidth}{@{}lQ@{}}
ACC & Accusative\\
ADV & Adverbial\\
ART & Article\\
BIO & Biographical texts (genre)\\
COMP & Complementiser\\
DAT & Dative \\
DEM & Demonstrative \\
DT & Discourse tradition \\
F & Feminine \\
FOCAL & Focalised (progressive reading) \\
GEN & Genitive  \\
IcePaHC & Icelandic Parsed Historical Corpus \\
INDET & Indeterminate (reading type)\\
IP-MAT & Matrix clause (IcePaHC tagset) \\
IP-SUB & Subordinate clause (IcePaHC tagset) \\
LAW & Legal texts (genre)\\
\end{tabularx}%
\begin{tabularx}{.45\textwidth}{@{}lQ@{}}
M & Masculine \\
N & Neuter \\
NAR & Narrative (genre) \\
NOM & Nominative  \\
OE & Old English \\
OI & Old Icelandic \\
PDE & Present-day English \\
PROG & Progressive (aspect)\\
PRS & Present\\
PST & Past\\
PTCP & Participle \\
REL & Religious texts (genre) OR Relativiser (glossing) \\
SCI & Scientific texts (genre) \\
SuB & Subordinate clause expressing temporal sequentiality\\
\\
\end{tabularx}

%%%%%%%%%%%%%%%%%%%%%%%%%%%%%%%%%%%%%%%%%%%%%%%%%%%%%%%%%%%%%%%%%%%%%

\sloppy\printbibliography[heading=subbibliography,notkeyword=this]

%%%%%%%%%%%%%%%%%%%%%%%%%%%%%%%%%%%%%%%%%%%%%%%%%%%%%%%%%%%%%%%
\newpage
\section*{Appendix}
\begin{paperappendix}

\begin{table}
\small
\caption{Classification of examples from Benediktsson (2002) }
\begin{tabularx}{\textwidth}{>{\raggedright\arraybackslash}p{3cm} >{\raggedright\arraybackslash}p{6cm} l}
\lsptoprule
Source & Example & Classification \\
\midrule
Njáls saga, ch. 11 & hann [Þorvaldur] var at at hlaða skútuna & Indet \\
Biskupa sögur, Jarteinabók Þorláks biskups helga, AM 645 4° A & MeN voro at. at tialda kirkio... & Durative \\
Íslendingasaga of Sturla Þórðarson & Einarr la ísarvm. oc var helgí. prestr. Scelivngs son at græða hann. & Durative \\

Bósa saga, ch. 8 & ...ok véru þeir allan dag upp at brjóta.... & Durative \\
Bjarnar saga Hítdælakappa, ch. 34 & Þorfinnr...kvað hann [Ásgrim] vera at telja silfr & Focal \\
Icelandic Book of Homilies & fyr óvinom sinom. oc þeím [er] at voro at pína hann. & Durative \\
Book of Homilies & En epter dagmól var hon at ver-alldlego verke... & Stative habitual \\
Book of Miracles, 1st Miracle & ...at þeim þore travt at aeínom dege mynde verþa. & Stative habitual \\
Book of Miracles, 33rd Miracle & þa var til at hogva biargit. Af oxanom. Oc veret at mikiN hlvt dags & Durative \\
Auðunar þætta vestfiryka Morkinskinna (Ed. 1932: 185) &  Einn dag er aleið varit gec Sveinn konungr... & Focal \\
Haralds saga hins hárfagra, ch. 8 & Þeir hofðu verit at þrjú sumur at gera haug einn & Durative \\
Eyrbyggja saga, ch. 37 & ...en þeir Arnkell voro þa at at gera annat hlassit & Focal \\
Jómsvíkinga saga, ch. 8 & En ec þottvmz vppe æga vef en þat var lín vefr... & Indet \\
Þiðriks saga af Bern, ch. 273 & ...oc erv þeir at allan dagin at taka hæstin oc geta æigi tekit & Durative \\
Örvar Odds saga & váru þeir þá at at herklæðaz... & Indet \\
Hákonar saga Hákonarsonar, ch. 236 & voro þa Varbelger at at taka af þa lag sem efter voro brvarinnar & Indet \\
\lspbottomrule
\end{tabularx}
\end{table}

\begin{table}
\small
\caption{Classification of examples from Benediktsson (2002) (continued)}
\begin{tabularx}{\textwidth}{>{\raggedright\arraybackslash}p{3cm} >{\raggedright\arraybackslash}p{6cm} l}
\lsptoprule
Source & Example & Classification \\
\midrule
Þorgils saga skarða, in Sturlunga saga & ...ok váru menn at, at kasta steinum í stofuvegginn & Durative \\
Egils saga einhenda ok Ásmundar berserkjabana, ch. 5 & Kerling var at at renna mjólk & Durative \\
Finnboga saga hins ramma, ch. 24 & Finnbogi var epter oc hrafn hinn litle hía honom... & Durative \\
Stjórn, ch. 183 & ...þiat hann ottaðiz at hann mvndi æigi na at briota & Indet \\
Codex Wormianus, Prologue to four grammatical treatises & ...þvi verðr spvrt hverr kvað þa er fra liðr enn æigi hversv lengi var at verit & Stative habitual \\
Fóstbræðra saga, ch. 8 & hann var at byrgja kviadyrrnar & Focal \\
Hrafnkels saga Freysgoða, 3. Kap & en konur váru at mjólka & Durative \\
Grettis saga, ch. 17 & Hafliði fór til þeira skipverja þar sem þeir váru at ausa, ok mælti & Focal \\
Reykdæla saga ok Víga-Skútu, ch. 27 & hann var at ok smíðaði skot um skála Skútu- & Durative \\
Droplaugarsona saga, ch. 3 & þeir foru brott (ok) þangat sem þeir uoru at ok gerdu hlauss(in) & Indet \\
Ragnars saga loðbrókar, ch. 16 & Ok er hann var at at reka flottann... & Focal \\
Völsunga saga, ch. 8 & Ok er þeir voru at tyrfa hauginn... & Focal \\
Morkinskinna, CCI 6 & ...oc taca nv oc havGva <s> is fvr scipom sinom oc er þeir voro at þa melti [maþr]... & Focal \\
Morkinskinna, CCI 6 & ...oc er þar ál lang vm baggann oc torsott at leysa oc er hann lengi at... & Indet \\
\lspbottomrule
\end{tabularx}
\end{table}


\begin{table}
\small
\begin{tabularx}{\textwidth}{rXrrrrl}
\lsptoprule
 & text & year & t.T & total\_rT & relative\_frequency & Genre \\
\midrule
1 & ntacts & 1540 & 1185 &  36 & 30.38 & rel \\
2 & ntjohn & 1540 & 1685 &  56 & 33.23 & rel \\
3 & eintal & 1593 & 1552 &  40 & 25.77 & rel \\
4 & okur & 1611 & 629 &  28 & 44.52 & rel \\
5 & olafuregils & 1628 & 906 &  65 & 71.74 & bio \\
6 & gerhard & 1630 & 701 &  25 & 35.66 & rel \\
7 & illugi & 1650 & 1929 &  20 & 10.37 & nar \\
8 & pislarsaga & 1659 & 324 &  41 & 126.54 & bio \\
9 & indiafari & 1661 & 1388 &  81 & 58.36 & bio \\
10 & armann & 1675 & 1018 &   8 & 7.86 & nar \\
11 & magnus & 1675 & 204 &   6 & 29.41 & bio \\
12 & modars & 1675 & 373 &   6 & 16.09 & nar \\
13 & skalholt & 1680 & 870 &  33 & 37.93 & nar \\
14 & vidalin & 1720 & 1112 &  73 & 65.65 & rel \\
15 & biskupasogur & 1725 & 1105 &  35 & 31.67 & nar \\
16 & klim & 1745 & 874 &  94 & 107.55 & nar \\
17 & fimmbraedra & 1790 & 1603 &  21 & 13.10 & nar \\
18 & jonsteingrims & 1791 & 1518 & 110 & 72.46 & bio \\
19 & hellismenn & 1830 & 1411 &   9 & 6.38 & nar \\
20 & jonasedli & 1835 & 163 &   9 & 55.21 & sci \\
21 & piltur & 1850 & 1440 &  30 & 20.83 & nar \\
22 & hugvekjur & 1859 & 1107 &  56 & 50.59 & rel \\
23 & orrusta & 1861 & 1804 &  12 & 6.65 & nar \\
24 & torfhildur & 1882 & 2000 &  25 & 12.50 & nar \\
25 & voggur & 1883 & 130 &   8 & 61.54 & nar \\
26 & grimur & 1888 & 625 &  24 & 38.40 & nar \\
27 & vordraumur & 1888 & 759 &  41 & 54.02 & nar \\
28 & fossar & 1902 & 1659 &  64 & 38.58 & nar \\
29 & leysing & 1907 & 1520 &  56 & 36.84 & nar \\
30 & ofurefli & 1908 & 1743 &  27 & 15.49 & nar \\
31 & arin & 1920 & 1149 &  28 & 24.37 & rel \\
32 & margsaga & 1985 & 1705 &  57 & 33.43 & nar \\
33 & sagan & 1985 & 2008 &  57 & 28.39 & nar \\
34 & mamma & 2008 & 1845 & 109 & 59.08 & nar \\
35 & ofsi & 2008 & 1210 &  53 & 43.80 & nar \\
\lspbottomrule
\end{tabularx}
\caption{Temporal subordinate clauses by text (IcePaHC)}
\label{ex:tempsubtable}
\end{table}


 \begin{figure}[ht]
        \includegraphics[width=\textwidth]{figures/SuBBySourceTextasafn.png}
        \caption{Prevalence of subordinate clauses expressing temporal sequentiality (\textit{SuB}) in \textit{Íslenskt textasafn} (16th-18th century subcorpus)}
        \label{fig:subtextasafnplot}
\end{figure}
\end{paperappendix}
\end{document}
