\chapter{The Separation of the Social and the Linguistic}
\label{ch:intro}
\date{}

\epigraph{I have resisted the term \textit{sociolinguistics} for many years, since it implies that there can be a successful linguistic theory or practice which is not social.}{\cite[xix]{labov1972sociolingpatterns}}

\noindent Despite the fact that language use occurs in a social realm, sociolinguistic findings are rarely incorporated into formal linguistic models; socially-conditioned linguistic variation has been treated as an epiphenomenon to grammatical and phonological variation.  This tendency had its beginnings over a century ago with Saussure's distinction between \textit{langue} (the knowledge of a language's structure that is shared across the speakers of that language) and \textit{parole} (the actual language used by an individual in their everyday life) \cite{saussure1916}.  Saussure believed that \textit{langue}, with its regularity and structure, should be the focus of linguistic study and that \textit{parole} was too erratic and variable to be of scholarly interest.  Half a century later, \citeasnoun[4]{chomsky1965} built on this with the distinction between  \textit{competence} (a speaker-hearer's knowledge of his or her language) and \textit{performance} (actual language use in everyday life), later making the differentiation between I-language (internalised language) and E-language (externalised language) \cite[20-22]{chomsky1986}.  The focus of structural linguistic theory has been \textit{langue}, competence, and I-language, treating language as invariant and linguistic categories as absolute.  Methodologies used to investigate internalised linguistic structure typically include eliciting data from a native speaker of a particular language or relying on the intuitions of the researcher.  Surveys are also sometimes conducted, while other studies use texts to determine whether certain structures are grammatical.  

In attempting to answer the question of \textit{how language works}, it is imperative that social effects on linguistic structure be investigated.  This cannot occur only by studying the homogeneous linguistic knowledge of an ``ideal'' speaker-hearer, nor can it occur only by investigating the relationship between linguistic variation and broad social categories.  Language is both social and individualistic; the construction of a symbol's meaning is a social enterprise and how this information is stored and used by a speaker-hearer is determined both by the unique experiences of that individual and by the experiences shared with others from the same community.  In an investigation of \isi{identity}, researchers must study both the community and the individual, ultimately examining the relationship between them \cite[146]{wenger1998}.  Similarly, language does not belong only to an individual or only to the society to which that individual belongs; language exists within and across both.  Linguistic variation that in Saussure's time was considered too messy to be investigated is now known to correlate with a number of factors, including social characteristics of the speaker and the formality of the situation \cite{labov1972sociolingpatterns}, \isi{token frequency} (the number of times a speaker has encountered a word) \cite{bybee2002}, and  how predictable a word is given its position in a sentence \cite{jurafskyetal2002}. Furthermore, there is evidence that this information is stored and affects speech processing \cite{strand1999,jurafsky2003}.  Variation is not somehow systematic ``noise'' that is filtered out; it is stored and used during the \isi{perception} and production of speech.

Sociolinguists have made \textit{parole}, performance, and E-language the focus of their investigation, examining the large amount of variation across different speakers and within the speech of a single individual.  While there is a great deal of variation, much of it is predictable based on social characteristics of the speaker, the persona that the speaker is projecting in a given situation, and the \isi{stance} that a speaker takes in interaction.  The variation is not only predictable but meaningful; it is a component of linguistic knowledge.  Researchers examining this variation argue that a speaker's \isi{communicative competence} is reflected in their behaviour \cite{hymes1972}.  Therefore, examining this behaviour (i.e. actual language in use) provides insight into how language is stored in the mind and accessed during speech production and \isi{perception}.  

Empirical methods of linguistic study allow researchers to

\begin{quote}
avoid the inevitable obscurity of texts, the self-con\-scious\-ness of formal elicitations, and the self-deception of introspection \cite[xix]{labov1972sociolingpatterns}.
\end{quote}

\noindent Empirical methods provide a means of examining speakers' behaviour with the intention of identifying patterns among the variation.  Traditionally in the investigation of sociophonetic patterns, these methods involve the quantitative analysis of variables from sociolinguistic interviews (see Section \ref{interview:method}), but a growing number of studies use experimental methodologies (e.g., \citealt{koopsetal2008,staumcasasantoetal2010,labovetal2011}).  Both methods help demonstrate how linguistic variation is dependent on both social and linguistic information.


Outside of sociolinguistics, there is a growing body of work by researchers who also use empirical methods to examine  language in use (e.g., the contributors to \citeasnoun{bodetal2003}). Like sociolinguists, they have made gradient ``messy'' variation the focus of their research and have shed new light on the nature of the variation.  This work provides strong evidence that language (at all levels of the grammar) is probabilistic; there is a great deal of variation in language and it is predictable if treated stochastically.\footnote{I would not argue that the study of language based on intuitions has no place in linguistics.  However, I do believe that this method can only come part-way in answering the multitude of questions that ultimately address how language works.}

Insights into how language is stored and accessed during production and \isi{perception} can be gained by investigating: \nocite{saussure1916}   

\begin{enumerate}
	\item how language is used in everyday life across different speakers, by individual speakers, and at all levels of the grammar and
	\item how perceivers are influenced by trends from production based on both linguistic and non-linguistic information.
\end{enumerate}

\noindent Patterns in the production and \isi{perception} of speech, regardless of whether they are conditioned by linguistic or social factors, can tell us something about a speaker's linguistic competence, blurring the traditional boundaries between competence and performance, \textit{langue} and \textit{parole}.  

Exciting advances in our understanding of the role of social factors on linguistic variation have arisen from work that combines an ethnographic (participant observation) approach with a variationist sociolinguistic one (e.g., \citeasnoun{eckert1989}, \citeasnoun{eckert2000}, \citeasnoun{lawson2014}, \citeasnoun{mendozadenton2008}, \citeasnoun{moorepodesva2009}).  Many of these studies have focused on \isi{high school} students; there are several reasons for this.  High school can be a difficult period as it marks the transition between childhood and adulthood.  As expressed during an interview with Katrina, one of the speakers in this study, "`It's kind of like there's \isi{youth culture} and then there's human beings and it's really nice to be like accepted as a human being."'  Adolescents are expected to take on additional responsibilities but are not yet treated like adults or, as Katrina expressed feeling, not yet treated like human beings.  This transitional period is marked by linguistic variation, as the teenagers "`try on"' different identities in an effort to construct their identities within the context of the world they live in.

Additionally, there is pressure from within the social make-up of the school, where an individual's style is often interpreted as a reflection of who she is \cite[2]{pomerantz2008}.  While \citeasnoun{pomerantz2008} focused on clothing styles, this is true of other aspects of style, where style is defined as a `socially meaningful clustering of features, within and across linguistic levels and modalities' \cite{campbellkibleretal2006} and non-linguistic levels and modalities. High school students construct their identities in relation to each other (in addition to the world around them) and in doing so, they make use of a multitude of stylistic components, including ways of dressing, ways of walking, and ways of talking.  

In this book, I examine the link between linguistic variation and \isi{identity} in order to develop our understanding of the ways in which language and social information are stored in the mind and accessed during the production and \isi{perception} of speech.  Specifically, I examine the degree to which \isi{lemma}-based phonetic variables are manipulated in the construction of social personae and I investigate the extent to which the relationship between social, phonetic, and \isi{lemma}-based information influences speech processing.  Within the context of data from an all girls' \isi{high school}, I argue that social theory needs to be incorporated into linguistic theory and I present a possible avenue in which to explore this unification of theories.  

Along with \citeasnoun{weinrichlabovherzog1968}, I believe that a

\begin{quote}
	nativelike command of heterogeneous structures is not a matter of multidialectalism or ``mere'' performance, but is part of unilingual linguistic competence \cite[101]{weinrichlabovherzog1968}.
\end{quote}

\noindent Empirical evidence can bring to light the richness and complexity of this competence, resulting in a better understanding of linguistic patterns found at all levels of the grammar.  Using empirical methods to inform a unified probabilistic model of \isi{identity} construction, speech production, and speech \isi{perception}, the research questions I explore here relate both to social theory and to how social information is stored in the mind and is indexed to linguistic representations.  The specific questions to be addressed are:

\begin{enumerate}
	\item Can lemmas that share a wordform have different realisations? 
	\item Do speakers manipulate their realisations of a \isi{lemma} in the construction and expression of their \isi{identity}?  
	\item What is the relationship between the phonetic realisation of a lexical item and how predictable that item is given who the speaker is?
	\item How is this construction of personae related to other speakers who share a similar \isi{stance}?
	\item And what role does this phonetic, \isi{lemma}, and social information play during speech processing?   
\end{enumerate}

\noindent In order to address these questions, I have employed the use of multiple methodologies within a single study, combining the qualitative method of ethnography with the quantitative methods of acoustic analysis and experimental design.
Because I used a number of methods (ethnography, acoustic analysis, and experimental design) and I address a number of theoretical issues (the role of gradience, speaker-specific probability of producing a word, accessing the \isi{lemma} versus the wordform, and the construction of an individual's \isi{identity}), this requires stepping through a vast amount of work from traditionally distinct linguistic subfields.  I begin by discussing the progression of social theory through the waves of variationist studies.  I then describe results from sociophonetic work that uses acoustic analysis and I discuss how this challenges some key assumptions made by popular linguistic theories.  Next I present relatively recent findings from speech \isi{perception} experiments that investigate the relationship between linguistic and non-linguistic information.  At this point, the discussion digresses from work in \isi{sociophonetics} and focuses on two questions of interest that have largely not been addressed in the sociolinguistic literature, namely the degree to which \isi{token frequency} influences phonetic realisations and the degree to which different lemmas that share a wordform can have different realisations. 

I spent a year at Selwyn Girls' High, the pseudonym for the all girls' \isi{high school} in Christchurch, New Zealand where I chose to conduct an ethnographic investigation of \isi{identity} construction.  The girls shared details of their lives with me and allowed me to record their conversations.  While there were a number of close-knit groups at the school, these groups could be categorised according to whether they embodied, created, and perpetuated the school's \isi{norms} (forming what I refer to as Common Room groups) or whether they dismissed, rejected, or failed to conform to these \isi{norms} (forming what I refer to an non-Common Room groups.  The qualitative findings from the ethnography are presented in Chapter \ref{ch:ethnography}.

The linguistic analysis focuses on the word \textit{like}, a word with a number of different functions including the quotative (and Mum's LIKE ``turn that stupid thing off''), the lexical verb (I don't really LIKE her that much), and the discourse particle (Lily was LIKE checking out my brother).  In Chapter \ref{ch:prod}, I discuss the frequency with which different girls and groups at the school used these different functions and I present results from acoustic analysis conducted on tokens of \textit{like} from the girls' speech.  I discuss the results within the context of theories of \isi{identity} construction and consider the possibility that colloquial words can serve as loci for socially-meaningful phonetic variation.


In Chapter \ref{ch:perc}, I present the method and results from three \isi{perception} experiments that I conducted at the school.  The experiments were designed with the aim of determining whether perceivers could use phonetic cues in the signal to identify a \isi{lemma} (here, a particular function of \textit{like}) and whether they could extract social information attributed to a speaker when exposed to only short clips of speech that contain phonetic and \isi{lemma}-based information.

In Chapter \ref{ch:disc}, I discuss the results within the context of two \isi{probabilistic models} of language use: one that replies on \isi{Bayesian} statistics \cite{jurafsky1996,narayananjurafsky2002} and an \isi{exemplar model} of speech production and \isi{perception}, where complete acoustically-detailed representations of encountered utterances are stored in the mind \cite{johnson1997,pierrehumbert2001}.  Finally, I argue for the need to incorporate theories of \isi{identity} construction into linguistic models and I propose a model in which to explore this unification. 


This book is an adaptation of my Ph.D. dissertation \cite{drager2009-thesis}.  Therefore, much of the background literature I discuss is reflective of the field at that time.  However, I have added a concluding chapter that discusses relevant developments in the field since that time and that situates this work within the current context.

Before discussing the methods and results, in the following chapter I turn to a discussion of the social make-up of the project site: an all girls' school which I refer to as Selwyn Girls' High.  I describe Selwyn Girls' High through a description of my experiences from the year I spent there.  We know from previous work that speakers' social characteristics and styles are complex as is the correlation between these styles and the phonetic variables produced \cite{eckert2005,eckert1996nailpolish,mendozadenton2008,zhang2005}.  I ask that readers take the time while reading Chapter \ref{ch:ethnography} to reflect on what life at Selwyn Girls' High was like and to recognise that while most of the girls belong to certain groups, each girl is a unique individual.  Through investigating individuals and how they construct their identities and through investigating variation not only in their production but also in their \isi{perception} of variables, I aim to provide further evidence that the observed variation and the indexical meanings are fundamental aspects of what constitutes a speaker's linguistic competence.

The ethnographic portion of this study was conducted at Selwyn Girls' High (SGH) in 2006 with the aim of becoming familiar with individuals at the school, determining what, if any, social categories were relevant for the girls, and identifying different styles and stances that were present at the school.  I was especially interested in how different individuals constructed their social identities through the manipulation of both linguistic and non-linguistic variables.  As will be discussed in Chapter \ref{ch:prod}, some phonetic variation at the school appears to be linked to the girls' active construction of their social personae.  

In the following chapter I will describe different experiences I had while at SGH.  Although I write from my point of view (and, in fact, start the narrative from my point of view), I have tried to focus the attention on the students rather than myself so that the reader may appreciate the richness of their lives and understand those aspects of life that the girls considered important.  These are real people, with real frustrations and real excitement.  But as explained by Narayan, we as ethnographers ``do not speak from a position outside `their' worlds, but are implicated in them'' (Narayan 1993:676).  Any results are only ``true'' insofar as they are understood in relation to ourselves being implemented within the reality of the speech community we are trying to describe.  Additionally, findings should be interpreted within the context of our biased observations.  We are not objective; our presence and previous biases are inseparable from ourselves.  Therefore, I have tried to remind the reader throughout the text that this is only my story, my ``truth'', of the situation at SGH, and I apologise to the girls for presenting them in a way that reflects at best only a part of who they are.  Still, though it fails to describe the girls entirely, I hope it reflects a part of each of them, however incompletely.

%\newpage
%\thispagestyle{empty}
%\mbox{}