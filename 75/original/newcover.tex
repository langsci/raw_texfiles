\documentclass[12pt]{article}
\usepackage{graphicx}
\usepackage{tikz}
\usetikzlibrary{positioning}
\usepackage{calc}
\usepackage{xstring}
\usepackage{pst-barcode}

\hyphenpenalty=10000

% CAUTION!


%\usetikzlibrary{calc}
\input{./LSP/lsp-seriesinfo/colorinfo.tex} 
\input{./LSP/lsp-seriesinfo/seriesinfo.tex} 
\title{Methods in prosody:\, \newlineCover A Romance language perspective}  
\author{Ingo Feldhausen\and Jan Fliessbach\lastand Maria del Mar Vanrell} 
\renewcommand{\lsSeriesNumber}{6}  
% \renewcommand{\lsCoverTitleFont}[1]{\sffamily\addfontfeatures{Scale=MatchUppercase}\fontsize{38pt}{12.75mm}\selectfont #1}

\renewcommand{\lsISBNdigital}{978-3-96110-104-7}
\renewcommand{\lsISBNhardcover}{978-3-96110-105-4}

\renewcommand{\lsSeries}{silp}          
\renewcommand{\lsSeriesNumber}{6}
\renewcommand{\lsURL}{http://langsci-press.org/catalog/book/183}
\renewcommand{\lsID}{183}
\renewcommand{\lsBookDOI}{10.5281/zenodo.1471564}

\typesetter{Jan Fliessbach\lastand Felix Kopecky}
\proofreader{Adrien Barbaresi, Amir Ghorbanpour, Aysel Saricaoglu, Brett Reynolds, Conor Pyle, Daniela Kolbe-Hanna, Jeroen van de Weijer, Sebastian Nordhoff\lastand Varun deCastro-Arrazola}

\BackBody{This book presents a collection of pioneering papers reflecting current methods in prosody research with a focus on Romance languages. The rapid expansion of the field of prosody research in the last decades has given rise to a proliferation of methods that has left little room for the critical assessment of these methods. The aim of this volume is to bridge this gap by embracing original contributions, in which experts in the field assess, reflect, and discuss different methods of data gathering and analysis. The book might thus be of interest to scholars and established researchers as well as to students and young academics who wish to explore the topic of prosody, an expanding and promising area of study.}


\StrLen{\subtitle}[\subtitleStrLen]


\newlength{\seitenbreite}
\newlength{\seitenhoehe}
\newlength{\spinewidth}
\newlength{\totalwidth}
%\newlength{\totalheight}
\setlength{\seitenbreite}{169.9mm}
\setlength{\seitenhoehe}{244.1mm}

\setlength{\spinewidth}{\csspine}			% Create Space Version. Value is in ./localmetadata.tex
%\setlength{\spinewidth}{\bodspine}   % BoD Version. Value is in ./localmetadata.tex

\setlength{\totalwidth}{\spinewidth+\seitenbreite+\seitenbreite}
%\setlength{\totalheight}{\seitenhoehe}
\usepackage[paperheight=\seitenhoehe, paperwidth=\totalwidth]{geometry}

	%
	% FONT MATTERS
	%
	
\usepackage{fontspec}
\newcommand{\fontpath}{./LSP/lsp-fonts/}
\setsansfont[
	%Ligatures={TeX,Common},		% not supported by ttf
	Scale=MatchLowercase,
	Path=\fontpath,
	BoldFont = Arimo-Bold_B.ttf ,
	ItalicFont = Arimo-Italic_B.ttf ,				
	BoldItalicFont = Arimo-BoldItalic_B.ttf 		
	]{Arimo_B.ttf}
\setmonofont[
	Ligatures={TeX},Scale=MatchLowercase,Path=\fontpath,
	BoldFont = FreeMonoBold_B.otf ,
	SlantedFont = FreeMonoOblique_B.otf ,				
	BoldSlantedFont = FreeMonoBoldOblique_B.otf 		
	]{FreeMono_B.otf}
\setmainfont[
	Ligatures={TeX,Common},
	Path=\fontpath,
	PunctuationSpace=0,							
	%Numbers={OldStyle,Proportional},				% for tables use \addfontfeatures{Numbers={Monospaced,Lining}}
	Numbers={Proportional},	% normal numbers			% for tables use \addfontfeatures{Numbers={Monospaced,Lining}}
	BoldFont = LinLibertine_RZ_B.otf ,				% semi-bold
	ItalicFont = LinLibertine_RI_B.otf ,			
	BoldItalicFont = LinLibertine_RZI_B.otf 		% semi-bold
	]{LinLibertine_R_B.otf}			

  \newcommand{\lsCoverTitleFont}[1]{\sffamily\addfontfeatures{Scale=MatchUppercase}\fontsize{52pt}{16.75mm}\selectfont #1}
  \newcommand{\lsCoverSubTitleFont}{\sffamily\addfontfeatures{Scale=MatchUppercase}\fontsize{25pt}{10mm}\selectfont}
  \newcommand{\lsCoverAuthorFont}{\fontsize{25pt}{12.5mm}\selectfont}
  \newcommand{\lsCoverSeriesFont}{\sffamily\fontsize{17pt}{7.5mm}\selectfont}			% fontsize?
  \newcommand{\lsCoverSeriesHistoryFont}{\sffamily\fontsize{10pt}{5mm}\selectfont}
  \newcommand{\lsInsideFont}{}	% obsolete, see \setmainfont
  \newcommand{\lsDedicationFont}{\fontsize{15pt}{10mm}\selectfont}
  \newcommand{\lsBackTitleFont}{\sffamily\addfontfeatures{Scale=MatchUppercase}\fontsize{25pt}{10mm}\selectfont}
  \newcommand{\lsBackBodyFont}{\lsInsideFont}
  \newcommand{\lsSpineAuthorFont}{\fontsize{16pt}{14pt}\selectfont}
  \newcommand{\lsSpineTitleFont}{\sffamily\fontsize{18pt}{14pt}\selectfont}
	\newcommand{\lsCoverFontColour}{white} % Insert the Colour for Text and Logos on Cover here.
	
	%
	% END FONT MATTERS
	%
	
\begin{document}
\pagestyle{empty}
\pgfdeclarelayer{lspcls_bg} % Please make sure to never use lspcls_... PGF layers in any document
\pgfsetlayers{lspcls_bg,main}<

	\begin{tikzpicture}[remember picture, overlay,bg/.style={outer sep=0}]
		\begin{pgfonlayer}{lspcls_bg} % background layer
			\node [bg, left = 7.5mm of current page.east, fill=\lsSeriesColor, minimum height=22.5cm, minimum width=15.5cm] (lspcls_bg1) {}; % Die können wir noch dynamisch bestimmen
			\node [bg, right = 7.5mm of current page.west, fill=\lsSeriesColor, minimum height=22.5cm, minimum width=15.5cm] (lspcls_bg2) {};
			\node at (current page.center) [bg, minimum height=24cm, minimum width=\spinewidth,dashed] (lspcls_bgspline) {}; % add draw option for preview mode 
		\end{pgfonlayer}
		
		% Text and Graphics Layer
		
			% Spine
			\node [above = 7.5mm of lspcls_bgspline.south] (lspcls_splinelogo) {\color{\lsSeriesColor}\includegraphics{./LSP/lsp-logos/langsci_spinelogo_nocolor.pdf}};
			\node [above left = 15mm and 4mm of lspcls_splinelogo.north, rotate=270] (lspcls_splinetitle) {\color{\lsSeriesColor} \lsSpineAuthorFont{\author} \hspace{13mm} \lsSpineTitleFont{\title}}; 
			
			% Book Cover
			\node [below right = 10mm and 7.5mm of lspcls_bg1.north west, text width=140mm, align=left] (lspcls_covertitle) {\color{\lsCoverFontColour}\lsCoverTitleFont{\title\par}}; % x = 15mm - 7.5mm ; y = 17.5mm - 7.5mm
			
			\ifnum\subtitleStrLen=0  % Is there a subtitle?
				\node [below = 11.2mm of lspcls_covertitle.south, text width=140mm] {\color{\lsCoverFontColour}\lsCoverAuthorFont{\author}}; % If not, just print the author
				\else
				\node [below = 8mm of lspcls_covertitle.south, text width=140mm] (lspcls_coversubtitle) {\color{\lsCoverFontColour} \lsCoverSubTitleFont \subtitle}; 
				\node [below = 11.2mm of lspcls_coversubtitle.south, text width=140mm] {\color{\lsCoverFontColour}\lsCoverAuthorFont{\author}};
			\fi
			
			\node [below right = 197.5mm and 117.1mm of lspcls_bg1.north west] {\color{\lsCoverFontColour}\includegraphics{./LSP/lsp-logos/langsci_logo_nocolor.pdf}};
			\node [below right = 201.5mm and -.1mm of lspcls_bg1.north west, rectangle, fill=white, minimum size=17pt] (lspcls_square) {}; % 2
			\node [right = 3mm of lspcls_square.east, text width=95mm] {\color{\lsCoverFontColour}\lsCoverSeriesFont{\lsSeriesTitle}};
			
			% Book Back Cover
			\node [below right = 16.5mm and 7.5mm of lspcls_bg2.north west, text width=11.5cm] (lspcls_backtitle) {\color{\lsCoverFontColour}\lsBackTitleFont{\BackTitle}};
			\node [below = 10mm of lspcls_backtitle, text width=11.5cm, align=justify] {\color{\lsCoverFontColour}\lsBackBodyFont{\parindent=15pt\BackBody}};
			%\node [below right = 192.5mm and 97.5mm of lspcls_bg2.north west] {\color{\lsCoverFontColour}ISBN \lsBackBodyFont{\lsISBN}};
			\node [below right = 192.5mm and 97.5mm of lspcls_bg2.north west, text width=4cm] {
				\colorbox{white}{
					\begin{pspicture}(0,0)(4.1,1in)
					\psbarcode[transx=.4,transy=.3]{\lsISBN}{includetext height=.7}{isbn}\end{pspicture}}};
			
			
	\end{tikzpicture}
\end{document}