Although the two measures give an indication of how frequently a speaker used quotative \textit{like}, they make different assumptions and they have different implications theoretically in terms of how they may influence phonetic realisations.  Assuming the amount of reported speech in each of the transcripts accurately reflects how often the speaker would use reported speech normally, the first measure reflects how frequently a speaker used quotative \textit{like} when they spoke. [but *still* doesn't control for how much someone talks]


Neither measure is comparable to token frequency as calculated by Bybee.  Instead, what these measures reflect is.  The second measure reflects the probability that, when producing a quotative, the quotative token will be \textit{like}.

First measure is problematic: not all speech acts created equal and not all speakers produce the same amount of speech.