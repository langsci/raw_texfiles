\chapter{Looking forward}
\label{ch:concl}


\epigraph{They may change their roles, their styles of acting, even the dramas in which they play; but -- as Shakespeare himself of course remarked -- they are always performing.}{\citep[35--36]{geertz1973}} 

\noindent Individuals manipulate linguistic variables in the construction of their identities, displaying their communicative competence within the context of the social world in which they participate.  Probabilistic models provide a means of uniting social and linguistic theory and this unification has been a driving force behind the methods and analyses used for this book.  

In Chapters \ref{ch:ethnography} and \ref{ch:prod}, the analysis concerned both the individual girls' styles constructed within the school and the components that make up this style, with a particular emphasis on stylistic phonetic variation of different words.  The results presented in \chapref{ch:perc} display how individuals can use phonetic information when identifying a word, even when the words are identical at the phonological level, and that listeners can identify the function of a word and, to some extent, attribute social characteristics to the speakers based on phonetic information in the stimulus.  In \chapref{ch:disc}, I presented a probabilistic model of speech production, perception and identity construction in which multidimensional stylistic components are indexed to a speaker's style.  My hope is that this model will serve as a stepping stone from which to explore the integration of social and linguistic theory in future work.  

In the remainder of this chapter, I present one possible avenue of inquiry that I believe will aid our understanding of the way in which mental representations of sociolinguistic variables are accessed during the production and perception of speech, ultimately leading to a better understanding of human cognition.

 

\section{Speakers as style-creators}

\noindent Variationist sociolinguists who examine style are increasingly turning their attention to the individual \citep{eckert1996,eckert2011,podesva2011}.  This is important because individuals create stance and style during interaction and, therefore, stance and style need to be examined as processes that emerge in context (see e.g., \citet{coupland2007,kiesling2009,rampton2013}. For example, \citet[110--115]{bucholtz2010} demonstrates how quotative variants used by high school students were influenced by a combination of the speaker's stance and their social group: preppy students were more likely to use \textit{be all} when taking a neutral stance, whereas non-preppy students were more likely to use \textit{be all} when taking a non-neutral stance.  In her study on phonetic realisations of bilingual children, \citet{khattab2013} shows how the children sometimes adopt phonetic features from their parents' non-native English accents to do social work, shifting between native-like and non-native-like realisations in socially-meaningful ways.  This work demonstrates how a single linguistic variant can be used to achieve multiple (and, in some cases, quite different) social goals. Work by Schilling-Estes demonstrates how speakers' sociolinguistic variants shift as a function of both their stances and the variants produced by their interlocutors \citep{schillingestes2004}. Finally, \citet{kirtley-diss} demonstrates how individual speakers use different phonetic variants for social purposes, many of which can be perceived both as being consistent with a particular trait (e.g., masculinity) and as highlighting different aspects of that trait (e.g., the many different ways of being masculine and of doing masculinity).  

When examining stance-taking and style-making by individuals, it is important to keep in mind the social groups to which these speakers orient.  It is believed that linguistic forms are directly indexed to stances and that the linguistic forms become indirectly indexed with social groups who regularly take or are believed to take those stances \citep{bucholtz2009,dubois2007}. Alternatively, it seems plausible that speakers wishing to construct a certain style or take a particular stance may do so by drawing on pre-existing indexations between (or ideologies around) linguistic forms and social groups.  The reality is likely a combination of these, where a variable that is ideologically linked with a group is adopted to take stances that are habitually taken or are associated with that group, and that those variables can then become indexed with new social categories via the stances they enact.

In this book, I have examined stance and style across social groups and, though I recognise that stances and styles are not stagnant, I have largely treated them this way in my analysis.  I have done this due to time restrictions and because I believe that variation in interaction has the most explanatory power when situated within a larger context for that variation.  During stancetaking, speakers do not select linguistic variants out of the blue: the variants are indexed to social meanings (social meanings that can be multiple and complex) through ideologies around what kinds of people produce what variants.  Therefore, understanding the ideologies and the ways in which variables pattern across social groups is important for understanding how the variables are manipulated in the construction of stance and style in interaction.

With this in mind, in \citet{dragerinpress-DPVC} I revisited the SGH data to gain insight into how the girls manipulated realisations of quotative and discourse particle \textit{like} when taking various stances in interaction.  The phonetic analysis was restricted to the tokens analysed for Chapter \ref{ch:prod} of this book and focused on tokens from narratives that included references to other groups or individuals.  The speakers' stances toward the referents were identified (removing those that were ambiguous or unknown) and the tokens were compared within the speech of a single individual.  
 

Three trends emerged from this analysis that are especially noteworthy.  The first is that, for some speakers, their interactional stances appear to be related to the frequency with which they produce discourse particle \textit{like}.  For example, The Goths (a non-Common Room group) frequently produced discourse particle \textit{like}, but they did not produce any tokens when making claims about how they were different from other ``normal'' girls at SGH.  Because the discursive functions of \textit{like} are highly associated with Common Room groups (and The PCs in particular), the absence of discourse particle \textit{like} in these segments helps to highlight that The Goths are different from the Common Room groups.  The second trend that emerged is that some speakers' realisations of discourse \textit{like} contained little phonetic variation.\footnote{I noticed this during the analysis presented in Chapter \ref{ch:prod} but, at the time, failed to comment on the importance of it for the construction of social meaning.}  The only speaker to produce both the quotative and discourse particle as [laik] for all of the analysed tokens was Mariah (The Geeks), who not only consistently realised the /k/ but produced it with a long release.  Mariah generally produced clear speech, including the release of other stops, such as /t/.  Articulation of stop releases is ideologically associated with intelligence, a trait that was valued by members of The Geeks, and strong releases have also been found to be used by geek girls in the United States \citep[125]{bucholtz1998}. Thus, Mariah's realisations of \textit{like} help to construct her personal style. 
\largerpage[-1]

A third trend that emerged concerns the realisations of quotative \textit{like} in which the speakers' evaluations of the referents differ, resulting in a change in footing \citep{goffman1981}.  The speech of two girls, Patricia and Kanani (both from The Sporty Group, a CR group), is especially revealing.  Patricia and Kanani were the two CR girls whose patterns of realisations of \textit{like} were most similar to the NCR girls (see  \sectref{theindividual}).  In Chapter \ref{ch:prod}, I attributed this to their particular social histories and backgrounds: Kanani was formerly a member of a NCR group (so may still have had some NCR group speech characteristics) and Patricia's closest friends went to other schools (and her friends' patterns of realisations of \textit{like} are unknown).  However, after examining how \textit{like} was used in interaction, I now believe Patricia and Kanani were doing something much more complicated and socially-meaningful: they seem to have been manipulating their realisations of quotative \textit{like} as a part of their stancetaking toward the person whose speech they were reporting.  An example from Kanani's speech is shown in Example \ref{ex:kanani-interaction}.   


\sent{
1 Kanani: 		I remember this chick rung up (.) for my brother 

2	\hspace{3em} and um (.)

3 \hspace{3em} hh she's like$_1$ ``hi is Kimo there please''

4 \hspace{3em} I'm like$_2$ ``oh he's in the toilet at the moment'' (.)

5 \hspace{3em} she's like$_3$ (.) ``thanks''

6  \hspace{3em} 					$<$laughter$>$

7 Rose: \hspace{1em} 		I would've been like ``oh actually he's just 
	
	\hspace{5em} [really constipated'' ]

8 KD: \hspace{2em} 	[``he's doing poos''  ]

9 Rose: \hspace{1em}		yeah

10 \hspace{3em}        $<$laughter:$>$

11 Kanani: 	and then my brother came and got the phone 

12 \hspace{3em} and he's like$_4$ $<$raises eyebrows$>$ 

13 \hspace{3em} he went off the phone he's like$_5$ ``what a dick''
 
}\label{ex:kanani-interaction}

\noindent In this example, Kanani is telling a story about her popular and attractive brother, Kimo, receiving a phone call from a woman, a ``chick'' (line 1) who she had never met.  Kanani was very close with her brother - she was very family-oriented in general - and didn't necessarily approve of the woman who called.  Within the context of the conversation, Kanani positions herself as family-oriented and down-to-earth, and the woman who rings up as coquettish and silly.  These positionings align with the realisations of quotative \textit{like}: when preceding her own or her brother's speech, the quotative is realised with a less diphthongal vowel and no /k/ (the ``CR realisation''), whereas when introducing the speech of the woman on the phone, the vowel is diphthongal and the /k/ is present (the ``NCR realisation''). These trends do not appear to result from phonological environment or speech rate: like$_5$ (line 13) is followed by /w/, an environment which promotes the realisation of /k/ in quotative \textit{like}, but the /k/ is not realised.  Likewise, given the relatively quick speech rate in like$_1$ (line 3), we might expect a monophthongal vowel but, in fact, the vowel is diphthongal.

\begin{table}[htbp]
\caption{Phonetic realisations of tokens of \textit{like} found in Example (\ref{ex:kanani-interaction}), adapted from Table 5 in \citet{drager2016}.  The Euclidean distance (EucD) is shown in Bark and speech rate is syllables per second in the IP surrounding the token but does not include the token itself.}	
	\label{tab:kanani-interaction}
	 \begin{center}
		\begin{tabular}{llrcr}\lsptoprule
	
   token & referent & EucD & /k/ present & syllables/sec\\
  \midrule
like$_1$ & woman & 2.24 & y & 4.88 \\
like$_2$ & Kanani & 1.36 & n & 6.47 \\
like$_3$ & woman & 2.53 & y & 2.78 \\
like$_4$ & brother & 0.31 & n & 5.26 \\
like$_5$ & brother & 1.53 & n & 4.08 \\

\lspbottomrule
		\end{tabular}
	
	\end{center}
\end{table} 

In other words, when reporting the speech of someone who she aligned with, Kanani produces quotative \textit{like} with a monophthongal vowel and non-realised /k/ (the CR realisation), whereas she produces it with a diphthongal vowel and [k] when reporting the speech of someone who she does not know or respect.  This suggests that Kanani does not have imperfect acquisition of the variable as a result of changing from a non-Common Room to a Common Room group; instead, she demonstrates a sophisticated knowledge of both systems and appears to use it as a narrative device. In fact, it is possible that her social history provided her with greater (unconscious) control of the variable as a result of having greater exposure.

Some other girls (e.g., Meredith, Isabelle, and Patricia) demonstrate analogous trends, with appropriate realisations for their group (CR or NCR).  For other speakers, a trend is less clear (or is non-existent as in the case of Mariah) and the analysis is complicated by ambiguous or unknown stances.  Additionally, other factors such as speech rate and pitch that are linked with the phonetic variables of interest are not controlled for in this type of analysis, which limits the data one can reliably use to examine variation of this type.\footnote{Variation in speech rate and pitch can, of course, be socially meaningful.  When they are, variation in related phonetic variables may in fact be an epiphenomenon.}  

In my estimation, the finding that phonetic realisations of quotative \textit{like} are conditioned by interactional stance is more suggestive than conclusive due to the relatively small number of speakers and data points that demonstrate the trend. But, given the fact that other researchers have demonstrated that the form of the quotative (e.g., \textit{be like} vs. \textit{be all}) varies by stance, we might actually expect that the realisations of these highly salient words would vary, particularly when - like in these data - \textit{be like} so strongly dominates the quotative system of all of the speakers.  It is worth noting that the variation in the realisation of discourse \textit{like} does not negate the quantitative findings presented in this book.  Because only some individuals seem to vary their realisations of quotative \textit{like} in this way and because the girls so frequently voice their own or their friend's speech, the trends reported in Chapter \ref{ch:prod} do seem to be the most frequent realisations.  In fact, the distinction between CR and NCR realisations is needed to interpret the stance-based variation reported by \citet{dragerinpress-DPVC}.  Taken together, the work suggests that speakers can use probabilistic patterns of sociolinguistic variables (including those tied with locally-constructed social categories) to help take stances during the course of an interaction, but much more work along these lines is needed.




\section{Concluding remarks}


The findings presented in this book demonstrate the benefits of combining qualitative and quantitative analysis and of examining variation in both production and perception.  While I do not advocate abandoning traditional variationist description by any means, I do believe that inroads will be made by variationists who choose to explore multiple avenues of inquiry.  In continuing the progression of social theory through the investigation of linguistic variation, sociolinguistics will benefit from increased focus on variation in speech perception in addition to production, using computational models to explore sociolinguistic assumptions and predictions, examining the behaviour of individuals within a single interaction, and using insights from other areas of empirical linguistics such as laboratory phonology.  Additionally, laboratory phonology will benefit from incorporating more socially-informed data and analyses, not only by examining speakers' and listeners' social characteristics but moving toward a more nuanced treatment of socially-conditioned variation that includes a focus on the individual and the individual's goals in interaction. As sociolinguists have long argued, the division of language into the social and the non-social is artificial; the time is ripe for social theory and linguistic theory to be examined together within the context of unified models of language use.
