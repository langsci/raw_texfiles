\date{}
\chapter{Variation in speech perception}
\label{ch:perc}
%\maketitle

\noindent This chapter presents results from three perception experiments. First, I provide a short review of the production data and discuss the hypotheses that the experiments set out to test. Then I present the methodology and results from each experiment. Finally, I briefly discuss the theoretical implications of the findings. A more in-depth theoretical discussion can be found in Chapter \ref{ch:disc}.

As discussed in Chapter \ref{ch:prod}, acoustic analysis of the girls' speech indicates that different girls produced phonetically different tokens of \textit{like} that varied systematically depending on the token's function and on whether or not the girl was in a group who ate lunch in the common room. Tokens with a higher F2 value at the nucleus target (i.e.~a fronter nucleus) were more likely to be a discourse particle than a grammatical function (i.e.~either the lexical verb or the adverb). A larger /l/ to vowel duration ratio, a lower mean pitch, and a more diphthongal vowel were more likely to be produced in both the grammatical functions of \textit{like} and the discourse particle than in the quotative. There were also two results that depend on an interaction between social group and the function of \textit{like}. (1) CR girls were more likely to realise the /k/ in the discourse particle than in the quotative, and NCR girls were more likely to realise the /k/ in the quotative than in the discourse particle. (2) CR girls were more likely to produce a long /l/ to vowel duration ratio in the discourse particle compared with the traditionally grammatical functions, and NCR girls were more likely to produce a long /l/ to vowel ratio in the grammatical functions. 

These findings provide evidence in fa\-vour of acous\-tically rich or acous\-ti\-cally-informed lemma-level representations and they raise questions about the degree to which the relationship between phonetic, social, and lemma-based information is stored in the mind. If perceivers are sensitive to this relationship during perception, this would provide further evidence that the mental representations are stored in such a way as to allow indexing between the different types of information. Exemplar Theory (see \S \ref{exemplar}) predicts that both lemma-conditioned and socially-conditioned phonetic variation observed in production should influence an individual's perception of the variants. A series of perception experiments was designed and conducted in order to test this hypothesis.

All Year 13 students were invited to take part in a series of three perception experiments. Forty-two girls chose to take part during the two weeks that the experiment was run. Additional girls offered to take part but, due to time constraints, were not able to participate before the debriefing I gave at the year's last assembly. 

The experiments were run using a Praat script and a Gateway laptop computer. Participants listened to the tokens over Sony Dynamic Stereo Headphones (MDR-V300).

As stimuli, all three experiments used clips of speech from informal interviews conducted with girls at the school, so some participants responded to stimuli comprised of their friends' or their own speech.  A recording of a male New Zealander reading the question numbers was played prior to an individual question's stimuli.
 
All auditory stimuli contained the word \textit{like}, where \textit{like} was either the discourse particle, the quotative, or a grammatical function. A token of lexical verb \textit{like} was used as the grammatical function whenever possible, but due to low token numbers for some speakers, the adverb was also sometimes used as it was found to be phonetically similar to lexical verb \textit{like}.\footnote{Other grammatical functions were not frequent enough in the data to be included in the preliminary phonetic analysis. Compared to all of the discursive functions, lexical verb \textit{like} and adverbial \textit{like} were most phonetically similar in terms of all phonetic factors tested in the production data.} I use the term grammatical \textit{like} to refer to the traditionally grammatical functions (as opposed to discourse pragmatic functions) used as stimuli. 

Tokens were spliced from the original signal using Praat. They were spliced at the nearest zero-crosspoint to the segment boundaries outlined for the production analysis in \sectref{sec:phoneticmethod}. All tokens that were labelled as having the /k/ present also had the /k/ released. There were no acoustic modifications made to the waveforms. 
 
After completion of all three perception experiments, participants were re\-cor\-ded reading the context sentences used in the second perception experiment. The list of sentences used as a production task is provided in Appendix \ref{appen:stimuli}. The production task was conducted with the intention of comparing the production and the perception of \textit{like} for a single speaker. However, during the reading task, girls consistently produced the diphthong and the /k/ for all tokens of \textit{like}, regardless of function or social group.\footnote{It was my impression that girls were not engaging with the meaning of the words when reading the passage. For example, they often read the first part of a sentence (e.g., \textit{I was like}), paused, and then continued on with the rest of the sentence (e.g., \textit{only two seconds behind}).} Therefore, acoustic phonetic analysis was not conducted on the reading passage and production trends from spontaneous speech were used instead to compare an individual's production and perception. 

A number of girls from a variety of groups were invited to take part in the experiment. Due to lack of interest on their part and time constraints on mine, not all girls who participated in interviews took part in the perception experiment. A total of 42 girls took part, 23 of whom were in groups that ate lunch in the CR.\footnote{One girl, Kristy (The BBs), was in a group that ate lunch in the CR, but she rarely ate lunch with her friends and would instead do school-sponsored activities during lunchtime. However, she is included as a CR girl in this analysis due to her choice of friends and her acceptance of similar values to the other CR girls.} \tabref{groupsperc} shows the number of girls from each group who took part in the experiment.



To test the degree to which speaker-specific phonetic trends in production influenced their perception, models were first tested on a subset of the data: responses from the 28 girls whose speech the production results were based on (Chapter \ref{ch:prod}). As discussed in \sectref{exemplar}, Exemplar Theory predicts that trends in a speaker's production will influence their perception. However, speaker-specific phonetic information was not found to influence perception significantly, either on its own or as part of an interaction.\footnote{Although speaker-specific production patterns did not reach significance in the model, the directionality of the patterns' relationship with perception in Experiment 1 suggests that there may be a link between the production and perception of /k/ realisation. Using the difference between the speaker-specific random effects' coefficients of discourse particle \textit{like} and quotative \textit{like} from a production model of /k/ realisation, the speaker-specific likelihood of producing /k/ in the discourse particle relative to the quotative was tested as a predictor in the perception model. Participants who were more likely to realise the /k/ in the discourse particle than the quotative were more likely to identify the first token as the quotative if the second token had the /k/ present, and participants who were less likely to realise the /k/ in the discourse particle than in the quotative were less likely to identify the first token as the quotative if the second token had the /k/ present. In Experiment 1, this trend is approaching significance (p$=$0.06). It is possible that if acoustic phonetic analysis was conducted on speech for a greater number of speakers, this interaction would reach significance. For questions in Experiment 2 where /k/ presence was mismatched across the two tokens, girls who were more likely to drop the /k/ in discourse particle \textit{like} were more likely to identify the token with the /k/ as the quotative than were girls who were more likely to drop the /k/ in the quotative. This trend is in the expected direction but is not approaching significance (p$=$0.37).} Therefore, the reported results were based on data from all 42 girls who took part in the experiments.\footnote{One participant (a CR girl) did not complete the last two tasks in Experiment 3 because the bell rang and she had to go to class. Her data were not included for Experiment 3.} 
\begin{table}[p]
\caption{The number of participants who took part in the perception experiment, by group}	
	\label{groupsperc}
	 \begin{center}
		\begin{tabular}{lrlr}
\lsptoprule
	      
          
CR & &NCR& \\
  \midrule
The PCs&  4    	&Pasifika Group& 1 \\
Sporty Girls& 2 &The Goths& 5 \\
Trendy Altern.&4&The Geeks& 4 \\
Rochelle's Group& 1 &Real Teenagers& 2\\
Relaxed Group& 4&Sonia's Group&1 \\
The BBs&8      	& Christians&2\\
     &         	& Sally's Group & 3 \\
     & 					&A Loner & 1 \\
     Total &  23&        &   19 \\
  \lspbottomrule
		\end{tabular}
	
	\end{center}
\end{table}

\begin{figure}[p]
 \begin{center}
\caption{Example question from Experiment 1: Participants matched two auditory tokens that contained \textit{like} (here, \textit{he was like} and \textit{he was like}) to different grammatical contexts provided on the answer-sheet.}\label{ExampleExp1} 

		\begin{tabular}{lllrr}\lsptoprule

 
 & a) He was like...	&	b) He was like... & \\
	\midrule
	\\
  & ...``what's that?'' 	&		&a	&b  \\

                    &...wearing this kind of visor thing.	& &a	&b  \\
                    \\	
		\lspbottomrule
		\end{tabular}
	
	\end{center}
\end{figure}
\clearpage


\section{Experiment 1}\label{exp1} 

\subsection{Methodology}

\subsubsection{The task}

In the first experiment, participants were played two clips of speech in a given question, each containing the word \textit{like} spoken by the same girl. The voices of seven girls were included, four of whom were members of CR groups. The stimuli for each question was made up of either the quotative and the discourse particle or a grammatical function of \textit{like} and the discourse particle. For example, in question 2, Isabelle was first heard saying \textit{he was like} where \textit{like} was a discourse particle in the source sentence \textit{and HE WAS LIKE singing along to music} (Interview, 02-05). Isabelle was then heard saying \textit{he was like}, where \textit{like} was the quotative: \textit{do I need to shave my legs and HE WAS LIKE ``naw''} (Interview, 02-05). Participants did not hear the disambiguating context. Upon hearing the clips (e.g., \textit{he was like}), they were asked to match each of the auditory stimuli with one of the contexts provided on the answer-sheet, as shown in \figref{ExampleExp1}. 



Participants were told that each of the sound clips was taken from a sentence similar to one of the contexts provided and that there was a one-to-one mapping between a context and an auditory token. In other words, one token was taken from a sentence similar to one of the contexts on the answer-sheet and the other token was taken from a sentence similar to the other context on the answer-sheet. They were asked to circle (a) for the context they felt went with the first sound clip and to circle (b) for the context they felt went with the second sound clip. The majority of girls circled the corresponding letter, but several girls chose instead to draw lines between the text representation of the auditory token on the answer-sheet and the context provided. I treated both response techniques as equivalent during analysis.

\subsubsection{The stimuli}

None of the contexts on the answer-sheet were actual excerpts from the interviews. This allowed for control of the phonological environment that followed \textit{like}. The first sound was matched between contexts of the same question in order to avoid response biases due to coarticulation in the source sentence.

The order of the different functions of \textit{like} was pseudo-randomised. Half of the auditory tokens of grammatical and quotative \textit{like} were played before the discourse particle. Ten of the questions compared grammatical-discourse particle pairs and twenty compared quotative-discourse particle pairs. Grammatical \textit{like} and quotative \textit{like} were not compared in Experiment 1 due to the low number of occurrences where their preceding context was matched.\footnote{For example, it would be possible to compare an adverb, as in \textit{oh no it was LIKE the coat tie} (Gina, The PCs, Interview, 16-05), with a quotative, as in \textit{he was dancing naked in my room last night and it was LIKE ``dih''} (Isabelle, The Real Teenagers, Interview, 02-05), where both the adverb and the quotative were preceded by \textit{it was}. However, at the time of designing the experiment, too few suitable adverbs were identified.} Additionally, because tokens were difficult to find given the low number of recordings I had transcribed at the time of designing the experiment, some tokens were used as stimuli in more than one question. The stimuli are listed in the order they were played in Appendix \ref{appen:stimuli}.

There was no training session for the experiment and it was hypothesised that participants may fail to respond to the first question. Therefore, the first pair of stimuli was repeated later in the experiment.  This resulted in a total of 31 pairs of tokens. After all 31 questions were played, the same 31 questions were repeated in the same order. In the second half of the experiment, the contexts for each question were presented in the same order on the page and the auditory tokens were played in the reverse order; if the discourse particle was played first in the first half, it was played second in the second half. This was done in order to remove a potential effect of token order.

Contexts were presented so that the context containing the discourse particle was first on the page for half of the questions. The context order and the order of auditory stimuli were mismatched, so that half of the time that the discourse particle context was first on the page, the auditory token of discourse particle \textit{like} was played second. 

The stimuli for a given question were matched as closely as possible. In some cases, the match was identical at the lexical level (\textit{He was like} and \textit{He was like}). However, in some cases the pair was not identical (\textit{They were like} and \textit{They're like}). This was due to the small number of identical phrases found in spontaneous speech within the recorded interviews for a single girl. Care was taken to match clips that were as similar as possible at the lexical level. In P\=akeh\=a (European) New Zealand English, quotative \textit{like} is more likely to occur with the historical present (i.e.~present tense morphology with a past temporal reference), as in \textit{he is like}, than with the past tense, as in \textit{he was like}. It is also most likely to occur with the first person singular (e.g., \textit{I was like}) \citep{buchstallerdarcy2009}. None of the experimental stimuli for a given question differed in both of these respects, but some differed in either tense or person. Questions where the contexts were lexically matched (matched preceding) were labelled separately from those where there was a mismatch. Mismatched questions for which the first token was either in the historical present or in the first person (i.e.~questions where the first token had the more frequently observed context for quotative \textit{like} than that of the second token) were labeled as `likely preceding', and mismatched questions for which the first token was the less frequently observed context compared with the second token were labelled as `unlikely preceding'.\footnote{There is some evidence that for M\=aori English speakers, quotative \textit{like} is more likely to be produced in the past tense than in the historical present \citep{darcy2010}. Because the vast majority of the participants in the current study were speakers of P\=akeh\=a English, I use the terms `likely' and `unlikely' to refer to the organization of the stimuli, although these terms would not be appropriate for an ethnicity-based investigation.}

The auditory stimuli were intentionally designed to represent a wide range of phonetic cues upon which the listeners could potentially rely instead of being representational tokens of the different types of \textit{like} from each group. This was done in order to determine whether participants would use particular phonetic cues to determine what word they had heard and whether their responses were congruent with trends in the girls' production. Phonetic characteristics of the different tokens in the quotative and discourse particle pairs are shown in \tabref{tab:cues1qd} and phonetic characteristics of the tokens in the grammatical and discourse particle pairs are shown in \tabref{tab:cues1gd}. Only phonetic characteristics that significantly predicted the functions in the production models are shown in the corresponding tables. All tokens were played twice.

\begin{table}[ht]
\begin{center}
\begin{tabular}{ld{2}d{2}cd{2}d{2}}
 \lsptoprule
 &	\multicolumn{2}{c}{CR} 		&			 & \multicolumn{2}{c}{NCR}  \\
 \cmidrule{2-3}\cmidrule{5-6}
  type					&	 \multicolumn{1}{c}{quote} & \multicolumn{1}{c}{dp} & &   \multicolumn{1}{c}{quote}  & \multicolumn{1}{c}{dp}    \\
  	 	  \midrule

 monophthong 	&	 2     &  2 	&\vline&   1  	  &  0    \\
 /k/ present 	&	 4     &  5  &\vline&   6 	  &  3     \\
 ave. mean pitch &   226.3 &	243.2 &\vline&  217.6	& 271.3  \\
 ave. duration ratio &  0.40 &  0.29  &\vline&  0.31 & 0.36 \\
 matched prec.   &   5   &    &\vline&     8   &  \\
 number of tokens	&  11    & 11& \vline&    10 	&  10   \\

   \lspbottomrule
   
\end{tabular}
\caption{Potential phonetic cues in Ex\-peri\-ment 1 for quota\-tive-dis\-course par\-ticle stimuli, by type and social group}\label{tab:cues1qd}
\end{center}
\end{table}	

% glottalised 	&	 5    	&  7 	\vline&   1 		&  3     \\

\begin{table}[ht]
\begin{center}
\begin{tabular}{lrrcrr}
  \lsptoprule   
  	 						&	\multicolumn{2}{c}{CR} 			&		& \multicolumn{2}{c}{NCR} \\
\cmidrule{2-3}\cmidrule{5-6}  	 	 
type						& gram & dp &	&  gram  & dp   \\
\midrule
%monophthong 		&    \\
%/k/ present 		&  1  &       &     &   \\
%matched prec.   &  4          &              \\
ave. nuc. F2 (Hz)        &  1574 &	1535 & &   1381  &	1451 \\
%matched preced.   &         &        \\
number of tokens	& 5  & 5  & & 5  & 5  \\

   \lspbottomrule
   
\end{tabular}
\caption{Potential phonetic cues in Ex\-peri\-ment 1 for quota\-tive-gram\-mati\-cal stimuli, by type and social group}\label{tab:cues1gd}
\end{center}
\end{table}	


%%%%%%%%%%%%%%%%%%%%%%%%%%%%%%%%%%


\subsection{Results}
Of the 2604 possible responses, 108 questions were not responded to and are not included in the analysis. Overall, participants performed at chance level when identifying the function of an auditory token of \textit{like} (50.7\% correct). A high accuracy rate was not anticipated given the non-representative phonetic features included in the stimuli.

In order to determine whether participants used phonetic cues to identify the word and whether these cues were consistent with trends in production, two mixed effects models were fit to the data from Experiment 1. The first model is based on responses to questions that compared the quotative with the discourse particle and the second model is based on responses to questions that compared the discourse particle to a grammatical function.


\subsubsection{Experiment 1, Model 1: The quotative and the discourse particle}

Model 1 includes responses to 1690 questions comparing the quotative and the discourse particle from all 42 girls who took part in the experiment. It models the likelihood of identifying the first token as the quotative. This was done instead of modelling accuracy in order to test whether participants relied on phonetic cues in the stimulus when identifying a token, independent of the actual function of that token. This was particularly important given the unequal distribution of phonetic cues across the different function types.

The data were fit using R (R Core Development Team 2007) and the lme4 package \citep{lme4}.  Participant and question num\-ber were in\-cluded as ran\-dom effects in the mo\-del and only factors reaching significance were included as fixed effects. Fac\-tors that were test\-ed but not in\-cluded in the mo\-del were de\-gree of mo\-noph\-thong\-isa\-tion, whether the participant was in a CR group, and whether the quotative stimulus had the /k/ realised. Also tested was how far through the experiment the participant was at the time of responding as well as whether the response was during the first or second half of the experiment. Fixed effects that were included in the model, shown in \tabref{qdcoeff1}, were whether the first context on the answer-sheet was the quotative (quote second) and the difference between the /l/ to vowel duration ratio of the first and second auditory token (duration ratio diff.). Also included was a three-way distinction between whether the preceding context of the first token was less frequently observed in production with the quotative (preced. unlikely), more frequently observed (preced. likely), or whether the preceding context was matched at the lexical level (preced. match). 

An estimated scale parameter is a measure of how the actual variance in the data compares to the variance assumed by the model. For a perfectly fit model, the value would be equal to 1. For this model, the estimated scale parameter is 0.9989158, which indicates that the model is a good fit. 

 
% latex table generated in R 2.6.1 by xtable 1.5-2 package
% Thu Aug 14 10:41:12 2008
\begin{table}[ht]
\begin{center}
\begin{tabular}{ld{5}d{5}d{3}d{7}}
 \lsptoprule
 & \multicolumn{1}{r}{Estimate} & \multicolumn{1}{r}{Std. Error} & \multicolumn{1}{r}{z value} & \multicolumn{1}{r}{Pr($>$$|$z$|$)} \\
 \midrule
(Intercept) & 0.8036   &  0.1337 &  6.009 & <0.0001\\
  quote second & -1.0213  &   0.1056 & -9.669  & <0.0001 \\
  preced. match & -0.2847  &   0.1387  & -2.053 & 0.04007 \\
  preced. unlikely & -0.4539  &   0.1706 & -2.660 & 0.00781 \\
  duration ratio diff. & -0.7200  &   0.2519  & -2.858 & 0.00426 \\
   \lspbottomrule
\end{tabular}
\caption{Experiment 1 coefficients of fixed effects from Model 1, comparing responses to the quotative and the discourse particle}
\label{qdcoeff1}
\end{center}
\end{table}	



The estimates provided for each factor in \tabref{qdcoeff1} are in log odds and can be taken as an indication of how robust the effect of each factor is. The estimate for the intercept is the likelihood of identifying the first token as the quotative given the default factors. The model assumes as defaults that the quotative context is listed first on the answer-sheet and that the first auditory token has a preceding context that is more likely than that of the second auditory token. It also assumes that the difference between the /l/ to vowel duration ratio of the first and second token is zero, which indicates that /l/ to vowel duration ratios of the first and second token were equal to one another. To determine the degree of a categorical factor's effect, that factor's estimate should be added to the intercept's estimate. For gradient factors, such as the difference in duration ratio, the product of the estimate and the value for a given token is added to the intercept's estimate.

Participants were significantly more likely to identify the first auditory token as the quotative if the quotative context was listed first on the answer-sheet (p$<$0.0001). This trend reflects an overall bias for participants to identify the first token heard with the first context on the sheet. This bias is the forced-choice equivalent of an acquiescence response set (the tendency for participants to answer `yes' for yes/no questions in experimental work), an effect which is commonly found in the psychology literature (cf. \cite{bentleretal1971}). The experiment design controlled for this by counterbalancing the auditory stimuli. Therefore, the influence of phonetic cues could be examined above and beyond the response bias. Additionally, including this factor in the model allowed for examination of other potential factors that influenced responses; the model held this constant when testing effects of the other factors. 



Participants were less likely to identify the first token as the quotative if the first auditory token of \textit{like} had an `unlikely preceding' context (p$<$0.01) than if it had a `likely preceding' context. This was in the expected direction given the trends described in \citet{buchstallerdarcy2009}. Responses to tokens that were matched for preceding context fell between the two mismatched question types. When identifying the function of a token, participants appear to have used their implicit knowledge about the syntactic distribution of contextual information that is associated with quotative \textit{like}. This finding provides evidence that individuals were sensitive to lemma-specific contextual information during perception. In order for perception to be influenced by the preceding context, chunks of speech that carry this syntactic information could be stored and indexed to the stored lemma. Chunks of speech that are larger than a single word could be stored as a cloud of exemplars or an abstract representation.\footnote{Interestingly, the two M\=aori English speakers who participated in the experiment responded in the opposite direction from the P\=akeh\=a participants with regard to this factor. This is consistent with trends in the production of quotative \textit{like} in M\=aori and P\=akeh\=a Englishes described by \citet{darcy2010}. Further work is needed to determine the extent to which perceivers from different social groups use lemma-specific contextual information that is consistent with socially-conditioned trends from production. All girls were included in the analysis presented in this chapter, regardless of ethnicity.} It is also possible that probabilities about context could be updated through experience. These possibilities are discussed further in Chapter \ref{ch:disc}.


In production, quotative \textit{like} was more likely to have a smaller ratio of /l/ to vowel duration than discourse particle \textit{like}: the /l/ was shorter in quotative \textit{like} than in discourse particle \textit{like}, relative to the duration of the vowel. In perception, participants were less likely to identify the first token as quotative \textit{like} if it had a larger duration ratio than the second token (p$<$0.01). In other words, perceivers' responses were consistent with trends in their production.


These results are discussed in more detail in Chapter \ref{ch:disc}. First, I present the second model for Experiment 1 and the methodology and results from Experiments 2 and 3. 



\subsubsection{Experiment 1, Model 2: Grammatical functions and the discourse particle}

A second model was fit to the data for questions comparing grammatical functions of \textit{like} with the discourse particle, modelling the likelihood of identifying the first token played as the grammatical token. Most of the same potential predictors that were tested in Model 1 were also tested in Model 2 and only those factors that reached significance were included in the model.\footnote{The likelihood of the preceding context was not tested due to a lack of previous work exploring the distribution of contextual information in production.} Model 2 was based on 806 responses from 42 different girls. The estimated scale parameter of the model is 0.9942788.



% latex table generated in R 2.6.1 by xtable 1.5-2 package
% Thu Aug 14 10:40:03 2008
\begin{table}[ht]
\begin{center}
\begin{tabular}{ld{5}d{5}d{5}d{5}}
  \lsptoprule
 & \multicolumn{1}{r}{Estimate} & \multicolumn{1}{r}{Std. Error} & \multicolumn{1}{r}{z value} & \multicolumn{1}{r}{Pr($>$$|$z$|$)} \\
  \midrule
(Intercept) & 0.3933  &   0.1003  & 3.923 & <0.0001 \\
  gram second & -0.8205  &   0.1789  & -4.586 & <0.0001 \\
   \lspbottomrule
\end{tabular}
\caption{Experiment 1 coefficients of fixed effects from Model 2, comparing responses to the discourse particle and grammatical functions of \textit{like}}
\label{gdcoeff1}
\end{center}
\end{table}

As shown in \tabref{gdcoeff1}, only one factor was included in the model: whether the grammatical context was listed first on the answer-sheet. The difference between the F2 values of the first and second tokens' nucleus target was not found to significantly predict responses.

Participants were more likely to identify the first token as the grammatical function if the grammatical function was listed first (p$<$0.0001). This parallels results from Model 1 where participants were more likely to identify the first token as the quotative if the quotative context was listed first. Both of these findings reflect an overall bias with identifying the first auditory token with the first context listed. 


There were no questions in Experiment 1 that compared the quotative with a grammatical function of \textit{like} due to the lack of tokens with comparable preceding contexts. In Experiment 2, the preceding context was not included in the auditory stimuli, making a comparison between grammatical functions and the quotative possible.




\section{Experiment 2} 

\subsection{Methodology}

\subsubsection{The task} 

As in Experiment 1, participants in Experiment 2 were asked to match the word \textit{like}, which had been spliced from spontaneous speech, with the contexts provided. In Experiment 2, however, participants were exposed only to the word \textit{like}. Additionally, the voices of only four girls were used. All were from different groups at the school. The girls were: Tracy (The PCs), Rose (Relaxed Group), Onya (Real Teenagers), and Meredith (The Goths). Two of the girls (Tracy and Rose) were in groups who ate lunch in the CR and two of the girls (Onya and Meredith) were in groups who did not. Voices were selected to cover a range of girls from different groups. Additionally, I used voices of girls for whom a larger amount of speech was transcribed, as this made the tokens more easily identifiable in an automated search.\footnote{To identify tokens to be used as potential stimuli, transcripts were searched using the tool ONZE Miner \citep{onzeminer}.} 

In contrast to the longer clips in Experiment 1, the shorter clips in Experiment 2 allowed for a three-way comparison between the different functions of \textit{like}. Five of the tokens for each voice were grammatical functions of \textit{like} (either a lexical verb or an adverb), five were quotative \textit{like}, and five were discourse particle \textit{like}. Participants were asked to distinguish between grammatical and quotative \textit{like}, grammatical and discourse particle \textit{like}, and quotative and discourse particle \textit{like}. The two auditory tokens for each question number were produced by the same speaker, and stimuli were blocked for each voice. After responding to 15 questions for each voice, participants were asked if they recognised the voice and were asked to identify the speaker if possible. 

The contexts provided on the response sheet differed for each question within a single voice. The same contexts were used across the different voices, but they differed in the order they appeared within a particular question and the order in which the context pairs were listed. For example, the contexts for speaker 1, question 3 were in the following order: \textit{I was like ``Only if he asks me himself''} and \textit{I was like only two seconds behind}, whereas they were in the opposite order for speaker 2, question 21. The question and context order are listed in Appendix \ref{appen:stimuli}. As in Experiment 1, participants were told that the contexts were not the actual contexts from the interview but that they were similar. The manner in which they were similar was not made explicit.

After playing stimuli for all four voices, the first half of the experiment was repeated. The questions were presented in the same order as during the first half, but the order in which the auditory tokens were played within each question was reversed in order to counterbalance potential effects from a response bias based on the tokens' order. The contexts were presented in the same order as found in the first half of the experiment.


\subsubsection{The stimuli}

Potential phonetic cues in the stimuli are shown in \tabref{tab:cues2} for each of the function types. Each token was played twice, once with each corresponding function type. For example, in the block for Tracy's voice, one discourse particle token (tracy-discp1) was compared with a grammatical token (tracy-like1) and a quotative token (tracy-quote1); and the quotative (tracy-quote1) and the grammatical (tracy-like1) were compared to each other. Only characteristics that were included in the production models are shown in the table. All tokens were played once in the first half of the experiment and then again in the second half. 

\begin{table}[ht]
\begin{center}
\resizebox{\textwidth}{!}{
  \begin{tabular}{ld{3}d{3}d{3}cd{3}d{3}d{3}}
  \lsptoprule
      
					  &	\multicolumn{3}{c}{CR}	& &	\multicolumn{3}{c}{NCR}	\\
  \cmidrule{2-4}\cmidrule{6-8}
  \multicolumn{1}{c}{type}				&	\multicolumn{1}{c}{quote}  &  \multicolumn{1}{c}{lex verb} & \multicolumn{1}{c}{dp}  &&	\multicolumn{1}{c}{quote}  &  \multicolumn{1}{c}{lex verb} & \multicolumn{1}{c}{dp} \\
					  \midrule
  monophthong &  4  &  0  &  0  & &    2	&  0  &  0  \\
  /k/ present	&   6  &  3  &  8 & &  6  &  4 &  6 \\
  %glottalised	&    4  &  3 &  3  \vline&  2   &  4 & 5 \\
  %mean pitch & 178.7 & look up & l.u. \vline& 206.2 & l.u. & l.u.\\
  ave. dur. ratio & 0.269	& 0.507 & 0.426 & & 0.286 &	0.505 &	0.354 \\
  ave. nuc. F2 & 1619.5 &	1628.5 &	1694.2 6& & 1598.8	& 1477.8	& 1583.5  \\
  number of tokens			&  10   &  10 & 10 & & 10 &  10   &  10 \\
    \lspbottomrule
    \end{tabular}
}
\caption{Potential phonetic cues in stimuli from Experiment 2, by type and social group}\label{tab:cues2}
\end{center}
\end{table} 

In Experiment 2, the 42 participants correctly identified the function of \textit{like} 54.1\% of the time. As there were only two possible answers in the task, participants' accuracy was roughly at chance. As with Experiment 1, a high rate of accuracy was not anticipated given the mix of phonetic cues included in the stimuli. Three mixed effects models were fit to the data in order to determine the extent to which perceivers relied on phonetic cues in the stimulus to identify the lemma. A number of factors were tested in the models and only those that reached significance were included as fixed effects.

\subsection{Results}

\subsubsection{Experiment 2, Model 1: The quotative and the discourse particle}

For questions that compared the quotative with the discourse particle, a model was fit that modelled the likelihood of identifying the first token as the quotative. It was based on 1650 responses from 42 participants. A number of factors were tested in the model, including the difference between the first and second tokens' Euclidean distance between the F1 and F2 of the vowel's nucleus and offglide. Also tested was whether the participant indicated that they recognised the voice. Only factors that reached significance were included in the model. As shown in \tabref{qdcoeff2}, the fixed effects that were included in the model were whether the quotative was listed first on the answer-sheet (quote first) and the difference in the /l/ to vowel duration ratio of the first and second auditory tokens (duration ratio diff.). The estimated scale parameter of the model is 0.994807. 

% [Greyed out to change this to the model where quote first is the default]
% latex table generated in R 2.6.1 by xtable 1.5-2 package
% Thu Aug 14 10:54:03 2008
%\begin{table}[ht]
%\begin{center}
%\begin{tabular}{lrrrr}
 %\lsptoprule
 %& Estimate & Std. Error & z value& Pr($>$$|$z$|$) \\
 % \midrule
%(Intercept) & $-$0.13415  &  0.09122 & $-$1.471 &  0.1414 \\
  %quote first & 0.29883 &   0.12915  & 2.314 &  0.0207\\
  %duration ratio diff. & $-$0.55792   & 0.27035  & $-$2.064  & 0.0390 \\
 %  \lspbottomrule
%\end{tabular}
%\caption{Experiment 2 coefficients of fixed effects from Model 1, comparing responses to the quotative and the %discourse particle.}
%\label{qdcoeff2}
%\end{center}
%\end{table}


% latex table generated in R 2.6.1 by xtable 1.5-2 package
% Wed Mar 04 11:42:56 2009
\begin{table}[t]
\begin{center}
\begin{tabular}{ld{6}d{6}d{4}d{5}}
  \lsptoprule
 & \multicolumn{1}{r}{Estimate} & \multicolumn{1}{r}{Std. Error} & \multicolumn{1}{r}{z value} & \multicolumn{1}{r}{Pr($>$$|$z$|$)} \\
  \midrule
(Intercept) & 0.16468 &   0.09152 &  1.799 &  0.0720 \\
  quote second & -0.29861  &  0.12929 & -2.310  & 0.0209 \\
  duration ratio diff. & -0.55828  &  0.27063  & -2.063 &  0.0391 \\
   \lspbottomrule
\end{tabular}
\caption{Experiment 2 coefficients of fixed effects from Model 1, comparing responses to the quotative and the discourse particle}
\label{qdcoeff2}
\end{center}
\end{table}

As for results from Experiment 1, participants were less likely to identify the first token as the quotative if the quotative context was listed second on the answer-sheet (p$<$0.05). Again, this reflects an overall bias toward matching the first auditory token with the first context on the sheet.

Also consistent with results from Experiment 1 was the effect of the difference in /l/ to vowel duration ratio between the first and second tokens. Participants were less likely to identify the first token as the quotative if the first token had a longer /l/ duration relative to its vowel than the second token (p$<$0.05). This is consistent with results from Experiment 1 and with results from production.


\subsubsection{Experiment 2, Model 2: Grammatical functions and the quotative}

A model was fit to the 1652 responses that compared grammatical functions with the quotative, modelling the likelihood that a token was the quotative. The estimated scale parameter for the model is 0.9881647. As shown in \tabref{qgcoeffExp2}, the only fixed effect that reached significance was the difference in F2 between the first and second tokens (F2 diff.). 


% latex table generated in R 2.6.1 by xtable 1.5-2 package
% Wed Aug 13 18:39:13 2008
\begin{table}[t]
\begin{center}
\begin{tabular}{ld{8}d{8}d{4}d{4}}
 \lsptoprule
 & \multicolumn{1}{r}{Estimate} & \multicolumn{1}{r}{Std. Error} & \multicolumn{1}{r}{z value} & \multicolumn{1}{r}{Pr($>$$|$z$|$)} \\
  \midrule
(Intercept) & 0.1724920 & 0.0954958  & 1.806   & 0.0709 \\
  F2 diff. & 0.0011511 & 0.0005074  & 2.269  & 0.0233\\
   \lspbottomrule
\end{tabular}
\caption{Experiment 2 coefficients of fixed effects from Model 2, comparing responses to the quotative and grammatical functions of \textit{like}}
\label{qgcoeffExp2}
\end{center}
\end{table}


F2 values were measured at the target of the nucleus in the stimulus tokens. Participants were more likely to identify the first token as the quotative if the first token had a greater F2 value (i.e.~a fronter vowel in the nucleus) than the second token (p$<$0.05). Again, the estimated coefficients are in log odds. To determine the robustness of the effect of F2 for a given question, the product of the difference in F2 and the factor's coefficient is added to the estimated coefficient of the intercept. This finding is consistent with production; speakers were more likely to produce a fronter vowel in the nucleus of quotative \textit{like} than in the nucleus of grammatical functions of \textit{like}.



\subsubsection{Experiment 2, Model 3: Grammatical functions and the discourse particle}

A model was also fit to the questions that compared grammatical functions with the discourse particle, modelling the likelihood of identifying the first token as grammatical \textit{like}.  There were a total of 1648 responses from 42 different girls. The estimated scale parameter of this model is 0.9883807. As shown in \tabref{dpgramcoeffExp2}, the only factor included as a fixed effect in the model was the difference in F2 between the first and second tokens.


% latex table generated in R 2.6.1 by xtable 1.5-2 package
% Wed Aug 13 17:56:11 2008
\begin{table}[ht]
\begin{center}
\begin{tabular}{ld{8}d{8}d{4}d{4}}
 \lsptoprule
 & \multicolumn{1}{r}{Estimate} & \multicolumn{1}{r}{Std. Error} & \multicolumn{1}{r}{z value} & \multicolumn{1}{r}{Pr($>$$|$z$|$)} \\
 \midrule
(Intercept) & -0.0416868 & 0.0756207 & -0.5513 & 0.5815 \\
 F2 diff. & -0.0006476 & 0.0003086 & -2.0983  & 0.0359 \\
   \lspbottomrule
\end{tabular}
\caption{Experiment 2 coefficients of fixed effects from Model 3, comparing responses to the discourse particle and grammatical functions of \textit{like}}
\label{dpgramcoeffExp2}
\end{center}
\end{table}

Participants were less likely to identify the first token as the grammatical function if the F2 of the first token's diphthong nucleus was greater than that of the second token (p$<$0.05). This is consistent with the production results; speakers were more likely to produce a fronter vowel (with a higher F2) when producing discourse particle \textit{like} than when producing a grammatical function. Though this factor did not reach significance in the model for Experiment 1, the relationship between response and the stimuli's F2 values in Experiment 1 is in the same direction as found in Experiment 2.

The difference in F2 can predict responses in three different models that compared grammatical functions with the discourse particle and the quotative. This is consistent with their behaviour in production. Crucially, the difference in F2 was not found to predict responses to questions that compared the discourse particle with the quotative, and this is also consistent with production.

The results from responses during Experiment 2 are very similar to those from Experiment 1. They provide supporting evidence that perceivers are sensitive to lemma-based phonetic variation. The third experiment investigated the relationship between phonetic, social, and lemma-based information in perception. 


\section{Experiment 3} 

Though there is growing evidence that perceivers attribute social information to a speaker based on phonetic cues in the stimuli \citep{gilespowesland1975,bayard2000,campbellkibler2007}, the extent to which social information can be accessed is less clear in cases when the target social groups are not explicitly discussed by participants. Experiment 3 was designed to test the degree to which perceivers would consistently identify the place where a given girl might eat based on phonetic cues in the stimuli.

\subsection{Methodology}

\subsubsection{The task}

In Experiment 3, participants were also asked to respond to isolated auditory tokens of \textit{like}. The tokens were either a quotative or a grammatical function.\footnote{At the time of designing the experiment, the production analysis had not yet been conducted. In hindsight, it would have been wise to include questions eliciting responses to the discourse particle.} The speech of ten girls was used and they were equally divided by lunch locale. The CR voices were: Tracy (The PCs), Betty (Sporty Girls), Rachel (Sporty Girls), Anita (Relaxed Group), and Rose (Relaxed Group). The NCR voices were: Vanessa (The Goths), Onya (Real Teenagers), Meredith (The Goths), Isabelle (Real Teenagers), and Sarah (Real Teenagers). 

The experiment was divided into three tasks, each with ten questions. Participants were told that the clips were from interviews conducted with Year 13 SGH students. For each question, they were asked to indicate whether they felt that the speaker was probably in a group that sometimes ate lunch in the common room (by circling ``Y'') or probably in a group that ate lunch outside the common room (by circling ``N''). An example question is shown in \figref{ExampleExp3}.  They were also asked if they recognised the voice and to identify the voice if possible. This information was collected in order to determine whether recognition had an effect on responses. The same voices were used in the three tasks and they were played in a different order in the different tasks. 

\begin{figure}

		\begin{tabular}{lrr}
\lsptoprule
`I like toast.'  & & \\
\midrule
\\
Does this person sometimes eat lunch in the Common Room?	 & Y	& N \\

Do you recognise this voice?		& Y	& N \\

If so, who do you think it is? & & \\

                   \\	
		\lspbottomrule
		\end{tabular}
\caption{Example question from Experiment 3}
\label{ExampleExp3}
\end{figure}


For the first task, all tokens were grammatical functions of \textit{like}. Participants were informed that the tokens they would hear were taken from sentences where \textit{like} had the lexical verb meaning, as in the sentence \textit{I like toast}.  

\newpage
For the second task, the tokens were quotative \textit{like}. The girls were informed that the tokens came from sentences similar to \textit{He was like, ``Yeah okay.''} 

For the third task, both the lexical verb and quotative tokens were played for each voice with the appropriate contexts provided. This was done in order to provide participants with a larger amount of lexically-conditioned phonetic information, as it was hypothesised that participant responses would be more accurate when more cues were provided. The token of quotative \textit{like} was played second for each question.

\subsubsection{The stimuli}

\tabref{tab:cues3} shows the number of tokens in Experiment 3 with phonetic cues that the participants may have used to identify a speaker's eating place. The lexical verb and quotative tokens were played once in the first and second task and these same tokens were repeated in task 3.


\begin{table}[ht]
\begin{center}
\begin{tabular}{lccccc}
 \lsptoprule
    
     	 						&	\multicolumn{2}{c}{CR} 		&			& \multicolumn{2}{c}{NCR}\\
\cmidrule{2-3}\cmidrule{5-6}
type						& quote & lexical verb 	& & quote & lexical verb   \\
\midrule
monophthong 		&  2 & 0 &\vline&  0 & 0 \\
/k/ present 		&  1 & 5 &\vline& 3 & 3  \\
%glottalised     &  2 & 2 \vline&  1 & 2 \\
number of tokens	& 5 & 5 &\vline&  5 & 5 \\


   \lspbottomrule
   
\end{tabular}
\caption{Potential phonetic cues in stimuli from Experiment 3, by type and social group}\label{tab:cues3}
\end{center}
\end{table}	


  

\subsection{Results}

Participants in Experiment 3 performed at chance level, correctly identifying the eating place of the girl who produced the stimulus 52.1\% of the time across all tasks. Participant responses were most accurate in the first task (with only grammatical functions) and least accurate during the second task (with only the quotative). However, the difference in accuracy between the tasks did not reach significance. 

In order to determine whether perceivers used lemma-based phonetic cues when identifying a speaker as someone who ate lunch in the CR or not, a binomial mixed effects model with both question number and participant as random effects was fit to responses from Experiment 3, modelling the likelihood of identifying the speaker as someone who ate lunch in the CR. Of the 1260 possible responses, 36 questions were not responded to and were not included in the analysis.

A number of factors were tested in the model, including whether the stimulus had a vowel that was monophthongal or had a /k/ that was realised. Also tested was the task that the question was in, and whether the participant and stimulus voice were CR girls. Two factors reached significance in the model: (1) whether the participant believed they recognised the voice and (2) whether the question contained a token of quotative \textit{like} with a monophthongal vowel. The coefficients for the model are shown in \tabref{coeff3}. 



\begin{table}[ht]
\begin{center}
\begin{tabular}{ld{6}d{6}d{4}d{4}}
 \lsptoprule
 & \multicolumn{1}{r}{Estimate} & \multicolumn{1}{r}{Std. Error} & \multicolumn{1}{r}{z value} & \multicolumn{1}{r}{Pr($>$$|$z$|$)} \\
  \midrule
(Intercept) & 0.46790  &  0.18834  & 2.484  & 0.0130 \\
  recognise = y &  0.97158  &  0.19703 &  4.931 & <0.0001 \\
  no quotative token &  0.04105  &  0.23599  & 0.174  & 0.8619 \\
  quote = monophong &  0.72789  &  0.33996 &  2.141 &  0.0323 \\
   \lspbottomrule
\end{tabular}
\caption{Experiment 3 coefficients of fixed effects}\label{coeff3}
\end{center}
\end{table}

Whether the participant believed they recognised the voice significantly predicted responses, even if they incorrectly identified the speaker. Participants who believed they recognised the voice were more likely to indicate that the speaker was someone who ate lunch in the CR (p$<$0.0001). Participants identified a speaker correctly only 58.7\% of the time that they believed they recognised the voice. While this is well above chance, it provides evidence that recognition of a voice does not equate with accurately knowing who the speaker was. In fact, when participants misnamed a speaker, they named someone from the same group as the actual speaker only 18.0\% of the time. This suggests that if a voice merely sounded familiar, the speaker was identified as someone who ate lunch in the CR. The tendency to believe that the recognised voices were CR girls is not entirely surprising, as CR girls were more involved in school activities. They were talkative in class, they played sport, and they had leadership roles at the school. With few exceptions, NCR groups interacted with each other rarely and many were actually more likely to interact with CR girls. Therefore, a wider variety of students had exposure to the speech of CR girls. CR girls had less exposure to NCR girls (and their speech) than to other CR girls, and NCR girls had more exposure to CR girls (and their speech) than to girls from other NCR groups. 

Also included in the model is whether the question contained a token of quotative \textit{like} that had a monophthongal vowel. Participants were significantly more likely to identify the voice as someone who ate lunch in the CR if the question contained a token of quotative \textit{like} with a monophthongal vowel than if it contained a token of quotative \textit{like} with a diphthongal vowel (p$<$0.05). In production, NCR girls were more likely to produce variants of quotative \textit{like} with a more diphthongal vowel (Wilcoxon, p$<$0.01).\footnote{This did not reach significance in the production model because it did not interact with function type; NCR girls were more likely to produce more diphthongal tokens, regardless of function type.} The fact that participants were significantly more likely to identify tokens with monophthongal vowels as having been produced by a CR girl suggests that they used their knowledge of sociophonetic trends in production to identify the eating place of the speaker; monophthongal vowels were more likely to be observed in the speech of CR girls and, in perception, tokens with this phonetic characteristic were more likely to be identified as having been produced by a CR girl. 

\section{Discussion}

In Experiments 1 and 2, participants were sensitive to the /l/ to vowel duration ratio in the stimuli when distinguishing between quotative \textit{like} and the discourse particle. This duration trend in perception was consistent with the duration trend in production. Likewise, when responding to questions that included a grammatical token of \textit{like}, participants in Experiment 2 responded to how fronted the diphthong's nucleus was in a way that was consistent with production. Taken together, these results suggest that perceivers can use their knowledge of lemma-conditioned phonetic variation from production to identify lemmas in perception.\footnote{If indeed the production trend regarding /l/ to vowel duration ratio is a result of prosodic position, individuals' sensitivity to the /l/ to vowel duration ratio in perception may reflect an ability to use this phonetic cue to extrapolate the likelihood of prosodic trends, which are then used during the experiment to determine the likelihood of the token being a particular function of \textit{like}. \citet{cutlerclifton1984} found little evidence to suggest that speakers use lexical stress to identify grammatical categories. This apparent conflict between their results and the results presented in this book may be due to differences between the methodologies, different types of stress, or other factors not discussed here.}

\largerpage
In Experiment 1, participants also relied on syntactic information to identify whether or not a given token was quotative \textit{like}. Their responses were consistent with lemma-conditioned syntactic variation in production. This suggests that participants were not only sensitive to lemma-conditioned phonetic variation but also lemma-conditioned contextual variation.

Participants in Experiment 3 were more likely to identify voices as NCR girls if the participant did not believe they recognised the voice and if the stimulus contained a token of quotative \textit{like} that had a monophthongal vowel. Here again, perception is consistent with production: CR girls were more likely to produce monophthongal vowels in the different functions of \textit{like}. This provides some evidence that perceivers were sensitive to sociophonetic trends from production when identifying the eating place of each girl.


\subsection{Lack of social effects in function identification tasks}

In the pro\-duction model, there is a signif\-icant inter\-action be\-tween whe\-ther the speaker was a CR girl and whether the /k/ was realised. In Chapter \ref{ch:prod}, I argued that the observed interaction was a result of the girls' identification with, and avoidance of, norms established by the CR girls. But why was there no evidence of perceivers' sensitivity to this socially-conditioned variation?

In the clips of speech that were used as stimuli, the recording ended directly after the token of \textit{like}. In Experiment 3, all of the tokens of \textit{like} where the /k/ was realised were also released (and were, therefore, easily identifiable as having the /k/ present). However, the participants had no way of knowing that if a /k/ was present, it would be released. Tokens with a velar closure but without a release are difficult to identify as having the /k/ present unless they are followed by another segment, particularly a vocalic segment. Without the following environment included in the stimuli, participants may not have been able to distinguish between tokens where the /k/ was realised and tokens where it was dropped, leading them to rely on phonetic information other than the presence or absence of /k/.\footnote{This may also be responsible for the lack of a significant correlation between a participant's production and their perception. Realisation of /k/ was the phonetic variable that varied most across different speakers; there was little variation for the other phonetic factors across the different discursive pragmatic functions. For example, with little deviation from the widespread trend of producing a monophthong in quotative \textit{like} and a diphthong in other functions, it was statistically unlikely to observe an effect of production on perception in regards to the diphthong.}

Another possibility is that observing sociophonetic effects in perception relies on the degree to which the perceivers are aware of the linguistic variable and possibly also its tendency to pattern with certain social characteristics. Results from \citet{haynolandrager2006} provide some evidence that sociophonetic trends in perception are stronger for variables for which the variation is above the level of consciousness in the community; vowels with realisations that are more stigmatised and commented on are affected more than other vowels and the effect is strongest for lexical items that are strongly associated with these highly salient realisations. At SGH, the girls were not aware of the variation in /k/ realisation across the different functions of \textit{like}; it was not commented on and they expressed surprise when I described some preliminary results regarding the differences in /k/ realisation across the different functions for the different groups. Awareness of a sociolinguistic variable does not appear to be necessary in order for that variable to covary with social group, stance, and style during speech production. In contrast, some level of awareness may be necessary to observe sensitivity to such trends in speech perception or else a larger number of tokens per experiment and a larger number of subjects may be required to observe the more subtle trends that we might expect when examining the perception of patterns that are below the level of consciousness.

A third possibility is that the lack of an effect is due to experiment design. I intentially chose tokens that were not representative of the two groups, thinking that it would help tease apart whether listeners were using the variables I identified during analysis or some other cues in the signal. However, the stimuli are not balanced for the different phonetic variants and the variants present in the stimuli may be ``at odds'' with other cues in the signal, making it less likely that listeners would/could use social information when completing the task.


\subsection{Theoretical implications}

These results provide evidence that perceivers are sensitive to the relationship between phonetic, social, and lemma-based information during perception. In production, phonetic variation depends on the social group of the individual and the function of the token. In perception, individuals are sensitive to the relationship between phonetic and lemma-based information. They also extract community-specific social information about the speaker, depending on whether they recognise the speaker's voice and whether a token has a monophthongal vowel. This suggests that social, phonetic, and lemma-based (syntactically/semantically-de\-fined) information is stored in, or indexed to, the lexicon and can be accessed during the perception of speech.

The results also provide evidence that individuals store and use information about the surrounding context. Quotative \textit{like} is most frequently found in the first person and in the historical present, and individuals appear to have used this information to identify the lemma. This suggests that not only are probabilities of contextual information beyond the word level stored in the mind but moreover that they are used during speech processing. 

%This provides evidence that varying amounts of exposure to the speech of an individual or a group will affect speech production and perception, presumably through the storage of their speech and the indexing to appropriate social information. These results are consistent with Clopper and Pisoni's (2004) finding that perceivers are more accurate at identifying the regional origin of a speaker if they have had a greater amount of exposure to speakers from that region.\nocite{clopperpisoni2004}


These findings are consistent with an exemplar-based model of speech perception and production in which utterances are stored in the mind complete with fine-grained phonetic detail and indexed with other social and contextual information observed at the time of the utterance \citep{johnson1997,pierrehumbert2001,pierrehumbert2002}. The results presented here indicate that such information must include the grammatical function of a token. Possibilities for how this information may be stored will be discussed in the following chapter.

%Whether this information is in the form of an independent label indexed to the lexical exemplar, is embedded in the lexical exemplar, or is derived on-line from indexing to higher-level information is beyond the scope of this thesis; the results are consistent with any of these representations. A discussion of the implications of these results can be found in the following chapter.




\newpage
\thispagestyle{empty}
\mbox{}