\chapter{Introduction} \label{ch:1}
\section{Aims and scope} \label{sec:1aims}
The core problem to be dealt with in this book is the syntax of functional left peripheries in West Germanic. In particular, I will concentrate on how sentence types are marked at the leftmost edge of the clause and how the presence of multiple visible markers can be accounted for. Regarding syntactic structure, I adopt a minimalist framework (as proposed by \citealt{chomsky2001, chomsky2004, chomsky2008}, among others), according to which syntactic structures are derived by merge (external or internal). Further, in line with the principles of mainstream generative grammar, I assume that the derivation of structures is constrained by economy, and hence the number of projections, as well as of syntactic processes, is as minimal as possible.

The study of various issues associated with the left periphery of the clause has always been central in generative grammar and it continues to be one of the most well-researched areas of syntax. Among other functions, left peripheries are associated with defining the type of the clause, and they are also responsible for establishing connections between clauses that make them into complex sentences. Apart from purely syntactic concerns, left peripheries raise a number of questions that make this domain extremely relevant for the interfaces of syntax, referred to as PF (Perceptible Form or, more traditionally, Phonological Form\footnote{Since generative theory was initially limited to the study of oral languages, the term ``Phonological Form'' was established, and many properties of this interface reflect the properties of oral languages, even though sign languages also evidently have an interface connected to their perceptible form. In this sense, as proposed by \citet{sigurdsson2004}, the term ``Perceptible Form'' is more appropriate as it does not treat sign languages as secondary. See also \citet{vanderhulst2015} for the distinction of the two. In this book, I will restrict myself to examining selected oral languages, mostly from Germanic.}) and LF (Logical Form, indicating the semantic component) in standard generative grammar. The interaction with the interfaces becomes evident when considering issues related to the left periphery beyond clause typing proper: certain phrases appear to be located in the left periphery due to their specific information structural status. Apart from that, clausal ellipsis is also related to various functional heads (see \citealt{merchant2001}).

It is most probably this diversity of problems that led to a significant interest in the left periphery of the clause in generative grammar already in the 1970s, most notably in \citet{chomskylasnik1977}, followed by the well-known cartographic enterprise from the 1990s onwards, especially by \citet{rizzi1997, rizzi2004} and various analyses with more or less shared concerns: for example, \citet{sobin2002}, \citet{poletto2006}, \citet{bayerbrandner2008}, \citet{brandnerbraeuning2013}. I will both rely on these previous findings and critically evaluate them. In addition, while many questions have indeed been answered by previous accounts, there are several others that have remained unresolved and have not received an adequate explanation which would hold both cross-linguistically and specifically for West Germanic as well. In addition, I assume that any proposal should follow from general principles of the grammar rather than by applying construction-specific mechanisms. In other words, the specific configuration of the left periphery of one construction should be comparable to the left periphery of other clause types within a single model by applying predictable properties of the grammar. The aim of this work is to provide such an analysis and to enable a better understanding of functional left peripheries.

In the following, I will briefly provide an overview of the most important issues concerning functional left peripheries and clause typing in West Germanic, and then I will provide a concise outline of the problems to be dealt with in this book.

\section{Functional left peripheries} \label{sec:1functional}
Clauses can fulfil various functions in discourse; in canonical cases, the form of the clause is indicative of its discourse function. Consider the following examples:

\ea \label{clauses}
\ea Ralph is interested in poetry. \label{declarative}
\ex Is Ralph interested in poetry? \label{interrogative}
\z
\z

In (\ref{declarative}), we have a statement and the type of the clause is declarative. By contrast, (\ref{interrogative}) is a question and the type of the clause is interrogative. In the first case, a proposition (\textit{p}) is true; in the second case, the truth of the proposition is asked (\textit{p} or $\neg$\textit{p}). The two utterances differ in their form. The declarative sentence represents the neutral, unmarked word order in English, which is SVO: crucially, the subject (\textit{Ralph}) precedes the aspectual auxiliary (\textit{is}). In the interrogative clause, these two elements have exactly the opposite order: the aspectual auxiliary has been moved to the front of the clause.

In many cases, the form of an utterance is not indicative of its discourse function in a straightforward way. Consider the example in (\ref{window}):

\ea Could you open the window? \label{window}
\z

In this case, the speaker does not ask the addressee about the truth of the proposition but expresses a request: a simple \textit{yes} answer, which is satisfactory in (\ref{interrogative}), would not be pragmatically appropriate in (\ref{window}) if it is not accompanied by the speaker also opening the window. The pragmatic function of sentences is thus not in a one-to-one correspondence with the observed grammatical form; these issues are examined extensively in speech act theory, going back to the work of \citet{austin1962}. As the present book is concerned with the formal properties, especially the syntax of functional left peripheries and clause typing, these issues will not be addressed here.  

The two clauses in (\ref{clauses}) differ not only in their word order but also regarding their intonation: declarative clauses have falling intonation, while interrogative clauses have rising intonation. However, there are discrepancies in this respect as well; consider:

\ea Ralph is interested in poetry? \label{declquest}
\z

The example in (\ref{declquest}) is a declarative question: formally the clause is declarative but it has a rising (interrogative) intonation; regarding its function, it constitutes a special type of question which does not ask about the truth of a proposition but rather asks for confirmation or expresses surprise. Again, these cases will not be discussed in the present thesis as they are not immediately relevant to the specific syntactic issues to be examined.

The clauses in (\ref{clauses}) are main clauses. Clause types are identified in slightly different ways in embedded clauses such as (\ref{embedded}):

\ea \label{embedded}
\ea I think [\textbf{that} Ralph is interested in poetry]. \label{that}
\ex I wonder [\textbf{if} Ralph is interested in poetry]. \label{if}
\ex It is important [\textbf{for} Ralph to study Byron]. \label{for}
\z
\z

The highlighted complementisers determine the type of the embedded clause: (\ref{that}) and (\ref{for}) are declarative, while (\ref{if}) is interrogative. Apart from clause type, complementisers can also determine whether the clause is finite, as in (\ref{that}) and (\ref{if}), or non-finite, as in (\ref{for}). Finiteness, as determined by the C head, has an effect on whether the clause contains a tensed element (e.g. \textit{is} in (\ref{that}) and (\ref{if}) above) above or not (in which case, as in (\ref{for}), English uses the element \textit{to} and the infinitival form of the verb). The incompatibility of finite complementisers with a non-finite clause, and vice versa, is illustrated in (\ref{finnonfin}) below:

\ea \label{finnonfin}
\ea[*]{I think [for Ralph is interested in poetry].}
\ex[*]{It is important [that Ralph to study Byron].}
\z
\z

Likewise, the type of a complement clause must also be compatible with the lexical properties of the matrix verb: verbs like \textit{think} select for declarative complements, while verbs like \textit{wonder} select for interrogative complements. If these sectional restrictions are violated, the result is ungrammatical:

\ea
\ea[*]{I think [if Ralph is interested in poetry].}
\ex[*]{I wonder [that Ralph is interested in poetry].} 
\z
\z

In other words, it is evident that the left periphery of the clause has a dual function. On the one hand, it connects the clause to the matrix clause (in the case of embedded clauses) or to the discourse (in the case of root clauses). On the other hand, it has an impact on the internal properties of the clause itself.

Besides complementisers, the CP is known to host other elements as well, such as \textit{wh}-phrases in interrogative clauses:

\ea \label{wh}
\ea I wonder [\textbf{who} Mary will invite]. \label{who}
\ex I asked Louisa [\textbf{which city} she was travelling to]. \label{whichcity}
\z
\z

In (\ref{who}), the \textit{wh}-element consists of a single operator (\textit{who}), while in (\ref{whichcity}) the \textit{wh}-phrase is visibly phrase-sized, containing not only the operator \textit{which} but also a lexical element, the NP \textit{city}. This indicates that \textit{wh}-phrases can occupy only a phrase position, namely [Spec,CP], and not C. Further, since they also fulfil a role in the TP, that is, they are arguments, it is assumed in generative grammar that they undergo movement from a clause-internal position to the CP-domain. This is illustrated in (\ref{wonderasked}) below:

\ea \label{wonderasked}
\ea I wonder [\textbf{who} Mary will invite \sout{\textbf{who}}].
\ex I asked Louisa [\textbf{which city} she was travelling to \sout{\textbf{which city}}].
\z
\z

In line with current minimalist theory, I assume that movement involves the copying of the moved constituent: by default, the higher copy is realised phonologically at the PF interface, while PF eliminates lower copies of a movement chain. In English, \textit{wh}-elements move to the left periphery in interrogatives, leaving the higher copy in the CP overt. Operators moving to the left periphery thus differ from complementisers not only with respect to their relative position in the CP but also in that they land there via movement, while complementisers are base-generated in the left periphery.

Relative clauses also contain operator movement:

\ea
\ea This is the linguist [\textbf{who} Mary will invite].
\ex The candidate [\textbf{who} we voted for] has already left the city.
\z
\z

Relative clauses differ from interrogative clauses in that they modify a nominal head, referred to as the head noun, while embedded interrogatives are complements of a matrix predicate (and interrogative clauses can also be root clauses). Again, relative operators undergo leftward movement:

\ea
\ea This is the linguist [\textbf{who} Mary will invite \sout{\textbf{who}}].
\ex The candidate [\textbf{who} we voted for \sout{\textbf{who}}] has already left the city.
\z
\z

Such operators (both in interrogative and relative clauses, and beyond) move to the left periphery because they have a function regarding clause typing: cases like (\ref{wh}) are identifiable as interrogative clauses precisely because there are overt interrogative elements in the left periphery, there being no distinctive interrogative intonation or word order changes (such as subject--auxiliary inversion) in embedded clauses.

\section{The problems to be discussed} \label{sec:1problems}
\subsection{The model} \label{sec:1model}
In current minimalist theory, the Complementiser Phrase (CP) is responsible for typing clauses and for encoding finiteness in finite clauses. Apart from complementisers, as pointed out in \sectref{sec:1functional} above, various operators can appear in this domain. Consider:

\ea
\ea I wonder \textbf{if} Ralph has arrived. \label{englishifch1}
\ex I wonder \textbf{whether} Ralph has arrived. \label{englishwhetherch1}
\z
\z

In (\ref{englishifch1}), \textit{if} is a complementiser and it types the subordinate clause as interrogative. In (\ref{englishwhetherch1}), there is no overt complementiser but the operator \textit{whether} is present. In such cases, it is assumed that a zero complementiser types the clause (since the CP can be projected only by a C head, which in this case is not visible, though; see \citealt[137--138]{bacskaiatkari2020jcgl} for discussion), yet a sound model of the CP-periphery must also clarify the role of the overt operator  in (\ref{englishwhetherch1}).

On the other hand, the CP is not restricted to hosting a single overt element: depending on the particular construction and the dialect, multiple elements may appear in the CP-domain. This is illustrated by (\ref{englishdfcch1}) for non-standard English and by (\ref{norwegiandfcch1}) for Norwegian (\citealt[175]{bacskaiatkaribaudisch2018}):

\ea \label{dfcch1}
\ea[\%]{I wonder \textbf{which book that} Ralph is reading. \label{englishdfcch1}}
\ex[]{\gll Peter spurte \textbf{hvem} \textbf{som} likte bøker. \label{norwegiandfcch1}\\
           Peter asked.\textsc{3sg} who that liked books\\
\glt `Peter asked who liked books.'}
\z
\z

A proper formal account of the CP-domain must be able to condition when multiple overt elements are allowed and when not. Further, it must be clarified whether the appearance of several overt elements requires multiple CP projections, and in cases where it does, how word order restrictions can be modelled. The generation of multiple functional layers is in principle possible, yet it should be appropriately restricted to exclude the generation of superfluous layers that are empirically not motivated. This question is likewise relevant in cases involving a single overt C-element, since then the question arises whether and to what extent covert elements and phonologically not visible projections are present. 

Apart from the exact position of various elements in the CP, their function(s) must also be addressed. For instance, interrogative complementisers regularly mark finiteness as well. Consider:

\ea \label{ifwhetherch1}
\ea[]{I don't know \textbf{if} I should call Ralph. \label{iffinitech1}}
\ex[]{I don't know \textbf{whether} I should call Ralph. \label{whetherfinitech1}}
\ex[*]{I don't know \textbf{if} to call Ralph. \label{ifnonfinitech1}}
\ex[]{I don't know \textbf{whether} to call Ralph.  \label{whethernonfinitech1}}
\z
\z

In (\ref{iffinitech1}), the complementiser \textit{if} introduces a finite embedded interrogative clause, and as the ungrammaticality of (\ref{ifnonfinitech1}) shows, it is incompatible with a non-finite clause, suggesting that it encodes finiteness apart from the interrogative property, too. By contrast, the operator \textit{whether} is compatible with both a finite clause, see (\ref{whetherfinitech1}), and with a non-finite clause, see (\ref{whethernonfinitech1}), indicating that the overt marking of the interrogative property is not incompatible with a non-finite clause in English. Since \textit{whether} is not specified for finiteness, it should be clear that finiteness is specified by some other element in (\ref{whetherfinitech1}); the question is whether there is a separate element encoding finiteness in (\ref{iffinitech1}) as well and, if so, how the restriction of \textit{if} to finite clauses can be explained.

Finally, the function(s) of various left-peripheral elements must be clarified also because there are some non-trivial combinations in which elements seem to be largely similar, as in the non-standard German example in (\ref{alswiech1}) below:

\ea
\ea[\%]{\gll Ralf ist größer \textbf{als} \textbf{wie} Maria. \label{alswiech1}\\
Ralph is taller than as Mary\\
\glt `Ralph is taller than Mary.'}
\ex[]{\gll Ralf ist größer \textbf{als} Maria. \label{alsch1}\\
Ralph is taller than Mary\\
\glt `Ralph is taller than Mary.'}
\ex[\%]{\gll Ralf ist größer \textbf{wie} Maria. \label{wiech1}\\
Ralph is taller as Mary\\
\glt `Ralph is taller than Mary.'}
\ex[]{\gll Ralf ist so groß \textbf{wie} Maria. \label{wieequatch1}\\
Ralph is so tall as Mary\\
\glt `Ralph is as tall as Mary.'}
\z
\z

In (\ref{alswiech1}), the elements \textit{als} and \textit{wie} both seem to mark the comparative nature of the clause, whereby single \textit{als} is the comparative particle in Standard German comparatives, as shown in (\ref{alsch1}), and single \textit{wie} is the comparative particle in equatives, see (\ref{wiech1}), and in certain dialects also in comparatives, see (\ref{wieequatch1}). In such cases, the question is to what extent there is genuine doubling at hand and how it can be modelled.

A central issue for the theory regarding the above-mentioned constructions is how the various properties associated with clause typing are encoded in the syntax. The occurrence of multiple overt elements in the left periphery indicates some complexity and raises the question whether a single CP projection is sufficient or whether multiple projections are necessary. In this respect, cartographic approaches (starting from \citealt{rizzi1997}) have a relatively clear answer, inasmuch as they assume a designated projection (generated in narrow syntax) for each feature, which necessarily leads to multiple projections in the above cases. In turn, this kind of approach is prone to reducing analysis to description, as the observed surface patterns are restated as syntactic projections; the question in this regard is whether such models are tenable or at least favourable to more minimalist approaches. These questions will be addressed in \chapref{ch:2}.

\subsection{Embedded interrogative clauses} \label{sec:1interrogative}
In Standard English, Standard German and Standard Dutch, there is no overt complementiser with an overt interrogative operator. This is illustrated in (\ref{whothatch1}) for English embedded interrogatives:

\ea	I don't know \textbf{who (*that)} has arrived. \label{whothatch1}
\z

As can be seen, the complementiser \textit{that} is not permitted in Standard English in embedded constituent questions. This phenomenon is traditionally termed as the ``Doubly Filled COMP Filter'' (going back to the work of \citealt{chomskylasnik1977}). By contrast, there are languages and also many West Germanic varieties that allow such patterns, as in (\ref{dfcch1}) above. Further examples are given in (\ref{dfcintch1}) below from non-standard English (\citealt[331, ex. 1]{baltin2010}) and from non-standard Dutch (\citealt[32]{bacskaiatkaribaudisch2018}):

\ea \label{dfcintch1}
\ea[\%]{They discussed a certain model, but they didn't know \textbf{which model that} they discussed.}
\ex[\%]{\gll Peter vroeg \textbf{wie} \textbf{dat} er boeken leuk vindt. \label{dutchdfcch1}\\
Peter asked.\textsc{3sg} who that of.them books likeable finds\\
\glt `Peter asked who liked books.'}
\z
\z

Such patterns are often referred to as doubling patterns, indicating that there are two overt elements in a single CP: the \textit{wh}-phrase in the specifier and the complementiser in C. Note that this is not exceptional: the specifier of the CP and the C head can be both lexicalised overtly in main clauses, as in T-to-C movement in English interrogatives, and in V2 clauses in German and Dutch main clauses. Consider the examples for main clause interrogatives in Standard English:

\ea \label{ttocch1}
\ea	\textbf{Who saw} Ralph? \label{whosawch1}
\ex	\textbf{Who did} Ralph see? \label{whodidch1}
\z
\z

In this case, doubling in the CP involves a \textit{wh}-operator in [Spec,CP] and a verb in C. T-to-C movement is visible by way of \textit{do}-insertion in (\ref{whodidch1}), though not in (\ref{whosawch1}): in principle, one might analyse (\ref{whosawch1}) as not involving the movement of the verb to C, but the CP is clearly doubly filled in (\ref{whodidch1}).

Similarly, in German (and Dutch) V2 declarative clauses a verb moves to C, while another constituent moves to [Spec,CP] due to an [edge] feature (see \citealt{thiersch1978diss}, \citealt{fanselow2002, fanselow2004isis, fanselow2004}, \citealt{frey2005}, \citealt{denbesten1989}). Consider:

\ea \label{v2ch1}
\ea \gll \textbf{Ralf} \textbf{hat} morgen Geburtstag.\\
Ralph has tomorrow birthday\\
\glt `Ralph has his birthday tomorrow.'
\ex \gll \textbf{Morgen} \textbf{hat} Ralf Geburtstag.\\
tomorrow has Ralph birthday\\
\glt `Ralph has his birthday tomorrow.'
\z
\z

As can be seen, the fronted finite verb is preceded by a single constituent in each case, and since the first constituent is not a clause-typing operator in either case, it is evident that doubling in the CP in V2 clauses is independent of the interrogative property.

It is therefore clear that the ``Doubly Filled COMP Filter'' should be more restricted in its application domain. In principle, one could say that an operator and a complementiser with largely overlapping functions are not permitted to co-occur in standard West Germanic languages, or that the Doubly Filled COMP Filter should be seen as some kind of an economy principle. Still, the problem remains that the notion of the Doubly Filled COMP Filter implies that the C head and [Spec,CP] position would be filled without the Filter, and the Filter is responsible for ``deleting'' the content of C. 

Regarding this, at least two major questions arise. First, it should be clarified what requirement is responsible for filling C even in the presence of an overt operator in [Spec,CP], as in (\ref{dfcintch1}). Second, the question is what kinds of elements may appear in C: in particular, if elements other than complementisers can satisfy the requirement of filling C, then the deletion approach is probably mistaken.

In addition, there is a theoretical problem with the notion of the Filter, which arises from a merge-based, minimalist perspective, while it is less problematic in X-bar theoretic terms. X-bar theoretic notions can at best taken to be descriptive designators that are derived from more elementary principles, in the vein of \citet{kayne1994} and \citet{chomsky1995}.\footnote{Note that I will also use X-bar structures for representational purposes in this book.} Under this view, the position of an element (specifier, head, complement) is a result of its relative position when it is merged with another element, and which element is chosen to be the label. By contrast, the notion of the Doubly Filled COMP Filter, as applied to a CP (as in \citealt{baltin2010}), implies that a phrase is generated with designated, pre-given head and specifier positions, and that there are additional rules on whether and to what extent they can be actually ``filled'' by overt elements. In a merge-based account, there are no literally empty positions, as no positions are created independent of merge: zero heads and specifiers reflect elements that are either lexically zero or have been eliminated by some deletion process (e.g. as lower copies of a movement chain or via ellipsis). In other words, Doubly Filled COMP effects should be accounted for in a way other than referring to a pre-given XP. These questions will be addressed in \chapref{ch:3}.

\subsection{Relative clauses} \label{sec:1relative}
West Germanic languages show considerable variation in terms of elements introducing relative clauses. There are two major strategies: the relative pronoun strategy and the relative complementiser strategy. In present-day Standard English, both of these strategies are attested. Relative pronouns are illustrated in (\ref{whrelativebasic}) below:

\ea \label{whrelativebasic}
\ea I saw the woman \textbf{who} lives next door in the park. \label{whosubjectch1}
\ex The woman \textbf{who/whom} I saw in the park lives next door. \label{whoobjectch1}
\ex I saw the cat \textbf{which} lives next door in the park. \label{whichsubjectch1}
\ex The cat \textbf{which} I saw in the park lives next door. \label{whichobjectch1}
\z
\z

As can be seen, relative pronouns show partial case distinction and distinction with respect to whether the referent is human or non-human. In particular, \textit{who}/\textit{whom} is used with human antecedents, as with \textit{the woman} in (\ref{whosubjectch1}) and (\ref{whoobjectch1}); the form \textit{who} can appear both as nominative and as accusative, while the form \textit{whom} used for the accusative is restricted in its actual appearance (formal/marked). With non-human antecedents, such as \textit{the cat} in (\ref{whichsubjectch1}) and (\ref{whichobjectch1}), the pronoun \textit{which} is used, which shows no case distinction. Note that apart from human referents, \textit{who(m)} is possible with certain animals: these are the ``sanctioned borderline cases'' (see \citealt[41]{herrmann2005}, quoting \citealt{quirkgreenbaumleechsvartvik1985}). On the other hand, non-standard dialects allow \textit{which} with human referents, as illustrated in (\ref{boywhichch1}) below (\citealt[42, ex. 4a]{herrmann2005}):

\ea {[}\ldots] And the boy \textbf{which} I was at school with [\ldots] \label{boywhichch1}\\
(\textit{Freiburg English Dialect Corpus} Wes\_019)
\z

At any rate, English relative pronouns are formed on the \textit{wh}-base and no longer on the demonstrative base: note that this is historically not so, and the present-day complementiser \textit{that} was reanalysed from a pronoun, while the \textit{wh}-based relative operators appeared only in Middle English (\citealt{vangelderen2009}).

Accordingly, the complementiser \textit{that} constitutes the second major strategy:

\ea
\ea I saw the woman \textbf{that} lives next door in the park.
\ex The woman \textbf{that} I saw in the park lives next door.
\ex I saw the cat \textbf{that} lives next door in the park.
\ex The cat \textbf{that} I saw in the park lives next door.
\z
\z

The complementiser \textit{that} is not sensitive to case and to the human/non-human distinction, which follows from its status as a C head. 

Given the availability of two strategies, a number of questions arise regarding their distribution. First, while it seems logical that the two strategies can be combined, doubling, as mentioned above, is less likely to appear in relative clauses than in embedded interrogatives, which raises the question what restrictions apply here. Second, as also mentioned above, there seems to be a preference for the complementiser strategy in West Germanic varieties that have a choice in the first place: it should be investigated why this should be so and why relative operators still exist even in dialects that have the complementiser strategy. Third, apart from their syntagmatic distribution (combinability), the paradigmatic distribution of the two strategies must likewise be examined, that is, whether the individual strategies can relativise all functions and how potential differences correlate with the featural properties of the respective items. These questions will be addressed in \chapref{ch:4}.

\subsection{Embedded degree clauses} \label{sec:1degree}
Embedded degree clauses fall into two major types: degree equatives, also called comparatives expressing equality, as given in (\ref{astallch1}), and comparatives expressing inequality, as given in (\ref{tallerthanch1}):

\ea \label{comparisonch1}
\ea Ralph is as tall \textbf{as} Mary is.\label{astallch1}
\ex Ralph is taller \textbf{than} Mary is.\label{tallerthanch1}
\z
\z

In (\ref{astallch1}), the subclause introduced by \textit{as} expresses that the degree to which Mary is tall is the same as to which Ralph is tall, while in (\ref{tallerthanch1}) the subclause introduced by \textit{than} expresses that the degree to which Mary is tall is lower than the degree to which Ralph is tall.

The comparison constructions presented in (\ref{comparisonch1}) above are instances of degree comparison: there is one degree expressed in the matrix clause and another one expressed in the subclause. The matrix degree morpheme is \textit{as} in degree equatives and it selects an \textit{as}-clause, while the matrix degree morpheme in degree comparatives is -\textit{er} (or \textit{more}, which is actually a composite of -\textit{er} and \textit{much}, see \citealt{bresnan1973}, \citealt{bacskaiatkari2014diss, bacskaiatkari2018langsci}). However, it is possible to have comparison without degree; consider:

\ea \label{nondegreecomparisonch1}
\ea[]{Mary is tall, \textbf{as} is her mother. \label{tallasch1}}
\ex[]{Mary is glamorous \textbf{like} a film-star. \label{glamorouslikech1}}
\ex[]{Farmers have other concerns \textbf{than} the farm bill. \label{otherthanch1}}
\ex[\%]{Life in Italy is different \textbf{than} I expected. \label{differentthanch1}}
\z
\z

In these cases, there is obviously no matrix degree element. The sentences in (\ref{tallasch1}) and (\ref{glamorouslikech1}) express merely similarity with respect to the property denoted by the adjective; in (\ref{glamorouslikech1}), the subclause is introduced by \textit{like} and not by \textit{as}, a further difference from degree equatives. Given the availability of non-degree equatives, \citet[35]{jaeger2018} suggests that comparison constructions can be grouped into three major categories: non-degree equatives, degree equatives, and comparatives; these constitute a markedness hierarchy in this order (non-degree equatives being the least marked). However, constructions like (\ref{otherthanch1}) and (\ref{differentthanch1}) indicate that there is in fact a fourth category as well: these are non-degree comparatives expressing difference. This category seems not to be productive as the availability of the \textit{than}-clause is dependent on the presence of a particular element expressing difference in the matrix clause: the word \textit{other} or, at least in American English, the adjective \textit{different} are potential candidates.

While the patterns in (\ref{comparisonch1}) suggest a relatively simple left periphery consisting of a single CP at first sight, further data indicate that comparatives regularly demonstrate doubling, similarly to the German pattern given in (\ref{alswiech1}) above, which seems to be present at least underlyingly in comparatives proper in all cases, while equatives may indeed have a single CP in the subclause. Further, the left periphery of degree clauses is also relevant in terms of polarity marking. In English, both degree equatives and comparatives are negative polarity environments, as illustrated by the following examples containing the negative polarity items \textit{any} and \textit{ever}:

\ea \label{englishch1}
\ea Sophia is as nice as \textbf{any} other teacher in the school. \label{asanych1}
\ex Sophia is nicer than \textbf{any} other teacher in the school. \label{thananych1}
\ex Museums are as popular as \textbf{ever} before. \label{aseverch1}
\ex Museums are more popular than \textbf{ever} before. \label{thaneverch1}
\z
\z

Negative polarity items are licensed in other negative polarity contexts (cf. \citealt{klima1964}) such as interrogatives, clausal negation and conditionals, but not in affirmative clauses (\citealt[531, ex. 11]{seuren1973}):

\ea
\ea[*]{\textbf{Any} of my friends could \textbf{ever} solve those problems.}
\ex[]{Could \textbf{any} of my friends \textbf{ever} solve those problems?}
\ex[]{At no time could \textbf{any} of my friends \textbf{ever} solve those problems.}
\ex[]{If \textbf{any} of my friends \textbf{ever} solve those problems, I'll buy you a drink.}
\z
\z

While the data in (\ref{englishch1}) suggest that English is symmetrical regarding negative polarity across the two major types of comparison clauses, German shows an asymmetric pattern: comparatives but not equatives have negative polarity:

\ea \label{germanch1}
\ea[*]{\gll Museen sind so beliebt wie \textbf{jemals} zuvor. \label{wiejemalsch1}\\
museums are so popular how ever before\\
\glt `Museums are as popular as ever before.'}
\ex[]{\gll Museen sind beliebter als \textbf{jemals} zuvor. \label{alsjemalsch1}\\
museums are more.popular as ever before\\
\glt `Museums are more popular than ever before.'}
\z
\z

The data point to the conclusion that the role of the left periphery in comparatives extends to marking polarity, not in terms of designated projections but as part of the featural makeup of the individual projections that are present in the derivation anyway due to independent clause-typing and semantic properties. These issues will be investigated in \chapref{ch:5}.

\subsection{Information structure and ellipsis} \label{sec:1information}
Certain constituents may undergo topicalisation or focalisation involving movement to the left periphery of the clause. Consider the following examples taken from \citet[285, ex. 1 and 2]{rizzi1997}:

\ea \label{englishrizzich1}
\ea {[}Your book]\textsubscript{i}, you should give \textit{t}\textsubscript{i} to Paul (not to Bill). \label{topiccommentch1}
\ex {[}YOUR BOOK]\textsubscript{i} you should give \textit{t}\textsubscript{i} to Paul (not mine). \label{focuspresuppch1}
\z
\z

The construction in (\ref{topiccommentch1}) illustrates topicalisation, and the one in (\ref{focuspresuppch1}) focalisation. Apart from interpretive differences, they crucially differ in their intonation patterns: a topic is separated by a so-called ``comma intonation'' from the remaining part of the clause (the comment), while a focus bears focal stress and is thus prominent with respect to presupposed information (see \citealt[258]{rizzi1997}).

Such movement operations are clearly instances of A-bar movement, and since they are apparently not driven by clause-typing features either, they raise the question what triggers movement in the first place. The cartographic model proposed by \citet{rizzi1997}, adopted by others such as \citet{poletto2006}, proposes that leftward movement in these cases targets designated left-peripheral positions: TopP and FocP. Movement is driven by specific features making reference to information-structural properties: this operator-like feature agrees with the functional head (Top or Foc). In essence, this kind of movement is supposed to be similar to ordinary operator movement involving \textit{wh}-operators or relative operators. Such an assumption is problematic, though: while [wh] and [rel] features are lexically determined, [topic] and [focus] features are obviously not. Taking the examples in (\ref{englishrizzich1}) above, in both cases the entire phrase \textit{your book} is topicalised or focussed, and the phrase as such, being compositional, is not part of the lexicon. This indicates that features like [topic] and [focus] would have to be added during the derivation. In addition, even if one were to assume that a lexical element like \textit{Mary} can be equipped with information-structural features in the lexicon (contrary to generally accepted views about the lexicon and lexical features, cf. \citealt{neelemanszendroei2004} and \citealt{dendikken2006}), this would leave us with various lexical entries for \textit{Mary}: a neutral entry (not specified for any information-structural category), a focussed one, a topicalised one, not to mention possible fine-grained categories such as contrastive topic or aboutness topic. 

Moreover, foci (and topics) can occur in non-fronted positions. This is illustrated by the following examples taken from \citet[172, ex. 6c and 6d]{fanselowlenertova2011}, both answering the question \textit{What happened?}:

\ea
\ea \gll \textbf{Eine} \textbf{LAWINE} haben wir gesehen!\\
a.\textsc{f.acc} avalanche have.\textsc{1pl} we seen\\
\glt `We saw an AVALANCHE!'
\ex \gll Wir haben \textbf{eine} \textbf{LAWINE} gesehen!\\
we have.\textsc{1pl} a.\textsc{f.acc} avalanche seen\\
\glt `We saw an AVALANCHE!'
\z
\z

This kind of optionality obviously contrasts with the behaviour of ordinary \textit{wh}-movement (and relative operator movement) in German, which always targets the CP-domain. Note also that there are certain fronted elements in the German CP (occupying the ``first position'') that clearly do not correspond to information structural categories such as topic and focus. Consider the following examples from \citet[173, ex. 7a]{fanselowlenertova2011}:

\ea \gll \textbf{Wahrscheinlich} hat ein Kind einen Hasen gefangen.\\
probably has a.\textsc{n.nom} child a.\textsc{m.acc} rabbit caught.\textsc{ptcp}\\
\glt `A child has probably caught a rabbit.'
\z

In this case, the adverb \textit{wahrscheinlich} `probably' is a sentential adverb that evidently lacks a discourse function such as topic or focus.

These considerations indicate that movement is not always driven by lexical features; if so, this has consequences regarding the way functional left peripheries are organised.

As mentioned above, clausal ellipsis is also closely connected to the issue of functional left peripheries. The prototypical case for this is sluicing, demonstrated in (\ref{sluicech1}) below:

\ea\label{sluicech1} Someone phoned grandma but I don't remember \textbf{WHO} \sout{phoned grandma}.\z

The elliptical clause is embedded in a clause conjoined with another main clause: this clause (\textit{someone phoned grandma}) contains the antecedents for the elided elements in the elliptical clause. The elliptical clause contains a single remnant, the subject \textit{who}, which bears main stress: it contains non-given information. Ellipsis is licensed as all elided information is recoverable. The assumption regarding the implementation of ellipsis in grammar (\citealt[55--61]{merchant2001} and \citealt[670--673]{merchant2004}) is that there is an ellipsis feature, [E]. This is merged with a functional head (such as C) and the complement of this head is elided. The [E] feature is specified as having either an uninterpretable [wh] or an uninterpretable [Q] feature, ensuring that it occurs only in (embedded) questions. As shown by \citet{vancraenenbroeckliptak2006} and \citet{hoytteodorescu2012}, this particular syntactic condition is highly unsatisfactory as many languages allow canonical ellipsis processes such as sluicing also from non-interrogative projections, including relative clauses and projections hosting foci. Rather, it seems that the [E] feature is not tied to a specific projection or features; indeed, \citet{merchant2004} also proposes that a functional projection, FP, can be headed by [E] in fragment answers, illustrated in (\ref{phonegrandma}) below:

\begin{exe}
\ex \label{phonegrandma}
\begin{xlist} 
\exi{A:} Who phoned grandma?
\exi{B:} \textbf{Liz} \sout{phoned grandma}.
\end{xlist}
\end{exe}

In this case, the remnant (\textit{Liz}) is the subject and the rest of the clause is elided. Since in English the subject DP in declarative clauses is located in [Spec,TP] and not in [Spec,CP], the ellipsis mechanism assumed for sluicing (the [E] feature located in C) does not automatically carry over. As \citet{merchant2004} assumes, there is an unspecified FP projection hosting the remnant in its specifier, landing there by movement. In this vein, it seems that leftward movement can target functional projections due to reasons other than clause-typing. This raises the question whether such functional projections may not ultimately have a more substantial role in the architecture of a clause than merely enabling ellipsis.

Questions related to information structure and ellipsis, particularly regarding their relevance for the proposed model, will be addressed in \chapref{ch:6}.

\section{Methodology}
This book aims at examining the syntax of functional left peripheries in West Germanic from a generative perspective, applying the minimalist framework in the analysis of syntactic structure. The main focus lies on the analysis of English and German, and to a lesser extent on Dutch. As language variation is a central issue, other Germanic languages will also occasionally be considered, as well as other European languages (mostly Romance and Slavic, and to some extent also Greek and Uralic). Language comparison can help to understand the cross-linguistic status of the West Germanic patterns: beyond that, however, the present investigation cannot carry out a more detailed analysis of these languages. 

Since the clausal left periphery is a well-studied area of linguistics (see \sectref{sec:1aims}), part of the investigation is dedicated to the analysis of already known patterns, also taken to be a basis for further inquiries. In addition, however, the book presents novel empirical data gained via corpus studies, questionnaires, and grammaticality judgement experiments. Regarding this, it must be kept in mind that the individual West Germanic languages (and their varieties) under scrutiny differ considerably in terms of how accessible the relevant data are and to what extent they have been discussed in the literature. 

As far as historical data are concerned, the present investigation relies on parsed corpora to identify which patterns were used in the given periods and what their frequency is. Regarding English, the Michigan Corpus of Middle English Prose and Verse was used; in addition, I compiled a database on relative clauses in the King James Bible and its modernised version. Regarding German, the DDD Referenzkorpus Altdeutsch was used. For present-day dialect data, the SyHD atlas on Hessian dialects and the SynAlm database on Alemannic dialects have been used.

As part of my project (BA 5201/1) on functional left peripheries, I obtained data from various Germanic languages (Dutch, Swedish, Danish, Norwegian, Icelandic) via an online questionnaire; this allows for a direct comparison of the languages involved. For each language, two informants were gathered who translated sentences from English as well as answered specific questions about the combinability of certain elements. The questionnaire contains 147 questions altogether. The results have been published in an open access database under \citet{bacskaiatkaribaudisch2018} and will be referenced throughout this work.

Finally, the book also presents the results of a grammaticality judgement experiment (see \citealt{schuetze2016} on the methodology) on elliptical comparative clauses in German. This allows for a more fine-grained analysis than the mere grammaticality judgements available thus far in the literature.

\section{Previous work}
The present book builds on results gained in my research projects and partly published in earlier papers; these works will be referenced in the relevant chapters as well. In this section, I would like to point out how the present investigation relates to and differs from these articles, to provide better orientation for the reader in this respect.

\chapref{ch:2} summarises the most important principles regarding the proposed non-cartographic model. The basic ideas were spelt out in \citet{bacskaiatkari2018sardis} regarding data from South German dialects and some major concerns regarding the cartographic model were also expressed in terms of the proposal made by \citet{baltin2010}. In the present book, the scope of the investigation is naturally larger; in addition, this chapter contains a detailed critical review of the literature, pointing out additional problems that were not discussed before, in particular regarding the original cartographic proposal by \citet{rizzi1997, rizzi2004}.

\chapref{ch:3} discusses embedded interrogative clauses. The core part of this chapter was published in \citet{bacskaiatkari2020jcgl}, with a particular emphasis on the relation between Doubly Filled COMP patterns in German and V2 syntax. The present investigation has a wider empirical and theoretical scope. In the original study, results from a corpus study on Middle English \textit{whether} were discussed: this was based on a smaller sample from the two versions of the Wycliffe Bible. The present study includes the results from the entire text (for both versions). Regarding the theoretical scope, the present study includes a detailed critical study of alternative analyses of Doubly Filled COMP effects, in particular that of the original proposal made by \citet{chomskylasnik1977}, which was not discussed before. In addition, the present book contains a section on long movement.

\chapref{ch:4} examines relative clauses. A core part of the discussion is centred on a corpus study carried out on the King James Bible. Some implications regarding the subject/object asymmetries observed in the choice of relativisation strategies were discussed in \citet{bacskaiatkari2020nordlyd}. This previous study was based on a smaller data set: for the present study, the entire King James Bible was taken into account, using the parallel loci to relative clause introduced by \textit{who(m)}, \textit{which} and \textit{that} in the modernised version. The present book also discusses some statistical findings that were not included in the previous investigations at all. In addition, the present study connects the findings to the general non-cartographic approach, as well as to the dialectal German data and it also presents a detailed account of equative relative clauses, also connecting the findings to the proposal made by \citet{brandnerbraeuning2013}.

\chapref{ch:5} is dedicated to embedded degree clauses. Some of the findings regarding German historical data and their diachronic development were discussed in \citet{bacskaiatkari2021oup}. The present study is more extensive in this respect and it also places the discussion of these data into a cross-linguistic setting, showing that the polarity differences between equative and comparative clauses hold across languages. The analysis is also connected to the model proposed in this book, showing the importance of analysing multiple left-peripheral projections in a non-cartographic model. The proposed account relies on many insights of \citet{jaeger2018}, yet there are some important differences in the syntactic structure between the two models: this issue is also discussed in detail.

\chapref{ch:6} analyses ellipsis processes in embedded clauses, concentrating on elliptical comparatives in German. The key idea behind the proposed analysis for German was expressed in \citet{bacskaiatkari2017atoh}; however, that study was entirely based on classical, introspective grammaticality judgements in very specific context, explicitly targeted at measuring ambiguity. The present study includes the results of a grammaticality judgement experiment and it also relates the findings to the general theory of ellipsis and information structure.

\section{Roadmap}
This book is structured as follows. In \chapref{ch:2}, I will introduce the basic assumptions regarding the proposed model. Following this, the book offers in-depth analyses of the three major constructions that will be examined here: \chapref{ch:3} addresses embedded interrogatives, \chapref{ch:4} addresses relative clauses, and \chapref{ch:5} addresses embedded degree clauses. In \chapref{ch:6}, I show that the analysis can be extended beyond the scope of clause typing proper, connecting it to issues related to information structure and ellipsis.
