\chapter{Conclusion} \label{ch:7}
The aim of this book was to examine the syntax of functional left peripheries in West Germanic, with a particular focus on how sentence types are marked at the leftmost edge of the clause and how the presence of multiple visible markers can be accounted for. Naturally, the analysis given here was restricted to a few selected issues and could not cover all questions in equal depth; still, the selected issues provide a sound basis for the theory developed in this work and can serve as a foundation for the proposed model. In addition to the specific constructions, a major interest of my research was to account for the observed cross-linguistic and dialectal variation in a formal way, with the aim of identifying both common West Germanic properties as well as language-specific constraints that may be responsible for the attested constraints. Ultimately, the goal was to tie the construction-specific constraints to more general properties of the languages under scrutiny, allowing for a grammar with as few construction-specific rules as possible.

\chapref{ch:2} presented the basic assumptions concerning a minimal, feature-based approach to the syntax of functional left peripheries, showing that the proposed analysis applies to various clause types, in each case correctly predicting the surface order of clause-typing elements appearing in combinations. Since the relevant combinations are restricted to embedded clauses in Germanic languages, this chapter focussed on subordinate clauses, while maintaining the assumption that the analysis is applicable to main clauses as well. In particular, I argued against cartographic approaches (in particular \citealt{rizzi1997, rizzi2004}), showing that clause-typing elements appearing on functional left peripheries are not in a one-to-one relationship with syntactic features, and the assumption that there are designated projections for the various semantic properties is fundamentally flawed. Instead, I proposed that functional left peripheries are as minimal as possible, and multiple projections are generated when the relevant semantic properties cannot be marked in a single projection; whether this is the case is ultimately dependent on the lexical properties of the individual clause-typing elements.

In current minimalist theory, the Complementiser Phrase (CP) is responsible for typing clauses and for encoding finiteness in finite clauses. Apart from complementisers, various operators can appear in this domain. Consider:

\ea
\ea I wonder \textbf{if} Ralph has arrived. \label{englishifch7}
\ex I wonder \textbf{whether} Ralph has arrived. \label{englishwhetherch7}
\z
\z

The standard assumption is that complementisers are by definition C heads, while operators move to the [Spec,CP] position. In (\ref{englishifch7}), the element \textit{if} is a complementiser and it types the subordinate clause as interrogative. In (\ref{englishwhetherch7}), there is no overt complementiser but the operator \textit{whether} is present. In (\ref{englishwhetherch7}), assuming that the operator is in the specifier, the clause is typed by a zero complementiser. Nevertheless, both kinds of elements (complementisers and operators) can overtly encode clause type; this was assumed to be the reason for the ban on the co-occurrence of the two in dialects like Standard English:

\ea[*]{I wonder \textbf{whether if} Ralph has arrived. \label{whetherifch7}}
\z

In other cases, however, we can observe a clear split in the function of two elements. In West Germanic, the CP is not restricted to hosting a single overt element: depending on the particular construction and the dialect, multiple elements may appear in the CP-domain. This is illustrated by (\ref{englishdfcch7}) for non-standard English and by (\ref{norwegiandfcch7}) for Norwegian (\citealt[175]{bacskaiatkaribaudisch2018}):

\ea \label{dfcch7}
\ea[\%]{I wonder \textbf{which book that} Ralph is reading. \label{englishdfcch7}}
\ex[]{\gll Peter spurte \textbf{hvem} \textbf{som} likte bøker. \label{norwegiandfcch7}\\
Peter	asked.\textsc{3sg} who	that liked books\\
\glt `Peter asked who liked books.'}
\z
\z

Such combinations raise the question whether the postulation of multiple CP projections is necessary in such cases. \chapref{ch:2} argued that while the generation of multiple functional layers is in principle possible, it should be appropriately restricted to exclude the generation of superfluous layers that are empirically not motivated. This question is likewise relevant in cases involving a single overt C-element, since then the question arises whether and to what extent covert elements and phonologically not visible projections are present. Adopting a minimal CP approach, in part following \citet{sobin2002}, I argued in \chapref{ch:2} that while the operator and the complementiser in cases like (\ref{dfcch7}) are associated with distinct functions, namely the overt marking of clause type and finiteness, respectively, they can still be located in a single CP, schematically represented in \figref{dfctreech7}.

\begin{figure} 
\caption{Doubly Filled COMP} \label{dfctreech7}
\begin{forest} baseline, qtree
[CP
	[which book\textsubscript{{[}wh{]}}]
	[C$'$
		[C\textsubscript{{[}wh{]},{[}fin{]}}
			[that\textsubscript{{[}fin{]}}]
		]
		[TP]
	]
]
\end{forest}
\end{figure}

In an embedded constituent question, two properties must be encoded by the CP: the interrogative clause-typing feature [wh] and the finiteness feature [fin], since both of these properties are required by the matrix predicate. The properties can be split between two elements, one merged as a specifier and one being the head, so that the features are stacked in a single projection. This is a more economical way than postulating two separate projections for encoding each property, as that would also presuppose the existence of additional empty elements, which are not motivated otherwise, and also a separate mechanism for the upward percolation of [fin], so that it remains visible for the matrix predicate. The ordering restrictions on the combination of a finite complementiser and an interrogative operator follow straightforwardly from the way specifiers are merged in the syntactic component. Note that ordering restrictions may differ if lower peripheries (following \citealt{poletto2006}) are involved, since in that case the separation of projections is additionally motivated by separate functional peripheries.

On the other hand, interrogative complementisers regularly mark finiteness as well. Consider:

\ea \label{ifwhetherch7}
\ea[]{I don't know \textbf{if} I should call Ralph. \label{iffinitech7}}
\ex[]{I don't know \textbf{whether} I should call Ralph. \label{whetherfinitech7}}
\ex[*]{I don't know \textbf{if} to call Ralph. \label{ifnonfinitech7}}
\ex[]{I don't know \textbf{whether} to call Ralph.  \label{whethernonfinitech7}}
\z
\z

\begin{sloppypar}
In (\ref{iffinitech7}), the complementiser \textit{if} introduces a finite embedded interrogative clause, and as the ungrammaticality of (\ref{ifnonfinitech7}) shows, it is incompatible with a non-finite clause, suggesting that it encodes finiteness in addition to the interrogative property. By contrast, the operator \textit{whether} is compatible both with a finite clause, see (\ref{whetherfinitech7}), and with a non-finite clause, see (\ref{whethernonfinitech7}), indicating that the overt marking of interrogativity is not incompatible with a non-finite clause in English. In this case, too, a single CP projection is sufficient for encoding the interrogative property (argued to be [Q] in polar questions) and [fin], as both are lexical properties of the same complementiser. In the case of \textit{whether}, just like with ordinary \textit{wh}-operators in dialects like Standard English, a phonologically empty complementiser is assumed for encoding finiteness: such complementisers are semantically motivated and, since they are attested in other constructions as well, independently motivated.
\end{sloppypar}

Finally, \chapref{ch:2} investigated the issue of certain non-trivial combinations in which elements seem to be largely similar, as in the non-standard German example in (\ref{alswiech7}):

\ea 
\ea[\%]{\gll Ralf ist größer \textbf{als} \textbf{wie} Maria. \label{alswiech7}\\
Ralph is taller than as Mary\\
\glt `Ralph is taller than Mary.'}
\ex[]{\gll Ralf ist größer \textbf{als} Maria. \label{alsch7}\\
Ralph is taller than Mary\\
\glt `Ralph is taller than Mary.'}
\ex[\%]{\gll Ralf ist größer \textbf{wie} Maria. \label{wiech7}\\
Ralph is taller as Mary\\
\glt `Ralph is taller than Mary.'}
\ex[]{\gll Ralf ist so groß \textbf{wie} Maria. \label{wieequatch7}\\
Ralph is so tall as Mary\\
\glt `Ralph is as tall as Mary.'}
\z
\z

In (\ref{alswiech7}), the elements \textit{als} and \textit{wie} both seem to mark the comparative nature of the clause, whereby single \textit{als} is the comparative particle in Standard German comparatives, see (\ref{alsch7}). Single \textit{wie} is the comparative particle in equatives, see (\ref{wieequatch7}), and in certain dialects also in comparatives, see (\ref{wiech7}). Since there is independent evidence for both elements being complementisers and for there being a separate comparative operator in the lower CP, \chapref{ch:2} argued that configurations like (\ref{alswiech7}) differ from the kind of doubling represented by (\ref{dfcch7}) in that two separate CPs are involved in the left periphery of the clause, whereby doubling is ultimately motivated by comparative semantics. The point is that while the model put forward in \chapref{ch:2} explicitly involves less structure in most of the examined patterns than would be postulated by cartographic approaches, there is no assumed commitment to there being always a single CP only. If motivated by the co-presence of overt elements and/or by independently established semantic properties, the CP-periphery can indeed be larger than a single CP. This assumption makes the proposed model not only restrictive in terms of the number of projections but also flexible enough to account for a number of phenomena.

Using the framework established in \chapref{ch:2}, \chapref{ch:3} was devoted to the left periphery of interrogative clauses, especially embedded ones. In particular, I examined various combinations of operators and complementisers in the left periphery that are allowed in certain dialects but not in others. The impossibility of the relevant combinations in standard West Germanic languages has been referred to as the ``Doubly Filled COMP Filter'' in the literature, suggesting some inherent syntactic ban on the configurations; however, the generalisation is not compatible with empirical data from non-standard dialects and from other languages allowing the combinations in question. I argued that the existence of such combinations does not require or justify the postulation of designated projections, as in cartographic approaches. Instead, I proposed that doubling patterns are compatible with a minimal CP and the insertion of a finite complementiser is not an indication of a separate projection for finiteness but merely the consequence of the regular Germanic pattern of lexicalising a finite C overtly, as can also be seen in V2 patterns.

In Standard English, Standard German and Standard Dutch, there is no overt complementiser with an overt interrogative operator. This is illustrated in (\ref{whothatch7}) for English embedded interrogatives:

\ea	I don't know \textbf{who (*that)} has arrived. \label{whothatch7}
\z

As can be seen, the complementiser \textit{that} is not permitted in Standard English in embedded constituent clauses. This restriction was captured by the notion of the Doubly Filled COMP Filter going back to \citet{chomskylasnik1977}, who assumed that one of the elements in COMP (which was analysed as CP in later approaches) must be deleted. However, as \citet{chomskylasnik1977} also mention, there are languages and also many West Germanic varieties that allow such patterns, as in the following examples from non-standard English (\citealt[331, ex. 1]{baltin2010}) and from non-standard Dutch (\citealt[32]{bacskaiatkaribaudisch2018}):

\ea \label{dfcintch7}
\ea[\%]{They discussed a certain model, but they didn't know \textbf{which model that} they discussed.}
\ex[\%]{\gll Peter vroeg \textbf{wie} \textbf{dat} er boeken leuk vindt. \label{dutchdfcch7}\\
Peter asked.\textsc{3sg} who that of.them books likeable finds\\
\glt `Peter asked who liked books.'}
\z
\z

Assuming the structure given in \figref{dfctreech7}, it is evident that the CP is doubly filled in these cases, both the specifier and the head containing overt elements. This, however, is by no means exceptional: as pointed out in \chapref{ch:3}, the specifier of the CP and the C head can be both lexicalised overtly in main clauses, as in T-to-C movement in English interrogatives, and in V2 clauses in German and Dutch main clauses. Consider the examples for main clause interrogatives in Standard English:

\ea \label{ttocch7}
\ea	\textbf{Who saw} Ralph? \label{whosawch7}
\ex	\textbf{Who did} Ralph see? \label{whodidch7}
\z
\z

In this case, doubling in the CP involves a \textit{wh}-operator in [Spec,CP] and a verb in C. T-to-C movement is visible by way of \textit{do}-insertion in (\ref{whodidch7}), though not in (\ref{whosawch7}): in principle, one might analyse (\ref{whosawch7}) as not involving the movement of the verb to C, but the CP is clearly doubly filled in (\ref{whodidch7}).

Similarly, in German (and Dutch) V2 declarative clauses, there is a verb moving to C, while another constituent moves to [Spec,CP] due to an [edge] feature (see \citealt{thiersch1978diss}, \citealt{fanselow2002, fanselow2004isis, fanselow2004}, \citealt{frey2005}, \citealt{denbesten1989}). Consider:

\ea \label{v2ch7}
\ea \gll \textbf{Ralf} \textbf{hat} morgen Geburtstag.\\
Ralph has tomorrow birthday\\
\glt `Ralph has his birthday tomorrow.'
\ex \gll \textbf{Morgen} \textbf{hat} Ralf Geburtstag. \label{v2adverbch7}\\
tomorrow has Ralph birthday\\
\glt `Ralph has his birthday tomorrow.'
\z
\z

As can be seen, the fronted finite verb is preceded by a single constituent in each case, and since the first constituent is not a clause-typing operator in either case, it is evident that doubling in the CP in V2 clauses is independent of the interrogative property.

\chapref{ch:3} therefore argued that any constraint underlying the Doubly Filled COMP Filter should be more restricted in its application domain. In principle, one could say that an operator and a complementiser with largely overlapping functions are not permitted to co-occur in standard West Germanic languages, or that there is some kind of an economy principle. On the other hand, the notion of the Doubly Filled COMP Filter implies that the C head and [Spec,CP] would be filled without the Filter, and the Filter is responsible for ``deleting'' the content of C: this approach was argued to be problematic in \chapref{ch:3}. 

Instead, \chapref{ch:3} proposed that the filling of C with overt material in Doubly Filled COMP patterns in Germanic, such as (\ref{dfcintch7}), is in line with the general syntactic paradigm, in which the C position is regularly lexicalised by an overt element, as in the patterns in (\ref{ttocch7}) and (\ref{v2ch7}) above. For a configuration like (\ref{v2adverbch7}), I assume the structure given in \figref{treev2barech7}.

\begin{figure} 
\caption{The structure of German V2} \label{treev2barech7}
\begin{forest} baseline, qtree
[hat
	[morgen\textsubscript{{[}edge{]}}]
	[hat
		[hat\textsubscript{{[}fin{]},{[}u:edge{]}}
		]
		[TP\textsubscript{{[}u:fin{]}}
			[Ralf Geburtstag,roof]
		]
	]
]
\end{forest}
\end{figure}

In \figref{treev2barech7}, the TP has an unchecked feature, [u:fin], which is ultimately projected by the verb but can only be checked by the fronting of the verb and re-merging it as a sister to the head, following \citet[309]{fanselow2004}. As there is no clause-typing feature that would trigger movement, an [edge] feature is responsible for the movement of the XP (here: \textit{morgen}) to the first position, resulting in surface V2. Doubly Filled COMP patterns differ only in that the movement of the specifier element is triggered by the interrogative feature, [wh] or [Q], anyway, and the finiteness feature of TP is checked off by inserting a complementiser. In dialects showing Doubly Filled COMP, there is no phonologically zero complementiser in the lexicon that would be compatible with the required features, resulting in overt doubling. This also indicates that elements other than complementisers can satisfy the requirement of filling C, indicating that the deletion approach to the lack of Doubly Filled COMP patterns is not adequate, as there is no underlying complementiser.

In addition, there is a theoretical problem with the notion of the Filter, which arises from a merge-based, minimalist perspective, while it is less problematic in X-bar theoretic terms. X-bar theoretic notions can at best taken to be descriptive designators that are derived from more elementary principles, in the vein of \citet{kayne1994} and \citet{chomsky1995}. Under this view, the position of an element (specifier, head, complement) is a result of its relative position when it is merged with another element, and which element is chosen to be the label. By contrast, the notion of the Doubly Filled COMP Filter, as applied to a CP (as in \citealt{baltin2010}), implies that a phrase is generated with designated, pre-given head and specifier positions, and that there are additional rules on whether and to what extent they can be actually ``filled'' by overt elements. In a merge-based account, there are no literally empty positions as no positions are created independent of merge: zero heads and specifiers reflect elements that are either lexically zero or have been eliminated by some deletion process (for instance, as lower copies of a movement chain or via ellipsis). This requirement is met by structures like \figref{treev2barech7} and the same applies to the analogous counterparts containing complementisers and clause-typing operators.

\chapref{ch:4} was devoted to the analysis of relative clauses, applying the framework established in \chapref{ch:2} and refined for interrogative clauses in \chapref{ch:3}. The notion of the ``Doubly Filled COMP Filter'' emerged in the literature primarily in connection with relative clauses in English. One of the most important questions addressed in \chapref{ch:4} was therefore whether and to what extent the conclusions drawn in \chapref{ch:3} for embedded interrogatives hold for relative clauses in West Germanic. On the one hand, combinations of operators and complementisers were examined; such combinations are particularly important as they are not compatible with traditional cartographic approaches. On the other hand, the question was addressed why and to what extent there seems to be a preference for relative complementisers over relative pronouns in Germanic. This preference was shown to make doubling patterns less likely to appear in relative clauses than in embedded constituent questions in dialects that allow the relevant patterns. 

West Germanic languages show considerable variation in terms of elements introducing relative clauses. There are two major strategies: the relative pronoun strategy and the relative complementiser strategy. In present-day Standard English, both of these strategies are attested. Relative pronouns are illustrated in (\ref{relpron}) below:

\ea \label{relpron}
\ea I saw the woman \textbf{who} lives next door in the park. \label{whosubjectch7}
\ex The woman \textbf{who/whom} I saw in the park lives next door. \label{whoobjectch7}
\ex I saw the cat \textbf{which} lives next door in the park. \label{whichsubjectch7}
\ex The cat \textbf{which} I saw in the park lives next door. \label{whichobjectch7}
\z
\z

As can be seen, relative pronouns show partial case distinction and distinction with respect to whether the referent is human or non-human. In particular, \textit{who}/\textit{whom} is used with human antecedents, as with \textit{the woman} in (\ref{whosubjectch7}) and (\ref{whoobjectch7}); the form \textit{who} can appear both as nominative and as accusative, while the form \textit{whom} used for the accusative is restricted in its actual appearance (formal/marked). With non-human antecedents, such as \textit{the cat} in (\ref{whichsubjectch7}) and (\ref{whichobjectch7}), the pronoun \textit{which} is used, which shows no case distinction. Note that apart from human referents, \textit{who(m)} is possible with certain animals: these are the ``sanctioned borderline cases'' (see \citealt[41]{herrmann2005}, quoting \citealt{quirkgreenbaumleechsvartvik1985}). On the other hand, non-standard dialects allow \textit{which} with human referents: five of the six dialect areas show this, while the proportion of \textit{which} is very low in Northern Ireland (see \citealt[41]{herrmann2005}). The construction is illustrated in (\ref{boywhichch7}) below (\citealt[42, ex. 4a]{herrmann2005}):

\ea {[}\ldots] And the boy \textbf{which} I was at school with [\ldots] \label{boywhichch7}\\
(\textit{Freiburg English Dialect Corpus} Wes\_019)
\z

At any rate, English relative pronouns are formed on the \textit{wh}-base and no longer on the demonstrative base. Note that this is historically not so, and the present-day complementiser \textit{that} was reanalysed from a pronoun, while the \textit{wh}-based relative operators appeared only in Middle English (\citealt{vangelderen2009}).

Accordingly, the complementiser \textit{that} constitutes the second major strategy:

\ea
\ea I saw the woman \textbf{that} lives next door in the park.
\ex The woman \textbf{that} I saw in the park lives next door.
\ex I saw the cat \textbf{that} lives next door in the park.
\ex The cat \textbf{that} I saw in the park lives next door.
\z
\z

The complementiser \textit{that} is not sensitive to case and to the human/non-human distinction, which follows from its status as a C head. 

Given the availability of two strategies, \chapref{ch:4} examined the question to what extent the two can be combined and what implications such combinations have for the theory. It was shown that while combinations are perfectly possible, they are less likely to occur than surface-similar doubling in embedded interrogatives. I argued that this is because the complementiser strategy already satisfies the lexicalisation requirement on C, so varieties that have this strategy are likely to prefer it over the use of relative pronouns, especially in functions high on the Noun Phrase Accessibility Hierarchy of \citet{keenancomrie1977}. Relative pronouns are recoverable (unlike interrogative pronouns), so they are not necessary overt: in functions lower on the Noun Phrase Accessibility Hierarchy, they are more likely to occur as they overtly identify the gap. Nevertheless, doubling is attested both in English and in German (in non-standard varieties); for such combinations, the structure given in \figref{treedfcenglishch7} was adopted.

\begin{figure} 
\caption{Doubling in relative clauses} \label{treedfcenglishch7}
\begin{forest} baseline, qtree
[CP
	[who(m)/which\textsubscript{{[}rel{]}}]
	[C$'$
		[C\textsubscript{{[}rel{]},{[}fin{]}}
			[that\textsubscript{{[}rel{]},{[}fin{]}}]
		]
		[TP]
	]
]
\end{forest}
\end{figure}

In essence, \figref{treedfcenglishch7} parallels \figref{dfctreech7} above. Notice, however, that there is no perfect split between the functions, as the complementiser not only encodes [fin] but it also carries a clause-typing feature, namely [rel]. This analysis gains support from German dialects, where the complementiser occurring in relative clauses is \textit{wo}, which is not surface-identical to the regular finite complementiser \textit{dass}, indicating that a split CP approach could not account for the combination. Regarding the combinability of the individual elements in West Germanic, a clear tendency was observed regarding the etymology of the individual elements: genuine doubling involving a relative operator and a relative complementiser is found only in the forms where a demonstrative-based pronoun is combined with a \textit{wh}-based complementiser (as in dialects of German), or where a \textit{wh}-based pronoun is combined with a demonstrative-based complementiser (as in English). This follows most probably from the interpretability of the [rel] feature on the individual elements: namely, it is interpretable on demonstrative-based elements and must be checked off on \textit{wh}-based elements.

Building on the theory put forward in the previous chapters, \chapref{ch:5} examined comparison constructions, including non-degree equatives (similatives), degree equatives, and comparatives expressing inequality. It was shown that while these constructions are similar in several respects, they show differences in ways that are slightly unexpected for analyses developed primarily for comparatives expressing inequality. The differences become evident especially when looking at the possible combinations of complementisers and operators at the left periphery of the subordinate clause. The various combinations are naturally relevant for the theory of functional left peripheries because they provide an ideal testing ground for whether designated projections are necessary, as is done in cartographic approaches, or whether a more minimal CP is favourable. Comparison constructions indeed provide evidence for the existence of multiple CP projections, yet the availability of overt combinations is subject to constraints that cartographic approaches cannot handle in an adequate way. Instead, I proposed that the restrictions and requirements on multiple marking should not be tied to designated projections but they follow from the semantic properties of the individual constructions and are also in interaction with the properties of the matrix element.\largerpage

Embedded degree clauses fall into two major types: degree equatives, also called comparatives expressing equality, as given in (\ref{astallch7}), and comparatives expressing inequality, as given in (\ref{tallerthanch7}):

\ea \label{comparisonch7}
\ea Ralph is as tall \textbf{as} Mary is.\label{astallch7}
\ex Ralph is taller \textbf{than} Mary is.\label{tallerthanch7}
\z
\z

In (\ref{astallch7}), the subclause introduced by \textit{as} expresses that the degree to which Mary is tall is the same as to which Ralph is tall, while in (\ref{tallerthanch7}) the subclause introduced by \textit{than} expresses that the degree to which Mary is tall is lower than the degree to which Ralph is tall.

The comparison constructions presented in (\ref{comparisonch7}) above are instances of degree comparison: there is one degree expressed in the matrix clause and another one expressed in the subclause. The matrix degree morpheme is \textit{as} in degree equatives and it selects an \textit{as}-clause, while the matrix degree morpheme in degree comparatives is -\textit{er} (or \textit{more}, which is actually a composite of -\textit{er} and \textit{much}, see \citealt{bresnan1973}, \citealt{bacskaiatkari2014diss, bacskaiatkari2018langsci}). However, it is possible to have comparison without degree; consider the following examples:

\ea \label{nondegreecomparisonch7}
\ea[]{Mary is tall, \textbf{as} is her mother. \label{tallasch7}}
\ex[]{Mary is glamorous \textbf{like} a film-star. \label{glamorouslikech7}}
\ex[]{Farmers have other concerns \textbf{than} the farm bill. \label{otherthanch7}}
\ex[\%]{Life in Italy is different \textbf{than} I expected. \label{differentthanch7}}
\z
\z

In these cases, there is obviously no matrix degree element. The sentences in (\ref{tallasch7}) and (\ref{glamorouslikech7}) express merely similarity with respect to the property denoted by the adjective; in (\ref{glamorouslikech7}), the subclause is introduced by \textit{like} and not by \textit{as}, a further difference from degree equatives. Given the availability of non-degree equatives, \citet[35]{jaeger2018} suggests that comparison constructions can be grouped into three major categories: non-degree equatives, degree equatives, and comparatives; these constitute a markedness hierarchy in this order (non-degree equatives being the least marked). However, constructions like (\ref{otherthanch7}) and (\ref{differentthanch7}) indicate that there is in fact a fourth category as well: these are non-degree comparatives expressing difference. This category seems not to be productive as the availability of the \textit{than}-clause is dependent on the presence of a particular element expressing difference in the matrix clause: the word \textit{other} or, at least in American English, the adjective \textit{different} are potential candidates.

While the patterns in (\ref{comparisonch7}) suggest a relatively simple left periphery consisting of a single CP at first sight, further data indicate that comparatives regularly demonstrate doubling, similarly to the German pattern given in (\ref{alswiech7}) above, which seems to be present at least underlyingly in comparatives proper in all cases, while equatives may indeed have a single CP in the subclause. \chapref{ch:5} argued that this is primarily related to comparative semantics: the maximality operator can be lexicalised either by a matrix element or by a higher complementiser, and the comparative operator is realised in a lower CP. Depending on which element expresses the maximality operator, the CP is doubled or remains single. This also has implications for the grammaticalisation processes affecting comparative clauses historically: equatives are more flexible in that they may also recategorise the original matrix element (\citealt{jaeger2018}).

\chapref{ch:5} also argued that this structural asymmetry underlies differences in polarity. In English, both degree equatives and comparatives are negative polarity environments, as illustrated by the following examples containing the negative polarity items \textit{any} and \textit{ever}:

\ea \label{englishch7}
\ea Sophia is as nice as \textbf{any} other teacher in the school. \label{asanych7}
\ex Sophia is nicer than \textbf{any} other teacher in the school. \label{thananych7}
\ex Museums are as popular as \textbf{ever} before. \label{aseverch7}
\ex Museums are more popular than \textbf{ever} before. \label{thaneverch7}
\z
\z

Negative polarity items are licensed in other negative polarity contexts (cf. \citealt{klima1964}) such as interrogatives, clausal negation and conditionals, but not in affirmative clauses (\citealt[531, ex. 11]{seuren1973}):

\ea
\ea[*]{\textbf{Any} of my friends could \textbf{ever} solve those problems.}
\ex[]{Could \textbf{any} of my friends \textbf{ever} solve those problems?}
\ex[]{At no time could \textbf{any} of my friends \textbf{ever} solve those problems.}
\ex[]{If \textbf{any} of my friends \textbf{ever} solve those problems, I'll buy you a drink.}
\z
\z

While the data in (\ref{englishch7}) indicate that English is symmetrical regarding negative polarity across the two major types of comparison clauses, German shows an asymmetric pattern: comparatives but not equatives have negative polarity:

\ea \label{germanch7}
\ea[*]{\gll Museen sind so beliebt wie \textbf{jemals} zuvor. \label{wiejemalsch7}\\
museums are so popular how ever before\\
\glt `Museums are as popular as ever before.'}
\ex[]{\gll Museen sind beliebter als \textbf{jemals} zuvor. \label{alsjemalsch7}\\
museums are more.popular as ever before\\
\glt `Museums are more popular than ever before.'}
\z
\z

As was shown, the data point to the conclusion that the role of the left periphery in comparatives extends to marking polarity, not in terms of designated projections but as part of the featural makeup of the individual projections that are present in the derivation anyway due to independent clause-typing and semantic properties. In particular, the data indicate that comparative \textit{als} and equative \textit{wie} occupy different kinds of projections regarding their relative positions in the left periphery (contrary to \citealt{jaeger2018}), which does not immediately correlate with their combinability in other constructions such as hypothetical comparatives. This again goes against a strict cartographic approach as the cross-constructional variation observed even in a single variety cannot be modelled by assuming designated projections.

After examining mostly finite, non-elliptical clauses in this book, concentrating on clause-typing elements in the CP-periphery, \chapref{ch:6} examined the role of information structure and ellipsis in terms of functional left peripheries. Functional left peripheries, both in the CP and in lower domains, may host elements associated with special information-structural roles (topics, foci). In addition, certain ellipsis processes, such as sluicing, are known to be associated with functional projections located at the left periphery. Naturally, the discussion of either issue (information structure and clausal ellipsis) would require more investigation than could be carried out in this work, and therefore I restricted myself to the discussion of some selected issues that bear immediate relevance to the general theory put forward in this book. I concentrated on elliptical interrogatives and reduced comparative constructions, showing that the proposed model can cast light upon interesting phenomena involving focalisation and clausal ellipsis.

\begin{sloppypar}
Certain constituents may undergo topicalisation or focalisation involving movement to the left periphery of the clause. Consider the following examples taken from \citet[285, ex. 1 and 2]{rizzi1997}:
\end{sloppypar}

\ea \label{englishrizzich7}
\ea {[}Your book]\textsubscript{i}, you should give \textit{t}\textsubscript{i} to Paul (not to Bill). \label{topiccommentch7}
\ex {[}YOUR BOOK]\textsubscript{i} you should give \textit{t}\textsubscript{i} to Paul (not mine). \label{focuspresuppch7}
\z
\z

The construction in (\ref{topiccommentch7}) illustrates topicalisation, and the one in (\ref{focuspresuppch7}) focalisation. Apart from interpretive differences, they crucially differ in their intonation pattern: a topic is separated by a so-called ``comma intonation'' from the remaining part of the clause (the comment), while a focus bears focal stress and is thus prominent with respect to presupposed information (see \citealt[258]{rizzi1997}).

Such movement operations are clearly instances of A-bar movement, and since they are apparently not driven by clause-typing features either, they raise the question what triggers movement in the first place. The cartographic model proposed by \citet{rizzi1997}, adopted by others such as \citet{poletto2006}, proposes that leftward movement in these cases targets designated left-peripheral positions: TopP and FocP. Movement is driven by specific features making reference to information-structural properties: this operator-like feature agrees with the functional head (Top or Foc). In essence, this kind of movement is supposed to be similar to ordinary operator movement involving \textit{wh}-operators or relative operators. As discussed in \chapref{ch:6}, such an assumption is problematic because while [wh] and [rel] features are lexically determined, [topic] and [focus] features are obviously not. Taking the examples in (\ref{englishrizzich7}) above, in both cases the entire phrase \textit{your book} is topicalised or focussed, and the phrase as such, being compositional, is not part of the lexicon. This indicates that features like [topic] and [focus] would have to be added during the derivation. In addition, even if one were to assume that a lexical element like \textit{Mary} can be equipped with information-structural features in the lexicon (contrary to generally accepted views about the lexicon and lexical features, cf. \citealt{neelemanszendroei2004} and \citealt{dendikken2006}), this would leave us with various lexical entries for \textit{Mary}: a neutral entry (not specified for any information-structural category), a focussed one, a topicalised one, not to mention possible fine-grained categories such as contrastive topic or aboutness topic. 

Moreover, foci (and topics) can occur in non-fronted positions. This is illustrated by the following examples taken from \citet[172, ex. 6c and 6d]{fanselowlenertova2011}, both answering the question \textit{What happened?}:

\ea
\ea \gll \textbf{Eine} \textbf{LAWINE} haben wir gesehen!\\
a.\textsc{f.acc} avalanche have.\textsc{1pl} we seen\\
\glt `We saw an AVALANCHE!'
\ex \gll Wir haben \textbf{eine} \textbf{LAWINE} gesehen!\\
we have.\textsc{1pl} a.\textsc{f.acc} avalanche seen\\
\glt `We saw an AVALANCHE!'
\z
\z

This kind of optionality obviously contrasts with the behaviour of ordinary \textit{wh}-movement (and relative operator movement) in German, which always targets the CP-domain. Note also that, as pointed out by \citet[173]{fanselowlenertova2011}, there are certain fronted elements in the German CP (occupying the ``first position'') that clearly do not correspond to information-structural categories such as topic and focus. Consider (\citealt[173, ex. 7a]{fanselowlenertova2011}):

\ea \gll \textbf{Wahrscheinlich} hat ein Kind einen Hasen gefangen.\\
probably has a.\textsc{n.nom} child a.\textsc{m.acc} rabbit caught.\textsc{ptcp}\\
\glt `A child has probably caught a rabbit.'
\z

In this case, the adverb \textit{wahrscheinlich} `probably' is a sentential adverb that evidently lacks a discourse function such as topic or focus.

These considerations indicate that movement is not always driven by lexical features. Following this line of argumentation, I adopted \chapref{ch:6} the view of \citet{fanselowlenertova2011} in that movement is driven by an unspecified [edge] feature in these cases and that information-structural effects arise as defined by the interfaces. Movement can target the CP but it can also target a functional projection, FP.

The FP has a crucial role in elliptical structures as well. As mentioned above, clausal ellipsis is also closely connected to the issue of functional left peripheries. The prototypical case for this is sluicing, demonstrated in (\ref{sluicech7}) below:

\ea Someone phoned grandma but I don't know \textbf{WHO} \sout{phoned grandma}. \label{sluicech7}
\z

In this case, the elliptical clause is embedded in a clause conjoined with another main clause: this first main clause (\textit{someone phoned grandma}) contains the antecedents for the elided elements in the elliptical clause. The elliptical clause contains only a single remnant, the subject \textit{who}, which bears main stress as it contains non-given information. Ellipsis is licensed because all elided information is recoverable. The standard assumption regarding the actual implementation of ellipsis in grammar (\citealt[55--61]{merchant2001} and \citealt[670--673]{merchant2004}) is that there is an [E] feature responsible for ellipsis. This feature is assumed to be merged with a particular functional head (such as C) and the complement of this head is elided. The [E] feature is specified as having either an uninterpretable [wh] or an uninterpretable [Q] feature, thus [u:wh] or [u:Q], ensuring that it occurs only in (embedded) questions. As shown by \citet{vancraenenbroeckliptak2006} for Hungarian and \citet{hoytteodorescu2012} for Romanian, this particular syntactic condition is highly unsatisfactory as many languages allow canonical ellipsis processes such as sluicing also from non-interrogative projections, including relative clauses and projections hosting foci.

If so, however, it seems that the [E] feature is not tied to a specific projection or features; indeed, \citet{merchant2004} also proposes that a functional projection, FP, can be headed by [E] in fragment answers, illustrated in (\ref{grandmaphone}) below:

\begin{exe}
\ex \label{grandmaphone}
\begin{xlist} 
\exi{A:} Who phoned grandma?
\exi{B:} \textbf{Liz} \sout{phoned grandma}.
\end{xlist}
\end{exe}

In this case, the remnant (\textit{Liz}) is the subject and the rest of the clause is elided. Since the subject DP in declarative clauses is located in [Spec,TP] and not in [Spec,CP] in English, the ellipsis mechanism assumed for sluicing (the [E] feature located in C) does not automatically carry over. As \citet{merchant2004} assumes, there is an unspecified FP projection hosting the remnant in its specifier, landing there by movement. In this vein, it seems that leftward movement can target functional projections due to reasons other than clause-typing. This of course also raises the question whether such functional projections may not ultimately have a more substantial role in the architecture of a clause than merely enabling ellipsis.

In \chapref{ch:6}, I argued that the availability of the [E] feature and the FP projection headed by [E] have consequences in terms of the organisation of left peripheries, especially regarding finiteness. In sluicing constructions like (\ref{sluicech7}), the presence of the [E] feature is not compatible with finiteness, [fin], so that even varieties that have Doubly Filled COMP patterns otherwise do not insert a finite complementiser in sluicing patterns. This gives us the  representation in \figref{whichmodelellipsis}.

\begin{figure} 
\caption{The [E] feature in sluicing} \label{whichmodelellipsis}
\begin{forest} baseline, qtree
[CP
	[which model\textsubscript{{[}wh{]}}]
	[C$'$
		[C\textsubscript{{[}wh{]}}
			[{[}E{]}]
		]
		[TP]
	]
]
\end{forest}
\end{figure}

The structure differs from \figref{dfctreech7} above precisely in that the C head contains an [E] feature and no [fin] feature, so that no finite complementiser is inserted. As seen in \chapref{ch:6}, the same restriction can be observed in lower peripheries in other languages, so the requirement is not even construction-specific. Ultimately, \chapref{ch:6} argued that the [E] feature can be seen as a syntactic object that has its own featural restrictions: it is compatible with [wh] but not with [fin]. If so, the fact that sluicing patterns do not demonstrate Doubly Filled COMP effects follows naturally from the properties of the [E] feature and there is no need to assume designated CP projections, as in \citet{baltin2010}.

On the other hand, \chapref{ch:6} argued that FP projections headed by the [E] feature can affect the clausal spine in that a TP projection that would undoubtedly be present in the non-elliptical counterpart is not generated and the complement is instead a tenseless PredP, rendering a non-isomorphic but recoverable structure (\citealt{vicente2018}). This configuration is subject to certain conditions and it does not rule out the possibility of full TPs either, but the underlying structure of the clause constrains the available readings. Further, as was seen in connection with German comparatives, a caseless subject remnant may not only appear in the default case but it may also get accusative case from the matrix predicate in the morphological component. This leads to surface patterns in languages like German that are unexpected based on the general distribution of accusative case, yet it can be fully explained by a model that treats FPs headed by an [E] feature an integer part of functional left peripheries.

