\chapter{Equatives, comparatives and the marking of polarity} \label{ch:5}
\section{Introduction} \label{sec:5introduction}
%\chaptermark{Equatives, comparatives and polarity}
Building on the feature-based theory put forward in this book so far, this chapter examines comparison constructions, including non-degree equatives (similatives), degree equatives, and comparatives expressing inequality. While these constructions are similar in several respects, they show differences in ways that are slightly unexpected for analyses developed primarily for comparatives expressing inequality. The differences become evident especially when looking at the possible combinations of complementisers and operators at the left periphery of the subordinate clause. The various combinations are naturally relevant for the theory of functional left peripheries because they provide an ideal testing ground for whether designated projections are necessary, as in cartographic approaches, or whether a more minimal CP is favourable. Comparison constructions indeed provide evidence for the existence of multiple CP projections, yet the availability of overt combinations is subject to constraints that cartographic approaches cannot handle in an adequate way. Instead, I will propose that the restrictions and requirements on multiple marking are not tied to designated projections, but follow from the semantic properties of the individual constructions, and are also in interaction with properties of the matrix element.

This chapter is structured as follows. Section \ref{sec:5comparison} discusses basic notions concerning comparison and degree. Section \ref{sec:5polarity} examines the role of polarity regarding the differences between comparatives expressing equality and ones expressing inequality. Section \ref{sec:5grammaticalisation} is dedicated to the grammaticalisation processes in German in this respect and how the relevant constructions can be integrated into the proposed model. Section \ref{sec:5polaritymarking} extends the conclusions to other languages, also beyond Germanic, showing that the asymmetries are not language-specific but stem from differences in the underlying comparative semantics. Section \ref{sec:5hypothetical} applies the analysis to hypothetical comparatives.

\section{Comparison and degree} \label{sec:5comparison}
Embedded degree clauses fall into two major types: degree equatives, also called comparatives expressing equality, as given in (\ref{astall}), and comparatives expressing inequality, as given in (\ref{tallerthan}):

\ea \label{comparison}
\ea Ralph is as tall \textbf{as} Mary is.\label{astall}
\ex Ralph is taller \textbf{than} Mary is.\label{tallerthan}
\z
\z

In (\ref{astall}), the subclause introduced by \textit{as} expresses that the degree to which Mary is tall is the same as to which Ralph is tall (or lower), while in (\ref{tallerthan}), the subclause introduced by \textit{than} expresses that the degree to which Mary is tall is lower than the degree to which Ralph is tall. While the examples in (\ref{comparison}) contain full subclauses, the subclause is often reduced to a single focused remnant, resulting in reduced clauses that can be derived by ellipsis (see \citealt{merchant2009}, \citealt{bhatttakahashi2011}, \citealt{bacskaiatkari2014alh, bacskaiatkari2014diss} among others). The elliptical counterparts of the clauses in (\ref{comparison}) are illustrated in (\ref{comparisonellipsis}) below:

\ea \label{comparisonellipsis}
\ea Ralph is as tall \textbf{as} Mary.\label{asellipt}
\ex Ralph is taller \textbf{than} Mary.\label{thanellipt}
\z
\z

As far as English is concerned, the DPs in (\ref{asellipt}) and (\ref{thanellipt}) are not directly selected by \textit{as} and \textit{than}, but are remnants of clauses that are present in the syntactic derivation. I will return to the issue of ellipsis in comparatives in \chapref{ch:6}; for the time being, suffice it to say that in the comparison constructions discussed in this chapter, the complement of \textit{as} and \textit{than} (and the relevant elements in other languages) is a clause, not simply a phrase.

The comparison constructions presented above are instances of degree comparison: there is one degree expressed in the matrix clause and another one expressed in the subclause. The matrix degree morpheme is \textit{as} in degree equatives and it selects an \textit{as}-clause, while the matrix degree morpheme in degree comparatives is -\textit{er} (or \textit{more}, which is actually a composite of -\textit{er} and \textit{much}; see \citealt{bresnan1973}, \citealt{bacskaiatkari2014diss, bacskaiatkari2018langsci}). However, it is possible to have comparison without degree; consider the following examples:\footnote{Regarding (\ref{differentthan}), it is worth mentioning that the acceptability of \textit{different than} shows some variation. As summarised by \citet[747]{hundt2001}, while \textit{different from} is accepted both in British English and in American English and constitutes the option usually recommended by grammars, \textit{different than} is far more common in American English (see also \citealt[213]{burchfield1996}). Especially if followed by a clause (and not a single DP), however, as in (\ref{differentthan}), it is more expected to occur in British English as well (\citealt[621]{fowlergowers1965}). Note that the availability of a clause following \textit{than} shows that \textit{than} cannot be a preposition (contrary to the assumption of \citealt[747--749]{hundt2001}); the status of \textit{than} as a complementiser will be discussed further in this chapter and in \chapref{ch:6}.}

\ea \label{nondegreecomparison}
\ea Mary is tall, \textbf{as} is her mother. \label{tallas}
\ex Mary is glamorous \textbf{like} a film-star. \label{glamorouslike}
\ex Farmers have other concerns \textbf{than} the farm bill. \label{otherthan}
\ex Life in Italy is different \textbf{than} I expected. \label{differentthan}
\z
\z

In these cases, there is obviously no matrix degree element. The sentences in (\ref{tallas}) and (\ref{glamorouslike}) express merely similarity with respect to the property denoted by the adjective. In (\ref{glamorouslike}), the subclause is introduced by \textit{like} and not by \textit{as}, a further difference from degree equatives. Given the availability of non-degree equatives, \citet[35]{jaeger2018} suggests that comparison constructions can be grouped into three major categories: non-degree equatives, degree equatives, and comparatives. These constitute a markedness hierarchy in this order (non-degree equatives being the least marked). However, constructions like (\ref{otherthan}) and (\ref{differentthan}) indicate that there is in fact a fourth category as well: these are non-degree comparatives expressing difference.\footnote{This is mentioned by \citet[35]{jaeger2018} as well, yet she does not pursue this distinction further, in line with previous analyses such as \citet{thurmair2001}, \citet{hahnemann1999}, \citet{kennedy1999}, \citet{zeilfelder2001habil}.} This category seems not to be productive, as the availability of the \textit{than}-clause is dependent on the presence of a particular element expressing difference in the matrix clause. The word \textit{other} or, at least in American English, the adjective \textit{different} are potential candidates.

Importantly, comparison and degree are not in a one-to-one relationship, and the two aspects rather produce a feature matrix allowing for various patterns. In addition to patterns like (\ref{nondegreecomparison}), the equative relative clauses discussed in \chapref{ch:4} also indicate that comparison is possible without degree. Consider the following example from Old Saxon (\citealt[138]{brandnerbraeuning2013}):

\ea \gll \textbf{sulike} gesidos \textbf{so} he im selbo gecos \label{relosch5}\\
such companions so he him self chose\\
\glt `such companions that he chose for himself'\\(\textit{Heliand} 16.1280)
\z

While the matrix clause contains the element \textit{so} and takes a clause introduced by \textit{so}, corresponding to the regular equative pattern in the language (cf. also \citealt{jaeger2010} for Old High German), the construction is evidently not a degree equative. However, as pointed out in \chapref{ch:4}, its structure bears some similarity to degree equatives as well.

\section{Polarity} \label{sec:5polarity}
As demonstrated already by \citet{seuren1973}, comparatives are negative polarity environments, as shown by the availability of negative polarity items such as \textit{any} and \textit{ever} (see also \chapref{ch:2}). This is demonstrated by (\ref{any}) below (\citealt[531, ex. 10]{seuren1973}):

\ea He solves problems faster than \textbf{any} of my friends \textbf{ever} could. \label{any}
\z

Such elements are licensed in other negative polarity contexts (cf. \citealt{klima1964}) including interrogatives, clausal negation and conditionals, but not in affirmative clauses (\citealt[531, ex. 11]{seuren1973}):

\ea
\ea[*]{\textbf{Any} of my friends could \textbf{ever} solve those problems.}
\ex[]{Could \textbf{any} of my friends \textbf{ever} solve those problems?}
\ex[]{At no time could \textbf{any} of my friends \textbf{ever} solve those problems.}
\ex[]{If \textbf{any} of my friends \textbf{ever} solve those problems, I'll buy you a drink.}
\z
\z

\begin{sloppypar}
Note that the same applies in non-degree comparative clauses as well, as shown by the availability of \textit{lift a finger} in (\ref{finger}) below:
\end{sloppypar}

\ea She would rather leave the party than \textbf{lift a finger} to help us. \label{finger}
\z

Comparative constructions are thus negative polarity environments. Interestingly, English shows a symmetrical pattern regarding degree equatives and comparatives, as demonstrated in (\ref{english}):

\ea \label{english}
\ea Sophia is as nice as \textbf{any} other teacher in the school. \label{asany}
\ex Sophia is nicer than \textbf{any} other teacher in the school. \label{thanany}
\ex Museums are as popular as \textbf{ever} before. \label{asever}
\ex Museums are more popular than \textbf{ever} before. \label{thanever}
\z
\z

German, on the other hand, shows an asymmetry in this respect, as shown in (\ref{german}) below (see also \citealt{hohauszimmermann2021}):\footnote{For the sake of consistency, I gloss \textit{als} as `as' and \textit{wie} as `how', reflecting their etymological counterparts in English. As will be shown later in \sectref{sec:5grammaticalisation}, both of these elements have various functions synchronically and diachronically, so that semantically more accurate translations would not be uniform.}

\ea \label{german}
\ea[*]{\gll Museen sind so beliebt wie \textbf{jemals} zuvor. \label{wiejemals}\\
museums are so popular how ever before\\
\glt `Museums are as popular as ever before.'}
\ex[]{\gll Museen sind beliebter als \textbf{jemals} zuvor. \label{alsjemals}\\
museums are more.popular as ever before\\
\glt `Museums are more popular than ever before.'}
\z
\z

\citet[531--532]{seuren1973} proposes that comparatives have negative polarity because there is some covert negation in the clause, in the sense that a proposition of the form `X is taller than Y' has the underlying semantics `X is tall to an extent to which Y is not'. This is unlikely to be the case (see also the discussion by \citealt{jaeger2018}); among other reasons, (\ref{asany}) and (\ref{asever}) are predicted to be impossible under this analysis, since degree equatives specifically express the equality of the two degrees and should therefore not contain an underlying negation. In other words, while the German pattern in (\ref{german}) fares well with this analysis, the English pattern in (\ref{english}) is highly problematic.

I assume that embedded degree clauses have negative polarity because they are downward entailing environments (see \citealt{ladusaw1979diss}, \citealt{vonstechow1984} and \citealt{heim1985, heim2000} on the connection). Taking the example in (\ref{thanany}), the sentence entails that Sophia is nice to a certain degree, call it \textit{d}, and for all the other teachers it is true that the degree to which they are nice, call it $d'$, is lower. In other words, $d'$ is always lower on a scale than \textit{d} is, while the exact value of \textit{d} is not necessarily known in the context. 

The downward entailing environment is due to the maximality operator. As \citet{hohauszimmermann2021} argue, comparative constructions involve a maximality operator and, in its scope, a comparative operator in the semantics, where\-by neither is tied to a particular syntactic projection or to the notion of degree (that is, they are also present in non-degree equatives). Note also that, according to standard (comparative) semantics, degree equatives and comparatives involve the same maximality operator (see, for instance, \citealt{beck2011}). This presupposes that both degree equatives and comparatives should be downward entailing environments, which would account for the grammaticality of both sentences in (\ref{english}), while the asymmetry in (\ref{german}) does not immediately follow. 

In degree equatives, \textit{d} is the same as or higher than $d'$: in (\ref{astall}), the degree to which Ralph is tall is the same as or higher than the degree to which Mary is tall. In comparatives, \textit{d} is higher than $d'$: in (\ref{tallerthan}), the degree to which Ralph is tall is higher than the degree to which Mary is tall. The degree \textit{d} associated with the matrix degree element in degree equatives (\textit{so}/\textit{as} in German/English) is thus the maximum for the value of $d'$. Given this relation, the matrix degree element can also lexicalise the maximality operator: alternatively, the maximality operator is lexicalised lower, that is, by the equative complementiser. Naturally, there can be only a single maximality operator in a single construction, and it depends on the specific language how this property is set. The variation regarding which element lexicalises the maximality operator gives us the difference between English and German, as demonstrated by (\ref{asever}) and (\ref{wiejemals}). In English, the equative complementiser \textit{as} lexicalises the maximality operator, and it licenses the negative polarity item in the same clause. In German, the matrix element \textit{so} lexicalises the maximality operator and as such it cannot license the negative polarity item across the clausal boundary.

\begin{sloppypar}
By contrast, the matrix degree element in comparatives (-\textit{er} in German\slash English) expresses merely a higher degree than $d'$ but does not set the maximal value of $d'$. This property has to be expressed by a lower syntactic projection, which is the comparative complementiser. Importantly, the maximum value of $d'$ is itself not equal to \textit{d}, and this property is reflected by the relevant element in the subclause. This suggests that the encoding of inequality is essentially compositional. Note that while the matrix degree determines the choice between equative \textit{as}/\textit{wie} and comparative \textit{than}/\textit{als}, there are no subtypes in comparative complements according to superiority/inferiority.
\end{sloppypar}

The German data in (\ref{german}) indicate that the difference between comparatives and degree equatives has far-reaching syntactic consequences. I suggest that further differences can also be traced to the properties described above. The maximality operator can be expressed by the matrix element in degree equatives but not in comparatives. As a consequence, the CP in the degree equative clause is associated with equality by default, while in comparative clauses it is associated with inequality. The property of equality/inequality is inherited from the matrix degree element. Complementisers differ in terms of their feature specification: some of them are specified either as marking equality, [EQ], or as inequality, [INEQ], while others are unspecified.

This property of inequality is similar to negation and expletive negation in that it has to be lexicalised by a phonologically visible element (see \citealt{dryer2013} on the necessity of lexicalising negation cross-linguistically). As there is no negative operator in the comparative subclause (there being no true negation involved when the inequality of the two degrees is expressed), this is carried out by the comparative complementiser. This kind of inequality marking (referred to as ``degree negation'' descriptively by \citealt{bacskaiatkari2016alh}) encoded by the complementiser is reflected by the fact that many languages contain a negative-like element in the complementiser (see \citealt{bacskaiatkari2016alh}). Complementisers that inherently contain a negative-like element are specific for comparative clauses. However, it is also possible to have complementisers in the topmost C head that are lexically specified as comparative, [compr], but do not express inequality. These complementisers can be shared between degree equatives and comparatives, as is the case in the history of German \textit{als} and \textit{wie} (see \citealt{jaeger2018}).

The differences in the properties to be encoded have consequences for the structure of the CP-periphery in the subclause, as already discussed in \chapref{ch:2}. In degree equatives, there is no degree inequality to be expressed, and as the matrix equative element can take up the function of lexicalising the maximality operator, it is possible to have a single CP in the subordinate clause. At the same time, a double CP is possible if the maximality operator is lexicalised by a complementiser above the CP containing the comparative operator. In comparatives, degree inequality has to be lexicalised and the matrix degree element cannot take up the function of lexicalising the maximality operator, meaning that a double CP is necessary in the subclause, whereby the higher CP is responsible for encoding inequality and the lower CP hosts the comparative operator (overt or covert). The head of this CP is either a comparative complementiser or a more general relative complementiser (see \citealt{bacskaiatkari2016alh}). As also shown in \chapref{ch:2}, comparative constructions differ from embedded interrogatives and relative clauses regarding the nature of doubling patterns. While doubling in embedded interrogatives and in relative clauses follows a classical Doubly Filled COMP pattern and can be analysed as involving a single CP (see \chapref{ch:3} and \chapref{ch:4}), doubling in comparatives always represents CP-doubling. Equatives may also show Doubly Filled COMP patterns but they may also involve doubling. All the differences ultimately follow from the semantic properties of the respective elements.

\section{Grammaticalisation in German} \label{sec:5grammaticalisation}
\subsection{Dialectal variation} \label{sec:5dialectal}
The considerations presented in \sectref{sec:5polarity} can be successfully applied to the diachronic and dialectal variation observed in German (see also \citealt{bacskaiatkari2021oup}). In the standard variety, degree equatives are introduced by \textit{wie} and comparatives are introduced by \textit{als}. Consider the examples in (\ref{standard}):

\ea \label{standard}
\ea \gll Ralf ist so groß \textbf{wie} Peter.\\
Ralph is so tall how Peter\\
\glt `Ralph is as tall as Peter.'
\ex \gll Ralf ist größer \textbf{als} Peter.\\
Ralph is taller as Peter\\
\glt `Ralph is taller than Peter.
\z
\z

By contrast, regional dialects show the availability of \textit{als}, \textit{wie} and the combination \textit{als wie} in both constructions, as shown by \citet{jaeger2018} in detail (see also \citealt{eggs2006}; \citealt{lipold1983}; \citealt{weise1918}). The examples in (\ref{germanequatives}) show the dialectal options for degree equatives: (\ref{lowgerman}) is from Low German, and (\ref{bavarianalswie}) and (\ref{bavarianwie}) are from Bavarian. 

\ea \label{germanequatives}
\ea \gll buten so still \textbf{as} binnen \label{lowgerman}\\
outside so silent as inside\\
\glt `outside as silent as inside'\\(\citealt[327, ex. 530a]{jaeger2018}, citing \citealt[170]{weise1918})
\ex \gll Dei Schweinsbraan schmeggd genau a so fad \textbf{ais} \textbf{wia} dei Schbinad \label{bavarianalswie}\\
your roast.pork tastes exactly \textsc{prt} so stale as how your spinach\\
\glt `Your roast pork tastes just as stale as your spinach.'\\(\citealt[327, ex. 531a]{jaeger2018}, citing \citealt[171]{merkle1975})
\ex \gll A Flugzeig is genauso deia \textbf{wiar} a Loggomodiv. \label{bavarianwie}\\
an aeroplane is just.as expensive how a locomotive\\
\glt `An aeroplane is just as expensive as a locomotive.'\\(\citealt[326, ex. 529a]{jaeger2018}, citing \citealt[171]{merkle1975})
\z
\z

As discussed by \citet{jaeger2018}, the pattern in (\ref{lowgerman}) has largely disappeared across dialects and it is attested only in traditional North German (Low German) dialects. The pattern given in (\ref{bavarianalswie}) is attested in dialects to the south of the Berlin--Braunschweig line, including southern dialect areas like Bavarian, Alemannic and Hessian, as well as mid-central varieties like Upper Saxon and Thuringian. The pattern in (\ref{bavarianwie}) is attested in the same areas as (\ref{bavarianalswie}), as well as in northern varieties (essentially in all regional dialects), and it corresponds to the standard pattern.

The examples in (\ref{germancomparatives}) show the dialectal options for comparatives: (\ref{lowgermancomp}) is from Low German, (\ref{uppersaxon}) is from Upper Saxon, and (\ref{thuringian}) is from Thuringian.

\ea \label{germancomparatives}
\ea \gll De Buu duur länger, \textbf{as} de Meister seggt harr. \label{lowgermancomp}\\
the construction lasts longer as the master said.\textsc{ptcp} has\\
\glt `The construction lasts longer than the master said.'\\(\citealt[291, ex. 492b]{jaeger2018}, citing \citealt[300]{lindowmoehnniebaumstellmachertaubkenwirrer1998})
\ex \gll Ich bin gresser \textbf{als} \textbf{wie} du \label{uppersaxon}\\
I am taller as how you\\
\glt `I am taller than you.'\\(\citealt[292, ex. 494b]{jaeger2018}, citing \citealt[174]{weise1918})
\ex \gll Da kommt de Brihe teirer \textbf{wie}'s Flääsch \label{thuringian}\\
there comes the broth more.expensive hpw.the meat\\
\glt `The broth is more expensive than the meat', fig. `it is not worth the effort'\\(\citealt[291, ex. 493]{jaeger2018}; Rudolstadt, ThWB 973)
\z
\z

As discussed by \citet{jaeger2018}, the pattern in (\ref{lowgermancomp}) is identical to the standard pattern. It is the only pattern attested in the dialect areas north of the Berlin--Braunschweig line, but it also occurs in the rest of the regional dialects. The patterns given in (\ref{uppersaxon}) and (\ref{thuringian}) are both attested in dialects to the south of the Berlin--Braunschweig line, including southern dialect areas like Bavarian, Alemannic and Hessian, as well as mid-central varieties like Upper Saxon and Thuringian.

\subsection{The data} \label{sec:5data}
The complementisers \textit{als} and \textit{wie} represent two options that differ diachronically, too: \textit{als} is the older form and \textit{wie} is more innovative (cf. \citealt{jaeger2010, jaeger2018}). Naturally, there are considerable overlaps (see \citealt[288--358]{jaeger2018}), yet it seems clear that Southern dialects are more innovative in allowing \textit{wie} in both constructions, while Northern dialects are more conservative and some of them still preserve the older equative pattern with \textit{als}. On the other hand, degree equatives are more innovative than comparatives, given that the newer pattern (with \textit{wie}) is well-established in most dialects and counts as the standard, while \textit{wie} in comparatives is non-standard and does not appear in all dialects. This raises the question why degree equatives are more innovative than comparatives in German and, if applicable, in other languages.

The asymmetry between the two constructions can be detected in earlier periods of German as well. The canonical West Germanic pattern involved \textit{so} in degree equatives and \textit{than}\footnote{English \textit{than}/\textit{then}, German \textit{dann}/\textit{denn} and Dutch \textit{dan} are etymologically the same (see \citealt{rutten2012} for West Germanic and Jacob and Wilhelm Grimm's \textit{Deutsches Wörterbuch}).} in comparatives. The complementiser \textit{so} was reinforced by \textit{all} into \textit{als(o)} in all the three West Germanic languages discussed here. Consider the Dutch pattern in (\ref{dutchpattern}):

\ea \label{dutchpattern}
\ea \gll Sanne is net zo groot \textbf{als} ik.\\
Sanne is just so tall as I\\
\glt `Sanne is as tall as I am.'
\ex \gll Sanne is groter \textbf{dan} ik.\\
Sanne is taller than I\\
\glt `Sanne is taller than I am.'
\z
\z

Dutch \textit{als} can be derived from \textit{also} (\textit{al}\,+\,\textit{so}). Note that the matrix degree element \textit{zo} in Dutch is not affected (of course, \textit{even} `just as' may also be used). In comparatives, (Standard) Dutch retains the original West Germanic pattern.

English \textit{as} derives from \textit{eallswa} (\textit{all}\,+\,\textit{so}). The forms \textit{swelce} (\textit{swilce}, \textit{such}) and \textit{so} (\textit{swa}) are also possible equivalents historically in \textit{as}-constructions (see \citealt[315--317]{kortmann1997}; see also \citealt[312--314]{lopezcousomendeznaya2014} and references there). This reinforced element occurs both as the matrix degree element and as the degree equative complementiser:

\ea
\ea Ralph is as smart \textbf{as} Peter.
\ex Ralph is smarter \textbf{than} Peter.
\z
\z

Just like Standard Dutch, English retains \textit{than} in comparatives.

German \textit{als} has an identical etymology. It derives from Old High German \textit{also} (\textit{all}\,+\,\textit{so}), whereby various forms of \textit{so} are possible historically in \textit{as}-con\-struc\-tions (see \citealt{jaeger2010}). Three examples are given from Old High German in (\ref{ohgch5}),\footnote{Notice that the Old High German sequence \textit{s\'o manag so} does not mirror the Latin original, which contains only \textit{quot}.} from Old Saxon in (\ref{os}),\footnote{The example in (\ref{ohgch5}) is taken from \citet[64, ex. 65]{jaeger2018} and the examples in (\ref{vogelweide}) is taken from \citet[22, ex. 12]{eggs2006}. The Old Saxon data are taken from the \textit{DDD Referenzkorpus Altdeutsch}, available at: https://korpling.german.hu-berlin.de/annis3/ddd/.} and from Middle High German in (\ref{vogelweide}):

\ea
\ea \textit{et dabit illi //quot habet necessarios}\\
\gll inti gibit imo // só manag \textbf{so} her bitharf. \label{ohgch5}\\
and give him.\textsc{dat} {} so much so he needs\\
\glt `and give him as much as he needs' (\textit{Tatian} 72, 28--29)\\
\ex \gll sô hôho afhuoƀi, \textbf{so} duot himilrîki \label{os}\\
so high elevate so does heaven\\
\glt `raise as high up as heaven does' (\textit{Heliand} 32.2626)
\ex \gll waer er s\^{o} milt \textbf{als} lanc, er hete tugende vil besezzen \label{vogelweide}\\
be.\textsc{cond.3sg} he so generous as tall he have.\textsc{cond.3sg} virtues many possess.\textsc{inf}\\
\glt `If he were as generous as he is tall, he would have had many virtues.'\\(Walther von der Vogelweide, \textit{Werke} Bd. 1, 118f)
\z
\z

Just as with degree equatives, comparatives show the regular West Germanic configuration in earlier periods of German. The example in (\ref{ohgcomp}) shows the relevant pattern for Old High German (\citealt[38, ex. 34]{jaeger2018}),\footnote{Notice also here that the Old High German comparative construction involving \textit{thanne} (which introduces a finite clause) does not mirror the Latin original, which contains a phrasal comparative involving a DP in the ablative case (\textit{illis}).} and the one in (\ref{oscomp}) for Old Saxon:

\ea \label{denn}
\ea \textit{Nonne vos magis plures estis illis?}\\
\gll Eno ni birut ir furirun \textbf{thanne} sie sín? \label{ohgcomp}\\
well not are.\textsc{2pl} you.\textsc{pl} greater than they are.\textsc{3pl}\\
\glt `Are you not much better than they are?' (\textit{Tatian} 70, 17)\\
\ex \gll that he sî betara \textbf{than} uui \label{oscomp}\\
that he is.\textsc{sbvj} better than we\\
\glt `that he is better than we are' (\textit{Heliand} 3.212)
\z
\z

As described by \citet[471--475]{jaeger2010}, Middle High German was mostly like Old High German, and the changes affecting the complementisers can be observed from Early New High German onwards, especially from the second half of the 16th century. In degree equatives, \textit{wie} came to replace \textit{als}: in this process, the incentive factor is the availability of \textit{wie} as a degree operator in another context (interrogatives) anyway.\footnote{Overt comparative operators tend to be surface-identical to their interrogative degree operator counterparts cross-linguistically (see \citealt[90--100]{bacskaiatkari2018langsci}).} In comparatives, \textit{als} came to replace \textit{denn}: in this process, analogy plays a crucial role, as \textit{als} was introduced into comparatives by way of analogical extension from degree equatives. This latter process is also attested in Dutch (see \citealt[377]{jaeger2018}). While Middle Dutch was much like Middle High German in not changing the original West Germanic pattern (cf. \citealt{postma2006}), \textit{als} came to replace \textit{dan} from especially the 16th century onwards (\citealt{vanderhorst2008deel1}, \citealt{postma2006}). This development was largely reversed from the 18th century onwards, mostly due to prescriptive pressure (\citealt{vanderhorst2008deel1}, \citealt{hubersdehoop2013}).

Modern German is innovative in at least two respects: it shows the original equative complementiser in comparatives, and the original operator \textit{wie} in equatives, dialectally also in comparatives. The changes can be schematically represented as follows (see \citealt[364, ex. 596]{jaeger2018} for a detailed summary):

\ea 
\ea als(/so)  $\rightarrow$ wie 
\ex dann/denn $\rightarrow$ als $\rightarrow$ wie
\z
\z

At any rate, the changes that ultimately affected comparatives in an analogical way started in degree equatives. These changes are referred to as the comparative cycle by \citet{jaeger2018}, referring to the observation that the particles introducing non-degree equatives can be extended to degree equatives and finally to comparatives, innovative patterns starting in non-degree equatives by default.\footnote{Just as with any other cyclic change, the question arises of what ultimately motivates the observed processes. As pointed out by \citet[398--400]{jaeger2018}, there are various approaches in the literature: for instance, it is often assumed that the phonological reduction of one element ultimately fosters the appearance of novel elements that reinforce the same meaning (as is generally assumed for the Jespersen cycle). On the other hand, the semantic bleaching of one element in itself may also foster the introduction of novel markers (see \citealt{willislucasbreitbarth2013} and \citealt{chatzopoulou2015} for such an analysis of the Jespersen cycle). In the case of comparatives, both are plausible explanations. The matrix equative marker is originally an element that reinforces similarity and is phonologically non-reduced. However, with its phonological reduction and semantic bleaching, it ceases to be a sufficient cue for the language learner to analyse it in its original function. Phonological and semantic reduction are paired with the loss of features, which is thought to be a general motivation behind grammaticalisation processes (see \citealt{vangelderen2004, vangelderen2008, vangelderen2009, vangelderen2011}, \citealt[428--441]{jaeger2018}).} This pattern can be observed across languages (\citealt[370--397]{jaeger2018}). According to \citet{jaeger2018}, this is related to the markedness hierarchy holding between the constructions in question: non-degree equatives are the least marked, while comparatives (expressing degree and inequality) are the most marked.

As discussed in \chapref{ch:4}, the element \textit{so} in German was also extended to ordinary relative clauses, via the intermediate stage of equative relative clauses. The analogical extension of an original non-degree equative complementiser (similative marker) into other clause types can thus proceed in two directions, both exemplified by German \textit{(al)so}:

\ea REL $\leftarrow$ EQUAT-REL $\leftarrow$ NON-DEG EQUAT $\rightarrow$ DEG-EQUAT $\rightarrow$ COMPR \label{schema}
\z

The schema in (\ref{schema}) is in fact exemplified by the elements \textit{wie}/\textit{wo} as well, as shown by \citet{brandnerbraeuning2013} in detail (see the discussion in \chapref{ch:4}).

It is not difficult to see how both processes can be described along the lines of grammaticalisation, accompanied by the semantic bleaching of the original similative marker. In non-degree equatives, the complementiser still expresses similarity in a transparent way. This is illustrated for English and German in (\ref{opencomp}) below:

\ea \label{opencomp}
\ea Peter is \textbf{like} Mary.
\ex \gll Peter ist so \textbf{wie} Maria.\\
Peter is so how Mary\\
\glt `Peter is like Mary.'
\z
\z

English uses \textit{like} in these cases rather than \textit{as}, though \textit{as} can appear especially in cases that are termed as ``comparison of factivity'' (\textit{Faktizitätsvergleich}) by \citet{jaeger2010}; see also \citet[165--182]{thurmair2001}. Consider the examples in (\ref{tallgross}):

\ea \label{tallgross}
\ea Peter is tall, \textbf{as} is Mary.
\ex \gll Peter ist groß, \textbf{wie} (auch) Maria.\\
Peter is tall how \phantom{[}too Mary\\
\glt `Peter is tall, as is Mary.'
\z
\z

This type is similar to additive coordination in that two facts (i.e. that Peter is tall and that Mary is tall) are compared. The examples in (\ref{opencomp}) represent canonical similative constructions that express similarity along the lines of a given property, without specifying any degree.

Degree equatives are more grammaticalised in that similarity is reduced to the equality of two degrees on a scale. Finally, the complementiser in comparatives proper does not encode equality: on the contrary, it merely encodes comparison, which in this case is paired up with degree inequality.

In the same vein, the similarity component is partially retained in equative relative clauses but not in ordinary relatives (see \chapref{ch:4}). Similative markers may thus grammaticalise in two distinct directions. As discussed in \chapref{ch:2} and in \chapref{ch:4}, a single CP can be sufficient in non-degree equatives, while the presence of the maximality operator always results in a double CP in comparatives proper and in most cases also in degree equatives. One piece of evidence was provided by doubling patterns involving the sequence \textit{als wie}, illustrated in (\ref{alswiedistr}) below for Upper Saxonian (\citealt[292, ex. 494b]{jaeger2018}, citing \citealt[174]{weise1918}) and Swiss German (\citealt[659]{friedli2012diss}), respectively:

\ea \label{alswiedistr}
\ea \gll Ich bin gresser \textbf{als} \textbf{wie} du.\\
I am taller as how you.\textsc{nom}\\
\glt `I am taller than you.'
\ex \gll S git ekäi esoo vercheerti Lüüt \textbf{as} \textbf{wi(e)} di gleerte\\
it gives none so crazy people as how the.\textsc{pl} learned.\textsc{pl}\\
\glt `There are no people as crazy as the learned.'
\z
\z

The configuration thus involves two functional heads. As discussed in \chapref{ch:2}, while it may at first be tempting to assume that \textit{wie} is an operator moving to [Spec,CP], the behaviour of this element with overt adjectives suggests that this cannot be the case.\footnote{Consider the following data in (\ref{germanzerogiven}) and (\ref{germanzerocontrast}):

\ea [?]{\gll Maria ist größer	als	(\% wie) Johann \textbf{groß} ist. \label{germanzerogiven}\\
Mary is	taller as	\phantom{(\%}how John tall	is\\
\glt `Mary is taller than John.'}
\ex	[]{\gll Der	Tisch	ist	länger als	(\% wie) das	Büro \textbf{breit} ist. \label{germanzerocontrast}\\
the.\textsc{m} table is	longer as	\phantom{(\%}how the.\textsc{n} office	wide is\\
\glt `The table is longer than the office is wide.'}
\z

The data show that the gradable adjective can be realised overtly in German (the markedness of (\ref{germanzerogiven}) is due to redundancy), at least as long as the AP remains in its base position. However, fronting the AP is not possible, as shown in (\ref{alswiefrontedgiven}) and (\ref{alswiefrontedcontrast}):

\ea	[*]{\gll Maria	ist	größer	als	\textbf{wie}	\textbf{groß} Johann	ist. \label{alswiefrontedgiven}\\
Mary is	taller	than	how	tall	John	is\\
\glt `Mary is taller than John.'}
\ex	[*]{\gll Der	Tisch	ist	länger als	\textbf{wie}	\textbf{breit}	das Büro ist. \label{alswiefrontedcontrast}\\
the.\textsc{m}	table	is	longer than	how	wide	the.\textsc{n} office is\\ 
\glt `The table is longer than the office is wide.'}
\z

The data indicate that \textit{wie} is not a comparative operator in German: if it were an operator, it would allow the pied-piping of the AP (just as its main clausal operator counterparts). See \citet[92--100]{bacskaiatkari2018langsci} for more discussion on this cross-linguistic restriction.} In addition, \citet[467--470]{jaeger2018} shows that \textit{wie} does not pattern with ordinary operator elements with respect to ellipsis and it occurs as a relative complementiser anyway dialectally (as also pointed out by \citealt{brandnerbraeuning2013}, see \chapref{ch:4}). Third, \textit{wie} may be subject to complementiser inflection in Bavarian, as shown in (\ref{equatinfl}) below for degree equatives (\citealt[328, ex. 537a]{jaeger2018}, citing Helmut Weiß p.c.):

\ea \gll D'Resl is genau so groass \textbf{ois} \textbf{wiest} du bisd. \label{equatinfl}\\
the.Resel is just so tall as how.\textsc{2sg} you are.\textsc{2sg}\\
\glt `Resel is just as tall as you are.'
\z

In Bavarian, the same applies in comparatives, as shown in (\ref{compinfl}) below (\citealt[293, ex. 502a]{jaeger2018}, citing \citealt[60]{fuss2014}):

\ea \gll D'Resl is gresser \textbf{ois} \textbf{wiest} du bisd. \label{compinfl}\\
the.Resel is taller as how.\textsc{2sg} you are.\textsc{2sg}\\
\glt `Resel is taller than you are.'
\z

Such agreement morphemes are restricted to complementisers (they do not occur with \textit{wh}-phrases in the specifier of the CP) and also to second person forms (\citealt{fuss2004}). At any rate, patterns like (\ref{equatinfl}) and (\ref{compinfl}) clearly indicate that \textit{wie} has head status.

\subsection{Analysis} \label{sec:5analysis}
In \chapref{ch:2}, I proposed the doubling structure shown in \figref{treealswie}.

\begin{figure} 
\caption{The structure of \textit{als wie}} \label{treealswie}
\begin{forest} baseline, qtree
[CP
	[C$'$
		[C\textsubscript{{[}compr{]}}
			[als\textsubscript{{[}compr{]},{[}MAX{]}}]
		]
		[CP
			[OP]
			[C$'$ [C\textsubscript{{[}rel{]},{[}compr{]},{[}fin{]}} [wie\textsubscript{{[}compr{]}}]] [TP]]
		]
	]
]
\end{forest}
\end{figure}

In this configuration, the higher C head encodes the maximality operator, which has the comparative operator (indicated as OP, referring to a covert operator) in its scope. As discussed in \chapref{ch:3}, semantic operators may or may not show operator properties like phrase movement in terms of their syntax. In the particular case, one might wonder whether the complementiser \textit{wie} would not be an optimal candidate for a semantic operator realised as a head in syntax, just like \textit{als} is a semantic operator and at the same time a complementiser. However, there is ample evidence from the literature on comparatives that the comparative operator actually undergoes movement (as evidenced by island violations; see e.g. \citealt{kennedy2002}), and as this element cannot be \textit{wie} (for the reasons mentioned above), assuming a covert operator is necessary. Note that the covert operator and its (non-)extractability from the degree expression has a bearing on the realisation of other elements in the clause, especially in the phenomenon traditionally referred to as ``Comparative Deletion'' (see \citealt[57--106]{bacskaiatkari2018langsci}).

As mentioned before, the maximality operator is always expressed by the (higher) C head in comparatives, necessarily leading to a double CP, while it may in principle be realised by the matrix degree element in equatives. The representation in \figref{treealswie} shows a doubling configuration and does not explicitly indicate whether the clause is degree equative or comparative proper. In fact, the idea is that in dialects allowing \textit{als wie} in both configurations (as is the case in Hessian; see \citealt[288--346]{jaeger2018} for a detailed discussion of the present-day dialect data), the complementisers are not specified either as [EQ] or as [INEQ], precisely because they can appear in either configuration. In these cases, then, there is nothing in the subclause itself that would determine which subtype of degree comparison is involved: it is only the matrix degree element (which takes the comparative CP as its complement, see \chapref{ch:4}) that disambiguates. Note that this is in fact expected in the present framework: not all properties related to clause typing are actually overtly encoded in a clause, as certain matrix elements can disambiguate as well. The same was seen in the case of embedded polar questions, which are formally specified as [Q] and may thus in principle be formally identical to conditional clauses (see \chapref{ch:3}).

In the earliest stage involving the complementiser \textit{denn} (in its various forms) in comparatives and the complementiser \textit{(al)so} (again attested in various forms) in equatives, we can assume specification for [EQ] and [INEQ]. The structure for comparatives is given in \figref{treecompdenn}.

\vfill
\begin{figure}[H]
\caption{Comparatives in the earliest stage} \label{treecompdenn}
\begin{forest} baseline, qtree
[CP
	[C$'$
		[C\textsubscript{{[}compr{]}}
			[denn\textsubscript{{[}compr{]},{[}MAX{]},{[}INEQ{]}}]
		]
		[CP
			[OP]
			[C$'$ [C\textsubscript{{[}rel{]},{[}compr{]},{[}fin{]}}] [TP]]
		]
	]
]
\end{forest}
\end{figure}
\vfill\pagebreak

What this says in addition to \figref{treealswie} is that the complementiser encodes inequality (or difference) and is thus not compatible with a matrix degree element encoding equality (or similarity). The matrix head (identified as Compr in \chapref{ch:4}) imposes selectional restrictions on the head of its complement. Note, however, that the property [INEQ] is more than merely a reflex of the matrix degree element: such subclauses can also occur without a matrix degree element in non-degree comparatives, as demonstrated by the example in (\ref{denne}) below (\citealt[112, ex. 142]{jaeger2018}):

\ea \gll ab' nach m\'iner eigennen forme enb\'in ich n\'iene \textbf{denne} alhie \label{denne}\\
but after my own form am I nowhere than here\\
\glt `but, according to my form, I am nowhere else but here'\\(Nikolaus von Strassburg, \textit{Predigten} 37vb, 12--14)
\z

This indicates that [INEQ] is truly a lexical property of the complementiser.

The structure of degree equatives in the same period is given in \figref{treealsdouble}.

\begin{figure} 
\caption{Degree equatives in the earliest stage} \label{treealsdouble}
\begin{forest} baseline, qtree
[CP
	[C$'$
		[C\textsubscript{{[}compr{]}}
			[al(so)\textsubscript{{[}compr{]},{[}MAX{]},{[}EQ{]}}]
		]
		[CP
			[OP]
			[C$'$ [C\textsubscript{{[}rel{]},{[}compr{]},{[}fin{]}}] [TP]]
		]
	]
]
\end{forest}
\end{figure}

This construction does not differ significantly from \figref{treecompdenn}: the only crucial difference is that the complementiser is specified as [EQ]. The degree equative complementiser was \textit{so} in Old High German and \textit{als} in Middle High German and Early New High German (as shown by \citealt{jaeger2018}, the element \textit{al} was reanalysed as part of the complementiser from the original matrix degree element, not shown here), and partly also in New High German (17th and 18th centuries). In these periods, there is no doubling of the form \textit{als wie} (see \citealt[360--361]{jaeger2018}) or an overt comparative operator following \textit{al(so)}: in principle, it is perfectly possible that the maximality operator was encoded by the matrix degree element, leading to a single CP in the subclause in the way attested in non-degree equatives in Old High German (see \chapref{ch:2}). This would give us the structure in \figref{treealssingle}.

\begin{figure}
\caption{The single CP in similatives} 
\label{treealssingle}
\begin{forest} baseline, qtree
[CP
	[OP]
	[C$'$
		[C\textsubscript{{[}compr{]},{[}rel{]}}
			[al(so)\textsubscript{{[}compr{]},{[}EQ{]},{[}rel{]}}]
		]
		[TP]
	]
]
\end{forest}
\end{figure}

While the split between the comparative and the degree equative complementiser held for Old High German and continued to be the case in Middle High German and in Early New High German as well, the main pattern in comparatives came to be \textit{als} in New High German, leading to a unified complementiser \textit{als} in both kinds of degree clauses in the 17th and 18th centuries. The complementiser \textit{als} started to appear in comparatives from the second half of the 16th century onwards and continued to be a slowly increasing minority pattern in Early New High German (\citealt[153--167]{jaeger2018}). Given that a single CP is not available in comparatives, we can suppose that a construction like \figref{treealsdouble} was valid in equatives by the time it started to be analogically extended to comparatives. In other words, while a construction like \figref{treealssingle} is not to be excluded for Old High German and Middle High German, the one in \figref{treealsdouble} is exclusively used from Early New High German onwards. This coincides with the proposal of \citet[448--467]{jaeger2018}, who argues for \textit{als} occupying a higher position from the second half of the 16th century onwards on independent grounds. Her arguments include the availability of lower elements such as \textit{was} `what' and \textit{dass} `that' in the lower position (I will return to this issue later in \sectref{sec:5polaritymarking}) as well as its relative position in hypothetical comparatives (as I will show later, this assumption is partly problematic, though).

In the 17th and 18th centuries, then, \textit{als} was used as a unified complementiser for both kinds of degree clauses, without being specified as [EQ] any longer. This gives us the representation in \figref{treeals}.

\begin{figure}
\caption{The position of \textit{als} as a unified complementiser} \label{treeals}
\begin{forest} baseline, qtree
[CP
	[C$'$
		[C\textsubscript{{[}compr{]}}
			[al(so)\textsubscript{{[}compr{]},{[}MAX{]}}]
		]
		[CP
			[OP]
			[C$'$ [C\textsubscript{{[}rel{]},{[}compr{]},{[}fin{]}}] [TP]]
		]
	]
]
\end{forest}
\end{figure}

In this case, \textit{als} lexicalises the maximality operator, and the interpretation of the clause as degree equative or comparative is contingent upon the matrix degree element.

As a minority pattern, the use of \textit{wie} in degree equatives is attested already in the 17th and 18th centuries, as extended from non-degree equatives, where it constituted the majority pattern (\citealt[243--254]{jaeger2018}), and this came to be the predominant pattern in the 19th century. This leads to a split between degree equatives using \textit{wie} and comparatives using \textit{als}, again leading to the specification of these elements as [EQ] and [INEQ], respectively. This is the pattern attested in the standard language, as in (\ref{standard}) above.

Thus, comparatives in Standard German have the representation given in \figref{treealsstandard}.

\begin{figure} 
\caption{Comparatives in Standard German} \label{treealsstandard}
\begin{forest} baseline, qtree
[CP
	[C$'$
		[C\textsubscript{{[}compr{]}}
			[als\textsubscript{{[}compr{]},{[}MAX{]},{[}INEQ{]}}]
		]
		[CP
			[OP]
			[C$'$ [C\textsubscript{{[}rel{]},{[}compr{]},{[}fin{]}}] [TP]]
		]
	]
]
\end{forest}
\end{figure}

The pattern is thus identical to the one reconstructed for earlier stages of German in \figref{treecompdenn} in that there is an [INEQ] specification, unlike in \figref{treeals}. The changes in the feature specification of \textit{als} can be conceived of in the following way:

\ea als\textsubscript{{[}compr{]},{[}EQ{]}} $\rightarrow$ als\textsubscript{{[}compr{]}} $\rightarrow$ als\textsubscript{{[}compr{]},{[}INEQ{]}}
\z

As indicated, the analogical extension of the original equative complementiser is subject to the loss of the lexical feature [EQ], which is not compatible with comparatives expressing degree inequality. If a new equative complementiser successfully wins the competition against the uniform complementiser, the uniform complementiser is reinterpreted as specified for [INEQ]. This featural enrichment follows from changes in the paradigm, i.e. it follows from the introduction of a new complementiser into the system and not from any syntactic change internal to degree comparatives.

Let us now turn to degree equatives in Modern Standard German. These involve the complementiser \textit{wie}, for which I assume the structure given in \figref{treewie}.

\begin{figure} 
\caption{Degree equatives in Standard German} \label{treewie}
\begin{forest} baseline, qtree
[CP
	[OP]
	[C$'$
		[C\textsubscript{{[}compr{]},{[}rel{]}}
			[wie\textsubscript{{[}compr{]},{[}EQ{]},{[}rel{]}}]
		]
		[TP]
	]
]
\end{forest}
\end{figure}

As indicated, there is only a single CP headed by \textit{wie}, specified for [EQ]; this projection also hosts the comparative operator. Crucially, the maximality operator is lexicalised by another element (that is, by the matrix degree element). This makes two predictions. On the one hand, \textit{wie} can appear as a second element in doubling patterns if there is an appropriate higher complementiser. This prediction is borne out since, as we have seen, the combination \textit{als wie} is attested in dialects of German, whereby \textit{als} was introduced later in addition to the already established complementiser \textit{wie} (\citealt{jaeger2018}). On the other hand, since the maximality operator is not present in the subclause itself, the clause is not a downward entailing environment and negative polarity items should not be available. This prediction is again borne out, as was discussed in \sectref{sec:5polarity} above, leading to the asymmetry between degree equatives and comparatives observed in (\ref{german}).

Table \ref{tablealswiesum} summarises the feature specifications of degree equatives and comparatives from Old High German to Modern Standard German. As can be seen, the features of comparatives are less flexible than of equatives, which show more variation over time.

\begin{table}
\begin{tabular}{lcc}
\lsptoprule
Period & Degree equatives & Degree comparatives \\\midrule
{} & \textit{al(so)}\textsubscript{{[}compr{]},{[}MAX{]},{[}EQ{]}} & \textit{denn}\textsubscript{{[}compr{]},{[}MAX{]},{[}INEQ{]}}\\
OHG -- ENHG & + & +\\
{} & $\emptyset$\textsubscript{{[}rel{]},{[}compr{]},{[}fin{]}} & $\emptyset$\textsubscript{{[}rel{]},{[}compr{]},{[}fin{]}}\\
\midrule
{} & \multicolumn{2}{c}{\textit{al(so)}\textsubscript{{[}compr{]},{[}MAX{]}}}\\
17th and 18th centuries & \multicolumn{2}{c}{+}\\
{} & \multicolumn{2}{c}{$\emptyset$\textsubscript{{[}rel{]},{[}compr{]},{[}fin{]}}}\\
\midrule
{} & {} & \textit{als}\textsubscript{{[}compr{]},{[}MAX{]},{[}INEQ{]}}\\
Mod. Standard German & \textit{wie}\textsubscript{{[}compr{]},{[}EQ{]},{[}rel{]}} & +\\
{} & {} & $\emptyset$\textsubscript{{[}rel{]},{[}compr{]},{[}fin{]}}\\
\lspbottomrule
\end{tabular}
\caption{The feature specification of complementisers}
\label{tablealswiesum}
\end{table}

\subsection{Discussion and predictions} \label{sec:5discussion}
In this respect, it is worth pointing out that the proposal differs from that of \citet{jaeger2018} crucially. \citet[448--482]{jaeger2018}, just like previously \citet{jaeger2010}, assumes a categorial difference between two functional projections: a lower CP and a higher ConjP (Conjunction Phrase). This not only serves to avoid a double CP (which in itself has no theoretical or empirical advantage) but also aims to account for certain properties of the comparative subclause, notably ellipsis patterns, that are similar to coordination (\citealt[491--517]{jaeger2018}). However, these properties are actually predicted also under an analysis in which the lower CP is missing and hence neither finiteness nor the [rel] feature are present, making comparatives dissimilar to relative clauses (see the discussion below and in \chapref{ch:6}). In other words, there is no need to postulate a categorial difference between various kinds of comparative markers: their differences arise from mere featural differences.

A second problem concerns the relative positions of \textit{als} and \textit{wie} in Standard German. Since both can be followed by \textit{wenn} in hypothetical comparatives, and since they never co-occur in this variety anyway, \citet[467, ex. 725c and 482, ex. 755d]{jaeger2018} assumes a uniform representation for both, as shown in \figref{treejaeger}.\footnote{As indicated, \citet{jaeger2018} assumes that the complement of C is a VP, not TP, unlike in the representations above. Nothing hinges on this difference, though.}

\begin{figure} 
\caption{The ConjP-analysis} \label{treejaeger}
\begin{forest} baseline, qtree
[ConjP
	[Conj$'$
		[Conj
			[als/wie]
		]
		[CP
			[OP]
			[C$'$ [C] [VP]]
		]
	]
]
\end{forest}
\end{figure}

The following problem arises. Since \textit{als} in comparative clauses lexicalises the maximality operator (see the discussion in \sectref{sec:5polarity} above), it is evident that the relevant projection, here ConjP, is able to encode this function. It is expected that any other element located in this position, including \textit{wie}, should be able to lexicalise the maximality operator as well. If so, one would expect \textit{als} and \textit{wie} to behave in a uniform fashion regarding the licensing of negative polarity items. This is, however, not the case, as shown in \sectref{sec:5polarity}. In other words, the observed asymmetry between \textit{als} and \textit{wie} is not borne out from the representation in \figref{treejaeger}.

As discussed in \sectref{sec:5polarity}, it follows from comparative semantics that the matrix degree element cannot lexicalise the maximality operator in comparatives, while it can do so in degree equatives. If \textit{wie} occupies the same position as \textit{als}, one would expect it to be able to function in the same way: that is, to lexicalise the maximality operator. Note that this also follows from the assumption underlying \figref{treejaeger} that the comparative operator is realised lower by a separate comparative operator. In other words, if one were to assume that the maximality operator is lexicalised by the matrix degree element (in the way proposed above), it would remain mysterious why the \textit{wie}-XP is generated at all, given that it would be associated with neither of the functions present in comparative subclauses otherwise. Marking the type of the clause (as comparative) is not a solid argument in the configuration in \figref{treejaeger} either: as this projection is factually above the CP, it is not expected to mark clause type proper at all.

The representations in Figures~\ref{treealsstandard} and \ref{treewie} predict the asymmetry in question and are therefore preferable as far as Standard German is concerned. This analysis also predicts that combinations like \textit{als wie} can occur, which is what we can observe dialectally. The question is rather how the proposed analysis can account for the way this combination arose and whether it carries over to dialects using \textit{wie} as a uniform comparative complementiser.

Let us start with the combination \textit{als wie}. For this, \citet[366]{jaeger2018} is forced to assume that the two elements in fact form a complex head, but this is not simply assumed to maintain \figref{treejaeger}. The corpus study of \citet{jaeger2018} clearly shows that \textit{als wie} first appeared in the 17th century and spread from non-degree equatives to degree equatives to comparatives, in line with the comparative cycle mentioned above. Further, \citet[255--259]{jaeger2018} argues that the combination is not an intermediate stage between \textit{als}-comparatives and \textit{wie}-comparatives, contrary to widespread assumptions in the previous literature (\citealt{jaeger2010}, \citealt{feldmann1901}, \citealt{dueckert1961}, also Grimm's \textit{Deutsches Wörterbuch}). The same applies to the assumption that \textit{als wie} would be a mixed dialect form between Northern dialects using \textit{als} and Southern dialects using \textit{wie} (\citealt[298]{jaeger2018}, contrary to \citealt{lipold1983}). \citet{jaeger2018} proposes that \textit{als} was reanalysed from a matrix correlative element as part of the comparative standard marker \textit{wie} at a time when \textit{wie} was already established as the standard marker in the relevant construction (that is, in non-degree equatives). The reanalysis of the matrix correlative element into the comparative subclause is in fact attested throughout the history of German: among other combinations, the combination \textit{also} (leading to present-day \textit{als}) arose this way in Old High German (\citealt[71--75]{jaeger2018}). Such processes start in non-degree equatives since the two elements (the matrix degree marker and the complementiser) are adjacent to each other, there being no gradable predicate in the matrix clause.

This does not automatically presuppose that the matrix element should always be reanalysed as part of the original complementiser, though. Unlike the combination of \textit{all} and \textit{so}, where \textit{all} was otherwise not attested as a complementiser of its own and could hence only be interpreted as part of another (now complex) complementiser, the combination \textit{als wie} contains two elements that are transparently available as complementisers in their own right in the system otherwise. The difference between the two cases is in fact supported by the overall diachronic data presented by \citet[360--361]{jaeger2018}. These data show that the comparative cycle from non-degree equatives to degree equatives to comparatives had clearly distinguishable starting and culmination points across the three constructions in the case of \textit{al(so)}: \textit{al(so)} appeared in non-degree equatives about 300 years earlier than in degree equatives, where in turn it appeared another 300 years earlier than in comparatives, the actual culmination into dominant patterns showing about the same differences in time. This clearly indicates that the complementiser had to undergo the reinterpretation (in features) discussed above. This contrasts with the behaviour of \textit{als wie}, where the data suggest that while the attestations of this combination indeed follow the comparative cycle, the actual spread of the combination occurred almost simultaneously in the three environments. The combination also occurred in a system (considering the period and the dialects involved) in which \textit{als} was otherwise already predominantly used as an [INEQ] complementiser and \textit{wie} was still predominantly used as an [EQ] complementiser, though the distinction was not as sharp as the present-day standard pattern would suggest. In any case, the appearance of \textit{als} in this configuration also presupposed that this element was underspecified for the [EQ]/[INEQ] distinction, and as such could be extended to other constructions. The same applies to \textit{wie}, which dialectally came to be underspecified for this property anyway.

Following the analysis for comparison constructions discussed in connection with equative relative clauses in \chapref{ch:4}, the two relevant structures for non-degree equatives can be schematically represented as follows. The diagram in \figref{treealswiecompcp} shows the configuration in which \textit{als} is a matrix correlative element.

\begin{figure} 
\caption{Doubling involving ComprP} \label{treealswiecompcp}
\begin{forest} baseline, qtree
[ComprP
	[Compr$'$
		[Compr
			[als\textsubscript{{[}MAX{]}}]
		]
		[CP
			[OP]
			[C$'$ [C [wie\textsubscript{{[}compr{]}}]] [TP]]
		]
	]
]
\end{forest}
\end{figure}

The diagram in \figref{treealswiecpcp} shows the configuration in which \textit{als} is a functional head in the subordinate clause.

\begin{figure} 
\caption{Doubling involving two CPs} \label{treealswiecpcp}
\begin{forest} baseline, qtree
[CP
	[C$'$
		[C
			[als\textsubscript{{[}MAX{]}}]
		]
		[CP
			[OP]
			[C$'$ [C [wie\textsubscript{{[}compr{]}}]] [TP]]
		]
	]
]
\end{forest}
\end{figure}

The representations only include the property [MAX], indicating which element lexicalises the maximality operator, and [compr], indicating an element expressing comparison in the projection hosting the comparative operator. The structure in \figref{treealswiecompcp} represents a stage in which \textit{wie} is already a complementiser, and \textit{als} is a matrix equative marker; the structure is the same as in Modern Standard German, involving the combination of the matrix equative marker \textit{so} and the complementiser \textit{wie}. The change from \figref{treealswiecompcp} to \figref{treealswiecpcp} involves a change in the label but not the reanalysis to a pre-existing lower position, which is not contrary to the grammaticalisation scheme of \citet{robertsroussou2003}, as it involves relabelling rather than a change from a less functional into a more functional element. 

The change is possible also because certain projections are not obligatory in non-degree equatives. The higher CP is generally not obligatory in equatives, as the maximality operator can also be realised by the matrix correlative element. On the other hand, non-degree equatives do not necessarily contain a matrix equative element (see also \citealt{jaeger2018}): the double CP in \figref{treealswiecpcp} is not embedded under a Compr head but is interpreted as a construction where \textit{als} is part of the subclause. The potential availability of both structures can be captured by the feature-based model proposed here, as the various functions are not tied to designated projections per se.

The structure in \figref{treealswiecpcp} above was extended both to degree equatives and to comparatives proper. As there is no syntactic difference in terms of which element lexicalises the maximality operator, the expectation is that both configurations should license negative polarity items. To my knowledge, this question has not been consistently investigated in the literature so far; in particular, it was not part of the large-scale dialect studies that otherwise provide reliable data on comparative constructions. Nevertheless, one can  find relevant examples from the 19th century.\footnote{Two examples are given in (\ref{alswiejemalsequat}) and (\ref{alswiejemalscompr}) below:

\ea \gll So völlig nichtig, gar nicht als wie \textbf{jemals} gewesen werden sie da sein, jene Schaaren, die Furcht vor ihnen eine ebenso effektlose, leere, nichtige, wie das Essen, Trinken im Traume. \label{alswiejemalsequat}\\
so fully void, really not as how ever been become.\textsc{3pl} they there be, those crowds the.\textsc{f} fear before them.\textsc{dat} a.\textsc{f} likewise inconsequential empty void how the.\textsc{n} eating drinking in.the dream\\
\glt `They, those crowds, will become so fully void, not what they once were, and the fear from them will be likewise inconsequential, empty, void, like eating and drinking in a dream.'\\(\textit{Der Prophet Jesaia, übersetzt und erklärt von D. Moritz Drechsler, Zweiter Theil, erste Hälfte}, Stuttgart, Verlag von Samuel Gottlieb Liesching, 1849, p. 50--51)
\ex \gll {}[\ldots{}] und es ergab sich das überraschende Resultat, daß die Wochenmärkte der Städte ungleich besuchter waren, als wie \textbf{jemals} zuvor. \label{alswiejemalscompr}\\
{} and it gave itself the.\textsc{n} surprising result that the.\textsc{pl} weekly.markets the.\textsc{pl.gen} towns unequal more.frequented were.\textsc{3pl} as how ever before\\
\glt `and the surprising results emerged that the weekly markets of the towns were remarkably more frequented than ever before.'\\(Heinrich Bodemer, \textit{Zehn Artikel zu Gunsten der Gewerbe}, Stuttgart, Beck \& Fränke, 1848, p. 16)
\z

The example in (\ref{alswiejemalsequat}) illustrates the use of \textit{jemals} in an equative clause (I have not yet found an example for a degree equative but degree does not seem to differentiate with respect to the licensing of negative polarity items in general), while (\ref{alswiejemalscompr}) illustrates the use of \textit{jemals} in (degree) comparatives.} This indicates that the presence of \textit{als} in the subclause can license negative polarity items. The results are thus compatible with the proposed structure in \figref{treealswiecpcp}. As discussed by \citet{jaeger2018}, the use of the combination \textit{als wie} and of the complementiser \textit{wie} in comparatives proper are attested in 19th-century texts, yet they were subject to prescriptive pressure already. Bearing this in mind, it is not surprising that the available written-language data are scarce.

Let us now turn to the status of \textit{wie} in dialects that use this element as a unified comparative complementiser, that is, both in equatives and in comparatives proper (see the examples in (\ref{bavarianwie}) and (\ref{thuringian}) in \sectref{sec:5data} above). Since this element can occur in comparatives proper, the analysis proposed here predicts that this should be an element located in a higher C position, that is, above the CP hosting the comparative operator. The structure should therefore be identical to the one proposed for 17th- and 18th-century \textit{als} given in \figref{treeals}. Consider the representation in \figref{treewiedialect}.

\begin{figure} 
\caption{The position of \textit{wie} in comparatives proper} \label{treewiedialect}
\begin{forest} baseline, qtree
[CP
	[C$'$
		[C\textsubscript{{[}compr{]}}
			[wie\textsubscript{{[}compr{]},{[}MAX{]}}]
		]
		[CP
			[OP]
			[C$'$ [C\textsubscript{{[}rel{]},{[}compr{]},{[}fin{]}}] [TP]]
		]
	]
]
\end{forest}
\end{figure}

In this case, \textit{wie} is not specified for [EQ] or [INEQ], which is what we see in its dialectal distribution, and it lexicalises the maximality operator. The structure in \figref{treewiedialect} is in a way similar to the one proposed by \citet{jaeger2018}, inasmuch as \textit{wie} occupies a higher position (setting the difference between CP and ConjP aside now). Note, however, that I assume this to be true only for the dialects that have apparently reanalysed \textit{wie} as a higher complementiser, and not for Standard German, where this assumption does not hold.

The question again arises whether negative polarity items are licensed in these contexts: this is expected to be possible under the present analysis. Again, just as with \textit{als wie}, this question has not been consistently investigated in the literature so far, and no proper dialect atlas data are available at the moment. Still, some examples from the 19th century can be found.\footnote{Two examples are given in (\ref{wiejemalsequat}) and (\ref{wiejemalscompr}) below:

\ea \gll Oder bin ich noch immer so ungeschickt und unvorbereitet darin, wie \textbf{jemals}! \label{wiejemalsequat}\\
or am I still always so clumsy and unprepared in.that how ever\\
\glt `Or am I still as clumsy and unprepared in that as ever?'\\(Matthew Henry, \textit{Des Communicanten Gefährte, oder, Anweisungen und Hülfsmittel zum würdigen Genuss des Heiligen Abendmahls}, Schippensburg: James Galbraith, 1847, p. 65)
\ex \gll {}[\ldots{}] der Zulauf zum Theater, zu Konzerten, Bällen, Maskeraden war größer wie \textbf{jemals} \label{wiejemalscompr}\\
{} the.\textsc{m} throng to.the theatre to concerts balls masquerades was.\textsc{3sg} greater how ever\\
\glt `the popularity of theatres, of concerts, balls, masquerades was greater than ever.'\\(\textit{Zeitung für die elegante Welt. Beilagen: Intelligenzblatt der Zeitung für die elegante Welt, Band 3}, 1803, p. 533)
\z

The example in (\ref{wiejemalsequat}) shows \textit{jemals} used in a degree equative clause introduced by \textit{wie}, while (\ref{wiejemalscompr}) shows \textit{jemals} in a degree comparative introduced by the same complementiser.} Such configurations are excluded from present-day Standard German, but can be found in earlier examples and in non-standard language use. This indicates that \textit{wie} indeed has a different syntactic status in these varieties from the one attested in the standard variety. The proposed analysis can account for these differences.

\section{Polarity marking cross-linguistically} \label{sec:5polaritymarking}
Having discussed the patterns in German historically and synchronically, let us briefly turn to the asymmetries between (degree) equatives and comparatives cross-linguistically. As discussed above, comparatives are always negative polarity contexts, as the matrix degree element cannot lexicalise the maximality operator. In addition, comparatives proper express inequality: this property is indirectly related to negation, as the non-equality of two degrees is expressed. Recall from \sectref{sec:5polarity} that \citet[531--532]{seuren1973} assumed some covert negation in the clause, which is, however, unlikely to be the case. Still, the question arises whether and how negation is related to comparatives in the left periphery.

Indeed, a negative head can occur within the comparative clause. The phenomenon can be observed in Italian (see also the discussion in \citealt[535]{seuren1973}), as shown by the example in (\ref{italiansubj}) below (\citealt[46, ex. 2--82]{grimaldi2009}):

\ea \gll Egli sapeva molto pi\`{u} che \textbf{non} dicesse. \label{italiansubj}\\
he knew.\textsc{3sg} much more that not said.\textsc{sbjv.3sg}\\
\glt `He knew much more than he said.' (Carlo Levi, \textit{Cristo si \`{e} fermato a Eboli})
\z

As can be seen, the negative element \textit{non} `not' appears with a finite verb in the subjunctive in (\ref{italiansubj}), which is associated with literary and/or formal style (\citealt[46]{grimaldi2009}, \citealt[535]{seuren1973}).\footnote{It is worth mentioning that the acceptability of such sentences is subject to much debate in the literature on Italian, though the fact that such examples are actually attested clearly shows that they are not unacceptable across speakers. As shown by \citet[45--48]{grimaldi2009}, this is altogether a restricted option (speakers preferring \textit{di quello che} `of that.\textit{dem} that' or \textit{di quanto} `of how.much' for clausal comparatives), whereby most examples occur with epistemic verbs. \citet[848]{belletti1991} claims that such sentences are altogether ungrammatical, while \citet[683--689]{schwarze1995}, \citet[205]{donati2000} and \citet[459]{wandruszka1991} express more nuanced opinions; \citet[519]{serianni1988} and \citet[150--159]{price1990} even treat it as a regular pattern. As pointed out by \citet[46]{grimaldi2009}, \textit{che}-comparatives were regular in Old Italian, suggesting that the occurrence of such examples especially in literary texts may well be due to the more conservative nature of this register. The element \textit{non} thus occurs in a largely fossilised (and for many speakers apparently ungrammatical or at least archaic) construction; this is certainly compatible with the fact that it does not encode clausal negation.}

A similar phenomenon can be detected in French, where \textit{ne} appears with finite verbs (\citealt[535, ex. 44]{seuren1973}):

\ea \gll Jean est plus grand que je \textbf{ne} pensais. \label{french}\\
John is more tall.\textsc{m} that I not thought.\textsc{1sg}\\
\glt `John is taller than I thought.'
\z

The French example in (\ref{french}) clearly shows that the overt marking of degree negation is not the same as clausal negation: in French, the polarity marker is \textit{ne}, while negation is carried rather by a negative particle such as \textit{pas} otherwise. Consider the example in (\ref{nepas}):

\ea \gll Je \textbf{(ne)} sais \textbf{*(pas)}. \label{nepas}\\
I \phantom{\textbf{(}}not know.\textsc{1sg} \phantom{\textbf{*(}}no\\
\glt `I don't know.'
\z

As can be seen, the element \textit{pas} must be overt, indicating that \textit{ne} cannot express negation on its own. By contrast, colloquial French allows \textit{ne} to be absent altogether, which shows that \textit{pas} is able to express negation by itself.

While both Italian and French show that the negative head does not express clausal negation, the position of this functional head is relatively low in the clause: as can be seen in (\ref{french}), the subject precedes the negative element. This kind of negation is essentially an instance of what is traditionally referred to as expletive negation, whereby a negative marker is present in the structure without actually expressing true clausal negation. The phenomenon can be observed in constructions other than comparatives as well; for instance, in French it can occur in complement clauses of the verbs \textit{craindre} `fear' and \textit{douter} `doubt', as well as in clauses introduced by \textit{avant que} `before' and \textit{\`{a} moins que} `unless'. Consider:

\ea \gll Je doute qu'il \textbf{(ne)} vienne ce soir.\\
I doubt.\textsc{1sg} that.he \phantom{\textbf{(}}not comes.\textsc{sbjv} this.\textsc{m} evening\\
\glt `I doubt that he will come tonight.'
\z

As indicated, the presence of the expletive \textit{ne} is not compulsory (it is more likely to appear in formal register), given that it does not express clausal negation. Similar patterns can be observed across Romance (see, for instance, \citealt{espinal2000} on Spanish). A common property of expletive negation structures is that the negative element is required by an element in the high CP-periphery of the clause. As \citet{abels2005} argues, there is some sort of negation involved in expletive negation, but it is unusually high in the clause. In our case, the licenser of the negative element is the higher C head lexicalising the maximality operator. There are also languages where the polarity head is high in the clause: Old Hungarian is such an example, where the original comparative C head \textit{hogy} `that, how' was immediately followed by the polarity marker \textit{nem} `not' in comparatives expressing inequality (see \citealt{bacskaiatkari2014dia, bacskaiatkari2014diss, bacskaiatkari2016alh}).

In Italian and French, the comparative complementiser is surface-identical to the general subordinator `that'. This differs from the German case, where we have seen that the complementisers are more specific (even though they are not necessarily restricted to a specific type). There are quite a few languages where the inequality comparative marker is negative-like also in the sense that it is transparently related to some negative/adversative element (or incorporates such an element). As described by the typological study of \citet[47--121]{stolz2013}, the adversative/contrastive source for comparative particles (complementisers, P heads) is quite common in European languages: it can be observed in Germanic languages, in the case of dialectal English \textit{nor}, Swiss German \textit{weder} `neither; than', historical (and Swiss) German \textit{wan} (see also \citealt{jaeger2018}) and in North Germanic (Swedish \textit{\"an}, Norwegian \textit{enn}, Danish \textit{end}, Icelandic \textit{en}). This pattern is also common in Slavic (e.g. Czech \textit{ne\v{z}}, Polish \textit{ni\.{z}}, Serbo-Croatian \textit{nego}/\textit{no}). The relatedness of negative/adversative elements and comparative markers was observed already in the 19th century, for instance by \citet{ziemer1884}. An example is given from Swedish in (\ref{swedishcomp}) below (based on \citealt[268]{bacskaiatkaribaudisch2018}):

\ea \gll	Astrid är	äldre	\textbf{än} Peter. \label{swedishcomp}\\
Astrid is	older	than Peter\\
\glt `Astrid is older than Peter.'
\z

This property is not the least surprising and congruent with the assumption made here that equative and comparative complementisers may contain lexical features such as [EQ] and [INEQ] beyond clause-typing features proper. Degree equatives tend to be reanalysed from similarity markers (see also \citealt{jaeger2018}), which also predictably leads to the presence of such a lexical feature.

As discussed in the previous section in connection with German, while comparatives always exhibit a double CP structure, equatives may also involve only a single CP. This leads to the prediction that there should be asymmetries in doubling effects, as far as they can be detected. Once doubling effect concerns the co-occurrence of the higher complementiser with an overt comparative operator in the lower CP; this can be detected in non-standard English as well, as shown in (\ref{englishsymmetrical}) below (\citealt{bacskaiatkari2018langsci}):\largerpage

\ea \label{englishsymmetrical}
\ea[\%]{Mary is as old \textbf{as how old} Susan is.}
\ex[\%]{Mary is older \textbf{than how old} Susan is.}
\z
\z

In English, the pattern is symmetrical: speakers find the two examples equally good or equally ungrammatical, depending on their dialect. This is, however, not necessarily the case. \citet[199, 205--206, 209--210, 216]{bacskaiatkaribaudisch2018} present data from Norwegian that suggest an asymmetrical pattern. The data are summarised in (\ref{norwegian}) below:

\ea \label{norwegian}
\ea \gll Maria er så gammel \textbf{som} (??/* \textbf{hvor} \textbf{gammel}) Peter var i~fjor. \label{norwegianequat}\\
Mary is so old as {} how old Peter was last.year\\
\glt `Mary is as old as Peter was last year.'
\ex \gll Katten er så feit \textbf{som} (\% \textbf{hvor} \textbf{vid}) kattedøra er. \label{somhvorvid}\\
the.cat is so fat as {} how wide the.cat.flap is\\
\glt `The cat is as fat as the cat flap is wide.'
\ex \gll Maria er eldr \textbf{enn} (\% \textbf{hvor} \textbf{gammel}) Peter var i~fjor. \label{ennhvorgammel}\\
Mary is older than {} how old Peter was last.year\\
\glt `Mary is older than Peter was last year.'
\ex \gll Katten er feitere \textbf{enn} \textbf{hvor} \textbf{vid} kattedøra er. \label{norwegiancompr}\\
the.cat is fatter than how wide the.cat.flap is\\
\glt `The cat is fatter than the cat flap is wide.'
\z
\z

As indicated, speakers have different judgements concerning the data.\footnote{One informant is from Rogaland and the other is from Vest-Agder. The data are uniformly given in Bokmål here. The informant from Rogaland accepts the operator in (\ref{somhvorvid}), and the informant from Vest-Agder accepts the operator in (\ref{ennhvorgammel}).} However, what matters for us here is not so much the absolute grammaticality of the sentences but rather their relative differences and the observed asymmetries. Most importantly, while (\ref{norwegianequat}) is ungrammatical or only marginally acceptable, (\ref{norwegiancompr}) is fully grammatical. There are two differences between these constructions: first, the adjective taken by the operator \textit{hvor} is non-contrastive in (\ref{norwegianequat}) and contrastive in (\ref{norwegiancompr}); second, (\ref{norwegianequat}) is a degree equative and (\ref{norwegiancompr}) is a comparative. Both of these factors apparently matter. Regarding the information structural status of the lexical AP, non-contrastive APs are redundant and speakers tend to prefer elliptical constructions: this is not specific to Norwegian but it can be observed cross-linguistically (see \citealt{bacskaiatkari2018langsci}). The phenomenon can be observed also by comparing (\ref{norwegianequat}) to (\ref{somhvorvid}) and (\ref{ennhvorgammel}) to (\ref{norwegiancompr}): both kinds of comparatives are more acceptable if the AP is contrastive.\largerpage[2]

Regarding the second difference, it should be clear that there is an asymmetry not attested in English (compare (\ref{englishsymmetrical}) above). The Norwegian data suggest that the lower CP is preferably not generated in equatives: in cases where the AP is contrastive and therefore cannot be left out, there may be a double CP structure, though not for all speakers. In comparatives, however, the lower CP seems to be generated easily: naturally, in cases where the AP is non-contrastive, the structure is not fully acceptable for all speakers due to redundancy. 

The idea that the difference primarily lies in the availability of a lower CP is reinforced by the data in (\ref{hva}), which contain a lower complementiser (the data are written in Nynorsk; based on \citealt[197--198, 208--209]{bacskaiatkaribaudisch2018}).

\ea \label{hva}
\ea[*]{\gll Maria er så gammel \textbf{som} \textbf{kva} Peter (er). \label{somhva}\\
Mary is so old as what Peter \phantom{(}is\\
\glt `Mary is as old as Peter.'}
\ex[]{\gll Maria er eldr \textbf{enn} \textbf{kva} Peter er. \label{ennhva}\\
Mary is older than what Peter is\\
\glt `Mary is older than Peter.'}
\z
\z

As can be seen, the lower complementiser \textit{kva} `what' (\textit{hva} in Bokmål) is permitted in comparatives like (\ref{ennhva}) but not in degree equatives, as shown in (\ref{somhva}). This contrasts with English, where various dialects allow \textit{what} both in equatives and in comparatives (see \citealt[91]{bacskaiatkari2018langsci} for arguments in favour of the complementiser status of \textit{what} in these cases; cf. also the data of \citealt{izvorski1995}). The ungrammaticality of (\ref{somhva}) suggests that \textit{som} and \textit{kva} (\textit{hva}) are in complementary distribution. Given that the lower C was identified as associated with the relative property, [rel], this implies that not only \textit{kva} (\textit{hva}) but also \textit{som} (unlike \textit{enn}) should be readily associated with [rel]. This expectation is borne out as the regular relative complementiser in Norwegian is in fact \textit{som}. Consider the following example (based on \citealt[185]{bacskaiatkaribaudisch2018}):

\ea \gll	Dette	er studenten \textbf{som} inviterte	Maria.\\
this is the.student	that invited.\textsc{pst} Mary\\
\glt `This is the student who invited Mary.'
\z

Norwegian \textit{som} is thus reminiscent of German \textit{wie} in many respects. Note that English \textit{what} is also available as a regular relative complementiser in dialects that allow the same element in comparatives, as illustrated in (\ref{whatrel}) below (\citealt[291]{kortmannwagner2007}):

\ea \% See he was the man \textbf{what} brought in decasualization during the war. (BNC H5H) \label{whatrel}
\z

The example in (\ref{whatrel}) contains a headed relative clause (the head noun is \textit{the man}); unlike in Standard English, \textit{what} is possible in many regional dialects.

In sum, Germanic data show that degree equatives may lack a higher CP (for the maximality operator) and that the lower CP is associated with a [rel] feature. Neither of these properties are restricted to Germanic (see \citealt{bacskaiatkari2016alh} for a detailed analysis). The following data in (\ref{sc}) are from Serbo-Croatian:

\ea \label{sc}
\ea \gll Pavao je visok \textbf{kao} \textbf{što} je visok Petar.\\
Paul is tall as what is tall Peter\\
\glt `Paul is as tall as Peter is.'
\ex \gll Pavao je viši \textbf{nego} \textbf{što} je visok Petar.\\
Paul is taller than what is tall Peter\\
\glt `Paul is taller than Peter is.'
\z
\z

In both cases, the higher complementiser is specified for [EQ]/[INEQ] and the lower complementiser is \textit{što} `what', similarly to the English cases given in (\ref{englishsymmetrical}) above. Consider the example in (\ref{sto}) below (\citealt[27, ex. 2]{gracaninyuksek2013}):

\ea \gll čovjek \textbf{što} puši \label{sto}\\
man that smokes\\
\glt `a/the man that smokes/is smoking'
\z

Doubling may also involve the combination of a higher complementiser and a lower overt operator. This can be observed in Czech comparative clauses, as shown in (\ref{czech}):

\ea \label{czech}
\ea[?]{\gll Ten stůl je delší, \textbf{než} \textbf{jak} \textbf{široká} je ta kancelář. \label{nezjaksiroka}\\
the table is longer than how wide is the office\\
\glt `The table is longer than the office is wide.'}
\ex[]{\gll Ten stůl je delší, \textbf{než} \textbf{jak} je ta kancelář \textbf{široká}. \label{nezjak}\\
the table is longer than how is the office wide\\
\glt `The table is longer than the office is wide.'}
\z
\z

As can be seen, the contrastive adjective is preferably stranded so that it occupies a position where it can receive main stress, as in (\ref{nezjak}); nevertheless, the grammaticality of (\ref{nezjaksiroka}) indicates that \textit{jak} is a regular comparative operator (unlike German \textit{wie}, see above) that can take a lexical AP (see \citealt{bacskaiatkari2015fdsl}, \citealt[93--94]{bacskaiatkari2018langsci} for more discussion). These configurations are possible only if the higher complementiser \textit{než} is present. This differs significantly from the pattern attested in equatives, as shown in (\ref{czechequat}):

\ea \label{czechequat}
\ea \gll Ten stůl je stejně dlouhý, \textbf{jak} \textbf{siroká} je ta kancelář.\\
the table is same long how wide is the office\\
\glt `The table is as long as the office is wide.'
\ex \gll Ten stůl je stejně dlouhý, \textbf{jak} je ta kancelář \textbf{siroká}.\\
the table is same long how is the office wide\\
\glt `The table is as long as the office is wide.'
\z
\z

In this case, there is no higher complementiser present at all: the operator suffices as far as the marking of the clause type is concerned. The maximality operator is lexicalised by the matrix degree element, which thus takes scope over the comparative operator. The asymmetry between degree equatives and comparatives in Czech thus clearly indicates that the two constructions differ in their left peripheries: this difference is predicted under the analysis proposed here.

In principle, one may suppose that the problem in Czech is simply that the higher complementiser is not compatible with degree equatives, which is why the operator is licensed on its own. Interestingly, Hungarian shows a similar case, also indicating that this alternative explanation does not suffice. Consider the examples in (\ref{hequat}) for degree equatives:

\ea \label{hequat}
\ea \gll Mari olyan magas, \textbf{mint} \textbf{amilyen} \textbf{(magas)} P\'eter. \label{asmintamilyen}\\
Mary so tall as how.\textsc{rel}  \phantom{\textbf{(}}tall Peter\\
\glt `Mary is as tall as Peter.'
\ex \gll Mari olyan magas, \textbf{mint} P\'eter. \label{asmint}\\
Mary so tall as Peter\\
\glt `Mary is as tall as Peter.'
\ex \gll Mari olyan magas, \textbf{amilyen} \textbf{(magas)} P\'eter. \label{asamilyen}\\
Mary so tall how.\textsc{rel} \phantom{\textbf{(}}tall Peter\\
\glt `Mary is as tall as Peter.'
\z
\z

As shown by (\ref{asmintamilyen}), Hungarian allows the co-presence of the overt complementiser \textit{mint} `as' and an overt operator such as \textit{amilyen} `how', whereby the latter may also occur together with a lexical adjective (note that the data were tested on several speakers, and the judgements were uniform and clear). It is also possible that only \textit{mint} is overt but not the operator, as in (\ref{asmint}): in this case, the finite verb is also elided (see \citealt[173--196]{bacskaiatkari2018langsci}). Finally, it is also possible that \textit{mint} is absent and the [compr] property is marked only by the operator, as in (\ref{asamilyen}). Hence, [compr] has to be encoded by at least one element, and doubling is also possible.

The picture is slightly different in comparative clauses, where \textit{mint} cannot be absent, as shown by (\ref{hcomp}):

\ea \label{hcomp}
\ea[]{\gll Mari magasabb, \textbf{mint} \textbf{amilyen} \textbf{(magas)} P\'eter. \label{thanmintamilyen}\\
Mary taller as how.\textsc{rel} \phantom{\textbf{(}}tall Peter\\
\glt `Mary is taller than Peter.'}
\ex[]{\gll Mari magasabb, \textbf{mint} P\'eter. \label{thanmint}\\
Mary taller as Peter\\
\glt `Mary is taller than Peter.'}
\ex[*]{\gll Mari magasabb, \textbf{amilyen} \textbf{(magas)} P\'eter. \label{thanamilyen}\\
Mary taller how.\textsc{rel} \phantom{\textbf{(}}tall Peter\\
\glt `Mary is taller than Peter.'}
\z
\z

Just like in degree equative clauses, Hungarian allows the co-presence of the overt complementiser \textit{mint} and an overt operator in comparative clauses, as shown by (\ref{thanmintamilyen}). Further, it is again possible that only \textit{mint} is present, as in (\ref{thanmint}), where the finite verb is again deleted. However, the configuration where only the operator is overt but the complementiser is absent is ungrammatical, as shown by (\ref{thanamilyen}).

Since the complementiser is the same in both kinds of constructions, this element is unspecified for [EQ]/[INEQ] and thus the observed asymmetry cannot be attributed to any difference in these lexical features. The full constructions in (\ref{asmintamilyen}) and (\ref{thanmintamilyen}) also indicate that the complementiser \textit{mint} occupies the same position (relative to the operator) in the CP, and thus no asymmetry like the one in Norwegian is observed. Still, it is clear that the complementiser must at all events be present in comparatives, since this encodes the maximality operator, whereas this function can be carried by the matrix degree element in equatives, allowing a single CP.

In sum, the data from various other languages discussed above indicate that the differences between degree equatives and comparatives observed in German hold cross-linguistically, having very similar effects on the complexity of the clausal left periphery.

\section{Hypothetical comparatives} \label{sec:5hypothetical}
\subsection{The data} \label{sec:5datahypothetical}
Let us now turn to another construction which involves multiple CPs. Hypothetical comparatives (briefly discussed already in \chapref{ch:2}) constitute a mixed clause type, as they share properties of ordinary conditional clauses and of comparative (more precisely similative) clauses. An example is given in (\ref{asif}) below:

\ea She behaves \textbf{as if} she were mad. \label{asif}
\z

Here the hypothetical comparative clause is introduced by the combination \textit{as if}. The first complementiser, \textit{as}, is used regularly in degree and non-degree equatives, while the complementiser \textit{if} is used in conditionals, as in (\ref{ifch5}) below:

\ea Mary would go mad \textbf{if} her daughter joined the army. \label{ifch5}
\z

Hypothetical comparatives are often referred to as ``conditional comparatives'' or ``unreal comparatives'' in the literature. I refer to the constructions as ``hypothetical comparatives'', for the following reasons. First, as opposed to the notion unreal comparatives, this term suggests that the clause type is a complex one involving a conditional/hypothetical and a comparative specification. Second, while the notion conditional comparative may seem even better in this respect, it has unfortunately been used in the literature for comparative correlatives that have a conditional meaning component, also called comparative conditionals or proportional correlatives (e.g. \textit{the richer you are, the more you can travel}).

Hence, at first sight, it appears that the combination \textit{as if} in (\ref{asif}) is compositional: it involves the mere combination of the regular equative complementiser expressing similarity and the regular conditional complementiser. One might wonder whether this is always the case. Regarding the various types of hypothetical comparatives attested in English and cross-linguistically, there are three major aspects that have to be taken into consideration: first, the transparency of the combination (if there is any combination at all); second, the reconstructability of the comparative clause; third, whether the conditional clause has realis or irrealis mood. English has two more variants regarding clause-typing elements alongside (\ref{asif}) above (see also the data in \citealt{pfeffer1985}):

\ea 
\ea[]{She behaves \textbf{as though} she were mad. \label{asthough}}
\ex[\%]{She behaves \textbf{like} she were mad. \label{like}}
\ex[]{She behaves \textbf{like} she's mad. \label{likes}}
\z
\z

As can be seen, the pattern in (\ref{asthough}) also involves a combination (\textit{as though}); the non-standard pattern with \textit{like} in (\ref{like})/(\ref{likes}) involves only a single element (and it preferably contains a reduced copula, as indicated). A full clause can be reconstructed if there is a combination that is transparent: this is possible in the case of \textit{as if} but not in the case of \textit{as though} (see \chapref{ch:2}). Consider the examples in (\ref{behave}):

\ea \label{behave}
\ea[]{She behaves \textbf{as} she behaved \textbf{if} she were mad. \label{asiffull}}
\ex[*]{She behaves \textbf{as} she behaved \textbf{though} she were mad. \label{asthoughfull}}
\z
\z

As described by \citet[388]{rudolph1996} and \citet[104]{chen2000}, in line with \citet{quirk1954} and contrary to \citet{koenig1985}, the element \textit{though} most probably started as a general concessive marker, appearing in both factual and hypothetical concessions: based on data from the OED, \citet[104]{chen2000} claims that the concessive use is attested in Old English already (around 888), while the conditional use in the combination \textit{as though} `as if' appears only around 1200. In this way, the combination \textit{as though} was never a transparent combination of a comparative complementiser and a conditional complementiser.

The difference between realis versus irrealis mood is illustrated in (\ref{realisirrealis}):

\ea \label{realisirrealis}
\ea She behaves \textbf{as if} she were afraid. \label{irrealis}
\ex	She behaves \textbf{as if} she is afraid. \label{realis}
\z
\z

As can be seen, the verb in the subclause has irrealis mood in (\ref{irrealis}) and realis mood in (\ref{realis}); there is no difference in the meaning. English is not exceptional in this respect: there are several languages where both the indicative and the subjunctive are licensed, without there being any difference in the meaning (see \citealt[393--394]{jensen1990} for Old French clauses introduced by the combination \textit{com se} `as if').

The possible German patterns were discussed in \chapref{ch:2} (cf. \citealt{jaeger2010}, \citealt{eggs2006}). Consider the examples in (\ref{hypothetical}):\largerpage

\ea \label{hypothetical}
\ea \gll	Sie	schreit	(so),	\textbf{als}	\textbf{wäre}	sie	beim	Zahnarzt. \label{alswaere}\\
she	shouts \phantom{(}so	as be.\textsc{sbjv.3sg}	she	at.the dentist\\
\glt `She is shouting as if she were at the dentist's.'
\ex \gll	Sie	schreit	(so),	\textbf{als}	\textbf{ob}	sie	beim	Zahnarzt	wäre. \label{alsobch5}\\
she	shouts \phantom{(}so	as if she	at.the dentist	be.\textsc{sbjv.3sg}\\
\glt `She is shouting as if she were at the dentist's.'
\ex \gll	Sie	schreit	(so),	\textbf{als}	\textbf{wenn}	sie	beim	Zahnarzt	wäre. \label{alswennch5}\\
she	shouts \phantom{(}so	as	if	she	at.the dentist	be.\textsc{sbjv.3sg}\\
\glt `She is shouting as if she were at the dentist's.'\\
\ex \gll	Sie	schreit	(so),	\textbf{wie}	\textbf{wenn}	sie	beim	Zahnarzt	wäre. \label{wiewenn}\\
she	shouts \phantom{(}so	how	if	she	at.the dentist	be.\textsc{sbjv.3sg}\\
\glt `She is shouting as if she were at the dentist's.'
\z
\z

As indicated, the matrix correlative element \textit{so} is optional in all these cases (cf. the data in \citealt[17]{jaeger2018}). This contrasts with degree equatives, where matrix \textit{so} is obligatory, appearing together with a gradable argument (see the discussion in \sectref{sec:5discussion}). In (\ref{hypothetical}), there is no gradable predicate in the matrix clause and \textit{so} is optional: this indicates that hypothetical comparatives are closer to non-degree equatives (similative constructions). Further, note also that all of these clauses contain a verb in irrealis mood (subjunctive): realis mood (indicative) is restricted in Standard German and rarely shows up in the written language in hypothetical comparatives.\footnote{This obviously does not apply to so-called complex comparatives (see also \citealt[167--168]{eggs2006}), which are surface-similar to proper hypothetical comparatives, yet do not constitute a single clause type. See \citet{bacskaiatkari2018jb} for further discussion.}

Importantly, all of the patterns in (\ref{hypothetical}) involve some combination: (\ref{alswaere}) is different in that the complementiser \textit{als} is followed by a fronted verb, while (\ref{alsobch5}--\ref{wiewenn}) all include the combination of two complementisers.

As discussed in \sectref{sec:5grammaticalisation} in detail, the equative complementiser in present-day (Standard) German is \textit{wie}, not \textit{als}: given this, it is obvious that only the combination \textit{wie wenn} in (\ref{wiewenn}) is transparent in the same was as English \textit{as if}. However, since \textit{als} used to be the equative complementiser, the combination \textit{als wenn}  in (\ref{alswennch5}) is also at least historically compositional. The conditional complementiser is \textit{wenn}; \textit{ob} is not available in this function:

\ea \gll Ich würde mich freuen, \textbf{wenn/*ob} du kommen würdest.\\
I would.\textsc{1sg} myself.\textsc{acc} rejoice.\textsc{inf} if you come.\textsc{inf} would.\textsc{2sg}\\
\glt `I would be glad if you came.'
\z\largerpage

The complementiser \textit{ob} as a conditional complementiser is attested in Old High German (see the data of \citealt[157--158]{schrodt2004}), and it continued to be the dominant pattern until Middle High German, when it started to be replaced by \textit{wenn}, see \citet[388]{rudolph1996}, citing \citet{paul1920band3}. As described by \citet[109]{ferrell1968}, citing the data of \citet[347--348]{behaghel1928}, there are instances of \textit{ob} as a conditional complementiser even in Early New High German, but the number of examples diminishes drastically in this period.\footnote{Interestingly, as pointed out by Lea Schäfer (p.c.), \textit{ob} as a conditional complementiser seems to have been preserved in Modern Eastern Yiddish to a certain degree, as shown in (\ref{oib1}) and (\ref{oib2}) below (\citealt[305]{birnbaum1979}):

\ea \gll ex volt ys im gyzugt, \textbf{oib} ex volt ym gytrofn\\
I would it he.\textsc{dat} tell.\textsc{ptcp} if I would he.\textsc{acc} meet.\textsc{ptcp} \label{oib1}\\
\glt `I would tell him if I were to meet him.'
\ex \gll \textbf{Oib} er vet dir ni\'st v\'eln zugn, darfstjym ni\'st fr\'eign kain sax. \label{oib2}\\
if he want you.\textsc{dat} not want.\textsc{inf} say.\textsc{inf} may.\textsc{2sg}.him not ask.\textsc{inf} no thing\\
\glt `If he should not want to tell you, you need not ask him many questions.'
\z

As Lea Schäfer (p.c.) mentions, similar examples occur also in the corpus \textit{Language and Culture Archive of Ashkenazic Jewry} (LCAAJ).} At any rate, this suggests that the combination \textit{als ob} was historically also compositional.

Still, synchronically only the combination \textit{wie wenn} allows for the reconstruction of a higher comparative clause, as shown in (\ref{wiewennfull}) below (see \chapref{ch:2} for the ungrammatical configurations):

\ea \gll	Sie	schreit	(so),	\textbf{wie}	sie schreien würde, \textbf{wenn}	sie	beim	Zahnarzt	wäre. \label{wiewennfull}\\
she	shouts \phantom{(}so	how	she shout.\textsc{inf} would.\textsc{3sg} if	she	at.the dentist	be.\textsc{sbjv.3sg}\\
\glt `She is shouting as she would be shouting if she were at the dentist's.'
\z

As can be seen, both \textit{wie} and \textit{wenn} take a finite clause of their own. This suggests that there can be two independent subordinate clauses in (\ref{wiewenn}) as well underlyingly. The reconstruction of the equative clause is not possible for (\ref{alswaere}--\ref{alswennch5}). In these cases, the lack of transparency and the impossibility of reconstruction suggest that the hypothetical comparatives in these cases represent a complex clause type involving multiple CPs in the same clausal periphery, just like the case of English \textit{as though}.

\subsection{The analysis} \label{sec:5analysishypothetical}
In \chapref{ch:2}, I proposed the representation in \figref{alsobtreech5} for combinations like \textit{als ob} and \textit{als wenn}.

\begin{figure} 
\caption{Hypothetical comparatives} \label{alsobtreech5}
\begin{forest} baseline, qtree
[CP
	[C$'$
		[C
			[als]
		]
		[CP
			[C$'$ [C [ob\\wenn]] [TP]]
		]
	]
]
\end{forest}
\end{figure}

This arrangement clearly does not hold for (\ref{wiewennfull}), though, since there is no way to locate a full finite clause in the left periphery. The biclausal configuration for \textit{wie wenn} can be represented as given in \figref{treebiclausal}.\footnote{If the higher TP is elliptical, the result is the string in (\ref{wiewenn}); the overt realisation of the underlying TP results in the configuration in (\ref{wiewennfull}).} The same configuration applies to English \textit{as if}, as demonstrated for (\ref{asiffull}) in \figref{treebiclausalenglish}.

\begin{figure}
\caption{The biclausal structure} \label{treebiclausal}
\begin{forest} baseline, qtree
[CP
	[C$'$
		[C
			[wie]
		]
		[TP
			[\phantom{xxx}]
			[\ldots{} [\phantom{xxx}] [CP [C$'$ [C [wenn]] [TP [sie	beim	Zahnarzt	wäre,roof]]]]]
		]
	]
]
\end{forest}
\end{figure}

\begin{figure}
\caption{The monoclausal structure} \label{treebiclausalenglish}
\begin{forest} baseline, qtree
[CP
	[C$'$
		[C
			[as]
		]
		[TP
			[\phantom{xxx}]
			[\ldots{} [\phantom{xxx}] [CP [C$'$ [C [if]] [TP [\phantom{xxx},roof]]]]]
		]
	]
]
\end{forest}
\end{figure}

It is evident that combinations in hypothetical comparatives may either involve two clauses (biclausal structure), as in \figref{treebiclausal}, or a single clause with a double CP (monoclausal structure), as in \figref{alsobtreech5}. Importantly, while there are two CPs in both \figref{treebiclausal} and \figref{alsobtreech5}, they are located in two different clauses in \figref{treebiclausal} but not in \figref{alsobtreech5}, where they constitute a complex left periphery of a single clause. Note that the higher clause indicated in \figref{treebiclausal} is typically elliptical (as it is redundant) and hence the element corresponding to \textit{as} is immediately followed by the element corresponding to \textit{if} in the linear string, as in (\ref{wiewenn}). Nevertheless, in underlyingly biclausal structures a full first clause is always an option. 

\begin{sloppypar}
Conditional clauses are known to be negative polarity environments, as pointed out already in \sectref{sec:5polarity}. Consider the example in (\ref{dream}):
\end{sloppypar}

\ea If you \textbf{ever} dreamed of travelling in space then this film is something for you. \label{dream}
\z

As can be seen, the negative polarity item \textit{ever} is licensed in the conditional clause. The same applies to German:

\ea \gll Wenn du \textbf{jemals} ganz alleine bist, denke an mich.\\
if you ever total alone are.\textsc{2sg} think.\textsc{imp.2sg} at me.\textsc{acc}\\
\glt `If you are ever completely alone, think of me.'
\z

Conditional clauses are also downward entailing environments, as demonstrated by the following examples (\citealt[504, ex. 6]{panizzachierchiaclifton2009}):

\ea
\ea If I eat pizza, I'll get sick. \label{pizza}
\ex If I eat pizza with anchovies, I'll get sick. \label{anchovies}
\z
\z

In this case, (\ref{pizza}) entails (\ref{anchovies}): whenever it is true that eating pizza makes me sick, eating pizza with anchovies will also make me sick. As \textit{pizza} is the superset of \textit{pizza with anchovies}, the superset entails the subset in this case. The entailment does not work the other way round.

Note that exactly the reverse holds in upward entailing environments, such as the consequence of a conditional, that is, the main clause (\citealt[504, ex. 4]{panizzachierchiaclifton2009}).

\ea
\ea If I go home, I'll eat pizza with anchovies.
\ex If I go home, I'll eat pizza.
\z
\z

In this case, the subset entails the superset: whenever it is true that I eat pizza with anchovies, it is also true that I eat pizza. The entailment does not hold the other way round: eating pizza does not entail eating pizza with anchovies.

Conditional clauses are embedded clauses that need to be licensed by a matrix clause, to which they are adjoined. Depending on the relative position of the conditional clause with respect to the matrix clause, the matrix clause may contain an anaphor such as German \textit{dann} `then', as shown in (\ref{dann}) below (see \citealt{bacskaiatkari2018lb} for a detailed analysis):

\ea \label{dann}
\ea \gll Ich rufe dich an, wenn ich die Lösung finde.\\
I call.\textsc{1sg} you.\textsc{acc} to if I the.\textsc{f} solution find.\textsc{1sg}\\
\glt `I will call you if I find the solution.'
\ex\gll Wenn ich die Lösung finde, (dann) rufe ich dich an.\\
if I the.\textsc{f} solution find.\textsc{1sg} \phantom{(}then call.\textsc{1sg} I you.\textsc{acc} to\\
\glt `If I find the solution, I will call you.'
\z
\z

In hypothetical comparatives with the structure given in \figref{treebiclausal}, the matrix clause of the conditional clause is the comparative subclause, which is mostly elliptical. Still, it can license the conditional clause in either case. The conditional clause is in many respects similar to the embedded polar interrogatives discussed in \chapref{ch:3}: most importantly, they are disjunctive and as such contain a disjunctive operator specified as [Q].

More problematic are the cases that have the structure in \figref{alsobtreech5}, since there is factually no comparative subclause. The matrix clause (\textit{sie schreit so} in all the examples in (\ref{hypothetical}) above) clearly cannot license the conditional clause in itself. Observe the example in (\ref{obungramm}):

\ea[*]{\gll Sie	schreit	(so),	\textbf{ob}	sie	beim	Zahnarzt	wäre. \label{obungramm}\\
she	shouts \phantom{(}so	if she	at.the dentist	be.\textsc{sbjv.3sg}\\
\glt `She is shouting as if she were at the dentist's.'}
\z

The construction is ungrammatical and it does not improve by using the subjunctive mood in the matrix clause either:

\ea[*]{\gll Sie	würde	(so) schreien,	\textbf{ob}	sie	beim	Zahnarzt	wäre. \label{obcondungramm}\\
she	would.\textsc{3sg} \phantom{(}so	shout.\textsc{inf} if she	at.the dentist	be.\textsc{sbjv.3sg}\\
\glt `She would be shouting if she were at the dentist's.'}
\z

Note that the two readings given for (\ref{obungramm}) and (\ref{obcondungramm}) differ, but this has no significance, as both constructions are unacceptable with either reading. Indeed, it is difficult to assign any meaning to (\ref{obungramm}) and (\ref{obcondungramm}) at all as they are ill-formed. The same considerations apply to the cases where a verb is fronted. In the case of \textit{wenn}, the construction in (\ref{obungramm}) is likewise unavailable; the configuration in (\ref{obcondungramm}) renders a regular conditional clause as \textit{wenn} is the regular conditional complementiser. This is expected as \textit{wenn} can appear in a biclausal structure anyway.

It follows that in monoclausal hypothetical comparatives, the highest clause cannot license the conditional clause and the presence of the equative complementiser is necessary: this indicates that the element actually licensing the conditional clause is the equative complementiser itself. It is precisely this element that licenses the disjunctive C head specified as [Q]. The [Q] element is lexicalised either by a complementiser (\textit{ob} or \textit{wenn}) or by a covert operator that is merged to a projection containing the verb, in exactly the same way as was established for polar questions in \chapref{ch:3}. Hypothetical comparatives differ from ordinary conditional clauses regarding the element licensing [Q].

On the other hand, there is a difference between hypothetical comparatives and ordinary comparison clauses. As discussed above, ordinary comparatives involve two important components: an element lexicalising the maximality operator, and a comparative operator. Naturally, the comparative operator is present in fully-fledged comparative clauses as in biclausal hypothetical comparatives, but it is expected to be absent from monoclausal structures (see \citealt{bacskaiatkari2018jb}). Given that there is no gradable predicate in the matrix clause, there is no semantic necessity of there being a comparative operator specifying degree either. In other words, monoclausal hypothetical comparatives do not require a doubling configuration in the way it is attested in ordinary comparison constructions.

As discussed above, the comparative complementiser in comparatives expressing inequality is essentially responsible for licensing a negative polarity context, since matrix comparative degree heads (e.g. -\textit{er}) cannot take over this function. Degree (and non-degree) equatives differ inasmuch as the matrix correlative element can lexicalise the maximality operator; otherwise it is perfectly possible that the equative complementiser takes over this function, which results in the equative clause being a negative polarity environment. We have also seen that in English, both \textit{as} and \textit{than} license negative polarity, whereas Standard German shows an asymmetry between \textit{als} and \textit{wie}.

Let us start with English \textit{as}. This element can introduce a negative polarity clause in comparatives in general and as such it is not surprising that it can do the same in hypothetical comparatives. The combination \textit{as though}, demonstrated in (\ref{asthough}), requires exactly this configuration as no full clause between \textit{as} and \textit{though} can be reconstructed, as shown in (\ref{asthoughfull}): the licensing of the conditional clause cannot be done by an intermediate clause.

In Standard German, the complementiser \textit{als} in hypothetical comparatives is essentially a fossil from a previous stage of the language, when \textit{als} regularly introduced equative subclauses (see also \citealt{jaeger2010, jaeger2018}). As discussed in \sectref{sec:5grammaticalisation} above, \textit{als} takes a position above the projection hosting the comparative operator, as evidenced by the fact that it is available as a complementiser in comparatives expressing inequality and in doubling constructions of the form \textit{als wie}. Negative polarity elements are also attested in the scope of this element. It is therefore expected, just like in the case of English \textit{as}, that this element should be available in monoclausal hypothetical comparatives. This prediction is borne out as \textit{als} occurs in configurations that cannot be assigned a biclausal structure and thus no intermediate clause could license the conditional clause (see the discussion above concerning the combinations \textit{als ob}, \textit{als wenn} and \textit{als}\,+\,fronted verb).

The question arises why reanalysis from a biclausal to a monoclausal structure takes place. On the one hand, as discussed by \citet{bacskaiatkari2018jb}, this has to do with structural economy: the comparative clause is generally elliptical in hypothetical comparatives (since, as mentioned above, it expresses redundant information that can be recovered from the conditional clause, too), hence the only remnant is the comparative C head itself, which cliticises onto the embedded (conditional) C head. The structure is more transparent if the higher C takes the lower CP as a complement and no ellipsis is needed, and it also involves generation of less structure.

On the other hand, transparency affects recoverability. With complementisers that are no longer available as equative complementisers in the language (as is the case with German \textit{als} but not with English \textit{as}), the configuration involving two phonologically adjacent complementisers can only be interpreted as a configuration involving a single left periphery, as there is no well-formed non-elliptical equivalent.\footnote{Transparency plays a role in reanalysis in that this principle is relevant for the language learner (cf. the Transparency Principle of \citealt{lightfoot1979}). For the biclausal configuration, while non-elliptical hypothetical comparatives are expected to be rare in the input (if present at all), there is substantial evidence from other constructions (e.g. ordinary equative clauses) that serves as a cue for the learner to assume a biclausal configuration. However, once this independent evidence is no longer present in the input, there are no relevant cues for the learner to assume a biclausal construction. The monoclausal configuration is more economical and closer to the surface input, and in this sense more transparent.} In this way, the comparative C head in hypothetical comparatives may fossilise a complementiser that is no longer used in equatives.

In this respect, the case of \textit{as if} in English is particularly interesting. Based on the observation that \textit{as} can license negative polarity environments and it can occur in an unambiguously monoclausal construction involving \textit{as though}, we can suppose that a monoclausal structure for \textit{as if} should be possible. At the same time, along with the fact that with the combination \textit{as if}, an intermediate clause can be reconstructed, we can suppose that in elliptical cases a biclausal structure is likewise possible. In other words, the sequence \textit{as if} is ambiguous between a monoclausal and a biclausal structure. This kind of structural ambiguity is in fact expected in a reanalysis scenario, since the syntactic reanalysis of an unchanged phonological sequence naturally arises from there being two possible underlying structures. Given that language change is gradual (see \citealt{traugotttrousdale2010}), the co-existence of two possibilities in one grammar is also expected.

This leads us to the last configuration involving \textit{wie}, with which only the conditional complementiser \textit{wenn} is licensed in the standard language: neither \textit{ob} nor verb movement constitutes an option. The former can be explained away easily as \textit{ob} was no longer available as a conditional complementiser when \textit{wie} started to appear in equatives and in hypothetical comparatives (see \citealt{jaeger2018}). Regarding verb movement, \citet[487]{jaeger2010} shows that a fronted verb in the subjunctive (but not in the indicative) is possible only if the comparative clause is not elliptical: the ban on indicative forms is possibly due to a surface condition ruling out the configuration that has the same linear form as interrogatives. The same argumentation can be carried over to the sequence \textit{wie}\,+\,fronted verb as well, since that would likewise be surface-identical to an interrogative clause. 

However, other factors may also play a role. Namely, \textit{wie}\,+\,fronted verb is possible dialectally, as shown in (\ref{wieverb}) for Rhine Franconian (\citealt[348, ex. 576]{jaeger2018}, citing \citealt[331]{steitz1981}):

\ea \gll De Vader dirmeld (so) \textbf{wie} \textbf{häd} er gesuf \label{wieverb}\\
the father tumbles \phantom{(}so how have.\textsc{cond} he drunk\\
\glt `The father is tumbling as if he had drunk.' (Saarbrücken)
\z

The dialect in question belongs to the High German dialects, which, as we saw above, use \textit{wie} as a regular comparative complementiser. There is no principled reason why the possible interference with interrogative clauses mentioned above would not hold here while it does in Standard German. I therefore suggest that such a condition may indeed hold for elliptical cases, since ellipsis, being a PF mechanism, can be curbed by surface constraints. The construction in (\ref{wieverb}) is, however, not an elliptical version of a biclausal structure but a monoclausal one, where the higher complementiser licenses a fronted finite verb as the head of its complement, just like \textit{als} does in Standard German.

The difference between the two varieties goes back to a more general difference concerning the status of \textit{wie}, as discussed in \sectref{sec:5grammaticalisation} above: \textit{wie} is located in the same CP projection as the comparative operator in Standard German and it cannot lexicalise the maximality operator, resulting in the lack of negative polarity in \textit{wie}-clauses (equative clauses). By contrast, Southern dialects regularly use \textit{wie} as a comparative complementiser as well, allowing also for negative polarity. In other words, in dialects that allow constructions like (\ref{wieverb}), \textit{wie} is predictably able to license a complement with negative polarity, making it available for monoclausal hypothetical comparatives. \citet[473--482]{jaeger2018} uses (\ref{wieverb}) as an argument for treating \textit{wie} as a Conj head above the CP, just like \textit{als}, more generally: according to her, the reanalysis of \textit{wie} from C to Conj made the lower position available for verb movement. I have discussed the potential problems with ConjP in \sectref{sec:5grammaticalisation} already; another problematic point that can be identified here is that this does not carry over to ordinary comparatives at all, which never show the fronting of the verb. In other words, while the status of the higher complementiser matters for both kinds of constructions, the properties of the lower projection differ: in ordinary comparatives, the lower projection hosts the comparative operator and possibly comparative (or relative) heads, while in hypothetical comparatives, the lower CP projection is conditional and hosts the appropriate operator and/or complementiser.

In dialects that allow (\ref{wieverb}), it is also plausible that in hypothetical comparatives introduced by the sequence \textit{wie wenn}, the two complementisers are located in the same left periphery, that is, such constructions are monoclausal. This naturally does not exclude the possibility of the co-existence of non-elliptical biclausal structures, which may undergo ellipsis as well. In this way, hypothetical comparatives involving \textit{wie wenn} in the relevant dialects are ambiguous between a monoclausal and a biclausal structure, as originally proposed by \citet{bacskaiatkari2018jb} for Standard German. Contrary to that analysis, however, I assume that comparatives formed with \textit{wie wenn} in Standard German are actually biclausal and undergo ellipsis. This follows from the general properties of \textit{wie} in related constructions. These properties predict that \textit{wie} cannot license a complement with negative polarity, as it cannot do so in unambiguously monoclausal constructions (equative clauses) either. It follows that the combination \textit{wie wenn} should be assigned the structure given in \figref{treebiclausal}, where the elided comparative subclause can function as a matrix clause for the conditional clause.

In sum, hypothetical comparatives provide an interesting testing ground for the relationship between clause typing and polarity. In many cases, a complex CP-periphery arises, encoding a mixed clause type: this is one of the configurations where a double CP is fully justified. In other cases, however, the properties of the individual elements prohibit such an analysis and suggest a biclausal structure that involves ellipsis as well. The differences between the given combinations, also within the same language, are in line with the polarity-marking properties of the individual elements observed in other clauses as well, and with the feature-based analysis proposed here in general.

\section{Summary} \label{sec:5summary}
Building on the theory put forward in the previous chapters, this chapter examined comparison constructions, including non-degree equatives (similatives), degree equatives, and comparatives expressing inequality. It was shown that while comparative semantics requires at least two projections, this does not necessarily result in  a complex left periphery: specifically, equatives may also rely on lexicalising the maximality operator in the matrix clause, while it was shown to be impossible in comparatives proper, due to constraints related to polarity. These constructions thus provide evidence for complex left peripheries; at the same time, no cartographic template is required -- in fact, such a template would be also problematic in accounting for the observed flexibility in grammaticalisation processes. The differences between equatives and comparative proper were also shown to exist cross-linguistically, providing further evidence for the differences being grounded in semantics. The presence of multiple projections in the left periphery also has a bearing on ellipsis phenomena and related information structural properties, as will be discussed in \chapref{ch:6}.
