\addchap{Acknowledgements}\largerpage[2]

This book is a revised an updated version of my habilitation thesis, completed in 2021 at the University of Potsdam. The research behind this book was funded by the German Research Fund (DFG), as part of my research project ``The syntax of functional left peripheries and its relation to information structure'' (BA 5201/1), carried out at the University of Potsdam, and later as part of my research project ``Asymmetries in relative clauses in West Germanic'' (BA 5201/2), carried out at the University of Konstanz.

Above all, I owe many thanks to Gisbert Fanselow  -- for all the discussions that we had over the years, for his support as my doctoral supervisor and well beyond that, and, indeed, for what I am as a linguist. Gisbert also reviewed my habilitation thesis and was part of the committee. The present version of this book benefited a lot from his comments; when writing up the revised version, I was hoping that we could discuss some of the new ideas together, though I already feared it would not be possible. Gisbert passed away in September 2022, following a serious illness. I hope the final version would have earned his good opinion.

For their extremely constructive and insightful remarks, I would like to thank Augustin Speyer, who was one of the reviewers of my habilitation thesis, as well as the two anonymous reviewers of the present book.

I owe many thanks to many other colleagues in Potsdam for assisting me in various ways during (and well beyond) the time I was working on this project. Extra thanks go to Malte Zimmermann for his insights concerning comparatives and polarity, as well as to Marta Wierzba for her indispensable help regarding setting up my experiments. I would also like to thank Doreen Georgi, Martin Salzmann, Radek Šimík, Boban Arsenijević, Craig Thiersch, Andreas Schmidt, Mira Grubic, Anne Mucha, Agata Renans, Flavia Adani, Joseph DeVeaugh-Geiss, and Teodora Radeva-Bork for all their questions and suggestions during these years, and for being such a nice bunch of fun people. Many thanks go to Ines Mauer, who had the task of administering my  project.

My work, especially regarding clause typing, benefited a lot from discussions with the late Ilse Zimmermann, whose incredible energy and wisdom have inspired me well beyond academic concerns. I would have very much liked to discuss the final work with her and can only hope that she would have indeed appreciated it as it is.

\begin{sloppypar}
I am indebted to Lisa Baudisch, my research assistant in the project ``The syntax of functional left peripheries and its relation to information structure'', whose help in evaluating the data from surveys and corpora were essential for the results presented in the book. Outside of Potsdam, I owe many thanks to my project cooperation partners for inspiring discussions and their useful suggestions concerning various parts of my research: Ellen Brandner (Konstanz\slash Stuttgart), Marco Coniglio (Berlin\slash Göttingen), Katalin É. Kiss (Budapest), Agnes Jäger (Köln\slash Jena), Marlies Kluck (Groningen), Svetlana Petrova (Wuppertal), Helmut Weiß (Frankfurt), Theresa Biberauer (Cambridge), Jeroen van Craenenbroeck (Leuven), Jürg Fleischer (Marburg\slash Berlin), Göz Kaufmann (Freiburg), and Lea Schäfer (Düsseldorf/Marburg).
\end{sloppypar}

As I was working at the University of Konstanz from 2018 to 2022, a substantial part of the work presented in this book goes back to this time. First and foremost, I would like to thank George Walkden for being such an extremely supportive and inspiring colleague. My work has benefited a lot from his useful insights and his indispensable help in managing bureaucratic nightmares, as well as for the syntax social gatherings organised in Seekuh.

Many thanks go to my colleagues Alexandra Rehn, Fernanda Barrientos Contreras, Tamara Rathcke, Katharina Kaiser, Svenja Schmid, Marc Meisezahl and David Diem for being such excellent fellows, including inspiring discussions in Seekuh and providing emotional support during digital semesters. I would also like to thank Hannah Booth, Henri Kauhanen, Georg Kaiser, Kajsa Djärv, Maria Francesca Ferin, Felix Frühauf, Frederik Hartmann, Natasha Korotkova, Erlinde Meertens, and Maryam Mohammadi for various questions and suggestions that proved to be highly inspirational. Special thanks go to Petr Biskup for being an ideal office mate and for donating me some chocolate.

I am highly indebted to Carmen Kelling for all her help and support in managing mostly administrative matters. Many thanks go to Anita Mademann for managing whatever counts as paperwork and to Achim Kleinmann for managing whatever is supposed to go electronically, including computers, printers and the occasionally mysterious connection between the two. Extra thanks go to Natalja Sander for her help in everyday matters. I would also like to thank Melanie Hochstätter and Florian Schönhuber for assisting me in teaching-related matters.

I am truly indebted to Julian Schlenker, my research assistant in the project ``Asymmetries in relative clauses in West Germanic'', who assisted me in finalising the corpus studies as well as in evaluating the data from questionnaires. I would also like to thank my teaching assistants for the lectures ``Structure and history of English I'', ``Structure and history of English II'', as well as the seminars ``English morphology'' and ``English syntax'', as well as my students of these and various other classes at the University of Konstanz, as their original and intelligent questions have been inspirational for my research.

Since I started working as an assistant professor at the University of Amsterdam in 2022, a smaller part of the work presented in this book was written in this period. I would like to thank Bertus van Rooy, Enoch Aboh and Monique Flecken for creating such a positive working environment and for Ronel Wasserman for being a wonderful colleague and office mate.  

My work has benefited substantially from the comments of anonymous reviewers I received for my papers and conference abstracts submitted during my doctoral project. I also owe many thanks to the audiences of various conferences I attended while I was working on this book, of which I would like to mention the following: the workshop ``Non-canonical Verb Positioning in Main Clauses'' of the ``39th Annual Conference of the German Linguistics Society'' in 2017 in Saarbrücken (and Werner Frey in particular), the ``Tenth International Conference on Middle English'' in 2017 in Stavanger (and Artur Bartnik and Rafał Molencki in particular), the ``Lund--Potsdam--Budapest Linguistics Colloquium'' in  2017 in Lund (and Hans-Martin Gärtner and Gunlög Josefsson in particular), the ``13th International Conference on the Structure of Hungarian'' in 2017 in Budapest (and Istv\'an Kenesei, Marcel den Dikken, Veronika Hegedűs and Adrienne Dömötör in particular), the ``6th International Conference on Late Modern English'' in 2017 in Uppsala (and Bianca Widlitzki and Victorina Gonz\'alez-D\'iaz in particular), the ``23. Arbeitstagung der Skandinavistik'' in 2017 in Kiel (and Christer Lindqvist in particular), ``SaRDiS 2017: Saarbrücker Runder Tisch für Dialektsyntax'' in 2017 in Saarbrücken (and Augustin Speyer, Melitta Gillmann, Andreas Pankau and Guido Seiler in particular), ``Cartography and Explanatory Adequacy'' in 2018 in Barcelona (and Klaus Abels, Volker Struckmeier and Dennis Ott in particular), the ``Germanic Sandwich 2019'' in 2019 in Amsterdam (and Stefan Sudhoff and Matthias Hüning in particular), the ``Comparative Germanic Syntax Workshop 34'' in 2019 in Konstanz (and Elly van Gelderen, Theresa Biberauer and Roland Hinterhölzl in particular), ``Generative Grammatik des Südens 2019'' in 2019 in Frankfurt (and Daniel Hole and Philipp Weisser in particular), the ``14th Annual Meeting of the Slavic Linguistics Society'' (and Luka Szucsich and Steven Franks in particular).

I owe many thanks to my informants for their valuable judgements. Many of them have been mentioned above; in addition, I am highly grateful to all my informants for completing my cross-Germanic survey: here I would also like to thank Ida Larsson for her help in finding Norwegian informants and Jóhannes Gísli Jónsson for his help in finding Icelandic informants.

Finally, I would like to thank my friends and my family for their support and encouragement during the time I was working on this book (and well beyond that). In particular, million thanks go to my parents for their love and for all the reindeer stuff (they will understand). I would also like to thank Ute and Norbert for all their love and kindness. And I am highly grateful to my turtles for being such sweet distractions from time to time. Finally, my most special thanks go to Ralf for everything but especially for just being as he is.

