\chapter{Doubly Filled COMP in interrogatives and the role of finiteness in V2} \label{ch:3}
\section{Introduction} \label{sec:3introduction}
%\chaptermark{Doubly Filled COMP in interrogatives}
Using the framework established in \chapref{ch:2}, this chapter is devoted to the left periphery of interrogative clauses, especially embedded interrogative clauses. In particular, I will examine various combinations of operators and complementisers in the left periphery that are allowed in certain dialects but not in others. The impossibility of the relevant combinations in standard West Germanic languages has been referred to as the ``Doubly Filled COMP Filter'' in the literature, suggesting some inherent syntactic ban on the configurations; however, the generalisation is not compatible with empirical data from non-standard dialects and from other languages that allow the combinations in question. I will argue that the existence of such combinations does not require or justify postulating designated projections, as in cartographic approaches. Instead, I propose that doubling patterns are compatible with a minimal CP. Further, I argue that the insertion of a finite complementiser is not an indication of a separate projection for finiteness but merely the consequence of the regular Germanic pattern of lexicalising a finite C overtly, as can be seen in V2 patterns as well. In order to understand the novelty of the proposed analysis better, I will first review some previous works addressing the same question.

This chapter is structured as follows. Section \ref{sec:3previous} reviews some previous accounts, notably the fundamental article by \citet{chomskylasnik1977}, which introduced the notion of Doubly Filled COMP, as well as the paper by \citet{bayerbrandner2008}, which made important steps towards a feature-based accounts regarding German dialect data. Section \ref{sec:3approaches} discusses two major approaches to Doubly Filled COMP that will be central to the discussions in this chapter. Section~\ref{sec:3embedded} examines variation in Doubly Filled COMP in embedded constituent questions. Section \ref{sec:3embeddedpolar} shows that similar variation can be found also in embedded polar questions, so that the conclusions regarding the classical Doubly Filled COMP setup have relevance beyond the particular construction. Section \ref{sec:3doubly} presents the core part of the analysis, establishing a connection between Doubly Filled COMP and V2. Section \ref{sec:3long} shows that the predictions made by the model hold in long movement constructions as well.

\section{Previous accounts} \label{sec:3previous}
\subsection{The problems to be discussed} \label{sec:3problems}
In Standard English, Standard German and Standard Dutch, there is no overt complementiser with an overt interrogative operator. This is illustrated in (\ref{whothat}) for English embedded interrogatives:

\ea	I don't know \textbf{who (*that)} has arrived. \label{whothat}
\z

As can be seen, the complementiser \textit{that} is not permitted in Standard English in embedded constituent clauses. However, there are languages and also many West Germanic varieties that allow such patterns, as in the following examples from non-standard English (\citealt[331, ex. 1]{baltin2010}) and from non-standard Dutch:\footnote{The Dutch data stem from the cross-Germanic survey of \citet[32]{bacskaiatkaribaudisch2018}. The informant who provided the example sentence marked it as grammatical but quite informal (this informant grew up predominantly in Amersfoort/Nijkerk and stated that she did not consider (\ref{dutchdfc}) to be part of her own variety).}

\ea \label{dfcint}
\ea[\%]{They discussed a certain model, but they didn't know \textbf{which model that} they discussed.}
\ex[\%]{\gll Peter vroeg \textbf{wie} \textbf{dat} er boeken leuk vindt. \label{dutchdfc}\\
Peter asked.\textsc{3sg} who that of.them books likeable finds\\
\glt `Peter asked who liked books.'}
\z
\z

On the other hand, the specifier of the CP and the C head can be both lexicalised overtly in main clauses, as in T-to-C movement in English interrogatives, and in V2 clauses in German and Dutch main clauses. Consider the examples for main clause interrogatives in Standard English:

\ea \label{ttoc}
\ea	\textbf{Who saw} Ralph? \label{whosawch3}
\ex	\textbf{Who did} Ralph see? \label{whodidch3}
\z
\z

In this case, doubling in the CP involves a \textit{wh}-operator in [Spec,CP] and a verb in C. T-to-C movement is visible by way of \textit{do}-insertion in (\ref{whodidch3}), though not in (\ref{whosawch3}): in principle, one might analyse (\ref{whosawch3}) as not involving the movement of the verb to C, but the CP is clearly doubly filled in (\ref{whodidch3}).

Similarly, in German (and Dutch) V2 declarative clauses a verb moves to C, while another constituent moves to [Spec,CP] due to an [edge] feature (see \citealt{thiersch1978diss, denbesten1989, fanselow2002, fanselow2004isis, fanselow2004, frey2005}). Consider:

\ea \label{v2}
\ea \gll \textbf{Ralf} \textbf{hat} morgen Geburtstag.\\
Ralph has tomorrow birthday\\
\glt `Ralph has his birthday tomorrow.'
\ex \gll \textbf{Morgen} \textbf{hat} Ralf Geburtstag.\\
tomorrow has Ralph birthday\\
\glt `Ralph has his birthday tomorrow.'
\z
\z

As can be seen, the fronted finite verb is preceded by a single constituent in each sentence, and since the first constituent is not a clause-typing operator in either case, it is evident that doubling in the CP in V2 clauses is independent of the interrogative property.

It is therefore clear that the Doubly Filled COMP Filter should be more restricted in its application domain. In principle, one could say that an operator and a complementiser with largely overlapping functions are not permitted to co-occur in standard West Germanic languages, or that the Doubly Filled COMP Filter should be seen as some kind of an economy principle. Still, the problem remains that the notion of the Doubly Filled COMP Filter implies that the C head and [Spec,CP] would be filled without the Filter and that the Filter is responsible for ``deleting'' the content of C. 

Regarding this, at least two major questions arise. First, it should be clarified what requirement is responsible for filling C even in the presence of an overt operator in [Spec,CP], as in (\ref{dfcint}). Second, the question is what kinds of elements may appear in C: in particular, if elements other than complementisers can satisfy the requirement of filling C, then the deletion approach is probably mistaken.

In addition, there is a theoretical problem with the notion of the Filter, which arises from a merge-based, minimalist perspective, while it is less problematic in X-bar theoretic terms. X-bar theoretic notions can at best be taken as descriptive designators that are derived from more elementary principles, along the lines of \citet{kayne1994} and \citet{chomsky1995}. Under this view, the position of an element (specifier, head, complement) is the result of its relative position when it is merged with another element, and which element is selected as the label. By contrast, the notion of the Doubly Filled COMP Filter, as applied to a CP (as in \citealt{baltin2010}), implies that a phrase is generated with designated, pre-given head and specifier positions, and that there are additional rules on whether and to what extent they can be actually ``filled'' by overt elements. In a merge-based account, there are no literally empty positions, as no positions are created independent of merge: zero heads and specifiers reflect elements that are either lexically zero or have been eliminated by some deletion process (e.g. as lower copies of a movement chain or via ellipsis). In other words, Doubly Filled COMP effects should be accounted for in a way other than by referring to a pre-given XP.

\subsection{Surface filters -- \citet{chomskylasnik1977}} \label{sec:3chomskylasnik}
In order to see the problem in the relevant context, I will first review the article by \citet{chomskylasnik1977}, which introduced the notion and the phenomenon of Doubly Filled COMP into the literature.

The key observation made by \citet[425]{chomskylasnik1977} is that ``it is necessary to develop some notion of well-formedness for surface structure'': they refer to this condition as a ``(surface) filter'', following previous investigations of \citet{chomsky1965, chomsky1973} and \citet{perlmutter1971}. Rules operating on the surface structure are not necessarily formulated as filters, though (\citealt[426]{chomskylasnik1977}).

One important area of filters is the complementiser domain; following \citet{bresnan1971, bresnan1972diss}, \citet[426]{chomskylasnik1977} assume that the basic structure of sentences is ``COMP + S''. Consider (\citealt[426, ex. 4]{chomskylasnik1977}):

\ea \label{forthatwhether}
\ea {[}\textsubscript{S$'$} [\textsubscript{COMP} \textbf{for}][\textsubscript{S} John to leave]] \underline{\hspace{1cm}} would be a mistake 
\ex {[}\textsubscript{S$'$} [\textsubscript{COMP} \textbf{that}][\textsubscript{S} John has left]] \underline{\hspace{1cm}} is obvious
\ex {[}\textsubscript{S$'$} [\textsubscript{COMP} \textbf{whether}][\textsubscript{S} John left]] \underline{\hspace{1cm}} is unclear
\z
\z

According to this, all the three boldfaced elements -- \textit{for}, \textit{that} and \textit{whether} -- fall under the category COMP.

In order to provide a sufficiently restrictive theory of grammar, with the ultimate aim of reaching explanatory adequacy, \citet[428]{chomskylasnik1977}, referring to \citet{jackendoff1972} and \citet{chomsky1972}, ``assume the general framework of the extended standard theory (EST)'', and ``that the grammar consists of a base with a categorial component and a lexicon, a transformational component, and two systems of interpretive rules, a phonological and a semantic component''. The categorial component generates abstract phrase markers, and by way of inserting lexical items in these abstract phrase markers, base phrase markers are derived, whereby rules associated with the transformational component serve to yield surface structures (\citealt[428]{chomskylasnik1977}). Surface structures must be well-formed: this can be achieved via surface filters and by the interpretive rules (\citealt[428]{chomskylasnik1977}). There are also base phrase markers that map directly into well-formed surface structures: these are referred to as ``deep structures'' (\citealt[428]{chomskylasnik1977}). Regarding the base component of core grammar, \citet[430--431]{chomskylasnik1977} assume the rules of X-bar theory to be operative. Base structures are transferred to the semantic component and to the phonological component independently: deletion, filters, rules of phonology and stylistic rules apply in the phonological component (\citealt[431]{chomskylasnik1977}). Filters apply after deletion has taken place, and phonological rules then assign a phonological representation, which may further be subject to stylistic rules (\citealt[433]{chomskylasnik1977}).

Regarding complementisers, \citet[434]{chomskylasnik1977} mention that apart from \textit{that} in tensed and \textit{for} in infinitival clauses, see (\ref{forthatwhether}), the COMP position may be empty: the assumption is that there is a rule for free deletion applying to complementisers. Consider (\citealt[434, ex. 13]{chomskylasnik1977}):

\ea I think [John left]
\z

On the other hand, \textit{wh}-movement also targets the COMP position: the \textit{wh}-element is placed to the left of the complementiser (\citealt[434]{chomskylasnik1977}). Consider now the following examples (taken from \citealt[435, ex. 17]{chomskylasnik1977}):

\ea \label{objrel}
\ea the man \textbf{who that} I saw \label{compwhothat}
\ex the man \textbf{that} I saw \label{compthat}
\ex the man \textbf{who} I saw 
\ex the man I saw \label{compdel}
\z
\z

The pattern in (\ref{compwhothat}) represents the underlying structure (and as such it contains no grammaticality markers). Assuming that elements in COMP can be deleted freely, the derivation of (\ref{compthat}--\ref{compdel}) is straightforward from an underlying (\ref{compwhothat}): however, while (\ref{compwhothat}) is possible in other languages and in earlier stages of English, it has to be excluded from Modern English (\citealt[434--435, 446]{chomskylasnik1977}). This is done in terms of a surface filter (\citealt[435, ex. 18]{chomskylasnik1977}, cf. \citealt{keyser1975}):

\ea[*]{{[}\textsubscript{COMP} \textit{wh}-phrase complementiser] \label{dfcf}}
\z

This is a filter ``blocking doubly-filled COMP'' (\citealt[461]{chomskylasnik1977}). The filter in (\ref{dfcf}) applies not only in finite clauses but also in infinitival clauses, and together with various other surface filters certain ungrammatical configurations can be ruled out (see \citealt[450--470]{chomskylasnik1977}).

Note that the clauses in (\ref{objrel}) are object relative clauses; the paradigm is different in subject relative clauses (\citealt[435, ex. 19]{chomskylasnik1977}):

\ea
\ea[*]{the man \textbf{who that} met you \underline{\hspace{1cm}} is my friend \label{subjwhothat}}
\ex[]{the man \textbf{that} met you \underline{\hspace{1cm}} is my friend}
\ex[]{the man \textbf{who} met you \underline{\hspace{1cm}} is my friend}
\ex[*]{the man met you \underline{\hspace{1cm}} is my friend \label{subjcompdel}}
\z
\z

The configuration in (\ref{subjwhothat}) is excluded by the rule in (\ref{dfcf}) but (\ref{subjcompdel}), contrasting with (\ref{compdel}), must be excluded by an additional rule. \citet[435, ex. 20]{chomskylasnik1977} formulate this as a filter:

\ea[*]{{[}\textsubscript{NP} NP tense VP] \label{npfilter}}
\z

The same rule is supposed to be operative in cases where a \textit{that}-clause is fronted and the complementiser cannot be deleted (\citealt[436, ex. 21]{chomskylasnik1977}):

\ea
\ea[]{I think that he left.}
\ex[]{I think he left.}
\ex[]{That he left is a surprise.}
\ex[*]{He left is a surprise.}
\z
\z

In essence, the rule under (\ref{npfilter}) is taken to be perceptual in nature: the linear sequence can be parsed as a main clause as well, which clashes with fact that they are actually embedded clauses (\citealt[436]{chomskylasnik1977}). The perceptual nature of the rule in (\ref{npfilter}) is supported by garden-path sentences (\citealt[438, ex. 25 and 26]{chomskylasnik1977}):

\ea
\ea The horse raced past the barn fell. \label{raced}
\ex The ball thrown past the barn fell. \label{thrown}
\z
\z

In (\ref{raced}), the verb \textit{raced} is ambiguous between the past tense form and the participle, and is naturally interpreted as a tensed verb preceded by the subject: this interpretation breaks down when the tensed verb \textit{fell} is reached in parsing. The sentence is therefore generally taken to be ungrammatical by speakers (\citealt[438]{chomskylasnik1977}). The same problem does not arise with (\ref{thrown}), where \textit{thrown} is unambiguously a participle and hence (\ref{npfilter}) does not apply. 

As noted by \citet[438]{chomskylasnik1977}, the rule under (\ref{npfilter}) is not universal: even English has dialects that allow the subject \textit{wh}-element to be deleted in (\ref{subjcompdel}). Further, there are certain configurations where \textit{that} must be overt, yet (\ref{npfilter}) would not be violated even if \textit{that} were covert. Consider the following example taken from \citet[484, ex. 174a--d]{chomskylasnik1977}:

\ea
\ea the fact [\textbf{that} John was here] surprised me
\ex it came as a surprise to me [\textbf{that} John was here]
\ex it is unlikely [\textbf{that} John is here]
\ex {[}\textbf{that} John is here], I have no reason to think
\z
\z

As \citet[484]{chomskylasnik1977} observe, the distribution of finite clauses is similar to that of infinitives. Therefore, there is a more general filter prohibiting the relevant sequence, as indicated in (\ref{thatfilter}) below (\citealt[485, ex. 178]{chomskylasnik1977}), where F is a subfeature of +V:

\ea *[\textsubscript{$\alpha$} NP tense VP], unless $\alpha$ $\neq$ NP and is adjacent to and in the domain of [+F], \textit{that}, or NP \label{thatfilter}
\z

In addition, there is a separate filter ruling out the appearance of an overt element in COMP in root clauses (\citealt[485]{chomskylasnik1977}).

Regarding COMP, \citet[444--445]{chomskylasnik1977} assume that $\pm$WH must be indicated: a clause must be identified either as declarative (which encompasses relative clauses besides declaratives) or as interrogative (which encompasses both direct and indirect questions). The idea is that \textit{for} and \textit{that}, both --WH, can be deleted freely, since --WH is not a lexical category: accordingly, these elements ``are not lexical items but rather semantically null feature sets generated by the categorial component of the base'' (\citealt[447]{chomskylasnik1977}). Apart from the complementiser, the \textit{wh}-element can be deleted in relative clauses: however, this does not apply for interrogative clauses (\citealt[447]{chomskylasnik1977}). \citet[447]{chomskylasnik1977} propose that this is so because the \textit{wh}-word has semantic content in interrogatives (it is a quantifier) but not in relative clauses.

The study by \citet{chomskylasnik1977} is important for various reasons. On the one hand, it offers an overview of phenomena that had not been substantially discussed previously in the literature. On the other hand, it indicates clearly that the requirements behind them are not necessarily strictly derivational in nature but are rather related to surface restrictions.

Nevertheless, there are certain problems that arise as well, both in terms of theory and empirical data. Regarding the overtness of \textit{that}, it was pointed out in the literature later that there are additional factors (such as the presence of adverbs) to consider and judgements are not necessarily clear; see the discussion in connection with \citet{rizzi1997} and especially the response of \citet{sobin2002}, as was discussed in \chapref{ch:2}. But even if one restricts oneself to the particular case of Doubly Filled COMP, certain questions arise, especially from a minimalist perspective.

First, the model used by \citet{chomskylasnik1977} is essentially X-bar theory, yet the complementiser layer is still referred to as merely COMP. This leaves only one option for \textit{wh}-movement, which is to adjoin the \textit{wh}-phrase to COMP itself. While this is indeed possible with a head-sized \textit{wh}-element like \textit{who}, as is indeed the case in all the examples given by \citet{chomskylasnik1977}, the adjunction of a complex phrase (such as \textit{whose books}) to a head element in COMP is problematic. This dilemma can be overcome by adopting the X-bar schema for COMP, that is, by treating it as CP, whereby the complementiser is the head of the CP and the \textit{wh}-element can move to the specifier, irrespective of whether the \textit{wh}-element is head-sized or phrase-sized. The same applies to a merge-based account, where the particular position COMP does not have to be assumed either.\largerpage

However, there is a second problem that arises either way and that applies to the filter itself. The Doubly Filled COMP Filter may effectively describe the ban on doubling in Standard English, yet the question remains why this should be so. More precisely, if one assumes the element \textit{that} to be an abstract feature set, it is not clear why it is assumed to be deleted rather than zero in the first place, given that zero elements are in principle possible in generative grammar. The deletion mechanism responsible for the elimination of certain elements is likewise not clear: it is assumed to take place in the phonological component, yet the filter rules are in part syntactic. Moreover, as can be seen especially in the case of (\ref{thatfilter}), the filter rules are very specific and do not follow from independently motivated factors (either in syntax or in the interfaces).\footnote{This is in fact also related to the next point of criticism, namely that the filter is not categorical. \citet[109--118]{fanselowcavar2001} argue that true partial \textit{wh}-movement in Bahasa Indonesia can be analysed as the result of surface filters in an optimality theoretic fashion, at least in a framework making use of cyclic optimisation, as proposed by \citet{mueller2000, mueller2002}. In this language, the \textit{meng}- prefix can appear on a verb if the \textit{wh}-element is postverbal, but if the \textit{wh}-element moves across the verb, landing either in the same clause or in a higher clause, \textit{meng}- has to be deleted. They attribute this to the \textit{wh}-phrase moving into the specifier of the phrase headed by \textit{meng}- (Agr-O), whereby either element has to be deleted: if \textit{meng}- is deleted, the higher copy of the \textit{wh}-element is realised, and if the higher copy of the \textit{wh}-element is deleted, \textit{meng} remains overt. This is ultimately carried out by some surface filter resembling the Doubly Filled COMP Filter (\citealt[115--116]{fanselowcavar2001}), not specified further; this raises the same concerns as expressed here regarding the Doubly Filled COMP Filter. More importantly, however, while the ban on the specific doubling in Bahasa Indonesia is apparently categorical, the same is not true for West Germanic doubly filled COMP patterns: these show not only dialectal but also intra-speaker variation and are also sensitive to the specific properties of the \textit{wh}-elements, leading to gradient variation and optionality in many cases (without triggering interpretive differences). This seriously questions the applicability of filters for these cases, since relevant data would automatically be filtered out. The same considerations of course also apply to the proposal made by \citet{pesetsky1998}, who assumes that complementisers are subject to constraints requiring them to be spelt out at the left edge of the CP, leading to the deletion of a non-initial complementiser or to the deletion of a fronted \textit{wh}-element in relative clauses. The non-doubling options in Pesetsky's system may all survive EVAL due to a ``constraint tie'', leading to optionality in effect, but in case one were to loosen the remaining constraint for non-standard varieties allowing doubling, this would amount to there being no constraints in this respect, leaving also no explanation for observable preferences.} In other words, the Doubly Filled COMP Filter does not explain either the insertion or the deletion of an overt \textit{that}.

Third, the way the filter is supposed to work is not entirely compatible with the empirical data. As \citet{chomskylasnik1977} also note, the Doubly Filled COMP Filter was not always operative in the history of English; similarly, subject relative clauses with no visible COMP are also attested in certain varieties. In fact, variation is attested both synchronically and diachronically in both cases. One may suppose that variation can be accounted for by assuming that the filter is operative in certain dialects (and languages) but not in others. However, the problem is that there is intra-speaker variation as well, and as will be discussed in the \sectref{sec:3bayerbrandner} in connection with the analysis of \citet{bayerbrandner2008}, there are speakers of Bavarian and Alemannic who make a fairly clear difference between head-sized and phrase-sized \textit{wh}-expressions with respect to the overtness of the complementiser. This sort of difference is not predicted by the filter, which is assumed to operate automatically; alternatively, one may try to apply more specific restrictions, but this would only make the surface filters even more descriptive in nature.

Fourth, related to this, while the doubling in COMP in English may involve the combination of a \textit{wh}-phrase (either interrogative or relative) and \textit{that} in many non-standard varieties, other combinations do not prevail; specifically, the combination with \textit{if} is not regularly attested.\footnote{An example for the combination \textit{whether if} is provided by \citet[96, ex. 82]{vangelderen2004}:

\ea The local authority will know \textbf{whether if} they let the council house to the tenant.\\
(BNC-FC3-80)
\z

In this case, \textit{whether} and \textit{if} are in the same left periphery; a more detailed discussion will follow in \sectref{sec:3embeddedpolar}. I have not found evidence for similar combinations in constituent questions, and the co-occurrence of \textit{wh}-elements and \textit{if} is banned in Standard English:

\ea *We were asked \textbf{who if} should be responsible for cleaning the kitchen.
\z

Note that the restriction obviously does not extend to cases like the following:

\ea We were asked \textbf{who}, if anyone, should be responsible for cleaning the kitchen.
\z

In this case, \textit{who} and \textit{if} do not occur in the same left periphery since \textit{if} introduces a parenthetical clause that is simply string-adjacent to the \textit{wh}-operator in the host clause.} While it seems plausible that a \textit{wh}-element does not combine with a COMP already specified as interrogative, Dutch dialectal data, as will be discussed later in \sectref{sec:3dutch}, indicate that this is possible in constituent questions. In addition, while in English the COMP element occurring in doubling patterns in relative clauses is \textit{that}, just like in embedded questions, this is not necessarily the case in other languages: in German dialects, relative pronouns may co-occur with the regular dialectal relative complementiser \textit{wo}, while in embedded questions the COMP involved is the finite subordinator \textit{dass} `that'. Given this difference, \textit{wo} can hardly be taken to be merely the overt realisation of the finite --WH COMP. Note also that treating relative operators as \textit{wh}-elements cannot be adopted universally: in German, relative pronouns in headed relative clauses (\textit{der}/\textit{die}/\textit{das}) are not identical to \textit{wh}-pronouns, as they in fact derive from demonstrative pronouns. These issues will be discussed in \chapref{ch:4} in detail. What matters for us here is that the actual distribution of Doubly Filled COMP patterns differs from what the theory of \citet{chomskylasnik1977} predicts, and therefore an alternative account is desirable.

\subsection{Variation in the CP -- \citet{bayerbrandner2008}} \label{sec:3bayerbrandner}
In this section, I will review the analysis of \citet{bayerbrandner2008}, which discusses some important empirical issues concerning Doubly Filled COMP patterns in embedded interrogatives in German dialects. This proposal is particularly important because it adopts a flexible approach to the CP that can avoid certain problems associated with the model used by \citet{chomskylasnik1977}.

As \citet[87]{bayerbrandner2008} note, while the Doubly Filled COMP Filter is operative in Standard English and Standard German, as well as other standard Germanic languages, earlier stages of these languages and dialects do not necessarily observe this rule (see \citealt{bayer1984} for Bavarian, \citealt{haegeman1992} for West Flemish, \citealt{pennerbader1995} and \citealt{schoenenberger2006} for Swiss German). The traditional assumption is that the insertion of the complementiser is optional and redundant but as descriptive works on Alemannic and Bavarian (such as \citealt{noth1993}, \citealt{schiepek1899}, \citealt{steininger1994}) indicated earlier, there seem to be certain restrictions.\footnote{As shown by \citet{schallertbidese2021}, Doubly Filled COMP patterns in interrogatives are not restricted to Germanic but they can be found in various Romance and even Slavic varieties in the Alpine region, constituting an areal (``Sprachbund'') phenomenon. They also show that Germanic and Romance varieties are structurally similar in embedded clauses and differ especially in root clauses. As will be discussed later, the asymmetries and restrictions observed in connection with Doubly Filled COMP show differences even across Germanic and the same applies to differences from other language groups. See also \citet{poletto2000} and \citet{polettovanelli1997} on Northern Italian varieties.}  

\citet[88--89]{bayerbrandner2008} conducted a questionnaire study on Bavarian and Alemannic; the judgements are not absolutely clear-cut but the relative differences can definitely be observed, on the basis of which a hierarchy can be established. The best results for Doubly Filled COMP patterns are achieved with ``genuine'' \textit{wh}-phrases that contain a DP or a PP in addition to the \textit{wh}-word, as illustrated by the following example from Bavarian (\citealt[88, ex. 3a]{bayerbrandner2008}):\footnote{Note that the same differences do not hold in all varieties: as will be discussed later, ``symmetric'' varieties tend not to make a distinction between kinds of \textit{wh}-elements in terms of Doubly Filled COMP. The Northern Italian data presented by \citet{polettovanelli1997} suggest that the preferences in Romance may in fact be the other way round; that is, doubling patterns may well rather occur with subjects. This is not unexpected because the rules underlying such patterns in Germanic (which will be shown to be related to a lexicalisation requirement on C) do not straightforwardly carry over to Romance. \citet{schallertbidese2021} suggest that doubling patterns in non-Germanic languages in the Alpine region may well be due to contact effects: if so, it seems that Romance varieties have adopted the surface syntactic pattern in the less marked functions (such as subjects) and have not necessarily extended it to the more marked functions (cf. the Noun Phrase Accessibility Hierarchy of \citealt{keenancomrie1977}).}

\ea \gll I frog-me, \textbf{fia} \textbf{wos} \textbf{dass}-ma an zwoatn Fernseher braucht. \label{wosdass}\\
I ask-\textsc{refl} for what that-one a second TV needs\\
\glt `I wonder what one needs a second TV for.'
\z

The lowest ratings for Doubly Filled COMP patterns are achieved with the word-sized \textit{wh}-elements \textit{wer} `who.\textsc{nom}', \textit{wen} `who.\textsc{acc}', \textit{was} `what', \textit{wie} `how', and \textit{wo} `where', as illustrated by the following example from Alemannic (\citealt[88, ex. 5b]{bayerbrandner2008}):

\ea[*]{\gll I wett gern wisse, \textbf{wa} \textbf{dass} i do uusfülle muss. \label{wadass}\\
I would gladly know what that I there out-fill must\\
\glt `I'd like to know what I have to fill out there.'}
\z

In addition, there are complex word-sized \textit{wh}-elements that have an intermediate status with respect to judgements: these are \textit{warum} `why', \textit{wieviel} `how much', \textit{wem} `who.\textsc{dat}' (\citealt[89]{bayerbrandner2008}). According to \citet[89]{bayerbrandner2008}, the intermediate status of these elements (that is, that they are more similar to complex \textit{wh}-phrases regarding the acceptability of \textit{dass}) is due to their complex internal structure: \textit{wieviel} is evidently composed of \textit{wie} `how' and \textit{viel} `much', \textit{warum} is underlyingly the combination of the preposition \textit{um} and \textit{was} `what', and \textit{wem} is similar to a PP in that it is internally complex, the dative acting as an adpositional head (cf. \citealt{bayerbadermeng2001}).

The conclusion is that word-sized \textit{wh}-elements are in complementary distribution with the complementiser \textit{dass}, and are thus located in C rather than in the specifier (\citealt[89]{bayerbrandner2008}). The assumption is that ``embedded questions must be syntactically typed'' for interrogativity, and this is possible either by the insertion of a Q-particle or by the movement of a \textit{wh}-element (\citealt[89]{bayerbrandner2008}, citing also the ``Clausal Typing Hypothesis'' of \citealt{cheng1991diss}). Moreover, \citet[89]{bayerbrandner2008} propose that \textit{wh}-words like \textit{wer} are not only lexically specified for [wh] but they also have a latent C-feature, which can (but does not have to) be activated in the derivation. The \textit{wh}-element then undergoes head movement. Regarding head movement, \citet[89]{bayerbrandner2008} follow \citet{koeneman2000diss, koeneman2002}, \citet{bury2002}, \citet{fanselow2002} and \citet{brandner2004} in assuming that head movement can be treated ``as self-attachment of a head to the highest maximal projection iff the head in question contains a (so-far unactivated) categorial feature by which this head is able to induce its own X-bar projection''.

According to \citet[90]{bayerbrandner2008}, a  \textit{wh}-word is a ``typing particle'' and simultaneously a complementiser, similarly to \textit{ob} `if' in polar questions; however, a \textit{wh}-word also expresses a semantic restriction and binds a variable in the VP. In complex \textit{wh}-phrases, however, the C-feature cannot be activated as the \textit{wh}-word merges with another element in its base position, and hence these complex phrases move to a maximal projection, making the insertion of \textit{dass} possible or even necessary, depending on the exact dialect (see \citealt[90]{bayerbrandner2008}). Similarly, the C-feature remains un-activated in \textit{wh}-in situ constructions and in multiple \textit{wh}-questions (\citealt[90]{bayerbrandner2008}).

The question arises how the proposal relates to Chain Uniformity, since the base position seems to be a phrase-sized (maximal) projection, while the landing site (the C head) is a head-sized, minimal projection. The phrase-sized nature of the base-generation site is indicated by extraction patterns involving adjuncts, as in the following example (\citealt[90, ex. 9]{bayerbrandner2008}):

\ea \gll Ich will wissen, \textbf{wen} sie [\sout{\textbf{wen}} aus Paris] gesehen hat. \label{wenparis}\\
I want.\textsc{1sg} know.\textsc{inf} who.\textsc{acc} she \phantom{[}who.\textsc{acc} from Paris seen has\\
\glt `I want to know who she saw from Paris.'
\z

In (\ref{wenparis}), the \textit{wh}-element \textit{wen} is extracted from a complex phrase that includes the adjunct \textit{aus Paris} `from Paris': extraction would not be permitted from a head position. \citet[90--91]{bayerbrandner2008} argue that \textit{wh}-elements like \textit{wen} can have a dual interpretation, that is, they are both minimal and maximal without there being any morphological difference between the two, thus satisfying a ``morphological condition of chain uniformity''.

\citet[91]{bayerbrandner2008} argue that the proposal is also in line with economy principles, such as the Head Preference (or Spec-to-Head) Principle of \citet[10]{vangelderen2004}: this essentially says that by merging an element as a head instead of as a specifier to a head, the configuration is more economical as it involves the merger of fewer elements. This is a standard way of operators moving to [Spec,CP] grammaticalising into complementisers. \citet[91]{bayerbrandner2008} suggest that the introduction of a latent [C] feature is the first step in such grammaticalisation processes.

Importantly, the proposal ``entails that a single lexical head may host several functional features that are projected to the maximal projection'' (\citealt[91]{bayerbrandner2008}), in line with the proposal of \citet{sobin2002} for a very minimal CP (cf. also \citealt{bobaljikthrainsson1998} for similar views in terms of the IP), and feature checking does not necessarily involve specifier--head agreement either (\citealt[91]{sobin2002}).

In addition to the syntactic behaviour, there is phonological evidence for the head status of the head-sized \textit{wh}-elements in question, such as \textit{n}-intrusion in Alemannic and \textit{r}-intrusion in Bavarian. Consider the following examples from Alemannic (\citealt[92, ex. 13a, 14 and 16a]{bayerbrandner2008}):

\ea \label{nintrusion}
\ea[]{\gll \ldots{} wa-\textbf{n}-er tuet \label{wasembedded}\\
{} what-N-he does\\
\glt `what he does'}
\ex[*]{\gll Wa-\textbf{n}-isch denn do passiert? \label{wasmain}\\
what-N-is PRT there happened\\
\glt `What has happened here?'}
\ex[*]{\gll \ldots{} [wege wa]-\textbf{n}-er sich so uffregt \label{wegenwasembedded}\\
{} \phantom{[}because what-N-he REFL so excites\\
\glt `because of what he gets so upset'}
\z
\z

As discussed by \citet[92]{bayerbrandner2008}, citing \citet{ortmann1998}, ``\textit{n}-in\-tru\-sion is only possible if the clitic pronoun is right-adjacent to a functional head''. Regarding \textit{wh}-elements, then, \textit{n}-in\-tru\-sion is possible only if the element is in C (the same holds for \textit{r}-intrusion in Bavarian). As shown by (\ref{nintrusion}), this is indeed the case: in main clause questions, such as (\ref{wasmain}), the \textit{wh}-element cannot be in C (that being the position where the finite verb moves), and hence \textit{n}-intrusion is not licensed. The same applies to complex \textit{wh}-phrases, such as \textit{wegen was} in (\ref{wegenwasembedded}). However, the \textit{wh}-element \textit{was} in an embedded clause like (\ref{wasembedded}) does allow \textit{n}-intrusion: obviously, the syntactic status of \textit{was} in this case must be different from that in (\ref{wasmain}).

Naturally, the availability of these \textit{wh}-elements in C does not exclude them as proper operators in certain cases even in the dialects that otherwise treat them as elements inserted into C. One such case is when the \textit{wh}-element is contrastively focussed, as in (\ref{wofocus}) below (\citealt[93, ex. 18]{bayerbrandner2008}, quoting \citealt[424]{noth1993}):

\ea \gll Ich woass \textbf{WO} \textbf{dass} er abfahrt aber noit WENN. \label{wofocus}\\
I know where that he leaves but not-yet when\\
\glt `I know WHERE it (the train) will leave but not WHEN.'
\z

In this case, \citet[93]{bayerbrandner2008} argue, following \citet{cardinalettistarke1999}, that ``focal stress requires strong pronouns and that strong pronouns have a richer syntactic structure than weak or clitic pronouns'', resulting in the same ban on merging them into C as with complex \textit{wh}-phrases.

There are some important cross-linguistic differences regarding these issues. Citing \citet{westergaardvangsnes2005} and \citet{vangsnes2005}, \citet[93--94]{bayerbrandner2008} point out that in North Norwegian dialects, simplex \textit{wh}-elements may be inserted into C even in main clauses.\footnote{See \citealt[21--22]{taraldsen1986} for the original observation.} Consider the following example from \citet[93, ex. 19 and 20]{bayerbrandner2008}:

\ea
\ea[\%]{\gll \textbf{Ka} \textbf{sa} han Ola? \label{kaverb}\\
what said the Ola\\
\glt `What did Ola say?'}
\ex[]{\gll \textbf{Ka} han Ola \textbf{sa}? \label{ka}\\
what the Ola said\\
\glt `What did Ola say?'}
\ex[]{\gll {[}\textbf{Ka} \textbf{slags} \textbf{bil}] \textbf{har} han Jens kjøpt sæ? \label{kaslagsbilverb}\\
\phantom{[}what kind car has the Jens bought.\textsc{ptcp} himself\\
\glt `What kind of car has Jens bought for himself?'}
\ex[*]{\gll {[}\textbf{Ka} \textbf{slags} \textbf{bil}] han Jens \textbf{har} kjøpt sæ? \label{kaslagsbil}\\
\phantom{[}what kind car the Jens has bought.\textsc{ptcp} himself\\
\glt `What kind of car has Jens bought for himself?'}
\z
\z

As can be seen, the simplex \textit{wh}-element \textit{ka} can occur in main clause questions without verb fronting, see (\ref{ka}), suggesting that it occupies the C position. Depending on the speaker, the proper operator used with verb fronting, see (\ref{kaverb}), is not even acceptable in these dialects. The pattern is altogether different when a complex \textit{wh}-phrase such as \textit{ka slags bil} is inserted: verb fronting in this case is obligatory, as demonstrated by the grammaticality of (\ref{kaslagsbilverb}) and the ungrammaticality of (\ref{kaslagsbil}). If, however, the \textit{wh}-element is focussed, the \textit{wh}-element cannot be inserted into C and verb fronting must occur (\citealt[93--94]{bayerbrandner2008}). Consider (\citealt[94, ex. 22]{bayerbrandner2008}):

\ea
\ea[]{\gll \textbf{KA} \textbf{sa} han Ola?\\
what said the Ola\\
\glt `What did Ola say?'}
\ex[*]{\gll \textbf{KA} han Ola \textbf{sa}?\\
what the Ola said\\
\glt `What did Ola say?'}
\z
\z

North Norwegian dialects are similar in this respect to Alemannic and Bavarian. 

In all the three groups, an asymmetric pattern can be observed: single \textit{wh}-words behave differently from complex \textit{wh}-phrases. This difference can be observed in Alemannic and Bavarian in embedded clauses only, while the asymmetry holds also in main clauses in North Norwegian dialects. Nevertheless, it is possible to have symmetric patterns as well. As noted by \citet[94]{bayerbrandner2008}, the standard varieties, including Standard German, generally do not use doubling. On the other hand, certain other dialects (including some varieties of Alemannic) seem to require the insertion of the complementiser with all \textit{wh}-elements, as is the case in West Flemish (\citealt[94]{bayerbrandner2008}, citing \citealt{haegeman1992}).

Regarding variation, \citet[94]{bayerbrandner2008} propose that symmetric varieties have no latent C-feature at all and \textit{wh}-elements are thus never inserted into C. In varieties like West Flemish, which always insert the complementiser, an element in C must be overt because this is the only way it can serve as a host to clitics: these varieties are similar to Alemannic and Bavarian in that they have a clitic system (\citealt[94]{bayerbrandner2008}). In dialects like Standard German, however, which do not have a genuine clitic system (cf. \citealt{cardinaletti1999}), the overtness of C does not make a difference: in these dialects, the insertion of an overt complementiser is ruled out due to economy considerations (\citealt[94]{bayerbrandner2008}). Ultimately, the analysis assumes that syntactic variation is the result of lexical variation, in line with \citet{borer1984}.

In sum, \citet{bayerbrandner2008} make an important contribution to the study of Doubly Filled COMP patterns in embedded questions, both from a theoretical and from an empirical perspective. The most significant empirical aspect is the existence of asymmetric patterns alongside symmetric patterns, which challenges not only cartographic approaches but also traditional X-bar theoretic terms. The proposal that head-sized \textit{wh}-elements in asymmetric patterns move to the C head instead of the specifier is convincing on empirical grounds, as these elements are in complementary distribution with the finite complementiser \textit{dass}, and the same can be observed with verb movement in North Norwegian dialects, suggesting that the generalisation genuinely holds for the C position and not just for the given complementiser.\largerpage

However, the proposal also raises some questions that need to be addressed. In particular, the very notion of the C-feature is problematic, especially because its occurrence is restricted to head-sized \textit{wh}-elements in asymmetric patterns, that is, precisely the pattern for which it served to account, rendering the argumentation circular. In addition, it should be clarified what the C-feature actually is: apparently, this feature is present on the given element and it can be activated or it can remain latent, but when it is activated, the \textit{wh}-element acts like a complementiser, even though it is not a grammaticalised complementiser proper. A major problem in this respect is how the C-feature is related to finiteness, which is encoded by a finite complementiser (e.g. \textit{dass}) but not by a \textit{wh}-operator: the moment a \textit{wh}-operator is equipped with a finiteness feature it is no longer an operator but a grammaticalised finite complementiser. It follows that if the head-sized \textit{wh}-element moves to C, finiteness is apparently not encoded. Finally, the availability of a latent C-feature seems to be tied to the particular dialect: asymmetric dialects have it, while others do not. This predicts that there is no optionality regarding non-focussed head-sized \textit{wh}-elements, which is, however, not the case (cf. the examples and discussion of \citealt[778]{weiss2013}).{\interfootnotelinepenalty=10000\footnote{For instance, the following example from Bavarian is marked as grammatical (\citealt[778, ex. 15a]{weiss2013}):

\ea \gll I woaß aa ned, \textbf{wer} \textbf{dass} do gwen is. \label{werdassbavarian}\\
I know.\textsc{1sg} also not who that there been is\\
\glt `I do not know either who was there.'
\z

\citet[778]{weiss2013} mentions that such patterns are subject to microvariation: one-syllable \textit{wh}-phrases such as \textit{wer} in (\ref{werdassbavarian}) are less likely to occur with \textit{dass} than larger \textit{wh}-phrases. This is, however, subject to individual preferences and is not to be taken a strict grammatical constraint. Similar observations were also made by \citet[265]{roedder1936} on South Franconian.}}

A further problem concerns the role of cliticisation. \citet{bayerbrandner2008} claim that C has to be overt in dialects that use Doubly Filled COMP (either symmetrically or asymmetrically) because in these dialects clitic pronouns need to cliticise onto an overt C head. The problem is that while this certainly applies in cases where there is a clitic pronoun in the relevant position, as in (\ref{wosdass}) and (\ref{wadass}), it is not a convincing argument in cases where there is no such clitic pronoun (as is also the case in the North Norwegian examples, and see also the Dutch example in (\ref{dutchdfc}) above). In other words, if the insertion of \textit{dass} were primarily a phonological matter, one would expect either (i) \textit{dass} to be absent in cases where no clitic is present altogether, or (ii) at least a significant improvement for \textit{dass}-less clauses (with complex \textit{wh}-phrases) in the absence of clitics. Regarding (i), \citet[41]{bayer2014} remarks that a phonological motivation for a dialect retaining doubly filled COMP patterns ``must be seen as affecting the grammar as a whole and not individual constructions''. This does not address (ii), though. Further, this explanation still leads to a second problem, which is that doubly filled COMP patterns in embedded interrogatives occur across West Germanic, also in dialects that are not known to use (subject) clitics in the Bavarian way. Specifically, the spoken (and dialectal, historical) English data mentioned in \sectref{sec:3problems} above do not indicate that cliticisation would play any role. In sum, while cliticisation as a factor may indeed be decisive in the (diachronic) emergence of the given system, it is not sufficient as an explanation for the entire (synchronic) system and it does not carry over to other varieties.

Finally, while the proposal made by \citet{bayerbrandner2008} breaks away from a strict X-bar theoretic framework and is hence more flexible, the mechanisms underlying \textit{wh}-elements landing in C are not without problems from a merge-based minimalist point of view. In particular, cliticisation cannot be built into the syntactic component directly, and, as mentioned above, the notion of the C-feature is problematic. Note that the same problem arises with verb fronting in V2 clauses in German (and across Germanic), as well as with T-to-C movement in English interrogatives, see (\ref{ttoc}) and (\ref{v2}). That is, the verb apparently takes the position of C, even though it is not categorised as a complementiser: in addition, it is highly unlikely that lexical verbs are equipped with a C-feature. Note that since V2 constructions appear in standard Germanic languages as well, they cannot be attributed to the availability of some dialect-specific feature either. Just as with embedded interrogatives, the role of finiteness should be clarified in these cases, too. This parallelism is not discussed by \citet{bayerbrandner2008}, and hence it is not clear how far the proposal extends regarding the Germanic left periphery.

\section{Approaches to Doubly Filled COMP} \label{sec:3approaches}
In principle, there are three possible scenarios regarding the Doubly Filled COMP Filter. One possibility is to say that the filter is subject to parametric variation: some dialects (such as standard West Germanic languages) have it, while others (such as Alemannic and Bavarian) do not. This is essentially in line with the original proposal of \citet{chomskylasnik1977}, who suggest that the filter holds in present-day Standard English but not in non-standard and historical varieties. However, as was also pointed out at the end of \sectref{sec:3chomskylasnik}, this is problematic for various reasons, notably because it allows no intra-dialect variation, which, however, exists, as pointed out by \citet{bayerbrandner2008} among others. Moreover, the specifier and the head of the CP are, strictly speaking, both overtly filled in standard West Germanic languages in V2 clauses and in T-to-C movement constituent questions in English. In addition to the notion of the filter being problematic from a minimalist perspective, its application domain should be clarified, and specific as it seems to be, it should not be treated as a parameter. 

Another possibility is to say that the filter is universal and apparent violations actually involve multiple CP projections, as proposed by \citet{baltin2010}, using the cartographic framework established by \citet{rizzi1997}.

A third possibility is that there is no such filter at all and dialects differ in whether they allow null complementisers or whether they require filling C with overt material: this approach would reduce the observed differences to lexical differences (in line with the proposal made by \citealt{borer1984}).

Let us consider doubling in embedded interrogatives; more specifically, in constituent questions, as illustrated in (\ref{dfcint}), repeated here as (\ref{dfcintrepeat}):\footnote{Note that the finite verb in the subclause in (\ref{vindt}) could also be in the past tense (\textit{vond} instead of \textit{vindt}. The informant providing the example generally found the present tense more natural (see \citealt[30]{bacskaiatkaribaudisch2018}).}

\ea \label{dfcintrepeat}
\ea[\%]{They discussed a certain model, but they didn't know \textbf{which model that} they discussed. \label{model}}
\ex[\%]{\gll Peter vroeg \textbf{wie} \textbf{dat} er boeken leuk vindt. \label{vindt}\\
Peter asked.\textsc{3sg} who that of.them books likeable finds\\
\glt `Peter asked who liked books.'}
\z
\z

Essentially, as was discussed already in \chapref{ch:2}, there are two possible structures. In the first scenario, there is a single CP hosting both the \textit{wh}-element and the finite complementiser. This is what was ultimately proposed in \chapref{ch:2} as well: this is in essence a true Doubly Filled COMP pattern since, once the COMP position of \citet{chomskylasnik1977} is translated into X-bar representations, both the specifier and the head of the CP are filled by overt elements, as proposed also by \citet{bayer1984}.\footnote{As pointed out by \citet[26]{bayer2015}, \textit{wh}-elements and complementisers were always taken to be in complementary distribution in the topological fields model (see \citealt{hoehle1986}). \citet[26]{bayer2015} also points out that this kind of approach was taken up by \citet{kathol2000} in the HPSG-framework, but while this line of argumentation makes good predictions for certain data, it fails to account for the doubling patterns attested in Bavarian, cf. \sectref{sec:3bayerbrandner}. Data like (\ref{dfcintrepeat}) also indicate that complementary distribution cannot be a satisfactory explanation; note that while \citet{chomskylasnik1977} assumed that both elements should be located in COMP, they explicitly did not use complementary distribution as an argument for the lack of doubling patterns but relied on complementiser deletion; see \sectref{sec:3chomskylasnik}.} Another option is to adopt a cartographic approach, supposing that there are two separate CPs with two distinct functions, as proposed by \citet{koopman2000} and more recently by \citet{baltin2010}, following \citet{rizzi1997}. The Doubly Filled COMP pattern involving a single CP is shown in \figref{treedfcsinglecp}.\footnote{Note that the opposition between a single CP and a double CP here is meant for the specific kind of construction. Specifically, the arguments presented here for a single CP analysis do not automatically carry over to all other constructions: as will be shown in \chapref{ch:5}, double CPs are attested (and in fact necessary) in certain constructions even in Germanic (and recall also that languages like Welsh can also have two complementisers, see \chapref{ch:2}). Crucially, no elements can intervene between the \textit{wh}-element and the complementiser in Doubly Filled COMP patterns (the same apparently holds for Romance, cf., for instance, the data provided by \citealt{polettovanelli1997} for Northern Italian varieties), which would be expected in a Force--Fin distinction in the sense of \citet{rizzi1997} and which would naturally imply the presence of multiple projections.} The split CP pattern is shown in \figref{treedfcdoublecp}.

\begin{figure}
\caption{The single CP analysis} \label{treedfcsinglecp}
\begin{forest} baseline, qtree
[CP 
	[which model\textsubscript{{[}wh{]}}]
	[C$'$
		[C\textsubscript{{[}wh{]},{[}fin{]}}
			[that\textsubscript{{[}fin{]}}]
		]
		[TP]
	]
]
\end{forest}
\end{figure}

\begin{figure}
\caption{The double CP analysis} \label{treedfcdoublecp}
\begin{forest} baseline, qtree
[CP
	[which model\textsubscript{{[}wh{]}}]
	[C$'$
		[C\textsubscript{{[}wh{]}}]
		[CP
			[C$'$ [C\textsubscript{{[}fin{]}} [that\textsubscript{{[}fin{]}}]] [TP]]
		]
	]
]
\end{forest}
\end{figure}

The features given here are interrogative, [wh], and finiteness, [fin], standing for the properties that have to be encoded in the CP-domain as determined by the matrix predicate (see also the discussion in \chapref{ch:2}). Note that the representation in \figref{treedfcdoublecp} is compatible not only with a classical cartographic analysis but also with the kind of CP-recursion proposed by \citet{vikner1995} and \citet{viknerchristensennyvad2017}; however, it is worth mentioning that CP-recursion was proposed primarily in order to account for embedded V2 and it is by no means necessary that the same kind of recursion applies in Doubly Filled COMP structures. In \chapref{ch:2}, I proposed the structure in \figref{treedfcsinglecp}, as it is more minimal and more congruent with a merge-based approach; let us now see more arguments in favour of \figref{treedfcsinglecp} and against \figref{treedfcdoublecp}. 

There are several problems with \figref{treedfcdoublecp} and the analysis of \citet{baltin2010}; see also the discussion of \citet{bayer2015}. First, the rigid split of functions between the two projections is highly questionable. Note that \citet{baltin2010} uses designated labels for these projections, but the differences are expressed here by features, as this is more compatible with the approach pursued in the present book and it allows for a more straightforward comparison of the two approaches. The rigid separation of Force and Fin essentially follows the cartographic approach (cf. \citealt{rizzi1997}), yet the analysis given by \citet{baltin2010} is fundamentally intended to be a minimalist one. In a merge-based account, the element \textit{that} should be directly merged with the \textit{wh}-phrase (here: \textit{which model}), which does not allow for \figref{treedfcdoublecp}, where an empty lower specifier and an empty higher C head are postulated: \figref{treedfcdoublecp} would be valid if there were evidence for empty elements in these positions. Note also that while \citet{baltin2010} refers to \citet{rizzi1997} regarding the Force--Fin distinction, his analysis places \textit{that} in Fin, contrary to what \citet[312--314]{rizzi1997} claimed, since he placed \textit{that} in Force (see the discussion in \chapref{ch:2}).\largerpage[-1]\pagebreak

In addition, \textit{wh}-operators are located in FocP (between Force and Fin) in \citet{rizzi1997}, unlike relative operators (see also in \chapref{ch:2}). In other words, the treatment of the cartographic left periphery by \citet{baltin2010} is not straightforward. One way to overcome this conflict would be to say that there are two lexical elements, \textit{that}-Force (the declarative complementiser of \citealt{rizzi1997}) and \textit{that}-Fin (the finite complementiser of \citealt{baltin2010}), but, there being no independent motivation,\footnote{In this respect, English (and Germanic in general) differs from Romance varieties that actually allow the co-occurrence of two such complementisers, as shown by \citet{paoli2009} for \textit{che} in North-Western Italian varieties. There is no comparable empirical evidence for a Force-Fin split for \textit{that}; recomplementation occurs only with phonologically heavy, complex phrases, most probably due to processing reasons, and it is compatible with the collapsing mechanism affecting the CP described by \citet{sobin2002}, see \chapref{ch:2} (thus, contrary to \citealt{villagarcia2019}, no cartographic template is necessary, given the ungrammaticality of recomplementation with single XPs).} this would be a fairly stipulative and circular argumentation. Moreover, as was discussed in \chapref{ch:2}, it is crucial for \citet{rizzi1997} that there should be no complementiser \textit{that} located in Fin when the CP is split, since the account for the \textit{that}-trace effect is contingent on the assumption that only the zero counterpart can be located in Fin. The fact that word order patterns like (\ref{dfcintrepeat}) contradict this assumption again indicates that the analysis of \citet{rizzi1997} is empirically problematic.

A rigid separation of the two CPs would indeed be needed for \figref{treedfcdoublecp} to work in order to avoid the violation of the Minimal Link Condition (see \citealt{fanselow1990, fanselow1991}, \citealt{chomsky1995}): the operator in \figref{treedfcdoublecp} does not move to the closest possible [Spec,CP]. Similar considerations regarding \textit{wh}-movement are expressed by \citet[241--243]{vancraenenbroeck2010}: while he assumes more or less designated CP projections for clause-typing and operator movement, movement is supposed to target the lower CP projection, the higher CP being potentially available for the direct merger of elements.

The problem may in principle be avoided by saying that the lower C head cannot attract the operator, and an additional complementiser has to be inserted. However, a rigid separation is not tenable for relative clauses, as was shown in \chapref{ch:2} (see also \chapref{ch:4}), and relative clauses showing Doubly Filled COMP effects would therefore violate the Minimal Link Condition. Finally, if \figref{treedfcdoublecp} is possible for non-standard varieties, it remains to be explained why it cannot appear in standard varieties, as finite subordinators are also available in these dialects.

In addition to the problems indicated above, it should be mentioned that the structure adopted by \citet{baltin2010} serves to avoid a potential problem regarding sluicing. The assumption, going back to \citet{merchant2001}, is that sluicing results from an ellipsis feature, [E], located on a functional head: this [E] feature instructs PF to eliminate the complement. Under this view, sluicing leaves the head itself intact. As also observed by \citet{baltin2010}, the complementiser head in Doubly Filled COMP patterns cannot be overt; taking the sluiced counterpart of (\ref{model}), the pattern is as follows:

\ea They discussed a certain model, but they didn't know \textbf{which model (*that)}. \label{sluice}
\z

Assuming that the complementiser \textit{that} is located in a lower CP projection and the [E] feature is located on a higher C head, \citet{baltin2010} claims that the obligatory elimination of \textit{that} follows naturally from a double CP, as in \figref{treedfcdoublecp}. The argumentation is contingent upon the assumption that sluicing does not affect the element in C. However, as pointed out by \citet[30--32]{bayer2015}, for instance, this is not necessarily true: one may equally assume that the head is affected by sluicing, except when the deletion of the head element would result in the loss of non-recoverable material. In addition, one may also argue that the non-elimination of the complementiser in cases like (\ref{sluice}) is prosodically ill-formed: the [E] feature also instructs PF to assign main stress to the element in the specifier (that is, the element preceding the [E] feature in the linear structure), which is to be followed immediately by the elided part: the overt complementiser violates this split pattern as it is neither silent nor stressed. Moreover, the complementiser normally forms one phonological unit with the following TP, which is again violated if it is overt when the TP is sluiced. Finally, this requirement may well be independent of the status of the element in the functional head as a complementiser: as shown by \citet[173--193]{bacskaiatkari2018langsci} for elliptical comparative clauses, the locus of the ellipsis feature and the projection to which a lexical verb moves show a correlation such that the [E] feature and an overt lexical verb seem to be in complementary distribution. Taking up this line of argumentation, it may be the case that in sluicing patterns like (\ref{sluice}) the presence of the [E] feature on C automatically implies the impossibility of an overt \textit{that} in the same head, making the insertion of a zero complementiser necessary. If so, one may even retain the idea that sluicing does not per se eliminate the head: it is rather that the head has to be empty in the first place (see \chapref{ch:6} for discussion).

In sum, a ``Doubly Filled COMP'' analysis involving a single CP (and hence the direct merger of the \textit{wh}-element to the complementiser) is favourable and this is the analysis I pursue in the rest of this chapter.

\section{Embedded constituent questions} \label{sec:3embedded}
Recall that there are essentially three possible scenarios regarding the Doubly Filled COMP Filter. First, much in the vein of \citet{chomskylasnik1977}, the Doubly Filled COMP Filter may be subject to parametric variation: under this view, some dialects (such as standard West Germanic languages) have it, while others do not. This is problematic, as the operation domain of the Doubly Filled COMP Filter should be more refined (see Sections~\ref{sec:3chomskylasnik} and \ref{sec:3approaches}); moreover, the Doubly Filled COMP Filter should not be a parameter in itself. Second, the Doubly Filled COMP Filter may be universal: accordingly, apparent violations of the Filter actually involve multiple CP projections (see, for instance, \citealt{baltin2010} discussed above). This is again problematic, as already pointed out in \sectref{sec:3approaches} in detail. Third, there may be no Doubly Filled COMP Filter at all, which is of course favourable in minimalist terms since it does not rely on filters in the syntactic derivation: in this approach, the economy of derivation versus the requirement to fill the head may be thought of as competing requirements, and Doubly Filled COMP patterns may be handled similarly to T-to-C or V2 patterns. In the present section, I argue in favour of this approach.

If one were to assume that a separate [wh] CP and a separate [fin] CP are available (and both are designated projections in a cartographic sense, as established by \citealt{rizzi1997}; see \citealt{grewendorf2002, grewendorf2008}, \citealt{frey2004, frey2005}, \citealt{bayer2004, bayer2006}, among others), as shown in \figref{treedfcdoublecp}, one would expect that doubling is available with all \textit{wh}-elements. However, as shown by \citet{bayerbrandner2008}, this is not universally the case as many Alemannic and Bavarian speakers show an asymmetric pattern, see the discussion in \sectref{sec:3bayerbrandner}. Consider the following examples (\citealt[88, ex. 3a, 4a, 5a and 5b]{bayerbrandner2008}):\largerpage

\ea
\ea[]{\gll I frog-me, \textbf{fia} \textbf{wos} \textbf{dass}-ma an zwoatn Fernseher braucht. \label{wosdassrepeat}\\
I ask-\textsc{refl} for what that-one a second TV needs\\
\glt `I wonder what one needs a second TV for.'}
\ex[]{\gll I hob koa Ahnung, \textbf{mid} \textbf{wos} \textbf{fia-ra} \textbf{Farb} \textbf{dass}-a zfrien waar. \label{wosfiarafarb}\\
I have no idea with what for-a colour that-he content would.be\\
\glt `I have no idea with what colour he would be happy.'}
\ex[*]{\gll I woass aa ned, \textbf{wer} \textbf{dass} allas am Sunndoch in da Kiach gwen is. \label{werdass}\\
I know too not who that all at Sunday in the church been is\\
\glt `I don't know either who all has been to church on Sunday.'}
\ex[*]{\gll I wett gern wisse, \textbf{wa} \textbf{dass} i do uusfülle muss. \label{wadassrepeat}\\
I would gladly know what that I there out-fill must\\
\glt `I'd like to know what I have to fill out there.'}
\z
\z

\begin{sloppypar}
There is a difference between \textit{wh}-elements for the speakers in question: phrase-sized \textit{wh}-elements, see (\ref{wosdassrepeat}) and (\ref{wosfiarafarb}), occur with \textit{dass} `that', while word-sized \textit{wh}-elements like \textit{wer} and \textit{was}, see (\ref{werdass}) and (\ref{wadassrepeat}), do not. As discussed in \sectref{sec:3bayerbrandner}, \citet{bayerbrandner2008} argue that the asymmetry arises because \textit{wer}/\textit{was} and \textit{dass} are in complementary distribution since these head-sized \textit{wh}-elements target the C head position instead of the specifier. Adopting this view, embedded interrogatives with a single \textit{wer} should be assigned the structure given in \figref{treewerc}.
\end{sloppypar}

\begin{figure} 
\caption{The position of word-sized \textit{wh}-elements} \label{treewerc}
\begin{forest} baseline, qtree
[CP
	[C$'$
		[C\textsubscript{{[}wh{]},{[}fin{]}}
			[wer\textsubscript{{[}wh{]}}]
		]
		[TP]
	]
]
\end{forest}
\end{figure}

As already pointed out in \sectref{sec:3bayerbrandner}, two potential problems seem to arise from a minimalist perspective. First, \figref{treewerc} represents a problem in terms of Bare Phrase Structure as \textit{wer} is not of the category C, and hence there seems to be a discrepancy between the lexical element and the descriptive label. In other words, if merging \textit{wer} with the TP results in a projection labelled as \textit{wer}, then it is not the same category as when the C head is filled by a complementiser proper. Matrix predicates like \textit{fragen} `ask' may take interrogative clauses with or without \textit{dass} and select for a CP-complement, but the relevant category cannot come from \textit{wer}. Second, \figref{treewerc} represents a problem for Chain Uniformity: apparently, \textit{wer} originates as a phrase and moves to a head position.

Regarding Chain Uniformity, as was discussed in \sectref{sec:3bayerbrandner}, \citet{bayerbrandner2008} propose that there is a morphological condition on Chain Uniformity: the phrase-sized (XP) \textit{wer} is morphologically identical to the word-sized (X) \textit{wer}.\largerpage

Regarding Bare Phrase Structure, the proposal of \citet{bayerbrandner2008} is that \textit{wh}-elements can be equipped with a latent C-feature in dialects that show asymmetrical patterns (see \sectref{sec:3bayerbrandner}). However, as was pointed out in the relevant discussion above, this assumption faces several problems. Among others, in order to define what a C-feature is, one should also have a clear definition for what belongs to the category C. Canonical complementisers (such as \textit{that} and \textit{if}) impose restrictions on whether the clause is finite or not (which is not the case with \textit{wh}-elements). They usually carry some clause-typing feature (such as interrogative), though specifically finite complementisers like \textit{that} or \textit{dass} which appear in Doubly Filled COMP patterns in embedded interrogatives seem to be underspecified in this respect: while they can undoubtedly appear in declarative clauses otherwise, it would be difficult to argue that in cases like (\ref{wosdassrepeat}) they type an interrogative clause as declarative. On the other hand, elements moving to the CP are known to be potential candidates for reanalysis into a complementiser (as discussed e.g. by \citealt{vangelderen2009}), blurring the boundary between complementisers and other elements related to clause typing.

Note that the same problem arises in the case of V2 in German main clauses by V moving to C, see \citet[10--32]{fanselow2004}. I will return to this issue later in this chapter and will provide a more refined analysis there.

At this point, I propose that the phenomenon in \figref{treewerc} is related to the general ability of C to host elements other than complementisers in the language. This is related to the V2 property of German and indeed most Germanic languages, including English historically; note that while Modern English is not V2, T-to-C movement in main clause interrogatives works exactly the same way in this respect. In other words, non-standard dialects with Doubly Filled COMP effects extend the ban on a phonologically empty C to embedded interrogatives.

\section{Embedded polar questions} \label{sec:3embeddedpolar}
\subsection{Basic properties} \label{sec:3basic}\largerpage

Before turning to the actual analysis, let us consider polar questions as well. Doubly Filled COMP effects are usually attested in constituent questions as \textit{wh}-operators are necessarily overt since they express non-recoverable information; they also correspond to the focussed constituent in question--answer sequences (\citealt[250]{krifka2008}, citing \citealt{paul1880}). However, polar interrogatives also contain an operator: this polar operator may be overt or covert, and it is inserted directly into the specifier of the CP (\citealt{bianchicruschina2016}); therefore, no movement is required from within the clause. According to \citet{larson1985nllt}, following \citet{roothpartee1982}, this operator essentially corresponds to \textit{whether} and has the properties of a scope-bearing element. Consider the following examples (\citealt[218, ex. 1]{larson1985nllt}, citing \citealt{roothpartee1982}):

\ea Mary is looking for a maid or a cook. \label{marymaidcook}
\z

The sentence in (\ref{marymaidcook}) demonstrates multiple ambiguity: apart from the de re reading (Mary is looking for a specific person), there are two de dicto readings that are relevant here:

\ea
\ea Mary is looking for [[a maid] or [a cook]].\\
`Mary is looking for a servant, who should be either a maid or a cook.'
\ex Mary is looking for [a maid] or Mary is looking for [a cook].\\
`Mary is looking for a maid or she is looking for a cook.'
\z
\z

Ambiguity arises because the scope of \textit{or} is not overtly marked: as \citet{larson1985nllt} argues, elements like \textit{either} or \textit{whether} may overtly mark scope. In polar questions, the element \textit{or} is mostly not overt:

\ea I don't know if/whether Mary has already arrived (or not).
\z

Disjunction comprises a proposition and its negation here (\citealt[225--227]{larson1985nllt}). 

In principle, an interrogative feature on C may be checked off by inserting an element equipped with the relevant feature either into the head itself or into the specifier, in line with the Clausal Typing Hypothesis of \citet{cheng1991diss}, see also \citet[89]{bayerbrandner2008} and \citet[86]{zimmermann2013} for German. As mentioned above, \textit{wh}-elements carrying the [wh] feature in constituent questions are necessarily overt since they carry new information. This is not the case in matrix polar questions, where word order and intonation are indicative of clause type.\footnote{In most dialects of German, the [Q] feature must be encoded morphologically in embedded questions, though: since German has no overt polar operator (see the discussion in \sectref{sec:3german} below), the interrogative/disjunctive complementiser \textit{ob} is inserted. In principle, however, the matrix predicate is indicative of the embedded clause type, which predicts that there can be varieties without an overt polar marker. This seems to be the case in Thuringian, as shown by the following example (\citealt[24, ex. 34]{schallertdroegepheiff2018}, quoting \citealt{loeschfahningwiegand1990}; the translation is mine):

\ea \gll ich soll frägn, \textbf{daß} sie heint zu uns kommen \label{dassint}\\
I should ask that they today to us come\\
\glt `I should ask if they are coming to our house today.'
\z

The feature [Q] is not overtly marked in this case, yet the insertion of \textit{dass} is necessary to lexicalise [fin] regularly. \citet[24]{schallertdroegepheiff2018} do not provide further insights regarding the distribution of patterns like (\ref{dassint}), but they take it as evidence for the underspecified nature of \textit{dass}. Further empirical investigations would be necessary to examine how robust this pattern is, especially because \textit{fragen} `ask' can take a \textit{dass}-clause as a complement even in Standard German as long as the embedded clause is interpreted a request and not as a question:

\ea \gll Wenn dich irgendjemand fragt, dass du für etwas bezahlen sollt, mach das nicht.\\
if you.\textsc{acc} someone asks that you for something pay.\textsc{inf} should.\textsc{2sg} do.\textsc{imp.2sg} that.\textsc{n} not\\
\glt `If anyone asks you to pay for something in advance, do not do that.'
\z

According to Hans-Martin Gärtner (p.c.), examples such as (\ref{dassint}) should also be interpreted as requests for this reason. Since the sentence in (\ref{dassint}) appears without context, the actual meaning cannot be identified.}

In English embedded polar interrogatives, either the complementiser \textit{if} or the operator \textit{whether} is overt; the configuration for \textit{whether} is represented in \figref{treewhether}. The configuration for \textit{if} is represented in \figref{treeif}.

\begin{figure} 
\caption{The position of \textit{whether}} \label{treewhether}
\begin{forest} baseline, qtree
[CP
	[whether\textsubscript{{[}wh{]}}]
	[C$'$
		[C\textsubscript{{[}Q{]},{[}fin{]}}
			[$\emptyset$\textsubscript{{[}fin{]}}]
		]
		[TP]
	]
]
\end{forest}
\end{figure}

\begin{figure} 
\caption{The position of \textit{if}} \label{treeif}
\begin{forest} baseline, qtree
[CP
	[C$'$
		[C\textsubscript{{[}Q{]},{[}fin{]}}
			[if\textsubscript{{[}Q{]},{[}fin{]}}]
		]
		[TP]
	]
]
\end{forest}
\end{figure}

Unlike in constituent questions, where the feature involved is [wh], I assume that the clause-typing feature required by the head is [Q], a disjunction feature; see \citet{bayer2004} on the separation of the two features. Crucially, [wh] implies disjunction, [Q], and hence inserting an operator equipped with [wh] also checks off [Q], at least in languages like English and German; other languages like Korean and Japanese consistently split the two features on two distinct elements in constituent questions (see \citealt{bayer2004}), a strategy also available in certain Dutch dialects (see \sectref{sec:3dutch} below). I will return to the difference later in this section. Importantly, the [Q] feature also appears in conditional clauses (see \citealt{bhattpancheva2006}, \citealt{arsenijevic2009}, \citealt{danckaerthaegeman2012} on the relatedness of polar questions and conditionals), and the chief difference between embedded polar questions and conditionals lies not so much in the clause-typing feature of the embedded clause but in the matrix element selecting the embedded clause. On the other hand, the element \textit{whether} is restricted to polar interrogatives and cannot appear in conditionals: it is reasonable to assume that this element is specified as interrogative proper, [wh], which is also morpho-phonologically transparent in English.\footnote{The word \textit{whether} itself clearly contains a \textit{wh}-base and it is a reflex of Proto-Germanic *\textit{hwaþeraz}/*\textit{hweþeraz}, for which \citet[154]{walkden2014} reconstructs a `which of two' reading, since this reading is present in all early Germanic languages and is in fact the only reading which is attested in Gothic (see \citealt[146--154]{walkden2014} for details). The cognates of \textit{whether} in other modern Germanic languages have different uses; for instance, German \textit{weder} `neither' is not used is interrogatives. The forms \textit{either} (and \textit{neither}) are also related to \textit{whether}, and all of these elements ultimately express disjunction.}

In \figref{treewhether}, the two properties, [fin] and [Q]/[wh], are carried by two separate elements, while in \figref{treeif} both are marked by the interrogative complementiser \textit{if}. In the latter case, no additional operator is necessary. I assume that semantic operators may or may not show operator properties like phrase movement in terms of their syntax; specifically, they may appear as complementisers or grammaticalise into ones, which may in certain cases lead to the reinforcement of the given semantic property by an additional element.\footnote{A well-known case showing similarities is negatives: referred to as the Jespersen cycle, a negative head may be reinforced by an additional negative operator, which may ultimately take over the function of marking negation altogether (see, for instance, \citealt{wallage2008}, \citealt{vanderauwera2009}, \citealt{hoeksema2009}).}\largerpage

Given the structure in \figref{treeif}, it is not difficult to see why a combination like *\textit{if that} is not possible in English: both elements are complementisers, and inserting \textit{if} satisfies the lexicalisation requirement on [fin] in C.{\interfootnotelinepenalty=10000\footnote{Note that this restriction on the impossibility of \textit{if that} only applies to configurations in which the two complementisers are on the same left periphery. Consider the following examples:

\ea We believe \textbf{that} [all will change [\textbf{if} we have continued efforts]]. \label{believethatif}
\ex We believe [\textbf{that} [\textbf{if} we have continued efforts], all will change]. \label{believeifthat}
\z

In (\ref{believeifthat}), \textit{that} and \textit{if} are string-adjacent; however, \textit{if} belongs to a different clause, as shown in (\ref{believethatif}), where the conditional clause is not inverted with its matrix clause. Therefore, cases like (\ref{believeifthat}) do not constitute a counterexample.}} The different syntactic positions of \textit{if} and \textit{whether} are also indicated by the fact that the combination \textit{whether if} is possible, even though it rarely occurs. Consider (\citealt[96, ex. 82]{vangelderen2004}):

\ea The local authority will know \textbf{whether if} they let the council house to the tenant.\\
(BNC-FC3-80)
\z

That such combinations are not typical is due to economy constraints: both overt elements essentially mark an embedded polar question and doubling is therefore redundant.\footnote{Unlike negation, however, where both elements (e.g. \textit{ne} and \textit{not} in Middle English, \textit{ne} and \textit{pas} in Modern French) are specified as [neg] only, the elements \textit{whether} and \textit{if} differ crucially in terms of the [fin] feature; in other words, doubling is not perfect as the head element still encodes a property that cannot be lexicalised by the specifier element.}

In German, the combination *\textit{ob dass} is ruled out for the same reason as in English: the two elements are in complementary distribution. Such a combination is indeed not possible in the Southern dialects discussed in the previous section: it is not available in Alemannic (Ellen Brandner, p.c.), and the SyHD-atlas for Hessian does not mention such instances either. The same applies to the detailed study of \citet{bayer1984} on Bavarian, which mentions several combinations with \textit{dass} but not \textit{ob dass}. According to \citet[226]{meinunger2011}, doubling patterns with \textit{dass}, including \textit{ob dass}, are sporadic and in these cases the two elements actually constitute a single word (essentially a complex complementiser), as in the case of \textit{sodass} `so that' (and hence \textit{ob dass} would technically be \textit{obdass}). \citet{gillmanndgfs} argues that while combinations of the form ``connector + \textit{dass}'' were possible in the 17th century, they gradually diminished till the end of the 18th century and have since then been restricted to a few grammaticalised cases.

As pointed out by \citet[778--779]{weiss2013}, most German dialects apply \textit{ob} in embedded polar questions, but in some dialects \textit{was} `what' can appear instead of \textit{ob}, see \citet{zimmermann2011} for Low German and \citet{luehr1989} for Upper Bavarian. However, as \citet[24]{schallertdroegepheiff2018} argue, no doubling is attested with \textit{was}.

The importance of this is primarily the following: an analysis with a separate designated interrogative CP and a finite CP, such as \figref{treedfcdoublecp}, as in \citet{baltin2010}, would predict that this is possible. Note also that there is no ban on multiple complementisers in Alemannic either, as the doubling pattern \textit{als wie} `than as' is possible in comparatives (as will be discussed in \chapref{ch:5}; see also \citealt{jaeger2010, jaeger2018}, \citealt{bacskaiatkari2014diss}); hence, the reason for the non-existence of *\textit{ob dass} is not a ban on double complementisers.

In principle, one might say that, in line with \citet{rizzi1997, rizzi2004}, constituent questions and polar interrogatives differ because [Q] and [wh] are associated with distinct projections: [wh] is associated with FocP and [Q] with IntP (see the discussion in \chapref{ch:2}), and while [Spec,FocP] can be filled when Fin is overt, this is apparently not the case with the IntP (disregarding now the problem whether placing \textit{that} in FinP is compatible with \citealt{rizzi1997} at all). In order for this to work, however, doubling in polar interrogatives should be uniformly impossible; as will be shown in this section, this is not the case. What appears to be decisive is not whether the feature is [Q] or [wh] but whether the interrogative element is a complementiser or not.

\subsection{English} \label{sec:3english}
Regarding polar questions in English, while the status of \textit{if} as a complementiser is quite straightforward throughout its history, the status of \textit{whether} can be disputed: it evidently differs from the complementiser \textit{if} in its distribution, yet it also does not fully pattern with proper \textit{wh}-elements (which occur in constituent questions). In this section, I am going to consider some (mainly historical) data that may shed light on the syntactic status of this element.

The combination \textit{whether that} (which essentially corresponds to a Doubly Filled COMP pattern) is attested historically and in modern non-standard varieties (see \citealt{vangelderen2009}). Consider the following Middle English examples from the \textit{Cursor Mundi} (based on \citealt[155, ex. 61 and 62]{vangelderen2009}):

\ea \label{cursor}
\ea \gll O þis watur he gert ilkan Drinc, \textbf{quer} he wald or nan \label{cursorwhether}\\
of this water he gives each drink whether he wanted or not\\
\glt `Of this water he gives each to drink whether he wanted it or not.'\\(\textit{Cursor Mundi} 5517--6618)
\ex \gll If þai ani child miht haue, \textbf{Queþer} \textbf{þat} it ware scho or he \label{cursorwhetherthat}\\
if they any child might have whether that it were she or he\\
\glt `If they might have any child, whether it were a she or he.'\\(\textit{Cursor Mundi} 10205)
\z
\z

As can be seen, the element \textit{whether} appears on its own in (\ref{cursorwhether}), while it is combined with \textit{that} in (\ref{cursorwhetherthat}). The proposed structure for the doubling pattern is given in \figref{treewhetherthat}.

\begin{figure} 
\caption{The structure of \textit{whether that}} \label{treewhetherthat}
\begin{forest} baseline, qtree
[CP
	[whether\textsubscript{{[}wh{]}}]
	[C$'$
		[C\textsubscript{{[}Q{]},{[}fin{]}}
			[that\textsubscript{{[}fin{]}}]
		]
		[TP]
	]
]
\end{forest}
\end{figure}

The doubling pattern is essentially the same as the one in \figref{treedfcsinglecp} for constituent questions. Given that ordinary \textit{wh}-elements may be merged with the TP directly (occupying the C position), as in \figref{treewerc}, this should intuitively be available for \textit{whether} as well, since it is also head-sized and is even directly inserted into the CP-domain anyway. This appears to be the case indeed. As \citet{vangelderen2009} points out, Doubly Filled COMP patterns with \textit{whether} are quite rare in modern dialects in comparison to Doubly Filled COMP patterns with ordinary \textit{wh}-elements (whereby complex \textit{wh}-elements are more likely to occur in Doubly Filled COMP constructions). This suggests a similar asymmetry as in Alemannic constituent questions, namely that \textit{whether} is preferably inserted under the C node in polar questions in the given dialects.

Regarding the relative position of \textit{whether}, there is another issue that needs to be addressed. In Modern English, \textit{whether} is restricted to embedded clauses, unlike ordinary \textit{wh}-elements, which are permitted in main clauses as well. However, \textit{whether} was used in main clauses even until Early Modern English (\textit{do}-insertion was reanalysed as a polarity marker in this period, see \citealt{wallage2015}).\footnote{In this respect, \textit{whether} differs from German \textit{ob}, which can only appear in main clauses if the clause expresses wondering on the part of the speaker rather than a genuine question (see \citealt{gutzmann2011} and \citealt{zimmermann2013}). As will be shown in this section, English main clause \textit{whether} had a wider distribution (see also \citealt{bacskaiatkari2019me} for discussion).} Consider the following examples from Old English (\citealt[141, ex. 15--16]{vangelderen2009}, quoting \citealt{allen1980}):

\ea \label{oepattern}
\ea \gll \textbf{Hwæðer} \textbf{wæs} iohannes fulluht þe of heofonum þe of mannum \label{whetherwas}\\
whether was John's baptism that of heavens or of man\\
\glt `Was the baptism of John done by heaven or by man?'\\(\textit{West Saxon Gospel})
\ex \gll \textbf{Hwæðer} ic mote lybban oðdæt ic hine geseo \label{whether}\\
whether I might live until I him see\\
\glt `Might I live until I see him?' (Aelfric \textit{Homilies})
\z
\z

As can be seen, verb fronting (involving verb movement to C) may co-occur with \textit{whether}, but it is not obligatory (see \citealt{fischer1992}, \citealt{vangelderen2009}).

Regarding \textit{whether} in Old English, \citet{vangelderen2009} treats it as a grammaticalised complementiser when it appears on its own, but not otherwise. In other words, \textit{whether} is supposed to be a complementiser in (\ref{whether}) but not in (\ref{whetherwas}), where it moves to [Spec,CP] as an operator. However, one major problem with such a view is that the non-complementiser patterns survive into Middle English and beyond, which is not what one would expect if the element in question had undergone grammaticalisation in Old English. Grammaticalisation follows from economy principles (feature economy), also in the framework of \citet{vangelderen2009}. If an element grammaticalises into a complementiser, it is unlikely to be preserved as an operator with exactly the same functions throughout the history of English (as doubling patterns are attested later as well, either with \textit{that} or with verb movement). This is true even when taking into account that language change (and variation) is gradient in nature (\citealt{traugotttrousdale2010}): similar reanalysis processes in the CP-domain took place in a much shorter time span during Old and Middle English (see, for instance, \citealt{vangelderen2009} for \textit{that} in relative clauses). In addition, the problem is that \citet[156]{vangelderen2009} explicitly states that \textit{whether} is an operator in Modern English and cannot be analysed as a complementiser.

However, variation like (\ref{oepattern}) is attested in Middle English as well. Consider:

\ea
\ea \gll Loke well aboute \& take consyderasion, / As I haue declaryd, \textbf{whether} hit so be.\\
look well about \& take consideration as I have declared whether it so be\\
\glt `Look about and taken consideration, {} as I have declared whether it is so.'\\(John Lydgate, \textit{The assembly of gods}, stanza 267)
\ex \gll \textbf{Whether} \textbf{art} thow double, or elles the same man / That thow were furst? \label{lydgatevmove}\\
whether are thou double or else the same man {} that thou were first\\
\glt `Are you doubled or the same man that you were first?'\\(John Lydgate, \textit{The assembly of gods}, stanza 200)
\z
\z

To gain a better insight into the Middle English data, I conducted a corpus study (see \citealt{bacskaiatkari2019me}) on the two versions of the Wycliffe Bible (using the ``Michigan Corpus of Middle English Prose and Verse''). Out of all Middle English texts in the corpus, this one contained the highest number of hits for \textit{whether}; also, the availability of the two versions allows a direct investigation of (potential) variation in the same contexts.\footnote{In this respect, some notes are in order. One major problem concerns the limitations of a historical investigation in general: the investigation has to rely on actually produced data and cannot test non-attested configurations, and hence negative evidence is impossible or at least scarce. The comparison of the two versions of the translation in this respect may at least shed some light on variation in the very same contexts but the differences observed in preferences cannot be translated into grammaticality judgements.} Regarding the two versions of the Wycliff Bible, it should be noted that the later version is the revision of the earlier version; in general, the earlier version is closer to the Latin original and the later version represents more idiomatic English (\citealt{bruce1984}), though the particular phenomenon investigated here does not seem to have been influenced by Latin (see below). 

Consider first the minimal pair in (\ref{cain}) below:

\ea \label{cain}
\ea \gll And the Lord seide to Caym, Where is Abel thi brother? The which answeryde, I wote neuere; \textbf{whether} \textbf{am} I the keper of my brother?\\
and the Lord said to Cain where is Abel thy brother the which answered I know never whether am I the keeper of my brother\\
\glt `And the Lord said unto Cain, Where is Abel thy brother? And he said, I know not: Am I my brother's keeper?'
(Wycliffe Bible older version, Genesis 4.9)
\ex \gll And the Lord seide to Cayn, Where is Abel thi brother? Which an|swerde, Y woot not; \textbf{whether} Y am the kepere of my brothir?\\
and the Lord said to Cain where is Abel thy brother which answered I know not whether I am the keeper of my brother\\
\glt `And the Lord said unto Cain, Where is Abel thy brother? And he said, I know not: Am I my brother's keeper?'
(Wycliffe Bible newer version, Genesis 4.9)
\ex \gll Et ait Dominus ad Cain: Ubi est Abel frater tuus? Qui respondit: Nescio: num custos fratris mei sum ego?\\
and said.\textsc{3sg} God to Cain	where is Abel brother your who answered.\textsc{3sg} not.know.\textsc{1sg} whether keeper brother.\textsc{gen} my.\textsc{gen} am I\\
\glt `And the Lord said unto Cain, Where is Abel thy brother? And he said, I know not: Am I my brother's keeper?'
\z
\z

The availability of both options (\textit{whether} with or without verb fronting) in the very same context shows that the two options are essentially equivalent regarding their function.\footnote{In patterns involving verb fronting, the verb is in C and \textit{whether} is in the specifier position. In patterns without verb fronting, \textit{whether} is either a specifier merged to a zero complementiser or it is merged directly with the TP and thus functions as C, in the way shown by \citet{bayerbrandner2008} for head-sized \textit{wh}-elements in constituent questions.} The Latin original clearly indicates that verb movement cannot be attributed to Latin influence, as there is no verb fronting in Latin. 

According to \citet[279]{fischer1992}, verb fronting patterns with \textit{whether} are far more frequent than non-fronting patterns in Middle English. This is, however, not borne out for the Wycliff Bible, as shown by the data in Table \ref{tablewycliffpolalt}.\footnote{The data set presented in Table \ref{tablewycliffpolalt} contains the entire text for both versions, which is considerably larger than in \citet[142]{bacskaiatkari2019me} and it shows the differences between the two versions more clearly. In addition, it contains the data for the combination \textit{whether if}.} Apart from patterns involving \textit{whether}, there are some examples with \textit{if} in embedded clauses and verb fronting in main clauses; the ``other'' option refers to cases where one of the translations uses a construction other than interrogatives.\largerpage

\begin{table}[H]
\begin{tabular}{l l r@{~}r r@{~}r r}
\lsptoprule
Question type & Element(s) in CP & \multicolumn{2}{c}{Earlier version} & \multicolumn{2}{c}{Later version} & Total\\\midrule
polar       & \textit{whether}      & 677 &  (81.57\%) & 804 & (96.87\%) & 1481\\
(830 items) & \textit{whether} + V  & 92  &  (11.08\%) & 7   & (0.84\%)   & 99\\
{}          & \textit{whether that} & 2   & (0.24\%)   & 2   & (0.24\%)   & 4\\
{}          & \textit{whether if}   & 1   & (0.12\%)   & 1   & (0.12\%)   & 2\\
{}          & \textit{if}           & 28           &            & 1            &         & 29\\
{}          & V                     & 13           &            & 7            &         & 20\\
{}          & other                 & 17           &            & 8            &         & 25\\\midrule
alternative & \textit{whether}      & 33 & (50.77\%) & 59 &  (90.77\%) & 92\\
(65 items)  & \textit{whether} + V  & 16 & (24.62\%) & 4  & (6.15\%)   & 20\\
{}          & \textit{if}           & 1           &           & 0           &            & 1\\
{}          & V                     & 1           &           & 0           &            & 1\\
{}          & other                 & 14          &           & 2           &            & 16\\
\lspbottomrule
\end{tabular}
\caption{Corpus results from the Wycliff Bible\label{tablewycliffpolalt}}
\end{table}

As can be seen, polar questions are considerably more frequent than alternative questions. Importantly, single \textit{whether} and \textit{whether} with fronting are possible in both types, and hence variation cannot be attributed to a polar/alternative difference either. In this respect, the findings for Middle English indicate that the polar/alternative dichotomy shown to be operative in Old English by \citet[149--150]{walkden2014} cannot be carried over to Middle English. \citet[149--150]{walkden2014} assumes that \textit{whether} did not grammaticalise into a complementiser, but it was instead always an operator in [Spec,CP]. However, he assumes that polar and alternative questions differed regarding the base-generation site of \textit{whether}. In polar questions, \textit{whether} was a base-generated yes-no operator inserted directly into the [Spec,CP] position, triggering no verb movement (unlike ordinary \textit{wh}-elements). In alternative questions, it was an operator with a `which of two' meaning: it was base-generated clause-internally and moved to [Spec,CP] like ordinary \textit{wh}-operators, triggering verb movement (just like ordinary \textit{wh}-elements). This analysis presupposes a difference between polar and alternative questions, which is not borne out by the Middle English data, as shown by Table \ref{tablewycliffpolalt}. Examples are given in (\ref{mepolalt}) below (see also (\ref{cain}) above):\largerpage

\ea \label{mepolalt}
\ea \gll And Rachel and Lya answeryden, \textbf{Whe|ther} \textbf{han} we eny thing of residewe in faculteis and erytage of the hows of oure fader?\\
and Rachel and Leah answered whether have we any thing of residue in faculties and heritage of the house of our father\\
\glt `Then Rachel and Leah answered and said to him, ``Is there still any portion or inheritance for us in our father's house?'''\\
(Wycliffe Bible older version, Genesis 31.14)
\ex \gll And Rachel and Lya answeriden, \textbf{Wher} we han ony thing residue in the catels, and eritage of oure fadir?\\
and Rachel and Leah answered whether we have any thing residue in the cattles and heritage of our father\\
\glt `Then Rachel and Leah answered and said to him, ``Is there still any portion or inheritance for us in our father's house?'''\\
(Wycliffe Bible newer version, Genesis 31.14)
\ex \gll sendith of ȝou oon, and bringe he him, ȝe forsothe shulen ben in boondis, to the tyme that the thingis that ȝe han seide ben proued, \textbf{whether} fals or soth thei ben; ellis bi the helth of Pha|rao aspies ȝe ben.\\
send of you one and bring he him you forsooth shall be in bonds to the time that the things that you have said be proved whether false or true they are else by the health of Pharaoh spy you are\\
\glt `Send one of you, and let him bring your brother; and you shall be kept in prison, that your words may be tested to see whether they are false or true; or else, by the life of Pharaoh, surely you are spies!'''\\
(Wycliffe Bible older version, Genesis 42.16)
\ex \gll sende ȝe oon of ȝou, that he brynge hym, forsothe ȝe schulen be in boondis, til tho thingis that ȝe seiden. ben preued, \textbf{whe|ther}. tho. ben false ether. trewe; ellis, bi the helthe of Farao, ȝe ben aspieris.\\
send you one of you that he bring him forsooth you shall be in bonds till those things that you said be proved whether thou be false or true else by the health of Pharaoh you are spy\\
\glt `Send one of you, and let him bring your brother; and you shall be kept in prison, that your words may be tested to see whether they are false or true; or else, by the life of Pharaoh, surely you are spies!'''\\
(Wycliffe Bible newer version, Genesis 42.16)
\z
\z

It appears that verb fronting in these cases is essentially optional; at the same time, it seems to be dependent on personal preferences, as indicated by the differences in the frequency of the fronting option between the earlier and the later versions (see Table \ref{tablewycliffpolalt} above). Note that \textit{whether} was available in main clause questions until Early Modern English, when it was eventually superseded by \textit{do}-insertion; the co-occurrence of \textit{whether} and \textit{do}-insertion is attested in this period:

\protectedex{
\ea \textbf{Whether} \textbf{did} he open the Basket?\\(\textit{The Tryal of Thomas Earl of Macclesfield})\\(source: Salmon, Thomas and Sollom Emlyn (1730) A complete collection of state-trials, and proceedings for high-treason, and other crimes and misdemeanours: 1715--1725) \label{whetherdid}
\z
}

The optionality of verb movement in polar questions in Early English is reminiscent of the South German dialectal patterns discussed in connection with polar questions in the previous section. Accordingly, I propose that \textit{whether} is a \textit{wh}-operator in all cases,\footnote{The relatively high number of patterns in which the C position is filled by a fronted verb, the complementiser \textit{that} (and even the complementiser \textit{if}) indicates that \textit{whether} cannot be treated as a complementiser. In this respect, it differs from \textit{if} and its cognates in Old High German (see \sectref{sec:3german} below). Crucially, it also differs from the Old High German question particles \textit{eno} and \textit{inu}: while these particles can co-occur with fronted verbs, they are not attested on their own in the CP (that is, when the verb is not in C) and they can also co-occur with \textit{wh}-elements in constituent questions (see \citealt[42--44]{axel2007}). The only similarity to these Old High German particles is the fact that there is no complementary distribution between verb fronting and the use of interrogative markers (see \citealt[46]{axel2007} for Old High German).} and that it is either merged as a specifier to a C head lexicalised by another element (\textit{that} in embedded questions and a finite verb in main clause questions), or it is merged with the TP directly (occupying the C position), as in \figref{treewhetherthat} above. Notice that this flexibility is necessary not only to account for the observed optionality but also because treating \textit{whether} as a grammaticalised complementiser would not be compatible with the verb fronting (and doubling) patterns, the number of which is considerable in Middle English, so that the clear patterning of \textit{whether} with ordinary \textit{wh}-elements cannot be treated as exceptional. In other words, while the present analysis assumes, in line with \citet{vangelderen2009}, that \textit{whether} can merge directly with TP and thus be located in C, it crucially differs from \citet{vangelderen2009} in that it does not assume that \textit{whether} was a grammaticalised complementiser in any stage.

\subsection{Old Saxon and Old High German} \label{sec:3german}
The behaviour of \textit{whether} thus contrasts with \textit{if}, which is a complementiser in all periods in polar questions. Since most of the doubling patterns attested with \textit{whether} are historical, one might wonder whether similar patterns can be detected in other West Germanic languages historically. Both Old English and Old Saxon belonged to the Ingvaeonic dialects (also called North Sea Germanic) of West Germanic (together with Old Frisian, see \citealt[14]{lass1994}), while Old High German was non-Ingvaeonic.\footnote{The traditional distinction between Ingvaeonic, Istvaeonic (associated with Old Low Franconian) and Erminonic (associated with Old High German) goes back to Tacitus; see \citet[14--15]{lass1994} for discussion. The point here is simply that differences between the old languages most probably have their origins in dialectal differences within West Germanic, rather than being coincidental.} It is therefore to be expected that Old Saxon should be more similar to English than to Old High German. 

In Old Saxon, both the operator \textit{(h)wedar} `whether' and the complementiser \textit{ef} `if' are attested (cf. \citealt{axel2007}, who categorises all these elements as complementisers, contrary to the assumption here). I carried out a corpus analysis, using the DDD Referenzkorpus Altdeutsch (Old German Reference Corpus). The results (comprising all hits from the corpus) are given in Table \ref{tableoldsaxon} (both texts are from the 9th century).

\begin{table}
\begin{tabular}{lccc}
\lsptoprule
{} & \textit{ef} & \textit{(h)wedar} & \textit{(h)wedar} + V\\\midrule
Genesis & 1 & 1 & {}\\
Heliand & 5 & 2 & 1\\
\lspbottomrule
\end{tabular}
\caption{Corpus results for Old Saxon}
\label{tableoldsaxon}
\end{table}

An example for \textit{ef} is given in (\ref{osef}):

\ea \gll endi frâgodun, \textbf{ef} he uuâri that barn godes \label{osef}\\
and asked.\textsc{3pl} if he was.\textsc{3sg} the son God's\\
\glt `and they asked whether he was the son of God' (\textit{Heliand} 11)
\z
  
Examples for \textit{(h)wedar} are given in (\ref{oshwedar}):

\ea \label{oshwedar}
\ea \gll ne rôkead, \textbf{huueðar} gi is ênigan thanc antfâhan\\
not worry.\textsc{imp.2pl} whether you it some thank receive.\textsc{2pl}\\
\glt `do not worry whether you get some reward' (\textit{Heliand} 18)
\ex \gll endi he frâgoda sân, huilic sie ârundi ûta gibrâhti, uueros an thana uuracsîð \textbf{huueðer} \textbf{lêdiad} gi uundan gold te geƀu huilicun gumuno?\\
and he asked.\textsc{3sg} instantly, which they.\textsc{acc} business out brought.\textsc{3sg} man in this.\textsc{acc} foreign.land whether bring.\textsc{2pl} you wrought gold to gift.\textsc{dat} some men.\textsc{gen}\\
\glt `and he instantly asked, what business had brought them out from their land into this foreign land and whether you are bringing wrought gold as a gift to someone?' (\textit{Heliand} 7)
\z
\z

Not surprisingly, the Old Saxon pattern is similar to the English one in that the distribution of \textit{whether} and \textit{if} shows the relevant difference: \textit{ef} is a finite complementiser that cannot co-occur with a fronted verb in C, while \textit{(h)wedar} is an operator that may occur with or without verb movement, just like in the case of \textit{whether}, see the Middle English data above. Naturally, as the number of all occurrences is very low (there are altogether ten examples), the results are not fully conclusive in terms of the exact behaviour of the respective elements. Nevertheless, it seems appropriate to conclude that Old Saxon shows essentially the same pattern as English.\footnote{The six examples from \textit{Heliand} containing \textit{whether} are also reported by \citet[150--151]{walkden2014}, who likewise concludes that Old Saxon essentially patterns with the Old English data.}

In Old High German, the cognates of \textit{if} are attested (\textit{ibu} and \textit{ob}). Again, I used the DDD Referenzkorpus Altdeutsch (Old German Reference Corpus); the results (comprising all hits from the corpus) are given in Table \ref{tableohg}.\footnote{As discussed by \citet[151--152, 155]{walkden2014}, the element \textit{whether} had a `which of two' interpretation in Old High German, contrasting with Old English and Old Saxon.}

\begin{table}
\begin{tabular}{ lccc }
\lsptoprule
{} & \textit{ibu} + V & \textit{ob} & \textit{ob} + V\\\midrule
Benediktiner Regel (9th c.)& 1 & {} & {}\\
Otfrid (9th c.) & {} & 11 & {}\\
Tatian (9th c.) & {} & 8 & 1\\
Ludwigslied  (9th c.)& {} & 2 & {}\\
Psalm 138 (9--10th c.)& {} & 1 & {}\\
St. Galler Schularbeit (11th c.)& {} & 1 & {}\\
Benediktbeurer Glaube und Beichte III (12--13th c.) & {} & 1 & {}\\
\lspbottomrule
\end{tabular}
\caption{Corpus results for Old High German\label{tableohg}}
\end{table}

Examples are given in (\ref{ohgibuv}--\ref{pilatus}) below; for each Old High German example, the Latin original follows: both texts rely heavily on the Latin original.

\ea
\ea \gll fona himile simblum sihit ubar parn manno, daz sehe, \textbf{ibu} \textbf{ist} farstantanti edo suahhanti cotan \label{ohgibuv}\\
from heaven always sees onto children.\textsc{pl} men's, that see.\textsc{sbjv.3sg} if is understood or sought.\textsc{acc} God.\textsc{acc}\\
\glt `from Heaven, he always sees onto men's children, to see if God is understood or sought' (\textit{Benediktiner Regel} 7)
\ex \gll de caelo semper respicit super filios hominum, ut videat, si est intellegens aut requirens deum\\
of heaven.\textsc{abl} always sees onto sons.\textsc{acc} men.\textsc{gen} that sees\textsc{sbvj} if is understanding or requiring God.\textsc{acc}\\
\glt `from Heaven, he always sees onto men's children, to see if God is understood or sought'
\z
\z

\ea
\ea \gll láz nu, gisehemes \textbf{oba} \textbf{come} Helias losenti inan \label{ohgobv}\\
let.\textsc{imp.2sg} now see.\textsc{1pl} if comes Elias save.\textsc{inf} he.\textsc{acc}\\
\glt `let us see if Elias will come to save him' (\textit{Tatian} 208)
\ex \gll sine, videamus an veniat Helias liberans eum\\
let.\textsc{imp.2sg} see.\textsc{sbjb.1pl} whether comes.\textsc{sbjv} Elias freeing he.\textsc{acc}\\
\glt `let us see if Elias will come to save him'
\z
\z

\ea
\ea \gll Pilatus uuntrota, \textbf{oba} her iu entoti \label{pilatus}\\
Pilate wondered.\textsc{3sg} if he already died.\textsc{3sg}\\
\glt `Pilate wondered if he was already dead.' (\textit{Tatian} 212)
\ex \gll Pilatus autem mirabatur, si iam obisset\\
Pilate however wondered.\textsc{3sg} if already died.\textsc{sbjv.3sg}\\
\glt `Pilate wondered if he was already dead.'
\z
\z

The question arises whether there was verb movement to C with \textit{ibu}/\textit{ob}. While this cannot be excluded, there are several factors that should prevent us from reaching the conclusion that verb movement in such cases constituted a productive pattern. First, this option was available in the earliest texts, and this pattern is altogether very rare indeed. Note also that no clear dialect differences can be established: both the \textit{Benediktiner Regel} and \textit{Tatian} come from the Upper German dialect area, like most of the texts in the table above (only the \textit{Ludwigslied} is Central German): the \textit{Benediktiner Regel} is Alemannic, \textit{Tatian} is East Franconian, and \textit{Otfrid} is South Rhine Franconian.

Second, while the clauses in (\ref{ohgibuv}) and (\ref{ohgobv}) may involve verb movement to C, whereby \textit{ibu}/\textit{ob} is an operator in [Spec,CP], it is also possible that there is no verb movement to C at all and the surface word order is a result of other factors, given that Old High German word order was considerably less fixed in this respect than Modern German (see, for instance, \citealt{hinterhoelzlpetrova2010} and \citealt{conigliolinderuette2017} on variation in verb position). Third, apart from internal reasons, the Old High German examples are translations from Latin: as can be seen by comparing these clauses to the Latin originals, the Old High German word order mirrors the Latin word order, so the observed patterns may be the result of (almost verbatim) translation. 

That is, the low number of examples from Old High German is not conclusive with respect to whether \textit{ibu}/\textit{ob} was available as an operator; in fact, the factors mentioned above seriously undermine such a possibility (though it cannot be completely excluded either). What is also evident is that even if it was an existing pattern, it was restricted to only a few early examples and it grammaticalised very early as a complementiser. This contrasts with the behaviour of English \textit{whether} and Old Saxon \textit{(h)wedar}, and based on the Old High German distribution, it should not be surprising that present-day \textit{ob} is not available as an operator in dialects either.

\subsection{Dutch} \label{sec:3dutch}
As \citet[14]{lass1994} remarks, Dutch seems to be a bit of a hybrid in terms of Ingvaeonic and non-Ingvaeonic patterns; the question arises what the status of \textit{of} `if' in Dutch is. In Standard Dutch, there is no doubling, similar to the case of English \textit{if} (see \citealt{bayer2004}, following \citealt{hoekstra1993}). However, the combination \textit{of dat} is possible in non-standard dialects (see, for instance, \citealt[65, ex. 14]{bayer2004}, quoting \citealt{hoekstra1993}). Consider (\citealt[27--28]{bacskaiatkaribaudisch2018}):

\ea \gll Peter vroeg \textbf{of} \textbf{dat} Mary houdt van boeken. \label{dutchofdat}\\
Peter asked.\textsc{3sg} if that Mary holds of books\\
\glt `Peter asked if Mary liked books.'
\z

\begin{sloppypar}
As can be seen, non-standard dialects of Dutch treat \textit{of} on a par with \textit{wh}-operators with respect to the availability of an overt finite complementiser: note that non-standard dialects allow Doubly Filled COMP with ordinary \textit{wh}-el\-e\-ments in Dutch (with considerable differences in the actual patterns, see \citealt[1612--1613]{barbiers2009}, \citealt{vancraenenbroeck2010}; see also \citealt{bayer2004}, quoting \citealt{hoekstra1993}). In line with \citet[141--142]{boef2013}, I assume that \textit{of} in these cases is the question operator: technically, this means that it is a specifier merged to a complementiser but not the complementiser itself. 
\end{sloppypar}

In other words, \textit{of} may or may not be equipped with a [fin] feature. In Standard Dutch, as well as in varieties that do not have the combination \textit{of dat}, \textit{of} is specified as [fin] and is incompatible with another finite complementiser (\textit{dat}). In those varieties however, which treat \textit{of} on a par with other interrogative operators, \textit{of} is not specified as [fin] and hence may co-occur with \textit{dat}. Co-occurrence with \textit{dat} seems to be largely optional (\citealt[1612]{barbiers2009}); this is expected if a head-sized element may either appear as a specifier or in the C head. \citet[1612--1613]{barbiers2009} reports that there is considerable -- inter-speaker and intra-speaker -- variation regarding the preferences in the relevant patterns: this is again expected in the present approach since elements like \textit{of} are not tied to a pre-given position in a syntactic template. Just like with head-sized \textit{wh}-elements in German dialects, the preference for the head position may be very strong or rather weak, resulting in different grammatical outputs.

This implies that the status of Dutch \textit{of} differs crucially from that of German \textit{ob}. Apart from the fact that German does not show constructions like (\ref{dutchofdat}), there are further differences justifying this distinction. First, as described by \citet{boef2013}, \textit{of} is a general disjunctive element in Dutch (in the sense of `or'). Second, \textit{of} may co-occur with ordinary \textit{wh}-operators in constituent questions. Consider (\citealt[66, ex. 17]{bayer2004}, citing \citealt{hoekstra1993}):

\ea \gll Ze weet \textbf{wie} \textbf{of} \textbf{dat} hij had willen opbellen. \label{wieofdat}\\
she knows who if that he had want call\\
\glt `She knows who he wanted to call.'
\z

I will turn to the analysis of (\ref{wieofdat}) in the next section; the point here is merely that there are various indicators in favour of treating dialectal Dutch \textit{of} differently from German \textit{ob}.

Importantly, Doubly Filled COMP patterns may arise in polar questions as well, since the requirement to lexicalise [fin] on C applies here just like it does in constituent questions. Naturally, doubling only arises when the polar interrogative marker can be merged as an operator; if it is a grammaticalised complementiser, it is sufficient to mark [fin] itself. In addition, the availability of verb movement to C with a polar operator in [Spec,CP] shows that Doubly Filled COMP patterns are not directly related to the clause-typing status of the finite complementiser corresponding to \textit{that} but the property is rather related to the requirement to merge a phonologically non-null element with the TP.

\section{Doubly Filled COMP and V2} \label{sec:3doubly}
\subsection{Declarative clauses} \label{sec:3declarative}
As pointed out previously in this chapter, the main idea underlying the proposed analysis is that Doubly Filled COMP effects stem from the necessity of filling the C head with an overt element (cf. also the descriptive observation made by \citealt[85--86]{lenerz1984} and the condition of ``C-visibility'' by \citealt{pittner1995}). The lexicalisation of the operator follows from independent reasons: clause-typing operators have to move to the left. Moreover, as \citet{fanselow2009} argues, the filling of [Spec,CP] is not directly related to the notion of V2. This assumption will be slightly modified in the present section.\footnote{The analysis follows the argumentation put forward in \citet{bacskaiatkari2020jcgl}.}

Let us start with German V2 clauses as exemplified in (\ref{v2}), repeated here as (\ref{v2repeat}):\largerpage[-1]

\ea \label{v2repeat}
\ea \gll \textbf{Ralf} \textbf{hat} morgen Geburtstag. \label{v2subject}\\
Ralph has tomorrow birthday\\
\glt `Ralph has his birthday tomorrow.'
\ex \gll \textbf{Morgen} \textbf{hat} Ralf Geburtstag. \label{v2adverb}\\
tomorrow has Ralph birthday\\
\glt `Ralph has his birthday tomorrow.'
\z
\z

As indicated, the first constituent can be of various categories: it is the subject DP in (\ref{v2subject}) and an adverb in (\ref{v2adverb}), but it is at any rate a phrase-sized constituent (XP). In line with the literature on V2 (see \citealt{thiersch1978diss, denbesten1989, fanselow2002, fanselow2004isis, fanselow2004, frey2005}; see also \citealt{westergaard2007, westergaard2008, westergaard2009}, \citealt{krochtaylor1997} and \citealt{lightfoot1999, lightfoot2006}), I assume that the verb moves to C,\footnote{The movement of the verb is related to clause typing, see also \citet{truckenbrodt2006}.} while the XP moves to [Spec,CP] due to an [edge] feature. The representation in \figref{treev2} shows the structure in X-bar theoretic terms.\footnote{In line with the model put forward in this book, the reflex of finiteness in the CP is taken to be a [fin] feature; in this respect, the model is similar to the approach articulated by \citet{rizzi1997}. As pointed out by \citet{chomsky2001, chomsky2008}, finiteness is probably a composite of features, including phi features (see also \citealt{cowper2016} for discussion). Since phi features are checked off in the TP, what remains relevant for the CP is the syntactic information that the clause is finite, as this affects the combinability of the clause as an embedded clause and/or as a main clause. I refer to this property as [fin].}

\begin{figure}
\caption{The structure of German V2}
\label{treev2}
\begin{forest} baseline, qtree
[CP
	[XP\textsubscript{{[}edge{]}}
		[Ralf/morgen,roof]
	]
	[C$'$
		[C\textsubscript{{[}fin{]},{[}edge{]}}
			[V [hat]]
			[C [$\emptyset$\textsubscript{{[}fin{]}}]]
		]
		[TP]
	]
]
\end{forest}
\end{figure}

The representation shows a head adjunction analysis,\footnote{The head adjunction analysis of head movement is controversial, as already pointed out by \citet{fanselow2004}; see also \citet{dekany2018} for a recent discussion.} which is almost the only way of representing a verb in C, other than simply labelling the verb as C, which is clearly not the right category label. Assigning the category V to the verb and not using a separate C head would either violate endocentricity (V being the head of CP) or would force us to assume that the entire clause is a VP regarding its category. The problem with this in representational terms is that the distribution and syntactic behaviour of finite, declarative main clauses pattern with other finite clauses rather than with mere verb phrases. In other words, while the X-bar schema is indeed useful for representation purposes (which is precisely the reason why I adopt it in this book), it should not be taken at face value.\footnote{The following representation shows the Bare Phrase Structure representation, using only the example in (\ref{v2adverb}) above:

\ea \label{treev2bare}
\begin{forest} baseline, qtree
[hat
	[morgen\textsubscript{{[}edge{]}}]
	[hat
		[hat\textsubscript{{[}fin{]},{[}u:edge{]}}
		]
		[TP\textsubscript{{[}u:fin{]}}
			[Ralf Geburtstag,roof]
		]
	]
]
\end{forest}
\z

In line with the general assumptions regarding minimal and maximal projections in Base Phrase Structure, the maximal (phrasal) status of the first constituent (here: \textit{morgen}) arises from the fact that it does not project further; minimal and maximal projections do not have to be structurally distinguished for head/sized phrases. The label of the phrase is provided by the element that projects further (see \citealt{chomsky1995, chomsky2013}); in other words, no external labels (e.g. CP) are used: Bare Phrase Structure is endocentric in this respect.}

The [u:fin] feature must be checked off on TP, and this is carried out by the finite verb (following \citealt[309]{fanselow2004}). There is no overt finite complementiser available for main clause declaratives in German (as is regularly the case in Germanic languages), resulting in a surface V2 pattern. In the particular implementation assumed here, a [fin] feature of TP has to be checked off: while TP was in fact projected from the verb, the strong feature cannot be checked automatically. The only possibility is to re-merge (move) the verb possessing the [fin] feature: this ultimately produces a finite clause (as the satisfied finiteness feature projects as a label), which is, without the addition of clause-type markers proper (e.g. interrogative elements) is declarative. In other words, there is no separate element or designated layer necessary for encoding clause type as long as it is the unmarked declarative. The representation in (25) above indicates that the verb is not a complementiser itself, yet it occupies the relevant position, and by virtue of the [fin] feature it makes the clause finite just as a finite complementiser would do.

English crucially differs here: one may either assume that no further layer above the TP is generated in main clause declaratives at all, or that a zero finite complementiser is available in the lexicon; at any rate, English does not show V2 patterns in simple declarative clauses.\footnote{Obviously, the interrogative patterns to be discussed below in this section are a residue of a former V2 grammar, just as other inversion structures (negative inversion, quotative inversion), as pointed out by \citet{rizzi1996} and \citet{roberts2010}. These V2 patterns do not involve lexical verbs and they are triggered by very specific elements; see also \citet{sailor2020} for discussion.}  

While the role of verb movement is thus straightforward, the movement of the XP to the specifier requires some explanation. At the point of re-merging the verb with TP, the [fin] feature is active on the head. \citet[171]{mueller2011} provides a modified definition of the Edge Feature Condition of \citet[109]{chomsky2000}, claiming that edge features ``can only be inserted as long as the phase head is active'', and a phase head ``is active as long as it has (structure-building or probe) features to discharge'', but ``otherwise it counts as inactive''. In (\ref{treev2bare}), the active phase head with a yet unchecked feature triggers the insertion of the [edge] feature, which in turn triggers the movement of an XP to the specifier. In this sense, the fact that a finite verb is re-merged and that a specifier in the same projection (traditionally referred to as CP) emerges are related: note that this does not mean any surface V2 requirement (see also \citealt{fanselow2009}), as the XP may in principle be covert (as will be discussed for certain clause-typing operators, but the same holds in topic drop constructions, see \citealt{trutkowski2016}).

In English, since the [fin] feature is interpretable on the zero declarative complementiser, there is no unchecked [fin] feature on C, and the C is not active: consequently, the [edge] feature is not inserted either. In other words, there is no verb movement to C and XP-movement to the specifier in English declaratives, resulting in the lack of V2, as opposed to other Germanic languages.

Let us now turn to embedded finite declarative clauses. An example for German is given in (\ref{germandass}):

\ea \gll Ich weiß, \textbf{dass} Ralf den Salat gemacht hat. \label{germandass}\\
I know.\textsc{1sg} that Ralph the.\textsc{m.acc} salad made.\textsc{ptcp} has\\
\glt `I know that Ralph has prepared the salad.'
\z

The structure is given in \figref{treedass} (page \pageref{treedass}).

In this case, the complementiser \textit{dass} is inserted, which is equipped with an interpretable [fin] feature. Given this, the feature [fin] on \textit{dass} does not make the C head active and thus no [edge] feature is inserted. The same applies to English \textit{that}-clauses as well. The clause is typed as finite by the complementiser: this information is necessary for the matrix predicate.\footnote{The diagram in \figref{treedass} uses traditional X-bar labels for ease of representation; the Bare Phrase Structure is largely identical:

\ea \label{treedassbare}
\begin{forest} baseline, qtree
[dass
	[dass\textsubscript{{[}fin{]}}]
	[TP\textsubscript{{[}u:fin{]}}
		[\phantom{xxx},roof]
	]
]
\end{forest}
\z

This is because the X-bar representation does not have to resort to head adjunction in this case.}

The difference between English and German lies in the availability of a zero declarative complementiser. The [fin] feature, just like in matrix clauses, is interpretable on the English zero complementiser. Descriptively, this results in the optionality of \textit{that} in non-fronted clauses:

\ea I think \textbf{(that)} Ralph likes turtles. \label{turtles}
\z

\begin{figure}[h]
\caption{Embedded declaratives} \label{treedass}
\begin{forest} baseline, qtree
[CP
	[C$'$
		[C\textsubscript{{[}fin{]}}
			[dass\textsubscript{{[}fin{]}}]
		]
		[TP]
	]
]
\end{forest}
\end{figure}

Note that English \textit{that} is not always interchangeable with the zero complementiser (for instance, it is not permitted in subject clauses), and authors such as \citet{kayne1984}, \citet{stowell1981diss} and \citet{pesetsky1995} have argued that the zero finite complementiser has the same distribution as traces or can even be treated as a trace (see also the discussion in \chapref{ch:2} in connection with \citealt{rizzi1997}). Nevertheless, the point is that the absence of an overt \textit{that} does not necessarily lead to ungrammaticality and it does not trigger verb movement either, as demonstrated by (\ref{turtles}).

German crucially differs here. Observe:\largerpage

\ea
\ea \gll Ich denke, \textbf{*(dass)} Ralf Schildkröten mag. \label{turtlesdass}\\
I think.\textsc{1sg} \phantom{\textbf{*(}}that Ralph turtles likes\\
\glt `I think that Ralph likes turtles.
\ex \gll Ich denke, Ralf \textbf{mag} Schildkröten. \label{turtlesv2}\\
I think.\textsc{1sg} Ralph likes turtles\\
\glt `I think Ralph likes turtles.'
\z
\z

As indicated, in German either \textit{dass} is used, as in (\ref{turtlesdass}), or verb movement occurs, as in (\ref{turtlesv2}): a silent complementiser without verb movement is not possible. It should be mentioned that verbs differ with respect to whether they allow embedded V2 or not: for instance, the verb \textit{bezweifeln} `doubt' allows only a \textit{dass}-clause but not verb fronting. There exist various hypotheses on how the two groups can be separated on formal grounds: a traditional idea is that embedded V2 is allowed by ``bridge verbs'' (\citealt{vikner1995}; see also \citealt{green1976}).\footnote{This distinction is problematic on empirical grounds, as pointed out by \citet{featherston2004} and \citet{meklenborgsalvesenwalkden2017}: notably, the ``bridge feature'' should be understood as a continuum and not as a categorial distinction (\citealt[205]{featherston2004}). See also \citet{hooperthompson1973} for discussion.}

If embedded V2 is possible, it is derived in the same way as \figref{treev2}. Note that there are also different analyses of embedded V2. For instance, \citet{denbesten1983} treats these clauses as main clauses (V2 being a ``Main Clause Phenomenon'' in asymmetric V2 languages like German and Dutch); there are various problems with this analysis, see also \citet{heycock2006}. On the other hand, there are analyses treating embedded V2 clauses as proper complement clauses (see \citealt{weerman1989}, \citealt{hooperthompson1973}). \citet{reis1997} takes a middle path in that she treats embedded V2 clauses as syntactically relatively unintegrated subclauses (essentially argument clauses that are not located in the complement position of the verb but adjoined to the VP). This is slightly problematic for a merge-based account, and the differences concern primarily the final syntactic position of the subclause and they do not undermine the fact that the matrix verb imposes restrictions on the left periphery of the subclause. For these reasons, I assume that embedded V2 clauses are selected by a matrix verb. Under this view, certain verbs select a complement headed by \textit{dass}, while others select a finite CP complement and do not impose further restrictions on the head. 

\subsection{Interrogative clauses} \label{sec:3interrogative}
Let us turn to matrix interrogatives. Constituent and polar questions are illustrated for German in (\ref{germanint}) below:

\ea \label{germanint}
\ea \gll \textbf{Wer} \textbf{hat} den Salat gemacht? \label{germanintwer}\\
who has the.\textsc{m.acc} salad made.\textsc{ptcp}\\
\glt `Who prepared the salad?'
\ex \gll \textbf{Hat} Ralf den Salat gemacht? \label{germaninthat}\\
has Ralph the.\textsc{m.acc} salad made.\textsc{ptcp}\\
\glt `Did Ralph prepare the salad?'
\z
\z

The X-bar structure of (\ref{germanintwer}) is shown in \figref{treeintgermanintwer}. The X-bar structure of (\ref{germaninthat}) is shown in \figref{treepolar}.

\begin{figure}
\caption{Main clause constituent questions} \label{treeintgermanintwer}
\begin{forest} baseline, qtree
[CP
	[wer\textsubscript{{[}wh{]}}]
	[C$'$
		[C\textsubscript{{[}fin{]},{[}wh{]}}
			[V [hat]]
			[C [$\emptyset$\textsubscript{{[}fin{]}}]]
		]
		[TP]
	]
]
\end{forest}
\end{figure}

\begin{figure}
\caption{Main clause polar questions} \label{treepolar}
\begin{forest} baseline, qtree
[CP
	[\textit{Op}.\textsubscript{{[}Q{]}}]
	[C$'$
		[C\textsubscript{{[}fin{]},{[}Q{]}}
			[V [hat]]
			[C [$\emptyset$\textsubscript{{[}fin{]}}]]
		]
		[TP]
	]
]
\end{forest}
\end{figure}

Again, just as with main clause declaratives, verb movement is represented as head adjunction in X-bar terms.\footnote{The Bare Phrase Structures of (\ref{germanintwer}) is given in (\ref{treeconstituentbare}) and the Bare Phrase Structure of (\ref{germaninthat}) in (\ref{treepolarbare}):

\begin{multicols}{2}\raggedcolumns
\ea \label{treeconstituentbare}
\begin{forest} baseline, qtree
[hat
	[wer\textsubscript{{[}wh{]}}]
	[hat\textsubscript{{[}u:wh{]}}
		[hat\textsubscript{{[}fin{]}}
		]
		[TP\textsubscript{{[}u:fin{]},{[}u:wh{]}}
			[\phantom{xxx},roof]
		]
	]
]
\end{forest}
\columnbreak\ex \label{treepolarbare}
\begin{forest} baseline, qtree
[hat
	[\textit{Op}.\textsubscript{{[}Q{]}}]
	[hat\textsubscript{{[}u:Q{]}}
		[hat\textsubscript{{[}fin{]}}
		]
		[TP\textsubscript{{[}u:fin{]},{[}u:Q{]}}
			[\phantom{xxx},roof]
		]
	]
]
\end{forest}
\z
\end{multicols}

Just as with V2 declaratives, the label is given provided by the verb.}

The [fin] feature is lexicalised by verb movement just like in German V2 declaratives, see \figref{treev2} above. Again, the C head is active, yet the [edge] feature is not inserted, since the operator feature -- [Q] or [wh] -- triggers movement anyway. The interrogative element is necessarily overt in constituent questions but not in polar questions (see the discussion in \sectref{sec:3embeddedpolar} above, especially regarding overt interrogative markers in matrix questions historically). In the case of \figref{treepolar}, inserting a covert operator results in a surface V1 order in German, as opposed to V2 in constituent questions and in declaratives.

Regarding English, verb movement to C from T is triggered in main clause interrogatives as well, unlike in declaratives. This indicates that the lexicalisation requirement is dependent on the exact features involved. While a zero declarative complementiser with a [fin] feature is available in English, it cannot type the clause as [wh]/[Q]. Assuming that an abstract feature bundle is
added in the syntax (see \citealt{chomskylasnik1977}) and lexicalised by a matching lexical element, if and to the extent that there is one, in the present case there would be simply no complementiser element in the English lexicon to satisfy these requirements. The resulting property of English interrogatives (traditionally referred to as T-to-C movement) is most probably a remnant of the original V2 property of the language; the point is that the lexicalisation of the finite C head may vary across clause types (not just across languages and dialects).

In embedded polar questions, German uses an overt complementiser:

\ea \gll Ich weiß nicht, \textbf{ob} Ralf den Salat gemacht hat.\\
I know.\textsc{1sg} not if Ralph the.\textsc{m.acc} salad made.\textsc{ptcp} has\\
\glt `I don't know if Ralph has prepared the salad.'
\z

The X-bar structure is shown in \figref{treeintpolar}.\footnote{As with embedded declaratives, the representation in Bare Phrase Structure is similar:

\ea 
\begin{forest} baseline, qtree
[ob
	[ob\textsubscript{{[}fin{]},{[}Q{]}}]
	[TP\textsubscript{{[}u:fin{]},{[}u:Q{]}}
		[\phantom{xxx},roof]
	]
]
\end{forest}
\z

The label is the complementiser \textit{ob}.}

\begin{figure}
\caption{Embedded polar questions} 
\label{treeintpolar}
\begin{forest} baseline, qtree
[CP
	[\textit{Op}.\textsubscript{{[}Q{]}}]
	[C$'$
		[C\textsubscript{{[}fin{]},{[}Q{]}}
			[ob\textsubscript{{[}fin{]},{[}Q{]}}]
		]
		[TP]
	]
]
\end{forest}
\end{figure}

In this case, the C head is lexicalised by an overt complementiser specified as [Q] and [fin]. The same configuration applies in \textit{if}-interrogatives in English. In German, this configuration matches the full syntactic paradigm that we have discussed in connection with main clauses and embedded clauses. In English embedded polar interrogatives containing \textit{if}, the same configuration matches the embedded paradigm and main clause interrogatives.

Regarding embedded constituent questions, Standard German differs from dialectal patterns that allow or even require \textit{dass} in C (the same difference holds in English and in Dutch). The phenomenon is illustrated in (\ref{germandfcvariation}) below:

\ea \gll Ich weiß nicht, \textbf{wer} \textbf{(\%dass)} den Salat gemacht hat. \label{germandfcvariation}\\
I know.\textsc{1sg} not who \phantom{\textbf{(\%}}that the.\textsc{m.acc} salad made.\textsc{ptcp} has\\
\glt `I don't know who has prepared the salad.'
\z

The X-bar structure of the (non-standard) version containing \textit{dass} is given in \figref{treeintwh}. The X-bar structure of the \textit{dass}-less (standard) version is given in \figref{treea}.

\begin{figure}
\captionsetup{margin=.05\linewidth}
\begin{floatrow}
\ffigbox
{\begin{forest} baseline, qtree
[CP
	[wer\textsubscript{{[}wh{]}}]
	[C$'$
		[C\textsubscript{{[}fin{]},{[}wh{]}}
			[$\emptyset$\textsubscript{{[}fin{]}}]
		]
		[TP]
	]
]
\end{forest}}
{\caption{Embedded constituent questions in Standard German} \label{treeintwh}}

\ffigbox
{\begin{forest} baseline, qtree
[CP
	[wer\textsubscript{{[}wh{]}}]
	[C$'$
		[C\textsubscript{{[}fin{]},{[}wh{]}}
			[dass\textsubscript{{[}fin{]}}]
		]
		[TP]
	]
]
\end{forest}}
{\caption{Doubling in embedded constituent questions} \label{treea}}
\end{floatrow}
\end{figure}

The representation in \figref{treeintwh} is the standard pattern, while the representation in \figref{treea} is the dialectal pattern.\footnote{The Bare Phrase Structure representations are given in (\ref{treesbare}) below:

\begin{multicols}{2}
\ea \label{treesbare}
\ea \label{treeintwhbare}
\begin{forest} baseline, qtree
[$\emptyset$
	[wer\textsubscript{{[}wh{]}}]
	[$\emptyset$\textsubscript{{[}u:wh{]}}
		[$\emptyset$\textsubscript{{[}fin{]}}
		]
		[TP\textsubscript{{[}u:fin{]},{[}u:wh{]}}
			[\phantom{xxx},roof]
		]
	]
]
\end{forest}
\ex \label{treeabare}
\begin{forest} baseline, qtree
[dass
	[wer\textsubscript{{[}wh{]}}]
	[dass\textsubscript{{[}u:wh{]}}
		[dass\textsubscript{{[}fin{]}}
		]
		[TP\textsubscript{{[}u:fin{]},{[}u:wh{]}}
			[\phantom{xxx},roof]
		]
	]
]
\end{forest}
\z
\z

\end{multicols}

The ultimate difference lies in whether the complementiser is overt or not.} In \figref{treeintwh}, unlike in all the other cases above, see Figures~\ref{treev2}, \ref{treedass}, \ref{treeintgermanintwer} and \ref{treeintpolar}, the [fin] feature is encoded by a zero complementiser. The assumption is that in Standard German, a zero complementiser with a [wh] and [fin] specification is part of the lexicon and is interpretable if it is licensed by a matrix predicate (in other words, such a complementiser is excluded from main clauses). At any rate, this feature specification makes sure that while the element in C (that is, the element directly merged with the TP) is not overt, at least the element merged as a specifier is (given that \textit{wh}-elements are necessarily overt, as discussed earlier): this prevents the generation of phonologically empty projections.

The English paradigm is different inasmuch as the availability of a zero complementiser depends primarily on clause type (e.g. declarative versus interrogative) and not so much on whether the clause is embedded or not, whereas this is crucial in German. The configuration in \figref{treeintwh} is in both languages exceptional with respect to the interrogative paradigm (in English, the same applies to interrogatives with \textit{whether}). Note also that relative clauses are also exceptional, especially in (Standard) German; these questions will be addressed in \chapref{ch:4}.

The structure in \figref{treea} is essentially the same as the one in \figref{treeintwh}, with the important difference that the complementiser is overt in \figref{treea} but not in \figref{treeintwh}. That is, the difference is not so much in the syntax but rather in the lexical elements. There is an underlying lexical difference between the standard language and dialects: in standard German, the [fin] feature on an embedded [wh] zero complementiser is interpretable, but not in dialects applying the strategy shown in \figref{treea}. 

I assume that an abstract feature bundle is inserted in syntax (cf. \citealt{chomskylasnik1977}), which is then replaced by a matching lexical item: this lexical item may fully match the features in question, as in \figref{treeintpolar}, or it may provide a partial match, as in \figref{treeintwh} and (\ref{treea}), in which case the remaining feature is uninterpretable on the inserted lexical item. While \textit{dass} is incompatible with the [wh] feature in Standard German and is therefore categorically excluded from interrogatives, it is not sensitive to this feature in dialects that allow its insertion in the relevant clauses. In either case, since the [wh] feature is uninterpretable on the complementiser, the movement of the \textit{wh}-element is triggered. Following the distinctions made by \citet{bayerbrandner2008}, as discussed in \sectref{sec:3bayerbrandner}, the analysis thus far covers symmetric dialects that uniformly allow or prohibit Doubly Filled COMP in embedded interrogatives.

As described by \citet{bayerbrandner2008}, there are also asymmetric dialects that require the insertion of \textit{dass} with phrase-sized \textit{wh}-elements but not with head-sized ones. In \sectref{sec:3bayerbrandner}, I discussed their proposal regarding locating such \textit{wh}-elements in C and evaluated its advantages and disadvantages. In particular, I argue that while the \textit{wh}-element is indeed in C in these cases, it is not the complementiser itself, as the notion of the (latent) C-feature is problematic. Instead, I propose that the \textit{wh}-element in these cases should be treated in the same way as verbs moving to C. In an X-bar representation, as illustrated in \figref{treeb}, this would translate as head adjunction:\footnote{The Bare Phrase Structure representation is given in (\ref{treewerbare}) below:

\ea \label{treewerbare}
\begin{forest} baseline, qtree
[wer\textsubscript{{[}u:fin{]}}
	[wer\textsubscript{{[}wh{]}}]
	[TP\textsubscript{{[}u:fin{]},{[}u:wh{]}}
		[\phantom{xxx},roof]
	]
]
\end{forest}
\z

In this case, \textit{wer} projects as a label.}

\begin{figure}
\caption{\textit{Wh}-elements in C} 
\label{treeb}
\begin{forest} baseline, qtree
[CP
	[C$'$
		[C\textsubscript{{[}fin{]},{[}wh{]}}
			[wer\textsubscript{{[}wh{]}}]
			[C [$\emptyset$\textsubscript{{[}fin{]}}]]
		]
		[TP]
	]
]
\end{forest}
\end{figure}

Just as in the case of \figref{treea}, the abstract feature bundle is lexicalised by a partially matching element, but instead of the finite complementiser, it is the fronted \textit{wh}-element: this element is crucially underspecified for the [fin] feature. Unlike the [wh] feature, which at any rate requires the fronting of the element it is located on, the same is not true for [fin], as in all cases where a complementiser is inserted, the fronting of the verb is not triggered in West Germanic. Configurations like in \figref{treeb} are licensed only in embedded clauses since the non-lexicalised [fin] feature has to be licensed; main clauses are obligatorily finite and cannot depend on a licensing element from a higher clause. The relevant features matter in terms of clause
typing, and additional labels such as C or V do not play a role in Bare Phrase Structure. 

The features (such as the interrogative feature or the finiteness feature) can be carried by other elements as well, as long as there is no categorial restriction from the matrix predicate. The configuration in \figref{treeb} is compatible with a matrix predicate requiring a [wh] complement, while it would not be possible with a matrix predicate requiring a [wh] complement headed by a C element specifically.

The same variation applies in polar questions in English with \textit{whether}, with the important difference that the feature involved is [Q] and not [wh]. Consider the cases where \textit{whether} is merged as a specifier, using X-bar representations:\footnote{The Bare Phrase Structure representations are given in (\ref{treesbareq}) below:

\begin{multicols}{2}
\ea \label{treesbareq}
\ea \label{treewhtherstandardbare}
\begin{forest} baseline, qtree
[$\emptyset$
	[whether\textsubscript{{[}Q{]}}]
	[$\emptyset$\textsubscript{{[}u:Q{]}}
		[$\emptyset$\textsubscript{{[}fin{]}}
		]
		[TP\textsubscript{{[}u:fin{]},{[}u:Q{]}}
			[\phantom{xxx},roof]
		]
	]
]
\end{forest}
\ex \label{treewhetehrdoublingbare}
\begin{forest} baseline, qtree
[that
	[whether\textsubscript{{[}Q{]}}]
	[that\textsubscript{{[}u:Q{]}}
		[that\textsubscript{{[}fin{]}}
		]
		[TP\textsubscript{{[}u:fin{]},{[}u:Q{]}}
			[\phantom{xxx},roof]
		]
	]
]
\end{forest}
\z
\z

\end{multicols}

The structures are altogether similar to the ones established for embedded constituent questions.}

\begin{figure}
\caption{The standard position of \textit{whether}}\label{treewhtherstandard}
\begin{forest} baseline, qtree
[CP
	[whether\textsubscript{{[}wh{]}}]
	[C$'$
		[C\textsubscript{{[}fin{]},{[}Q{]}}
			[$\emptyset$\textsubscript{{[}fin{]}}]
		]
		[TP]
	]
]
\end{forest}
\end{figure}

\largerpage[-1] The structure in \figref{treewhtherstandard} represents the standard pattern, where a zero complementiser encodes [fin] feature. The non-standard pattern in \figref{treewhetehrdoubling} differs only in the lexical element inserted as a complementiser: it is an overt \textit{that}, in line with the rest of the English interrogative paradigm. As was mentioned in \sectref{sec:3embeddedpolar}, even non-standard dialects seem to prefer single \textit{whether}, even if they otherwise show Doubly Filled COMP patterns in embedded constituent questions. As I argued, such cases are instances of \textit{whether} inserted into C; in the X-bar structure, this is represented as in \figref{treewhetherc}.\footnote{The Bare Phrase Structure representation is as in \REF{treewhetherbare}.

\ea
\label{treewhetherbare}
\begin{forest} baseline, qtree
[wer\textsubscript{{[}u:fin{]}}
	[wer\textsubscript{{[}wh{]}}]
	[TP\textsubscript{{[}u:fin{]},{[}u:wh{]}}
		[\phantom{xxx},roof]
	]
]
\end{forest}
\z

Again, the label is the \textit{wh}-element \textit{wer}.}

\begin{figure}
\captionsetup{margin=.05\linewidth}
\begin{floatrow}
\ffigbox
{\begin{forest} baseline, qtree
[CP
	[C$'$
		[C\textsubscript{{[}fin{]},{[}Q{]}}
			[whether\textsubscript{{[}wh{]}}]
			[C [$\emptyset$\textsubscript{{[}fin{]}}]]
		]
		[TP]
	]
]
\end{forest}}
{\caption{The operator \textit{whether} in C}\label{treewhetherc}}

\ffigbox
{\begin{forest} baseline, qtree
[CP
	[whether\textsubscript{{[}wh{]}}]
	[C$'$
		[C\textsubscript{{[}fin{]},{[}Q{]}}
			[that\textsubscript{{[}fin{]}}]
		]
		[TP]
	]
]
\end{forest}}
{\caption{Doubling with \textit{whether}}\label{treewhetehrdoubling}}
\end{floatrow}
\end{figure}

As mentioned in \sectref{sec:3embeddedpolar}, the difference between constituent questions and polar questions is expected since they differ in their feature specification, [wh] versus [Q]. 

Let us now turn to the triple combination attested in Dutch dialects that was mentioned in \sectref{sec:3dutch}, as exemplified in (\ref{wieofdat}), repeated here as (\ref{wieofdatrepeat}):\largerpage[2]

\ea \gll Ze weet \textbf{wie} \textbf{of} \textbf{dat} hij had willen opbellen. \label{wieofdatrepeat}\\
she knows who if that he had want call\\
\glt `She knows who he wanted to call.'
\z

I propose the construction in \figref{treewieofdat} for the combination \textit{wie of dat}.\footnote{The structure relies on the idea that in minimalist syntax, multiple specifiers are possible; this can be extended to V3 orders in main clauses, see \citet[148--149]{bacskaiatkari2020jcgl}. \citet{lahne2009} also proposes multiple specifiers instead of separate left-peripheral projections, as an attractive alternative for the cartographic approach. Note, however, that her system generally relies on multiple specifiers so that apparent left peripheral heads are generally assumed to be affixes on displaced constituents. This differs crucially from the system proposed here, as I do not exclude the possibility of multiple projections (see \chapref{ch:5}).}\pagebreak

\begin{figure}
\caption{Triple combinations in Dutch}\label{treewieofdat}
\begin{forest} baseline, qtree
[CP
	[wie\textsubscript{{[}wh{]}}]
	[C$'$
		[of\textsubscript{{[}Q{]}}]
		[C$'$
			[C\textsubscript{{[}fin{]},{[}wh{]}} [dat\textsubscript{{[}fin{]}}]]
			[TP]
		]
	]
]
\end{forest}
\end{figure}

The structure in \figref{treewieofdat} differs from that of \citet[75]{bayer2004}, who considers \textit{of} to be the head of a separate Disjunction Phrase. As I argued in \sectref{sec:3dutch}, there are reasons to believe that \textit{of}, at least dialectally, is the disjunctive operator itself, see also \citet{boef2013}. By merging the Q-element with the finite complementiser, the [wh] feature is not checked off and hence the phrase remains active, allowing a second merger operation that involves the movement of the \textit{wh}-element. Essentially, both operators are specifiers regarding their relative positions to the head (neither of them is adjoined via head adjunction). This is naturally possible in a merge-based model, while it would be ruled out by strict X-bar rules. Note that the relative position of the \textit{wh}-element with respect to the disjunctive operator does not violate the Minimal Link Condition: the \textit{wh}-element moves to the closest specifier available, as there is no skipped position since \textit{of} and \textit{dat} are not heads of separate projections. 

Structures like in \figref{treewieofdat} are of relevance here since the proposed account can accommodate more complex combinations as well, without resorting to a rigid cartographic distinction between designated phrases. Indeed, in none of the cases showing double or triple combinations is a cartographic template necessary; moreover, as was argued in detail, the nature of the combinations seriously challenges the possibility of a pre-defined template and of the notion of separate designated projections.

The final question to be discussed here concerns the (non-)availability of verb movement to C in embedded interrogatives. As was discussed in connection with embedded declaratives, this option is not entirely ruled out in Modern Standard German, and examples from Old German also suggest that this may have been an option in embedded polar interrogatives as well (see \sectref{sec:3embeddedpolar}). As mentioned there, the key factor is the matrix verb, which may impose selectional restrictions on its complement clause: it may require a CP headed by \textit{dass} (as is the case with \textit{bezweifeln} `doubt'), but it may simply require a finite CP, which allows V2 patterns as well (as is the case with \textit{denken} `think').

Restrictions from the matrix clause can be observed in other dependent clause types that are not taken by a matrix predicate. Consider the following examples for German hypothetical comparatives:

\ea
\ea \gll Anna verhält sich (so), \textbf{als} \textbf{wäre} sie im Kindergarten. \label{alsv}\\
Anna behaves herself \phantom{(}so as be.\textsc{cond.3sg} she in.the nursery.school\\
\glt `Anna behaves as if she were at nursery school.'
\ex \gll Anna verhält sich (so), \textbf{als} \textbf{ob} sie im Kindergarten wäre. \label{alsobch3}\\
Anna behaves herself \phantom{(}so as if she in.the nursery.school be.\textsc{cond.3sg}\\
\glt `Anna behaves as if she were at nursery school.'
\z
\z

In this case, as far as lexicalisation of [fin] in C is concerned, verb movement in (\ref{alsv}) and the insertion of the complementiser \textit{ob} are equivalent options. As indicated, an optional degree-like element \textit{so} `so' can be inserted in the matrix clause, but this does not serve as a predicate in the way lexical verbs taking finite clauses do.

The optionality between complementiser-insertion and verb movement applies to conditionals as well, illustrated for English in (\ref{conditionals}):

\ea \label{conditionals}
\ea \textbf{If} water should leak out, check the tube connections.
\ex \textbf{Should} water leak out, check the tube connections.
\z
\z

Again, both strategies lexicalise C and check off the [fin] feature and in this sense they are equivalent. In other words, verb movement is not excluded from embedded clauses per se, but it is rather restricted by certain elements appearing or not appearing in the matrix clause. The ban on verb movement in embedded interrogatives can ultimately be related to selectional restrictions.

To conclude this section, it can be established that Doubly Filled COMP effects arise due to a lexicalisation requirement on C, which follows from the general syntactic paradigm in West Germanic. Essentially, the differences observed between standard varieties of West Germanic and dialects can be drawn back to lexical differences, in line with \citet{borer1984}. Importantly, Doubly Filled COMP patterns are not seen as exceptional in the proposed model but they are in fact consistent with the more general syntactic properties of the respective languages.

\section{Long movement} \label{sec:3long}
The last issue that I would like to examine briefly concerns long movement, since doubling is also relevant for this phenomenon. Consider first the following example:

\ea You said [\textbf{that} they saw the new students]. \label{thatdecl}
\z

In (\ref{thatdecl}), the embedded clause (bracketed) is a declarative clause (as selected by the matrix verb) and the matrix clause is also declarative. In canonical matrix questions, we have the configuration given in (\ref{thatint}):

\ea \textbf{Who} said [\textbf{that} they saw the new students]? \label{thatint}
\z

In this case, the \textit{wh}-element originates in the (interrogative) matrix clause: the embedded clause is not affected. The relevant derivation process (disregarding issues not relevant for our purposes here) is given in (\ref{derivation}) below:

\ea \label{derivation}
\ea {[}\textsubscript{vP} who said [\textsubscript{CP} that \textit{opaque}]]
\ex {[}\textsubscript{TP} who [\textsubscript{vP} who said [\textsubscript{CP} that \textit{opaque}]]]
\ex {[}\textsubscript{CP} who [\textsubscript{TP} who [\textsubscript{vP} who said [\textsubscript{CP} that \textit{opaque}]]]]
\ex {[}\textsubscript{CP} who [\textsubscript{TP} \sout{who} [\textsubscript{vP} \sout{who} said [\textsubscript{CP} that \textit{opaque}]]]] \label{derivationdel}
\z
\z

As indicated, the subject \textit{who} originates in the VP and moves (via TP) to the specifier of the CP. The lower copies are regularly eliminated at PF, as shown in (\ref{derivationdel}). The embedded clause plays no role in these cases, as no operation targets the edge of this clause. In line with the Phase Impenetrability Condition (\citealt{chomsky2000}), I assume that only the head (here: \textit{that}) and the specifier of the clause remain active after the clause has been spelt out, the rest of the clause being opaque.

Likewise, if the \textit{wh}-element originates in the embedded (interrogative) clause and moves to the left periphery of the same clause, this does not affect the matrix clause:

\ea You asked [\textbf{who} they saw]. \label{whointembedded}
\z

In (\ref{whointembedded}), the \textit{wh}-element originates as an object and undergoes regular operator movement; this does not affect the declarative main clause. The derivation is as follows:

\ea \label{derivationwhointembedded}
\ea {[}\textsubscript{TP} they saw who]
\ex {[}\textsubscript{CP} who [\textsubscript{TP} they saw who]]
\ex {[}\textsubscript{CP} who [\textsubscript{TP} they saw \sout{who}]]
\z
\z

The movement of the \textit{wh}-element in this case is triggered by a [wh] feature on the C head, which is checked off.

In long movement, the \textit{wh}-element originates in the embedded clause and ends up in a higher clause:\footnote{As discussed in \chapref{ch:2}, such extractions are subject to various constraints in English: the extraction of a subject \textit{wh}-element triggers the so-called Comp-trace or \textit{that}-trace effect, while the extraction of an object \textit{wh}-element is unproblematic. The following examples are two test sentences from \citet[557, ex. 1a and 3a]{sobin2002} that were used in a grammaticality judgement experiment:

\ea Who did you say that Mary saw last week? \label{testthatsubject}
\z

\ea Who did you say that saw Elvis last week? \label{testthatobject}
\z

For the sentence in (\ref{testthatobject}) reports that 64\% or the participants marked it as `good', 27\% as `maybe', and 9\% as `impossible'; this contrasts with 100\% `good' for the sentence in (\ref{testthatsubject}). This subject-object asymmetry is due to independent constraints, presumably related to processing, and is not directly relevant to the discussion on long movement here.}

\ea \textbf{Who} did you say [\textbf{that} they saw]? \label{long}
\z

According to the standard view (\citealt{chomsky1981}, see also the discussion in \citealt{fanselow2017}), \textit{wh}-elements move in a cyclic fashion in long-distance dependencies, so that the \textit{wh}-element moves first to the specifier of the embedded clause and subsequently to the specifier of the higher clause. The derivation for (\ref{long}) is shown in (\ref{longderivation}) below:

\ea \label{longderivation}
\ea {[}\textsubscript{TP} they saw who]
\ex {[}\textsubscript{CP} who that [\textsubscript{TP} they saw who]] \label{whothattp}
\ex {[}\textsubscript{CP} who that \textit{opaque}]
\ex {[}\textsubscript{vP} you said [\textsubscript{CP} who that \textit{opaque}]] 
\ex {[}\textsubscript{TP} you [\textsubscript{vP} you say [\textsubscript{CP} who that \textit{opaque}]]]
\ex {[}\textsubscript{CP} who did [\textsubscript{TP} you [\textsubscript{vP} you say [\textsubscript{CP} who that \textit{opaque}]]]]
\ex {[}\textsubscript{CP} who did [\textsubscript{TP} you [\textsubscript{vP} say [\textsubscript{CP} \sout{who} that \textit{opaque}]]]] 
\z
\z

In essence, the derivation is largely a logical combination of the derivations given in (\ref{derivation}) and (\ref{derivationwhointembedded}) above, though some remarks are in order here, especially regarding the intermediate landing site. First, note that long movement can occur in multiply embedded environments, as illustrated in (\ref{multiplethat}) below:

\ea \textbf{Who} do you think [\textbf{that} Peter said [\textbf{that} they saw]]? \label{multiplethat}
\z

\begin{sloppypar}
Second, the intermediate landing site does not constitute an interrogative clause. On the one hand, this is unproblematic since the \textit{wh}-element does not undergo feature-checking regarding [wh] in a lower clause so it can do so in the highest clause. On the other hand, the question arises what triggers the movement of the \textit{wh}-element to the intermediate landing site, if there is no triggering [wh] feature.
\end{sloppypar}

Third, related to this, notice the presence of \textit{that} in the embedded clause. This is not obligatory, so that the zero complementiser counterpart of (\ref{long}) is also possible, as illustrated in (\ref{longzero}):\footnote{Note again that overt \textit{that} and the zero complementiser are not always interchangeable: in cases like (\ref{long}) and (\ref{longzero}), the extracted \textit{wh}-element is an object. If the extracted \textit{wh}-element is a subject, this leads to the so-called Comp-trace effect or \textit{that}-trace effect, as discussed in \chapref{ch:2}.}

\ea \textbf{Who} did you say [they saw]? \label{longzero}
\z

As discussed in this chapter, the complementiser \textit{that} is not allowed in embedded interrogatives in English in the standard variety, so that its presence in sentences like (\ref{long}), which is undoubtedly part of the standard, indicates yet more clearly that the embedded clause cannot be interrogative. Note also that the matrix verb is \textit{say}, which selects declarative, not interrogative complements. Given the absence of a specific clause-typing feature normally triggering the movement of an operator, it seems reasonable to assume that the movement of the \textit{wh}-element to the intermediate landing site is most probably triggered by an [edge] feature (cf. \citealt{georgi2013} on [edge] features as trigger sin intermediate [Spec,TP] landing sites). Note that, except for echo questions (which I take to be instances of focus, in line with \citealt{boskovic2002} and \citealt{artstein2002}), \textit{wh}-elements need to be fronted in Germanic; in other words, the [wh] feature, as an interrogative clause-typing feature, normally requires the \textit{wh}-element to move to a relevant [Spec,CP] position or to be preceded by such a \textit{wh}-element in the same clause. Note that the same cannot happen in embedded declaratives, as that would cause clashes at LF: the \textit{wh}-element takes scope over the clause, yet a declarative clause is assumed to be a complete proposition. Since the \textit{wh}-element is available for further operations in the [Spec,CP] position, its movement triggered by the [wh] feature on the highest C head is regularly triggered. Failing to insert the [edge] feature on the intermediate C head causes problem for the interfaces, as the \textit{wh}-element would fail to move up to the highest clause, leaving the [u-wh] feature unchecked.

It is evident that there is an intermediate step in the derivation, see (\ref{whothattp}), which constitutes a Doubly Filled COMP pattern in the sense that both the specifier and the head of the same CP are filled by overt (clause-typing) material. However, note that in this case the clause is not typed as interrogative: consequently, it does not require the complementiser to be underspecified for [wh] and thus crucially differs from canonical Doubly Filled COMP patterns.

Let us now turn to German. German also allows long-distance \textit{wh}-movement, as illustrated in (\ref{germanlong}) below:

\ea \gll \textbf{Wen} denkst du, [\textbf{dass} sie liebt]? \label{germanlong}\\
who.\textsc{acc} think.\textsc{2sg} you \phantom{[}that she loves\\
\glt `Who do you think that she loves?'
\z

Just like in English, it is possible to extract across multiple embedded clauses, as in (\ref{multiplegerman}) below (\citealt[25]{fanselow2017}):\footnote{In German, long extraction is not banned for subjects, see \citet{fanselow2017}, \citet{brandnerbucheliberger2018}, \citet{weiss2016}, though \citet{weiss2016} notes that the acceptance of long extraction seems to be considerably higher for objects than for subjects in Hessian, and similar differences are also detected by \citet[36]{brandnerbucheliberger2018} for Alemannic.}

\ea \gll \textbf{Wen} denkst du [\textbf{dass} sie glaubt [\textbf{dass} Fritz meint [\textbf{dass} sie liebt]]]? \label{multiplegerman}\\
who.\textsc{acc} think.\textsc{2sg} you \phantom{[}that she believes \phantom{[}that Fritz means that she loves\\
\glt `Who do you think she believes that Fritz means that she loves?'
\z

While such constructions are possible across German, \citet{fanselowweskott2010} observed not only that these constructions seem to be more acceptable in dialects than in the standard but also that there seems to be a North-South divide (when examining regional standards), such that the construction is more widespread in Southern areas, especially in Bavarian. The acceptance of such extraction patterns is confirmed to be high (around 74\%, with 45\% mentioning other options as possible alternatives) in Alemannic by \citet[34]{brandnerbucheliberger2018}, it appears to be less widespread in Hessian, as it is not the preferred option (it amounts to only 33\% for objects, \citealt{weiss2016}). Note that while the Bavarian and Alemannic areas strongly employ Doubly Filled COMP in embedded interrogatives (see \citealt{bayerbrandner2008}, among others), the same doubling patterns constitute a minority option in Hessian (\citealt{weiss2016doubly}).

This seems to suggest that there is at least some correlation between Doubly Filled COMP patterns and long movement, though this is not categorical, as the possibility of long movement does not imply the availability of Doubly Filled COMP patterns (or vice versa). It is, however, possible that in dialects that allow \textit{dass} to be underspecified for [wh], combining [edge] with \textit{dass} is more readily an option, since the filling of the specifier position together with \textit{dass} is in fact a regular pattern in these varieties, so that the intermediate structure containing the sequence \textit{wen dass} conforms to a more generally available syntactic configuration. Since this configuration does not ultimately surface, due to the elimination of lower copies, this apparent correlation again suggests that constraints related to the presence or absence of Doubly Filled COMP patterns are not governed by surface filters but rather follow from the underlying syntactic features of the respective elements.

\section{Summary} \label{section:3summary}
Using the framework established in \chapref{ch:2}, this chapter analysed the left periphery of interrogative clauses, especially embedded interrogatives clauses. In particular, doubling effects that go against the so-called ``Doubly Filled COMP Filter'' were discussed in detail. It was shown that variation in Germanic cannot be successfully described (and especially explained) by a surface filter. It was shown that the properties underlying such combinations stem from  the necessity of overtly realising the operator in constituent questions and from the preference of lexicalising a finite C in Germanic. In polar questions, doubling effects are also attested, yet they are far less common, which is expected since the polar operator is recoverable. In some varieties, as is the case in certain Dutch dialects, triple combinations are also possible: these also do not necessarily require multiple CP projections, as the minimalist model allows multiple specifiers. The availability of such patterns is constrained by semantics. Finally, it was also shown that the proposed account is compatible with basic observations concerning long-distance \textit{wh}-movement. Since the present chapter was restricted to embedded interrogatives, the question that arises at this point is whether the analysis can be carried over to other clause types. In \chapref{ch:4}, I will turn to the analysis of relative clauses, which, as was pointed out in connection with the Doubly Filled COMP Filter, have often been treated in a parallel fashion, and \chapref{ch:5} will address embedded degree clauses, which differ in terms of doubling. I will return to basic questions concerning clausal ellipsis in embedded interrogatives and the relevance of information structure in \chapref{ch:6}.
