\chapter{Ellipsis and the role of information structure in left peripheries} \label{ch:6}
\section{Introduction} \label{sec:6introduction}
%\chaptermark{Ellipsis and the role of information structure}
So far I have mostly examined finite, non-elliptical clauses in this book, concentrating on clause-typing elements in the CP-periphery. As was discussed in \chapref{ch:2}, functional left peripheries may host elements associated with special information-structural roles (topics, foci), and functional left peripheries are not restricted to the high CP-periphery but may appear clause-internally as well. In addition, certain ellipsis processes, such as sluicing, are known to be associated with functional projections located at the left periphery. Naturally, the discussion of either issue (information structure and clausal ellipsis) would require more investigation than can possible be carried out within a single chapter, and therefore I will restrict myself to the discussion of some selected issues that bear immediate relevance to the general theory put forward in this book. I will chiefly consider reduced comparative constructions but elliptical interrogatives will also be discussed. The ultimate aim is to show that the proposed model can cast light upon interesting phenomena involving focalisation and clausal ellipsis.

This chapter is structured as follows. Starting with the theoretical foundations, \sectref{sec:6information} examines the interaction between information structure and leftward movement, while \sectref{sec:6ellipsis} examines the relationship between leftward movement and ellipsis. Section \ref{sec:6lower} discusses both of these issues in the context of lower peripheries. Section \ref{sec:6sluicing} extends the insights from lower peripheries to sluicing patterns and argues that the presence or absence of tense has a crucial effect on the reconstructed clause. This line of thinking is applied to a study on ellipsis in comparatives, presented in \sectref{sec:6comparatives}.

\section{Information structure and leftward movement} \label{sec:6information}
As was discussed in \chapref{ch:2} in detail in connection with the account of \citet{rizzi1997}, certain constituents may undergo topicalisation or focalisation involving movement to the left periphery of the clause. Recall the following examples taken from \citet[285, ex. 1 and 2]{rizzi1997}:

\ea \label{englishrizzich6}
\ea {[}Your book]\textsubscript{i}, you should give \textit{t}\textsubscript{i} to Paul (not to Bill). \label{topiccommentch6}
\ex {[}YOUR BOOK]\textsubscript{i} you should give \textit{t}\textsubscript{i} to Paul (not mine). \label{focuspresuppch6}
\z
\z

The construction in (\ref{topiccommentch6}) illustrates topicalisation, and the one in (\ref{focuspresuppch6}) focalisation. Apart from interpretive differences, they crucially differ in their intonation patterns: a topic is separated by a so-called ``comma intonation'' from the remaining part of the clause (the comment), while a focus bears focal stress and is thus prominent with respect to presupposed information (see \citealt[258]{rizzi1997}).

The cartographic model proposed by \citet{rizzi1997}, adopted by others such as \citet{poletto2006}, proposes that leftward movement in these cases targets designated left-peripheral positions: TopP and FocP. Movement is driven by specific features making reference to information-structural properties: this operator-like feature agrees with the functional head (Top or Foc). In essence, this kind of movement is supposed to be similar to ordinary operator movement involving \textit{wh}-operators or relative operators. As pointed out already in \chapref{ch:2}, such an assumption is problematic because while [wh] and [rel] features are lexically determined, [topic] and [focus] features obviously are not. Taking the examples in (\ref{englishrizzich6}) above, in both cases the entire phrase \textit{your book} is topicalised or focussed, and the phrase as such, being compositional, is not part of the lexicon. This indicates that features like [topic] and [focus] would have to be added during the derivation (cf. \citealt{selkirk1984} in focus features); this is certainly possible in principle, yet the addition of discourse-relevant features (relevant primarily for the interfaces) in the core syntactic component requires additional assumptions.\footnote{To avoid this problem, one could locate such features in the postsyntactic component (notice that the addition of such features has no bearing on truth conditions, though they certainly affect interpretation, as, for instance, in terms of exhaustivity). Yet by doing so, the motivation for designated syntactic projections (in narrow syntax) disappears. Regarding postsyntactic operations related to information structure, future research will have to clarify to what extent the syntactic structure generated by core syntax is relevant and to what extent purely prosodic features matter. In the present investigation, I will restrict myself to issues immediately relevant to the model proposed here on clause typing.} On the other hand, if one were to assume that a lexical element like \textit{Mary} can be equipped with information-structural features in the lexicon (contrary to generally accepted views about the lexicon and lexical features, cf. \citealt{neelemanszendroei2004} and \citealt{dendikken2006}), this would leave us with various lexical entries for \textit{Mary}: a neutral entry (not specified for any information-structural category), a focussed one, a topicalised one, not to mention possible fine-grained categories such as contrastive topic or aboutness topic.

As pointed out by \citet{fanselowlenertova2011}, the cartographic approach is problematic from a theoretical perspective as well. In merge-based minimalist approaches, as spelt out by \citet{chomsky2008}, syntax should not make direct reference to information-structural notions;\footnote{The problem is obvious in the case of cartographic approaches, where such features are assumed to project a matching phrase on the left periphery. Somewhat less problematic are approaches such as that of \citet[163--176]{miyagawa2017}, in which a [focus] feature is assumed to be available on C, without resorting to an additional projection. Another interesting alternative is employed by \citet[40--45]{biberauervankemenade2011}, who posit an extra [Person] feature for discourse-old subjects in Old English: this effectively evades the problem of introducing additional features, but it raises the question whether such features are independently motivated.} including [topic] and [focus] features in the derivation violates the inclusiveness condition (\citealt{chomsky1995}). According to \citet{fanselowlenertova2011}, movement to the left periphery is generally triggered by an unspecified [edge] feature (in the sense of \citealt{chomsky2008}); whether an element receives some accent depends on other factors, including linearisation, but syntax does not include accentuation features directly. The proposed model can account for the movement of pragmatically unmarked constituents to the left periphery, essentially in the way the ``formal fronting'' of \citet{frey2004} works.\footnote{Note that in the system of \citet{frey2004, frey2005, frey2010}, formal movement has no semantic or pragmatic effect, while other leftward movement operations targeting designated topic (TopP) and contrast (ContrP/KontrP) projections do. In this respect, this system takes over some properties of the cartographic model, leading to the problems mentioned in connection with the cartographic approach in general. A detailed discussion of these issues would fall outside the scope of the present investigation; see e.g. \citet{wierzba2017diss} for discussion.} In this way, the following two structures are similar in their syntax in that the left periphery involves a simple CP rather than specific projections:

\ea
\ea \gll \textbf{Wen} \textbf{hat} deine Mutter eingeladen? \label{germanwh}\\
who.\textsc{acc} has your.\textsc{f} mother invited.\textsc{ptcp}\\
\glt `Who has your mother invited?'
\ex \gll \textbf{Den} \textbf{Schuldirektor} hat meine Mutter eingeladen. \label{germanv2}\\
the.\textsc{m.acc} schoolmaster has your.\textsc{f} mother invited.\textsc{ptcp}\\
\glt `My mother has invited the schoolmaster.'
\z
\z

The example in (\ref{germanwh}) shows canonical \textit{wh}-movement, involving a [wh] feature in syntax: \textit{wh}-movement is linked to the [wh] criterion (\citealt[172]{fanselowlenertova2011}, citing \citealt{rizzi1991}). By contrast, (\ref{germanv2}) involves no specific feature to trigger the movement of the fronted DP, especially as constituents appearing in the first position cannot be associated with a uniform information-structural notion and there is no agreement between the specifier and the head in terms of some information-structural feature (\citealt[172]{fanselowlenertova2011}, contrary to \citealt{grewendorf1980} and \citealt{rizzi1997}). In (\ref{germanwh}), the \textit{wh}-element moves to the specifier of the CP, while C is filled by the verb (see also the discussion in \chapref{ch:3}); verb movement takes place in (\ref{germanv2}) as well, whereby the fronted XP is located in the specifier (an assumption going back to \citealt{thiersch1978diss}). German is not unique in this respect: \citet{fanselowlenertova2011} argue that Czech has the same structure in these cases (following the observations made by \citealt{toman1999}, \citealt{lenertova2004}, and \citealt{meyer2004}). Fronted elements like \textit{den Schuldirektor} in (\ref{germanv2}) can be associated with various information-structural notions such as topic and focus; in turn, topics and foci can occur in non-fronted positions. This is illustrated by the following examples taken from \citet[172, ex. 6c and 6d]{fanselowlenertova2011}, both answering the question \textit{What happened?}:

\ea
\ea \gll \textbf{Eine} \textbf{LAWINE} haben wir gesehen!\\
a.\textsc{f.acc} avalanche have.\textsc{1pl} we seen\\
\glt `We saw an AVALANCHE!'
\ex \gll Wir haben \textbf{eine} \textbf{LAWINE} gesehen!\\
we have.\textsc{1pl} a.\textsc{f.acc} avalanche seen\\
\glt `We saw an AVALANCHE!'
\z
\z

This kind of optionality obviously contrasts with the behaviour of ordinary \textit{wh}-movement (and relative operator movement) in German, which always targets the CP-domain. Note also that this is true the other way round as well: a \textit{wh}-element moving to [Spec,CP] is interpreted as interrogative. Consider the following examples from German:

\ea
\ea \gll \textbf{Was} hast du gefunden? \label{wasfronted}\\
what have.\textsc{2sg} you found.\textsc{ptcp}\\
\glt `What have you found?'
\ex \gll Schau, ich habe \textbf{was} gefunden. \label{wasbase}\\
look.\textsc{imp.2sg} I have.\textsc{1sg} what found.\textsc{ptcp}\\
\glt `Look, I have found something.'
\z
\z

Certain \textit{wh}-words like \textit{was} in German can be interpreted as indefinite pronouns if they feature in their base positions, as in (\ref{wasbase}), where \textit{was} has the interpretation `something'. This interpretation is not available if the \textit{wh}-element is fronted, as in (\ref{wasfronted}).

Finally, as pointed out by \citet[173]{fanselowlenertova2011}, there are certain fronted elements in the German CP (occupying the ``first position'') that clearly do not correspond to information-structural categories such as topic and focus. Consider (\citealt[173, ex. 7a]{fanselowlenertova2011}):

\ea \gll \textbf{Wahrscheinlich} hat ein Kind einen Hasen gefangen.\\
probably has a.\textsc{n.nom} child a.\textsc{m.acc} rabbit caught.\textsc{ptcp}\\
\glt `A child has probably caught a rabbit.'
\z

In this case, the adverb \textit{wahrscheinlich} `probably' is a sentential adverb that evidently lacks a discourse function such as topic or focus.

Partial fronting, discussed extensively by \citet{wierzba2017diss}, constitutes another problem. The phenomenon is illustrated by the following examples, both answering a question like \textit{What did Maria do in the afternoon?} (\citealt[1, ex. 1]{wierzba2017diss}):

\ea
\ea \gll \textbf{Ein} \textbf{Buch} hat sie [\sout{ein} \sout{Buch} gelesen]. \label{focus}\\
a.\textsc{n.acc} book has she \phantom{[}a\textsc{n.acc} book read.\textsc{ptcp}\\
\glt `She read a book.'
\ex \gll \textbf{Ein} \textbf{Buch} hat sie jedenfalls nicht [\sout{ein} \sout{Buch} gelesen]. \label{topic}\\
a.\textsc{n.acc} book has she anyway not \phantom{[}a\textsc{n.acc} book read.\textsc{ptcp}\\
\glt `As for reading a book, that's not what she did.'
\z
\z

In both cases, only a direct object is fronted to the left periphery. Nevertheless, as pointed out by \citet[1]{wierzba2017diss}, the whole VP is interpreted as the focus in (\ref{focus}) above (also observed already by e.g. \citealt{hoehle1982} and \citealt{krifka1998}, tested empirically by e.g. \citealt{ferydrenhaus2008}) and as a contrastive topic in (\ref{topic}) above (also observed already by \citealt{buering1997} and \citealt{jacobs1997}, tested empirically by \citealt{wierzba2011}). This indicates that the landing site of the constituent as such does not define its information-structural status. Rather, the particular elements have specific prosodic properties.

In this vein, I follow \citet{fanselowlenertova2011} in assuming that information-structural properties are primarily related to prosody and that the syntax--pros\-o\-dy mapping does not need to make reference to syntactic features present in designated left-peripheral projections.\footnote{In languages like English or German, the prosodic properties of contrastive topics and foci are evidently marked by the specific stress and intonation patterns associated with these elements. In other languages, such properties are associated more clearly with specific syntactic positions (e.g. relative to the verb) and/or to the presence of specific morphemes (as the ``focus markers'' in Chadic, Kwa and Gur languages, see \citealt[687]{fery2013nllt}). Regarding focus in particular, \citet{fery2013nllt} argues that it can be best understood as alignment, in that the focused element is prosodically aligned with the right or left edge of a prosodic domain. Languages have various means to achieve this: alignment can be marked by, for instance, pitch accent, morpheme insertion, or syntactic movement. In this sense, the prosodic approach put forward by \citet{fanselowlenertova2011} is not specific to Germanic (and Slavic). In what follows I will concentrate on West Germanic only, since the discussion of different marking strategies would go beyond the scope of the present investigation.} This is also in line with the general approach put forward in the present thesis, namely that left-peripheral projections do not conform to a cartographic template -- which does not in any way mean that there would be no ordering restrictions, as semantic and prosodic constraints still apply. Regarding information-structure related left-peripheral movement, I will henceforth assume that it is triggered by an unspecific [edge] feature in the syntactic component (in the way discussed for V2 clauses in German in \chapref{ch:3}). The relevant configuration is schematically represented in \figref{treefp} (YP representing the complement of F, e.g. TP).

\begin{figure} 
\caption{Movement to FP} \label{treefp}
\begin{forest} baseline, qtree
[FP
	[XP\textsubscript{{[}edge{]}}
		[\phantom{xxx},roof]
	]
	[F$'$
		[F]
		[YP]
	]
]
\end{forest}
\end{figure}

In this configuration, FP stands for functional projection (comprising, for instance, the CP): the relevant head does not trigger the movement of an argument and there is no specific information-structural feature involved either. In certain configurations, such as German V2 clauses, the movement of some XP is necessary for independent reasons in the syntax (see \chapref{ch:3}): this, however, does not impose any information-structural constraints. In other cases, such as in the English in (\ref{englishrizzich6}) above, there are no such independent reasons in the syntactic component: that is, the FP would not be generated otherwise (unlike the German CP layer to encode finiteness); still, there are no information-structural features present either as the precise interpretation is defined by phonological constraints. The specific constraints related to prosody will not be discussed here, as there is ample literature on this topic and it is not the main point to be examined in this book.\footnote{See, for instance, \citet{ferydrenhaus2008}, \citet{fanselowlenertova2011} \citet{wierzba2017diss} for prosodic accounts; see also \citet{fanselow2016} and \citet{wierzbafanselow2020} for an overview.}

\section{Ellipsis and leftward movement} \label{sec:6ellipsis}
As mentioned at the beginning of this chapter, clausal ellipsis is also relevant for functional left peripheries. The prototypical case for this is sluicing, demonstrated in (\ref{sluicech6}) below:

\ea Someone phoned grandma but I don't remember \textbf{WHO} \sout{phoned grandma}. \label{sluicech6}
\z

In this case, the elliptical clause is embedded in a clause conjoined with another main clause: this first main clause (\textit{someone phoned grandma}) contains the antecedents for the elided elements in the elliptical clause. The elliptical clause contains only a single remnant, the subject \textit{who}, which bears main stress as it contains non-given information (see \chapref{ch:3} and \chapref{ch:4}). Ellipsis is licensed because all elided information is recoverable. Recoverability is more than merely \textsc{given}ness; consider the following example:

\ea[*]{Someone phoned grandma; it was a sunny afternoon and Peter fed the cat but I don't remember who \sout{phoned grandma}.}
\z

In this case, the elided part is actually \textsc{given} in the discourse; nevertheless, it is not recoverable and as such it does not constitute an appropriate antecedent for the elided string. The reason is that there are two intervening clauses containing new information and the original information is not salient enough to serve as an antecedent.

For reasons of this kind, \citet[25--36]{merchant2001} proposes that elided elements should be e-\textsc{given} (ellipsis-\textsc{given}). Apart from the salience condition, this also implies mutual entailment between the elided part and its antecedent (\citealt[26, ex. 42]{merchant2001}):

\begin{exe}
\ex e-\textsc{given}ness\\
An expression E counts as e-\textsc{given} iff E has a salient antecedent A and, modulo $\exists$-type shifting, \label{egivenness}
\begin{xlist} 
\exi{(i)} A entails F-clo(E), and
\exi{(ii)} E entails F-clo(A).
\end{xlist}
\end{exe}

Regarding the actual implementation of ellipsis in grammar, \citet[55--61]{merchant2001} and \citet[670--673]{merchant2004} argue that there is an [E] feature responsible for ellipsis. This feature is assumed to be merged with a particular functional head (such as C) and the complement of this head is elided. Since this feature contains information not only relevant in narrow syntax but also for both interfaces, it has not only syntactic but also phonological and semantic properties (\citealt[670--673]{merchant2004}). The semantics is essentially the same as the e-\textsc{given}ness condition mentioned above, see (\ref{egivenness}). The phonological condition amounts to saying that the complement of the functional head will be realised as phonologically zero if it follows the [E] feature. Finally, the syntactic condition is that the [E] feature is specified as having either an uninterpretable [wh] or an uninterpretable [Q] feature, thus [u:wh] or [u:Q], ensuring that it occurs only in (embedded) questions. As shown by \citet{vancraenenbroeckliptak2006} for Hungarian and \citet{hoytteodorescu2012} for Romanian, this particular syntactic condition is highly unsatisfactory as many languages allow canonical ellipsis processes such as sluicing also from non-interrogative projections, including relative clauses and projections hosting foci. In fact, the analysis proposed by \citet{merchant2004} for fragments also suggests that this feature specification does not always hold.

Namely, clausal ellipsis can not only take the form of sluicing, as in (\ref{sluicech6}), but it can also be observed in fragments. Consider the following example:

\begin{exe}
\ex
\begin{xlist} 
\exi{A:} Who phoned grandma?
\exi{B:} \textbf{Liz} \sout{phoned grandma}.
\end{xlist}
\end{exe}

In this case, the remnant (\textit{Liz}) is the subject and the rest of the clause is elided. Since in English the subject DP in declarative clauses is located in [Spec,TP] and not in [Spec,CP], the ellipsis mechanism established for sluicing does not automatically carry over. While it may at first be tempting to assume that T can also host the [E] feature, just like C can, examples like (\ref{fragment}) clearly show that this is not a viable option:

\begin{exe}
\ex \label{fragment}
\begin{xlist} 
\exi{A:} Who did Liz phone?
\exi{B:} \sout{Liz phoned} \textbf{grandma}.
\end{xlist}
\end{exe}

In this case, the underlying structure of the clause does not match the direction of sluicing: the remnant is located on the right, while the ellipsis site seems to be on the left.

\citet{merchant2004} proposes that fragments involve a functional projection, FP, which hosts the [E] feature in its head: the remnant moves up to the specifier of this projection and the complement is elided.\footnote{Note that this also implies that the particular leftward movement of the remnant is triggered in elliptical environments (in the presence of the [E] feature) but not otherwise: English is a language that does not have focus fronting otherwise. Taking the example in (\ref{fragment}), this leads to the following contrast:

\ea[*]{Grandma\textsubscript{i} Liz phoned t\textsubscript{i}. \label{grandmafoc}}
\ex[]{Grandma\textsubscript{i} \sout{Liz phoned t\textsubscript{i}}. \label{grandmafocelided}}
\z

As indicated, (\ref{grandmafoc}), involving overt focus movement, is ungrammatical, while the elliptical version in (\ref{grandmafocelided}) is well-formed. This suggests that grammatical elliptical constructions are not isomorphic to their grammatical non-elided counterparts. I will return to this question in \sectref{sec:6sluicing}.} This analysis is able to unify the template for sluicing and fragments (and VP-ellipsis). The structure is given in \figref{treefpschematic} below (following \citealt[675]{merchant2004}; YP simply represents the complement of F, which is TP in \citealt{merchant2004}).

\begin{figure} 
\caption{Ellipsis and the FP} \label{treefpschematic}
\begin{forest} baseline, qtree
[FP
	[XP
		[\phantom{xxx},roof]
	]
	[F$'$
		[F\\{[}E{]}]
		[YP]
	]
]
\end{forest}
\end{figure}

Comparing now this structure to the one in \figref{treefp}, a striking similarity arises. In both cases, leftward movement targets an unspecific left-peripheral position (FP), whose primary role is to establish a configuration that is congruent with the requirements of the interfaces, specifically with those of PF. There is no specific clause-typing feature involved in the leftward movement of the element landing in the specifier in either case; still, movement takes place in narrow syntax as it has effects on PF. In both cases, the FP is generated in such a way that its head requires the movement of a constituent into the specifier; this follows from minimalist assumptions inasmuch as extra projections are not generated per se but they host phonologically visible material and/or are semantically motivated.\footnote{In addition, it appears that the unspecific projection has no label, which differs from the setup of ordinary projections, in which a head projects into a phrase. I will further elaborate in this chapter on the nature of heads containing an [E] feature, suggesting that the FP is by no means headless. \citet{emonds2004, emonds2007, emonds2012} proposes that certain discourse-related projections, which he calls ``Discourse Shells'', may be without a label; in his system, this is related to Main Clause Phenomena. Crucially, the FPs related to ellipsis are available also in embedded clauses.}

The issue of the [edge] feature was addressed in \chapref{ch:3} already: the [edge] feature can be inserted into the derivation relatively freely (see the Edge Feature Condition of \citealt[109]{chomsky2000}). The [E] feature seems to be similar in the sense that it is not a lexically specified feature and it is inserted on top of the basic syntactic derivation. One question regarding this is how it potentially interacts with other elements: while the [edge] feature is assumed to be checked off by moving a relevant phrase to the specifier and being thus not further relevant for the derivation, the [E] feature cannot be eliminated by the movement of the XP in the specifier, as the [E] feature carries information relevant for PF. Still, the movement of the specifier element is obviously triggered.

One way to look at this is to say that the [E] feature is a phonologically zero element that is lexically specified as [edge]; in this sense, the [E] feature is strictly speaking not a (syntactic) feature attached to other lexical items (for instance, in the way the [edge] feature can appear on any fronted phrase) but a lexical item. This would automatically give us the observed distribution, namely that whenever the [E] feature is inserted, there is movement to the specifier. Treating the [E] feature as a lexical item (rather than a syntactic feature) has the advantage that all the phonological and semantic information associated with the presence of this feature can be specified as lexical information. An obvious point of objection is that the [E] feature in itself does not carry lexical information but it merely makes reference to that of others. This kind of property is, however, familiar from placeholder elements:\footnote{In addition, it should be noted that the notion of lexical information as a criterion is not without problems either. Functional elements (such as complementisers, determiners, negation heads) can be treated as the realisations of feature bundles (and in some cases single features, for instance with negation heads projecting a NegP), as is also proposed in the present book.}

\ea
\ea \textbf{There} is a box on the table.
\ex \textbf{There} are boxes on the table.
\z
\z

The element \textit{there} is inserted into the [Spec,TP] position and as such it holds the place of the canonical subject; the logical subject itself comes after the copula in both cases. As can be seen, \textit{there} does not show agreement with the verb: agreement is governed by the logical subject (at least as far as the standard variety is concerned; in many non-standard varieties, \textit{there} is not treated as a dummy element proper).

In essence, treating the [E] feature as a lexical item projecting a phrase of its own has the obvious consequence that the entire phrase inherits the properties of the head: phrases are endocentric in Bare Phrase Structure. In other words, what is referred to as FP is in fact an ellipsis phrase, meaning that the phrase is not merely some unspecific FP hosting [E] but a phrase generated by [E].\footnote{For the purposes of the present investigation, I will continue using the FP label in the tree diagrams, in line with the original proposal of \citet{merchant2001}. Note that the notion “ellipsis phrase” is sometimes used in the literature for the elided string (see, for instance, \citealt{hardtromero2004}): this differs from the proposal here, as the ellipsis phrase equivalent to the FP contains not only the elided string but also the remnant.}

The question arises what eventually constrains the insertion of such an element. I assume that it lies primarily in information structure. The complement of the F head is eliminated: as such, it must be recoverable (e-\textsc{given}) and it cannot contain any contrastive elements. On the other hand, the [E] feature requires the element in the specifier to bear stress; the resulting configuration is well-formed and discourse-congruent only if main stress on the given element is justified by its information-structural status. Consider:

\ea Mary painted the picture. \label{marypicturefull}
\z

Uttered out of the blue (as an all-new sentence), this sentence has the main stress on the object (\textit{the picture}). Now suppose there is a question asking about the agent:

\ea Who painted the picture?
\z

Obviously, (\ref{marypicturefull}) would be a congruent answer, with the modification that the stress should be shifted to the subject (\textit{Mary}), as this is the element that provides new information; the rest of the clause can be destressed. It is, however, also possibly to give an elliptical answer, as shown in (\ref{marygramamtical}) below:

\begin{exe}
\ex \label{marygramamtical}
\begin{xlist} 
\exi{A:} Who painted the picture?
\exi{B:} {[}\textsubscript{FP} Mary \sout{[\textsubscript{TP} \textit{t} painted the picture]}].
\end{xlist}
\end{exe}

In this case, \textit{Mary} is located in [Spec,FP] and the rest of the clause is eliminated. The complement TP is e-\textsc{given} and non-contrastive; \textit{Mary} is supposed to bear main stress anyway, as discussed above in connection with the full structure. In other words, (\ref{marygramamtical}) is congruent because it satisfies basic requirements concerning information structure.

The state of affairs is very different if another DP is fronted in the same context:

\begin{exe}
\ex \label{maryungramamtical}
\begin{xlist} 
\exi{A:}[]{Who painted the picture?}
\exi{B:}[*]{{[}\textsubscript{FP} The picture \sout{[\textsubscript{TP} Mary painted \textit{t}]}].}
\end{xlist}
\end{exe}

Strictly speaking, the answer is not syntactically ill-formed: the object may as well be attracted to [Spec,FP], as neither the [edge] feature nor the [E] feature imposes any restriction on this. However, (\ref{maryungramamtical}) is not congruent: the elided TP contains non-recoverable information and the element in [Spec,FP] is non-contrastive and not supposed to bear main stress. The only congruent interpretation of the utterance in B reconstructs a different sentence, which is nonsensical:

\ea[\#]{The picture painted the picture.}
\z

In a different context, the utterance in B is obviously congruent:

\begin{exe}
\ex 
\begin{xlist} 
\exi{A:} What did Mary paint?
\exi{B:} {[}\textsubscript{FP} The picture \sout{[\textsubscript{TP} Mary painted \textit{t}]}].
\end{xlist}
\end{exe}

In other words, what actually constrains the insertion of the [E] feature is not a rule in narrow syntax but a requirement set by the interfaces: the results must be prosodically well-formed and congruent with the information-structural properties of the utterance, as set by the discourse. This is similar to what can be said about the prosodic marking of elements in general (see \sectref{sec:6information}) and it thus does not require additional assumptions in the grammar.

\section{Lower peripheries} \label{sec:6lower}\largerpage
As mentioned in \chapref{ch:2}, there are languages that have functional left peripheries lower than the CP-domain proper, as shown by \citet{poletto2006} for Old Italian. The existence of this lower functional periphery (the vP-periphery) is closely tied to the notion of focus. As discussed briefly in \chapref{ch:2}, Hungarian is similar in this respect: the canonical focus position is lower then the CP-domain. This lower functional domain may host clause-type markers as well, such as the interrogative marker -\textit{e} that appears in polar questions. Moreover, as shown by \citet{liptakzimmermann2007}, a Hungarian clause may host a \textit{wh}-element clause-internally (in the FP) and a relative operator in the CP, and the \textit{wh}-operator can be extracted without triggering an island violation effect, indicating that the CP is not a landing site for the \textit{wh}-element. All this provides additional support for the idea that left peripheries proper are not tied to the CP-domain per se.

The idea that the FP is a focus projection (see \citealt{vancraenenbroeckliptak2008}) goes back to the observation that focussed elements normally occupy a preverbal position in the language as well (see, for instance, \citealt{ekiss2002}). There are two problematic points here, however. First, there are instances of polar interrogatives where there is evidently no focussed XP undergoing leftward movement. Second, there are certain non-standard patterns that indicate that the FP is iterable (see \citealt{bacskaiatkari2018kenesei} for a detailed analysis, using the original data of \citealt{kenesei1994}): designated focus phrases do not seem to be iterable otherwise (see also \citealt{rizzi1997}). The basic patterns are illustrated in (\ref{fphung}) below:

\ea \label{fphung}
\ea \gll Azt kérdeztem, \textbf{(hogy)} (tegnap) \textbf{ki} hívta fel Marit. \label{kiverb}\\
that.\textsc{acc} asked.\textsc{1sg} \phantom{\textbf{(}}that \phantom{(}yesterday who called.\textsc{3sg} up Mary.\textsc{acc}\\
\glt `I asked who called Mary yesterday'
\ex \gll Azt kérdeztem, \textbf{(hogy)} (tegnap) Péter felhívta\textbf{-e} Marit. \label{felverbe}\\
that.\textsc{acc} asked.\textsc{1sg} \phantom{\textbf{(}}that \phantom{(}yesterday Peter up.called.\textsc{3sg}-Q Mary.\textsc{acc}\\
\glt `I asked if Peter called Mary yesterday.'
\ex \gll Azt kérdeztem, \textbf{(hogy)} (tegnap) Péter hívta\textbf{-e} fel Marit. \label{verbe}\\
that.\textsc{acc} asked.\textsc{1sg} \phantom{\textbf{(}}that \phantom{(}yesterday Peter called.\textsc{3sg}-Q up Mary.\textsc{acc}\\
\glt `I asked if it was Peter who called Mary yesterday.'
\z
\z

In (\ref{verbe}), the verb is adjoined to the interrogative head -\textit{e} and the specifier of the FP hosts a focussed subject (the DP \textit{Péter}); adjunction happens from the left, resulting in an inverted word order (see the Linear Correspondence Axiom of \citealt{kayne1994} and the Mirror Principle of \citealt{baker1985, baker1988}). As indicated by (\ref{kiverb}), \textit{wh}-operators occupy the same preverbal positions; this is not surprising, as the \textit{wh}-element corresponds to the focussed element in question--answer sequences. In (\ref{felverbe}), however, there is no focussed element proper, yet it is evident that the verb left-adjoins to -\textit{e} just as in (\ref{verbe}). In this case, the element in [Spec,FP] is the verbal particle \textit{fel} `up'. One might wonder why the verbal particle moves up at all. It seems that this element is relevant in terms of polarity marking in yes-no questions, as evidenced by the fact that it can appear instead of \textit{igen} `yes' as a positive counterpart to the negative \textit{nem} `not' in an answer to yes-no questions:

\begin{exe}
\ex 
\begin{xlist} 
\exi{A:} \gll Elment már Mari?\\
			  off.went.\textsc{3sg} already Mary\\
\glt `Has Mary already left?'
\exi{B:} \gll El. / Nem.\\
		 off {} not\\
\glt `Yes./No.'
\end{xlist}
\end{exe}

Whatever is located in the [Spec,FP] position bears main stress and the movement of such elements can thus be captured by general rules of information-structurally determined movement (see \sectref{sec:6information}; see \citealt{ekiss2002, ekiss2008li} and \citealt{szendroi2001diss} for analyses of Hungarian). The question that arises in connection with (\ref{fphung}) is why the verb moves up to the F head in the first place. According to \citet{ekiss2008li}, the constituent in [Spec,FP] (her FocP) moves from within the VP to [Spec,PredP], then to [Spec,TP] and subsequently to [Spec,FP]: the verb moves along into the respective heads. Verb movement occurs generally in finite clauses, not just in interrogatives (see also \citealt{brody1990, brody1995}), so the trigger cannot be specific to interrogatives.

I suggest that the triggering feature is [fin], similarly to what we observed in Germanic (see \chapref{ch:3}). The features [wh]/[Q] and [fin] are passed on from C to F (\citealt{bacskaiatkari2018kenesei}). Evidence for verb movement being related to finiteness comes from the fact that verb movement to F is obligatory in finite clauses but not in infinitival clauses (which also allow focussing). Consider the following examples for finite clauses containing focussed elements with \textit{csak} `only':

\ea
\ea[*]{\gll Csak MARIT fel\textbf{hívtam}. \label{focusungr}\\
only Mary.\textsc{acc} up.called.\textsc{1sg}\\
\glt `I called up ONLY MARY.'}
\ex[]{\gll Csak MARIT \textbf{hívtam} fel. \label{focusgramm}\\
only Mary.\textsc{acc} called.\textsc{1sg} up\\
\glt `I called up ONLY MARY.'}
\z
\z

As can be seen, (\ref{focusungr}) is ungrammatical, as the verb does not move up to be adjacent to the focussed element, while (\ref{focusgramm}), where this movement has taken place, is well-formed. The same asymmetry does not hold for infinitival clauses (\citealt[448, ex. 20]{ekiss2008li}):

\ea
\ea \gll Szeretném \textbf{csak} \textbf{MARIT} fel\textbf{hívni}.\\
like.\textsc{cond.1sg} only Mary.\textsc{acc} up.call.\textsc{inf}\\
\glt `I would like to call up ONLY MARY.'
\ex \gll Szeretném \textbf{csak} \textbf{MARIT} \textbf{hívni} fel.\\
like.\textsc{cond.1sg} only Mary.\textsc{acc} call.\textsc{inf} up\\
\glt `I would like to call up ONLY MARY.'
\z
\z

The requirement to fill the F head by overt material is similar to the requirement of filling a C specified as [fin] in German and a C specified as [fin] and [wh] in English. This gives further support to the idea that lower peripheries can be fully-fledged and not reduced to hosting elements moving due to their special information-structural status. In addition, it shows that the lexicalisation requirement on [fin] is more general than merely applying to the Germanic CP-domain.

The element -\textit{e} is a  clitic that requires a host; this is satisfied by verb movement, which occurs also in cases where no element moves to [Spec,FP], that is, when there is no focussed element or polarity marker (such as the verbal particle). This can be observed with embedded non-negated questions that contain a lexical verb without a particle:

\ea \gll Azt kérdeztem, \textbf{(hogy)} l\'attad\textbf{-e} Marit. \label{lattade}\\
that.\textsc{acc} asked.\textsc{1sg} \phantom{\textbf{(}}that saw.\textsc{2sg}-Q Mary.\textsc{acc}\\
\glt `I asked if you have seen Mary.'
\z

The structure for the subclause in (\ref{verbe}) is shown in \figref{treeverbe}. The structure for the subclause in (\ref{lattade}) can be represented as in \figref{fig:6:ex27}.


\begin{figure}
\begin{floatrow}
\captionsetup{margin=.05\linewidth}
\ffigbox[.475\textwidth]
{\begin{forest} baseline, qtree
[CP
	[C$'$
		[C\textsubscript{{[}Q{]},{[}fin{]}}
			[(hogy)\textsubscript{{[}fin{]}}]
		]
		[\ldots
			[FP [DP [P\'eter,roof]] [F$'$ [F\textsubscript{{[}Q{]},{[}fin{]}} [h\'ivta\textsubscript{i}-e\textsubscript{{[}Q{]}}]] [TP [\phantom{xxx},roof]]]]
		]
	]
]
\end{forest}}
{\caption{Verbs with a particle} \label{treeverbe}}

\ffigbox[.525\textwidth]
{\begin{forest} baseline, qtree
[CP
	[C$'$
		[C\textsubscript{{[}Q{]},{[}fin{]}}
			[(hogy)\textsubscript{{[}fin{]}}]
		]
		[\ldots
			[FP [F$'$ [F\textsubscript{{[}Q{]}} [l\'attad\textsubscript{i}-e\textsubscript{{[}Q{]},{[}fin{]}}]] [TP [\phantom{xxx},roof]]]]
		]
	]
]
\end{forest}}
{\caption{Verbs without a particle} \label{fig:6:ex27}}
\end{floatrow}
\end{figure}

In cases where the clause is elliptical, we can observe that -\textit{e} attaches to the focussed remnant:

\ea \gll Tudom, hogy valaki l\'atta Marit, de nem tudom, hogy P\'eter-e.\\
know.\textsc{1sg} that someone saw.\textsc{3sg} Mary.\textsc{acc} but not know.\textsc{1sg} that Peter-Q\\
\glt `I know that someone saw Mary but I don't know if it was Peter.'
\z

In this case, the DP \textit{P\'eter} is in [Spec,FP] and -\textit{e} is in F. This follows automatically from the general properties of focussing and the element -\textit{e}, as discussed above. What seems to be somewhat peculiar is the fact that there is no verb located in F: if the finite verb were in F, then it would escape deletion, just as -\textit{e} does. The state of affairs is schematised in \figref{treehogyfp}.

\begin{figure}
\caption{Attachment to a remnant}
\label{treehogyfp}
\begin{forest} baseline, qtree
[CP
	[C$'$
		[C\textsubscript{{[}Q{]},{[}fin{]}}
			[(hogy)\textsubscript{{[}fin{]}}]
		]
		[\ldots
			[FP [DP [P\'eter,roof]] [F$'$ [F\textsubscript{{[}Q{]}} [{[}E{]}-e\textsubscript{{[}Q{]}}]] [TP [\sout{l\'atta Marit},roof]]]]
		]
	]
]
\end{forest}
\end{figure}

The non-elliptical counterpart (with the same word order) would be ungrammatical, as the movement of the verb is otherwise triggered. The difference, then, lies solely in the presence of the [E] feature, which suggests that what blocks the movement of the verb is this feature itself. Note that this does not equal saying that whenever [E] is present on a functional head, there can be no element in that functional head: in this particular case, the element -\textit{e} is base-generated in this position. The movement of the verb, normally triggered by the [fin] feature on F, seems not to be allowed.

This constraint is apparently not a unique property of embedded polar questions but it can be observed in comparative clauses as well. As discussed by \citet[174--175]{bacskaiatkari2018langsci}, Hungarian allows comparative subclauses to contain both an overt quantified expression (or a DP containing a quantified expression) and an overt lexical verb, even if both elements are non-contrastive:\pagebreak

\ea \gll Mari	több macskát vett, mint	\textbf{ahány} \textbf{macskát} Péter \textbf{vett}. \label{hungnomfull}\\
Mary more	cat.\textsc{acc} bought.\textsc{3sg} than	how.many cat.\textsc{acc} Peter bought.\textsc{3sg}\\
\glt `Mary bought more cats than Peter did.'
\z

In this case, both the DP containing the quantified expression \textit{ah\'any macsk\'at} `how many cats' and the lexical verb \textit{vett} `bought' are overt. There is evidence that the remnant (e.g. \textit{Péter} in (\ref{hungnomellipsis}) above) moves to [Spec,FP], as it bears main stress and its movement in constructions with a verbal particle triggers the inversion of the verbal particle and the verb. The structure of cases like (\ref{hungnomfull}) can be schematised as in \figref{fig:6:ex29} (cf. \citealt[185--192]{bacskaiatkari2018langsci}).

\begin{figure}
\caption{Comparative subclauses}
\label{fig:6:ex29}
\begin{forest} baseline, qtree
[CP
	[C$'$
		[C\textsubscript{{[}compr{]},{[}fin{]}}
			[mint\textsubscript{{[}compr{]},{[}fin{]}}]
		]
		[CP
			[DP\textsubscript{i} [ah\'any macsk\'at,roof]]
			[C$'$ [C] [FP [DP\textsubscript{j} [P\'eter,roof]] [F$'$ [F\textsubscript{{[}fin{]}} [vett\textsubscript{k}]] [TP [t\textsubscript{i} t\textsubscript{j} t\textsubscript{k},roof]]]]]
		]
	]
]
\end{forest}
\end{figure}

Since both the element containing the quantified expression and the lexical verb are non-contrastive and redundant in these cases, they can also be eliminated:

\ea \gll Mari	több macskát vett, mint	Péter. \label{hungnomellipsis}\\
Mary more	cat.\textsc{acc} bought.\textsc{3sg} than Peter\\
\glt `Mary bought more cats than Peter did.'
\z

This results in the structure shown in \figref{treehungariantpellipsis} (see \citealt[179]{bacskaiatkari2018langsci}).

\begin{figure}
\caption{Elliptical comparatives}
\label{treehungariantpellipsis}  
\begin{forest} baseline, qtree, for tree={align=center}
[CP
	[C$'$
		[C\textsubscript{{[}compr{]},{[}fin{]}}
			[mint\textsubscript{{[}compr{]},{[}fin{]}}]
		]
		[FP
			[DP\textsubscript{i} [P\'eter,roof]]
			[F$'$ [F [{[}E{]}]] [TP [\sout{t\textsubscript{i} vett ah\'any macsk\'at},roof]]]
		]
	]
]
\end{forest}
\end{figure}

As the [E] feature is again located on F, the complement is eliminated, resulting in the deletion of the DP containing the quantified expression and of the lexical verb (see \citealt{bacskaiatkari2016alh} for a cross-linguistic investigation of why the lower CP is not generated in this case, resulting in the absence of movement for the phrase containing the comparative operator). The representation in (\ref{treehungariantpellipsis}) suggests that the verb does not move up to F in this case; this is descriptively adequate, as the lexical verb should escape deletion otherwise, which is not what we see in (\ref{hungnomellipsis}). Indeed, deleting the DP containing the quantified expression but not the lexical verb is ungrammatical:

\ea[*]{\gll Mari	több macskát vett, mint	Péter \textbf{vett}. \label{verbmov}\\
Mary more	cat.\textsc{acc} bought.\textsc{3sg} than	Peter bought.\textsc{3sg}\\
\glt `Mary bought more cats than Peter did.'}
\z

In this configuration, the verb escapes deletion, which suggests that the verb should be in F, as [E] is regularly in F; however, this is apparently illicit. In this case, there is no overt element in F at all, unlike in embedded polar questions, so the only reason for the ungrammaticality of the relevant elliptical constructions is that the verb cannot move to F because the [E] feature is already there.

This raises the question why the movement of the verb is blocked, though the [E] feature itself is compatible with an overt element in F. Recall that the movement of the verb in F is triggered by the [fin] feature and as such it is obligatory in finite clauses (while it may optionally occur in non-finite clauses). It follows that if we assume that the F head contains both a [fin] feature and the [E] feature, there should be contradictory requirements on verb movement: the [fin] feature would require verb movement to F, while the [E] feature would ban this movement. In other words, not only (\ref{verbmov}) is expected to be ungrammatical, since verb movement goes against the requirement set by the [E] feature, but also (\ref{hungnomellipsis}), where the lack of verb movement would leave the [fin] feature unchecked. However, (\ref{hungnomellipsis}) is grammatical: this in fact suggests that in this case, there is no [fin] feature on the F head that would trigger verb movement.

In this way, the [E] feature is not only specified as [edge] but it is incompatible with [fin]. This is not even surprising, as clausal ellipsis regularly eliminates the finite verb, so that there is ultimately nothing in the clause that would suggest that it would be finite in any way. If this view is correct, we expect these properties of the [E] feature to be constant also in other languages and in other constructions, specifically in other functional left peripheries. In the remainder of this chapter, I will show that this is indeed the case and the analysis proposed here can be carried over to Germanic languages.

\section{Sluicing} \label{sec:6sluicing}
In \chapref{ch:3}, I briefly discussed the issue of complementiser deletion in sluicing patterns in dialects that otherwise allow Doubly Filled COMP. This is illustrated in (\ref{sluicethat}) below:

\ea They discussed a certain model, but they didn't know \textbf{which model (*that)}. \label{sluicethat}
\z

I suggested in \chapref{ch:3} that the ungrammaticality of (\ref{sluicethat}) may also have to do with the prosodic properties of the complementiser, that is, with the fact that it normally cliticises onto the following complement TP, which is evidently violated in (\ref{sluicethat}); in addition, however, it seems that the [E] feature responsible for sluicing is simply incompatible with the feature specification of \textit{that}.

Recall that the regular Doubly Filled COMP pattern involves the co-occurrence of an overt complementiser, specified as [fin], in C, and an operator element in the specifier, checking off the [wh] feature on the C head, as shown in \figref{treedfcfiniteness}.

\begin{sloppypar}
Once sluicing occurs, the [E] feature must be located on the C head that projects the CP hosting the \textit{wh}-element in its specifier. Since this C head is equipped with a [wh] feature triggering movement, the [edge] feature requirement of the [E] feature is immediately satisfied. As the [E] feature is incompatible with [fin], the complementiser \textit{that} cannot occur in these constructions. The relevant structure is illustrated in \figref{treewhichmodele}.
\end{sloppypar}

\begin{figure}
\caption{Doubly Filled COMP} \label{treedfcfiniteness}
\begin{forest} baseline, qtree
[CP
	[which model\textsubscript{{[}wh{]}}]
	[C$'$
		[C\textsubscript{{[}fin{]},{[}wh{]}}
			[that\textsubscript{{[}fin{]}}]
		]
		[TP]
	]
]
\end{forest}
\end{figure}

\begin{figure} 
\caption{The [E] feature blocking \textit{that}} \label{treewhichmodele}
\begin{forest} baseline, qtree
[CP
	[which model\textsubscript{{[}wh{]}}]
	[C$'$
		[C\textsubscript{{[}wh{]}}
			[{[}E{]}]
		]
		[TP]
	]
]
\end{forest}
\end{figure}

One might wonder whether the complement of C is factually TP in this case, as a subordinate clause without a [fin] specification is by default not finite. The full variant is of course finite:

\ea They discussed a certain model, but they didn't know \textbf{which model (that)} they discussed.
\z

However, the restrictions holding on the full variant and the elliptical variant are not necessarily the same. For one thing, as has been discussed in this section, the specification of C is different in each case. Moreover, it seems that sluicing is not necessarily isomorphic anyway in general, as discussed by \citet[484--486]{vicente2018}.\footnote{As mentioned already in \sectref{sec:6ellipsis}, this is not surprising inasmuch as leftward movement targeting the FP also involves an underlying structure that is necessarily different from the antecedent clause and, as far as the surface string is concerned, it can also be markedly different from non-elliptical clauses.} This is evident from certain constructions that are assumed to suggest that sluicing can repair island violations (\citealt{merchant2001}). Observe the following asymmetry:

\ea
\ea[]{{[}How diligent a worker]\textsubscript{i} did they hire [\textsubscript{DP} t\textsubscript{i}]? \label{howdiligent}}
\ex[*]{{[}How diligent]\textsubscript{i} did they hire [\textsubscript{DP} t\textsubscript{i} a worker]? \label{howisland}}
\z
\z

The construction in (\ref{howisland}) is ungrammatical as it involves the extraction of a degree expression out of a DP-island; the configuration in (\ref{howdiligent}) is grammatical, since the entire DP moves to the front of the clause (see \citealt{kennedymerchant2000} and \citealt[132--139]{bacskaiatkari2018langsci} on inversion within the DP).

Consider now the following examples (based on \citealt[484, ex. 11]{vicente2018}):

\ea
\ea[*]{They hired a diligent worker, but I don't know [how diligent]\textsubscript{i} they hired [\textsubscript{DP} t\textsubscript{i} a worker]. \label{howislandembedded}}
\ex[]{They hired a diligent worker, but I don't know how diligent [ ]. \label{howsluice}}
\z
\z

Just like in (\ref{howisland}), the construction in (\ref{howislandembedded}) is not licit as it involves an island violation. Its elliptical counterpart in (\ref{howsluice}), however, can apparently violate the constraint on extraction as long as ellipsis takes place. Under this view (see \citealt{merchant2001}), the underlying structure of (\ref{howsluice}) is the same as in (\ref{howislandembedded}). An alternative approach is proposed by \citet{barroselliottthoms2014} and \citet{vicente2018}, who assume that the underlying clause is actually predicative: the phrase \textit{how diligent} originates as the predicate of the clause. This represents an evasive analysis: rather than saying that an ungrammatical syntactic configuration is repaired by ellipsis, the authors argue that the underlying structure is also grammatical in the first place. Consider (based on \citealt[485, ex. 14]{vicente2018}):

\ea They hired a diligent worker, but I don't know [how diligent]\textsubscript{i} [\textsubscript{IP} that worker is t\textsubscript{i}]. \label{howdiligentpred}
\z

In this case, no repair is needed since the extraction of the \textit{wh}-expression from the predicative position constitutes no island violation. This analysis has thus the advantage of not resorting to repair but the predictable isomorphic structure is also lost. The question arises how we can decide between (\ref{howsluice}) and (\ref{howdiligentpred}).

As pointed out by \citet[484--485]{vicente2018}, the desired repair effect does not seem to be borne out in certain cases. The following example shows that adjectives with non-intersective readings do not lead to a repair effect (\citealt[485, ex. 12a]{vicente2018}):

\ea[*]{They hired a hard worker, but I don't know how hard [ ]. \label{howhard}}
\z

According to the repair analysis, the underlying structure should be parallel to (\ref{howislandembedded}):

\ea[*]{They hired a hard worker, but I don't know [how hard]\textsubscript{i} they hired [\textsubscript{DP} t\textsubscript{i} a worker]. \label{howislandembeddedhard}}
\z

Just as (\ref{howislandembedded}), (\ref{howislandembeddedhard}) is predictably ungrammatical due to an island violation. But if (\ref{howsluice}) is grammatical simply because deletion has taken place, we expect (\ref{howhard}) to be grammatical for the same reason, which is evidently not the case. Assuming the evasive analysis, however, what we have to consider is whether the underlying predicative structure is licit or not. The following minimal pair clearly shows that there is a relevant difference in this respect (\citealt[485, ex. 13]{vicente2018}):

\ea
\ea[]{The worker is diligent.}
\ex[*]{The worker is hard. \label{hard}}
\z
\z

Under this analysis, we expect (\ref{howhard}) to be ungrammatical because \textit{hard} cannot be used as a predicate in this construction, as shown by (\ref{hard}). This indicates that the non-isomorphic approach is favourable not merely on theoretical grounds but also because it makes empirically more adequate predictions.

The same conclusion can be drawn from adjectives that have both an intersective and a non-intersective interpretation, such as \textit{old} (\citealt[485, ex. 12b]{vicente2018}):

\ea Jack is visiting an old friend, but I don't know how old [ ].\\
{[}= I don't know the age of Jack's friend.]\\
{[}$\neq$ I don't know how long this friendship has been going on.]
\z

In this case, only the intersective reading is available.

In principle, one might wonder whether the ungrammaticality of (\ref{howhard}) is due to \textit{hard worker} being a compound instead, as members of a compound cannot be extracted. This is, however, not a satisfactory explanation. In languages like Serbo-Croatian, which generally allow Left Branch Extractions, the relevant construction is possible (based on \citealt[486, ex. 16]{vicente2018}):

\ea \gll Jovan je zaposlio tvrdog radnika ali ne znam koliko tvrdog\\
Jovan \textsc{aux} hired hard.\textsc{acc} worker.\textsc{acc} but not know.\textsc{1sg} how hard.\textsc{acc}\\
\glt `Jovan hired a hard worker but I don't know to what extent he is hard-working.'
\z

This suggests that the difference between English and Serbo-Croatian can be drawn back to a more general property of the respective languages, namely whether Left Branch Extractions are allowed. If sluicing could repair island violations, the asymmetry should not arise (\citealt[486]{vicente2018}).

In other words, there is independent evidence supporting the assumption that the complement of a C head containing an [E] feature can be different from a TP identical to the one in a preceding clause: more precisely, under certain circumstances it can also be an underlying predicative structure. As also shown by \citet{vicente2018}, this possibility does not arise at random; considering the relevant examples above, it should be obvious that in all these cases, there is an attributive adjective in the antecedent clause, which can then be reconstructed as a predicative adjective in the elided clause. This apparently violates isomorphism but it does not violate recoverability: a predicative construction is recoverable from an attributive construction (but not vice versa).

The last question to be addressed in this respect concerns tense. In cases like (\ref{howdiligentpred}), the antecedent clause is marked for the past tense, but the reconstructed elided clause appears to be in the present tense; note that a past tense reconstruction is also possible, but not obligatory. The optionality is illustrated in (\ref{workerreading}) below:

\ea They hired a diligent worker, but I don't know how diligent. \label{workerreading}\\
Reading A: `They hired a diligent worker, but I don't know how diligent that worker is.'\\
Reading B: `They hired a diligent worker, but I don't know how diligent that worker was.'
\z

While Reading B is unproblematic as the antecedent clause is also in the past tense, Reading A seems to be problematic inasmuch as present tense is not recoverable on the basis of past tense. 

Consider now the following examples:

\ea
\ea I know Peter. And Agnes, too.
\ex I knew Peter. And Agnes, too.
\z
\z

In both cases, the second clause is elliptical; the tense that is reconstructed in each case matches the one in the first clause:

\ea
\ea I know Peter. And I know Agnes, too.
\ex I knew Peter. And I knew Agnes, too.
\z
\z

In cases like (\ref{howdiligentpred}) there seems to be an optionality that does not necessarily arise, at least not in clauses where the predicative/attributive effect mentioned above does not hold. If so, however, the present vs. past interpretation in clauses like (\ref{howdiligentpred}) may be context-dependent and pragmatic in nature, in the sense that it is not syntactically encoded. In other words, it seems that the complement of C in such cases is not necessarily a TP but rather a tenseless projection encoding predication, call it PredP (in effect, this is much in the sense of \citealt{bowers1993, bowers2010} and \citealt{dendikken2006}, in that predication is not tied to tense). This gives us a modified reconstruction for the elliptical clause:

\ea They hired a diligent worker, but I don't know [how diligent]\textsubscript{i} [\textsubscript{PredP} that worker BE t\textsubscript{i}]. \label{howdiligentpredbe}
\z

The reconstruction of a tenseless PredP instead of a tensed TP arises in cases like (\ref{howdiligentpredbe}) as the antecedent predicative relation is tenseless as well, since the adjective (\textit{diligent}) is an attribute to the noun (\textit{worker}). Note that such tenseless PredPs are contingent upon the FP projected by [E]: the non-elliptical version of (\ref{howdiligentpredbe}) is ill-formed. In other words, the final string is licit precisely because the remnant has undergone leftward movement and landed above the PredP, which must be elided in such cases. This is in line with the general idea that elliptical clauses differ in their derivation from non-elliptical ones.

The difference between (\ref{howdiligentpredbe}) and (\ref{howdiligentpred}) lies solely in tense and the resulting difference in the projection that serves as a complement to C. The point is that the ellipsis feature [E], located on C, can in this way have a syntactic effect since it can in principle change the label of the complement to C. This again reinforces the assumption that the [E] feature is more than a mere additional feature of syntax but it behaves in fact like a proper syntactic head that has an effect beyond ellipsis proper. Its incompatibility with the [fin] feature also results in the fact that the complement of C in these cases is not necessarily TP: this is borne out only if the [E] feature can impose a ban on [fin], as [fin] would otherwise be expected to be carried over due to reconstruction effects from the antecedent clause.

\section{Ellipsis in comparatives} \label{sec:6comparatives}
\subsection{The basic data} \label{sec:6basic}
As mentioned in \chapref{ch:5}, embedded degree clauses are often elliptical. This is illustrated in (\ref{paul}) below for German comparatives expressing inequality (the same conclusions apply to equatives, not discussed here separately):

\ea \gll Ralf ist größer als Paul. \label{paul}\\
Ralph is taller as Paul\\
\glt `Ralph is taller than Paul.'
\z

In (\ref{paul}), the complementiser \textit{als} is followed by a single remnant (the DP \textit{Paul}). It is evident that \textit{als} can take a full TP complement, as illustrated in (\ref{germanfull}) below (see also \chapref{ch:5}):

\ea \gll Der Tisch ist länger als das Büro breit ist. \label{germanfull}\\
the.\textsc{m.nom} table is longer as the.\textsc{n.nom} office wide is\\
\glt `The table is longer than the office is wide.'
\z

While the case of the remnant is not visible on the proper name remnant in (\ref{paul}), a pronominal remnant is indicative of case:

\ea \label{germancomp}
\ea[]{\gll Ralf ist größer als \textbf{ich}. \label{ich}\\
Ralph is taller as I\\
\glt `Ralph is taller than I am.'}
\ex[*]{Ralf ist größer als \textbf{mich}. \label{mich}\\
Ralph is taller as me\\
\glt `Ralph is taller than I am.'}
\z
\z

As indicated, the nominative remnant in (\ref{ich}) is grammatical, while the accusative remnant in (\ref{mich}) is not. This is expected as the complementiser \textit{als} does not assign accusative case to the DP subject remnant, which bears nominative case regularly as the subject of a tensed clause (TP). In other words, the underlying clause is a full, tensed clause:

\ea \gll Ralf ist größer als \textbf{ich} x-groß bin.\\
Ralph is taller as I x-tall am\\
\glt `Ralph is taller than I am.'
\z

Assuming that the remnant moves to FP, in line with \citet{merchant2001}, the structure is schematically represented as in \figref{fig:6:ex:alsichgrossbin} (see also the discussion in \sectref{sec:6lower} above).

\begin{figure}
\caption{A predicative underlying clause}
\label{fig:6:ex:alsichgrossbin}
\begin{forest} baseline, qtree
[CP
	[C$'$
		[C
			[als]
		]
		[FP
			[DP\textsubscript{i} [ich,roof]]
			[F$'$ [F [{[}E{]}]] [TP [\sout{\textit{t}\textsubscript{i} x-groß bin},roof]]]
		]
	]
]
\end{forest}
\end{figure}

The movement of the remnant DP to [Spec,FP] is triggered by way of the [edge] feature, which is an inherent property of the [E] feature heading its own projection.

English differs from German regarding the case of the remnant. In non-el\-lip\-ti\-cal comparative clauses, the subject is in the nominative case:

\ea Ralph is taller than I am.
\z

In elliptical clauses, both a nominative and an accusative remnant is possible:

\ea \label{thanime}
\ea[?]{Ralph is taller than \textbf{I}. \label{thani}}
\ex[]{Ralph is taller than \textbf{me}. \label{thanme}}
\z
\z

As can be seen, the remnant is preferably in the accusative case, possibly also due to phonological reasons: the remnant bears extra (focal) stress. Note also that in English, the default case is the accusative (\citealt{schuetze2001}). At the same time, as pointed out by \citet[618]{bhatttakahashi2011}, the nominative remnant is not excluded either (contrary to \citealt{pancheva2006}), indicating that \textit{than} is not a preposition assigning accusative case to the pronoun (contrary to \citealt{hankamer1973}). The appearance of the accusative case on the remnant is rather due to the absence of the TP projection in the subclause (see also \citealt{bacskaiatkari2014alh, bacskaiatkari2018langsci}); in such cases, English allows the default accusative case on DPs. This property of English is not directly related to comparative constructions and will not be discussed further here.

What is of interest to us is a peculiar constellation in German attributive comparatives that is not expected on the basis of (\ref{germancomp}). Consider first the following example from English:

\ea I saw a taller woman than my mother. \label{ambiguity}\\
External reading: `I saw a taller woman than my mother saw.'\\
Internal reading: `I saw a taller woman than my mother is.'
\z

As indicated, the sentence in (\ref{ambiguity}) has two readings, which \citet{lernerpinkal1995} refer to as DP-external and DP-internal readings (abbreviated here as external and internal). In the DP-external reading, the reconstructed comparative clause parallels the matrix clause, in that a lexical verb (here: \textit{see}) is reconstructed and the gradable adjective is reconstructed as an attribute to a noun (see \citealt{kennedymerchant2000} and \citealt[125--139]{bacskaiatkari2018langsci} on the inversion involving \textit{x-tall}):

\ea I saw a taller woman than [\textsubscript{FP} my mother \sout{[\textsubscript{TP} \textit{t} saw x-tall a woman]}]. \label{englishext}
\z

By contrast, the DP-internal reading involves the reconstruction of a predicative construction, where the gradable adjective is a predicate and tense is not encoded (see \citealt{bacskaiatkari2017icgl}):

\ea I saw a taller woman than [\textsubscript{FP} my mother \sout{[\textsubscript{PredP} \textit{t} BE x-tall]}]. \label{englishint}
\z

The German equivalent of (\ref{ambiguity}) is likewise ambiguous (see also \citealt{bacskaiatkari2017atoh}):

\ea \gll Ich habe eine größere Frau \textbf{als} meine Mutter gesehen. \label{germanfem}\\
I have.\textsc{1sg} a.\textsc{f.acc} taller.\textsc{f.acc} woman than my.\textsc{f.nom}/my.\textsc{f.acc} mother seen\\
\glt External reading: `I saw a taller woman than my mother saw.'\\
Internal reading: `I saw a taller woman than my mother is.'
\z

Note that the remnant DP \textit{meine Mutter}, as indicated in (\ref{germanfem}) above, is case-ambiguous between the nominative and the accusative (just like in English, but English is much less reliable regarding morphological case, as discussed above). With overt case distinction between the nominative and the accusative, the ambiguity disappears (\citealt{bacskaiatkari2017atoh}):

\ea \label{germanmasc}
\ea \gll Ich habe einen größeren Mann \textbf{als} mein Vater gesehen. \label{germanlexical}\\
I have.\textsc{1sg} a.\textsc{m.acc} taller.\textsc{m.acc} man than my.\textsc{m.nom} father seen\\
\glt External reading: `I saw a taller man than my father saw.'\\
\ex \gll Ich habe einen größeren Mann \textbf{als} meinen Vater gesehen. \label{germanacc}\\
I have.\textsc{1sg} a.\textsc{m.acc} taller.\textsc{m.acc} man than my.\textsc{m.acc} father seen\\
\glt Internal reading: `I saw a taller man than my father is.'
\z
\z

While (\ref{germanlexical}) is expected on the basis of (\ref{germancomp}), (\ref{germanacc}) is not: it seems that while German generally does not allow subject remnants in the accusative case, in the particular configuration in (\ref{germanacc}) this is possible.

\subsection{Experimental methodology} \label{sec:6experimental}
In order to gain more insight into this matter, I carried out an acceptability rating experiment at the University of Potsdam in 2020.\footnote{I owe many thanks to Marta Wierzba for her help with setting up the experiment and recruiting the informants, as well as for her suggestions regarding the items and her indispensable help with the platform L-Rex.} The aim of this experiment was to examine the acceptability of elliptical comparatives with a single remnant across three major conditions: (i) case-ambiguous (feminine) remnants, (ii) nominative (masculine) remnants, and (iii) accusative (masculine) remnants, similarly to (\ref{germanfem}) and (\ref{germanmasc}) above. Importantly, the individual target sentences were presented in a context that allowed only a DP-external or a DP-internal reading: in other words, ambiguity was not tested explicitly, unlike in the study mentioned above, which measured the ambiguity of sentences out of context. Since this experimental study is to be discussed in a designated paper in detail, in what follows I am going to concentrate on the aspects that are immediately relevant to the purposes of the present thesis.

Altogether 48 informants took part in the study, which was designed and made available via L-Rex (\citealt{lrex}). The items were distributed over 12 questionnaires via a Latin Square design; each participant had to rate 64 items. The items were randomised. The experiment contained altogether 720 different stimuli and 48 fillers (the fillers contained target sentences with a gender mismatch in contexts where only an internal reading was available), so that each item was evaluated by 4 informants. The participants had to rate the target sentences on a scale from 5 (fully acceptable) to 1 (fully unacceptable). The informants were all born between 1979 and 2002.

\subsection{The results for the basic condition} \label{sec:6results}
Let us consider the basic contrast illustrated by the following items:

\ea Kontext: Ich habe mit [meiner Schwester / meinem Bruder] beschlossen, unseren Eltern dieses Jahr selbstgemalte Bilder zu schenken. \label{extpresfull}\\
`Context: I have decided with [my sister / my brother] to give self-painted pictures to our parents this year.'\\
\ea \gll Ich male ein schöneres Bild als meine Schwester. \label{extpresfullambig}\\
I paint.\textsc{1sg} a.\textsc{n} nicer.\textsc{n} picture as my.\textsc{f.nom/acc} sister\\
\glt `I am painting a nicer picture than my sister.'
\ex \gll Ich male ein schöneres Bild als mein Bruder. \label{extpresfullnom}\\
I paint.\textsc{1sg} a.\textsc{n} nicer.\textsc{n} picture as my.\textsc{m.nom} brother\\
\glt `I am painting a nicer picture than my brother.'
\ex \gll Ich male ein schöneres Bild als meinen Bruder. \label{extpresfullacc}\\
I paint.\textsc{1sg} a.\textsc{n} nicer.\textsc{n} picture as my.\textsc{m.acc} brother\\
\glt `I am painting a nicer picture than my brother.'
\z
\z

\begin{sloppypar}
In the example above, (\ref{extpresfullambig}) represents the case-ambiguous configuration, while in (\ref{extpresfullnom}) the remnant is overtly marked as nominative and in (\ref{extpresfullacc}) as accusative. In all these cases, the remnant is a full DP, and the reduced \textit{als}-clause immediately follows the direct object of the matrix clause (the lexical verb moves up to the C position). The results for the items of the type in (\ref{extpresfull}) are shown in \tabref{tableextpresfull}.
\end{sloppypar}

\begin{table}
\begin{tabular}{l *3{S[table-format=1.2]}}
\lsptoprule
{} & {Case-ambiguous} & {Nominative} & {Accusative}\\\midrule
Mean & 4.58 & 4.60 & 1.88\\
Median & 5 & 5 & 1\\
Variance & 0.70 & 0.36 & 2.15\\
Standard deviation & 0.85 & 0.61 & 1.48\\
\lspbottomrule
\end{tabular}
\caption{External reading, full DP remnants, basic condition}
\label{tableextpresfull}
\end{table}

As can be seen, the results are expected on the basis of what was said about (\ref{germanfem}) and (\ref{germanmasc}) above, in that the unambiguously accusative remnant has a low average rating compared to the cases that can be interpreted as nominative. The difference between the nominative and accusative masculine remnants is significant at $p<0.05$: I carried out a simple comparison of means calculation and this gives $p<0.0001$ (the 95\% confidence interval is $-3.1788$ to $-2.2612$). This also follows from German not allowing accusative case remnants in the English way, as discussed above.

The basic condition was tested for pronominal remnants as well, illustrated in (\ref{extprespro}) below:\pagebreak

\ea Kontext: Ich habe mit [meiner Schwester / meinem Bruder] beschlossen, unseren Eltern dieses Jahr selbstgemalte Bilder zu schenken. \label{extprespro}\\
`Context: I have decided with [my sister / my brother] to give self-painted pictures to our parents this year.'\\
\ea \gll Ich male ein schöneres Bild als sie. \label{extpresproambig}\\
I paint.\textsc{1sg} a.\textsc{n} nicer.\textsc{n} picture as she.\textsc{nom/acc}\\
\glt `I am painting a nicer picture than her.'
\ex \gll Ich male ein schöneres Bild als er. \label{extprespronom}\\
I paint.\textsc{1sg} a.\textsc{n} nicer.\textsc{n} picture as he.\textsc{nom}\\
\glt `I am painting a nicer picture than him.'
\ex \gll Ich male ein schöneres Bild als ihn. \label{extpresproacc}\\
I paint.\textsc{1sg} a.\textsc{n} nicer.\textsc{n} picture as him.\textsc{acc}\\
\glt `I am painting a nicer picture than him.'
\z
\z

The results for the items of the type in (\ref{extprespro}) are shown in \tabref{tableextprespro}.
The picture is altogether similar to what we found for full DP remnants; again, the difference between the nominative and accusative masculine remnants is significant: $p<0.0001$ (the 95\% confidence interval is $-3.5437$ to $-2.9763$). In this case, the unambiguously accusative condition was judged even worse than for full DP remnants: this difference is statistically significant ($p=0.0006$; 95\% confidence interval $-1.2134 to -0.3466$), while there is no significant difference between the nominative remnants. The difference between full DPs and pronouns in the accusative is not a genuine grammatical contrast but it is can be explained by assuming that the full DP remnant was processed as a nominative by some of the informants (but not the others, hence the relatively high variance), while such a misinterpretation is not possible with the pronouns. This would indicate that the way morphological marking (suffixation versus suppletion) affects perception is also relevant.

\vfill
\begin{table}[H]
\begin{tabular}{l *3{S[table-format=1.2]}}
\lsptoprule
{} & {Case-ambiguous} & {Nominative} & {Accusative}\\\midrule
Mean & 4.60 & 4.36 & 1.10\\
Median & 5 & 5 & 1\\
Variance & 0.61 & 0.86 & 0.93\\
Standard deviation & 0.79 & 0.94 & 0.31\\
\lspbottomrule
\end{tabular}
\caption{External reading, pronominal remnants, basic condition}
\label{tableextprespro}
\end{table} 
\vfill\pagebreak


Consider the basic condition with an internal reading (full DP remnants):

\ea Kontext: [Deine Schwester / Dein Bruder] ist ganz schön groß, jedoch nicht [die größte Frau / der größte Mann] der Welt. \label{intpresfull}\\
`Context: [Your sister / Your brother] is fairly tall but not [the tallest woman / the tallest man] in the world.'\\
\ea \gll Ich kenne eine größere Frau als deine Schwester. \label{intpresfullambig}\\
I know.\textsc{1sg} a.\textsc{f} taller.\textsc{f} woman as your.\textsc{f.nom/acc} sister\\
\glt `I know a taller woman than your sister.'
\ex \gll Ich kenne einen größeren Mann als dein Bruder. \label{intpresfullnom}\\
I know.\textsc{1sg} a.\textsc{m} taller.\textsc{m} man as your.\textsc{m.nom} brother\\
\glt `I know a taller man than your brother.'
\ex \gll Ich kenne einen größeren Mann als deinen Bruder. \label{intpresfullacc}\\
I know.\textsc{1sg} a.\textsc{m} taller.\textsc{m} man as your.\textsc{m.acc} brother\\
\glt `I know a taller man than your brother.'
\z
\z

The results for the items of the type in (\ref{intpresfull}) are shown in \tabref{tableintpresfull}.
As can be seen, in this case the nominative remnant is judged to be worse than the accusative one. Just as with the external reading, the difference between the nominative and accusative masculine remnants is significant ($p<0.0001$; 95\% confidence interval 1.1713 to 2.2487). At the same time, it is also judged to be better than accusative remnants (full DPs) in the external reading context.

\begin{table}
\begin{tabular}{l *3{S[table-format=1.2]}}
\lsptoprule
{} & {Case-ambiguous} & {Nominative} & {Accusative}\\\midrule
Mean & 4.23 & 2.56 & 4.27\\
Median & 5 & 2 & 5\\
Variance & 1.38 & 2.00 & 1.45\\
Standard deviation & 1.19 & 1.43 & 1.22\\
\lspbottomrule
\end{tabular}
\caption{Internal reading, full DP remnants, basic condition}
\label{tableintpresfull}
\end{table} 


The following items illustrate the basic condition with an internal reading, involving pronominal remnants:

\ea Kontext: [Deine Schwester / Dein Bruder] ist ganz schön groß, jedoch nicht [die größte Frau / der größte Mann] der Welt. \label{intprespro}\\
`Context: [Your sister / Your brother] is fairly tall but not [the tallest woman / the tallest man] in the world.'\\
\ea \gll Ich kenne eine größere Frau als sie. \label{intpresproambig}\\
I know.\textsc{1sg} a.\textsc{f} taller.\textsc{f} woman as she.\textsc{nom/acc}\\
\glt `I know a taller woman than her.'
\ex \gll Ich kenne einen größeren Mann als er. \label{intprespronom}\\
I know.\textsc{1sg} a.\textsc{m} taller.\textsc{m} man as he.\textsc{nom}\\
\glt `I know a taller man than him.'
\ex \gll Ich kenne einen größeren Mann als ihn. \label{intpresproacc}\\
I know.\textsc{1sg} a.\textsc{m} taller.\textsc{m} man as him.\textsc{acc}\\
\glt `I know a taller man than him.'
\z
\z

The results for the items of the type in (\ref{intprespro}) are shown in \tabref{tableintprespro}.
Just as with full DP remnants, the difference between the nominative and accusative masculine remnants is significant ($p<0.0001$; 95\% confidence interval 2.3032 to 3.1568). Again, just as in the external reading conditions, the pronominal remnant is more explicit than the full DP variant: while there is no significant difference between the full DP and the pronominal remnants in the preferred accusative version, the difference between the two is significant ($p=0.0255$; 95\% confidence interval 0.0803 to 1.1997) in the nominative. This again points to a role of morphological marking in processing.

\begin{table}
\begin{tabular}{l *3{S[table-format=1.2]}}
\lsptoprule
{} & {Case-ambiguous} & {Nominative} & {Accusative}\\
\midrule
Mean & 4.35 & 1.92 & 4.65\\
Median & 5 & 1 & 5\\
Variance & 0.98 & 1.74 & 0.44\\
Standard deviation & 1.00 & 1.33 & 0.67\\
\lspbottomrule
\end{tabular}
\caption{Internal reading, pronominal remnants, basic condition}
\label{tableintprespro}
\end{table} 


\subsection{The results for the perfective condition} \label{sec:6resultsperfective}
Apart from the basic condition illustrated above, the experiment also examined the same phenomena (external vs. internal, full DP vs. pronoun) with perfective verbs (where the lexical verb is in the base position). This is illustrated in (\ref{extperftfullbase}) below for full DP remnants:

\ea Kontext: [Deine Mutter / Dein Vater] hat schon ein Buch geschrieben, jedoch nie veröffentlicht, weil es nicht so gut gelungen ist. Du wirst aber eine Veröffentlichung schaffen. \label{extperftfullbase}\\
`Context: [Your mother / Your father] has already written a book but has never published it as it is not so good. You will make it to a publication, though.'\\
\ea \gll Du hast ein besseres Buch als deine Mutter geschrieben. \label{extperftfullbaseambig}\\
you have.\textsc{2sg} a.\textsc{n} better.\textsc{n} book as your.\textsc{nom/acc} mother written\\
\glt `You have written a better book than your mother.'
\ex \gll Du hast ein besseres Buch als dein Vater geschrieben. \label{extperftfullbasenom}\\
you have.\textsc{2sg} a.\textsc{n} better.\textsc{n} book as your.\textsc{nom} father written\\
\glt `You have written a better book than your father.'
\ex \gll Du hast ein besseres Buch als deinen Vater geschrieben. \label{extperftfullbaseacc}\\
you have.\textsc{2sg} a.\textsc{n} better.\textsc{n} book as your.\textsc{acc} father written\\
\glt `You have written a better book than your father.'
\z
\z

The results for the items of the type in (\ref{extperftfullbase}) are shown in \tabref{tableextperftfullbase}.
As can be seen, the results are quite similar to the ones reported for the corresponding basic condition in \tabref{tableextpresfull}; the difference between the nominative and accusative masculine remnants is significant ($p<0.0001$; 95\% confidence interval $-2.9993$ to $-2.0407$). That is, whether the lexical verb stays in its base position or moves to C seems to make no major difference. The same applies to pronouns; the patterns are illustrated in (\ref{extperftprobase}) below.

\begin{table}
\begin{tabular}{l *3{S[table-format=1.2]}}
\lsptoprule
{} & {Case-ambiguous} & {Nominative} & {Accusative}\\
\midrule
Mean & 4.41 & 4.54 & 2.02\\
Median & 5 & 5 & 1\\
Variance & 0.74 & 0.54 & 2.19\\
Standard deviation & 0.87 & 0.74 & 1.50\\
\lspbottomrule
\end{tabular}
\caption{External reading, full DP remnants, perfective condition}
\label{tableextperftfullbase}
\end{table} 


\ea Kontext: [Deine Mutter / Dein Vater] hat schon ein Buch geschrieben, jedoch nie veröffentlicht, weil es nicht so gut gelungen ist. Du wirst aber eine Veröffentlichung schaffen. \label{extperftprobase}\\
`Context: [Your mother / Your father] has already written a book but has never published it as it is not so good. You will make it to a publication, though.'\\
\ea \gll Du hast ein besseres Buch als sie geschrieben. \label{extperftprobaseambig}\\
you have.\textsc{2sg} a.\textsc{n} better.\textsc{n} book as she.\textsc{nom/acc} written\\
\glt `You have written a better book than her.'
\ex \gll Du hast ein besseres Buch als er geschrieben. \label{extperftprobasenom}\\
you have.\textsc{2sg} a.\textsc{n} better.\textsc{n} book as he.\textsc{nom} written\\
\glt `You have written a better book than him.'
\ex \gll Du hast ein besseres Buch als ihn geschrieben. \label{extperftprobaseacc}\\
you have.\textsc{2sg} a.\textsc{n} better.\textsc{n} book as him.\textsc{acc} written\\
\glt `You have written a better book than him.'
\z
\z

The results for the items of the type in (\ref{extperftprobase}) are shown in \tabref{tableextperftprobase}.
Again, the difference between the nominative and accusative masculine remnants is significant ($p<0.0001$; 95\% confidence interval $-3.4313$ to $-2.7087$). Just like in the base condition, the results for the pronominal remnants are again clearer and show less variation: there is no significant difference between the full DP and the pronominal remnants in the preferred nominative version, while the difference between the two is significant ($p=0.0001$; 95\% confidence interval $-1.3705$ to $-0.4695$) in the accusative. This is possibly due to the repair effect in processing mentioned above. The results are thus also quite similar to the ones for the corresponding basic condition in \tabref{tableextprespro}.

\begin{table}
\begin{tabular}{l *3{S[table-format=1.2]}}
\lsptoprule
{} & {Case-ambiguous} & {Nominative} & {Accusative}\\
\midrule
Mean & 4.42 & 4.17 & 1.10\\
Median & 5 & 5 & 1\\
Variance & 0.78 & 1.35 & 0.22\\
Standard deviation & 0.90 & 1.17 & 0.47\\
\lspbottomrule
\end{tabular}
\caption{External reading, pronominal remnants, perfective condition}
\label{tableextperftprobase}
\end{table} 


Consider now the following items for the perfective condition with an internal reading, involving full DP remnants:

\ea Kontext: [Deine Schwester / Dein Bruder] ist ganz schön groß, jedoch nicht [die größte Frau / der größte Mann] der Welt. \label{intperftfullbase}\\
`Context: [Your sister / Your brother] is fairly tall but not [the tallest woman / the tallest man] in the world.'\\
\ea \gll Ich habe eine größere Frau als deine Schwester gesehen. \label{intperftfullbaseambig}\\
I have.\textsc{1sg} a.\textsc{f} taller.\textsc{f} woman as your.\textsc{f.nom/acc} sister seen\\
\glt `I have seen a taller woman than your sister.'
\ex \gll Ich habe einen größeren Mann als dein Bruder gesehen. \label{intperftfullbasenom}\\
I have.\textsc{1sg} a.\textsc{m} taller.\textsc{m} man as your.\textsc{m.nom} brother seen\\
\glt `I have seen a taller man than your brother.'
\ex \gll Ich habe einen größeren Mann als deinen Bruder gesehen. \label{intperftfullbaseacc}\\
I have.\textsc{1sg} a.\textsc{m} taller.\textsc{m} man as your.\textsc{m.acc} brother seen\\
\glt `I have seen a taller man than your brother.'
\z
\z

The results for the items of the type in (\ref{intperftfullbase}) are shown in \tabref{tableintperftfullbase}.
Again, the difference between the nominative and accusative masculine remnants is significant ($p<0.0001$; 95\% confidence interval 1.3278 to 2.2922). Just like in the case of the basic condition with internal readings (and full DP remnants), see \tabref{tableintpresfull}, the differences are less clear-cut than for the external readings. Nevertheless, the preference for the accusative remnant in this position is evidently there.

\begin{table}
\begin{tabular}{l *3{S[table-format=1.2]}}
\lsptoprule
{} & {Case-ambiguous} & {Nominative} & {Accusative}\\
\midrule
Mean & 3.92 & 2.63 & 4.44\\
Median & 4 & 3 & 5\\
Variance & 1.41 & 2.03 & 0.77\\
Standard deviation & 1.20 & 1.44 & 0.87\\
\lspbottomrule
\end{tabular}
\caption{Internal reading, full DP remnants, perfective condition}
\label{tableintperftfullbase}
\end{table} 


Consider now the following items for the perfective condition with an internal reading, involving full DP remnants:

\ea Kontext: [Deine Schwester / Dein Bruder] ist ganz schön groß, jedoch nicht [die größte Frau / der größte Mann] der Welt. \label{intperftprobase}\\
`Context: [Your sister / Your brother] is fairly tall but not [the tallest woman / the tallest man] in the world.'\\
\ea \gll Ich habe eine größere Frau als sie gesehen. \label{intperftprobaseambig}\\
I have.\textsc{1sg} a.\textsc{f} taller.\textsc{f} woman as she.\textsc{nom/acc} seen\\
\glt `I have seen a taller woman than her.'
\ex \gll Ich habe einen größeren Mann als er gesehen. \label{intperftprobasenom}\\
I have.\textsc{1sg} a.\textsc{m} taller.\textsc{m} man as he.\textsc{nom} seen\\
\glt `I have seen a taller man than him.'
\ex \gll Ich habe einen größeren Mann als ihn gesehen. \label{intperftprobaseacc}\\
I have.\textsc{1sg} a.\textsc{m} taller.\textsc{m} man as him.\textsc{acc} seen\\
\glt `I have seen a taller man than him.'
\z
\z

The results for the items of the type in (\ref{intperftprobase}) are shown in \tabref{tableintperftprobase}.
Again, the difference between the nominative and accusative masculine remnants is significant ($p<0.0001$; 95\% confidence interval 2.0157 to 2.9443). Just as with the basic condition, the pronominal remnants deliver clearer result, whereby the unambiguously nominative remnant is judged to be worse than with full DPs: there is no significant difference between the full DP and the pronominal remnants in the preferred accusative version, while the difference between the two is significant ($p=0.0001$; 95\% confidence interval 0.5489 to 1.5911) in the nominative, but at the same time the judgements are better than for accusative remnants in the external reading contexts.  

\begin{table}
\begin{tabular}{l *3{S[table-format=1.2]}}
\lsptoprule
{} & {Case-ambiguous} & {Nominative} & {Accusative}\\
\midrule
Mean & 4.54 & 1.56 & 4.04\\
Median & 5 & 1 & 4\\
Variance & 0.62 & 1.20 & 1.37\\
Standard deviation & 0.80 & 1.11 & 1.18\\
\lspbottomrule
\end{tabular}
\caption{Internal reading, pronominal remnants, perfective condition}
\label{tableintperftprobase}
\end{table} 


The same conclusions apply to cases where the \textit{als}\,+\,DP sequence follows the lexical verb. A set of examples for the external reading with full DP remnants is shown in (\ref{extperftfullextr}) below:

\ea Kontext: [Deine Mutter / Dein Vater] hat schon ein Buch geschrieben, jedoch nie veröffentlicht, weil es nicht so gut gelungen ist. Du wirst aber eine Veröffentlichung schaffen. \label{extperftfullextr}\\
`Context: [Your mother / Your father] has already written a book but has never published it as it is not so good. You will make it to a publication, though.'\\
\ea \gll Du hast ein besseres Buch geschrieben als deine Mutter. \label{extperftfullextrambig}\\
you have.\textsc{2sg} a.\textsc{n} better.\textsc{n} book written as your.\textsc{nom/acc} mother\\
\glt `You have written a better book than your mother.'
\ex \gll Du hast ein besseres Buch geschrieben als dein Vater. \label{extperftfullextrnom}\\
you have.\textsc{2sg} a.\textsc{n} better.\textsc{n} book written as your.\textsc{nom} father\\
\glt `You have written a better book than your father.'
\ex \gll Du hast ein besseres Buch geschrieben als deinen Vater. \label{extperftfullextracc}\\
you have.\textsc{2sg} a.\textsc{n} better.\textsc{n} book written as your.\textsc{acc} father\\
\glt `You have written a better book than your father.'
\z
\z

The results for the items of the type in (\ref{extperftfullextr}) are shown in \tabref{tableextperftfullextr}.
Again, the difference between the nominative and accusative masculine remnants is significant ($p<0.0001$; 95\% confidence interval $-2.3388$ to $-1.3212$). 

\begin{table}
\begin{tabular}{l *3{S[table-format=1.2]}}
\lsptoprule
{} & {Case-ambiguous} & {Nominative} & {Accusative}\\
\midrule
Mean & 4.52 & 4.46 & 1.92\\
Median & 5 & 5 & 1\\
Variance & 0.87 & 1.00 & 2.08\\
Standard deviation & 0.94 & 1.01 & 1.46\\
\lspbottomrule
\end{tabular}
\caption{External reading, full DP remnants, perfective condition, extraposed}
\label{tableextperftfullextr}
\end{table} 


A set of examples for the external reading with pronominal remnants is shown in (\ref{extperftproextr}) below:

\ea Kontext: [Deine Mutter / Dein Vater] hat schon ein Buch geschrieben, jedoch nie veröffentlicht, weil es nicht so gut gelungen ist. Du wirst aber eine Veröffentlichung schaffen. \label{extperftproextr}\\
`Context: [Your mother / Your father] has already written a book but has never published it as it is not so good. You will make it to a publication, though.'\\
\ea \gll Du hast ein besseres Buch geschrieben als sie. \label{extperftproextrambig}\\
you have.\textsc{2sg} a.\textsc{n} better.\textsc{n} book written as she.\textsc{nom/acc}\\
\glt `You have written a better book than her.'
\ex \gll Du hast ein besseres Buch geschrieben als er. \label{extperftproextrnom}\\
you have.\textsc{2sg} a.\textsc{n} better.\textsc{n} book written as he.\textsc{nom}\\
\glt `You have written a better book than him.'
\ex \gll Du hast ein besseres Buch geschrieben als ihn. \label{extperftproextracc}\\
you have.\textsc{2sg} a.\textsc{n} better.\textsc{n} book written as he.\textsc{acc}\\
\glt `You have written a better book than him.'
\z
\z

The results for the items of the type in (\ref{extperftproextr}) are shown in \tabref{tableextperftproextr}.
Again, the difference between the nominative and accusative masculine remnants is significant ($p<0.0001$; 95\% confidence interval $-3.7511$ to $-3.2889$). Just like in the non-extraposed conditions, the difference between the full DP versus pronominal remnants is not significant in the preferred nominative case, while there is a significant difference ($p=0.0001$; 95\% confidence interval $-1.2840$ to $-0.4360$) in the accusative case, the more explicit pronominal remnant being less acceptable.

\begin{table}
\begin{tabular}{l *3{S[table-format=1.2]}}
\lsptoprule
{} & {Case-ambiguous} & {Nominative} & {Accusative}\\
\midrule
Mean & 4.58 & 4.58 & 1.06\\
Median & 5 & 5 & 1\\
Variance & 0.45 & 0.58 & 0.06\\
Standard deviation & 0.68 & 0.77 & 0.24\\
\lspbottomrule
\end{tabular}
\caption{External reading, pronominal remnants, perfective condition, extraposed}
\label{tableextperftproextr}
\end{table} 


Consider now the following items for the perfective condition with an internal reading, involving extraposed full DP remnants:

\ea Kontext: [Deine Schwester / Dein Bruder] ist ganz schön groß, jedoch nicht [die größte Frau / der größte Mann] der Welt. \label{intperftfullextr}\\
`Context: [Your sister / Your brother] is fairly tall but not [the tallest woman / the tallest man] in the world.'\\
\ea \gll Ich habe eine größere Frau gesehen als deine Schwester. \label{intperftfullextrambig}\\
I have.\textsc{1sg} a.\textsc{f} taller.\textsc{f} woman seen as your.\textsc{f.nom/acc} sister\\
\glt `I have seen a taller woman than your sister.'
\ex \gll Ich habe einen größeren Mann gesehen als dein Bruder. \label{intperftfullextrnom}\\
I have.\textsc{1sg} a.\textsc{m} taller.\textsc{m} man seen as your.\textsc{m.nom} brother\\
\glt `I have seen a taller man than your brother.'
\ex \gll Ich habe einen größeren Mann gesehen als deinen Bruder. \label{intperftfullextracc}\\
I have.\textsc{1sg} a.\textsc{m} taller.\textsc{m} man seen as your.\textsc{m.acc} brother\\
\glt `I have seen a taller man than your brother.'
\z
\z

The results for the items of the type in (\ref{intperftfullextr}) are shown in \tabref{tableintperftfullextr}.
Again, the difference between the nominative and accusative masculine remnants is significant ($p<0.0001$; 95\% confidence interval 0.9526 to 2.0474). 

\begin{table}
\begin{tabular}{l *3{S[table-format=1.2]}}
\lsptoprule
{} & {Case-ambiguous} & {Nominative} & {Accusative}\\
\midrule
Mean & 4.02 & 2.63 & 4.13\\
Median & 5 & 2 & 5\\
Variance & 1.69 & 1.90 & 1.69\\
Standard deviation & 1.31 & 1.39 & 1.31\\
\lspbottomrule
\end{tabular}
\caption{Internal reading, full DP remnants, perfective condition, extraposed}
\label{tableintperftfullextr}
\end{table} 


Consider now the following items for the perfective condition with an internal reading, involving extraposed pronominal remnants:

\ea Kontext: [Deine Schwester / Dein Bruder] ist ganz schön groß, jedoch nicht [die größte Frau / der größte Mann] der Welt. \label{intperftproextr}\\
`Context: [Your sister / Your brother] is fairly tall but not [the tallest woman / the tallest man] in the world.'\\
\ea \gll Ich habe eine größere Frau gesehen als sie. \label{intperftproextrambig}\\
I have.\textsc{1sg} a.\textsc{f} taller.\textsc{f} woman seen as she.\textsc{nom/acc}\\
\glt `I have seen a taller woman than her.'
\ex \gll Ich habe einen größeren Mann gesehen als er. \label{intperftproextrnom}\\
I have.\textsc{1sg} a.\textsc{m} taller.\textsc{m} man seen as he.\textsc{nom}\\
\glt `I have seen a taller man than him.'
\ex \gll Ich habe einen größeren Mann gesehen als deinen Bruder. \label{intperftproextracc}\\
I have.\textsc{1sg} a.\textsc{m} taller.\textsc{m} man seen as he.\textsc{acc}\\
\glt `I have seen a taller man than him.'
\z
\z

The results for the items of the type in (\ref{intperftproextr}) are shown in \tabref{tableintperftproextr}.
Again, the difference between the nominative and accusative masculine remnants is significant ($p<0.0001$; 95\% confidence interval 2.2557 to 3.0443). Just like in the non-extraposed conditions, the difference between the full DP versus pronominal remnants is significant in the nominative case ($p=0.0023$; 95\% confidence interval 0.3012 to 1.3388), the more explicit pronominal remnant being less acceptable, while there is no significant difference in the preferred accusative case.

\begin{table}
\begin{tabular}{l *3{S[table-format=1.2]}}
\lsptoprule
{} & {Case-ambiguous} & {Nominative} & {Accusative}\\
\midrule
Mean & 3.88 & 1.81 & 4.46\\
Median & 4 & 1 & 5\\
Variance & 1.44 & 1.32 & 0.54\\
Standard deviation & 1.21 & 1.16 & 0.74\\
\lspbottomrule
\end{tabular}
\caption{Internal reading, pronominal remnants, perfective condition, extraposed}
\label{tableintperftproextr}
\end{table} 


In addition, I examined patterns with particle verbs (where the particle remains in the base position; the \textit{als} + DP sequence can either precede or follow the particle). The conclusions pointed out above, however, remain across these conditions. As eventual further differences are not relevant for the purposes of the present discussion, I will not evaluate the results for these conditions here.

\subsection{Discussion} \label{sec:6discussion}
When interpreting the acceptability ratings reported above, it should be clear that they do not straightforwardly translate into grammaticality judgements. Moreover, the ratings for the case-unambiguous items indicate that the same kind of sentence can be acceptable in one condition and not acceptable in another, just as discussed for (\ref{germanmasc}) already. In other words, the constructions under scrutiny are in fact all grammatical in isolation and their acceptability in a given context depends on whether they are compatible with that context.

That being said, the following can be established regarding the case-distinction patterns (the conclusions regarding the underlying syntax carry over to case-ambiguous patterns as well): (i) the clear case distinction established based on simple grammaticality judgements (see the discussion regarding (\ref{germanmasc}) at the beginning of this section) is confirmed, and (ii) there are significant differences between full DP and pronominal remnants (which are not readily determined without experimental investigation). The fact that accusative remnants receive very low ratings with the DP-external reading is expected on the basis of the baseline (predicative) pattern in German, given in (\ref{germancomp}), since the accusative case is not available as a default case (that is, in the absence of an overt case marker). This suggests that, just like in (\ref{englishext}) for English above, the DP-external reading involves an underlying TP. Taking the sentence in (\ref{extpresfullnom}), this is illustrated as follows:

\ea \gll Ich male ein schöneres Bild als [\textsubscript{FP} mein Bruder [\textsubscript{TP} \textit{t} ein x-schönes Bild malt]].\\
I paint.\textsc{1sg} a.\textsc{n} nicer.\textsc{n} picture as {} my.\textsc{m.nom} brother {} {} a.\textsc{n} x-nice.\textsc{n} picture paints\\
\glt `I am painting a nicer picture than my brother.'
\z

An underlying PredP would be problematic for semantic reasons, as that configuration would reconstruct a completely different meaning (`I am painting a nicer picture than my brother is x-nice'), which is infelicitous.

On the other hand, accusative remnants are judged to be better than nominative remnants in the DP-internal reading, though nominative remnants are slightly better in this condition than accusative remnants in the DP-external reading. The high acceptability and the preference for accusative remnants are not expected on the basis of the baseline (predicative) pattern in (\ref{germancomp}), since German predicative comparatives appear to either involve a full tensed TP underlyingly or to use the default nominative case in the absence of a case assigner, both possibilities resulting in a surface nominative remnant. Nevertheless, the accusative remnant in the DP-internal condition can be modelled in a way similar to (\ref{englishint}) above for English, namely that the subclause involves a tenseless PredP rather than a TP. Taking the sentence in (\ref{intpresfullacc}), this is illustrated as follows: 

\protectedex{
\ea \gll Ich kenne einen größeren Mann als [\textsubscript{FP} deinen Bruder [\textsubscript{PredP} \textit{t} x-groß BE]]. \label{germanstructure}\\
I know.\textsc{1sg} a.\textsc{m} taller.\textsc{m} man as {} your.\textsc{m.acc} brother {} {} x-tall BE\\
\glt `I know a taller man than your brother.'
\z
}

As discussed above, the PredP is headed by a tenseless abstract element (in line with what \citealt{dendikken2006} refers to as relators); I take PredP to be head-final in German (cf. \citealt{salzmannschaden2019}). By way of looking at (\ref{germanstructure}), it is evident that there is no nominative case assigner in the subordinate clause; in other words, when examining the comparative subclause here, it is reasonable to claim that the remnant DP can be viewed as caseless. If so, however, we would actually expect the nominative case to appear, as this is the default case in German.

This strongly suggests that the accusative case comes from somewhere else, namely from outside the clause: were the accusative available as a default or were the complementiser \textit{als} a potential case assigner, (\ref{mich}) should be possible.

I assume, in line with \citet{bacskaiatkari2017atoh}, that the accusative case is ultimately governed by the matrix verb (\textit{kenne} in (\ref{germanstructure}) above) and the caseless remnant receives the same morphological case as the matrix direct object (\textit{einen größeren Mann} in (\ref{germanstructure}) above). Indeed, the remnant is taken to be part of the direct object DP inasmuch as it is extraposed to the edge of the DP but not (necessarily) beyond that: recall that in (\ref{germanfem}) and (\ref{germanmasc}) above, the \textit{als}\,+\,DP sequence precedes the lexical verb.

As discussed in \chapref{ch:5}, the comparative subclause (of the category CP) is base-generated as a complement of the comparative (Compr) head (taken to be Deg in various earlier analyses). In predicative comparatives, the underlying structure is reflected by the surface word order:

\ea \gll Ralf ist [größer als ich]. \label{germanpred}\\
Ralph is \phantom{[}taller as I\\
\glt `Ralph is taller than me.'
\z

In this case, the Compr head is lexicalised by -\textit{er} and the comparative subclause \textit{als ich} follows the matrix degree expression immediately. The entire degree expression forms a constituent; this can also be seen in patterns where it is fronted to [Spec,CP] as a whole:

\ea \gll {[}Viel größer als ich] war sie nicht.\\
\phantom{[}much taller as I was.\textsc{3sg} she not\\
\glt `She was not much taller than me.'
\z

In other cases, however, the subclause has to be extraposed; note that this is not a specific property of German attributive comparatives but can be observed more generally and it is apparently related to how phases are spelt out (see the discussion in \citealt[53--56]{bacskaiatkari2018langsci} and references there). The underlying order given in (\ref{germanpred}) is not possible in German attributive comparatives (irrespective of the reading):

\ea[*]{\gll Ich kenne eine [größere als meine Mutter] Frau.\\
I know.\textsc{1sg} a.\textsc{f} \phantom{[}taller.\textsc{f} as my.\textsc{f} mother woman\\
\glt `I know a taller woman than my mother.'}
\z

As can be seen, the degree expression containing the comparative subclause is not licensed in this case. The comparative subclause must be extraposed:

\ea \gll Ich kenne [eine größere \textit{t} Frau [als meine Mutter]].\\
I know.\textsc{1sg} \phantom{[}a.\textsc{f} taller.\textsc{f} {} woman \phantom{[}as my.\textsc{f} mother\\
\glt `I know a taller woman than my mother.'
\z

Still, the DP containing the degree expression forms a constituent with the comparative subclause, as also indicated by examples like the following, where the entire constituent is fronted to the [Spec,CP] position:

\ea \gll {[}Eine klügere Frau als sie] wäre vorsichtiger gewesen.\\
\phantom{[}a.\textsc{f} smarter.\textsc{f} woman than she] be.\textsc{cond.3sg} more.cautious been\\
\glt `A smarter woman than her would have been more cautious.'
\z

This is possible only if the extraposed subclause can be attached to the DP. Since the DP is assumed to be a phase (see \citealt{chomsky2008}, citing \citealt{svenonius2004} and \citealt{hiraiwa2005diss}), it is not surprising that extraposition can target the DP-edge.

In cases like (\ref{germanstructure}), what happens is that the remnant of the comparative subclause containing a PredP as the complement of F (instead of a TP) is caseless in the given position; it is part of the matrix DP as it is extraposed onto the DP edge. This DP receives accusative case as it is the direct object of the matrix verb: the accusative case is extended onto the caseless remnant. Note that this is strictly subject to both conditions, i.e. (i) that it is caseless in the syntactic derivation and (ii) that it forms a constituent with an accusative-marked DP. In all other configurations, it should appear as nominative, either because it receives nominative case in a TP or because it is assigned the default nominative case: otherwise, predicative comparatives like (\ref{mich}) should be licensed.

This suggests a late insertion approach to the insertion of lexical items in the derivation, as in Distributed Morphology. In Distributive Morphology, it is assumed that terminal nodes in all components of grammar except for phonology consist only of morphosyntactic/semantic features: in other words, they lack phonological features (\citealt{hallemarantz1993}, \citealt{embicknoyer2007}). The terminal nodes used to build the underlying structure (in traditional notions: D-structure) come from the component called the Lexicon: lexical items are essentially viewed as feature bundles. A functional morpheme is a feature bundle consisting of syntactic/semantic features, and a content morpheme is a category-neutral lexical root. Phonological content is added only after Spell-Out, in the Morphological Structure (MS) from the component termed Vocabulary. The inserted vocabulary items need not match the original terminal nodes in a one-to-one fashion: terminal nodes may undergo various operations (such as morphological merger and fusion) that result in cumulative expression. Further, while inherent inflection is added into the syntax, contextual inflection (such as agreement) is added also in the Morphological Structure only (referred to as ``ornamental'' by \citealt{embicknoyer2007} precisely because such inflection does not affect the semantics). 

For our purposes, what matters is that since case assignment is contextual inflection, it does not affect the semantics either and can therefore be added late. Note that this of course presupposes a certain syntactic configuration licensing the individual case values. Crucially, however, while the syntactic configuration is such that the direct object DP can only receive accusative case, the caseless remnant could in principle be assigned the default nominative if it were not part of the larger DP containing both the direct object DP and the remnant. Indeed, as the ratings for the nominative remnants with the DP-internal reading indicate, which are higher than the ratings for the accusative remnants with the DP-external reading, this scenario is not altogether excluded. In other words, while an underlying nominative subject cannot be ``re-assigned'' accusative case, an underlying caseless subject can be assigned accusative case if it falls into the accusative domain of the direct object.

It appears that ellipsis, as defined by the [E] feature on F, precedes case assignment: this is straightforward in a late insertion approach, as the point is precisely that the string generated by the syntactic component lacks phonological features: the part marked to be elided already receives no phonological realisation. In the complex DP containing the direct object and the remnant, then, no predicate is present any more.

What matters for the present discussion is not so much the morphology--syntax interface, though, but rather what implications these findings have for the FP. Just as with the sluicing patterns discussed in \sectref{sec:6sluicing}, it is evident that the FP allows an underlying syntactic configuration that does not match the overt configuration and does not parallel the antecedent clause syntactically. The key assumption is that these underlying configurations are fully recoverable; in addition, they involve less structure than the overt counterparts. This suggests that recoverability and the economy of derivation interact in a way that actually favours the emergence of more condensed syntactic patterns in elliptical clauses. In this way, the FP has an important impact on the syntax since it has an immediate effect on the projections involved in the structure of the elliptical clause, which goes well beyond the addition of the FP projection itself.

\begin{sloppypar}
Regarding German comparatives, the presence of the FP licenses a tenseless PredP as a complement in certain attributive configurations; this projection would not be possible with the same properties overtly. As a consequence, the subject remnant can be accusative in these instances even though subject accusative remnants are otherwise not possible. In fact, this is not limited to German but it can be observed in Icelandic as well. Icelandic is a language that has a nominative/accusative system (see \citealt[115--116]{hroarsdottir2000}), contrary to the Mainland Scandinavian nominative/oblique system. Just like in German, the subject remnant of a predicative comparative is in the nominative:\footnote{I owe many thanks to Jóhannes Gísli Jónsson for his help with the Icelandic data.}
\end{sloppypar}

\ea \label{icelandicpred}
\ea[]{\gll Egill er hærri en þú. \label{icelandic}\\
Egill is taller than you.\textsc{nom}\\
\glt `Egill is taller than you.'}
\ex[*]{\gll Egill er hærri en þig\\
Egill is taller than you.\textsc{acc}\\
\glt `Egill is taller than you.'}
\z
\z

Just like in German, the potentially ambiguous attributive cases are disambiguated by the nominative/accusative distinction on the remnant:

\ea \label{icelandicnomacc}
\ea \gll Ég sá hærri konu en móðir mín. \label{icelandicnom}\\
I saw.\textsc{1sg} taller woman than mother.\textsc{nom} my.\textsc{nom}\\
\glt External reading: `I saw a taller woman than my mother saw.'
\ex \gll Ég sá hærri konu en móður mína. \label{icelandicacc}\\
I saw.\textsc{1sg} taller woman than mother.\textsc{acc} my.\textsc{acc}\\
\glt Internal reading: `I saw a taller woman than my mother is.'
\z
\z

As indicated, the DP-external reading is associated with the nominative remnant, as in (\ref{icelandicnom}), while the DP-internal reading is associated with the accusative remnant, as in (\ref{icelandicacc}). The latter would not be expected on the basis of (\ref{icelandicpred}) but it is expected based on the German experiment data discussed in \sectref{sec:6results} and \sectref{sec:6resultsperfective}.

In languages like English, accusative remnants are available in basic predicative structures. Recall the basic pattern from (\ref{thanime}), repeated here as (\ref{thanimerepeat}):

\ea \label{thanimerepeat}
\ea[?]{Ralph is taller than \textbf{I}. \label{thanirepeat}}
\ex[]{Ralph is taller than \textbf{me}. \label{thanmerepeat}}
\z
\z

Cases like (\ref{thanirepeat}) work in the same was as ordinary tensed comparatives like (\ref{englishext}) and can be modelled as follows:

\ea Ralph is taller than [\textsubscript{FP} I \sout{[\textsubscript{TP} \textit{t} am x-tall]}].
\z

This is the same kind of construction as for German predicative comparatives; note, however, that a German predicative comparative without an overt case assigner would mark the remnant as nominative since the default case in German is the nominative. In English, a non-tensed clause gives an accusative remnant per default case:

\ea Ralph is taller than [\textsubscript{FP} me \sout{[\textsubscript{PredP} \textit{t} BE x-tall]}].
\z

The difference from the German cases like (\ref{germanstructure}) lies in the context-independent nature of the accusative case in English: there needs to be no accusative case licensor since the accusative appears per default. This is contingent upon the assumption that the complement clause is a PredP and not a TP, in line with the suggestion of \citet{pancheva2006}, who also assumed that reduced clausal comparatives can be smaller (in her analysis, small clauses). Unlike \citet{pancheva2006}, however, the present analysis assumes that the PredP-structure is not unprecedented in elliptical configurations and is not an idiosyncratic property of comparatives.

In English, however, the availability of such caseless remnants does not appear to be contingent upon the presence of a genuine PredP. Consider the following example for a non-elliptical nominal comparative:

\ea Peter has more cats than Mary has dogs.
\z

In this case, only the degree marker (\textit{x-many}) is covert but since it it zero anyway (\citealt[78--80]{bacskaiatkari2018langsci}), there is no reason to assume any designated ellipsis process. The structure can therefore be modelled as follows:

\ea Peter has more cats than [\textsubscript{TP} Mary has (x-many) dogs].
\z

Once the direct object is not contrastive, the comparative subclause can be elliptical in nominal comparatives as well, producing (\ref{nominal}):

\ea Peter has more cats than Mary (has). \label{nominal}
\z

With an underlying TP, there is a straightforward ellipsis process involving the FP:

\ea Peter has more cats than [\textsubscript{FP} Mary \sout{[\textsubscript{TP} \textit{t} has (x-many) dogs]}].
\z

If this were the only possible derivation, then we might in principle expect the remnant DP to be in the nominative, since as a subject of a TP it receives nominative case. This is, however, not always the case:

\ea Peter has more cats than her. \label{her}
\z

Taking the analysis for predicative constructions above, one might be tempted to carry over the conclusions and suggest that there is a PredP instead of a TP in such cases:

\ea[\#]{Peter has more cats than [\textsubscript{FP} her \sout{[\textsubscript{PredP} \textit{t} BE x-many cats]}].}
\z

As indicated, however, this is infelicitous and does not match the purported output in (\ref{her}). As there is no attributive adjective in the matrix clause in the way it was attested in the attributive comparative structures, it is evident that no predicative subclause can be reconstructed in an analogous way. In other words, PredP is not an available option (either in the sense taken above or in the sense of \citealt{pancheva2006}).

I follow \citet{pancheva2006} in assuming that English \textit{than}-comparatives are underlyingly clausal in all cases, since they semantically pattern with ordinary clausal comparatives and not with genuine phrasal comparatives; this is also evidenced by the fact that in the attributive constructions discussed in \sectref{sec:6results} above for German, they allow the external reading, which is banned in genuine phrasal comparatives involving PPs or inherently case-marked DPs (\citealt{bacskaiatkari2017icgl, bacskaiatkari2017atoh}). In other words, the \textit{than}-XP is not a PP and \textit{than} is not a case assigner.

If so, however, the question arises what sort of projection we have in the \textit{than}-clause below the FP. One way to think of it is that there is, in fact, none: the FP hosts the remnant in its specifier and the complement, consisting of a quantified DP, is elided since the F head contains the [E] feature. The structure is represented in \figref{treec}.

\begin{figure}
\caption{Deriving the accusative remnant in English} 
\label{treec}
\begin{forest} baseline, qtree
[CP
	[C$'$
		[C
			[than]
		]
		[FP
			[DP [her,roof]]
			[F$'$ [F [{[}E{]}]] [DP [x-many cats,roof]]]
		]
	]
]
\end{forest}
\end{figure}

This, of course, involves a reinterpretation of the original FP, as the properties of the original FP and the PredP are conflated: the remnant does not undergo movement but is merged directly as a specifier in the FP, and the quantified DP is merged as a complement. The F head is more abstract than the Pred head, since it does not actually introduce a predicative relation between the two; this goes beyond the small clause analysis of \citet{pancheva2006} as well, which includes predicative comparatives only. 

In cases like (\ref{her}) the complement of F is expected to be zero, since it is logically identical to its antecedent in the matrix clause (\citealt[78--80]{bacskaiatkari2018langsci}). The point is, however, that the configuration is not available in a non-elliptical version in cases where the complement would have to be overt either:

\ea[*]{Peter has bought more cats than her dogs.}
\z

The complement of F cannot be contrastive as it could not be elided in that case.

The role of the FP is altogether similar to the FP in Hungarian, as discussed in \sectref{sec:6lower}, in that the projection is not restricted to ellipsis: indeed, the specifier serves to host a constituent that bears main stress in both cases and the presence of the [E] feature on F is incompatible with the marking of finiteness -- which, however, has different conclusions for the two language types, as the English (and more generally, Germanic) FP is restricted to elliptical and non-finite contexts and is not associated with focus-marking in any other contexts. It is very likely, though, that elliptical patterns have a considerable impact on the establishment of the FP as a lower functional periphery (cf. \citealt{bacskaiatkaridekany2014oup}) and it remains to be investigated whether comparable processes can be detected in English (and across Germanic). What matters for us here is that the FP, taken as a functional projection that can host constituents not specified for clause-typing in its specifier and containing the [E] feature as a head, can have an impact on the syntactic architecture of (embedded) clauses in a way that goes beyond its originally assumed functions.

\section{Summary} \label{sec:6summary}
While most of this book examined finite, non-elliptical clauses, concentrating on clause-typing elements in the CP-periphery, this chapter investigated in\-for\-ma\-tion-struc\-tur\-al movement to the left periphery, as well as clausal ellipsis. Regarding this, it was shown that information-structural notions are not directly built into the narrow syntax in terms of designated projections and distinguishing information-structurally marked constituents is primarily a prosodic matter in the languages under scrutiny. Regarding ellipsis, it was shown that information-structural notions are crucial and that not only the high CP-periphery but also lower functional projections are relevant. The feature-based model proposed here can successfully integrate these questions as well, making also interesting predictions in terms of the realisation of remnants. 
