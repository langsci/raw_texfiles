\chapter{A feature-based approach to functional left peripheries} \label{ch:2}
\section{Introduction} \label{sec:2introduction}
%\chaptermark{A feature-based approach}
In this chapter, I am going to present the basic assumptions concerning a minimal, feature-based approach to the syntax of functional left peripheries, showing that the proposed analysis applies to various clause types, in each case correctly predicting the surface order of clause-typing elements appearing in combinations. Since the relevant combinations are restricted to embedded clauses in Germanic languages, this chapter will be focusing on subordinate clauses, even though, as will be indicated, the analysis is also applicable to main clauses. In particular, I will be arguing against cartographic approaches, showing that clause-typing elements appearing on functional left peripheries are not in a one-to-one relationship with syntactic features, and the assumption that there are designated projections for the various semantic properties is fundamentally flawed. Instead, I propose that functional left peripheries are as minimal as possible, and multiple projections are generated when the relevant semantic properties cannot be marked in a single projection; whether this is the case is ultimately dependent on the lexical properties of the individual clause-typing elements. To put the analysis into an appropriate context, I am first going to review some previous proposals of relevance: the papers discussed here are not meant to be a representative summary of the state of the art but they are selected analyses that have been particularly influential and/or are of particular interest for the analysis pursued here.

This chapter is structured as follows. Section \ref{sec:2previous} provides an overview of some previous accounts. Section \ref{sec:2introducing} introduces the basic ideas regarding the flexible approach to left peripheries put forward in this book. This basic proposal will be refined with more details in the subsequent sections: \sectref{sec:2interrogatives} discusses embedded interrogatives, \sectref{sec:2relative} discusses relative clauses, and \sectref{sec:2degree} discusses embedded degree clauses. These clauses types will be dealt with in more details in the rest of this book.

\section{Previous accounts} \label{sec:2previous}
\subsection{The problems to be discussed} \label{sec:2problems}
In current minimalist theory, the Complementiser Phrase (CP) is responsible for typing clauses and for encoding finiteness in finite clauses.\footnote{See, for instance, \citet[283]{rizzi1997}, for anchoring finiteness in the CP-system. Note that finiteness is ultimately inherited from the inflectional system (see \citealt{chomskylasnik1977} and \citealt{denbesten1983}). This also means that a clause can be finite without a CP layer, as is the case for English main clause declaratives, which are standardly assumed to be TPs.} Apart from complementisers, various operators can appear in this domain. Consider:

\ea
\ea I wonder \textbf{if} Ralph has arrived. \label{englishif}
\ex I wonder \textbf{whether} Ralph has arrived. \label{englishwhether}
\z
\z

In (\ref{englishif}), the element \textit{if} is a complementiser and it types the subordinate clause as interrogative. In (\ref{englishwhether}), there is no overt complementiser but the operator \textit{wheth\-er} is present. In such cases, it is generally assumed that the zero complementiser types the clause, yet a sound model of the CP-periphery must also clarify the role of the overt operator in (\ref{englishwhether}), especially because its appearance in dialects like Standard English is tied to the absence of the overt complementiser:

\ea[*]{I wonder \textbf{whether if} Ralph has arrived. \label{whetherifch2}}
\z

On the other hand, the CP is not restricted to hosting a single overt element: depending on the particular construction and the dialect, multiple elements may appear in the CP-domain. This is illustrated by (\ref{englishdfc}) for non-standard English and by (\ref{norwegiandfc}) for Norwegian\footnote{The Norwegian data stem from the cross-Germanic survey of \citet[175]{bacskaiatkaribaudisch2018}. Both of the informants marked the sentence in (\ref{norwegiandfc}) as grammatical.}:

\ea \label{dfc}
\ea[\%]{ I wonder \textbf{which book that} Ralph is reading. \label{englishdfc}}
\ex[]{ \gll Peter spurte \textbf{hvem} \textbf{som} likte bøker. \label{norwegiandfc}\\
Peter	asked.\textsc{3sg} who	that liked books\\
\glt `Peter asked who liked books.'}
\z
\z

A proper formal account of the CP-domain must be able to condition when multiple overt elements are allowed and when not. Further, it must be clarified whether the appearance of several overt elements requires multiple CP projections, and in cases where it does, how word order restrictions can be modelled. The generation of multiple functional layers is in principle possible, yet it should be appropriately restricted to exclude the generation of superfluous layers that are empirically not motivated. This question is likewise relevant in cases involving a single overt C-element, since then the question arises whether and to what extent covert elements and phonologically invisible projections are present.

Apart from the exact position of various elements in the CP, their function(s) must also be addressed. For instance, interrogative complementisers regularly encode finiteness as well, imposing finiteness restrictions on the complement TP. Consider:

\ea \label{ifwhether}
\ea[]{I don't know \textbf{if} I should call Ralph. \label{iffinite}}
\ex[]{I don't know \textbf{whether} I should call Ralph. \label{whetherfinite}}
\ex[*]{I don't know \textbf{if} to call Ralph. \label{ifnonfinite}}
\ex[]{I don't know \textbf{whether} to call Ralph.  \label{whethernonfinite}}
\z
\z

\begin{sloppypar}
In (\ref{iffinite}), the complementiser \textit{if} introduces a finite embedded interrogative clause, and as the ungrammaticality of (\ref{ifnonfinite}) shows, it is incompatible with a non-finite clause, suggesting that it encodes finiteness apart from the interrogative property. By contrast, the operator \textit{whether} is compatible with both a finite clause, as shown in (\ref{whetherfinite}), and with a non-finite clause, as shown in (\ref{whethernonfinite}), indicating that the overt marking of interrogativity is not incompatible with a non-finite clause in English. Since \textit{whether} is not specified for finiteness, it should be clear that finiteness is specified by some other element in (\ref{whetherfinite}); the question is whether there is a separate element encoding finiteness in (\ref{iffinite}) as well and, if so, how the restriction of \textit{if} to finite clauses can be explained.
\end{sloppypar}

Finally, the function(s) of various left-peripheral elements must be clarified also because there are some non-trivial combinations in which elements seem to be largely similar, as in the non-standard German example in (\ref{alswie}) below:

\ea
\ea[\%]{\gll Ralf ist größer \textbf{als} \textbf{wie} Maria. \label{alswie}\\
Ralph is taller than as Mary\\
\glt `Ralph is taller than Mary.'}
\ex[]{\gll Ralf ist größer \textbf{als} Maria. \label{als}\\
Ralph is taller than Mary\\
\glt `Ralph is taller than Mary.'}
\ex[\%]{\gll Ralf ist größer \textbf{wie} Maria. \label{wie}\\
Ralph is taller as Mary\\
\glt `Ralph is taller than Mary.'}
\ex[]{\gll Ralf ist so groß \textbf{wie} Maria. \label{wieequat}\\
Ralph is so tall as Mary\\
\glt `Ralph is as tall as Mary.'}
\z
\z

In (\ref{alswie}), the elements \textit{als} and \textit{wie} both seem to mark the comparative nature of the clause, whereby single \textit{als} is the comparative particle in Standard German comparatives, see (\ref{als}), and single \textit{wie} is the comparative particle in equatives, see (\ref{wie}), and in certain dialects also in comparatives, see (\ref{wieequat}). In such cases, the question is to what extent there is genuine doubling at hand and how it can be modelled.

\subsection{The cartographic approach -- \citet{rizzi1997, rizzi2004}} \label{sec:2rizzi}
I will start reviewing the relevant literature with Rizzi's work, since it is generally taken to be the foundation of cartographic approaches.\footnote{The original model was extended by later work by several scholars working in the cartographic framework, such as \citet{frascarelli2000, frascarelli2008}, \citet{paoli2003diss, paoli2007}, \citet{benincapoletto2004}, \citet{polettopollock2004}, \citet{beninca2006}, \citet{frascarellihinterhoelzl2007}, \citet{cinquerizzi2008, cinquerizzi2009}, \citet{bocci2013}, \citet{polettozanuttini2013}, \citet{bianchiboccicruschina2017, bianchiboccicruschina2018}, \citet{boccicruschina2018}, \citet{rizzibocci2017}, \citet{boccicruschinarizzi2021}, \citet{boccibianchicruschina2021}. As these works do not fundamentally differ from the original idea in spirit (in fact, they explicitly adopt Rizzi's framework), the concerns expressed here in connection with \citet{rizzi1997, rizzi2004} also apply to them. The aim of this section is not to provide an overview of the cartographic approach but rather to focus on the motivating factors underlying the original idea, as well as potential problematic points.} While his model was primarily developed for Romance languages (and for Italian in particular), the model implies a universal applicability; indeed, the Germanic left periphery has been analysed in a (partial) cartographic fashion as well (see, for instance, \citealt{haegeman2007, haegeman2012, haegeman2013, haegeman2014, haegeman2017} and \citealt{hinterhoelzlpetrova2010, hinterhoelzlpetrova2010focus}).\footnote{While there are certainly differences in the exact combinations that are attested in the two language groups, the similarities are altogether overwhelming. In both Germanic and Romance, clause-typing elements such as complementisers and operators (e.g. interrogative and relative operators) can occur in the left periphery, as well as other XPs that are fronted to the CP-domain without encoding clause type. In \chapref{ch:6}, I argue that XP-fronting is largely due to an unspecified [edge] feature. In this respect, Germanic seems to be more restrictive, as there is generally only a single XP fronted to the CP (leading to the canonical V2 pattern); however, this is not necessarily the case, as will be discussed in connection with V3 patterns in \chapref{ch:3}. Note also that while the Romance left periphery appears to be able to host multiple fronted XPs, fronting is not the only option for marking information structure: in fact, as shown by \citet{sameklodovici2015} for Italian, contrastive focus occurs in situ by default.}

The basic observation underlying Rizzi's model is that while in the 1980s the layers VP, IP and CP were taken to be composed of single projections each, there is evidence for there being a more intricate structure underlying these domains, as was already established for the VP and the IP\footnote{In this book, I will restrict myself to the discussion of cartographic approaches to the CP-domain; note that such approaches have also been proposed for the IP-domain, see, for instance, \citet{cinque1999}, \citet{belletti2004}, \citet{cardinaletti2004}.} towards the end of the 1980s (\citealt[281]{rizzi1997}). Essentially, \citet{rizzi1997} assumes that the same holds for the left periphery of the clause, that is, the domain above IP.

According to \citet[283]{rizzi1997}, the CP-domain has two major functions. On the one hand, it relates the clause to the outside, that is, either to a superordinate structure or, in the case of root clauses, to the articulation of the discourse. This kind of information expresses whether the clause is, for instance, a question or a declarative, and is referred to as the clausal Type by \citet{cheng1991diss} and the specification of Force by \citet{chomsky1995}, whereby \citet[283]{rizzi1997} adopts the latter term. As pointed out by \citet[283]{rizzi1997}, ``Force is expressed sometimes by overt morphological encoding on the head (special C morphology for declaratives, questions, relatives etc.), sometimes by simply providing the structure to host an operator of the required kind, sometimes by both means''. The last option is considered to be rare by \citet{rizzi1997}, who attributes this to economy principles on representation, following \citet{cheng1991diss} among others.

On the other hand, the CP-domain has an effect on its complement domain, namely the IP, and \citet[283--285]{rizzi1997}, following \citet{holmbergplatzack1988}, among others, assumes that the CP is responsible for encoding finiteness. That is, contrary to Den \citet{denbesten1983}, \citet[283--284]{rizzi1997} claims that the C is not specified for tense as such, the selection of the C not making any selection on the particular tense (that is, whether it is present or past, etc.) but it rather encodes whether there is tense at all, correctly accounting for the observation going back to \citet{chomskylasnik1977} that in English the complementiser \textit{that} co-occurs with tensed verbs while the complementiser \textit{for} co-occurs with infinitives. Some languages replicate additional information from the IP in the CP, such as subject agreement in various Germanic varieties (\citealt{haegeman1992}, \citealt{bayer1984}, \citealt{shlonsky1993}), yet this is far from being obligatory and the exact content of replication (e.g. mood, negation) shows considerable cross-linguistic variation (\citealt{rizzi1997}). Regarding the distinction between the IP and the CP, \citet[284--285]{rizzi1997} argues that the CP cannot be regarded as an extension of the verbal domain (as opposed to the IP) since the ``inflectional'' properties expressed by C are carried rather by free functional morphemes that are more nominal than verbal (cf. the resemblance between certain demonstratives and complementisers): the CP is therefore not V-related.

Apart from Force and finiteness, \citet[285]{rizzi1997} claims that the C-system ``can have other functions which are by and large independent from selectional constraints''. For instance, sentences can have a topic--comment articulation, as in (\ref{topiccomment}), and they can also have a focus--presupposition articulation, as in (\ref{focuspresupp}), examples taken from \citet[285, ex. 1 and 2]{rizzi1997}:

\ea \label{englishrizzi}
\ea {[}Your book]\textsubscript{i}, you should give \textit{t}\textsubscript{i} to Paul (not to Bill). \label{topiccomment}
\ex {[}YOUR BOOK]\textsubscript{i} you should give \textit{t}\textsubscript{i} to Paul (not mine). \label{focuspresupp}
\z
\z

While both constructions involve fronting an element to the left periphery, they differ in their intonation and their interpretation. A topic is separated by a so-called ``comma intonation'' from the remaining part of the clause and it normally expresses ``old information, somehow available and salient in previous discourse'', whereas ``the comment is a kind of complex predicate, an open sentence predicated of the topic and introducing new information'' (\citealt[258]{rizzi1997}). A focus bears focal stress and it ``introduces new information, whereas the open sentence expresses contextually given information, knowledge that the speaker presupposes to be shared with the hearer'' (\citealt[258]{rizzi1997}). As \citet[258]{rizzi1997} points out, ``the interpretive relation of the preposed element to the open sentence is (\ldots) virtually the opposite in the two cases''. Other languages demonstrate a similar difference, notably Italian. Consider the examples taken from \citet[286, ex. 3 and 4]{rizzi1997} in (\ref{rizzilibro}) below (the glosses are mine):

\ea \label{rizzilibro}
\ea \gll Il tuo libro, lo ho letto. \label{italiantopic}\\
the.\textsc{m} your.\textsc{m} book that.\textsc{m.acc} have.\textsc{1sg} read.\textsc{ptcp}\\
\glt `Your book, I have read it.'
\ex \gll IL TUO LIBRO ho letto (, non il suo). \label{italianfocus}\\
the.\textsc{m} your.\textsc{m} book have.\textsc{1sg} read.\textsc{ptcp} {} not the.\textsc{m} his.\textsc{m.acc}\\
\glt `Your book I read (, not his).'
\z
\z

The topic--comment construction in (\ref{italiantopic}) shows Clitic Left Dislocation (CLLD), a term introduced by \citet{cinque1990}, and this involves ``a resumptive clitic coreferential to the topic'' (\citealt[285]{rizzi1997}). The focus--presupposition articulation in (\ref{italianfocus}) involves a special kind of stress (called focal stress), on the preposed element: this construction is restricted to contrastive focus in Italian,\footnote{See \citet{paoli2009} for a mroe fine-grained study of focus in varieties of Italian.} while in other languages fronting is also possible with other kinds of foci (\citealt[286]{rizzi1997}, see \citealt{tsimpli1995} for Greek and \citealt{horvath1986}, \citealt{ekiss1987} and \citealt{brody1990} for Hungarian).

\citet[286]{rizzi1997} assumes that both constructions involve a designated left-peripheral position, TopP and FocP, respectively, which conform to the X-bar schema (whereby the X-bar schema is not necessarily taken to be a primitive but as derived from more elementary principles, in the vein of \citealt{kayne1994} and \citealt{chomsky1995}).\footnote{The same applies to other discourse-related left-peripheral positions (\citealt[237]{rizzi2004}).} Accordingly, a TopP hosts the topic in its specifier and the complement of the Top head is the comment, while the FocP hosts the focussed constituent in its specifier and the complement of the Foc head expresses presupposed information (\citealt[286--287]{rizzi1997}). Further, \citet[286]{rizzi1997} assumes that the Top head defines a ``higher predication'', that is, ``a predication within the Comp system'', and its function is analogous to that of the AgrSP in the IP system, with the important difference that the specifier of TopP is an A$'$-position (\citealt[286]{rizzi1997}). Regarding FocP, \citet[287]{rizzi1997} suggests that foci move to the specifier of this projection either before spellout or at LF, whereby the second type is an instance of lower focalisation and can be observed in languages like Italian, where focal stress can appear on an element remaining in situ (cf. \citealt{antinuccicinque1977}, \citealt{calabrese1982}, \citealt{cinque1993}, \citealt{bellettishlonsky1995}).

While in English and Italian the Top and Foc heads are phonologically null, there are languages where topic and focus markers are located here (\citealt[287]{rizzi1997}, \citealt[238]{rizzi2004}), such as the topic particle \textit{ya} and the focus particle \textit{w\`{e}} in Gungbe (see \citealt{aboh1999diss}).\footnote{This is illustrated by (\ref{gungbe}) below (\citealt[238, ex. 47]{rizzi2004}, cf. \citealt{aboh1999diss}):

\ea \gll \ldots do Kofi ya gankpa me we kponon le su I do \label{gungbe}\\
\phantom{\ldots}that Kofi \textsc{top} prison in \textsc{foc} policemen \textsc{pl} shut him there\\
\glt `\ldots that Kofi was shut into PRISON by policemen'
\z

\citet[238]{rizzi2004} concludes ``that other languages use analogous structures with null heads'' and they differ ``from Gungbe and similar languages in the morphological manifestation of a fundamentally uniform syntactic system''.} The heads are also relevant in terms of specifier--head agreement: \citet[287]{rizzi1997} assumes that topicalised and focussed constituents are equipped with topic and focus features which must be checked against a head, just like in the case of interrogative and negative features. \citet[287--288]{rizzi1997} assumes that TopP and FocP are integrated into the C-system and are present in all non-truncated clauses; however, if there is no topic or focus to be fronted, these layers are not activated. They are always located in between ForceP and FinP, since these two ``must terminate the C system upward and downward, in order to meet the different selectional requirements and properly insert the C system in the structure'' (\citealt[288]{rizzi1997}). Ultimately, \citet[297, ex. 41]{rizzi1997} suggests the structure given in \figref{rizzitree} for the split CP.

\begin{figure} 
\caption{The cartographic left periphery} \label{rizzitree}
\begin{forest} baseline, qtree
[ForceP
	[\phantom{xxx}]
	[Force$'$ 
		[Force] 
		[TopP*
			[\phantom{xxx}]
			[Top$'$ [Top] [FocP [\phantom{xxx}] [Foc$'$ [Foc] [TopP* [\phantom{xxx}] [Top$'$ [Top] [FinP [\phantom{xxx}] [Fin$'$ [Fin] [IP]]]]]]]]
		]
	]
]
\end{forest}
\end{figure}

The star indicates that TopPs are iterable; otherwise, the order of the phrases is fixed (\citealt[288--298]{rizzi1997}). The ordering restrictions are based on the observed patterns in Italian (and, to a minor extent, other languages, mainly English).\largerpage[-1]

In all the examples provided by \citet{rizzi1997}, either only the Force or only the Fin head is filled by overt material but not the two at the same time. Indeed, \citet{rizzi1997} uses examples only from Germanic and Romance languages, and as \citet[237]{rizzi2004} points out, ``Romance and Germanic typically overtly express type Force head in finite clauses'' (take, for instance, Italian \textit{che} `that' or English \textit{that} introducing embedded declarative clauses), but it is possible that Fin is expressed overtly, as with prepositional complementisers like Italian \textit{di} `of' in Romance in non-finite clauses.\footnote{Consider the following example, taken from \citet[288, ex. 10b]{rizzi1997}:

\ea \gll Credo [\textbf{di} apprezzare molto il tuo libro].\\
believe.\textsc{1sg} \phantom{[}of appreciate.\textsc{inf} much the.\textsc{m} your.\textsc{m} book\\
\glt `I believe to appreciate your book very much.'
\z

\citet{rizzi1997} assumes that \textit{di} in such cases is in Fin.} However, ``Celtic languages like Irish appear to normally express Fin in finite clauses as well'', so the complementiser \textit{go} `that' follows other material in the left periphery (\citealt[237]{rizzi2004}, following \citealt{mccloskey1996} and \citealt{roberts2004}). Consider (\ref{irishex}) from Irish (\citealt[237, ex. 45]{rizzi2004}):

\ea \gll Is do\'iche [faoi cheann c\'upla l\'a [go bhf\'eadfai\'i imeacht]] \label{irishex}\\
is probable \phantom{[}about end couple day \phantom{[}that could leave\\
\glt `It is possible to leave after a couple of days.'
\z

Apart from patterns involving an overt Fin head in finite clauses, there are languages such as Welsh that allow both Force and Fin to be lexicalised (\citealt[237]{rizzi1997}, quoting \citealt{roberts2004}). This is illustrated by the following example (taken from \citealt[122, ex. 8]{roberts2005}, identical to the example quoted by \citealt[237]{rizzi1997}), where both \textit{mai} and \textit{a} are overt:

\ea \gll Dywedais, i [mai 'r dynion fel arfer a [werthith y ci]]. \label{welsh}\\
say I \phantom{[}that the men as usual that \phantom{[}sell the dog\\
\glt `I said that it’s the men who usually sell the dog.'
\z

Again, the two clause-typing heads Force and Fin are separated by topics.

There are several important differences between topics and foci, which affect not only their interpretation but also their syntactic behaviour. First, as \citet[289--290]{rizzi1997} shows, while topics ``can include a resumptive clitic within the comment'', foci are incompatible with resumptive clitics (see \citealt{cinque1990} regarding foci). Second, topics never give rise to Weak Cross-Over effects, while such effects are detectable with foci (\citealt[290]{rizzi1997}, cf. \citealt{culicover1992} regarding English foci). Third, bare quantificational elements cannot be topics but they can be foci (\citealt[290]{rizzi1997}). These first three differences can be traced back to the basic difference that focus is quantificational, while topic is not (\citealt[291--295]{rizzi1997}, based on \citealt{cinque1990}). Fourth, while multiple topics can be fronted, there is only one structural focus position (\citealt[290--291]{rizzi1997}, \citealt{beninca1988}). \citet[295--300]{rizzi1997} suggests that this is due to an interpretive distinction between the constructions. Fifth, a \textit{wh}-operator in main clause questions is compatible with a preceding topic but not with focus (\citealt[291, ex. 24a and 25a]{rizzi1997}):

\ea \label{gianni}
\ea[]{\gll A Gianni, che cosa gli hai detto?\\
to Gianni what thing he.\textsc{dat} have.\textsc{2sg} said.\textsc{ptcp}\\
\glt `To Gianni, what did you tell him?'}
\ex[*]{\gll A GIANNI che cosa hai detto (,non a Piero)?\\
to Gianni what thing have.\textsc{2sg} said.\textsc{ptcp} \phantom{(,}not to Piero\\
\glt `What did you tell GIANNI (, not to Piero)?'}
\z
\z

By contrast, both topics and foci are compatible with relative operators (\citealt[291]{rizzi1997}).\footnote{This is illustrated by the examples in (\ref{topicrel}) and (\ref{focusrel}) below (\citealt[289 and 298, ex. 12a and 44a]{rizzi1997}):

\ea \gll Un uomo a cui, il premio Nobel, lo daranno senz'altro. \label{topicrel}\\
a.\textsc{m} man to whom the.\textsc{m} prize Nobel it.\textsc{m.acc} give.\textsc{fut.3pl} undoubtedly\\
\glt `A man to whom, the Nobel Prize, they will give it undoubtedly.'
\ex \gll Ecco un uomo a cui IL PREMIO NOBEL dovrebbero dare (non il premio X). \label{focusrel}\\
here a.\textsc{m} man to whom the.\textsc{m} prize Nobel should.\textsc{3pl} give.\textsc{inf} \phantom{(}not the.\textsc{m} prize X\\
\glt `Here is a man to whom they should give THE NOBEL PRIZE (not prize X).'
\z
}

Based on the observed patterns regarding ordering restrictions, \citet[291, 298--299]{rizzi1997} concludes that relative pronouns are located in [Spec,ForceP], while question operators are located lower in the structure, namely in [Spec,FocP], which is why they are in complementary distribution with foci.

The TopP projection is also relevant in terms of adverb preposing: here the analysis of \citet{rizzi1997} differs crucially from his later views expressed in \citet{rizzi2004}. Rather than assuming that adverbs are adjuncts to the IP, \citet[300--301, 308--309]{rizzi1997} argues that adverbs move to [Spec,TopP], satisfying a Topic Criterion, just as in the case of argumental topicalisation. In this way, topicalisation is triggered properly as any other movement process, and the fact that topics appear in an IP-peripheral position (but not within the IP or above the CP) can be accounted for by assuming TopP to be an integral part of the clause (\citealt[300--301]{rizzi1997}). This view is contested by \citet[238--243]{rizzi2004}, in that the most typical position of left-peripheral adverbs is the specifier of a dedicated modifier phrase, ModP, while under certain circumstances an adverb may act as a regular topic (in TopP) or be focussed (in FocP). The reason behind this is partly interpretive (topics express given information, adverbs not necessarily), partly distributional (adverbs occupy different relative positions from ordinary topics), see \citet[238--239]{rizzi2004}. The assumption that adverbs move to specifiers of designated left-peripheral positions is in line with the general spirit of the cartographic approach and with the implementation of \citet{cinque1999} for adverb positions in particular.

The revised theory of the fine structure of the left periphery is hence as follows (\citealt[242, ex. 60]{rizzi2004}):

\ea Force \phantom{\ldots}Top* \phantom{\ldots}Int \phantom{\ldots}Top* \phantom{\ldots}Focus \phantom{\ldots}Mod* \phantom{\ldots}Top* \phantom{\ldots}Fin \phantom{\ldots}IP
\z

The novelty lies in the introduction of an iterable ModP for various adverbs and also the IntP, interrogative phrase, which hosts higher \textit{wh}-elements such as \textit{perch\'e} `why' in Italian (see \citealt[242]{rizzi2004}; see also \citealt{rizzi2001}) for details. The importance of the various positions lies in a revised analysis of Relativized Minimality. \citet[247]{rizzi2004} claims that the ``positional system is amenable to a typology of few featurally defined natural classes: argumental, quantificational, and modificational elements''.

An important point made by \citet[312--315]{rizzi1997} concerns the actual size of the CP-periphery. Namely, in ``simple cases (\ldots) the force--finiteness system can be expressed on a single head'', such as \textit{that} in English embedded declaratives or its zero counterpart (\citealt[312]{rizzi1997}). More precisely, \citet[312]{rizzi1997} assumes that \textit{that} ``expresses declarative force and may optionally express finiteness'', while its zero counterpart ``expresses finiteness, and may optionally express declarative force''. \citet[312]{rizzi1997} distinguishes between ``simple cases'' and ``complex cases'': in simple cases, there are no TopP or FocP projections and hence ``the force--finiteness system can be expressed on a single head'' (in which case \textit{that} and the zero complementiser are functionally equivalent), while in complex cases ``force and finiteness must split because the topic--focus system is activated'' (in which case \textit{that} occupies Force and the zero complementiser occupies Fin). This kind of split can be observed in the following example (\citealt[313, ex. 91]{rizzi1997}):

\ea \ldots [that [next year [$\emptyset$ John will win the prize]]]
\z

Importantly, the higher specification (Force) cannot be realised as zero and the lower specification (Fin) cannot be realized as \textit{that} in such ``splitting'' cases (\citealt[313]{rizzi1997}, following \citealt{rochemont1989} and \citealt{grimshaw1997} among others). If, however, the topic--focus field is not activated, the split between Force and Finiteness is barred by an economy constraint that can be referred to as ``Avoid structure'' (\citealt[314]{rizzi1997}, in line with analogous proposals made by \citealt{crisma1992}, \citealt{safir1993}, \citealt{speas1994}, \citealt{grimshaw1997}, among others, as well as with the economy constraints of \citealt{chomsky1991, chomsky1993, chomsky1995}). Ultimately, this is taken to be responsible for the following extraction asymmetry (based on \citealt[312 and 314, ex. 88 and 97]{rizzi1997}):

\ea
\ea[*]{Who do you think [that [\textit{t} $\emptyset$ [\textit{t} will win the prize]]]? \label{thattrace}}
\ex[]{Who do you think [\textit{t} $\emptyset$ [\textit{t} will win the prize]]? \label{emptytrace}}
\z
\z

The idea is that (\ref{thattrace}) is a violation of the \textit{that}-trace filter, while (\ref{emptytrace}) is not, and that while (\ref{thattrace}) involves a separate ForceP and a FinP, in (\ref{emptytrace}) there is only one CP projection (\citealt[313--314]{rizzi1997}). As \citet[313--314]{rizzi1997} assumes, the FinP projection must be generated for agreement purposes if the subject is extracted, but this is possible only with the zero complementiser and not with \textit{that}, an assumption made by \citet[312]{rizzi1997} regarding the feature specification of the respective complementisers. Hence, the insertion of \textit{that} implies that a separate ForceP is present. This is licensed if there is a topic in between the two, which is why the \textit{that}-trace effect does not arise if there is a topic. However, if the topic--focus field is not activated, the generation of a separate ForceP is not licensed, due to the economy principles described above. In his later analysis, \citet[241]{rizzi2004} points out that the ``anti-adjacency effect'' can be observed with adverbs but not with regular topics, which again speaks for different positions for adverbs and topics in the left periphery mentioned above.

Regarding the exact mechanism of the economy principle, \citet[314--315]{rizzi1997} argues that the blocking effect making (\ref{thattrace}) cannot be due to the numeration (as the economy principles of \citealt{chomsky1995} would suggest) but it rather follows from a principle allowing the insertion of functional elements only if they are necessary, as maintained by \citet{grimshaw1993} for \textit{do}-support: in this sense, functional elements are not part of the reference set in the numeration. \citet[315]{rizzi1997} assumes that a similar principle may underlie the distribution of expletives in languages like German and Icelandic, where the expletive is licensed (and required) by the V2-constraint.

Importantly, the C head can host verbs as well, and this can also be observed in English to some extent. One such context is negative inversion, where \citet{rizzi1997}, following \citet{culicover1992, culicover1993}, discusses a difference between patterns where a subject has been extracted and ones where there is no subject extraction. Consider the following examples involving the preposed negative element \textit{only in that election} (\citealt[315--316, ex. 104 and 105]{rizzi1997}):

\ea\judgewidth{??}
\ea[??]{Leslie is the person who I said that only in that election did run for public office. \label{relinv}}
\ex[]{Leslie is the person who I said that only in that election ran for public office. \label{relnoinv}}
\ex[]{I think that only in that election did Leslie run for public office. \label{intinv}}
\ex[*]{I think that only in that election Leslie ran for public office. \label{intnoinv}}
\z
\z

In (\ref{relinv}) and (\ref{relnoinv}), the subject is extracted: as demonstrated by the grammaticality of (\ref{relnoinv}), no inversion is required, while the inversion pattern involving \textit{do}-insertion in (\ref{relinv}) is degraded. The exact opposite can be observed if no subject extraction applies, as in (\ref{intinv}) and (\ref{intnoinv}): the inversion pattern in (\ref{intinv}) is grammatical, while the absence of inversion in (\ref{intnoinv}) results in unacceptability. As \citet[316]{rizzi1997} summarises, it seems that ``inversion with a preposed negative element must apply except in case the subject has been extracted''. In fact, the same asymmetry can be observed in main clause interrogatives, as pointed out by \citet[317, ex. 106 and 107]{rizzi1997}:

\ea
\ea[]{Who did you see \textit{t}? \label{whodid}}
\ex[*]{Who you saw \textit{t}?}
\ex[*]{Who did see you?}
\ex[]{Who saw you? \label{whosaw}}
\z
\z

As pointed out by \citet[317]{rizzi1997}, a \textit{wh}-element has to move to [Spec,CP], regardless of whether it is a subject or an object. The difference lies in whether there is I-to-C movement. This is obligatory in (\ref{whodid}): the [wh] feature is generated under T and it has to move to C in order for the Wh Criterion to be satisfied. By contrast, in (\ref{whosaw}) the subject moves ultimately from [Spec,TP] to [Spec,CP] and hence C agrees with a specifier that is coindexed with the specifier of T, where the [wh] feature is located. Hence, ``they are in the appropriate local relation (no other head intervenes)'' and ``they can form a representational chain which possesses the Wh feature (still sitting under T)'' (\citealt[317]{rizzi1997}). The same option is not available for (\ref{whodid}) because the specifiers of C and T ``are contra-indexed, so that the heads are contra-indexed, too, and no representational chain connecting C to T can be built'' (\citealt[317]{rizzi1997}).

By analogy, \citet[317]{rizzi1997} assumes that I-to-C movement (more precisely, movement to Foc) in negative preposing structures ``is triggered by the Negative Criterion'' (based on \citealt{rizzi1991}, \citealt{haegemanzanuttini1991}, \citealt{haegeman1995}), whereby the Neg feature is ``generated under T on a par with the Wh feature'' and it ``must be brought up to the C system if a negative element is preposed in order to create the required Spec/Head configuration''. This involves the insertion and movement of \textit{do} to C (Foc) in constructions like (\ref{intinv}), since the specifier of the CP (FocP) is not coindexed with the specifier of the TP, while no verb movement is required in constructions like (\ref{relnoinv}), where the subject has been extracted and hence a representational chain can be created (\citealt[317--318]{rizzi1997}).\footnote{Naturally, languages show differences with respect to the projections generated in the CP-domain, as well as regarding the properties of various elements located there. For instance, \citet[318--325]{rizzi1997} argues that in French an independent AgrP can be projected, which results in a lack of anti-adjacency effects of the English type. This property of French also follows from the properties of the finite complementiser, which cannot be zero, unlike in English (\citealt[320, ex. 114]{rizzi1997}), as shown in (\ref{englishzero}) and (\ref{frenchzero}):

\ea I think (that) John will come. \label{englishzero}
\ex \gll Je crois *(que) Jean viendra. \label{frenchzero}\\
I think.\textsc{1sg} \phantom{*(}that John come.\textsc{fut.3sg}\\
\glt `I think that John will come.'
\z

The distribution of the zero complementiser is restricted in English: it is allowed in internal argument clauses, as in (\ref{internalarg}), but not in subject clauses, see (\ref{subjectclause}), or in preposed clauses, as in (\ref{preposed}) below (\citealt[320, ex. 115]{rizzi1997}):

\ea[]{I didn't expect [$\emptyset$ [John could come]]. \label{internalarg}}
\ex[*]{[$\emptyset$ [John will come]] is likely. \label{subjectclause}}
\ex[*]{[$\emptyset$ [John could come]], I didn't expect. \label{preposed}}
\z

As pointed out by \citet{kayne1984} and \citet{stowell1981diss}, the zero finite complementiser has the same distribution as traces, and \citet{pesetsky1995} actually claims that there is a trace involved: the zero complementiser is affixal and it incorporates onto the higher V head (\citealt[320]{rizzi1997}).}

The approach proposed by \citet{rizzi1997, rizzi2004} is important for various reasons. First, it recognises the availability of a complex left periphery, indicating that a single CP is not always tenable. Second, it shows that not only clause-typing elements but also topics and foci may move to the left periphery, and that the two also differ in their syntactic behaviour. Third, it is evident that while multiple elements may be allowed to co-occur, there are certain ordering restricting applying to them.

However, there are also some problems that cannot be overlooked. While TopP and FocP are taken to be designated left-peripheral projections that appear along genuine CPs, it is clear that the movement operations targeting these must be different from the movement of clause-typing operators. Namely, while the movement of a relative operator is tied to its lexical property (call it a [rel] feature), topicalised and focussed elements do not have lexically inherent [topic] and [focus] features (cf. \citealt{fanselowlenertova2011}), unless one were to assume that the different occurrences of the DP \textit{Gianni} in (\ref{gianni}) feature a lexical item \textit{Gianni}\textsubscript{[topic]} and a lexical item \textit{Gianni}\textsubscript{[focus]}, while a neutral lexical item \textit{Gianni} must also exist. Moreover, while the fronting of topicalised and focussed elements is indeed triggered in certain languages, it is not the case in others: in English, for instance, the sentences in (\ref{englishrizzi}) are less natural versions and normally topics and foci would not be fronted. In other words, while the notions of topic and focus are not completely independent of syntax, it is evident that they cannot be reduced to the insertion of syntactic features. On the other hand, non-operator material may be fronted to the CP-domain in certain languages, such as in German V2 clauses, which is not tied to any specific informational structural property (termed ``formal movement'' by \citealt{frey2004, frey2005}; see also \citealt{denbesten1989}, \citealt{fanselow2002, fanselow2004, fanselow2004isis, fanselow2009} on German V2). This kind of movement is not covered by any of the designated positions of \citet{rizzi1997, rizzi2004}.

A second problem concerns selectional restrictions, which was also pointed out earlier by \citet[534--536]{sobin2002} and \citet[231]{abels2012}, among others (see also \citealt{lahne2009}, following \citealt{newmeyer2003}). \citet{rizzi1997, rizzi2004} assumes that TopP and FocP are essentially independent of selectional restrictions, yet if the left periphery is structured in the way given in \figref{rizzitree}, a Force head selects a TopP as its complement, and the Top head selects a FocP, and ultimately a FinP is selected by a Top head. While the ForceP and the FinP are the core projections of the functional left periphery, if a complex periphery is generated, there is no way for them to be in a direct selectional relationship. It is unclear how designated Top and Foc heads can be equipped with features responsible for selection of other CP-type projections. \citet{rizzi1997, rizzi2004} argues that the TopP and FocP layers are present in all clauses, though they may not be activated. If they are not activated in the syntax, this affects selectional restrictions, and the question arises how such variability can be modelled, as sometimes a given head appears to select for diverse complements. To give one example, a Top head may select a FocP, but if there are multiple topics, another TopP is supposed to be generated and selected by the higher Top head, and if the FocP is not generated, the complement of Top is either again a TopP, or a FinP. In addition, the notion of activating layers is problematic from a minimalist perspective since elements that are not merged into the structure cannot be taken to be part thereof.

Third, related to this, the ForceP and the FinP are apparently not always split: if there is only a single \textit{that} on the left periphery, it can express both Force and Finiteness. However, this sort of optionality is problematic inasmuch as finiteness is part of the lexical entry of \textit{that}, since it clearly cannot appear in non-finite clauses. This is supposed to happen if there is no intervening topic (or focus) between the two layers, and in essence this would be a ban on two adjacent complementisers (in this case \textit{that} and its zero counterpart). However, complementiser combinations are actually possible across languages, such as the German comparative in (\ref{alswie}) above containing the combination \textit{als wie} `than as', where the two complementisers (see \citealt{jaeger2010, jaeger2018}, \citealt{bacskaiatkari2014dia, bacskaiatkari2014diss} on the status of \textit{wie} as a complementiser) have largely overlapping functions (just as in the case of \textit{that} and it zero counterpart).

This leads to the fourth problem, which is the separation of Force and Fin and whether complementisers are inserted according to this template. As mentioned above, \textit{that} is lexically specified for finiteness, and while examples for complementiser reduplications such as (\ref{welsh}) above indicate that doubling is indeed possible, it is highly questionable to claim that a finite declarative complementiser encodes finiteness in certain constructions but not in others. In addition, examples like (\ref{alswie}) with the doubling of two comparative complementisers indicate that the separation is also problematic because both elements have largely identical lexical properties, and the lower complementiser \textit{wie} is not a finiteness marker (which would be \textit{dass} `that').

Fifth, the relative position of various left-peripheral elements does not seem to conform to the template in general. This was already pointed out by \citet{neelemanvandekoot2008} in connection with scrambling in Dutch: many word order patterns involving discourse functions (topics, foci) in Dutch are not borne out by the template. I will briefly return to scrambling in \chapref{ch:6}; for now, the point is simply that the template in many cases does not generate certain patterns, while it does not restrict others. Similar problems arise in connection with clause-typing elements as well. \citet{rizzi1997} assumes that relative operators are in [Spec,ForceP], while interrogative operators are in [Spec,FocP]. If English \textit{that} is in the Force head, it is expected that interrogative operators appear lower: however, Doubly Filled COMP patterns such as (\ref{englishdfc}) above (and similar pattern across Germanic) show that this is empirically untenable as the \textit{wh}-operator precedes \textit{that}. As pointed out by \citet[534--536]{sobin2002}, one way to overcome this for \citet{rizzi1997} is to say that interrogative operators target [Spec,FocP] in root clauses but they target [Spec,ForceP] in embedded clauses, which would give the right order in Doubly Filled COMP structures, yet the separation is problematic and unmotivated. Moreover, the IntP of \citet{rizzi2001, rizzi2004} does not solve the restrictions either: this is the position where interrogative complementisers such as Italian \textit{se} `if' are supposed to be located, yet applying this to English \textit{if} raises the question why \textit{whether} cannot be inserted simultaneously into [Spec,ForceP] in embedded clauses, as demonstrated by the ungrammaticality of (\ref{whetherifch2}). In short, while the cartographic template is able to describe certain ordering restrictions, it cannot account for the possibility or the impossibility of others.

In this way, the cartographic template of \citet{rizzi1997, rizzi2004} runs into problems in terms of both descriptive and explanatory adequacy. As pointed out by \citet{abels2012}, the descriptive gains predicted by the template (as mush as this is indeed the case), are borne out also on the basis of locality constraints, that is, without the need to postulate a template as a theoretical primitive: rather, what appears to be a template is merely the consequence of independently motivate locality constraints. In his analysis, \citet{abels2012} concentrates on ordering constraints involving topics and foci (also in interaction with clause-typing projections proper, such as interrogative and relative operators). The question arises whether an alternative analysis for the cartographic approach in the same spirit  is possible in the domain of clause-typing elements only; in Sections~\ref{sec:2introducing}--\ref{sec:2degree}, I will show that this is indeed the case.

\subsection{A minimal CP -- \citet{sobin2002}} \label{sec:2sobin}
As mentioned at the end of \sectref{sec:2rizzi}, \citet{sobin2002} expressed criticism towards the cartographic approach of \citet{rizzi1997}. In this section, I am therefore going to review and evaluate his proposal. This approach involves a minimal CP in accounting for the Comp-trace effect (also known as \textit{that}-trace effect) and the adverb effect, based on the proposal made by \citet{carnie2000} ``that constituents may adjoin to lexical heads forming complex lexical heads'' (\citealt[527]{sobin2002}).

Recall that the insertion of \textit{that} next to a subject trace is marked (\textit{that}-trace effect), while the construction improves if there is an adverb between \textit{that} and the trace (adverb effect),\footnote{The same effect was observed by \citet{bruening2010} in various constructions; notably, \citet[55]{bruening2010} assumes that the adverb effect arises because the constraint is not about the subject but about the highest constituent.} as illustrated in (\ref{thattracesobin}) below (\citealt[528, ex. 1a, 2a and 3]{sobin2002}):

\ea \label{thattracesobin}
\ea[\%]{ Who did you say \textbf{that} would hate the soup? \label{thatwould}}
\ex[]{Who did you say would hate the soup?}
\ex[]{Who did you say \textbf{that} without a doubt would hate the soup? \label{thatwithout}}
\z
\z

A traditional explanation (see \citealt{sobin1991}, \citealt{culicover1993}, \citealt{browning1996}; see also the discussion in \sectref{sec:2rizzi}) for (\ref{thatwould}) was that the trace of the subject and C (which would license the trace) are not co-indexed; however, the adverb effect in (\ref{thatwithout}) constitutes a problem for this approach (\citealt[528]{sobin2002}).

\citet{browning1996} adopts CP-recursion and in her analysis, (\ref{thatwould}) is not licensed because a lexical complementiser (as opposed to a zero complementiser) is not allowed to be coindexed. The problem with this is, as pointed out by \citet[530--531]{sobin2002}, that lexical complementisers can be coindexed: the Dutch counterpart of (\ref{thatwould}) is grammatical, and French exhibits similar phenomena (see \citealt{perlmutter1971} and \citealt{malingzaenen1978} for Dutch and \citealt{kayne1981} for French). Moreover, the same indexing seems to be licensed in relative clauses with \textit{that} and a subject trace (\citealt[537, ex. 7]{sobin2002}):

\ea The person \textbf{that}\textsubscript{i} t\textsubscript{i} likes anchovies ordered the pizza. 
\z

In fact, as pointed out by \citet[535]{sobin2002}, either the complementiser or a relative pronoun (\textit{who}) is required in subject relative clauses, and hence these constructions show exactly the opposite of what can be observed in clauses exhibiting the \textit{that}-trace effect.

Regarding the adverb effect, \citet{browning1996}, following the analyses of \citet{cheng1991diss} and \citet{watanabe1992}, assumes that a [Spec,CP] position is generated only in \textit{wh}-clauses, and since the adverbial is located in [Spec,CP], the complementiser \textit{that} has to move up to a higher C head position in order to satisfy the requirement of the highest CP-node lacking a base-generated specifier. The rest of the analysis is reminiscent of the arguments presented by \citet{rizzi1997}. However, unlike in the cartographic approach, the CP is by default minimal: CP-recursion is limited by Greed, following \citet{chomsky1993}. With respect to the adverb effect, \citet[531]{sobin2002} notes that the position of the adverb assumed by \citet{browning1996} is probably wrong: the adverbials do not involve agreement with the C head, unlike \textit{wh}-phrases, and there is no reason to assume that they are located in this position. This is further strengthened by the experimental data given by \citet[540--545]{sobin2002}.

Concerning the split CP account of \citet{rizzi1997}, \citet[534--536]{sobin2002} criticises the general approach, especially the problems regarding selectional restrictions and the feasibility of given elements always targeting the same designated projections: see the discussion at the end of \sectref{sec:2rizzi}.

Importantly, \citet[536--537]{sobin2002} points out that there is considerable variation concerning the \textit{that}-trace effect and the adverb effect. Empirical studies suggest that English speakers differ with respect to the acceptability of these constructions and that judgements are not rigid either (see, for instance, \citealt[328]{pesetsky1982} and \citealt{sobin1987} on American English). It seems plausible that ``for adults, the \textit{that}-trace effect in English may be `softer' and more variable than much of the literature anticipates'' and that the ``\textit{that}-trace effect appears to be weak or absent from the grammars of learners of English'', its acquisition being comparatively late (\citealt[537]{sobin2002}). The problem with previous approaches is, then, that while they may be aware of variability, they fail to account for it, categorically blocking or allowing the constructions in question instead (\citealt[537]{sobin2002}).

As \citet[537]{sobin2002} posits, the \textit{that}-AdvP sequence may form one prosodic unit, depending on the preferences of the speaker. In fact, the possibility of coordinating such units indicates that they may well be constituents. Consider (\citealt[538, ex. 21a]{sobin2002}):

\ea John claimed \textbf{that in the last election and that in all earlier ones} ballot boxes were stuffed.
\z

Following the idea proposed by \citet{carnie2000}, \citet[538]{sobin2002} claims that ``the phrase/head distinction may be derived rather than primitive'' and hence ``phrases and heads may have overlapping properties'', so that it is possible ``that lexical heads may combine with phrase-like sequences, projecting a lexical category''. The complex C head has the following structure (based on \citealt[539, ex. 22]{sobin2002}):

\ea {[}\textsubscript{C} [\textsubscript{C} that] AdvP]
\z

This complex head constitutes, according to \citet{carnie2000}, an extraction island.

Interestingly, adverbials may interact with Doubly-Filled Comp, as the insertion of the adverbial may license Doubly Filled COMP patterns for speakers who otherwise do not accept it. Consider (\citealt[539--540, ex. 25a and 25e]{sobin2002}):

\ea
\ea[]{I just saw a person \textbf{WHO, that for all intents and purposes}, could pass for Albert Einstein!}
\ex[*]{I just saw a person \textbf{who that} could pass for Albert Einstein!}
\z
\z

In order to account for the observed phenomena, \citet[545]{sobin2002} introduces the notion of Fuse. In essence, this means that ``under specific conditions, the Spec and head elements of CP collapse or fuse together into a single indexed element'' (\citealt[545]{sobin2002}, following \citealt{pesetsky1982} and \citealt{sobin1987}). According to \citet[546]{sobin2002}, phenomena like the Doubly Filled COMP Filter or the interchangeability of \textit{who} and \textit{that} in relative clauses are indicative of there being strong pressure on the CP to collapse, something that is not attested in, for instance, the IP, where the subject and the I head are not prohibited to be spelt out simultaneously. Fuse is triggered when a chain head is merged in [Spec,CP] and if either the specifier or the C head is overt (\citealt[546--547]{sobin2002}). This is also supposed to account for asymmetries between subject and object relative clauses (\citealt[547--548]{sobin2002}). In subject relative clauses, the trace of the subject in [Spec,IP] must be properly governed: this is possible if either \textit{who} or \textit{that} is inserted, since by way of Fusion the resulting element can be indexed and can therefore properly govern the trace. However, if neither the relative pronoun nor the complementiser is overt, Fusion cannot take place, and since the complementiser cannot govern the subject trace by its own virtue, the subject trace remains ungoverned and the structure is ungrammatical. The same problem does not arise in object relative clauses because the trace  is located elsewhere.

Regarding Doubly Filled COMP, the idea is that Fuse allows a more economical configuration than Doubly Filled COMP, and hence the latter is blocked in favour of the former option (\citealt[548--549]{sobin2002}). If there is an adverb, then Fuse either applies to the sequence \textit{who that} and eliminates \textit{that}, or it applies to \textit{who} and the complex C element (consisting of \textit{that} and the AdvP), in which case it cannot apply fully and it leaves \textit{who} adjoined to the already complex C head (\citealt[549]{sobin2002}). Crucially, this does not lead to the collapse of the CP, as with a simple initial C head (\citealt[550]{sobin2002}). The variability of the adverb effect lies in the weighting of the derivational cost of a more complex (non-collapsed) syntactic structure versus a relatively complex C head (\citealt[553]{sobin2002}).

Importantly, Fuse operates differently if the element in the specifier is not a chain head but merely a trace: in this case, Fuse is possible also if there is no overt element in the CP (\citealt[550--551]{sobin2002}). This accounts for the difference between subject relative clauses, where an overt element in the CP is necessary, and subject extractions, where an overt element in the CP is not licensed (\citealt[550]{sobin2002}). More precisely, a collapse of the CP is possible if the complementiser is covert, but when it is overt, like in (\ref{thatwould}), the trace fails to collapse with it, leaving the subject trace unlicensed: this is subject to variation (\citealt[551]{sobin2002}). That is, there are speakers of English for whom the non-phonetic condition on traces is suspended and they hence allow the insertion of \textit{that} in constructions like (\ref{thatwould}), and similar patterns can be observed in language acquisition data and in languages like Dutch and French (\citealt[552]{sobin2002}). Fuse is ultimately an operation creating simpler structures and it can be viewed as an extreme form of agreement; it is most probably restricted to apply more broadly by recoverability conditions (\citealt[556]{sobin2002}).

The proposal made by \citet{sobin2002} in favour of a minimal CP is altogether favourable and the relative flexibility of this approach is also able to handle fine-grained variation and gradience in terms of acceptability. It is also more in line with a minimalist perspective in that there is no predefined template with a large number of possibly non-activated projections. At the same time, there are some problems that arise both on theoretical and empirical grounds.

On the one hand, the approach seems to build very strongly on X-bar theoretic assumptions, that is, on the notion that there is a pre-given specifier and a pre-given head position, which may Fuse and ultimately collapse together. In the constructions under scrutiny, there is always some (overt or covert) specifier element, but it is perfectly possible to have configurations where no specifier is merged, such as in English embedded declaratives introduced by \textit{that}. The question arises how these notions can be considered by syntactic derivations. As Fuse operates on already merged elements, this operation is introduced on top of Merge in syntax\footnote{In this respect, it is reminiscent of incorporation (cf. \citealt{baker1988} or of the operation fusion in Distributed Morphology (see \citealt{hallemarantz1993}). However, the exact location of Fuse in the grammatical system remains unclear.} and it remains unclear whether there is any clear advantage of this. Of course, similar considerations arise also in connection with the analysis of \citet{browning1996}, where it is assumed that only \textit{wh}-CPs have a specifier: apart from the question to what extent this is X-bar specific, the problem is that V2 languages like German are known to have an active [Spec,CP] in declarative clauses as well.

On the other hand, \citet{sobin2002} explicitly states that Fuse is favourable over Doubly Filled COMP, yet data like (\ref{dfc}), repeated here as (\ref{dfcrepeatch2}) clearly indicate that Doubly Filled COMP patterns do exist in Germanic languages:

\ea \label{dfcrepeatch2}
\ea[\%]{ I wonder \textbf{which book that} Ralph is reading. \label{englishdfcrepeatch2}}
\ex[]{ \gll Peter spurte \textbf{hvem} \textbf{som} likte bøker. \label{norwegiandfcrepeatch2}\\
Peter	asked.\textsc{3sg} who	that liked books\\
\glt `Peter asked who liked books.'}
\z
\z

This pattern is completely acceptable in Norwegian, see (\ref{norwegiandfcrepeatch2}), and subject to dialectal variation in English and other West Germanic languages, whereby the standard West Germanic languages prohibiting Doubly Filled COMP patterns contrast with many regional dialects. In other words, while Fuse is supposed to be compatible with variation as well, one of its major operation domains is empirically refuted by a number of Germanic varieties, including non-standard English. Further, it is not clear how double complementisers like German \textit{als wie} in (\ref{alswie}) can be handled since multiple CPs are not discussed by \citet{sobin2002}, who explicitly assumes that the CP is very minimal. Finally, regarding subject relatives, it is assumed throughout by \citet{sobin2002} that subject relative clauses are uniformly introduced by an overt element (either \textit{who} or \textit{that}). However, this is in fact also subject to variation: zero subject relatives constitute a historical pattern that is on the retreat but nevertheless available for many speakers of British English (for instance, in the dialects of Northern Ireland, see \citealt[55--56]{herrmann2005}; see also \citealt{kortmannwagner2007}). As no variation is supposed to be available regarding Fuse on chain heads, the dialect data are not covered by the analysis.

\subsection{Lower left peripheries -- \citet{poletto2006}} \label{sec:2poletto}
So far I have concentrated on the CP-domain when discussing functional left peripheries. I will now briefly review the study of \citet{poletto2006}, which argues for the availability of a lower functional left periphery, at the functional vP-edge. As \citet[261]{poletto2006} mentions, similar views were expressed by \citet{jayaseelan2001}, \citet{bellettishlonsky1995} and \citet{belletti2004} for Modern Italian,\footnote{\citet{belletti2004}, just like \citet{poletto2006}, maionly concentrates on the availability of a focus projection in a clause-internal position, which she claims to be potentially surrounded by topic projections, in the same way as originally proposed by \citet{rizzi1997} for the CP-periphery.} and by \citet{paul2002} for Chinese. Apart from considering data that are not covered by the analysis of \citet{rizzi1997}, this proposal is important because it extends the cartographic approach beyond the CP-periphery. As pointed out by \citet[17--18]{belletti2004}, the importance in recognising a similarity between the CP and the vP in this respect lies also in the fact that these projections, as proposed by \citet{chomsky2001}, are considered to be phases in the Minimalist Programme. I am concentrating on the analysis of \citet{poletto2006} here, since her particular instantiation of focus in a lower periphery is immediately relevant for certain embedded interrogatives, as will be discussed in \sectref{sec:2interrogatives} and in \chapref{ch:6}.\footnote{Notably, this analysis addresses a problem in the vP-periphery in Old Italian that is analogous to the V2 requirement in the CP-domain. Naturally, the vP-periphery also constitutes a well-researched area of syntax; see \citet{bonan2021} for a recent analysis (and references there). The considerations not only apply to focus: A similar analysis is suggested by \citet{hinterhoelzl2006} for Old High German topicalisation, presenting evidence from verb clusters. \citet{hinterhoelzl2018} also argues that the vP-periphery contains an AspP at its left edge, and that the movement operations targeting the vP-periphery altogether conform to an analysis of the German AspP/vp/VP as head initial, which is essentially in line with the model proposed by \citealt{kayne1994} regarding basic assumptions about headedness. (Note that the same conclusions would apply to further verbal projections, such as VoiceP, which is standardly located below AspP. The importance of AspP in the aforementioned analysis is that \citealt[249]{hinterhoelzl2006} assumes AspP to be a phase head.) For similar view regarding OV orders in Old English, see also \citet{struikvankemenade2022}, relying on \citet{struikvankemenade2020} and \citet{biberauerroberts2005}. See also \citet{roberts1997} on Old English, \citet{hinterhoelzl2004, hinterhoelzl2009, hinterhoelzl2010, hinterhoelzl2015} and \citet{hinterhoelzlpetrova2010focus} on Old High German, and \citet{hroarsdottir2000} on Old Icelandic.}

Old Italian demonstrates properties of a V2 language, which involves the movement of a finite verb to C (\citealt[261--262]{poletto2006}, citing \citealt{beninca1984}). Consider the following example from Old Italian (\citealt[262, ex. 2a]{poletto2006}):

\ea \gll \textbf{quali} \textbf{denari} \textbf{avea} Baldovino lasciati loro \label{italianv2}\\
which.\textsc{m.pl} money.\textsc{pl} had.\textsc{3sg} Baldovino left.\textsc{m.pl} them\\
\glt `(\ldots) which money Baldovino had left them' (\textit{Doc. fior}. 437)
\z

Similar patterns are also attested with non-finite verbs, which do not move to C, raising the question how these patterns should be treated (\citealt[261, ex. 1]{poletto2006}):

\ea \gll Allora il cavalero, che 'n sì alto mestero avea \textbf{la} \textbf{mente} \textbf{misa} \label{lamente}\\
then the.\textsc{m} knight that in so high.\textsc{m} work had.\textsc{3sg} the.\textsc{f} mind set.\textsc{f.ptcp}\\
\glt `then the knight, who had set his mind in so high a work'\\(Brunetto Latini, \textit{Tesoretto}, v. 1975)
\z

Apart from the surface word order in examples like (\ref{italianv2}), pro drop patterns strongly suggest that the verb in Old Italian moved higher than in Modern Romance languages: pro drop is attested in main clauses but not in embedded clauses (cf. \citealt{vanelli1987}), and the standard analysis for this is ``that pro can only be licensed when the verb has moved to the CP layer'' (\citealt[263]{poletto2006}, quoting \citealt{beninca1984} for Old Italian and \citealt{roberts1993} for Old French).\footnote{This idea has been questioned by other authors as well; see \citet{cognolawalkden2019, cognolawalkden2021} for a recent account relying on different types of Agree in asymmetric \textit{pro}-drop languages such as Old High German and Old Italian.} However, Old Italian was not a strict V2 language as V3 orders are frequently attested (\citealt[236, ex. 4a]{poletto2006}):

\ea \gll E \textbf{dall'} \textbf{altra} \textbf{parte} \textbf{Aiaces} \textbf{era} uno cavaliere franco\\
and on.the other.\textsc{f} side Ajax was.\textsc{3sg} a.\textsc{m} knight courageous.\textsc{m}\\
\glt `and, on the other hand, Ajax was a courageous knight'\\(Brunetto Latini, \textit{Rettorica} 94)
\z

Following \citet{beninca2006}, \citet[263]{poletto2006} assumes that the verb in these cases moves to the head of a FocP position, and hence the topic positions above FocP are still available for fronting operations. Apart from V3 orders, \citet[263--264]{poletto2006} notes that Old Italian demonstrates the widespread use of V1 orders: this is compatible with V2 in Germanic languages as well, and in fact used to be more frequent in Old Germanic languages than in their modern counterparts (\citealt{fuss2005diss}).

Interestingly, however, there are examples in which a focussed object cannot be located in the CP-periphery, as it follows the finite auxiliary; consider (\citealt[264, ex. 7a]{poletto2006}):

\ea \gll i nimici avessero già \textbf{il} \textbf{passo} \textbf{pigliato} \label{ilpasso}\\
the.\textsc{m.pl} enemies had.\textsc{3pl} already the.\textsc{m} pass taken.\textsc{m}\\
\glt `the enemies had already occupied the pass'\\(Bono Giamboni, \textit{Orosio} 88, r. 15)
\z

As can be seen, the object precedes the auxiliary, similarly to (\ref{lamente}), and the finite auxiliary is preceded by the subject: note that this cannot be due to Old Italian being an OV language as unmarked word orders clearly show that Old Italian, similarly to the general Romance pattern, was a VO language (\citealt[265]{poletto2006}). Given this, it is logical to suppose that the object has undergone scrambling to the left: following \citet{belletti2004}, \citet[267]{poletto2006} assumes that this position is located at edge of the low vP phase, and that it can host essentially any type of constituent (arguments, adverbials, verbal modifiers), which is a freedom frequently observed with left peripheral positions. \citet[267]{poletto2006} suggests that the left periphery of any phase is fundamentally construed by merging a Topic-Focus field below the highest projection of the given phase, e.g. Force in the CP (in the sense of \citealt{rizzi1997}).

Citing \citet{egerland1996}, \citet[267--268]{poletto2006} observes that in OV orders like (\ref{lamente}) and (\ref{ilpasso}), agreement between the object and the participle (masculine plural in (\ref{lamente}) and masculine singular in (\ref{ilpasso}) above) is obligatory, whereas this agreement is optional in VO orders; the difference can be observed in certain modern dialects like Friulian (see \citealt{paoli1997ma}). Further, the eventual loss of OV orders went in parallel with the eventual loss of agreement with the past participle (\citealt[268]{poletto2006}). The differences in object agreement are similar to what can be observed in subject agreement: preverbal subjects (moving to AgrSP) show more agreement morphologically than postverbal subjects (\citealt[268--269]{poletto2006}, citing \citealt{guastirizzi2002}). Hence, \citet[269, 271]{poletto2006} concludes that in OV orders the object has undergone fronting to AgrOP, while in VO cases the null hypothesis is that there is no movement; moreover, since OV patterns are related to focus, movement ultimately targets a lower FocP, the verb moving to the Foc head. Importantly, \citet[271]{poletto2006} points out that ``Focus in Old Italian maintains the same property throughout all the phases where it occurs'' in that ``it must be filled by a verbal head in all phases'', whereby the Focus head is filled by the inflected verb in the high phase and it is filled by the past participle in the low phase.

There are a number of properties that are similar to what can be observed in the high periphery. First, just as V3 orders exist in the CP-domain, multiple elements can occur in front of the past participle (\citealt[275]{poletto2006}). Consider (\citealt[275, ex. 25a]{poletto2006}):

\ea \gll ed ha'mi \textbf{la} \textbf{cosa} \textbf{molte} \textbf{volte} \textbf{ridetta}\\
and has.to.me the.\textsc{f} thing many.\textsc{f.pl} times retold.\textsc{f}\\
\glt `and has retold me the thing many times' (Bono Giamboni, \textit{Trattato} 131)
\z

In addition to the similarity to V3, the low periphery allows orders akin to V1 and just like in the case of V1, enclisis is attested (\citealt[276]{poletto2006}).

Apart from V2 and IP scrambling, Old Italian shows scrambling phenomena in the DP, which is likewise not possible in Modern Italian: this indicates that functional properties are independent of the particular phase (\citealt[277]{poletto2006}). Namely, Old Italian allows modified adjectives in a prenominal position and also adjectives that are possible only in a postnominal position in Modern Italian (\citealt[277--278]{poletto2006}). Consider (\citealt[277, ex. 30c]{poletto2006}):

\ea \gll il \textbf{ben} \textbf{usato} \textbf{cavaliere} disidera battaglia\\
the.\textsc{m} well behaved.\textsc{m} knight wants battle\\
\glt `the well-behaved knight wants battle' (Bono Giamboni, \textit{Vegezio} 70, r. 6)
\z

The prenominal position is a result of fronting; this can be seen in examples where the adjective moves to the left but leaves its modifier behind (\citealt[278, ex. 31a]{poletto2006}):

\ea \gll e di \textbf{gentile} aspetto \textbf{molto}\\
and of kind appearance very\\
\glt `and of a very kind appearance' (Dante, \textit{Vita nuova}, cap. 8, par. 1, v. 11)
\z

Importantly, the word order patterns that result from a Focus position on the left periphery (V2, IP-scrambling, DP-scrambling) were lost at the same time and at the same rate: the examination of Renaissance texts shows that they were already limited to few constructions in this period (\citealt[278--285]{poletto2006}).

The proposal made by \citet{poletto2006} for Old Italian is crucial especially because it shows convincingly that focus fronting is not tied to a single position in the CP-domain and that once new information focus can be fronted, it is true in all functional domains. At the same time, the analysis makes use of the cartographic approach and assumes that there are designated Focus projections, and thus the concerns expressed in connection with the analysis of \citet{rizzi1997, rizzi2004} apply here as well. In particular, if the notion of focus (just like topicalisation) is independent of a particular projection, inasmuch as it is clearly not just a type of CP that can be associated with focus fronting, the question arises whether focus fronting is not better treated in a model where the interface conditions on focusing (such as prosodic requirements) are taken into consideration without trying to transform them into syntactic rules (see the proposal made by \citealt{fanselowlenertova2011}).

In addition, Old Italian V2 is related to focusing in this proposal, while this is clearly not transferable to Germanic V2, where no information-structural constraint can be observed regarding the first constituent in [Spec,CP]. Likewise, while V2 is attested in German and most Germanic languages, the presence of an analogous vP-periphery (and DP-periphery) is questionable. Importantly, when discussing peripheries, \citet{poletto2006} exclusively considers focus and topic fronting but not complementisers, even though complementisers are canonical elements in the CP-periphery. The analogy seems to be straightforward in the case of the DP-domain, where determiners may well have similar functions; the question is rather whether there are functional elements in the lower periphery (vP-domain) that would provide additional evidence for the existence of a periphery analogous to the higher periphery.\footnote{This problem of course relates to the more general problem of what exactly belongs to the vP-periphery and which projections count as phase boundaries. Projections such as AspP or VoiceP are either treated as separate from vP or as subtypes of vP. Likewise, vP is standardly assumed to be a phase boundary (\citealt{chomsky2001}), but VoiceP (\citealt{baltin2007}) and AspP (\citealt{hinterhoelzl2006}) have also been proposed as phase boundaries.}

\section{Introducing a flexible approach} \label{sec:2introducing}
As was mentioned in \sectref{sec:2previous}, the cartographic approach, as implemented by \citet{rizzi1997, rizzi2004} and \citet{poletto2006} among others, is problematic for a number of reasons, especially concerning the one-to-one relationship between syntactic position and function, and the designated positions regarding information structure. For these reasons, I am going to propose a more minimal, feature-based model. The goal is ultimately similar to that of \citet{sobin2002}, yet I will not resort to operations like Fuse or the notion of collapse but will argue that minimal structures directly follow from the way Merge operates and from the lexical properties of the individual elements. 

In this section, I am going to show that a flexible approach to the CP-domain is needed. Recall that in the model given by \citet{rizzi1997, rizzi2004}, there are ideally two CP projections enclosing the CP-periphery, encoding Force and Finiteness. As I partly pointed our earlier, this is problematic inasmuch as the empirical data do not always conform to this pattern. Specifically, there are (i) patterns that clearly involve a single complementiser encoding both clause type and finiteness, (ii) patterns that involve a combination of two complementisers that do not conform to the Force--Finiteness split, and (iii) patterns that involve three complementisers.

Let us consider the first scenario. As was pointed out in connection with the Force--Finiteness distinction made by \citet{rizzi1997}, the complementiser \textit{that} encodes both declarative Force and finiteness, and even though \citet{rizzi1997} assumes that it encodes finiteness only optionally, this is counter-intuitive as \textit{that} cannot occur in non-finite environments\footnote{At first sight, subjunctive clauses introduced by \textit{that} may seem to be a counter-example:

\ea I demand that he leave. \label{demand}
\z

In (\ref{demand}), the verb is not morphologically inflected for 3Sg, present tense, as it would be in the indicative mood. Note, though, that this merely indicates that the subjunctive paradigm is different from the indicative one: in language with more verb inflection, such as German, the present and the past tense paradigm of the subjunctive differ and they are both inflected for person and number. Note also that the subject of the embedded clause in (\ref{demand}) is in the nominative case: this indicates that the TP must be present (cf. \citealt[86--87]{kannonomura2012} for a similar observation), since in English, the nominative is not the default case (see \chapref{ch:6} for discussion).} and it is therefore logical to assume that finiteness is part of its lexical entry (see the discussion at the end of \sectref{sec:2rizzi}). The same applies to other complementisers, such as \textit{if}, as illustrated in (\ref{ifwhether}), repeated here as (\ref{ifwhetherrepeat}):

\ea \label{ifwhetherrepeat}
\ea[]{I don't know \textbf{if} I should call Ralph. \label{iffiniterepeat}}
\ex[]{I don't know \textbf{whether} I should call Ralph. \label{whetherfiniterepeat}}
\ex[*]{I don't know \textbf{if} to call Ralph. \label{ifnonfiniterepeat}}
\ex[]{I don't know \textbf{whether} to call Ralph. \label{whethernonfiniterepeat}}
\z
\z

Obviously, since both \textit{if} and \textit{whether} can occur as overt markers of an interrogative clause (in embedded polar interrogatives), the ungrammaticality of \textit{if} in (\ref{ifnonfiniterepeat}) cannot be related to the clause type (or Force, in the sense of \citealt{rizzi1997}), especially because \textit{whether} is licensed in (\ref{whethernonfiniterepeat}). Rather, the difference is related to finiteness: \textit{whether} is not specified for finiteness, while \textit{if} is: \textit{if} is thus incompatible with a non-finite clause. It is therefore logical to assume that finiteness is part of its lexical entry and not a property that it marks only optionally.

Without discussing this issue in detail at this point, the left periphery of (\ref{iffiniterepeat}) is assumed to have the structure in \figref{treeiffiniterepeat} and the left periphery of (\ref{whetherfiniterepeat}) is assumed to have the structure in \figref{treewhetherfiniterepeat} (note that both representations show overt elements only).

\begin{figure} 
\caption{The position of \textit{if}} \label{treeiffiniterepeat}
\begin{forest} baseline, qtree
[CP 
	[\phantom{TP}]
	[C$'$
		[C
			[if]
		]
		[TP]
	]
]
\end{forest}
\end{figure}

\begin{figure}
\caption{The position of \textit{whether}} \label{treewhetherfiniterepeat}
\begin{forest} baseline, qtree
[CP
	[whether]
	[C$'$
		[C]
		[TP]
	]
]
\end{forest}
\end{figure}


In either case, the CP encodes both the interrogative nature of the clause and finiteness, as defined by the head; there is no need for a separate overt element to encode finiteness, and the interrogative nature of the clause can be encoded either by the complementiser or by the operator located in the specifier. I will return to the features involved here later in this chapter. Note that in this section I adopt X-bar theoretic structures for representational purposes, but I will return to the issue of what the structures actually mean in a merge-based minimalist account.

Let us now turn to the second scenario, namely the combination of two complementisers that do not follow the Force--Fin distinction. One such example was discussed already: in certain dialects of German (see \citealt{jaeger2010, jaeger2018, eggs2006, lipold1983, weise1918} on dialectal variation), comparative clauses can be introduced by the combination \textit{als wie} `than as'. This is illustrated in (\ref{alswie}), repeated here as (\ref{alswierepeat}):

\ea[\%]{\gll Ralf ist größer \textbf{als} \textbf{wie} Maria. \label{alswierepeat}\\
Ralph is taller than as Mary\\
\glt `Ralph is taller than Mary.'}
\z

As was pointed out earlier, there is independent evidence that both \textit{als} and \textit{wie} are heads in configurations like (\ref{alswierepeat}), see \citet{jaeger2010, jaeger2018} and \citet{bacskaiatkari2014dia, bacskaiatkari2014diss} especially regarding the arguments against treating \textit{wie} as the comparative operator. Note that \citet{jaeger2010, jaeger2018} treats \textit{als} as a Conj head and \textit{wie} as a C head, while I assume that both elements are C heads, as a (partial) coordination analysis of comparatives (as in \citealt{lechner1999diss, lechner2004}) is rather problematic (see the argumentation in \citealt[78--84]{bacskaiatkari2014diss}). Based on this, the structure of the left periphery in (\ref{alswierepeat}) can be represented as given in \figref{alswietree}.

\begin{figure} 
\caption{The structure of \textit{als wie}} \label{alswietree}
\begin{forest} baseline, qtree
[CP
	[C$'$
		[C
			[als]
		]
		[CP
			[C$'$ [C [wie]] [TP]]
		]
	]
]
\end{forest}
\end{figure}

The configuration does not match the Force--Finiteness distinction as clearly both elements mark the comparative nature of the clause in some way. At the same time, the combination shows that CP-doubling is certainly possible; similar patterns are found in many languages (see \citealt{bacskaiatkari2014diss}, \citealt{bacskaiatkari2016alh} for details).\footnote{As will be shown in \chapref{ch:5}, the combination \textit{als wie} involves two separate syntactic heads and not one complex head (*\textit{alswie}). The fusion of such heads is in principle possible and indeed attested in the history of German: present-day \textit{als}, for instance, goes back to the original combination \textit{all so} `just as' (see \chapref{ch:5}). However, there is no evidence for such a change regarding \textit{als wie}. Both components are independently attested in German embedded degree clauses and there is no phonological reduction indicative of fusion either. Most importantly, the behaviour of the combination in terms of polarity also points to the conclusion that it cannot be a single C head (see \chapref{ch:5}).}

Ordinary comparatives are not the only syntactic environment where the doubling of the CP can be observed. Hypothetical (or irreal) comparatives involving the combination \textit{as if} are such an environment, where there are apparently two Force heads. Consider:

\ea
\ea Mary speaks \textbf{as if} she were afraid.
\ex Mary speaks \textbf{as though} she were afraid.
\z
\z

One might wonder whether the two complementisers are in a single left periphery or whether they are in separate clauses, whereby the higher (equative or similative) clause is elliptical. However, a higher clause is available only with \textit{as if} but not with \textit{as though}:

\ea
\ea[]{Mary speaks \textbf{as} she would speak \textbf{if} she were afraid.}
\ex[*]{Mary speaks \textbf{as} she would speak \textbf{though} she were afraid.}
\z
\z

This indicates that \textit{as} and \textit{though} must be in the same clause; there are reasons to believe that the generation of the equative/similative clause is not necessary with \textit{as if} either. The German pattern is even clearer in this respect (\citealt[469, ex. 4]{jaeger2010}):

\ea \label{irrealcomps}
\ea \gll Tilla läuft, \textbf{als} \textbf{liefe} sie um ihr Leben. \label{alsliefe}\\
Tilla runs as run.\textsc{sbjv.3sg} she for her.\textsc{n} life\\
\glt `Tilla is running, as if she were running for her life.'
\ex \gll Tilla läuft, \textbf{als} \textbf{ob} sie um ihr Leben liefe. \label{alsob}\\
Tilla runs as if she for her.\textsc{n} life run.\textsc{sbjv.3sg}\\
\glt `Tilla is running, as if she were running for her life.'
\ex \gll Tilla läuft, \textbf{als} \textbf{wenn} sie um ihr Leben liefe. \label{alswenn}\\
Tilla	runs as if she for her.\textsc{n} life run.\textsc{sbjv.3sg}\\
\glt `Tilla is running, as if she were running for her life.'
\ex \gll Tilla läuft, \textbf{wie} \textbf{wenn} sie um ihr Leben liefe.\\
Tilla runs as if she for her.\textsc{n} life run.\textsc{sbjv.3sg}\\
\glt `Tilla is running, as if she were running for her life.'
\z
\z

The availability of \textit{als} and \textit{ob} shows that the relevant patterns cannot be the combination of a reduced \textit{as}-clause and an ordinary \textit{if}-clause; the only pattern that can involve ellipsis is \textit{wie wenn} `as if', which is transparent, \textit{wie} being the canonical equative complementiser and \textit{wenn} being the canonical conditional complementiser in Modern (Standard) German,	cf. also \citet{jaeger2010}. Observe (\citealt[487, ex. 43 and 45a]{jaeger2010}):

\ea
\ea[]{\gll Tilla läuft, \textbf{wie} sie laufen würde, \textbf{wenn} sie um ihr Leben liefe.\\
Tilla runs as she run.\textsc{inf} would.\textsc{3sg} if seh for her.\textsc{n} life run.\textsc{sbjv.3sg}\\
\glt `Tilla is running, as if she were running for her life.'}
\ex[*]{\gll Tilla läuft, \textbf{als} sie laufen würde, \textbf{ob} sie um ihr Leben liefe.\\
Tilla runs as she run.\textsc{inf} would.\textsc{3sg} if she for her.\textsc{n} life run.\textsc{sbjv.3sg}\\
\glt `Tilla is running, as if she were running for her life.'}
\ex[*]{\gll Tilla läuft, \textbf{als} sie laufen würde, \textbf{wenn} sie um ihr Leben liefe.\\
Tilla runs as she run.\textsc{inf} would.\textsc{3sg} if she for her.\textsc{n} life run.\textsc{sbjv.3sg}\\
\glt `Tilla is running, as if she were running for her life.'}
\z
\z

This indicates that hypothetical comparatives of the type in (\ref{alsob}) represent a complex clause type involving multiple CPs in the same clausal periphery (and not two independent clauses). As indicated by (\ref{alsliefe}), the lower C is available for movement as well: this is in line with the Minimal Link Condition, according to which movement should target the first available position (see \citealt{fanselow1990, fanselow1991, chomsky1995}).

The structure of the left periphery of (\ref{alsob}) and (\ref{alswenn}) can be represented as given in \figref{alsobtree}.

\begin{figure} 
\caption{Doubling in hypothetical comparatives} \label{alsobtree}
\begin{forest} baseline, qtree
[CP
	[C$'$
		[C
			[als]
		]
		[CP
			[C$'$ [C [ob\\wenn]] [TP]]
		]
	]
]
\end{forest}
\end{figure}

Again, similarly to \figref{alswietree}, there are two complementisers on the same left periphery, and the configuration is not compatible with the Force--Fin model.

Finally, let us turn to the question of triple combinations. Since the combination \textit{als wie} is available in various German dialects, one might suppose that these dialects allow a triple combination in hypothetical comparatives. This is indeed the case: as \citet[279]{jaeger2016habil} reports, the combination \textit{als wie wenn} is indeed attested in present-day dialects (cf. also \citealt[62]{thurmair2001}). The unavailability of \textit{als wie ob} has historical reasons. The pattern involving \textit{als wenn} is attested since Early New High German (\citealt[178]{eggs2006}; see also \citealt{jaeger2010}) and the combination \textit{wie wenn} is attested since the 17th century, first only in complex comparatives (in parallel with the replacement of \textit{als} by \textit{wie} in equatives), then also in hypothetical comparatives  and in comparatives expressing equality (\citealt[178]{eggs2006}; see also \citealt{jaeger2010}). At the time of the appearance of \textit{wie} in comparatives, \textit{ob} was already obsolete in conditional clauses; hence, combinations such as \textit{wie ob} and \textit{als wie ob} were not possible. Consider now the following Bavarian example for \textit{als wie wenn} (\citealt[280, ex. 582]{jaeger2016habil}, citing \citealt[19]{alber1994}):

\ea \gll Er locht, \textbf{als} \textbf{wia} \textbf{wenn} er nimmr aufhearn kannt. \label{alswiewenn}\\
he laughs as as if he no.more stop.\textsc{inf} could.\textsc{3sg}\\
\glt `He is laughing as if he could never stop it.'
\z

The structure on the basis of Figures~\ref{alswietree} and \ref{alsobtree} should involve three complementiser heads. In the system of \citet{rizzi1997, rizzi2004}, the two CP projections marking the left periphery are not the only projections hosting clause-typing elements: for instance, an IntP can be located in between the two CP layers. However, in combinations like (\ref{alswiewenn}), the middle head is clearly not interrogative and hence the entire configuration is incompatible with basic cartographic assumptions. The structure of \textit{als wie wenn} is represented in \figref{treealswiewenn}.

Importantly, a rigid cartographic approach is not tenable for modelling clause typing, in addition to the problems mentioned in connection with information structure. I propose that the size of the CP-domain is flexible and it depends on the particular features involved in the given clause type and the availability of lexical elements specified for these features, as well as overtness requirements (essentially interface requirements) regarding the lexicalisation of the relevant features. Apart from complementisers, clause-typing operators are crucial because they can overtly encode clause-typing features. They are categorically distinct from complementisers heads, but both take part in overt encoding: features of a head can be checked off either by inserting an element into the head or by inserting a phrase into the specifier.\largerpage

\begin{figure} 
\caption{The combination \textit{als wie wenn}} \label{treealswiewenn}
\begin{forest} baseline, qtree
[CP
	[C$'$
		[C
			[als]
		]
		[CP
			[C$'$ [C [wie]] [CP [C$'$ [C [wenn]] [TP]]]]
		]
	]
]
\end{forest}
\end{figure}

At this point, three major questions might arise. First, given the possibility of accommodating two elements in a single CP, one might wonder why this is not possible for cases like \textit{als wie} (that is, \textit{als} in the specifier and \textit{wie} in the head). After all, a minimal CP would be undoubtedly more economical, in line with general principles of avoiding superfluous structure. I will argue that multiple CPs arise if they are semantically necessary: certain features cannot be encoded in a single CP (due to semantic incompatibility on a single lexical element). Notice that surface doubling may be underlyingly more complex, in that phonologically zero elements (if they are independently motivated by semantics) also occupy positions in the syntax. These questions will be elaborated in \sectref{sec:2degree} and in \chapref{ch:5}.

Second, given the relative flexibility of generating structure (there being no pre-given template), the question arises how the system can prevent overgeneralisation (the same concerns also applying to cartographic models, see \citealt{grewendorf2002} and \citealt{speyer2009}), among others). Just like in the case of the previous question, the answer lies in semantic restrictions: further CP layers are generated only if clause-typing features cannot be accommodated in a single projection, as required by the underlying semantic properties.

The third questions concerns to what extent the proposed model is specific for clause typing and whether it can be extended to other domains as well. As discussed already in \sectref{sec:2previous}, similar considerations have been discussed in the literature concerning projections related to information structure. I will return to information structure in \chapref{ch:6}; in essence, however, a flexible approach to such left-peripheral elements is indeed desirable, especially as there is no on-to-one correspondence between certain information structural categories and left peripheral positions (in West Germanic languages, foci are typically realised in situ). In addition, some information structural categories such as contrastive topics, which are readily assumed to have designated projections in cartographic approaches, seem to be by definition complex: topicality is not associated with contrast per se. According to \citet[267--268]{krifka2008}, a contrastive topic is essentially a combination of a topic and a focus (see also \citealt{roberts1996} and \citealt{buering1997, buering2003}): more precisely, in these cases an aboutness topic contains a focus. Consider the following example (\citealt[268, ex. 44]{krifka2008}):

\begin{exe}
\ex
\begin{xlist} 
\exi{A:} What do your siblings do?
\exi{B:} {[}My [SISter]\textsubscript{Focus}]\textsubscript{Topic} [studies MEDicine]\textsubscript{Focus}, and\newline [my [BROther]\textsubscript{Focus}]\textsubscript{Topic} is [working on a FREIGHT ship]\textsubscript{Focus}.
\end{xlist}
\end{exe}

In cartographic terms, a [topic] feature and a [focus] feature should have separate projections, yet it is clear that the highlighted phrases do not contain two separate constituents: rather, the very same phrase has both discourse properties.\footnote{In principle, the phrase in question could undergo movement from one position to the other, but this stance would introduce an otherwise unmotivated step in the derivation and it would not account for the prosodic properties associated with contrastive topics. Likewise, assuming a contrast feature (as done by \citealt{frascarellihinterhoelzl2007}) does not immediately account for the mixed properties of contrastive topics.} In other words, the combination of discourse-related semantic properties is also possible, similarly to the combination of clause-typing features.

Another domain where a non-cartographic approach seems to be fruitful is that of sentential adverbs. \citet{cinque1999} extended the cartographic framework to the syntax of adverbs, and this approach was later taken up by others such as \citet{alexiadou1997}, \citet{laenzlinger2004}, and \citet{haumann2007}. This approach faces various problems that are similar to the ones raised in this chapter regarding clause typing (see \citealt{ernst2014} for an overview). An alternative approach is termed the ``scopal'' approach, going back to \citet{jackendoff1972}; this framework was later extended by others such as \citet{ernst1984, ernst2002} and \citet{haider1998, haider2000, haider2004}; under this view, the observed ordering arise naturally based on the scopal restrictions, ultimately thus going back to semantic constraints. While the cartographic template suggests a rigid order for adverbs, some adverbs such as \textit{often} show considerable flexibility regarding their positions, without creating interpretive differences. This is problematic for the cartographic approach inasmuch as positional variation is assumed to be the surface result of two different projections (which, however, would imply interpretive differences), or to the fact that movement has taken place (which would in such cases be unmotivated), so that the cartographic template fails to provide the desired descriptive and explanatory adequacy (see also \citealt[119--120]{ernst2014}). The scopal approach provides a more flexible way to deal with ordering constraints, essentially building on the lexical properties of the individual elements rather than on pre-defined categories. The same considerations can be raised in connection with the ordering of adjectives.

These considerations indicate that a flexible approach is not restricted to clause typing but can be viewed as a more general principle underlying the generation of syntactic structures. Since it would be impossible to deal with all of these issues in a single book, I will restrict myself to discussing clause typing. In the following sections, I will sketch the proposed structure for the types of embedded clauses to be discussed in more detail in subsequent chapters. The reason why embedded clauses are discussed is that several functional elements appearing on the left periphery occur only in embedded clauses, which have therefore been much more discussed in the relevant literature in this respect.

\section{Embedded interrogatives} \label{sec:2interrogatives}
In embedded interrogatives, there are two important properties to be encoded: the interrogative nature of the clause, [Q] or [wh], and finiteness, [fin]. I will return to the distinction between [Q] and [wh] later: at this point, suffice it to say that [Q] is a disjunctive feature appearing in polar interrogatives, while [wh] is associated with constituent questions (see \citealt{bayer2004}). Both will be referred to as interrogative features in this section. In addition, certain restrictions apply since the clause is subordinated: in contrast to main clause polar questions, English requires an overt complementiser or an operator in embedded interrogatives; the clause must be syntactically typed (cf. \citealt[89]{bayerbrandner2008}, citing the Clausal Typing  Hypothesis of \citealt{cheng1991diss}). This is relatively straightforward as no distinctive interrogative intonation is available in embedded interrogatives in the languages under scrutiny. In turn, complementisers like \textit{if} are restricted to embedded clauses. Subordination itself does not have to be treated as a syntactic feature: the matrix predicate can impose selectional restrictions on the head of the argument CP. Finiteness does not have to be marked overtly in the CP.

In principle, an uninterpretable interrogative feature of a functional head can be checked off by merging an interrogative operator (on which the feature is interpretable), or an interrogative complementiser is inserted with an interpretable interrogative feature. Consider the following English examples:

\ea
\ea	I asked \textbf{if} Anthony had eaten the cheese.
\ex	I asked \textbf{who} had eaten the cheese.
\z
\z

The structure showing the relative position of \textit{if} is given in \figref{iftree}. The structure showing the relative position of \textit{who} is given in \figref{whotree}.

\begin{figure}
\caption{Clause typing with \textit{if}}\label{iftree}
\begin{forest} baseline, qtree
[CP 
	[\phantom{who\textsubscript{{[}wh{]}}}]
	[C$'$
		[C\textsubscript{{[}Q{]},{[}fin{]}}
			[if\textsubscript{{[}Q{]},{[}fin{]}}]
		]
		[TP]
	]
]
\end{forest}
\end{figure}

\begin{figure}
\caption{Clause typing with a \textit{wh}-element}\label{whotree}
\begin{forest} baseline, qtree
[CP
	[who\textsubscript{{[}wh{]}}]
	[C$'$
		[C\textsubscript{{[}wh{]},{[}fin{]}}]
		[TP]
	]
]
\end{forest}
\end{figure}

There is only a single CP, which can fulfil the function of marking the interrogative nature of the clause and finiteness: there is no need to postulate a separate projection for finiteness, since either \textit{if} or a zero complementiser can encode this property.\largerpage

Let us now turn to the Doubly Filled COMP data given in (\ref{dfc}) for English and Norwegian, repeated here as (\ref{dfcrepeat}):

\ea \label{dfcrepeat}
\ea[\%]{I wonder \textbf{which book that} Ralph is reading. \label{englishdfcrepeat}}
\ex[]{\gll Peter spurte \textbf{hvem} \textbf{som} likte bøker. \label{norwegiandfcrepeat}\\
Peter	asked.\textsc{3sg} who	that liked books\\
\glt `Peter asked who liked books.'}
\z
\z

Doubly Filled COMP effects can be observed across Germanic, at least dialectally. In German, such patterns are attested in several regional dialects, for instance in Alemannic and Bavarian (\citealt{bayerbrandner2008}). Consider the following examples from Alemannic (\citealt[88, ex. 3b and 4b]{bayerbrandner2008}):

\ea
\ea \gll I frog mich \textbf{wege} \textbf{wa} \textbf{dass} die zwei Autos bruchet. \label{wegenwasdass}\\
I ask \textsc{refl} for what that they two cars need\\
\glt `I wonder why they need two cars.'
\ex \gll I ha koa Ahnung, \textbf{mid} \textbf{wa} \textbf{für-e} \textbf{Farb} \textbf{dass}-er zfriede wär. \label{wasfuernefarbe}\\
I have no idea with what for-a colour that-he content would.be\\
\glt `I have no idea with what colour he would be happy.'
\z 
\z

In all of these cases, the complementiser is specified for finiteness but not for [Q] or [wh]. Based on the representations in \figref{iftree} and \figref{whotree} showing the relative position (syntactic status) of complementisers and \textit{wh}-operators, the Doubly Filled COMP patterns should have the representation of \figref{dfctree}.

\begin{figure}
\caption{Doubly Filled COMP}
\label{dfctree}
\begin{forest} baseline, qtree
[CP
	[which book\textsubscript{{[}wh{]}}]
	[C$'$
		[C\textsubscript{{[}wh{]},{[}fin{]}}
			[that\textsubscript{{[}fin{]}}]
		]
		[TP]
	]
]
\end{forest}
\end{figure}

The point is that the two elements, \textit{that} and the \textit{wh}-phrase, can be merged directly and it is not necessary to assume two separate projections for the two distinct functions: the complementiser marks finiteness in the head and the \textit{wh}-phrase checks off the [wh] feature in the specifier. In traditional X-bar terms, this is compatible with the assumption that a phrase has distinct head and specifier positions that can be filled by overt elements each (but note that a strict distinction is not necessary in minimalist terms and will be revised in \chapref{ch:3}).

One clear advantage of this analysis is that it is minimal, as opposed to a rigid cartographic template. Naturally, one of the main reasons for cartographic approaches is that they can describe word order restrictions as the order of projections in the template is supposed to match the observed empirical data. However, the fact that the order of the interrogative phrase and the finite complementiser is fixed is also predicted by \figref{dfctree} since the specifier precedes the complementiser head, which is directly merged with TP. In sum, there is no need to assume multiple projections to account for word order restrictions.

The question arises whether the linear order of \textit{wh}-elements and finite complementisers is universal: the reverse order might indicate a doubling of the CP. As far as linear order is concerned, Hungarian seems to constitute a counterexample to the linear restriction, as demonstrated by (\ref{hungthatwho}):

\ea \label{hungint}
\ea \gll Nem tudom, \textbf{(hogy)} \textbf{ki} ette meg a sajtot. \label{hungthatwho}\\
not know.\textsc{1sg} \phantom{\textbf{(}}that who ate.\textsc{3sg} \textsc{prt} the cheese.\textsc{acc}\\
\glt `I do not know who has eaten the cheese.'
\ex \gll Nem tudom, \textbf{(hogy)} Mari ette\textbf{-e} meg a sajtot. \label{hungthatq}\\
not know.\textsc{1sg} \phantom{\textbf{(}}that Mary ate.\textsc{3sg-Q} \textsc{prt} the cheese.\textsc{acc}\\
\glt `I do not know if it was Mary who has eaten the cheese.'
\z 
\z

In (\ref{hungthatwho}), the \textit{wh}-element immediately follows the finite complementiser in the linear structure. However, as indicated by (\ref{hungthatq}), the interrogative element is not necessarily adjacent to the C head: a topic (the subject DP) and the finite verb appear in between the two elements. In fact, topics\footnote{Topics are iterable in the language, see \citet{ekiss2002}.} are available in front of the \textit{wh}-element in constituent questions as well:

\ea \gll Nem tudom, \textbf{(hogy)} tegnap \textbf{ki} ette meg a sajtot. \label{cheeseki}\\
not know.\textsc{1sg} \phantom{\textbf{(}}that yesterday who ate.\textsc{3sg} \textsc{prt} the cheese.\textsc{acc}\\
\glt `I do not know who has eaten the cheese.'
\z 

This indicates that the [wh]/[Q] property in Hungarian is located considerably lower in the clause than the CP, as pointed out already by \citet{horvath1986} and \citet{ekiss2002}. Importantly, \textit{wh}-operators differ from relative operators in this respect, which target the CP (see \citealt{horvath1986}). Moreover, as shown by \citet{liptakzimmermann2007}, a Hungarian clause may host a \textit{wh}-element clause-internally as in (\ref{cheeseki}) above and a relative operator in the CP, and the \textit{wh}-operator can be extracted without triggering an island violation effect, indicating that the CP is not a landing site for the \textit{wh}-element. This indicates that high clause-typing markers (including the subordinator \textit{hogy} and relative operators) are not in the same left periphery as low clause-typing markers (including the interrogative marker -\textit{e} and interrogative operators), so that the elements under scrutiny in (\ref{hungint}) cannot be captured by postulating a split CP in the sense of \citet{rizzi1997}.

The projection hosting interrogative elements is generally assumed to be a FocP (see \citealt{ekiss2002} and \citealt{vancraenenbroeckliptak2006}), which is a projection located above TP (see also \citealt{ekiss2008li, ekiss2008} on its exact relative position). There are, however, reasons to believe that interrogative elements, such as the question particle -\textit{e} occurring in polar questions, appear in this position independently of focus (see also \citealt{bacskaiatkari2017atoh}). For this reason, I am going to refer to this projection simply as FP, standing for functional projection. Note that the appearance of such a functional projection (hosting e.g. foci) is not surprising in the light of \citet{poletto2006}, see \sectref{sec:2poletto} above. 

The structure for the subclause in (\ref{hungthatwho}) is given in \figref{treehungthatwho}. The structure for the subclause in (\ref{hungthatq}) is given in \figref{treehungthatq}.

\begin{figure}
\caption{The CP and the FP in constituent questions} \label{treehungthatwho}
\begin{forest} baseline, qtree
[CP
	[C$'$
		[C\textsubscript{{[}wh{]},{[}fin{]}}
			[(hogy)\textsubscript{{[}fin{]}}]
		]
		[\ldots
			[FP [ki\textsubscript{j{[}wh{]}}] [F$'$ [F\textsubscript{{[}wh{]}} [ette\textsubscript{i}]] [TP [t\textsubscript{i} t\textsubscript{j} meg a sajtot,roof]]]]
		]
	]
]
\end{forest}
\end{figure}

\begin{figure}
\caption{The CP and the FP in polar questions} \label{treehungthatq}
\begin{forest} baseline, qtree
[CP
	[C$'$
		[C\textsubscript{{[}Q{]},{[}fin{]}}
			[(hogy)\textsubscript{{[}fin{]}}]
		]
		[\ldots
			[FP [Mari\textsubscript{j}] [F$'$ [F\textsubscript{{[}Q{]}} [ette\textsubscript{i}-e\textsubscript{{[}Q{]}}]] [TP [t\textsubscript{i} t\textsubscript{j} meg a sajtot,roof]]]]
		]
	]
]
\end{forest}
\end{figure}

The dots between the CP and the FP indicate the optional topic field. As can be seen, the complementiser \textit{hogy} is regularly located in the CP, while the overt marking of the interrogative property is tied to a lower functional domain (FP). Naturally, the typing of the clause is tied to the CP and hence the [Q]/[wh] feature is rather passed onto the F head from the C head, establishing some kind of agreement between the two. One might wonder whether the FP is not rather a lower CP, resembling the analysis of \citet{rizzi1997, rizzi2004} in that topics can appear between two CPs. However, there are various problems with such a stance and it cannot be maintained. First, as was pointed out in connection with \citet{rizzi1997, rizzi2004}, configurations containing topics between two genuine complementisers are generally not attested in the form predicted by the cartographic template (cf. also the observations of \citealt{beninca2001}). Second, the higher functional head containing \textit{hogy} is related to finiteness and the overt marking of the clause type is tied to the lower functional head, which again would not conform to the Force--Fin distinction. Third, historical data from Middle Hungarian indicate that the marking of [Q] could be split between the CP and the FP: in this period, the interrogative C head in polar questions was still \textit{ha} `if' (inherited from Old Hungarian, where the pattern was similar to English), and the F head contained -\textit{e}.\footnote{This is shown by the following example (\citealt[121, ex. 15a]{bacskaiatkari2022oup}):

\ea \gll kérdette tülle \textbf{ha} nyughatik\textbf{e}\\
asked.\textsc{3sg} (s)he.\textsc{abl} if rest.\textsc{possib.3sg.q}\\
\glt `(s)he asked him/her whether (s)he could rest' (Witch Trial 13; from 1724)
\z

The interrogative complementiser is \textit{ha} `if' and the question particle -\textit{e} appears cliticised onto the finite verb in the lower periphery.} This is a doubling pattern that is actually incompatible with the cartographic template, as it is indeed counter-intuitive that the same property, [Q], would be marked twice on the same periphery (see \citealt{bacskaiatkaridekany2014oup} for details). 

\begin{sloppypar}
In sum, Hungarian polar questions seem to demonstrate a pattern where clause typing is distributed among two distinct peripheries: in the CP and in a lower functional domain immediately above TP. This is reminiscent rather of the lower periphery proposed by \citet{poletto2006}, even though the FP is not equivalent to a functional vP. Nevertheless, the stress pattern of the Hungarian clause, as demonstrated by \citet{szendroi2001diss}, suggests that the FP constitutes an Intonational Phrase that is sent to the interfaces as a unit, and in this sense it is probable that the phase boundary is the FP, whereas topics are extrametrical. The point is that the word order seen in (\ref{hungint}), which is the opposite of the Germanic word order, is simply the result of there being two distinct functional layers, and there is no reason to assume a cartographic template.
\end{sloppypar}

I will return to the left periphery of embedded interrogatives in \chapref{ch:3}; what matters at this point is that the marking of finiteness and [wh]/[Q] does not require multiple CPs with designated functions. In Germanic, a single CP suffices for overtly marking both, and the co-occurrence of a \textit{wh}-element and a finite complementiser constitutes a classical example of Doubly Filled COMP.

\section{Relative clauses} \label{sec:2relative}
In relative clauses, the relative nature of the clause has to be encoded: this is treated as a clause type by \citet{rizzi1997}, and I will simply refer to this property as [rel] for the time being. Overt marking can happen either by a complementiser or by an operator, though in languages like English overt marking is not necessary in all cases. 

I will restrict myself to the discussion of finite relative clauses, though it is worth mentioning that in several languages there also non-finite relative clauses (see e.g. \citealt{ackermannikolaeva2013} for a typological perspective). To a limited degree, this is true for English as well, as pointed out already by \citet{chomsky1977}.\footnote{This is altogether a very restricted option and it seems to be confined to complements of prepositions: \citet[13]{radford2019} reports that this pattern did not prove to be productive in his corpus data.} Consider the following sentences (see also the examples of \citealt[161--162]{law2000}, \citealt[199]{aarts2011}):

\ea 
\ea A desk is a dangerous place [\textbf{from which} to view the world].\\(John le Carré, \textit{Tinker Tailor Soldier Spy})
\ex London is becoming a cheaper place [\textbf{in which} to live and work], according to a new survey.\\(\citealt[13, ex. 18a]{radford2019})
\z
\z

Regarding the [rel] feature on lexical elements, it is important to stress that [rel] apparently comes with an [edge] feature and triggers the movement of the given operator: there is no ``relative in situ'' (at least in the languages under scrutiny, cf. the discussion in \citealt[122]{bacskaiatkari2014diss}). For this reason, the relative operator moves to the left periphery even if a complementiser with an interpretable [rel] feature is inserted as the [edge] feature has to be satisfied by merging a further element. This suggests that the doubling of two overt relative elements is possible, contrasting with Doubly Filled COMP patterns in embedded interrogatives in the sense that doubling in interrogatives regularly involves the combination of an operator specified as [wh] and a complementiser specified as [fin].

Consider the following examples from English:

\ea
\ea	This is the book \textbf{that} explains the difference between cats and tigers. \label{relwhich}
\ex	This is the book \textbf{which} explains the difference between cats and tigers. \label{relthat}
\z 
\z

English has two possibilities for the overt marking of relative clauses in the standard variety: either by lexicalising the operator, as in (\ref{relwhich}), or by inserting the complementiser \textit{that}, as in (\ref{relthat}). The canonical analysis of the position of these elements is given in \figref{relheadtree} for relative complementisers and in \figref{reloperatortree} for relative pronouns (again, only overt elements are indicated).

\begin{figure} 
\caption{Relative complementisers} \label{relheadtree}
\begin{forest} baseline, qtree
[CP 
	[\phantom{that}]
	[C$'$
		[C\textsubscript{{[}rel{]},{[}fin{]}}
			[that\textsubscript{{[}rel{]},{[}fin{]}}]
		]
		[TP]
	]
]
\end{forest}
\end{figure}

\begin{figure} 
\caption{Relative operators} \label{reloperatortree}
\begin{forest} baseline, qtree
[CP
	[which\textsubscript{{[}rel{]}}]
	[C$'$
		[C\textsubscript{{[}rel{]},{[}fin{]}}]
		[TP]
	]
]
\end{forest}
\end{figure}


As can be seen, there is no need to generate a further layer for marking finiteness and hence a single CP suffices. In \figref{relheadtree}, a covert operator moves to [Spec,CP] as there is no relative in situ, and in \figref{reloperatortree}, a zero finite complementiser is located in C, but these elements do not play a role in overt marking.

The structures in Figures~\ref{relheadtree}  and \ref{reloperatortree} suggest that the doubling of the operator and the complementiser may be possible. In English, Doubly Filled COMP patterns are attested in relative clauses as well, though apparently less frequently than in embedded questions (this issue will be discussed in \chapref{ch:4} in detail). Consider (\citealt[59, ex. 85]{vangelderen2013}):

\ea
\ea	This program \textbf{in which that} I am involved is designed to help low-income first generation attend a four year university and many of the resources they\ldots
\ex	It's down to the community \textbf{in which that} the people live.
\z 
\z

Doubling patterns are assigned the structure given in \figref{treeinwhichthat} in the current model.

\begin{figure} 
\caption{Doubling in relative clauses} \label{treeinwhichthat}
\begin{forest} baseline, qtree
[CP
	[in which\textsubscript{{[}rel{]}}]
	[C'
		[C\textsubscript{{[}rel{]},{[}fin{]}}
			[that\textsubscript{{[}rel{]},{[}fin{]}}]
		]
		[TP]
	]
]
\end{forest}
\end{figure}

As indicated, the inserted complementiser is also specified as [rel] and does not only mark finiteness, unlike what was seen in interrogatives. As was mentioned above in connection with the obligatory leftward movement of relative operators, this is expected. In principle, \textit{that} could also be merely a finiteness marker: English \textit{that} is ambiguous between the relative complementiser and the general finite subordinator. 

However, this ambiguity does not necessarily arise in other languages. In German, the general finite subordinator is \textit{dass} `that'. Standard German uses the relative pronoun strategy in relative clauses and not the complementiser strategy. However, South German dialects regularly form relative clauses (with a nominal head) using the complementiser \textit{wo}, see \citet{brandner2008} and \citet{brandnerbraeuning2013}, among others. This is illustrated for Alemannic in (\ref{relwo}) below (\citealt[140, ex. 23]{brandnerbraeuning2013}):

\ea \gll Ich suech ebber \textbf{wo} mer helfe künnt. \label{relwo}\\
I search someone \textsc{rel} I.\textsc{dat} help.\textsc{inf} could\\
\glt `I am looking for someone who could help me.'
\z

Relative operators (\textit{d}-pronouns) are essentially borrowings from Standard German, and they either trigger V2 or they can co-occur with \textit{wo} in the CP (cf. \citealt{weise1917}). In any case, the relative head is filled by an overt element; further, \textit{wo} is specified as [rel] and cannot be treated as a mere finiteness marker, as that particular element would be \textit{dass} even in these dialects. An example from Hessian illustrating doubling is given in (\ref{deswofleischer}) below (\citealt[ex. 3d]{fleischer2016}):

\ea \gll Des Geld, \textbf{des} \textbf{wo} ich verdiene, des geheert mir. \label{deswofleischer}\\
the.\textsc{n} money that.\textsc{n} \textsc{rel} I earn.\textsc{1sg} that.\textsc{n} belongs I.\textsc{dat}\\
\glt `The money that I earn belongs to me.'
\z 

The importance of this is that doubling in South German relative clauses does not conform to a Force--Finiteness distinction, since the element to the right cannot be identified as a designated finiteness marker: it is a relative complementiser. Given that the relative operator marks the same property, [rel], a double CP in this case would involve two designated relative CPs in a cartographic approach, which contradicts the very idea of the cartographic template consisting of distinct functions distributed over distinct projections.

What Germanic relative clauses with doubling patterns demonstrate is rather the consequence of the given dialects having no genuine relative operators by default, as these varieties regularly employ an overt complementiser. This not only applies to German but also to English: in Middle English, \textit{wh}-based relative operators were an innovation alongside the already existing \textit{that} head, see \cite{vangelderen2004, vangelderen2009}. There are also other languages that regularly use relative complementisers: this is true for Mainland Scandinavian languages using \textit{som} and Modern Icelandic using \textit{sem} and \textit{er} (see \citealt{thrainsson2007} on the loss of relative operators in Icelandic).

I will return to the left periphery of relative clauses in \chapref{ch:4}; what matters at this point is that the marking of finiteness and [rel] does not require multiple CPs with designated functions, just like what was established for embedded interrogatives. In Germanic, a single CP suffices for overtly marking both, and the co-occurrence of a relative operator and a finite relative complementiser is not compatible with cartographic templates; rather, it constitutes a regular example of Doubly Filled COMP.

\section{Embedded degree clauses} \label{sec:2degree}
In the constructions examined thus far (embedded interrogatives and relative clauses), there was no evidence for the necessity of a double CP layer, as a single CP can accommodate both the operator and the complementiser, accounting also for the order (operator\,+\,complementiser). This indicates that a minimal, feature-based analysis is tenable and in fact preferable to a pre-defined cartographic template. However, the question arises whether CP-doubling is possible and if so, how the proposed model can accommodate it. In this section, I am going to argue that a double CP is necessary in comparative subclauses (see also the discussion in \sectref{sec:2introducing}), which follows from semantic reasons.

Embedded degree clauses fall into two major types: degree equatives (expressing the equality of two degrees) and comparatives (expressing the inequality of two degrees), as illustrated in (\ref{paradigm}):

\ea \label{paradigm}
\ea Ralph is as tall \textbf{as} Mary is. \label{paras}
\ex Ralph is taller \textbf{than} Mary is. \label{parthan}
\z 
\z

In (\ref{paras}), the subclause introduced by \textit{as} expresses that the degree to which Mary is tall is the same as to which Ralph is tall, while in (\ref{parthan}) the subclause introduced by \textit{than} expresses that the degree to which Mary is tall is lower than the degree to which Ralph is tall. In line with my previous approach (see \citealt{bacskaiatkari2014diss}), I assume that \textit{as} and \textit{than} are complementisers. They are selected by the matrix degree elements (\textit{as} in equatives and -\textit{er} in comparatives; see \citealt[22--23]{lechner2004} and \citealt[45--53]{bacskaiatkari2014diss} on selectional restrictions).

In addition to the equative and the comparative complementiser, non-standard English varieties may allow an overt degree operator (appearing together with a gradable predicate). Consider the examples in (\ref{asthanhow}) below (see \citealt[91--92]{bacskaiatkari2018langsci} for discussion):

\ea \label{asthanhow} 
\ea[\%]{Ralph is as tall \textbf{as how tall} Mary is.}
\ex[\%]{Ralph is taller \textbf{than how tall} Mary is.}
\z 
\z

In these cases, the comparative operator is overt in the form of \textit{how}. Note that the comparative operator is a relative operator (see \citealt{bacskaiatkari2014diss} for details on this; the original insight goes back to \citealt{chomsky1977}, who detected the availability of \textit{wh}-operators in comparatives) and hence its movement is triggered by a [rel] feature, in line with what was said about relative clauses in \sectref{sec:2relative}.

\begin{sloppypar}
The phenomenon is not restricted to English but it can be observed in languages\slash dialects generally that permit overt comparative operators (\citealt[98--129]{bacskaiatkari2014diss}). Consider the following examples from Dutch\footnote{The Dutch data stem from the cross-Germanic survey of \citet[51--52, 61--62]{bacskaiatkaribaudisch2018}. The informants show differing judgements regarding (\ref{alshoe}) and (\ref{danhoe}). The inter-speaker variation with respect to comparatives involving an overt operator was also pointed out in a previous questionnaire, see \citet[115]{bacskaiatkari2014diss}.}:
\end{sloppypar}

\ea
\ea[]{\gll Mary is even oud \textbf{als} Peter vorig jaar was.\\
Mary is just.as old as Peter last year was\\
\glt `Mary is as old as Peter was last year.'}
\ex[]{\gll Mary is ouder \textbf{dan} Peter vorig jaar was.\\
Mary is older than Peter last year was\\
\glt `Mary is older than Peter was last year.'}
\ex[\%]{\gll Mary is even oud \textbf{als} \textbf{hoe} \textbf{oud} Peter vorig jaar was. \label{alshoe}\\
Mary is just.as old as how old Peter last year was\\
\glt `Mary is as old as Peter was last year.'}
\ex[\%]{\gll Mary is ouder \textbf{dan} \textbf{hoe} \textbf{oud} Peter vorig jaar was. \label{danhoe}\\
Mary is older than hoe old Peter last year was\\
\glt `Mary is older than Peter was last year.'}
\z 
\z

In this way, it appears that embedded degree clauses demonstrate a ``complementiser\,+\,operator'' order that is exactly the reverse of what was seen in doubling patterns in embedded interrogatives and in relative clauses. Evidently, doubling in these cases cannot be simply represented as a result of the ``specifier\,+\,head'' order, given that in the languages under scrutiny upward movement of elements results in merging them to the left of other (higher) elements. This suggests that there is a double CP in these configurations.

Accordingly, the structure of the left periphery of clauses such as (\ref{asthanhow}) is represented in \figref{asthanhowtree}.

\begin{figure} 
\caption{Doubling in comparative clauses} \label{asthanhowtree}
\begin{forest} baseline, qtree
[CP
	[C$'$
		[C\textsubscript{{[}compr{]}}
			[as\textsubscript{{[}compr{]},{[}MAX{]}}\\than\textsubscript{{[}compr{]},{[}MAX{]}}]
		]
		[CP
			[how\textsubscript{{[}rel{]},{[}compr{]}} tall]
			[C$'$ [C\textsubscript{{[}rel{]},{[}compr{]},{[}fin{]}}] [TP]]
		]
	]
]
\end{forest}
\end{figure}

As indicated, the comparative nature of the clause, [compr], is marked both by the overt operator and by the overt complementiser; in line with \citet{rizzi1997}, comparative can be treated as a clause type (Force in Rizzi's analysis). The lower head is specified as [rel], and this property induces the leftward movement of the operator. The properties [rel] and [compr] are not tied together: there are naturally ordinary relative clauses without a comparative feature, and comparative complements are not necessarily clauses -- Italian, for instance, has PP-complements headed by the preposition \textit{di} `of'.

The question arises why a second C head is inserted. This has to do with comparative semantics. As shown by \citet{hohauszimmermann2021}, comparative constructions involve a maximality operator (given as {[}MAX{]} in \figref{asthanhowtree} above) and a comparative operator in the semantics; importantly, the maximality operator is not tied to a particular syntactic projection (or to the notion of degree) but in English it is expressed by the complementisers \textit{as} and \textit{than}. Since it is not a clause type, the representation in \figref{asthanhowtree} does not mark it on ``C'', indicating that this is not a syntactic feature to be checked off. The presence of the maximality operator is necessary semantically, and the comparative operator is in the scope of the maximality operator.

The presence of the maximality operator can be detected in comparatives because comparatives are downward entailing environments. Such environments in turn license negative polarity items, as pointed out already by \citet{seuren1973}; see \citet{ladusaw1979diss} on the relation between downward entailment and negative polarity contexts. Consider the following examples:

\ea
\ea He would rather continue complaining than \textbf{lift a finger} to improve his life.
\ex Ralph has spent more time travelling than \textbf{any} other member of the family (has). \label{comptravel}
\z 
\z

Taking the example in (\ref{comptravel}), the sentence entails that Ralph has spent a certain amount of time travelling, call it \textit{d}, and for all the other members of the family it is true that the amount they travelled is lower, call it $d'$, hence $d'$ is always lower on a scale than \textit{d} is, while the exact value of \textit{d} is not necessarily known in the context. These issues will be discussed in \chapref{ch:5} in detail.

Regarding the lower CP, a complementiser is in a feature-checking relationship with the operator, in essentially the same way as what was established for relative clauses. The Minimal Link Condition is satisfied in that the operator moves to the closest possible projection. By contrast, there is no feature-checking mechanism taking place in the higher CP in the same way and the inserted element is a head (see a cross-linguistic overview in \citealt{bacskaiatkari2016alh}).

While the comparative operator always moves from within the clause, it is not always overt, as is evident from Standard English, see (\ref{paradigm}) above (see also the discussion in \citealt[98--129]{bacskaiatkari2014diss}). Nevertheless, even if the lower CP contains covert elements only, its presence is justified by the semantics, without thus resorting to a predefined template. Note that it is also possible to have an overt lower complementiser in certain languages, as in non-standard German, see (\ref{alswie})/(\ref{alswierepeat}), repeated here as (\ref{alswierepeat2}):

\ea[\%]{\gll Ralf ist größer \textbf{als} \textbf{wie} Maria. \label{alswierepeat2}\\
Ralph is taller than as Mary\\
\glt `Ralph is taller than Mary.'}
\z

As pointed out at the end of \sectref{sec:2relative}, this involves two CPs with two overt C heads. Since the element \textit{wie} `how' in interrogatives corresponds to English \textit{how} and is a regular degree operator, it cannot be treated as the comparative operator in comparative subclauses (see the discussion in \citealt[497--499]{bacskaiatkari2014dia} and \citealt[117--118, 223--226]{bacskaiatkari2014diss}). One of the main arguments is that patterns like (\ref{asthanhow}) are not allowed with \textit{wie}:

\ea
\ea[*]{\gll Ralf ist größer \textbf{als} \textbf{wie} \textbf{groß} Maria ist. \label{alswieap}\\
Ralph is taller than as tall Mary is\\
\glt `Ralph is taller than Mary.'}
\ex[\%]{\gll Ralf ist größer \textbf{als} \textbf{wie} Maria \textbf{groß} ist. \label{alswiestrand}\\
Ralph is taller than as Mary tall is\\
\glt `Ralph is taller than Mary.'}
\z 
\z

As can be seen, the construction in (\ref{alswieap}) is not acceptable even for speakers who otherwise allow the combination \textit{als wie}; on the other hand, (\ref{alswiestrand}) clearly demonstrates that the overtness of the adjective in itself is not problematic.

Taking this into consideration, the structure of \textit{als wie} involving features is given in \figref{treealswiefeatures}.

\begin{figure} 
\caption{Features in comparatives} \label{treealswiefeatures}
\begin{forest} baseline, qtree
[CP
	[C$'$
		[C\textsubscript{{[}compr{]}}
			[als\textsubscript{{[}compr{]},{[}MAX{]}}]
		]
		[CP
			[\phantom{TP}]
			[C$'$ [C\textsubscript{{[}rel{]},{[}compr{]},{[}fin{]}} [wie\textsubscript{{[}compr{]}}]] [TP]]
		]
	]
]
\end{forest}
\end{figure}

Since the operator moves to the lower specifier and checks off the relative feature there, the two complementisers are not merged directly together. Just as in \figref{asthanhowtree}, a double CP is needed, which it ultimately follows from comparative semantics.

The question arises whether a single CP is possible in comparative constructions at all. In order to do that, the maximality operator would have to be located outside the subordinate clause, and the comparative head should be a relative head, so that the comparative operator can move. As I will show in \chapref{ch:5} in detail, the maximality operator can be lexicalised by the matrix degree element as well under certain conditions (in line with the assumption of \citealt{hohauszimmermann2021} that the maximality operator is not tied to a particular syntactic category). This is in principle available in non-degree equatives (similatives) and in degree equatives, but not in comparatives proper.

There is direct evidence from Old High German that at least in non-degree equatives a single CP was sufficient. The element \textit{wie} appears in Early New High German in equatives (first in non-degree equatives, then in degree equatives, see \citealt{jaeger2010, jaeger2018}), and it goes back to Middle High German \textit{swie}, which in turn stems from Old High German \textit{so wie so}, see \cite[488]{jaeger2010}. This is illustrated in (\ref{sowieso}) below (\citealt[488, ex. 46]{jaeger2010}, quoting \citealt{schrodt2004}):

\ea \gll er bi unsih tod thulti, \textbf{so} \textbf{wio} \textbf{so} er selbo wolti \label{sowieso}\\
he by us death suffered as how as he self wanted\\
\glt `he suffered death by us, as he himself wished' (\textit{Otfrid} V, 1, 7)
\z

The combination \textit{so wie so} appears in free relatives, just as \textit{so wer so} or \textit{so waz so} in non-comparative free relatives, where the \textit{so}+WH combination is in [Spec,CP] and \textit{so} is in C, see \citet[488]{jaeger2010}, cf. \citet{behaghel1928} and \citet{paul1920band3}. In addition, \textit{so} was used as a C head in \textit{as}-clauses in Old and Middle High German, see \citet[470--472]{jaeger2010}, as demonstrated by (\ref{socomp}) below (\citealt[472, ex. 14]{jaeger2010}):

\ea \gll ir scult wesen fruot. \textbf{so} die natrun. \label{socomp}\\
you.\textsc{pl} should.\textsc{2pl} be cunning as the.\textsc{pl} snakes\\
\glt `you should be cunning as snakes' (\textit{Physiologus} 142v, 6)
\z

Moreover, \textit{so} was available as a relative complementiser on its own in Middle High German (and beyond, see \citealt{brandnerbraeuning2013}), see \citet[405]{paul2011}, \citet[98]{ferraresiweiss2011}. Consider (\citealt[98, ex. 30]{ferraresiweiss2011}, quoting \citealt[414]{paul2007}):

\ea \gll ich hete ir doch vil~lihte ein teil geseit, der vil grossen liebe \textbf{so} min herze an si hat \label{sorel}\\
I have.\textsc{cond.1sg} she.\textsc{dat} \textsc{prt} perhaps a.\textsc{m.acc} part say.\textsc{ptcp} the.\textsc{f.gen} much great.\textsc{f.gen} love as my.\textsc{n} heart at she has\\
\glt `perhaps I should have expressed to her a part of the great love that my heart has towards her' (\textit{Rudolf von Rotenburg} VII, 2,1--2)
\z

The examples in (\ref{socomp}) and (\ref{sorel}) illustrate that there is independent evidence for the availability of \textit{so} as a similative complementiser and as a relative complementiser. Regarding (\ref{sowieso}), then, it is justified to assume that \textit{so} is equipped with both a [rel] and a [compr] feature. The relevant structure is shown in \figref{treesowie} below.

\begin{figure} 
\caption{Doubling in  similatives} \label{treesowie}
\begin{forest} baseline, qtree
[CP
	[so wie\textsubscript{{[}compr{]},{[}rel{]}}]
	[C'
		[C\textsubscript{{[}compr{]},{[}rel{]}}
			[so\textsubscript{{[}compr{]},{[}rel{]}}]
		]
		[TP]
	]
]
\end{forest}
\end{figure}

In this case, there is only one CP, and it is an instance of a Doubly Filled COMP pattern, just like what was attested in ordinary relative clauses. Importantly, this construction is available in non-degree equatives (similatives): note that there is no matrix degree element, and there is no maximality operator needed at all, since the construction expresses similarity but not (degree) equality. I will argue in \chapref{ch:5} that degree equatives can also have a single CP, whereas this pattern is not attested in comparatives proper. At this point, what matters for us is that doubling patterns can arise in comparative constructions as well, and while they sometimes do indeed require a double CP, this is not necessarily the case. In other words, while comparative constructions provide evidence for a complex left periphery, they do not support the existence of a cartographic template.

\section{Summary} \label{sec:2summary}
This chapter presented the basic assumptions concerning a minimal, feature-based approach to the syntax of functional left peripheries, showing that the proposed analysis applies to various clause types, in each case correctly predicting the surface order of clause-typing elements appearing in combinations. This approach contrast with the cartographic approach, which postulated designated functional projections in narrow syntax, both in the CP-periphery and in a lower functional vP-periphery. While there have been calls for a more minimal CP in the literature (see \citealt{sobin2002}, building on \citealt{pesetsky1982} and \citealt{sobin1987}), the present proposal aims at providing a unified framework applicable across clause types and languages. In particular, it has been shown that a single CP is satisfactory for doubling phenomena in embedded interrogatives and in relative clauses, while doubling in embedded degree clauses normally requires a double CP (without, however, requiring a pre-defined cartographic template). So far, only the basic outline of the model has been presented; further investigation of the data reveals that there are various asymmetries that should also be accounted for. In this vein, \chapref{ch:3} will have a more thorough look at embedded interrogatives and \chapref{ch:4} offers a detailed investigation of relative clauses. \chapref{ch:5} discusses embedded degree clauses, also pointing out differences between equatives and comparatives proper, while \chapref{ch:6} addresses questions related to information structure and ellipsis, extending the model to domains that go beyond clause typing in a narrow sense.
