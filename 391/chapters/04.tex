\chapter{The left periphery of relative clauses} \label{ch:4}
\section{Introduction} \label{sec:4introduction}
This chapter is dedicated to the analysis of relative clauses, applying the framework established in \chapref{ch:2} and refined for interrogative clauses in \chapref{ch:3}. As discussed in \chapref{ch:3}, the notion of the ``Doubly Filled COMP Filter'' emerged in the literature primarily in connection with relative clauses in English. One of the most important questions to be dealt with in this chapter is therefore whether and to what extent the conclusions drawn in \chapref{ch:3} for embedded interrogatives hold for relative clauses in West Germanic. On the one hand, combinations of operators and complementisers will be examined; as was pointed out in \chapref{ch:2} already, such combinations are particularly important as they are not compatible with traditional cartographic approaches. On the other hand, the question will be addressed why and to what extent there seems to be a preference for relative complementisers over relative pronouns in Germanic.\footnote{Unlike in the case of embedded interrogatives and, as will be shown in \chapref{ch:5}, hypothetical comparatives, verb movement will not be discussed in connection with relative clauses. The reason is that verb movement to C does not occur in genuine relative clauses in West Germanic. There are so-called V2 relative clauses in German, yet these are syntactically paratactic configurations, as shown by \citet{gaertner1998, gaertner2001}, \citet{endrissgaertner2005}, \citet{ebertendrissgaertner2007}; see also \citet[144--150]{coniglio2019}.} As will be discussed, this preference also makes doubling patterns less likely to appear in relative clauses than in embedded constituent questions in dialects that allow the relevant patterns. Since the data from South German dialects are of especial relevance in this respect, I am first going to summarise the most important findings of \citet{brandnerbraeuning2013}, who discuss the syntactic status of the most widespread complementiser in South German, namely \textit{wo}. Following this, I will discuss other elements (complementisers and operators, as well as the combination of the two).

This chapter is structured as follows. Section \ref{sec:4relative} discusses the findings of \citet{brandnerbraeuning2013}. Section \ref{sec:4relativepronouns} examines the differences between relative pronouns and relative complementisers in the framework put forward in this book. Section \ref{sec:4variation} offers further insights in this respect, by looking at changes and variation in English. The interaction between pronouns and complementisers is also examined in terms of combinations in German: \sectref{sec:4doubling} discusses doubling in ordinary relative clauses, \sectref{sec:4doublingfree} discusses doubling in free relatives, and \sectref{sec:4triple} discusses triple combinations. Section \ref{sec:4equative} examines equative relative clauses, a construction that highly resembles comparison and is therefore also relevant for the further discussion in this book.

\section{Complementisers -- Brandner \& Bräuning (2013)} \label{sec:4relative}
In South German dialects, relative clauses are generally introduced by the element \textit{wo},\footnote{As will be discussed in \sectref{sec:4relativepronouns}, there is some variation in this respect (see \citealt{fleischer2004, fleischer2016}, \citealt{weiss2013}); nevertheless, \textit{wo} is by far the most frequent complementiser and its status as well as its historical development should be discussed in more detail, since it raises some questions especially due to the fact that it is phonologically identical to the adverbial \textit{wo}, which is attested as a relative pronoun in locative relative clauses across varieties of German (including the standard variety).} while Standard German uses demonstrative pronouns; depending on the dialect and the exact syntactic configuration (cf. \citealt{bayer1984}, \citealt{salzmann2006, salzmann2009}), the demonstrative pronoun can occur additionally in the dialectal patterns as well (\citealt[131--132]{brandnerbraeuning2013}). The difference is illustrated in (\ref{germanparadigmrel}) below (\citealt[132, ex. 1 and 2]{brandnerbraeuning2013}):

\ea \label{germanparadigmrel}
\ea \gll \ldots{} der Mann \textbf{der} seine Schuhe verloren hat\\
{} the man that.\textsc{m} his shoes lost has\\
\glt `the man who has lost his shoes'
\ex \gll \ldots{} dea Mo \textbf{(dea)} wo seine Schu verlora hot\\
{} the man \phantom{\textbf{(}}that.\textsc{m} \textsc{prt} his shoes lost has\\
\glt `the man who has lost his shoes'
\z
\z

The main question posed by \citet{brandnerbraeuning2013} is what the historical development behind the particle \textit{wo} is, which they synchronically treat as a complementiser specific to relative clauses. This strategy is common across languages, and it can be detected historically in German with the equative particle \textit{so}, as demonstrated in (\ref{sorelpaul}) below (\citealt[132, ex. 3 and 4]{brandnerbraeuning2013}, citing \citealt{paul1920band3}):

\ea \label{sorelpaul}
\ea \gll d\"er Sache \textbf{s\^{o}} ir meinent \label{reinfried}\\
the thing so you mean\\
\glt `the thing that you mean'\\(\textit{Reinfried von Braunschweig}, 14th century)
\ex \gll hier das Geld \textbf{so} ich neulich nicht habe mitschicken können\\
here the money so I recently not have with.send can\\
\glt `here the money that I recently could not send'\\(Schiller to Goethe 127)
\z
\z

South German dialects like Alemannic would use \textit{wo} in these cases (\citealt[132--133]{brandnerbraeuning2013}). In line with this, \citet[133]{brandnerbraeuning2013} propose that the change from \textit{so} into \textit{wo} in the relevant dialects involves no reanalysis but simply a change from the \textit{d}-series of pronouns (\textit{so} being a deictic element originally) to the \textit{w}-series, whereby \textit{so}/\textit{wo} is an equative particle. \citet[133]{brandnerbraeuning2013} mention three empirical facts supporting this approach. First, as described by \citet[238]{paul1920band3}, \textit{so}-relatives were most widespread precisely in the areas that nowadays use \textit{wo}-relatives (Upper German areas). Second, \textit{wo}-relatives appeared at the same time when the particle in equatives changed from \textit{als} (derived from \textit{also}) to \textit{wie}, which belongs to the \textit{w}-series, see \citet{jaeger2010}. Third, equative particles are used in other Germanic languages in relative clauses as well, notably in Scandinavian languages (\textit{som}-relatives). According to \citet[133]{brandnerbraeuning2013}, \textit{wo} is a complementiser, which also allows for the Doubly Filled COMP patterns described by \citet{bayer1984}.

The use of \textit{d}-pronouns as relative pronouns can be observed in all Germanic languages, at least historically, illustrated for Old High German and Old English in (\ref{ohgoe}) below (\citealt[134, ex. 7a and 7b]{brandnerbraeuning2013}):

\ea \label{ohgoe}
\ea \gll See miin sunu, \textbf{den} ich gechos\ldots \label{ohgrel}\\
see my son that.\textsc{acc} I chose\\
\glt `See my son, who I have chosen\ldots''\\(Monseer, Matth.12.18)
\ex \gll gelaðede Cenred þone cyning \textbf{þam} he Myrcna rice sealde\\
invited Cenred the king that.\textsc{dat} he Myrcna kingdom gave\\
\glt `Cenred invited the king whom he had given the kingdom of Mercia.'\\(Bede, Hist.Ecc. 464/7)
\z
\z

German still preserves the pattern given in (\ref{ohgrel}), while English is exceptional among Germanic languages in later using \textit{w}-pronouns in relative clauses (\citealt[134--135]{brandnerbraeuning2013}). Apart from \textit{d}-pronouns, the particle strategy is attested from the earliest records as well: this was \textit{the} in Old High German and \textit{ðe} in Old English, and in Middle High German \textit{und} `and' was possible (\citealt[135--136]{brandnerbraeuning2013}, citing \citealt{ferraresiweiss2011}). Middle High German also allowed the particle \textit{als} `as', illustrated in (\ref{mhgals}) below (\citealt[136, ex. 13]{brandnerbraeuning2013}, citing \citealt{ebertreichmannsolmswegera1993}):

\ea \gll \ldots{} und begerten solichen schaden \textbf{als} sie deshalben gelitten \label{mhgals}\\
{} and demanded such damage as they because.of.that suffered\\
\glt`And they demanded such damage that they had suffered because of that.'\\(Chr. V. Mainz 220)
\z

The elements \textit{so} and \textit{als} (a shortened form of \textit{al-so}) occur in equatives as well (\citealt[136]{brandnerbraeuning2013}, quoting \citealt{jaeger2010}). The use of this particle was common in Early New High German and was possible with all types of head nouns (\citealt[137]{brandnerbraeuning2013}, contrary to \citealt{paul1920band3}). In addition, it could appear even in appositives (\citealt[137, ex. 19]{brandnerbraeuning2013}):

\ea \gll \ldots{} das land Moesia \textbf{so} iezo Bulgarei heist\ldots\\
{} the land Moesia so now Bulgarei is.named\\
\glt `the land Moesia which is now called Bulgarei\ldots'\\(\textit{Deutsches Wörterbuch} vol. 16, col. 1381--1388)
\z

According to \citet[137]{brandnerbraeuning2013}, \textit{so}-relatives occur scarcely in the older texts, especially when compared to Early New High German, which may be due to the particle strategy being rather a spoken and less formal phenomenon in Germanic languages than the pronoun strategy typical of written and more formal contexts (cf. \citealt{fiorentino2007} on Germanic and Romance). Note that \textit{so}-relatives were possible very early on, as shown by the Old Saxon example taken from the \textit{Heliand} (about 830) in (\ref{heliand}) below (\citealt[138, ex. 20]{brandnerbraeuning2013}):

\ea \gll sulike gesidos \textbf{so} he im selbo gecos \label{heliand}\\
such companions so he him self chose\\
\glt `such companions that he chose himself'\\(\textit{Heliand} 16.1280)
\z

\citet[138--146]{brandnerbraeuning2013} discuss three proposals from the previous literature for the origin of \textit{wo}-relatives, all of which are, however, empirically not tenable.

In the first scenario, relative \textit{wo} has its origin in the locative adverb \textit{wo} `where', as taken by \citet{bidesecognolapadovan2012} and by the cross-linguistic study of \citet{fiorentino2007}. As \citet{brandnerbraeuning2013} point out, the exact mechanisms behind this idea have not been spelt out precisely, and while the transfer of locational expressions to certain other domains like temporal expressions is plausible (\citealt{hoppertraugott1993}), the extension to all types of domains is not straightforward. In principle, one may suppose that every DP has a silent ``location argument'',\footnote{See \citet[65--79]{kayne2005} in this respect, who shows that locatives like \textit{there} can also have demonstrative uses (as is evident from non-standard examples such as \textit{that there book}). On a different note, \citet{landau2010} argues that experiencer arguments are related to locatives, as experiencers are `mental locations'. \citet{brody2013} and \citet{sluckinkastner2022} also argue that locatives can introduce a person element. Against this background, the objections raised by \citet{brandnerbraeuning2013} against this first scenario may turn out to be weaker.} which is why \textit{wo} may relativise DPs, but the location argument is unlikely to be present in expressions like \textit{somebody} (\citealt[139--140]{brandnerbraeuning2013}). In addition, \citet[140--141]{brandnerbraeuning2013} point out that no other \textit{w}-pronoun grammaticalised into a complementiser in relative clauses in German, so that the loss of the location argument and the shift from \textit{so} to \textit{wo} should have occurred at the same time, for which there is no evidence. In essence, \citet[141]{brandnerbraeuning2013} do not dismiss this kind of proposal as entirely impossible but they stress that there is no positive evidence supporting it either.\footnote{In this respect, it should be mentioned that while the conclusion of \citet{brandnerbraeuning2013} regarding the change from \textit{so} to \textit{wo} is convincing, it is still possible that South German \textit{wo} had multiple sources. Original locative relative markers can be extended to a more general grounding function, as illustrated in (\ref{eis}) below:

\ea \gll Der Eisladen hat ganz schräge Sorten, wo ich sage, das kann doch keiner mögen. \label{eis}\\
the ice-cream.shop has very weird sorts where I say that can but nobody like\\
\glt `The ice-cream shop has very weird flavours, where I say, nobody can possibly like that.'
\z

In (\ref{eis}), the nominal head is \textit{ganz schräge Sorten} `very weird flavours': it expresses no location meaning and not even a semantically related type (e.g. temporal). This grammaticalisation path, as suggested by \citet{ballareinglese2022}, operates independently of the so-called Noun Phrase Accessibility Hierarchy (see the discussion in \sectref{sec:4results}), also given that it targets adverbial (and not nominal) elements. In other words, the development of the relative complementiser \textit{wo} in South German may well be the result of a conspiracy of two processes: as both of these were available in South German, the complementiser arose there, while the absence of one of them (namely the paradigmatic effect regarding the change from \textit{so} to \textit{wo}) in North German varieties results in the predictable absence of the same complementiser.}

In the second scenario, relative \textit{wo} has its origin in the so-called split R-pro\-noun construction, which is attested during the history of German, as shown by \citet{fleischer2008}. This idea was suggested by \citet{staedele1927} and \citet{paul1920band3}. In R-pronouns, ``the argument of a preposition occurs as an invariable particle \textit{da}- resp. \textit{wo}- linearly before the preposition by which they are selected'' (\citealt[141]{brandnerbraeuning2013}). For instance, \textit{damit} is composed of \textit{da} `there' and \textit{mit} `with', and \textit{woran} is composed of \textit{wo} `where' and \textit{an} `at', whereby the -R- ``is inserted to avoid hiatus'' (\citealt[141]{brandnerbraeuning2013}), giving the name of R-pronouns (going back to \citealt{vanriemsdijk1978}). Split R-pronouns are illustrated in (\ref{splitr}) below (\citealt[141, ex. 25]{brandnerbraeuning2013}):

\ea \label{splitr}
\ea \gll \textbf{Da} weiss ich nichts \textbf{von}.\\
there know I nothing of\\
\glt `I know nothing of this.'
\ex \gll \textbf{Wo} hast du nichts \textbf{von} gehört?\\
where have you nothing of heard\\
\glt `What did you hear nothing about?'
\ex \gll Ich weiss nicht \textbf{wo} er das \textbf{mit} bezahlen will.\\
I know not where he that with pay wants\\
\glt `I don't know with what he wants to pay this.'
\z
\z

The split R-pronoun construction is also attested in relative clauses (\citealt[142, ex. 26]{brandnerbraeuning2013}, citing \citealt{staedele1927}):

\ea \gll des isch s' messer \textbf{wo}-n i Brot mit abe koue ha\\
this is the knife wo-N I  bread with \textsc{prt} cut have\\
\glt `This is the knife with which I cut off (some) bread.'
\z

However, while there is evidence, for instance from \citet{behaghel1928}, for the split R-pronoun construction to have been relatively frequent in Old High German and Middle High German (with \textit{da} belonging to the \textit{d}-series), it was restricted to certain head nouns only and, as shown by \citet{fleischer2008}, it disappeared after Early New High German (\citealt[142--143]{brandnerbraeuning2013}). In addition, split R-pronouns are restricted to certain dialect areas only (\citealt{fleischer2002, fleischer2008}), and Alemannic (a dialect using \textit{wo}-relatives) is not one of these areas, and it is unlikely that the source construction should be altogether absent (\citealt[143]{brandnerbraeuning2013}).

In the third scenario, relative \textit{wo} has its origin in free relatives, and hence crucially involves no change from the \textit{d}-series to the \textit{w}-series. The construction is illustrated for Middle High German in (\ref{psalm}) below (\citealt[144, ex. 30]{brandnerbraeuning2013}, citing \citealt{luehr1998}):

\ea \gll \textbf{So} \textbf{ware} \textbf{so} (se) ich cherte minen zoum\ldots{} \label{psalm}\\
so where so \phantom{(}se I guided my rein\\
\glt `Wherever I guided my rein'\\(\textit{Bairischer Psalm} 138)
\z

Eventually both \textit{so} elements were dropped and/or incorporated, so that single \textit{wo} emerged by the 14th century in free relatives, yet there is no evidence for the same element appearing in relatives with proper nominal heads, apart from some (scarce) examples where the head noun is locative (\citealt[144--146]{brandnerbraeuning2013}). On the other hand, the change from headless relatives to headed relatives is problematic, as the two clause types are in fact quite different both syntactically and semantically (\citealt[144--146]{brandnerbraeuning2013}, citing \citealt{caponigro2003diss}).

\citet[147--150]{brandnerbraeuning2013} argue that the use of \textit{so} and \textit{wo} in relative clauses is possible because of an inherent similarity between equative clauses and relative clauses: both clauses are embedded and contain a ``gap'' that is connected via an equation relation to another element in the matrix clause (cf. the characterisation of relative clauses by De \citealt{devries2006}). Note that the gap can be present without there being movement to the left periphery (\citealt[150--151]{brandnerbraeuning2013}).

In equative constructions, both the comparee and the standard (\citealt{haspelmathbuchholz1998}) are marked by special particles, as illustrated in (\ref{equative}) below (\citealt[150, ex. 42]{brandnerbraeuning2013}):

\ea \gll Hans ist \textbf{so} groß \textbf{wie} Maria. \label{equative}\\
Hans is so tall as Maria\\
\glt `Hans is as tall as Maria.'
\z

In (\ref{equative}), \textit{so} is the parameter marker (or degree marker) and \textit{wie} is the standard marker; in Old High German, both markers surfaced as \textit{so} (\citealt[150, ex. 43]{brandnerbraeuning2013}, citing \citealt{schlosser1998}):

\ea \gll Sie sind \textbf{so} s\'ama chuani s\'elb \textbf{so} thie Romani\\
they are so same keen self so the Romans\\
\glt `They are as keen as the Romans themselves.'
\z

Importantly, the parameter marker can be dropped and the standard can also be omitted if the parameter marker is used purely deictically; otherwise the parameter marker is a phoric element that builds a correlative construction with the standard (\citealt[150--151]{brandnerbraeuning2013}).

Regarding the difference between \textit{wo} and \textit{wie}, \citet[152]{brandnerbraeuning2013} assume that both express equation and there is dialectal variation regarding their distribution, such that \textit{wie} is not restricted to equative clauses and \textit{wo} is not restricted to relative clauses. In certain Bavarian dialects, \textit{wie} can introduce relative clauses (\citealt[153, ex. 50]{brandnerbraeuning2013}, quoting \citealt{eroms2005}):

\ea \gll So dass ma do ned iba de norm koma san, de \textbf{wia} se aufgschaid ham.\\
such that we there not above the norm come be which as they up.set have\\
\glt `Such that we did not exceed the norm that had been set up.'
\z

On the other hand, in certain Swiss German dialects \textit{wo} can introduce equative clauses (\citealt[153, ex. 51]{brandnerbraeuning2013}):

\ea \gll der isch so gross \textbf{wo}-n-i\\
he is so big as-N-I\\
\glt `He is as big as I.'
\z

In addition, dialects differ in the element appearing in temporal clauses expressing simultaneity: Alemannic uses \textit{wo}, while Bavarian uses \textit{wie}, as shown in (\ref{wowia}) below (\citealt[153, ex. 52]{brandnerbraeuning2013}).

\ea \label{wowia}
\ea \gll \textbf{Wo} er hom gloffe isch het 's grenglet.\\
when he home walked is has it rained\\
\glt `It was raining when he was going home.'\\
\ex \gll \textbf{Wia} ar heim glauffen is hot 's gregnet.\\
when he home walked is has it rained\\
\glt `It was raining when he was going home.'
\z
\z

In essence, \citet[153]{brandnerbraeuning2013} assume that the equative element \textit{so} has a \textit{w}-variant that is spelt out as \textit{wo} in certain dialects, while in others it is spelt out as \textit{wie}, whereby the equative/relative differentiation is a matter of surface phonology. The change from \textit{so} to \textit{wo} is thus merely a change from the \textit{d}-series to the \textit{w}-series, which is a recurring phenomenon (attested also in non-Germanic languages, see \citealt{diessel2003}) and can also be observed in the case of \textit{wenn} `when' and \textit{denn} `then',\footnote{\citet[155, ex. 54]{brandnerbraeuning2013} provide the following example for an earlier use of \textit{denn} from Old High German:

\ea \gll Pidiu ist dem manne so guot, \textbf{denn} er zu demo mahale quimit\\
by.this is the.\textsc{dat} man.\textsc{dat} so good then he to the.\textsc{dat} court goes\\
\glt `Because of this it is good for the man if he goes to the court.'\\
(\textit{Mûspilli} 63,64; from 870)
\z

The element \textit{denn} `then' is used here in the sense of `if': in Present-Day German, the \textit{wh}-counterpart \textit{wenn} would be used.} and, as described by \citet{jaeger2010}, in ordinary equatives from \textit{als} (\textit{al-so}) to \textit{wie} (\citealt[154--156]{brandnerbraeuning2013}). Note that both the change from \textit{so} to \textit{wo} and the change from \textit{als} to \textit{wie} were completed after the Early New High German period (\citealt[154]{brandnerbraeuning2013}, citing \citealt{jaeger2010} and \citealt{behaghel1928}). The change from \textit{so} to \textit{wo} is also shown by the fact that in different manuscripts of the Nibelung legend, the earlier \textit{so} forms were replaced by \textit{wo} (\citealt[155--156]{brandnerbraeuning2013}). Interestingly, the earlier \textit{so} in relative clauses survives apparently till the present, as evidenced by Alemannic dialect data (\citealt[154, ex. 53a]{brandnerbraeuning2013}):\footnote{In principle, one might wonder whether \textit{so} in (\ref{maedle}) is not a resumptive element: \citet{meklenborgsalvesen2020} shows that \textit{så} is a possible adverbial resumptive particle in Mainland Scandinavian. However, (\ref{maedle}) is a possessive relative clause and not an adverbial clause; in addition, German \textit{so} generally does not seem to be available as a resumptive (unlike \textit{dann} `then' and \textit{da} `there', as also shown by \citealt{meklenborgsalvesen2020}). Apart from possessive relatives, \textit{so} seems to be possible to a limited degree in subject and object relatives as well. In the project SynAlm (``Syntax des Alemannischen''), the following examples were used in an acceptability judgement task (FB2/B2-1-7, Question ID: FB2-24, Column: X $\rightarrow$ Q\_B2-1-7 and FB2/B2-1-8, Question ID: FB2-25, Column: X $\rightarrow$ Q\_B2-1-8):

\ea \gll Das Mädchen, so in Deggingen wohnt\ldots \label{sosubject}\\
the.\textsc{dat} girl, so in Deggingen lives\\
\glt `the girl who lives in Deggingen'
\ex \gll Das Haus, so man jetzt kaufen kann \label{soobject}\\
the.\textsc{dat} house, so one now buy.\textsc{inf} can.\textsc{3sg}\\
\glt `the house which one can now buy'
\z

On a scale from 1 to 3 (where 1 is the best), the sentence in (\ref{sosubject}) was rated as 1 by 0.53\% of all informants, while the sentence in (\ref{soobject}) was rated as 1 by 2.13\% of all participants. The distribution shows regional differences as well: in Baden-Württemberg, the 1-ratings were somewhat more frequent, 0.9\% for (\ref{sosubject}) and 3.15\% for (\ref{soobject}). Neither of these contexts is adverbial.}

\ea \gll Dem Maedle \textbf{so} ses Fahrrad gstohle hen\ldots \label{maedle}\\
the.\textsc{dat} girl so they.the bicycle stolen have\\
\glt `The girl whose bicycle was stolen\ldots'
\z

\begin{sloppypar}
As noted by \citet[158]{brandnerbraeuning2013}, the availability of the \textit{so}-element both in equative and in relative clauses is not unique to German: this is in fact standard in Scandinavian languages, and goes back to Old Norse. Consider (\citealt[158, ex. 63 and 64]{brandnerbraeuning2013}, citing \citealt{faarlund2004}):
\end{sloppypar}

\ea
\ea \gll Allum guðs vinum ok sinum þeim \textbf{sem} þetta bref sj\'a eðr heyra\\
all god.\textsc{gen} friends.\textsc{dat} and his.\textsc{refl} those so this letter see.\textsc{3pl} or hear.\textsc{3pl}\\
\glt `to all God's friends and those of his own who see or hear this letter'
\ex \gll sv\'a þr\'ottaust folk \textbf{sem} þetta er\\
so powerless people so this.\textsc{n} is\\
\glt `powerless as this people is'
\z
\z

Importantly, however, the present-day \textit{som}-particle in Scandinavian differs from what can be observed in German dialectally and historically: \textit{som} can appear in embedded questions (together with a \textit{wh}-element), and in relative clauses, it can be omitted, it can co-occur with another finite complementiser (\textit{at} `that'), and it allows preposition stranding (\citealt[158--162]{brandnerbraeuning2013}). Apart from these differences, however, the Scandinavian patterns are similar in their syntax to the Upper German \textit{wo}-relatives, and therefore it seems reasonable to assume that the latter also contain \textit{wo} as a complementiser, which is the continuation of the previous pattern involving \textit{so} (\citealt[163--165]{brandnerbraeuning2013}).

\section{Relative pronouns versus complementisers} \label{sec:4relativepronouns}
As should be obvious from the analysis of \citet{brandnerbraeuning2013}, West Germanic languages show considerable variation in terms of elements introducing relative clauses. In this section, I am going to review the various patterns in English and German briefly, concentrating on the differences between relative pronouns and complementisers, indicating the questions to be discussed later in this chapter.

The first strategy to be mentioned here is relative pronouns. In present-day Standard English, they show partial case distinction and distinction with respect to whether the referent is human or non-human. Observe:

\ea
\ea I saw the woman \textbf{who} lives next door in the park. \label{whosubject}
\ex The woman \textbf{who/whom} I saw in the park lives next door. \label{whoobject}
\ex I saw the cat \textbf{which} lives next door in the park. \label{whichsubject}
\ex The cat \textbf{which} I saw in the park lives next door. \label{whichobject}
\z
\z

As can be seen, \textit{who}/\textit{whom} is used with human antecedents, as with \textit{the woman} in (\ref{whosubject}) and (\ref{whoobject}); the form \textit{who} can appear both as nominative and as accusative, while the form \textit{whom} used for the accusative is restricted in its actual appearance (formal/marked). With non-human antecedents, such as \textit{the cat} in (\ref{whichsubject}) and (\ref{whichobject}), the pronoun \textit{which} is used, which shows no case distinction. Note that apart from human referents, \textit{who(m)} is possible with certain animals: these are the ``sanctioned borderline cases'' (see \citealt[41]{herrmann2005}, quoting \citealt{quirkgreenbaumleechsvartvik1985}). On the other hand, non-standard dialects allow \textit{which} with human referents: five of the six dialect areas show this, while the proportion of \textit{which} is very low in Northern Ireland (see \citealt[41]{herrmann2005}). The construction is illustrated in (\ref{boywhich}) below (\citealt[42, ex. 4a]{herrmann2005}):

\ea {[}\ldots] And the boy \textbf{which} I was at school with [\ldots] \label{boywhich}\\
(\textit{Freiburg English Dialect Corpus} Wes\_019)
\z

At any rate, English relative pronouns are formed on the \textit{wh}-base and no longer on the demonstrative base: note that this is historically not so, and the present-day complementiser \textit{that} was reanalysed from a pronoun, while the \textit{wh}-based relative operators appeared only in Middle English (\citealt{vangelderen2009}).

In Standard German, relative clauses are introduced by demonstrative-based relative pronouns (\textit{d}-pronouns) or, less typically, by \textit{wh}-based relative pronouns (\textit{w}-pronouns);\footnote{As noted by \citet[140]{coniglio2019}, the \textit{wh}-based pronouns constitute a later development: they are first attested in Early New High German (cf. \citealt[717ff]{behaghel1928}), while \textit{d}-pronouns in relative clauses are attested much earlier (cf. \citealt[114ff]{fleischmann1973}).} the variation between \textit{d}-pronouns and \textit{w}-pronouns is expected on the basis of \citet{brandnerbraeuning2013}. Consider:

\ea \label{germanrelhuman}
\ea \gll Das ist die Frau, \textbf{die} das Buch geschrieben hat.\\
that.\textsc{n} is the.\textsc{f} woman that.\textsc{f} the.\textsc{n} book written has\\
\glt `That is the woman who built the house.'
\ex \gll Das ist die Frau, \textbf{welche} das Buch geschrieben hat.\\
that.\textsc{n} is the.\textsc{f} woman which.\textsc{f} the.\textsc{n} book written has\\
\glt `That is the woman who built the house.'
\z
\z

The examples in (\ref{germanrelhuman}) above contain human referents. Unlike in English, German relative pronouns are not sensitive to a human versus non-human distinction. Consider:

\ea
\ea \gll Das war die Idee, \textbf{die} der Lösung zugrunde lag.\\
that.\textsc{n} was the.\textsc{f} idea that.\textsc{f} the.\textsc{f.dat} solution beneath lay.\textsc{3sg}\\
\glt `That was the idea behind the solution.'
\ex \gll Das war die Idee, \textbf{welche} der Lösung zugrunde lag.\\
that.\textsc{n} was the.\textsc{f} idea which.\textsc{f} the.\textsc{f.dat} solution beneath lay.\textsc{3sg}\\
\glt `That was the idea behind the solution.'
\z
\z

Note that, as discussed by \citet{vangelderen2004, vangelderen2009} and \citet{robertsroussou2003}, while relative pronouns may stem from interrogative pronouns or from demonstrative pronouns cross-linguistically (and that thus German represents both patterns), it appears that the use of relative pronouns is a typically European strategy and is otherwise rare cross-linguistically (see \citealt{comrie2002}; see also \citealt{vangelderen2009} for discussion in a minimalist framework).\footnote{This only applies to relative pronouns proper (that is, operators that are base-generated within the clause and move to the CP) and not to strategies such as pronoun retention (see \citealt{comriekuteva2013}). Similarly, complementisers that originally derive from pronouns by definition do not count as instances of the pronoun strategy (these are generally subsumed under relative particles in typological work). Regarding the relative pronoun strategy, \citet{comriekuteva2013subject} identify it for 12 languages in subject relatives (of a sample of 166 languages) and \citet{comriekuteva2013oblique} identify it for 13 languages in oblique relatives (of a sample of 112 languages). Out of these 12/13 languages, only 2 are non-European: Acome (spoken in North America) and Georgian (which counts as non-European in the sense that it does not belong to the Sprachbund ``Standard Average European'', as defined by \citealt{haspelmath2001}).}

The standard assumption in generative grammar is that the relative pronoun occupies a specifier position in the CP, as shown in \figref{treepronounenglish} for English.

\begin{figure} 
\caption{English relative pronouns} \label{treepronounenglish}
\begin{forest} baseline, qtree
[CP
	[who(m)/which\textsubscript{{[}rel{]}}]
	[C$'$
		[C\textsubscript{{[}rel{]},{[}fin{]}}
			[$\emptyset$\textsubscript{{[}rel{]},{[}fin{]}}]
		]
		[TP]
	]
]
\end{forest}
\end{figure}

Essentially the same holds for German, as given in \figref{treepronoungerman}.

\begin{figure} 
\caption{German relative pronouns} \label{treepronoungerman}
\begin{forest} baseline, qtree
[CP
	[der/die/das\textsubscript{{[}rel{]}}\\welcher/welche/welches\textsubscript{{[}rel{]}}]
	[C$'$
		[C\textsubscript{{[}rel{]},{[}fin{]}}
			[$\emptyset$\textsubscript{{[}rel{]},{[}fin{]}}]
		]
		[TP]
	]
]
\end{forest}
\end{figure}

Using the syntactic model proposed in \chapref{ch:2} and the findings about embedded interrogatives in \chapref{ch:3}, the C position contains a phonologically empty complementiser, while the phonologically visible relative pronoun moves to the specifier, checking off the [rel] feature (taking relative to be a clause type, see \citealt{rizzi1997}); the relative operator has to move up since there is no relative-in-situ in the respective languages (see the discussion in \citealt[100]{bacskaiatkari2018langsci}). This contrasts with \textit{wh}-elements, which may remain in situ (which of course results in a focus interpretation rather than a true interrogative one, see \citealt{boskovic2002}).

The second strategy involves the use of a relative complementiser. In Standard English, this is possible with the complementiser \textit{that}:

\ea
\ea I saw the woman \textbf{that} lives next door in the park.
\ex The woman \textbf{that} I saw in the park lives next door.
\ex I saw the cat \textbf{that} lives next door in the park.
\ex The cat \textbf{that} I saw in the park lives next door.
\z
\z

The complementiser \textit{that} is not sensitive to case and to the human/non-human distinction, which follows from its status as a C head. As discussed in section (\ref{sec:4relative}), Standard German does not allow such patterns but in Southern (High German) dialects the complementiser \textit{wo} is the usual strategy (see also \citealt{weiss2013}).\footnote{Apart from present-day dialects, the complementiser strategy is also prevalent in the history of German, also beyond \textit{so}/\textit{wo}, see \citet{coniglio2019}.} This is illustrated for Alemannic in (\ref{relwoalemannic}), taken from \citet[140, ex. 23]{brandnerbraeuning2013} and in (\ref{zurichwo}) for Zurich German, taken from \citet[337, ex. 1]{salzmann2017}:

\ea
\ea \gll Ich suech ebber \textbf{wo} mer helfe künnt. \label{relwoalemannic}\\
I search someone \textsc{rel} I.\textsc{dat} help.\textsc{inf} could\\
\glt `I am looking for someone who could help me.'
\ex \gll Das isch s Buech, \textbf{won} i geschter poschtet ha. \label{zurichwo}\\
this is the.\textsc{n} book \textsc{rel} I yesterday bought.\textsc{ptcp} have.\textsc{1sg}\\
\glt `This is the book I bought yesterday.'
\z
\z

Apart from its uniform availability, the status of \textit{wo} as a complementiser is evident from the fact that, unlike relative pronouns, it cannot co-occur with a preposition. Consider (\citealt[337, ex. 2]{salzmann2017}):

\ea[*]{\gll de Maa, mit \textbf{won} i gredt ha\\
the man with \textsc{rel} I spoken have.\textsc{1sg}\\
\glt `the man I talked to'}
\z

In North Bavarian dialects, as shown by \citet{weiss2013}, the complementiser may also be realised as \textit{was}; I will return to this question later. 

Since both \textit{that} and \textit{wo} are complementisers, they appear in C with respect to their relative position, as shown in \figref{treerelcompthat} and \figref{treerelcompwo}.

\begin{figure} 
\caption{The relative complementiser \textit{that}} \label{treerelcompthat}
\begin{forest} baseline, qtree
[CP
	[$\emptyset$\textsubscript{{[}rel{]}}]
	[C$'$
		[C\textsubscript{{[}rel{]},{[}fin{]}}
			[that\textsubscript{{[}rel{]},{[}fin{]}}]
		]
		[TP]
	]
]
\end{forest}
\end{figure}

\begin{figure} 
\caption{The relative complementiser \textit{wo}} \label{treerelcompwo}
\begin{forest} baseline, qtree
[CP
	[$\emptyset$\textsubscript{{[}rel{]}}]
	[C$'$
		[C\textsubscript{{[}rel{]},{[}fin{]}}
			[wo\textsubscript{{[}rel{]},{[}fin{]}}]
		]
		[TP]
	]
]
\end{forest}
\end{figure}


In line with the findings in \chapref{ch:3}, these patterns lexicalise the C head and thus correspond to the regular West Germanic pattern. The operator corresponds to the gap in the relative clause and is semantically dependent on the head noun in the matrix clause: as it is recoverable precisely on the basis of the head noun, it does not have to be overt. In principle, however, it can be realised overtly, resulting in Doubly Filled COMP patterns, as was pointed out already in \chapref{ch:3}. I will return to the discussion of relative complementisers and doubling patterns later in this chapter.

It is worth mentioning that, in Standard English (but not in German, see also \sectref{sec:4relative}), zero relatives are possible with object relative clauses:

\ea
\ea[*]{I saw the woman lives next door in the park.\label{womanlives}}
\ex[ ]{The woman I saw in the park lives next door.}
\ex[*]{I saw the cat lives next door in the park. \label{catlives}}
\ex[ ]{The cat I saw in the park lives next door.}
\z
\z

Zero subject relative clauses are not allowed in Standard English. They are, however, possible in certain dialects (see \citealt[55--56]{herrmann2005}). This strategy is traditionally considered as not regionally restricted, though, as \citet[26--27]{herrmann2005} shows, there are considerable regional differences, such that zero relatives are the most frequent in the Northern areas of Britain (see also the discussion of \citealt{kortmannwagner2007} of this data set). Consider the example in (\ref{grandmotherzero}) below (\citealt[64, ex. 25b]{herrmann2005}):

\ea {[}\ldots] It was my grandmother owned this bit of land [\ldots] \label{grandmotherzero}\\
(\textit{Northern Ireland Transcribed Corpus of Speech} A13.3)
\z
 
Naturally, in zero relatives, the CP-periphery contains a zero complementiser and a zero operator, as shown in \figref{treezerorel}.

\begin{figure}
\caption{Zero relatives} \label{treezerorel}
\begin{forest} baseline, qtree
[CP
	[$\emptyset$\textsubscript{{[}rel{]}}]
	[C$'$
		[C\textsubscript{{[}rel{]},{[}fin{]}}
			[$\emptyset$\textsubscript{{[}rel{]},{[}fin{]}}]
		]
		[TP]
	]
]
\end{forest}
\end{figure}

Given the availability of zero subject relatives like (\ref{grandmotherzero}), as well as language acquisition data of the same type, I follow \citet[537]{sobin2002} in assuming that the apparent ban on \textit{that}-trace effects is a soft constraint and it is rather perceptional in nature, as pointed out by \citet{chomskylasnik1977} already (see the relevant discussions in \chapref{ch:2} and \chapref{ch:3}). As it is, the distribution of zero relatives is not of immediate relevance to the discussion in the present chapter and I will therefore not examine zero relatives any further.

The last pattern to be discussed here is that of relative complementisers that are surface-identical to equative complementisers (see also \citealt{brandnerbraeuning2013}). Such complementisers include historical German \textit{so} and English \textit{as}, which can introduce relative clauses in dialects, especially in the more Southern dialects in Britain (\citealt{herrmann2005}, \citealt{kortmannwagner2007}).\footnote{For her study, \citet{herrmann2005} examined six dialect areas (Central Southwest, East Anglia, Central Midlands, Central North, Scotland). Out of these, only three had examples for \textit{as}-relatives, with different proportions (in relation to all relative clauses): Northern Ireland (0.5\%), Central North (1.4\%) and Central Midlands (2.4\%). \citet{wagner2008} reports \textit{as}-relatives as a traditional feature in the Southwest of England and \citet{anderwald2008} reports them for the Southeast of England.} An example is given in (\ref{allas}) below (\citealt[64, ex. 26d]{herrmann2005}):

\ea {[}\ldots] so all \textbf{as} he had to do were go round in a circle all the time [\ldots] \label{allas}\\
(\textit{Freiburg English Dialect Corpus} Som\_001)
\z

\citet{koenig2015} identifies manner deictic elements like \textit{so} as a potential source of various grammatical markers across Indo-European languages; in this sense, the West-Germanic patterns are not unique. Further, regarding the relatedness of relative and equative clauses, \citet{brandnerbraeuning2013} suggest that there is a common underlying semantics: this assumption is altogether questionable (as \citealt[54]{koenig2015} puts it, there is ``hardly any semantic similarity'' between the uses in relative clauses and in comparison clauses). Another option would be to assume that equative markers may grammaticalise into relative markers. If so, equative-based relative complementisers behave exactly like other relative complementisers (such as English \textit{that} and German \textit{wo}) and can thus be attributed a structure analogous to Figures~\ref{treerelcompthat} and \ref{treerelcompwo}. It follows that they also regularly lexicalise [fin] on C and are in line with the general West Germanic pattern. In other cases, as an immediate stage in the grammaticalisation of equative complementisers into proper relative complementisers, we find relative clauses that are tied to the presence of an equative marker in the matrix clause: I will refer to these as equative relative clauses and will discuss them separately in \sectref{sec:4equative}.

\section{Variation and change in English} \label{sec:4variation}
\subsection{Background and methodology} \label{sec:4background}
As established in \sectref{sec:4relativepronouns}, English allows both the relative pronoun and the relative complementiser strategy. However, there are important differences between the standard variety and non-standard dialects in this respect. According to \citet[163]{vangelderen2009}, \textit{wh}-pronouns are promoted by prescriptive rules but English speakers prefer \textit{that} over a \textit{wh}-pronoun ``by at least a 4:1 ratio'' (see also \citealt{romaine1982}, \citealt{montgomerybailey1991}, \citealt{vangelderen2004}, \citealt{tagliamontesmithlawrence2005}). In line with this, the study of \citet{herrmann2005} shows that the use of the relative pronouns \textit{who} and \textit{which} is not very frequent in the regional dialects of Britain. In essence, non-standard varieties show a wider distribution of \textit{that}, which is interchangeable even with PPs involving a \textit{wh}-element, such as \textit{from which} in (\ref{partythat}) below (\citealt[161, ex. 8]{vangelderen2009}, citing \citealt[112]{miller1993}):

\ea I haven't been to a party yet \textbf{that} I haven't got home the same night. \label{partythat}
\z

As described by \citet{kortmannwagner2007} and \citet{herrmann2005}, the dialectal patterns discussed above (\textit{which}-relatives with human referents, \textit{as}-relatives, zero subject relatives, and a higher frequency of \textit{that}-relatives) are attested historically (unlike \textit{what}-relatives with nominal heads, which count as an innovation). It appears that the present-day standard pattern shows the effect of conscious standardisation (beyond mere diachronic change attested across dialects), since non-standard varieties are not necessarily affected by the same constraints. As such, the changes responsible for the present-day pattern are at least in part due to changes in Late Modern English. 

In order to gain a better idea of the relevant changes, I conducted a corpus study (see \citealt{bacskaiatkari2020lmec}) comparing the King James Bible (1611/1769)\footnote{The original version dates from 1611, and the standardised spelling by Benjamin Blayney dates from 1769.} and the New King James version (1989). The new version essentially adheres to the original version, as far as the original construction is grammatical in present-day Standard English. The comparison between the Early Modern English text and modernised version offers a good comparison between the two language stages, even though some caveats must be taken into account (see also the remarks in \chapref{ch:3}).

In particular, it is difficult to compare data for various reasons. First, the issue of optionality cannot be neglected: namely, the choice of one strategy does not imply the impossibility of other strategies. Second, the context or the particular construction may influence the choice: comparing highly different sentences, even in a large corpus, is not conclusive. Third, register has an influence as well: it is evidently difficult to compare texts from Early Modern English and ones from Late Modern English due to varying degrees of standardisation and/or the differences in the influence of prescriptive rules, not to mention the different requirements of diverse genres.

Against this background, the advantages of comparing the two versions of the King James Bible are quite straightforward. First, the same loci are compared, and hence the differences in relative markers cannot be due to the sentences or the context being different; this ultimately allows some quantitative comparison. Second, the same register is used in both texts: the new version is not an instance of radical modernisation, and forms that are partly archaic are not necessarily ruled out. What matters is not so much the distribution of the individual markers in the new version in itself but rather the difference between the original and the modernised version, which reflects conscious deviations from the previous pattern in line with prescriptive rules and language change. Note also that the original version may also be more archaic in general than other texts from the period (as, for instance, in using -\textit{th} instead of -\textit{s} for 3Sg on verbs, see \citealt[173]{vangelderen2014}); what matters for us is rather the fact that it can be dated back to a period when the prescriptive pressure disfavouring \textit{that} was not yet active.

Regarding the present study, the following methodology was applied. The hits for  the forms ``who'', ``whom'', ``which'' and ``that'' in the New King James version were taken as the basis of the corpus. In each case, the corresponding element in the original version was examined. Given that there is a preference for the relative pronoun strategy with \textit{who(m)} with human referents in present-day Standard English, it is expected that many of these occurrences have different equivalents in the original, and that there are unlikely to be many changes the other way round. It should be noted that the New King James version is strongly norm-oriented: \textit{who} is consistently used for subjects, while objects (and complements of prepositions) invariably appear in the form \textit{whom}. This strict split does not truly reflect the actual present-day standard language (see the discussion in \sectref{sec:4relativepronouns}), but it certainly facilitates the corpus study.

\subsection{The results} \label{sec:4results}
There are altogether 5606 hits for \textit{who} and 761 hits for \textit{whom}. The hits were manually checked, so the figures above include relative clauses only and do not include interrogative uses but they include loci where the original King James version uses constructions other than relative clauses. Subject relatives are clearly more frequent than object relatives, in line with the Noun Phrase Accessibility Hierarchy of \citet[66--67]{keenancomrie1977}.\footnote{The original observation of \citet{keenancomrie1977} pertained to the occurrence of resumptive pronouns: these are more likely to appear lower in the hierarchy, such that if resumptive pronouns are obligatory at a given point, then they will be obligatory for all lower functions (as far as they are available in the given language), but they may be optional or even prohibited in higher functions. Conversely, if resumptive pronouns are prohibited at a given point, then they will be prohibited in all higher functions as well, but they may be optional or obligatory in lower functions. Resumptive pronouns are rare in the subject function, which is the highest-ranked function. A further implication concerns the occurrence of relative clauses in a given language: the subject function can always be relativised, while lower functions can only be relativised if all the functions ranked higher are. To provide a simple example: if a language relativises obliques, we can be sure that it also relativises subjects, direct objects and indirect objects. See the discussion in \citet[105--107]{bacskaiatkari2020nordlyd}.} Before turning to the detailed frequency data, let us first consider some examples that show the relevant parallels.

First, \textit{who} can have the equivalent \textit{who} in the original version, and \textit{whom} can have the equivalent \textit{whom} in the original version:

\ea
\ea And it was found written, that Mordecai had told of Bigthana and Teresh, two of the king's chamberlains, the keepers of the door, \textbf{who} sought to lay hand on the king Ahasuerus.\\
(King James Bible; Esther 6:2)
\ex And it was found written that Mordecai had told of Bigthana and Teresh, two of the king's eunuchs, the doorkeepers \textbf{who} had sought to lay hands on King Ahasuerus.\\
(New King James version; Esther 6:2)
\ex Why is light given to a man whose way is hid, and \textbf{whom} God hath hedged in?\\
(King James Bible; Job 3:23)
\ex Why is light given to a man whose way is hidden, And \textbf{whom} God has hedged in?\\
(New King James version; Job 3:23)
\z
\z

Second, \textit{who}/\textit{whom} can have the equivalent \textit{which} in the original version:

\ea
\ea And it came to pass, that when the Jews \textbf{which} dwelt by them came, they said unto us ten times, From all places whence ye shall return unto us they will be upon you.\\
(King James Bible; Nehemiah 4:12)
\ex So it was, when the Jews \textbf{who} dwelt near them came, that they told us ten times, ``From whatever place you turn, they will be upon us.''\\
(New King James version; Nehemiah 4:12)
\ex Have the gods of the nations delivered them \textbf{which} my fathers have destroyed, as Gozan, and Haran, and Rezeph, and the children of Eden which were in Telassar?\\
(King James Bible; Isaiah 37:12)
\ex Have the gods of the nations delivered those \textbf{whom} my fathers have destroyed, Gozan and Haran and Rezeph, and the people of Eden who were in Telassar?\\
(New King James version; Isaiah 37:12)
\z
\z

Third, \textit{who}/\textit{whom} can have the equivalent \textit{that} in the original version:

\ea
\ea And all they \textbf{that} were about them strengthened their hands with vessels of silver, with gold, with goods, and with beasts, and with precious things, beside all that was willingly offered.\\
(King James Bible; Ezra 1:6)
\ex And all those \textbf{who} were around them encouraged them with articles of silver and gold, with goods and livestock, and with precious things, besides all that was willingly offered.\\
(New King James version; Ezra 1:6)
\ex So all the people \textbf{that} Ishmael had carried away captive from Mizpah cast about and returned, and went unto Johanan the son of Kareah.\\
(King James Bible; Jeremiah 41:14)
\ex Then all the people \textbf{whom} Ishmael had carried away captive from Mizpah turned around and came back, and went to Johanan the son of Kareah.\\
(New King James version; Jeremiah 41:14)
\z
\z

Fourth, \textit{who} can have the equivalent \textit{as} in the original version:

\ea
\ea And the king said unto Ziba, What meanest thou by these? And Ziba said, The asses be for the king's household to ride on; and the bread and summer fruit for the young men to eat; and the wine, that such \textbf{as} be faint in the wilderness may drink.\\
(King James Bible; 2 Samuel 16:2) \label{kjas}
\ex And the king said to Ziba, ``What do you mean to do with these?'' So Ziba said, ``The donkeys are for the king's household to ride on, the bread and summer fruit for the young men to eat, and the wine for those \textbf{who} are faint in the wilderness to drink.''\\
(New King James version; 2 Samuel 16:2)
\z
\z

Note that only one such example was found for object relatives, the rest are subject relatives; this may well be due to the fact that there are far more examples for subject relatives than for object relatives (see above and also the discussion below).

Fifth, \textit{who} can have a zero relative equivalent in the original version:

\ea \label{kjzero}
\ea Moreover the soul that shall touch any unclean thing, as the uncleanness of man, or any unclean beast, or any abominable unclean thing, and eat of the flesh of the sacrifice of peace offerings, which pertain unto the LORD, even that soul shall be cut off from his people. \\
(King James Bible; Leviticus 7:21)
\ex Moreover the person who touches any unclean thing, such as human uncleanness, an unclean animal, or any abominable unclean thing, and \textbf{who} eats the flesh of the sacrifice of the peace offering that belongs to the Lord, that person shall be cut off from his people.\\
(New King James version; Leviticus 7:21)
\z
\z

Such examples were again found only for subject relatives but not for object relatives (with \textit{whom}). In addition, it should be noted that these instances of zero occur (with one questionable exception) in coordinate constructions, as can also be seen in (\ref{kjzero}). These instances do not provide good evidence for the availability of true zero relatives, as the omission of an overt element (either the operator or the complementiser) in coordinated constructions can be licensed by an appropriate antecedent in the preceding relative clause (compare the true zero subject relative in (\ref{grandmotherzero}) in \sectref{sec:4relativepronouns} above). This is also possible in modern Standard English:

\ea These are the students \textbf{*(who)} study linguistics and \textbf{(who)} play basketball. \label{relcoord}
\z

As indicated, in the first subject relative clause in (\ref{relcoord}) above, the relative pronoun \textit{who} cannot be left out, while in the second subject relative clause its presence is optional. Since the behaviour of present-day Standard English does not differ from what can be observed in the King James Bible, zero relatives will not be discussed here any further, especially as they are not immediately relevant to the present investigation anyway (see \sectref{sec:4relativepronouns}).

Apart from the patterns of major interest concerning historical change and dialectal variation presented above, \textit{who} in the new version may correspond to \textit{whoso} and \textit{whosoever} in the original version, both appearing in free relatives (the new version in these cases has a head noun or a pronoun).\footnote{These are illustrated in (\ref{whosokj}--\ref{whosoevernkjv}) below:

\eanoraggedright \label{whosokj}
And I find more bitter than death the woman, whose heart is snares and nets, and her hands as bands: \textbf{whoso} pleaseth God shall escape from her; but the sinner shall be taken by her.\hbox{}\hfill\hbox{(King James Bible; Ecclesiastes 7:26)}
\ex And I find more bitter than death The woman whose heart is snares and nets, Whose hands are fetters. He \textbf{who} pleases God shall escape from her, But the sinner shall be trapped by her.\hbox{}\hfill\hbox{(New King James version; Ecclesiastes 7:26)}
\ex All the king's servants, and the people of the king's provinces, do know, that \textbf{whosoever}, whether man or woman, shall come unto the king into the inner court, who is not called, there is one law of his to put him to death, except such to whom the king shall hold out the golden sceptre, that he may live: but I have not been called to come in unto the king these thirty days.\hbox{}\hfill\hbox{(King James Bible; Esther 4:11)}
\ex \label{whosoevernkjv} ``All the king's servants and the people of the king's provinces know that any man or woman \textbf{who} goes into the inner court to the king, who has not been called, he has but one law: put all to death, except the one to whom the king holds out the golden scepter, that he may live. Yet I myself have not been called to go in to the king these thirty days.''\\\hbox{}\hfill\hbox{(New King James version; Esther 4:11)}
\z}
Since the differences here are rather due to whether headed or headless relatives are used, these patterns will not be discussed any further here; they are altogether not very frequent in the corpus results (see Table \ref{tablekjsubject}) and appear only in subject relatives.

Let us now turn to the distribution of the various patterns. Table \ref{tablekjsubject} shows the distribution of the elements corresponding to \textit{who}.\footnote{The original study presented in \citet[100]{bacskaiatkari2020lmec} contains only the data from the Old Testament for the elements corresponding to ``who''. The data from the entire text confirm the previously reported results.} The cases subsumed under ``other'' refer to instances where either the role of the relative pronoun is not a subject in the original or the original text contains no relative clause in the given locus. The instances of ``zero'' occur in coordinated constructions.

\begin{table}
\begin{tabular}{ll r@{~}r}
\lsptoprule
Role in KJB    & {Element in KJB}   & \multicolumn{2}{c}{Occurrences}\\\midrule
subject (5405) & \textit{who}       & {478}  & (8.84\%)\\
{}             & \textit{which}     & {1194} & (22.09\%)\\
{}             & \textit{that}      & {3667} & (67.84\%)\\
{}             & \textit{as}        & {26}   & (0.48\%)\\
{}             & zero               & {23}   & (0.43\%)\\
{}             & \textit{whoso}     & {10}   & (0.19\%)\\
{}             & \textit{whosoever} & {7}    & (0.13\%)\\
\addlinespace
other & -- & 202 & \\
\midrule
Total &    & 5607 & \\
\lspbottomrule
\end{tabular}
\caption{The elements corresponding to \textit{who} in the KJB}
\label{tablekjsubject}
\end{table}

Table \ref{tablekjobject} shows the distribution of the elements corresponding to \textit{whom} in the original King James Bible. The cases subsumed under ``other'' refer to instances where either the role of the relative pronoun in the original does not match the one in the new version or the original text contains no relative clause in the given position.

\begin{table}
\begin{tabular}{ll r@{~}r}
\lsptoprule
Role in KJB & Element in KJB & \multicolumn{2}{c}{Occurrences}\\\midrule
direct object (398) & \textit{whom}  & {312} & (78.39\%)\\
{}                  & \textit{which} & {76}  & (19.10\%)\\
{}                  & \textit{that}  & {10}  & (2.51\%)\\
\addlinespace
indirect object (2) & \textit{whom} & {2} & (100\%)\\
\addlinespace
PP complement (265) & P + \textit{whom} & {256} & (96.60\%)\\
{}                  & P + \textit{which} & {7} & (2.64\%)\\
{}                  & \textit{that} & {2} & (0.75\%)\\
\addlinespace
other & -- & 39 & \\
\midrule
Total & 704  & \\
\lspbottomrule
\end{tabular}
\caption{The elements corresponding to \textit{whom} in the KJB}
\label{tablekjobject}
\end{table}

The data indicate clearly that the present-day dialectal patterns discussed in \sectref{sec:4relativepronouns} are attested and in fact quite substantial in the King James Bible.\footnote{This applies to the use of \textit{as}-relatives, and also to the fact that \textit{that}-relatives represent a dominant strategy} This applies especially to the case of \textit{that}, while the pattern with \textit{as} is clearly a minority pattern. The proportion of \textit{that} is especially high in the case of subject relatives (67.84\%), while it is considerably lower in the case of object relatives (2.51\%). Note that the total number of indirect object relative clauses is very low: the Noun Phrase Accessibility Hierarchy (\citealt{keenancomrie1977}) would predict that they are between direct objects and prepositional complements. The low number of indirect object relative clauses is not a peculiar property of the King James Bible: as \citet{fleischer2004wien} points out, relative clauses with indirect object relatives are generally very rare in corpora. The proportion of \textit{which} is about the same in both (22.09\% in subject relatives and 19.10\% in object relatives).

Table \ref{tablekjwhich} shows the distribution of the elements corresponding to \textit{which} in the original King James Bible. The cases subsumed under ``other'' refer to instances where either the role of the relative pronoun in the original does not match the one in the new version or the original text contains no relative clause in the given position.

\begin{table}
\begin{tabular}{ll r@{~}r}
\lsptoprule
Role in KJB & Element in KJB & \multicolumn{2}{c}{Occurrences}\\\midrule
{subject (925)} & \textit{who}     & {2}   &  (0.22\%)\\
                    {} & \textit{which}   & {833} & (90.05\%)\\
                    {} & \textit{that}    & {89}  &  (9.62\%)\\
                    {} & \textit{whether} & {1}   &  (0.11\%)\\
\addlinespace
{direct object (1222)} & \textit{whom}       & {8}    &  (0.65\%)\\
{}                            & \textit{which}      & {1135} & (92.88\%)\\
{}                            & \textit{that}       & {78}   &  (6.38\%)\\
{}                            & \textit{whatsoever} & {1}    &  (0.08\%)\\
\addlinespace
{PP complement (116)} & P + \textit{whom}       & {3}  &  (2.59\%)\\
                          {} & P + \textit{which}      & {99} &  (85.34\%)\\
                          {} & \textit{that}           & {13} &  (11.21\%)\\
                          {} & P + \textit{whatsoever} & {1}  &  (0.86\%)\\
\addlinespace
other & -- & 606 & \\
\midrule
Total & & 2869 &\\
\lspbottomrule
\end{tabular}
\caption{The elements corresponding to \textit{which} in the KJB}
\label{tablekjwhich}
\end{table}

As can be seen, the overall distribution of relative clauses with non-human antecedents is very similar to that of relative clauses with human antecedents. The predominant pattern is \textit{which} in the original version, with some examples of \textit{that}-relatives in subject relatives and with complements of prepositions, there being only a single example for a direct object relative with \textit{that}. The data also suggest that as far as the subject/object asymmetry is concerned, the human/non-human distinction may play a role in that the proportion of \textit{that}-relatives with human referents is predictably lower in the new version than with non-human referents: in other words, more changes are expected in the direction of \textit{who}/\textit{whom}. In order to test this, it is also necessary to take \textit{that}-relatives in the new version into account.

Table \ref{tablekjthat} shows the distribution of the elements corresponding to \textit{that} in the original King James Bible. As in Table \ref{tablekjwhich}, the cases subsumed under ``other'' are those where either the role of the relative pronoun in the original does not match the one in the new version or the original text contains no relative clause in the given position.

\begin{table}[p]
\begin{tabular}{ll r@{~}r}
\lsptoprule
{Role in KJB} & {Element in KJB} &\multicolumn{2}{c}{Occurrences}\\\midrule
{subject (970)} & \textit{who}   & {2}   & (0.21\%)\\
                    {} & \textit{which} & {46}  & (4.74\%)\\
                    {} & \textit{that}  & {921} & (94.95\%)\\
                    {} & zero           & {1}   & (0.10\%)\\
\addlinespace
{direct object (552)} & \textit{whom} & {3} & (0.54\%)\\
{} & \textit{which} & {10} & (1.81\%)\\
{} & \textit{that} & {536} & (97.10\%)\\
{} & \textit{as} & {1} & (0.18\%)\\
{} & \textit{whatsoever} & {2} & (0.36\%)\\
\addlinespace
{PP complement (123)} & \textit{that} & {123} & (100\%)\\
\addlinespace
other & -- & 75\\
\midrule
Total & & 1720 &\\
\lspbottomrule
\end{tabular}
\caption{The elements corresponding to \textit{that} in the KJB}
\label{tablekjthat}
\end{table}

There are very few exceptions where an original \textit{wh}-element was changed into \textit{that} in the new version. The few instances of PP-relatives with \textit{that} in the new version may seem surprising at first since this pattern (unless with preposition stranding) is not normally attested in Standard English (see the discussion at the beginning of this section). However, all the occurrences are either instances of preposition stranding or appear with set phrases involving either \textit{the day that} or \textit{the time that}, where the \textit{that}-relative is a lexicalised part of the set phrase. By looking at Table \ref{tablekjthat}, there seems to be no particular asymmetry regarding subjects and objects regarding the frequency of \textit{that}-relatives: \textit{that}-relatives occur in the new version almost exclusively in cases where the original version also contained \textit{that}-relatives. Note that Table \ref{tablekjthat} includes relative clauses with both human and non-human referents, but as we saw above, the human/non-human distinction does not seem to be relevant regarding the subject/object asymmetry.

In order to present a more direct comparison between the two versions, Table \ref{tablekjgenesis} summarises the distribution of the relative markers \textit{who}, \textit{whom}, \textit{which} and \textit{that} across subtypes in the original version.\footnote{I have disregarded cases marked as ``other'', as well as some minor options including \textit{as} and the zero strategy, for this table, so that a more direct comparison with the new version can be applied regarding the major options under scrutiny.}

\begin{table}[p]
\tabcolsep=.75\tabcolsep
\fittable{\begin{tabular}{l *4{r@{~}r}}
\lsptoprule
Role & \multicolumn{2}{c}{\textit{who}} & \multicolumn{2}{c}{\textit{whom}} & \multicolumn{2}{c}{\textit{which}} & \multicolumn{2}{c}{\textit{that}}\\\midrule
subject (7232) & 482 & (6.66\%) & -- & {} & 2073 & (28.66\%) & 4677 & (64.67\%)\\
direct object (2168) & -- & {} & 323 & (14.90\%) & 1221 & (56.32\%) & 624 & (28.78\%)\\
indirect object (2) & -- & {} & 2 & (100\%) & -- & {}  & -- & {}\\
PP complement (503) & -- & {} & 259 & (51.49\%) & 106 & (21.07\%) & 138 & (27.44\%)\\
\lspbottomrule
\end{tabular}}
\caption{The distribution of relative markers in the KJB}
\label{tablekjgenesis}
\end{table}

\begin{sloppypar}
The data indicate a clear preference for \textit{that}-relatives in subject relative clauses, while \textit{wh}-relatives are preferred in direct object relative clauses and in relative clauses where the relative pronoun corresponds to the complement of a preposition. I carried out a chi-square test on the distribution of \textit{who(m)}/\textit{which}/\textit{that} in the three types of relative clause: this test reveals that the differences are significant at $p<0.05$, namely $\chi^2 (4, N = 9903) = 1786.8714,\allowbreak p < 0.00001$, meaning that the choice of relative marker is dependent on the relativised function. This is in line with the prediction made by the Noun Phrase Accessibility Hierarchy.\footnote{As relative pronouns lexicalise the gap, they may be similar to resumptive pronouns in that they can ease processing for less accessible gaps (see also \citealt[440]{romaine1984}, \citealt[230]{fleischer2004}). This was formulated by \citet[252--258]{hawkins1999} as the Filler-Gap-Complexity Hypothesis: according to this, [--case] elements are expected to occur in functions that are higher in the Accessibility Hierarchy, while [$+$case] elements are expected especially in lower functions. Under this view, we are expected to find cut-off points analogous to the ones with resumptive pronouns. Indeed, there are some remarkable similarities that arise, while there are obvious differences as well. For one thing, the occurrence of resumptive pronouns is compared to the non-occurrence of the same element (pronoun vs. zero); clause typing is independently carried out by a complementiser in the left periphery (so that no choice in the form ``pronoun vs. complementiser'' arises). Relative pronouns, however, primarily compete with overt complementisers (that is, the relative clause is either introduced by an overt relative pronoun or by an overt relative complementiser), so that the question ``pronoun vs. complementiser'' is more sensible to ask. See \citet[107]{bacskaiatkari2020nordlyd} for more discussion.} The same holds for the fact that \textit{as}-relatives are attested only in subject relative clauses (where they also form a minority pattern; see \sectref{sec:4equative} for further discussion). As mentioned above, indirect object relative clauses are rare in corpora.
\end{sloppypar}

\begin{sloppypar}
Table \ref{tablenkjgenesis} summarises the distribution of the relative markers \textit{who}, \textit{whom}, \textit{which} and \textit{that} across subtypes in the Five Books of Moses and in the Historical Books in the new version.
\end{sloppypar}

\begin{table}[p]
\tabcolsep=.75\tabcolsep
\fittable{\begin{tabular}{l *4{r@{~}r}}
\lsptoprule
Role & \multicolumn{2}{c}{\textit{who}} & \multicolumn{2}{c}{\textit{whom}} & \multicolumn{2}{c}{\textit{which}} & \multicolumn{2}{c}{\textit{that}}\\\midrule
subject (7232) & 5339 & (73.82\%) & -- & {} & 924 & (12.78\%) & 969& (13.40\%)\\
direct object (2168)& -- & {} & 398 & (18.36\%) & 1221 & (56.32\%) & 549& (25.32\%)\\
indirect object (1)& -- & {} & 2 & (100\%) & -- & {} & --& {}\\
PP complement (503) & -- & {}& 265 & (52.68\%) & 115 & (22.86\%) & 123& (24.45\%)\\
\lspbottomrule
\end{tabular}}
\caption{The distribution of relative markers in the new version}
\label{tablenkjgenesis}
\end{table}

Table \ref{tablenkjgenesis} includes the same set of data as Table \ref{tablekjgenesis} (that is, the mismatches subsumed under ``other'' in Tables \ref{tablekjsubject}--\ref{tablekjthat} are disregarded, as well as the cases in which the original version contains an element other than \textit{who(m)}/\textit{which}/\textit{that}). As can be seen, no changes occur in the case of PP complements, but there are considerable changes affecting subject and direct object relative clauses (indirect relative clauses cannot be measured). The proportion of \textit{that}-relatives remains the same in object relatives; however, \textit{which}-relatives decrease in favour of \textit{whom}-relatives, which can be attributed to the fact that \textit{which} is no longer possible with human referents in the standard language. In subject relatives, there are two major changes, both resulting in an increase in the proportion of \textit{who}-relatives. On the one hand, the proportion of \textit{that}-relatives decreases in favour of \textit{wh}-relatives, though it remains slightly higher than in object relatives, in line with the prediction made by the Noun Phrase Accessibility Hierarchy. On the other hand, just as in object relatives, original \textit{which}-relatives with a human referent were changed to \textit{who}-relatives: still, due to the general decrease in the use of \textit{that}-relatives, the proportion of \textit{which}-relatives in subject relatives is actually higher than in the original version. Despite these crucial differences, the asymmetry between the functions remains. Again, I carried out a chi-square test on the distribution of \textit{who(m)}/\textit{which}/\textit{that} in the three types of relative clauses: this test reveals that the differences are significant: $\chi^2 (4, N = 9903) = 2400.8996,\allowbreak p < 0.00001$, meaning that the choice of relative marker is dependent on the relativised function. This indicates that the subject/object asymmetry is quite robust in the language.\footnote{This provides additional support for the hypothesis expressed by \citet{bacskaiatkari2020nordlyd}, according to which the English case system (contrasting nominative with oblique) is ultimately responsible for the observed differences: the case system is ultimately unchanged in the two periods under scrutiny.}

\subsection{Discussion} \label{sec:4discussion}\largerpage
Let us start with the discussion of the corpus data with respect to the variation between \textit{who(m)} and \textit{which}. Note that in the case of \textit{which}-relatives with human referents, all cases had to be altered in the new version since \textit{which} is not possible in these cases in modern Standard English. The fact that the proportion of \textit{which} is about the same in subject and object relatives indicates that this element was probably not sensitive to a subject/object asymmetry.

As pointed out earlier in this section, the distinction is quite clear in Standard English: \textit{who} is used with human referents (including the ``sanctioned borderline cases''), while \textit{which} is used with non-human referents. The situation is somewhat different in regional dialects. \citet[41--42]{herrmann2005} reports that while \textit{who} is restricted to human referents just like in Standard English, \textit{which} is preferably but not exclusively used with non-human referents: \textit{which} with human (personal) referents occurs in five of the six dialect areas she examined (Central Southwest, East Anglia, Central Midlands, Central North, Scotland). In the sixth dialect area, Northern Ireland, there were only very few instances of \textit{which} occurring with non-human referents, but these dialects hardly use \textit{wh}-pronouns in relative clauses (\citealt[41]{herrmann2005}). It appears that the occurrence of \textit{which} with human referents in dialects is not regionally bound, but altogether not very frequent. The data given by \citet[41, Table 3]{herrmann2005} show that out of all occurrences of \textit{who} as a relative pronoun, the referent is human in 96.4\% of the cases and non-human in 3.6\% of the cases (all ``sanctioned borderline cases''), while in the case of \textit{which} as a relative pronoun, the referent is human in 4.2\% of the cases and non-human in 95.8\% of the cases. 

It should be clear that the use of \textit{which} with human referents is in fact very restricted in dialects and altogether much less attested than in the King James Bible. I assume that the results in the King James Bible are indicative of a previous stage in the grammaticalisation of \textit{which} as [--human], and that significant changes took place in Late Modern English afterwards, leading to the present-day distribution. The relevant change has its roots earlier in the history of English. As \citet[41]{herrmann2005} points out, \textit{which} was possible with human referents in Middle English (cf. \citealt{mosse1991}) and the grammaticalisation of \textit{which} as [--human] started in the 16th century (cf. \citealt{nevalainenraumolinbrunberg2002}). It appears that while the grammaticalisation process is evidently completed in the standard variety, there are still exceptions in regional dialects; at the same time, the dialectal pattern suggests that \textit{which} strongly tends towards [--human] and hence the grammaticalisation process has affected regional dialects as well, albeit not to the same degree as the standard variety. Naturally, the gradual change that can be observed in dialects is in line with the assumption that language change (and variation) is gradual (see \citealt{traugotttrousdale2010}).\largerpage

Regarding the distribution of \textit{that}, it should be kept in mind that the use of \textit{that} in relative clauses is part of the standard variety, though its distribution is somewhat different from non-standard varieties. In subject and object relative clauses, such as the ones examined in the corpus study presented above, the use of \textit{that} is in line with the standard pattern (as opposed to cases where the standard variety would use PPs), and hence the restrictedness of \textit{that} in the new version can be attributed to a strongly norm-oriented use that goes beyond mere standardisation. This is naturally an important factor that must be considered when evaluating the data from the new version. 

Importantly, the results show a strong subject/object asymmetry: the question is whether this difference should necessarily be attributed to the King James Bible or whether it may also be due to the new translation. \citet[48--59]{herrmann2005} shows that the Noun Phrase Accessibility Hierarchy of \citet[66--67]{keenancomrie1977} is relevant in the spread of the relative particles \textit{that} and \textit{as}: subjects are more accessible than objects, which predicts not only that subject relative clauses are more frequent but also that relative complementisers are more frequent in subject relative clauses than in object relative clauses (which is ultimately related to processing reasons). This may be a reason behind \textit{that}-relatives being more frequent in subject relatives in the King James Bible than in object relatives, and \textit{as}-relatives being attested in subject relatives but not in object relatives.

In the case of \textit{that}-relatives, however, it is perfectly possible that not all instances were changed to \textit{who}/\textit{whom} in the new version, and as \textit{that}-relatives were not included in the search results for the new version, the proportion of \textit{that}-relatives may eventually be different when considering all relative clauses. Observe the following examples:\largerpage

\ea \label{peoplerel}
\ea And Abram took Sarai his wife, and Lot his brother's son, and all their substance that they had gathered, and the souls \textbf{that} they had gotten in Haran; and they went forth to go into the land of Canaan; and into the land of Canaan they came. \label{peoplerelkjwh}\\
(King James Bible; Genesis 12:5)
\ex Then Abram took Sarai his wife and Lot his brother's son, and all their possessions that they had gathered, and the people \textbf{whom} they had acquired in Haran, and they departed to go to the land of Canaan. So they came to the land of Canaan. \label{peoplerelnkjwh}\\
(New King James version; Genesis 12:5)
\ex Then Jacob was greatly afraid and distressed: and he divided the people \textbf{that} was with him, and the flocks, and herds, and the camels, into two bands; \label{peoplerelkjthat}\\
(King James Bible; Genesis 32:7)
\ex So Jacob was greatly afraid and distressed; and he divided the people \textbf{that} were with him, and the flocks and herds and camels, into two companies.\\
(New King James version; Genesis 32:7) \label{peoplerelnkjthat}
\z
\z

In both of the loci given in (\ref{peoplerel}), the head noun is \textit{people} (or its synonym \textit{souls}): the relative clause is introduced by \textit{that} in the original version both in (\ref{peoplerelkjwh}) and in (\ref{peoplerelkjthat}). The new version, however, uses a \textit{wh}-pronoun only in the case of the object relative, see (\ref{peoplerelnkjwh}), but not in the case of the subject relative, see (\ref{peoplerelnkjthat}), which contains the complementiser \textit{that}. The asymmetry attested in (\ref{peoplerel}) is due to the newer version and not to the original. Hence, in order to achieve reliable conclusions regarding the new version, all the occurrences of \textit{that} should be considered as well. Since the examination of the asymmetry is not immediately relevant to the present study, I will not investigate this question any further.

Considering subject relatives, however, it is evident that the frequency of \textit{that}-relatives is quite high and in fact higher than could be expected based on the present-day dialectal data. \citet[24]{herrmann2005} argues that this is overall the most typical strategy in dialects. At the same time, it is much more dominant in the North: its share is 50.1\% in Northern Ireland, 46.2\% in Scotland, 43.5\% in the Central North, and 40.3\% in the Central Midlands, while it is less frequent in the South (below 30\% in the areas of East Anglia and Central Southwest), see \citet[27, Table 1]{herrmann2005}. This is in line with the assumption that traditional forms in relative clauses seem to be on the retreat (see \citealt[291--292]{kortmannwagner2007}), as opposed to the spread of innovative \textit{what} in dialects, which is more dominant in the South than in the North and correlates with the frequency of \textit{that}-relatives (see \citealt[27, Table 1]{herrmann2005}). The results of the present corpus study indicate that the proportion of \textit{that} was apparently indeed higher in Early Modern English subject relative clauses. Note, however, that the innovative pattern \textit{what} also involves a uniform relative particle (syntactically a grammaticalised complementiser) and as far as the syntax of relative clauses is concerned, the lexicalisation of C is still fulfilled.

I will return to the issue of \textit{as}-relatives later in this chapter. What matters for us at this point is that the preference for the complementiser strategy in English is not only attested in dialects but also historically, including data from Early Modern English, contrasting with the norm-oriented standard language. On the other hand, it is worth noting that not all non-standard patterns are attested: there were no true subject zero relatives in the King James Bible, and Doubly Filled COMP patterns do not occur in relative clauses in the corpus either. This is in line with the claim made in \sectref{sec:4relativepronouns} that West Germanic languages (and apparently Germanic languages more generally) tend to lexicalise the C head in relative clauses, and this requirement is already met by inserting the complementiser, while the insertion of an overt relative pronoun is redundant in these cases. This is important especially because in the literature on Doubly Filled COMP patterns going back to \citet{chomskylasnik1977}, embedded interrogatives and relative clauses are often treated on a par with each other, yet there seems to be an important asymmetry between the two constructions regarding doubling, which clearly indicates that doubling is the result of other processes and requirements, and not simply the elimination of one element due to some surface filter.

\section{Doubling in relative clauses} \label{sec:4doubling}
As pointed out in \sectref{sec:4relativepronouns} already, once it is assumed that relative pronouns are in [Spec,CP] and relative complementisers are in C, it is expected that the two should be able to co-occur. In \chapref{ch:2}, I briefly discussed the issue of doubling in English and German. Consider again the English examples (\citealt[59, ex. 85]{vangelderen2013}):

\ea \label{dfcrelenglish}
\ea	This program \textbf{in which that} I am involved is designed to help low-income first generation attend a four year university and many of the resources they\ldots
\ex	It's down to the community \textbf{in which that} the people live.
\z
\z

As \citet{vangelderen2013} notes, while such examples are attested, they are altogether not very frequent (contrasting with Doubly Filled COMP patterns in embedded interrogatives); see also the discussion of the Early Modern English data in \sectref{sec:4variation}. In \chapref{ch:2}, I argued that patterns like (\ref{dfcrelenglish}) represent true Doubly Filled COMP rather than the combination of a projection hosting the relative operator and another one encoding merely finiteness (contrary to \citealt{baltin2010}). The structure is shown in \figref{treedfcenglish}.

\begin{figure} 
\caption{Doubling in relative clauses} \label{treedfcenglish}
\begin{forest} baseline, qtree
[CP
	[who(m)/which\textsubscript{{[}rel{]}}]
	[C$'$
		[C\textsubscript{{[}rel{]},{[}fin{]}}
			[that\textsubscript{{[}rel{]},{[}fin{]}}]
		]
		[TP]
	]
]
\end{forest}
\end{figure}

One piece of evidence for the implausibility of two designated projections comes from German dialects involving \textit{wo}: as \textit{wo} is clearly the canonical relative complementiser in these dialects (see \citealt{brandnerbraeuning2013} and the discussion in \sectref{sec:4relative} above) and not a general finite complementiser, it is unlikely that it would merely mark finiteness. The canonical pattern with \textit{wo} was illustrated in (\ref{relwoalemannic}), repeated here as (\ref{relwoalemannicrepeat}) below (\citealt[140, ex. 23]{brandnerbraeuning2013}):

\ea \gll Ich suech ebber \textbf{wo} mer helfe künnt. \label{relwoalemannicrepeat}\\
I search someone \textsc{rel} I.\textsc{dat} help.\textsc{inf} could\\
\glt `I am looking for someone who could help me.'
\z

Such patterns are attested in Alemannic (\citealt{brandnerbraeuning2013}, \citealt{weiss2013}), in Hessian (\citealt{fleischer2004, fleischer2016}) and in Bavarian (\citealt{weiss2013}). In addition, in Northern Bavarian the complementiser \textit{was} is used (\citealt{weiss2013}). This is illustrated in (\ref{was}) below (\citealt[780, ex. 19c]{weiss2013}):

\ea \gll Röslen (\ldots), \textbf{was} oben am hohlen Wege stehn \label{was}\\
roses \phantom{\ldots} \textsc{rel} above at.the empty road stand.\textsc{3pl}\\
\glt `roses, which are above by the empty road'
\z 

Note that while Hessian also uses \textit{was} in relative clauses, it is a very limited pattern and in many dialects it is restricted to neuter antecedents (\citealt{fleischer2004, fleischer2016}, \citealt{weiss2013}); therefore, contrary to Northern Bavarian, \textit{was} is rather an operator in Hessian.

Doubling in Alemannic and Hessian involves \textit{wo} and a \textit{d}-pronoun (\citealt{brandnerbraeuning2013}, \citealt{fleischer2016}), as shown for Hessian in (\ref{dfchessian}) below (\citealt{fleischer2016}):

\ea \gll Des Geld, \textbf{des} \textbf{wo} ich verdiene, des geheert mir. \label{dfchessian}\\
the.\textsc{n} money that.\textsc{n} \textsc{rel} I earn.\textsc{1sg} that.\textsc{n} belongs I.\textsc{dat}\\
\glt `The money that I earn belongs to me.'
\z

In Bavarian, the combination of \textit{was} and a \textit{d}-pronoun is possible, as in (\ref{dfcbavarian}) below (\citealt[780, ex. 19d]{weiss2013}):

\ea \gll Mei Häusl (\ldots), \textbf{dös} \textbf{wos} dorten unten (\ldots) steht \label{dfcbavarian}\\
my house.\textsc{dim} {} that.\textsc{n} \textsc{rel} there below {} stands\\
\glt `My little house, which stands down there'
\z

As described by \citet{fleischer2016}, the same is not possible in Hessian; the complementary distribution of \textit{was} and the \textit{d}-pronoun in Hessian again indicates that \textit{was} is a relative operator and not a grammaticalised complementiser, unlike in Bavarian.

The appearance of the pronouns in relative clauses with nominal heads indicates that they cannot be treated as matrix elements; they belong to the left periphery of the relative clause. In \chapref{ch:2}, I argued that such patterns are incompatible with a Force--Fin distinction in cartographic approaches, due to the complementiser (\textit{wo}/\textit{was}) being the canonical relative complementiser and not a finiteness marker (unlike \textit{dass} in embedded interrogatives). Instead, they are instances of Doubly Filled COMP involving the direct merge of the pronoun to the clause headed by the complementiser, as in \figref{treedfcgerman}.

\begin{figure}
\caption{Doubling in German relative clauses} \label{treedfcgerman}
\begin{forest} baseline, qtree
[CP
	[der/die/das\textsubscript{{[}rel{]}}]
	[C$'$
		[C\textsubscript{{[}rel{]},{[}fin{]}}
			[wo/was\textsubscript{{[}rel{]},{[}fin{]}}]
		]
		[TP]
	]
]
\end{forest}
\end{figure}

As far as the connection between the relative clause and the matrix clause is concerned, I adopt a matching analysis rather than a head raising analysis. Under this view, the DP headed by the relative pronoun contains a phonologically invisible NP that matches (is identical to) the head noun NP; that is, the two NPs are not connected via movement but are both base-generated (see \citealt[55--179]{salzmann2017} and \citealt{pankau2018} on arguments in favour of the matching analysis, and see also \citealt{lees1960, lees1961as}, \citealt{chomsky1965}, \citealt{sauerland1998diss, sauerland2003} for similar views, as well as \citealt{bhatt2005lot} for a comparative summary).\footnote{One serious advantage of the matching analysis over the head-raising analysis is that ``it adheres to the more traditional constituency and does not involve raising of the head'' (\citealt[174]{salzmann2017}). In other words, by adopting this kind of analysis, the relative operator undergoes regular operator movement in the embedded CP just like in embedded constituent questions, and there is no further raising operation to the matrix clause. The similarities between the two clause types, especially with respect to doubling, are thus expected to arise.} However, instead of deleting the NP in the subclause, I will assume it to be zero, though nothing crucial hinges on this.

Consider the following example from Standard German:

\ea \gll Der Mann, der Kartoffeln schält, ist mein Bruder. \label{kartoffeln}\\
the.\textsc{m.nom} man who.\textsc{m.nom} potatoes peels is my.\textsc{m.nom} brother\\
\glt `The man who is peeling potatoes is my brother.'
\z

The structure for the relative clause in (\ref{kartoffeln}) is shown in \figref{treeder}.\footnote{As should be obvious, the complex DP shown in \figref{treeder} functions as the subject of the matrix clause in (\ref{kartoffeln}); it undergoes movement from the vP to the [Spec,CP] position, as is regularly the case in main clause German declaratives. Since the position of the head noun DP within the matrix clause is not relevant for the discussion here, \figref{treeder} shows only the structure of the relative clause. The same applies to all other representations in this section.}

\begin{figure} 
\caption{The structure of headed relative clauses} \label{treeder}
\begin{forest} baseline, qtree
[DP
	[D
		[der]
	]
	[NP
		[NP [Mann,roof]]
		[CP [DP [D [der]] [NP [$\emptyset$,roof]]] [C$'$ [C] [TP [Kartoffeln schält,roof]]]]
	]
]
\end{forest}
\end{figure}

As can be seen, the relative pronoun \textit{der} is a D head and takes an empty NP as its complement; the NP takes its reference from the NP in the matrix clause. (Coindexing is not used in the tree diagram as it might create the impression that movement is involved.) The C head is an empty complementiser.

Essentially, dialectal patterns are very similar in their syntax. The representation in \figref{treewo} below shows ordinary \textit{wo}-relatives (the same applies to Bavarian \textit{was}-relatives).

\begin{figure} 
\caption{Headed relatives clauses with \textit{wo}} \label{treewo}
\begin{forest} baseline, qtree
[DP
	[D
		[der]
	]
	[NP
		[NP [Mann,roof]]
		[CP [DP [D [\textit{Op}.]] [NP [$\emptyset$,roof]]] [C$'$ [C [wo]] [TP]]]
	]
]
\end{forest}
\end{figure}

The representation in \figref{treederwo} shows the doubling of a \textit{d}-pronoun and \textit{wo} (the same applies to doubling involving \textit{was} in Bavarian).

\begin{figure} 
\caption{Doubling in headed relatives} \label{treederwo}
\begin{forest} baseline, qtree
[DP
	[D
		[der]
	]
	[NP
		[NP [Mann,roof]]
		[CP [DP [D [der]] [NP [$\emptyset$,roof]]] [C$'$ [C [wo]] [TP]]]
	]
]
\end{forest}
\end{figure}

In \figref{treewo}, the complementiser is overt but the relative pronoun is not; however, the relative pronoun contains no information that could not be recovered, and the type of the clause is also overtly marked by the complementiser. In \figref{treederwo}, both the complementiser and the relative pronoun are overt. In either case, the empty NP takes its reference from the matrix NP, just like in \figref{treeder}.

The difference between \figref{treewo} and \figref{treederwo} lies in whether the relative pronoun is overt or not. The assumption is that the lexicalisation of the pronoun is essentially possible in dialects that have relative pronouns in the first place; note that in the South German dialects under scrutiny, the insertion of the overt complementiser is the default option, unlike in Standard English, where relative pronouns in themselves constitute a canonical option.\footnote{Whether constructions with zero operators are actually available depends on various factors; the absence of an overt relative pronoun leads either to zero relatives or to relative clauses introduced by an overt complementiser. Zero relatives constitute a restricted option (see \sectref{sec:4relative} and \sectref{sec:4relativepronouns}). The complementiser strategy is clearly not an option in varieties that do not permit relative complementisers in the first place (e.g. Standard German and Standard Dutch). But restrictions arise even in varieties that have both the pronoun and the complementiser strategy. As discussed in \sectref{sec:4variation}, the syntactic function of the gap makes an important difference here: the complementiser strategy (as a single option) tends to be more available in certain functions than in others. The case of the head noun may also be decisive. \citet[215--217, 221--222]{bayer1984} reports for Bavarian that the presence or absence of the relative pronoun alongside the relative complementiser \textit{wo} depends on the morphological case of the relative operator and the head noun: if the cases match, the definite article and the relative pronoun are phonologically identical and the relative pronoun is optional. The picture is more interesting with case mismatches: relative pronouns in the dative are obligatory with nominative head nouns; in these cases, there is no form identity for any of the genders (or for the plural). This is not true vice versa: nominative relative pronouns are merely optional with dative head nouns (suggesting that the unmarked form is always fully recoverable). Accusative relative pronouns show mixed behaviour with case mismatches: a masculine accusative relative pronoun (\textit{den}) is obligatory with nominative head nouns (marked by the article \textit{der}) but the same does not hold for feminine (\textit{die}) and neuter (\textit{das}) ones, where nominative and accusative forms are phonologically identical.}

Interestingly, the insertion of relative pronouns is possible even in languages that otherwise use, or at least strongly prefer, complementisers. Consider the following examples from Norwegian (\citealt[185--187]{bacskaiatkaribaudisch2018}):

\ea
\ea \gll Dette er studenten \textbf{som} inviterte	Mary. \label{norwegiansom}\\
this is	the.student	that invited.\textsc{pst}	Mary\\
\glt `This is the student who invited Mary.'
\ex \gll Dette	er byen	\textbf{der} \textbf{som}	eg vart	fødd. \label{norwegiandersom}\\
this	is	the.city	which	that	I	was	born\\
\glt `This is the city where I was born.'
\z
\z

The option in (\ref{norwegiansom}) is the ordinary option showing the relative complementiser \textit{som}; the doubling option in (\ref{norwegiandersom}) including a \textit{d}-pronoun \textit{der} was indicated as possible by the informant from Rogaland county but not by the one from Vest-Agder county in the study quoted above. The difference lies in whether the \textit{d}-pronoun is acceptable as a relative pronoun or not. The same applies to Swedish (\citealt[246--247]{bacskaiatkaribaudisch2018}):

\ea
\ea \gll Detta	är studenten \textbf{som} bjöd in Mary. \label{swedishsom}\\
this is	the.student	who	invites	in Mary\\
\glt `This is the student who invites Mary.'
\ex \gll Detta	är studenten \textbf{vilken} \textbf{som} bjöd in	Mary. \label{swedishvilkensom}\\
this is	the.student	which that invites in	Mary\\
\glt `This is the student who invites Mary.'
\z
\z

Again, the option in (\ref{swedishsom}) is the ordinary option showing the relative complementiser \textit{som}. The doubling option in (\ref{swedishvilkensom}) including the \textit{wh}-based pronoun \textit{vilken} shows variation between the two informants involved in the questionnaire: it was indicated as possible by the informant from the Färgelanda municipality but not by the one from Göteborg in the study quoted above. The difference lies in whether the \textit{wh}-pronoun is acceptable as a relative pronoun.

While Norwegian and Swedish show variation concerning doubling patterns, the situation appears to be different in Danish. In Danish, the complementiser \textit{som} occurs on its own in relative clauses as in (\ref{danishrel}) but not in combination with other elements (\citealt[89--91]{bacskaiatkaribaudisch2018}):

\ea \gll Dette er bogen \textbf{som} Mary købte. \label{danishrel}\\
this is the.book that Mary bought.\textsc{pst}\\
\glt `This is the book which Mary bought.'
\z

There is thus a strong tendency in Mainland Scandinavian languages for the complementiser strategy, which is also the standard option (unlike what can be generally observed in West Germanic, see above), yet the relative pronoun may also be lexicalised in some cases. The observed variation and the lack of clearly defined syntactic rules in terms of when the pronoun appears are in line with the assumption that the insertion of the relative pronoun does not generate a new projection in the syntax but it merely lexicalises a position that is covertly present anyway but is essentially redundant.

This leads to the last question to be addressed in this section, which concerns the differences between embedded interrogative clauses and relative clauses in terms of doubling patterns. As discussed in \chapref{ch:3} already, the two constructions show a remarkable surface similarity in this respect, especially in English, where the specific elements involved are also surface-identical. Indeed, the two clause types have also been treated analogously in the literature (see, for instance, \citealt{chomskylasnik1977} and \citealt{chomsky1977}). However, there are some important asymmetries to be observed here. First, as discussed in the present section and in \sectref{sec:4relativepronouns}, it seems that Germanic languages generally favour the complementiser strategy over the pronoun strategy (which can, depending on the language and the variety, show considerable differences). In embedded interrogatives, the complementiser strategy is favoured in polar interrogatives but it is impossible in constituent questions. Second, while doubling is widespread in embedded constituent questions across Germanic, it seems to be altogether less frequent in relative clauses (and, as we saw, it is also a marginal option in embedded polar questions). Given the rather unified syntactic template (namely, a single CP for all of these constructions), the asymmetries may seem somewhat surprising at first.

I suggest that the reason for these differences lies in the information structural properties of the operators.\footnote{I will discuss the relevance of information structure for left peripheries in \chapref{ch:6} in more detail.} The relevant distinction can best be formulated as discourse-new vs. discourse-old. In interrogatives, the operator is associated with discourse-new information;\footnote{Note that discourse-new does not equal new information. In fact, it is possible that \textit{wh}-phrases represent old information (both in terms of the speaker and in terms of the reader), yet it is not necessarily the case that the relevant information is present in the preceding discourse. In addition, newness cannot be equalled with focusing either, as discussed by \citet[255--257]{krifka2008}, so that focus-like properties are not even necessarily expected to be related to newness (contrary to the ``information focus'' proposed by \citealt{halliday1967}).} in the classical scenario, the \textit{wh}-part of a constituent question corresponds to a focused element in the answer (see \citealt[250]{krifka2008}, citing \citealt{paul1880}). The \textit{wh}-phrase is associated with the presence of alternatives and it regularly bears main stress.\footnote{There is a strong correlation between discourse-new and stress, yet no one-to-one correspondence, as discussed by \citet[874--876]{buering2013}. One reason behind this is that the relevant properties represent non-prosodic information that is mapped onto the prosodic component from syntax rather than being prosodic properties (see \citealt[860--861]{buering2013}). \citet[248]{krifka2008} suggests that a focus property indicates the presence of alternatives (this idea in turn goes back to \citealt{vonstechow1981} and to \citealt{rooth1985diss}, and it was adopted by later analyses, see \citealt{buering2013}). See also \citet[197--198]{bacskaiatkari2022jb} for more discussion.} Ordinary (headed) relative clauses differ in that the relative operator expresses discourse-old information: it is co-referent with the head noun and is hence recoverable. The polar operator is also recoverable and does not need to be overt.

It follows that doubling in embedded constituent questions results from the interplay of two independent factors: first, the operator must be overt due to its discourse function; second, the C head is preferably lexicalised by an overt element in the languages under scrutiny. In relative clauses, however, the first requirement does not hold (due to the information structural properties of relative operators) and the second requirement is satisfied by inserting a relative complementiser. We can thus establish that there is no doubling requirement per se in either of these constructions, and the observed differences between the two clause types can be answered by considering the information structural properties of the respective elements (see \citealt{bacskaiatkari2022jb} for more details).

\section{Doubling in free relatives} \label{sec:4doublingfree}
So far I have chiefly concentrated on relative clauses with nominal heads. Interestingly, doubling patterns are also possible in free relatives. In West Germanic languages, free relatives are introduced by \textit{wh}-pronouns, as illustrated in (\ref{egd}) below for English, German and Dutch, respectively (all examples from the standard varieties):\footnote{Given the similarity between interrogative and free relative clauses, the two have been claimed to have similar syntactic and semantic properties (see \citealt{caponigro2003diss}, \citealt{polettosanfelici2019}). What seems to be somewhat surprising is therefore not that \textit{wh}-elements appear in free relatives but rather that they appear in headed relative clauses. \citet{watanabe2009} argues that the indefinite \textit{wh}-base in Old English was also quantificational, making the clause into a complete proposition, which was incompatible with headed relatives. Once this property was lost, the \textit{wh}-base became available for relative clauses. (The relationship between the indefinite \textit{wh}-bases and the interrogative \textit{wh}-pronouns constitutes a long-standing debate that cannot be discussed here; see \citealt[348]{brugman1911} and \citealt{gonda1954}.) Note that \textit{wh}-pronouns are more related to the indefinite use (\citealt{paul1920band4}, \citealt{gonda1954}, \citealt{gisbornetruswell2017}) than to the definite use, unlike demonstrative-based pronouns, which stem from (definite) demonstrative pronouns. \citealt{bacskaiatkaritoappearfdsl} hypothesises that the original definite/indefinite distinction has a reflex in terms of the interpretability of the [rel] feature: demonstrative-based relative markers are regularly equipped with an [i-rel] feature, while \textit{wh}-based relative markers are regularly equipped with an [u-rel] feature; this arrangement restricts the possible combinations as well, so that doubling patterns regularly show asymmetric patterns (that is, \textit{wh}+\textit{d} or \textit{d}+\textit{wh}). This is in line with the assumption that the feature properties of ordinary relative clauses and free relatives differ.}

\ea \label{egd}
\ea You should finish \textbf{what} you have begun. \label{what}
\ex \gll Ich nehme \textbf{was} du nimmst.\\
I take.\textsc{1sg} what you take.\textsc{2sg}\\
\glt `I'll take what you take.'
\ex \gll \textbf{Wie} zoiets doet, is gek.\\
who such does is crazy\\
\glt `Whoever does such a thing is crazy.'
\z
\z

In theses cases, there is no lexical head; the matrix clause contains an empty DP (see \citealt{vanriemsdijk2006}). In German, the finite complementiser \textit{dass} `that' can be inserted dialectally, as in the following example from Bavarian (\citealt[781, ex. 21c]{weiss2013}):

\ea \gll \textbf{wem} \textbf{dass} des zvei is, kann aa wenger zoin \label{germanwemdass}\\
who.\textsc{dat} that that.\textsc{n} too.much is can.\textsc{3sg} also less pay.\textsc{inf}\\
\glt `Whoever finds it too much can pay less as well.'
\z

Importantly, \textit{dass} is not a relative marker otherwise in these dialects (that being \textit{wo}, see \sectref{sec:4doubling}); the insertion of \textit{dass} takes place to satisfy the lexicalisation requirement on [fin] in C and happens exactly the same way as in embedded constituent questions (see \chapref{ch:3}). 

A similar doubling pattern can also be observed in Flemish, as illustrated in the following example (\citealt[358]{zwart2000}, citing \citealt[143]{vanacker1948}):

\ea \gll \textbf{Wie} \textbf{dat} er nou trouwt zijn stommerike.\\
who that there now marries are stupid.ones\\
\glt `Whoever gets married nowadays is stupid.'
\z

The example is from South Brabant (the relevant territory is today Vlaams-Brabant, Flemish Brabant). According to \citet[357]{zwart2000}, Dutch dialects do not have \textit{dat}-relatives; however, \citet[141]{boef2013} reports, on the basis of the SAND1 data, that relative clauses with the complementiser \textit{dat} but without a visible relative pronoun are actually possible in this territory. The appearance of \textit{dat} in headless relative clauses, however, is rather due to \textit{dat} being a finite complementiser in these cases. In English, \textit{that} is universally acceptable as a relative complementiser, yet it does not introduce free relatives. Essentially the same applies to Mainland Scandinavian \textit{som}-relatives.

Doubling patterns like (\ref{germanwemdass}) in German have the same structure as embedded constituent questions showing Doubly Filled COMP effects: the complementiser is inserted regularly as the [fin] marker, and the [wh] feature is checked off by the operator. The structure for (\ref{germanwemdass}) is shown in \figref{treefreerel}.

\begin{figure} 
\caption{Doubling in headless relatives} \label{treefreerel}
\begin{forest} baseline, qtree
[CP
	[wem\textsubscript{{[}wh{]}}]
	[C$'$
		[C\textsubscript{{[}wh{]},{[}fin{]}}
			[dass\textsubscript{{[}fin{]}}]
		]
		[TP]
	]
]
\end{forest}
\end{figure}

I assume that in free relatives, a [wh] feature and not a [rel] feature is involved (see \citealt{groosvanriemsdijk1981}). This predicts that relative complementisers and relative pronouns equipped with a [rel] feature cannot appear in free relatives: this is indeed borne out, as the German complementisers \textit{wo} and \textit{was} do not occur in free relatives, and \textit{d}-pronouns are not attested either.\footnote{As discussed by \citet{fussgrewendorf2014}, on the surface German has free relatives with \textit{d}-pronouns; however, they show that these constructions differ from free relatives with \textit{w}-pronouns in terms of their syntactic and semantic properties and that they are in fact headed relative clauses: in this sense, there are no free relatives involving \textit{d}-pronouns in German. Note that the above generalisation holds for \textit{d}-pronouns and does not include the particle \textit{da}: as noted by \citet[782]{weiss2013}, this occurs in the dialect of Leipzig in addition to the \textit{w}-pronoun in free relatives (constituting a Doubly Filled COMP effect) and it is also attested in the same doubling patterns in embedded interrogatives. This again underlines the parallelism between free relatives and embedded constituent questions; the particle \textit{da} in this respect shows similar behaviour to the complementiser \textit{dass}.}

The more detailed structure is given in \figref{treefreereldetailed}.

\begin{figure} 
\caption{The detailed structure of headless relatives} \label{treefreereldetailed}
\begin{forest} baseline, qtree
[CP
	[DP 
			[D [wem]] [NP [$\emptyset$,roof]]
	]
	[C$'$
		[C
			[dass]
		]
		[TP]
	]
]
\end{forest}
\end{figure}

The zero NP complement, just like in interrogative clauses, requires no overt antecedent. This construction is, however, not available with all \textit{wh}-elements. For instance, \textit{which} in English takes an overt NP complement in interrogatives like (\ref{whichint}) and is not licensed in free relatives in the way shown in (\ref{whichfreerel}):

\ea
\ea[]{\textbf{Which book} should I read next? \label{whichint}}
\ex[*]{You should finish \textbf{which} you have begun. \label{whichfreerel}}
\ex[]{This is the book \textbf{which} I should read next. \label{whichrel}}
\z
\z

As indicated by (\ref{whichrel}), \textit{which} is perfectly possible as a relative pronoun if there is an overt antecedent (\textit{book}) in the matrix clause that licenses the covert NP complement, but not otherwise.

Again, just as in the case of ordinary relative clauses and interrogative clauses (see \chapref{ch:3}), there is no need for a cartographic split of the CP to accommodate the various overt elements in the structure.

\section{Triple combinations} \label{sec:4triple}
Against this background, the question of triple combinations is especially interesting: while double combinations are, as was shown in this chapter, compatible with a single canonical CP, the insertion of any further element into the CP-periphery requires further explanation. According to \citet{weiss2013} and \citet{grewendorfpoletto2015}, Bavarian allows combinations of the form ``\textit{d}-pronoun + \textit{wo} + \textit{dass}''. The example in (\ref{derwodass}) shows the combination in a relative clause with a lexical NP (\citealt[781]{weiss2013}):

\ea \gll dea Mã, \textbf{dea} \textbf{wo} \textbf{dass} des gsogd hod \label{derwodass}\\
the.\textsc{m} man that.\textsc{m} \textsc{rel} that that.\textsc{n} said.\textsc{ptcp} has\\
\glt `the man who said it'
\z

The example in (\ref{derwodassheadless}) shows the combination in free relatives (\citealt[781, ex. 21e]{weiss2013}):

\ea \gll \textbf{dem} \textbf{wo} \textbf{dass} des zvei is, kann aa wenger zoin \label{derwodassheadless}\\
that.\textsc{m.dat} \textsc{rel} that that.\textsc{n} too.much is can.\textsc{3sg} also less pay.\textsc{inf}\\
\glt `Whoever finds it too much can pay less as well.'
\z

Such constructions do not appear to be predominant, though. In the literature, they are only mentioned for Bavarian. Regarding Alemannic, in the project SynAlm (``Syntax des Alemannischen''), the combination was tested only for long movement (e.g. \textit{solche Blumen wüsste ich niemanden, der bei uns verkauft} `such flowers I do not know anyone who here sells', FB2/15; see FB2-251, Column: IT $\rightarrow$ Q\_15-4 and FB2-258, Column: IX $\rightarrow$ Q\_16-4 of the database). \citet[336--343]{salzmann2017} provides a detailed overview of relative clauses in Swiss German but mentions no constructions like (\ref{derwodass}) and (\ref{derwodassheadless}): the combination of \textit{wo} and \textit{dass} is generally not possible in relative clauses (unlike in interrogatives), which indicates that \textit{wo} in Swiss German is clearly a complementiser.

In addition, the combination is apparently not frequent. Regarding the combination of the \textit{d}-pronoun and \textit{wo} and (\ref{derwodassheadless}), \citet[781]{weiss2013} reports that for some speakers of Bavarian (referring to H. Altmann, personal communication), this combination is possible and even \textit{dass} can be added, which is generally possible in Bavarian, see (\ref{derwodass}). \citet{grewendorfpoletto2015}, who also mention the combination in (\ref{derwodass}), cite the same example as given in (\ref{derwodass}) and another one with complementiser agreement. At any rate, it should be kept in mind that Bavarian relative clauses can be headed not only by \textit{wo} but also by \textit{was}. Based on all this, it is quite probable that \textit{wo} has a different status in dialects that allow (\ref{derwodass}) and (\ref{derwodassheadless}) from what we can observe in Alemannic and in Hessian.

The cited sources also agree in locating the \textit{d}-pronoun in the relative clause and not in the matrix clause, which can be verified, for instance, by intonation. In addition, in (\ref{derwodass}) there is already a lexical head in the matrix clause (the DP \textit{der Mann} `the man'), and hence the \textit{d}-pronoun \textit{der} cannot be regarded as the matrix head. In (\ref{derwodassheadless}), the case of the pronoun (dative) shows that it belongs to the relative clause and not to the matrix clause. Moreover, the combination \textit{wo dass} in relative clauses is possible only if the \textit{d}-pronoun is inserted as well (see the data of \citealt[781]{weiss2013}). In other words, the insertion of \textit{dass} is allowed only if the status of \textit{wo} is different from what can be observed in ordinary \textit{wo}-relatives. This difference is at the same time responsible for the insertion of the \textit{d}-pronoun.

The idea is that \textit{wo} in these cases is not treated as a complementiser but as an operator; however, as an adverbial element it cannot have an NP complement and marks only clause type. The \textit{d}-pronoun is an internal head, similarly to \textit{that} in English free relatives. Consider:

\ea
\ea You should finish \textbf{what} you have begun. \label{whatrepeat}
\ex You should finish \textbf{that which} you have begun. \label{thatwhich}
\z
\z

In English, the \textit{wh}-pronoun \textit{what} can take an empty NP complement, as in (\ref{whatrepeat}). This is not possible with \textit{which} on its own, see (\ref{whichfreerel}) above, but if there is an overt antecedent in the form of the demonstrative pronoun \textit{that}, as in (\ref{thatwhich}), which functions as an internal head, the construction is grammatical. Since \textit{that} in Modern English is not a relative pronoun, its status in examples like (\ref{thatwhich}) is clearly different from relative pronouns.

The constructions in (\ref{derwodass}) and (\ref{derwodassheadless}) should have a similar structure. Importantly, \textit{d}-pronouns in German can be both demonstrative pronouns and relative pronouns. In (\ref{derwodass}), the \textit{d}-pronoun functions as a relative pronoun, but in (\ref{derwodassheadless}) it is inserted as a demonstrative pronoun, since relative pronouns with the feature [rel] are not possible in free relatives, see \sectref{sec:4doublingfree}. The element \textit{wo} is specified as [wh] in constructions like (\ref{derwodass}) and (\ref{derwodassheadless}): this allows it to appear in free relatives, but in ordinary relative clauses the insertion of a relative pronoun with a [rel] feature is necessary in addition. The structure for (\ref{derwodass}) is given in \figref{treederwodass}.

\begin{figure}
\caption{Triple combinations}\label{treederwodass}
\begin{forest} baseline, qtree
[CP
	[dea\textsubscript{{[}rel{]}}]
	[C$'$
		[wo\textsubscript{{[}wh{]}}]
		[C$'$
			[C\textsubscript{{[}rel{]},{[}fin{]}} [dass\textsubscript{{[}fin{]}}]] [TP]
		]
	]
]
\end{forest}
\end{figure}

Under this view, multiple specifiers are possible; the structure is in this respect similar to triple combinations in Dutch constituent questions and to V3 main clauses in German (see \chapref{ch:3} and \citealt[148]{bacskaiatkari2020jcgl}). This is in line with the basic operation Merge, which does not make reference to traditional X-bar notions. The complementiser \textit{dass} in \figref{treederwodass} marks only finiteness; the \textit{wh}-operator \textit{wo} can be inserted but it cannot check off the [rel] feature. Importantly, the insertion of the finite complementiser does not require a separate FinP: if that were possible, then the finite complementiser in Fin should be available regardless of whether there is a \textit{d}-pronoun in a higher specifier (possibly in ForceP). In other words, a separate, designated FinP would be expected to occur in \textit{wo}-relatives as well (\textit{wo dass}) or in relative clauses with a single \textit{d}-pronoun (\textit{der dass}), but neither of these options is empirically justified. In this way, not even triple combinations provide support for a cartographic analysis. The structure for (\ref{derwodassheadless}) is given in \figref{treederwodassheadless}.\largerpage

\begin{figure}
\caption{Triple combinations in free relatives} \label{treederwodassheadless}
\begin{forest} baseline, qtree
[CP
	[dem]
	[C$'$
		[wo\textsubscript{{[}wh{]}}]
		[C$'$
			[C\textsubscript{{[}wh{]},{[}fin{]}} [dass\textsubscript{{[}fin{]}}]] [TP]
		]
	]
]
\end{forest}
\end{figure}

Unlike in \figref{treederwodass}, the clause type in this case is not [rel] but [wh]; hence, \textit{wo} can mark clause type. The demonstrative pronoun is inserted so that the empty NP complement can be inserted, since \textit{wo} in nominal relative clauses cannot function as an internal head. The \textit{d}-pronoun is hence not a relative pronoun but it appears in the relative clause: its status is reminiscent of an intermediate stage in the reanalysis of demonstrative pronouns into relative pronouns, which can be detected in various languages (see \citealt{vangelderen2004, vangelderen2009}; see also \citealt{coniglio2019}, \citealt{axel2009}, \citealt{axeltober2012}).\footnote{This reanalysis process is taken to be part of the so-called relative cycle by \citet[77--99]{vangelderen2004} and \citet[161--168]{vangelderen2009}. The process is referred to as a cycle because relative pronouns may further be reanalysed into relative complementisers, leaving the [Spec,CP] position phonologically empty and thus available for novel relative pronouns.} In this respect, the structure given in \figref{treederwodassheadless}, just like the one in \figref{treederwodass}, is not an idiosyncratic feature of German.

The detailed structure for (\ref{derwodass}) and (\ref{derwodassheadless}) is given in \figref{treetriple}.

\begin{figure} 
\caption{The detailed structure of triple combinations} \label{treetriple}
\begin{forest} baseline, qtree
[CP
	[DP 
			[D [dea/dem]] [NP [$\emptyset$,roof]]
	]
	[C$'$
		[AdvP
			[wo,roof]
		]
		[C$'$
			[C [dass]]
			[TP]
		]
	]
]
\end{forest}
\end{figure}

As indicated, the relative position of the \textit{d}-pronoun is the same in both relative clauses with a nominal head and in free relatives. The difference lies in whether the \textit{d}-pronoun is specified as [rel] or not.

The question arises why such complex structures arise at all if double combinations are sufficient as well, see Sections~\ref{sec:4doubling} and \ref{sec:4doublingfree}. It appears that \textit{wo} in dialects that otherwise use \textit{was} rather than \textit{wo} as a relative complementiser has not (or not completely) grammaticalised as a [rel] complementiser, and hence \textit{wo} can be inserted as a \textit{wh}-operator with a [wh] feature. Consequently, other elements like the finite complementiser \textit{dass} and the \textit{d}-pronoun must be inserted to check off all necessary features. The complex syntactic structure is thus related to the syntactic status of the element \textit{wo}. However, since \textit{wo} can generally appear as a complementiser and there are other options for the formation of relative clauses, such complex constructions are expected to be rare since they are functionally equivalent to more economical configurations.

\section{Equative relative clauses} \label{sec:4equative}
As mentioned before (see Sections~\ref{sec:4relativepronouns} and \ref{sec:4variation}), in present-day English dialects, relative clauses with \textit{as} are still possible. This was exemplified in (\ref{allas}), repeated here as (\ref{allasrepeat}) below (taken from \citealt[64, ex. 26d]{herrmann2005}):

\ea {[}\ldots] so \textbf{all as} he had to do were go round in a circle all the time [\ldots] \label{allasrepeat}\\
(\textit{Freiburg English Dialect Corpus} Som\_001)
\z

Similar patterns were found in the King James Bible as well, illustrated in (\ref{kjpatternsas}):

\ea \label{kjpatternsas}
\ea And \textbf{such as} do wickedly against the covenant shall he corrupt by flatteries: but the people that do know their God shall be strong, and do exploits. \label{kjsuchas}\\(King James Bible; Daniel 1:4) 
\ex And his brother's name was Jubal: he was the father of \textbf{all such as} handle the harp and organ. \label{kjallsuchas}\\(King James Bible; Genesis 4:21) 
\z
\z

All the examples from the King James Bible (altogether 23 instances given in Table \ref{tablekjsubject}) contain the element \textit{such} in the matrix clause. Out of these, 19 instances are similar to (\ref{kjsuchas}) in that they do not contain an additional \textit{all}, while in 4 examples \textit{all} is also present (immediately preceding \textit{such}), as demonstrated in (\ref{kjallsuchas}). In essence, the presence of \textit{all} in these constructions appears to be optional in the King James Bible.

The studies on \textit{as}-relatives in present-day dialects cite examples containing \textit{all}, as in (\ref{allasrepeat}) above; another example is given in (\ref{asrel}) below (\citealt[72, ex. 8]{kjellmer2008}):

\ea They come back from the football or wherever we've been on a Sunday afternoon bath the kids get the telly on the fire on and get them a bit of tea and try
and sit and watch the telly and \textbf{all as} you hear is effing and blinding and screaming and shouting and threatening. He hates baths.\\(ukspok/04. Text: S9000001271) \label{asrel}
\z

As shown by \citet{kjellmer2008}, the contracted form \textit{alls} (deriving from the sequence \textit{all as}) is also possible, both in American English and apparently also in British English. The phenomenon is illustrated in (\ref{alls}) below (\citealt[69, ex. 3]{kjellmer2008}):

\ea \textbf{Alls} he needs is a bit of help and that you know.\\(ukspok/04. Text: S9000000507) \label{alls}
\z

It is evident from the examples of \citet{kjellmer2008} that \textit{alls} is used in free relatives (essentially in the sense of \textit{all that}) and not in ordinary relative clauses. At any rate, the availability of the contracted form strongly suggests that \textit{all} came to be a grammaticalised marker, which is in this respect similar to \textit{such} in the King James Bible. The point is that unlike other relative complementisers (such as English \textit{that} and German \textit{wo}), English relative \textit{as} is in many (but not all) cases contingent upon the presence of a particular equative-like element in the matrix clause.\footnote{Note that equative relative clauses in this respect constitute a bridging context between ordinary equative clauses and ordinary relative clauses. This contradicts the assumption of \citet[54]{koenig2015} that there are no plausible bridging contexts between equatives/comparatives and relative clauses, making his criticism of \citet{brandnerbraeuning2013} considerably weaker.} Note also that the very form \textit{as} derived from \textit{eallswa}, the combination of \textit{all} and \textit{so}, whereby historically the \textit{so}-forms (\textit{swelce}, \textit{swilce}, \textit{such} and \textit{so}, \textit{swa}) are also attested in \textit{as}-constructions (see \citealt[315--317]{kortmann1997}, \citealt[312--314]{lopezcousomendeznaya2014} and references there). Similarly, German \textit{als} derives from Old High German \textit{also} (\textit{all} + \textit{so}), whereby various forms of \textit{so} are possible historically in \textit{as}-constructions (\citealt{jaeger2010}).

Turning now to German, recall from \sectref{sec:4relative} that \textit{so}-comparatives are attested in German historically, as discussed in the study of \citet{brandnerbraeuning2013}. This is illustrated in (\ref{heliand}), repeated here as (\ref{relos}), for Old Saxon (\citealt[138, ex. 20]{brandnerbraeuning2013}):

\ea \gll \textbf{sulike} gesidos \textbf{so} he im selbo gecos \label{relos}\\
such companions so he him self chose\\
\glt `such companions that he chose for himself'\\(\textit{Heliand} 1280)
\z

As can be seen, the matrix clause contains a \textit{so}-element and the relative clause is introduced by \textit{so}. Similar patterns are also found with relative \textit{als}; an example for Middle High German is given in (\ref{mhgals}), repeated here as (\ref{mhgalsrepeat}) below (\citealt[136, ex. 13]{brandnerbraeuning2013}, citing \citealt{ebertreichmannsolmswegera1993}):

\ea \gll \ldots{} und begerten \textbf{solichen} schaden \textbf{als} sie deshalben gelitten \label{mhgalsrepeat}\\
{} and demanded such damage as they because.of.that suffered\\
\glt `And they demanded such damage that they had suffered because of that.'\\(Chr. V. Mainz 220)
\z

However, as is also obvious from \citet{brandnerbraeuning2013}, \textit{so} eventually grammaticalised as a relative complementiser; an example for this from Middle High German was given in (\ref{reinfried}), repeated as (\ref{reinfriedrepeat}) below (\citealt[132]{brandnerbraeuning2013}, citing \citealt{paul1920band3}):

\ea \gll d\"er Sache \textbf{s\^{o}} ir meinent \label{reinfriedrepeat}\\
the thing so you mean\\
\glt `the thing that you mean'\\(\textit{Reinfried von Braunschweig})
\z

Constructions like (\ref{reinfriedrepeat}) do not represent a problem for any analysis accommodating relative complementisers; I adopt the view of \citet{brandnerbraeuning2013} that \textit{so} is a regular relative complementiser in these cases, and this results in a structure analogous to \figref{treewo}, as shown in \figref{treeso}.

\begin{figure} 
\caption{Relatives clauses containing \textit{so}} \label{treeso}
\begin{forest} baseline, qtree
[DP
	[D
		[d\"er]
	]
	[NP
		[NP [Sache,roof]]
		[CP [DP [D [\textit{Op}.]] [NP [$\emptyset$,roof]]] [C$'$ [C [so]] [TP]]]
	]
]
\end{forest}
\end{figure}

However, the structure in \figref{treeso} cannot straightforwardly accommodate the matrix equative element. In the German examples in (\ref{relos}) and (\ref{mhgalsrepeat}), there is a matrix nominal head following the element \textit{so}, but in the English example in (\ref{kjsuchas}), there is no lexical noun, yet \textit{such} can still take an \textit{as}-clause. This suggests that relative clauses with a matrix equative element are essentially equative clauses and have a syntax different from ordinary relative clauses.

Regarding the structure of equatives, I start from the proposal made by \citet{bacskaiatkari2014diss, bacskaiatkari2018langsci} and \citet{lechner2004} for comparatives expressing inequality. A comparative construction is given in (\ref{ralphcomp}) below:

\ea Ralph is \textbf{more intelligent than} Peter is. \label{ralphcomp}
\z

The proposed structure is shown in \figref{qptree}.

\begin{figure} 
\caption{The QP-analysis for comparatives} \label{qptree}
\begin{forest} baseline, qtree
[QP
	[Q$'$
		[Q
			[-er\textsubscript{i} + much]
		]
		[DegP
			[AP [intelligent,roof]]
			[Deg$'$ [Deg [t\textsubscript{i}]] [CP [than Peter is,roof]]]
		]
	]
]
\end{forest}
\end{figure}

As can be seen, the comparative subclause (headed by \textit{than})\footnote{The comparative subclause contains a degree expression (here: \textit{x-intelligent}, where \textit{x} refers to the degree to which Peter is intelligent). This degree expression is regularly deleted in Standard English if the lexical phrase in the degree expression (here: \textit{intelligent}) is not contrastive. The phenomenon is traditionally referred to as ``Comparative Deletion'' (see \citealt[57--106]{bacskaiatkari2018langsci} for discussion).} is the complement of the matrix degree element -\textit{er}; this is in line with the fact that the degree element imposes selectional restrictions on the subclause (for instance, -\textit{er} can take only a \textit{than}-clause but not an \textit{as}-clause). The gradable adjective (\textit{intelligent}) is in the specifier of the DegP. There is an additional layer, QP, on top of the DegP; this is not immediately relevant for our purposes here.\footnote{The Q head determines whether comparison is to a higher or to a lower degree (\textit{more} versus \textit{less}); the specifier of the QP can host modifiers such as \textit{very} and \textit{far} that show agreement with the degree element (see \citealt[42]{bacskaiatkari2018langsci}).} At any rate, the movement of the degree head to Q results in the correct surface word order (the dummy element \textit{much} is inserted to host -\textit{er}).

Degree comparison may, however, not only express inequality but also equality, as in (\ref{equat}) below:

\ea Ralph is \textbf{as intelligent as} Peter is. \label{equat}
\z

Based on \figref{qptree}, it seems logical to suppose a structure like in \figref{qpequattree} for (\ref{equat}).

\begin{figure}
\caption{The QP-analysis for equatives}
\label{qpequattree}
\begin{forest} baseline, qtree
[QP
	[Q$'$
		[Q
			[as\textsubscript{i}]
		]
		[DegP
			[AP [intelligent,roof]]
			[Deg$'$ [Deg [t\textsubscript{i}]] [CP [as Peter is,roof]]]
		]
	]
]
\end{forest}
\end{figure}

Unlike -\textit{er}, the degree element \textit{as} is not a clitic and no \textit{much} is inserted; otherwise, \figref{qpequattree} is essentially the same as \figref{qptree}, in line with much of the literature going back to \citet{bresnan1973} that treats comparatives and equatives in an analogous way.

However, it is difficult to see how a structure like \figref{qpequattree} could stand for patterns like (\ref{kjpatternsas}), which do contain an \textit{as}-clause taken by a matrix equative element (\textit{such}) but are not associated with degree and do not contain a gradable AP either. In fact, even canonical similative or non-degree equative constructions (cf. \citealt{haspelmathbuchholz1998}) have this property. Consider the example in (\ref{similative}):

\ea \gll Es ist \textbf{so} \textbf{wie} es ist. \label{similative}\\
it is so as it is\\
\glt `It is what it is.'
\z

As discussed by \citet{bacskaiatkari2019fanselow}, the major difference between similatives like (\ref{similative}) and degree equatives like (\ref{equat}) lies in the fact that the gradable AP argument is present in the latter but not in the former construction. Obviously, the label of the relevant projection headed by \textit{so} in these cases could hardly be taken to be a DegP, since no degree is involved, merely comparison. The structure can thus be represented as in \figref{comprptree} (see \citealt[102]{bacskaiatkari2019fanselow}).

In degree equatives, the comparative functional head takes another argument, namely the AP in its specifier. Such Compr heads are equipped with a degree feature {[}deg{]}, which can be checked off by upward movement, creating the Deg position and ultimately projecting DegP (\citealt[103--104]{bacskaiatkari2019fanselow}, building on the Münchhausen-style verb movement to C proposed by \citealt{fanselow2004}). The structure in \figref{qpequattree} is modified to \figref{degpfinaltree}.

\begin{figure}
\caption{The ComprP-analysis} \label{comprptree}
\begin{forest} baseline, qtree, for tree={align=center}
[ComprP
	[Compr$'$
		[Compr
			[so]
		]
		[CP
			[wie es ist,roof]
		]
	]
]
\end{forest}
\end{figure}

\begin{figure}
\caption{The combination of ComprP and DegP} \label{degpfinaltree}
\begin{forest} baseline, qtree, for tree={align=center}
[DegP
	[Deg$'$
		[Deg
			[as\textsubscript{i}]
		]
		[ComprP
			[AP [intelligent,roof]]
			[Compr$'$ [Compr [t\textsubscript{i}]] [CP [as Peter is,roof]]]
		]
	]
]
\end{forest}
\end{figure}

What matters for us here is that elements like \textit{as} and \textit{so} can take also a complement clause argument in non-degree constructions. Observe again the equative relative clause given in (\ref{kjsuchas}), repeated here as (\ref{kjsuchasrepeat}):

\ea And \textbf{such as} do wickedly against the covenant shall he corrupt by flatteries: but the people that do know their God shall be strong, and do exploits.\\(King James Bible; Daniel 1:4) \label{kjsuchasrepeat}
\z

Just like in ordinary non-degree equatives, comparison here identifies a given property rather than relating degrees to one another. The structure can be therefore represented as in \figref{comprptreerel}.

\begin{figure} 
\caption{Equative relatives} \label{comprptreerel}
\begin{forest} baseline, qtree, for tree={align=center}
[ComprP
	[Compr$'$
		[Compr
			[such]
		]
		[CP
			[as do wickedly against the covenant,roof]
		]
	]
]
\end{forest}
\end{figure}

Unlike ordinary relative clauses, which are adjoined to a matrix nominal expression, equative relative clauses occur as complements of a matrix degree-like element. Given the syntactic distinction between the two, it follows naturally that the \textit{as}-relatives attested in the King James Bible have a markedly different distribution from that of ordinary relative clauses. Namely, the equative relative head is not a grammaticalised relative complementiser in Early Modern English but it is contingent upon the matrix equative element. The fact that the possibilities of occurrence of \textit{as}-relatives are restricted anyway, together with the observation that in the given corpus they appear to be restricted to subject relatives (the most unmarked type), means that they essentially cannot compete with ordinary relatives that have a far wider distribution. This also has a consequence for the distribution of \textit{as}-relatives in Present-day English: the standard variety has eradicated this construction completely, while regional dialects only have it to a limited degree.

\section{Summary} \label{sec:4summary}
This chapter was dedicated to the analysis of relative clauses, applying the framework established in \chapref{ch:2} and refined for interrogative clauses in \chapref{ch:3}. It was shown that Germanic relative clauses tend to apply the relative complementiser strategy, in line with the preference for lexicalising a finite C head. Regarding this issue, we find considerable differences between varieties, not only geographically but also related to register. Relative pronouns may not only occur as the sole overt markers of clause type but they may also co-occur with overt relative complementisers, resulting in the doubling effects familiar from embedded interrogatives. However, doubling is apparently less common than in interrogatives; this, together with the overall preference for complementisers, can be attributed to the fact that the relative operator is essentially recoverable. Triple combinations are also attested in some South German varieties; in these cases, a single CP is sufficient under the minimalist assumption that multiple specifiers are allowed. Given the findings so far, it appears that a single CP is appropriate even for cases that are complex on the surface; this raises the question whether the left periphery can be complex (in the sense of containing multiple projections) at all. In \chapref{ch:5}, I will turn to the analysis of embedded degree clauses in this respect, which, despite many similarities with relative clauses, show different behaviour, and \chapref{ch:6} will show how multiple CPs are relevant not only in terms of information structure but also in terms of ellipsis.
