\documentclass[output=paper]{langscibook}
\ChapterDOI{10.5281/zenodo.15148162}
\author{Cormac Anderson\orcid{}\affiliation{Surrey Morphology Group, University of Surrey; Max Planck Institute for Evolutionary Anthropology, Leipzig} and Natalia Kuznetsova\orcid{}\affiliation{Università Cattolica del Sacro Cuore, Milan; Institute for Linguistic Studies, Russian Academy of Sciences} and Shelece Easterday\orcid{}\affiliation{University of Hawaiʻi, Mānoa}}
\title[Rarities in phonetics and phonology]{Rarities in phonetics and phonology: Exploring rarities}
\abstract{This chapter introduces theoretical controversies around studies on rare phonetic and phonological phenomena, and affirms the importance of rarities for the development of phonological theory. We highlight the attempt of our volume to chart a middle ground between two extremes in existing approaches to rarities: radical exoticisation (emphasising the uniqueness of rare patterns to the point of undermining efforts to compare them) and radical normalisation (explaining away rare patterns without accounting for their peculiarity). We favour the de-exoticisation of rare phenomena and attempts to refine existing typologies to better account for them, as this can help to expand and improve phonological theory and typology. The chapter goes on to outline the main themes explored in the volume and to explore how the 17 individual chapters contribute to our better understanding of sound patterns in human language.

\keywords{uncommon sound patterns, phonological typology, language comparison, de-exoticisation, phonological theory}
}
\IfFileExists{../localcommands.tex}{
  \addbibresource{../localbibliography.bib}
  \usepackage{langsci-optional}
\usepackage{langsci-gb4e}
\usepackage{langsci-lgr}

\usepackage{listings}
\lstset{basicstyle=\ttfamily,tabsize=2,breaklines=true}

%added by author
% \usepackage{tipa}
\usepackage{multirow}
\graphicspath{{figures/}}
\usepackage{langsci-branding}

  
\newcommand{\sent}{\enumsentence}
\newcommand{\sents}{\eenumsentence}
\let\citeasnoun\citet

\renewcommand{\lsCoverTitleFont}[1]{\sffamily\addfontfeatures{Scale=MatchUppercase}\fontsize{44pt}{16mm}\selectfont #1}
  
  %% hyphenation points for line breaks
%% Normally, automatic hyphenation in LaTeX is very good
%% If a word is mis-hyphenated, add it to this file
%%
%% add information to TeX file before \begin{document} with:
%% %% hyphenation points for line breaks
%% Normally, automatic hyphenation in LaTeX is very good
%% If a word is mis-hyphenated, add it to this file
%%
%% add information to TeX file before \begin{document} with:
%% %% hyphenation points for line breaks
%% Normally, automatic hyphenation in LaTeX is very good
%% If a word is mis-hyphenated, add it to this file
%%
%% add information to TeX file before \begin{document} with:
%% \include{localhyphenation}
\hyphenation{
affri-ca-te
affri-ca-tes
an-no-tated
com-ple-ments
com-po-si-tio-na-li-ty
non-com-po-si-tio-na-li-ty
Gon-zá-lez
out-side
Ri-chárd
se-man-tics
STREU-SLE
Tie-de-mann
}
\hyphenation{
affri-ca-te
affri-ca-tes
an-no-tated
com-ple-ments
com-po-si-tio-na-li-ty
non-com-po-si-tio-na-li-ty
Gon-zá-lez
out-side
Ri-chárd
se-man-tics
STREU-SLE
Tie-de-mann
}
\hyphenation{
affri-ca-te
affri-ca-tes
an-no-tated
com-ple-ments
com-po-si-tio-na-li-ty
non-com-po-si-tio-na-li-ty
Gon-zá-lez
out-side
Ri-chárd
se-man-tics
STREU-SLE
Tie-de-mann
}
  \togglepaper[1]%%chapternumber
}{}

\begin{document}
\maketitle
%\shorttitlerunninghead{}%%use this for an abridged title in the page headers
% ATTENTION: Diacritics on the following phonetic characters might have been lost during conversion: {'ə'}




\section{Introduction}
\label{sec:Introduction}

The patterns of sounds of the world's languages are many and varied. Although the human articulatory, auditory, and perceptual apparatus limits and shapes the ways in which spoken languages harness sound to convey meaning, considerable diversity can be observed in phonological systems cross-linguistically. The current volume gathers chapters exploring various facets of this diversity, with a focus on its extreme cases.

During the reviewing process for this volume, we learned of the sad death of Bob Blust. This volume is dedicated to his memory, as he was one of the most insightful researchers of phonetic and phonological rarities. Blust was a prolific linguist, authoring nearly 300 publications in his 52-year career \citep{LobelEtAl2022}. He had an enormous impact on the field of linguistics at large through his descriptive and reconstructional work in the Formosan, Philippine, Bornean, and Central and Eastern Malayo-Polynesian subgroups of the Austronesian family. Yet even as he worked through big-picture historical problems in Austronesian, he frequently noted and puzzled over rarities present in individual languages, including rare sources of geminates in Berawan \citep{Blust1995}, the typologically anomalous distribution of nasal and oral vowels in Seimat \citep{Blust1998}, putative vowelless words in Selau \citep{Blust2003}, and velar nasal intrusion in Palauan \citep{Blust2009}. His chapter in this volume on a rare sound change in Berawan appears largely as he submitted it to us.

\begin{sloppypar}
This introductory chapter presents the topic of rarities in phonetics and phonology as we understand it. In \cref{sec:Why-Rarities}, we explain why we think the study of rarities is important and topical, and frame this study within current linguistic research. In \cref{sec:Hist-Concept-Volume}, we give some background to the genesis of this book (\cref{sec:Hist-Volume}) and elaborate on its conceptual foundations (\cref{sec:Concept-Volume}). In \cref{sec:Structure-Content-Volume}, we lay out the general organisation of the volume (\cref{sec:Structure-Volume}) and introduce individual chapters (\cref{sec:Content-Volume}).
\end{sloppypar}

\section{Why rarities?}
\label{sec:Why-Rarities}

Rare phenomena play a key role in forming and challenging linguistic theory. In formal approaches, the identification of rare phenomena has long been one of the primary means of expanding and modifying the theoretical acquis. Within linguistic typology, while much research involves large cross-linguistic samples, a focus on rare phenomena has emerged since the beginning of this millennium, see \textit{Das Grammatische Raritätenkabinett} \citep{Plank2000} and the proceedings of several conferences on the topic (\citealt{WohlgemuthCysouw2010a,WohlgemuthCysouw2010b, SimonWiese2011, GolovkoEtAl2015}).

Two types of rarities have been distinguished in that work: absolute rarities and phenomena that are rare worldwide but frequent in certain genetic groupings (\citealt{WohlgemuthCysouw2010b}: 1–2). Three ways of approaching those rarities which pose challenges for general linguistic theory have been outlined. Rarities can be ignored by linguistic theory, re-analysed as regular, or incorporated by changing the theory (\citealt{SimonWiese2011}: 9–14).

Originally, rarities were seen only as something unnatural with respect to so-called “natural” language laws. However, with the rise of research on the evolutionary aspects of both universals and rarities, a trend towards “de-exoticising” rarities by expanding existing theories has been observed. Importantly, studies in this vein start from the premise that rarities are valid patterns which theoretical frameworks must grow to accommodate. As rarities are comprehensively described and studied, the limitations and weaknesses of current synchronic and diachronic models of language structure and change are revealed. Confronted with a growing expectation that they should be able to account for patterns on the extremes, theories are challenged to widen their scope and explanatory power. This approach has resulted in progressively more linguistic, cognitive, biological, cultural, and other factors being acknowledged as critical in the emergence and functioning of linguistic rarities.

Within phonology, the number of thematic works taking the approach described above has been constantly growing. Older examples include studies on nasalised laryngeals \citep{Blust1998}, non-modal vowels \citep{Gordon1998}, and inhibited sound change (\citealt{BlevinsWedel2009}). More recent studies of the same kind include the analysis of vertical vowel systems \citep{Anderson2016}, voiceless sonorants \citep{Blevins2018} and voiced word-final obstruents (\citealt{EgurtzegiUllrich2020}), uncanonical word prosody \citep{Kuznetsova2018}, metathesis and “un-metathesis” \citep{Edwards2020}, highly complex syllable structure \citep{Easterday2019}, ternary stress \citep{Golston2021}, and ternary quantity \citep{Kuznetsova_forth}.

This work appears in the context of an expansion of research into the evolutionary aspects of both universals and rarities, with attempts to explain typical pathways of the emergence, maintenance, and disappearance of linguistic phenomena (\citealt{Blevins2004}: 192–214; \citealt{SimonWiese2011}: 15–20; \citealt{Round2019}). This research follows the foundational work of Joseph Greenberg, who developed the systematic typological study of evolutionary aspects of linguistic phenomena. In outlining a probabilistic approach to diachronic change, \citet{Greenberg1978} made an important distinction between two dimensions along which linguistic features can be classified: (a) frequency of emergence and (b) stability. The synchronic rarity of a pattern does not necessarily imply that it arises infrequently and is unstable once it arises. Just as patterns which arise frequently may be diachronically unstable (e.g. nasal vowels), patterns which arise rarely may be diachronically rather stable (e.g. clicks and ejectives). In the evolutionary sense, we can consider the latter patterns to be different in important ways from those which arise rarely and are unstable (e.g. velar implosives). Greenberg's approach shows that properties of both emergence and stability must be considered in order to fully understand the diachrony of rarities.

This “evolutionary” trend, which has appeared within functional and typological research on phonology, is currently at odds with formal phonological analyses based on the concept of markedness. Certain phenomena, like word-final obstruent voicing as discussed by Schwartz and Ulfsbjorninn (Chapter 6), are considered rare but possible in Evolutionary Phonology, where the possibility for sound change is limited only by physiological biases related to language acquisition and use (\citealt{Blevins2006a,Blevins2006b,EgurtzegiUllrich2020}). On the contrary, such phenomena are seen as impossible in the “amphichronic” Universal Grammar, which considers language change, and resulting synchronic distributions, to be consequences of markedness constraints (\citealt{Kiparsky2006,Kiparsky2008}). The constant search for and challenging of such rare pivotal cases constitutes an important line of debate in the field of phonological rarities.

The trend of de-exoticising rarities is directly linked to the growing degree of description, documentation, and inclusion of various world languages in scientific research. It has been noted that rarities, including those in phonetics and phonology, are often concentrated in lesser-studied languages (\citealt{Mithun2007, WhalenMcDonough2019, TuckerWright2020, ZellouEtAl2022}). Suggested reasons for this include a sociolinguistic distance observed between those varieties and “big” languages (\citealt{MansfieldStanford2017}), differences in input on performance, which affect language users’ knowledge base, processing manner, and expectations \citep{Lev-Ari2016}, and an especially strong motivation on the part of speakers of smaller and under-resourced languages to be different (\citealt{BirdKell2017}). Statistically, fewer contexts for language use might also increase the probability for preserving the quirks of random language drift (\citealt{Bird2008, JägerForthcoming}). The prevalence and acceleration of language obsolescence itself might also be a factor behind the concentration of rare features in endangered languages. For example, some rarities happen to be documented in the middle of an unfinished sound change. These stages may exist for just a brief period in the life of a big vital language. However, if a vanishing variety is documented at one of these stages, the rarity will enter into the cross-linguistic typology and remain forever unchanged in that state (\citealt{Kuznetsova2021}: 164; \citealt{KuznetsovaMarkus2022}).

Some important aspects of rarities may only come to light in the context of comprehensive global surveys which include many under-represented languages. A phenomenon which appears rare as a firmly established primary contrastive phonological feature might nevertheless be quite common as a purely phonetic phenomenon or process, or as a secondary cue relevant only in certain contexts. However, we can only arrive at this conclusion after thorough cross-linguistic investigation. A good example is preaspiration, discussed in detail in this volume (see especially Iosad’s and Hejná’s contributions). Such cross-linguistic findings raise the key question of why certain relatively frequent phonetic phenomena phonologise so rarely.

On the other hand, certain features might also seem rare and “unnatural” only from the perspective of the best-studied languages, while in fact they are more frequent than commonly thought. The initial database of the World Atlas of Language Structures (WALS; \citealt{DryerEtAl2005}), for example, showed that at least part of Europe (more specifically, English, German, Dutch, Frisian, and French) is actually one of the geographical areas with a “high level of rarity” (\citealt{Cysouw2011}: 424; see also \citealt{Dahl1990}). Therefore, it is probable that many features, including phonetic and phonological ones, which seem exotic from the perspective of Standard Average European (\citealt{Whorf1941}: 78; \citealt{Haspelmath2001, Auwera2011}) or LOL (Literate, Official, Lots of users; \citealt{Dahl2015}) languages can be actually common worldwide. A parallel observation has been also made in psychology (\citealt{HeineNorenzayan2010, Henrich2021}). Experimental subjects coming from WEIRD (Western, Educated, Industrialised, Rich, Democratic) countries showed many unusual psychological and behavioural features, as compared to the rest of human species. The problem here, as Henrich argues, is that such subjects constitute 96\% of all experimental participants in psychology, so these often quite unusual results are typically extrapolated and considered universal for all human beings.

In sum, a greater inclusion and finer investigation of lesser-studied languages and cultures will likely progressively challenge our understanding of what is rare or typical, possible or impossible. 

A further factor contributing to the de-exoticisisation of rarities is a spread of parametric methods in typology and a shift from languages as the minimal units of analysis to the features of languages, and further to the parameters of these features. Multidimensional approaches include Canonical Typology (\citealt{Corbett2007, Hyman2014, RoundCorbett2020}) and Multivariate (or Distributional) Typology (\citealt{Bickel2015, Tallman2020}), cf. \citet{Forker2016} for a comparison of the two. In these approaches, marginal cases are interpreted as possible but rare combinations of universal parameters rather than as logically impossible cases. The reasons for the rarity of certain combinations, however, still have to be explained and accounted for within those frameworks.

\largerpage
The general approach we take in the present book to two other common treatments of rarities – radical exoticisation and radical normalisation – is returned to in \cref{sec:Concept-Volume}.

\section{History and conceptual basis of this volume}
\label{sec:Hist-Concept-Volume}
\subsection{History of the volume}
\label{sec:Hist-Volume}

After founding the series \textit{Topics in Phonological Diversity} at Language Science Press in 2021 (\url{https://langsci-press.org/catalog/series/tpd}), we conceived of a book on phonological rarities as one way of launching the series. The three of us had been conducting typologically oriented research of various rarities in phonetics and phonology for some time, including in our doctoral studies (see references in \cref{sec:Why-Rarities}), and shared an interest in the topic. To gauge whether there would be sufficient interest from the community to fill a volume on rarities in phonetics and phonology, we organised a thematic session at the Poznań Linguistic Meeting in 2021 (\url{http://wa.amu.edu.pl/plm_old/2020/PLM2021_Programme}).

The session invited submissions for papers including individual studies or thematic surveys of specific sounds, features, systems, structures, or phenomena. We declared the workshop open to a wide range of topics and frameworks addressing phonological and phonetic rarities, unusual phenomena in phonological typology, explanatory factors for rare phenomena (structural, acoustic, articulatory, perceptual, cognitive, statistical, sociolinguistic, etc.), the evolution and the social aspects of rarities.

The call attracted considerable interest, and fourteen papers were accepted to the thematic session, six of which are represented in this book. Following on from this success, we made an open call for chapter proposals, to which we received over thirty-five responses. In the end, seventeen chapters appear in this volume.

In reviewing proposals for both the workshop and the book, we considered contributions focusing on synchronic rarities in individual languages, especially lesser-studied ones, and on studies with a broader typological or evolutionary perspective. Over the course of this process, we formulated our general approach to the study of linguistic rarities, outlined in the next section.

\subsection{Conceptual basis of the volume}
\label{sec:Concept-Volume}
When considering submissions to this volume, we identified two major pitfalls that linguists often fall into when studying languages which seem to differ from the cross-linguistic norm (in addition to just ignoring such phenomena).

On the one hand, there is the danger of (radical) \textsc{normalisation}, whereby the researcher attempts to explain away an apparently unusual linguistic phenomenon by forcing it to fit into an existing framework, often through some creative reinterpretation of the pattern. This is usually carried out in defence of the theoretical acquis, to protect the universality of an existing generalisation and dependent theories or typologies.

On the other hand, there is the peril of (radical) \textsc{exoticisation}, whereby the researcher insists on the peculiarities of a given phenomenon in a particular language, to the point of underplaying or even denying the extent to which it can be validly compared to similar phenomena in other languages. Very often, this is done in the defence of linguistic diversity, to protect the unique status of a given language variety in the linguistic (and sometimes broader) discourse. In other cases, a researcher resorts to radical exoticisation simply because current theoretical devices and conventional definitions, especially when narrowly formulated and interpreted, leave little to no room for the pattern observed. 

In accepting manuscripts for this volume, we preferred studies that attempted to chart a path between these extremes, examining ways in which rare phenomena are both similar to and different from those found in more familiar languages. This can be done, for example, by decomposing rare phenomena and the paths of their emergence into many single parameters, as in Canonical or Multivariate Typology, or in Evolutionary Phonology (cf. \cref{sec:Why-Rarities}), by observing at which exact point the rarity occurs.

What many such approaches have in common is that they do not attempt to remove the rarity, but try to move its locus. This may be, for  example, a shift from phonological units to their organisation (e.g. Brandão De \citealt{Carvalho2006,DresherEtAl2018}, \citetv{chapters/07_Wolff}) or from one level of description to another (\citealt{Kiparsky2018}; see also \citetv{chapters/07_Wolff}). Some contributions to this volume illustrate this shift of locus also with the relationship of phonetics and phonology. Sometimes a rarity turns out to be an unusual phonetic manifestation: word-final devoicing in Lakota \parencitetv{chapters/06_SchwartzUlfsbjorninn}, a word prosody contrast manifested by a laryngeal feature \parencitetv{chapters/08_DoBui}, fricative and uvularised vowels (\citetv{chapters/10_HugteEtAl} and \citetv{chapters/11_Guan} respectively). An extreme example of uncommon phonetic realisation is presented in silently realised sonorants \parencitetv{chapters/12_WatsonEtAl}. This is quite a unique phenomenon in that there is an articulatory gesture, but no or little acoustic output. The gestures, observed through instrumentation, are different for each type of sonorant, and they are apparently transmitted between generations. Such phenomena seem especially challenging for most phonetic and phonological theories and might open exciting new paths for theoretical advancement in various fields.

However, rarity can also be due to an uncommon type of phonologisation (such as preaspiration addressed in Part~\ref{part:VI}, or the contrastive palatalisation of rhotics in \citetv{chapters/14_KavitskayaWandl}), or because of the unusual place of a segment in a phonemic system (ejective fricatives as analysed by in \citetv{chapters/15_Puderbaugh}), or due to an atypical dialectal variability (consonantal epenthesis in \citetv{chapters/05_CulhaneEdwards}). At times, the parameters may appear fairly regular in themselves, but their combination, for certain reasons, is rare, as the evolutionary approach, represented in the chapter by Blevins, claims for a series of Austronesian sound changes. In a similar vein, palatalised rhotics \parencitetv{chapters/14_KavitskayaWandl} and word-initial voiceless sonorant geminates \parencitetv{chapters/13_ShinhoaraEtAl} represent atypical combinations of relatively common synchronic features. The phenomenon of tone-driven vowel epenthesis \parencitetv{chapters/09_Rolle} may also be included here.

It is also entirely valid to re-evaluate the status of a rarity when new data come to light or by bringing a wider range of factors to bear on analysis of the phenomenon at hand (as in chapters by \citetv{chapters/02_Iosad}, \citetv{chapters/15_Puderbaugh}, and \citetv{chapters/06_SchwartzUlfsbjorninn}). We accept the usefulness, even the necessity, of doing this, but by the same token also wish to learn more about what exactly is genuinely rare in phonetics and phonology, not just about what is not.

At the other extreme from normalisation, as mentioned above, is radical exoticisation, where the researcher focuses on the distinctiveness of a given phenomenon in a particular language to a degree that does not admit placement in broader typologies or dialogue with general phonological theories. To varying degrees, we support the maxim that languages should be described in their own terms, but we are also interested in the wider picture afforded by cross-linguistic study. When assessing submissions, we repeatedly insisted that authors make an effort to firmly establish the rarity of the phenomenon at hand, but also to actively seek out and discuss comparanda in other languages.

Such exoticisation is perhaps most frequently encountered when it comes to rare segments. Several proposals to the volume described exquisite sound patterns, often but not always in understudied languages. We accepted such proposals only when they made an effort to contextualise these segments in terms of what other attested sound patterns might serve as comparanda and what we know about these types of sound in general. We considered the language-specific description of something that is (supposedly) rare to be insufficient. The rarity had to be viewed through both a cross-linguistic and a general theoretical lens.

We also evaluated with some scepticism cases in which researchers claimed a clear constellation of different phenomena to constitute a rarity. The problem here is that it can be statistically unlikely for a number of phenomena that are not in themselves particularly uncommon to co-occur in the same language. For example, a language with uvulars, fixed initial stress, lateral obstruents, and \mbox{/θ/} could be considered to have a pretty interesting phonology. Given that each of these phenomena is more or less uncommon, their statistical cooccurrence in the same language is unlikely. However, this is an expected rarity and a rather trivial one: after all, every language variety, taken holistically, is not just rare, but unique. In such cases, we are interested to look only at individual ingredients, not the whole stew.

We are aware, however, that, on the one hand, different frameworks disagree as to what is and what is not a phonological primitive and, on the other, that it may anyway be difficult to identify what individual ingredients actually are. For example, many phonologists would argue that phonemes are not single units, but rather combinations of distinctive features, which are themselves the more fundamental building blocks  (viz. \citealt{VeerEtAl2023}). Even in this view, palatalised rhotics or word-initial voiceless sonorant geminates still require explanation in terms of why certain feature combinations are more uncommon than others. 

In addition to the theoretical differences over the base unit of analysis for phonology, we also find analytical disagreements over the simple or complex nature of given phenomena (cf. “criterial conflicts” in \citealt{Round2023}). In some cases, these disagreements are core to the controversies over the status of certain rarities, such as the pre-aspirated or ejective consonants, or word-initial geminates discussed in this volume. 

Similarly, whether or not some sound changes are actually rare depends on whether they are seen as occurring in a single step (as \citetv{chapters/03_Blust} proposes) or consisting of an infrequent combination of multiple more common changes (as suggested in \citetv{chapters/04_Blevins} and by \citetv{chapters/05_CulhaneEdwards}) and/or acting upon rare initial conditions. We agree with Blust when he argues that it is unprincipled to assume that sound changes have been telescoped when there is no other evidence for any intermediate stages. It will nearly always be possible, with sufficient ingenuity, to come up with a plausible succession of sound changes that yield a certain result. On the other hand, we cannot accept a given sound correspondence as evidence of a rare sound change without examining whether evidence of a sequence of less rare sound changes may have occurred instead.

In spite of different viewpoints on the fundamental building blocks in phonology or the nature of rare sound changes, there is also likely to be considerable convergence over what is \textit{trivially} rare more generally. In particular, we expect that all current phonological frameworks would exclude from consideration a whole language as a rarity.

In sum, while we are sympathetic to the impulses underlying the pitfalls of radical normalisation and radical exoticisation, namely, to defend generalisations valid across human languages and to affirm the individuality of each language variety, we were most inclined to accept proposals that struck a middle ground. We favoured chapters that firmly establish the empirical status of a given rarity and describe it in a way that facilitates comparison, the elaboration of a typology, and theory development. 

We also encouraged authors to be explicit about the terminology and the theoretical implications they relied on, especially in cases where those might not be familiar to all readers (e.g. in some less common formal phonological accounts) or rather technical (e.g. in phonetic experiments). In such ways, we sought to make our volume accessible to as broad an audience as possible. In the same vein, we also accepted chapters which provided comprehensive up-to-date overviews of individual phenomena (rather than brand new theoretical or experimental breakthroughs). The hope here was to raise general interest towards those phenomena and to establish a new baseline for future research.

We actively solicited cross-referencing between chapters, because many of the analytical issues regarding rarities and even of the phenomena chosen for analysis by individual authors manifested considerable overlap. The particular analytical angles, however, were always slightly different, which enabled us to provide multi-faceted coverage of certain phenomena (e.g. of preaspiration or of some Austronesian sound changes).

An outline of the chapters that met the criteria discussed above appears in the next section. Almost all rarities discussed in the book are phonological rather than phonetic, with the exception of silent sonorants in \citetv{chapters/12_WatsonEtAl}. All of them concern segments and word-level prosody, while post-lexical prosody is not discussed. This is unlikely to be coincidental. Reviewing some of the initial proposals concerning intonation, we recognised that it is still extremely difficult and perhaps premature to firmly establish any rare types in this highly variable and yet much understudied field.

\section{Structure and content of the volume}
\label{sec:Structure-Content-Volume}
\subsection{Structure of the volume}
\label{sec:Structure-Volume}
The seventeen chapters of the volume cover diverse rare phonetic and phonological phenomena from a wide variety of geographical areas. The contributions can be roughly divided into the following main types: 

\begin{itemize}
\sloppy
\item[(1)]
    \label{ex:anderson:1}
         broad cross-linguistic analyses of certain phenomena (most notably: preaspiration and fricative vowels);

\item[(2)]
    \label{ex:anderson:2}
         descriptive case-studies, often field-based, providing a state-of-the-art comprehensive overview of particular rare phenomena (Otomanguean ballisticity, Queyu uvularised vowels, Ghomala’ tone-driven vowel epenthesis, Ecuadorian Siona preaspiration);

\item[(3)]
    \label{ex:anderson:3}
         comprehensive summary chapters primarily devoted to the diachrony of rare sound changes (the rise of Chadic prosodies, Meto epenthetic consonants, Slavic rhotics);

\item[(4)]
    \label{ex:anderson:4}
         critical re-analysis of certain rarities (diachronic and synchronic) as more regular phenomena, or a rebuttal thereof (Austronesian and Lakota sound changes, Upper Necaxa Totonac ejective fricatives, partially also preaspiration in Chapter 17);

\item[(5)]
    \label{ex:anderson:5}
         experimental phonetic studies (of Mehri and Shehret silent sonorants, Ikema Miyako Ryukyuan voiceless nasal geminates, partially also of Upper Necaxa Totonac ejective fricatives).
\end{itemize}

We divided the volume into a number of thematic parts based on the main types of phonetic and phonological rarities considered in each of them. Brief descriptions of the contents of each part follow below.

Part~\ref{part:I} of the volume includes, apart from this introductory chapter, a chapter by Iosad dedicated to the analytical status of phonological rarities, with a discussion of some examples. It raises many of the same concerns as those outlined above in \cref{sec:Why-Rarities} and \cref{sec:Concept-Volume} about what a phonological rarity is and how it should be analysed typologically.

The remainder of the volume is organised along the general types of phonological rarities. Part~\ref{part:II}, opening with a chapter by Bob Blust, provides analyses of rare sound changes. The chapters by Blust, Blevins, and Culhane and Edwards are especially tightly interconnected, as they discuss partially overlapping rare sound changes in Austronesian languages. The chapter by Schwartz and Ulfsbjorninn re-assesses a supposedly rare word-final obstruent voicing in a Siouan language of North America and re-analyses it as a manifestation of a common type of lenition.

Part~\ref{part:III} includes rare types of word prosody and prosodic processes: word-level “prosodies” in Chadic languages \parencitetv{chapters/07_Wolff}, Otomanguean stress ballisticity \parencitetv{chapters/08_DoBui}, and tone-induced vowel epenthesis in a Grassfields language of Cameroon \parencitetv{chapters/09_Rolle}. In all these cases, phenomena are presented as rare in themselves, but a broader comparative context and some potential alternative analyses are also provided. Wolff’s chapter discusses at length the historical evolution of prosodies and minimal vowel systems in Chadic with a keen eye to typological comparanda elsewhere.

In Part~\ref{part:IV}, two rare types of vowels are described. More specifically, both chapters address vowels with properties that are phonetically more consonantal than what would be expected of a typical vowel. The chapter by \textcitetv{chapters/10_HugteEtAl} analyses fricative vowels in Sinitic and other languages. \textcitetv{chapters/11_Guan} discusses uvularised vowels in languages of the Qiangic branch of Tibeto-Burman. Here too, the rarity of the phenomena themselves is accepted, but they are placed in a broad cross-linguistic and historical context, and an array of different interpretations of the phenomena in question is presented.

Part~\ref{part:V} is dedicated to rare types of consonants. Three of the four chapters discuss rare sonorants. Two of those provide detailed experimental phonetic data (articulatory and acoustic) on “non-sonorous” sonorants: silent sonorants in two Modern South Arabian languages \parencitetv{chapters/12_WatsonEtAl} and voiceless geminate sonorants in the Ryukyuan Japonic varieties \parencitetv{chapters/13_ShinhoaraEtAl}. The historically oriented chapter by  \textcitetv{chapters/14_KavitskayaWandl} explores reasons for the preservation of a rare contrast of plain and palatalised rhotics in Slavic languages. These three chapters dedicated to sonorants do not challenge the rare status of the discussed phenomena. The chapter by \textcitetv{chapters/15_Puderbaugh}, on the other hand, which explores rather rare ejective fricatives in a Totonacan language of Mexico, re-analyses these as consonant clusters.

The concluding Part~\ref{part:VI} presents an array of different analyses of the rare consonantal feature of preaspiration. The chapters by \textcitetv{chapters/16_Hejná} and \textcitetv{chapters/17_Craioveanu} provide typological analyses of the phenomenon, but from divergent perspectives (see in this regard also Iosad’s Chapter 2 in Part~\ref{part:I}). Both tend to de-exoticise preaspiration, but in different ways. Hejná argues that even phonological preaspiration is more common than typically thought, while phonetic preaspiration is actually frequent. Craioveanu, in turn, reanalyses preaspiration rather as a consonant cluster, using its separate syllabification from the following consonant as the main criterion. Finally, the chapter by \textcitetv{chapters/18_vantveer} discusses the phonological status of preaspiration in a Tukanoan language of Ecuador. The sum of evidence leads the authors to maintain the status of preaspiration as a consonantal property in this particular case.

The following section provides more in-depth overviews of findings across individual chapters.

\subsection{Outlines of individual chapters}
\label{sec:Content-Volume}
\textsc{Chapter 2} by Pavel Iosad (University of Edinburgh) “Why the search for rarities must take phonology seriously” takes issue with the general notion of a phonological rarity as a unit of cross-linguistic comparison. Iosad calls for an explicit definition of a “phonological phenomenon” and an explicit threshold for “rarity”, in order to ensure the commensurability of compared phenomena. The first plea is illustrated with a case study of final devoicing, where similar surface phonetic effects might reflect very different phonological systems, some of which can be rare. The second plea is discussed as applied to preaspiration, which is phonetically frequent but phonologically much less so. The definition of a “phonological phenomenon” also requires a necessary engagement of phonological typology with theoretically informed phonology. Iosad, much like \citet{HymanPlank2018}, notes an uneasy relationship between these two fields and explicates the need for their mutually fruitful collaboration. This is also in line with a call by \citet[97]{Kiparsky2018} that “the search for better linguistic descriptions, more illuminating typologies, and stronger cross-linguistic generalisations and universals should go hand in hand”. Following the typological advances in areas other than phonology (cf. \cref{sec:Why-Rarities}), Iosad also proposes to explicitly include historical and areal dimensions into both phonological typology and phonological theory, including those concerning rarities.

\textsc{Chapter 3} “\textit{*b} > \textit{-k}{}-: A Berawan sound change for the ages” by the late Robert Blust (University of Hawaiʻi, Mānoa) is a scrupulous analysis of all the attempts by previous authors to normalise this idiosyncratic sound change, i.e. to re-ana\-lyse it as one or several more “natural” (phonetically or phonologically motivated) changes. The conclusion is that each of these analyses is problematic, especially because they are too speculative and do not adhere enough to the principle of Occam’s razor (simplicity of explanation). Blust also tentatively suggests that such a bizarre sound change could have even been introduced by speakers consciously, for certain sociolinguistic reasons (cf. also \citetv{chapters/05_CulhaneEdwards} below).

Juliette Blevins (The Graduate Center, CUNY) in \textsc{Chapter 4} “Linguistically motivated sound change: Revisiting some of the world’s rarest wonders” takes an opposing stance regarding the aforementioned Berawan change and nine other rare Austronesian sound changes analysed by Bob Blust throughout his career. She sees all of them as ultimately phonetically motivated, i.e. as “natural”. In particular, Blevins interprets the Berawan \textit{*b} > \textit{-k}{}- change as resulting from several regular processes of lenition and fortition (more specifically, glide strengthening). She nevertheless acknowledges the importance of Blust’s plea for the maximal simplicity of analysis (Occam’s razor). Additionally, in line with the general spirit of this volume, her analysis is multi-parametric, as it is informed by research in phonetics, phonological typology, historical linguistics, and contact phonology from the past several decades – and Blevins calls for more such analyses.

\begin{sloppypar}
The processes of glide insertion and fortition in Austronesian are discussed also in \textsc{Chapter 5} “Consonant epenthesis in Meto: Typologically rare but diachronically explicable” by Kirsten Culhane (University of Canterbury) and Owen Edwards (University of Cologne; Language and Culture Unit (UBB), Kupang). In line with Blevins, they see rare types of epenthetic consonants in Meto, an Austronesian language of Western Timor, through the lens of these phonetically “natural” emergence processes. Like Blust, the authors also propose taking potential sociolinguistic factors into account when analysing consonant epenthesis in Meto. Different epenthesis patterns seem to serve as group identity exemplifications cited by Meto speakers themselves in support of their inter-dialectal differences. This desire to be different (cf. \citealt{BirdKell2017} cited in \cref{sec:Why-Rarities}) might be an additional factor in the phonologisation and entrenchment of diverse rare types of epenthetic consonants across Meto varieties. The chapter contains a supplement with a small comparative database abundantly illustrating various types of epenthesis in Meto.
\end{sloppypar}

\textsc{Chapter 6} “Reconciling the debate about final obstruent voicing: The phonology of Lakota obstruent lenition” by Geoffrey Schwartz (Adam Mickiewicz University in Poznań) and Shanti Ulfsbjorninn (Memorial University of Newfoundland) takes an amphichronic stance, opposing the evolutionary one, in the conceptual debate on word-final obstruent voicing (see \cref{sec:Why-Rarities}). This subject is at the very core of the principal disagreement between the two equally influential theoretical paradigms. Such a process is considered impossible in the amphichronic programme. The authors, therefore, re-analyse the newest case of this sort attested by \citet{EgurtzegiUllrich2020} in Lakota (Siouan) as consonant lenition, which is a cross-linguistically common process. In this case, lenition is claimed to be manifested as sonorisation of stops and devoicing of fricatives. Importantly, word-final obstruent voicing is considered within a broader picture of the Lakota phonological system.

H. Ekkehard Wolff (Leipzig University) opens the word-prosodic part of the book with his foundational \textsc{Chapter 7} “Typology and evolution of minimal vowel systems in Central Chadic (Afroasiatic)”. The author has been studying the topic for several decades, and the chapter is a comprehensive summary of this life-long work. Proto-Central Chadic is reconstructed with a rare minimal system of just two vowels *a and *ə, where the latter is actually a non-phonemic intrusive vowel. Modern Chadic languages, however, have extremely variable vowel systems, presented in a range from just one vowel /a/ to 17 different vowels. Wolff describes the development between these two stages as the de-segmentalisation and prosodification of the consonantal features of palatalisation, labialisation, nasalisation, and laryngealisation. These features become associated with the whole stem and “colour” not only all stem vowels, but also many stem consonants. Later, this “colouring” phonologises in the stem’s vowels and consonants in different ways and to different degrees across Central Chadic varieties.

In \textsc{Chapter 8} “A model of non-modal phonation: Ballisticity in Otomanguean languages”, Bien Dobui (Université de Picardie Jules Verne) discusses a non-ca\-non\-i\-cal phonetic cue engaged in a word-prosodic (syllable level) contrast. This contrast is observed in some Otomanguean languages and is commonly referred to as “ballistic stress” vs. “controlled stress”. The feature is analysed in the chapter as a contrast of spread vs. constricted glottis. It is independent of contrasts in tone, nasalisation, and vowel length. An application of a multi-dimensional approach to word prosody proposed by \citet{Hyman2006,Hyman2014} within Canonical Typology leads the author to conclude that “ballisticity” does not correspond either to canonical stress or to canonical tone, while it is still a lexicalised prosodic feature. This conclusion is very similar to that made for other rare laryngeal-based or quantity-based features in Danish, Estonian, and Udihe by \citet{Kuznetsova2018, kuznetsova2022}. Such features challenge the established dichotomy of stress and tone as the only word-prosodic types, because they are “neither stress, nor tone” but still highly word-prosodic.

\textsc{Chapter 9} by Nicholas Rolle (Princeton University; Leibniz-ZAS) “A tonological rarity: Tone-driven epenthesis in Ghomala’” stands out with respect to the other prosodic chapters, because it considers a rare prosodic process rather than a phenomenon. What is rare in this case, is a motivation for final vowel epenthesis. According to the author, epenthesis can usually be accounted for by segmental or syllabic well-formedness. However, in Ghomala’, a Bamileke Grassfields language, the motivation is rather to avoid a contour tone on a single vowel. Unlike most other chapters, this chapter also discusses post-lexical prosody in its typological overview. The author draws parallels between tone-driven and intonation-driven vowel epenthesis. The latter is likewise rare but still appears to be significantly more frequent than the former.

Thom van Hugte (Leipzig University), Yiya Chen (Leiden University), and Li Guo (Shanghai International Studies University) open the vocalic part of the book with their \textsc{Chapter 10} “The contradictory nature of fricative vowels in Chinese and beyond”. This is a comprehensive typological overview of fricative vowels. These segments are at a crossroad between vowels (according to their phonological functions and history) and consonants (according to their phonetics). A multi-parametric approach, which includes the analysis of many phonetic and phonological dimensions, synchronic and diachronic factors, and the cross-linguistic diversity observed for these segments, allows one to see a fuller and much more nuanced picture than just a simple binary labelling of a segment as a vowel or consonant.

\textsc{Chapter 11} “Uvularization in Queyu phonology” by Xuan Guan (University of Oregon) is dedicated to another type of less vowel-like vowels. This is more of a case study, because uvularised vowels are discussed for several varieties of the Qiangic language Queyu (Tibeto-Burman). The approach is also multi-dimensional: Queyu uvularised vowels are described from the phonetic (articulatory and acoustic), phonological, morphophonological (vowel harmony), and historical points of view. They are placed within other similar cross-linguistic phenomena (uvularised, velarised, and pharyngealised vowels). The origin of uvularised vowels appears to be quite different from that of fricative vowels \parencitetv{chapters/10_HugteEtAl}. The latter are the result either of an assimilation by manner of articulation with preceding consonants, or of vowel raising. Uvularised vowels rather represent a merger of a vowel and a preceding or following velar or uvular consonant. Some fricative vowels outside Sinitic (in Lakes Plain, Papuan), however, might also originate from a vowel and a following obstruent, as argued by \textcitetv{chapters/10_HugteEtAl}.

The consonantal part of the book opens with \textsc{Chapter 12} “Silent sonorant articulations in Mehri and Shehret” by Janet C. E. Watson (Sultan Qaboos University, Muscat; University of St. Andrews), Barry Heselwood, Gisela Tomé Lourido (University of Leeds), Amer al-Kathiri (University of Technology and Applied Sciences, Salalah), and Abdullah al-Mahri (Dhofar). This chapter is outstanding in several respects. It discusses word-final sonorant realisations with no or very little acoustic output. The authors argue that such sonorants nevertheless contrast in manner of articulation and laryngeal specifications (plain vs. breathy). The analysis is backed up with the experimental results of several articulatory studies. An ELG study shows that phonetic realisations of preceding vowels contain some acoustic cues at least to the laryngeal contrast of silent sonorants. The authors discuss such silent realisations within the typology of consonantal lenition, arguing that these pronunciations constitute an unusual case. They have neither become more sonorous, nor has their articulation shown signs of decrease in the degree of articulatory constriction, as observed in a EPG study. This study raises important theoretical questions on the precise mechanisms of perception by speakers, transmission, preservation, and overall stability of such contrasts, as well as it contributes to the typology of lenition.

\textsc{Chapter 13} “Aerodynamic and acoustic correlates of word-initial voiceless nasal geminates of Ikema Miyako Ryukyuan” by Shigeko Shinohara (Laboratoire de Phonétique et Phonologie, CNRS/Sorbonne Nouvelle), Qandeel Hussain (Universität Bamberg; University of Toronto; North Carolina State University), and Angélique Amelot (Laboratoire de Phonétique et Phonologie, CNRS/Sorbonne Nouvelle) is the other chapter in the volume which presents new experimental phonetic data. Experiments studied duration, voice quality features, and oral and nasal airflow for rare word-initial voiceless geminate sonorants in the Ikema dialect of Miyako Ryukyuan (Japonic). These sounds do not contrast with singleton voiceless sonorants in the system. They originate from combinations of high vowels (plain or fricative) and preceding obstruents or obstruent and fricative clusters (cf. Chapter 10). In addition to phonetic data, the authors provide phonological, historical, and typological analyses of voiceless geminate obstruents. A question can be raised here whether these sounds are best analysed as single geminate phonemes (as the authors propose) or as sequences of voiceless and voiced sounds. Only their initial portion ({\textasciitilde}1/3) is voiceless, and their overall duration is significantly longer than that of voiced sonorant geminates.

\textsc{Chapter 14} “Preservation and loss of a rare contrast: Palatalization of rhotics in Slavic” by Darya Kavitskaya (UC Berkeley) and Florian Wandl (University of Zurich) is quite different from the two previous chapters in Part~\ref{part:V}. It discusses the history of a rare contrast between plain and palatalised rhotics observed in several Slavic languages. The contrast was lost or transformed into a different contrast in some other Slavic varieties. The authors explicate potentially conflicting articulatory or aerodynamic demands on palatalisation and rhotic trills as a possible underlying source for the rarity of the emergence of this phonological contrast. They argue then that this contrast has been preserved in those Slavic languages where its functional load was especially high, including a very broad range of its phonological contexts of occurrence and great extent of contrastive palatalisation in the system in general.

Rebekka Puderbaugh (University of Edinburgh) in \textsc{Chapter 15} “Ejective fricatives in Upper Necaxa Totonac: Complex segments or consonant clusters?” discusses the synchronic phonological status of these units on the basis of several different parameters. The very path of their emergence from fricative plus stop clusters through a series of sound changes was quite unremarkable. What is rare is that those processes did not apply to other sounds, so only fricatives have ejective counterparts in the synchronic system. A multi-parametric analysis based on the distributional properties of ejective fricatives in comparison with other Upper Necaxa Totonac consonants, and on their phonetic properties leads the author to conclude that it is still best to treat these rare sounds as clusters rather than as single complex segments.

A comprehensive \textsc{Chapter 16} “On the rarity of pre-aspirated consonants” by Michaela Hejná (Aarhus University) opens the last part of the book. The text is accompanied by an extensive supplement, listing all languages for which phonetic or phonological preaspiration or similar phenomena have been claimed, with references. This chapter discusses the questions of how to define and quantify preaspiration and whether it is actually rare. In this, it echoes Chapter 2 by Iosad, which considers the analytical challenges of cross-linguistic comparison of rarities in general. Hejná aims at a maximally broad cross-linguistic survey of preaspiration based on its very lax definition (including also pre-affrication, pre-spirantisation, sonorant and vowel devoicing, /s/ debuccalisation, /hC/ clusters, vowel aspiration, and glottalisation). The result is that phonological preaspiration taken very broadly is still rare, although phonetically it is a much more frequent phenomenon. In many cases, it serves as a secondary phonetic cue for fortis/lenis and similar contrasts. A firmly established phonological contrast between pre-aspiration and post-aspiration has not been attested in any language. However, the author argues that the two phenomena are still phonetically very distinct, so such a contrast is not impossible in principle. Hejná further discusses potential evolutionary reasons for this rarity, as well as possible analytical biases present in the very description of data containing preaspiration.

\textsc{Chapter 17} “Weighing preaspiration” by Radu Craioveanu (University of To\-ron\-to) is yet another typologically instructed view on consonantal preaspiration. The author analyses in reasonable detail a subset of 12 genetically diverse languages, mostly of Europe and North America, and mentions a few additional languages. The general conclusion is that preaspiration in these languages is always syllabified separately from the following consonant. The author also argues that preaspiration associated with the coda of the preceding syllable always bears prosodic weight (a mora). One typological question which arises here is whether the mora is a necessary or at least useful concept for all the languages for which preaspiration has been ever claimed. Also, are syllable boundaries always uncontroversially identified for all these languages? Another concern is whether just one criterion (in this case, of syllabification) enables one to say anything about the phonological status of preaspiration as a consonantal feature or as a separate consonant /h/. The chapter clearly sets up this and other theoretical questions regarding the formal phonological status of preaspiration across individual languages, forming a basis for future research in this direction.

\textsc{Chapter 18} “Pre-aspiration in Ecuadorian Siona” by Marijn van ’t Veer (University of Amsterdam), Martine Bruil, and Oleksandra Damonte-Matveienko (Leiden University) concludes the volume by discussing preaspiration in a Western Tukanoan language and considering a diverse range of evidence. The authors discuss the distributional and phonetic properties of Siona preaspiration, as well as its emergence, proposed to be through the gemination of voiceless consonants. Preaspiration is also seen in the context of other laryngeal features of the language (in particular, the status of the glottal fricative and stop), word prosody (stress and moraic structure), and nasal harmony. This multi-parametric approach leads the authors to conclude that Siona preaspiration is best seen as a consonantal feature.

\section*{Author contributions}

The draft text of this chapter was written by CA (Sections~\ref{sec:Introduction} and~\ref{sec:Hist-Concept-Volume}) and NK (Sections~\ref{sec:Why-Rarities} and~\ref{sec:Structure-Content-Volume}) and subsequently read and revised by all authors. Both CA and NK are the first authors of the chapter.

\printbibliography[heading=subbibliography,notkeyword=this]
\end{document}
