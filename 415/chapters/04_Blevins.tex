\documentclass[output=paper]{langscibook}
\ChapterDOI{10.5281/zenodo.15148168}
\author{Juliette Blevins\orcid{}\affiliation{The Graduate Center, The City University of New York}}
\title[Linguistically motivated sound change]{Linguistically motivated sound change: Revisiting some of the world’s rarest wonders}
\abstract{Historical linguists of the last several centuries have come to widespread agreement that the majority of regular sound changes in the world’s languages have clear phonetic and/or structural motivations. It is surprising then, that on the basis of ten typologically rare sound changes in Austronesian, \citet{Blust2005} takes an opposing position: sound change is not phonetically motivated, and need not be linguistically motivated at all. In this chapter, \citegen{Blust2005} analyses are questioned, taking into account research in phonetics, phonological typology, and contact linguistics of the past few decades. In some cases, multiple innovations are telescoped, while in others, articulatory, perceptual, aerodynamic, and structural factors, as well as language contact, are shown to play a role. The general conclusion is that sound change is indeed linguistically motivated.

\keywords{sound change, phonetically motivated sound change, structurally motivated sound change, contact-induced sound change, Austronesian sound change}
}
\IfFileExists{../localcommands.tex}{
  \addbibresource{../localbibliography.bib}
  \usepackage{langsci-optional}
\usepackage{langsci-gb4e}
\usepackage{langsci-lgr}

\usepackage{listings}
\lstset{basicstyle=\ttfamily,tabsize=2,breaklines=true}

%added by author
% \usepackage{tipa}
\usepackage{multirow}
\graphicspath{{figures/}}
\usepackage{langsci-branding}

  
\newcommand{\sent}{\enumsentence}
\newcommand{\sents}{\eenumsentence}
\let\citeasnoun\citet

\renewcommand{\lsCoverTitleFont}[1]{\sffamily\addfontfeatures{Scale=MatchUppercase}\fontsize{44pt}{16mm}\selectfont #1}
   
  %% hyphenation points for line breaks
%% Normally, automatic hyphenation in LaTeX is very good
%% If a word is mis-hyphenated, add it to this file
%%
%% add information to TeX file before \begin{document} with:
%% %% hyphenation points for line breaks
%% Normally, automatic hyphenation in LaTeX is very good
%% If a word is mis-hyphenated, add it to this file
%%
%% add information to TeX file before \begin{document} with:
%% %% hyphenation points for line breaks
%% Normally, automatic hyphenation in LaTeX is very good
%% If a word is mis-hyphenated, add it to this file
%%
%% add information to TeX file before \begin{document} with:
%% \include{localhyphenation}
\hyphenation{
affri-ca-te
affri-ca-tes
an-no-tated
com-ple-ments
com-po-si-tio-na-li-ty
non-com-po-si-tio-na-li-ty
Gon-zá-lez
out-side
Ri-chárd
se-man-tics
STREU-SLE
Tie-de-mann
}
\hyphenation{
affri-ca-te
affri-ca-tes
an-no-tated
com-ple-ments
com-po-si-tio-na-li-ty
non-com-po-si-tio-na-li-ty
Gon-zá-lez
out-side
Ri-chárd
se-man-tics
STREU-SLE
Tie-de-mann
}
\hyphenation{
affri-ca-te
affri-ca-tes
an-no-tated
com-ple-ments
com-po-si-tio-na-li-ty
non-com-po-si-tio-na-li-ty
Gon-zá-lez
out-side
Ri-chárd
se-man-tics
STREU-SLE
Tie-de-mann
} 
  \togglepaper[4]%%chapternumber
}{}

\begin{document}
\maketitle 
%\shorttitlerunninghead{}%%use this for an abridged title in the page headers
% ATTENTION: Diacritics on the following phonetic characters might have been lost during conversion: {'ɔ', 'ɑ', 'ɪ', 'ə', 'ɨ'}




\section{Understanding sound change}\label{sec:blevins:1}
Centuries of work in historical linguistics demonstrate that the great majority of regular sound changes in the world’s languages have clear phonetic and\slash or structural motivations (e.g. \citealt{DuanYucai1815, OsthoffBrugman1878, Grammont1933, Ohala1974,Ohala1981, Ohala1983, Ohala1989, Ohala1993, Blevins2004,Blevins2008a, Garrett2015}). Phonetic motivations can be articulatory, acoustic\slash perceptual, and\slash or aerodynamic. Structural factors include systems of contrast (e.g. \citealt{DeCheneAnderson1979, Blust2004}), functional load (e.g. \citealt{Martinet1952, WedelEtAl2013}); predictability (e.g. \citealt{Blevins2005a}), frequency (e.g. \citealt{Blust2007}), and contact (e.g. \citealt{Thurgood1999, Blevins2017a,Blevins2017b}), and can all act as inhibitors of change, or as facilitators or attractors in sound pattern categorization. Nevertheless, in light of the ten typologically rare sound changes in Austronesian languages listed in \tabref{tab:from-ex:blevins:1}, \citet{Blust2005} argues that sound change need not be \textit{linguistically} motivated.\footnote{\citet{Blust2005} concludes that social factors must trigger regular sound change in these special cases, though no particular social scenarios are put forth, and his general method is to continue to seek phonetic explanations for regular sound change, even when changes are highly unusual (\citealt{Blust2018,LobelEtAl2021}).  In \tabref{tab:from-ex:blevins:1} and throughout, I follow Austronesianist practice: <y> represents IPA [j] and <j> represents IPA [ɟ]. However, all symbols inside square brackets are IPA symbols. In \tabref{tab:from-ex:blevins:1}, ISO 639-3 language codes are included after each language name.}

\begin{table}%1
\caption{Some bizarre sound changes in Austronesian languages \citep[221]{Blust2005}}
\label{tab:from-ex:blevins:1}
\fittable{\begin{tabular}{l@{~~}ll}
\lsptoprule
   &   Sound change(s)               & Language(s)\\\midrule
a. &  *w/y > -\textit{p}                      & Drehet [tlx], Levei [tlx] (Khehek varieties) \\
b. &  *w/b > \textit{c}-, -\textit{nc}-                & Sundanese [sun]\\
c. &  intervocalic devoicing         & Berawan [zbc], [zbe], [zbw], Kiput [kyi]\\
d. &  *dr > \textit{kʰ}                       & Drehet\\
e. &  *-b/d/g > -\textit{m/n/ŋ}               & Berawan, Karo Batak [btx]\\
f. &  C > C: /{\longrule}V\#                & Berawan \\
g. &  *b > -\textit{k}-                       & Berawan\\
h. &  *g > \textit{p-, -j-, -p}               & Sa’ban [snv]\\
i. &  *an/aŋ > -\textit{ay}, *em/en/eŋ > -\textit{aw}  & Iban [iba]\\
j. &  postnasal devoicing            & Murik [mxr], Buginese [bug]\\
\lspbottomrule
\end{tabular}}
\end{table}

Each change in \tabref{tab:from-ex:blevins:1} is claimed to be regular (in the sense of Neogrammarian regularity), and each change is argued to be a single step, as opposed to a sequence of multiple changes. In later work, \citet{Blust2018} holds fast to this position:

\begin{quote}
In an earlier publication \citep{Blust2005} I drew attention to a number of surprising innovations in Austronesian languages, and concluded that few if any of them can convincingly be explained as due to previously unknown phonetic mechanisms, or to a telescoping of more typical changes over time. This has not stopped some scholars from continuing to speculate about possible phonetic conditions that apply very rarely, or about transitional steps that have never been observed… Needless to say, my purpose in discouraging arguments based on speculation is not to be anti-explanatory, but rather to insist that explanations be empirically grounded… With enough persistence and freewheeling imagination, some contrived explanation can no doubt be found. But my point in trying to raise awareness about the importance of such ‘unnatural’ cases is that an acceptable explanation (i.e. one based on phonetic motivation) can usually be salvaged only by proposing a series of steps for which no evidence exists. Rather than ‘whitewash’ such apparent changes or hide them in a closet of silence, it seems much more in the interest of science to recognize that they challenge a widely-held assumption, namely the assumption that because \textit{most} sound changes are phonetically motivated \textit{all} sound changes are phonetically motivated. In short, only by incorporating deviations from expectation in a theory of sound change can we hope to gain a deeper understanding of why expected changes work the way they do.\hfill\hbox{\citep[1--2]{Blust2018}}
\end{quote}

A similar position is echoed in Blust’s contribution to this volume.

This study, informed by research in phonetics, phonological typology, and contact phonology of the past several decades, reviews the sound changes in \tabref{tab:from-ex:blevins:1}, questioning each step of Blust’s original argumentation. In line with Blust’s remarks above, I evaluate the extent to which these changes challenge widely held assumptions about sound change: (i) Must the change be viewed as an instance of sound change, or are there alternative analyses? (ii) Is there evidence that the change is a single-step reflex, as opposed to a cumulative product of multiple innovations? (iii) Is there really no evidence for articulatory, perceptual, aerodynamic, or structural motivation of the change? And, since significant language contact can result in otherwise unexpected sound changes that mimic internal change (\citealt{Hamp1996}, \citealt{BabelEtAl2013}, \citealt{YaoChang2016}, \citealt{Blevins2017a,Blevins2017b})\footnote{Precursors to this work include \citet[213]{Sapir1921} \citet{Hock1975}, \citet{Desphande1979}, and \citet{Hamp1979}. \citet{AnderssonEtAl2017} refer to these as “Sprachbund” effects.}, (iv) Has language contact played a role in any of these changes? Multiple innovations are suggested for several cases, and arguments are made for articulatory, perceptual, aerodynamic, and structural motivation, as well as influences of language contact. The general conclusion is that sound change \textit{is} linguistically motivated, but in line with Blust’s intent, exploration of these rare deviations from common and predicted patterns allows us to gain a deeper understanding of sound change in all its complexity.

In \tabref{tab:blevins:1}, the sound changes listed in \tabref{tab:from-ex:blevins:1} are reordered and annotated to reflect the order of discussion in this study, specifics of target segments, and previous work on these recalcitrant problems (not including Blust’s subsequent publications which are cited throughout this chapter). In the first column, sound changes are numbered for easy reference throughout the chapter, and in the last column, the section of this chapter where the sound change is discussed is listed. The order of Blust’s original list in \tabref{tab:from-ex:blevins:1}, to my knowledge, had no particular rationale, and seemed to reflect the order in which he found himself working on these problems for the 2005 \textit{Diachronica} publication. Blust’s claim is that these sound changes are bizarre or unnatural, lacking clear phonetic or phonological motivation. Let us see if this is the case. Are any of these processes natural, where by natural we mean having a clear phonetic basis \citep{Blevins2008b}? If so, what factors obscure this naturalness?

\begin{table}[t]
\small
\begin{tabularx}{\textwidth}{lQQll}
\lsptoprule
{SC\#} & {Sound} & Language(s) & Previous  & This \\
       & change(s) &           & work         & work\\
\midrule
SC1 & \mbox{*-w, *-y > -\textit{p}} & Khehek (Drehet, Levei) & \citet{Goddard2007} & \sectref{sec:blevins:2.1}\\
SC2 & \mbox{*-b/-d/-g > \textit{-m/-n/-ŋ}} & Berawan, Karo Batak & \citet{Blevins2007} & \sectref{sec:blevins:2.2}\\
SC3 & \mbox{*b > g / V\_V} & Berawan &  & \sectref{sec:blevins:2.3}\\
SC4 & \mbox{*D > T/ V\_V} & Berawan, Kiput & \citet{Beguš2018, Beguš2019} & \sectref{sec:blevins:2.3}\\
SC5 & \mbox{*ND > NT} & Murik, Buginese & \citet{Beguš2018, Beguš2019} & \sectref{sec:blevins:2.4}\\
SC6 & \mbox{*g > \textit{p}-, -\textit{j}-, -\textit{p}} & Sa’ban (Kelabit) &  & \sectref{sec:blevins:3.1}\\
SC7 & \mbox{*-an/-aŋ > -\textit{ay}}, \mbox{*-em/-en/-eŋ > -\textit{aw}} & Iban &  & \sectref{sec:blevins:3.2}\\
SC8 & \mbox{*C > C:/\_V\#} & Berawan &  & \sectref{sec:blevins:3.3}\\
SC9 & \mbox{*dr- > \textit{kʰ}-} & Drehet &  & \sectref{sec:blevins:3.4}\\
SC10 & \mbox{*w-/b- > \textit{c}-,} \mbox{*-w-/-b- > -\textit{nc}-} & Sundanese &  & \sectref{sec:blevins:3.5}\\
\lspbottomrule
\end{tabularx}
\caption{\label{tab:blevins:1}“Unmotivated” sound changes from \citet{Blust2005}}
\end{table}

\section{Earlier replies to \citet{Blust2005}}\label{sec:blevins:2}
Since 2005, at least four of Blust’s “unexplainable” one-step changes in \tabref{tab:blevins:1} have been challenged, with sequences of natural phonetically motivated changes suggested instead. Under “Previous work” in \tabref{tab:blevins:1} are published studies that have explored some of the sound changes further, including the extensive work of \citet{Beguš2018,Beguš2019,Beguš2020} in forging a deeper understanding of unnatural phonology more generally. In reviewing the proposed sound changes in \tabref{tab:blevins:1}, I consider sound changes reviewed in earlier literature in \sectref{sec:blevins:2} (including SC3, which must feed SC4 in Berawan), and then, in \sectref{sec:blevins:3}, turn to the remaining proposed sound changes that have not received significant attention outside of Blust’s own work.

\subsection{*-w, *-y  > p in Levei and Drehet}\label{sec:blevins:2.1}
\citet{Goddard2007} reviews the apparent one-step merger of Proto-Oceanic *-w and *-y (IPA *[j]) to -\textit{p} in Drehet and Levei (SC1), repeated in \REF{ex:blevins:2}. Drehet and Levei are two dialects of Khehek, an Oceanic language of Manus Island within the Admiralties subgroup. Data on Khehek (Drehet-Levei) includes Blust’s published work and unpublished fieldnotes, the organized phonology data of \citet{Beard1992}, and an updated version of that \citep{SIL2004}.

\ea%2
    \label{ex:blevins:2}
         Glide fortition in Khehek (Drehet and Levei dialects)\\
*-w, *-y > -\textit{p}
\z

\begin{table}[b]
\begin{tabular}{llllll}
\lsptoprule
{Proto-Oceanic} & {Proto-Manus} & {Lindrou} & {Levei} & {Drehet} & {Gloss}\\
\midrule
a. *boRok & *powo & bow & pup & pup & `pig'\\
b. *kanawe & *kanawe & kanaw & kanap &  & `seagull'\\
c. *koe & *koe & ow & op & op & \textsc{2sg} pronoun\\
d. *pakiwak & *paʔiwa & beʔew & peʔep & peʔep & `shark'\\
e. *pitaquR & *pitawu & besew &  & pʷisip & \textit{`Calophyllum'}\\
f. *qayawan & *qaiwa & ew & ep & ep & `banyan'\\
\tablevspace
g. *kayu & *kayu & key & kep & kep & `tree, wood'\\
h. *laqia\footnote{In \textcite[231, Table 1]{Blust1995} there are two typographical errors: Proto Oceanic *laqia and Proto Manus *laqia ‘ginger’ are both written as *qalia.} & *laqia & ley & lip & lip & `ginger'\\
i. *layaR & *palea & baley & pelep & pelep & `sail'\\
i. *paRi & *payi & bey & pep &  & `stingray'\\
j. *puqaya & *puaya & pʷiyey & puep & puip & `crocodile'\\
k. *waiwai & *weweyi & ewey & owip & owip & `mango'\\
\lspbottomrule
\end{tabular}
\caption{\label{tab:blevins:2}Drehet and Levei data from \citet[230--231]{Blust2005}}
\end{table}

\largerpage
Blust’s central contention is that, while *w > (β > b) > \textit{p} is arguably a natural fortition process, a parallel change of the palatal glide *-y, and, in particular, *-y > -\textit{w}, is decidedly unnatural and unexpected. Data from \citet[230--231]{Blust2005} supporting the sound change in \REF{ex:blevins:2} is shown in \tabref{tab:blevins:2}, including Proto-Oceanic reconstructions (widely agreed upon) and Blust’s Proto-Manus reconstructions as well (cf. \citealt[659--660]{Blust2013}). Data from neighboring Lindrou [lid] illustrates continuations of inherited word-final glides, and Levei and Drehet, show *-w continued as -\textit{p} in the top half of the table, and *-\textit{y} continued as -\textit{p} in the bottom half.


\citet{Goddard2007} reviews the apparent one-step merger of Proto-Oceanic *-w and *-y to -\textit{p} in Drehet and Levei. He suggests a sequence of two natural phonetically motivated sound changes, where each step is claimed to be natural and supported by parallel developments in other languages. First, word-final *-w, *-y merge as -\textit{w} \REF{ex:blevins:3i}; then *-w strengthens to -\textit{p} \REF{ex:blevins:3ii}.

\ea%3
\label{ex:blevins:3}Suggested glide merger and glide strengthening in Khehek (Drehet and Levei dialects)
\NumTabs{4}
\ea *-y > *-w \tab  Glide merger\label{ex:blevins:3i}
\ex *-w > -p \tab  Glide fortition\label{ex:blevins:3ii}
\z
\z

However, the example of glide merger that Goddard refers to is not a word-final neutralization of *-y > *-w, but the Proto-Algonquian post-consonantal *w > *y/C\_ in Proto-Arapaho-Atsina. As such, it differs from \REF{ex:blevins:3i} in two important ways: the target is *w, not *y; and the context is post-consonantal (syllable-initial), not word-final. \citet[117]{Goddard2007} suggests that since the merger of labial and palatal glides is attested in one direction in one language family, there is no reason to doubt that it could have happened in the opposite direction in another. 

I agree with \citet{Blust2005,Blust2018} that there are reasons to doubt the naturalness of glide merger as stated in \REF{ex:blevins:3i}. First, many sound changes, like *θ > \textit{f}, are strongly unidirectional, so A > B should not be thought to imply B > A \citep{Blevins2019}. Second, \citet{Blust2018} notes that not a single case of regular context-free *y > \textit{w} is known in Austronesian. Finally, an attempt to collect all reported cases of non-assimilatory glide mergers, summarized in \tabref{tab:blevins:3}, suggests that non-assimilatory *y > \textit{w} is rare:\footnote{Assimilatory *w > \textit{y/\_e} occurs in Tabasco Chontal \citep{Justeson1985}, while assimilatory ?*y > \textit{w/u} may be found in Middle Indo-Aryan \citep[243]{Kümmel2007}.} Semitic, Bodish and Algonquian show *w > \textit{y}; Kuki-Chin suggests merger, but with no evidence of directionality. The only case with a clear non-assimilatory *y > \textit{w} merger is East Franconian, which is questionable, since the glide may be optional or variable.

\begin{table}
\begin{tabularx}{\textwidth}{QQQQ}
\lsptoprule
{Proto-language} & {Sound} {change} & {Evidenced} {in} & {source}\\
\midrule
Proto-Semitic & \mbox{*w- > \textit{y}-} & Northwest Semitic & \citet{Suchard2019}\\
Proto-Bodish & \mbox{*w- > \textit{y}-} & Old Tibetan & \citet{Hill2019}\\
\mbox{Proto-Algonquian} & \mbox{*w > \textit{y/C\_};} \mbox{*w,*y > \textit{n}} elsewhere & Arapaho-Atsina & \citet[106--107]{Goddard1974}\\
Proto-Kuki-Chin & \mbox{*-ay, *-aw > -\textit{ɘ}}; \mbox{*-aay/*-aaw > -\textit{e}} & Monsang & \citet{Konnerth2018}\\
Central German & \mbox{?*y > \textit{w/V\_V}} & East Franconian & \citet[243]{Kümmel2007}, \citet{Armborst1979}\\
\lspbottomrule
\end{tabularx}
\caption{\label{tab:blevins:3} Some non-assimilatory *w, *y (partial) mergers}
\end{table}

Given the rarity of non-assimilatory y > w mergers, it might be fruitful to consider distributional factors that could have played a role in a hypothesized -y > -w change in Khehek.  In \REF{ex:blevins:4} we see generalizations holding of word-final consonants and VG sequences in Khehek \textit{prior} to the hypothesized glide merger.

\ea%4
    \label{ex:blevins:4}Khehek word-final segments prior to *-y > -\textit{w}
    \ea Final -C\\\relax*-h, -ŋ, -k      
    \ex Final -VG\\\relax *-\{i, e, u, o, a\}w\\\relax*-\{i, e, a\}y\footnote{There are few clear examples of directly inherited *-iy sequences. One possible example is Levei \textit{kaalip} ‘rock cod, grouper’ < POC *kalika (cf. Lou \textit{keliy}, Ere \textit{kaliy}, Titan \textit{kaliy}, Lindrou \textit{kalik}) (\citealt[230--231]{Blust1995}; ACD).}
    \z
\z

Apart from *h which, under most accounts, lacks specified place features, prior to the proposed glide-merger, final Cs \{*ŋ, *k, *w\} are all dorsal\slash\relax [+back] with the exception of *-y. The high frequency of [+back] segments in final position might lead the listener to expect a final segment to be [+back]. In addition, since final *-y is only preceded by non-round (and, possibly, non-back) vowels, \footnote{That the central vowel \mbox{/a/} is treated as [-back] in Khehek is suggested by the evolution of \mbox{/ɔ/} from \mbox{/a/} under backing/rounding as in: \textit{buɔ} ‘water’ < *bua;  \textit{ɲuɔh} ‘steam’ < *ɲuah;  \textit{n’druɔ} ‘possum’ <  *ndrua (cf. POC *k(ʷ)adroRa ‘phalanger, cuscus’). These Khehek forms are from \citet{SIL2004}, where it is stated that \mbox{/ɔ/} (written as <ø>) is included as a sixth vowel phoneme “because it carries a fairly heavy semantic load, with several minimal pairs.” Note also the Levei form \textit{e-tæŋ} ‘to weep, cry’ (ACD), written with a front low vowel \mbox{/æ/}. In contrast, \citet{Beard1992} writes the Khehek low vowel as \mbox{/ɑ/}, and places it towards the back of his vowel chart.} one can understand the Khehek shift of *-y > -\textit{w} as a dissimilatory process: perceptual hypercorrection \citep{Ohala1981} occurs when listeners interpret [j] as the result of fronting \REF{ex:blevins:5.1} or unrounding \REF{ex:blevins:5.2} of [w] in the context of a preceding non-back or unround vowel respectively. Subsequent to \REF{ex:blevins:5}, glide fortition, as stated in \REF{ex:blevins:3ii}, occured.

\ea%5
    \label{ex:blevins:5}
Khehek dissimilatory *-y-backing or *-y-rounding 
\ea *-y > -\textit{w}/ [-back]\_\#  \hspace{1em}   (assuming /a/ is [-back]) \label{ex:blevins:5.1}
\ex *-y > -\textit{w}/ [-round]\_\#   \label{ex:blevins:5.2}
\z
\z

Strengthening or fortition of final *-w >(*β >)(*b >) -\textit{p} was accepted by \citet{Blust2005} as a natural fortition process based on similar processes in other Austronesian languages (e.g. Miri *w > \textit{b}, Kiput *w > \textit{fʷ}) and parallel cases could be added in other language families, e.g. proto-Siouan *waC > \textit{pC} in Mandan \citep[69]{Larson2016}.\footnote{A reviewer notes word-initial *w- > \textit{gw}- in some Admiralties languages \citep[321]{Ross1988} and asks whether *gw might be an intermediate stage for Khehek, given Ross’s proposal of *w- > \textit{b}- in Levei-Tulu. First, strengthening of *w- > \textit{gw}- is limited to initial position: cf. Likum \textit{gway} < *waiR ‘water’, vs. \textit{kaano-lew} ‘spider’ (\textit{lew} < PMP *lawaq ‘spider’), Likum \textit{sa:kow} ‘long variety of Malay apple’ < POC *sakau. In addition, Ross’s proposal for Levei-Tulu appears to be mistaken, with *w- continued either as \mbox{/w/} or zero in Khehek: cf. Drehet \textit{weyi pʷehea} ‘saliva’ [water mouth] (\textit{weyi} < *waiR), and Levei, Drehet \textit{owip} ‘mango’ < POC *waiwai.}  Since all final consonants are voiceless in Khehek, devoicing is expected. A further structural factor suggested by \citet[236]{Blust2005} as an additional component of the shift to \mbox{/p/} is a possible “pull-chain”: prior to this change, \mbox{/p/} was absent word-finally but present elsewhere. Hence, the shift of *-w >(β >)(b>)-p could be seen to fill a phonotactic gap. In sum, the reinterpretation of \REF{ex:blevins:2} as the two-step sequence of sound changes in \REF{ex:blevins:3}, and elaborated in \REF{ex:blevins:6}, is supported, with each change phonetically motivated, and having potential structural motivation as well.

\ea%6
    \label{ex:blevins:6}
         Suggested glide dissimilation and glide strengthening in Drehet and Levei
    \ea\label{ex:blevins:6a}
        \ea\label{ex:blevins:6ai} *-y > *-w/ [-back]\_\#  \\ Glide backing (phonetic + distributional/structural factors) \textit{or}
        \ex *-y > *-w/ [-round]\_\# \\   Glide rounding (phonetic + distributional/structural factors)
        \z
    \ex *-w > (*β >)(> *b) > -\textit{p}  \\ Glide fortition (phonetic + distributional/structural factors)
    \z
\z


The unique distribution of final consonants prior to this sound change may have played a catalytic role in \REF{ex:blevins:6a}, with non-syllabic back segments acting as attractors in the backing of *-y, and limited *-Vy sequences resulting in reinterpretation of *-y as a fronted/unrounded *-w. The pre-existing /-p/ gap in word-final position, may also have given rise to a pull-chain for *-w. Reference to these unique distributional properties of pre-Khehek may allow one to better understand why changes like \REF{ex:blevins:6} are typologically rare.

\subsection{*-b,-d,-g > \textit{-m,-n,-ŋ} in Berawan and Karo Batak}\label{sec:blevins:2.2}
Just as the shift of *-y > -\textit{w} might strike one as natural, arguably involving a shift in only one distinctive feature ([+back], [+round], or [+labial]), another sound change, SC2 in \tabref{tab:blevins:1}, also, at first glance, appears to be natural as well. This is a sound change where, in both Berawan, a language of northern Borneo, and in Karo Batak (aka Northern Batak) of Northern Sumatra, original word-final voiced stops become nasals:  *-b,*-d,*-g > -\textit{m},-\textit{n},-\textit{ŋ}.\footnote{In fact, as \textcite[249, 250]{Blust2005} notes, no examples are available in Berawan illustrating reflexes of *-g.}  Representative data from each language along with relevant reconstructions are provided in \tabref{tab:blevins:4} from \citet[248--251]{Blust2005}, with Berawan represented by the Batu Belah variety.

\begin{table}
\begin{tabularx}{\textwidth}{XXl}
\lsptoprule
 {Proto}  {North}  {Sarawak} &  {Batu Bela Berawan} &  {Gloss}\\
 \midrule
*eleb & lem    & `knee'\\
*uab & ŋ-uam   & `to yawn'\\
*laled & dilan & `housefly'\\
*kuyad & kuyan & `grey macaque'\\
*pused & pusan & `navel'\\
\midrule
{Proto} {Batak} & {Karo} {Batak} & Gloss\\
\midrule
*abab & abam     & `fine burning ashes'\\
*saŋkeb & saŋkem & `lid of clay pot'\\
*sahud & sahun   & `take place, occur'\\
*sered & seren   & `stinger of insect'\\
*talag & talaŋ   & `be open, as a door'\\
\lspbottomrule
\end{tabularx}
\caption{\label{tab:blevins:4}Final voiced stop to nasal in Berawan and Karo Batak \citep[248--249]{Blust2005}}
\end{table}

Karo Batak is unique among Batak languages in showing this sound change. Final *-b, *-d, *-g are continued without change in Simalungun, while they are devoiced in Southern Batak dialects like Toba Batak \citep{Adelaar1981}. Similarly, within the group of North Sarawak languages, most Dayic languages, like Lun Dayeh, continue final *-b, *-d without change \citep[235]{Smith2017}, while the same proto-sounds are devoiced in others, like Sa’ban. What makes these two languages stand out, both areally and genetically, is the apparent nasalization of final voiced obstruents. \citet[249--251]{Blust2005} summarizes the problem this way:

\begin{quote}
…why should voiced stops become nasals only in wordfinal position, where the more common development is for stops to devoice?... No obvious linguistic motivation... is apparent... it might be surmised that the change of voiced stops to nasals word-finally passed through an intermediate stage in which these stops were prenasalized... Prenasalization of both voiced and voiceless obstruents is a common phenomenon in Austronesian languages, but is almost always confined to medial position. Since prenasalized final obstruents are unknown in any of the hundreds of languages of the Philippines or western Indonesia it would be unjustified to assume a historically intermediate stage in which they were present…
\end{quote}

In earlier work, I suggested the sequence of sound changes shown in \REF{ex:blevins:7}, where voiced stops are pre-nasalized \REF[i]{ex:blevins:7}, and subsequently, prenasalized stops simplify to plain nasals \REF[ii]{ex:blevins:7} \citep{Blevins2007}.

\ea%7
    \label{ex:blevins:7}Voiced stop pre-nasalization followed by simplification
\ea *-b, *-d, *-g   >   *-ᵐb, *-ⁿd, *-ᵑg\label{ex:blevins:7i}
\ea via nasal-spread in N(C)V\_\#
\ex via hypervoicing
\z
\ex   *-ᵐb, *-ⁿd, *-ᵑg >  \textit{-m, -n, -ŋ}  \\
      non-release (or non-audible oral phase)\label{ex:blevins:7ii}
\z
\z

While the focus of that analysis was \REF{ex:blevins:7ii}, phonetic support for \REF{ex:blevins:7i} is twofold. First, at least half of the Karo Batak words showing final nasalization of *-b, *-d, *-g (and -\textit{g} < *-j) have historical nasal stops in the preceding syllable, allowing one to view some cases of \REF{ex:blevins:7i} as instances of local syllable internal-nasal spread: \textit{cernem} < PWMP *ce(R)ñeb ‘dive, immerse’; \textit{keneŋ} < PMP *keñej ‘sink, drown’; \textit{kuniŋ} < PMP *kunij ‘tumeric’; \textit{laŋkem} < PWMP *la(ŋ)keb ‘lie face down’; \textit{saŋkem} < PWMP *sa(ŋ)keb ‘lid, cover’; \textit{sumban} < PWMP *sumbad ‘plug, stopper’ (ACD).\footnote{Interestingly, there also appear to be cases in Karo Batak where nasalization spreads from the end of the word back: \textit{ambaŋ} < *abaŋ ‘ambush; block the way’; \textit{endem} < *edem ‘overcast; dull luster’;  \textit{me-linduŋ} < *liduŋ ‘shelter, cover, protection’.} Second, the intermediate pre-nasalized stage shown in \REF{ex:blevins:7i} may have led to interpretation of pre-nasalized voiced stops as variants of plain voiced stops as a consequence of hypervoicing. On the basis of phonetic studies of similar sound patterns, pre-nasalization is argued to be a type of hypervoicing, with nasal airflow making it easier to maintain a voiced oral stop in word-final position (\citealt{IversonSalmons1996, WetzelsNevins2018}). After historical pre-nasalization, simple non-release in final position, or a short inaudible oral phase, could result in a percept of a plain nasal \REF{ex:blevins:7ii}. Further support for \REF{ex:blevins:7} as a natural process is typological. The same process of final nasalization is found in some Kayanic languages, and other languages of Borneo, including Lebo and Sagai Basap \citep[201]{Smith2017}. In Long Nah Kayan (\citealt{Smith2017}: 53, 66), *-b, *-d > -\textit{m}, -\textit{n}, while inherited *mb, *nd, *nj, *ŋg > \textit{b}, \textit{d}, \textit{j}, \textit{g}, supporting general simplification of ND articulations. Within Oceanic, where inherited *-b, *-d, *-g are phonetically prenasalized, a syllable-final instance of \REF{ex:blevins:7ii} is reported for Lihir, a Western Oceanic language, and in Mwotlap, a language of Vanuatu, \textit{ᵐ}\textit{b, ⁿ}\textit{d} vary with \textit{m, n} word-finally, while two languages of Vanuatu (Neve’ei and Avava) and two languages of New Caledonia (Nemi and Jawe) show \REF{ex:blevins:7ii} word-finally. What appears to make the sound change rare is not the simplification in \REF{ex:blevins:7ii}, but, rather, the rarity of finding prenasalized voiced stops in word-final position to begin with.

\subsection{Intervocalic devoicing in Berawan (and Kiput) and Berawan \mbox{*-b-  > -\textit{k}-}}\label{sec:blevins:2.3}
Intervocalic devoicing of voiced stops has been argued by \citet{Blust2005}, and more recently, by \citet{Beguš2018,Beguš2019} to be an unnatural sound change.\footnote{\citet[12--13]{Beguš2019} defines “unnatural” processes in terms of “universal phonetic tendencies” (UPT). Universal Phonetic Tendencies are “phonetic pressures motivated by articulatory or perceptual mechanisms that passively operate in speech production cross-linguistically and result in typologically common phonological processes.” Unnatural processes operate against UPTs, or are not UPTs. Within this model, examples of unnatural processes include intervocalic devoicing, post-nasal devoicing (see \sectref{sec:blevins:2.4}), and final voicing (but cf. \citealt{BlevinsEtAl2020}). Compare this with the definition of natural sound changes in \sectref{sec:blevins:1} as those with clear phonetic bases; unnatural processes are all others \citep{Blevins2008b}.} On articulatory, aerodynamic and perceptual grounds, there is a natural phonetic tendency for stops to be voiced intervocalically, and intervocalic voicing is common cross-linguistically \citep[21--22]{Beguš2019}. Nevertheless, intervocalic devoicing appears to have occurred in at least two Austronesian languages of North Sarawak in Borneo: in Berawan (\citealt{Blust2005, Blust2013, Burkhardt2014}) and in Kiput (\citealt{Blust2002, Blust2005}) (SC4). \citet{chapters/03_Blust} offers a critique of Beguš’s general analysis of intervocalic devoicing, and of the more specific analysis proposed for Berawan where, in addition, *-b- shifts to velar articulation (SC3). In the spirit of Blust’s critique, what I offer here is a simple alternative to Beguš’s Blurring Account \citep{Beguš2019}, with a focus on Berawan. The treatment of intervocalic devoicing extends to Kiput, and to other languages exhibiting apparent one-step intervocalic devoicing.

Pre-Berawan bilabial and velar stops *b and *g devoice intervocalically, but remain voiced word-initially. The bilabial stop *b also undergos a change of place intervocalically, becoming a velar: \mbox{*b > \textit{k} / V\_V}. Assuming intervocalic devoicing, one could propose intervocalic *-b- > *-g-, followed by devoicing. \tabref{tab:blevins:5} provides four examples of Berawan intervocalic devoicing of *g, where *g is from Proto North Sarawak *R.\footnote{The symbol *R in historical Austronesian is usually interpreted as an alveolar or uvular trill, though there is continued debate regarding its phonetic status; see \citet[Chapter 8]{Blust2013}. Velar reflexes like those in Berawan are not uncommon.} Gemination in the last two rows of \tabref{tab:blevins:5} is a consequence of a later sound change discussed in \sectref{sec:blevins:3.3}.

\begin{table}
\begin{tabularx}{\textwidth}{lXll}
\lsptoprule
{Proto} {North} {Sarawak} & {after} *R> *g &   {Berawan} & {Gloss}\\
\midrule
*Rabun & *gabun & gikuŋ         & `cloud, fog'\\
*Ratas & *gatas & gitaʔ         & `milk'\\
*Ratus & *gatus & gitoʔ         & `hundred'\\
*Rusuk & *gusuk & gusok         & `chest'\\
*aRem & *agem & akem            & `pangolin'\\
*kaRaw & *kagaw & kikiw         & `scratch an itch'\\
*baRa & *baga & bikkeh < *bikeh & `shoulder'\\
*duRi & *dugi & dukkih < *dukih & `thorn'\\
\lspbottomrule
\end{tabularx}
\caption{\label{tab:blevins:5} Berawan *R> g, *-g-> -k-, C>C:/\_V\# \citep[667--668]{Blust2013}}
\end{table}

\largerpage[2]
\tabref{tab:blevins:6} illustrates *-b- > *-g- > \textit{k}- with data from \citet[667--668]{Blust2013}.\footnote{For a thorough review of the data, and a critique of \citet{Beguš2019}, see \textcitetv{chapters/03_Blust}.} Blust assumes the intermediate stage *-g-, and so, Berawan presents itself, in his view, with two inexplicable instances of sound change: *-b- > *-g-; and general intervocalic obstruent devoicing. Gemination in \tabref{tab:blevins:6}, like that in \tabref{tab:blevins:5}, is a consequence of a later sound change discussed in \sectref{sec:blevins:3.3}.
\clearpage

\begin{table}
\begin{tabularx}{\textwidth}{XXl}
\lsptoprule
{Proto} {North} {Sarawak} & {Berawan} & {Gloss}\\
\midrule
*abaŋ & akiŋ       & `ilipe nut'\\
*abu & akkuh       & `ashes'\\
*balabaw & belikiw & `rat'\\
*babuy & bikuy     & `pig, boar'\\
*bubu & bukkuh     & `fish trap'\\
*bubuŋ & bukuŋ     & `ridge of roof'\\
*Ribu & gikkuh     & `thousand'\\
*Rabun & gikuŋ     & `cloud, fog'\\
*kabiŋ & kakiŋ     & `left side'\\
*lubaŋ & lukiŋ     & `hole'\\
*mabuk & makuk     & `drunk'\\
*nibuŋ & nikuŋ     & `nibong palm'\\
*tuba & tukkih     & `derris root'\\
\lspbottomrule
\end{tabularx}
\caption{\label{tab:blevins:6}*-b- > *-g- and intervocalic devoicing in Berawan \citep[667--668]{Blust2013}}
\end{table}


We address general intervocalic devoicing first. While intervocalic devoicing is rare cross-linguistically, under certain conditions, it can be viewed as phonetically natural. If an intervocalic consonant is in a prosodically strong position, fortition is expected to take place. For the data in \tabref{tab:blevins:5}, where we hypothesize a historical [ws] iambic foot structure, the prosodically strong position is the onset of the strong syllable of the foot. Prosodically strong positions range from initial position of a stressed syllable, to initial position of a foot, to initial position of the prosodic word, and higher level units \citep[155]{Gordon2016}. In these positions, phonetic lengthening may occur as a form of fortition.\footnote{For a typological survey of the range of fortition processes associated with increased stricture, see \citet{BybeeEasterday2019}.  The possible typological parallels between Berawan fortition and fortition/lenition processes in Balto-Finnic and Saamic consonant gradation are discussed in \citet{Blust2018} with specific reference to \citegen{Gordon1997} proposal of historical onset fortition in Finnic. Blust considers an analysis very similar to that proposed here and in \sectref{sec:blevins:3.3}, but rejects it on the grounds that one is unable to explain the absence of similar fortition in the hundreds of other Austronesian languages have similar prosodic structures. As suggested by the comparison with Haroi here, and the brief discussion in \sectref{sec:blevins:4}, fortition may be due to contact with Mon Khmer languages with sesquisyllabic structure. In sesquisyllables or uneven weak-strong iambs, the initial syllable is severely restricted in terms of segmental and prosodic features, while the strong syllable shows all possible contrasts.} In intervocalic position, the association between prosodically strong positions and devoicing is most likely related to increased stop duration: the longer the stop closure, the higher the probability that voicing will decay \citep{Ohala2003}. A recent corpus-based study of synchronic word-initial devoicing in five Romance languages supports an association between obstruent devoicing and prosodically strong word-initial position, and, for Portuguese, shows a clear effect in intervocalic, as well as other positions \citep{HutinEtAl2021}.

Apart from Berawan and Kiput, historical cases of prosodically conditioned intervocalic devoicing in Austronesian might include devoicing and aspiration of proto-Chamic voiced obstruents in the main syllable onset of Hainan and Coastal Chamic languages \citep[82]{Thurgood1999}.\footnote{Outside of Austronesian, a parallel change can be found in the devoicing of proto-Samurian voiced stops in the onsets of stressed syllables in pre-Lezgian, where the same segments were geminated (\citealt{Topuria1974, Yu2004}). It is notable that in Berawan as well, as discussed in \sectref{sec:blevins:3.3}, intervocalic gemination occurred, albeit in a more specific prosodic context.} 

\begin{table}
\begin{tabularx}{\textwidth}{XXXl}
\lsptoprule
{Proto} {Chamic} & {Jarai} [jra] & {Haroi} [hro] & {Gloss}\\
\midrule
*ʔabih & abih & aphih     & `all; finished, done'\\
*ʔiduŋ & aduŋ & athuŋ     & `nose'\\
*babuy & bəbui & pəphui   & `wild pig'\\
*habow & həbau & aphɨau   & `ashes'\\
*hadaŋ & hədaŋ & athɨaŋ   & `charcoal'\\
*hubey & həbəi & aphui    & `taro; tuber; yam'\\
*huda:ŋ & həda:ŋ & athiaŋ & `shrimp, lobster'\\
*labuh & rəbuh & ləphuh   & `fall down'\\
*muda & məda & məthɨa     & `young; unripe; tender'\\
*pagar & pəga & pəkhɨa    & `fence'\\
\lspbottomrule
\end{tabularx}
\caption{\label{tab:blevins:7} Intervocalic devoicing and aspiration in Haroi, a coastal Chamic language  \citep[82]{Thurgood1999}}
\end{table}

The Chamic cases are not described in terms of the intervocalic environment, because the main syllable onset is a clear prosodic position in the evolution of Chamic sequisyllables through documented contact with languages of mainland southeast Asia \citep{Thurgood1999}; however, as the data in \tabref{tab:blevins:7} illustrate, the position, is, indeed, intervocalic. Another Austronesian language with clear evidence of intervocalic devoicing is Sula, a language of north Maluku \citep{Bloyd2020}, where stem-initial consonants have devoiced under prefixation and compounding.

In short, there is evidence that intervocalic devoicing can and does occur in prosodically strong positions. For this reason, following Blust’s quest for the simplest analysis (\cite{chapters/03_Blust}), we hypothesize the sound change in \REF{ex:blevins:8} for Berawan (and Kiput).

\ea%8
    \label{ex:blevins:8}
         Intervocalic devoicing in Berawan\\
*g  > \textit{k} / V\_V  \hspace{1em}  (Hypothesis: the position is prosodically strong)
    \z

A final piece of evidence supporting the sound change in \REF{ex:blevins:8} as direct devoicing under phonetic lengthening in prosodically strong position is that fact that it targets velar *g, but not stops at other points of articulation. This is more evident in Kiput, where *b is continued as a bilabial stop and “intervocalic stops *b and *d generally did not devoice, but *g did” \citep[668]{Blust2013}. The aerodynamics of stop voicing appear to be at work: velar stops are more likely to undergo devoicing than stops at more advanced positions in the oral cavity because there is less air volume behind the point of closure, and limited capacity for passive enlargement to lower supraglottal pressure (\citealt{OhalaRiordan1980}). If general, across-the-board devoicing, of the kind suggested by \citet{Beguš2019} was at work, this pattern would be unexpected.

Let us now turn to the apparent *-b- > *-g- sound change in pre-Berawan. I suggest, based on parallel developments in other languages of Borneo, that this is not a single labial-to-velar sound change, but rather a sequence of lenition and fortition processes, as stated in (\ref{ex:blevins:9i}) and (\ref{ex:blevins:9ii}). A later process of glide-insertion and strengthening \REF{ex:blevins:9iii}, illustrated in \textcitetv{chapters/03_Blust} for Proto Berawan, post-dates glide-strengthening \REF{ex:blevins:9ii}, producing a new set of bilabial glides/fricatives.

\ea%9
\label{ex:blevins:9}Proto-Berawan *-b- > *-g- as a lenition + fortition sequence, followed by *w-insertion\footnote{A reviewer notes that \citet{Burkhardt2014} also reconstructs *β to proto-Berawan as the outcome of intervocalic glide insertion. One can still propose *b > *w > \textit{g} after this insertion, provided that the inserted *w in \REF{ex:blevins:9iii} had already become /β/.}
\NumTabs{3}
\ea\label{ex:blevins:9i} *-b- > *w/ V\_V     \tab intervocalic lenition
\ex\label{ex:blevins:9ii} *w > *g/V\_V       \tab glide strengthening
\ex\label{ex:blevins:9iii} *ua > \textit{uwa} (> \textit{uβa)} \tab glide-insertion + glide strengthening
\z
\z

Similar sound changes to those in \REF{ex:blevins:9} have been proposed for other languages of Borneo. In \tabref{tab:ex:blevins:10}, \citegen{Smith2017} treatment of historical developments in Segai-Modang, a subgroup of Kayanic, is illustrated, involving a different time-line of the same developments suggested for Berawan in \REF{ex:blevins:9}.

\begin{table}%10
\caption{Proposed historical developments in Segai-Modang \citep[76--77]{Smith2017}}
\label{tab:ex:blevins:10}
\begin{tabularx}{\textwidth}{XXXl}
\lsptoprule
               &   \multicolumn{3}{c}{PMP}\\\cmidrule(lr){2-4}
Sound change       &  *abu ‘ashes’ & *sawa ‘spouse’  & *lubaŋ ‘hole’\\\midrule
i.   *uba > \textit{uwa}    &   -           &   -             & *luwaŋ\\
ii. *-w- > *gw\footnote{Though \citet[77]{Smith2017} writes the sound change as *-w- > \textit{g}, the data suggests *w > *gw.}
                    &  -      & *səgwa     &  *ləgwaŋ\\
iii. *-b- > -\textit{w}-     &  awu    &   -        &  -\\
Modang [mxd]        &  əwaɵ̯  &     səguə̯n  &   guə̯ŋ\\
Long Gelat [mxd]    &  awaː   &   səgün    & guə̯ŋ\\
Gaai [sge]          & awaw    &  sguə̯n     &  guə̯ŋ\\
Kelai [sge]         & awəw    &            & guə̯ŋ\\
\lspbottomrule
\end{tabularx}
\end{table}

As a consequence of these changes, there is an apparent shift of *-b- > *-g- in the *u\_a environment (e.g. Modang \textbf{\textit{g}}\textit{uə̯ŋ} < *lu\textbf{b}aŋ ‘hole’, \textit{tə}\textbf{\textit{g}}\textit{awʔ} < *tuba ‘derris root’) though the context, as well as the strengthening of inherited *-w- argues for earlier (contextual) weakening of *-b- to *-w- (10i), as opposed to a single-step change. Compare the developments in \tabref{tab:ex:blevins:10}, with the proposed sequence of changes for Berawan in \tabref{tab:ex:blevins:11}.

\begin{table}%11
\caption{Proposed developments in Berawan (from \citealt{Blust2005}, except where noted)}
\label{tab:ex:blevins:11}
\begin{tabularx}{\textwidth}{XXXl}
\lsptoprule
                 & \multicolumn{3}{c}{PNS}\\\cmidrule(lr){2-4}
Sound change     & *abu ‘ashes’ &  *lubaŋ ‘hole’ &  *dua ‘two’\\\midrule
i.   *-b- > *-w- & *awu  &   *luwaŋ  & \\
ii. *-w- > *g    & *agu  &   *lugaŋ  & \\
iii. *ua > *uwa  &  -    &  -        &  duwa\\
(ex. 7) *-g- > -\textit{k}-\\
Batu Belah      & akkuh  &  lukiŋ      & duβeh\\
Long Terawan    & akkuh  &  luyə̯ŋ      & lebih\\
Long Jegan      &        &  lukiŋ [AS] & duve [AS]\\
\lspbottomrule
\end{tabularx}
\end{table}

A brief comparison of the proposed account above with that of \citet{Beguš2019} highlights both its simplicity and improved plausibility, in line with the desiderata of \textcitetv{chapters/03_Blust}.

\citet{Beguš2019} offers an analysis of intervocalic devoicing in Berawan and Kiput that involves a sequence of three arguably natural and well motivated sound changes in the context of what he calls the “Blurring Account”. The Blurring Process is a term for a special combination of sound changes that gives rise to an unnatural sound pattern in the technical sense of footnote 9. In \tabref{tab:ex:blevins:12} we see the proposal for sound changes to account for Berawan intervocalic devoicing and labial-to-velar shift under Beguš’s Blurring Account. The three definitive sound changes under the Blurring Account are I, II, III in that order, defining (rare) intervocalic devoicing patterns cross-linguistically. Note that the second of these, including ɸ > \textit{x}, is empirically questionable, at least within Austronesian (a family of at least 1,200 languages), where \textcitetv{chapters/03_Blust} cannot think of a single case.

\begin{table}%12
\caption{Berawan intervocalic devoicing and labial-to-velar shift after \citet{Beguš2019}}
\label{tab:ex:blevins:12}
\begin{tabularx}{.8\textwidth}{X c@{~}c@{~}c @{\qquad\qquad} c@{~}c@{~}c}
\lsptoprule
  Context                        &  \multicolumn{3}{l}{\#{\longrule}}  &  \multicolumn{3}{c}{V\_V}\\\midrule
  Pre-Berawan                    &  *b  & *d  & *g    &  *b  & *d  & *g\\\addlinespace
  I Intervocalic spirantization  &      &     &       &  *β  & *ð  & *ɣ\\
  +Rhoticization *ð  > r         &      &     &       &      & *r  &   \\
  II Fricative devoicing         &      &     &       &  *ɸ  &     & *x\\
  +Place change  *ɸ > x          &      &     &       &*x      &     & \\
  III Fricative occlusion        &      &     &       &  *k  &     & *k\\\addlinespace
  Berawan                        &   \textit{b}- &  \textit{d}- &  \textit{g}-   &  -\textit{k}- & -\textit{r}- & -\textit{k}-\\
  \lspbottomrule
\end{tabularx}
\end{table}

Compare Beguš’s analysis of *g > \textit{k} stop devoicing in \tabref{tab:ex:blevins:12}, with three stages, to the account in \REF{ex:blevins:8}, with one. In addition to this added complexity, Beguš must rely on modifications of I, II, III to restrict devoicing to *g in Kiput, while this feature follows from aerodynamic properties of velar stops under the phonetic lengthening account, as noted above. While details of \REF{ex:blevins:8} and \REF{ex:blevins:9} remain to be worked out, \textit{gw} < *-w- in Segai-Modang provides preliminary support for glide-strengthening, as opposed to fricative occlusion. Under the proposed account, the rarity of the Berawan development is a simple consequence of the rarity of the four sound changes in \REF{ex:blevins:8} and \REF{ex:blevins:9} co-occuring in this particular sequence.

\subsection{\label{sec:blevins:2.4}Post-nasal devoicing in Murik, Buginese and elsewhere.}
 Postnasal devoicing of voiced stops has been argued by \citet[258--262]{Blust2005}, and, more recently, by \citet{Beguš2018,Beguš2019} to be an unnatural sound change. On articulatory and aerodynamic grounds, there is a natural phonetic tendency for stops to be voiced after nasals, and post-nasal voicing is common cross\hyp linguistically. Nevertheless, *ND > NT (D a voiced stop, T a voiceless stop) appears to have occurred as a regular sound change in Murik-Merap, Kayanic languages of Borneo (dialects include Merap, Ngorek, and Mpraa), and in Buginese in its development from Proto South Sulawesi \citep{Mills1975}. I believe that \citet[258--262]{Blust2005} has already offered a clear solution to the origins of post-nasal devoicing in both of these cases, and an explanation for why the process is not more common. The central observation is that prior to the *ND > NT sound change, there is no contrast between ND vs. NT in these languages (in Buginese, earlier *NT > TT). As a consequence, \citet[261--262]{Blust2005} suggests two conceivable scenarios leading *ND to devoice. In the first, speakers assume post-nasal voicing to be automatic, and, essentially, hypercorrect by devoicing. In the second scenario, which he seems to endorse, the absence of a voicing contrast in *ND leads to natural variability in the extent to which the cluster is fully voiced, with the voiceless variant ultimately prevailing. Both of these proposals are plausible, and provide a testable hypothesis regarding true one-step sound changes of ND > NT: they will occur only where obstruent voicing is non-contrastive after nasals.

Needless to say, adopting these proposals for post-nasal voicing in these Austronesian languages does not preclude other cases that may reflect telescoped changes. The alternative in \citet{Beguš2019}, is, like the analysis of intervocalic devoicing, a three-step process: (i) voiced stops become fricatives in all environments, except after nasals; (ii) general oral stop devoicing; (iii) occlusion of voiced fricatives to stops. This may be appropriate for some languages, but it does not fit with the well studied phonological developments in either Kayanic \citep{Smith2017} or South Sulawesi languages \citep{Mills1975}.

\section{\label{sec:blevins:3}New cases of phonetically motivated sound change (or not).}
\largerpage
Since the publication of \citet{Blust2005}, six of the “bizarre” changes in \tabref{tab:blevins:1} have, to my knowledge, been revisited only in Blust’s own work (\citealt{Blust2013,Blust2016, Blust2018}). In this section, I review each of the remaining proposed sound changes in \tabref{tab:blevins:1}, beginning with the case of Sa’ban (SC6) where supporting data appears to be misanalysed, and then turning to Iban (SC7), where the sound change may be misstated and better viewed as a case of nasal dissimilation. \sectref{sec:blevins:3} continues with discussion of Berawan gemination (SC8), Western Manus trill obstruentization (SC9), and Sundanese labial palatalization (SC10), which, it is argued, can all be seen to reflect natural phonetically motivated processes.

\subsection{*g > \textit{p}-, -\textit{j}-, -\textit{p} in Sa’ban}\label{sec:blevins:3.1}
\largerpage
Sa’ban is a highly innovative dialect of Kelabit spoken at the headwaters of the Baram River in the phonological ‘hot spot’ of northern Sarawak, where several languages show extensive and unusual sound changes \citep{Blust2001}. Among many innovations classified as “bizarre” by \citet{Blust2005} is the Sa’ban shift of *g to \textit{j} ([dʒ]) intervocalically and initially in some forms, but to \textit{p} word-finally, and, in two forms, word-initially \parencites[255--257]{Blust2005}[28--29]{Blust2018}. Forms supporting *g > \textit{j-, -j-} are shown in \tabref{tab:blevins:8}, where PKLD is Proto-Kelabit-Lun Dayeh. The shift of velar to palatal place is natural from an articulatory perspective, involving advancement of the tongue body. While context-sensitive shifts are common in front-vowel contexts (see, e.g. \citealt[215--216]{Kümmel2007}), recent work suggests that velar palatalization can be triggered by articulatory strengthening, where tongue-to-palate contact increases significantly in non-front vocalic environments \citep{Recasens2020}; in a least one variety of Catalan spoken on Majorca, the standard velars \mbox{/k/}, \mbox{/g/} are realized as palatal [c], [ɟ], except before back rounded vowels \citep[10]{Wheeler2005}. In Sa’ban, the shift from velar to palatal may be part of a chain-shift, since Sa’ban, like other Dayic languages, has undergone a merger of PMP palatals with coronals \citep{Smith2017}. In particular, PMP *d and *z (an alveopalatal affricate) are continued as \mbox{/d/} in Dayic.

\begin{table}
\begin{tabularx}{\textwidth}{XXX}
\lsptoprule
{PKLD} & {Sa’ban} & {Gloss}\\
\midrule
*gatəl & jatəl      & `itchy'\\
*gənuluh & jənləw   & `empty rice head'\\
*gəramih & jəlaməy  & `rice straw'\\
*gituʔən & jɪntoʔɔn & `star'\\
*ŋ-agap & m-ajeəp   & `to startle'\\
*sagət & ajɪt       & `quickly'\\
*iguʔ & jəuʔ        & `shame'\\
*pəgamuŋ & pəjamuəŋ & `tangled'\\
*təgəkər & təjaʔəl  & `to shiver'\\
*təgəraŋ & təjareəŋ & `ribs'\\
\lspbottomrule
\end{tabularx}
\caption{\label{tab:blevins:8}Velar palatalization: *g > j-, -j- in Sa’ban \citep[28--29]{Blust2018}}
\end{table}

What makes Sa’ban remarkable, in Blust’s view, is the apparent sound change of *g > \textit{p} initially in some words, and finally in others. 

Let us first review the cases where Blust suggests a regular change of word-initial *g- > \textit{p}-. The only two examples noted are shown in \tabref{tab:blevins:9}. Given the seemingly regular correspondence of *g- with \textit{j}- in \tabref{tab:blevins:8}, it would appear that these examples do not constitute a case of regular sound change.

\begin{table}
\begin{tabularx}{\textwidth}{XXX}
\lsptoprule
 {PKLD} &  {Sa’ban} &  {Gloss}\\
\midrule
*gərawat & pelawet & ‘tangled, complicated’  \\
*giləg & pelep & ‘skittish, easily startled’\\
\lspbottomrule
\end{tabularx}
\caption{\label{tab:blevins:9}*g- > \textit{p}- in Sa’ban from \citet{Blust2005, Blust2018}}
\end{table}

In the first example, Sa’ban \textit{pelawet} ‘complicated’ appears to reflect a \mbox{/pe-/} prefixed form of an earlier root *lawat ‘cross over, go across’: cf. Kayan \textit{lawat} ‘to cross over’; Gah \textit{lawat} ‘bridge’ [AS]; Sawa \textit{lawat} ‘bridge’ [AS]; Kelabit \textit{gərawat} ‘tangled, complicated’ [BD]; \textit{pəlawət} ‘religious travellers’ [BD] (i.e. ‘those that go across’); Bidayuh Serian \textit{kaawat} ‘wire’ [BD] < *ka-lawat (i.e. ‘crossed over, tangled’).\footnote{The prefix \mbox{/pe-/} appears to be a productive reciprocal prefix, among other functions. Compare Kelabit \textit{pebala} ‘to tell each other’ (\textit{bala} ‘news, fame, reputation’), \textit{pebisung} ‘to shove each other about’ (\textit{bisung} ‘shoved, pushed’), \textit{pebukut} ‘to punch one another’ (\textit{bukut} ‘a punch’) [BD].} Sa’ban \textit{pelawet} ‘complicated’ appears to derive from earlier *pe-lawet ‘crossing over each other’, and be cognate with Kelabit \textit{pəlawət} ‘religious travellers’.  Blust’s PKLD reconstruction *gərawat appears to correspond with a distinct derived form with prefix *gə-. If this is the case, there is no sound correspondence between initial *g- and Sa’ban \textit{p}- in this word pair. In the second example, Sa’ban \textit{pelep} ‘skittish’ is compared with Kelabit \textit{gileg} ‘feel nauseated, feel like vomiting when seeing something disgusting; squeamish about something; make a startled movement, as when someone pokes you in the ribs’[BD]. Although there is some semantic overlap, there also appears to be evidence for a root *leb/lep ‘jump, flicker’ in the languages of North Sarawak: cf. Kayan \textit{lepseu} ‘jump’, Bidayuh Serian \textit{leb-leb} ‘a flickering light’, \textit{lep-lep} ‘lightning flash’, Lun Dayeh \textit{məkələp} ‘blink’, \textit{lep:u} ‘faint’; Ketapang \textit{solap} ‘faint’ [AS]; Sanggau \textit{mbilap} ‘blink’ [AS]. On this basis, Sa’ban \textit{pelep} < *pe-lep ‘jumpy, skittish’seems just as plausible as Blust’s comparison. Neither of these comparisons, then, is convincing.

Turning to reflexes of word-final *-g, \citet{Blust2005} provides one example of *-g > -\textit{k:} Sa’ban \textit{malok} ‘to trick’ < PKLD *m-alug (cf. Kelabit \textit{malug} \mbox{/em-alug/}). Sa’ban \textit{luwék} ‘chest (anat.)’ may also show the same correspondence if this form is cognate with Bidayuh Serian \textit{adŭg, aadŭg} ‘chest’ (but cf. Kelabit \textit{ruuk} ‘chest’). While PWMP *-g is continued as \mbox{/-g/} in Kelabit (e.g. Kelabit \textit{aag} ‘finely chopped meat or vegetables’ < PWMP *sagsag ‘to chop, to mince, as meat, fish or vegetables preparatory to cooking’; Kelabit \textit{teteg} ‘stick used for beating a gong’ < PMP *-teg ‘hit, beat’), there are no Sa’ban cognates presented by Blust for these common roots, and examples of *-g > -{p} in Sa’ban are not altogether convincing. Consider Sa’ban \textit{arep} < *areg ‘crumbs, rubbish’ (cf. Kelabit \textit{areg} ‘crumbs’). The problem with this comparison is that there is another Kelabit word, \textit{arep} ‘rubbish, litter’ which is also a potential cognate (cf. Malay \textit{sarap} ‘dry rubbish’). Another comparison is Sa’ban \textit{ppap} ‘a slap’ < PKLD *pepag (cf. Kelabit \textit{pepag}). But here, again, there are other potential comparisons, including PWMP *epap ‘slap’, Proto-Kayan *ne-bip ‘slap’, Proto-Kenyan *tepap [AS], Bidayuh Serian \textit{nipaap, ñipap} ‘to slap’, \textit{tipaap} ‘slap’. Two additional comparisons appear to involve the same root: Sa’ban \textit{eləp} ‘to separate, divorce’ < *iləg (cf. Kelabit \textit{ileg} \mbox{/e-i-leg/} ‘was stopped by’) and Sa’ban \textit{ləp} ‘to stop, as working’ < *ələg (cf. Kelabit \textit{eleg} ‘cessation, divorce’). In these cases, reconstructions appear to be based on the Kelabit form alone. For ‘divorce, separate; stop, finish’, compare Dusun Witu \textit{ulep} ‘stop’, Seputan \textit{palop} ‘finish’, Sekapan \textit{kaʔarəp} ‘divorce’, and Pawe \textit{pələpək} ‘finish’, all consistent with derivations from *elep ‘stop, cease’. The only comparison set which shows a clear bilabial reflex of *-g is Sa’ban \textit{ajiəp} ‘rice sieve’ < *agag (cf. Kelabit \textit{agag} ‘sift’).\footnote{Two other cases from \citet[256]{Blust2005} may also be misanalyzed. In *rurug > \textit{hrop} ‘to fall, pour out’, *rurug is based on Kelabit \textit{rurug} ‘come out all at once, as when~pouring something that you want to come out slowly’ [BD]. However, a better root comparison for Sa’ban \textit{hrop} may be Kelabit \textit{ru’eb}, Lun Bawang \textit{rueb} ‘waterfall’. For *beluqug > \textit{belʔup} ‘wasp, hornet’, where *beluqug is based on Kelabit \textit{belu’ug} ‘wasp’s nest’, Kelabit \textit{belubpuk} ‘wood borers, all varieties of insects etc which eat wood, bamboo, etc.’ [BD], with stem-final \mbox{/b/} seems a better comparison.} In this case, a dissimilatory sound change is suggested taking \textit{gVg} > \textit{gVb}, followed by Blust’s other proposed regular sound changes: *agag > \textit{agab} > \textit{ajab} > \textit{ajap} > \textit{ajiep}. While this proposal may appear ad-hoc, ACD *agag is reconstructed to PWMP on the basis of Kelabit and Agutaynen alone. Of interest is that Agutaynen shows \textit{mag-agag} ‘to sift’, but also \textit{maŋ-agat} ‘to sift’, where final \mbox{/-t/} also appears to involve place-dissimilation. Compare also the doublet PWMP *qayag, *ayak ‘sift, separate by sifting’, which may reflect an earlier dissimilation of *gVg > \textit{yVg}. A final confusing factor are pairs where the seemingly bizarre correspondence goes in the opposite direction: Kelabit \textit{alap} ‘to get, fetch’ (cf. Malay \textit{alap}), but Sa’ban \textit{alak}, where the PAN doublet *alaq, *alap ‘fetch, get, take’ may be continued, and/or the Kelabit form may be a Malay loan.

In sum, though there are many unusual features of Sa’ban historical phonology \citep{Blust2001}, there does not appear to be a rare context-free shift of *-g > -\textit{b}. Rather, *g-, *-g shifted to \textit{j}-, -\textit{j}-, while final *-g devoiced to -\textit{k}, except in *gVg sequences, where sporadic dissimilation took *gVg > \textit{gVb}, with final devoicing of *-b > -\textit{p}.

\subsection{\label{sec:blevins:3.2}*an/aŋ >  {-\textit{ay}, } {*em/en/eŋ} >  {-\textit{aw} in Iban}}
Another sound change for which Blust’s own reconstructions and lexical comparisons do not support his characterization of a regular sound change is the shift of final non-high vowel sequences + nasals to vowel-glide sequences in Iban, a Malayic language of southwest Sarawak. Compare Iban \textit{jalay} ‘road’ with Malay \textit{jalan} < *jalan, Iban \textit{makay} ‘eat’ with Malay \textit{makan} < *ma-kan ‘eat’, and Iban \textit{diaw} ‘quiet, silent’ with Malay \textit{diam} < *hidem. According to \citet[257--258]{Blust2005}, final nasals show a surprising development in weakening to glides, though he notes that the process in Iban “…has not affected all potentially available forms.” This appears to be a gross understatement. In going through Iban etymologies in the ACD, there are many more exceptions than examples. Looking just at ACD entries for Iban headwords beginning with \mbox{/a/} and \mbox{/b/}, there are 24 exceptions to the proposed change, and only two forms that conform to the generalization. The exceptions are: \textit{acaŋ} ‘pigeon’ < *acaŋ; \textit{alaŋ} ‘beam’ < *qalaŋ; \textit{alaŋ-alaŋ} ‘insufficient’ < *alaŋ-alaŋ; \textit{amaŋ} ‘menace, threaten’ < *amaŋ amaŋ; \textit{anaŋ} ‘don’t’ < *anaŋ; \textit{ansaŋ} ‘gills of fish’ < *hasaŋ; \textit{antam} ‘strike’ < *qantem;  \textit{asam} ‘sour, acid’ < *qalesem; \textit{awan} ‘cloud’ < *hawan; \textit{bambaŋ} ‘large, broad’< *baŋbaŋ;  \textit{baraŋ} ‘any, about; whatever’ < *baraŋ; \textit{bataŋ} ‘tree, tree trunk’ < *bataŋ; \textit{bebaŋ} ‘stop, hinder, check’< *beŋbeŋ; \textit{belakaŋ} ‘back, rear, behind, after’< *balakaŋ; \textit{belaŋ} ‘Leucoderma’< *belaŋ; \textit{belian} ‘perform a rite of shamanistic curing or exorcism’ < *balian; \textit{bentaŋ} ‘rotan or cord stretched across a river with charms and offereings attached to it’ < *benteŋ; \textit{benuaŋ} ‘quick growing softwood riparian tree’< *benuaŋ; \textit{beraŋan} ‘small trees yielding a small chestnut’< *baraŋan; \textit{beruaŋ} ‘Malay bear’ < *baRuaŋ; \textit{bidaŋ} ‘numeral classifier for things spread out when in use’ < *bidaŋ; \textit{binuaŋ} ‘quick growing softwood riparian tree’< *binuaŋ; \textit{bulan} ‘moon, month’ < *bulan; \textit{bulaŋ} ‘cord with which an artificial spur is tied to a fighting cock’ < *bulaŋ. From this sample of several hundred Iban words, only two appear to undergo the sound change proposed by Blust: \textit{bembai} ‘rush or reed with fragrant white flower\textit{: Clinogyne dichotoma}, and \textit{Donax}, \textit{Cyperus} spp.’ < *benban; and \textit{bukai} ‘other, another’ < *buken. Even with a majority of words not undergoing the change, \citet[257]{Blust2005} notes that “the change clearly is recurrent, and the probability that it is a product of analogy or some other mechanism of secondary change is virtually nil.”

\newpage
\citet[193--196]{Smith2017} revisits the Iban data and describes the process as one of “diphthongization … a process where final *-a(C) became -\textit{ay} or -\textit{aw} in a number of lexemes”. In his comparison sets, he includes cases where a glide is seemingly accreted (e.g. Iban \textit{kitai} ‘1\textsc{pl.incl}; we, us’; cf. Malay \textit{kita} < *k-ita), as well as cases where the Iban glide corresponds to a coda rhotic or -\textit{s} in a few Malay words (e.g. Iban \textit{bəsay} ‘big’, cf. Malay \textit{bəsar} ‘big’; Iban \textit{ataw} ‘above’, cf. Malay \textit{atas} < *atas). Smith’s view is that diphthongization shows inconsistent application in Iban, and in Ibanic languages more generally, and for this reason, he excludes it as a sound change defining possible Malayic subgroups.

However, a careful reconsideration of the data suggests that there was a regular sound change, but that it was more specific than the changes suggested by Blust or Smith. The sound change appears to have been one of nasal-dissimilation as stated in \REF{ex:blevins:13}, where inherited word-final *-aN sequences (N a nasal) were denasalized when preceded by a nasal or nasalized segment: *N > [-nas]/N(C)a\_\#.

\ea%13
\label{ex:blevins:13}Nasal dissimilation in the history of Iban \\
*N > [-nas]/N(C)a\_\# (Does not apply in *CVC-CVC reduplicated forms)
\z

The hypothesis is that a phonetically nasalized sequence NãN was interpreted or perceived as NãG under the general theory of perceptual hypercorrection \citep{Ohala1981}.\footnote{The role of \mbox{/a/} vs. other vowels in this sound change merits further study, but is consistent with what is known of vowel nasalization: low vowels may undergo spontaneous nasalization, and low vowels are typically more nasalized than non-low vowels \citep[9]{Ruhlen1973}. On spontaneous nasalization associated with other segments, including aspirates and glides, see also \citet{Ohala1974, Ohala1975, Matisoff1975, BlevinsGarrett1993}, and \citet{Johnson2019}.} Denasalized [n], [ŋ] were realized as [j] and denasalized [m] was realized as [w]. Comparative data consistent with this sound change are shown in \tabref{tab:blevins:10}.

\begin{table}[t]
\begin{tabularx}{\textwidth}{QlQQQ}
\lsptoprule
 {{PWMP}} & {{Iban}} & {{Other Malayic}} & {{Other Borneo}} & {{Gloss}}\\
\midrule
*benban & bembay & bemban (Malay) &  & ‘herb used in making baskets’\\
\tablevspace
*ka-/beŋis-an & binsai [BD] & ka-beŋis-an (Malay) &  & ‘anger, cruelty/aggressive’\\
\tablevspace
*kembaŋ & kemay & kembaŋ  (Malay) &  & ‘swollen/expansion/swollen’\\
\tablevspace
*hinzam & injaw & m-injam (Kendayan) &  & ‘borrow/lend/borrow’\\
\tablevspace
*ñaman & ñamai [BD] & ñaman (Malay) & main (Lung Bawang/ BD) & ‘tasty, delicious’\\
\tablevspace
*deŋan & ŋaw & deŋan (Kendayan) & doŋam (Kadorih) & ‘with’\\
\tablevspace
*tilanzaŋ & telañay & telañjaŋ (Malay) &  & ‘naked’\\
\tablevspace
*qu(n)daŋ Proto-Malay & unay & undaŋ, (h)udaŋ (Malay) & hundaŋ (Ketapang) & ‘shrimp’\\
\lspbottomrule
\end{tabularx}
\caption{\label{tab:blevins:10}  Evidence for nasal dissimilation: *N > [-nas]/N(C)a\_\# (data from ACD except where noted otherwise)}
\end{table}

\largerpage[-1]
In some cases, the regular sound change formulated in \REF{ex:blevins:13} seems to apply to a phonetic form where the nasal trigger of dissimilation must be viewed as, itself, a consequence of allophonic (or phonetic) nasalization: in these cases a phonological nasal appears farther from the word-final nasal, for example, in word-initial position of a disyllable.\footnote{\citet[51]{Omar1969} describes nasal harmony in Iban as follows: vowels “are nasalized in the environment of preceding nasals; the nasalization of a vowel continues within the word until it is checked by a following consonant which is not a semi-vowel”, where, in his examples on p.~52, semi-vowels include [w], [j], and [l]. See \citet[239]{Blust2013} for a summary of segments transparent to nasal spreading in six Western Malayo-Polynesian languages.} For example, from Malayic *ŋətam (< Pan-Borneo *getem ‘harvest’), there is Seberuang \textit{ŋətam} and Iban \textit{ŋətaw}: the final glide in Iban is not predicted by \REF{ex:blevins:13}, unless the input to the sound change is a phonetic form like *[ŋəⁿtãm]. Phonetic forms of this kind are found in the modern language \citep{Blust1997} and are also suggested by phonologization in other languages of Borneo, as in Basap \textit{nanem} ‘harvest’.
\largerpage[-1]
A similar extension of \REF{ex:blevins:13} will account for Iban \textit{makay} ‘eat’ from *ma-kan ‘to eat’ (Malay \textit{makan}). It is even possible to consider that in words with no nasal consonants at all preceding the final nasal, phonetic nasal harmony across medial liquids or glides may give rise to triggers. Consider, for example, three reflexes of PMP *zalan ‘road’ in Borneo: Malayic Keninjal \textit{jalan}; Iban \textit{jalay}; and Mpraa \textit{ñalãə̯} < Proto-Kayanic *jalan. The Mpraa form suggests that nasalization can spread across a medial lateral. If this is the case, it is possible to understand Iban \textit{jalay} < *[jã\~{l}ãn], where the medial nasalized \mbox{/l/} triggers dissimilation.\footnote{In some languages of Borneo, the allophonic nasalization is phonologized. Compare Bidayuh Bau \textit{janaŋ} ‘beam’ with Iban \textit{alaŋ}, Malay \textit{alaŋ}.  In at least one case \citet[195]{Smith2017} may have misanalyzed an Iban form. He compares Indonesian \textit{ada} ‘exist’ (< PWMP *wada) to Mualang \textit{aday}, and we could add Iban \textit{aday}. However, in addition to PWMP *wada, the ACD shows PWMP *wadai-i based on Iban and Cebuano \textit{waray}.}

\begin{sloppypar}
A final observation with respect to what may have been a regular sound change regards a later dissimilatory change of Iban *yVy > *yVw. Two examples of this kind are:  Iban \textit{ayaw} < *ayay < *ayaŋ ‘shadow’ (cf. Seberuang \textit{bayaŋ}, Keninjal \textit{kemayaŋ} < PWMP *bayaŋ); and Iban \textit{sayaw} < *sayay < *sayaŋ ‘pity’ (cf. Ngaju Dayak \textit{ma-ñañaŋ}, PWMP *sayaŋ).
\end{sloppypar}

In addition to regular sound change, analogical change may have played an important role in the distribution of final glides in modern Iban. A productive source of nasal-initial stems is nasalization associated with the formation of transitive verbs, and denominal verbs. Under these derivational processes, allomorphs are \mbox{/ŋe-/}, zero, or nasalization of the initial consonant. By this last process \mbox{/p, b/} are replaced by \mbox{/m/}, \mbox{/t, d/} are replaced by \mbox{/n/}, \mbox{/k, g, ʔ/} are replaced by \mbox{/ŋ/}, and \mbox{/c, j, s/} are replaced by \mbox{/ɲ/} (\citealt{Omar1969}: 78, 82--85). This means that a historic form like *dataŋ ‘come’ (cf. Malay \textit{dataŋ}) would often have had at least two basic forms in pre-Iban: *dataŋ and *nataŋ. Nasal harmony in the second would yield allophonic *[nãⁿtãŋ], which then, could feed \REF{ex:blevins:13}, resulting in \mbox{/natay/}, which, by analogy with other pairs (\textit{dilat/nilat} ‘to lick/licks’, \textit{dedat/nedat} ‘to beat/beats’, etc.) would yield \textit{natay/datay}.\footnote{Lexicalization of nasal-prefixed forms of this particular stem is found elsewhere in Borneo. Compare Ambai Sembuak (Murutic) \textit{matoŋ} ‘arrive’ and Bulusu \textit{matoŋ} ‘come’ (< *d-um-ateŋ), both from the same stem.} A similar account is possible for Iban \textit{pulay} ‘return’ (cf. Malay \textit{pulaŋ} ‘return home’) where Iban has also \textit{mulay}-, as in \textit{mulayka} ‘return (tr.)’.

Another complicating factor in understanding this sound change is the influx of Malay loans into the Malayic languages of Borneo, as well as diffusion among Malayic languages of Borneo, as discussed by \citet[193--196]{Smith2017}.  Consider Iban \textit{kumaŋ} in \textit{kumaŋ meñaruŋ} ‘hermit crab’, (\textit{meñaruŋ} from \textit{saruŋ} ‘sheath, case, covering’). By the regular dissimilatory sound change \REF{ex:blevins:13}, we expect Iban **kumay < PWMP *kumaŋ ‘hermit crab’, not the attested form eding in -\textit{ŋ}. As far as I can tell, there are no other reflexes of this term for ‘hermit crab’ in Borneo. However, there is a term \textit{kumaŋ} ‘beetle’ in other Malayic languages of Borneo, including Kendayan, Keninjal, and Mualang. It is possible, then, that Iban \textit{kumaŋ} was not directly inherited.\footnote{Other Iban forms that are likely loans, and therefore fail to undergo the sound change in \REF{ex:blevins:13} are: \textit{amaŋ} ‘menace, threaten’ (cf. Malay \textit{amaŋ} ‘defiance’); \textit{anaŋ} ‘don’t’ (with no known cognates in Borneo); \textit{antam} ‘strike’ (cf. Malay \textit{hantam} ‘slamming, slapping, bumping against’); and \textit{bentaŋ} ‘rotan or cord stretched across a river with charms and offerings attached to it’ (cf. Makassarese \textit{bantaŋ} ‘stretched thread or cord’). None of these terms have known cognates in other languages of Borneo.  Other sporadic changes are in evidence as well. For example, the unexpected final glide in Iban \textit{tuay} ‘old, mature’ (cf. Malay \textit{tua} < *tuqah) is likely due to the influence of Iban \textit{tuay} ‘leader, head, chief, senior’ < *tuqay ‘leader in a group’.}

In sum, this proposal supports \citegen{Blust2005} intuition that a regular sound change underlies the word-final nasal-to-glide change in Iban: in nearly all inherited PWMP (non-reduplicated) words ending in *-N(C)aN\#, a final nasal is continued as a homorganic glide.

\subsection{\label{sec:blevins:3.3}C  > C: /{\longrule}V\# in Berawan}



The Sa’ban sound change discussed in \sectref{sec:blevins:3.1} is suspect, and Iban final VN\# > VG\# has been reanalyzed in \sectref{sec:blevins:3.2} as dissimilatory. In contrast, gemination of intervocalic consonants in inherited Berawan \#(C)VCV\# words is strongly supported by evidence like that in \tabref{tab:blevins:11}, as offered by \citet{Blust2005,Blust2016,Blust2018}. In \tabref{tab:blevins:11}, medial singleton consonants have undergone gemination in some instances; however, if the etymon has a final closed syllable there is no gemination.

\begin{table}[t]
\begin{tabularx}{.8\textwidth}{XXl}
\lsptoprule
{PMP} & {Long} {Terawan} & {Gloss}\\
\midrule
*asu & accoh   & `dog'\\
*aku & akkoh   & `1\textsc{sg} , I'\\
*bana & binneh & `husband'\\
*batu & bittoh & `stone'\\
*kali & kalléh & `dig'\\
*kutu & kuttoh & `head louse'\\
*laki & lakkéh & `man, male'\\
*siku & sikkoh & `elbow'\\
*batuk & bitoʔ & `neck'\\
*buluq & bulu  & `bamboo'\\
*likud & likon & `back'\\
*putiq & puté  & `white'\\
*qatay & atay  & `liver'\\
*tanaq & tana  & `earth'\\
*tukud & tukon & `prop, support'\\
*utaq & uta    & `vomit'\\
\lspbottomrule
\end{tabularx}
\caption{\label{tab:blevins:11}Berawan medial C gemination in *(C)V\_V\# \parencites[252]{Blust2005}[29--30]{Blust2018}}
\end{table}



However, \citet{Blust2005} was not satisfied with this description and sought to determine whether a phonetic motivation for this gemination could be found. Comparing this process with widespread gemination of consonants after schwa in Austronesian, Blust came up emptyhanded: if historical lengthening in Berawan was, essentially, the same type of syllable-bulking as post-schwa gemination, he reasoned, “…we would expect to find some languages that automatically geminate consonants after unstressed vowels other than schwa”, though he knew of none at the time. What continued to puzzle \citet[252]{Blust2005} was the fundamental phonetic basis of the process: “What linguistic factor, if any, might drive consonant onsets to geminate only if they initiate an open final syllable…”. 


Blust’s original assumption was that, because Berawan stress is consistently word-final in the modern language, it was final at the time gemination applied. However, \citet{Blust2018}, revisits Berawan gemination with a refreshing take on historical prosody. There, Blust suggests that, perhaps, the process of medial gemination was prosodically conditioned after all. Assuming a trochaic [sw] stress pattern at the time gemination took place, the initial open syllable would geminate in the strong position of the foot, -- a case of post-tonic gemination. Fleshing out this analysis, a general word stress algorithm can be suggested for \mbox{(pre-)}Berawan: stress a final syllable if it is heavy (VV, GC, VC); otherwise, stress the penultimate syllable. With this stress pattern in place, allophonic gemination occurred as a consequence of the universal tendency for stressed syllables to be heavy (the “stress-to-weight” condition). The historical sequence is illustrated in \REF{ex:blevins:14} with a prosodic minimal pair.

\largerpage
\ea%14
Tonic gemination in pre-Berawan\label{ex:blevins:14}
    \ea \TabPositions{0pt,.5\linewidth,.8\linewidth}
        Pre-Berawan 1         \tab\relax  *batu ‘stone’ \tab\relax   *batuk ‘neck’\\
        \ea \TabPositions{0pt,.5\linewidth,.8\linewidth}
            Stress rule\footnote{Stress the final syllable if heavy; otherwise, stress the penultimate.}  
                             \tab\relax *bátu   \tab\relax   *batúk    \\
        \ex Stress-to-weight \tab\relax *báttu  \tab\relax   *batúk     \\
        \z
    \ex 
        Pre-Berawan 2\\
        \ea \TabPositions{0pt,.5\linewidth,.8\linewidth}
            Final stress  \tab\relax  *battú             \tab\relax  *batúk\\
        \ex Other changes \tab  \textit{bittóh}    \tab   \textit{bitóʔ}\\
        \z
    \z
\z

While \citet{Blust2018} seemed open to considering this possibility, he remained skeptical for one central reason. If such a historical process were truly natural, why did it not occur elsewhere in the language family, or in other language families? For the moment, let us consider the conditions under which tonic gemination does apply, both within Austronesian and elsewhere. Phonetic gemination in Austronesian languages is widespread in one particular context: after schwa in open syllables. Languages showing this sound change (with subsequent phonologization of geminates after vowel mergers) include: Buginese; Dupaningan Agta \citep{Robinson2008}; Isneg; Guinaang Bontok; Kelabit; Konjo; Makasarese; Sangir; Sri Lankan Malay \citep{Adelaar1991}; and Talaud (northern Sulawesi). However, there are also many languages where this process is simply allophonic and non\hyp neutralizing. For example \citet[133]{Blust1995} notes that in Long Terawan Berawan itself, “consonants are geminated after schwa in oxytone citation forms,” while similar sound patterns are found far and wide, as, for example in Hawu where all consonants are geminated after stressed schwa (\citealt{Walker1982, Blust2008,Blust2012}). As for whether there are languages (outside Austronesian) with tonic gemination after all vowels, the answer appears to be yes. One of these is Kugu Nganhcara, a Pama-Nyungan language of Australia, as described by \citet{SmithJohnson2000}. In this language, with consonants /p, t, t̪ , c, k, b, d, d̪ , j, g, m, n, n̪ , ɲ, ŋ, l, r, w, j/, all consonants except the tap, \mbox{/r/}, have geminate allophones optionally in intervocalic position following a short initial stressed syllable. Another language with predictable consonant length after stressed vowels is Karuk as described by \citet{Bright1957}, where in certain prosodic positions (e.g. after long accented vowels), consonant length (less than a true geminate) is allophonic \citep{Blevins2005b}. Given the general view that post-tonic consonant gemination serves to make stressed syllables longer (\citealt{Gordon2002,Gordon2004, Blevins2005b}), if this kind of lengthening is not taking place in many Austronesian languages, it is likely because tonic  vowels are already long enough to give necessary weight to the syllable. In contrast, a working hypothesis is that the initial vowels of CVCV disyllables in pre-Berawan were significantly shorter than in other Austronesian languages. We return to this issue in \sectref{sec:blevins:4}.

\subsection{\label{sec:blevins:3.4}*dr  > \textit{kʰ} in Drehet}
In \sectref{sec:blevins:2.1}, the apparent one-step merger of Proto-Oceanic *-w, *-y to -\textit{p} in Drehet and Levei, two dialects of Khehek of Manus Island, was shown to be amenable to reanalysis as a sequence of two phonetically motivated changes. Khehek was also singled out as having a bizarre context-free change of an inherited word-initial trill. An additional sound change took the Proto Admiralties pre-nasalized alveolar trill *dr- (a continuation of the same sound in Proto Oceanic) to a “strongly aspirated” velar stop \mbox{/kʰ/} in Drehet, corresponding to \mbox{/c/} a voiceless post-alveolar/palatal affricate in Levei.\footnote{\citet[246]{Blust2005} describes the Levei /c/ as “a voiceless palatal affricate”, though, by this, in many of his descriptions, he seems to often mean IPA [t͡ʃ], the voiceless post-alveolar affricate \citep[601]{Blust2013}. Since \citet{Beard1992}\slash SIL transcribe the Levei \mbox{/c/} as [t͡ʃ], I assume this is what Blust means.} Correspondences are shown in \tabref{tab:blevins:12}, where Drehet and Levei data is from \citet[247]{Blust2005} and Nali forms are from the ACD. Note that Nali, a language of eastern Manus, continues the pre-nasalized trill without change.\footnote{Some Proto-Admiralties *dr- continue POC *dr-, and others are innovations. Outside of the Admiralties, POC *dr- is continued as a post-alveolar affricate in Trukic languages.}

\begin{table}
\fittable{%
\begin{tabular}{llllll}
\lsptoprule
{POC} & {Proto-Manus} & {Nali} & {Drehet} & {Levei} & {Gloss}\\
\midrule
*drali & *dranV &  & kʰaŋ & coŋ & ‘slitgong’\\
*draRaq & *dra & dray & kʰa & ca & ‘blood’\\
*na topu & *druhu & druh & kʰuh & cuh & ‘sugarcane’\\
*ruRi & *drui- & drui- & kʰui- & cui- & ‘bone’\\
*ruyuŋ & *druyu &  & kʰu & cu & ‘dugong’\\
*tapuRi & *drapui & drah & kʰah & coh & ‘conch shell’ \\
*tau-mata & *dramata &  & kʰamak & camok & ‘person, human being’\\
*t<in>aqi & *drine- & drina- & kʰini & cini & ‘intestines’\\
*tokon & *droko &  & kʰo & co & ‘punting pole’\\
*tola & *drolV & droy & kʰoŋ & coŋ & ‘outrigger canoe’\\
(*turuR) & *droV &  & kʰep & cep & ‘to sleep/sleeping mat’\\
*tuRu & *dru & dru & kʰu & cu & ‘housepost\\
\lspbottomrule
\end{tabular}}
\caption{\label{tab:blevins:12}Alveolar, velar and palatal reflexes of the Proto-Manus pre-nasalized alveolar trill (all forms but Nali from \citealt[247]{Blust2005}; Nali from ACD)}
\end{table}

Before exploring the phonetic basis of this change, it is important to note that Blust’s “strongly aspirated” velar stop is written by \citet{Beard1992} as \mbox{/kx/} and described as a “voiceless velar affricate”, written <kh> in the SIL orthography. Two other Admiralties languages have velar reflexes of Proto Manus *dr-: Seimat \textit{k}- and Wuvulu \textit{x}-, as in Seimat \textit{kawa}-, Wuvulu \textit{xawa}- ‘forehead’ < POC *dramʷa. These reflexes support [kx] as the more conservative phone in Drehet, as does comparative data from unrelated languages shown in \tabref{tab:blevins:13}.

\begin{table}
\begin{tabularx}{\textwidth}{lXlXl}
\lsptoprule
{Proto-Tibeto-Burman} & {Khezha} & {Mongsen} {Ao} & {Sumi} & {Gloss}\\
\midrule
*rus {\textasciitilde} *rew & è-ru & [tə]-ɹət & à-ɣì & ‘bone’\\
*ran & è-ri &  & a-ɣi & ‘war’\\
*rey {\textasciitilde} *rwi & è-rü{}  & [a]-hɹə & à-ɣìyí & ‘rattan/cane’\\
*d-ruk & sàrü & təɹuk & tsɨɣò & ‘six’\\
*ruk & rho &  & xo & ‘pick (fruit)/pluck’\\
*s-rik & è-rhi & [a]-tshək & à-xɨ & ‘head louse’\\
\lspbottomrule
\end{tabularx}
  \caption{\label{tab:blevins:13} Alveolar trill to velar fricative in Sumi \citep[52--53]{Teo2009}}
\end{table}

For example, in Sumi, a Tibeto-Burman language of Nagaland, the velar fricative \mbox{/x/} corresponds to the voiceless alveolar trill \mbox{/rh/} in related languages, while the velar fricative \mbox{/ɣ/} corresponds to \mbox{/r/} \citep[52--53]{Teo2009}.

Blust is explicit in questioning the phonetic basis of this sound change, but at the same time, recognizes that parallel shifts of *dr- to palatals and velars are suggestive of naturalness:

\begin{quote}
What kind of phonetic or phonological bridge can be built between a prenasalized alveolar trill and either a voiceless palatal affricate or a voiceless aspirated velar stop? … If there were phonetically intermediate steps in the transition from PM *dr to \textit{c} and \textit{kh}, they are difficult to infer. A development *dr > \textit{c} > \textit{kh} is perhaps favored by the fact that several other Admiralty languages, including Likum, in southwest Manus, and Nauna in the eastern Admiralties, also reflect PM *dr as \textit{c}. However, Seimat, spoken in the Ninigo Lagoon some 270 km west of Manus, reflects *dr as \textit{k} or \textit{x} …This parallelism, between historically independent changes of *dr to a voiceless palatal affricate in some languages and a voiceless velar stop or fricative in others, suggests that the change paths leading from *dr to its reflexes were constrained by some type of linguistic motivation, although what this motivation might be remains completely obscure \citep[248]{Blust2005}.
\end{quote}

In this context, work on the acoustic and aerodynamic properties of alveolar trills has uncovered certain natural tendencies in these speech sounds which are illuminating. First, trills have a natural tendency to devoice, since aerodynamic requirements to both sustain tongue-tip trilling and vocal fold vibration fall within narrow limits \citep{Solé2002}. Second, trills have a tendency to produce frication, or become fricatives when oropharyngeal pressure and subglottal pressure are reduced below a certain threshold \citep[682--684]{Solé2002}. There is also evidence for perceptual similarity between trills and fricatives (\citealt{LadefogedMaddieson1996}: 241). Finally, alveolar trills exhibit more predorsum lowering and postdorsum retraction than taps, with more retracted alveolar closure (\citealt{RecasensPallarès1999}), and, like other rhotics, often give rise to retraction of adjacent coronal stops \citep{SmithEtAl2019}.

Given these phonetic tendencies, the context-free shift of pre-nasalized trill to palatal, alvopalatal, or velar affricate or fricative might look like the sequence of changes outlined in \tabref{tab:ex:blevins:15}, where Drehet-1 is the variety described by \citet{Beard1992}, and Drehet-2 the variety described by Blust. Levei, Likum, Nauna, and Pelipowai are listed twice, as they may show variation between palatal and alveopalatal articulations.

\begin{table}%15
\caption{Trill devoicing, retraction, and frication in some Admiralties languages}
\label{tab:ex:blevins:15}
\small
\begin{tabularx}{\textwidth}{X@{~}c@{~}c@{~}c@{~}c@{~}c@{~}c@{~}cl}
 \lsptoprule
         &    1     &  2   &    3                 & \{4a  &  4b\} &  \{5a   &  5b\}\\\midrule
  *[ⁿᵈr] & > [ᵗr̥] > & [t̠r̥] & > [t̠ç]  \phantom{>}  &       &       &         &     &   (Levei, Likum, Nauna, Pelipowai)\\
  *[ⁿᵈr] & > [ᵗr̥] > & [t̠r̥] & > [t̠ç]  >            & \phantom{\{}[t͡ʃ]  &       &         &     &    Levei, Likum, Nauna, Pelipowai\\
  *[ⁿᵈr] & > [ᵗr̥] > & [t̠r̥] & > [t̠ç]  >            &       & [k͡x] \phantom{>}  &         &     &   Drehet-1\\
  *[ⁿᵈr] & > [ᵗr̥] > & [t̠r̥] & > [t̠ç]  >            &       & [k͡x] >            &   [kʰ] &     &   Drehet-2, Seimat\\
  *[ⁿᵈr] & > [ᵗr̥] > & [t̠r̥] & > [t̠ç]  >            &       & [k͡x] >            &        & [x] &   Wuvulu\footnote{Loss of the closure phase of the original pre-stopped trill may have occurred earlier.}\\
  \lspbottomrule
\end{tabularx}
\end{table}

In stage 1, the trill devoices with associated loss of pre-nasalization (pre\hyp nasalization is a feature of the voiced consonant series in these languages). In stage 2, the initial stop portion undergoes retraction triggered by the trilled rhotic. In stage 3, the trill is strengthened to a fricative: while this is transcribed as a palatal fricative [ç] it could be a complex sound involving tongue tip retraction and/or tongue root retraction. The output of stage 3 is unstable, involving a sequence of tongue blade-body articulations that are expected to coarticulate, merge and/or undergo perceptual assimilation. Change 4a shows the shift of this unstable articulation to the palato-alveolar affricate, which is already a phoneme in these languages, and hence, may act as a perceptual magnet, or an articulatory basin of attraction. In contrast, step 4b shows the shift of the same unstable articulation from palatal to velar: whether this is related to the loss of initial *k- in Drehet, Seimat and Wuvulu is unclear. In stage 5a, the fricated release of the velar affricate is weakened to aspiration, while stage 5b shows simplification of the velar affricate to a velar fricative. Only by invoking a phonetic explanation for this context-free change are we able to explain how Proto-Oceanic *dr- came to be realized as [t͡ʃ] in four languages of the Admiralties, as well as in several Chuukic (Trukic) languages, as well as the three languages with velar reflexes of the same sound, keeping in mind that, within the Admiralities, Levei and Drehet are two dialects of Khehet, and that Wuvulu and Seimat are spoken on remote islands over 400 km from Manus island.

\subsection{*w/b > \textit{c-, -nc}- in Sundanese}
\label{sec:blevins:3.5}
The final rare sound change explored in this study is an apparent labial to palatal shift in Sundanese, SC10 of \tabref{tab:blevins:1}, repeated in \REF{ex:blevins:16} with separation of word-initial \REF{ex:blevins:16a} and word-medial \REF{ex:blevins:16b} contexts. Sundanese is a Malayo-Polynesian language native to western Java, spoken today by over 40 million people, and thought to be closely related to the Malayic languages. Unless noted otherwise, all Austronesian data in this section is from the ACD.

\ea%16
    \label{ex:blevins:16}
    Sundanese labial to palatal glide fortition
  \ea \label{ex:blevins:16a}
  *w-, *b- > \textit{c}-
  \ex \label{ex:blevins:16b}
  *-w-, *-b- > -\textit{nc}-
  \z
\z

Correspondences supporting these seemingly bizarre sound changes are given in \tabref{tab:blevins:14}, arranged alphabetically by Sundanese lexeme, with data from \citet{Blust2005} updated to offer every known example of the correspondence, current etymologies, additional comparanda, and a first column with schematic environments related to the discussion that follows.\footnote{One set from \citet{Blust2005} that is not included here is Sundanese \textit{kenca} (Low), \textit{kiwa} (High) ‘left side’ < PMP *kiwa. The ACD proto-form is *ka-wiRi ‘left side or direction’, with the Sundanese form relegated to the “near comparisons” section under ‘left (not right)’, suggesting that a reconstruction is not possible.}  Where data is not from \citet{Blust2005, Blust2018} it is from the ACD.

Blust's most recent published notes on the observed data take a similar position to his 2005 view:

\begin{quote}
… *w, *b and *mb are recurrently reflected as \textit{c}-, -\textit{nc}- ([ʧ], [nʧ]) in Sundanese, a major Austronesian language of west Java that has been studied by linguists for well over a century. To overcome the featural absurdity of such an apparent change one might speculate that this situation came about through a series of innovations *b > *w > *y > \textit{c}-, -\textit{nc}-. However, since *y remained unchanged, and *-mb- became -nc- in at least two known cases it quickly becomes difficult to find a way to make this work. With enough persistence and freewheeling imagination, some contrived explanation can no doubt be found. But my point in trying to raise awareness about the importance of such "unnatural" cases is that an acceptable explanation (i.e. one based on phonetic motivation) can usually be salvaged only by proposing a series of steps for which no evidence exists. \citep{Blust2018}
\end{quote}

The questions we address here are: (i) Is there a reasonable, non-contrived explanation for the observed sound change in precisely the environment where it is found? (ii) If a series of steps are involved, is there evidence for these intermediate steps? However, before doing this we review several aspects of the data which, to my knowledge, have not been observed earlier, and which may be important to understanding phonetic motivations behind the change.

\begin{sidewaystable}
\small
\begin{tabularx}{\textwidth}{llllll}
\lsptoprule
{Context} & {PMP} & {Sundanese} & {Malay} & {Other} {WMP} & {Gloss}\\
\midrule
\_a…aH & *\textbf{b}adas & \textbf{c}adas &  & \textbf{w}aɖas (Javanese) & ‘grit, gravel’\\
\_aHY & *\textbf{w}ahir & \textbf{c}ai & air & \textbf{v}aig (Kadazan Dusun) & ‘water, river’\\
\_a..YN & *\textbf{b}aliŋbiŋ & \textbf{c}aliŋciŋ & \textbf{b}elimbiŋ &  & ‘starfruit/tree with sour fruit’\\
\_aNY &  & \textbf{c}anir & \textbf{b}anir & *\textbf{b}aŋil  (Proto-Philippines) & ‘buttress root/wedge, prop’\\
\_aN & \textbf{*ba}ŋkudu & \textbf{c}aŋkudu & \textbf{b}eŋkudu & \textbf{w}aŋkuɖu   (Old Javanese) & ‘\textit{Morinda} spp.’\\
\_a..N & *\textbf{b}ataŋ & \textbf{c}ataŋ & \textbf{b}ataŋ & \textbf{w}ataŋ   (Old Javanese) & ‘tree trunk, fallen log’\\
\_a…YN &  & \textbf{c}ariŋin & \textbf{b}eriŋin\footnotemark{} &  & ‘banyan, fig tree’\\
&  & \textbf{c}aruluk & \textbf{b}elulok\footnotemark{} &  & ‘fruit of the sugar palm’\\
\_a…N &  & \textbf{c}auŋ & \textbf{b}auŋ\footnotemark{} & \textbf{b}auŋk (Bidayuh Bau) & ‘catfish’\\
& *b-in-ahi & (see below) & \textbf{b}ini &  & ‘woman’\\
\_a…YNaHY & *\textbf{b}a-b-in-ahi & \textbf{c}a-wene &  & \textbf{b}awine (Sangir) & ‘woman/virgin’\\
\_aY & *\textbf{b}ayaD & \textbf{c}aya & \textbf{b}ayar & \textbf{w}aer  (Tontemboan) & ‘pay, compensate’\\
\_aYa & *\textbf{b}ayuR & taŋkal \textbf{c}ayur & \textbf{b}ayur & \textbf{w}ajur  (Manggarai) & ‘ko timber tree’\\
\_aHaH & *\textbf{b}ahaq & \textbf{c}aʔah & \textbf{b}ah & \textbf{w}ah, \textbf{w}aah  (Old Javanese) & ‘floodwaters’\\
a\_aH & (*ka\textbf{w}aq) & ka\textbf{nc}ah & ka\textbf{w}ah & ka\textbf{v}aʔ  (Kadazan Dusun) & ‘vat, cauldron’\\
N\_a &  & katu\textbf{nc}ar & ketu\textbf{mb}ar &  & ‘coriander seed (Sanskrit loan)’\\
a\_aH & *la\textbf{w}aq & la\textbf{nc}ah &  & paŋga-la\textbf{f}a  (Tajio) & ‘spider’\\
\_a\_aN & *la\textbf{b}an & la\textbf{nc}an & la\textbf{w}an & la\textbf{w}an (Old Javanese) & ‘oppose/ opponent’\\
N\_aN &  & li\textbf{nc}aŋ & le\textbf{mb}aŋ\footnotemark{} &  & ‘swollen with water’\\
a\_aH & (*ra\textbf{w}aq) & ra\textbf{nc}a & ra\textbf{w}a(h) & ɾa\textbf{w}aŋ (Kadorih) & ‘swamp, morass/ lake’\\
Ha\_a & *sa\textbf{w}a & sa\textbf{nc}a & sa\textbf{w}a & sa\textbf{b}a (Mentawai) & ‘python’\\
\lspbottomrule
\end{tabularx}

\footnote{ Compare PWMP *biRiŋ ‘running sore’, *biRiŋ-en ‘have a running sore’, with possible reference to fig milk, and/or irritation caused by fig milk.}
\footnote{ Compare PWMP *balulaŋ ‘thick or hard skin’, with possible reference to outer hard skin of sugar palm fruit.}
\footnote{ Compare PWMP *bauk ‘chin whiskers’, with possible reference to fish with chin whiskers.}
\footnote{Compare PWMP *la(m)baŋ ‘to beyond, go over, go past’, *labeŋ ‘abundance, surplus’.}

\caption{\label{tab:blevins:14} *w, *b > c-, -nc- in Sundanese where N is nasal, H is an aspirate and Y is a palatal (data from \citealt{Blust2005,Blust2018}; ACD)}
\end{sidewaystable}

The first noticeable feature of the reconstructed forms in \tabref{tab:blevins:14} is that the target labial is always followed by *a, the low vowel. Recall from our discussion in \sectref{sec:blevins:3.2} of nasal dissimilation in Iban that a similar condition occurred. In the case of Iban, it was suggested (see footnote 17) that the presence of a low vowel could be associated with enhanced nasalization.

Since spontaneous nasalization occurs intervocalically in this data as well (e.g. \textit{sanca} < *sawa), the vocalic context may be relevant. This observation is supported by potential near-minimal pairs where palatalization has not taken place in Sundanese. For example, compare \textbf{c}ataŋ < *bataŋ, but \textit{bilaŋ} < *bilaŋ ‘count’, where *b is followed by *i. The distribution and significance of *a is highlighted in \tabref{tab:blevins:14} in the first column, in the schematic environment of the sound change. In addition to the low-vowel context, all proto-forms in \tabref{tab:blevins:14} include a nasal segment, an aspirate (*h or *s), or a palatal vowel or glide (*y), abbreviated as N, H, and Y respectively in the schemata of the first column. These observations lead to a new question: what do nasalization, aspiration and potential palatalization have to do with the apparent rare shift of  *w, *b > \textit{c}-, -\textit{nc}- in Sundanese? In order to address this question, let’s first step back and look at labial palatalization as it occurs in other languages.

To date, the most comprehensive study of labial palatalization\footnote{In this discussion, “labial palatalization” refers to a major place of articulation shift, from labial to palatal, not to secondary palatalization of labial sounds.} is \citet{Bateman2010}. Typological evidence as well as detailed diachronic studies of three distinct cases of apparent labial to palatal shifts in Moldavian, Tswana (as well as related Romance and Bantu languages), and variation in modern Polish \citep{Kochetov1998}, strongly support a surprising result: apparent full palatalization (and delabialization) of labials is the result of a series of sound changes that do not target the labial itself. Rather, full labial palatalization arises due to hardening of a palatal glide that followed the labial, with subsequent deletion of the labial. The general pathway of labial palatalization is illustrated in \tabref{tab:ex:blevins:17}: after labial deletion, palatal approximants or fricatives phonologize as palato-alveolar fricatives or affricates, in particular when these sounds pre-exist in the language.

\begin{table}%17
\caption{Labial palatalization as glide hardening + labial deletion \citep{Bateman2010}}
\label{tab:ex:blevins:17}
\begin{tabularx}{\textwidth}{l lX lX lX ll }
\lsptoprule
  Palatalization           & p > & pj  & f > & fj  & b > & bj  & v > & vj\\
  Voicing assimilation     &     & pɕ  &     & fɕ  &     & bʝ  &     & vʝ\\
  Labial deletion          &     & ɕ   &     & ɕ   &     & ʝ   &     & ʝ \\
  (Possible strengthening) &     & t͡ʃ  &     & t͡ʃ  &     & d͡ʒ  &     & d͡ʒ\\
\lspbottomrule
\end{tabularx}
\end{table}

This broad outline of labial palatalization may provide a better understanding of probable phonetic developments in Sundanese. Let us begin with Blust’s first hypothesis which is the suggestion that the apparent instances of *w < *b in Sundanese are due to borrowing from Javanese, where an apparent unconditioned split of *b into \mbox{/b/}, \mbox{/w/} occurred \citep[239]{Blust2005}.\footnote{\citet[239]{Blust2005} discards this hypothesis on the basis of the Sundanese cognates with \mbox{-\textit{nc}-,} saying: “The major obstacle … is the two examples which Nothofer gives of Malay forms with -\textit{mb}- corresponding to apparent Sundanese cognates with -\textit{nc}-. Since the change *b > \textit{w} following a nasal is unattested in Austronesian languages it appears necessary to assume that in at least these two forms the cluster *-mb- changed directly to -\textit{nc}-.” However, one form is a Sanskrit loan, Sundanese \textit{katuncar} (Malay \textit{ketumbar}) ‘coriander seed’, from an Indic form (cf. Prakrit \textit{kutthuṁbharī}~) with an \emph{anusvara \textup{nasal (syllable-final nasal glide), involving nasalization of the preceding vowel preceding the labial stop, while the other, Sundanese} }\textit{lincaŋ} (Malay \textit{lembaŋ}) ‘swollen with water’, appears to derive from a stem with a doublet *labeŋ ‘abundance, surplus’ (cf. Kayan \textit{laveŋ,} Maranao \textit{labeŋ}), allowing for the possibility of a -nc- < *w here as well. In all other cases, \emph{\textup{-}nc\textup{- in Sundanese corresponds to Old Javanese \mbox{/w/} < *b, as hypothesized.} }} If this is the case, we can rewrite the rare sound change of interest as in \REF{ex:blevins:18}.

\ea%18
    Sundanese labial to palatal glide fortition\\\label{ex:blevins:18}
    \ea *w- > \textit{c}-
    \ex *-w-  > -\textit{nc}-
    \z
\z

Given the trajectories of labial palatalization in \tabref{tab:ex:blevins:17}, we suggest that pre-Sundanese *w (from Javanese borrowings) was nasalized and aspirated to *w̥̃, and subsequently strengthened to a palatalized labial fricative, with subsequent labial deletion, as shown in \tabref{tab:ex:blevins:19}.

\begin{table}%19
\caption{Labial palatalization of *w in Sundanese}
\label{tab:ex:blevins:19}
\begin{tabularx}{\textwidth}{l l@{\qquad\qquad} XXXl}
\lsptoprule
Stage &  Process                & \multicolumn{4}{c}{Context}\\\cmidrule(lr){3-6}
      &                         &                              &      &   V{\longrule}V                     &  \#{\longrule}\\\midrule
1     & Nasalization/aspiration &  *w                          &  >   & *w̥̃                           &  *w̥̃\\
2     & Glide strengthening-I   &  *w̥̃                          &  >   & *\textsuperscript{\~{h}}fʲ   &  *fʲ\\
3     & Glide strengthening-II  &  *\textsuperscript{\~{h}}fʲ  &  >   & *\textsuperscript{\~{h}}fɕ   &  *fɕ\\
4     & Labial Deletion         &  *\textsuperscript{\~{h}}fɕ  &  >   & *\textsuperscript{\~{h}}ɕ    &  *ɕ\\
5     & Affrication             &  *\textsuperscript{\~{h}}ɕ   &  >   & *\textsuperscript{\~{h}}t͡ʃ   &  t͡ʃ\\
6     & Nasal strengthening     &  *\textsuperscript{\~{h}}t͡ʃ  &  >   &  nt͡ʃ\\
\lspbottomrule
\end{tabularx}
\end{table}

In stage 1, nasalization and aspiration are induced by ambient nasalization (from nasal consonants) or from spontaneous nasalization associated with low vowels and/or aspirates and glides (see footnote 17). This voiceless nasalized segment is then strengthened to a fricative in stage 2: since frication and nasalization are generally incompatible \citep{WarnerEtAl2015}, the initial labial fricative loses nasalization, while nasalization is maintained in pre-aspiration of the medial fricative. Stage 2 also shows a decoupling or linearization of the labial and high tongue body gestures of the labial glide: labialization is realized during the fricative stricture phase, while high tongue body is realized at release in the form of a voiceless palatal offglide.\footnote{An alternative way of conceptualizing this is in terms of (mis)perception, as opposed to articulation, - a case of \textit{chance} in Evolutionary Phonology \citep[32]{Blevins2004}. The height feature is associated with [j], the labial feature with [f], and spread glottis with the entire (complex) segment. A potential phonological parallel is the widespread palatalization of labials before \mbox{/w/} in Southern Bantu languages as recently summarized in \citet[2--4]{BennettBraver2020}, where the labialization of \mbox{/w/} is attributed by the listener to the preceding labial. Within the typology of \citet{Ohala1993}, all of these could be treated as cases of hypercorrection.}  Subsequent changes follow the widely attested trajectories documented by \citet{Bateman2010}: strengthening of the off-glide in stage 3; labial deletion in stage 4; and affrication in stage 5. Nasal strengthening in stage 6 is of special interest: recall from footnote 31 that it occurs not only in inherited forms, but in at least one clear Indic borrowing, where an \emph{anusvara} nasal, is realized as a nasal stop.

Labial palatalization is relatively uncommon, since palatalized labials often persist as sequences of independent articulatory gestures \citep{Bateman2010}, and since glide strengthening (as opposed to weakening), is relatively less common (\citealt{BybeeEasterday2019}). Spontaneous nasalization is also uncommon, and the apparent combination of inherent nasal, aspirate, low-vowel, and glide-induced effects in the history of Sundanese is especially notable. Finally, the influx of Indic terms with \textit{anusvara} nasals may have played a role as well, with ambient vowel nasalization phonologized in loans as pre-nasalized stops. In sum, the unique status of the Sundanese sound change in \REF{ex:blevins:18} can be understood as a rare confluence of all of these factors.

\section{\label{sec:blevins:4}Rare sound change and language contact}
Of the ten sound changes in \tabref{tab:blevins:1} classified as “bizarre”, most, as argued above, can be seen as phonetically natural: Sundanese labial palatalization (SC10); intervocalic devoicing in Berawan and Kiput (SC4); *dr > \textit{k\textsuperscript{x}} in Drehet (SC9); \mbox{*-b/d/g > \textit{ᵐb/ⁿd/ᵑg} > -\textit{m/n/ŋ}} in Berawan and Karo Batak (SC2); tonic gemination in Berawan (SC8); Berawan *-b- > *-g-, (SC3) which telescopes weakening of *b > \textit{w}, and fortition of *w > \textit{g}; and the Iban shift of final nasals to glides (SC7), reinterpreted as nasal dissimilation \REF{ex:blevins:13}. The neutralzation of Khehek *-y > *-w (SC1), and the development of post-nasal devoicing in Murik and Buginese (SC5) appear to involve non-phonetic factors: in the first case, pre-Khehek final consonant distribution appears to play a role in both glide-backing, and, later, in filling the \mbox{/p/} gap; in the development of post-nasal devoicing, the absence of NT clusters may have allowed ND to vary in pronunciation, giving rise to voiceless variants which dominated over time. Finally, one of the bizarre changes listed, Sa’ban *g > \textit{p}-, -\textit{p} (SC6), does not appear to be a true sound change (or sequence of sound changes) at all.

In the cases of rare developments, where multiple sound changes are involved, rarity may reflect the low frequency of the individual changes that co-occur.\footnote{See \citet{Beguš2020} on estimating the probability of natural and unnatural sound change.}

It is also possible that some of these developments are not as rare as they might seem. For example, if intervocalic devoicing in Berawan and Kiput are analyzable as phonetically natural instances of lengthening in a strong prosodic position as suggested in \sectref{sec:blevins:2.3}, these sound changes are unremarkable. The same is true of Berawan gemination: if the analysis in \sectref{sec:blevins:3.3} of gemination as weight-to-stress is correct, it aligns with many similar cases of tonic lengthening and/or gemination in the world’s languages. Blust’s (\citeyear*{Blust2005,Blust2018}) concern in both of these cases was the rarity of the sound pattern within Austronesian: if such sound changes were possible, why were they restricted to just two languages of over 500 Austronesian languages having both open and closed syllables?\footnote{Many Austronesian languages lack closed syllables. In languages without closed syllables, both post-tonic gemination and weight-to-stress will be absent by definition.}

\hspace*{-2.5pt}Five languages in this study are spoken in Borneo: Berawan, Iban, Kiput, Sa’ban, and Murik. Within the Austronesian language family, the languages of Borneo are widely recognized as being phonologically and lexically divergent, with a range of properties that are uncommon elsewhere within the family. In the realm of segmental contrasts, contrastive palatal consonants are found, final nasals may be pre-ploded, medial nasals may be post-ploded, and rare laryngeal series, like the voiced aspirates described for several dialects of Kelabit \citep{Blust2006}, the voiceless sonorants described for the Sa’ban dialect of Kelabit, and the implosives of Bintulu are attested (\citealt{Blust2013}:67, 182, 184--185). Even more remarkable are the word-level properties which make some languages of Borneo almost unrecognizable as Austronesian stock. These include variability and neutralization of vowel quality in nonfinal syllables, and bulking of final syllables, with a shift towards iambic or even monosyllabic words. These word-level properties can be illustrated with some of the Borneo continuations of PAN *asu ‘dog’ shown in \tabref{tab:blevins:15}.

While the final syllable remains light in the majority of Austronesian languages (cf. Bunun \textit{asu}, Siraya \textit{asu}; Ilokano \textit{áso}; Malaweg \textit{asú}; Toba Batak \textit{asu}; Kambera \textit{ahu}), the languages of Borneo are clearly different, with a range of distinct changes yielding a light-heavy or monosyllabic heavy syllable pattern. In this way, the languages of Borneo are similar to Chamic languages, where parallel shifts towards iambic and monosyllabic word types have been documented in detail by \citet{Thurgood1999} and attributed, in large part, to intense contact with Mon-Khmer languages in mainland Southeast Asia. For PAN *asu ‘dog’, compare the Borneo continuations in \tabref{tab:blevins:15} with Jarai \textit{asəu}, W. Cham \textit{saw}, and Wr. Cham \textit{asuŋ, suw}, all from proto Chamic *ʔasɔw \citep[281]{Thurgood1999}.

\begin{table}[t]
\begin{tabularx}{\textwidth}{QllQQ}
\lsptoprule
{Subgroup} & {Language} & {*asu ‘dog’} & {Bulking of second syllable} & {Reduction/loss of first syllable}\\
\midrule
{Kayanic} & {Busang} & asoʔ & {C-epenthesis} & {--}\\
& {Kelai} & asaw & {V\textsubscript{2}-breaking} & {--}\\
& {Data Dian} & asoːʔ & {C-epenthesis,} \mbox{{V\textsubscript{2}-lengthening}} & {--}\\
& {Mpraa} & hawʔ & {C-epenthesis,} {V\textsubscript{2}-breaking} & {V\textsubscript{1}-loss}\\
& {Modang} & saɵ̯ & {V\textsubscript{2}-breaking} & {V\textsubscript{1}-loss}\\
& {Long Gelat} & saː & \mbox{{V\textsubscript{2}-lengthening}} & {V\textsubscript{1}-loss}\\
& {Bahau} & hoːʔ & {C-epenthesis,} \mbox{{V\textsubscript{2}-lengthening}} & {V\textsubscript{1}-loss}\\
\midrule
{Punan} & {Punan Tuvu’} & {auh} & {Metathesis} & {Metathesis}\\
\midrule
{Land Dayak} &  & {*kasu} &  & \\
& {Benyadu} & kasuʔ & {C-epenthesis} & {--}\\
& {Sungkung} & kasokŋ & {C-epenthesis} & {--}\\
& {Hliboi Bidayuh} & kisúə̯kŋ & {C-epenthesis,}  {V\textsubscript{2}-breaking} & {V\textsubscript{1}-reduction}\\
& {Sanggau} & {kiu}ʔ & {C-epenthesis,} {C-lenition} & {V\textsubscript{1}-reduction}\\
\midrule
{Kenyah} & {E. Penan} & asəwʔ & {C-epenthesis,} {V\textsubscript{2}-breaking} & {--}\\
\lspbottomrule
\end{tabularx}
{\caption{\label{tab:blevins:15} Some Borneo reflexes of PAN *asu ‘dog’ illustrating shift toward iambic word}}
\end{table}

While all of these features of the Austronesian languages of Borneo suggest contact with Mon Khmer speaking populations, language contact is not considered in the major works on the historical phonology of Borneo, including the extensive work of Blust cited here, and the recent work of \citet{Smith2017}. This is due to the absence of clear evidence for these populations being present in Borneo at or after the arrival of the Austronesians.\footnote{On linguistic evidence for a Mon-Khmer presence in Borneo at the time of Austronesian settlement, see \citet{Adelaar1995}, \citet{Blench2010} and \citet{BlevinsKaufman2023}.} Future studies may lend further support to the view that some of the many unusual sound changes in the languages of Borneo are due to contact with Mon-Khmer, and, more specifically, to the prosodic and segmental impact of sequisyllables and complex laryngeal contrasts on canonical western Austronesian forms. 

\begin{sidewaystable}
\begin{tabularx}{\textwidth}{lllQQ}
\lsptoprule
\textbf{SC\#} & \textbf{Sound change(s)} & \textbf{Language(s)} & \textbf{Phonetic motivation} & \textbf{Other factors}\\
\midrule
SC1 & *-y >  *-w, *-w > *-b > -\textit{p} & Khehek (Drehet, Levei) & hypercorrection in glide dissimilation; fortition & VG distributions;  final *-p gap\\
SC2 & *-D > *-ND, *-ND > -N & Berawan, Karo Batak & nasal spread; hyper-voicing; non-release & analogical extension\\
SC3 & *b > \textit{w} > \textit{gw} > \textit{g} / V\_V & Berawan & lenition + fortition & \\
SC4 & *D > T/ V\_V & Berawan, Kiput & post-tonic fortition & iamb/sesquisyllable via contact?\\
SC5 & *ND > NT & Murik, Buginese & variable voicing & lack of contrast\\
SC6 & *g > -\textit{j}- & Sa’ban (Kelabit) & articulatory & \\
SC7 & *N > G/N(C)\textit{a}\_\# & Iban & hypercorrection in nasal dissimilation & exceptions in Malay loans\\
SC8 & *C > C:/\_V\# & Berawan & syllable bulking by weight-to-stress & iamb/sesquisyllable via contact?\\
SC9 & *dr- > \textit{tʃ}- > \textit{k\textsuperscript{x}} > \textit{kʰ}- & Drehet & articulatory & \\
SC10 & *w- …> \textit{c}-, *-w- …> -\textit{nc}- & Sundanese & perceptual confusion; rhinoglottophilia; labial palatalization & pre-Sundanese *w from Javanese loans\\
\lspbottomrule
\end{tabularx}
\caption{\label{tab:blevins:16}. Motivated Sound Changes adapted from \citet{Blust2005}}
\end{sidewaystable}


In the meantime, whether phonetic or structural, internal or external, completely transparent, or clouded by diffusion and confusion, the big picture remains one of regular linguistically motivated sound change. As summarized in \tabref{tab:blevins:16}, Blust’s recalcitrant cases of regular sound change in Austronesian can, for the most part, be viewed as natural phonetically motivated changes, or sequences of these changes. At the same time, recognition of phonotactic regularities, phonotactic gaps, absence of contrasts, prosodic factors, contact-induced change, and exceptions in loan vocabulary, may be necessary in order to make sense of the rarity of these sound patterns within the Austronesian language family, and, for some patterns, more widely.


At the same time, there are theoretical implications. While Beguš’s approach to unnatural sound change (\citeyear{Beguš2018,Beguš2019}) treats intervocalic devoicing as decidedly unnatural, we argue that direct devoicing under phonetic lengthening in prosodically strong positions is a natural process. Apart from cross-linguistic phonetic evidence for this kind of lengthening additional arguments support this approach: first Beguš’s analysis fails to make the right predictions regarding intermediate stages of change; second, devoicing appears to target sounds in order of their aerodynamic phonetic tendency to devoice, suggestive of direct stop devoicing, as opposed to indirect devoicing under the Blurring Account, where devoicing targets a critical intermediate fricative stage (for all points of articulation.)

There is clearly more work to be done in understanding fine details of many of the historical developments discussed here. The role of contact cannot be underestimated, nor can structural properties of phonological systems, nor the phonetic dimensions of speech that we have yet to fully understand. Taking all of these factors into account, with Blust’s contributions to inspire and enlighten, new aspects of sound change are bound to be discovered, and with them, explanations for changes that puzzle us today.

\subsection*{Acknowledgements}

This chapter is dedicated to the memory and scholarship of Bob Blust, with respect, appreciation, awe and renewed inspiration. Most of the chapter was written prior to Bob’s passing, and summarizes years of fruitful conversations with him on related topics. \sectref{sec:blevins:2.3} and \sectref{sec:blevins:2.4} were revised after that time and have benefited from reading his contribution to this volume. An earlier version of this chapter was presented at the Understanding Sound Change Workshop at The Old World Phonology Conference, Donostia-San Sebastián, January 27, 2022. I am grateful to the audience there, an anonymous reviewer, and Daniel Kaufman for comments, and to the editors of this volume for sharing Bob’s submission with me, and for their suggestions.

\subsection*{Abbreviations}
\begin{tabularx} {.46\textwidth}{lQ}
1     &   first person\\
ACD   &   Austronesian Comparative Dictionary \citep{bltr}\\
AS    &   \citet{Smith2017}\\
C     &   consonant or glide\\
D     &   voiced stop\\
BD    &   Borneo Dictionary\\
G     &   glide\\
H     &   aspirate\\
\textsc{incl}     &   inclusive\\
N     &   nasal stop\\
\end{tabularx}%
\begin{tabularx} {.52\textwidth}{lQ}
PAN   &   Proto Austronesian\\
\textsc{pl}     &  plural\\
PMP   &   Proto Malayo-Polynesian\\
POC   &   Proto Oceanic\\
PWMP  &   Proto Western Malayo-Polynesian\\
T     &   voiceless stop\\
V     &   vowel\\
Y     &   palatal vowel or glide\\
WMP   &   Western Malayo-Polynesian\\
*     &   reconstructed form\\
**    &   {expected but unattested form}
\end{tabularx}

\printbibliography[heading=subbibliography,notkeyword=this]
\end{document}
