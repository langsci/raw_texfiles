\documentclass[output=paper]{langsci/langscibook.cls} 
\author 
{Katrin Menzel\affiliation{Saarland University}\and
Ekaterina Lapshinova-Koltunski\affiliation{Saarland University}\lastand
Kerstin Kunz\affiliation{Heidelberg University}
}
\title{Cohesion and coherence in multilingual contexts} 
% \rohead{Cohesion and Coherence in multilingual contexts}
% \lehead{Katrin Menzel et al.}

%\epigram{Change epigram in chapters/01.tex or remove it there }
%\abstract{Change the  abstract in chapters/01.tex \lipsum[1]}
\abstract{\vspace*{-2\baselineskip}}
\maketitle

\ChapterDOI{10.5281/zenodo.814458}
\begin{document}
\section{Introduction}
	
The volume will investigate textual relations of \isi{cohesion} and \isi{coherence} in translation and multilingual text production with a strong focus on innovative methods of empirical analysis as well as technology and computation. Given the amount of multilingual computation that is taking place, this topic is important for both human and \isi{machine translation} and further multilingual studies. 

Coherence and \isi{cohesion}, the two concepts addressed by the papers in this book, are closely connected and are sometimes even regarded as synonymous \citep[see e.g.][]{Brinker2010}. We draw a distinction concerning the realization by linguistic means. 

\textsc{Coherence} first of all is a cognitive phenomenon. Its recognition is rather subjective as it involves text- and reader-based features and refers to the \isi{logical flow} of interrelated topics (or experiential domains) in a text, thus establishing a mental textual world. \textsc{Cohesion} can be regarded as an explicit indicator of relations between topics in a text. It refers to the text-internal relationship of linguistic elements that are overtly linked via lexical and grammatical devices across sentence boundaries. The main types of \isi{cohesion} generally stated in the literature are \isi{coreference}, substitution/ ellipsis, conjunction and lexical \isi{cohesion} (\citet{HallidayHasan1976}). They create relations of identity or comparison, logico-semantic relations or similarity. In the case of \isi{coreference} and lexical \isi{cohesion}, \textsc{cohesive chains} may contain two or more elements and may span local or global stretches of a text \citep{HallidayHasan1976, Widdowson1979}. 

There is another linguistic phenomenon dealt with in several studies of this book, which interacts with \isi{cohesion} and which also contributes to the overall \isi{coherence} and topic continuity of a text: \textsc{Information structure} concerns the linguistic marking of textual information as new/ relevant/ salient or old/ less relevant/ less salient \citep{Krifka2007,Lambrecht1994}. The information in question is presented through linear arrangement of syntactic constituents as either theme or theme, topic or focus or, more generally speaking, in sentence-initial or sentence-final position.

Hence, \isi{coherence} may or may not be signaled by linguistic markers at the text surface, while \isi{cohesion} and \isi{information structure} are explicit linguistic strategies which enhance the recognition of conceptual continuity and the \isi{logical flow} of topics in texts \citep{LouwerseGraesser2005,HallidayMatthiessen2004}.

One major task involved in the process of translation is to identify the linguistic triggers employed in the \isi{source text} to develop, relate and change topics. Moreover, the conceptual relations in the mental textual world have to be transferred into the \isi{target text} by using strategies of \isi{cohesion} and \isi{information structure} that conform to target-language conventions. Empirical knowledge about language contrasts in the use of these explicit means and about adequate/ preferred translation strategies is one essential key to systematize the \isi{logical flow} of topics in human and \isi{machine translation}. 
	The aim of this volume is to bring together scholars analyzing the \isi{cohesion} and \isi{information structure} from different research perspectives that cover translation-relevant topics: language contrast, \isi{translationese} and \isi{machine translation}. What these approaches share is that they investigate instantiations of discourse phenomena in multilingual contexts. Moreover, language comparison in the contributions of this volume is based on empirical data. The challenges here can be identified with respect to the following methodological questions:
	
	\begin{enumerate}
		
		\item What is the best way to arrive at a cost-effective operationalization of the annotation process when dealing with a broader range of discourse phenomena?
		\item Which statistical techniques are needed and are adequate for the analysis? And which methods can be combined for data interpretation?
		\item Which applications of the knowledge acquired are possible in multilingual computation, especially in \isi{machine translation}?
	\end{enumerate}
	
	The contributions of different scholars and research groups involved in our volume reflect these questions. All contributions have undergone a rigorous double blind peer reviewing process, each being assessed by two external reviewers.
    On the one hand, some contributions will concentrate on procedures to analyse \isi{cohesion} and \isi{coherence} from a corpus-linguistic perspective (M. Rysov\'{a}; K. Rysov\'{a}). On the other hand, our volume will include papers with a particular focus on textual \isi{cohesion} in parallel corpora that include both originals and translated texts (Kerremans; Kutuzov, Kunilovskaya). Finally, the papers in the volume will also include discussions on the nature of \isi{cohesion} and \isi{coherence} with implications for human and \isi{machine translation} (Lapshinova-Koltunski; Sim Smith, Specia). 
	
	Targeting the questions raised above and addressing them together from different research angles, the present volume will contribute to moving empirical translation studies ahead.
	
	\section{Phenomena under analysis: Cohesion and coherence}
	
	What unifies all of the studies gathered in this volume is that they deal with explicit means of \isi{coherence}: some works are concerned with particular types of \isi{cohesion} (M. Rysov\'{a}; Lapshinova-Koltunski; Sim Smith, Specia), some of them look into the interplay of these different types (Kerremans; Lapshinova-Koltunksi), and some investigate their \isi{interaction} with \isi{information structure} (K. Rysov\'{a};  Kunilovskaya, Kutuzov; Sim Smith, Specia) In most studies, the focus is on the cohesive devices triggering a cohesive relation (M. Rysov\'{a}; Lapshinova-Koltunski; Kunilovskaya, Kutuzov), others also take account of the relations between cohesive elements (K. Rysov\'{a}; Kerremans; Sim Smith, Specia).

M. Rysov\'{a} considers discourse connectives from an etymological perspective in order to set up a structural classification of different connective types for her corpus-linguistic analysis of the Prague Discourse \isi{Treebank}. Taking account of their degree of \isi{grammaticalization}, she draws a main distinction between primary and secondary discourse connectives. While both types share their textual function of signaling logico-semantic relations between different textual passages (clauses, clause complexes and larger chunks), they differ in terms of their internal structure as well as their syntactic function. 

K. Rysov\'{a} looks into the interplay of \isi{coreference} and \isi{information structure}. She analyses whether different types of coreferential expressions occur in the topic or the focus of a sentence. More precisely, coreferential anaphors or antecedents may collide with syntactic elements that are non-contrastive contextually bound (typically given information), contrastive contextually bound (information on some alternative that can be derived from the context but may not be explicitly given), or non-contextually bound (textually new information). 

Kerremans focuses on the \isi{interaction} of \isi{coreference} and lexical \isi{cohesion} in order to determine terminological variants of the same conceptual entity. He groups all nominal elements referring to the same entity in \isi{coreference} chains and merges these chains with corresponding chains in other texts of the same language. Assigning the \isi{coreference} chains in the English source texts to the corresponding chains in the Dutch and \isi{French} target texts eventually permits enriching a terminological database. 

Kunilovskaya, Kutuzov consider the mapping of given and new information onto syntactic structure. They train machine learning models to compare originals and translations in terms of (a-) typical patterns at sentence boundaries. For this purpose, they analyze a set of cohesive devices (e.g. pronouns and conjunctions) and other features (e.g. parts of speech, word length) in \isi{Russian} translations from English and in \isi{Russian} original texts. Contrasts are identified in terms of where and in which linear order these features occur before and after sentence starts. 

Lapshinova compares the distribution of various types of \isi{cohesion} in human and \isi{machine translation}. Her focus is on cohesive devices indicating identity of reference (\isi{coreference}) and logico-semantic relations (conjunction). Within \isi{coreference}, she distinguishes devices serving as nominal heads (e.g. personal and demonstrative pronouns) and those functioning as modifiers (e.g. the definite article, demonstrative determiners). Conjunctions are classified in terms of their syntactic function (e.g. subordinating or coordinating conjunction and the logico-semantic relation they indicate (e.g. additive or temporal). Translations from English into \isi{German} and original texts of the two languages. 

\largerpage
Sim Smith, Specia investigate the textual distribution of lexical \isi{cohesion} for improving statistical \isi{machine translation}. They apply two statistical techniques in order to assess the lexical \isi{coherence} of texts in a multilingual \isi{parallel corpus} (English, \isi{French} and \isi{German}). Contrasts between languages and between translations and originals are identified by analyzing nominal elements contained in lexical chains of one and the same document. The criteria of comparison included in the research are a) in which sentences these elements appear and b) in which syntactic function (subject vs. other).
	
	\section{Corpora and languages}
	This volume has much to offer to the reader interested in electronic corpora as language resources. It provides information on current research into textual characteristics and discourse structures in different types of language corpora and suggests solutions to questions related to annotation procedures, the quantitative analysis and interpretation of data and \isi{machine translation} for various languages.
	
	Several types of corpora were used for the studies in this volume. Some contributions focus on large-scale monolingual corpora with the purpose of analyzing a particular language and developing methods that can be applied to other languages as well where similar corpora are available. Some researchers demonstrate the pedagogical and scientific value of native and \isi{learner} corpora that help to reveal differences between native speakers of a given language and non-native speakers in their ways of creating textuality. Finally, some contributions use bi- or multilingual parallel or comparable corpora consisting either of texts in a language and their translations in another language or of original texts in several languages that are similar with regard to their sampling frame, balance and representativeness.	
	
	The annotation of discourse relations and the frequency of discourse connectives in large monolingual corpora such as the the Prague Discourse \isi{Treebank} 2.0 (PDiT) consisting of \isi{Czech} newspaper texts as a particular type of written texts are discussed in the chapter by M. Rysov\'{a}. She examines the \isi{historical origin} of prototypical discourse connectives in \isi{Czech}, English and \isi{German} and demonstrates how these findings can help translators to produce more accurate translations of connectives in these languages. Furthermore, her observations are helpful for the annotation of connectives in large corpora of these languages. Discourse connectives arose from various parts of speech in \isi{Czech}, English and \isi{German} and display different stages of \isi{grammaticalization}. In corpus data for modern stages of the languages investigated in this chapter, they can occur, for instance, in the form of conjunctions, particles, prepositional phrases or fixed collocations. Her chapter provides an angle to address such challenges to annotators of discourse connectives as groups of expressions that may not seem straightforward to define in various languages.
	
	K. Rysov\'{a}'s chapter also addresses the analysis of texts from the Prague Dependency \isi{Treebank} as a large monolingual corpus and focuses on coreferential relations and \isi{information structure} in \isi{Czech}. Her chapter demonstrates that the complexity of text \isi{coherence} demands extensive language resources of authentic texts from a given language. Large monolingual corpora with multilayer annotation are still relatively rare for many languages. K. Rysov\'{a}'s analysis encourages research into other languages and recommends applying the methodology she used for the annotation and analysis of coreferential relation and \isi{information structure} to other languages for which similar resources exist.
	
	Kerremans' chapter demonstrates the invaluable contribution of multilingual parallel corpora including both originals and translated texts as a resource for comparative linguistics and translation studies. The corpus created for Kerremans' study is comprised of written English original texts and their translations into \isi{French} and Dutch. Terminological variants and coreferential relations from the English source texts have been analyzed from a contrastive perspective. The translation equivalents of these phenomena were retrieved from the \isi{French} and Dutch target texts in order to create a useful terminological database of translation units and their target-language equivalents for the English-\isi{French} and the English-Dutch language pairs. 
	
	The chapter by Kunilovskaya, Kutuzov deals with the benefits which can be gained from the conjoined use of native and \isi{learner} corpus data. It compares native and \isi{learner} varieties of the \isi{Russian} language with regard to the use of sentence boundaries in a subcorpus of mass media texts from the \isi{Russian} Learner Translator Corpus. The corpus includes English-\isi{Russian} \isi{learner} translations and a genre-comparable subcorpus of the \isi{Russian} National Corpus, aiming at uncovering differences between native \isi{Russian} and its \isi{learner} translated variant. 

	The chapter by Sim Smith, Specia provides a compelling example of how \isi{multilingual corpus} data can be used to improve the \isi{translation quality} in machine-translation models. In this study, original and translated news excerpts in English, \isi{French} and \isi{German} from a \isi{parallel corpus} from the Workshop on Statistical Machine Translation (\isi{WMT}) were used as well as translations of from \isi{French} into English from the \isi{LIG} corpus, which contains news excerpts drawn from various \isi{WMT} years. The translations that were used for the analysis were provided by human professional translators. They were analyzed with regard to the realisation of lexical \isi{coherence}, and a multilingual comparative entity-based grid was developed that consists of various types of documents covering the three languages under comparison.
	
	The chapter by Lapshinova-Koltunski describes innovative corpus-based methods to analyze the frequencies and distributions of cohesive devices in multilingual data. Her bilingual corpus contains comparable English and \isi{German} data for various written text types as well as multiple translations into \isi{German} which were produced by human translators with different levels of expertise and by different \isi{machine translation} systems. This contribution has its focus on the analysis of \isi{cohesion} in texts from different languages which vary along dimensions such as text-production type, translation method involved and systemic contrasts between source and \isi{target language}.
	
	\section{Methods of investigation}
	The contributions to this volume cover a wide range of different methods of analysis, starting from manual investigation of previously annotated data, across semi-automatic procedures supporting manual analysis towards fully computational approaches such as entity-grid calculation and automatic sentence segmentation with machine-learning techniques.
	
	Annotation of corpora with information on cohesion- or coherence-related phenomena play a significant role in various descriptive studies based on corpora. They receive particular attention in chapters 2, 3 and 4, in which research design relies to a large extent on annotation. In chapters 5, 6 and partly 7, automatic procedures are used to identify \isi{cohesion} and \isi{coherence} phenomena.
	
	Issues of annotation of \isi{explicit discourse} relations (i.e. relations expressed by concrete language means) in the PDiT are addressed in the study by M. Rysov\'{a}. She uses the data from PDiT for her analysis to illustrate the difficulty of delineating the boundaries between connectives and non-connectives. 
	For instance, she discusses if frozen lexical forms are a sufficient argument for excluding \isi{multiword} phrases from discourse connectives and their annotation in the corpus. These phrases clearly signal discourse relations within a text, but they significantly differ from the ``prototypical'', lexical connectives. The author provides an analysis of historical formation of discourse connectives, justifying their claim that discourse connectives are not a closed class of expressions but rather a scale mapping the \isi{grammaticalization} of the individual connective expressions. The author believes that this justification may help with the annotation of discourse in large corpora, as was done for PDiT.
	
	The Prague Dependency \isi{Treebank} was used in the analyses by K. Rysov\'{a}, who demonstrates how different annotation layers can be used to examine text \isi{coherence}. The author concentrates on the interplay of two annotation layers: text \isi{coreference} and \isi{sentence information structure}. The \isi{annotation of sentence information structure} is related to \isi{contextual boundness}, whereas text \isi{coreference} is understood as the use of different language means for marking the same object of textual reference (the antecedent and the anaphor referents are identical). The author defines all mutual possibilities of \isi{coreference} relations among contextually bound and contextually \isi{non-bound sentence} items, and analyzes their corpus occurrences. The client-server PML Tree Query \citep{StepanekPajas2010} was used to extract the frequency information. The client part is an extension of the tree editor TrEd2 \citep{PajasStepanek2008}. K. Rysov\'{a} analyzes the proportion of various mutual possibilities on the basis of corpus occurrences in PDT.
	
	Kerremans uses \isi{coreference} analysis to study inter- and intralingual \isi{terminology} variation in a \isi{parallel corpus}. He proposes a \isi{semi-automatic method} to annotate terminological patterns that belong to the same \isi{coreference} chain (called coreferential terminological variants) as an alternative to fully manual labeling, which turns out to be a labour-intensive process. Kerremans method is aimed at supporting manual identification of coreferential terminological variants in the English source texts, annotating these variants according to a common \isi{cluster label}, extracting them from the text and storing them in a separate database. The automated procedures are implemented in a Perl script ensuring completeness, accuracy and \isi{consistency} in the data obtained.
	
	Kunilovskaya, Kutuzov also apply semi-automatic procedures to a \isi{multilingual corpus} that contains both parallel and comparable texts. These semi-auto\-mat\-ic procedures are applied to detect divergences in sentence structures between translations into \isi{Russian} and \isi{Russian} non-translations.  
	The authors deploy statistical techniques from machine learning: they train a decision-tree model to describe the contextual features of sentence boundaries in the \isi{reference corpus} of \isi{Russian} texts, which are considered to be an approximation of the \isi{standard language} \isi{variety}. The model is then applied to the translation \isi{learner} corpus, and translated sentences that are different from the \isi{standard language} \isi{variety} are identified through the \isi{evaluation} of predictors and their combinations. Kunilovskaya, Kutuzov use a number of contextual features in sentence-boundary environments for \isi{evaluation}. The initial set of 82 features was reduced to 48 with the help of \isi{feature selection} procedures, allowing them to keep only predictive ones. The results of their analysis permit, on the one hand, to manually inspect cases of the model failing to predict sentence boundaries and possibly find the route causes, and on the other hand, to train another model which predicts not sentence boundaries, but inconsistencies between the first-model decisions and what a translator did in a particular context.
	
	Sim Smith, Specia perform an exploratory analysis of lexical \isi{coherence} in a \isi{multilingual context} with a view to identifying patterns that could later be used to improve overall \isi{translation quality} in \isi{machine translation} models. They use an entity-grid model and an entity-graph metric -- two entity-based frameworks that have previously been used for assessing \isi{coherence} in a \isi{monolingual setting}. The authors try to understand how lexical \isi{coherence} is realized across different languages and apply these techniques in a \isi{multilingual setting} for the first time. The entity-grid approach is applied to a \isi{parallel corpus}. Simply tracking the existence or absence of entities allows for direct comparison across languages. However, \isi{entity transition} patterns may vary from language to language, while retaining an overall degree of \isi{coherence}. In order to illustrate the differences between the distributions of entity transitions over the different languages, the authors compute divergence scores. They also analyze the reasons for the observed divergence by taking a closer look at their data.
	
	Lapshinova-Koltunski uses a number of \isi{visualisation} and statistical techniques to investigate the distributional characteristics of subcorpora in terms of occurrences of cohesive devices in human and \isi{machine translation}. The cohesive features chosen for the comparative analysis were obtained on the basis of automatic linguistic annotation: tokenisation, lemmatisation, \isi{part-of-speech} tags and segmentation into syntactic chunks and sentences. Cohesive features are operationalized with the Corpus Query Processor (CQP) queries \citep{Evert2010}. This tool allows definition of language patterns in the form of regular expressions that can integrate string, \isi{part-of-speech} and chunk tags, as well as further constraints, e.g. position in a sentence. With the help of CQP queries, frequencies of various cohesive features are extracted from a corpus containing translation varieties. Then, various descriptive techniques are used to observe and explore differences between groups of texts and subcorpora under analysis.
	
	
	\section{Conclusion}
The contributors to this volume are experts on discourse phenomena and textuality who address these issues from an empirical perspective. We hope that this volume provides an innovative and useful contribution to the advancement of linguistic theory and discourse-oriented corpus studies. This volume also aims at addressing the challenges for human and \isi{machine translation} arising from the interplay of grammatical and lexical indicators of textual \isi{cohesion} and \isi{coherence}. 
		
The chapters in this volume are written in an accessible style. They epitomize the latest research, thus making this book useful to both experts of discourse studies and computational linguistics, as well as advanced students with an interest in these disciplines. We hope that this volume will serve as a catalyst to other researchers and will facilitate further advances in the development of cost-effective annotation procedures, in the application of statistical techniques for the analysis of linguistic phenomena, the elaboration of new methods for data interpretation in \isi{multilingual corpus} linguistics and \isi{machine translation}.
		
	
	%PART 1: Contrastive Aspects of Cohesion and Coherence
	%PAPER 1:
	%Title: How to Annotate Multiword Discourse Connectives in Large Corpora 
	%Author: Magdaléna Rysová, Charles University in Prague (\isi{Czech} Republic)
	%PAPER 2: 
	%Title: Interaction of Coreference and Sentence Information Structure in the Prague Dependency \isi{Treebank}
	%Author: Kateřina Rysová, Charles University in Prague (\isi{Czech} Republic)
	
	%PART 2:  Textual Cohesion and Translation
	%PAPER 3:
	%Title: Terminological variation in multilingual parallel corpora: a \isi{semi-automatic method} involving co-referential analysis
	%Author: Koen Kerremans, Vrije Universiteit Brussel (Belgium)
	
	%PAPER 4:
	%Title: Testing \isi{target text} \isi{cohesion}: An attempt at machine learning model to predict acceptability of sentence boundaries in English-\isi{Russian} translation
	%Authors: Andrey Kutuzov, Linguistic Lab on Corpus Technologies, Higher School of Economics (Moscow, Russia) and Maria Kunilovskaya, Tyumen State University (Russia)
	
	%PART 3: Aspects of Cohesion and Coherence in Human vs. Machine Translation
	%PAPER 5: 
	%Title: Cohesion and Translation Variation: Corpus-based Analysis of Translation Varieties
	%Author: Ekaterina Lapshinova-Koltunski, Saarland University (Germany)
	
	%PAPER 6: 
	%Title:   Examining Lexical Coherence in a Multilingual Setting
	%Authors: Karin Sim Smith, University of Sheffield (UK) and Lucia Specia (University of Sheffield)
	
%	\bibliographystyle{apalike} 
%	\bibliography{biblio}

%{\sloppy
%\printbibliography[heading=subbibliography,notkeyword=this]
%}	
{\sloppy
\printbibliography[heading=subbibliography,notkeyword=this] 
}
\end{document}