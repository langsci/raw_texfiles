\documentclass[output=paper]{langsci/langscibook.cls}
\title{Possibilities of text coherence analysis in the Prague Dependency Treebank}
\author{Kateřina Rysová \affiliation{Charles University, Faculty of Mathematics and Physics}}
\shorttitlerunninghead{Text coherence in the PDT}
% \lehead{Kateřina Rysová}
\abstract{The aim of this paper is to examine the interplay of text coreference and sentence information structure and its role in text coherence. The study is based on the analysis of authentic Czech texts from the Prague Dependency Treebank 3.0 (PDT; i.e. on almost 50 thousand sentences). The corpus contains manual annotation of both text coreference and information structure -- the paper tries to demonstrate how these two different corpus annotations may be used in examination of text coherence. In other words, the paper tries to describe where these two language phenomena meet and how important the interplay is in making text well comprehensible for the reader. Our results may be used not only in a theoretical way but also practically in automatic corpus annotations, as they may give us an answer to the general question whether it is possible to annotate the sentence information structure automatically in large corpora on the basis of text coreference. 
}
\maketitle
\begin{document}
\section{Introduction and theoretical background\label{rysova_k:sec:Introduction}}

Studying text \isi{coherence} is dependent on studying several individual language phenomena like \isi{coreference}, \isi{anaphora}, \isi{sentence information structure} or discourse (mainly in terms of semantico-pragmatic discourse relations). In other words, a text may be imagined as a net of many different kinds of relations that are mutually interconnected and possibly influence each other. 

So far, these phenomena have been studied primarily in isolation but recently, there is a growing need for more complex studies focusing on \isi{interaction} (see, for example, \citealt{Hajicova2006, Hajicova2011}; \citealt{Eckert2010}; \citealt{Rysova2015Analyzing}). In other words, if we want to analyze text \isi{coherence} deeply (i.e. to help to answer the question what are the general properties of a text), we have to pay closer attention to the interactions of several individual phenomena at once (operating both inter- and intra-sententially).\footnote{For complex studying of \isi{coherence} phenomena, see Accessibility Theory \citep{ariel1988referring} or Centering Theory \citep{joshi1981control, grosz1986attention}.}  

The theme of interplay between \isi{coreference} or anaphoric relations and \isi{sentence information structure} has been studied recently especially in \citet{NedoluzhkoHajicova2015} and \citet{Nedoluzhko2015} who linguistically investigated contextually bound nominal expressions (explicitly present in the sentence) that do not have an anaphoric (bridging, \isi{coreference} or \isi{segment}) link to a previous (con)text. They draw the conclusion that three cases may be found when contextually bound expressions may not be linked by any \isi{coreference} or anaphoric relation: (i) contextually bound nominal groups related to \isi{previous context} (semantically or pragmatically) but not specified as bridging relations in the Prague Dependency \isi{Treebank} (PDT); (ii) noun groups referring to secondary circumstances (like temporal, local, etc.) and (iii) nominal groups having low referential potential.

In this respect, this paper follows their work. It investigates a narrower data sample (only expressions interlinked by text \isi{coreference}) with the aim to bring an overview of density of text \isi{coreference} relations according to the \isi{sentence information structure} values of the interlinked expressions.

The complex analysis of text \isi{coherence} demands extensive language material of authentic texts, i.e. large language corpora with multilayer annotation. Such corpora are rather rare (cf., for example, \citealt{komen2012coreferenced, Stede2014, Chiarcos2014}). The corpus with one of the richest (i.e. multilayer) annotation is the Prague Dependency \isi{Treebank} (PDT) for \isi{Czech} (see \citealt{Bejcek2013}). The PDT contains detailed annotation on morphological, analytical (surface syntactic) and tectogrammatical (deep syntactic) level as well as the \isi{annotation of sentence information structure}, \isi{coreference} and anaphoric relations, discourse relations and text genres. The PDT thus offers suitable language material for studies focusing on the annotated language phenomena in \isi{interaction}.

The paper concentrates on the interplay of two of them -- text \isi{coreference} and \isi{sentence information structure} (mainly in terms of \isi{contextual boundness}) -- as well as on the fact how and to what extent this interplay is projected into text \isi{coherence}.

\section{Main objectives\label{rysova_k:sec:MainObjectives}}

Generally, as said above, the paper focuses on the relation between text \isi{coreference} and \isi{sentence information structure}. It describes where and in which aspects these two phenomena meet in the text and how they influence each other. It also presents methods that may be used for analyzing language interplays in general (demonstrated using the PDT data). Finally, the paper demonstrates whether and how the present (manual) annotation of text \isi{coreference} in the PDT may be used for improving automatic \isi{annotation of sentence information structure}.

To meet the goals, the paper focuses on the specific tasks concerning the relation of text \isi{coreference} and \isi{sentence information structure} (in sense of \isi{contextual boundness} -- see \sectref{rysova_k:sec:SentenceInformation}). The paper explores whether the text \isi{coreference} relations (in the PDT texts) connect rather contextually bound or \isi{non-bound sentence} members (mutually) or both of them in the same way, see Examples \ref{ex:1} and \ref{rysova_k:example:2} and \figref{rysova_k:fig:1} below.

Since the contextually bound sentence items usually carry information that is deducible from the previous (con)text (in contrast to the contextually non-bound items), we assume the higher number of text \isi{coreference} links leading right from them. In other words, the assumption is that text \isi{coreference} and \isi{sentence information structure} meet especially in sentence items related somehow to the previous (con)text.

\section{Methods and material\label{rysova_k:sec:MethodsAndMaterial}}

\subsection{Sentence information structure in the PDT\label{rysova_k:sec:SentenceInformation}}

The analysis uses the language data of the Prague Dependency \isi{Treebank}. The PDT contains almost 50,000 sentences (833,195 word tokens in 3,165 documents) of \isi{Czech} newspaper texts that are (mostly manually) annotated on several language levels at once. 

The theoretical framework for \isi{sentence information structure} in the PDT is based on Functional Generative Description (FGD) introduced by \citet{Sgall1967} and further developed especially by \citet{hajicova1998topic}. 

The annotation is carried out on tectogrammatical trees. Each relevant node of the tree is labeled with one of the three values of \isi{contextual boundness}.\footnote{In addition, the communicative dynamism is annotated -- as deep order of the nodes in the tree.} Contextual boundness has the following possible values: non-contrastive contextually bound nodes (marked as ``t''), contrastive contextually bound nodes (marked as ``c'') and contextually non-bound nodes (marked as ``f''). 

Non-contrastive contextually bound nodes represent units that are considered deducible from the broad (not necessarily verbatim) context and are known for the reader (or presented as known for him or her). Contrastive contextually bound nodes also are expressions related to the broad context and moreover, they usually represent a choice from a set of alternatives. They often occur at the beginning of paragraphs, in enumerations etc. In spoken language, such units carry an optional contrastive stress. Contextually non-bound expressions are not presented as known and are not deducible from the \isi{previous context} -- on the contrary, they represent new facts (or known facts in new relations). The particular occurrences of \isi{contextual boundness} values can be found in Example \ref{ex:1}. \ 

\ea
\label{ex:1}
{\ob}\textit{Jane is my friend.{\cb}} She.t is.f very.f fine.f. However, her.t brother.c is.t boring.f. I.t like.f rather.f her.f. \\
\z

On the basis of \isi{contextual boundness}, the division of the sentence into Topic and Focus is realized (Topic is formed especially by contextually bound items and Focus typically by non-bound items). In the first sentence, the Topic is \textit{she} and the Focus \textit{is very fine}. In the second sentence, the Topic part includes \textit{however, her brother is} and the Focus part \textit{boring}. The participant \textit{I} is the Topic of the third sentence and the part \textit{like rather her} is the Focus.\footnote{For more details about \isi{annotation of sentence information structure} in English texts, see \citet{Rysova2015Topic}.}

For further examples of ``t'', ``c'' and ``f'' nodes, see Example \ref{rysova_k:example:2} in \sectref{rysova_k:sec:ExampleOfADependencyTree}. For more details about Topic-Focus Articulation, see \citet{hajicova1998topic}.

\subsection{Text coreference in the PDT\label{rysova_k:sec:TextCoreference}}

Annotation principles of text \isi{coreference} in the PDT were done according to \citet{Nedoluzhko2011}. In this concept, the text \isi{coreference} is understood as the use of different language means for marking the same object of textual reference. The basic principle of text \isi{coreference} is that the antecedent and the anaphor referents are identical (e.g. \textit{a house} -- \textit{the house}; \textit{Jane} -- \textit{she} --\textit{ her}; \textit{Jane} -- \textit{0}; \textit{problem} -- \textit{this} -- \textit{that}).

The general aspect of text \isi{coreference} is that the coreferential relation is symmetric (if A is coreferential with B, B is coreferential with A) and transitive (if A is coreferential with B and B is coreferential with C, then A is coreferential with C).

Text \isi{coreference} relations in the PDT are represented especially by personal or possessive pronouns (\textit{Jane} -- \textit{she} -- \textit{her}), ellipsis (\textit{Jane }-- \textit{0}), demonstratives (\textit{problem} -- \textit{this }-- \textit{that}) or by referential nominal phrases (concerning mainly nouns with specific, abstract or generic reference -- for more details see \citet{Nedoluzhko2011}) and they operate both inter- and intra-sententially.

\subsection{Example of a dependency tree from the Prague Dependency Treebank\label{rysova_k:sec:ExampleOfADependencyTree}}

Example \ref{rysova_k:example:2} illustrates the most common corpus occurrence -- the \isi{text \isi{coreference} connection} leading from a non-contrastive contextually bound node to another non-contrastive contextually bound node (i.e. from ``t'' to ``t'').

\ea
\label{rysova_k:example:2}
[Jestliže ve státě New Hampshire začne geometricky narůstat kriminalita mladistvých, veřejnost ocení svou přízní vládní akt zvýšení výdajů na boj se zločinností.] {\cb} 
Takové dobré \textbf{opatření} nakonec udělá každá druhá vláda, zvlášť půjde-li o \textbf{opatření} předvolební. \\
\glt [If the juvenile delinquency will increase in the state of New Hampshire, the public will appreciate the government act to increase spending on the fight against crime.]
Every other government eventually makes such good \textbf{measure}, regarding especially a pre-election \textbf{measure}.
\z


\begin{figure}[h]
\includegraphics[scale=0.7]{figures/KR_Figure_1_Dependency_Tree}
\caption{Dependency tree from the Prague Dependency Treebank depicting the sentence \textit{Takové dobré opatření nakonec udělá každá druhá vláda, zvlášť půjde-li o opatření předvolební. -- Every other government eventually makes such good measure, regarding especially a pre-election measure.}}
\label{rysova_k:fig:1}
\end{figure}

\figref{rysova_k:fig:1} represents the sentence from Example \ref{rysova_k:example:2}. The \isi{text \isi{coreference} arrow} leads from the second occurrence of the word \textit{measure} (non-contrastive contextually bound (``t'')) to the first occurrence of the word \textit{measure} (that is also non-contrastive contextually bound (``t''), i.e. deducible from the \isi{previous context}). 

Another \isi{coreference relation} is between the nodes\textit{ government} (\figref{rysova_k:fig:1}) and \textit{government} (\textit{act}) from the previous sentence, see Example \ref{rysova_k:example:2}. In \figref{rysova_k:fig:1}, only the starting position of this \isi{coreference relation} can be seen. The final position of the \isi{coreference arrow} is in the previous tree in the treebank and it is not displayed in \figref{rysova_k:fig:1}.



\subsection{PML Tree Query\label{rysova_k:sec:PMLTreeQuery}}
                                                                                                                                                                          
Our analysis of the \isi{interaction} between \isi{information structure} and text \isi{coreference} was carried out with the client-server PML Tree Query (PML-TQ; the primary format of the PDT is called Prague Markup Language) \citep{Stepanek2010}. The client part has been implemented as an extension to the tree editor TrEd \citep{Pajas2008} that may be used also for editing data. 

Using PML-TQ engine, all the occurrences of text \isi{coreference} relations in the PDT (annotated as arrows -- see \figref{rysova_k:fig:1}) have been collected and we have examined the \isi{information structure} of the sentence items (nodes in \isi{dependency} trees) where the text \isi{coreference} relations start and where they lead to. In other words, identifying whether the items participating in text \isi{coreference} are rather contextually bound or non-bound.

\section{Results and evaluation\label{rysova_k:sec:ResultsAndEvaluation}}

\tabref{rysova_k:tab:1} shows text \isi{coreference} relations connecting contextually bound and \isi{non-bound sentence} items (nodes) in the PDT.\footnote{The distributions of ``f'', ``t'' and ``c'' nodes in the PDT are presented below.}

\begin{table}
\caption{Contextually bound and non-bound sentence items interconnected with text coreference relation in the Prague Dependency Treebank}
\begin{tabularx}{\textwidth}{Xrrrr}
\lsptoprule
 &
f (from) &
t (from) &
c (from) &
To (in~total)\\  
\midrule
f (to) &
19,571 &
20,354 &
2,754 &
42,679\\ 
t (to) &
7,980 &
27,109 &
1,762 &
36,851\\ 
c (to) &
2,322 &
3,671 &
1,067 &
7,060\\ 
\midrule
>From (in~total) &
29,873 &
51,134 &
5,583 &
\textbf{86,590}\\
\lspbottomrule
\end{tabularx}
\label{rysova_k:tab:1}
\end{table}





From the comparison of \figref{rysova_k:fig:2} and \ref{rysova_k:fig:3}, we may observe that among all the 86,590 text \isi{coreference} relations marked in the PDT, mainly the non-contrastive contextually bound sentence items (``t'' nodes) (60\%) are referring to the previous text (51,134 within 86,590). On the contrary, mainly the contextually \isi{non-bound sentence} items (``f'' nodes) (49\%) serve as recipients of text \isi{coreference} relations (42,679 within 86,590), see \figref{rysova_k:fig:2}. More specifically, if there is the \isi{coreference} text relation between the words \textit{Jane} and\textit{ she} (i.e. from \textit{she} to \textit{Jane}), \textit{she} is mostly (in 60\%) ``t'' node (i.e. non-contrastive contextually bound \isi{sentence item}) and\textit{ Jane}, on the other hand, ``f'' node (i.e. contextually \isi{non-bound sentence item}) in 49\%, see \figref{rysova_k:fig:3}.

\begin{figure} 
\includegraphics[height=.3\textheight]{figures/KR_Figure_2_graph.pdf}
\caption{Percentage of individual node types participating in text coreference as the sender of the coreference arrow (its starting point)}
\label{rysova_k:fig:2}
\end{figure}


\begin{figure}
\includegraphics[height=.3\textheight]{figures/KR_Figure_3_graph}
\caption{Percentage of individual node types participating in text coreference as the recipient of the coreference arrow (its ending point)}
\label{rysova_k:fig:3}
\end{figure}


The particular ``c'', ``t'' and ``f'' node types are not distributed with the same frequency in the PDT, see \tabref{rysova_k:tab:2} reflecting the ratio of occurrences of particular node types in the data (the PDT contains 354,841 contextually non-bound nodes (``f''), 176,225 non-contrastive contextually bound nodes (``t'') and 30,312 contrastive contextually bound nodes (``c'')).



\begin{table}
\caption{The PDT distribution of ``f'', ``t'' and ``c'' interconnected with a text coreference relation}
\begin{tabularx}{\textwidth}{Xrrr}
\lsptoprule
\% &
f (from) &
t (from) &
c (from)\\ 
\midrule
f (to) &
5.52 &
11.55 &
9.09\\ 
t (to) &
2.25 &
15.38 &
5.81\\ 
c (to) &
0.65 &
2.08 &
3.52\\
\lspbottomrule
\end{tabularx}
\label{rysova_k:tab:2}
\end{table}



\begin{figure}
\includegraphics[height=.3\textheight]{figures/KR_Figure_4_graph}
\caption{The PDT distribution of ``f'', ``t'' and ``c'' interconnected with a text coreference relation}
\label{rysova_k:fig:4}
\end{figure}



The contextually bound nodes (``t'' and ``c'' nodes) generally have higher probability that the \isi{text \isi{coreference} arrow} will lead from them and also to them than contextually non-bound nodes (``f'' nodes). Based on this, the most typical \isi{text \isi{coreference} connection} leads from a non-contrastive contextually bound node to another non-contrastive contextually bound node (i.e. from ``t'' to ``t''), see Example \ref{rysova_k:example:2} in \sectref{rysova_k:sec:ExampleOfADependencyTree}. The second most typical \isi{text \isi{coreference} connection} leads from a non-contrastive contextually bound node to a contextually non-bound node (i.e. from ``t'' to ``f''). The third most typical \isi{text \isi{coreference} connection} leads from a contrastive contextually bound node to a contextually non-bound node (from ``c'' to ``f''). Generally, the most favored ``starting'' position for a \isi{text \isi{coreference} arrow} is a non-contrastive contextually bound \isi{sentence item} (``t'').



\begin{table}
\caption{Percentage of all ``f'' or ``t+c'' nodes interlinked with a text coreference relation in the PDT}
%\label{tab:interlinkedNodes}
\begin{tabular}{lrr}
\lsptoprule
\% &
(from) &
t+c (from)\\
\midrule
f (to) &
5.52 &
11.19\\
t+c (to) &
2.90 &
16.27\\
\lspbottomrule
\end{tabular}
\label{rysova_k:tab:3}
\end{table}

\begin{figure} 
\includegraphics[height=.3\textheight]{figures/KR_Figure_5_graph}
\caption{Percentage of all ``f'' or ``t+c'' nodes interlinked with a text coreference relation in the PDT}
\label{rysova_k:fig:5}
\end{figure}


Contextually bound sentence items (both contrastive and non-contrastive that are mostly part of sentence Topic) are interlinked with text \isi{coreference} relations more often than contextually non-bound (i.e. from the context non-deducible) items that are mostly part of sentence Focus, see \tabref{rysova_k:tab:3} and \figref{rysova_k:fig:5}. Thus, the two described language phenomena, text \isi{coreference} and \isi{sentence information structure}, mutually cooperate in building the text \isi{coherence}.




The \isi{individual node} types differ in the fact where they find their parts of \isi{coreference} chains. While the non-contrastive contextually bound nodes (``t'') most likely are interconnected with contextually bound nodes, the contextually non-bound nodes (``f'') mostly interconnected with contextually non-bound nodes (in terms of text \isi{coreference}). The contrastive contextually bound nodes stand between these two tendencies -- they are connected both with contextually bound and non-bound nodes (in relatively equal way). Such inclinations also demonstrate that it is worth distinguishing two different kinds of contextually bound nodes (contrastive and non-contrastive) because they contribute to the text \isi{coherence} in different ways.




The \isi{individual node} types (``t'', ``c'' and ``f'') have in common that they all refer to the contrastive contextually bound nodes (``c'') in the slightest degree (among them, the ``c'' nodes have the highest tendency to be interlinked with other ``c'' nodes).



\begin{table}
\caption{Percentage of ``f'', ``t'', ``c'' or ``t+c'' nodes interlinked with a text coreference relation in the PDT}
\begin{tabular}{lrrrr}
\lsptoprule
\% &
f &
t &
c &
t+c\\
\midrule
from &
8.42 &
29.02 &
18.42 &
27.46\\
to &
12.03 &
20.91 &
23.29 &
21.26\\
\lspbottomrule
\end{tabular}
\label{rysova_k:tab:4}
\end{table}


\begin{figure} 
\includegraphics[height=.3\textheight]{figures/KR_Figure_6_graph}
\caption{Percentage of ``f'', ``t'', ``c'' or ``t+c'' nodes interlinked with a text coreference relation in the PDT}
\label{rysova_k:fig:6}
\end{figure}


\tabref{rysova_k:tab:4} and \figref{rysova_k:fig:5} shows a percentage of bound vs. non-bound nodes participating in text \isi{coherence} relations (either as ``recipients'' or ``senders''). The biggest text \isi{coreference} ``recipient'' and also ``sender'' are contextually bound nodes (without further distinguishing between contrast and non-contrast) -- 27.46 \% within all of them (i.e. 56,717 within 206,537) serve as a ``text \isi{coreference} sender'' and 21.26\% of them (i.e. 43,911 within 206,537) as a ``text \isi{coreference} recipient''.




Based on the presented analysis, the following conclusions can be drawn:



\begin{itemize}
\item 

Generally, a \isi{text \isi{coreference} arrow} (i) starts in every 5th--6th and leads to every 4th contrastive contextually bound \isi{sentence item} (``c'' node); (ii) starts in every 3rd--4th and to every 5th non-contrastive contextually bound \isi{sentence item} (``t'' node) and (iii) starts in every 12th and to every 8th contextually \isi{non-bound sentence item} (``f'' node).


\item 

The contextually non-bound nodes (``f'') as well as contrastive contextually bound (``c'') nodes serve more often as text \isi{coreference} ``recipient'' than ``sender''. 


\item 

Conversely, the non-contrastive contextually bound nodes (``t'') serve more often as a text \isi{coreference} ``sender'' than ``recipient''.


\end{itemize}

\section{Conclusions\label{rysova_k:sec:Conclusions}}

The paper has examined the \isi{correlation} between \isi{sentence information structure} and text \isi{coreference} on the data of the Prague Dependency \isi{Treebank}. Altogether, the PDT contains 86,590 text \isi{coreference} relations interconnecting contextually bound or \isi{non-bound sentence} items. The analysis shows that the text \isi{coreference} relations operate rather within contextually bound nodes, i.e. if a \isi{sentence item} is contextually bound (in terms of \isi{sentence information structure}), it has a relatively high probability to be interconnected with another \isi{sentence item} in a \isi{text \isi{coreference} relation}.




On the other hand, there is also a relatively significant part of contextually \isi{non-bound sentence} items interconnected with another part of text through text \isi{coreference}. The \isi{text \isi{coreference} arrow} leads from every 12th contextually \isi{non-bound sentence item} (``f'' node). It means that every 12th contextually \isi{non-bound sentence item} clearly refers to the previous language context (in terms of text \isi{coreference}). However, these two facts are not in contradiction. It is well known that entities mentioned in the previous text can be used in a new perspective (i.e. as contextually non-bound items) and they can bring new and unknown information to the text addressee (cf. \textit{Do you want tea or coffee? -- Tea, please.}).




In this context, contextually bound sentence items cannot be defined simply as coreferentially referring to the previous language context. They refer to the previous text (through text \isi{coreference}) clearly more often than the contextually non-bound items. However, such kind of text referring is also not rare -- according to the PDT, the contextually non-bound items participate in the text \isi{coreference} in about 35\%, non-contrastive contextually bound items in 60\% and contrastive contextually bound items in 5\%.




In this respect, the corpus-based research also demonstrates that the annotation of text \isi{coreference} cannot be (without further specification) a reliable basis for the automatic \isi{annotation of sentence information structure} in large corpora. If every \isi{sentence item} annotated as referring to the \isi{previous context} (in terms of text \isi{coreference}) were automatically annotated also as contextually bound, it would constitute a large degree of error (based on the data from the PDT, the error rate would be about 35\%).



\section*{Acknowledgments\label{rysova_k:sec:Acknowledgments}}

The author acknowledges support from the Ministry of Culture of the \isi{Czech} Republic (project n.~DG16P02B016 \textit{Automatic Evaluation of Text Coherence in Czech}). This work has been using language resources developed, stored and distributed by the LINDAT/CLARIN project of the Ministry of Education, Youth and Sports of the \isi{Czech} Republic (project LM2015071). 

{\sloppy
\printbibliography[heading=subbibliography,notkeyword=this]
}
 \end{document}