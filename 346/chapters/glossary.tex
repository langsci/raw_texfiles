\chapter*{Glossary of linguistic terms}\label{glossary}
\addcontentsline{toc}{chapter}{Glossary of linguistic terms}
\ohead{Glossary of linguistic terms}

\begin{longtable}{rp{8cm}}
        \label{gl-agreement}\makecell[r]{\textsc{agreement}\\\textsc{(concord)}} & A syntactic configuration in which a morphosyntactic feature on one word systematically covaries with the same feature on another. For instance, in many varieties of English, the subject\is{subjects} agrees with the verb for person and number (-\emph{s} for third person singular; no ending elsewhere).\\
        \label{gl-allophone}\textsc{allophone} & A variant realization of a phoneme which is not contrastive. For example, in most varieties of Present Day English \emph{limb} has an initial [l] produced primarily with the movement of the front part of the tongue. On the other hand, at the end of syllables, the /l/ phoneme typically has a dark or a darker allophone, [ɫ], as in \emph{mill}, which includes the raising of the back part of the tongue in addition to the raising of its front part. But the distinction is never important for lexical meaning, i.e. in English we don't find pairs of words where the darkness, or velarization, of /l/ leads to two different words in \textit{mill}: [mɪɫ] and [mɪl] would still refer to a `place where grains are ground'. There are many more examples you can think of, e.g. /t/ being pronounced as a glottal stop,\is{glottalling} as in \textit{water} [woːtə] vs. [woːʔə], or even as a tap [wɔːɾə].\\
        \label{gl-case}\textsc{case}\is{case} & A morphological category for nominals that is related to the grammatical function played by the nominal constituent (subject, direct object, etc.). In Old English, the four major cases we find are nominative, accusative, dative and genitive. \\
        \label{gl-clitic}\textsc{clitic}\is{clitics} & A morpheme that behaves syntactically like a free word but is phonologically dependent on another word or phrase. A Present Day English example is possessive \emph{'s}.\\
        \label{gl-compounding}\textsc{compounding} & A type of morphology alongside \glossterm{gl-inflection}{inflection} and \glossterm{gl-derivation}{derivation}. Compounding is the process of forming a new word by putting two independent words together, e.g. \emph{green} + \emph{house} = \emph{greenhouse}. Compounding can sometimes create unpredictable meanings; a greenhouse is not the same as a green house!\\
        \label{gl-corpus}\textsc{corpus}\is{corpora} & A collection of linguistic material (e.g. texts) suitable for use as a data source in linguistic research. An example is the Corpus of Contemporary American English \citep{Davies2008}, which contains more than a billion words of text from 1990 onwards (25+ million words per year). A corpus (plural \emph{corpora}) is usually intended to be \textsc{representative} of a particular variety (e.g. American English),\il{English, American} and \textsc{balanced} in its selection of material (e.g. with respect to the gender of the author).\\
        \label{gl-deixis}\textsc{deixis}\is{deixis} & The use of words for a specific time, place, person or direction whose reference is dependent on the context. For instance, \emph{tomorrow} refers to different days, and \textit{you} refers to different people, depending on when the word is used. Pronounced /daɪksɪs/ or /dɛɪksɪs/.\\
        \label{gl-derivation}\textsc{derivation} & A type of morphology alongside \glossterm{gl-inflection}{inflection} and \glossterm{gl-compounding}{compounding}. Derivation is the process of forming a new word by adding an affix (e.g. \emph{punch} > \emph{punchability}) or altering the root (e.g. \emph{sing} > \emph{song}). Unlike inflection, derivation can change the category of the word: in the two examples given here, a noun is formed from a verb.\\
        \label{gl-diachronic}\textsc{diachronic} & Across time. A study of how English verbs change between Old English and Present Day English, for example, would be a diachronic study. The opposite of \glossterm{gl-synchronic}{synchronic}.\\
        \label{gl-diacritic}\textsc{diacritic} & A distinguishing mark added to a grapheme, such as a dot (e.g. <ċ>) or a macron (e.g. <ū>). Usually used to indicate differences in pronunciation compared to the unmarked grapheme (e.g. <c> or <u>).\\
        \label{gl-geminate}\textsc{geminate}\is{geminates} & Long consonant,\is{consonants} as in Old English \textit{cwellan} /kwɛlːan/ `to kill'. Geminates are phonemic in Old English, i.e. they distinguish lexical meaning; compare \textit{cwellan} with \textit{cwelan} /kwɛlan/ `to die'.\\
        \label{gl-given}\makecell[r]{\textsc{given}\\\textsc{information}} & Information that has already been mentioned in the discourse, e.g. \emph{the woman} in reference to someone who has previously been mentioned. The opposite of \glossterm{gl-new}{new} information. \\
        \label{gl-grammaticalization}\makecell[r]{\textsc{grammatical-}\\\textsc{ization}} & Historical\is{grammaticalization} process by which lexical material becomes grammatical, open-class words (like nouns and verbs) become closed-class words (like pronouns and complementizers), and free words become bound morphemes (often through an intermediate \glossterm{gl-clitic}{clitic} stage). Very common in language change (see \citealp{HopperTraugott2003}). \emph{See also} \glossterm{gl-univerbation}{univerbation}. \\
        \label{gl-hdrop}\textsc{/h/-dropping}\is{/h/-dropping} & The non-pronunciation of /h/ in words that historically had an /h/ (and which often still do in standard spelling). See §\ref{LModE-hdrop} and §\ref{ME-hdrop}. \\
        \label{gl-indicative}\textsc{indicative} & A type of \glossterm{gl-mood}{mood} found in all stages of English. The indicative is the default mood and does not convey any particular stance towards the meaning of the clause. \\
        \label{gl-inflection}\textsc{inflection}\is{inflection} & A type of morphology alongside \glossterm{gl-derivation}{derivation} and \glossterm{gl-compounding}{compounding}. Inflection is the process of forming a word-form within a \glossterm{gl-paradigm}{paradigm}, e.g. adding the grammatical morpheme -\emph{s} to the verb \emph{love} to get the third-person singular form \emph{loves}. Unlike derivation, inflection cannot change the category of the word. \\
        \label{gl-lexicon}\textsc{lexicon} & Quite simply, the words of a language (as opposed to its phonetics, syntax, etc.). The corresponding adjective is \textsc{lexical}.\\
        \label{gl-mood}\textsc{mood} & A morphological category applying to verbs, allowing speakers to express their attitude towards an event or statement. All stages of English have an \glossterm{gl-indicative}{indicative} and an imperative mood; Middle and Old English also have a \glossterm{gl-subjunctive}{subjunctive} mood.\\
        \label{gl-morpheme}\textsc{morpheme} & The smallest meaningful morphological unit that cannot be further divided. The word \emph{played} consists of two morphemes: the root \emph{play} and the \glossterm{gl-inflection}{inflectional} morpheme -\emph{ed}. The word \emph{replayable} consists of three morphemes: the \glossterm{gl-derivation}{derivational} prefix \emph{re}-, the root \emph{play}, and the derivational suffix -\emph{able}.\\
        \label{gl-negative-concord}\makecell[r]{\textsc{negative}\\\textsc{concord}}\is{negation} & The use of multiple negative-marked words in a single clause to express only one semantic negation, e.g. \emph{I didn't do nothing to no one} rather than the ``standard'' \emph{I didn't do anything to anyone}. A type of \glossterm{gl-agreement}{agreement}.\\
        \label{gl-new}\makecell[r]{\textsc{new}\\\textsc{information}} & Information that is newly introduced into the discourse, e.g. \emph{a woman} who has not been mentioned before in the discourse. The opposite of \glossterm{gl-given}{given} information.\\
        \label{gl-orthography}\textsc{orthography} & How a language is written: primarily, spelling and punctuation.\is{orthography}\\
        \label{gl-overdotting}\textsc{overdotting} & The practice of adding a \glossterm{gl-diacritic}{diacritic} dot over certain graphemes in editions of Old English texts, representing the palatal\is{palatalization} consonants\is{consonants} /ʃ/, /tʃ/, /dʒ/, and /j/ in Old English as <sċ>, <ċ>, <ċġ>, and <ġ> respectively. Overdotting is purely editorial and is not found in the original Old English manuscripts used in this way. \\
        \label{gl-paradigm}\textsc{paradigm}\is{paradigms} & The set of word-forms that belong to a particular lexical word. For the verb \emph{LOVE}, for instance, the possible forms are \emph{love}, \emph{loves}, \emph{loved}, and \emph{loving}. Paradigms are often displayed as tables. Determining the members of a paradigm is the job of \glossterm{gl-inflection}{inflectional} morphology.\\
        \label{gl-passive}\textsc{passive}\is{passive} & A syntactic construction in which the grammatical subject\is{subjects} expresses the theme or patient of the lexical verb. Often passive sentences have a corresponding active sentence: \emph{the building was destroyed (by him)} is passive, and \emph{he destroyed the building} is active. A major difference between passive and active constructions is that the passive construction is not required to express the agent: in the previous example, \emph{by him} is optional. \\
        \label{gl-pejoration}\textsc{pejoration}\is{pejoration} & Semantic process whereby a word or a phrase acquires negative value judgement connotations.\\
        \label{gl-phoneme}\textsc{phoneme} & A perceptually distinct unit of sound that distinguishes lexical meaning. The classic test for phoneme status is the presence of minimal pairs: since the words \emph{pat} and \emph{bat} differ only in their initial consonant,\is{consonants} we can be sure that /p/ and /b/ are distinct phonemes and not just \glossterm{gl-allophone}{allophones} because their substitution with one another changes the lexical meaning. \\
        \label{gl-phrasal-clitic}\makecell[r]{\textsc{phrasal clitic}\\\textsc{(phrasal affix)}} & A \glossterm{gl-clitic}{clitic}\is{clitics} that attaches to a whole syntactic phrase, rather than to a single word. A Present Day English example is possessive \emph{'s}.\\
        \label{gl-prestige}\textsc{prestige}\is{prestige} & The level of regard or value that a feature, variety, or language is associated with, i.e. whether it is viewed positively or negatively. Typically, the term is used for \textsc{overt} prestige associated with standard varieties. However, other varieties can also be associated with \textsc{covert} prestige within particular groups, associated with group identity.\\
        \label{gl-preterite-present}\makecell[r]{\textsc{preterite-}\\\textsc{present}\is{preterite-presents}} & Class of verb whose present tense form derives historically from a past tense form. All of the survivors from this class in Present Day English are modals:\is{modals} \emph{can}, \emph{shall}, etc.\\
        \label{gl-purism}\textsc{purism}\is{purism} & The belief that a language should only contain elements that are not borrowed\is{borrowings} from other languages.\\
        \label{gl-RP}\makecell[r]{\textsc{Received}\\\textsc{Pronunciation}\\\textsc{(RP)}}\il{English, Received Pronunciation} & The pronunciation characteristic of the Present Day British English\il{English, British} standard variety. Traditionally associated with the south of England, RP is now found all over the world, though it has few if any native speakers. \\
        \label{gl-standardization}\textsc{standardization}\is{standardization} & The historical process by which a standard variety of a language emerges, involving selection, elaboration, codification, and acceptance. See §\ref{EModE-standardization}.\\
        \label{gl-synchronic}\textsc{synchronic} & At a specific moment in time. A study of English verbs in the present day, for example, or a study of the vocabulary used in one of Shakespeare's plays, would be a synchronic study. The opposite of \glossterm{gl-diachronic}{diachronic}.\\
        \label{gl-stative}\textsc{stative} & A class of verbs that convey a condition or state rather than an activity or event. The verbs \emph{LOVE} and \emph{KNOW} are typical stative verbs. \\
        \label{gl-subjunctive}\textsc{subjunctive}\is{subjunctive} & a) The ``subjunctive construction'' in Modern English is a type of embedded clause without a finite verb form, e.g. ... \emph{that he be good}; see §\ref{LModE-subjunctive}. b) The subjunctive in Middle and Old English is a type of \glossterm{gl-mood}{mood}, a category of verb forms; see §\ref{OE-verbs}. Both the modern subjunctive construction and the older subjunctive mood are used to express \textsc{irrealis} meaning: they're typically used to describe a situation or event that is not known to the speaker to have happened at the time of utterance, i.e. not a known fact. \\
        \label{gl-sv}\textsc{s.v.} & Stands for \ili{Latin} \emph{sub voce} `under the expression' or \emph{sub verbo} `under the word'. It means that what follows s.v. is what you should look up in a dictionary. For instance, ``OED, s.v. bumblebee'' is an invitation to look up the word \emph{bumblebee}\is{bumblebees} in the Oxford English Dictionary.\\
        \label{gl-syncretism}\textsc{syncretism} & When the same morphological form is used to express multiple different combinations of features within a paradigm. The form \emph{cut}, for instance, is the infinitive and finite present tense form of the verb in Present Day English, but also the finite past form (\emph{Yesterday I cut my finger}) and the past participle (\emph{I have cut my finger}). \\
        \label{gl-thorn}\textsc{thorn}\is{thorn} & Name of the grapheme <þ>, found in Old English and early Middle English texts, and originating from the runic\is{runes} alphabet. Represents a dental fricative phoneme. Alternates with <ð> ``eth'' in Old English. \\
        \label{gl-univerbation}\textsc{univerbation} & Historical process by which two independent words become a single word, as in \emph{police} $+$ \emph{man} $>$ \emph{policeman}, where the independent word /man/ becomes the bound -/mən/ suffix. \emph{See also} \glossterm{gl-grammaticalization}{grammaticalization}. \\
        \label{gl-V2}\makecell[r]{\textsc{verb-second}\is{verb-second}\\\textsc{(V2)}} & Syntactic rule according to which the finite verb must occupy the second position in the clause, with only one constituent preceding it. An example is modern English \emph{wh}-questions: \emph{Which book \textbf{should} I read?}, not *\emph{Which book I should read?}. Old and early Middle English were characterized by a variety of V2. \\
        \label{gl-yogh}\textsc{yogh}\is{yogh} & Name of the character <ȝ>, found in Old English and early Middle English texts. Pronounced /jɒx/. In Old English, yogh is the normal way of writing the <g> of the Latin alphabet, and so editions of Old English texts usually normalize it to <g>. In Middle English, yogh is a grapheme in its own right, distinct from <g>.\\
    \end{longtable}