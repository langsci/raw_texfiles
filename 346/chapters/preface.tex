\chapter{About this book}

Dear student (and dear teacher),\smallskip\\

First of all, welcome to the history of the English language! We hope it's going to be an interesting ride for all passengers travelling with us this semester.

In this part of the book, we would like to explain what this book is about, why it is the way it is, and how it is intended to be used in the course on the history of the English language which you've most likely just started.

All books on the history of English necessarily focus on how English changes, but ours does so a bit differently. First and foremost, our book differs from most other history of the English language books by not proceeding in chronological order. This means that rather than starting with a language called Indo-European or Germanic and making our way to English as we speak it today, we instead start precisely with English as spoken and written today, with the changes it's been currently undergoing in various parts of the world. And we will be following at least some of the variation we encounter today as we plunge back in time in the course of the semester. There are several reasons why this is the approach we've decided to adopt.

\begin{itemize}
\item We are surrounded by variation in language all the time in our everyday lives. Where does this variation come from? More relevantly for this course, where does Present Day English variation come from? In order to answer questions of this type, we need to go back in time and uncover the history of English. We believe that standing at the (current!) end-result state and moving backwards in time to unravel the origins of the English language is in many ways, though perhaps a bit surprisingly, more logical than starting with Indo-European or Germanic.

\item Many, if not most degrees focusing on English Linguistics and/or English Studies include the study of Present Day English variation. Thus, students often first experience instances of variation found in recent varieties of English even from the point of view of your formal studies. Taking a course on the history of English which starts with Present Day English variation should therefore work better within currently offered programmes in English Studies, and help you to make the history of the English language make more sense to you. Because this is, as we hope you'll find, a course useful for anyone who's going to engage with the English language, for example as a future teacher.

\item Crucially, however, the history of the English language is still being written. As long as (native or non-native) speakers of English exist, the language is bound to vary and, as a result, potentially also change. This is important to bear in mind -- and it's easier to do so if we start our journey in varieties of English used today, and if we have a look at how these have been changing within our own lifespans.

\item Finally, many of you will be taking classes in contemporary or modern literature, written in the English language, standard as well as non-standard. Your literary classes will likely \emph{not} focus primarily on the literary works of Chaucer or Old English literary pieces of the \emph{Beowulf} type. This again makes it likely that your first-hand experience with language variation and change\is{language variation and change (field)} will be found in English of the 20th and the 21st centuries, so that's where our story of English starts as well.
\end{itemize}

\noindent Another important aspect of this book is that it is aimed as an \emph{introduction} to the history of English. In many English programmes currently offered, the history of English is limited to a one-semester course, and this is the setup this book is aimed at. It's definitely not easy to tackle all that there is to tackle within a couple of weeks (see, for example, \citealp[4]{HayesBurkette2017intro}; \citealp[60]{Giancarlo2017}; \citealp[195]{Hayes2017}; \citealp[107]{Hayes2017b}), and it's certainly not easy to teach a course like this within the time given and the range of students taking this course, with so many different experiences and expectations.

Just consider the following. There has been a distinct language that we can call ``English'' for approximately fifteen centuries. The history of English is not limited to how words of the language have been changing, no -- the history of the language importantly involves discussions of various linguistic levels, indeed all of the conceivable levels relevant for our study of language (phonetics, phonology, morphology, syntax, semantics, pragmatics, lexicon). If that sounds like a lot, also bear in mind that to fully comprehend how the language has changed, we must look into all of these levels at \emph{multiple points in time}!

And now think of your own English, or the English you know from films and social media,\is{media} or songs. Do these represent one and the same type of English? Most likely not. So here's another dimension to our history of English: it's found in a range of sociocultural situations, and it's always been found in a range of sociocultural situations. Today's variation is but one slice of the historical cake we could eat. If you're thinking this sounds a bit overwhelming, especially for a one-semester course, then we fully agree with you. This means that the scope necessarily has to be limited to those aspects from the history of the language that are most important. And deciding which aspects those should be is no easy task (see also \citealp[47]{Buck2003}). However, because we focus on explaining the present state and making the prior states accessible to the readers of older texts, we won't be introducing some of the phenomena traditionally taught in classes on the history of English.\footnote{Such as breaking, back umlaut,\is{umlaut} weak and strong adjectives, and Verner's Law.} But please don't worry -- we do make sure to point the student to further references if you/they would like to pursue the subject matter at hand further, on your own or in more specialized courses on the history of the English language, should you have the chance.

So this is why the timeline of the materials presented in this book moves back through time. Another important thing to know about this book is that we don't necessarily use the same texts for the different periods in the history of English as those used in traditional textbooks dealing with this subject. Why is that? Like some other teachers of this course (e.g. \citealp[47]{Buck2003}), we feel that the choice of the texts should be motivated by the following factors. Do the texts demonstrate the phenomena discussed in the relevant chapter? Do the texts represent the language found in a broader range of social and stylistic contexts? Do the texts also reflect any issues that are currently topical within non-linguistic fields, such as various topics covered within gender studies\is{gender studies} or transmedia\is{media} storytelling? We truly hope that considering these questions during our text selection is going to make the texts, and thus also the subject, more obviously useful as well as appealing to as many of you as possible.

And now for the slightly less interesting but still important aspects of this textbook. First and foremost, this is a stand-alone textbook. All the student is intended to need is this open-access textbook and the internet.

\begin{wrapfigure}[2]{l}{0.04\textwidth}
    \vspace{-10pt}
    \centering
    \includegraphics[scale=0.085]{chapters/img/chili.png}
\end{wrapfigure}

\noindent In each chapter you will find exercises, at least one of which presents a possible written type of exercise. There are also exercises that include the use of various online databases. And, of course, there are analytical exercises you may be familiar with from your syntax, morphology, and phonetics and phonology classes. More challenging exercises are marked with a chili pepper, like the one to the left of this paragraph. It's up to your teacher which of these exercises will be used in classes, but the exercises are there, and so is the \hyperref[answers]{key} at the end of the textbook (for those exercises that have right and wrong answers -- not all of them do!). Each chapter is also accompanied by text samples, which can be read for their own interest, investigated for the linguistic features that are discussed in the chapter, used along with the preceding exercises, or ignored entirely -- it's up to the reader and the teacher.

In the hope of making the life of students as well as teachers a bit easier, we also provide you with a fairly detailed \hyperref[glossary]{glossary} of linguistic terms, which you can find at the end of this book. So, if you need to remind yourself of what a \glossterm{gl-phoneme}{phoneme} is, or what a \glossterm{gl-corpus}{corpus}\is{corpora} is, the glossary is one of the places you can turn to.

\subsection*{For the teacher}
Finally, we'd like to share a couple of words aimed primarily at the teacher. All linguistic levels are covered in each chapter of this textbook. In contrast to many history of English textbooks, however, we also explicitly discuss pragmatics and do our best to include at least one example relevant to pragmatics and discourse in each chapter. We feel historical pragmatics\is{historical pragmatics} deserves its rightful place in this textbook, considering the recent work done in the area (and see e.g. \citealp{JuckerTaavitsainen2013}, \citealp{Arnovick2017}). Regarding a related field, semantics, we diverge from the common history of English textbook practice. Instead of focusing on the traditional semantic categories (such as semantic narrowing), we approach semantics opportunistically and only focus on those semantic phenomena that make most sense to discuss within the individual chapters -- and only if they do. For instance, in Chapter \ref{LModE}, where we deal with prescriptivism,\is{prescriptivism} amelioration\is{amelioration} and pejoration\is{pejoration} are also introduced as they work well with this overarching topic. Other than that, semantic differences are touched upon without employing much semantics-specific terminology.
\vspace{10pt}

\noindent We hope you find this book useful on your journey through the history of English and wish you safe travels!
