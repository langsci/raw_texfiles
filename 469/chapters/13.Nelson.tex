\documentclass[output=paper,colorlinks,citecolor=brown]{langscibook}
\ChapterDOI{10.5281/zenodo.14264552}
\author{Brett C. Nelson\affiliation{University of Calgary}\orcid{0000-0001-6665-7193}}
\title{Insertion of [spread glottis] at the right edge of words in Kaqchikel}
\abstract{This paper examines a set of allophonic processes in Kaqchikel (ISO 639-3: cak) to determine the status of the laryngeal feature [spread glottis] in the language. In word final position: underlyingly plain voiceless stops surface as aspirated voiceless stops while sonorants surface as voiceless fricatives. I argue that these are the results of prosodic domain marking whereby [+spread glottis] is inserted at the right edge of every prosodic word. This is despite [spread glottis] not being active in contrasting the phonemes of Kaqchikel. Thus, I claim that a language’s phonology may still manipulate features which are non-contrastive.}
  
\IfFileExists{../localcommands.tex}{%hack to check whether this is being compiled as part of a collection or standalone
   \usepackage{langsci-optional}
\usepackage{langsci-gb4e}
\usepackage{langsci-lgr}

\usepackage{listings}
\lstset{basicstyle=\ttfamily,tabsize=2,breaklines=true}

%added by author
% \usepackage{tipa}
\usepackage{multirow}
\graphicspath{{figures/}}
\usepackage{langsci-branding}

   
\newcommand{\sent}{\enumsentence}
\newcommand{\sents}{\eenumsentence}
\let\citeasnoun\citet

\renewcommand{\lsCoverTitleFont}[1]{\sffamily\addfontfeatures{Scale=MatchUppercase}\fontsize{44pt}{16mm}\selectfont #1}
  
   %% hyphenation points for line breaks
%% Normally, automatic hyphenation in LaTeX is very good
%% If a word is mis-hyphenated, add it to this file
%%
%% add information to TeX file before \begin{document} with:
%% %% hyphenation points for line breaks
%% Normally, automatic hyphenation in LaTeX is very good
%% If a word is mis-hyphenated, add it to this file
%%
%% add information to TeX file before \begin{document} with:
%% %% hyphenation points for line breaks
%% Normally, automatic hyphenation in LaTeX is very good
%% If a word is mis-hyphenated, add it to this file
%%
%% add information to TeX file before \begin{document} with:
%% \include{localhyphenation}
\hyphenation{
affri-ca-te
affri-ca-tes
an-no-tated
com-ple-ments
com-po-si-tio-na-li-ty
non-com-po-si-tio-na-li-ty
Gon-zá-lez
out-side
Ri-chárd
se-man-tics
STREU-SLE
Tie-de-mann
}
\hyphenation{
affri-ca-te
affri-ca-tes
an-no-tated
com-ple-ments
com-po-si-tio-na-li-ty
non-com-po-si-tio-na-li-ty
Gon-zá-lez
out-side
Ri-chárd
se-man-tics
STREU-SLE
Tie-de-mann
}
\hyphenation{
affri-ca-te
affri-ca-tes
an-no-tated
com-ple-ments
com-po-si-tio-na-li-ty
non-com-po-si-tio-na-li-ty
Gon-zá-lez
out-side
Ri-chárd
se-man-tics
STREU-SLE
Tie-de-mann
}
    \bibliography{../localbibliography}
    \togglepaper[13]
}{}


\begin{document}
\maketitle \label{ch13}

\section{Background}\label{Background}
In spoken languages, all speech sounds are made using air that passes through the larynx, specifically the glottis~-- the opening between the vocal folds in the larynx~-- which can be in one of several different states (see e.g. \citealt{ladefoged_1971}, \citealt{ladefoged_1983}). These laryngeal states determine the phonation of the speech sound. Critically, spoken languages make lexical contrasts based on these different phonation states, with many of these contrasts occuring at the segmental or phonemic level.

To account for these phonological contrasts made in the larynx, various sets of laryngeal features have been proposed (e.g. \citealt{halle_stevens_1971, clements_1985, lombardi_1994, iverson_salmons_2007}). Critical to the current study are the features of glottal width [spread glottis] and [constricted glottis], as well as the feature of [voice].

The feature [spread glottis] ([sg]), as defined by \citet{halle_stevens_1971}, refers to the outward displacement of the vocal folds, and thus a widening of the glottis. Sounds typically represented by [+sg] include aspirated stops and voiceless fricatives \citet{vaux_1998} (see also \citealt{ridouane_2006} and especially \citealt{avery_idsardi_2001} for [sg]’s relation to glottal fricative /h/). This is opposed to [constricted glottis] ([cg]), which corresponds to “adduction of the arytenoid cartilages relative to the position of normal voicing” \citep[201--202]{halle_stevens_1971}. [+cg] typically results in ejectives, implosives, glottal stop, and creak \citep{fallon_2002}. “Normal voicing” is called for by the feature [+voice], and refers to the continuous vibration of the vocal folds as air passes through the glottis \citep{lombardi_1994}. 

Languages may use any of these in distinguishing their segments. For example, Spanish (ISO 639-3: spa) uses [voice] in distinguishing its voiced stops /b d g/ from its voiceless stops /p t k/, while Korean (kor) stops exhibit a three-way contrast using two features: [cg] and [sg] to distinguish [+cg] [−sg] fortis stops /p’ t’ k’/ from [−cg] [+sg] aspirated stops /pʰ tʰ kʰ/ from [−cg] [−sg] plain (lenis) stops /p t k/ (see \citetv{chapters/09.HamannMiatto} for a discussion of Korean stop perception). Indeed it is even possible for a language to not utilize any of these laryngeal features contrastively. For example, Plains Cree (crk), has just a single series of three stops differentiated only by Place /p t k/.

Critically for the current study, laryngeal features may be implemented to strengthen a boundary between prosodic domains \citep{cho_2016} and, in the extreme case, inserted by the prosodic structure \citep{iosad_2016}, in effect turning a laryngeally unmarked segment into a marked one (e.g. [+sg] turning [t] into [tʰ] as is argued for final fortition in German (deu) \citep{iverson_salmons_2007}. 

A question remains, however, in that the set of laryngeal features that are inserted for enhancement or edge-marking purposes in a given language is typically restricted to the set of laryngeal features that are contrastive in that language. Can non-contrastive features, or those not active in distinguishing pho-nemes, be active in the marking or enhancement of prosodic domains?

I investigate this question in the rest of this paper, beginning with the next section in which I introduce the language of study, Kaqchikel (cak), and relevant parts of its phonological systems. In \sectref{Kaqchikel allophony}, I present the allophony that occurs among several subsets of Kaqchikel’s phonemes, specifically in different word positions. These cases of allophony lead to my proposal in \sectref{Proposal}. I then discuss this proposal in \sectref{Discussion} in relation to alternative explanations as well as similar cases in other languages. Finally, in \sectref{Conclusion}, I summarize and conclude the study.

\section{Kaqchikel}\label{Kaqchikel}
\subsection{Kaqchikel language background}\label{Kaqchikel language background}
Kaqchikel is a K’ichee’an language in the Mayan language family. It is spoken by about 400,000 first language speakers, primarily in south-central Guatemala (see \figref{map}), but with speakers in diaspora across North America \citep{heaton_xoyón_2016}.

\begin{figure}
\includegraphics[width=.75\textwidth]{figures/Idiomasmap_Guatemala.pdf}
\caption{Map of languages of Guatemala \tiny (adapted from \url{http://commons.wikimedia.org/wiki/File:Idiomasmap_Guatemala.svg} CC BY-SA 3.0 Ignacio Icke)}
\label{map}
\end{figure}

Considered an at risk language by both UNESCO \citep{unesco} and Ethnologue \citep{ethnologue16}, it has co-official language status (with Spanish) in the regions where it is spoken \citep{decreto_19_2003}. As such, Kaqchikel people are guaranteed the right to use their language in public spheres, and to have their language developed, used, and interpreted in educational, medical, and legal domains. Additionally, the Guatemalan government recognizes the official orthography of Kaqchikel and all other Mayan languages in Guatemala as developed and endorsed by the \emph{Academia de las Lenguas Mayas de Guatemala} ‘Academy of the Mayan Languages of Guatemala’ (ALMG) \citep{decreto_1046-87}.
The data in this paper come from ongoing thesis study of the third language acquisition of Kaqchikel’s sound system, specifically its stop consonants. All spoken data presented are produced by first language speakers of Kaqchikel who also speak Spanish and English as additional languages.

\subsection{Kaqchikel phonology}\label{Kaqchikel phonology}
\subsubsection{Phoneme inventory}\label{Phoneme inventory}
An understanding of the phoneme inventory is crucial to answering the question regarding the phonological insertion of a laryngeal feature. The phoneme inventory of Kaqchikel consists of up to 32 distinct phonemes. Of these 32, 22 are consonants, which are shown in Table \ref{consonants}.

\begin{table}
\caption{Kaqchikel consonant phonemes}
\label{consonants}
 \begin{tabular}{l cccccc}
  \lsptoprule
                        & Labial & Alveolar  & Palatal & Velar & Uvular & Glottal\\
  \midrule
  Plain stop            &   p  &    t   &           & k    & q  & ʔ \\
  Glottalized stop      &   ɓ̥   &   t’  &          & k’    & ʛ̥ \\
  Plain affricate       &       &   ts  &    tʃ    &       & \\
  Glottalized affricate &       &   ts’ &    tʃ’   &       & \\
  Fricative             &       &   s   &    ʃ     & x     & \\
  Nasal                 &   m   &   n   &          &       & \\
  Lateral approximant   &       &   l   &          &       & \\
  Approximant           &       &   ɾ   &          &       & \\
  Glide                 &       &       &   j      & w     & \\
  \lspbottomrule
 \end{tabular}
\end{table}

The majority of the consonants of Kaqchikel are obstruents, as there is one series of fricatives at three places of articulation (alveolar /s/, palato-alveolar /ʃ/, and velar /x/), two pairs of affricates, and two series of stops at four places of articulation (bilabial, alveolar, velar, and uvular), as well as a phonemic glottal stop. There is no phonemic glottal fricative, indicative of the absence of contrastive [sg] in Kaqchikel.

Kaqchikel stops and affricates, like those of all other Mayan languages \citep{Bennett:2016}, are differentiated by the laryngeal feature [cg]. One series~-- what I’ll call the plain stops~-- have a [−cg] feature. These are the stops /p t k q/ and the affricates /ts/ and /tʃ/. The other series, which I’ll refer to as glottalized, is specified [+cg] and features mainly ejectives /t’ k’/, but voiceless implosives do surface, particularly the bilabial /ɓ̥/ and uvular /ʛ̥/ \citep{patal_majzul_2000}. The glottalized affricates are both ejectives /ts’ tʃ’/.

The six remaining sonorants can be grouped in a number of different ways, but most relevant to the current study is the distinction between nasals /m n/ on the one hand and non-nasal sonorants /l ɾ j w/ on the other.


\begin{figure}
\includegraphics[height=.2\textheight]{figures/vowels}
\caption{Kaqchikel vowel phonemes; Vowels indicated by a star are lax vowels not distinguished in all varieties of Kaqchikel}
\label{vowels}
\end{figure}

Kaqchikel minimally has a standard five-vowel system with high front /i/, mid front /e/, high back /u/, and mid back /o/ plus a single low vowel /a/. In addition to these five vowels are a series of five lax vowels, each of which has a tense counterpart among those five standard vowels. This tense-lax distinction developed from a short-long distinction in proto-K’ichee’an, one still exhibited in other K’ichee’an languages \citep{Bennett:2016}. However, the lax vowels of Kaqchikel are limited as to where within a word they may surface, namely only in stressed, word-final syllables \citep{rill_2013}. If a lax vowel were to be dislocated outside of that position, it would neutralize to its tense counterpart. Furthermore, the distinction between each of the tense-lax pairs is not present in all varieties of Kaqchikel. Therefore, in stressed, word-final syllables, Kaqchikel has 5--10 tense and lax vowel phonemes, as shown in \figref{vowels}, but only 5 tense vowels outside of those prominent contexts.

\subsubsection{Words and syllables}\label{Words and syllables}
\citet[138]{brown_maxwell_little_2006} describe Kaqchikel as having word stress and that that word stress “is generally on the final syllable of a word”. \citet{Bennett:2016}, reporting on Mayan languages more broadly, states that “final stress is the norm in K’iche[e’]an languages” (\citeyear[495]{Bennett:2016}) (example \ref{chiköp}). This is true even when the final syllable is wholly or partly a suffix (example \ref{chikopi’}).
\begin{multicols}{2}

\ea
    \ea\label{chiköp}
    \gllll \emph{chiköp}    \\
    {}[tʃi.ˈkɔpʰ]   \\
    chiköp  \\
    animal \\
    \glt ‘animal’  
\columnbreak
    \ex\label{chikopi’}
    \gllll \emph{chikopi’} \\
    {}[tʃi.ko.ˈpiʔ]    \\
    chiköp-i’ \\
    animal-\textsc{pl} \\
    \glt ‘animals’ 
    \z
\z

\end{multicols}

In multi-word phrases, \citet{brown_maxwell_little_2006} describe Kaqchikel as having phras-al prominence on the final syllable of the final word of every phrase. The realization of this prominence differs depending on the type of phrase, with declaratives and content (wh-)questions having a falling tone over the final syllable of the final word \citep{nelson_2020}, while polar (yes/no) questions instead have a rising tone \citep{brown_maxwell_little_2006}.

The prosodic emphases above show that Kaqchikel consistently places prominence on the right edge of prosodic domains. This becomes crucial when discussing allophony at the right edge in the following section.

\section{Kaqchikel allophony}\label{Kaqchikel allophony}
\largerpage
\subsection{Stop allophony}\label{Stop allophony}
In Kaqchikel, the right edge of words is also the conditioning environment for position-based allophony of several sound classes. In this subsection I present and discuss allophony of stops. Then, in the following section I show sonorants and their allophony.

As mentioned in \sectref{Phoneme inventory}, there are four plain stop consonants in Kaq-chikel. These are bilabial /p/, alveolar /t/, velar /k/, and uvular /q/. Plain stops may and do appear in either position of a syllable (onset or coda) and any position of a word (initial, medial, or final), and appear in roots as well as affixes. However, in word-final position, the allophone that surfaces is not a plain, voiceless, unaspirated stop such as [p] (\figref{p_onset}), but instead an aspirated stop [pʰ] (\figref{p_final}).

\begin{figure}
\includegraphics[width=0.8\textwidth]{figures/p_onset.pdf}
\caption{\emph{pich’} /pitʃ’/ [ˈpitʃ’] ‘tender corn’ spoken by NKS1}
\label{p_onset}
\end{figure}

\begin{figure}
\includegraphics[width=0.8\textwidth]{figures/p_final.pdf}
\caption{\emph{sip} /sip/ [ˈsipʰ] ‘tick’ spoken by NKS1}
\label{p_final}
\end{figure}

Of course this is not restricted to just the bilabial stop /p/. It also occurs with alveolar /t/: [t] surfaces in non-final positions (\figref{t_onset}) while [tʰ] does so in word final position (\figref{t_final}); the plain velar stop /k/: unaspirated or lightly aspirated [k] at the beginning of a word (\figref{k_onset}) vs. heavily aspirated [kʰ] at the end of a word (\figref{k_final}); and the plain uvular stop /q/: lightly aspirated [q] in onset (\figref{q_onset}) and heavily aspirated [qʰ] in word-final coda (\figref{q_final}).

\begin{figure}
\includegraphics[width=0.8\textwidth]{figures/t_onset.pdf}
\caption{\emph{tix} /tiʃ/ [ˈtiʃ] ‘elephant’ spoken by Kawoq}
\label{t_onset}
\end{figure}

\begin{figure}
\includegraphics[width=0.8\textwidth]{figures/t_final.pdf}
\caption{ \emph{xet} /ʃet/ [ˈʃetʰ] ‘hair whorl’ spoken by Kawoq}
\label{t_final}
\end{figure}


\begin{figure}
\includegraphics[width=0.8\textwidth]{figures/k_onset.pdf}
\caption{ \emph{kem} /kem/ [ˈkem] ‘weaving’ spoken by NKS1}
\label{k_onset}
\end{figure}

\begin{figure}
\includegraphics[width=0.8\textwidth]{figures/k_final.pdf}
\caption{ \emph{ach’ek} /ʔatʃ’ek/ [ʔa.ˈtʃ’ekʰ] ‘dream’ spoken by NKS1}
\label{k_final}
\end{figure}


\begin{figure}
\includegraphics[width=0.8\textwidth]{figures/q_onset.pdf}
\caption{ \emph{qey} /qej/ [ˈqeç] ‘our teeth’ spoken by Kawoq}
\label{q_onset}
\end{figure}

\begin{figure}
\includegraphics[width=0.8\textwidth]{figures/q_final.pdf}
\caption{ \emph{t’uq} /t’uq/ [ˈt’uqʰ] ‘setting hen’ spoken by Kawoq}
\label{q_final}
\end{figure}

Through these examples, we see that all plain stops in Kaqchikel exhibit positional allophony. In onset position, stops surface as voiceless and unaspirated (T). In word-final position, the same stop phonemes instead surface as voiceless and heavily aspirated (Tʰ).

\subsection{Sonorant allophony}\label{Sonorant allophony}
The four non-nasal sonorants (continuant sonorants) /l ɾ j w/  also exhibit positional allophony at the ends of words in Kaqchikel. However, it is realized differently than adding an aspiration burst to the end of the articulation of the voiceful sonorant, as is the case for the plain stops.

Examine the following examples, both of which contain the root \emph{-söl} /-sɔl/ ‘to tie’. The key segment here is the lateral /l/. In \figref{l_onset}, I show the third person plural antipassive \emph{yesolon}, where /l/ surfaces as [l] in medial onset position. Compare this to \figref{l_final} \emph{nkisöl}, where /l/, now in final position, surfaces as a voiceless alveolar lateral fricative [ɬ]. 

\begin{figure}
\includegraphics[width=0.8\textwidth]{figures/l_onset.pdf}
\caption{ \emph{yesolon} /jesolon/ [je.so.ˈlon] ‘they untie’ spoken by B’alam}
\label{l_onset}
\end{figure}

\begin{figure}
\includegraphics[width=0.65\textwidth]{figures/l_final.pdf}
\caption{ \emph{nkisöl} /nkisɔl/ [nki.ˈsɔɬ] ‘they untie it’ spoken by B’alam}
\label{l_final}
\end{figure}

Again, looking to other members of this class, similar patterns emerge. With the rhotic /ɾ/, a tap (optionally a trill) surfaces in onset position, as seen in \figref{r_onset}, but in word-final coda, this phoneme surfaces as a voiceless fricative [ʂ] (\figref{r_final}).

\begin{figure}
\includegraphics[width=0.65\textwidth]{figures/r_onset.pdf}
\caption{\emph{rut’} /ɾut’/ [ˈɾut’] ‘receipt’ spoken by Yab’un}
\label{r_onset}
\end{figure}

\begin{figure}
\includegraphics[width=0.65\textwidth]{figures/r_final.pdf}
\caption{\emph{q’or} /ʛ̥oɾ/ [ˈʛ̥oʂ] ‘atole (corn beverage)’ spoken by Yab’un}
\label{r_final}
\end{figure}

Next, we have the glides /j/ and /w/. \figref{y_onset} shows /j/ surfacing as [j] in onset, while \figref{y_final} shows this sonorant phoneme becoming a voiceless obstruent [ç] in word-final position. 
Similarly, /w/ is a (voiceful) sonorant glide as an onset (\figref{w_onset}), but a voiceless fricative as a word-final coda (\figref{w_final}).

\begin{figure}
\includegraphics[width=0.65\textwidth]{figures/y_onset.pdf}
\caption{\emph{yot’} /jot’/ [ˈjot’] ‘dimple’ spoken by B’alam}
\label{y_onset}
\end{figure}

\begin{figure}
\includegraphics[width=0.8\textwidth]{figures/y_final.pdf}
\caption{\emph{k’oy} /k’oj/ [ˈk’oç] ‘spider monkey’ spoken by B’alam}
\label{y_final}
\end{figure}

\begin{figure}
\includegraphics[width=0.8\textwidth]{figures/w_onset.pdf}
\caption{ \emph{tew} /tew/ [ˈtef] ‘cold’ spoken by Kawoq}
\label{w_onset}
\end{figure}

\clearpage

\begin{figure}
\includegraphics[width=0.8\textwidth]{figures/w_final.pdf}
\caption{ \emph{ntewär} /ntewəɾ/ [nte.ˈwəʂ] ‘it gets cold’ spoken by Kawoq}
\label{w_final}
\end{figure}


\subsection{Other sounds in final position}\label{Other sounds in final position}
In the previous subsection, I showed that plain stops and non-nasal sonorants both exhibit particular cases of positional allophony where word-final occurrences differ substantially from non-word-final counterparts. What happens to the other sound classes in word final position? The answer to this question helps circumscribe the main pattern of allophony at the right edge of words in Kaqchikel. Therefore, in this subsection, I show examples of these sound classes in these positions.

First, let’s look at a glottalized stop. In \figref{t’} the glottalized alveolar stop /t/ occurs in both (word-medial) onset and word-final coda. Both occurrences result in surface forms that are voiceless ejectives [t’]. Other glottalized stop phonemes behave similarly.
\newpage

\begin{figure}
\includegraphics[width=0.65\textwidth]{figures/t'.pdf}
\caption{ \emph{it’ot’} /ʔit’ot’/ [ʔi.ˈt’ot’] ‘y’all’s conch’ spoken by NKS1}
\label{t’}
\end{figure}

Next, consider the palato-alveolar fricative /ʃ/. It appears as an onset in \figref{x_onset}, but as a word-final coda in \figref{x_final}. Again, both surface forms are the same: [ʃ], the voiceless palato-alveolar fricative. There is no positional allophony for fricatives, though word-final fricatives are considerably longer than fricatives in other word positions.

\begin{figure}
\includegraphics[width=0.65\textwidth]{figures/x_onset.pdf}
\caption{\emph{xit’} /ʃit’/ [ˈʃit’] ‘It was filled well.’ spoken by Aq’ab’al}
\label{x_onset}
\end{figure}

\begin{figure}
\includegraphics[width=0.65\textwidth]{figures/x_final.pdf}
\caption{\emph{b’ix} /ɓ̥iʃ/ [ˈɓ̥iʃ] ‘song’ spoken by Aq’ab’al}
\label{x_final}
\end{figure}

Finally, I show an example of the labial nasal /m/. \figref{m_onset} has /m/ as it appears in onset, while \figref{m_final} has it in word-final coda. There is no difference between them other than the longer duration exhibited by the final [m] in \figref{m_final}. The other nasal phoneme /n/ behaves similarly.

\begin{figure}
\includegraphics[width=0.65\textwidth]{figures/m_onset.pdf}
\caption{\emph{nimeq’} /nimeʛ̥/ [ni.ˈmeʛ̥] ‘It is warmed.’ spoken by Aq’ab’al}
\label{m_onset}
\end{figure}

\begin{figure}
\includegraphics[width=0.8\textwidth]{figures/m_final.pdf}
\caption{ \emph{kem} /kem/ [ˈkem] ‘weaving’ spoken by Aq’ab’al}
\label{m_final}
\end{figure}

Finally, there is the case of the alveolar and palato-alveolar affricates.\footnote{As pointed out by an anonymous reviewer, affricates were notably missing from this analysis when I originally presented it at the 2021 Epenthesis Workshop. With many thanks to this reviewer, I have added the following data on glottalized and plain affricates to complete this survey of Kaqchikel's consonant inventory.} As with the previously discussed stops, these affricates do contrast on [cg], with the glottalized affricates surfacing as ejectives [ts’ tʃ’] in both onset and word-final coda positions. Examples of the glottalized palato-alveolar affricate /tʃ’/ are shown in Figures \ref{ch’_onset} and \ref{ch’_final}.

\begin{figure}
\includegraphics[width=0.65\textwidth]{figures/ch'_onset.pdf}
\caption{\emph{ach’ek} /ʔatʃ’ek/ [ʔa.ˈtʃ’ekʰ] ‘dream’ spoken by B’alam}
\label{ch’_onset}
\end{figure}

\begin{figure}
\includegraphics[width=0.65\textwidth]{figures/ch'_final.pdf}
\caption{\emph{pich’} /pitʃ’/ [ˈpitʃ’] ‘tender corn’ spoken by B’alam}
\label{ch’_final}
\end{figure}

The plain, non-glottalized affricates /ts tʃ/, on the other hand do exhibit positional variation, with word-final affricates being aspirated, aligning with descriptions of other Mayan languages such as Ch’ol (ctu) \citep{warkentin_brend_1974} and Yukatek (yua) \citep{anderbois_2011} which have all plain oral stops and affricates aspirating in word-final position. However, as can be seen in Figures \ref{ch_onset} and \ref{ch_final}, the noisy period of frication and aspirtion after the release burst is longer in word-final positions. Thus, plain affricates, patterning phonologically with plain stops, do participate in a similar process to those stops, aspirating in word-final positions.

\begin{figure}
\includegraphics[width=0.65\textwidth]{figures/ch_onset.pdf}
\caption{\emph{nuchäp} /nutʃəp/ [nu.ˈtʃəpʰ] ‘I grab it.’ spoken by B’alam}
\label{ch_onset}
\end{figure}


\begin{figure}
\includegraphics[width=0.65\textwidth]{figures/ch_final.pdf}
\caption{\emph{b’uch} /ɓ̥utʃ/ [ˈbutʃʰ] ‘belly of an animal’ spoken by B’alam}
\label{ch_final}
\end{figure}


The examples in this subsection have shown that despite the difference between the final and non-final allophones of the plain stop (including affricates) and sonorant classes, the other consonantal classes of the language do not exhibit similar types of allophony. The glottalized stops, (voiceless) fricatives, and (modally voiced) nasals are just that: glottalized stops, fricatives, and nasals in all positions. Thus, the positional allophony of Kaqchikel consonants is phonologically restricted to those first two classes: plain stops and non-nasal sonorants. In the next section, I offer a proposal as to why this is the case.

\section{Proposal}\label{Proposal}
\subsection{[+spread glottis] insertion}\label{[spread glottis] insertion}
To account for the facts that in Kaqchikel, certain sound classes (plain stops and sonorants), but not others (glottalized stops, fricatives, nasals) exhibit positional allophony at the right edge of words, I propose that this prosodic domain boundary is being marked (cf. \citetv{chapters/05.RubinKaplan} for discussion of prosodic domain specific epenthesis). Specifically, the right edge of every word in Kaqchikel is potentially marked by the insertion of a [+sg] feature.

Similar to languages like German \citep{iverson_salmons_2007}, [+sg] insertion in Kaqchikel causes unmarked stops to become aspirated. In that language, [+sg] causes final fortition of laryngeally unmarked stops [T] into aspirated stops [Tʰ]. In German, this neutralizes the distinction between them and the already [+sg] marked aspirated stops [Tʰ] present in the language’s consonant inventory. In Kaqchikel, when [+sg] is inserted into a plain stop [T], it also creates an aspirated stop [Tʰ] (i.e. /p/ becomes [pʰ], /t/ becomes [tʰ], /k/ becomes [kʰ], and /q/ becomes [qʰ]). However, as the contrastive laryngeal feature in Kaqchikel is [cg], there is no contrast neutralization as there is in German. Instead, the contrast between plain stops [T] and glottalized stops [T’] is enhanced in word final position. These differences in positional allophony are shown in Table \ref{insertions}.

\begin{table}
\caption{Final [+sg] insertion comparison}
\label{insertions}
 \begin{tabular}{l cccc}
  \lsptoprule
        &    \multicolumn{2}{c}{German}   &    \multicolumn{2}{c}{Kaqchikel} \\\cmidrule(lr){2-3}\cmidrule(lr){4-5}
                &    Non-final   &   Final   &   Non-final  &   Final\\\midrule
    Unmarked    &   T   &   \multirow{2}{*}{Tʰ} &   T   &   Tʰ  \\
    Marked      &   Tʰ  &                       &   T’  &   T’  \\
  \lspbottomrule
 \end{tabular}
\end{table}

For the Kaqchikel non-nasal sonorants, when they appear in the word-final position where they receive the [+sg] feature, they cannot so easily become aspirated stops. Instead, these continuant sounds become obstruents. Continuant obstruents are fricatives, and fricatives with [+sg] are voiceless fricatives \citep{vaux_1998}. Thus each sonorant becomes a voiceless fricative at the same place of articulation: /l/ becomes [ɬ], /ɾ/ becomes [ʂ], /j/ becomes [ç], and /w/ becomes [f].

This insertion of [+sg] fails to cause positional allophony of the other classes of consonants, even though they can surface in word-final positions. This can be explained for each of the consonant classes. First, the glottalized consonants already have an antithetical laryngeal specification of [+cg] making insertion of [+sg] impossible for those consonants. Next, the fricatives are already continuant obstruents. Insertion of [+sg] on these fricatives does not cause any change to the sounds, as voiceless fricatives are the end result of continuants being specified for [+sg] \citep{vaux_1998}. Finally, the nasals are inherently voiced, despite [voice] not being a distinctive feature for consonants in Kaqchikel. This is due to licensing restrictions against laryngeal specification when [nasal] is specified \citep{ito_mester_padgett_1995}. Thus [+nasal] sounds may not accept the insertion of [+sg] at the ends of words in Kaqchikel.

\subsection{Implications}\label{Implications}
The proposal given in \sectref{[spread glottis] insertion} carries several key implications for Kaqchikel phonology and phonological theory more generally. These implications arise from three points: the non-contrastiveness of [sg] in Kaqchikel, the selective insertion of [+sg] dependent upon consonantal class, and the enhancement of the laryngeal contrast between the plain and glottalized stops in word-final position. This subsection addresses each of these points in turn.

The main research question raised in \sectref{Background} asks whether non-contrastive features can be active in the marking of prosodic domains. The data shown in the previous section demonstrate that [sg] is active in domain marking, despite not being active in distinguishing the phonemes of Kaqchikel. Unless this is a purely post-phonological, prosodic process, then this would challenge the tenet of the Contrastivist Hypothesis that only contrastive features are active during phonological computation \citep{hall_2007, Dresher:2009}. I discuss alternative analyses that maintain the Contrastivist Hypothesis in \sectref{Alternative explanation: Contrastivist Hypothesis}.

The possibility that this is a post-phonological, prosodic process is complicated by the fact that [+sg] insertion is phonologically selective. Only some of the consonantal classes readily accept the new laryngeal feature, namely the plain (laryngeally unspecified) stops and non-nasal (continuant) sonorants. The other consonantal classes do not accept [+sg]; the glottalized stops are already specified for an opposing [+cg] laryngeal feature, while the nasals do not license [+sg], but instead [+voice] in all positions. This shows that the insertion process is more phonologically controlled than it would be under a simple post-phonological, prosodic process.

Finally, the Kaqchikel data illustrate a process of contrast enhancement, differing from the case of German contrast neutralization. Enhancement, after \citet{keyser_stevens_2006}, refers to cases where distinctive features, which are always accompanied by phonetic gestures, receive secondary gestures in order to supplement its insufficiently salient primary gestures. In this case, the aspiration gestures of [+sg] are introduced to plain stops in final position, leading to enhancement of the contrast between those plain stops and the laryngeally marked glottalized stops in that position. Whether this enhancement was the initial impetus for the insertion of [+sg], that subsequently spread to the sonorant class, is a matter for future research.

\section{Discussion}\label{Discussion}
\subsection{Alternative explanation: Element Theory}\label{Alternative explanation: Element Theory}
\subsubsection{Introduction to Element Theory}
The insertion of [+sg] is not the only possible explanation for the positional allophony observed in Kaqchikel. Indeed, \citet{nasukawa_backley_2018} examine Kaqchi-kel positional allophony of stops and sonorants under the framework of Element Theory. In this subsection I discuss Element Theory and their proposal so that I can then compare it to the one I offer in \sectref{Proposal}.

\citet{harris_lindsay_1995} present Element Theory as a theoretical framework of phonology that represents phonological contrasts by way of elements rather than features. Instead of more than a dozen \emph{SPE}-like features, Element Theory uses six elements to represent all of the contrasts of spoken languages. To accomplish this feat, instead of a feature having a relatively restricted set of accompanying gestures and cues, multiple instances of an element may combine within a segment to call for the necessary cues of the language \citep{nasukawa_backley_2018}. Therefore, the elements are founded upon the acoustic properties of sounds.

Each of the six elements of Element Theory is typically represented as single uppercase letter (e.g. |A|) and can be bundled together with other elements using brackets (e.g. [AH], \citealt{harris_lindsay_1995}. Each element is associated with one or more consonant categories as well as one or more vowel categories. 

Three of the elements are founded upon properties of resonance \citep{nasukawa_backley_2018}. These are |A|, |I|, and |U|. The element |A| is labeled by \citet{nasukawa_backley_2018} as “Mass” for its mass of spectral energy and is associated with uvular or pharyngeal consonants and non-high vowels. |I| is called “Dip” for its spectral trough and is associated with dental and palatal consonants as well as front vowels. The final resonance element is |U| “Rump”, which is associated with both labial and velar consonants plus rounded vowels. These elements together perform roles similar to place features in featural theories of phonology.

The remaining three elements, |ʔ|, |H|, and |N| ,  are for non-resonant properties and are of primary importance to the current issue of Kaqchikel allophony \citep{nasukawa_backley_2018}. |ʔ| carries the label “Edge” and provides the occlusion property for obstruents, as well as laryngealization of creaky voice vowels. |H| is called “Noise” and is associated with voicelessness, aspiration, and frication in consonants, but high tone in vowels. Finally |N| “Murmur” refers to nasality and voicing of consonants, and when it is part of a vowel it may also signal nasality or low tone. These elements act in ways that manner and laryngeal features do in featural theories.

\subsubsection{Element Theory in Kaqchikel}
Based on these descriptions of the elements alone, one can see that |ʔ| and |H| are going to be most relevant for Kaqchikel allophony. |ʔ| is indeed the difference between the plain and glottalized stops of Kaqchikel, with the glottalized series having two (one for occlusion and one for glottalization) and the plain series having one (for occlusion), Additionally all voiceless obstruents also carry |H|.

\cite{nasukawa_backley_2018} propose, however, that in addition to these base attributions, the elements are also at work in creating the positional allophony observed in Kaqchikel. Specifically, they claim that the right edge of the domain is prominent in Kaqchikel, similar to how the left edge of prosodic domains is prominent in English. Thus, similar to how English voiceless/lenis stops aspirate at the left edge of words, Kaqchikel plain stops aspirate at the right edge of words. Therefore, they propose, a fortition element of |H| is inserted in the prominent, domain-final position.

It should be noted, as \citet[629]{nasukawa_backley_2018} do: “not all versions of Element Theory allow multiple tokens of an element in an expression”, however the version \citet{nasukawa_backley_2018} present for their analysis of Kaqchikel (named Precedence-free Phonology) does. While other models of Element Theory, including \citet{harris_lindsay_1995}, mark headedness on elements (visualized as underlining of the element) in order to account for finer distinctions in acoustic properties among segments, \citet{nasukawa_backley_2018} accomplish this distinction through the use of multiple tokens of an element. In the case of unaspirated voiceless stops versus aspirated voiceless stops, the former has a single |H| while the latter has two |H|, corresponding to the extra noise component present in stop aspiration.

The insertion of the fortition element |H|, by its nature within Element Theory, causes various outcomes. The clearest of these is aspiration of stops. Moreover, \citet{nasukawa_backley_2018} propose, |H| causes the change of voiceful sonorants into voiceless fricatives. However, one |H| element is not enough to cause this change, as the sonorants do not carry a |H| on their own. One |H| element would lead to voiceful/lenis fricatives. But voiceful fricatives are not produced; voiceless ones are. As \citet[630]{nasukawa_backley_2018} propose: “voiceless fricatives and aspirated stops are represented ... each with two tokens of |H|”. Therefore, an additional |H| element would be required to form the observed voiceless (fortis) fricatives.

\begin{figure}
    \begin{forest} GP1
    sn edges/.style={for tree={ parent anchor=south,child anchor=north}}, 
    sn edges
    [,phantom,for tree={s sep=5ex, l sep=.5ex}
    [|A|, for tree={calign=last}
    [|I|
    [|H|
    [|U|
    [|ʔ|
    [\textcolor{red}{|H|},tikz={\node [draw,red,inner sep=0,fit to=tree]{};}]
    [|ʔ|]]
    [|U|]]
    [|H|]]
    [|I|]]
    [|A|]]]
    \end{forest}
    \caption{Change in structure of [ik] to [ikʰ] via boundary marking, cf Figure 6 in \citet{nasukawa_backley_2018}. The added element |H| (shown and boxed in red) indicates aspiration of the stop.}
    \label{ETaspiration}
\end{figure}


Through this fortition process, |H| is involved in the domain-final allophony of both stops and sonorants, but in different quantities. One fortition element |H| is added to turn the plain, voiceless stops into aspirated stops, but one structural element |H| is needed to turn the sonorants into fricatives and then another fortition element |H| is added to form voiceless fricatives. This contradicts the proposal of \citet{nasukawa_backley_2018} that the addition of a \emph{single} fortition element |H| domain-finally can account for allophony in both plain voiceless stops and sonorants in Kaqchikel. Furthermore, the tree structure diagram for [ikʰ], shown here as \figref{ETaspiration}, shows \emph{three} |H| element nodes: one for obstruency, one for voicelessness (doubled together in the tree), and one for aspiration. The linear representation they give for the consonantal portion of this syllable (i.e. [kʰ]) is [[[Hʔ]\textsubscript{ʔ}U]\textsubscript{U}H]\textsubscript{H} (\citeyear[634]{nasukawa_backley_2018}). However, they give neither a tree structure nor a linear representation for the structure of any of the spirantized sonorants.

Expanding outside of the observed allophonic patterns in Kaqchikel, the analysis of \citet{nasukawa_backley_2018} does not attempt to account for the fact that the other Kaqchikel consonant classes \emph{do not} participate in similar cases of word-final allophony. If the Noise element |H| were introduced to any segment that appears in the relevant position (word-finally), it should have no problem combining with the structural elements already present. 

For the glottalized stops, the structural element bundle is minimally [ʔ ʔ H]: one |ʔ| for the occlusion, another |ʔ| for the glottalization, and one |H| for their voicelessness.\footnote{The only mention of the structure of glottalized stops appears in a footnote to show that doubling of elements is not only permitted, but necessary to build up the phonemic contrasts exhibited by Kaqchikel \citep[630]{nasukawa_backley_2018}.} The fortition element should attach to the structural elements of the glottalized stops. However, no allophony or change in elements is observed; Kaqchikel glottalized stops are still glottalized and unaspirated in final position. Therefore the fortition element |H| either has no effect on glottalized stops or is somehow prevented from being inserted when these sounds occur in the prominent position. However, there is no \emph{a priori} reason Element Theory would prohibit this combination. Moreover, the structural elements making up voiceless glottalized stops already consist of this very combination.\footnote{Rather it is |A| that \citet{nasukawa_backley_2018} identify as incompatible with |H| due to incompatibility of their articulatory gestures.}

Phonemic voiceless fricatives appearing at the right edge should, however, have no issues combining their structural |H H| with the fortition element |H| from the domain boundary. Nevertheless, \citet{nasukawa_backley_2018} make no mention of how the phonemic fricatives actually play into their analysis.\footnote{\citet{nasukawa_backley_2018} describe Kaqchikel as having five contrastive fricative phonemes /s~ʃ~x~χ~h/, yet there are only three: /s~ʃ~x/ (cf Table \ref{Phoneme inventory} and \cite{brown_maxwell_little_2006}).}

Finally, we have the case of the nasals, the two of which do not exhibit allophony at the right edge.\footnote{\citet{nasukawa_backley_2018} also describe Kaqchikel as having three contrastive nasals, adding /ŋ/ to the two /m~n/ that I show in Table \ref{consonants}. This is not true of Kaqchikel. Alveolar /n/ \emph{is} produced as [ŋ] in some environments, by some speakers, but not contrastively.} Again, no discussion of them is given in \citet{nasukawa_backley_2018}, but presumably, the nasality of these sounds is represented by the |N| element. The possible effects of fortition element |H| on structural element |N| are neither explained nor predicted by \citet{nasukawa_backley_2018}, and I struggle to provide an explanation of my own. I would expect, however, that devoiced\slash aspirated nasals are a possible result of |N| + |H|. Devoiced nasals, however, do not occur here or in any position in Kaqchikel. So again, the domain boundary fortition element |H| appears not to have any effect on the class of nasal consonants in Kaqchikel.

To summarize, \citet{nasukawa_backley_2018} propose that Kaqchikel plain stops and sonorants become their word-final allophones of aspirated stops and voiceless fricatives because the right-edge is prominent in Kaqchikel. This prominence is realized through the addition of a fortition element |H|. However, in order for this analysis to work, one |H| must be added to the voiceless stops to induce aspiration, while two |H| are added to the sonorants to induce obstruency and then voicelessness. This lack of parsimony, combined with the absence of explanation for the other consonants in Kaqchikel not exhibiting any similar patterns of allophony, makes the Precedence-free Phonology analysis of \citet{nasukawa_backley_2018} less preferable than the current analysis that [+sg] insertion adequately derives the surface forms of all Kaqchikel consonant classes. This does not, however, preclude the possibility that another version of Element Theory can provide an adequate explanation for the observations of Kaqchikel. 

\subsection{Alternative explanation: Contrastivist Hypothesis}\label{Alternative explanation: Contrastivist Hypothesis}
Further alternatives to the analysis presented here may support the Contrastivist Hypothesis of \citet{Dresher:2009}.

As mentioned above, the Contrastivist Hypothesis holds that only contrastive features may participate in the lexical phonological processes of a language. “Lexical phonology,” as \citet[118]{Dresher:2009} explains, “interacts with the morphology and the lexicon, and exhibits properties such as cylcic application, restriction to derived environments, and exceptions”. In order for [sg] to participate in lexical phonology, it must be one of the contrastive features of the language's phonology. I do not take this to be the case in Kaqchikel. 

However, as  contrastive [sg] can be achieved within the Contrastivist Hypothesis. The Successive Division Algorithm (SDA) may produce feature sets not immediately apparent upon first glance at a language's phoneme inventory. If [sg] is used in dividing the contrastive segments of Kaqchikel early in the algorithm, namely in separating the three phonemic fricatives /s ʃ x/ from the rest of the [−sg] consonants by specifying the fricatives as [+sg]. \citet{wax_cavallaro_2020} points this out as a possible solution for a similar case in Tz’utujil (described in greater detail in \sectref{Tz’utujil}).

If this were the SDA path used in Kaqchikel, [sg] would indeed be a contrastive feature of Kaqchikel, and would therefore be able to participate within the lexical phonology of the language. However, separate positive evidence for the presence of contrastive [sg], other than the processes described in the current research, would be desirable before continuing with this analysis. For instance, if [sg] were a contrastive feature, then the lack of a phonemic glottal fricative /h/, as the placeless [+sg] consonant, would be curious (cf. the distribution of /h/ and aspiration in English and Korean as evidence for contrastive [sg] in those languages \citep{davis_cho_2003}). 

The basal specification of [+sg] for Kaqchikel fricatives additionally contradicts the proposal of \citet[2]{kehrein_golston_2004} that “[a]n onset, nucleus, or coda has a single, unordered set of laryngeal features”. If both this proposal and the [sg] division of Kaqchikel fricatives were true, then clusters containing a fricative and a non-fricative would not be possible in Kaqchikel. However, these clusters are allowed to surface in both native words (example \ref{native_xcluster}) and loanwords (example \ref{loan_xcluster}). Thus, I conclude, the proposed division of fricatives using [+sg] as their contrastive feature is untenable for Kaqchikel.

\begin{multicols}{2}
\ea
    \ea\label{native_xcluster}
    \glll \emph{\textbf{xk’}is}    \\
    {}[ˈʃk’is]   \\
    ʃ-\emptyset-k’is  \\
    \glt ‘it was finished’  
\columnbreak
    \ex\label{loan_xcluster}
    \glll \emph{wa\textbf{kx}} \\
    {}[ˈwakx]    \\
    wakx \\
    \glt ‘cow’ 
    \z
\z
\end{multicols}

Another solution, which avoids the issue of [sg] being a contrastive feature of Kaqchikel altogether, is to place the processes of aspiration and spirantization outlined in \sectref{Proposal}, into the postlexical phonology of Kaqchikel. As further explained in \citet[118]{Dresher:2009}: “Postlexical phonology follows the lexical phonology and does not observe the above restrictions having rather properties one would associate with phonetics rules”. This could be the case in Kaqchikel, as contrast enhancement is often a phonetic, rather than phonological, process. However, as explained in \sectref{Implications}, there are exceptions to the aspiration and spirantization rules of Kaqchikel. Exceptions are indicative of lexical phonology as opposed to the exceptionless rules of postlexical phonology. Indeed, the processes described in the current research depend on the phonological structure of the segments at the right edge of the word in Kaqchikel, placing them firmly within the scope of lexical phonology.



\subsection{Other languages}\label{Other languages}
\subsubsection{Tz’utujil}\label{Tz’utujil}
Processes similar, yet not identical, to Kaqchikel’s allophony can be observed in related languages. \citet{Bennett:2016} reports that word-final plain stops are typically aspirated across the Mayan family, claiming it’s a regular, obligatory process for all but two Mayan languages. Additionally, final devoicing/spirantization of sonorants is found across the K’ichee’an, Tseltalan, and Huastecan branches of the family \citep{Bennett:2016}.

\citet{wax_cavallaro_2020} analyzes a specific case very similar to the Kaqchikel case in Tz’utujil (tzj), a closely related K’ichee’an language, citing data from \citet{dayley_1985}. Of its identical consonant inventory, she also observes that plain stops aspirate in word-final position (example \ref{tzj_tut}), but also in word-internal codas (i.e. syllable-final position) (example \ref{tzj_saqb’ach}).
\begin{multicols}{2}
\ea
    \ea\label{tzj_tut}
    Tz’utujil   \citep{dayley_1985}\\
    Word-final aspiration \\
    \gll\emph{tut}    \\
    {}[tʊtʰ]   \\
    \glt ‘palmera’ 
    \columnbreak
        \ex\label{tzj_saqb’ach}
        Tz’utujil \citep{dayley_1985}\\
        Syllable-final aspiration \\
        \gll\emph{saqb’ach} \\
        {}[saqʰɓat͡ʃʰ]  \\
        \glt ‘hailstone’ 
    \z    
\z
\end{multicols}

Similarly, Tz’utujil also has word-final spirantization/devoicing (example \ref{tzj_way}) but also coda spirantization/devoicing more generally (example \ref{tzj_moysees}). Thus, seemingly, the insertion of [+sg] is occurring at the right edge of every syllable in Tz’utujil, as opposed to the process only occurring at the right edge of every word (or every word-final syllable) in Kaqchikel.
\begin{multicols}{2}
\ea
    \ea\label{tzj_way}
    Tz’utujil  \citep{dayley_1985} \\
    Word-final devoicing \\
    \gll\emph{way}    \\
    {}[waj̥]   \\
    \glt ‘tortilla’ 
    \columnbreak
        \ex\label{tzj_moysees}
         Tz’utujil \citep{dayley_1985}\\
        Syllable-final devoicing \\
        \gll\emph{Moysees} \\
        {}[moj̥seːs]  \\
        \glt ‘Moses’ 
    \z
\z
\end{multicols}

A further point of departure from the Kaqchikel data, \citet{wax_cavallaro_2020} also notes word-final devoicing is not restricted to the non-nasal sonorants: na-sals also \emph{partially} devoice in that position (example \ref{tzj_naan}). However, the general coda devoicing rule does not apply to nasals, as medial codas do not (example \ref{tzj_xinwa’i}). 
\begin{multicols}{2}
\ea
    \ea\label{tzj_naan}
    Tz’utujil \citep{dayley_1985}\\
    Nasal devoicing \\
    \gll\emph{naan} \\
    {}[naːn͡n̥] \\
    \glt ‘lady’ 
    \columnbreak
        \ex\label{tzj_xinwa’i}
        Tz’utujil \citep{dayley_1985}\\
        Nasal non-devoicing \\
        \gll\emph{xinwa’i} \\
        {}[ʃinwaʔi] \\
        \glt ‘I ate’ 
    \z
\z
\end{multicols}

\citet{wax_cavallaro_2020} provides a constraint ranking to derive all of these observations within an Optimality Theory framework. I do not discuss that part of her analysis here. More relevant to~-- and in agreement with~-- the current analysis of Kaqchikel is that [sg] is being manipulated by Tz’utujil phonology despite not being a contrastive feature in the language. However, the process of [+sg] insertion in Tz’utujil is less restricted/more general than that of Kaqchikel: [+sg] is inserted at all codas, not just word-final ones, though constraints exist that block its insertion on word-medial nasals.

\subsubsection{Blackfoot}\label{Blackfoot}
In addition to the similar cases of word-final allophony in related Mayan languages, there are cases of final aspiration/devoicing/spirantization across the world, including many other Indigenous languages of North America, including Plains Cree \citep{flynn_hodgson_windsor_2019}, Cherokee (chr) \citep{montgomery-anderson_2008}, Nahuatl (nah) \citep{launey_1992}, and Mistanlta Totonac (tlc) \citep{mackay_1994}.\footnote{See \citet{campbell_kaufman_smith-stark_1986} for discussion of final devoicing as a potential areal feature of Meso-America}

In Blackfoot (bla), an Algonquian language spoken in the northwestern prai-ries of North America, \citet{windsor_2016} analyzes a case of [+sg] insertion affecting phrase-final vowels, rather than word- or syllable-final consonants observed in Kaqchikel and Tz’utujil.\footnote{The fact that [+sg] insertion is not observed to affect Kaqchikel vowels is due to morphological constraints that all lexical words and suffixes end in a consonant.} In Blackfoot, vowels devoice at the ends of orthographic words, which map to phonological words prosodically (example \ref{bla_devoicing}).

\ea\label{bla_devoicing}
    Blackfoot vowel devoicing \citep[64 (3)]{windsor_2016}\\
    \glll  \emph{Ánniksi} \emph{ákaímahkihkinaiksi} \emph{inókiwa} \\
    ann-iksi áka-íímahkihkinaa-iksi ino-okiwa \\
    [ánːiksi̥ ákɛ́ːmaxkiçkinɛksi̥ inókiʍḁ] \\
    \glt ‘those old sheep see us’
\z

Moreover, \citet{windsor_2016} explains, stops in Blackfoot are actually aspirated in the same phrase-final positions, though these instances are less frequent due to the relatively few consonant-final suffixes in the language (example \ref{bla_asp}).

\ea\label{bla_asp}
    Blackfoot stop aspiration \citep[68 (based on spectrogram in 7b)]{windsor_2016}\\
    \glll \emph{Piit!} \\
    pii-t \\
    {}[piːtʰ] \\
    \glt ‘enter!’
\z

In previous work, Windsor found that the average duration of the devoiced period in vowels and aspirated period in stops were roughly equivalent \citep{windsor_cobler_2013}. This leads to the notion that these two effects at the same phrase-final position are in fact part of the same process: one of [+sg] insertion.

\citet{windsor_2016}, in developing this analysis, proposes that [sg] is in fact contrastive in Blackfoot, despite the language’s lack of contrastive aspiration or /h/ (or for that matter, voicing or glottalization indicative of other laryngeal features; cf. \citealt{elfner_2006}). This claim is based on an reinterpretation of sequences previously described as velar fricative /x/ + stop /T/ as pre-aspiration.\footnote{Blackfoot stops contrast three ways for place: labial /p/, coronal /t/, and dorsal /k/ \citep{windsor_2016}.} This allows Windsor’s proposal to satisfy the requirement of the Contrastivist Hypothesis that only contrastive features be active in phonology \citep{hall_2007, Dresher:2009}.\footnote{This step would end up being an unnecessary step in the analysis if the current proposal regarding Kaqchikel and Wax Cavallaro’s (\citeyear{wax_cavallaro_2020}) proposal regarding Tz’utujil are borne out.}

From there, \citet{windsor_2016} proposes that [+sg] is inserted at the right edge of phonological phrases (orthographic words), leading to the observed allophony of voiced vs. devoiced vowels and plain vs. aspirated stops in phrase-final position.

\subsubsection{Generalizing [spread glottis] insertion}
The two additional cases of [+sg] insertion presented in the previous two subsections show that [+sg] insertion is found at the right edge of prosodic domains other than the word-final pattern observed in Kaqchikel. In Tz’utujil, [+sg] is inserted at the edge of every syllable, while in Blackfoot, it is inserted only at the edge of phonological phrases.

\citet{iverson_salmons_2007}, building upon work in Evolutionary Phonology by \citet{Blevins2004, blevins_2006}, show that marking of phrase boundaries with laryngeal cues (e.g. aspiration or glottalization) may spread to smaller domains through a process of generalization. So a process specific to the right (or left) edge of a phonological phrase can be generalized first to the respective edge of words within that phrase, and then to the edge of syllables within the word.

The final proposal of the current study is that there exists a typology of [+sg] insertion processes at play in various languages’ phonologies. Specifically, [+sg] may be inserted at the edge of various prosodic domains (e.g. the phrase in Blackfoot, the word in Kaqchikel, and the syllable in Tz’utujil.) This aspect of the typology comes about due to generalization of [+sg] insertion from the highest level (phrase) down to lower levels within the prosodic hierarchy. Additionally, this typology can be built upon which segments within the phonological inventory of the language are affected by the insertion of [+sg]. In other words, for some languages, only some classes of consonants are affected, while in other languages, vowels may be affected by [+sg] insertion. Again we see that this may develop/evolve over time: only plain stops and non-nasal sonorants are affected in Kaqchikel, while in Tz’utujil, nasal sonorants have begun to show phrase-final allophony as well.

\section{Conclusion}\label{Conclusion}
\largerpage
The pattern of Kaqchikel allophony discussed in this paper, along with a similar pattern in Tz’utujil, show that the languages consistently mark the right edge of prosodic domains with laryngeal cues. These cues may be explained by the insertion of a laryngeal feature [+spread glottis]. This featural insertion has several implications for phonological theory. 

The main implication of this proposal is that [+sg] is inserted despite [sg] not being a contrastive feature in the language. This conflicts with the Contrastivist Hypothesis of \citet{hall_2007} and \citet{Dresher:2009}. Under the current proposal, non-contrastive laryngeal features are available to be manipulated, or in this case inserted, by the phonology, with variable effects depending on the features already present in the segment receiving the feature.

Secondly, following \citet{iverson_salmons_2007}, edge marking is predicted to spread and generalize in a single direction: from higher prosodic domains down to lower prosodic domains. Thus, I expect that historically, Kaqchikel only exhibited [+sg] insertion at the right edge of phrases, similar to the pattern seen in Blackfoot. This has since spread so that the insertion now occurs at the end of every word as well. In the future, this may generalize further to occur in every syllable coda, as it does in Tz’utujil.

Finally, this study surveyed three different patterns of [+sg] insertion to show that patterns can differ in various ways, including which prosodic domain the featural insertion applies to and which sounds are affected by the insertion. Future work on this topic will explore other cases of [spread glottis] or even [constricted glottis] insertion in order to explore and expand the breadth of the typology of laryngeal feature insertion. 

\section*{Abbreviations}
\begin{tabularx}{\textwidth}{@{}lQ@{}}
ALMG       & Academia de Lenguas Maya de Guatemala\\
{[cg]}     & {[constricted glottis]}\\
{[sg]}     & {[spread glottis]}\\
SDA        & Successive Division Algorithm\\
\emph{SPE} & \emph{The Sound Pattern of English} \citep{ChomskyHalle:1968}\\
UNESCO     & UN Educational, Scientific and Cultural Organization\\
\end{tabularx}

\section*{Acknowledgements}
\subsection*{Personal acknowledgements}
I offer special thanks to my Kaqchikel friends and teachers, Ixchel, Kawoq, Aq’a-b’al, B’alam, Yab’un, and an anonymous contributor, among many others, whose productions I could analyze. This study is impossible without their devotion to their language and culture. I am also grateful to the attendees of various conferences and workshops where I’ve presented parts of this work. I specifically thank the attendees of the 2020 Meeting of the Society for the Study of the Indigenous Languages of the Americas for their thoughts on and references to similar patterns observed in other languages. I thank the attendees of the 2021 Epenthesis and Beyond workshop at Stony Brook University for discussion on possible causes for the observed allophony. Additionally, I thank the linguists at the University of Calgary: the faculty of the Linguistics Division, the graduate students of \emph{A Higher Clause}, and the undergraduate students of \emph{Verbatim} for listening to me rant about these patterns for nearly half a decade. Finally, I especially thank my supervisor, Darin Flynn, for his time, energy, and thoughts that helped me develop this analysis and this paper.

\subsection*{Territorial acknowledgements}
I prepared this paper from the area where the Bow and Elbow rivers meet in Southern Alberta, a place known in Blackfoot as Mohkinstsis, a name now used to refer to the city of Calgary. This land lies within the territory of Treaty 7, which was signed by this land’s stewards, representatives of the Blackfoot Confederacy, the Tsuut’ina First Nation, and the Stoney Nakoda. Additionally, this land is home to speakers of Cree and Ktunaxa, as well as members of the Métis Nation of Alberta.

I am a newcomer to this land, coming from Bulbancha, the land of many tongues at the mouth of the Mississippi, the Great River. I strive to learn more about the land, its peoples, their languages, and their history every day. 

This research also took place in Iximulew (Guatemala) the traditional territory of the Kaqchikel, between Lake Atitlán, Armita (Guatemala City) and Junajpu’ (Volcán de Agua). I am thankful to have been welcomed into those lands to learn and to share within the ways of knowing and living of the Kaqchikel Maya people. The data analyzed come from my fieldwork that took place during my stay there with them.


\printbibliography[heading=subbibliography,notkeyword=this]

\end{document}
