\documentclass[output=paper,colorlinks,citecolor=brown]{langscibook}
\ChapterDOI{10.5281/zenodo.14264536}

\author{Edward J. Rubin\affiliation{University of Utah} and Aaron Kaplan\affiliation{University of Utah}}
\title{Segmental and prosodic influences on Bolognese epenthesis} 
\abstract{Bolognese, the Gallo-Italic grammar of Bologna, eliminates illicit coda clusters via epenthesis.  This process is noteworthy for two reasons.  First, as in other closely related Romance varieties such as Donceto \citep{cardinalettirepetti:clitics}, Bolognese prosodic structure and phonotactic patterns determine whether an epenthetic vowel appears: certain clusters are permitted within a PWd but trigger epenthesis when they straddle a PWd boundary, and sonorant-final coda clusters are always subject to epenthesis.  Second, Bolognese displays two epenthetic vowels.  [u] appears before [v, m], while [e] appears elsewhere.  In closely related grammars like Donceto, only one epenthetic vowel ([\textschwa]) appears.  We build on \citeauthor{cardinalettirepetti:clitics}'s (\citeyear{cardinalettirepetti:clitics}) analysis of coda clusters in Donceto to account for the Bolognese facts.}

\IfFileExists{../localcommands.tex}{
   \addbibresource{../localbibliography.bib}
   % add all extra packages you need to load to this file

\usepackage{tabularx,multicol}
\usepackage{url}
\urlstyle{same}

\usepackage{listings}
\lstset{basicstyle=\ttfamily,tabsize=2,breaklines=true}

\usepackage{langsci-basic}
\usepackage{langsci-optional}
\usepackage{langsci-lgr}
\usepackage{langsci-osl}
% \usepackage{./langsci/styles/langsci-lgr}
% \usepackage{./langsci/styles/langsci-osl}
% \usepackage{langsci-gb4e}

\usepackage{tikz}
\usetikzlibrary{patterns,calc}
\pgfdeclarepatternformonly{south east lines}{\pgfqpoint{-0pt}{-0pt}}{\pgfqpoint{3pt}{3pt}}{\pgfqpoint{3pt}{3pt}}{
    \pgfsetlinewidth{0.6pt}
    \pgfpathmoveto{\pgfqpoint{0pt}{3pt}}
    \pgfpathlineto{\pgfqpoint{3pt}{0pt}}
    \pgfpathmoveto{\pgfqpoint{.2pt}{-.2pt}}
    \pgfpathlineto{\pgfqpoint{-.2pt}{.2pt}}
    \pgfpathmoveto{\pgfqpoint{3.2pt}{2.8pt}}
    \pgfpathlineto{\pgfqpoint{2.8pt}{3.2pt}}
    \pgfusepath{stroke}}
    
\usepackage{stmaryrd}
\usepackage{wasysym}
\usepackage{multirow}
\usepackage{caption}
\usepackage{subcaption}
\usepackage{mathrsfs}
\usepackage{qtree}

\usepackage{linguex}


   %pminos do not split footnotes
% \interfootnotelinepenalty=10000 %Footnote in Laporte chapters has to be split SN


%\DeclareIndexNameFormat{default}{%
%\nameparts{#1}%
%\usebibmacro{index:name}%
%{\index[names]}%
%{\namepartfamily}%
%{\namepartgiveni}%
% {}% L1
% {}% L2
%{\namepartprefix}% generates spurious space L3
%{\namepartsuffix}% generates spurious space L4
%}

%  {\DeclareIndexNameFormat{default}{%
%     \usebibmacro{index:name}{\index[names]}{#1}{#3}{#5}{#7}}}

%\DeclareIndexNameFormat{default}{%
%  \usebibmacro{index:name}{\sindex[nom]}{#1}{#3}{#5}{#7}}

%\DeclareIndexNameFormat{default}{%
%  \usebibmacro{index:name}{\sindex[person]}{#1}{#3}{#5}{#7}}
%\DeclareIndexNameFormat{default}{%
%\nameparts{#1} \usebibmacro{index:name}{\sindex[person]]}{\namepartfamily}{‌​\namepartgiven}{\nam‌​epartprefix}{\namepa‌​rtsuffix}}

%\newcommand{\smiley}{:)}

%\renewbibmacro*{index:name}[5]{%
%\usebibmacro{index:entry}{#1}%
%{\iffieldundef{usera}{}{\thefield{usera}\actualoperator}\mkbibindexname{#2}{#3}{#4}{#5}}}

% \newcommand{\noop}[1]{}

%remove for final
%\overfullrule=1mm

\newcommand{\tobi}[2]}}
\renewcommand{\S}[1]{\tobi{#1}{\textsc{*}}}

% this volume references
% puts: [this volume]
% already defined: \citetv
%\newcommand{\citepv}[1]{(\citeauthor{#1} \citeyear*{#1} [this volume])}
\newcommand{\citealtv}[1]{\citeauthor{#1} \citeyear*{#1} [this volume]}

%parentheses around example number
\newcommand{\pref}[1]{(\ref{#1})}

% in-text examples

\newcommand{\lnex}[1]{\textit{#1}} %target lang word
\newcommand{\lnlit}[1]{(lit.: `#1')} %literal reading
\newcommand{\lnlat}[1]{(#1)} % latinization
\newcommand{\lntrans}[1]{`#1'} %translation
\newcommand{\lnexl}[2]%
{\lnex{#1}{} \lnlat{#2}} % ex with latinization
\newcommand{\lnexlat}[3]{\lnex{#1}{} \lnlat{#2}{} \lntrans{#3}} % ex with latinization and tranl.

%ch01
\newcommand{\co}[1]{\mbox{\textbf{#1}}}

%ch09

\newcommand{\cyrbulg}[1]{\begin{otherlanguage*}{bulgarian}#1\end{otherlanguage*}}


%ch10
\newcommand{\nlp}{{\small NLP}}
\newcommand{\mwe}{{\small MWE}}
\newcommand{\rae}{{\small RAE}}
\newcommand{\lvc}{{\small LVC}}
\newcommand{\pos}{{\small P}o{\small S}}
%\newcommand{\todo}[1]{ \textcolor{red}{#1} }

%\renewcommand{\labelenumi}{\theenumi}
%\ainamefmt{{vv}{ll}{, ff}{, jj}} % fullname

\newcommand{\biberror}[1]{{\color{red}#1}}

\newcommand{\osenovaitem}{--~}
   %% hyphenation points for line breaks
%% Normally, automatic hyphenation in LaTeX is very good
%% If a word is mis-hyphenated, add it to this file
%%
%% add information to TeX file before \begin{document} with:
%% %% hyphenation points for line breaks
%% Normally, automatic hyphenation in LaTeX is very good
%% If a word is mis-hyphenated, add it to this file
%%
%% add information to TeX file before \begin{document} with:
%% %% hyphenation points for line breaks
%% Normally, automatic hyphenation in LaTeX is very good
%% If a word is mis-hyphenated, add it to this file
%%
%% add information to TeX file before \begin{document} with:
%% \include{localhyphenation}
\hyphenation{
    Beck-man
    Ngu-yen
    back-chan-nel
    back-chan-nels
    mo-not-o-nous
    ste-reo-typ-i-cal
}

\hyphenation{
    Beck-man
    Ngu-yen
    back-chan-nel
    back-chan-nels
    mo-not-o-nous
    ste-reo-typ-i-cal
}

\hyphenation{
    Beck-man
    Ngu-yen
    back-chan-nel
    back-chan-nels
    mo-not-o-nous
    ste-reo-typ-i-cal
}

   \boolfalse{bookcompile}
   
   \togglepaper[5]%%chapternumber
}{}

\begin{document}
\maketitle \label{ch5}

\section{Two contexts for epenthesis}
In many ways, pronominal clitics in Bolognese (Romance; Italy) are typical of Romance languages, making the usual morphosyntactic distinctions and mostly exhibiting a typical set of consonants, as highlighted in Table \ref{tab:1:CLs}. These clitics vary phonologically according to (i) whether they appear as proclitics or enclitics (though the 2\textsc{p} subject clitic (\textsc{scl}) is only enclitic, as in related grammars) and (ii) whether they are adjacent to vowels or consonants. On the other hand, the complexity Bolognese permits in onsets and codas is unusual among Romance languages. Our focus in this paper is these clitics, all of which (in their enclitic form) display a V $\sim\emptyset$ alternation.\footnote{The gaps in Table \ref{tab:1:CLs} represent non-existent clitics (e.g.\ reflexive \textsc{scl}s) or would contain clitics that are or include other vowels and therefore do not participate in this alternation (e.g.\ the third person dative clitic (\textsc{dcl}) and other third person plural clitics [i], and the third person feminine \textsc{scl} and all non-reflexive third person accusative clitics (\textsc{acl}s) which contain [a]). First and second person \textsc{dcl}s and \textsc{acl}s are both reflexive and non-reflexive. \textsc{prt} is unspecified for gender and number. All Bolognese data in this paper are drawn from \citet{caneparivitali:bolognese}, \citet{bologneseprimertwo}, and \citet{leprivitali:dict}, or from extensive consultation with native speakers.}  We argue that this vowel is epenthetic, breaking some clusters up that are permissible in other contexts. In support of this claim we examine other contexts for epenthesis in the language, presenting a unified account for epenthesis in postverbal clitics and these other contexts.

%In many ways, pronominal clitics in Bolognese (Romance; Italy) are typical of Romance languages, making the usual morphosyntactic distinctions and mostly exhibiting a typical set of consonants, as highlighted in Table \ref{tab:1:CLs}. These clitics vary phonologically according to (i) whether they appear as proclitics or enclitics (though the 2\textsc{p} subject clitic (\textsc{scl}) is only enclitic, as in related grammars) and (ii) whether they are adjacent to vowels or consonants. Our focus in this paper is these highlighted clitics, all of which (in their enclitic form) display a V $\sim\emptyset$ alternation.\footnote{The gaps in Table \ref{tab:1:CLs} represent non-existent clitics (e.g.\ reflexive \textsc{scl}s) or would contain clitics that are or include other vowels and therefore do not participate in this alternation (e.g.\ the third person dative clitic (\textsc{dcl}) and other third person plural clitics [i], and the third person feminine \textsc{scl} and all non-reflexive third person accusative clitics (\textsc{acl}s) which contain [a]). First and second person \textsc{dcl}s and \textsc{acl}s are both reflexive and non-reflexive. \textsc{prt} is unspecified for gender and number. All Bolognese data in this paper are drawn from \citet{caneparivitali:bolognese}, \citet{bologneseprimertwo}, and \citet{leprivitali:dict}, or from extensive consultation with native speakers.} We argue that this vowel is epenthetic.  In support of this claim we examine other contexts for epenthesis in the language, presenting a unified account for epenthesis in postverbal clitics and these other contexts.

\begin{table}
\caption{Clitic pronouns in Bolognese}
\label{tab:1:CLs}
 \begin{tabular}{l  cc cc cc c}
  \lsptoprule
  		& \multicolumn{2}{c}{\textsc{scl}} & \multicolumn{2}{c}{\textsc{dcl}} & \multicolumn{2}{c}{\textsc{acl}} & \textsc{prt}\\
		& \textsc{Sing} & \textsc{Plur} & \textsc{Sing} & \textsc{Plur} & \textsc{Sing} & \textsc{Plur} & \\
  \midrule
  1  &    &   & \textbf{m} & \textbf{s} & \textbf{m} & \textbf{s} & \\% {\color{gray}a / ja}
  2  &   \textbf{t} & ꞊\textbf{v} &  \textbf{t} & \textbf{v} &  \textbf{t} & \textbf{v} & \\
  3\textsc{msg} & {\color{gray}(a)}\textbf{l} &  & & & {\color{gray}(a)}\textbf{l} &  & \textbf{n}\\
%  3f & {\color{gray}l(a)} & {\color{gray}(\ae l - \ae\textturny)} & {\color{gray}i} & {\color{gray}i} & {\color{gray}l(a)} & {\color{gray}(\ae l - \ae\textturny)}&\\
  3\textsc{rflx} &  &   & \textbf{s} & \textbf{s} & \textbf{s} & \textbf{s} &\\
  \lspbottomrule
 \end{tabular}
\end{table}




In (\ref{subjectpostverbal}--\ref{objectpostverbal}) we provide data showing the enclitics that participate in the noted alternation. Like in other Gallo-Italic varieties, subject clitics in Bolognese appear only with tensed verbs and are postverbal in interrogatives \REF{subjectpostverbal}. The object clitics (dative/indirect (\textsc{dcl}), accusative/direct (\textsc{acl}), and partitive (\textsc{prt})) appear postverbally with tenseless verbs (infinitives, imperatives, gerunds) \REF{objectpostverbal}. In both sets of data, we observe the mentioned V $\sim\emptyset$ alternation, and the vowels that appear before the relevant consonants, [e] and [u], shall be a main focus of our attention below.

\NumTabs{3} %divides text into 3 tab spaces
\begin{exe}
		\ex \label{subjectpostverbal}
			\begin{xlist}
				\ex 
                    \begin{xlisti}
                       \ex ˈdoːrm꞊\textbf{e}t\tab `Are you sleeping?'\tab 2\textsc{sg}
                       \ex durmiˈrɛː꞊t\tab `Will you sleep?'\tab
                    \end{xlisti}

                \ex  \begin{xlisti}
                       \ex ˈdoːrm꞊\textbf{e}l\tab `Is he sleeping?'\tab 3\textsc{sg}
                       \ex durˈme꞊l\tab `Did he sleep?'
                    \end{xlisti}

                \ex \begin{xlisti}
                    \ex \_ \footnote{There is a predictable gap here: The Vs in question occur only after consonants, and all 2\textsc{pl} tensed verb-forms in Bolognese are V-final. See also footnote \ref{historicalfootnote}.}\tab \tab 2\textsc{pl}
                    \ex durˈmi꞊v\tab `Are you.\textsc{pl} sleeping?'
                \end{xlisti}
			\end{xlist}
\end{exe}


\NumTabs{3} %divides text into 3 tab spaces
\begin{exe}
		\ex\label{objectpostverbal}
			\begin{xlist}
				\ex 
                        \begin{xlisti}
                           \ex ˈdɛr꞊\textbf{u}m\tab `to give me.\textsc{dcl}'  \tab  \hspace{5mm} 1\textsc{sg}
                           \ex arspuŋˈdiː꞊m\tab `Answer.\textsc{pl} me.\textsc{dcl}!'
                           \ex truˈvand꞊\textbf{u}m iŋ ˈka\tab `finding me.\textsc{acl} at home'
                           \ex ɡwaˈrdɛː꞊m\tab `Watch.\textsc{pl} me.\textsc{acl}!'
                        \end{xlisti}

                    \ex  \begin{xlisti}
                           \ex diˈɡaŋd꞊\textbf{e}t\tab  `saying to you.\textsc{dcl}'\tab \hspace{5mm} 2\textsc{sg}
                           \ex ˈda꞊t\tab `Give.\textsc{sg} yourself.\textsc{dcl} ... !'
                           \ex kaˈtɛr꞊\textbf{e}t\tab `to find/visit you.\textsc{acl}'
                           \ex ˈftes꞊\textbf{e}t\tab `Dress.\textsc{sg} yourself.\textsc{acl}!'
                        \end{xlisti}

                    \ex \begin{xlisti}
                        \ex \emph{Non-reflexive \textsc{dcl} [i] never alternates this way}\tab  \hspace{5mm} 3\textsc{sg}
                        \ex purˈtɛːr꞊\textbf{e}l\tab `to carry it/him.\textsc{acl}'
                        \ex studˈjɛ꞊l\tab `Study.\textsc{pl} it/him.\textsc{acl}!'
                        \ex \textsc{refl}:ˈdɛr꞊\textbf{e}s\tab `to give to oneself/himself/herself/ \\themselves.\textsc{dcl}'
                        \ex \textsc{refl}:ˈftaŋd꞊\textbf{e}s\tab `dressing oneself/himself/herself/ \\ themselves.\textsc{acl}'
                    \end{xlisti}
                    
                    \ex \begin{xlisti}
                        \ex ˈdɛr꞊\textbf{e}s\tab `to give us.\textsc{dcl}'\tab \hspace{5mm} 1\textsc{pl}
                        \ex arspuŋˈdiːs\tab `Answer.\textsc{pl} us.\textsc{dcl}!'
                        \ex kaˈtɛr꞊\textbf{e}s\tab `to find/visit us.\textsc{acl}'
                        \ex aspˈtɛː꞊s\tab `Wait.\textsc{pl} for us.\textsc{acl}!'
                    \end{xlisti}
                    
                    \ex \begin{xlisti}
                        \ex arspuŋˈdaŋd꞊\textbf{u}v\tab `responding to you.\textsc{pl}.\textsc{dcl}'\tab \hspace{5mm} 2\textsc{pl}
                        \ex ˈdɛː꞊v\tab `Give.\textsc{pl} yourselves.\textsc{dcl} ... !'
                        \ex θarˈkɛːr꞊\textbf{u}v\tab `to look-for you.\textsc{pl}.\textsc{acl}'
                        \ex liˈvɛː꞊v\tab `Get up!/lift yourselves.\textsc{acl}!'
                    \end{xlisti}
                    \ex \begin{xlisti}
                        \ex ˈfɛːr꞊\textbf{e}ŋ ˈduː\tab `to make two of them'\tab \hspace{5mm} \textsc{prt}
                        \ex ʦkuˈræŋn꞊\textbf{e}ŋ ˈdɑp\tab `Let's talk about it later!'
                        \ex ˈdɛː꞊ŋ ˈduː a ˈðvaŋ\tab `Give.\textsc{pl} two to John!'
                    \end{xlisti}
                    
			\end{xlist}
\end{exe}


As is apparent, these clitics have the shape [C] following a vowel-final stem, but the shape [eC] following a consonant-final stem (except that we find [um] and [uv], not *[em] and *[ev], for 1\textsc{sg} and 2\textsc{pl} object clitics, respectively; we examine this in Section \ref{epentheticu}).


Though we will largely ignore preverbal clitics, \REF{preverbal} shows that the [C] form of most of these clitics also appears preverbally (\textsc{acl}s are shown, and relevant \textsc{dcl}s are identical). In addition, like in many Romance varieties, the preconsonantal \textsc{acl}.3\textsc{msg} [al] and \textsc{acl}.3\textsc{fsg} [la] clitics distinguish gender \REF{aclgender}, but this distinction is leveled before a vowel \REF{aclgenderleveled}. The corresponding \textsc{scl}.3\textsc{sg}s [al] / [la] behave identically ([al/la=ˈvad] `he/she sees', [l꞊e] `he/she is'), though the \textsc{scl}.3\textsc{msg} has additional allomorphs in preverbal clitic clusters (see \cite{rubinkaplanallomorphywithls} for an analysis of preverbal clitic allomorphy). The vowel in both of these is distinct from the epenthetic vowels that we discuss below. Note that Bolognese differs importantly from Donceto (another Romance variety spoken near Bolognese in Italy), where the \textsc{scl}.3\textsc{msg} includes the epenthetic vowel of that language ([\textschwa]) according to \citet{cardinalettirepetti:clitics}, but the \textsc{scl}.3\textsc{fsg} includes [a], and undergoes the same pre-V / pre-C allomorphic variation as the two \textsc{scl}.3\textsc{sg}s in Bolognese. We conclude that Bolognese preverbal \textsc{scl}.3\textsc{msg} [al] is due to allomorphy, not epenthesis.


\NumTabs{3} %divides text into 3 tab spaces
\begin{exe}
		\ex\label{preverbal}
			\begin{xlist}
				\ex 
                        \begin{xlisti}
                           \ex i꞊\textbf{m}=ˈʦaːmeŋ\tab `they call me'\tab 1\textsc{sg}
                           \ex i꞊\textbf{m}꞊abˈraːθeŋ\tab `they hug me'
                        \end{xlisti}
                        
                    \ex
                        \begin{xlisti}
                           \ex  i꞊\textbf{t}=ˈʦaːmeŋ\tab `they call you.\textsc{sg}'\tab  2\textsc{sg}
                           \ex i꞊\textbf{t}꞊abˈraːθeŋ\tab `they hug you.\textsc{sg}'
                        \end{xlisti}
                        
                    \ex 
                        \begin{xlisti}
                            \ex \label{aclgender} i꞊\textbf{al}=ˈʦaːmeŋ\tab `they call him' \tab  3\textsc{sg}\\
                            i꞊\textbf{la}=ˈʦaːmeŋ\tab `they call her'\\
                            \ex \label{aclgenderleveled}  i꞊\textbf{l}꞊abˈraːθeŋ\tab `they hug her/him'
                        \end{xlisti}
                        
                    \ex 
                        \begin{xlisti}
                           \ex i꞊\textbf{s}=ˈʦaːmeŋ\tab `they call us'\tab  1\textsc{pl}
                           \ex i꞊\textbf{s}꞊abˈraːθeŋ\tab `they hug us'
                        \end{xlisti}

                    \ex 
                        \begin{xlisti}
                            \ex i꞊\textbf{v}=ˈʦaːmeŋ\tab `They call you.\textsc{pl}'\tab  2\textsc{pl}
                            \ex i꞊\textbf{v}꞊abˈraːθeŋ\tab `they hug you.\textsc{pl}'
                        \end{xlisti}

                    \ex
                        \begin{xlisti}
                            \ex i꞊\textbf{i}=ˈʦaːmeŋ\tab `they call them'\tab  3\textsc{pl}
                            \ex i꞊\textbf{i}꞊abˈraːθeŋ\tab `they hug them'
                            \ex i꞊\textbf{s}=ˈʦaːmeŋ\tab `they call each other'
                            \ex i꞊\textbf{s}꞊abˈraːθeŋ\tab `they hug each other'
                        \end{xlisti}

                    \ex 
                        \begin{xlisti}
                            \ex i꞊\textbf{ŋ꞊}ˈʦaːmeŋ ˈduː\tab `they call two of them'\tab \textsc{prt}
                            \ex i꞊\textbf{n}꞊abˈraːθeŋ ˈduː\tab `they hug two of them'
                        \end{xlisti}
			\end{xlist}
\end{exe}


The choice between the [C] and [VC] forms of the enclitics in (\ref{subjectpostverbal}--\ref{objectpostverbal}) is dictated by Bolognese's coda cluster phonotactics.  Two phonotactic requirements are relevant: a prohibition on coda clusters in certain prosodic domains and a prohibition on sonorant-final coda clusters.  We begin with the former.

A variety of coda clusters is attested in the language; this includes clusters ending with [s] or [t] as we see in \REF{wordInternalstclusters}.  Interestingly, though, these clusters are not permitted when the final [s] or [t] is a clitic, as \REF{epenthesiswithclitics} shows.  The bolded epenthetic vowels in \REF{epenthesiswithclitics} break up the clusters in these examples, and in some examples it is the minimal difference with a correspondent in \REF{wordInternalstclusters}.

\NumTabs{3}
\begin{exe}
    \ex \label{wordInternalstclusters} [Cs] and [Ct] can occur word-finally...
        \begin{xlist}
            \ex skɛːrs\tab `scarce'
            \ex sɛːlt\tab `(a) jump'
            \ex ˈt꞊sɛːlt\tab `you jump'
            \ex a=ˈpæŋs\tab `I think'
            \ex a꞊trˈavers\tab `I cross'
            \ex peːrs\tab `lost'
            \ex t꞊iŋˈvæŋt\tab `you invent'
            \ex a꞊ɡˈwaːst\tab `I spoil'
        \end{xlist}
\end{exe}

\begin{exe}
    \ex \label{epenthesiswithclitics} ...but epenthesis occurs when the [s] or [t] is a clitic
        \begin{xlist}
            \ex ˈskɛːr꞊\textbf{e}s\tab `to dry us/ourselves'
            \ex ˈsɛːl꞊\textbf{e}t\tab `do you salt (something)?'
            \ex amiˈrɛːr꞊\textbf{e}s\tab `to admire us/ourselves'
            \ex liˈvɛːr꞊\textbf{e}s\tab `to get us/ourselves up'
            \ex liˈvɛːr꞊\textbf{e}t\tab `to get you/yourself up'
            \ex truˈvɛːr꞊\textbf{e}s\tab `to find us/ourselves'
            \ex truˈvɛːr꞊\textbf{e}t\tab `to find you/yourself'
            \ex ˈrɑŋf꞊\textbf{e}t\tab `do you snore?'
        \end{xlist}
\end{exe}


The contrast between \REF{wordInternalstclusters} and \REF{epenthesiswithclitics} indicates that enclitics and the verb are (immediate) constituents of distinct prosodic units that differ in whether they permit epenthesis.  \citet{cardinalettirepetti:clitics} document similar facts for subject clitics in Donceto, but they do not provide an explicit analysis.  They argue that clitics are outside the prosodic word (PWd) but within the phonological phrase (PP), and we adopt that position here.\footnote{For present purposes, the identities of the relevant prosodic categories are unimportant.  They might be PWd and PP, or perhaps recursive PWds \citep{itomestercompounds}.  We adopt the former to follow \citeauthor{cardinalettirepetti:clitics}'s precedent.}  To illustrate, the structure of [ˈsɛːl꞊\textbf{e}t] `do you salt (something)?' is given in \figref{doyousaltstructure}, setting aside the epenthetic vowel.

\begin{figure}
\caption{Prosodic structure of [ˈsɛːl=et] ‘do you salt (something)?’}
\label{doyousaltstructure}
\begin{tikzpicture}[auto]
  \node (PP)    at (0,0)    {PP};
  \node (PWd) at (-1,-1)  {PWd};
  \node (clitic)  at (1,-1)   {t};
  \node (root)  at (-1,-2)   {sɛːl};
  \draw (PP.south) to node {} (PWd.north);
  \draw (PP.south) to node {} (clitic.north);
  \draw (PWd.south) to node {} (root.north);
\end{tikzpicture}
\end{figure}


Such an analysis leads to the following generalization: a C + [s]/[t] coda is permitted PWd-internally, but when it straddles a PWd boundary it is banned.  To account for this, we posit that \textsc{*Complex} outranks \textsc{Contiguity}(PP) but not \textsc{Contiguity}(PWd).\footnote{Onset clusters behave somewhat differently, suggesting a distinction between \textsc{*ComplexOnset} and \textsc{*ComplexCoda}.  Because we will not analyze onset clusters here, we will simply use \textsc{*Complex}.}  In \tabref{youjump}, the /lt/ cluster is contained within the root and is thus PWd-internal; \textsc{Contiguity}(PWd) blocks epenthesis  (because the PWd is a constituent of the PP, the cluster is also PP-internal, hence candidate (b)'s \textsc{Contiguity}(PP) violation).  But in \tabref{doyousalt}, the cluster is not wholly within the PWd (because the /t/ is a clitic) and is therefore subject only to the low-ranking \textsc{Contiguity}(PP); this time, \textsc{*Complex} compels epenthesis.\footnote{\textsc{Contiguity}(PP) and \textsc{Contiguity}(PWd) are in a stringent relationship \citep{delacy:conflation}: assuming PWds are always, or at least usually, contained within PPs (whether one adopts the strict layer hypothesis (e.g.\ \citealt{selkirk:1984}) or something else), any configuration subject to \textsc{Contiguity}(PWd) is also subject to \textsc{Contiguity}(PP).  A prediction of this analysis is therefore that whatever the ranking between these two constraints, if epenthesis or any other \textsc{Contiguity}-violating process is blocked in elements outside a PWd but within a PP, it will also be blocked inside a PWd.  But the opposite does not hold: as in Bolognese, epenthesis may occur within a PP even if it is blocked within a PWd.}

\begin{table}
\caption{/ˈsɛːlt/ `(a) jump', from \REF{wordInternalstclusters}}
\label{youjump}
\begin{center}
\ShadingOn
\begin{tableau}{c|s|s} 
\inp{/ˈsɛːlt/}      \const*{\textsc{Contig}(PWd)}  \const{\textsc{*Complex}} \const{\textsc{Contig}(PP)}
\cand[\Optimal]{ˈsɛːlt}   \vio{}     \vio{*}  \vio{}
\cand{ˈsɛːlet}             \vio{*!}   \vio{}  \vio{*}
\end{tableau}
\end{center}
\end{table}

\begin{table}
\caption{/ˈsɛːlet/ ‘Do you salt (something)?’, from \REF{epenthesiswithclitics}, \figref{doyousaltstructure}}
\label{doyousalt}
\begin{center}
\ShadingOn
\begin{tableau}{c|c|s} 
\inp{/ˈsɛːl=t/}      \const*{\textsc{Contig}(PWd)}  \const{\textsc{*Complex}} \const{\textsc{Contig}(PP)}
\cand{ˈsɛːlt}                       \vio{}     \vio{*!}  \vio{}
\cand[\Optimal]{ˈsɛːlet}             \vio{}   \vio{}  \vio{*}
\end{tableau}
\end{center}
\end{table}

\hspace*{-2.5pt}Evidence that the epenthetic vowels in these forms are indeed epenthetic comes from two sources.  First, as we have seen, these clitics do not always surface with [e] \REF{cliticsnoepenthesis}, appearing as just [s] or [t] when doing so does not violate \textsc{*Complex}.  Furthermore, some object clitics, including the ones at issue here, have a [VC] allomorph that appears, for example, after the second singular subject clitic, but the vowel that appears in this allomorph is [a], not [e] \REF{VCallomorph}.


\NumTabs{3}
\begin{exe}
    \ex \label{cliticsnoepenthesis} 
        \begin{xlist}
            \ex i꞊\textbf{s}꞊ˈsakeŋ\tab `they dry us'
            \ex al꞊\textbf{s}꞊aˈmiːra\tab `he admires us'
            \ex a꞊\textbf{s}꞊iŋdurmiŋˈtæŋ\tab `we fall asleep'
            \ex a꞊\textbf{t}꞊ˈtroːv\tab `I find you'
            \ex ˈ\textbf{t}꞊sɛːlt\tab `you jump'
            \ex ˈ\textbf{t}꞊rɑŋf\tab `you snore'
        \end{xlist}
\end{exe}

\begin{exe}
    \ex \label{VCallomorph}
        \begin{xlist}
            \ex t꞊\textbf{at}꞊iŋdurˈmæŋt\tab `you fall asleep'
            \ex t꞊\textbf{as}꞊ˈtroːv\tab `you find us'
        \end{xlist}
\end{exe}



Second, epenthesis in contexts not involving clitics uses the same vowel that we see in \REF{epenthesiswithclitics}.  For example, despite the ranking \textsc{Contiguity}(PWd) \rank\ \textsc{*Complex}, PWd-internal epenthesis to break up coda clusters is attested; some examples are given in \REF{sonorityepenthesis} (we address the cause of this epenthesis below).  Each of these roots contains a root-final cluster.  In the first form on a line, a suffix allows the second of those consonants to surface as an onset, avoiding a coda cluster.  But in the second form on a line, in the absence of suffixes, [e] is epenthesized between the consonants.  Aside from regular exceptions to be discussed in Section \ref{epentheticu}, the vowel that appears in these contexts is always [e].


\begin{exe}
    \ex \label{sonorityepenthesis}
        \begin{xlist}
            \ex \textsc{fsg} [-a] / \textsc{fpl} [-$\emptyset$]
                \begin{xlisti}
                    \ex ˈtɛːvla / ˈtɛːv\textbf{e}l\tab `table' / `tables'
                    \ex laŋˈteːrna / laŋˈteːr\textbf{e}ŋ\tab `lantern' / `lanterns'
                    \ex ˈliːvra /ˈliːv\textbf{e}r\tab `hare' / `hares'
                \end{xlisti}
            \ex \textsc{infinitive} [-ˈɛr] / \textsc{pres.1sg} [-$\emptyset$]
                \begin{xlisti}
                    \ex sfitˈlɛːr / a=ˈsfat\textbf{e}l\tab `to slice' / `I slice'
                    \ex urdˈnɛːr / a=ˈɑʊrd\textbf{e}ŋ\tab `to order' / `I order'
                    \ex lusˈtrɛːr / a=ˈlost\textbf{e}r\tab `to polish' / `I polish'
                \end{xlisti}
\newpage
            \ex \textsc{adj\textsubscript{fsg}} [-a] / \textsc{adj\textsubscript{msg}} [-$\emptyset$]
                \begin{xlisti}
                    \ex ˈdabla / ˈdab\textbf{e}l\tab `weak.\textsc{fs}' / `weak.\textsc{ms}'
                    \ex ˈðɑʊvna / ˈðɑʊv\textbf{e}ŋ\tab `young.\textsc{fs}' / `young.\textsc{ms}'
                    \ex ˈvɔːstra / ˈvɔːst\textbf{e}r\tab `your.\textsc{fs}' / `your.\textsc{ms}'
                \end{xlisti}
        \end{xlist}
\end{exe}



The evidence therefore suggests that the [e] seen in \REF{epenthesiswithclitics} is epenthetic.  This conclusion ties the appearance of this vowel to other patterns of epenthesis in Bolognese, and it is simpler than an alternative that posits two [VC] allomorphs for these clitics, one with [e] that appears only word-finally and one with [a] that appears elsewhere.

More must be said about \REF{sonorityepenthesis}.  We attribute the epenthesis illustrated there to phonotactic requirements.  It is tempting to view that epenthesis as a manifestation of sonority sequencing principles (see \cite{selkirk:majorclassfeatures} and \cite{sonoritycycle} for overviews) that prohibit rising-sonority coda clusters (e.g.\ *[a=ˈsfatl], *[a꞊ɑʊrdŋ]) or clusters that do not have an adequate fall in sonority (*[laŋˈteːrŋ], plausibly).  But it is actually unclear to what extent Bolognese obeys sonority sequencing constraints.  A representative sample of the language's coda clusters is given in \tabref{obstruentclusters}; see also \REF{wordInternalstclusters}.  Most clusters conform to sonority sequencing expectations, but not all do (e.g.\ [rbz], [dɡ]); onset clusters are even more dramatic in their disregard for sonority sequencing ([zbdɛl] `hospital', [ˈftleŋna] `slice', [ˈʦkɲɔser] `to disavow', [vdand] `seeing').  One clear generalization, though, is that sonorant-final coda clusters are unattested, and we therefore adopt a constraint against such clusters, *C[+\textsc{son}]]$_\sigma$, and this constraint drives epenthesis in \REF{sonorityepenthesis}.

\begin{table}
\caption{Licit obstruent coda clusters}
\label{obstruentclusters} 
\begin{minipage}{.5\textwidth}
\begin{tabular}{lll}
\lsptoprule
rbz & fo\textbf{rbz} & `scissors' \\
rb & tɑʊ\textbf{rb} & `cloudy' \\
rp & auˈzuː\textbf{rp} & `I usurp' \\
rd & sɑʊ\textbf{rd} & `deaf' \\
rdɡ & poː\textbf{rdɡ} & `portico' \\
rt & pɛː\textbf{rt} & `part' \\
rʦ & kwɛː\textbf{rʦ} & `lid' \\
rθ & pɔː\textbf{rθ} & `pig'\\
\lspbottomrule
\end{tabular}
\end{minipage}%
\begin{minipage}{.5\textwidth}
\begin{tabular}[t]{lll}
\lsptoprule
rð & zɡɛː\textbf{rð} & `wool comb' \\
dɡ & ˈapɛː\textbf{dɡ} & `I walk' \\
mb & strapˈjɑ\textbf{mb} & `overhang' \\
mɡ &ˈstɑ\textbf{mɡ} & `stomach' \\
mt &ˈɡɑ\textbf{mt} & `elbow' \\
ŋdɡ & pɑ\textbf{ŋdɡ} & `mouse'\\
ŋf & ɡre\textbf{ŋf} & `claw'\\
ŋp & ka\textbf{ŋp} & `field'\\
\lspbottomrule
\end{tabular}
\end{minipage}
\end{table}

*C[+\textsc{son}]]$_\sigma$ outranks \textsc{Contig}(PWd), as illustrated in \tabref{haretableau}.


\begin{table}
\caption{ˈliːver `hares', from \REF{sonorityepenthesis}}
\label{haretableau}
\begin{center}
\ShadingOn
\begin{tableau}{c|s|s|s} 
\inp{/ˈliːvr-$\emptyset$/}      \const*{\textsc{*C[+son]]$_\sigma$}}  \const{\textsc{Contig}(PWd)} \const{\textsc{*Complex}} \const{\textsc{Contig}(PP)}
\cand{ˈliːvr}               \vio{*!}     \vio{}  \vio{*}    \vio{}
\cand[\Optimal]{ˈliːver}    \vio{}      \vio{*}  \vio{}     \vio{*}
\end{tableau}
\end{center}
\end{table}


To summarize, we have identified two considerations that drive epenthesis in Bolognese.  The first is \textsc{*Complex}, whose effect is visible outside the PWd, triggering epenthesis in final clusters involving consonantal enclitics.  Within the PWd, epenthesis eradicates sonorant-final coda clusters.

So far we have dealt only with examples in which the epenthetic vowel is [e], but in certain situations [u] appears instead.  We turn now to those contexts.


\section{Epenthetic [u]}\label{epentheticu}

As we have said, the primary epenthetic vowel in Bolognese is [e], which surfaces in a variety of contexts.  But when followed by a labial consonant, the epenthetic vowel is instead [u].  For example, [m]~-- being a sonorant~-- unsurprisingly triggers epenthesis in coda clusters, just like the other sonorants shown in \REF{sonorityepenthesis}.  But the epenthetic vowel that precedes [m] is [u]:

\NumTabs{3}
\begin{exe}
    \ex \label{epenthesism}
    \begin{xlist}
        \ex \textsc{fsg} [-a] / \textsc{fpl} [-$\emptyset$]
            \begin{xlisti}
                \ex ˈaːnma / ˈaːn\textbf{u}m\tab `soul' / `souls'
                \ex ˈfɑʊrma / ˈfɑʊr\textbf{u}m\tab `form' / `forms'
                \ex baˈtaɪzma / baˈtaɪz\textbf{u}m\tab `baptism' / `baptisms'
            \end{xlisti}
        \ex \textsc{infinitive} [-ˈɛr] / \textsc{pres.1sg} [-$\emptyset$]
            \begin{xlisti}
                \ex kalˈmɛːr / a꞊ˈkɛːl\textbf{u}m\tab `to calm' / `I calm'
                \ex laɡarˈmɛːr / a꞊ˈlɛːɡr\textbf{u}m\tab `to weep'  / `I weep'
                \ex farˈmɛːr / a꞊ˈfaɪr\textbf{u}m\tab `to stop' / `I stop'
            \end{xlisti}
        \ex \textsc{adj\textsubscript{fsg}} [-a] / \textsc{adj\textsubscript{msg}} [-$\emptyset$]
            \begin{xlisti}
                \ex ˈuːltma / ˈuːlt\textbf{u}m\tab `last.\textsc{fs}' / `last.\textsc{ms}'
                \ex ˈsɛːtma / ˈsɛːt\textbf{u}m\tab `seventh.\textsc{fs}' / `seventh.\textsc{ms}'
            \end{xlisti}
    \end{xlist}
\end{exe}



We attribute the appearance of [u] in \REF{epenthesism} to \textsc{Agree}(lab)-rime \REF{agreedef}  (see, e.g., \cite{Lombardi:1999} for discussion of \textsc{Agree} constraints).  Under the assumption that the distinction between round and unround vowels is captured formally by the feature [labial] (as opposed to [round]; \citealt{Clements:1991}), this constraint can compel epenthesis of a round vowel like [u] when the following coda consonant is [+labial].  \textsc{Agree}(lab)-rime holds only for segments appearing in the same rime; evidence for this restriction on the constraint's effect is presented below.


\ea
\label{agreedef} 
\textsc{Agree}(lab)-rime: within a rime, adjacent segments must match for [labial].
\z


Epenthesis of [u] is an example of The Emergence of the Unmarked \citep{mccarthy:1994}.  \textsc{Agree}(lab)-rime is outranked by \textsc{Ident}(labial), which prevents underlying vowels from becoming round to match a following labial coda.  As \REF{Vm} shows, vowel quality before [m] is not generally restricted.  But epenthetic vowels have no input correspondent, and \textsc{Agree}(lab)-rime can influence their realization.


\begin{exe}
    \ex \label{Vm}
        \begin{xlist}
            \ex θ\textbf{i}mˈzɛːra\tab `bedbug infestation'
            \ex ʣ\textbf{e}mˈleŋ\tab `gem.\textsc{dim}'
            \ex pr\textbf{e}m\tab `first'
            \ex krizaŋˈt\textbf{eː}m\tab `chrysanthemum'
            \ex ˈ\textbf{ɑ}mbra\tab `shadow'
            \ex eˈkɔn\textbf{o}m\tab `treasurer'
            \ex ˈ\textbf{o}md\tab `humid'
        \end{xlist}
\end{exe}

The effect of \textsc{Agree}(lab)-rime is illustrated in \tabref{soulsprelim}. \footnote{To keep the tableau simple, \textsc{*Complex} and the \textsc{Contiguity} constraints are omitted from most subsequent tableaux in this section.  As \tabref{soulsprelim} shows, they are ranked too low to affect the outcome in the kinds of cases presently under consideration.} \textsc{*C[+son]]$_\sigma$} compels epenthesis, and \textsc{Agree}(lab)-rime selects the candidate with an epenthetic [u].


\begin{table}
\caption{ˈaːn\textbf{um} `souls', from \REF{epenthesism}}
\label{soulsprelim}
\begin{center}
\ShadingOn
\begin{tableau}{c:c|c:s|s|s|s:s} 
\inp{/ˈaːnm-∅/}      \const*{\rotatebox{90}{\textsc{*C[+son]]$_\sigma$}}}  \const*{\rotatebox{90}{\textsc{Ident}(lab)}} \const*{\rotatebox{90}{\textsc{Agr}(lab)-rime~}} \const*{\rotatebox{90}{\textsc{Contig}(PWd)}} \const*{\rotatebox{90}{*\textsc{Complex}}} \const*{\rotatebox{90}{\textsc{Contig (PP)}}} \const*{\rotatebox{90}{*[V, +lab]}} \const*{\rotatebox{90}{*[V, +hi]}}
\cand{ˈaːnm}               \vio{*!}     \vio{}  \vio{}    \vio{}                                    \vio{*}     \vio{}  \vio{}    \vio{}
\cand{ˈaːnem}              \vio{}     \vio{}  \vio{*!}    \vio{*}                                      \vio{}     \vio{*}  \vio{}    \vio{}
\cand[\Optimal]{ˈaːnum}    \vio{}      \vio{}  \vio{}     \vio{*}
                             \vio{}     \vio{*}  \vio{*}    \vio{*} 
\end{tableau}
\end{center}
\end{table}



Epenthesis of [e] when the relevant coda consonant is not [+labial] has two possible explanations.  Either \textsc{Agree}(lab)-rime requires the epenthetic vowel to match the coda consonant's [--labial] specification, or \textsc{Agree}(lab)-rime is ambivalent in the face of a [--labial] coda consonant and the constraints *[V, +lab] and *[V, +hi] favor [e].  For purposes of illustration, we adopt the former approach.

Curiously, labial consonants trigger the appearance of the high vowel [u], not a mid vowel [o] or [\o], either of which would satisfy *[V, +hi] and be more similar to the default [e].  To account for this we adopt \textsc{*RoLo} and \textsc{*RoFro} \citep{groundedphono, kaun, kaun:typology}, which prohibit round non-high vowels and round front vowels, respectively.  Their effect is visible in \tabref{souls}: candidate (a) is eliminated by \textsc{*C[+son]]$_\sigma$}, and of the remaining candidates, only candidate (f) satisfies \textsc{Agree}(lab)-rime, \textsc{*RoLo}, and \textsc{*RoFro}.

\begin{table}
\caption{ˈaːn\textbf{um} `souls', from \REF{epenthesism}}
\label{souls}
\begin{center}
\ShadingOn
\begin{tableau}{c:c|c:c:c|s:s} 
\inp{/ˈaːnm-∅/}      \const*{\rotatebox{90}{\textsc{*C[+son]]$_\sigma$}}}  \const*{\rotatebox{90}{\textsc{Ident}(lab)}} \const*{\rotatebox{90}{\textsc{Agr}(lab)-rime~}} \const*{\rotatebox{90}{*\textsc{RoLo}}} \const*{\rotatebox{90}{*\textsc{RoFro}}} \const*{\rotatebox{90}{*[V, +lab]}} \const*{\rotatebox{90}{*[V, +hi]}}
\cand{ˈaːnm}               \vio{*!}     \vio{}  \vio{}    \vio{}                                    \vio{}     \vio{}  \vio{}
\cand{ˈaːnem}              \vio{}     \vio{}  \vio{*!}    \vio{}                                      \vio{}     \vio{}  \vio{}
\cand{ˈaːnom}               \vio{}     \vio{}  \vio{}    \vio{*!}                                    \vio{}     \vio{*}  \vio{}
\cand{ˈaːn\o m}              \vio{}     \vio{}  \vio{}    \vio{*!}                                      \vio{*!}     \vio{*}  \vio{}
\cand{ˈaːnym}              \vio{}     \vio{}  \vio{}    \vio{}                                      \vio{*!}     \vio{*}  \vio{*}
\cand[\Optimal]{ˈaːnum}    \vio{}      \vio{}  \vio{}     \vio{}
                             \vio{}     \vio{*}  \vio{*}     
\end{tableau}
\end{center}
\end{table}

To our knowledge, the only other labial consonant that triggers a preceding epenthetic vowel is [v] (cf. forms with coda-cluster-final [b, p, f] in \tabref{obstruentclusters}).  As our analysis predicts, that epenthetic vowel is [u]:


\begin{exe}
    \ex \label{vepenthesis}
        \begin{xlist}
            \ex \textsc{fsg} [-a] / \textsc{fpl} [-$\emptyset$]
                \begin{xlisti}
                    \ex ˈseːrva / ˈseːr\textbf{u}v\tab `servant' / `servants'
                    \ex ˈkaˈteːrva /  kˈateːr\textbf{u}v\tab `multitude' / `multitudes'
                \end{xlisti}

            \ex \textsc{infinitive} [-ˈɛr] / \textsc{pres.1sg} [-$\emptyset$]
                \begin{xlisti}
                    \ex kurˈvɛːr / a=ˈkuːr\textbf{u}v\tab `to bend' / `I bend'
                    \ex userˈvɛːr / t꞊uʼseːr\textbf{u}v\tab `to observe' / `I observe'
                \end{xlisti}

            \ex \textsc{N\textsubscript{fsg}} [-a] / \textsc{N\textsubscript{msg}} [-$\emptyset$]
                \begin{xlisti}
                    \ex ˈvadva /ˈvad\textbf{u}v\tab `widow' / `widower'
                \end{xlisti}

            \ex \textsc{N\textsubscript{msg}.dim} [-ˈeŋ] / \textsc{N\textsubscript{msg}} [-$\emptyset$]
            \begin{xlisti}
                \ex narˈveŋ /ˈneːr\textbf{u}v\tab `little nerve' / `nerve'
            \end{xlisti}
        \end{xlist}
\end{exe}



As with [m], \textsc{Agree}(lab)-rime favors [u] to match the [+labial] [v].  What is most notable about these examples, however, is that [v] triggers epenthesis in the first place.  As shown in \REF{wordInternalstclusters} and \tabref{obstruentclusters}, epenthesis does not usually occur when a cluster ends with an obstruent.  We argue in Section \ref{vsonorant} that [v] is in fact a sonorant in Bolognese, so the examples in \REF{vepenthesis} simply further illustrate epenthesis in sonorant-final clusters.  Before turning to that argument, however, some loose ends need attention.

As discussed above, \textsc{Ident}(labial) prevents \textsc{Agree}(lab)-rime from causing non-epenthetic vowels to change.  This is illustrated in \tabref{groove} with the form [iŋˈkɛːv] `groove.'  The underlying /ɛː/ surfaces faithfully despite the following [v].


\begin{table}
\caption{iŋˈkɛːv `groove'}
\label{groove}
\begin{center}
\ShadingOn
\begin{tableau}{c:c|s:s:s|s:s} 
\inp{/iŋˈkɛːv/}      \const*{\rotatebox{90}{\textsc{*C[+son]]$_\sigma$}}}  \const*{\rotatebox{90}{\textsc{Ident}(lab)}} \const*{\rotatebox{90}{\textsc{Agr}(lab)-rime~}} \const*{\rotatebox{90}{*\textsc{RoLo}}} \const*{\rotatebox{90}{*\textsc{RoFro}}} \const*{\rotatebox{90}{*[V, +rnd]}} \const*{\rotatebox{90}{*[V, +hi]}}
\cand[\Optimal]{iŋˈkɛːv}     \vio{}     \vio{}  \vio{*}    \vio{}                                    \vio{}     \vio{}  \vio{*}
\cand{iŋˈkuːv}              \vio{}     \vio{*!}  \vio{}    \vio{}                                      \vio{}     \vio{*}  \vio{**}
\end{tableau}
\end{center}
\end{table}

Furthermore, examples such as [ˈliːv꞊\textbf{et]} `Get up!' (with an epenthetic [e]) show that labial consonants trigger rounding only on preceding epenthetic vowels, not following ones.  The difference, we suggest, is that when an epenthetic vowel precedes a labial (or any) consonant, that consonant is invariably a coda~-- hence \textsc{Agree}(lab)-rime's requirement that only segments in the same rime must match for labiality.

The evaluation of [ˈliːv꞊\textbf{et]} is shown in \tabref{getup}.  \textsc{*Complex} compels epenthesis here because the form contains an enclitic, and this time \textsc{Agree}(lab)-rime favors an unround vowel, eliminating candidate (c); recall that if \textsc{Agree}(lab)-rime is interpreted to be ambivalent in this case, lower constraints favor [e], too.

\begin{table}
\caption{ˈliːv꞊\textbf{et} `Get up!'}
\label{getup}
\begin{center}
\ShadingOn
\begin{tableau}{c:c|c:s:s|s:s} 
\inp{/ˈliːv=t/}      \const*{\rotatebox{90}{\textsc{*C[+son]]$_\sigma$}}}  \const*{\rotatebox{90}{\textsc{Ident}(lab)}} \const*{\rotatebox{90}{\textsc{Agr}(lab)-rime~}} \const*{\rotatebox{90}{*\textsc{RoLo}}} \const*{\rotatebox{90}{*\textsc{RoFro}}} \const*{\rotatebox{90}{*[V, +rnd]}} \const*{\rotatebox{90}{*[V, +hi]}}
\cand{ˈliːv꞊t}                \vio{*!}     \vio{}  \vio{}                                       \vio{}   \vio{}     \vio{}  \vio{*}
\cand[\Optimal]{ˈliːv꞊et}     \vio{}     \vio{}  \vio{}                                      \vio{}   \vio{}     \vio{}  \vio{*}
\cand{ˈliːv꞊ut}                \vio{}     \vio{}  \vio{*!}                                      \vio{}   \vio{}     \vio{*}  \vio{**}
\end{tableau}
\end{center}
\end{table}


Finally, the data in \REF{sonorantclitics} show the convergence of the two environments for epenthesis that we have focused on here.  These examples show epenthesis in word-final clusters involving clitics, driven by \textsc{*Complex}.  Here, though, the clitics are sonorants and are therefore subject to \textsc{*C[+son]]$_\sigma$}.  Epenthesis occurs exactly as expected: [u] appears before [m] and [v], and [e] appears elsewhere.  A representative tableau is shown in \tabref{lifting}.

\begin{exe}
    \ex \label{sonorantclitics}
        \begin{xlist}
            \ex ˈliːv꞊\textbf{e}l\tab `Is he lifting (something) up?' / `lift him up!'
            \ex ˈliːv꞊\textbf{e}ŋ\tab `lift some up!'
            \ex ˈliːv꞊\textbf{u}m\tab `lift me up!'
            \ex liˈvɛːr꞊\textbf{u}v\tab `to lift you up'
        \end{xlist}
\end{exe}


\begin{table}
\caption{ˈliːv꞊\textbf{el} `Is he lifting (something) up?' / `lift him up!'}
\label{lifting}
\ShadingOn
\begin{tableau}{c|s|s|s} 
\inp{/ˈliːv꞊l/}      \const*{\textsc{*C[+son]]$_\sigma$}}  \const*{\textsc{Contig}(PWd)} \const*{\textsc{*Complex}} \const*{\textsc{Contig}(PP)}
\cand{ˈliːvl}               \vio{*!}     \vio{}  \vio{*}    \vio{}
\cand[\Optimal]{ˈliːvel}    \vio{}      \vio{}  \vio{}     \vio{*}
\end{tableau}
\end{table}



\section{The status of [v]}\label{vsonorant}
\citet{padgett:v} argues that in Russian and possibly other languages, the segment transcribed as [v] is more properly treated as a sonorant.  Bolognese appears to belong to this group of languages.  The contrast between, on the one hand, \REF{vepenthesis}, with [v]-final clusters, and \REF{sonorityepenthesis} and \REF{epenthesism}, with sonorant-final clusters, and, on the other hand, \REF{wordInternalstclusters}/\tabref{obstruentclusters}, with clusters with other obstruents in final position (including [b, p, f]), is just one piece of evidence for this position. % \textbf{(I added a couple more below. Are these examples the ones we want? (I mean, does a liquid/nasal + obstruent coda violate \textsc{*C[+son]]$_\sigma$}?) More, fewer, other examples? Should they be better ordered? Highlight [b, p, f] for contrast with [v]?)  AK: Example moved; see \REF{obstruentclusters}. I think what's there now is perfect. ER: Alrighty, then.}


In addition, Bolognese [v] sometimes alternates with [w], as in \REF{vw}.

\begin{exe}
    \ex \label{vw}
        \begin{xlist}
            \ex ˈak\textbf{w}a / ˈaku\textbf{v}\tab `water' / `waters'
            \ex iŋsiŋˈ\textbf{w}ɛr / t꞊iŋˈsiːnu\textbf{v}\tab `to insinuate' / `you insinuate'
            \ex koŋˈtiːɡ\textbf{w}a / koŋˈtiːɡu\textbf{v}\tab `contiguous.\textsc{fs}' / `contiguous.\textsc{ms}'
        \end{xlist}
\end{exe}

\noindent

Moreover, \citet[148]{caneparivitali:bolognese} write:

\begin{quote}
/v/ often vanishes: [farˈa(v)ɑŋna] `guinea fowl', [(v)ɲ o] `come (past part.)' (or also [farˈaʋɑŋna]); occasionally it becomes [w]: [asˈwad] `si vede/one sees'.
\end{quote}
Some sources (including \citeauthor{caneparivitali:bolognese} in the excerpt just provided) transcribe this sound as [ʋ], indicating that some listeners hear the sound as an approximant, not a fricative.\footnote{Historically, Bolognese [$\emptyset$/v/ʋ/w] comes from Latin [w]; perhaps it has not (yet?) fully transitioned from a sonorant to an obstruent. Perhaps relatedly, across all conjugations, in the imperfect and conditional, the stress is penultimate, and not final as with all other 2\textsc{pl} verb-forms. With those two forms, in the interrogative, no enclitic 2\textsc{pl} \textsc{scl} [v] is present, e.g. [maɲaˈresi꞊$\emptyset$] `Would you.\textsc{pl} eat?' With all other 2\textsc{pl} verb-forms, with final stress, an enclitic [v] does appear, whether \textsc{scl}, \textsc{dcl}, or \textsc{acl}. Adjacency to primary stress plays a clear role, perhaps both diachronically and synchronically. \label{historicalfootnote}}

\section{Discussion and conclusion}


Like other Romance languages, Bolognese shows epenthesis that is sensitive to morphological and prosodic structure.  Our account of this extends the analysis of \citet{cardinalettirepetti:clitics} to account for a collection of facts that are peculiar to Bolognese, such as variation in the quality of the epenthetic vowel, the avoidance of sonorant-final coda clusters, and [v]'s patterning with sonorants.

Chief among our claims is that the alternation between [C] and [VC] seen in Bolognese's enclitics involves epenthesis rather than deletion or suppletion.  This position has two major benefits.  First, it connects clitic allomorphy to broader epenthetic processes in the language.  Second, these clitics show extensive allomorphy \citep{rubinkaplanallomorphywithls}, and treating some of this allomorphy as epenthesis reduces the number of allomorphs in the lexicon and/or the number of clitic-specific processes that must be posited.

The analysis presented here represents just a first attempt to account for the interaction between vowel epenthesis and Bolognese's clitics.  Our focus has been on word-final clusters, but proclitics are also subject to epenthesis in familiar ways: [e] is epenthesized, except that [u] appears when sharing a rime with [m] or [v]:

\begin{exe}
    \ex \label{procliticepenthesis}
        \begin{xlist}
            \ex al꞊v\textbf{e}꞊ˈdɛːva\tab `He was waking you up.'
            \ex al꞊m\textbf{e}꞊ˈʦ{ftes}\tab `He's undressing me.'
            \ex l꞊\textbf{u}m꞊ˈda\tab `he gives me'
            \ex i꞊s\textbf{e}꞊ˈfteːveŋ\tab `they were getting dressed'
            \ex a꞊s\textbf{e}=ˈfteːveŋ\tab `we were getting dressed'
        \end{xlist}
\end{exe}



Notice that [e], not [u], appears before [f] in the final two examples in \REF{procliticepenthesis}.  Several explanations are available: perhaps [f] is syllabified as an onset here ([ft] clusters are attested in Bolognese); perhaps the \textsc{Agree} constraint used above might be further restricted to \emph{sonorants} that share a rime; or perhaps [e] is not epenthetic here, but rather part of the underlying representation of the reflexive clitic [se] and other relevant clitics.  Furthermore, alongside the similarities between proclitics and enclitics are substantive differences.  In particular \citet{rubinkaplanallomorphywithls} argue that proclitics exhibit rather extensive allomorphy that enclitics do not. %RELEVANT DATA HERE, PERHAPS 3MS.NOM. ED ASKS: BUT IS IT ABOUT ALLOMORPHY ONLY (THE [A] FORM) OR ABOUT HOW [A] DOESN'T OCCUR WITH ENCLITICS. OR ABOUT HOW [AL] VS [L] [LE] WITH PROCLITICS BUT [L] VS [EL] WITH ENCLITICS?  AARON ANSWERS: ALL OF IT, OR WHATEVER WE CAN REASONABLY PRSENT HERE.  ENOUGH TO SHOW THAT PROCLITICS DO THINGS ENCLITICS DON'T.
One example, of many similar, concerns the 3\textsc{msg} subject and direct object proclitics (as noted in the discussion above \REF{preverbal}), which regularly exhibit a form [al] that never occurs as an enclitic ([t꞊al=ˈvad]/*[t꞊el=ˈvad] `you see him' vs.\ [ˈvadr꞊el]/*[ˈvadr꞊al] `to see him') even though as an enclitic [al] would satisfy \textsc{*C[+\textsc{son}]]$_\sigma$} just as well as [e] epenthesis does. Another issue arises when both the subject and object clitics are 3\textsc{ms} and adjacent, which can only occur preverbally. The [al] allomorph is not permitted here for both, and epenthesis arises, though after the object clitic, rather than before it ([al꞊le=ˈvad]\slash *[al꞊al=ˈvad]\slash *[al꞊el=ˈvad] `he sees him').

Our supposition is that at least some of these differences arise from differences in combinatorial possibilities, as with the consecutive 3\textsc{ms} clitics, which occur only preverbally.  As another example, the 3\textsc{ms.nom} proclitic appears as [l] before vowels; when 3\textsc{ms.nom} is an enclitic it is always verb-final and therefore never prevocalic in isolation.  But in context, when the following word begins with a vowel, [l] appears.%  Other proclitic allomorphy is conditioned by particular sequences of clitics, and because only certain clitics can appear postverbally, those sequences never arise in the enclitic domain.

In sum, the analysis developed here supplies a foundation on which a broader treatment of clitic alternations in Bolognese, and indeed perhaps those found in other languages, can be built.

 

\section*{Acknowledgements}
The data in this paper were mostly collected or checked in fieldwork over several years with prominent members of the native speaker community that centers in the cultural association ``Al Pånt dla Biånnda.” Data were also drawn from \citet{caneparivitali:bolognese}, \citet{bologneseprimertwo}, and \citet{leprivitali:dict}. We thank our consultants for their patient help with the data, and we thank Roberto Serra and Daniele Vitali for extensive discussion of it. Thanks also to the audience at \textit{Epenthesis and Beyond} for feedback on this work.


{\sloppy\printbibliography[heading=subbibliography,notkeyword=this]}
\end{document}
