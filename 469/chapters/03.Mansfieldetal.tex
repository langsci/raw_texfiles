\documentclass[output=paper,colorlinks,citecolor=brown]{langscibook}
\ChapterDOI{10.5281/zenodo.14264532}
\author{John Mansfield\affiliation{The University of Melbourne}\orcid{} and Rosey Billington\affiliation{The Australian National University}\orcid{} and  Hywel Stoakes\affiliation{The University of Melbourne}\orcid{0000-0002-5543-8750}}
\title{Vowel predictability and omission in Anindilyakwa}

\abstract{The Australian language Anindilyakwa has some vowels that are to a large extent contextually predictable, and arguably epenthetic. In previous work on the language there are differing views on the segmental contrasts, phonotactic patterns and lexical representations of these vowels. Drawing on information-theoretic approaches, we investigate the predictability of vowel occurrence across different consonant environments in Anindilyakwa words, using orthographic representations from an existing wordlist, and speech production data collected with seven Anindilyakwa speakers. We find that there is a high level of predictability in the word-internal occurrence of non-low vowels compared to low vowels. At the same time, the word-final low vowel \textit{-a} is completely predictable. In our speech production data we find that the predictable vowels (both word-internal non-low vowels, and the final \textit{-a}) are quite frequently omitted even in a relatively careful speech register, while the unpredictable vowels are never omitted. Our findings support previous research that draws a connection between segmental predictability and phonetic reduction or deletion, and we show that this association extends to segments that can be analysed as epenthetic.}
  
\IfFileExists{../localcommands.tex}{
   \addbibresource{../localbibliography.bib}
   % add all extra packages you need to load to this file

\usepackage{tabularx,multicol}
\usepackage{url}
\urlstyle{same}

\usepackage{listings}
\lstset{basicstyle=\ttfamily,tabsize=2,breaklines=true}

\usepackage{langsci-basic}
\usepackage{langsci-optional}
\usepackage{langsci-lgr}
\usepackage{langsci-osl}
% \usepackage{./langsci/styles/langsci-lgr}
% \usepackage{./langsci/styles/langsci-osl}
% \usepackage{langsci-gb4e}

\usepackage{tikz}
\usetikzlibrary{patterns,calc}
\pgfdeclarepatternformonly{south east lines}{\pgfqpoint{-0pt}{-0pt}}{\pgfqpoint{3pt}{3pt}}{\pgfqpoint{3pt}{3pt}}{
    \pgfsetlinewidth{0.6pt}
    \pgfpathmoveto{\pgfqpoint{0pt}{3pt}}
    \pgfpathlineto{\pgfqpoint{3pt}{0pt}}
    \pgfpathmoveto{\pgfqpoint{.2pt}{-.2pt}}
    \pgfpathlineto{\pgfqpoint{-.2pt}{.2pt}}
    \pgfpathmoveto{\pgfqpoint{3.2pt}{2.8pt}}
    \pgfpathlineto{\pgfqpoint{2.8pt}{3.2pt}}
    \pgfusepath{stroke}}
    
\usepackage{stmaryrd}
\usepackage{wasysym}
\usepackage{multirow}
\usepackage{caption}
\usepackage{subcaption}
\usepackage{mathrsfs}
\usepackage{qtree}

\usepackage{linguex}


   %pminos do not split footnotes
% \interfootnotelinepenalty=10000 %Footnote in Laporte chapters has to be split SN


%\DeclareIndexNameFormat{default}{%
%\nameparts{#1}%
%\usebibmacro{index:name}%
%{\index[names]}%
%{\namepartfamily}%
%{\namepartgiveni}%
% {}% L1
% {}% L2
%{\namepartprefix}% generates spurious space L3
%{\namepartsuffix}% generates spurious space L4
%}

%  {\DeclareIndexNameFormat{default}{%
%     \usebibmacro{index:name}{\index[names]}{#1}{#3}{#5}{#7}}}

%\DeclareIndexNameFormat{default}{%
%  \usebibmacro{index:name}{\sindex[nom]}{#1}{#3}{#5}{#7}}

%\DeclareIndexNameFormat{default}{%
%  \usebibmacro{index:name}{\sindex[person]}{#1}{#3}{#5}{#7}}
%\DeclareIndexNameFormat{default}{%
%\nameparts{#1} \usebibmacro{index:name}{\sindex[person]]}{\namepartfamily}{‌​\namepartgiven}{\nam‌​epartprefix}{\namepa‌​rtsuffix}}

%\newcommand{\smiley}{:)}

%\renewbibmacro*{index:name}[5]{%
%\usebibmacro{index:entry}{#1}%
%{\iffieldundef{usera}{}{\thefield{usera}\actualoperator}\mkbibindexname{#2}{#3}{#4}{#5}}}

% \newcommand{\noop}[1]{}

%remove for final
%\overfullrule=1mm

\newcommand{\tobi}[2]}}
\renewcommand{\S}[1]{\tobi{#1}{\textsc{*}}}

% this volume references
% puts: [this volume]
% already defined: \citetv
%\newcommand{\citepv}[1]{(\citeauthor{#1} \citeyear*{#1} [this volume])}
\newcommand{\citealtv}[1]{\citeauthor{#1} \citeyear*{#1} [this volume]}

%parentheses around example number
\newcommand{\pref}[1]{(\ref{#1})}

% in-text examples

\newcommand{\lnex}[1]{\textit{#1}} %target lang word
\newcommand{\lnlit}[1]{(lit.: `#1')} %literal reading
\newcommand{\lnlat}[1]{(#1)} % latinization
\newcommand{\lntrans}[1]{`#1'} %translation
\newcommand{\lnexl}[2]%
{\lnex{#1}{} \lnlat{#2}} % ex with latinization
\newcommand{\lnexlat}[3]{\lnex{#1}{} \lnlat{#2}{} \lntrans{#3}} % ex with latinization and tranl.

%ch01
\newcommand{\co}[1]{\mbox{\textbf{#1}}}

%ch09

\newcommand{\cyrbulg}[1]{\begin{otherlanguage*}{bulgarian}#1\end{otherlanguage*}}


%ch10
\newcommand{\nlp}{{\small NLP}}
\newcommand{\mwe}{{\small MWE}}
\newcommand{\rae}{{\small RAE}}
\newcommand{\lvc}{{\small LVC}}
\newcommand{\pos}{{\small P}o{\small S}}
%\newcommand{\todo}[1]{ \textcolor{red}{#1} }

%\renewcommand{\labelenumi}{\theenumi}
%\ainamefmt{{vv}{ll}{, ff}{, jj}} % fullname

\newcommand{\biberror}[1]{{\color{red}#1}}

\newcommand{\osenovaitem}{--~}
   %% hyphenation points for line breaks
%% Normally, automatic hyphenation in LaTeX is very good
%% If a word is mis-hyphenated, add it to this file
%%
%% add information to TeX file before \begin{document} with:
%% %% hyphenation points for line breaks
%% Normally, automatic hyphenation in LaTeX is very good
%% If a word is mis-hyphenated, add it to this file
%%
%% add information to TeX file before \begin{document} with:
%% %% hyphenation points for line breaks
%% Normally, automatic hyphenation in LaTeX is very good
%% If a word is mis-hyphenated, add it to this file
%%
%% add information to TeX file before \begin{document} with:
%% \include{localhyphenation}
\hyphenation{
    Beck-man
    Ngu-yen
    back-chan-nel
    back-chan-nels
    mo-not-o-nous
    ste-reo-typ-i-cal
}

\hyphenation{
    Beck-man
    Ngu-yen
    back-chan-nel
    back-chan-nels
    mo-not-o-nous
    ste-reo-typ-i-cal
}

\hyphenation{
    Beck-man
    Ngu-yen
    back-chan-nel
    back-chan-nels
    mo-not-o-nous
    ste-reo-typ-i-cal
}

   \boolfalse{bookcompile}
   \togglepaper[3]%%chapternumber

}

\begin{document}
\maketitle 
\label{ch3}

\hypertarget{introduction}{%
\section{Introduction}\label{introduction}} 

In this chapter we focus on one particular aspect of epenthesis, namely the \emph{predictability} of epenthetic segments. We present new data and analysis on Anindilyakwa, a northern Australian language in which the presence of particular vowel phones is to a large extent contextually predictable. In part this involves canonically epenthetic vowels, inserted predictably at morphological boundaries and the end of words. But vowel predictability in Anindilyakwa also involves morpheme-internal positions, where vowels that can be grouped as “non-low” are to a large extent predictable according to consonantal context. We can thus posit redundancy\hyp free representations in which all boundary vowels and many internal vowels are omitted, as in \REF{ex:mansfield:1}.\footnote{Our morphological glosses follow the Leipzig Glossing Rules, with one language-particular addition: \textsc{mut}(ual), a marker of ‘symmetrical information access from speaker perspective’ (\citealt{Bednall2021}) (see example \REF{ex:mansfield:1}).} 
In this chapter we investigate both types of predictability, following an information-theoretic approach to phonological analysis.

\ea
\label{ex:mansfield:1}
\glll {nəŋərəŋkənama}\\
  nŋ-rŋk-na-m\\
 \textsc{1s}{}-see-\textsc{npst-mut}\\
\glt ‘I see.’
\z
\il{Anindilyakwa}


We begin this chapter by briefly surveying the common ground between information theory and phonological analysis, showing that this has particular relevance to epenthesis (Section \ref{sec:mansfield:2}). We then introduce the Anindilyakwa language and its segmental phonology (Section \ref{sec:mansfield:3}), before presenting new data on this language. Our first original contribution investigates the extent to which the appearance of non-low vowels is predictable according to context, based on orthographic forms in a wordlist (Section \ref{sec:mansfield:4}). Our second contribution uses new field recordings to analyse the variable production of vowels in elicited sentences, showing that more predictable segments are more likely to be omitted (Section \ref{sec:mansfield:5}). In the final section we interpret our findings with respect to theories of lexical representation, and point out some directions for further research (Section \ref{sec:mansfield:6}).

\section{Phonological predictability, phonetic reduction and epenthesis}\label{sec:mansfield:2}

The core concerns of phonology have substantial overlap with those of information theory (IT). Both disciplines analyse how sets of distinct symbols are combined into sequences to convey meanings, with concomitant issues such as contrast, redundancy and contextual predictability (\citealt{Hockett:1967,Goldsmith2000}). One salient difference between standard phonological analysis, and IT analysis, is the use of discrete vs gradient methods. Phonology is mostly concerned with categorical distinctions (if a pair of phones ever contrast in the same environment, then they are distinct phonemes), whereas IT quantifies \textit{degrees} of contrast, based on the probabilistic distribution of symbols in context. An IT approach is useful for studying Anindilyakwa vowels because they are largely predictable according to context, while nonetheless exhibiting contrasts in a few lexemes. IT provides formulae for mapping out this ‘intermediate’ territory between the fully predictable and the fully contrastive (\citealt{Hall2009,hallk2013typology,Parker2015}).  While IT inherently lends itself to analysing gradient phenomena, there has as yet been little discussion of how it relates to gradient models of phonology such as exemplar theory (\citealt{bybee2000phonology,pierrehumbert2001exemplar}). It is beyond the scope of the current study to bridge the gap between IT and gradient theories of phonology, but in our closing discussion we will argue that IT and gradient lexical representations are fundamentally compatible.

Information theory treats communication as the transmission of messages from a sender to a receiver via symbolic sequences (\citealt{Hartley1928,Shannon1948}). The messages could for example be words selected from a lexicon, where each word is uniquely associated with a sequence of phonological symbols. A fundamental constraint of communication is that the symbolic sequences must be relatively short to be usable, and IT captures this constraint by calculating the shortest possible encoding scheme for an array of messages. The calculation of these efficient encodings leads to more general formulae for the information-content of symbols, and the effects of context in symbolic sequences.\footnote{For mathematical details, see \citet{Shannon1948} and \citet{CoverThomas2002}.}

To some extent, IT provides an alternative theoretical path to the same types of conclusions as standard phonology. If we have a lexicon where many words contain a substring [tək], but there are no words containing the substring [tk], the intervening schwa does not distinguish between any possible messages. It thus conveys zero information, and can be omitted from an efficient encoding. Phonological analysis might similarly conclude that there is a redundancy-free underlying sequence /tk/, and the schwa is inserted at some level of derivation.

But IT additionally quantifies the amount of information in phonological symbols that are only partially predictable. Given a lexicon where we frequently encounter [tək], but there are also a few words containing [tk], the intervening schwa now carries some quantity of information, because it contributes to distinguishing one word from another. This degree of predictability is formulated as `surprisal', that is \emph{negative log probability}, where a low surprisal value captures the highly predictable nature of schwa insertion. Surprisal can also be thought of as a quantity of `information', because unsurprising segments have little capacity to distinguish between messages, and are therefore relatively uninformative \citep{Shannon1951}. Using a binary logarithm, quantities of information are often expressed as ‘binary digits’ or ‘bits’, reflecting the fact that the surprisal of a message is equal to its binary character length in an optimally efficient encoding \citep{Shannon1948}. 

IT facilitates various predictions about natural language phonology and phonetics. Among other things, it predicts that phones or phone-sequences should be hyper-articulated when they contain a greater quantity of information (i.e. more discriminative of distinct messages), but hypo-articulated when they contain less information \citep{Lindblom1990}. In recent years, a substantial body of work has shown that when a phone is less informative, its phonetic cues are indeed more likely to be reduced or deleted (e.g. \citealt{vanSonvanSanten2005,Hall2009,CohenPriva2015,CohenPriva2017,ShawKawahara2017,HallEtAl2018}).

Epenthesis is typically defined as a relationship between underlying lexical representations and surface phonology, whereby a segment is present on the surface but not in the underlying representation. While much of the research on epenthesis focuses on its function to satisfy phonological constraints \citep{Hall2011}, in this chapter we focus on its predictable character. In general, segments are only considered epenthetic if they occur predictably in some context, whereas segments that are distinctive to particular lexical items are considered to be part of the underlying lexical representation. If a purportedly epenthetic vowel ceases to be predictable, for example through sound changes in the conditioning consonants, then its epenthetic status is called into question \citep[1579]{Hall2011}. Since epenthetic segments are by their nature predictable, IT would therefore suggest that epenthetic segments should be more subject to phonetic reduction or omission, compared to more lexically informative phones – and this is exactly what has been found, at least in some cases of epenthesis (\citealt{HallN2013}). There has also been some research connecting IT directly with epenthesis, for example arguing that the unusual occurrence in French of [ø] as an epenthetic vowel may be explained by the low informativity of this segment in lexical representations (\citealt {hume_anti-markedness_2011}; see also \citealt{tily_rational_2012}).

The current study draws on methods developed in IT phonology, applying them to the arguably epenthetic vowels of Anindilyakwa. While previous studies have mostly focused on the contextual predictability between pairs of phones, or the general information content of one particular phone type, we instead measure the predictability of vowel presence vs absence in consonantal contexts. Additionally, we investigate the proposed association of predictability with phonetic reduction and omission, by investigating vowel omission patterns in speech production.


\section{The Anindilywaka language and its segmental phonology}\label{sec:mansfield:3}

Anindilyakwa is a Gunwinyguan language (Glottocode: anin1240), owned and spoken by the Warnumamalya people of the Groote Archipelago in northern Australia. Warnumamalya generously shared information about Anindilyakwa that is the basis for this study, provided recordings of their speech to the first author, and supported this research especially through the Groote Eylandt Language Centre.

There are several previous studies of Anindilyakwa phonology (\citealt{Heath:2020aa,Stokes1981,Leeding1989,vanEgmond2012,vanEgmondBaker2020}). In this chapter we draw on all these previous works, but perhaps the most important is the earliest, that being Heath’s unpublished sketch where he observes that many Anindilyakwa vowels are largely predictable in their occurrence and quality. This is relatively unusual among Australian languages (but see also the proposals of ‘vertical vowel systems’ in Arandic languages, e.g. \citealt{breen2005central}).

Anindilyakwa has a small phonemic inventory, highly constrained phonotactics, and unusually long words. All words begin with a consonant or a low vowel, and in citation form all words end in \textit{-a}. Word-internally there is a relatively constrained range of consonant clusters (\citealt{Heath:2020aa,vanEgmond2012}), which partly motivates the analysis of vowel epenthesis to be described below. Words tend to be long (\citealt[68]{Leeding1989}), even when they denote relatively basic concepts, e.g. \textit{nɛɲcarŋaʎiʎa} `boy', \textit{ɛŋkəparŋʷarŋʷa} `heavy' and \textit{al̪uŋkᵚuwaruwaʎa} `shade' .

There have been various analyses of the underlying vowel phonemes and patterns of surface allophony in Anindilyakwa, with previous work noting vowel frontness and rounding is conditioned by the place-of-articulation of neighbouring consonants. Partly due to the differing analyses, the same lexical items are sometimes represented with different orthographic vowels in different sources. Despite different phonemic analyses, there is broad agreement that (surface) vowel phones in Anindilyakwa include two \textsc{low} vowels [ɛ, a], and three \textsc{non-low} vowels [i, ə, u]. 

There are also various analyses of the consonant phonemes, but for our study we use the system represented in \tabref{tab:mansfield:1}. It is broadly typical of Australian languages, with a large number of place contrasts, and more sonorants than obstruents \citep{butcher_placearticulation_2006}. The most recent analysis of the consonant inventory posits complex segments, namely a prenasalised stop series /mp, nt, n̪t̪, ɳʈ, ɲc, ŋk, ŋkʷ/, and labio-velar /kp, ŋm, ŋp/ (\citealt{vanEgmond2012};  \citealt{vanEgmondBaker2020}), but we treat these instead as clusters.\footnote{Also our ‘Anterior’ place of articulation merges potential distinctions between apical-alveolar and lamino-dental articulations, since we did not observe any contrasts between these in our field data, and in any case such contrasts are considered marginal (for discussion see  \citealt{vanEgmond2012}). } 
This is firstly because like other clusters they appear only in word-internal positions; secondly because their articulation seems to us much like clusters in other Australian languages; and thirdly because of phonotactic patterns that form the core of the analysis below (see Section \ref{sec:mansfield:4.1}). 
 
\begin{table}
\caption{Anindilyakwa consonant inventory}
\label{tab:mansfield:1}
\begin{tabular}{l *6{c}}
\lsptoprule
& Labial & Ante- & Retro- & Alveo- & Dorsal & Labialised \\
&        & rior  & flex   & palatal &       & dorsal\\
\midrule
Stop & p & t̪ (t) & ʈ & c & k & kʷ\\
Nasal & m & (n̪) n & ɳ & ɲ & ŋ & ŋʷ\\
Lateral &  & l̪ (l) & (ɭ) & ʎ & & \\
Trill &  & r &  &  &  & \\
Approx & w &  & ɻ & j &  & \\
\lspbottomrule
\end{tabular}
\end{table} 

As observed by Heath and subsequent analysts, non-low vowels are generally realised as [u] when adjacent to a labial or labialised dorsal consonant, and [i] when adjacent to an alveo-palatal consonant. Both contexts are exemplified in [jipuɻat̪a] ‘wallaby’.\footnote{There is no consensus on how they are realised when these two conditions overlap, e.g. following a labial and preceding a palatal, and there are differences in the observations of vowel distributions relative to specific consonant segments within broad place categories.} Elsewhere, non-low vowels are generally realised as [ə], as for example in [t̪ənəna] `mosquito'. At the same time, various lexical exceptions have been observed \citep{Stokes1981}, especially in instances where [i] has no conditioning palatal, as in [məɻirpa] ‘back’.\footnote{Our production data includes several of these lexemes with ‘un-conditioned’ close front vowels. Some, such as \textit{ɛniŋapa} ‘good’, show phonetic variation [ɛniŋapa {\textasciitilde} ɛnəŋapa]. But \textit{məɻirpa} ‘back’ appears to have a quite consistent close front vowel. Further investigation of vowel quality is an important topic for further research.} There is also some evidence for conditioning in the low vowels, where [ɛ] frequently appears adjacent to a palatal and is arguably an allophone of [a] in this context, as in [micijɛʎa] ‘beach’, though this pattern appears to be weaker than the non-low vowel conditioning.

The predictability of non-low vowel quality suggests a phonological analysis where there is a single underlying non-low vowel, with surface allophony accounting for [i {\textasciitilde} ə {\textasciitilde} u] \citep{Leeding1989}. But Heath goes further than this, additionally proposing that since Anindilyakwa has pervasive cluster constraints in its surface phonology, the very presence of non-low vowels can be considered epenthetic – a product of surface articulation, but absent in underlying lexical representations such as /jpɻat̪/ → [jipuɻat̪a].
A more canonical type of vowel epenthesis is also observed at word-internal morphological boundaries, where non-low vowels are predictably interpolated as in /k-wʈa-n-m/ → [kuwuʈanəma] ‘will climb (\textsc{irr}-climb-\textsc{npst-mut})’. Heath also reports that the insertion of non-low vowels between consonants is somewhat variable, and their phonetic character is ‘brief and indistinct’ (\citealt{Heath:2020aa}, section 1.8). These claims are supported in our speech production study below.

As shown in the examples above, there is another another aspect of vowel predictability in Anindilyakwa: every word in the lexicon ostensibly ends with \textit{{}-a}. This suggests that while non-low vowels are epenthesised word-internally, the low vowel \textit{{}-a} is epenthesised word-finally (see \citealt{vanEgmond2012} for further discussion).

To summarise the existing analyses of Anindilyakwa vowels, the top line in \REF{ex:mansfield:100} shows the surface form of a noun with a case suffix. The next line down reflects the view shared by all previous work that both the non-low vowel at the morphological boundary, and the low vowel at the end of the word, are epenthetic. The line below that reflects an additional proposal (as in \citealt{Leeding1989}) that there is a single underlying non-low vowel, with surface allophones conditioned by neighbouring consonants. The line below that reflects Heath's additional proposal that morpheme-internal non-low vowels might be considered epenthetic. Since non-low vowels are uncontroversially epenthesised at word-internal morphological boundaries, the implication is that all (word-internal) non-low vowels may be treated as epenthetic. This proposal is the most efficient from an IT point of view, since it reduces the length of symbolic representations, and it is this proposal that we examine below through the lens of predictability and informativity.

% \ea
% \label{ex:mansfield:100}
% \glllll Surface: {jipuɻat̪əjat̪a}\\
%   Boundary epenthesis: jipuɻat̪-jat̪\\
%   Non-low allophony: jəpəɻat̪-jat̪\\
%   Non-low epenthesis: jpɻat̪-jat̪\\
%   wallaby-\textsc{purp}\\
% \glt ‘For wallaby.’\z
% \il{Anindilyakwa}


\ea
\label{ex:mansfield:100}
\glllll {jipuɻat̪əjat̪a}  \jambox {(surface)}\\
  jipuɻat̪-jat̪  \jambox {(boundary epenthesis)}  \\
  jəpəɻat̪-jat̪  \jambox {(non-low allophony)}    \\
  jpɻat̪-jat̪    \jambox {(non-low epenthesis)}  \\
  wallaby-\textsc{purp}       \\
\glt ‘For wallaby.’\z
\il{Anindilyakwa}

While Anindilyakwa non-low vowels can be analysed as synchronically epenthetic, historically they derive from reduction of full vowels *i, *u and *a in coarticulatory and unstressed environments, as shown in recent work reconstructing the ancestor of Anindilyakwa and closely-related Wubuy (\citealt[159]{vanEgmondBaker2020}). This suggests an alternative synchronic analysis of non-low vowels as being variably deleted, rather than variably inserted. While this analysis is quite plausible, for most of this chapter we focus instead on the epenthetic analysis as this reflects our interest in the efficiency of redundancy-free coding. However, in our closing discussion we will further probe the question of lexical representations, and argue that probabilistic, gradient approaches ultimately avoid having to make an analytic choice between insertion and deletion.

Other elements of Anindilyakwa phonology remain somewhat under-studied, especially prosodic phonology (but see \citealt[138--141]{Leeding1989}; \citealt[27ff.]{vanEgmond2012}). While we suspect that higher prosodic structures may have an important role in the patterning of non-low vowels, this dimension of the system must await further research.


\section{Predictability of vowels in Anindilywaka lexical data}\label{sec:mansfield:4}


In this section we investigate vowel predictability in the Anindilyakwa lexicon, using a wordlist of 3038 orthographically-represented lexical items drawn from a dictionary \citep{Waddy1989}. From this, we extract 6943 consonant contexts that are the potential environments for non-low vowels to occur.\footnote{An estimation of the discourse frequency of these lexical items would also be of interest for an IT analysis, but unfortunately data of this type is not currently available, and we therefore treat all lexemes as equiprobable messages in a communication channel.}
In this section we illustrate wordlist citation forms using italic text as in \textit{jipuɻat̪a} ‘wallaby’, transformed into IPA to enable cross-reference to the phonemic inventory outlined above. We use angled brackets to indicate redundancy-free phonological representations as in /jpɻat̪/.

Based on previous analyses, we assume that word-internal low vowels [ɛ, a] are encoded in lexical representations, and therefore wherever two consonants occur in sequence \textit{without} a low vowel in between, we have a consonant context that is the possible site of non-low vowel interpolation. We label such contexts ‘\textsc{C\_C}’, and analyse the predictability of non-low vowel interpolation in the 6943 instances of \textsc{C\_C} extracted from the wordlist. For example from the wordlist item \textit{jipuɻat̪a} ‘wallaby’ we extract the contexts j\_p and p\_ɻ. We also calculate the predictability of low vowels, given a representation in which non-low vowels are omitted, which confirms that low vowels are much less predictable (i.e., more informative) in this representation.

Technically, we could reverse our analysis and assume that non-low vowels are encoded in lexical representations, then calculate the predictability of low vowel interpolation in the remaining C\_C contexts. However this would be more difficult to justify, given the previous analyses proposing that non-low vowels are epenthesised word-internally, and low vowels word-finally. We return to this issue in Section \ref{sec:mansfield:4.2} below.

It would also be interesting to investigate the predictability of non-low vowel \textit{quality} in the wordlist data, i.e. the extent to which [i, ə, u] can be predicted by the place of articulation of \textsc{C\_C} contexts, as suggested in previous work. However, we are not able to do this because the qualities [i, ə] are not distinguished in the wordlist, which presents both [i] and [ə] using the character <i>. Conversely, we suspect that some entries may ‘over-distinguish’ vowel types, for example the r\_p context is attested with both <i> and <u> as in \textit{\nobreakdash-ripijena} ‘see dimly’ vs \textit{\nobreakdash-rupuriŋkina} ‘watch over’, but we suspect that such words may have a single vowel type that is phonetically intermediate between [i {\textasciitilde} u]. Analysing the predictability of non-low vowel quality will require quantitative analysis of vowel formants in production data, which is beyond the scope of this chapter, but the subject of preliminary analyses currently underway \citep{billington_acoustic_nodate}. 

\subsection{Predicting non-low vowels by consonantal context}\label{sec:mansfield:4.1}

Our analysis of the orthographic wordlist data shows that the presence or absence of a non-low vowel is to a large extent predictable from the neighbouring consonants. To some extent this is implied by earlier observations that Anindilyakwa permits only a limited range of consonant clusters (both within and across syllables), which suggests that for any other \textsc{C\_C} context, a non-low vowel is predictably interpolated. However we here add a further observation: not only are there certain \textsc{C\_C} contexts where a vowel is predictably present, but there are also \textsc{C\_C} contexts where a vowel is predictably \textit{absent}. That is to say, for a context like m\_p that can form a cluster, not only do we find many instances of clusters as in \textit{ŋampuwa} ‘where to?’, but we almost never find interpolated non-low vowels like *\textit{mup}. Thus both presence and absence of non-low vowels is generally predictable from \textsc{C\_C} context, and this greatly reduces their overall information value.

The simplest types of \textsc{C\_C} context are those in which non-low vowels generally do not occur. These are the most frequent cluster types noted in previous work, namely homorganic nasal-stop sequences [mp, nt, ɳʈ, ɲc, ŋk, ŋkʷ], and dorsal-labial sequences of equal or increasing sonority [kp, ŋm, ŋp]. For homorganic nasal-stop sequences, the wordlist attests just a handful of exceptional vowel interpolations such as <i> in the n\_t̪ context in \textit{ɛnit̪ira} ‘fins’. For dorsal-labial sequences there is also a handful of exceptions such as <u> in the ŋ\_p context in \textit{aŋupina} ‘cloud’. Some of these may be explained by borrowing, for example \textit{aŋupina} < Yolŋu \textit{waŋupini} ‘cloud’, while other apparent exceptions can be explained as a constraint against triple-consonant clusters *[ŋkp], for example \textit{ɛŋkuparŋʷarŋʷa} ‘heavy’.

The predictable absence of non-low vowels in contexts such as m\_p and k\_p is another reason why we treat these as clusters rather than complex segments (as in \citealt{vanEgmond2012};  \citealt{vanEgmondBaker2020}). By positing bisegmental sequences such as m\_p and k\_p, in a system of generalised predictability for non-low vowels, we can explain both the prevalence of clusters such as [mp] and [kp], and the scarcity of sequences such as *[mup] and *[kəp]. The scarcity of interpolated sequences would remain unexplained if we instead posit complex segments such as /mp/ and /kp/. If these were phonemic segments, existing alongside simple segments such as /m, k, p/, then we would need an additional explanation as to the scarcity of *[mup], *[kəp] \textit{etc.} in the lexicon.

Consonant contexts that predictably do have an interpolating non-low vowel require more detailed specification. There is a large set comprising any context with equal or increasing sonority (e.g. t̪ət̪, t̪ən, t̪əl̪, nən, nəl̪…), and these account for about half of all the \textsc{C\_C} contexts in the dataset (52\%, N=3628). There are just a handful of exceptions to the pattern of vowel interpolation in this context, some of which conform to Heath’s (n.d.: 1) observation of coronal stop clusters [ʈc, t̪c], e.g. \textit{\nobreakdash-kpə}\textit{t̪ciji} ‘be on an edge’. Others involve a retroflex nasal followed by a palatal glide, e.g. \textit{{}-ʎaŋkaɳjɛra}, ‘hard’.

There are also several types of decreasing-sonority \textsc{C\_C} that are predictably interpolated. These are listed in \tabref{tab:mansfield:2} (parenthetic labels like ‘\textsc{M\_T}’ will be used to refer back to these classes below). 

\begin{table}
\small
\begin{tabularx}{\textwidth}{Ql@{~~}l@{}}
\lsptoprule
{Description} & {Label} & {Example}\\
\midrule
Peripheral or palatal nasal followed by heterorganic stop & M\_T & \textit{mamut̪akpa} ‘tail’\\
\tablevspace
Palatal lateral followed by stop or nasal & ʎ\_T & \textit{{}-ŋkaʎikəna} ‘wet’\\
\tablevspace
\mbox{Anterior lateral followed by non-labial nasal} & l\_N & \textit{{}-l̪al̪əna} ‘sit’\\
\tablevspace
Trill followed by coronal nasal & r\_N & \textit{{}-kʷuʎarəna} ‘shine’\\
\tablevspace
Approximant followed by nasal (except ɻ\_m, see below) & Y\_N & \textit{awuɲampa} ‘anger’\\
\tablevspace
Approximant followed by liquid & Y\_L & \textit{{}-l̪awul̪awɛna} ‘be stretched out’\\
\tablevspace
\mbox{Non-retroflex approximant followed by stop} & Y\_T & \textit{wijit̪a} ‘straight’\\
\tablevspace
\mbox{Retroflex approximant followed by dorsal stop} & ɻ\_k & \textit{akʷaɻaɻəkaja} ‘vines’\\
\lspbottomrule
\end{tabularx}
\caption{Decreasing-sonority \textsc{C\_C} types with predictable vowel interpolation.}
\label{tab:mansfield:2}
\end{table}

Finally, there are the other types of decreasing-sonority \textsc{C\_C}, in which non-low vowels are unpredictably present or absent (listed with examples in \tabref{tab:mansfield:3}). Whereas the predictable \textsc{C\_C} types described above can be efficiently encoded without non-low vowel symbols, as in /ŋampw/ → [ŋampuwa] and /jpɻat̪/ → [jipuɻat̪a], the unpredictable types require non-low vowels to provide complete lexical representations, for example \textit{nəp} in /akʷa\-nɛnəpk/ ‘life’ vs \textit{np} /l̪amanpcn/ ‘be absorbed’.


\begin{figure}
    \includegraphics[height=.4\textheight]{figures/macro-transition-classes.png}
    \caption{Token counts in the wordlist of \textsc{C\_C} contexts grouped according to predictability of non-low vowel interpolation}
    \label{fig:mansfield:1}
\end{figure}

\begin{table}
\begin{tabularx}{\textwidth}{L{3.75cm}L{1.1cm}L{2.7cm}L{2.6cm}}
\lsptoprule
{Description} & {Label} & {CƏC example} & {CC example}\\
\midrule
Anterior nasal followed by non-heterorganic stop & N\_P & {\textit{akʷanɛnəpəka} ‘life’} & {\textit{{}-l̪amanpəcina} ‘be absorbed’}\\
\tablevspace
Anterior liquid followed by stop & L\_T & {\textit{{}-ɛrəpuɻan̪t̪ə} ‘not want’} & {\textit{{}-ɛrpal̪ici} ‘separate’}\\
\tablevspace
Anterior liquid followed by bilabial nasal & L\_m & {\textit{{}-ŋʷurməl̪ɛna} ‘grumble’} & {\textit{{}-ŋʷurumɛjina} ‘keep quiet’}\\
\tablevspace
Trill followed by dorsal nasal & r\_ŋ & {\textit{\nobreakdash-wɛrəŋɛkpuɻakəna} ‘comfort’} & {\textit{{}-alkarŋi} ‘cut grass’}\\
\tablevspace
Retroflex approx followed by a non-dorsal stop & ɻ\_T & {\textit{{}-ŋʷuciɻət̪əna} ‘become deep’} & {\textit{{}-aŋmaɻt̪ɛ} ‘hate’}\\
\tablevspace
Retroflex approx followed by bilabial nasal & ɻ\_m & {\textit{{}-apaɻumə} ‘search’} & {\textit{amaɻmara} ‘sore’}\\
\lspbottomrule
\end{tabularx}
\caption{Decreasing-sonority \textsc{C\_C} types with unpredictable vowel interpolation.}
\label{tab:mansfield:3}
\end{table}

\figref{fig:mansfield:1} shows the token counts of \textsc{C\_C} non-low vowel interpolation contexts in the wordlist, grouped according to whether they are predictably clustered ‘CC’, predictably interpolated ‘CəC’, or ‘Unpredictable’. The figure shows that the largest group is those that are predictably interpolated (N=4445, 64\% of the data), while smaller groups constitute the predictable clusters (N=1267, 18\%)  and unpredictable contexts (N=1231, 18\%).


While \figref{fig:mansfield:1} already suggests that non-low vowels are to a great extent predictable in the Anindilyakwa lexicon, we can provide an information-theoretic measurement of this predictability. As outlined above, surprisal (negative log probability) can be used as a measure of predictability for a segment in a given context. When we additionally consider the predictability of a choice between phonological possibilities, in this instance clustering versus non-low vowel interpolation, we use the \textit{weighted average surprisal} of the possible outcomes. This quantity is often referred to as “entropy”, and can be thought of as the information\hyp value of a paradigmatic choice, such as a phonological contrast.  \tabref{tab:mansfield:4} illustrates the CəC and \textsc{CC} counts for each context, as well as the entropy value in bits, $H(\text{ə}\sim\emptyset)$, quantifying information value of non-low vowel interpolation in that context.\footnote{The formula for this entropy, i.e. weighted average surprisal, is:
\[
H(\text{ə}\sim\emptyset) = (p(\text{ə}) \times -\log_2 p(\text{ə})) + (p(\emptyset) \times -\log_2 p(\emptyset)).
\]}
The table also gives the weighted average entropy for \textsc{C\_C} contexts in the entire lexicon, where each \textsc{C\_C} context is weighted by the probability of this context being encountered. This overall weighted average is 0.23 bits, which is a low figure given that complete predictability would be zero, and the maximum informativity would be 1. %The highly frequent \textsc{C\_C} contexts such as ‘Equal or increasing’ sonority therefore have a greater influence on this average than do infrequent contexts such as ‘N\_P’.%

\begin{table}[t]
\begin{tabular}{l rrr}
\lsptoprule
{Context type} & {CəC count} & {CC count} & $H(\text{ə}\sim\emptyset)$\\
\midrule
\textit{Predictable} \textit{interpolation} &  &  & \\
Equal or increasing & 3615 & 13 & 0.03\\
\quad sonority \\
l\_N & 55 & 0 & 0.00\\
{}ʎ\_T & 162 & 0 & 0.00\\
M\_T & 105 & 1 & 0.08\\
r\_N & 52 & 0 & 0.00\\
{}ɻ\_k & 60 & 2 & 0.21\\
Y\_L & 133 & 2 & 0.11\\
Y\_N & 177 & 0 & 0.00\\
Y\_T & 68 & 0 & 0.00\\
\midrule
\textit{Predictable clustering} &  &  & \\
Dorsal\_labial & 23 & 392 & 0.31\\
Homorganic nasal\_stop & 16 & 836 & 0.13\\
\midrule
\textit{Unpredictable clustering} &  &  & \\
L\_m & 61 & 36 & 0.95\\
L\_T & 309 & 364 & 0.99\\
N\_P & 52 & 25 & 0.91\\
r\_ŋ & 63 & 114 & 0.94\\
{}ɻ\_m & 39 & 17 & 0.89\\
{}ɻ\_T & 46 & 105 & 0.89\\
\midrule
\textsc{weighted average} &  &  & 0.23\\
\lspbottomrule
\end{tabular}
\caption{Predictability of non-low vowels in various \textsc{C\_C} types}
\label{tab:mansfield:4}
\end{table}

In fact there is some evidence suggesting that 0.23 bits is actually an over-estimate of the informativity of non-low vowels in Anindilyakwa. We will see below (Section \ref{sec:mansfield:5}) that non-low vowels in production data are variably omitted in decreasing-sonority contexts, e.g. for [u] in [mamut̪akpa {\textasciitilde} mamt̪akpa] ‘tail’. This suggests that some of the lexical unpredictability attested in the wordlist for decreasing-sonority contexts may actually reflect phonetic variation, rather than lexical contrasts.

\subsection{Predictability of low vowels}\label{sec:mansfield:4.2}

To put the information measurement of non-low vowels in perspective, we calculate the amount of information carried by low vowels in the proposed underlying representations. Given that non-low vowels have been shown to be largely predictable, we assume representations in which these are absent from underlying representations, e.g. /jpɻat ̪/ ‘wallaby’. This leaves every C\_C sequence as a context for potential interpolation of a low vowel, e.g. j\_p, p\_ɻ, ɻ\_t̪. To maximise comparability with the non-low vowels, we set aside the difference between low vowel qualities [a, ɛ], and ask only whether the presence/absence of a low vowel is predictable in various C\_C types.

Using the same C\_C context types that were identified above, we find that low vowels are much less predictable than the non-low vowels, that is to say they carry much more information. As shown in \tabref{tab:mansfield:5}, the presence vs absence of a low vowel is relatively unpredictable in almost every C\_C context type, including the most frequent contexts such as ‘Equal or decreasing sonority’. The only contexts that have relatively low entropy are dorsal-labial and homorganic nasal-stop sequences, which again show tendencies to form clusters rather than undergo vowel interpolation, though even in these contexts we find greater entropy for low vowels than non-low vowels.

\begin{table}[t]
\begin{tabular}{lrrr}
\lsptoprule
{Context type} & {CaC count} & {CC count} & $H(\text{ə}\sim\emptyset)$\\
\midrule
Equal or increasing & 2495 & 3628 & 0.97 \\ 
\quad sonority \\
l\_N & 95 & 55 & 0.95\\
{}ʎ\_T & 147 & 162 & 0.99\\
M\_T & 159 & 106 & 0.97\\
r\_N & 34 & 52 & 0.97\\
{}ɻ\_k & 42 & 62 & 0.97\\
Y\_L & 275 & 135 & 0.91\\
Y\_N & 224 & 177 & 0.99\\
Y\_T & 197 & 68 & 0.82\\
Dorsal\_labial & 69 & 415 & 0.59\\
Homorganic nasal\_stop & 60 & 852 & 0.35\\
L\_m & 34 & 97 & 0.83\\
L\_T & 152 & 673 & 0.69\\
N\_P & 38 & 77 & 0.92\\
r\_ŋ & 28 & 177 & 0.58\\
{}ɻ\_m & 18 & 56 & 0.80\\
{}ɻ\_T & 30 & 151 & 0.65\\
\midrule
\textsc{weighted average} &  &  & 0.86\\
\lspbottomrule
\end{tabular}
\caption{Predictability of low vowels in various \textsc{C\_C} types}
\label{tab:mansfield:5}
\end{table}

As noted above, our treatment of non-low and low vowels could technically be reversed. We could assume only non-low vowels to be present in underlying representations (e.g. /jipuɻt̪/ → [jipuɻat̪a]/), then measure the predictability of word-internal low-vowel interpolation. As noted by a reviewer, these calculations would in fact show low vowels to be substantially predictable, almost as much so as non-low vowels in our model. What this ultimately shows is that for most C\_C contexts in Anindilyakwa, it is highly predictable whether a vowel will be interpolated, or not. For example given the consonant sequence /t̪ŋn̪t̪ŋ/ `sharp', it is highly predictable that word-internal vowels will be interpolated as in [t̪VŋVn̪t̪Vŋ]. The dimension of unpredictability is largely a binary choice as to whether each of these vowels is either low or non-low. This means that a redundancy-free encoding only needs to represent one vowel type or the other, since whichever type is not represented can be predictably interpolated according to C\_C context. For example by specifying the low vowels, we can predictably insert non-low vowels in the remaining contexts that require a vowel, as in /t̪ŋan̪t̪ŋ/ → [t̪əŋan̪t̪əŋ]. The analysis pursued here selects low vowels for underlying representation, rather than non-low vowels, in keeping with the earlier proposal by \citet{Heath:2020aa}. But besides evaluating Heath's analysis, there are two reasons to model word-internal epenthesis of the non-low vowels, rather low vowels. Firstly, at word-internal morphological boundaries it is non-low vowels that are epenthesised, rather than low vowels. Secondly, it is non-low vowels that are cross-linguistically attested as word-internal epenthetic segments, rather than low vowels.

\newpage
We have thus far shown that the presence/absence of non-low vowels in the Anindilyakwa lexicon is to a great extent predictable, based on wordlist data. Not only are there certain C\_C contexts in which non-low vowels are predictably interpolated (as observed in earlier works), but there are also C\_C contexts in which non-low vowels are predictably absent. We used an IT measurement to quantify the information value of non-low vowels, and compare this to the much greater information value of low vowels in the resulting lexical representations. In the following sections we will investigate how these findings on vowel predictability align with patterns of vowel omission in speech production data.

\section{Variable omission of predictable vowels in production data}\label{sec:mansfield:5}

In this section we turn to speech production data, focusing in particular on the extent to which vowels are omitted. As noted above, previous research suggests that highly predictable segments should be articulated with reduced phonetic cues, including complete omission, in comparison to more informative segments (\citealt{Lindblom1990}; \citealt[119]{Hall2009,HallEtAl2018}). We here present evidence that this is indeed the case with respect to Anindilyakwa vowels.

Since predictable segments are expected to undergo phonetic reduction, up to and including omission, one might also investigate whether predictable vowels in Anindilyakwa exhibit more gradient reduction in comparison to unpredictable vowels. However, this approach is confounded by other phonetic factors. Epenthetic final \textit{{}-a} vowels in Anindilyakwa, although they are completely predictable, tend to be phonetically long when they are present. However, this observation is difficult to disentangle from the fact that final lengthening is a widespread, if not universal, property of speech (\citealt{fletcher2010prosody};  \citealt{seifart2021extent}). As for the predictable non-low vowels, these \textit{do} exhibit shorter durations in comparison to the unpredictable word-internal low vowels. In preliminary analyses, the average durations of word-internal low vowels are around 100ms, while non-low vowels are around 50--60ms \citep{billington_acoustic_nodate}. However in this case, the shorter duration of non-low vowels cannot easily be disentangled from the fact that low vowels are cross-linguistically longer than non-low vowels \citep{lindblom_vowel_1967}. In addition, there are likely to be influences of word-level prominence patterns on vowel realisation, but these are difficult to take into account based on current knowledge of Anindilyakwa prosody. To avoid these confounds, the analysis below focuses on the complete omission of vowels, rather than gradient phonetic reduction.

\subsection{Data and method}\label{sec:mansfield:5.1}
\largerpage
% Note I have changed the intro for this para due to some Latex issues - HS
All data for this study were collected using field elicitation of Anindilyakwa utterances, comprising a combination of picture prompts and spoken English prompts. Sets of around 50–100 utterances were collected in this way from seven Anindilyakwa speakers, and audio recorded. The same prompts were used for each speaker, so that their data is largely comparable and contains repeats of many of the same lexical items; however speakers were not coerced into using specific sentence frames or lexical items, so there is also substantial diversity in how they chose to translate the prompts.

The speakers’ ages range from approximately 25 to 80. Five are women and two are men. They all speak Anindilyakwa as their main daily language, and all are multilingual in Kriol, English and other regional languages (especially Wubuy and Yolngu Matha). Sentences were transcribed by the first author, then converted into an \textsc{EMU-SDMS} hierarchical database \citep{Winkelmann2017} by the third author, following automated phone segmentation via WebMAUS \citep{kisler2017multilingual} using the Australian Aboriginal Language model and by manual correction by all three authors. The resulting database contains a total of 493 sentences and 5668 vowel tokens. Non-low vowel tokens (N=1918) were labelled [i, ə, u] according to perceived vowel quality, but as in the preceding section they are treated as a single class for the purposes of the present analysis focusing on whether they are produced or not.

\subsection{Omission of final \textit{{}-a}} \label{sec:mansfield:5.2}

Beginning with the category of low vowels [a, ɛ] (N=3674), there is evidence for an association between predictability and omission. As we saw above, the presence of word-internal low vowels is lexically informative, and must therefore be maintained in representations such as /jpɻat̪/ ‘wallaby’. We should therefore expect these vowels to retain robust cues, and be reliably present, and this is supported by the production data. Our data includes 2071 instances where word-internal low vowels are expected to occur, and we did not identify any instances of vowel omission in these contexts.

The opposite situation should occur with word-final \textit{{}-a}, which as we saw is completely predictable in the orthographic wordlist data. Indeed, in the speech production data, we find that final \textit{{}-a} is quite often omitted (supporting informal observations by \citealt[139]{Leeding1989}). It is consistently present in sentence{}-final words (N=493),\footnote{We have observed just a handful of exceptions, for example (RaLa\_nest\_MED\_01; RaLa\_river\_FIN\_01).} but it is only present in 67\% of non-final words (total N=1110). Leeding associates the omission of final \textit{{}-a} with an initial low vowel in the following word. While we do find many such examples in our data, as for \textit{kuʈanəm} followed by [ɛ] in \REF{ex:mansfield:2}, we also find many examples of \textit{{}-a} omission where the following word is consonant-initial, as for \textit{niŋɛn} followed by [k] in the same example. (Example \REF{ex:mansfield:2} also exhibits word-internal non-low vowel omissions; to be discussed below). 

\ea\label{ex:mansfield:2}
\glll {niŋɛn} {kuʈanəm} {ɛnmaɲc} {ɛjka} \\
       niŋɛn k-wʈa-n-m ɛn-maɲc ɛjk\\
       \textsc{1s} \textsc{irr}{}-climb-\textsc{npst-mut} \textsc{prox.neut-loc} tree\\
\glt   ‘I’m going to climb this tree.’ (JoMa\_tree\_FIN\_01)\z


In our production data \textit{{}-a} omission is especially frequent at the boundary between a demonstrative and the following noun (as observed by \citealt{Heath:2020aa}), for example \textit{t̪ak} in \REF{ex:mansfield:3}. This suggests that its presence may be disfavoured within some constituent types such as NPs, which are likely to form a phonological phrase. 

\ea\label{ex:mansfield:3}
\glll [{t̪ak} {t̪urkʷarəkʷa}]\textsubscript{NP} {təŋan̪t̪əŋa}\\
       {\db}t̪ak trkʷarkʷ t̪ŋan̪t̪ŋ\\
       \textsc{dist.fem} spear.grass sharp\\
\glt   ‘That spear-grass is sharp.’ (RaLa\_speargrass2\_MED\_01)\z

The fact that final \textit{{}-a} is most consistently present utterance-finally, and is frequently absent within NPs, which appear to form phonological phrases, suggests that it may have a prosodic boundary-marking function. Its presence likely also interacts with pause phenomena. Further research would be required to disentangle how final \textit{{}-a} interacts with prosodic or syntactic phrasing, since at present there is no description of syntax-prosody mapping in Anindilyakwa. However, the crucial point for our study is that final \textit{{}-a} is freely omissable because it does not carry any lexical information – its main functions appear to relate to boundary marking, and it makes no contribution to distinguishing lexemes.


\subsection{Omission of non-low vowels}\label{sec:mansfield:5.3}

Finally, we turn to non-low vowels [i, ə, u] in our speech production data, which as we saw above are largely predictable in the wordlist data, and therefore can be expected to be more subject to omission compared to word-internal low vowels. This is indeed the case in the production data. We focus on C\_C types that are identified from the wordlist as either predictably interpolated, or as lexically unpredictable (Section \ref{sec:mansfield:4.1}). Our data contains 2582 tokens of these C\_C types, and in 28\% of these the non-low vowel is omitted.\footnote{Out of these 723 instances of non-low vowel omission, approximately 50 instances show marginal evidence of vocalic-like acoustic material (< 10ms in duration) intervening between consonant segments. We annotated these as vowel omission, since they are auditorily difficult to distinguish from transitional phenomena in clusters. This does not materially affect our claims, since the marginal tokens account for less than 10\% of non-low vowel omissions, and in any case are quite consistent with our hypothesis regarding predictability and phonetic reduction.} We find that non-low vowels are often omitted both morpheme-internally (where the wordlist suggests that they are largely predictable), and at morpheme boundaries (where all sources concur that they are totally predictable).

As we saw above, C\_Cs with equal or increasing sonority are attested with almost exceptionless non-low vowel interpolation in the wordlist data. But in the production data we find that vowels are variably omitted in these contexts, including in lexemes that the wordlist attests with vowel interpolation. Examples of vowel omission in increasing-sonority contexts can be seen for [pm] in \REF{ex:mansfield:4} and [kʷm] in \REF{ex:mansfield:5}. 
%Since these clusters occur word-internally we may assume that they are heterosyllabic, though nonetheless of a cross-linguistically dispreferred type (\citealt{MurrayVenneman1983,Seo2011}),

\ea\label{ex:mansfield:4}
\glll {jikarpma} \\
       jkarp-m\\
       woomera-\textsc{instr}\\
\glt   ‘with a woomera’ (EdMa, woomera\_FIN\_01)
\ex\label{ex:mansfield:5}
\glll {jɛʎukʷmaɲca} \\
     jɛʎkʷ-maɲc\\
     rain-\textsc{loc}\\
\glt   ‘in the rain’ (JuLa, rain)
\z

We also find examples of increasing-sonority clusters that involve a labial stop or nasal followed by either ɻ or l̪, as in [pɻ] in \REF{ex:mansfield:6}, [pl̪] in \REF{ex:mansfield:7}, and [ml̪] in \REF{ex:mansfield:8}.

\ea\label{ex:mansfield:6}
\gll {jipɻat̪a}\\
       jpɻat̪\\
\glt   ‘wallaby’ (JoMa\_kangaroo\_MED\_01)
\ex\label{ex:mansfield:7}
\glll {mɛmɛrpa} {məl̪arkpl̪al̪a}\\
       mɛmɛrp m-l̪arkpl̪al̪\\
       calf fem-thin\\
\glt   ‘thin calves’ (EdMa\_lowerleg2\_MED)
\ex\label{ex:mansfield:8}
\glll {at̪aʎəml̪aŋa}\\
       at̪aʎm-l̪aŋʷ\\
       river-\textsc{gen}\\
\glt   ‘across the river’ (KaMa, river)
\z

There are also many examples of nasal-nasal, nasal-lateral or lateral-lateral clusters, as in \REF{ex:mansfield:9}–\REF{ex:mansfield:12}. In \REF{ex:mansfield:12} we also observe that the palatal glide attested in the wordlist form \textit{micijɛʎa} ‘beach’, is omissible in speech.

\ea\label{ex:mansfield:9}
\gll {jinmamuwa}\\
       jnmamw\\
\glt   ‘egg’ (CoMa\_egg\_FIN\_01)
\ex\label{ex:mansfield:10}
\glll {niŋɛn} {ŋmarəŋnam}\\
       niŋɛn ŋ-ma-rŋk-na-m\\
       \textsc{1s} \textsc{1s-fem}{}-see-\textsc{npst-mut}\\
\glt   ‘I see it (\textsc{fem}).’ (JoMa\_road\_FIN\_01)
\ex\label{ex:mansfield:11}
\glll {ɛnl̪aŋʷa}\\
       ɛn-l̪aŋʷ\\
       \textsc{prox.neut-gen}\\
\glt   ‘for this (\textsc{neut})’ (CoMa\_shoulder\_MED\_01)
\ex\label{ex:mansfield:12}
 \glll {micɛʎʎaŋʷuja} \\
       mcjɛʎ-l̪aŋʷ-wj\\
       beach-\textsc{gen-com}\\
\glt   ‘along the beach’ (CoMa\_beach\_FIN\_01)\z


We also find widespread non-low vowel omission in C\_Cs where the second consonant is a glide, as in [rj] in \REF{ex:mansfield:13} and [jw] in \REF{ex:mansfield:14}. In some of these cases it can be difficult to define whether a vowel is phonetically present or not, given the acoustic similarities between vowels and glides.\footnote{In tokens of this type, we annotate vowels based on either intensity or formant changes, either of which suggests a vocalic interval that can be distinguished from the glide.}

\ea\label{ex:mansfield:13}
\gll {mijɛrja}\\
       mjɛrj\\
\glt   ‘nest’ (KaMa\_nest\_FIN\_01)
\ex\label{ex:mansfield:14}
\gll {ŋajwa}\\
       ŋajw\\
\glt   ‘I…’ (JuLa\_water\_FIN\_01)\z

Overall, these data present an interesting situation where Anindilyakwa can be said to have very strict cluster constraints, based on citation forms in the wordlist. But in natural speech samples, even in a relatively careful register used in elicitation, Anindilyakwa is much more permissive of clusters.

We turn now to decreasing-sonority \textsc{C\_C}s, which in the written wordlist data comprise a complex mixture of types that are predictably interpolated, and unpredictable types that have lexemes listed both with and without vowels. Our speech production data is not extensive enough to investigate these subtypes in detail, but overall we find that decreasing-sonority \textsc{C\_C}s often appear as clusters. For example, we find clustering in non-homorganic nasal-stop sequences such as [mkʷ] in \REF{ex:mansfield:15} and [mt̪] in \REF{ex:mansfield:16}.

\ea\label{ex:mansfield:15}
\glll {nuŋumkʷul̪ama}\\
       nŋ-mkʷl̪a-m\\
       1s-stay.\textsc{npst}{}-\textsc{mut}\\
\glt   ‘I’m staying’ (JuLa\_shelter2\_FIN\_01)
\ex\label{ex:mansfield:16}
\gll {mamt̪akpa}\\
       mamt̪akp\\
\glt   ‘tail’ (KaBa\_tail\_FIN\_01)\z

We also find clustering of liquids followed by stops \REF{ex:mansfield:17} or nasals \REF{ex:mansfield:18}, though these again present some examples where it is difficult to define what counts as an intervening vowel (cf. \citealt[25ff.]{vanEgmond2012}):

\ea\label{ex:mansfield:17}
\gll {jɛʎkʷa}\\
       jɛʎkʷ\\
\glt   ‘rain’ (JoMa\_rain\_MED\_01)
\ex\label{ex:mansfield:18}
 \glll {al̪ŋacira}\\
       a-l̪ŋacr\\
       \textsc{neut}{}-tall\\
\glt   ‘tall (\textsc{neut})’ (CoMa\_tree\_MED\_01)\z

In our analysis of the wordlist we described how specific decreasing-sonority C\_C types have lexically unpredictable non-low vowels. These constituted 18\% of all C\_C tokens in the lexicon and contributed most of the 0.23 bits of weighted average entropy. But the production data shows that there is variable omission of non-low vowels in these contexts, including evidence of variable production of the same lexeme. We have also noted that in some decreasing-sonority C\_C types involving liquids and approximants (e.g. r\_ŋ, l\_m, ɻ\_m, r\_j), it is difficult to define whether a brief intervening vowel is present or not. We note that these types are also strongly represented in the purportedly unpredictable C\_C contexts in the lexicon (see Table 3). This suggests that the apparent lexical unpredictability of non-low vowels in these contexts could instead reflect phonetic variation. For example if a vowel is variably present or absent in lexemes such as /mɛrŋʷ/ → [mɛr(u)ŋʷa] ‘yellow clay’, and /aml̪rŋʷ/ → [amul̪ur(u)ŋʷa] ‘heel’, this might result in apparent lexical distinctions in the wordlist, which attests r\_ŋʷ vowel interpolation in \textit{mɛruŋʷa}, but clustering in \textit{amul̪urŋʷa}. Thus there is reason to suspect that even the limited informativity of non-low vowels represented in the wordlist may be an over-estimate, though more extensive phonetic research is required to test this.

\section{Discussion and conclusions}\label{sec:mansfield:6}


In this chapter we have investigated vowel predictability in Anindilyakwa, analysing types of consonant contexts in which non-low vowels are predictably present, predictably absent, or lexically specified as either present or absent. We used an IT approach to conceptualise and quantify the information content of these vowels, and to compare this to the word-internal low vowels, which are much more informative.

We then examined evidence for the proposed association between low information content and reduced phonetic realisation, focusing on complete omission of vowels. Our speech production data provides support for this proposal, since we observed that the low-information vowel types, namely non-low vowels, and word-final \textit{{}-a} vowels, are quite frequently omitted even in a relatively careful register of elicited speech. This is in contrast to the highly informative, word-internal low vowels, which are never omitted in our production data.

Our findings support a theory of predictable vowels in terms of informational vacuity and phonetic facilitation. On the one hand, predictable vowels do not in themselves contribute to the phonological contrasts that distinguish an intended word from all the others in the lexicon. On the other hand, highly predictable vowels may nonetheless play a role in speech production and perception, as has been argued for epenthetic vowels in previous work (\citealt{cote2000consonant,tily_rational_2012}). The non-low vowels of Anindilyakwa may be largely uninformative in their own right, but they nonetheless may play a role in the accurate transmission of their information-rich neighbouring consonants, as has been proposed for vowels in Australian languages more generally \citep{butcher_placearticulation_2006}. On the level of phonological categories and contrasts, non-low vowels in Anindilyakwa are uninformative because they rarely contribute to distinguishing one lexeme from another. But on the level of phonetic articulation, they facilitate the transmission of cues to consonant place of articulation. In IT terms, this is an issue of channel capacity, where a certain amount of redundancy in signal encoding (such as the interpolation of uninformative vowels) can help to maintain transmission fidelity. It has been noted in previous work that Anindilyakwa words seem unusually long in general \citep[68]{Leeding1989}, and this is exactly what we might expect in an encoding system that includes very low-information segments \citep{Nettle1995, Nettle1998}.

As for the word-final epenthesis of \textit{{}-a}, previous analyses already showed that this is completely non-contrastive because it can occur on all words, and therefore is completely uninformative with respect to the lexicon. But final \textit{{}-a} likely has other functions. For one thing, it provides a sonorous substrate for intonational boundary tones, expressing pragmatic intent (a different kind of `information'). In some contexts, lengthening of clause-final vowels is suggested to carry aspectual information \citep{bednall2019}. In our observations of low-vowel omission we noted that it is almost never omitted utterance-finally, and is omitted most frequently within noun phrases. This suggests the possibility that final \textit{{}-a} may also help signal phrasal structure, though further research would be required to disentangle syntactic and prosodic phrases. 

\subsection{Predictability in gradient lexical representations}\label{sec:mansfield:6.1}

As pointed out near the beginning of this chapter, one advantage of IT is that it can capture gradient as well as categorical effects. Yet throughout this chapter we have associated the predictability of non-low vowels with their categorical omission from redundancy-free lexical representations as in /jpɻat̪/ → [jipuɻat̪a]. Other gradient approaches to phonology, such as exemplar theory (\citealt{bybee2000phonology,pierrehumbert2001exemplar}), instead propose phonetically detailed, gradient lexical representations. In these approaches, each word has its own phonetically detailed representation, rather than being merely a conjunction of abstract phonological segments or features. This explains why phonetic variation shows lexically specific effects (e.g. \citealt{jurafsky2002role,walker2012form}). Furthermore, the memory traces of specific phonetic tokens are said to be stored as `exemplars', so that a lexical representation is a collective of all the exemplars experienced for that word. In this approach to phonology, phonemes and features are not the basic building blocks, but instead are epiphenomena that reflect the convergence of lexical representations on similar gestures and timing patterns \citep[72]{bybee2000phonology}.

While the focus of this study has been discrete phonological representations, we believe that both our conceptual framework and our findings are also compatible with gradient lexical representations. For example, if we assume phonetically gradient representations along the lines of the `gestural scores' used in articulatory phonology \citep{browman1992articulatory}, the IT approach would imply that certain gestures are more informative than others. In a word like [mamut̪akpa] `tail', the vowel [u] would be present in the lexical representation as a (weak) vocalic gesture; but crucially, this gesture would be represented as having low surprisal with respect to the rest of the lexicon. And if lexical representations encompass probabilistic variation over phonetically distinct exemplars, then this would be reflected in a distribution of exemplar tokens encompassing both presence and absence of the weak vocalic gesture, i.e. [mamut̪akpa {\textasciitilde} mamt̪akpa], as observed in our production data. An IT-enriched exemplar theory would make the connection between this phonetic variation and the low informativity of the vocalic gesture (e.g. \citealt{vanSonvanSanten2005,Hall2009,CohenPriva2015,CohenPriva2017,ShawKawahara2017,HallEtAl2018}). It is beyond the scope of this discussion to flesh out how exactly gestural scores and exemplar distributions might be enriched by representations of gestural surprisal, but the approaches do appear to be fundamentally compatible.

In this chapter we have discussed redundancy-free representations such as /mamt̪akp/,
in which non-low vowels are treated as epenthetic. However, if we instead assume that lexical representations are phonetically gradient exemplar distributions, then we do not need to make an analytic choice between deletion and epenthesis. Instead, both vowel presence and absence are possibilities within a continuous space of gestural timing and magnitude. Furthermore, this approach supports the analysis of vowel reduction as a dynamic process, and one that may lead to changes over time \citep{Wedel:2013, HallEtAl2018}.

\subsection{Directions for further research}\label{sec:mansfield:6.2}
The current study leaves open many questions. One of the most important is the matter of vowel quality in Anindilyakwa non-low vowels. In this study we have focused solely on how consonants condition vowel presence/absence, while setting aside the issue of how the quality of non-low vowel phones is conditioned by consonantal place-of-articulation. Investigation of this issue may further demonstrate the contextual predictability of these vowels, or it may reveal dimensions of lexical specificity that were not accessible in the current study. Detailed acoustic analyses would also further inform our understandings of gradient patterns in Anindilyakwa vowel production, as would analyses that are able to take into account the frequency of phones and lexemes based on natural speech data. Another issue that requires further research is the role of prosodic prominence, both in terms of clarifying prosodic patterns in Anindilyakwa, and examining how these interact with segmental phenomena.

Finally, the interpretation of our results will be greatly enhanced by comparative research on other languages. We have measured degrees of predictability in Anindilyakwa non-low vowels, comparing this to the low vowels. This has been partly motivated by debates about phonemic status in the Anindilyakwa vowel system, and the language’s unusual structural patterns relative to the areal typology. But would equivalent measurements for other languages show that Anindilyakwa is in fact unusual in terms of vowel predictability? Do non-low vowels typically exhibit more contextual predictability than their louder and longer counterparts in the low vowel space? A good place to begin such research would be to compare data from other Australian languages, many of which have roughly similar segmental inventories to Anindilyakwa, but in which there has been no comparable debate about which vowels are lexical or epenthetic.


\section*{Acknowledgments}

We wish to thank all the Anindilyakwa speakers who taught us about their language, and especially the seven who spoke sentences for this study: Ramesh Lalara, Kathleen Mamarika, Katelynn Bara, Edith Mamarika, Judy Lalara, Coleen Mamarika and Joel Marawili. We also thank staff at the Groote Eylandt Language Centre \-– James Bednall, Brighde Collins and Carolyn Fletcher – who provided generous support and hospitality to the first author in his field visits. Kathleen Currie Hall and two anonymous reviewers gave us valuable comments on an earlier draft. We gratefully acknowledge funding support from the University of Melbourne Research Unit for Indigenous Language, and the Australian Research Council grant DE180100872.

\printbibliography[heading=subbibliography,notkeyword=this]

\end{document}
