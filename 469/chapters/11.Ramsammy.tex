\documentclass[output=paper,colorlinks,citecolor=brown]{langscibook}
\ChapterDOI{10.5281/zenodo.14264548}

\title{On the diachrony of lateral epenthesis} 
\author{Michael Ramsammy\orcid{}\affiliation{University of Edinburgh}}

\abstract{This paper discusses the phenomenon of lateral epenthesis from a diachronic perspective.  Within the theoretical context of the life cycle of phonological processes, I briefly discuss two cases that are frequently mentioned in general studies on epenthesis, namely English /l/-intrusion and /l/-epenthesis in Motu. Shifting the focus to Indo-Aryan, /l/-insertion in Hindi causative verbs is then discussed in detail.  After tracing diachronic developments through Old and Middle Indo-Aryan, I argue that Hindi /l/-causatives are not the product of a historical pattern of allophony in which [p] surfaces after vowel-final bases in causative forms.  Similarly, I show that evidence for an analogical development of /l/-causatives, as has been assumed within the philological tradition, is relatively weak.  Consequently, I propose an alternative account under which Hindi /l/-causatives emerge due to sonority-driven optimisation of historical /j/-epenthesis.}


\IfFileExists{../localcommands.tex}{
   \addbibresource{../localbibliography.bib}
   \usepackage{langsci-optional}
\usepackage{langsci-gb4e}
\usepackage{langsci-lgr}

\usepackage{listings}
\lstset{basicstyle=\ttfamily,tabsize=2,breaklines=true}

%added by author
% \usepackage{tipa}
\usepackage{multirow}
\graphicspath{{figures/}}
\usepackage{langsci-branding}

   
\newcommand{\sent}{\enumsentence}
\newcommand{\sents}{\eenumsentence}
\let\citeasnoun\citet

\renewcommand{\lsCoverTitleFont}[1]{\sffamily\addfontfeatures{Scale=MatchUppercase}\fontsize{44pt}{16mm}\selectfont #1}
  
   %% hyphenation points for line breaks
%% Normally, automatic hyphenation in LaTeX is very good
%% If a word is mis-hyphenated, add it to this file
%%
%% add information to TeX file before \begin{document} with:
%% %% hyphenation points for line breaks
%% Normally, automatic hyphenation in LaTeX is very good
%% If a word is mis-hyphenated, add it to this file
%%
%% add information to TeX file before \begin{document} with:
%% %% hyphenation points for line breaks
%% Normally, automatic hyphenation in LaTeX is very good
%% If a word is mis-hyphenated, add it to this file
%%
%% add information to TeX file before \begin{document} with:
%% \include{localhyphenation}
\hyphenation{
affri-ca-te
affri-ca-tes
an-no-tated
com-ple-ments
com-po-si-tio-na-li-ty
non-com-po-si-tio-na-li-ty
Gon-zá-lez
out-side
Ri-chárd
se-man-tics
STREU-SLE
Tie-de-mann
}
\hyphenation{
affri-ca-te
affri-ca-tes
an-no-tated
com-ple-ments
com-po-si-tio-na-li-ty
non-com-po-si-tio-na-li-ty
Gon-zá-lez
out-side
Ri-chárd
se-man-tics
STREU-SLE
Tie-de-mann
}
\hyphenation{
affri-ca-te
affri-ca-tes
an-no-tated
com-ple-ments
com-po-si-tio-na-li-ty
non-com-po-si-tio-na-li-ty
Gon-zá-lez
out-side
Ri-chárd
se-man-tics
STREU-SLE
Tie-de-mann
}
   \boolfalse{bookcompile}
   \togglepaper[11]%%chapternumber
}{}

\begin{document}
\maketitle \label{ch11}

\renewcommand{\rowletter}{\stepcounter{tableaurow}\roman{tableaurow}.}
\section{Introduction} \label{intro}
Theoretical accounts of epenthetic patterns involving consonants aim to respond to two main questions. The first is that of what constrains epenthesis: i.e. what environmental and structural factors trigger the insertion of consonantal material, and how is this determined on a language-specific basis. The second question is what consonants can be epenthetic: i.e. what specific qualities or feature values favour the occurrence of particular consonants in epenthesis environments. Research around this second theme has revealed typologically common patterns, and also some that are much less well attested.

This paper discusses a relatively rare phonological phenomenon, namely lateral epenthesis.  Being rare, cases of lateral epenthesis are typically mentioned in more general theoretical discussions of epenthesis, which may be either language\hyp specific or cross-linguistic in nature.  In this paper, the focus is given to the diachronic emergence of epenthetic laterals.  I first discuss some examples that are better known in the epenthesis literature, specifically cases in English and Oceanic languages (Section \ref{intro}).  Thereafter, the core of the paper is dedicated to Indo-Aryan.  In Section \ref{hindi}, I present data from Hindi causative verbs in which /l/ occurs preceding the first causative suffix, /-ɑ/, after vowel-final bases: e.g. /so-/ `sleep' vs [sʊlɑ] `put to sleep'.  As a hiatus-resolution strategy, /l/-insertion presents explanatory challenges in both synchronic and diachronic dimensions.

Section \ref{histo} of the paper traces the morpho-phonological history of causative formations in Indo-Aryan and discusses proposals that have previously been put forward to explain the occurrence of Hindi /l/-causatives.  These rely strongly on the assumption of analogical processes of change.  Whilst these claims are philologically well-founded, it is also generally acknowledged that the emergence of lateral epenthesis in causative verbs remains somewhat mysterious: as \citet[141]{Bloch1965} puts it, ``the real history of these suffixes is hidden from us".  

In view of this, Section \ref{altern} presents a reconstruction of a pathway of change leading to the development of lateral epenthesis over time.  The analysis is situated theoretically within the life cycle of phonological processes \citep{Bermúdez-Otero2015, Bermúdez-OteroTrousdale2012, Iosad2020, Ramsammy2015, Roberts2012, Sen2016, Turton2016, Turton2017}. Thus, the overarching aim is to construct a plausible scenario through which the development of lateral epenthesis in Indo-Aryan can be explained on the basis of core phonological principles and without exclusive reliance on analogy.

\subsection{Cases of lateral epenthesis}
\subsubsection{English}\label{english}
\citet{gickphono} describes the behaviour of intrusive /l/ in American English in the context of /ɔ/ and /a/ (see also \citealt{Gick2002}).  In varieties such as those spoken by working-class communities around Philadelphia, /l/-intrusion operates in a similar manner to intrusive /r/, which has been more widely discussed in the phonological literature (e.g. \cite{Barras2011, Hall2013, Mompeán-GonzálezMompeán-Gillamón2009, Sóskuthy2013, Uffman2007b}, inter alia).  Gick highlights the relationship between /l/-vocalisation and /l/-intrusion.\footnote{Relatedly, \citet{JohnsonBritain2007} highlight the relationship between the development of vocalised variants of /l/ in British dialects and pre-existing clear$\sim$dark lateral allophony. There are, however, cases of /l/-vocalisation in other languages that do not depend on an allophonic pattern of this type: e.g. Cibae\~{n}o Spanish \citep{Alba1979}.} Where /l/ undergoes vocalisation in coda position, e.g. [drɔː] \textit{drawl}, but is retained intervocalically, e.g. [drɔːlɪŋ] \textit{drawling}, intrusive /l/s emerge subsequently in non-etymological contexts: e.g. [drɔːlɪŋ] \textit{drawing}, [braːlɪz] \textit{bra is}.\footnote{Some speakers do not epenthesise /l/ in examples like \textit{bra is} (cf. \citealt[741]{RBO&Börjars2006}). As \citet[178]{Gick2002} notes, this pattern is observed for speakers who merge /ɔ/ and /ɑ/, hence [dɹɑːlɪŋ]. Further to this, \citet{Weissmann1970} describes a similar pattern of /l/-linking in Bristol English, e.g. [æfrɪkələneɪʃə] \textit{Africa and Asia}, which is an example of a broader /l/-\textit{epithesis} pattern \citep{gickphono} that causes homophony between pairs such as \textit{idea$\sim$ideal}, \textit{Eva$\sim$evil}, etc. \citep{Trudgill1999}.}

\begin{figure}
\includegraphics[width=\textwidth]{figures/life_cycle}
\label{fig_1}
\caption{The life cycle of phonological processes (adapted from \citealt[38]{Ramsammy2015})}
\end{figure}

/l/-intrusion of this type represents a classic case of rule inversion \citep{venneinversion, RBO&Börjars2006} that is also consistent with the core assumptions of the life cycle of phonological processes.  The life cycle is illustrated in Figure 1: the initial stages are of particular relevance here.  In brief, a fundamental claim of the life cycle is that all language-internal phonological innovations begin as low-level phonetic phenomena that are below the level of speaker awareness.  Over time, this can change: a gradient phonetic effect can transcend the level of speaker awareness, and thus, come under the speaker's cognitive control. This is the process of \textit{phonologisation}.  Following phonologisation, the innovation is still gradient in nature.  In a further phase of development known as \textit{stabilisation}, it can evolve into a discrete operation, governed by e.g. featural or representational changes. The latter phases of the life cycle involve changes to the domain of application of the phonological innovation, which gradually shrink over time through \textit{domain narrowing}.  The final phase is \textit{lexicalisation}, which involves restructuring of underlying representations, or alternatively \textit{morphologisation}, which involves the conversion of a phonological operation into a morphological one.

In this connection, the development of /l/-epenthesis is illustrated in \tabref{eng_latep}.  Before the onset of change, /l/ occurs freely both in word-medial and word-final positions.  At Stage 1, /l/ in word-final position begins to weaken: Gick notes that this involves partial vocalisation through undershoot or a reduction in magnitude of the tongue tip gesture.  As predicted by the life cycle, /l/-weakening is a gradient phenomenon in this phase.  It then proceeds through the phonologisation and stabilisation phases leading to Stage 2, at which point /l/-vocalisation is reinterpreted as a discrete phonological rule.


\begin{table}
\caption{English lateral epenthesis by rule inversion}
\label{eng_latep}
\begin{tabular}{ll}
\lsptoprule
Stage 0: & \\
\enspace Consonantal coda /l/ & Consonantal onset /l/ \\
\enspace /drɔːl/ → [drɔːɫ] & /drɔːl+ɪŋ/ → [drɔːlɪŋ] \\
    \midrule
Stage 1: & \\
\enspace Weakening of coda /l/ & No word-medial weakening \\
\enspace /drɔːl/ → [drɔː\textsuperscript{ɫ}] & /drɔːl+ɪŋ/ → [drɔːlɪŋ] \\
    \midrule
Stage 2: & \\
\enspace Vocalisation of coda /l/ & Preservation of word-medial /l/ \\
\enspace /drɔːl/ → [drɔː] & /drɔːl+ɪŋ/ → [drɔːlɪŋ] \\
    \midrule
Stage 3: & \\
\enspace Input restructuring & Reinterpretation: /l/ is epenthetic \\
\enspace /drɔː/ → [drɔː] (\textit{drawl}) & /drɔː+ɪŋ/ → [drɔːlɪŋ] (\textit{drawling}) \\
    \midrule
Stage 4: & \\
\enspace Underlying ∅ & Extension of /l/-epenthesis \\
\enspace /drɔː/ → [drɔː] (\textit{draw}) & /drɔː+ɪŋ/ → [drɔːlɪŋ] (\textit{drawing}) \\
\lspbottomrule
\end{tabular}
\end{table}


\newpage
The critical reanalysis happens at Stage 3, which can be assumed to be a learner-driven development.  A child acquiring the grammar at this stage reaches the generalisation that, rather than an underlying /l/ vocalising word-finally, the underlying form for \textit{drawl} in fact contains no /l/.  The child therefore postulates that the [l] in forms like \textit{drawling} is epenthetic.  The Stage 2 deletion rule has thus undergone inversion to a Stage 3 epenthesis rule.  The final phase in the development is the generalisation of this rule to all VV sequences (whether derived through affixation, as in the case of \textit{drawing}, or dialect-dependently, through concatenation of words into phrases, as in \textit{the bra is}).

\subsubsection{Oceanic}
In addition to English, \citet[7]{Vaux2001} and \citet[15]{Morley2015} mention Motu (Oceanic, Papua New Guinea), which displays instances of word-initial /l/ that are present neither in cognate forms in related languages nor in Proto-Oceanic reconstructed forms.  Data (adapted from \citealt{Blust1999}) are given in \tabref{motu_fij}.  As shown, elimination of word-initial onsetless syllables through epenthesis occurs in both Motu and Fijian in the context of the vowel /a/.  Blust comments that Motu /l/ is a reflex of historical /j/, such that the Fijian data here can be said to resemble a prior phonological stage for Motu.  

\begin{table}
\caption{/l/-epenthesis in Motu and /j/-epenthesis in Fijian}
\label{motu_fij}
\begin{tabularx}{\textwidth}{lllQ}
\lsptoprule
  & {Proto-Oceanic} & {Motu} & \\\midrule
a.& ansan & lada & `name' \\
b.& asaŋ & lada & `gills' \\
c.& apaʁat & lahara & `North-West wind and season' \\
d.& api & lahi & `fire' \\
e.& aja & lala & `father's sister; woman's brother's child' \\
f.& aku & lau & `I' \\\midrule
  & {Proto-Oceanic} & {Fijian} & \\\midrule
g.&  ansan & jaca & `name' \\
h.&  asaq & jaca & `grate, grind, sharpen' \\
i.&  aɲam & janajana & `loosely plaited' \\
j.&  ane & jane & PO: `termite'; Fij. 'moth species' \\
k.&  asi & jasi & `sandalwood' \\
\lspbottomrule
\end{tabularx}
\end{table}

Unlike /l/-epenthesis in English, Oceanic /j/-epenthesis is not the result of rule inversion.  Instead, we might reconstruct the change as the result of vowel breaking.  Under this scenario~-- and in keeping with the life cycle's fundamental claim that phonological changes begin as low-level phonetic effects~-- the inception of the change might have involved onsetless word-initial syllables being articulated with something like a short onglide: e.g. [\textsuperscript{ə}asi].  As this pattern then underwent phonologisation and rose above the level of speaker awareness, it could gradually have become more /j/-like: i.e. involving a larger articulatory movement and a greater acoustic distance from the original /a/.\footnote{A parallel to this development is the Great Vowel Shift, specifically the emergence of /aɪ/ and /aʊ/ from historical /iː/ and /uː/, respectively.  This is thought to have involved vowel breaking and progressive stages of change in which the vocalic onset became more phonetically distinct from the original high vowel over time: e.g. [iː] > [iɪ] > [əɪ] > [aɪ] (cf. \citealt[246ff.]{Krug2017}).}  Following the life-cycle trajectory, a rule of /j/-insertion before /a/ would have emerged in the grammar as the gradient phonetic pattern stabilised historically: i.e. /asi/ → [asi] > [\textsuperscript{ə}asi] > [\textsuperscript{j}asi]~> [jasi].

The additional innovation in Motu is an extension of this pathway of change, i.e. ∅ > /j/ > /l/.  What, then, could cause lateralisation of the historical epenthetic /j/?  This development is most probably perceptually driven. Whereas phonologisation and stabilisation of vowel breaking may lead to the emergence of word-initial glides, glides make relatively poor onsets because of their acoustic similarity to vowels.\footnote{Note that the etymological /j/ in \tabref{motu_fij}e also lateralises in Motu. In addition to /lala/, Blust lists the Motu forms /huala/ `crocodile' and /mala/ `tongue', from Proto-Oceanic /puqaja/ and /maja/, respectively.} This fact underlies the prohibition that some languages enforce on syllable-initial glides.  It is also the cause of fortition effects, such as glide hardening, which reduce the sonority of glides typically in syllable-initial positions (cf. (4) below).  Within the OT literature, this has been formalised as a markedness hierarchy for onset sonority \citep{Gopal2018, Gouskova2004}.\footnote{See \textcitetv{chapters/04.Bokhari} on the implementation of a similar hierarchy in the analysis of Hijazi Arabic vowel epenthesis.}

\ea \label{ons_son}
Onset sonority markedness hierarchy (adapted from \citealt[84]{Gopal2018})\\\relax

{\small \mbox{*O\textsc{ns}/j $\gg$ *O\textsc{ns}/r $\gg$ *O\textsc{ns}/l $\gg$ *O\textsc{ns}/n $\gg$ *O\textsc{ns}/z $\gg$ *O\textsc{ns}/d $\gg$ *O\textsc{ns}/s $\gg$ *O\textsc{ns}/t}}
\z

As indicated, high-sonority glides (represented in (\ref{ons_son}) by the shorthand /j/) are highly marked when they occur in syllable-initial position, whereas low-sonority voiceless stops (represented here by /t/) are optimal onsets.  In this regard, (\ref{ons_son}) is relevant for Fijian too.  \citet[10--11]{Blust1999} argues that /j/-epenthesis in Fijian was a lexically gradual change.  The words listed in \tabref{motu_fij} are relatively late examples.  Words that were early targets for epenthesis show evidence of subsequent glide hardening, i.e. /j/ > /c/.  In the examples in \tabref{ep_fij}, the historical epenthetic /j/ was maximally optimised with regard to onset sonority.  In accordance with (\ref{ons_son}), hardening of /j/ to /c/ maximally increases the perceptual distinctiveness between the onset and following vowel.

\begin{table}
\caption{Epenthesis and glide hardening in Fijian}
\label{ep_fij}
\begin{tabularx}{\textwidth}{lXXl}
\lsptoprule
   & {Proto-Oceanic} & {Fijian} & \\
   \midrule
a. & asam & caca & `fern species' \\
b. & aŋin & caɡi & `wind' \\
c. & aŋo & caɡo & PO `yellow'; Fij. `turmeric' \\
d. & aʁu & cau & `shore tree' \\
e. & apaʁat & cava & `North-West monsoon, storm wind' \\
\lspbottomrule
\end{tabularx}
\end{table}

In Motu, optimisation of historical epenthetic /j/ was not as extreme.  Rather than reducing the sonority of the epenthetic onset to the lowest possible level, as in Fijian, the change /j/ to /l/ in Motu represents a sonority decrease of just two increments in the onset sonority hierarchy.  This confirms the fact that the threshold of onset well-formedness with regard to sonority is set language-specifically in regard to (\ref{ons_son}).  Indeed, I shall argue that sonority reduction from /j/ to /l/ is of relevance for Hindi too: this is discussed in Section \ref{altern}.  Before proceeding to discussion of that phenomenon, Section \ref{hindi} below first introduces the Hindi data that form the basis of the remainder of the paper.

\section{Hindi causative verbs}\label{hindi}
Modern Standard Hindi forms causative verbs with two suffixes, /-ɑ/ and /-ʋɑ/.  /-ɑ/ forms the so-called first or direct causatives, and /-ʋɑ/ is used for the second or indirect causatives.  Some verbs select only one suffix or the other: for example, /puʧʰ/ `ask' can take the second formative (i.e. /pʊʧʰ-ʋɑ/ `cause to be asked') but not the first (*/pʊʧʰ-ɑ/).  A second subset of verbs can take both suffixes, thereby yielding semantically distinct triplets (cf. \citealt{BhattEmbick2017}): e.g. /kud/ `jump' $\sim$ /kʊd-ɑ/ `bounce' $\sim$ /kʊd-ʋɑ/ `make jump'.

In this paper, I focus exclusively on the first causative forms.  As illustrated in the data below, formation of first causatives triggers a range of phonological processes.  \tabref{non_alt} lists verb stems containing the lax vowels, /ɪ, ʊ, ʌ/, and /ɛ, ɔ/, which are derived from historical diphthongs.  In these verbs, the stem vowel does not alternate following affixation of causative /-ɑ/.  However, stems containing the tense vowels, /i, u, ɑ/, or /e, o/~-- which, unlike reference varieties of English, pattern as lax vowels in Hindi~-- are subject to neutralisation in causative derivations.  As shown in \tabref{stem_lax}, stems containing tense vowels display laxing in their corresponding causative forms, whereas /e/ and /o/ raise to [ɪ] and [ʊ], respectively, as in the examples in \tabref{stem_rais}.

\begin{table}
\caption{Non-alternating stems in /ɪ, ʊ, ʌ, ɛ, ɔ/}
\label{non_alt}
\begin{tabularx}{\textwidth}{lQQQl}
  \lsptoprule
   & Stem & & \multicolumn{2}{l}{Causative stem} \\\midrule
a. & lɪkʰ & `write' & lɪkʰ-ɑ & `dictate' \\
b. & sʊn & `hear' & sʊn-ɑ & `tell, cause to hear' \\
c. & ʧʌl & `move, go' & ʧʌl-ɑ & `drive' \\
d. & pʰɛl & `stretch' & pʰɛl-ɑ & `extend, cause to stretch out' \\
e. & dɔɽ & `run' & dɔɽ-ɑ & `urge on, cause to run' \\
\lspbottomrule
  \end{tabularx}
\end{table}

\begin{table}
\caption{Laxing: stems in /i, u, ɑ/}
\label{stem_lax}
\begin{tabularx}{\textwidth}{lQQQl}
  \lsptoprule
   & \multicolumn{2}{@{~}l}{Stem} & \multicolumn{2}{l}{Causative stem} \\\midrule
a. &  ʧʰin & `snatch' & ʧʰɪn-ɑ & `cause to snatch' \\
b. &  ɡ\textsuperscript{ɦ}um & `turn, tour' & ɡ\textsuperscript{ɦ}ʊm-ɑ & `tell, cause to hear' \\
c. &  mɑn & `accept' & mʌn-ɑ & `persuade' \\
\lspbottomrule
\end{tabularx}
\end{table}

\begin{table}
\caption{Vowel raising: stems in /e, o/}
\label{stem_rais}
\begin{tabularx}{\textwidth}{lQQQl}
\lsptoprule
   & \multicolumn{2}{@{~}l}{Stem}  & \multicolumn{2}{l}{Causative stem} \\\midrule
a. & leʈ & `lie down' & lɪʈ-ɑ & `make lie down' \\
b. & bol & `speak' & bʊl-ɑ & `call, invite' \\
\lspbottomrule
\end{tabularx}
\end{table}

Note that the stems listed in Tables~\ref{non_alt}–\ref{stem_rais} are all consonant-final.  By contrast, the verbs listed in \tabref{caus_l} have stem-final vowels.  Their causative forms in \tabref{caus_l}a--e are of principal interest because /l/ occurs as a hiatus-breaker between the base and suffix vowels.  Whilst the laxing patterns are not critical for the operation of /l/-insertion, vowel neutralisation can also be observed here: /i/ laxes to [ɪ] in examples \tabref{caus_l}a--b and the stems containing the mid vowels /e/ and /o/ display raising in their causative counterparts in \tabref{caus_l}c--e.

\begin{table}
\caption{/l/ after vowel-final bases in first causative forms}
\label{caus_l}
\begin{tabularx}{\textwidth}{lXXXl}
\lsptoprule
   & \multicolumn{2}{@{~}l}{Stem}  & \multicolumn{2}{l}{Causative stem} \\\midrule
a. & pi & `drink' & pɪ-lɑ & `water, irrigate' \\
b. & si & `sew' & sɪ-lɑ & `caused to sew' \\
c. & de & `give' & dɪ-lɑ & `cause to be given' \\
d. & ro & `cry' & rʊ-lɑ & `cause to cry' \\
e. & so & `sleep' & sʊ-lɑ & `put down to sleep' \\
\lspbottomrule
\end{tabularx}
\end{table}

What is particularly interesting about these forms is that /l/-insertion is not a regular hiatus-breaking strategy in Hindi.  The forms in Tables~\ref{futs} and~\ref{perfs} provide a point of comparison. In \tabref{futs}, hiatuses formed by the addition of future-tense suffixes to vowel-final bases are tolerated on the surface. \citet[72--74]{Ohala1983} notes that /j/-epenthesis in these types of VV-sequences is sometimes observed (e.g. [pɪjeɡɑ], [sɪjeɡɑ], etc.), but this is not obligatory.\footnote{Glottal stops or [w] may also occur variably in /e+e/ sequences: e.g.   [kʰee], [kʰeʔe], [kʰewe] `row' (\textsc{2sg.fut.subj}). Like variable /j/-insertion, these patterns may be partially dependent on dialectal factors that remain to be fully explored.}

\begin{table}
\caption{Hiatus after vowel-final bases in future forms. Formation of future forms triggers laxing of stem /i/ in \tabref{futs}a--b in the same way that causative /-ɑ/ does in \tabref{stem_lax}a and \tabref{caus_l}a--b.  The same pattern is noted with stem /u/ in \tabref{futs}g. /de/ `give' is irregular, but note that hiatus after stem-final /ɪ/ is unrepaired, as shown.}
\label{futs}
\begin{tabularx}{\textwidth}{lQQQQ}
\lsptoprule
   & \multicolumn{2}{@{~}l}{Stem} & \textsc{2pl.fut.infor} & 3\textsc{sg.fut.m} \\\midrule
a. &  pi & `drink' & pɪ-oɡ-e & pɪ-eɡ-ɑ \\
b. &  si & `sew' & sɪ-oɡ-e & sɪ-eɡ-ɑ \\
c. &  de & `give' & diʤɪ-oɡ-e & diʤɪ-eɡ-ɑ \\
d. &  ro & `cry' & ro-oɡ-e & ro-eɡ-ɑ \\
e. &  so & `sleep' & so-oɡ-e & so-eɡ-ɑ \\
f. &  ɑ & `come' & ɑ-oɡ-e & ɑ-eɡ-ɑ \\
g. &  ʧʰu & `touch' & ʧʰʊ-oɡ-e & ʧʰʊ-eɡ-ɑ \\
\lspbottomrule
\end{tabularx}
\end{table}

Similarly, hiatuses in stem-suffix sequences occur in a subset of the perfective forms listed in \tabref{perfs}. In the feminine singular and masculine plural, formed through suffixation of /-i/ and /-e/, respectively, hiatus is observed after vowel-final stems. Interestingly, this is not the case in the masculine singular forms.  Here, addition of perfective /-ɑ/, which is syncretic with causative /-ɑ/, triggers the emergence of a pre-suffixal [j].  Note that [j] is obligatory in these forms (unlike optional /j/-epenthesis in forms like [pɪeɡɑ$\sim$pɪjeɡɑ]) and that [j] never occurs after consonant-final bases, as in \tabref{perfs}d--e. These facts confirm that it is specifically hiatus before /-ɑ/ that is targeted for obligatory repair in Hindi.  In causatives, the outcome is a pre-suffixal [l], whereas in masculine perfectives, it is a pre-suffixal [j].\footnote{Thus, causative perfectives exhibit a double repair: e.g. [pɪlɑjɑ] `irrigated' (\textsc{3sg.caus.perf.m}).}

\begin{table}
\caption{Perfective forms}
\label{perfs}
\begin{tabularx}{\textwidth}{lQQQQl}
\lsptoprule
   & Stem & & \textsc{3sg.perf.f} & \textsc{3pl.perf.m} & \textsc{3sg.perf.m}  \\
   \midrule
a. &  pi & `drink' & pi & pi-e & pɪ-jɑ, *pi-ɑ \\
b. &  ro & `cry' & ro-i & ro-e & ro-jɑ, *ro-ɑ  \\
c. &  ɑ & `come' & ɑ-i & ɑ-e & ɑ-jɑ, *ɑ-ɑ \\
d. &  leʈ & `lie down' & leʈ-i & leʈ-e & leʈ-ɑ, *leʈ-jɑ \\
e. &  bol & `speak' & bol-i & bol-e & bol-ɑ, *bol-jɑ \\
\lspbottomrule
\end{tabularx}
\end{table}


It is therefore clear that regular /l/-insertion and /j/-insertion are morphologically conditioned (cf. \citealt{Vaux2001}).  Regarding the causative pattern, I refer to a derivational sketch in Section 4 which assumes that /-lɑ/ is an allomorph of /-ɑ/ in the present-day language.  In this sense, and depending upon how restrictively one wishes to define the term ``epenthesis", /-lɑ/-\textit{selection} may be the most appropriate label for the synchronic Hindi patterns outlined in \tabref{caus_l}.  Nevertheless, I also shall make the claim that the occurrence of [l] in causative forms is governed by phonological factors.  As will also be discussed in Section \ref{altern}, synchronic morphologically-conditioned allomorphy may reflect patterns of epenthesis that were phonologically active in earlier forms of a language.  In this connection, I now turn to discussing some of the key historical facts that are relevant for the Hindi patterns, beginning with causative formations in Sanskrit.

\section{Historical considerations}\label{histo}
Causative verbs were formed with the suffix /-ɑjɑ/ in Classical Sanskrit, as illustrated in \tabref{skrt_caus}a--d below (data adapted from \citealt[§129]{Mayrhofer1978}).  Vowel-final roots show special behaviour: note that in \tabref{skrt_caus}e--g, the consonant /p/ intervenes between the base and suffix vowels.  The parallel between pre-suffixal /p/ in these forms and the Hindi causatives with pre-suffixal /l/ listed in \tabref{caus_l} is clear.  However, beyond the fact that they serve a hiatus-breaking function, the phonological correspondence between these consonants is not obvious.

\begin{table}
\caption{Sanskrit causative formation}\label{skrt_caus}
\begin{tabularx}{\textwidth}{lQQQl}
\lsptoprule
   & \multicolumn{2}{@{~}l}{Root} & \multicolumn{2}{l}{Causative forms (\textsc{3sg.pres.indic})} \\\midrule
a. & kr̩ & `do' & k\=ɑr-ɑjɑ-ti & `causes to do' \\
b. & ɟɑn- & `be born' & ɟɑn-ɑjɑ-ti & `begets, procreates' \\
c. & dr̩ʃ & `see' & dɑrʃ-ɑjɑ-ti & `causes to see' \\
d. & budh- & `wake' & bodh-ɑjɑ-ti& `causes to wake' \\
e. & ɡ\=ɑ & `sing' & ɡ\=ɑ-pɑjɑ-ti & `causes to sing' \\
f. & ɟɲ\=ɑ & `know' & ɟɲ\=ɑ-pɑjɑ-ti & `causes to know' \\
g. & d\=ɑ & `give' & d\=ɑ-pɑjɑ-ti & `causes to give' \\
\lspbottomrule
\end{tabularx}
\end{table}

Regarding the origin of the Hindi pattern, a direct phonological change, /p/ > /l/, seems unlikely for two reasons.  Firstly, lateralisation of labial consonants is not attested in Indo-Aryan.  Cases of lateral excrescence after labial consonants have been documented in other language families.  Most notably, Common Slavic /pj/-clusters developed into /pl\textsuperscript{j}/: this is represented dialectally in examples like Belorussian \textit{kanópli}, Slovenian \textit{konóplja} vs Upper Sorbian \textit{konopje}, Polish \textit{konop$\sim$konopie} `hemp' (see \citealt[219ff.]{Shevelov1964} for further examples).  Baltic shows a similar development, particularly Latvian: e.g. Latvian \textit{pl'aūt} `mow' vs Lithuanian \textit{piáuti} `cut' (\citealt[§84]{Endzelin1922}).\footnote{I am grateful to Florian Wandl for bringing these facts to my attention.} This raises the question of whether a lateral consonant could have emerged in the /p/ causative allomorph in the history of Indo-Aryan: e.g. /-pɑjɑ/ > \textsuperscript{?}/-pjɑ/ > \textsuperscript{?}/pl(j)ɑ/ > /-lɑ/.  Indeed /pj/- and /pl/-clusters in Hindi are rare (at least outside of English loanwords) relative to the very high frequency of /pr/.  However, whereas /pj/ and /mj/ do occur medially in Sanskrit (\citealt[161]{Masica1991}), these clusters never lateralise in any Indo-Aryan variety.  A development resembling Balto-Slavic /pj/ > /pl/ with subsequent loss of /p/ is therefore an unlikely source of the Hindi /l/-causatives.

This connects to the second reason for rejecting /p/ > /l/ as a plausible pathway of change, namely that the /-pɑjɑ/ allomorph was repurposed as the second causative suffix via a development /-pɑjɑ/ > /-pe/ > /-ʋɑ/.  This occurred without the emergence of a prevocalic lateral consonant at any recorded stage.  

In fact, the change /-pɑjɑ/ > /-ʋɑ/ involved intermediate stages that played out differently across Indo-Aryan dialects. For example, /ɑj/-monophthongisation had a significant morpho-phonological impact in the transition from Old (OIA) to Middle (MIA) Indo-Aryan.  In Pāli (early MIA), /p/-causatives were retained despite simplification of OIA /-ɑjɑ/ to /-e/: e.g. /ɲɑ̄-pē-ti/ `causes to know', cf. \tabref{skrt_caus}f~-- additional examples in \citet[§52]{Oberlies2001}.  By the middle MIA period, this suffix had passed through a phase of intervocalic lenition to /-ʋe/ (e.g. Māhārāṣṭrī /ʤɑ̄ɳɑ̄ʋēi/ `causes to know'), and a lower suffix vowel is observed in later MIA varieties such as Apabhraṁśa: e.g. /kɑrɑ̄ʋɑi/ `causes to do' (cf. \citealt[230]{Bubenik2003}).  These changes are also reflected in the patterning of causative suffixes across present-day New Indo-Aryan (NIA) languages, with some exhibiting both first and second causatives that resemble Hindi /-ɑ/ and /-ʋɑ/, and others showing only one type or the other (cf. \citealt[§9.6]{Masica1991} for a complete overview).\footnote{For example, Odiya has only first causatives in /-ɑ/ and Sinhalese has only /-ʋɑ/.  Some varieties, such as Maithili, have second-causative suffixes with /b/ instead of /ʋ/, which presumably arose from voicing but not spirantisation of MIA /-pe/.} 

Given the low probability that /l/-causatives originate from historical /p/\hyp causatives, other potential sources have been suggested.  One claim is that the Sanskrit form \textit{pālayati} `saves, brings across' may be the source of Hindi /l/-causatives.  \textit{pālayati} is a causative formation on the root /pr̩-/ `cross over', which, as \citet[85]{Jamison1983} notes, is a variant of the well-attested regular formation, \textit{pārayati}.  Dialectal variation between /r/ and /l/ is also well documented for OIA: \citet[280]{Norman2012} comments that ``[...] there must have been OIA dialects which turned all -r- and -l- sounds into -l-, others which turned them all into -r-, and still others which mingled the two sounds in different proportions".\footnote{Interestingly, /l/ in Kwa \parencitetv{chapters/02.Sande} has multiple possible surface realisations, including [l, ɾ, n].}  Thus, despite its status as a dialectal variant of \textit{pārayati}, the argument goes that the pre-suffixal /l/ from \textit{pālayati} generalised as an epenthetic segment that was inserted after vowel-final bases in causative formations.  \citet[§156]{Sen1960} gives the Apabhraṁśa 1\textsc{sg} causative form \textit{dālayami} (from /d\=ɑ/, `give') as a possible instance of this development.  

Nevertheless, beyond a very small set of examples, there is no strong evidence from Old or Middle Indo\hyp Aryan that supports an analogical development of /l/\hyp causatives based on a dialectal variant of \textit{pārayati} with /r/\hyp lateralisation.  A further point in this connection is that merger of /l/ and /r/ in favour of /l/ is more typical of eastern varieties of Indo-Aryan, for example Eastern Aśokan (early MIA), with Western Aśokan maintaining the /l/$\sim$/r/ contrast \citep[176]{Oberlies2003}.  This is at least potentially problematic since Modern Hindi evolved from varieties of MIA, particularly Śaurasenī, which originate in the central-western region.\footnote{Note that other Western NIA languages such as Panjabi also display /l/-causatives that closely resemble the Hindi forms.}  This fact presents a problem also for a second possible historical source of /l/-causatives.  A number of Eastern NIA languages exhibit /l/ in perfective forms: examples for `seen' (from \citealt[267]{Bloch1965}) include Bengali \textit{dekkhila}, Maithili \textit{dekhal} and Marathi and Oriya \textit{dekhilā}, inter alia.  \citet{Bloch1965} identifies this /l/ as a potential source of /l/-causatives.  However, /l/ is not observed in perfective forms elsewhere (cf. Hindi /dekʰɑ/ `seen').  The situation is therefore one of geographical discontinuity: two possible sources of /l/ that could have provided an analogical basis for the development of /l/-causatives in central-western NIA languages are restricted to eastern varieties, both historically and synchronically.

\section{An alternative pathway: Epenthesis and onset sonority}\label{altern}
\subsection{Reconstructing the pre-history of lateral epenthesis}
Since evidence for an analogical development is not strongly supported either by historical sources or by the geographical/dialectal facts, it is worthwhile to consider other mechanisms of change that could have led to the Hindi pattern. In this section, I explore an alternative scenario that draws both on the core principles of the life cycle and the analyses of lateral epenthesis in English and Motu already discussed.

\begin{figure}
\caption{Diachronic development of Hindi causative suffixes}
\label{dev_hindi}
% \includegraphics[scale=0.5]{figures/pathway_final}
\begin{forest}
  [,phantom
  [\textit{Stage 1}:, for tree={no edge}
      [\textit{Stage 2}:
         [\textit{Stage 3}:
            [\textit{Stage 4}:
               [\textit{Stage 5}:
                  [\textit{Stage 6}:]
               ]
            ]
         ]
      ]
   ]
   [OIA, for tree={no edge}
      [Early MIA
         [Late MIA
            [NIA/Pre-Hindi 1
               [NIA/Pre-Hindi 1
                  [Hindi]
               ]
            ]
         ]
      ]
   ]
   [\{-ɑjɑ, for tree={edge=dashed}
      [\{-e
         [\{-ɑ
            [\{-ɑ\textsubscript{1}
               [\{-ɑ\textsubscript{1}, name=a]
            ]
            [~, no edge
               [~, no edge
                  [{$\left\{\begin{tabular}{c}-ɑ1\\-lɑ1\end{tabular}\right\}$}, name=ala, no edge]
               ]
            ]
            [-\textbf{j}ɑ\textsubscript{1}\}
               [-\textbf{l}ɑ\textsubscript{1}\}, name=la]
            ]
         ]
      ]
   ]
   [-pɑjɑ\}, for tree={edge=dashed}
      [-pe\}
         [-ʋɑ\}
            [-ʋɑ\textsubscript{2},name=va21
               [~,phantom
                  [-ʋɑ\textsubscript{2},name=va22]
               ]
            ]
         ]
      ]
   ]
  ]
\draw[dashed](a)--(ala);
\draw[dashed](la)--(ala);
\draw[dashed](va21)--(va22);
\end{forest}

\end{figure}

\figref{dev_hindi} illustrates the pathway of change leading to the development of pre-suffixal laterals in Hindi causative verb forms that I shall assume.  Stages 1--3 refer to the pre-history of Hindi.  As discussed in Section 3, monophthongisation of OIA \mbox{/-ɑjɑ/} to MIA /-e/ -- i.e. the change from Stage 1 to Stage 2 -- is well documented.  Similarly, textual evidence supports the reconstruction of a later development in which the quality of the MIA causative suffix, /-e/, lowered to /-ɑ/.  The post-vocalic allomorph, /-pe/, was similarly affected, in addition to a change in consonantal quality through spirantisation and voicing: i.e. /-pe/ > \mbox{/-ʋɑ/}.

It is then at Stage 4 that the most important innovations for the development of the Hindi pattern happen.  Firstly, the /-ʋɑ/ causative suffix that developed from OIA /-pɑjɑ/ acquires a grammatically distinct function: at this point, it is used to form second causatives that are semantically distinct from first causative forms.  This would have involved a fundamental reanalysis of /-ʋɑ/, i.e. a \textit{change in status} \citep{Anderson1981}.  Whereas /-ʋɑ/ occurs only as an allomorph of the general causative suffix, /-ɑ/, at Stage 3, it has acquired a new status as an independent, lexically listed verbal suffix by the completion of Stage 4. As noted in Section \ref{histo}, this development is particular to certain NIA languages; and whilst it is semantic and grammatical in nature, this change entails a crucial phonological consequence, namely that a consonant-initial suffix allomorph is no longer available for first-causative formations.

At this point, it is reasonable to assume that a new phonological repair to VV-sequences generated by affixation of first-causative /-ɑ/ to vowel-final bases would have evolved.  In accordance with the reconstruction in \figref{dev_hindi}, one possibility is that this repair could have taken the form of an innovative pattern of /j/-epenthesis in the first instance.  This is a somewhat tentative proposal, though it is not without foundation. Similar to the reconstruction of Fijian /j/-epenthesis, such a development may have come about through vowel breaking, or through a gradient liaison of the type /ɑ-ɑ/ → [ɑ\textsuperscript{j}ɑ] that, as already noted, is commonly observed synchronically in Hindi. Furthermore, verbal stems terminating in /-ɑ/ have a particularly high frequency of occurrence in the language.  The life cycle postulates that any phonological innovation begins as a low-level phonetic effect: thus, productions resembling [ɑ\textsuperscript{j}ɑ] may have occurred, for example, as some sort of articulatory or perceptual artifact of speakers producing /ɑ-ɑ/-sequences generated through the formation of causatives. 

Whereas the phonetic precursor of the pattern cannot be precisely pinpointed, phonologisation and stabilisation of a gradient pattern of /j/-liaison of this type would lead to a new, categorical rule of /j/-epenthesis specifically targeting hiatus in the context of a suffixal /-ɑ/.  Interestingly, under the assumption that a /j/-epenthesis pattern is a plausible precursor to lateral epenthesis, then the developments between Stages 1 and 4 in \figref{dev_hindi} could be said to represent a kind of ``long-distance'' rule inversion.  In the English patterns discussed in Section \ref{english}, it was observed that lateral epenthesis emerged from a historical pattern of coda /l/-reduction and deletion.  Under the scenario presented in \figref{dev_hindi}, loss of /j/ through diachronic truncation of OIA /-ɑjɑ/ to MIA /-e/ is undone at Stage 4: i.e. /j/ is re-supplied through epenthesis in NIA in the same context that it disappeared from in the development from OIA to MIA.

\subsection{From /j/-epenthesis to /l/-insertion}
\largerpage
The second phase in the development of /l/-insertion occurs at Stage 5.  Here, the /j/-epenthesis process innovated at Stage 4 evolves further into a pattern of lateral epenthesis. This parallels what \citet{Blust1999} reconstructs for Motu. In line with the generalisation that /j/ is a poor onset from a perceptual point of view, fortition to /l/ reduces the sonority of the epenthetic material.\footnote{The proposal here is therefore for a development that follows from similar principles to Uffmann's (\citeyear{Uffman2007b}) treatment of /r/-insertion in English as intervocalic sonority optimisation.}  In the context of the onset sonority hierarchy in (\ref{ons_son}), this development can be modelled straightforwardly as a diachronic re-ranking of constraints penalising sonorant onsets.  This is illustrated in \figref{jtol}.\footnote{\figref{jtol} abstracts away from the vowel-neutralisation process that causes raising of /o/ to [ʊ] in causative formations.  I assume that this is governed by other constraints that are omitted from the tableaux here.}  

\begin{figure}
\caption{/j/-epenthesis > /l/-epenthesis by re-ranking of onset sonority constraints}
\label{jtol}
% \includegraphics[scale=0.3]{figures/tableaux_jtol}
% \hrule
\subfigure[Stage 4]{
% \ShadingOn
\begin{tableau}{c|c|s|s|s}
\inp{/so+ɑ/ }           \const*{\textsc{*hiatus}}  \const*{\textsc{Dep}-seg\textsubscript{[+cons]}}  \const*{*\textsc{Ons}/j} \const*{*\textsc{Ons}/l}
\cand{[sʊɑ]}            \vio{*!}   \vio{}   \vio{} \vio{}
\cand[\Optimal]{[sʊjɑ]} \vio{}     \vio{}   \vio{*} \vio{}
\cand{[sʊlɑ]}           \vio{}     \vio{*!} \vio{}  \vio{*}
\end{tableau}
}

\subfigure[Stage 5]{
% \ShadingOn
\begin{tableau}{c|c|s|s|s}
\inp{/so+ɑ/ }           \const*{\textsc{*hiatus}}  \const*{*\textsc{Ons}/j}  \const*{\textsc{Dep}-seg\textsubscript{[+cons]}}  \const*{*\textsc{Ons}/l}
\cand{[sʊɑ]}            \vio{*!}   \vio{}      \vio{} \vio{}
\cand{[sʊjɑ]}           \vio{}     \vio{*!}   \vio{} \vio{}
\cand[\Optimal]{[sʊlɑ]} \vio{}     \vio{}      \vio{*}  \vio{*}
\end{tableau}
}
\end{figure}

At Stage 4, a superordinate constraint targeting forms with hiatus enforces a repair by epenthesis to VV-sequences generated by suffixation of causative \mbox{/-ɑ/} to vowel-final bases like /so/ `sleep'.  The quality of the epenthetic consonant is regulated by lower-ranked constraints.  As shown, a constraint such as \textsc{Dep}-seg\textsubscript{[+cons]} prevents the insertion of [+cons] segments: this eliminates candidate \figref{jtol}a-iii with lateral epenthesis (and any candidate exhibiting additional consonantal material).  Accordingly, candidate \figref{jtol}a-ii with /j/-epenthesis is selected as the winner despite its violation of low-ranked *O\textsc{ns}/j.  The change from Stage 4 to Stage 5 happens through promotion of *O\textsc{ns}/j.  As this constraint militates against syllable-initial /j/, \figref{jtol}b-ii is eliminated by the later grammar.  Under this ranking, therefore, the form with lateral epenthesis, i.e. candidate \figref{jtol}b-iii, wins.

In fact, diachronic promotion of a constraint like *O\textsc{ns}/j is well motivated for Indo-Aryan on independent grounds.  This is confirmed by a number of other phonological developments that conspired to eliminate /j/-onsets historically.  For example, OIA /j/ hardened in MIA in a similar way to the Fijian change, /j/ > /c/, resulting in a sonority reduction in word-initial contexts.  The outcome of this in MIA is <j> orthographically (i.e. /ʤ/ in present-day Hindi), which, as \citet[169]{Masica1991} notes, had ``at least a fricative pronunciation in the Early MIA period": e.g. Skt. /jɑʋɑ/ > Pāli /ʒɑʋɑ/ `barley'; Skt. /juddʰɑ/ > Pāli /ʒuddʰɑ/ `battle, war'.\footnote{In some cases, preservation or re-introduction of Sanskrit forms has created doubles: e.g. Hindi /ʤʊʤʰ/ from MIA /ʒuddʰɑ/, which exists alongside /jʊdʰ/ from Skt. /juddʰɑ/.}  Furthermore, word-medial stop gemination before /j/ had a similar effect on sonority: e.g. Skt. /drɑʋjɑ/ > Pāli /dɑbbɑ/ `property'.  In examples like these, the high-sonority /j/-onset in the Sanskrit form was eliminated by gemination and hardening of syllable-final /ʋ/ and elimination of the /j/ in a way that is reminiscent of the well-known case of West Germanic gemination (e.g. Proto-Germanic [bid.jan] > Old English [bid.dan] `to ask').

The final development at Stage 6 in \figref{dev_hindi} involves a reanalysis of /l/\hyp epenthesis.  As was also the case with lexicalisation of /-ʋɑ/ in the transition from MIA to NIA, Stage 6 represents a change in status for causative /-lɑ/.  Whereas as /l/-causatives are generated by the phonology at Stage 5 through epenthesis of /l/ to VV-sequences that occur because of suffixation of causative /-ɑ/ to vowel-final bases, \figref{dev_hindi} assumes that this pattern entered a new phase in its life cycle at Stage 6. The outcome of this is a lexical listing of the first causative suffix allomorph in which /l/ is present underlyingly.\footnote{Relatedly, see \textcitetv{chapters/12.BaronianRoyerArtuso} regarding the claim that schwa epenthesis in Armenian may, in some cases, be understood as the result of lexical restructuring, i.e. lexicalisation of forms containing underlying schwas.}

As already alluded to, lexicalisation of causative /-lɑ/ as an allomorph of /-ɑ/ in the final stage of the life cycle represents what is probably the best characterisation of these patterns synchronically.  The schematisation shown in \figref{sync_caus} illustrates causative-suffix \textit{selection} based on the phonological shape of the base.

\begin{figure}
\caption{Synchronic derivation of first causative stems}
\label{sync_caus}
% \includegraphics[scale=0.3]{figures/example_sync_caus}

\begin{tabular}{rcc}
~& a. /sʊn/ `hear'& b. /so/ `sleep'\\
\textit{input:} &
   sʊn + $\left\{\begin{tabular}{ll} ɑ\\lɑ \end{tabular}\right\}$ &
      so + $\left\{\begin{tabular}{ll} ɑ\\lɑ \end{tabular}\right\}$\\
       & $\Downarrow$ & $\Downarrow$\\
\textit{output:} & [sʊn\textbf{ɑ}] & [sʊ\textbf{lɑ}] \\
\end{tabular}
\end{figure}

In \figref{sync_caus}a, the /-ɑ/ causative allomorph is selected because the root /sʊn/ is consonant\hyp final.  This yields the well-formed causative stem, [sʊnɑ], in contrast to ill-formed *[sʊnlɑ].  However, affixation of the /-ɑ/ allomorph in \figref{sync_caus}b would generate an ill-formed output with hiatus, i.e. *[sʊɑ].  Since the root /so/ is vowel-final, selection of the /-lɑ/ allomorph yields an optimal output with /l/-insertion, i.e. [sʊlɑ].  

As shown in \figref{a_la}, these operations can be derived from the constraint ranking already established in \figref{jtol}b.  For input \figref{sync_caus}b, in which the /-ɑ/ suffix allomorph is selected~-- i.e. \figref{a_la}i--iii~-- all candidates are eliminated by the top-ranked markedness constraints.  Conversely, candidate \figref{a_la}vi, in which the /-lɑ/ allomorph is selected and faithfully mapped to the surface form,\footnote{In this connection, see \textcitetv{chapters/06.Uffmann} for discussion of epenthesis as an outcome of faithfulness operations.} [sʊlɑ], incurs no violations of the superordinate constraints.

\begin{figure}
\caption{Stage 6: Suffix allomorph selection in /so+\{-ɑ, -lɑ\}/}
\label{a_la}
% \includegraphics[scale=0.3]{figures/tableau_allomorph}
\begin{tableau}{c|c|c|s}
\inp{/so+\{-ɑ, -lɑ\}/ }           \const*{\textsc{*hiatus}}  \const*{*\textsc{Ons}/j}  \const*{\textsc{Dep}-seg\textsubscript{[+cons]}}  \const*{*\textsc{Ons}/l}
\cand{/so+ɑ/ $\to$ [sʊɑ]}            \vio{*!}   \vio{}      \vio{} \vio{}
\cand{/so+ɑ/ $\to$ [sʊjɑ]}           \vio{}     \vio{*!}   \vio{} \vio{}
\cand{/so+ɑ/ $\to$ [sʊlɑ]} \vio{}     \vio{}      \vio{*!}  \vio{*}
\\
\hline
\\[-2em]
\cand{/so+lɑ/ $\to$ [sʊɑ]}            \vio{*!}   \vio{}      \vio{} \vio{}
\cand{/so+lɑ/ $\to$ [sʊjɑ]}           \vio{}     \vio{*!}   \vio{} \vio{}
\cand[\Optimal]{/so+lɑ/ $\to$ [sʊlɑ]} \vio{}     \vio{}      \vio{}  \vio{*}
\end{tableau}
\end{figure}

\subsection{Some residual issues}
The foregoing analysis has argued in favour of a pattern of causative-stem formation in synchronic Hindi in which a lexically listed suffix allomorph, /-lɑ/, is selected for vowel-final bases as this maximally satisfies constraints penalising hiatus and the generation of forms with syllable-initial /j/. This sketch analysis illustrates how such forms are derived synchronically under the assumption that a historical \textit{phonological} pattern of /l/-epenthesis in causative formations was reanalysed diachronically as allomorphy. In this connection, there are other factors that merit comment, not least the fact that Hindi does also exhibit pre-suffixal /j/ in masculine perfective forms, as shown in \tabref{perfs}. Space does not permit full discussion of how this pattern coexists and interacts synchronically with the causative-formation patterns.  However, the derivations in \tabref{caus_perf} below briefly lay out a possible solution to this.

\begin{table}
\caption{Causative and perfective formation in synchronic Hindi}
\label{caus_perf}
\begin{tabular}{lcccc}
\lsptoprule
 & \multicolumn{2}{c}{Perfective formation} & \multicolumn{2}{c}{Causative-perfective formation}\\\midrule
Root: & a. /sʊn/ & b. /so/ & c. /sʊn/ & d. /so/ \\
 & ↓ & ↓ & ↓ & ↓ \\
SL: & sʊn & so & sʊn-ɑ\textsubscript{\textsc{caus}} & sʊ-lɑ\textsubscript{\textsc{caus}} \\
 & ↓ & ↓ & ↓ & ↓ \\
WL: & sʊn-ɑ\textsubscript{\textsc{perf}} & so-jɑ\textsubscript{\textsc{perf}} & sʊnɑ-jɑ\textsubscript{\textsc{perf}} & sʊlɑ-jɑ\textsubscript{\textsc{perf}} \\
\lspbottomrule
\end{tabular}
\end{table}

\tabref{caus_perf} assumes the same stratified phonological architecture as the life cycle (cf. \citealt{Bermúdez-Otero2011, Bermúdez-Otero2017, Kiparsky:2000}): the distinction between stem-level (SL) and word-level (WL) processes is particularly important here.  More specifically, I assume that perfective formation as a general inflectional pattern is confined to the word level.  By contrast, the formation of causative stems is handled by the stem-level grammar.  Thus, in examples (a) and (b), the underlying forms of the verbal stems /sʊn/ and /so/ received a faithful mapping at the stem level (i.e. in the absence of other morphological material).  At the word level, the perfective suffixes attach to the outputs generated at the stem level. This yields forms in [-ɑ] in cases like [sʊnɑ] `heard' (formed on a consonant-final base) and forms in [-jɑ] with vowel-final bases, as in [sojɑ] `slept'.

In the case of the causatives, \tabref{caus_perf}c is similar to \tabref{caus_perf}b.  The difference is that affixation of the causative suffix -- specifically the /-ɑ/ allomorph -- occurs at the stem level. This generates a causative-stem output formed on the root /sʊn/ that is vowel-final, i.e. [sʊnɑ]. Perfective formation at the word level then applies in the same way as in \tabref{caus_perf}b, thereby yielding an output in [-jɑ]: [sʊnɑjɑ] `caused to hear'. \tabref{caus_perf}d presents a case of double repair.  Here, formation of a causative stem with /-ɑ/ at the stem level would produce an ill-formed output with hiatus, i.e. *[sʊ-ɑ]. Accordingly, the /-lɑ/ allomorph is selected instead: this generates the /l/-causative form, [sʊlɑ].  In the same way as \tabref{caus_perf}c, this causative stem is vowel-final. Thus, the word-level grammar generates the perfective form [sʊlɑjɑ] `put to sleep', *[sʊlɑɑ].\footnote{Assuming the derivation proceeds in this way also accounts for the opaque patterning of the surface vowels in perfective forms like [sojɑ] vs [sʊlɑjɑ].}  

An obvious question that arises regarding these patterns is whether the /j/ in perfective forms is epenthetic.  This is not something that can be dealt with decisively here; however, the historical literature at least suggests that examples like \tabref{caus_perf}b–d are cases of epenthesis.  \citet[269]{Masica1991} comments that the perfective suffixes developed from a productive late Sanskrit suffix, /-itɑ/.  Then, ``[b]y the regular process of phonological attrition in MIA and NIA this became \textit{-ia} and thence \textit{-i} or \textit{-y}. [...] The extreme weakness of this element made it prone to disappear entirely [...]. In Standard Hindi [...] it is retained only with vowel stems: \textit{khā-y-ā} `eaten', \textit{gā-y-ā} `gone'".  Similarly, \citet[§439]{Mayrhofer1951} refers to a \textit{Bindevokal} that developed from diachronic weakening of /-itɑ/.  However, if it is the case that /j/ in perfective forms is epenthetic, it remains unclear why it surfaces only in the context of /-ɑ/ (i.e. in masculine forms) and not preceding other inflectional vowels (cf. the data in \tabref{perfs}).

Despite this, the development of the perfective suffixes from /-itɑ/ is important in connection with the proposed historical reconstruction shown in \figref{dev_hindi}. As this assumes that /l/-causatives developed from historical /j/-epenthesis, a relative chronology of the related patterns is suggested. This is outlined in (\ref{relchron}).
\largerpage

\ea Relative chronology of /j/-related patterns\label{relchron}
  \ea Elimination of /j/-onsets through fortition and gemination in MIA. 
  \ex Emergence of /j/-epenthesis in causative formations. 
  \ex Change from /j/ > /l/-epenthesis as sonority optimisation. 
  \ex Lexicalisation of /-lɑ/ as an allomorph of causative /-ɑ/. 
  \ex Emergence of /j/-epenthesis in perfective formations. 
  \z
\z

As discussed, the elimination of /j/ through hardening and gemination were early patterns already visible in MIA. Further to this, if the the re-introduction of /j/ in causative forms through historical epenthesis did indeed occur, then this, and the subsequent lateralisation of the epenthetic /j/, must also have been relatively early developments.  Under this scenario, a second, later innovation then resulted in /j/-epenthesis in the context of \textit{perfective} /-ɑ/ via progressive weakening of the /-itɑ/ suffix over time.  Crucially, the suggestion here is that this later pattern of /j/-epenthesis in perfectives parallels the earlier development which targeted \textit{causative} /-ɑ/.\footnote{Recall Blust's claim that /j/-epenthesis in Fijian was a gradual change that operated in identifiable phases.} 

To close the discussion, it must be reiterated that some of these suggestions are tentative, and the historical developments summarised in \figref{dev_hindi} and the chronology in (\ref{relchron}) constitute an attempt at synthesising various pieces of historical evidence, many of which are not fully understood independently. In addition to these historical concerns, there is also certainly more to say about the synchronic interaction between /l/-causatives and /j/-perfectives.  However, this is something that must be left for now for future research to address.


\section{Conclusion}\label{ramsammy:concl}
The goal of this paper was to discuss the diachrony of lateral epenthesis, which, as a rare phonological phenomenon, has received little attention in the theoretical literature to date.  I have argued that the development of /l/-epenthesis in English dialects is best understood as the result of rule inversion, which is consistent with previous accounts and those of the closely related phenomenon of /r/-sandhi.  In Oceanic, initial /j/-epenthesis in Motu and Fijian had phonological knock-on effects.  The development of initial /l/ in Motu and /c/ in Fijian has been explained in relation to language-specific optimisation of onset sonority.  

Regarding Indo-Aryan, I have argued that the origin of /l/-insertion in Hindi first causative formations cannot reliably be traced~-- at least directly~-- to OIA /p/-causatives.  Other explanations based on analogy from \textit{pālayati} or /l/ in participial forms that occur in Eastern NIA languages are similarly unsatisfactory.  An alternative pathway of change that does not assume an analogical development has therefore been proposed.  This reconstruction assumes a \textit{language-internal} development that involves the emergence of a pattern of /j/-epenthesis in causative forms and a subsequent change from epenthetic /j/ to /l/, paralleling the evolution of epenthesis in Motu. The proposed reanalysis has therefore aimed to account for the facts in a way that is informed by the core principles of the life cycle and that does not rely exclusively on historical dialectal patterns that may well have had no influence on the development of /l/-causatives in Hindi. 

\section*{Acknowledgements}
A previous version of the analysis of Hindi causative verbs was presented at the \textit{Fifth Edinburgh Symposium on Historical Phonology} in December 2021, in addition to the \textit{Epenthesis and Beyond Workshop}. I am grateful to attendees at both meetings who provided feedback on the developing analysis. I would like to personally thank Nimisha Upadhyay and Aarushi Asthana who kindly provided judgements on some of the Hindi data.  Thanks are also due to the two anonymous reviewers who commented on a previous draft of this chapter and offered helpful suggestions. I am responsible for any remaining errors.

Whilst preparing this paper, I was saddened to hear of the deaths in early 2022 of both Robert Blust and Colin Masica. Although I never had the opportunity to meet them, I wish to acknowledge the extraordinary contributions that both made to their respective fields of expertise. It will be immediately apparent that this paper would not have been possible without their ground-breaking work.

{\sloppy\printbibliography[heading=subbibliography,notkeyword=this]}
\renewcommand*{\rowletter}{\stepcounter{tableaurow}\alph{tableaurow}. }
\end{document}
