\documentclass[output=paper,colorlinks,citecolor=brown]{langscibook}
\ChapterDOI{10.5281/zenodo.14264528}
\title{Epenthesis and beyond: An overview} 
\author{Ji Yea Kim\orcid{}\affiliation{Gyeongsang National University} and Veronica Miatto\orcid{}\affiliation{Stony Brook University} and Andrija Petrović\orcid{}\affiliation{Stony Brook University} and Lori Repetti\orcid{}\affiliation{Stony Brook University}}

\abstract{}

\IfFileExists{../localcommands.tex}{
   \addbibresource{../localbibliography.bib}
   % add all extra packages you need to load to this file

\usepackage{tabularx,multicol}
\usepackage{url}
\urlstyle{same}

\usepackage{listings}
\lstset{basicstyle=\ttfamily,tabsize=2,breaklines=true}

\usepackage{langsci-basic}
\usepackage{langsci-optional}
\usepackage{langsci-lgr}
\usepackage{langsci-osl}
% \usepackage{./langsci/styles/langsci-lgr}
% \usepackage{./langsci/styles/langsci-osl}
% \usepackage{langsci-gb4e}

\usepackage{tikz}
\usetikzlibrary{patterns,calc}
\pgfdeclarepatternformonly{south east lines}{\pgfqpoint{-0pt}{-0pt}}{\pgfqpoint{3pt}{3pt}}{\pgfqpoint{3pt}{3pt}}{
    \pgfsetlinewidth{0.6pt}
    \pgfpathmoveto{\pgfqpoint{0pt}{3pt}}
    \pgfpathlineto{\pgfqpoint{3pt}{0pt}}
    \pgfpathmoveto{\pgfqpoint{.2pt}{-.2pt}}
    \pgfpathlineto{\pgfqpoint{-.2pt}{.2pt}}
    \pgfpathmoveto{\pgfqpoint{3.2pt}{2.8pt}}
    \pgfpathlineto{\pgfqpoint{2.8pt}{3.2pt}}
    \pgfusepath{stroke}}
    
\usepackage{stmaryrd}
\usepackage{wasysym}
\usepackage{multirow}
\usepackage{caption}
\usepackage{subcaption}
\usepackage{mathrsfs}
\usepackage{qtree}

\usepackage{linguex}


   %pminos do not split footnotes
% \interfootnotelinepenalty=10000 %Footnote in Laporte chapters has to be split SN


%\DeclareIndexNameFormat{default}{%
%\nameparts{#1}%
%\usebibmacro{index:name}%
%{\index[names]}%
%{\namepartfamily}%
%{\namepartgiveni}%
% {}% L1
% {}% L2
%{\namepartprefix}% generates spurious space L3
%{\namepartsuffix}% generates spurious space L4
%}

%  {\DeclareIndexNameFormat{default}{%
%     \usebibmacro{index:name}{\index[names]}{#1}{#3}{#5}{#7}}}

%\DeclareIndexNameFormat{default}{%
%  \usebibmacro{index:name}{\sindex[nom]}{#1}{#3}{#5}{#7}}

%\DeclareIndexNameFormat{default}{%
%  \usebibmacro{index:name}{\sindex[person]}{#1}{#3}{#5}{#7}}
%\DeclareIndexNameFormat{default}{%
%\nameparts{#1} \usebibmacro{index:name}{\sindex[person]]}{\namepartfamily}{‌​\namepartgiven}{\nam‌​epartprefix}{\namepa‌​rtsuffix}}

%\newcommand{\smiley}{:)}

%\renewbibmacro*{index:name}[5]{%
%\usebibmacro{index:entry}{#1}%
%{\iffieldundef{usera}{}{\thefield{usera}\actualoperator}\mkbibindexname{#2}{#3}{#4}{#5}}}

% \newcommand{\noop}[1]{}

%remove for final
%\overfullrule=1mm

\newcommand{\tobi}[2]}}
\renewcommand{\S}[1]{\tobi{#1}{\textsc{*}}}

% this volume references
% puts: [this volume]
% already defined: \citetv
%\newcommand{\citepv}[1]{(\citeauthor{#1} \citeyear*{#1} [this volume])}
\newcommand{\citealtv}[1]{\citeauthor{#1} \citeyear*{#1} [this volume]}

%parentheses around example number
\newcommand{\pref}[1]{(\ref{#1})}

% in-text examples

\newcommand{\lnex}[1]{\textit{#1}} %target lang word
\newcommand{\lnlit}[1]{(lit.: `#1')} %literal reading
\newcommand{\lnlat}[1]{(#1)} % latinization
\newcommand{\lntrans}[1]{`#1'} %translation
\newcommand{\lnexl}[2]%
{\lnex{#1}{} \lnlat{#2}} % ex with latinization
\newcommand{\lnexlat}[3]{\lnex{#1}{} \lnlat{#2}{} \lntrans{#3}} % ex with latinization and tranl.

%ch01
\newcommand{\co}[1]{\mbox{\textbf{#1}}}

%ch09

\newcommand{\cyrbulg}[1]{\begin{otherlanguage*}{bulgarian}#1\end{otherlanguage*}}


%ch10
\newcommand{\nlp}{{\small NLP}}
\newcommand{\mwe}{{\small MWE}}
\newcommand{\rae}{{\small RAE}}
\newcommand{\lvc}{{\small LVC}}
\newcommand{\pos}{{\small P}o{\small S}}
%\newcommand{\todo}[1]{ \textcolor{red}{#1} }

%\renewcommand{\labelenumi}{\theenumi}
%\ainamefmt{{vv}{ll}{, ff}{, jj}} % fullname

\newcommand{\biberror}[1]{{\color{red}#1}}

\newcommand{\osenovaitem}{--~}
   %% hyphenation points for line breaks
%% Normally, automatic hyphenation in LaTeX is very good
%% If a word is mis-hyphenated, add it to this file
%%
%% add information to TeX file before \begin{document} with:
%% %% hyphenation points for line breaks
%% Normally, automatic hyphenation in LaTeX is very good
%% If a word is mis-hyphenated, add it to this file
%%
%% add information to TeX file before \begin{document} with:
%% %% hyphenation points for line breaks
%% Normally, automatic hyphenation in LaTeX is very good
%% If a word is mis-hyphenated, add it to this file
%%
%% add information to TeX file before \begin{document} with:
%% \include{localhyphenation}
\hyphenation{
    Beck-man
    Ngu-yen
    back-chan-nel
    back-chan-nels
    mo-not-o-nous
    ste-reo-typ-i-cal
}

\hyphenation{
    Beck-man
    Ngu-yen
    back-chan-nel
    back-chan-nels
    mo-not-o-nous
    ste-reo-typ-i-cal
}

\hyphenation{
    Beck-man
    Ngu-yen
    back-chan-nel
    back-chan-nels
    mo-not-o-nous
    ste-reo-typ-i-cal
}

   \boolfalse{bookcompile}
   %\usepackage{xr}
    %\externaldocument[\usepackage{xr}
    %\externaldocument[1-]]{chapterI}
   \togglepaper[1]%%chapternumber
}{}

\begin{document}
\maketitle \label{ch1}

\section{Introduction} 
Epenthesis, or the insertion of a non-etymological segment, has been an object of linguistic inquiry for centuries. The specific terms used to refer to the insertion of a segment at the beginning of a word (\textit{pro(s)thesis}), within a word (\textit{anaptyxis} or \textit{svarabhakti}), or the end of a word (\textit{paragoge} or \textit{epithesis}), reflect the study of these processes within the Sanskrit, Greek, and Latin traditions (see \citealt{Kiparsky2022}, \citealt{Sen2022}, \citealt{OnigaRe2021}, etc.). Recently, the notion of insertion of non-etymological material has been expanded to include patterns that are not transparently phonologically motivated, but are conditioned phonetically, morphologically, morphosyntactically, and lexically. In the most familiar cases, the trigger for insertion is phonological in nature – for example, to reduce complexity in syllable structure – and the quality of the inserted segment is also determined phonologically – a featurally simple or predictable segment is used. However, an increasing body of research identifies cases where the motivation for insertion and the choice of the inserted segment lie beyond phonology. We outline some of these studies below: section \ref{intrusion} considers phonetic “intrusions”, section \ref{can_ep} reviews canonical phonological epenthesis, and section \ref{morph} looks at morphological and morphosyntactic interactions. This is followed in section \ref{outline} by an overview of the articles included in this volume. These articles are a selection of the papers presented at the virtual workshop “Epenthesis and Beyond”, held at Stony Brook University in 2021.

\section{Phonetic considerations: intrusion} \label{intrusion}
Not all insertions are borne out of the same processes. While some segments are inserted phonologically, some are considered phonetic artifacts (see \citealt{Ohala1974}, \citealt{Alietal1979}, \citealt{Hall2006}, to name a few). Systematic accounts of the distinction between phonologically and phonetically inserted segments is fairly recent (see \citealt{Hall2004}) and traditionally binary. However, in the last few years, some authors (\citealt{Griceetal2018}, \citealt{Karlin2021}, \citealt{Hutinetal2021}, \citetv{chapters/08.Hall}) have challenged a binary distinction, stating that the dichotomy between epenthetic and intrusive segments is not always clear-cut.

Some inserted consonant-like sounds appear to be the result of a co\hyp articulatory process. For example, \citet[359]{Ohala1974} claims that [t]-insertion in a word like \textit{false} /fɔls/, resulting in [fɔlts], is the result of articulatory transitions. In fact, the author states that the articulatory points of contact between /l/ and /s/ are somewhat complementary, so the transition between the two sounds might cause complete stoppage of the airflow, resulting in a t-like sound. Likewise, \citet{Alietal1979} note that a stop might appear in nasal-fricative clusters, such as \textit{warmth} /waɹmθ/, which can be pronounced as [waɹmpθ], and that the stop-insertion is due to delays in articulatory transitions. 

In general, the consensus from these authors on English consonant intrusion seems to be that the “intrusive consonants” that appear in consonant clusters are optional (not always inserted), but there are some constraints on the process. \citet{Ohala1974} reports that intrusive consonants are less likely to appear if the following consonant is voiced, while \citet{Alietal1979} note that the inserted stop tends to be homorganic with the cluster’s place of articulation. \citet{FourakisPort1986} find that intrusive and ``intended" stops are phonetically different, as the first is significantly shorter than the second. Regarding the perception of intrusive consonants, \citet{WarnerWeber2001} find that these inserted stops are perceived by listeners only about 50\% of the time in nonce words, and they are perceived differently, as shown by longer reaction times with intrusive stops. Finally, \citet{Wetzels1985} claims that the addition of an intrusive consonant leaves the original syllable structure unaltered.

Vowel insertion, and its phonological status, is a much more hotly debated topic, as it has significant consequences for syllable structure. Some inserted vowels, called “excrescent vowels” or “intrusive vowels”, have been argued to be a phonetic artifact, as they lack target gestures and do not interact with phonological processes. While these kinds of vowels have been identified in studies as early as \citet{MattesonPike1958} and \citet{Levin1987}, it is not until \citet{Hall2004, Hall2006, Hall2011} that we have a systematic account of the difference between intrusive and epenthetic vowels. According to \citet{Hall2006}, intrusive vowels are distinguished from epenthetic vowels because (i) they have a schwa-like quality or are a copy of a neighboring vowel, (ii) they occur in heterorganic clusters where at least one consonant is voiced, (iii) are usually optional, and (iv) speakers might not be aware of the presence of this vowel. Moreover, (v) their function is not to repair an illicit syllabic structure, but they might in fact serve a perceptual function, and they do not constitute the nucleus of a syllable.

Epenthetic vowels, on the other hand, (a) have a fixed or copied quality, (b) they occur in marked clusters that might or might not be voiced, (c) their presence is obligatory, and (d) speakers are usually aware of them. Moreover, (e) they constitute the nucleus of a syllable and are inserted to repair an illicit syllabic structure (\citealt{Hall2006, chapters/08.Hall} [this volume], for a more complete review of their diagnostics). According to the author, intrusive vowels and epenthetic vowels are, therefore, fundamentally different in their nature, the former being the result of articulatory retiming, while the latter being the result of a repair mechanism. Although intrusive vowels look like vowels phonetically, they do not participate in phonological processes, and they cannot be syllable nuclei. Since \citet{Hall2006}, there has been a plethora of studies looking more closely at inserted vowels and exploring this concept further, both in well-studied and understudied languages (see for example \citealt{Bellik2019a, chapters/07.Bellik} [this volume] for work on Turkish, \citealt{Burkeetal2019} on Lamkang, \citealt{Cavirani2015} on two Lunigiana dialects, \citealt{Griceetal2015Berber} on Tashlhiyt Berber, \citealt{Heselwoodetal2015} on Libyan Arabic, \citealt{Karlin2021} on Finnish, \citealt{LancienCôté2019} on Quebec French, \citealt{Nogita2011} on Japanese, \citealt{Pariente2010} on Sephardic Hebrew).

Hall’s specific research and the abovementioned studies address the dichotomy between epenthetic and intrusive vowels in consonant clusters. The nature of an intrusive vowel is in fact tied to the fact that it occurs in a consonant cluster, since it is due to articulatory retiming. However, there have been studies that claim that intrusive-like vowels can be found in word-final position as well. One example is Cavirani’s (\citeyear{Cavirani2015}) work on the Lunigiana dialects of Italy, where Pontremolese and Carrarese are claimed to have schwa-like vocalic releases after consonants in word-final position. Another example is Italian where a vowel-like element is found after consonant-final words. Although the analysis is not undisputed (see \citealt{Bafile2002,Bafile2005}, \citealt{Broniś2016} and \citealt{Passino2008}, who treat this vowel as the nucleus of a syllable, and therefore epenthetic), some accounts treat this vowel as non-syllabic (see \citetv{chapters/09.HamannMiatto}, \citealt{Miattoetal2019}, \citealt{Miatto2020}, \citealt{Repetti2012}), and therefore much closer to intrusive than epenthetic vowels. This is due to their similarities with intrusive vowels in quality (highly variable, usually transcribed as schwa, which is not a phoneme in the language), optionality (the vowel may or may not be inserted), and awareness of insertion (speakers are not aware that they are producing it and do not identify it as syllabic) (see in particular \citealt{Miatto2020}).

Recently, many authors have acknowledged that the distinction between epenthetic and intrusive vowels is not clear cut. For example, while \citet{Griceetal2018} recognize that word-final inserted vowels in Italian resemble intrusive vowels in their appearance and seem not to be syllable nuclei, they claim that the inserted vowels are influenced by phonological and intonational pressures, and they state that the dichotomy between epenthetic and intrusive vowels is not completely satisfactory. In the same vein, \citet{Karlin2021} reports that Finnish inserted vowels that have been previously called epenthetic, are actually intrusive vowels that are becoming phonologized. Likewise, \citet{Hutinetal2021} claim that French word-final inserted schwas, while sharing many properties with intrusive vowels, are ultimately epenthetic. These very recent studies, as well as \textcitetv{chapters/08.Hall}, all agree that the distinction between intrusive and epenthetic is not always clear.

\section{Phonology and epenthesis: canonical epenthesis} \label{can_ep}
Epenthesis has usually been studied as a phonological process, and focus has been on the properties of the structure that is repaired and on the quality of the inserted segment(s). The generalization appears to be that unmarked segments are inserted to improve well-formedness. This type of epenthesis is called phonological or canonical epenthesis.\footnote{Epenthesis has been described in sign languages as a means of ensuring syllable well-formedness \citep{Brentari1990}.} In this volume, we use the terms “marked", ``unmarked", and ``markedness” in the ways they are usually employed in the phonological literature, although we are fully aware of their complexity and the controversy that surrounds them. A question that can be raised in relation to epenthesis is what structures and segments are unmarked, under the assumption that unmarked vowels and consonants are inserted to repair marked structures. For example, /ə/ is universally an unmarked vowel and inserted to break illicit consonant clusters (\citealt{DavidsonStone2003}). Similarly, pharyngeals, the least marked consonants, such as glottal stop, are inserted to repair onsetless syllables \citep{Lombardi2002}. Both cases espouse the Emergence of the Unmarked (TETU; \citealt{McCarthyPrince1994}) and are related to phonological naturalness.

On the other hand, \citet{Haspelmath2006} provides an overview of markedness in different subfields of linguistics, such as phonetics, phonology, morphology, and semantics and gives alternative approaches to phonological (un)markedness. For instance, unmarked segments are inserted because they are more frequent and thus more predictable \citep{Hume2004}. Another perspective highlights the role of language change in accounting for the appearance of synchronically unexpected segments \citep{Blevins2004}, which is to be discussed in more detail later in this section.

Epenthetic processes usually result in phonological structures that are more acceptable in a given language, and have been discussed in the context of TETU, within the framework of Optimality Theory (OT, \citealt{PrinceSmolensky1993}). While OT is the most common contemporary framework used when analyzing epenthesis, Government Phonology (\citealt{Kaye1990}, \citealt{Kayeetal1990}) and the related CVCV Phonology \citep{Scheer2004} frame epenthesis, or the lack thereof, as the ability of a certain language to govern empty nuclei. Within this approach, final consonants are not thought of as codas, but as onsets of empty nuclei (\citealt{HarrisGussmann1998}). Therefore, if a language that does not allow for word-final consonants borrows loanwords with word-final consonants, its inability to govern empty nuclei will make it so that the nucleus will be phonetically realized, hence ``epenthesized". See, for example, \citet{Bafile2001, Bafile2002} for an analysis of Emilian and Florentine dialects within a Government Phonology Framework, and \citet{Passino2008} for an analysis of word-final gemination and schwa insertion in Italian within a CVCV framework.

No matter the framework, vowel epenthesis has been analyzed as improving syllable structure by resolving clusters or avoiding coda consonants, and the study of vowel epenthesis has played a key role in our understanding of syllable structure (\citealt{Broselow1982}, \citealt{Ito1989}, \citealt{Piggott1995}, just to name a few). Consonant epenthesis also illustrates this point: it has been argued that epenthesis between vowels \REF{mor} and before a word-initial vowel \REF{rhys} is a means of filling a missing onset resulting in an optimal CV(C) syllable.

\NumTabs{8} %divides text into 8 tab spaces
\begin{exe}
		\ex
			\begin{xlist}
				\ex \label{mor}  Persian\tab  /sekei/\tab [sekeʔi]\tab   ‘coin’-\textsc{indf}\tab \citep{Moradi2017}\\
				    Koryak\tab  /alaal/\tab [alaʔal]\tab ‘summer’\tab \tab \citep{Kurebito2004}\\
                    English\tab  \textit{druid}\tab [dɹuwɪd]\\
                    English\tab \textit{fire}\tab [fajəɹ]\\

                    \ex \label{rhys} Persian\tab /abru/\tab [ʔabru]\tab 'eyebrow'\tab (\citealt{DehghanKambuziya2012})\\
                    Koryak\tab /ajatək/\tab [ʔajatək]\tab ‘to fall’\tab (\citealt{Kenstowicz1976}, \citealt{Lombardi2002})\\
                    English\tab \textit{apple}\tab [ʔæpəl]
			\end{xlist}
	\end{exe}

The non-etymological segment can be analyzed as inserted by a rule (\citealt{Zwicky1972}, \citealt{Dinnsen1980}) or resulting from the interaction and relative ranking of constraints. The latter is encoded in OT, which is the most frequently used framework in the phonology literature for epenthesis, with \textsc{dep} and \textsc{max} constraints that militate against insertion and deletion, respectively, in output forms in comparison to input forms. \citet{McCarthyPrince1995} conceive of these constraints as applying to segments, and \citet{Lombardi1998}, \citet{Krämer2001}, \textcitetv{chapters/06.Uffmann}, and others extend to include insertion/deletion of features.

The target of the insertion process is a marked structure, and the quality of the epenthetic segment is often claimed to be unmarked. The “unmarked” quality of the inserted segment was analyzed as determined by language-specific rules in early generative work (\citealt{Zwicky1972}, \citealt{Dinnsen1980}), and more recently by general principles such as the following universal hierarchy based on Place within the OT framework: \textsc{*Dorsal, *Labial >> *Coronal >> *Pharyngeal} \citep{Lombardi2002}. Lombardi’s hierarchy accounts for the insertion of a glottal stop, the least marked consonant, and one that is frequently observed cross-linguistically; if the least marked consonant is not available in a language, the next least marked consonant is inserted. Another type of hierarchy is Uffmann’s (\citeyear{Uffmann2007a}a, \citeyear{Uffman2007b}b), in which syllabic positions and sonority are taken into consideration, and the optimal segment in a particular position is selected by the relative ranking of key constraints such as \textsc{Dep}(feature) and \textsc{*Multiple}. 

A perception-based account is proposed by \citet{Jun2015, Jun2021}, following Steriade’s (\citeyear{Steriade2001, Steriade2009}) P-map hypothesis: n-insertion in Korean compounds (for example, /som-ipul/ [somnipul] ‘cotton sheet’) is due to the fact that [n] is the perceptually least marked consonant before the high front vocoids /i/ and /j/ \citep[34]{Jun2021}. Alternatively, the inserted segment can be a copy of a nearby segment \citep{KittoLacy1999} or “split” from an adjacent input segment \citep{Staroverov2014}.

There are other approaches in addition to markedness-based ones. Historical explanations deal with cases in which the quality of the inserted segment is clearly not “unmarked” from a synchronic perspective. Evolutionary Phonology (\citealt{Blevins2004,Blevins2008}) in particular accounts for phonologically opaque phenomena such as the emergence of [x] in Land Dayak \citep{Blevins2008}. This is explained diachronically by a series of phonological processes that have taken place over time.

We have just discussed the insertion of a segment for phonetic or phonological reasons. The inserted segment can, over time, become part of the lexical item itself, as in the case of the /b/ in French \textit{trembler} ‘tremble’ < Latin \textit{tremulare}. Furthermore, it can even become a morphological marker, e.g., the active imperfective past morpheme in Modern Greek which originated as an epenthetic segment (\citealt{JosephRalli2021}). In the next section, we explore other ways in which epenthesis interacts with the grammar of the language beyond phonetics and phonology.

\section{Morphological and morphosyntactic interactions} \label{morph}
While there has been general agreement on the properties of canonical epenthesis, there are cases in which phonetics and phonology alone cannot account for the phenomena. \citet{Żygis2010} provides an overview of epenthetic phenomena, focusing on consonant insertion, and notes that, in addition to canonical epenthesis, there exists a category of “grammatical insertions”, consisting of morphologically, syntactically, and morphosyntactically conditioned insertion processes. Her paper provides a typology of consonantal insertions and reviews previous treatments, and she concludes that there exist very different analyses of consonant insertion because the processes that they model are fundamentally different. 

\citet{Staroverov2014} similarly makes a distinction between phonological epenthesis (which in his work is a result of Splitting, the operation that draws a correspondence between one input segment and multiple output segments), and morphologically restricted consonant-zero alternations. This is directly related to the difference between phonological epenthetic segments (which are predictable; in Staroverov’s work, they share features with segments directly surrounding them) and morphologically restricted insertions (which are not predictable, so they permit a greater variety of segments).

Recent studies have focused on the factors influencing these other types of insertions. In some cases of conditioned epenthesis, the result is a phonologically improved structure, and one way to account for these insertions is to represent such segments as “ghosts” which are part of the underlying form and surface only when their presence is phonologically optimizing, as in French liaison: [le \underline{z} ami] ‘the friends’ vs. [le \_ tami] ‘the sieves’.\footnote{“Liaison” segments are segments that were deleted historically, and their quality is therefore diachronically explicable.} This is a purely phonological solution, circumventing the need for reference to morphosyntactic conditions, thus allowing for a strictly modular view of phonology (see, for instance, the analysis of Italian articles in \citealt{Faustetal2018}). Also, since these segments are an example of deletion but not insertion, especially when they disappear on the surface, the quality of the segments is therefore explicable in a diachronic aspect (i.e., not something that is inserted synchronically). Other work, on the other hand, defends the necessity of capturing the conditioning on the processes. In \citet{Zimmermann2019}, French liaison segments are examples of “appearing ghosts”; for the same phenomenon, \citet{Fukazawa1999} and \citet{Pater2010} employ lexically-indexed constraints, while \citet{InkelasZoll2007} argue for a co-phonology approach. Other analyses use mechanisms that more overtly refer to morphological structure in the input. For example, alignment constraints proposed by \citet{Jun2015} and \citet{Blaylock2017} penalize misalignment of morphological structure and phonological structure.

Cases of morphosyntactically conditioned epenthesis that result in phonologically more complex structures are surprisingly widespread. For example, Korean [n]-epenthesis results in the creation of a coda; crucially, [n]-epenthesis marks a morpheme boundary and is exclusive to compounds. Korean sometimes epenthesizes an [s] with the conjunctive suffix {}-\textit{iraŋ} (for example, /pap-iraŋ/ [papsiraŋ] ‘rice and’; \citealt{Kim2022a}). Similarly, we find [s]-epenthesis in diminutives in Spanish (for example, /amoɾ-it-o/ [amoɾsito] ‘love-\textsc{dim}’; \citealt{Kim2022a}) and English (for example, \textit{Betsy} [bɛtsi] ‘Elizabeth-\textsc{dim}’; \citealt{Kim2022b, Kim2022a}). In each of these cases, a closed syllable is created, but syllable-morpheme alignment is optimized.

The quality of the non-canonical epenthetic segment can be determined by a variety of factors: [n] (used in Korean compounds) for perceptual markedness constraints (\citealt{Jun2015,Jun2021}), [s] (in Korean {}-\textit{iraŋ} suffixation, and Spanish and English diminutives) for frequency-based or analogical reasons \citep{Kim2022a}, [o] (in final position in some Romance processes) since it is the morphologically neutral vowel (\citealt{AronoffRepetti2021}), and other segments that are determined historically or lexically \citep{Moradietal2023}. We also find cases of copy epenthesis at morpheme boundaries: the Korean innovative suffix {}-\textit{lʌ} (/sʌul-ʌ/ [sʌullʌ] ‘a person from Seoul’) is formed by adding a copy of the final consonant of the stem \textit{sʌul} ‘Seoul’ to the English agentive suffix \textit{{}-er} \citep{Kim2022c}, and examples of copy epenthesis at a word boundary are observed in many languages: Italian (/tram-elɛttriko/ [tram.me.lɛt.tri.ko] ‘electric tram’), Jeju Korean (/kacuk-os/ [ka.cuk.kot] ‘leather clothes’) \citep{Kim2022a}.

Finally, within the domain of syntax, the realization of a functional head with ``default" phonological material, such as schwa as in the northern Italian dialect of Donceto [(ə) be:v] {}`I drink', has been referred to as “syntactic epenthesis” (\citealt{CardinalettiRepetti2004}).

Recent efforts have been made to provide a uniform account of the diverse epenthesis/insertion processes that have been observed. \citet{Moradi2017} provides an overview of conditioned insertion processes in various languages, which have received different treatments, unifying them under the umbrella of non\hyp canonical epenthesis and identifying the generalizations that characterize all of these phenomena. \citet{AronoffRepetti2021} extend this survey to related processes in many Romance varieties. \citet{Petrovic2023} proposes a non-canonical epenthesis treatment for a pattern in Serbo-Croatian noun inflection, and provides a formalization couched in Boolean Monadic Recursive Schemes (BMRS; \citealt{ChandleeJardine2021}). This work abstracts away from more common approaches partly due to the need to capture the fact that the epenthetic process under consideration is not necessarily phonologically optimizing. The formalism allows for direct reference to the input as well as the output forms, while at the same time the system does not surpass the computational complexity of phonological processes. (See also \citealt{Moradietal2023}.)

\section{Outline of the volume}\label{outline}
The virtual workshop “Epenthesis and Beyond”, held at Stony Brook University September 17--19, 2021, provided a forum for scholars who approach epenthesis and other types of insertion from new perspectives. The Workshop featured five invited speakers, fourteen 20-minute talks, and nine 5-minute blitz talks which substituted a poster session (a format more suitable for a virtual conference, held over Zoom).\footnote{Details of the Workshop can be found at https://www.stonybrook.edu/epenthesis/.} The twelve articles included in this volume, which we summarize below, were selected from the many excellent papers that were presented at the Workshop, and they represent exciting new approaches to epenthesis.

\subsection{Insertion or deletion?}
Any analysis of a phenomenon as epenthesis must, of course, be superior to a competing deletion analysis of the same alternation. The differentiation of deletion and insertion processes is a topic that has sparked substantial discussion in the literature (e.g., \citealt{Morley2015}), and even well-known and frequently referenced examples of epenthesis may merit a preferable deletion analysis (e.g., \citealt{Staroverov2015} on Ajyíninka Apurucayali, also pejoratively referred to as Axininka Campa). The chapters in Part I focus on such issues, and compare and evaluate competing hypotheses.

Words that are variably realized as CVCV or CCV can be analyzed as cases of optional epenthesis (CCV $>$ CVCV) or optional deletion (CVCV $>$ CCV). Hannah Sande investigates such alternations in her contribution “Insertion or deletion? CVCV/CCV alternations in Kru languages” (ch. \ref{ch2}). She offers diagnostics to determine if the alternation is best characterized as deletion or insertion, and concludes that in some Kru languages the alternation is the result of optional deletion (Guébie), while in others it is due to optional insertion (Dida) which Sande characterizes as vowel intrusion rather than epenthesis. She concludes the article with proposals on the diachronic development of these CVCV/CCV alternations within a broader areal context.

The contribution by John Mansfield, Rosey Billington, and Hywel Stoakes, “Vowel predictability and omission in Anindilyakwa” (ch. \ref{ch3}), investigates vowels in the Australian language Anindilyakwa that are fully predictable (/a/ in word-final position) and highly predictable (non-low vowels in word-internal position), and can therefore be considered epenthetic. The authors find that the predictable vowels are frequently omitted in speech, while the unpredictable ones are never omitted; their investigation of a written wordlist is consistent with these findings. They adopt an information-theoretic approach to vowel predictability, which contributes to our understanding of segmental predictability and deletion.

\subsection{Quality of epenthetic vowels}
Another dimension to note when it comes to epenthesis is the quality of the inserted segments. In section \ref{can_ep}, we discussed the cross-linguistically “unmarked” nature of some epenthetic segments (e.g., \citealt{Zwicky1972}, \citealt{Dinnsen1980}, \citealt{Lombardi2002}). However, not all epenthetic segments can be explained by markedness, since there may be more than one epenthetic segment, and more importantly, not all epenthetic segments can be considered “unmarked”. This leads our attention to other new approaches to epenthetic segment quality that are elaborated on in the following three articles. 

Hassan Bokhari’s paper, “The patterning of epenthesis in Urban Hijazi Arabic” (ch. \ref{ch4}), explores epenthesis in a variety of Arabic spoken in Saudi Arabia which has two different types of epenthesis using various epenthetic segments. “Syllable-structure-driven epenthesis” repairs a word-internal illegal string (such as *CVCCCV or *CV:CCV) with epenthetic [a], and “sonority-driven epenthesis” inserts [i], [u], or [a] to repair a word-final consonant cluster of rising sonority. In the latter case, the quality of the inserted vowel is determined by the stem vowel or the place feature of one of the consonants of the cluster.

Edward Rubin and Aaron Kaplan (“Segmental and prosodic influences on Bolognese epenthesis”, ch. \ref{ch5}) report on epenthesis processes in a Romance variety spoken in northern Italy, Bolognese, that also involve more than one epenthetic vowel: [e] is the default segment, and [u] is used before [m] and [v]. Rubin and Kaplan propose that [v] is best treated as a sonorant since it patterns with [m] rather than the labial obstruents [p b f] in epenthesis processes, and since it alternates with [w]. In addition, they note that some obstruent-final clusters that are permitted word-internally, are instead repaired across word boundaries (such as verb-enclitic sequences). Their treatment of these latter cases as epenthesis rather than C/VC clitic allomorphy simplifies the analysis of clitics and unifies it with the analysis of epenthesis. 

The article by Christian Uffmann, “Epenthesis as a matter of Faith” (ch. \ref{ch6}), provides an Optimality Theoretic account of the quality of default epenthetic segments. Cross-linguistically, certain epenthetic consonants (such as glottal stop) and vowels (such as schwa) are very common, and this fact has been accounted for by identifying these segments as “unmarked”, violating few markedness constraints. Instead, Uffmann proposes an analysis based on Faithfulness. These particular segments are optimal because they violate few Faithfulness constraints, i.e., few violations of \textsc{Dep(F)}: schwa is featureless, and glottal stop has laryngeal but not oral features.

\subsection{Phonetics-phonology interface}
Hall’s (\citeyear{Hall2004, Hall2006}) seminal work on the distinction between epenthetic and intrusive vowels has sparked work that strives to determine the phonological status of inserted vowels in different languages, or that challenges the epenthetic-intrusive vowel dichotomy as too restrictive. The articles in this section address such questions with different approaches and frameworks, contributing to our understanding of insertion at the interface of phonology and phonetics. 

Epenthesis strategies within Turkish (Turkic language family) onset and coda clusters differ. Vowel insertion in coda clusters is obligatory, reflected in the orthography, and invariant, and the inserted vowel harmonizes with the preceding lexical vowel. Alternatively, insertion in onset clusters is variable, not reflected in the orthography , and optional, and the inserted vowel is intermediate between a schwa-like vowel and a copy vowel. Using findings from an acoustic study (\citealt{Bellik2018, Bellik2019b},b) and an ultrasound study inspired by \citet{DavidsonStone2003}, Jennifer Bellik (“Gestural characteristics of vowel intrusion in Turkish onset clusters: An ultrasound study”, ch. \ref{ch7}) proposes that coda epenthesis is actual vowel insertion driven by syllable structure constraints, while insertion in onsets should be considered intrusion due to gestural alignment since the inserted vowel differs acoustically from lexical vowels and does not undergo harmony.

Nancy Hall (“Intrusive and epenthetic vowels revisited”, ch. \ref{ch8}) reviews recent work on the distinction between epenthetic (phonologically active) vowels and intrusive (phonologically invisible) vowels, as well as subclasses of each. The main distinguishing feature between the two groups is the presence vs. absence of a new vowel gesture, respectively, but the typology of vowel insertion processes is actually much richer, and other factors need to be considered in order to characterize the full range of insertions.

Silke Hamann and Veronica Miatto expand the typology of intrusive vowels in their paper “Three language-specific phonological interpretations of release bursts and short vowel-like formants” (ch. \ref{ch9}). They show how the same phonetic material (word-final release burst) can have different phonological interpretations in different languages. In American English, a word-final release burst is interpreted as a plosive, and in Korean, as a vowel, despite the lack of vowel-like formants. In Italian, release bursts are followed by vowel-like formants which, however, are not perceived as vowels, making them more akin to intrusive vowels than epenthetic vowels. Hamann and Miatto adopt the Bidirectional Phonetics and Phonology model which distinguishes three levels of phonetic and phonological representations to account for these patterns.

Martin Krämer (“Prokaryotic syllables and excrescent vowels in two Yuman languages”, ch. \ref{ch10}) also discusses the typology of intrusive segments in his investigation of two Yuman languages, Cocopa and Jamul Tiipay (Northwest of Mexico and Southwest of the United States). He analyzes certain structures in these languages as syllables consisting of a consonant or consonants (onset and optional coda) but without a nucleus or a mora, rendering them prosodically invisible. These types of structures have been referred to as degenerate, minor, or semisyllables, but he borrows the term “prokaryote” from biology to propose a new name for these nucleus-less syllables which provide evidence for another type of intrusive vowel.

\subsection{Epenthesis and beyond}
The ultimate goal of our workshop and of this collection of articles is to explore different kinds of insertion and different methods for analyzing epenthesis. The following articles are perhaps the embodiment of our overarching theme. They in fact complete our collection with analyses of epenthesis on understudied languages, using diverse methods and challenging the concept itself of insertion and epenthesis. 

As the title of Michael Ramsammy’s paper suggests, “On the diachrony of lateral epenthesis” (ch. \ref{ch11}) investigates the historical development of a particular type of consonant epenthesis in various languages including English, Motu (Oceanic language of Papua New Guinea), and Hindi. Ramsammy focuses on Hindi /l/-causatives, arguing against analyses involving allophony and analogy, and in support of an analysis whereby epenthetic /j/ changed to /l/ due to sonority optimization.

In their contribution “Textsetting the case for epenthesis in Armenian” (ch. \ref{ch12}), Luc Baronian and Nicolas Royer-Artuso show how textsetting, or the study of the way poets map their text onto a metrical grid, can be used as a tool to better understand the synchronic status of phonological processes such as schwa epenthesis. They use Armenian (Indo-European language family) as a test case, and build on the claim in \citet{Baronian2017} that some cases of Western Armenian schwa are historically derived from epenthesis but are now part of the underlying form, and question whether schwa epenthesis is still productive. They map the lyrics of a song to the beats of the song, and conclude that schwa epenthesis is indeed used productively.

Brett Nelson’s contribution, “Insertion of [spread glottis] at the right edge of words in Kaqchikel” (ch. \ref{ch13}), explores allophonic alternations in non-final and final position in Kaqchikel (a Mayan language spoken in Guatemala): plain stops are realized as aspirated, and non-nasal sonorants are realized as fricatives in final position; other consonants (glottalized stops, fricatives, nasals) do not alternate. Nelson proposes that the feature [+spread glottis] is inserted at the end of the word to mark the prosodic boundary. As a result, plain stops with [+sg] added become aspirated, and continuant (non-nasal) sonorants become obstruents (i.e., fricatives). The other consonants do not alternate because [+sg] would not change the segment (fricatives), or because [+sg] is blocked in certain contexts (glottalized stops, nasals). Nelson highlights the fact that even though [spread glottis] is not a contrastive feature in Kaqchikel, it plays an active role in its phonology.

\section{Conclusion}
The study of epenthesis is continuing to develop in significant ways; as we move forward, it remains a crucial area of research in linguistics. This introductory chapter has provided an overview of the past, present, and future of the study of epenthetic processes, highlighting the different types of epenthesis, including phonetic intrusion, canonical phonological epenthesis, morphological and morphosyntactic epenthesis, and the challenges and questions that arise from each.

The volume is organized into four parts, each of which explores different aspects of epenthesis. Part 1 focuses on the relationship between insertion and deletion processes, while Part 2 focuses on the quality of epenthetic segments. In Part 3, our attention turns to the phonetics-phonology interface and the interaction between the two domains. Finally, Part 4 expands the scope of epenthesis research by investigating understudied languages, employing diverse methodologies, and challenging the fundamental notions of epenthesis itself. These articles illustrate the complexity and richness of epenthesis, and its continued significance to phonological theory.

Overall, this volume seeks to contribute to our understanding of epenthesis and its role in language and linguistic theory, and we hope it will be a valuable resource for scholars interested in this area of research.



\section*{Acknowledgements}
We would like to express our sincere thanks to participants of the Workshop “Epenthesis and Beyond”, the reviewers of the talk abstracts and Proceedings papers, the Department of Linguistics at Stony Brook University, and in particular Alex Yeung for his technical support, and the Offices of the Provost, Vice President for Research, and the Dean of the College of Arts and Sciences at Stony Brook University for providing support for this Workshop.

%\section*{Contributions}
%John Doe contributed to conceptualization, methodology, and validation.
%Jane Doe contributed to the writing of the original draft, review, and editing.

{\sloppy\printbibliography[heading=subbibliography,notkeyword=this]}
\end{document}
