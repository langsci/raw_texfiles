\documentclass[output=paper,colorlinks,citecolor=brown]{langscibook}
\ChapterDOI{10.5281/zenodo.14264538}
\author{Christian Uffmann\affiliation{HHU Düsseldorf}\orcid{}}
\title{Epenthesis as a matter of \textsc{faith}}
\abstract{In Optimality Theory, the observation that default epenthetic segments are sourced from a heavily restricted set of segments across languages, such as the glottal stop [ʔ] for consonants, schwa [ə] or [i] for vowels, is generally analysed as a markedness effect: Optimal epenthetic segments are maximally unmarked segments. This chapter highlights several conceptual and empirical problems with this claim and proposes a faithfulness-based alternative instead: Optimal epenthetic segments involve minimal epenthesis at the level of distinctive features, that is, minimal violations of \textsc{Dep(F)}. We will show how this proposal accounts for crosslinguistically common epenthetic segments as segments that are in some way underspecified vis-à-vis other segments, and it will be shown how the problems encountered in a markedness-based approach can be resolved.}


\IfFileExists{../localcommands.tex}{%hack to check whether this is being compiled as part of a collection or standalone\
   \addbibresource{../localbibliography.bib}
   \usepackage{langsci-optional}
\usepackage{langsci-gb4e}
\usepackage{langsci-lgr}

\usepackage{listings}
\lstset{basicstyle=\ttfamily,tabsize=2,breaklines=true}

%added by author
% \usepackage{tipa}
\usepackage{multirow}
\graphicspath{{figures/}}
\usepackage{langsci-branding}

   
\newcommand{\sent}{\enumsentence}
\newcommand{\sents}{\eenumsentence}
\let\citeasnoun\citet

\renewcommand{\lsCoverTitleFont}[1]{\sffamily\addfontfeatures{Scale=MatchUppercase}\fontsize{44pt}{16mm}\selectfont #1}
  
   %% hyphenation points for line breaks
%% Normally, automatic hyphenation in LaTeX is very good
%% If a word is mis-hyphenated, add it to this file
%%
%% add information to TeX file before \begin{document} with:
%% %% hyphenation points for line breaks
%% Normally, automatic hyphenation in LaTeX is very good
%% If a word is mis-hyphenated, add it to this file
%%
%% add information to TeX file before \begin{document} with:
%% %% hyphenation points for line breaks
%% Normally, automatic hyphenation in LaTeX is very good
%% If a word is mis-hyphenated, add it to this file
%%
%% add information to TeX file before \begin{document} with:
%% \include{localhyphenation}
\hyphenation{
affri-ca-te
affri-ca-tes
an-no-tated
com-ple-ments
com-po-si-tio-na-li-ty
non-com-po-si-tio-na-li-ty
Gon-zá-lez
out-side
Ri-chárd
se-man-tics
STREU-SLE
Tie-de-mann
}
\hyphenation{
affri-ca-te
affri-ca-tes
an-no-tated
com-ple-ments
com-po-si-tio-na-li-ty
non-com-po-si-tio-na-li-ty
Gon-zá-lez
out-side
Ri-chárd
se-man-tics
STREU-SLE
Tie-de-mann
}
\hyphenation{
affri-ca-te
affri-ca-tes
an-no-tated
com-ple-ments
com-po-si-tio-na-li-ty
non-com-po-si-tio-na-li-ty
Gon-zá-lez
out-side
Ri-chárd
se-man-tics
STREU-SLE
Tie-de-mann
}
   \boolfalse{bookcompile}
    \togglepaper[6]
}{
}

\begin{document}
\maketitle \label{ch6}

\section{Introduction}

Crosslinguistically, the set of possible default segments in epenthesis seems quite heavily restricted. Typical default consonants are glottals, especially the glottal stop [ʔ], while typical vowels are schwa [ə] or [i]. This raises the question of what principles restrict the set of possible epenthetic segments, or put differently: what makes [ʔ] or [ə] a good epenthetic segment?\footnote{There is the open question to what extent there are also marked or `unnatural' epenthetic segments (see e.g. \citealt{vauxsamuels17}). In this chapter I will leave the question open but simply note that the claim is contentious.}

Before we continue, a few brief words of clarification. When I discuss default epenthesis in this chapter, I mean the phonologically motivated insertion of a contextually invariant segment. By contextual invariance I mean that segment quality is not influenced by adjacent segments (as in vowel copy or glide insertion). In addition, epenthesis has to be phonologically motivated, that is, epenthesis is prosodically or phonotactically optimising. This includes cases of consonant epenthesis to resolve hiatus or to satisfy an onset requirement, or cases of vowel epenthesis to break up illicit consonant clusters or to satisfy coda requirements. Crucially, this definition excludes cases of epenthesis that are either morphologically conditioned or morphologically restricted (such as cases of epenthesis that only occur with certain affixes).\footnote{For an overview of other types of epenthesis and a discussion of phonological and morphological epenthesis, the reader is referred to \citet{zygis10}.} 

As an example of phonologically motivated epenthesis of a default segment, consider glottal stop epenthesis in German. A glottal stop is inserted (a) in hiatus position, before a stressed syllable, (b) word-initially, and (c) stem-initially (see the examples in (\ref{glottal}).


\TabPositions{2.5cm,5.5cm,8cm}
\begin{exe}
    \ex \label{glottal} Glottal stop epenthesis in German \citep{alber01, wiese98}
    \begin{xlist}
        \ex\relax [ʁuˈʔiːn]   \tab  `ruin'    \tab [koˈʔɑːlɐ]\tab `koala'
        \ex{} [ʔɛlç]        \tab `moose'    \tab [ʔoˈʔɑːzə] \tab `oasis'
        \ex{} [ˈʔɑpˌʔɑʁtɪç]  \tab `deviant'  \tab [ˈmɪtˌʔɑʁbɑɪt] \tab `cooperation'
    \end{xlist}
\end{exe}
% \todo{check ɑaɐ against original submission}


\NumTabs{4}

So why is the glottal stop inserted in German (and many other languages)? In Optimality Theory (\citealt{ps93}, henceforth OT), the standard answer has been that default epenthesis constitutes a case of the emergence of the unmarked: optimal epenthetic segments are universally unmarked segments \citep{lombardi02, lombardi03, delacy06}. In this chapter, I want to propose a different account of default epenthesis, one that is based on faithfulness instead, more precisely feature faithfulness. Optimal epenthetic segments are not unmarked; instead they insert as few distinctive features as possible. 

The next section will briefly introduce the markedness approach to epenthesis and raise a few questions and problems with respect to this approach. \sectref{uffmann:epfaith} will introduce the alternative, the faithfulness-based account and discuss how it addresses the questions and problems raised by the markedness approach. In \sectref{uffmann:disc} I will discuss how the faithfulness-based approach fits in with other approaches to epenthesis and conclude.


\section{Epenthesis of the unmarked?}

In Optimality Theory, default segment epenthesis is commonly analysed as a markedness effect, thereby also explaining why the set of possible epenthetic consonants seems to be quite heavily restricted \citep{lombardi02, lombardi03, delacy06}. Default segment epenthesis is epenthesis of the least marked segment, following universal markedness hierarchies \citep{lombardi02, lombardi03} or markedness scales \citep{delacy06}. De Lacy is adamant that it can only be markedness that is responsible for the selection of the optimal epenthetic segment, as the other main constraint family, faithfulness, cannot play a role in this selection: All segments violate the segmental anti-insertion constraint \textsc{Dep} while vacuously satisfying the feature faithfulness constraint \textsc{Ident} -- since no features in the inserted segment have an input correspondent. Let us now look at how this model accounts for the limited set of segments found as default epenthetic segments, that is glottal segments, especially [ʔ], in the case of consonants and [ə, i] in the case of vowels. 

To account for glottals as default epenthetic consonants, \citet{lombardi02} and \citet{delacy06} propose to extend the established place markedness hierarchy, according to which [coronal] is less marked than [labial] and [dorsal] by adding [pharyngeal] (Lombardi) or [glottal] (de Lacy) at the bottom of the hierarchy. Glottals are thus the least marked consonants in terms of place of articulation and therefore selected for epenthesis. 

This approach also makes a second prediction, namely that in cases where glottals are unavailable for independent reasons, coronal epenthesis will be found instead, and it is here that we find a first potential wrinkle with the markedness approach to epenthesis: Coronal default epenthesis is only marginally attested, if at all. The textbook case of [t]-epenthesis in Axininca Campa has been convincingly reanalysed as deletion in the inverse context \citep{staro15}, and the other examples \citet{lombardi02} mentions also generally do not qualify as instances of purely phonologically driven default epenthesis or are amenable to different analyses, by Lombardi's own admission. The markedness approach thus overpredicts variation in consonant epenthesis.

Quite the opposite situation holds when we look at default epenthetic vowels, where considerable variation is found. \citet{lombardi03} nevertheless attempts to provide a markedness-based account of this variation. She argues that [ɨ] is the least marked vowel, followed by [ə] and then [i, a], and if one vowel is unavailable in a language, the next one will be selected. This finding again translates into universal markedness scales. The unmarkedness of the central vowels is formalised by stating that [+back, --round] vowels are least marked (central vowels being analysed as back unrounded), and the preference of [i, ɨ] is analysed as the preference of high vowels over mid vowels. Yet, despite allowing variation, some attested default vowels are not predicted by Lombardi's model, such as [e], found for example in Spanish and Gengbe \citep{Archangeli1988, abaglo-archangeli} or Bolognese \parencitetv{chapters/05.RubinKaplan}. \citet{delacy06} concedes that markedness only has a limited role to play in the selection of default epenthetic vowels (mostly by excluding rounded vowels) and suggests sonority as an additional factor, with different languages preferring high-, low- or mid-sonority vowels. We thus see the opposite situation for vowels than for consonants: Markedness alone underpredicts the range of attested epenthetic vowels, and this raises the question of where this asymmetry between consonants and vowels comes from: Why is there considerably more crosslinguistic variation regarding default vowels compared to consonants?

It also raises a more general question: What evidence is there, outside epenthesis, that default segments actually are the crosslinguistically least marked segments?\footnote{Note also that Lombardi has to classify pharyngeals as the least marked consonants and back unrounded vowels as the least marked vowels, which is especially surprising, as the general assumption is that rounding is unmarked on back vowels, and that back vowels are more marked than front vowels.}  The concept of markedness is, of course, complex (see e.g. \citealt{rice07} for an overview), and a detailed discussion would go beyond the scope of this chapter, but let us use crosslinguistic frequency as a proxy for evaluating the markedness of a segment. Frequency is a well-motivated diagnostic especially in the OT framework, as we should expect segments to appear in many languages when the constraints against them are universally very low ranked  (given that there are no conditions on inputs, a principle known as Richness of the Base, and phoneme systems follow epiphenomenally from constraint rankings). I thus used the phoneme database PHOIBLE \citep{phoible} to establish how crosslinguistically common default epenthetic segments are in phoneme systems (\tabref{tab:1:frequencies}).


\begin{table}
\caption{Crosslinguistic frequencies of default segments in PHOIBLE}
\label{tab:1:frequencies}
 \begin{tabular}{c c c}
  \lsptoprule
             segment & in \% of languages  & rank \\
  \midrule
  {}[ʔ]  &   37  &    19  \\
  \midrule
  {}[ɨ]  &   16  &    17  \\
  {}[ə]  &   22  &    11  \\
  {}[i]  &   99  &    1  \\
  \lspbottomrule
 \end{tabular}
\end{table}

We can see that the glottal stop is crosslinguistically not particularly frequent, found in little more than a third of languages, being only the 19th most frequent consonant, a somewhat disappointing showing for what is supposedly the least marked consonant of all.

The results for vowels are even more perplexing. The allegedly least marked vowel, [ɨ], is rather infrequent, found in only 16\% of all languages surveyed, and schwa only fares marginally better. [i] aligns well with the markedness approach, though: It is the most frequent of all vowels and practically universally present in vowel inventories. Otherwise, frequencies are the opposite of what the mar\-kedness scale should lead us to expect,  the least marked vowel (according to Lombardi) actually being the least frequent. 

It is not just crosslinguistic frequency, however. In languages that have glottal segments or schwa, these are often restricted in their distribution. In English and German, for example, schwa is found in unstressed syllables only, while the phonemic glottal /h/ is restricted to word- or foot-initial singleton onsets. Such contextual restrictions are also surprising for allegedly unmarked segments.

Moreover, in some languages these segments are in fact  restricted to epenthetic contexts. There are languages in which a default epenthetic segment is not otherwise found in the phoneme inventory, for example [ʔ] in German \citep{alber01, wiese98} or schwa in Italian \citep{repetti12}, Western Aramaic \citep{eid21} and Anindilyakwa \parencitetv{chapters/03.Mansfieldetal}. As \citet{kramer06} first noticed, this creates a ranking paradox: In order to be selected as the optimal epenthetic segment, the constraints against that segment have to be ranked low(est), while in order to be excluded elsewhere (if posited as underlying), the same constraints need to rank relatively high in order to prevent the segment from surfacing faithfully. So in German the fact that there is no phonemic glottal stop would standardly be analysed by assuming a high-ranked constraint *[ʔ]. In order to be epenthetic, however, *[ʔ] needs to rank below all other segmental (consonantal) markedness constraints.\footnote{Krämer's analysis relies on Comparative Markedness. Alternatively, one could invoke positional markedness to prohibit the segment from all contexts except those where it happens to be epenthetic. Both approaches look unsatisfying, though. While Krämer has to rely on essentially arbitrary rerankings of what should be universal markedness scales, the alternative approach has to stipulate that the epenthetic segment is only permissible in epenthesis contexts via  a brute force mechanism.} The fact that the default epenthetic segment may not be permissible elsewhere in a language thus poses a serious problem for the markedness-based approach to epenthesis.

There is one final issue with the markedness-based approach I want to discuss here. Default epenthesis is not the only available epenthesis strategy. Instead, epenthetic segment quality 
can also be determined, fully or partially, by spreading or copying features from adjacent segments.\footnote{I will use the term `spreading' without necessarily implying autosegmental spreading but using it as a theory-neutral term instead.} There are cases of vowel copy and vowel harmony (see e.g. \citealt{stantonzukoff}), there is spreading from vowels to consonants, as in glide insertion, where the glide typically agrees in backness with a neighbouring vowel, and there are also cases where consonants spread features to vowels, as in labial attraction, where vowels are rounded next to consonants (as in Bolognese; \citetv{chapters/05.RubinKaplan}).

The markedness-based approach cannot account for these alternative types of epenthesis, and additional mechanisms have to be introduced to handle these, without there being a connection between the two types of epenthesis or a principled explanation when which kind is found. That such a connection exists is clear, however, from cases where the different types of epenthesis interact. \citet{uffbook, uff06} shows that in loanword adaptation, all three processes -- vowel harmony, consonantal spreading, and default epenthesis -- are frequently found. This is not limited to loanword adaptation, however. A case in point is Urban Hijazi Arabic, where we find all three strategies in vowel epenthesis into final clusters of rising sonority (for a detailed discussion, see \citetv{chapters/04.Bokhari}). The relevant data from \citet{almohanna21} are given in (\ref{epuha}); generalisations are my own.

\ea Epenthesis in Urban Hijazi Arabic \citep{almohanna21}
\label{epuha}
\ea \begin{tabular}{p{1cm}p{1cm}p{2cm}p{1cm}p{1cm}l}
{}ʔisim & /ism/  & `name'    &ʃukur & /ʃukr/& `gratitude'
\end{tabular}
\ex \begin{tabular}{p{1cm}p{1cm}p{2cm}p{1cm}p{1cm}l}
{}ʔakil   & /akl/& `food'    &madiħ &/madħ/ & `praise'
\end{tabular}
\ex \begin{tabular}{p{1cm}p{1cm}p{2cm}p{1cm}p{1cm}l}
{} tamur   &/tamr/& `dates'    &baħar &/baħr/ &`sea'
\end{tabular}
\z 
\z

The data show that vowel copy is the most common strategy, but it is restricted to the high vowels [i, u] (examples in (a)). When the stem vowel is /a/, however (Urban Hijazi Arabic has the common Arabic 3-vowel system), as in (b), we generally find epenthesis of default [i] (also found in other epenthesis contexts), unless the epenthetic vowel is preceded by a labial or pharyngeal consonant; in these cases we find spreading of the consonantal place feature, yielding [u] after labials and [a] after pharyngeals (c). How can such a conspiracy of epenthesis strategies be analysed insightfully, if default epenthesis and copy or spreading epenthesis constitute separate phenomena?

To summarise the foregoing, the markedness approach into epenthesis runs into a number of empirical and theoretical problems, listed here again:

\begin{itemize}
    \item There is an asymmetry between consonants and vowels: the markedness approach predicts more variation regarding consonant epenthesis than is attested, but cannot account for the full range of variation attested for default vowels.
    \item Typologically common epenthetic segments are probably not unmarked; they are crosslinguistically not very frequent and often positionally restricted in languages that have them.
    \item The default vowel or consonant in a language may otherwise be absent from that language's phoneme inventory.
    \item There is no unified approach to default and copy epenthesis.
\end{itemize}

The next section will therefore introduce an alternative proposal, arguing that default epenthesis is not based on markedness but on faithfulness instead, more precisely on feature faithfulness. I will propose that optimal epenthetic segments minimally insert features, and I will then show how this approach can account for the problems raised above.



\section{Epenthesis of the faithful!} \label{uffmann:epfaith}

The alternative proposal I am going to outline in this section is that epenthesis in general minimally violates the feature faithfulness constraint \textsc{Dep}(F) \citep{zollphd}: when segments are inserted, a minimal amount of features is inserted along with the segment. Default segment epenthesis then is not a case of the emergence of the unmarked but a case of the emergence of the unspecified or underspecified. This, of course, presupposes that segments differ in their featural complexity, so that less complex segments are better candidates for epenthesis. I will motivate this assumption in a moment, but let us first take a brief historical detour, as the idea that epenthesis is connected to underspecification is not new.

This idea was explicitly pursued in underspecification theory, most notably radical underspecification \citep{Archangeli1988, pulley88}. In this theory, segments are specified minimally; predictable feature values are inserted by rule in the course of the derivation. As a consequence, there is one vowel and one consonant in every language that carries no feature specifications in the underlying representations but receives them via rule instead. When a segment is inserted, only the segmental slot is inserted, and then the feature fill-in rules that exist independently will take care of the segmental content of the epenthetic segment. As a result, the default epenthetic segment of a language is always the segment that is underlyingly unspecified. For example, in a language like Spanish where the default vowel is [e], underlying /e/ is unspecified, and then there is a set of rules that fills in default feature specifications, in Spanish [-high, -back, -round], yielding surface [e], when no specifications are present. Differences in default segments between languages stem from differences in these fill-in rules.

This general approach is, of course, not available in OT. There are no serial derivations in the course of which feature specifications can be inserted, and there are no constraints on inputs that could, for example, prohibit fully specified input segments. The general idea that epenthesis involves underspecified segments can  be transferred to an OT approach, however, as I will argue now.

In OT, segments can also show different degrees of feature specification, in three ways. First, there is considerable evidence that some features at least are privative, not binary (see e.g. \citealt{clemhume, lombardi96, andersonewen, beckjessring}). Privative features are also a prime reason to invoke a class of \textsc{Dep}(F) constraints in addition to (or replacing) \textsc{Ident} constraints, which operate on binary features. Now if there are indeed privative features, different degrees of segmental complexity follow naturally: segments are or are not specified for some feature; some segments will consequently carry very few features.

Second, while radical underspecification with feature fill-in is not an option in OT, the same cannot be said about theories of contrastive underspecification, such as \citet{dresher09}, especially if we assume that underspecified segments remain surface-underspecified (no fill-in; the interface to phonetics interprets underspecified segments, as in \citealt{HallD:2011}). Consequently, not all segments are specified for all features. Features that are redundant for a class of segments (say, [voice] for sonorants) may be absent in the feature specifications of this class, again yielding different degrees of segmental complexity. Feature privativity and contrastive underspecification can also be combined as in \citet{iosadphd}.

Even if readers remain skeptical of these additional assumptions (and motivating them in detail would go beyond the scope of this chapter), there is a third way in which segments can show different degrees of specification or complexity, one that does not need additional and perhaps controversial assumptions about the nature of phonological representations. Some segments are intrinsically less specified than others. It is commonly assumed that glottal segments are not specified for oral features (or do not have an Oral / Supralaryngeal node in feature geometry), thus making them intrinsically less complex than oral segments, while schwa is often analysed as a featureless segment \citep[e.g.][]{oostendorp00, crosswhite04}. I want to argue that this is the true reason for them being good default segments, rather than their unmarkedness. I will now develop this argument and show how it also addresses the questions and problems with the markedness approach discussed above.

For consonants, glottals are selected for epenthesis not because they are the least marked segments (they may in fact be fairly marked), but because they are intrinsically less complex than oral segments, lacking oral features. Consequently, they violate the constraint against feature insertion \textsc{Dep(F)} less than oral segments (no insertion of place and manner features). With this proposal we can also address and explain the ranking paradox mentioned above, that the default epenthetic segment in a language may otherwise be absent from the phoneme inventory of that language. Let us illustrate this with a toy language example, modelled on the German facts mentioned earlier.

To begin with, our toy language has no phonemic glottal stop. Hypothetical underlying glottal stops will be deleted.\footnote{As the alternative, buccalisation, i.e change into an oral segment, is robustly unattested typologically, it is not considered here.} In OT this is easily modelled by ranking a constraint against glottal stops *ʔ above the no-deletion constraint \textsc{Max} (see \tabref{tab:1:tbl1}).


\begin{table}
\caption{Glottal stop deletion...}
\label{tab:1:tbl1}
\ShadingOn
\begin{tableau}{c|s} 
\inp{/aʔ/}     \const*{*ʔ}  \const{\textsc{Max}} 
\cand{[aʔ]}        \vio{*!}  \vio{} 
\cand[\Optimal]{[a]}        \vio{}   \vio{*}  
\end{tableau}
\end{table}


Now assume our toy language also inserts glottal stops to satisfy an onset requirement. We will thus add \textsc{Onset} to the tableau as a trigger for epenthesis. Now what drives the selection of the glottal stop as an optimal epenthetic consonant is the constraint \textsc{Dep(F)} ranked above *ʔ, as in \REF{tableau2:tbl1}.


\begin{table}
\caption{...and epenthesis}
\label{tableau2:tbl1}
\ShadingOn
\begin{tableau}{c|c|c|s|s|s} 
\inp{/aʔ/}    \const{Onset} \const{Dep(F)} \const*{*ʔ}  \const{\textsc{Max}} \const{Dep} \const*{*t}
\cand{[aʔ]}                \vio{*!}  \vio{} \vio{*}                                           \vio{} \vio{}  \vio{} 
\cand{[a]}                          \vio{*!}  \vio{} \vio{}                                           \vio{*} \vio{}  \vio{}  
\cand{[ʔaʔ]}    \vio{}  \vio{\{Lar\}} \vio{**!}                                       \vio{} \vio{*}  \vio{}
\cand[\Optimal]{[ʔa]}     \vio{}  \vio{\{Lar\}} \vio{*}                                           \vio{*} \vio{*}  \vio{}
\cand{[ta]}                         \vio{}  \vio{\{Lar, Oral!\}}                                    \vio{} \vio{*} \vio{*}  \vio{*}
\end{tableau}
\end{table}


The first two candidates replicate (3) but do not satisfy \textsc{Onset} and are therefore eliminated. The last two candidates are relevant, [ʔa] vs. [ta]. Although *ʔ outranks *t -- assuming that [t] is part of the language's segment inventory -- [ʔ] is selected for epenthesis because it fares better on \textsc{Dep}(F): it only incurs violations for the insertion of laryngeal features, while [t]-epenthesis also incurs violations for inserting oral features. Finally, the third candidate [ʔaʔ] shows that *ʔ is still relevant: even when a glottal stop is inserted, underlying glottal stops will still be deleted; the second violation of *ʔ proves fatal for this candidate. In sum, the feature faithfulness approach to epenthesis can explain why a segment can be epenthetic even though it is not in a language's phoneme inventory (and thus relatively marked in that language).

A closer look at the proposed analysis reveals that it does not preclude marked\-ness-driven epenthesis: With \textsc{Dep}(F) sufficiently low-ranked, markedness constraints can still determine the quality of the epenthetic segment. While this is a theoretical possibility, such a ranking is very unlikely to ever arise in practice, however. To see why, consider briefly the consequences of low-ranked \textsc{Dep}(F), beyond epenthesis. It would mean that all segments are changed whenever adding a feature decreases their markedness. This is an across-the-board change, and it will thus trigger Lexicon Optimisation: the altered segments will be posited as underlying (because in the absence of alternations learners have no evidence to the contrary), but this in turn removes evidence for low-ranked \textsc{Dep}(F). In short, even if there is a stage where \textsc{Dep}(F) is ranked low, the changes this induces on underlying forms will ensure that the constraint is quickly promoted. It is therefore highly unlikely that there could ever be a stable period of low-ranked \textsc{Dep}(F) that would be necessary for a markedness-based epenthesis process to emerge.

But let us turn to vowels now. Recall that vowels display considerably greater variation when it comes to the quality of the default epenthetic vowel, another challenge for the markedness-based approach. How can we explain this? I propose that this is because there is no class of vowels that is inherently less specified than others, unlike the class of glottals in the set of consonants, with one exception: There is schwa, which is featureless; hence schwa epenthesis does not incur any \textsc{Dep}(F) violations, making it an ideal candidate for epenthesis. The reason that we do not find schwa in many languages can be motivated by a constraint against featureless vowels (such as *[] in \citealt{jjm18}). Once schwa is out of the race, there is no obvious alternative candidate that is underspecified. Instead, underspecification and markedness may both play a role in selecting the optimal epenthetic vowel.

Both [i] and [e] are attested epenthetic vowels. While the insertion of [i] can be explained as a markedness effect, the insertion of [e] cannot, as mid vowels are considered more marked than both high and low vowels (see e.g. \citealt{lombardi03}). However, [e]-insertion could be explained as an underspecification effect. If we assume that features are privative and that [i] is [high], then [e] is underspecified for height (neither [high] nor [low]), and this favours its insertion. Let us look at a little toy language again to see how epenthesis of [e] can be modelled as an underspecification effect.

In our toy language, vowel (see the tableau in \REF{tableau3:tbl1}) epenthesis is triggered by a high-ranked \textsc{NoCoda} constraint. Consider three candidate vowels: [i], [e], [ə]. Assume that both [i] and [e] are specified as [coronal] (front) vowels, and [i] is additionally specified as [high], while [ə] is featureless. While [i] is less marked than [e] (the ranking \textsc{*Mid \frqq{} *High}, see \citealt{lombardi03}), [e] is nevertheless the more harmonic candidate vowel, as it violates \textsc{Dep}(F) less. Schwa satisfies \textsc{Dep}(F) but is ruled out by a high-ranked constraint against featureless segments, *[].


\begin{table}
\caption{Default epenthesis of [e]}
\label{tableau3:tbl1}
\ShadingOn
\begin{tableau}{c|c|c|s|s} 
\inp{/pat/}    \const{NoCoda} \const{*[]} \const{Dep(F)}  \const{*Mid} \const{*High} 
\cand{[pat]}                \vio{*!}  \vio{} \vio{}                                           \vio{} \vio{}   
\cand{[pati]}              \vio{}  \vio{} \vio{\{Cor, Hi!\}}                                           \vio{} \vio{*}    
\cand[\Optimal]{[pate]}    \vio{}  \vio{} \vio{\{Cor\}}                                       \vio{*} \vio{}  
\cand{[patə]}     \vio{}  \vio{*!} \vio{}                                           \vio{} \vio{}  
\end{tableau}
\end{table}


It is not clear to what extent markedness is necessary at all to explain the variation between [i] and [e] in epenthesis. Recall an older (and still ongoing) debate in phonology whether vowel height is marked by the feature [high] or by [open] (leaving high vowels underspecified), with arguments going both ways (see e.g. \citealt{clements90}). The choice of aperture feature may hence well be language-specific with evidence coming from the phonological behaviour of the vowels. Are high vowels or mid vowels phonologically active in a language? The underspecification approach to epenthesis makes an interesting prediction here (though one that goes well beyond the remit of this chapter): that languages where [high] is active should prefer a mid vowel for epenthesis, and vice versa. Romance languages may be a case in point: High vowels trigger metaphony processes in many varieties, and mid vowels are generally found epenthetically. At any rate, language-dependent specifications (dependent on the phonological activity of vowels, see also \citealt{dresher09} for discussion) may also yield different default epenthetic vowels. Further research should be able to shed light on this question and the potential link between feature activity and default epenthesis.

There is one more point we need to address: the connection between default epenthesis and copy or spreading-based epenthesis that a markedness-based approach cannot provide. Faithfulness provides such a connection, as both default epenthesis and spreading are different ways of minimising \textsc{Dep}(F) violations, assuming that spreading does not involve the insertion of a new feature but just the extension of its domain, for example by adding an association line.

Assume further that spreading comes at a cost, violating anti-spreading constraints. The interaction we find between spreading-based and default epen\-thesis then follows from the relative ranking of anti-spreading constraints with \textsc{Dep(F)}. \citet{uff06, uffbook} suggests families of anti-spreading constraints, also taking into account non-local spreading and feature sharing between consonants and vowels. For the present purpose, assume a simple constraint *\textsc{Share}.

\ea *\textsc{Share}(F) \\ Every feature F is associated with exactly one segment 
\z

Let us briefly return to the Urban Hijazi data discussed earlier. High vowels spread to an epenthetic vowel slot (/i, u/ are copied) while there is epenthesis of the default vowel [i] after low /a/. This indicates that [high] spreads more easily than [low]. We can therefore posit a ranking *\textsc{Share}(lo) \frqq{} *\textsc{Share}(hi) with \textsc{Dep}(F) sandwiched in between: Spreading high vowels is more harmonic than inserting a feature, which in turn is more harmonic than low vowel spreading.\footnote{This raises the question to what extent the proposed ranking *\textsc{Share}(lo) \frqq{} *\textsc{Share}(hi) is language-specific or universal. A detailed discussion of this question would go beyond the scope of this chapter, but note that \citet{uffbook} proposes universal spreading hierarchies, derived from established markedness hierarchies.} The tableaux in (\ref{highvsp}-\ref{defep}) illustrate this. For each input form there are three candidates: a faithful candidate with no epenthesis violating the Sonority Sequencing Principle (SSP), one candidate where the epenthetic vowel results from spreading (indicated by the tie bar between the vowels), and one candidate with default [i] insertion.

\begin{table}
\caption{High vowel spreading...}
\label{highvsp}
\ShadingOn
\begin{tableau}{c|c:c|s} 
\inp{/ʃukr/ `gratitude'}    \const{SSP} \const*{*\textsc{Share}(lo)} \const*{\textsc{Dep}(hi)}  \const*{*\textsc{Share}(hi)} 
\cand{[ʃukr]}              \vio{*!}  \vio{} \vio{}                                           \vio{}  
\cand[\Optimal]{[ʃukur]}   \vio{}  \vio{} \vio{}                                   \vio{*}     
\cand{[ʃukir]}    \vio{}  \vio{} \vio{*!}                                       \vio{}   

\end{tableau}
\end{table}

Epenthesis is triggered by the high-ranked SSP constraint, eliminating the fully faithful candidate. In \REF{highvsp} spreading is optimal vis-a-vis default epenthesis because *\textsc{Share}(hi) ranks below \textsc{Dep(F)}. In \REF{defep} default epenthesis is optimal because \textsc{Dep(F)} ranks below *\textsc{Share}(lo).


\begin{table}
\caption{...but default epenthesis after low vowels}
\label{defep}
\ShadingOn
\begin{tableau}{c|c|s|s} 
\inp{/ʔakl/ `food'}    \const{SSP} \const*{*\textsc{Share}(lo)} \const*{\textsc{Dep}(hi)}  \const*{*\textsc{Share}(hi)} 
\cand{[ʔakl]}              \vio{*!}  \vio{} \vio{}                                           \vio{}  
\cand{[ʔakal]}   \vio{}  \vio{*!} \vio{}                                   \vio{}     
\cand[\Optimal]{[ʔakil]}    \vio{}  \vio{} \vio{*}                                       \vio{}   
\end{tableau}
\end{table}

Note that this approach makes an interesting prediction: that such interactions between default epenthesis and spreading should be restricted to languages where the epenthetic vowel is not schwa. As schwa epenthesis satisfies both \textsc{Dep(F)} and \textsc{*Share}, it should always be optimal vis-a-vis spreading. A preliminary survey seems to confirm this prediction, but further research could shed more light on this question. 

To summarise, I proposed that the selection of the default epenthetic segment is not determined by markedness but by faithfulness, more precisely satisfaction of \textsc{Dep(F)}, the constraint against feature insertion. Under this view, glottals are optimal epenthetic consonants because they are inherently featurally less complex, not having any oral segments. For default vowels more variation is expected as there is no vowel, bar featureless schwa, that is inherently less specified than others. As epenthesis is uncoupled from markedness, this can also explain why there are languages where the default segment is not part of the phoneme inventory. Finally, this approach can unify default epenthesis and spreading or copy epenthesis, as both result from the same drive, to minimise \textsc{Dep(F)} violations. 


\section{Discussion and conclusions} \label{uffmann:disc}

To conclude this chapter, I want to cast the net wider and discuss briefly how the faithfulness-based account of epenthesis relates to other approaches to epenthesis, and I want to argue that it fits in well with observations made in these approaches. I want to discuss three approaches briefly: gradual epenthesis in serial OT, perceptually-based epenthesis and the relationship between epenthetic and excrescent or intrusive segments, with a look at Articulatory Phonology.

In serial versions of OT, the candidate generator function \textsc{Gen} can only perform one change to the input at a time. The winning candidate is then iteratively resubmitted to evaluation until no more improvements are possible. This raises the question of what a single change is, and \citet{jjm08, jjm18} argues that segment deletions are not a single change, but the end point of a series of deletions of features or feature-geometric nodes. He furthermore proposes that consonant deletion always goes through a stage of debuccalisation before full deletion. Epenthesis should therefore also be gradual, a point taken up by \citet{almohanna21}. Now if epenthesis is the mirror image of deletion, and debuccalisation is the final step before deletion, as argued by McCarthy, then adding laryngeal specifications should be the first step in epenthesis, and after that it should be hard or even impossible to add any additional features and still remain harmonically optimising, as buccalisation is hardly attested in the world's languages,\footnote{The only convincing case I can think of is in Northern Pame \citep[226]{berthiaume-phd}, where a sequence of two glottal segments turns into a glottalised lateral.} while debuccalisation is common (and improves harmony, to allow for deletions). In other words, once laryngeal specifications are inserted, further harmonic improvement is unlikely and the derivation stops. Glottal epenthesis is therefore predicted as a consequence of gradual feature insertion. For vowels, no comparable `brake' exists, and feature additions are possible as long as they are harmonically optimising, but they will still be minimal: Serial OT predicts that the featurally least specified possible surface segment is selected and thus supports the faithfulness-based approach to epenthesis.

Another approach to epenthesis that has received particular attention in loanword adaptation is perceptually based. The idea is that a segment is inserted that is perceptually closest to zero. In other words, the epenthetic segment should be as imperceptible as possible (see e.g. \citealt{kenstowicz07}). \citet{Steriade:2009} introduces the idea of the P-map to this effect, a map of segment confusabilities. These are translated into perceptibility scales, which can then be implemented as scalar constraints. Steriade proposes scalar faithfulness constraints: the more perceptible a segment is, the less likely is its insertion or deletion. Hence, for example, the optimality of schwa epenthesis would derive from the lower perceptibility of schwa compared to other vowels, formalised as scalar \textsc{Dep}-IO constraints: \textsc{Dep}(i) \frqq{} \textsc{Dep}(ə). The P-map approach thus also appeals to faithfulness, but differently, by using faithfulness constraints as anchors for the perceptibility scales rather than by appealing to faithfulness per se.

I would argue instead that the insights from the perceptual approach to epenthesis can be accounted for in a faithfulness-based approach without having to appeal to additional sets of constraints. Assuming -- uncontroversially, I hope -- that features are at the interface to phonetic pairings of articulatory gestures with salient acoustic cues,\footnote{For a formalisation see e.g. \citet{boersha09} within the framework of Bidirectional Phonology.} the fewer features a segment has, the fewer salient cues its articulation contains, and the less generally perceptible (or confusable with zero) it will be. It may thus well be possible to subsume the perceptual approach under my proposal.

Finally, there is an ongoing and renewed debate regarding the distinction between intrusive or excrescent segments and true phonological epenthesis (see e.g. \citealt{Hall2006, chapters/08.Hall} [this volume], \citetv{chapters/10.Kramer}, \citetv{chapters/07.Bellik}), which can sometimes be difficult to distinguish. Besides, epenthesis can diachronically result from phonologising intrusion \citep{Hall2006, karlin21}, which is commonly analysed as gesture retiming, especially gestural underlap, leading to the perception of an additional segment although no articulatory gestures are added. When intrusion phonologises, gesture addition should remain minimal, and thus feature insertion should remain minimal (as features correspond to gestures on the phonological level). This is in fact the strong view of Articulatory Phonology \citep{browman1992articulatory}: no gestures may be added to an underlying form \citep{gickphono}, that is, all epenthetic segments must make use of underlyingly present gestures, and all that happens in a phonological derivation is the retiming and reorganisation of an underlying gestural score. While this claim may after all be too strong, it too ties in with the idea pursued in this chapter, that epenthesis involves the minimal insertion of features.  The markedness-based approach, in contrast, has nothing to say about the relationship between epenthesis and intrusion; why should the phonologisation of gestural underlap result in a universally unmarked segment? 

To conclude, I argued in this chapter that the standard markedness-based approach to epenthesis in Optimality Theory is beset with theoretical and empirical problems, and I sketched, however briefly, an alternative approach that is based on minimal feature insertion instead. This was formally captured in OT by the \textsc{Dep(F)} constraint, addressing and resolving the problems that the markedness-based approach encounters, and tying in with independent work on epenthesis. Future research should test the validity of this approach further.

\section*{Acknowledgements}

I would like to thank audiences at the Epenthesis Workshop and at the University of Düsseldorf and two anonymous reviewers for discussion and feedback. All remaining errors are my own.

\printbibliography[heading=subbibliography,notkeyword=this]

\end{document}
