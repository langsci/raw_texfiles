\documentclass[output=paper,colorlinks,citecolor=brown]{langscibook}
\ChapterDOI{10.5281/zenodo.14264540}

\author{Jennifer Bellik\affiliation{UC Santa Cruz}}
\title[Gestural characteristics of vowel intrusion in Turkish onset clusters]{Gestural characteristics of vowel intrusion in Turkish onset clusters: An ultrasound study}
\abstract{Onset clusters in Turkish loanwords have previously been described as being repaired with an optional epenthetic vowel, as in Turkish \textit{spor} [s\textbf{ɯ}por] ‘sport’ (\citealt{ClementsSezer:1982}, \textit{inter alia}). However, the percept of an inserted vowel can also result from gestural timing \citep{Gafos2002, Hall:2003}. \citet{Bellik2019a} presents acoustic evidence that the vocoids in Turkish onset clusters are gradiently present and differ acoustically from their underlyingly present counterparts, arguing that they arise from gestural timing, not epenthesis. This chapter presents ultrasound data from five Turkish speakers that further support this interpretation. An SSANOVA analysis shows that the tongue body is in a significantly different position during the interconsonantal interval (ICI) in underlying \#CC words compared to in underlying vowels in \#CVC words. Anticipatory coarticulation with the following vowel appears to influence tongue position in the ICI in \#CC more than in \#CVC. I argue that the process is best understood as vowel intrusion combined with coarticulation.}

\IfFileExists{../localcommands.tex}{
   \addbibresource{../localbibliography.bib}
   % add all extra packages you need to load to this file

\usepackage{tabularx,multicol}
\usepackage{url}
\urlstyle{same}

\usepackage{listings}
\lstset{basicstyle=\ttfamily,tabsize=2,breaklines=true}

\usepackage{langsci-basic}
\usepackage{langsci-optional}
\usepackage{langsci-lgr}
\usepackage{langsci-osl}
% \usepackage{./langsci/styles/langsci-lgr}
% \usepackage{./langsci/styles/langsci-osl}
% \usepackage{langsci-gb4e}

\usepackage{tikz}
\usetikzlibrary{patterns,calc}
\pgfdeclarepatternformonly{south east lines}{\pgfqpoint{-0pt}{-0pt}}{\pgfqpoint{3pt}{3pt}}{\pgfqpoint{3pt}{3pt}}{
    \pgfsetlinewidth{0.6pt}
    \pgfpathmoveto{\pgfqpoint{0pt}{3pt}}
    \pgfpathlineto{\pgfqpoint{3pt}{0pt}}
    \pgfpathmoveto{\pgfqpoint{.2pt}{-.2pt}}
    \pgfpathlineto{\pgfqpoint{-.2pt}{.2pt}}
    \pgfpathmoveto{\pgfqpoint{3.2pt}{2.8pt}}
    \pgfpathlineto{\pgfqpoint{2.8pt}{3.2pt}}
    \pgfusepath{stroke}}
    
\usepackage{stmaryrd}
\usepackage{wasysym}
\usepackage{multirow}
\usepackage{caption}
\usepackage{subcaption}
\usepackage{mathrsfs}
\usepackage{qtree}

\usepackage{linguex}


   %pminos do not split footnotes
% \interfootnotelinepenalty=10000 %Footnote in Laporte chapters has to be split SN


%\DeclareIndexNameFormat{default}{%
%\nameparts{#1}%
%\usebibmacro{index:name}%
%{\index[names]}%
%{\namepartfamily}%
%{\namepartgiveni}%
% {}% L1
% {}% L2
%{\namepartprefix}% generates spurious space L3
%{\namepartsuffix}% generates spurious space L4
%}

%  {\DeclareIndexNameFormat{default}{%
%     \usebibmacro{index:name}{\index[names]}{#1}{#3}{#5}{#7}}}

%\DeclareIndexNameFormat{default}{%
%  \usebibmacro{index:name}{\sindex[nom]}{#1}{#3}{#5}{#7}}

%\DeclareIndexNameFormat{default}{%
%  \usebibmacro{index:name}{\sindex[person]}{#1}{#3}{#5}{#7}}
%\DeclareIndexNameFormat{default}{%
%\nameparts{#1} \usebibmacro{index:name}{\sindex[person]]}{\namepartfamily}{‌​\namepartgiven}{\nam‌​epartprefix}{\namepa‌​rtsuffix}}

%\newcommand{\smiley}{:)}

%\renewbibmacro*{index:name}[5]{%
%\usebibmacro{index:entry}{#1}%
%{\iffieldundef{usera}{}{\thefield{usera}\actualoperator}\mkbibindexname{#2}{#3}{#4}{#5}}}

% \newcommand{\noop}[1]{}

%remove for final
%\overfullrule=1mm

\newcommand{\tobi}[2]}}
\renewcommand{\S}[1]{\tobi{#1}{\textsc{*}}}

% this volume references
% puts: [this volume]
% already defined: \citetv
%\newcommand{\citepv}[1]{(\citeauthor{#1} \citeyear*{#1} [this volume])}
\newcommand{\citealtv}[1]{\citeauthor{#1} \citeyear*{#1} [this volume]}

%parentheses around example number
\newcommand{\pref}[1]{(\ref{#1})}

% in-text examples

\newcommand{\lnex}[1]{\textit{#1}} %target lang word
\newcommand{\lnlit}[1]{(lit.: `#1')} %literal reading
\newcommand{\lnlat}[1]{(#1)} % latinization
\newcommand{\lntrans}[1]{`#1'} %translation
\newcommand{\lnexl}[2]%
{\lnex{#1}{} \lnlat{#2}} % ex with latinization
\newcommand{\lnexlat}[3]{\lnex{#1}{} \lnlat{#2}{} \lntrans{#3}} % ex with latinization and tranl.

%ch01
\newcommand{\co}[1]{\mbox{\textbf{#1}}}

%ch09

\newcommand{\cyrbulg}[1]{\begin{otherlanguage*}{bulgarian}#1\end{otherlanguage*}}


%ch10
\newcommand{\nlp}{{\small NLP}}
\newcommand{\mwe}{{\small MWE}}
\newcommand{\rae}{{\small RAE}}
\newcommand{\lvc}{{\small LVC}}
\newcommand{\pos}{{\small P}o{\small S}}
%\newcommand{\todo}[1]{ \textcolor{red}{#1} }

%\renewcommand{\labelenumi}{\theenumi}
%\ainamefmt{{vv}{ll}{, ff}{, jj}} % fullname

\newcommand{\biberror}[1]{{\color{red}#1}}

\newcommand{\osenovaitem}{--~}
   %% hyphenation points for line breaks
%% Normally, automatic hyphenation in LaTeX is very good
%% If a word is mis-hyphenated, add it to this file
%%
%% add information to TeX file before \begin{document} with:
%% %% hyphenation points for line breaks
%% Normally, automatic hyphenation in LaTeX is very good
%% If a word is mis-hyphenated, add it to this file
%%
%% add information to TeX file before \begin{document} with:
%% %% hyphenation points for line breaks
%% Normally, automatic hyphenation in LaTeX is very good
%% If a word is mis-hyphenated, add it to this file
%%
%% add information to TeX file before \begin{document} with:
%% \include{localhyphenation}
\hyphenation{
    Beck-man
    Ngu-yen
    back-chan-nel
    back-chan-nels
    mo-not-o-nous
    ste-reo-typ-i-cal
}

\hyphenation{
    Beck-man
    Ngu-yen
    back-chan-nel
    back-chan-nels
    mo-not-o-nous
    ste-reo-typ-i-cal
}

\hyphenation{
    Beck-man
    Ngu-yen
    back-chan-nel
    back-chan-nels
    mo-not-o-nous
    ste-reo-typ-i-cal
}

   \usepackage{makecell}
   \boolfalse{bookcompile}
   \togglepaper[7]%%chapternumber
}{}

\begin{document}
\maketitle \label{ch7}

\section{Introduction}
Turkish phonology is well-known for its highly regular vowel harmony system. Canonically, all non-initial vowels in a word match the preceding vowel in backness, and high vowels additionally match the preceding vowel in rounding. Synchronically, this harmony applies to vowels in (most) suffixes: \textit{ip-ler-in-iz} ‘your (pl.) ropes’ but \textit{pul-lar-ın-ız} ‘your (pl.) stamps’. Vowel harmony has not applied within loanwords, however, so Turkish includes many words where vowels disagree in backness, rounding, or both, such as \textit{kitap} ‘book’ (Arabic), \textit{broşür} ‘brochure’ (French), and \textit{İstanbul} (Greek). When a word contains disharmonic vowels, harmonizing vowels match the immediately preceding vowel: \textit{kitap-lar-ınız} ‘your (pl.) books’, \textit{gid-iyor-uz} ‘we are going’. Turkish is typically described as permitting some complex codas but prohibiting all complex onsets (\citealt{ClementsSezer:1982}); both types of consonant cluster occur only in loanwards. Vowel insertion occurs in complex codas with rising sonority and in any onset cluster. These two vowel insertion processes differ in several ways. Vowel insertion in coda clusters (Table \ref{Vinsertion}) is obligatory, written, and invariant. The inserted vowel can receive primary stress, and harmonizes in backness and rounding with the preceding lexical vowel, like other high vowels in Turkish suffixes. Coda clusters generally occur in words of Arabic or Farsi origin.


\begin{table}
\caption{Vowel insertion in Turkish coda clusters}
\label{Vinsertion}
\begin{tabular}{llllll}
\lsptoprule
Spelling & \makecell{Nom.\\(epenthesis,\\ no suffix)} & \makecell{Acc.\\ (-ɯ/i/u/y)} & \makecell{Dat.\\ (-e/a)} & Root & Gloss \\\midrule
\textit{sabır}	& [sa.bɯr] &	[sab.ˈrɯ]	& [sab.ˈra]	& /sabr/	& `patience'\\
\textit{cebir}	& [dʒe.ˈbir]	& [dʒeb.ˈri]	& [dʒeb.ˈre]	& /dʒebr/	& `algebra'\\
\textit{burun}	& [bu.ˈrun]	& [bur.ˈnu]	& [bur.ˈna]	& /burn/	& `nose'\\
\textit{ömür}	& [ø.ˈmyr] &	[øm.ˈry] &	[øm.ˈre] &	/ømr/	& `life' \\
\lspbottomrule                
\end{tabular}
\end{table}


Vowel insertion also occurs in onset clusters, which usually appear in borrowings from European languages (Table \ref{Vonset}; \citealt{Yavas1980}, \citealt{ClementsSezer:1982}, \citealt{Kaun1999}, \citealt{Yildiz2010}, \citealt{Kabak2011}). These inserted vowels are variable and unwritten, and have been characterized as optional \citep{Yildiz2010} or style-dependent (\citealt{ClementsSezer:1982}). They sometimes match the following lexical vowel in backness and/or rounding, but are always [+back] following /g/ or /k/ (\citealt{ClementsSezer:1982}, \citealt{Kabak2011}).

 
\begin{table}
\caption{Vowel insertion in Turkish onset clusters}
\label{Vonset}
\begin{tabular}{llll}
\lsptoprule
Spelling & With insertion & Without insertion & Gloss \\\midrule
\textit{branda} &	[bɯranda] &	[branda] &	`canvas'\\
\textit{prens} &	[pirens] &	[prens]	& `prince'\\
\textit{prova} &	[purova] &	[prova] &	`test'\\
\textit{blujin} &	[buluʒin] \sim [byluʒin] & 	[bluʒin] &	`blue jeans'\\
\textit{grip} & 	[gɯrip]	& [grip]	& `flu'\\
\lspbottomrule                
\end{tabular}
\end{table}

Though both types of insertion have previously been characterized as epenthesis, this unified description does not explain the differences between them. These differences would be explained, however, if the presence of vowels in codas is due to actual vowel insertion (that is, epenthesis driven by syllable structure), while vowels in onsets are due only to gestural alignment, as will be illustrated in \figref{abcd}, which I have argued previously (\cite{Bellik2018,Bellik2019b,Bellik2019a}).

\begin{figure}
\caption{Four possible gestural scores for /CrV/}
% \includegraphics[width=\textwidth]{figures/bellik_figure01_abcd.png}

\subfigure[Epenthesis (V gesture inserted between C and r)]{
\includegraphics[height=2cm]{figures/bellik_figure01_a.png}
}\hfill
\subfigure[Copy-vowel intrusion (Open C-r transition, more V-ICI overlap)]{
\includegraphics[height=2cm]{figures/bellik_figure01_b.png}
}

\subfigure[Schwa-like intrusion (Open C-r transition, less V-ICI overlap)]{
\includegraphics[height=2cm]{figures/bellik_figure01_c.png}
}\hfill
\subfigure[Transparent complex onset (Close C-r transition)]{
\includegraphics[height=2cm]{figures/bellik_figure01_d.png}
}
\label{abcd}
\end{figure}


\figref{abcd} shows four possible gestural scores that could be the output of phonology, given /CrV/ as the input, ranging from epenthesis in \figref{abcd}a to a canonical complex onset in \figref{abcd}d. In a canonical complex onset, the first consonant's gesture is immediately followed by the second consonant's. No vowel sound heard between them, because /r/ achieves its closure before /C/'s release. In vowel intrusion (\figref{abcd}b,c), the two consonant gestures are likewise adjacent, with no distinct vowel gesture intervening. Vowel intrusion results when the first consonant's closure is released before the second consonant achieves its closure, opening the vocal tract during the transition between consonants. This open transition allows an overlapping vowel gesture to be heard before the second consonant recloses the vocal tract (\cite{BrowmanGoldstein1993, Gafos2002, Hall:2003, Hall2006}). Intrusion can be realized in a variety of ways, depending on the precise gestural alignments involved. If /V/ achieves its target during the interconsonantal interval (ICI), and the ICI is long enough, the resulting open transition can sound like a copy of the following /V/ (\figref{abcd}b). When the ICI is short, and/or /V/ does not attain its target during the ICI (\figref{abcd}c), the open transition sounds more schwa-like, and may be more affected by the preceding consonant than a longer “copy” vowel. Since inserted vowels in Turkish onset clusters reportedly take on the backness, but not the height, of V2 (\citealt{Yavas1980}, \citealt{ClementsSezer:1982}, \citealt{Kaun1999}, \cite{Yildiz2010}, \citealt{Kabak2011}), they can be considered intermediate between schwa-like vowels and copy vowels.

Intrusion resembles epenthesis (\figref{abcd}a) in that the first consonant's closure is released before the second consonant's closure is achieved, and in that acoustically, a period of high amplitude periodicity intervenes between consonants. They differ, however, in that in epenthesis, an additional vowel gesture intervenes between the consonants. Consequently, tongue position in the ICI reflects a phonologically specified tongue body target. This is also true for an underlying /CVC/ sequence. By contrast, in intrusion, the acoustic “vowel” between the consonants is only a preview of the following vowel, audible during the open transition between consonants. Therefore, tongue position during the ICI reflects the transition between the preceding consonant and the following vowel. Unlike an epenthetic vowel, the intrusive vowel is not a phonological segment, and therefore not a target for any segment-level vowel harmony.

The phonological status of vowels in Turkish onset clusters has implications for our understanding of Turkish syllable structure, vowel harmony, and their interaction. Turkish may offer an example of vowel intrusion in a language that also has a phonological process of vowel harmony, broadening the range of descriptions of vowel harmony processes. Morever, the phonological status of these inserted vowels is of broader theoretical importance, because a variety of claims about the Turkish or cross-linguistic vowel harmony have been founded on the interpretation that inserted vowels in Turkish onset clusters are epenthetic targets for vowel harmony (\citealt{ClementsSezer:1982}, \citealt{Yavas1980}, \citealt{Kaun1999}, \citealt{Yildiz2010}).

I hypothesize that vowels in Turkish onset clusters are intrusive, resulting from gestural timing, and therefore differ gesturally and acoustically from underlying vowels. I tested this hypothesis with an ultrasound and acoustic production study. The acoustic study (\cite{Bellik2018, Bellik2019b}) found, firstly, that the duration of the ICI in /CC/ words has a unimodal distribution, indicating that the vowels heard in onset clusters are gradiently present. If they were optional epenthetic vowels with a specified durational target, they would be either categorically present (one mode at a positive vocalic duration) or categorically absent (another mode at zero vocalic duration), creating a bimodal distribution of durations instead. Secondly, the F1 and F2 values of vowels in onset clusters before /i/ and /o/ differ from both harmonizing vowels /i/ and /u/ (V2 determines which vowel is harmonic) and the non-harmonizing vowel /ɯ/. If vowels in onset clusters are epenthetic and had acoustic targets, we would expect them to hit the same targets that underlyingly present vowels hit: either the same F1{\textasciitilde}F2 values as harmonizing vowels in the same context, if vowel harmony applies; or the same F1{\textasciitilde}F2 values as the lexical /ɯ/, which is the Turkish phoneme that these vowels, impressionistically, most resemble, as well as being the most schwa-like Turkish vowel, suggesting that it could function as a neutral vowel. But in fact, vowels in onset clusters differ acoustically from both the phonemes harmony predicts and the phoneme they most sound like, suggesting that vowels in onset clusters do not share the acoustic targets of any lexical vowels. Taken together, this is evidence that vowels in underlying onset clusters in Turkish lack durational and gestural targets. The present paper presents the ultrasound study, which provides direct evidence that vowels in Turkish onset clusters differ gesturally from phonologically targeted vowels, supporting the hypothesis that they result from an open transition between consonants.


\section{Method}

The present ultrasound study was inspired by \citet{DavidsonStone2003}'s use of ultrasound to determine whether articulation of /zg/ clusters by English speakers as [z\textsf{ə}g] represents phonological epenthesis, or phonetic intrusion. They compared [z\textsf{ə}g] to English words containing a lexical schwa (\textit{succumb}), and words with sC clusters (\textit{scum}), according to the logic that if the inserted vowel is epenthetic and has a gestural target, its gestural sequence will resemble lexical schwa’s more closely than the insertionless cluster’s. If the inserted vowel is instead intrusive, then its gestures will more closely resemble the insertionless cluster’s.

Following the same logic, this study compares the articulation of underlying clusters /Cr/ in Turkish, with and without inserted vowels, to that of /CVr/ sequences, which contain underlying vowels.

\subsection{Design}
The primary independent variable in the experiment was the underlying structure of the target word, and hence the lexical status of the vowel between /C/ and /r/: non-lexical vowels occurred in words beginning with a stop+/r/ onset cluster (/Cr/), and lexical vowels occurred in words beginning with a simple onset followed by an underlying vowel and /r/ (/CVr/). To ensure that the findings extend across all consonant and vowel places, and investigate claims of vowel harmony in the inserted vowel, three stop consonants (/b d g/ – voiced stops were chosen to avoid aspiration) and three vowels (/i a o/\footnote{Originally /u/ was included as the third V2 value, rather than /o/, but no familiar words of the shape /bru-/ could be found, and so /o/ was selected.}) were included. Due to a mistaken syllabification during the experimental design, the d-o condition did not contain a true onset cluster, and was therefore dropped from the analysis, leaving eight C1-V2 conditions; see \citet[39]{Bellik2019a} for details. Both real and nonce words were included, to assess the productivity of insertion. No acoustic differences between insertion in real and nonce words were found, nor were any systematic gestural differences between inserted vowels in real and nonce words found, so I set aside the real/nonce distinction from here on.

Vowel insertion is reported to occur in /s/+stop and obstruent+/l/ clusters in Turkish, as well as obstruent+/r/ (\citealt{Yavas1980}, \citealt{ClementsSezer:1982}). The Turkish /r/ is phonetically a voiced alveolar tap, while the /l/ can be a palatalized post-alveolar lateral or a velarized dental lateral, depending on context (\citealt{GokselKerslake2005}).  This experiment uses stop+/r/ clusters because they have a higher rate of insertion in the Turkish Electronic Living Lexicon (TELL; \cite{Inkelasetal2000}) than /sC/ clusters (71\% vs. 42\%). Also, surface harmonic effects resulting from vowel overlap are more likely to occur across a sonorant like Turkish /r/ (phonetically a tap) than across a stop (\citealt{Hall:2003, Hall2006}; see also \citealt{Bradley2004}). /Cl/ clusters were avoided for ease of segmentation using spectrograms.

Thus, the study had a 2 by 3 by 3 design \REF{exfactors}.

\begin{exe}
    \ex \label{exfactors} Experimental factors: \\
    \textsc{W}ord shape = [/Cr/ vs. /CVr/]\\
× [C1 = /b d g/]\\
× [V2 = /i a o/]\\
    
\end{exe}


The full experiment, detailed in \citet[ch. 2]{Bellik2019a}, included both careful and casual speech. The ultrasound results reported here are restricted to careful, hyper-articulated speech; see \citet{Bellik2019b} for acoustic differences across speech styles.

\subsection{Materials}
A list of real and nonce words beginning with stop+/r/ clusters was generated. All onset clusters in Turkish occur in borrowed words; a familiarity-rating task with three Turkish speakers ensured that all real words in the study were familiar, not novel. Control words of the form /CVrV/ were created for every condition so that non-lexical vowels in /Cr/ words could be compared to lexical vowels in the same context. The apparent insertion of [ɯ] is attested before all qualities of V2, so /CɯrV/ controls were included for every V2 condition. In addition, /Ciri/ and /Curo/ controls were included, since insertion of [i] and [u] is reported before /i/ and /o/, respectively. Most control words are nonce words,\footnote{Surface productions of /\#CrV/ sequences often sound like [CɯrV], both in this experiment and according to TELL. However, across all values of V2, underlying /\#CɯrV2/ sequences are rare (\citealt[ch. 5]{Bellik2019a}). In fact, such sequences where V${\neq}$a are completely unattested. This suggests that Turkish phonology prohibits these disharmonic sequences of segments, and can be taken as distributional evidence that there is no vowel segment/gesture between /C/ and /r/ in underlying clusters.} although real words were included where possible. \tabref{stimuli} shows the experimental items and controls. See \citet[Ch.2]{Bellik2019a} for further details, including a list of the seventeen fillers.

Participants were instructed to speak carefully. All words were presented in a carrier sentence \REF{carrier}, which includes slots for two target words (X and Y). The sentence was designed to elicit contrastive focus on the target words, to promote hyperarticulation.

\begin{exe}
\ex \label{carrier} Carrier sentence:\\
\gll  Bana	X	deme, 		bana  	Y de.\\
me.\textsc{dat}	X 	say.\textsc{neg} me.\textsc{dat} Y say.\\
\glt `Don't say X to me, say Y to me.'
\end{exe}


\begin{table}
\caption{Stimuli for the production experiment}
\label{stimuli}
\begin{tabular}{llll}
\lsptoprule
C1 & V2 & \makecell{Experimental: /Cr/\\ Real, nonce} & Controls: /CVr/ \\
\midrule
\textbf{b} &	\textbf{/i/}	& bri.fing `briefing' & bɯ.ri.pis \\
 & &                            bri.mi.ti	  & bi.ri.m-in `unit.your' \\\addlinespace
& & & bi.ri.bis\\ 
&	\textbf{/a/} &	bran.ʃ-ɯ `subject.\textsc{acc}' & bɯ.ran.ʤɯ \\\addlinespace
& & brat.ʧi.ten	\\
& \textbf{/o/} &	bro.ʃyr `brochure' & bɯ.ro.ʒyn\\
& & bro.ʒør.le	& bu.ro.ʧyp\\\midrule
\textbf{d} &	\textbf{/i/} & 	drip.ling `dribbling' & dɯ.rib.le\\
& & drip.li.ke	& di.rim.-ler `life.\textsc{pl}'\\\addlinespace
& & & di.rib.rit\\
& \textbf{/a/} &	dra.ma `drama' & dɯ.rap\\
& & dra.fa	\\\midrule
\textbf{g} &	\textbf{/i/} &	grip (5) `influenza' &  gɯ.rif\\
& & gri.vi	& gi.rim `penetration'\\\addlinespace
& & & gi.riv\\
& \textbf{/a/} &	gram (5) `gram' & gɯ.rap\\\addlinespace
& & gra.bɯ	\\
& \textbf{/o/} &	gro.s-u (2.7) `gross.\textsc{acc}' & gɯ.ron \\
& & gro.dol	 & gu.rot\\
\lspbottomrule                
\end{tabular}
\end{table}

 

\subsection{Participants}

Seven native speakers of Turkish (4 female: S1, S4, S5, S7; age range 18--35) were recruited from the University of California at Santa Cruz. S1 participated in the pilot experiment with voiceless rather than voiced obstruents, and S2's ultrasound data were uninterpretable for anatomical reasons, leaving five speakers' ultrasound data to be discussed here. S3 is bilingual in French and Turkish. S6 lived in New Jersey, USA, for a year (age 4-5), but in Turkey otherwise. The remaining speakers all studied English in school during adolescence, but lived in Turkey, using Turkish as their primary language at home and work, until age 18 or later. Participants were paid \$20 for their time.


\subsection{Procedure}

Participants wore an Articulate Instruments Ultrasound Stabilization Headset \citep{Wrench2008} to stabilize the ultrasound probe. Recordings were made in a sound-attenuated booth using a shotgun microphone with a USB pre-amplifier connected to the ultrasound machine (Terason T3000 ultrasound system with a model 8MC3 probe, 45--60 frames per second). Stimuli were presented to subjects on a laptop screen, one sentence at a time. Participants read through a list of 27 sentences five times in careful speech.

Acoustic recordings were annotated in Praat (\citealt{BoersmaWeekink}) in order to identify the time range of the ICI. The left edge of the interconsonantal interval (ICI) was marked from the beginning of the C1 release burst, identified by a dramatic increase in amplitude. The right edge of the ICI was identified by the decrease in amplitude accompanying the onset of /r/. Then the ultrasound frame best corresponding to the midpoint of the ICI was selected using a Python script, and tongue tracings were made in Edgetrak \citep{Lietal2005}.

R (\citealt{R}) and the ssanova function in the gss package \citep{Gu2014} were used to create smoothing spline ANOVAs (SSANOVA; \citealt{Gu2002}, \citealt{Davidson2005}) for each word, within subject. An SSANOVA is essentially the mean of multiple curves, plus a confidence interval around it. Within each C1-V2 combination, the SSANOVAs for words with underlying clusters and with underlying vowels were plotted together.

\subsection{Predictions}

Lexical vowels have phonologically specified gestural targets, while intrusive vowels are targetless. Hence, the position of the articulators during an intrusive vowel will be determined by the gestural demands of the surrounding vowels and consonants. According to the hypothesis that the inserted vowels result from an open transition between consonants, the surrounding consonants and the following vowel should shape inserted vowels more than underlying vowels.

The phonological status of the inserted vowel shapes the organization of the syllable, which in turn shapes tongue position during the ICI. Let's take /gram/ [gɯram] `gram' as an example. If the inserted vowel [ɯ] is epenthetic and has a gestural target, the word is syllabified as [gɯ.ram]. Thus, [gɯ] forms a syllable with a simplex onset and the gestures for [g] and [ɯ] are therefore coordinated with each other. The [r] and [a] gestures are part of a second syllable and are not directly coordinated with [g] and [ɯ], since they belong to distinct syllables, just as in a two-syllable /CVrV/ control word like /gɯrap/ [gɯ.rap] (nonce). This syllabification limits the overlap of the second vowel /a/ with the first vowel [ɯ]. On the other hand, if [ɯ] in [gɯram] is an intrusive vowel resulting from gestural timing alone, then /gram/ is a single syllable with a complex onset, meaning that the gesture for the nucleus /a/ is coordinated with both [g] and [r] gestures and is expected to overlap the interval between them (\citealt{Byrd1996}, \citealt{BrowmanGoldstein1988}), pulling the tongue body toward /a/'s target during the ICI when [ɯ] is heard. We predict, then, that if a word-initial consonant cluster /Cr/ is syllabified as a complex onset with vowel intrusion, then the vowel that follows the consonant cluster will affect tongue position in the ICI in /Cr/ words more than in control /CVr/ words. But if a word-initial consonant cluster is broken up by epenthesis and syllabified as two consecutive syllables with simplex onsets, then this predicts that the vowel following the consonant cluster will equally affect experimental /CrV/ and control /CVrV/ words.

The degree to which the vowel following /Cr/ can overlap the ICI will be shaped by C’s demands on the tongue body. Vowel-based differences between intrusive and underlying vowels are predicted to be most pronounced when the preceding consonant is labial, since the lips are able to move independently of the tongue body, meaning that an overlap of V2 with /b/ minimally interferes with /b/'s articulation. When the preceding consonant is coronal, its tongue tip target will limit movement of the tongue body toward V2’s target, since the tongue tip is coupled to the tongue body, limiting V2’s impact during the ICI. Finally, V2-driven differences between underlying and inserted vowels will be least pronounced when the preceding consonant is dorsal (/g/), since /g/’s tongue body target will most severely limit anticipatory movement toward the following vowel’s tongue body target.

\section{Results}

The vowel that drives anticipatory coarticulation in the ICI determines the direction of expected differences between intrusive and underlying vowels. In the analysis that follows, I compare SSANOVAs of tongue body positions at the midpoint of the ICI for intrusive vowels with those for /ɯ/ (the most schwa-like Turkish vowel) and for the high vowels that regressive vowel harmony would demand, based on claims of harmony in the previous literature (\citealt{Yavas1980}, \citealt{ClementsSezer:1982}, \citealt{Kaun1999}, \citealt{Yildiz2010}), within each /C+V2/ condition. Each speaker is plotted separately since there can be significant interspeaker variation. Only a representative sample of the SSANOVAs is included here; see \citet[ch.2]{Bellik2019a} for additional plots. Each plot shows a contour for tongue body position in the underlying cluster (orange line), another for underlying /ɯ/ (light blue), and a third for harmonic /i/ (dark blue). The legend shows the specific C(V)rV conditions for the plot, and the number of tokens that each curve represents. In most cases, there are half as many tokens of the condition where the V1 /ɯ/ is disharmonic with V2, because there are no real Turkish words with this shape; only nonce words were available. Where they have more tokens, the /Cr/ and harmonic V1 conditions combine the repetitions of a real and a nonce word.

The plots also include 99\% confidence intervals, shown by dashed lines. When the confidence intervals for two curves do not overlap, the curves represent significantly different tongue positions. In most plots, the confidence intervals are so close to the main curve that they are hard to see. As is typical in ultrasound results, however, the position of the tongue tip and root is less certain, due to jaw shadow which limits imaging of those areas. This decreased certainty is reflected by the flared confidence intervals around the ends of the curves. The flaring at endpoints also reflects the method of calculating confidence intervals, which is sensitive to the square of the distance from the midpoint on the x-axis.

Because each SSANOVA curve represents only one to two lexical items from the same speaker, in interpreting the results, I did not consider the existence of a region of non-overlapping confidence intervals for a given pair of curves to be a sufficient basis for concluding that the articulation of the relevant cluster and /CVC/ sequence differed systematically across lexical items. I consider two curves to indicate a meaningful difference across conditions (not just lexical items) for a speaker only if there are statistically significant differences in tongue position across the majority of the length of the two curves and those differences are in the direction predicted by coarticulation with the surrounding consonants and vowel.

The results, detailed below, largely bore out the predictions above. Tongue position in underlying clusters differs systematically from tongue position in underlying vowels in the same context, in ways that show the greater influence of the following vowel on underlying clusters than on underlying vowels. Not every speaker conforms to the predictions in every condition, however. Also, a preceding /g/ obscures the effect of the following vowel on tongue body position.

Speakers fall into three groups, according to their acoustic and gestural results. The early bilinguals S3 and S6 tend to show the greatest differences between underlying clusters and underlying vowels. Late bilinguals S5 and S7 tend to show intermediate levels of gestural difference. Lastly, S4, the only monolingual in the study, has no acoustic differences between underlying clusters and underlying vowels (see \citealt{Bellik2018}), but nonetheless displays gestural differences in most conditions.


\subsection{/i/ conditions}

When the following vowel is /i/, coarticulation will raise and front the tongue body in anticipation of /i/’s [+high, –back] target. Consequently, before /i/, the tongue body should be fronter and potentially higher during intrusive <v> than during underlying [+high, +back] /ɯ/. (Although both /i/ and /ɯ/ are phonologically [+high], \citet{Kilic2004} found that /ɯ/ has a lower tongue body position than /i/.) However, the tongue should be less front and high during a targetless <v> than during an underlying /i/: in an underlying /Cri/ sequence, /i/’s [+high, –back] target does not need to be attained until after both consonants, in contrast to an underlying /Ciri/ sequence. Therefore, if the vowel within the cluster is intrusive, we expect tongue body position in underlying clusters before /i/ to be intermediate between that of underlying /ɯ/ and /i/.

When the preceding consonant is /b/, four out of five speakers clearly bore out this prediction, as exemplified by S5 (\figref{b1s5}). The highest point in the tongue body at the midpoint of the ICI for /bri/ is intermediate in backness between that for /biri/ and for /bɯri/. For the last speaker (S4, \figref{b1s4}), the highest point of the tongue body in /bri/ is indeed much backer than in /biri/, but it is not fronter than the highest point in /bɯri/, which is markedly lower than the other two curves'.


\begin{figure}
\caption{Tongue body position in the /b-i/ condition, S5}
\includegraphics[height=.35\textheight]{figures/bellik_figure02_s5_bi.png}
\label{b1s5}
\end{figure}


\begin{figure}
\caption{Tongue body position in the /b-i/ condition, S4}
\includegraphics[height=.35\textheight]{figures/bellik_figure03_s4_bi_midpoint.png}
\label{b1s4}
\end{figure}

When the preceding consonant is /d/, the difference between tongue position in /i/ and /ɯ/ is less dramatic, likely due to the fronting effect of the coronal consonant (\figref{d1s5}). For all subjects, the peak of the tongue body during /dri/ is lower than its peak in one (S4, S7) or both (S3, S5, S6) of the underlying vowels~-- an effect not predicted by within-word coarticulation, although conceivably related to the final /a/ of the preceding word in the carrier phrase. For four out of five subjects, the confidence intervals around the /dri/ curve only overlap those for /dɯri/ or /diri/ when the curves intersect, and the peak of the /dri/ curve at the midpoint of the ICI is fronter than that of /dɯri/ but backer than that of /diri/. For three subjects, the tongue body's highest point during the ICI for /dri/ is intermediate in frontness between that of /diri/ and /dɯri/. For S7, however, the curve for /dri/ largely overlaps that of /dɯri/.

\begin{figure}
\caption{Tongue body position in the /d-i/ condition, S5}
\includegraphics[height=.35\textheight]{figures/bellik_figure04_s5_di_midpoint.png}
\label{d1s5}
\end{figure}

\newpage

\begin{figure}[b!]
\caption{Tongue body position in g-i condition, S7}
\includegraphics[height=.35\textheight]{figures/bellik_figure05_s7_gi_midpoint.png}
\label{g1s7}
\end{figure}


\begin{figure}[b!]
\caption{Tongue body position in g-i conditions, S6}
\includegraphics[height=.35\textheight]{figures/bellik_figure06_s6_gi_midpoint.png}
\label{g1s6}
\end{figure}


As expected, the pattern of clusters being intermediate between /i/ and /ɯ/ before /i/ is much less discernible when the preceding consonant is /g/ (\figref{g1s7}). The highest point of the tongue body during the ICI in /gri/ is intermediate in backness and height between that of /giri/ and /gɯri/ for S6 and S3, for both of whom /gri/'s peak is higher than that of /gɯri/, perhaps indicating that /g/ is having a greater impact during the targetless interval than during underlying vowels. As a velar consonant, /g/ can be expected to contribute backing and/or raising to an adjacent vowel \citep{Padgett2011}; this occurs, for example, in intrusive vowels in Maxakalí (\citealt{Gudschinsky1970}, \citealt{Clements1991}, cited in \citealt{Padgett2011}). For S5, S7, and S4, meanwhile, tongue positions in /gri/ and /gɯri/ overlap for much of the curves' length.


To summarize results for the conditions where V2=/i/, tongue body position during the ICI in /bri/ and /dri/ is intermediate between that of /ɯ/ and /i/, while tongue body position in /gri/ is less clearly so, as expected.


\subsection{/a/ conditions}

When the following vowel is /a/, anticipatory coarticulation lowers the tongue body toward /a/’s [–high] target during the ICI. This predicts that a targetless vowel before /a/ will be lower than the [+high] /ɯ/. Since /ɯ/ matches /a/ in both backness and rounding, it is also harmonic with /a/, so no other vowels are predicted before /a/.

The prediction is most clearly borne out after /b/. Tongue position in the ICI for /bra/ is significantly lower than in /bɯra/ for all subjects at least at the peak and usually throughout the length of the curve (\figref{bas5}). The height difference is greatest for early bilinguals S3 and S6.

\begin{figure}
\caption{Tongue body position in the /b-a/ condition}
\includegraphics[height=.35\textheight]{figures/bellik_figure07_s5_ba_midpoint.png}
\label{bas5}
\end{figure}


The same pattern holds after /d/. The entire length of the tongue body, apart from any intersections, is significantly lower during the ICI in /dra/ than in /dɯra/ (\figref{das5}) for all subjects but S7.


\begin{figure}
\caption{Tongue body position in the /d-a/ condition}
\includegraphics[height=.35\textheight]{figures/bellik_figure08_s5_da_midpoint.png}
\label{das5}
\end{figure}

S4 is the only subject who shows the predicted lowering in /gra/ (\figref{gas7}). For S3, S6 and S7, the peak of the tongue body during the ICI in /gra/ is significantly higher than in /gɯra/, suggesting that /g/’s velar closure has a greater effect on inserted vowels for these speakers. For S4 and S6, the peak of the tongue body is fronter in /gra/'s ICI than in /gɯra/'s, perhaps reflecting movement toward /a/’s more central articulation.

\begin{figure}
\caption{Tongue body position in the /g-a/ condition, S7 (S3 is similar)}
\includegraphics[height=.35\textheight]{figures/bellik_figure09_s7_ga_midpoint.png}
\label{gas7}
\end{figure}

\begin{figure}
\caption{Tongue body position in the /g-a/ condition, S4 (S6 is similar)}
\includegraphics[height=.35\textheight]{figures/bellik_figure10_s4_ga_midpoint.png}
\label{gas4}
\end{figure}
 


Across consonant conditions, tongue position in clusters with a following /a/ tends to be lower than in an /ɯ/; however, a preceding /g/ usually blocks this effect. As in /i/ conditions, the following vowel /a/ has the greatest effect when /b/ precedes.



\subsection{/o/ conditions}

Anticipatory coarticulation with a following /o/ will move the tongue body toward a [+back, –high] target, possibly resulting in a lower tongue position for intrusive vowels than for high vowels /ɯ/ and /u/. No differences in backness during the ICI in /Cro/ vs. /Cɯro/ vs. /Curo/ are predicted. Differences in rounding are possible, but not observable from ultrasound.

Tongue body position during the ICI in /bro/ does tend to be lower than in /bɯro/ or /buro/ (\figref{bos7}). For S3 and S7, the tongue body during the ICI of /bro/ is significantly lower than during both /bɯro/ and /buro/, at its peak and along most of its length. For S5 and S6, the tongue body position in /bro/ is significantly lower than in /bɯro/ but higher than in /buro/, while for S4, the curve for /bro/ is lower than /u/ but higher than /ɯ/ for most of its length.

\begin{figure}
\caption{Tongue body position in the /b-o/ condition, S7}
\includegraphics[height=.35\textheight]{figures/bellik_figure11_s7_bo_midpoint.png}
\label{bos7}
\end{figure}

A complicating factor is that S4 and S6 often fronted the back vowel preceding /o/, probably because the target words contained front rounded vowels in the syllable following /o/, in order to match the real word \textit{broşür} /bro.ʃyr/ ‘brochure’. The nonce words in the condition were \textit{brojörle} /bro.ʒør.le/, \textit{buroçüp} /bu.ro.ʧyp/, \textit{bırojün} /bɯ.ro.ʒyn/.

\begin{figure}
\caption{Tongue body position in the /b-o/ condition, S4}
\includegraphics[height=.35\textheight]{figures/bellik_figure12_s4_bo_midpoint.png}
\label{bos4}
\end{figure}

\begin{figure}
\caption{Tongue body position in the /g-o/ condition, S4}
\includegraphics[height=.35\textheight]{figures/bellik_figure13_s4_go_midpoint.png}
\label{gos4}
\end{figure}


In the /g-o/ condition, most of the length of the tongue body during the ICI in /gro/ is significantly lower than in /gɯro/ and /guro/ for S4 (\figref{gos4}), and lower than in /guro/ only for S7. For the other subjects, however, there are no significant differences in tongue body position between <v>, /ɯ/, and /u/ (\figref{gos5}).

\begin{figure}
\caption{Tongue body position in the /g-o/ condition, S5}
\includegraphics[height=.35\textheight]{figures/bellik_figure14_s5_go_midpoint.png}
\label{gos5}
\end{figure}


\section{Discussion}
This ultrasound study found that tongue body position during the ICI in underlying clusters differs significantly from that of underlying vowels. Generally speaking, anticipatory coarticulation with the following vowel appeared to influence tongue position in the ICI in /Cr/ more than in /CVr/. V2's influence was clearest when the preceding consonant was labial, whereas a preceding dorsal consonant largely blocks the lowering or fronting effect of the following vowel. This is expected, given that /g/ imposes its own target on the tongue body, unlike /b/ and /d/. These ultrasound results support the hypothesis that coarticulatory effects, rather than a gestural target, determine tongue position during acoustic vowels in complex onsets.

The tongue body positions found here indicate that, in underlying clusters, the tongue body is already moving toward its V2 target during the ICI. However, it has not yet attained the following V2’s backness target, as shown by the fact that in /Cri/ words, the tongue is significantly backer during the ICI than it is during /Ciri/. Furthermore, the tongue has not attained its height target yet, since the tongue body is fairly high during the ICI even before a following mid or low vowel (/o/ or /a/). While a following non-high vowel does have a lowering effect, this effect is only clearly seen in the /b/ and to a lesser extent /d/ conditions. In the /g/ conditions, tongue body position during the ICI of clusters is not significantly lower than tongue body position in underlying high vowels for most subjects. This is particularly true for the /gro/ condition.

These gestural findings accord with the acoustic results in \citet{Bellik2018} and \citet{Bellik2019b}, which found that the formant values of the ICI in /Cr/ words differed significantly from those of underlying vowels in the same context, particularly before /i/. Relatedly, the speech style comparison \citep{Bellik2019b} found that intrusive vowels before /i/ become more /i/-like in hypoarticulated speech, where greater gestural overlap is expected. A natural extension of the present study could examine the ultrasound data that were also collected in the hypoarticulated speech style condition to identify the gestural correlates of these acoustic results.


\subsection{Gestural organization of Turkish onsets}
\largerpage
These gradient distinctions between underlying vowels and underlying clusters, which vary according to the consonant context, imply that the gesture for V2 is still in its onset phase during the ICI; it has not attained its target yet. In the language of \citet{Gafos2002} or \citet{Hall:2003}, this suggests a gestural alignment in which the release of the C1 gesture is aligned with the onset of the V2 gesture. Indeed, if C1C2 is syllabified as a complex onset, then we would expect the V2 gesture to be coordinated with the C-center (\citealt{Shawetal2009}, \citealt{BrowmanGoldstein1988}). Firmer conclusions about the relative timing of the gestures involved here, however, require a study of tongue movement over the course of the C(V)rV sequence, going beyond the analysis of tongue position at a single point in time (the midpoint of the ICI, or C-center, examined here). Optical Flow Analysis (\citealt{Barbosa2014}, \citealt{Halletal2015}), for example, could indicate a stable point corresponding to each gestural target in the sequence, and show when the fastest vertical or horizontal changes in tongue position are occurring.

The gestural coordination that produces intrusive vowels seems to be grammaticized in some languages (\citealt{Gafos2002}, \citealt{Hall:2003}), and this is likely the case in Turkish as well. One possibility is that the Turkish grammar of gestural timing prioritizes an anti-phase coordination between the two consonants in the cluster (which pushes them apart in time), over an in-phase coordination between C1 and V (which seeks to synchronize their onsets). The interspeaker variation found in this study suggests that Turkish speakers vary in the gestural coordination they employ in onset clusters. This variation is manifested in the acoustic characteristics of the ICI as well \citep{Bellik2018}. This could reflect individual differences in the coupling strengths assigned to the competing gestures in the onset, as well as individual differences in phonetic implementation, although a study of a greater number of speakers is needed for any solid conclusions about the nature of interspeaker variation on this question.

\subsection{Harmony and syllable structure}
\largerpage
If onset cluster repairing vowels arise from gestural timing relations, then their behavior is irrelevant to the segmental phonology of Turkish, particularly vowel harmony. An intrusive vowel cannot be a target for phonological harmony since it is not a phonological object. Its apparent harmonization (actually coarticulation) does not indicate that Turkish phonological vowel harmony can proceed from right to left, but its failure to harmonize also does not indicate that Turkish phonological vowel harmony is strictly limited to spreading from left to right.

Moreover, since the onset-repairing vowel is not epenthetic, it would seem that Turkish phonology does not categorically prohibit complex onsets in borrowed words, at least in bilinguals like the participants in this study. More work remains to be done, however, to uncover the role that experience with the source languages for these loanwords plays in Turkish speakers' grammatical restrictions on syllable structure. There is some initial evidence that these vowels do not have the same metrical status as full vowels, for a broader range of speakers, based on their text-setting (\citealt[ch.6]{Bellik2019a}).

\subsection{Methodological contribution}

Finally, this project bears on the extensibility of \citet{DavidsonStone2003}'s methodology, in adapting their experimental design but combining it with a more modern statistical technique, SSANOVAs. This comparative ultrasound methodology corroborated the acoustic study that probed the phonological status of vowels in Turkish onset clusters. This study finds a great deal of interspeaker variability in the articulation of these sequences. Further research with more speakers could illuminate factors that may structure this non-contrastive variability.


{\sloppy\printbibliography[heading=subbibliography,notkeyword=this]}
\end{document}
