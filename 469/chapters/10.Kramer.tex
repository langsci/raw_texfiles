\documentclass[output=paper,colorlinks,citecolor=brown]{langscibook}
\ChapterDOI{10.5281/zenodo.14264546}
\author{Martin Krämer\affiliation{UiT The Arctic University of Norway}}

\title{Prokaryotic syllables and excrescent vowels in two Yuman languages}

\abstract{Excrescent vowels in two Yuman languages (Cocopa and Jamul Tiipay) and the phonotactic restrictions for their occurrence show that vowels that fulfil some criteria of excrescent vowels are not always a phonetic reflex without repercussions for syllabification \citep{Hall2006}, but rather signal the presence of an additional, albeit non-canonical, syllable. They are inserted in syllables without a nucleus/mora, which renders them inaccessible for higher level prosodic computation. Degenerate, minor or semisyllables, i.e., syllables without a nucleus, have elsewhere been postulated for stray consonants that add a beat/mora accessible to foot construction. The two Yuman languages discussed here add to the typology of minor syllables by contributing minor syllables with an onset and an optional coda, but without a nucleus or a mora. They also provide evidence for a second type of intrusive vowel.}

\IfFileExists{../localcommands.tex}{
   \addbibresource{../localbibliography.bib}
   \usepackage{langsci-optional}
\usepackage{langsci-gb4e}
\usepackage{langsci-lgr}

\usepackage{listings}
\lstset{basicstyle=\ttfamily,tabsize=2,breaklines=true}

%added by author
% \usepackage{tipa}
\usepackage{multirow}
\graphicspath{{figures/}}
\usepackage{langsci-branding}

   
\newcommand{\sent}{\enumsentence}
\newcommand{\sents}{\eenumsentence}
\let\citeasnoun\citet

\renewcommand{\lsCoverTitleFont}[1]{\sffamily\addfontfeatures{Scale=MatchUppercase}\fontsize{44pt}{16mm}\selectfont #1}
  
   %% hyphenation points for line breaks
%% Normally, automatic hyphenation in LaTeX is very good
%% If a word is mis-hyphenated, add it to this file
%%
%% add information to TeX file before \begin{document} with:
%% %% hyphenation points for line breaks
%% Normally, automatic hyphenation in LaTeX is very good
%% If a word is mis-hyphenated, add it to this file
%%
%% add information to TeX file before \begin{document} with:
%% %% hyphenation points for line breaks
%% Normally, automatic hyphenation in LaTeX is very good
%% If a word is mis-hyphenated, add it to this file
%%
%% add information to TeX file before \begin{document} with:
%% \include{localhyphenation}
\hyphenation{
affri-ca-te
affri-ca-tes
an-no-tated
com-ple-ments
com-po-si-tio-na-li-ty
non-com-po-si-tio-na-li-ty
Gon-zá-lez
out-side
Ri-chárd
se-man-tics
STREU-SLE
Tie-de-mann
}
\hyphenation{
affri-ca-te
affri-ca-tes
an-no-tated
com-ple-ments
com-po-si-tio-na-li-ty
non-com-po-si-tio-na-li-ty
Gon-zá-lez
out-side
Ri-chárd
se-man-tics
STREU-SLE
Tie-de-mann
}
\hyphenation{
affri-ca-te
affri-ca-tes
an-no-tated
com-ple-ments
com-po-si-tio-na-li-ty
non-com-po-si-tio-na-li-ty
Gon-zá-lez
out-side
Ri-chárd
se-man-tics
STREU-SLE
Tie-de-mann
}
   \boolfalse{bookcompile}
   \togglepaper[10]%%chapternumber
}{}

\begin{document}
\maketitle \label{ch10}

\section{Introduction} 
Long sequences of consonants defying the Sonority Sequencing Generalisation (\citealt{selkirk:1984,Zec1988, Zec2007,sonoritycycle}) are attested in a number of languages. They have been analyzed in a range of ways from syllabic, moraic to unsyllabified or in an appendix position (\citealt{Bagemihl1991,Lin1997,Ridouane2008,VauxWolfe2009,Zimmermann2013} among others). Some of the consonant sequences in Cocopa (Yuman; \citealt{Crawford1966}) are described as interrupted by excrescent vocalic offglides. This matches descriptions of similar sequences in other languages with alleged syllabic obstruents, such as Tashlhiyt Berber or Georgian (see \citealt{Easterday2019} for an overview). \citet{Crawford1966} analyzes these as syllables that just contain an onset or an onset and a coda but no phonological nucleus. Such syllables without a nucleus have independently been proposed by \citet{MccarthyPrince1990}, \citet{Broselow1992}, \citet{Shaw1994} and \citet{Repetti1994}. New arguments for this analysis for Cocopa come from recent insights into the distinction between excrescent and epenthetic vowels. Excrescent vowels are determined in their quality by surrounding consonants, and are shorter and less prominent than epenthetic and lexical, i.e., phonologically present vowels (\cite{Hall2006, Hall2011}; see also the discussion in \citealt{Easterday2019}).

If Crawford’s analysis is adopted, the schwa vowels or vocalic offglides are excrescent vowels without a phonological affiliation. \citet{Hall2006}'s assertion that such intrusive vowels are not accompanied by additional syllable structure cannot be upheld (see also \citetv{chapters/08.Hall}, for further refinement of the typology). The excrescent vowels in Cocopa and Jamul Tiipay, which do not have a prosodic association or even a root node, are placed in and signal the presence of an additional syllable \-– albeit one without a nucleus or mora, a prokaryotic syllable. I present data from Cocopa and its sister language Jamul Tiipay in evidence of the presence of an additional syllable containing the consonant(s) separated from a cluster by an apparently excrescent vowel and conclude that the prosodic invisibility of such syllables must be caused by the lack of a nucleus and mora rather than the lack of a syllable. 

Cocopa displays sonority sequencing defying sequences that are not broken up by excrescent vowels, such as presented in \REF{ex1a}, as well as those that receive such a vocalic offglide \REF{ex1b} and \REF{ex1c}. The choice is determined by the presence of intervening sonorants that then serve as codas of the prokaryotic syllables, in avoidance of syllable-internal sonority roller coaster rides, as in \REF{ex1c} or other phonological criteria, such as the avoidance of sequences of identical manner. 

\NumTabs{4}
\begin{exe}
    \ex Cocopa consonant clusters (\citealt{Crawford1966}; accent indicates stress)\footnote{Transcriptions from both sources, \citet{Crawford1966} and \citet{Miller2001}, are adapted to IPA by the author. The excrescent vowels are transcribed as superscript vowels also indicating their quality as transcribed and described by Crawford in Cocopa examples. In examples from Jamul Tiipay, the excrescent vowels are transcribed as schwa, corresponding to the <e> used in Miller's orthographic transcriptions.}
    \begin{xlist}
        \ex \label{ex1a} ps̪kʷá  \tab     ‘I gossip about him’\\
        ksca.ʔárk    \tab ‘dry!’\\
        scxuʔá:k    \tab  ‘she hangs up several (things)’
        \ex \label{ex1b}  ɬʲpᵃm.wák \tab    ‘you are to ride him’\\
        rⁱxúp    \tab    ‘tin can’
        \ex \label{ex1c} pᵃmⁱnṭⁱmá:k \tab  ‘we abandon them’
    \end{xlist}
\end{exe}


The presence of uninterrupted clusters as well as those broken up by vocalic offglides suggests that Cocopa employs both appendixes/complex onsets as well as degenerate syllables to prosodify its consonant sequences.

The situation is similar in the related language Jamul Tiipay \citep{Miller2001}, even though Miller does not claim that they signal the presence of prokaryotic syllables. Jamul Tiipay displays optional as well as obligatory vowel insertion. Vowel insertion is obligatory if sonorants are involved \REF{ex2a} and optional between obstruents \REF{ex2b}, suggesting that the sonority requirements on onsets of degenerate syllables are stricter than on full syllables. Sequences of sibilant and stop are never broken up \REF{ex2c}, suggesting that Tiipay also permits appendixes/complex onsets, though in a much more restricted fashion than Cocopa.

\begin{exe}
    \ex Jamul Tiipay consonant clusters
    \begin{xlist}
        \ex \label{ex2a} /m-ʃ{}-jaːj/   \tab məʃəjaːj     \tab   ‘to be afraid’\\
        /kʷ{}-n-maːw/  \tab kʷənəmaːw  \tab    ‘his/her father's mother’
        \ex \label{ex2b} /x-t̪at̪/   \tab   xt̪at̪ /xət̪at̪  \tab    ‘(someone's) back’\\
        /k-ʃ{}-uː-pit/  \tab  kʃuːpit / kəʃuːpit \tab ‘close it!’\\
        /t̪-t̪-k-juːt̪/  \tab  t̪ət̪əkjuːt̪    \tab    ‘to greet (pl)’
        \ex \label{ex2c} /ʃ-puk/   \tab   ʃpuk  \tab        ‘to lay head on pillow’\\
        /s-pir/   \tab   spir    \tab      ‘to be strong’ \\
        /s-t̪u/     \tab     st̪u     \tab       ‘to pick up, gather, get’\\
        /s-kan/  \tab    skan     \tab     ‘to flee’   
    \end{xlist}
\end{exe}


Tiipay inserted schwa, described as a “non-organic vowel” by \citet{Miller2001}, is also not stressable, as is claimed to be typical for excrescent vowels. 

The excrescent vowels in Yuman are a phonetic side effect of adjustments in syllabic structure to integrate excess consonants into syllables and they are present in non-canonical or prokaryotic syllables. Unstressability is either an effect of the absence of a nucleus (and accordingly a mora) or just the absence of a mora and thus a unit that can receive stress or can be recognized in mora or syllable counting for foot formation.   

The paper is organized as follows. In Section \ref{theoretical} I provide the theoretical background by first summarizing the state of the art of degenerate syllables, concluding that the more appropriate term is prokaryotic, and then discussing the distinction between excrescent and epenthetic vowels. Section \ref{excre} first provides relevant background information on Cocopa and Jamul Tiipay phonology and morphology. It continues with a more detailed discussion of the nature and location of excrescent vowels and argues for strict phonotactics of prokaryotic syllables in Cocopa and more strict phonotactics of prokaryotic syllables in Jamul Tiipay. Section \ref{implic} puts the degenerate syllables proposed for Yuman into a larger typological and theoretical context and Section \ref{kramer:concl} concludes.

\section{Theoretical background}\label{theoretical}
\subsection{Degenerate syllables}\label{degen}

Degenerate, minor or semisyllables have been proposed in several empirical contexts. \citet{MccarthyPrince1990} propose to re-analyze superheavy syllables at the right word edge in Arabic as a heavy maximally bimoraic syllable plus an extrametrical syllable consisting of a consonant only. \citet{Broselow1992} adds to this by postulating degenerate syllables with an onset and those with a rhyme. A moraic consonant in a degenerate consonantal syllable is detected by \citet{Repetti1994} at the end of some words in Friulian. Discussing syllable phonotactics and reduplication in Mon Khmer languages, \citet{Shaw1994} proposes to enrich the impoverished model of the syllable in Moraic Theory (\citealt{Hayes:1989,Zec1995}) by reintroducing the nucleus constituent. In full syllables this constituent is obligatory and associated with at least one mora (Figure \ref{syll_type}a). In minor syllables, that is, stray consonants or consonant sequences, the nucleus is missing (Figure \ref{syll_type}b) or both nucleus and mora are absent (Figure \ref{syll_type}b').

\begin{figure}
\caption{Full and minor syllables \citep{Shaw1994}}
\label{syll_type}
\subfigure[\hfill Full syllable]{
	\begin{forest}
		[σ,s sep=1cm
		[(C),tier=lowest]
		[nuc [µ [V,tier=lowest]]]
		[(C),tier=lowest]
		]
	\end{forest}
}
~
\subfigure[Minor syllable with missing nucleus]{
	\hspace*{.9cm}\begin{forest}
		[σ
			[,phantom[,phantom [C]]]
			[,phantom [µ]]
		]
		\draw (.parent anchor) -- (!111.child anchor);
		\draw (.parent anchor) -- (!21.child anchor);
	\end{forest}\hspace*{.9cm}
}
~
\renewcommand\thesubfigure{(b')}%
\subfigure[Minor syllable with missing nucleus and mora]{
	\hspace*{1.3cm}\begin{forest}
		[σ [,phantom [,phantom [C]]]]
		\draw (.parent anchor) -- (!111.child anchor);
	\end{forest}\hspace*{1.3cm}
	}
\end{figure}

These defective syllables can be referred to as templates in reduplication processes, as well as the prosodification of excess consonants that do not fit into regular syllables for violating Sonority Sequencing \citep{sonoritycycle}, or because they show properties such as compensatory lengthening or otherwise unexpected stress placement that justify their analysis as a separate syllable.

\citet{ChoHolloway2003} propose the term semisyllable, defined as a syllable without a mora and without a coda. They list the following properties of semisyllables.

\begin{exe}
    \ex Properties of semisyllables \citep[187]{ChoHolloway2003}
    \begin{xlist}
        \ex No nucleus
        \ex No codas
        \ex No stress/accent/tone
        \ex Prosodically invisible
        \ex Well-formed onset clusters (observing SSP)
        \ex Restricted to morpheme peripheral positions
    \end{xlist}
\end{exe}


What all these proposals have in common is that the syllable types proposed lack a nucleus, which is why the term \textit{prokaryotic syllable} is more appropriate than the familiar terms \textit{degenerate, minor} or \textit{semisyllable}. To my knowledge, the term prokaryotic has not been proposed before. Given that the consonants in such a syllable are subject to different phonotactic restrictions, as are the onset and the coda of major or full syllables, i.e., those that have a nucleus, it can be assumed that some more subsyllabic structure is present. This goes beyond what Shaw so carefully proposed and is also more than what \citeauthor{ChoHolloway2003}  propose, since, as I will show, prokaryotic syllables may have a coda. Cho \& King listed the absence of codas as one of the characteristics of minor syllables. Accordingly, the constituents rime and coda are given in parentheses in Figure \ref{proka} to indicate their optionality. 

\begin{figure}
\caption{Prokaryotic syllable}
\label{proka}
% % % \includegraphics[width=0.5\textwidth]{figures/kr_proka.png}
\begin{forest}
[σ
    [onset]
    [(rime) [(coda)]]
]
\end{forest}
\end{figure}

\subsection{Epenthetic and excrescent vowels}

\textcitetv{chapters/08.Hall} observes that predictable or inserted vowels behave differently in different contexts and languages. Furthermore, some are invisible to phonological processes such as stress placement or tone assignment. The differences line up into properties that are typical of what she calls intrusive vowels on the one side, and epenthetic vowels on the other. Epenthetic vowels are phonological in the sense that they are affiliated with phonological structure, such as a mora or a syllable nucleus, while excrescent vowels are not. She gives the following catalogue of properties. 

\begin{exe}
    \ex Properties of phonologically invisible inserted vowels (intrusive vowels) \citep[391]{Hall2006}
    \begin{xlist}
        \ex The vowel’s quality is either schwa, a copy of a nearby vowel or influenced by the surrounding consonants.
        \ex If the vowel copies the quality of another vowel over an intervening consonant, that consonant is a sonorant or guttural.
        \ex The vowel generally occurs in heterorganic clusters.
        \ex \label{ex4d}The vowel is likely to be optional, have a highly variable duration or disappear in fast speech rates.
        \ex The vowel does not seem to have the function of repairing illicit structures. The consonant clusters in which the vowel occurs may be less marked, in terms of sonority sequencing, than clusters which surface without vowel insertion in the same language.
    \end{xlist}
\end{exe}

According to Hall, intrusive vowels are not stressable/do not affect stress placement and do not participate in other phonological processes. This is why they are assumed to be devoid of syllable structure. She contrasts these properties with those she attributes to properly phonological epenthetic vowels:

\begin{exe}
    \ex Properties of phonologically visible inserted vowels (epenthetic vowels) \citep[391]{Hall2006}
    \begin{xlist}
        \ex The vowel’s quality may be fixed or copied from a neighboring vowel. A fixed-quality epenthetic vowel does not have to be schwa.
        \ex If the vowel’s quality is copied, there are no restrictions as to which consonants may be copied over.
        \ex The vowel’s presence is not dependent on speech rate.
        \ex The vowel repairs a structure that is marked, in the sense of being cross-linguistically rare. The same structure is also likely to be avoided by means of other processes within the same language. 
    \end{xlist}
\end{exe}

Furthermore, phonologically visible inserted vowels occupy a syllable nucleus, usually are moraic and participate in stress assignment and other phonological processes (e.g., vowel harmony). 

I will show in the following sections that the excrescent vowels of Cocopa and Jamul Tiipay have most of the properties of intrusive vowels, including phonological invisibility, but do signal the separate syllabification of a consonant sequence that would be an illicit tautosyllabic cluster. Consonants are separated by an intrusive vowel when they cannot coinhabit an onset. The separate syllables created for these consonants are not stressable, are ignored in reduplication, but are subject to constraints of syllable phonotactics. We are thus dealing with prokaryotic syllables of the type outlined in Section \ref{degen} that consist of an onset and optionally a coda but no nucleus or mora.

\section{Excrescent and epenthetic vowels in Yuman}\label{excre}

The two languages Cocopa and Jamul Tiipay both belong to the Yuman family. Cocopa is spoken by around 400 people who live north and south of the borders between Mexico, California and Arizona. Jamul Tiipay or just Tiipay is spoken in a neighboring area west of the Cocopa area. For 2007, Ethnologue reported approximately 100 remaining speakers. Diegueño or Kumeyaay, which will be discussed briefly at the end of Section \ref{back}, is spoken north of Tiipay north and south of the border between California and Mexico. In the 1990s there were an estimated 50 native speakers. All Cocopa and Jamul Tiipay data used here come from \citet{Crawford1966} and \citet{Miller2001}, respectively. The inventories of contrastive segments of the two languages are extremely similar, which is why I will present them together in section 3.1. In this section I also discuss relevant aspects of syllable phonotactics, stress and affixation as well as reduplication. Section \ref{cocopa} presents the details of vowel intrusion in Cocopa and their analysis. Section \ref{jamul} presents the vowel intrusion patterns of Jamul Tiipay. Section \ref{summy} summarizes the section.

\subsection{Background} \label{back}

To understand the role of excrescent vowels in these two Yuman languages it is essential to first learn about the basic facts of their phonology and morphology. I will first discuss their segment inventories, with special focus on the phonetics of the vowels, consonant cluster phonotactics and stress and close this subsection with a short discussion of reduplication.

\tabref{tab:Kramer:1} displays the Cocopa consonants as described by Crawford.

\begin{table}
\small\tabcolsep=.66\tabcolsep
\caption{Cocopa consonants (adapted from \cite[25]{Crawford1966})}
\label{tab:Kramer:1}
\begin{tabular}{l *9{l}} 
\lsptoprule
& Labial & Dental & Alveol. & Alveo- & Velar & Labio- & Uvular & Labio- & Glottal\\
&        &        &         & palatal &      & velar  &        & uvular & \\ \midrule
Stops & {p} & {t̪} & {t} & {c} & {k} & {kʷ} & {q} & {qʷ} & {ʔ}\\
Nasals & \multicolumn{1}{r}{m} & \multicolumn{1}{r}{n̪} &  & \multicolumn{1}{r}{ɲ} &  &  &  &  & \\
{Affric.} &  &  &  & {t͡ʃ} &  &  &  &  & \\
{Fricat.} &  & {s̪} & {s} & {ʃ} & {x} & {xʷ} &  &  & \\
{Lat.\,fric.} &  &  & {ɬ} & {ɬʲ} &  &  &  &  & \\
{Lateral} &  &  & \multicolumn{1}{r}{l} & \multicolumn{1}{r}{ʎ} &  &  &  &  & \\
{Rhotic} &  & \multicolumn{1}{r}{r̪} & \multicolumn{1}{r}{r} &  &  &  &  &  & \\
{Glides} &  &  &  & \multicolumn{1}{r}{j} &  & \multicolumn{1}{r}{w} &  &  & \\
\lspbottomrule
\end{tabular}
\end{table}

Cocopa has a slightly larger consonant system than Jamul Tiipay, which lacks the uvular stops and does not distinguish between two rhotics. Crawford also describes an additional coronal stop and fricative for Cocopa.

\tabref{tab:Kramer:2} is adapted from \citet[39ff]{Miller2001}. Neither language has a laryngeal contrast, but nevertheless both have a sizeable consonant inventory.

\begin{table}
\small
\caption{Jamul Tiipay consonants (adapted from \cite[39]{Miller2001})}
\label{tab:Kramer:2}
\begin{tabular}{l *7{l}} 
\lsptoprule
& Labial & Dental & Alveol. & Alveo- & Velar & Labio- & Glottal\\
&        &        &         & palatal &      & velar  & \\\midrule
{Stops} & {p} & {t̪} &  &  & {k} & {kʷ} & {ʔ}\\
{Nasals} & \multicolumn{1}{r}{m} & \multicolumn{1}{r}{n̪} &  & \multicolumn{1}{r}{ɲ} &  &  & \\
{Affricate} &  &  &  & {t͡ʃ} &  &  & \\
{Fricatives} &  & {s̪} &  & {ʃ} & {x} & {xʷ} & \\
{Lat. fric.} &  &  & {ɬ} & {ɬʲ} &  &  & \\
{Lateral} &  &  & \multicolumn{1}{r}{l} & \multicolumn{1}{r}{ʎ} &  &  & \\
{Rhotic} &  &  & \multicolumn{1}{r}{r} &  &  &  & \\
{Glides} &  &  &  & \multicolumn{1}{r}{j} &  & \multicolumn{1}{r}{w} & \\
\lspbottomrule
\end{tabular}
\end{table}

Crawford describes three contrastive vowels for Cocopa and Miller discusses the status of a fourth one. Some schwas, she claims, are not predictable and therefore have to be analyzed as present in the lexicon. \tabref{tab:Kramer:3} is adapted from \citet[12]{Miller2001}.

\begin{table}
\caption{Cocopa and Jamul Tiipay vowels (\citealt{Crawford1966,Miller2001})}
\label{tab:Kramer:3}
\begin{tabular}{ccc}  
\lsptoprule
{Front} & {Central} & {Back}\\
{i, iː} &         & {u, uː}\\
        & {ə}     & \\
        & {a, aː} & \\
\lspbottomrule
\end{tabular}
\end{table}


\citet[13]{Crawford1966} discusses two additional vowels. /e/ occurs in Spanish loans and is consistently mid to upper-mid unrounded, unless it is replaced by Cocopa /i/, which many speakers do. The other vowel, /o/, is only found in one interjection that expresses “frustration or disappointment”. 

In both languages, the realization of the three contrastive vowels depends considerably on their environment. In Cocopa, the front high vowel is a bit centralized when preceded or followed by an alveolar consonant. The long front vowel is a bit higher than the short one. This difference is not reported for the back high vowels. These vary in height and are lowered slightly when followed by another vowel with only one intervening consonant or when followed by a sibilant. Crawford only gives examples with following low vowels for the first condition. There might actually be some kind of height harmony operative here. The low vowel is lowest when preceding the stressed vowel, only separated from it by one consonant, and slightly raised and fronted when preceded by a palatal consonant and even more so when surrounded by palatals. Elsewhere, it is a central low vowel with the long one a bit lower than the short.

The intrusive vowel is described as a vowel similar to /u/ when followed by a labiovelar consonant, including /w/, an /i/-like vowel when followed or preceded by a palatal or dental before any consonant except the labiovelars, and as a schwa-like vowel in all other environments (\cite[38]{Crawford1966}, see also \citetv{chapters/03.Mansfieldetal}, for similar environmental colouring of inserted vowels). He transcribes them as superscript \textit{i}, \textit{u} and \textit{a}, respectively.

\citet[20]{Miller2001} gives a similar description of what she calls “inorganic” schwa, the vowel that “is inserted between consonants to break up clusters” in Tiipay: it is never long and never stressed. Its quality is determined by surrounding consonants, resulting in [ɪ], [ʊ], [ə]. Whenever two conditions overlap schwa may be realized as any of the available options, e.g., [ɬʲəxʷiːw] ‘skunk’; this vowel could be realized as [ɪ] or [ə] because it is preceded by a palatal consonant (a palatal voiceless lateral fricative) or it can be realized as [ʊ] because it is followed by a labialized velar [xʷ]. If schwa is only separated from the following vowel by a glottal stop, it may be realized as a copy of the short allophone of that vowel. The three lexical vowels vary according to environment in a similar way as in Cocopa. Since some schwas are unpredictable, such as those in the left column in \REF{ex:Kramer:6}, Miller assumes them to be present in lexical representations. The forms on the right serve to illustrate the same phonotactic context without a schwa. The last form shows a context for optional vowel intrusion, i.e., a prokaryotic syllable.

\NumTabs{4}
\begin{exe}
    \ex \label{ex:Kramer:6}Unpredictable schwas in Jamul Tiipay \citep[21]{Miller2001}\\
    aləmi\tab    ‘beard’\tab       xalma \tab        ‘gourd rattle’\\
    wanəpu\tab   ‘buttocks’\tab     xənpaɬ\tab         ‘tongue’\\
    xəmuk\tab  ‘to be three’\tab  xəmuɬ / xmuɬ\tab    ‘to be foamy’
\end{exe}


Cocopa complex onsets can consist of almost any sequence of up to four obstruents, with the following restrictions. If a voiceless lateral is involved, it is cluster-initial. If a glottal stop is involved, it is final in the cluster. Sequences of stops are not attested, but sequences of fricatives are. If there is a nasal it is final too. The only examples for complex onsets with a sonorant other than a nasal are loanwords and have the sonorant as the internal member. Complex onsets of unstressed syllables are slightly more restricted. For example, they do not contain a glottal stop.

\begin{exe}
    \ex \label{ex7}Cocopa stressed complex onsets
    \begin{xlist}
        \ex Sequences of obstruents\\
        xps̪íw\tab     ‘be blue, green’\\
        ps̪kʷá \tab  ‘I gossip about him’\\
        ɬʲksís  \tab   unidentified plant species \\
        pscʔáːw  \tab ‘I have them as daughters’ \\
        scxʔúːɲ  \tab  ‘yellowshafted flicker’\\
        xs̪áːm \tab    ‘be almost’\\
        s̪xʈú  \tab   ‘I spit’        
        \ex Rising sonority \\
         ʃmá \tab    ‘I sleep’\\
         ɬʲmár  \tab   ‘I light a fire’ \\ ɬʲs̪míx \tab ‘I intend to lay something big in’\\
         ɬʲjúːm  \tab ‘I think’\\
         tréːn  \tab   ‘train’\\
         krúːs̪   \tab  ‘cross’
    \end{xlist}
\end{exe}


Jamul uninterrupted initial clusters, as already indicated in the introduction, are much more restricted. They all start in a sibilant and the internal consonant is always a stop.

\begin{exe}
\NumTabs{5}
    \ex  \label{ex8} Jamul Tiipay uninterrupted initial consonant clusters \\
    /ʃ-puk/ \tab   ʃpuk   \tab ‘to lay head on pillow’\\
    /s-pir/ \tab   spir  \tab  ‘to be strong’\\
    /s-tu/  \tab      stu  \tab    ‘to pick up, gather, get’\\
    /s-kan/   \tab skan  \tab  ‘to flee’    \\  
    /ʃ{}-ʈu/  \tab    ʃʈu  \tab    ‘to shove (with hands or instrument)’
\end{exe}


All other sequences are broken up at least optionally by a schwa vowel, as will be discussed in the next subsection. 

\begin{sloppypar}
On stress, \citet[28]{Crawford1966} states that there are three levels, strongly stressed, stressed and unstressed. Within what he calls a “macrosegment”, which I interpret to roughly coincide with a word, there is usually only one stressed syllable. However, he also states that within a macrosegment with more than one syllable preceding the stressed syllable, “the first unstressed syllable has a slightly stronger stress than the following unstressed syllables” \citep[29]{Crawford1966}. Crawford is very clear about the unstressability of prokaryotic syllables: “A stressed or strongly stressed syllable can only be one which contains a vowel.” And on syllables he states that “[a] syllable can be entirely consonantal and consist of an onset only or of an onset and a coda with a predictable ‘murmur’ vowel following the onset as phonetic peak.” \citep[34]{Crawford1966} According to \citet{Miller2001}'s description of stress in Jamul, and in Yuman in general \citep{Langdon1975}, stress on schwa syllables is not an option because of the morphological nature of stress placement. Stress is always placed on the morphological root of a word, which predominantly has the shape (C)V(C). Since the morphology is mostly prefixing, this results in word-final stress in most cases. In the few cases of bigger roots, as illustrated in \REF{ex8}, stress is still on the last vowel and schwas are not stressed. 
\end{sloppypar}

Cocopa verb root reduplication is a semi-productive process. I consider it semi-productive because many reduplicated verbs do not have a non-reduplicated base form. Many do, however, and we can observe some regularities that indicate that it is impossible to reduplicate consonantal syllables. The preferred verb root for reduplication is of the form CVC(C). Roots with complex codas can be reduplicated, while verbs with complex onsets are not reduplicated. Initial consonant clusters arise only when an instrumental prefix is added to the reduplicated form. In this case inflection for person is possible, while the other reduplicated forms are impersonal uninflected forms. Inflection is realized on an adjacent auxiliary. One instrumental prefix exemplified is of the form CV- and the other is a sibilant. This sibilant causes either deletion of the reduplicant-initial (root) consonant or its alternation from a fricative into a stop, creating either a simple onset or a cluster, adhering to the restrictions for such clusters found in Jamul \REF{ex8}. 

\begin{exe}
\NumTabs{4}
    \ex \label{ex9}Cocopa prefix-reduplicant interaction\\
    ʃírmír\tab  ‘I take aim’ (probably from \textit{mírmír i} ‘to be straight’)\\
    skárxár/sxárxár \tab ‘I break into small pieces’
\end{exe}

Thus, none of the many consonantal prefixes that would create complex onsets or prokaryotic syllables are reduplicated or even used to further derive or inflect reduplicated forms.

Jamul reduplication is similarly restricted and unproductive. The base is maximally CVCC, as in \REF{ex10a} and \REF{ex10b}.  Of the 23 reduplicated verb stems Miller collected, only two forms have a prefix, and even there it is prefixed to the reduplicated form, as shown in \REF{ex10c}. 

% \TabPositions{2cm,6cm}
\begin{exe}
    \ex \label{ex10}
    Reduplication in Jamul Tiipay
    \begin{xlist}
        \ex \label{ex10a}milmil \tab    ‘to be narrow’
        \ex \label{ex10b}aʃkaʃk-i  \tab   ‘to go up and down, back and forth’
        \ex \label{ex10c}t͡ʃəxəlxul \tab   ‘to gargle’ cf.        təkəlkul      ‘to pile (things) up’
    \end{xlist}
\end{exe}


Verbs such as [txiːl] ‘to get dressed, wear clothes’ do not seem to undergo reduplication, not even with a reduced reduplicant (e.g., *[xiːl-txiːl])

The alternations observed in Cocopa \REF{ex9} suggest that the restriction of reduplication on verb roots with simple onsets is a phonological one, and that the many stems that are formed with derivational consonantal prefixes, such as causatives, do not undergo reduplication because complex onsets and minor syllables are banned in the reduplicant.  

Compare these reduplication patterns with those found in the sister language Diegueño/Kumeyaay \citep{Langdon1966}, which displays almost identical patterns of schwa insertion. Langdon reports to have found many reduplicated forms. Almost all reduplicate only the stem syllable, as in the other two languages. There is, however, a very small set of bisyllabic reduplicants. Interestingly, in three of the four forms Langdon found, the initial vowel is a schwa.

\TabPositions{2.2cm,6cm}
\begin{exe}
    \ex Bisyllabic reduplicants in Kumeyaay \citep[202]{Langdon1966}\\
    kuɬaːɬ kuɬaːɬ\tab    ‘to go up and down (like when riding a horse)’\\
      ɬəxup ɬəxuːp \tab    ‘holes all over’ (cf. ɬəxup  ‘hole, cave’)\\
      səkap səkaːp  \tab  ‘half and half, to be more than half full’ (cf. səkap ‘to be half’)\\
      xəkaɬ xəkaːɬ \tab   ‘to be scalloped, uneven at the edges, to have teeth missing’
\end{exe}

The fact that there are bisyllabic reduplicants in Kumeyaay and that they contain schwas can be counted as weak evidence that the schwas in Kumeyaay are phonological, unlike those in Cocopa and Jamul Tiipay. Presumably, reduplication targets prosodic structure, such as moras.

As illustrated in \REF{ex7} and \REF{ex8}, the restrictions on initial clusters differ in Cocopa and Jamul. In the following we will more closely examine the insertion sites for intrusive vowels in both varieties and conclude that also the phonotactic constraints on prokaryotic syllables differ slightly.

\subsection{Cocopa intrusive vowel landing sites}\label{cocopa}

The intrusive vowel prevents a sonority rise and consecutive fall, as illustrated in \REF{ex12}. Minor syllables in Cocopa can have complex onsets, just like major syllables, i.e., containing two obstruents \REF{ex12a}. A vowel is inserted if either a sonorant is followed by an obstruent or vice versa \REF{ex12b}. The forms in \REF{ex12b} could theoretically be syllabified with fewer inserted vowels, i.e., fewer prokaryotic syllables, as indicated by the conceivable but unattested forms marked with a question mark in \REF{ex12b}. This would, however, result in obstruent codas followed by sonorant onsets, as the question marked forms show. Such rising sonority profiles across syllables violate the Syllable Contact Law \citep{MurrayVenneman1983}, according to which sonority should fall from one syllable to the next. Clusters with variable intrusion sites involve sonorants and either syllable contact created is wellformed \REF{ex12c}.

\TabPositions{4cm,6cm,8cm,10cm}
\begin{exe}
    \ex \label{ex12} Prokaryotic syllable phonotactics
    \begin{xlist}
        \ex \label{ex12a} sxᵃm.pá   \tab ‘yellowjacket’\\
        ɬʲpᵃm.wák   \tab ‘you are to ride him’\\
        pʃkᵘ.wáːkˣ \tab ‘we intend to return him’
        \ex \label{ex12b} mᵃ.kⁱ.ɲáːp \tab    {}'you relate'  \tab   \tab        ?mᵃk.ɲáːp \\ pᵃ.mⁱn.ʈⁱ.máːk \tab 'we abandon them' \tab    ?pᵃm.nⁱʈ.má:\\ mⁱ.cⁱm.pᵃ.káːwc \tab 'you meet each other' \tab  ?mⁱc.mⁱp.káːwc
        \TabPositions{5cm}
        \ex \label{ex12c} ɲⁱm.ɲⁱ.kʷájs / ɲⁱ.mⁱɲ.kʷájs   \tab  ‘we are your mother's brothers’\\
        ɲⁱɬʲ.mwa.jáːc / ɲⁱ.ɬ\textsuperscript{ji}m.wa.jáːc  \tab  ‘you are around in it’
    \end{xlist}
\end{exe}

As in major syllables, sequences of stops are avoided (unless the last stop is a glottal stop). Obstruent-sonorant sequences are avoided too, suggesting that rising sonority in complex onsets is marked and restricted to loanwords. 


\TabPositions{2cm, 4cm,6cm,8cm,10cm}
\begin{exe}
    \ex \label{ex13}Stop-stop and obstruent-sonorant onsets are avoided\\
    ʈⁱ.ʈʔá:p  \tab   ‘I turn something upside down’\\
    pᵃ.qⁱ.la.ʃáw  \tab ‘he cleaned him’\\
    ʈⁱ.má:j  \tab   ‘waves of the ocean’\\
    pⁱ.lík  \tab    ‘I taste’
\end{exe}

Cocopa allows both complex onsets as well as, presumably, appendix plus onset initial clusters, with the appendix filled with a sibilant. Clusters that exceed these structures with maximally three consonants, and clusters that do not conform to the sonority restrictions and the manner \textsc{ocp} banning consecutive stops, are divided up into prokaryotic syllables, which can have complex onsets and codas. The minor syllables are more restricted in Jamul Tiipay as are regular syllables.

\subsection{Jamul Tiipay prokaryotic syllable phonotactics}\label{jamul}
The schwa vowel emerges between stops and between sonorants, between stops and sonorants but not glides, between sonorants and obstruents, and sonorants and sonorants \REF{ex14a}. It does not occur between sibilants and stops \REF{ex8}, but between sibilants and glides, as illustrated in the second example in \REF{ex14}. \REF{ex14b} shows that also string-internally sibilant-stop sequences are tolerated, as word-initially \REF{ex8}.

\NumTabs{4}
\begin{exe}
    \ex \label{ex14} Jamul Tiipay cluster resolution
    \begin{xlist}
        \ex \label{ex14a} /t-ɲur/    \tab       təɲur   \tab      ‘to curl (hair), to decorate’\\
        /m-ʃ{}-jaːj/   \tab     məʃəjaːj   \tab   ‘to be afraid’\\
        /kʷ{}-n-maːw/  \tab    kʷənəmaːw  \tab  ‘his/her father's mother’\\
        /t-t-k-juːt/   \tab     tətəkjuːt  \tab    ‘to greet (pl)’\\
        /ɲ-ʃ{}-k-ʔ{}-mak/    \tab  ɲəʃkəʔmak  \tab  ‘s/he took it away from me’
        \ex \label{ex14b} /k-s-kan/    \tab     kəskan    \tab  ‘run away!’\\
        /m-ʃ{}-t-uː-jaj/  \tab     məʃtuːjaj \tab   ‘to be afraid (pl)’\\
        /ɲ{}-ʃ{}-p-aː{}-ʔ{}-ʔáːw-a/  \tab ɲəʃpaʔáːwa \tab ‘they made us stand up’
    \end{xlist}
\end{exe}

The glottal stop behaves differently in that it can precede any consonant but not follow a consonant.

\begin{exe}
    \ex \label{ex15} No insertion between glottal stop and other Cs\\
    /ɲkʔ-wiːw/   \tab      ɲəkəʔwiːw \tab   ‘look at me!’\\
    /ɲ-ʃ{}-k-ʔ{}-mak/   \tab    ɲəʃkəʔmak  \tab  ‘s/he took it away from me’
\end{exe}

Unlike Crawford, Miller distinguishes between obligatory schwa insertion (\ref{ex14} and \ref{ex15}) and optional intrusion \REF{ex16}. Miller does not always give two forms in these examples. From the fact that she lists them in this context, I conclude that they display optional schwa.

\begin{exe}
    \ex \label{ex16} Variable epenthesis between two voiceless obstruents\\
    /x-tat/    \tab    xtat /xətat      \tab     ‘(someone's) back’\\
    /p-ʔaw/  \tab        pʔaw     \tab           ‘to stand, step; (for rain) to fall’\\
    /t-k-aː-xaːp/  \tab    tkaːxaːp  \tab           ‘bracelet’\\
    /k-ʃ{}-uː-pit/    \tab   kʃuːpit/ kəʃuːpit \tab      ‘close it!’\\
    /t͡ʃxlkaj/   \tab      t͡ʃxəlkaj   \tab           ‘kidneys’\\
    /t͡ʃ-k-uːjaw-a/  \tab   t͡ʃkuːjawa    \tab        ‘to teach’\\
    /s-naːj/   \tab      snaːj      \tab         ‘to dip up (water)’\\
    /t͡ʃ-mi/   \tab      t͡ʃmi         \tab      ‘to lay (long or large object) down’
\end{exe}

In longer sequences we find fewer intrusive vowels than expected, suggesting that there are limits to the number of prokaryotic syllables in a row. The occasional transcription of an optional schwa in string-internal sequences of sonorants followed by obstruents shows that internal clusters are preferably syllabified tautosyllabically as onsets and not heterosyllabically as coda-onset sequences. However, it is also noteworthy that none of the given forms has a word-initial complex onset to a prokaryotic syllable. Word-initial clusters are all followed by a full vowel. This differs from Cocopa, where we find up to three consonants followed by an intrusive vowel \REF{ex12a}.


\TabPositions{2.5cm, 6.4cm,8cm,10cm}
\begin{exe}
    \ex \label{ex17}Bigger clusters\\
    /t-t-xʷak/   \tab   tətxʷak     \tab   ‘to break (brittle object) (pl)’\\
    /m-ɲ{}-kurʔak/ \tab  məɲkurʔak    \tab     ‘your husband’\\
    /m-ʎ{}-piʃ/  \tab  məʎpiʃ/məʎəpiʃ     \tab  ‘you are small’\\
    /t͡ʃ-k-piːk/  \tab   t͡ʃəkpiːk     \tab       ‘to squash many’\\
    /t-k-xap/   \tab    təkxap   \tab         ‘to put on, wear (bracelet, ring,\tab \tab \tab shirt,  eyeglasses)’\\
    /k-t-k-xap/  \tab   kətkəxap/kətəkxap  \tab   ‘put (bracelet, ring, shirt, eyeglasses) \tab \tab \tab  on!’\\
    /m-m-ʃ{}-jaːj/  \tab    məmʃəjaːj məməʃəjaːj \tab  ‘you are afraid’\\
    /ɲk-m-ʃ{}-ʔ{}-jaːj/ \tab ɲəkəmʃəʔjaːj    \tab       ‘be afraid of me!’\\
    /m-m-ʃ{}-kʷaɬʲ/   \tab məmʃəkʷaɬʲ/məməʃkʷaɬʲ \tab ‘you bother him/her;\tab \tab \tab \tab  s/he bothers you’\\
    /ɲm{}-m-ʃ{}-kʷaɬʲ/ \tab ɲəməməʃkʷaɬʲ  \tab    ‘you bother me’
\end{exe}

\citet{Miller2001} analyzes the cluster patterns with an across{}-the{}-board schwa epenthesis rule and several schwa deletion rules that apply optionally, resulting in optionality for most of the inserted schwas. In addition, these rules also apply either rightwards or leftwards in a cluster. Thus, the two forms of ‘you bother him/her; s/he bothers you’ in \REF{ex17} are the result of rightward or leftward application of a late schwa deletion rule scanning the result of the general schwa insertion rule {\textbar}mə{}-mə{}-ʃə{}-kʷaɬʲ{\textbar}. We can thus add another property of intrusive invisible vowels, optionality \REF{ex4d}.

\subsection{Summary}\label{summy}

The stress and reduplication patterns of Cocopa and Jamul Tiipay do not show any sign that the syllables with intrusive vowels are accessible by higher level prosody. Closer inspection of the distribution of intrusive vowels and where they are optional and where not reveals restrictions on onsets and codas, which differ between regular and prokaryotic syllables as well as across the two languages. Cocopa regular onsets display complexity, allowing for proper complex onsets with rising or plateauing sonority and an \textsc{ocp} constraint that bans sequences of stops as well as sequences of fricatives. Prokaryotic syllables in Cocopa can have complex onsets consisting of obstruents alternating in continuancy and they can have simple codas. Regular onset phonotactics are stricter in Jamul Tiipay, permitting basically only sibilant-stop clusters and prokaryotic syllables seem to allow complex onsets only under duress, i.e., to avoid codas or too many consecutive prokaryotic syllables. 

\section{Theoretical implications}\label{implic}

The Yuman patterns discussed here contribute to the understanding of minor syllables on the one hand and of excrescent vowels on the other. 

\subsection{ Non-canonical syllables}

I will not try to integrate all proposals of consonants dominated only by a syllable node or a mora or no syllable structure or an appendix into one model of a typology of prokaryotic syllables. These are in many cases competing proposals. However, a few remarks are in order. 

The appendix (e.g., \citealt{VauxWolfe2009} and references cited there) is an additional position that can be attached to the syllable, preceding the onset or a higher-level prosodic category, such as the foot or the word. In the analysis of Indo-European languages, this is usually invoked to account for sibilant-initial clusters that violate sonority sequencing. Clusters of rising sonority are treated as complex onsets. The only segment allowed in this position is accordingly a sibilant, and usually only one, as in English. The first segment in complex onsets is usually an obstruent and the second segment is a sonorant or in more restrictive languages a non-nasal sonorant. We have seen the former, i.e., sibilant-initial clusters, in Jamul Tiipay. They are not only highly restrictive in the first position, allowing only sibilants, but also in the second position, in which we find only stops. The cross-linguistically widespread rising sonority clusters are not attested or only in recent loanwords. In Cocopa, we find more combinations of obstruents in initial clusters. Only consecutive stops are avoided. Such obstruent clusters can have more than two members. 

An analysis of the sibilant-initial clusters in Jamul Tiipay as appendix plus simple onset is an obvious choice. Whether the Cocopa obstruent sequences are appendixes plus onset or complex onsets with a strict requirement for flat and low sonority is a more intricate issue. We will not solve this here, since these elaborations only serve to rule out an appendix analysis for the clusters broken up by excrescent vowels. If an intrusive vowel can or must be inserted between an appendix and the following onset consonant, it should also be attested in the sibilant + stop sequences in Jamul Tiipay. This is not the case. The vowel intrusion patterns can thus not be analyzed by the stipulation of an appendix position for each consonant preceding an intrusive vowel. Assuming that several appendixes can precede the first syllable of a word, one would also not expect any clusters of two consonants inside a sequence of appendixes, as in the second form in \REF{ex12b} or those in \REF{ex12c}. This suggests that there is more elaborate syllable structure than just a sequence of appendixes.  

The Yuman prokaryotic syllable is also different from the minor syllables that have been proposed as the weak part of the sesquisyllable in Southeast Asian languages (\citealt{Matisoff1973,Shaw1994}, for a recent discussion see \citealt{Butler2014}). The minor (or half) syllable in a sesquisyllabic word precedes a full or major syllable. These arguably form iambic feet together and the minor syllable might even bear a tone (\citealt{SvantessonKarlsson2004,Butler2014}). The minor syllable in sesquisyllables thus has to have prosodic structure that makes it visible for footing and that makes it a licit tone-bearing unit. The mora is usually assumed to be the relevant unit in iambic feet and tone association.

Yuman prokaryotic syllables contribute neither to foot construction nor can they be said to be prosodically active in any other way. Their only purpose is to avoid illicit consonant sequences within canonical syllables.

\subsection{ Excrescence and epenthesis}
\largerpage
As has been argued at length here, the schwa vowels of Cocopa and Jamul Tiipay emerge between the constituents of syllables without nuclei. They are not inserted to repair an illicit structure, but they do emerge as a side effect of such an adjustment in avoidance of marked or ungrammatical phonotactics. They thus do not comply with one of the central criteria invoked by Hall for the diagnosis of excrescent/intrusive vowels. They do, however, fulfil other important criteria. They are short, reduced, unstable, and variable, and their quality is dependent on that of neighboring consonants, i.e., their immediate phonetic neighborhood. 

Considering how the schwas of Cocopa and Jamul Tiipay have properties of both phonetic and phonological inserted vowels and lack properties of each type, it is tempting to propose the existence of a third category. Using the existing terminology, we could distinguish between three types of inserted vowels – epenthetic vowels that are integrated in the phonological structure, excrescent vowels that signal additional phonological structure, and intrusive vowels, which are phonologically irrelevant. The latter two categories are both phonologically invisible in that they are not available for the phonology, e.g., to house tones, contribute to foot construction or participate reliably in vowel harmony. However, this would be premature. The most important distinction is whether a vowel is affiliated with phonological structure, i.e., parsed in a syllable nucleus, or not. In addition, I think one can reasonably argue that the emergence of the excrescent vowels is a side effect of gestural timing, similar to that of Hall’s intrusive vowels.

The presence of the additional onset to house the first consonant(s) of a sequence that would not constitute a well-formed onset with this initial consonant, has phonetic consequences. The final consonant in an onset is expected to have a substantial release phase, especially if it is a stop, which is expected to turn into a vowel, which always occupies the nucleus of a major or canonical syllable. The consonant in a prokaryotic syllable is also in an onset, it is just not followed by a nucleus. One can, however, assume that the articulatory targets are determined by its position in the syllable structure and that it behaves in the same way as a consonant in a major syllable does. The only difference is that the onset consonant in a prokaryotic syllable is not followed by a nucleus and thus no phonological vowel. It is thus the automatic articulatory mechanics at the end of the consonant in an onset that makes observers perceive a vocalic offglide or schwa-like vowel. A coda consonant on the other hand is not expected to have much of a release. In many languages, word-final stops do not have an audible release at all. The articulatory “habits” for onsets thus facilitate the emergence of a following transitional or excrescent vocoid, while the articulation patterns for coda consonants do not easily provide space for such a non-phonological vocoid. 

An excrescent vowel in a prokaryotic syllable is thus not the result of an overlap of the transition between two consonants and the opening gesture of a phonological vowel. Its perception is, however, the result of gesture coordination determined by the prosodic structure associated with the surrounding segments and thus a phonetic by-effect of abstract phonological representations.

Accordingly, it is more appropriate to divide inserted vowels into epenthetic and excrescent vowels, and the latter into those caused by mere gestural coordination and those caused by the mapping of abstract phonological representations to articulatory actions.   

\section{Conclusions} \label{kramer:concl}
\largerpage
Excrescent vowels in Cocopa emerge in response to sonority fluctuation inside consonant clusters and thus signal the presence of a pro\-karyotic syllable. Jamul Tiipay schwa insertion and variability is similarly conditioned by the sonority of surrounding consonants. Hall's main argument for assuming the consonants flanking excrescent vowels to not project a separate syllable was their inactivity in stress placement and other prosodic patterns. This inactivity of prokaryotic syllables is explained here by their prosodic deficiency causing their inability to contribute to higher level prosodic structure: Prokaryotic syllables, i.e., those with an optional excrescent vowel, do not have a nucleus and do not project a mora, as proposed by \citeauthor{ChoHolloway2003}. In contrast to what Cho \& King stipulated, however, Cocopa prokaryotic syllables can have a coda. If the distinction between obligatory and optional schwa observed by \citeauthor{Miller2001} in Jamul Tiipay signals a phonological difference, we are most probably dealing with defective syllables which contain a nucleus but do not project a mora in the case of obligatory schwas. Thus, they too are ignored in the computation of stress, feet, or other syllable counting operations. 

There are two main results of this study. First, with the help of excrescent or intrusive vowels, we can recognize prokaryotic syllables. These syllables are inaccessible for prosodic computation because they lack a nucleus and a mora, but they are syllables because they are subject to syllable phonotactic constraints on onsets and codas. Cho \& King’s definition of what they call semisyllables thus must be broadened to include prokaryotic syllables with a coda.

Second, there are two types of intrusive (non-phonological) vowels. The first type are those described by \citeauthor{Hall2006} as stemming from gestural overlap of the vowel within a syllable with the transitions between consonants within that syllable. The second type, described here, emerges as a phonetic effect of standard articulatory patterns in the realization of consonants in specific syllable positions, in this case the rightmost consonantal position in an onset.



%\section*{Contributions}
%John Doe contributed to conceptualization, methodology, and validation.
%Jane Doe contributed to the writing of the original draft, review, and editing.

{\sloppy\printbibliography[heading=subbibliography,notkeyword=this]}
\end{document}
