\documentclass[output=paper,colorlinks,citecolor=brown]{langscibook}
\ChapterDOI{10.5281/zenodo.14264550}
\author{Luc Baronian\affiliation{UQAC} and Nicolas Royer-Artuso\affiliation{UQAC/EFMI}}

\title{Textsetting the case for epenthesis in Armenian}

\abstract{The authors analyze the textsetting of an Armenian song (\textit{Ooska gukas}, recorded by the Gomidas Band in Philadelphia on Roulette Records, 1963) that shows schwas within words where they are unexpected when compared to the standard language and known dialects of Armenian. The beat of the song is a 10/16 djurdjuna in Ottoman music. The authors demonstrate that schwa epenthesis is used by the singer (Roger Mgrdichian 1930--2019) as a the main strategy to fill in beats with additional syllables in the textsetting process, within certain consonantal contexts. Besides the intrinsic interest of the textsetting process in an Armenian dialect that is now nearly extinct, this case study strengthens the point of view that schwa epenthesis is an active and productive process in Armenian, suggesting also that moraicity plays a role in the language’s prosody. This is not to say that some Armenian schwas cannot be lexicalized or morphologized, but that epenthesis is live enough to be used in creative ways by speakers when playing with language.}


\IfFileExists{../localcommands.tex}{
   \addbibresource{../localbibliography.bib}
   % add all extra packages you need to load to this file

\usepackage{tabularx,multicol}
\usepackage{url}
\urlstyle{same}

\usepackage{listings}
\lstset{basicstyle=\ttfamily,tabsize=2,breaklines=true}

\usepackage{langsci-basic}
\usepackage{langsci-optional}
\usepackage{langsci-lgr}
\usepackage{langsci-osl}
% \usepackage{./langsci/styles/langsci-lgr}
% \usepackage{./langsci/styles/langsci-osl}
% \usepackage{langsci-gb4e}

\usepackage{tikz}
\usetikzlibrary{patterns,calc}
\pgfdeclarepatternformonly{south east lines}{\pgfqpoint{-0pt}{-0pt}}{\pgfqpoint{3pt}{3pt}}{\pgfqpoint{3pt}{3pt}}{
    \pgfsetlinewidth{0.6pt}
    \pgfpathmoveto{\pgfqpoint{0pt}{3pt}}
    \pgfpathlineto{\pgfqpoint{3pt}{0pt}}
    \pgfpathmoveto{\pgfqpoint{.2pt}{-.2pt}}
    \pgfpathlineto{\pgfqpoint{-.2pt}{.2pt}}
    \pgfpathmoveto{\pgfqpoint{3.2pt}{2.8pt}}
    \pgfpathlineto{\pgfqpoint{2.8pt}{3.2pt}}
    \pgfusepath{stroke}}
    
\usepackage{stmaryrd}
\usepackage{wasysym}
\usepackage{multirow}
\usepackage{caption}
\usepackage{subcaption}
\usepackage{mathrsfs}
\usepackage{qtree}

\usepackage{linguex}


   %pminos do not split footnotes
% \interfootnotelinepenalty=10000 %Footnote in Laporte chapters has to be split SN


%\DeclareIndexNameFormat{default}{%
%\nameparts{#1}%
%\usebibmacro{index:name}%
%{\index[names]}%
%{\namepartfamily}%
%{\namepartgiveni}%
% {}% L1
% {}% L2
%{\namepartprefix}% generates spurious space L3
%{\namepartsuffix}% generates spurious space L4
%}

%  {\DeclareIndexNameFormat{default}{%
%     \usebibmacro{index:name}{\index[names]}{#1}{#3}{#5}{#7}}}

%\DeclareIndexNameFormat{default}{%
%  \usebibmacro{index:name}{\sindex[nom]}{#1}{#3}{#5}{#7}}

%\DeclareIndexNameFormat{default}{%
%  \usebibmacro{index:name}{\sindex[person]}{#1}{#3}{#5}{#7}}
%\DeclareIndexNameFormat{default}{%
%\nameparts{#1} \usebibmacro{index:name}{\sindex[person]]}{\namepartfamily}{‌​\namepartgiven}{\nam‌​epartprefix}{\namepa‌​rtsuffix}}

%\newcommand{\smiley}{:)}

%\renewbibmacro*{index:name}[5]{%
%\usebibmacro{index:entry}{#1}%
%{\iffieldundef{usera}{}{\thefield{usera}\actualoperator}\mkbibindexname{#2}{#3}{#4}{#5}}}

% \newcommand{\noop}[1]{}

%remove for final
%\overfullrule=1mm

\newcommand{\tobi}[2]}}
\renewcommand{\S}[1]{\tobi{#1}{\textsc{*}}}

% this volume references
% puts: [this volume]
% already defined: \citetv
%\newcommand{\citepv}[1]{(\citeauthor{#1} \citeyear*{#1} [this volume])}
\newcommand{\citealtv}[1]{\citeauthor{#1} \citeyear*{#1} [this volume]}

%parentheses around example number
\newcommand{\pref}[1]{(\ref{#1})}

% in-text examples

\newcommand{\lnex}[1]{\textit{#1}} %target lang word
\newcommand{\lnlit}[1]{(lit.: `#1')} %literal reading
\newcommand{\lnlat}[1]{(#1)} % latinization
\newcommand{\lntrans}[1]{`#1'} %translation
\newcommand{\lnexl}[2]%
{\lnex{#1}{} \lnlat{#2}} % ex with latinization
\newcommand{\lnexlat}[3]{\lnex{#1}{} \lnlat{#2}{} \lntrans{#3}} % ex with latinization and tranl.

%ch01
\newcommand{\co}[1]{\mbox{\textbf{#1}}}

%ch09

\newcommand{\cyrbulg}[1]{\begin{otherlanguage*}{bulgarian}#1\end{otherlanguage*}}


%ch10
\newcommand{\nlp}{{\small NLP}}
\newcommand{\mwe}{{\small MWE}}
\newcommand{\rae}{{\small RAE}}
\newcommand{\lvc}{{\small LVC}}
\newcommand{\pos}{{\small P}o{\small S}}
%\newcommand{\todo}[1]{ \textcolor{red}{#1} }

%\renewcommand{\labelenumi}{\theenumi}
%\ainamefmt{{vv}{ll}{, ff}{, jj}} % fullname

\newcommand{\biberror}[1]{{\color{red}#1}}

\newcommand{\osenovaitem}{--~}
   %% hyphenation points for line breaks
%% Normally, automatic hyphenation in LaTeX is very good
%% If a word is mis-hyphenated, add it to this file
%%
%% add information to TeX file before \begin{document} with:
%% %% hyphenation points for line breaks
%% Normally, automatic hyphenation in LaTeX is very good
%% If a word is mis-hyphenated, add it to this file
%%
%% add information to TeX file before \begin{document} with:
%% %% hyphenation points for line breaks
%% Normally, automatic hyphenation in LaTeX is very good
%% If a word is mis-hyphenated, add it to this file
%%
%% add information to TeX file before \begin{document} with:
%% \include{localhyphenation}
\hyphenation{
    Beck-man
    Ngu-yen
    back-chan-nel
    back-chan-nels
    mo-not-o-nous
    ste-reo-typ-i-cal
}

\hyphenation{
    Beck-man
    Ngu-yen
    back-chan-nel
    back-chan-nels
    mo-not-o-nous
    ste-reo-typ-i-cal
}

\hyphenation{
    Beck-man
    Ngu-yen
    back-chan-nel
    back-chan-nels
    mo-not-o-nous
    ste-reo-typ-i-cal
}

   \boolfalse{bookcompile}
   \togglepaper[12]%%chapternumber
}

\epigram{\raggedleft In memory of Roger Mgrdichian, 1930--2019.}

\begin{document}
\maketitle \label{ch12}

\section{Introduction} 
In this paper, we present a case study from textsetting that offers one argument for the synchronic status of schwa epenthesis in Armenian.  We do not claim to have definitely proven that schwa epenthesis is active in all varieties of Armenian for every speaker; we offer a case study of a 20\textsuperscript{th} century Armenian diaspora speaker who used schwa epenthesis productively in the process of textsetting his lyrics to the (djurdjuna) beat of a song.  Analyzing such a process in a language other than English requires a lengthy enough exposition that we feel makes our contribution worthwhile, even though it is only one brick in the enterprise to validate the synchronic status of schwa epenthesis in Armenian.  In the general view of epenthesis as a prosodic phenomenon, we feel also that this contribution provides a new tool, which, to the best of our knowledge, has not been used so far for this purpose.  As the reader will notice in most contributions to this volume, but especially in the papers by \textcitetv{chapters/08.Hall}, \textcitetv{chapters/10.Kramer}, \textcitetv{chapters/03.Mansfieldetal}, \textcitetv{chapters/13.Nelson}, \textcitetv{chapters/05.RubinKaplan} and \textcitetv{chapters/02.Sande}, a discussion of epenthesis without mentioning linguistic prosody is almost impossible.  In this sense, we thought that the interaction of epenthesis with musical and rhythmical prosody offered something different but in line with the other contributions.

What lead us to delve into this problem is the assumption by \citet{Vaux1998} that schwa epenthesis is a completely systematic and synchronic process, while \citet{Baronian2017} adopts a more nuanced view.  He analyzes cases where the presence of schwas must be part of the underlying form in Modern Western Armenian, even, in some cases, where there historically was an epenthesis.  Even Baronian, however, still treats most schwas as synchronically epenthetic.  The question behind this paper is thus: given that at least some schwas can be analyzed as part of the underlying representation, is the synchronic status of this process still valid in Modern Armenian? We intend to convince the reader that the textsetting of an Armenian folk song that we analyzed argues for considering Armenian schwa epenthesis as a still productive synchronic phonological process in the language, living next to underlying schwa, which itself can even sometimes be the result of a historical epenthesis.  In doing so, we highlight the ties of epenthesis as a phenomenon to the prosody of a language.

Epenthetic processes, while being easy to define a priori, ``vary enormously in their characteristics, and many aspects of their typology are still not well understood" \citep[1]{Hall2011}. As in the case of any phonological process that linguists analyze, frameworks and/or theories will often guide the solution adopted. Competing frameworks/theories will generally offer one of two different solutions and it is sometimes difficult to decide between them.  In the present case, the possibilities are: 1) phonological insertion of a schwa by an epenthetic process during the derivation from underlying representations (UR) to surface representation (SR); or 2) presence of the schwa in the underlying representation.  We find it useful to search for methods and evidence external to~-- one might say neutral from~-- strictly linguistic frameworks in order to better understand a specific process.

The “external” evidence that we propose to use is the process of textsetting and its formal analysis in Generative Metrics/Generative Textsetting. That is, roughly, the analysis of the way poets and/or songwriters put their words onto metrical grids when they compose poems and songs. For example, we can see in \REF{baronian:shake} that in the English iambic pentameter, Shakespeare aligns stressed syllables (in bold) with the strong positions (s) of superimposed iambs and stressless syllables with weak positions (w):\footnote{Function words can be placed in either strong or weak positions.}

\NumTabs{16}
\begin{exe}
    \ex \label{baronian:shake} w\tab  s\tab   {\textbar}\tab  w\tab  s\tab   {\textbar}\tab  w\tab  s\tab   {\textbar}\tab  w\tab  s\tab   {\textbar}\tab  w\tab  s\tab\\
    But\tab	thy\tab \tab		 e-\tab	\textbf{ter-}\tab \tab		nal\tab 	\textbf{sum-}\tab 	mer\tab \textbf{shall}\tab \tab 		not\tab 	\textbf{fade},
\end{exe}


\noindent As \citet[1]{desisto2020} puts it:

\begin{quote}
The characteristics of poetic metre recreate what is attested in the phonology of the language in which verse is written \citep{Kiparsky1973, Hayes1989meter, Fabb1997, GolstonRiad1999}. Metre is, therefore, an abstract structure which is constructed by mirroring phonological structure and which is filled by phonological material. 
\end{quote}


The rationale of our argument is therefore the following: 

\begin{enumerate}
\item Some processes can receive different analyses depending on the model\slash theory we work with; 
\item Some constraints imposed by textsetting that are not part of the constraints of the language might activate such a process (e.g. by creating a different kind of context, or by creating rare or unexpected structures); 
\item The speaker’s reaction to this new type of context can allow us to better understand what this process is, thereby giving us some cues about the competence of the native speakers with regards to this process.
\end{enumerate}

\section{Aims of the contribution}
Our contribution has two main goals: 
\begin{enumerate}
    \item The first goal concerns method: we want to show how using textsetting as data can help the phonologist decide if a phonological element is underlyingly represented (or not), and thus, if a phonological process~-- in the present case, \textit{epenthesis}~-- is or is not involved in the surface variation that we observe in our data.
    \item The second goal is specific to Western Armenian: we want to help advance the answer to an important question in the phonology of Armenian, namely whether some schwas in this language are truly epenthesized synchronically.
\end{enumerate}


\section{About Armenian and Western vs. Eastern dialects}

Armenian has its own alphabet, which, tradition holds, was invented in the 5\textsuperscript{th} century for Classical Armenian by a monk named Mesrop.  Classical Armenian or Grabar ‘the written word’ is the oldest attested written variety of Armenian.  The alphabet continues to be used for the two modern standards, Western and Eastern Armenian, but has also been used for many dialects of the language and even for several other languages spoken by Armenians, most famously Ottoman Turkish up until the beginning of the 20\textsuperscript{th} century.\footnote{For example, the US Library of Congress holds in its catalog several 19\textsuperscript{th} century texts in what it terms Armeno-Turkish.}  In the 20\textsuperscript{th} century, there was a series of spelling reforms in Soviet Armenia \citep{Dum2009}, which affected the spelling of words for Eastern Armenian in Armenia, but not for Eastern Armenian communities in Iran, nor for Western Armenian in the diaspora.  The song we analyze in this paper is sung in a non standard dialect that belongs to the Western group, which consists mostly of dialects once spoken in the Ottoman Empire.  As a result, Standard Western Armenian and Western Armenian dialects are mainly spoken by descendants of the 1915 Armenian genocide survivors.

Armenian reformed spelling is rather phonetic.  The more traditional spelling still favored by Western Armenian speakers has, as one would expect, a greater discrepancy with pronunciation, but is still much closer to pronunciation than English or French might be to their respective spelling systems.  One example of a discrepancy is that final Yi (\armenian{Յ/յ}) is often silent word-finally in the traditional orthography, representing a former 3\textsc{sg} suffix once pronounced [j] or sometimes part of a case suffix\footnote{The suffixed form is genitive, possessive or ablative.  In the song analyzed, it is used once as an ablative.} once pronounced [-ɑj], but now pronounced [-ɑ]:

\NumTabs{4}
\begin{exe}
    \ex \label{baronian:ex2} Traditional spelling\tab \tab				\armenian{Կարդայ}\tab	`He reads’\\
Reformed spelling\tab	\tab			\armenian{Կարդա}\tab\\
Classical Armenian SR\tab 			[karday]\tab \\
Standard Eastern Armenian SR\tab		[kɑɾdɑ]\tab \\
Standard Western Armenian SR\tab		[ɡɑɾt\textsuperscript{h}ɑ]\tab 
\end{exe}


These silent letters were removed in the reformed spelling.  Another example of discrepancy is that Yi (\armenian{Յ/յ}), Vo (\armenian{Ո/ո}) and Ech (\armenian{Ե/ե}) are respectively pronounced [h], [vɔ] and [jɛ] word-initially in the traditional orthography, but [j], [ɔ] and [ɛ] word-medially.  Initial Yi was thus replaced by Ho (\armenian{Հ}) in the reformed spelling, but the other two letters have been preserved in this position even in the reformed spelling.


\begin{exe}
    \ex \label{baronian:ex3} Traditional spelling\tab \tab				\armenian{Յակոբ}\tab	`Jacob’\\
Reformed spelling\tab	\tab			\armenian{Հակոբ}\tab\\
Classical Armenian SR\tab 			[jɑkob]\tab \\
Standard Eastern Armenian SR\tab		[hɑkob]\tab \\
Standard Western Armenian SR\tab		[hɑɡop\textsuperscript{h}]\tab 
\end{exe}

One discrepancy that is specific to Standard Western Armenian and some dialects is the merger of two series of stop consonants and affricates: Standard Western Armenian opposes voiceless aspirates to voiced stops and affricates, while Standard Eastern Armenian and more conservative dialects have a trilateral opposition, usually\footnote{\citet{Pisowicz1976} lists a total of seven voicing patterns.  Some dialects have voiced aspirated stops and affricates, sometimes termed murmured.  For details, see \citet{Baronian2017}.  For the phonetic nature of voiced aspirates, see \citet{Khachaturian1992}, \citet{SeyfarthaGarellek2018}.} between voiced, voiceless unaspirated and voiceless aspirated stops and affricates.  In this case, making orthography correspond to pronunciation would involve removing five letters from the alphabet, which is not likely to be viewed favorably by most Armenians.  The dialect used in the song under study in this paper, however, has a different merger than Standard Western Armenian as illustrated in \REF{baronian:ex4}.

\NumTabs{7}
\begin{exe}
    \ex \label{baronian:ex4} Illustration of the stop and affricate mergers in Standard Western Armenian and in the dialect used in the song under study\smallskip\\
    \tab \tab \tab ‘petal’\tab    ‘still’\tab    ‘father (addressing a priest)’\\
    Traditional spelling\tab    \armenian{Թեր}\tab    \armenian{Դեռ}\tab    \armenian{Տէր}\\
    Reformed spelling\tab     \armenian{Թեր}\tab    \armenian{Դեռ}\tab    \armenian{Տեր}\\
    Standard Eastern SR\tab    [tʰɛɾ]\tab    [dɛr]\tab    [tɛɾ]\\
    Standard Western SR\tab  [tʰɛɾ]\tab    [tʰɛɾ]\tab    [dɛɾ]\\
    Dialect of the song SR\tab  [tʰɛɾ]\tab    [dɛɾ]\tab    [dɛɾ]
    
\end{exe}

Another relevant feature of the alphabet for this paper is the fact that there exists an Armenian letter for schwa, called Et (\armenian{Ը/ը}).  The rule of thumb in both the traditional and reformed spellings is that this letter is used whenever it is not predictable by epenthesis. This has given phonologists a handle on deciding which schwas are part of the UR and which schwas are derived through epenthesis.  As we will see, some phonologists tend to posit less UR schwas than orthography suggests, but there is also nothing preventing us from positing more schwas than orthography suggests if we assume that orthography is generally more conservative than the spoken language.

\section{The phonological problem: Underlying schwa vs schwa-epenthesis in Western Armenian}
As highlighted by \citet{Baronian2017}, some Armenian schwas can be analyzed as part of the underlying representation. For example, the definiteness or specificity suffix of Western Armenian \citep{Sigler1996} makes consonant-final nouns alternate with schwa-less forms, whereas vowel-final nouns use the allomorph /-n/: 

\NumTabs{6}
\begin{exe}
    \ex T/RS\footnote{Following comments by two anonymous reviewers, we tried to use either TS (traditional spelling) or RS (reformed spelling) based on which we thought was most useful to help the reader understand how we determined the UR.  Our transliteration system WT is basically the ISO 9985 romanization system for Armenian, with two exceptions: 1) we transliterate the digraph \armenian{ու} as u instead of ow, because this digraph always represented a single vowel; 2) we switched to Western Armenian values for unaspirated stops and affricates, because the dialect is Western and not doing so would have distracted the reader from voicing issues not relevant to the question of epenthesis.} \tab		\armenian{մատ}	\tab	\armenian{մատը}\tab		\armenian{լեզու}	\tab	\armenian{լեզուն}\\
WT\tab		mad\tab		madë\tab		lezun\tab		lezu\\
U/SR\tab		mɑd\tab		mɑdə\tab		lɛzu\tab		lɛzun\\
 \tab \tab ‘finger’\tab	‘the finger’\tab	‘tongue’\tab	‘the tongue’

\end{exe}


\citet{Vaux1998}’s earlier analysis posited a unified suffix -\textit{n} that triggered epenthesis when forming a cluster and a special rule that deleted the -\textit{n} later (therefore in the example above, underlying /mad-n/ would become [madən] before resulting in surface [madə]).  We favor the less abstract allomorphic analysis, because the n-deletion rule proposed by Vaux, while it almost certainly corresponds to what happened historically, does not appear to have survived elsewhere in the language.\footnote{Except, as \citet{Baronian2017} points out, before the verb for ‘be’ and before the word \textit{al} ‘also’.}  In our view, positing a suffix-special rule does not place any less burden on memory than positing V/C-sensitive allomorphs. Therefore, minimizing the level of abstraction in the derivation should be favored by Occam’s razor.

\NumTabs{5}
\begin{exe}
    \ex  \tab T/RS\tab   \armenian{մատը} \\
    \tab \tab  WT\tab madë\tab \\
    \citeauthor{vaux03}’s analysis \tab \tab  \citeauthor{Baronian2017}’s analysis\\
    UR mɑd-n\tab \tab \tab UR mɑd-ə\\
    \tab \tab  SR\tab mɑd-ə\tab 
    
\end{exe}

Whatever one’s view on the definiteness or specificity suffix is, the schwas that have attracted most attention in phonology are those that interrupt a consonant sequence otherwise unattested in Armenian and assumed to be impossible to pronounce by native speakers.  Examples from Western Armenian are given in \REF{baronian:ex7}:

\NumTabs{7}
\begin{exe}
    \ex \label{baronian:ex7} T/RS\tab	\armenian{նկար}\tab	\tab		\armenian{պտտիլ}\\
	WT\tab	ngar\tab		\tab	bddil\\
UR\tab	nɡɑr\tab			\tab bddil\\
SR\tab	nəɡɑr\tab		\tab	bədədil\\
\tab \tab ‘portrait’\tab	\tab		‘to stroll’\\
\tab \tab *[nɡ…] unattested\tab	*[bdd…] unattested

\end{exe}

As \citet{Baronian2017} points out, if we assume a sonority hierarchy Stops < Fricatives < Nasals < Liquids < Glides < Vowels, only the onset C-glide clusters and some of the C-liquid clusters are allowed to remain as such in the SR.  The other onsets (whether sonority rises or falls) break up the cluster by inserting a schwa, except in the context of sC, where epenthesis precedes the cluster, as it does in Modern Spanish or Old French, for example.\footnote{For example, Armenian \armenian{Ստեփան} /sdepan/ ‘Stephen’ is pronounced [əsdepʰɑn], similar to the Spanish cognate Esteban and the French cognate Etienne (Old French Estienne).}  In codas, only clusters with raising sonority are broken up by epenthesis:

\NumTabs{7}
\begin{exe}
    \ex \label{baronian:ex8} T/RS\tab		\armenian{վագր}\tab \tab 	\tab \armenian{Ակն}\\
WT\tab		vakr\tab \tab		\tab		agn\\
UR\tab		vɑkr\tab \tab	\tab			ɑɡn	\\	
SR\tab		vɑkər	\tab	\tab \tab		ɑɡən\\	
\tab \tab ‘tiger’	\tab \tab		\tab	town’s name \\
\tab \tab*[…kr] coda unattested	\tab *[…ɡn] coda unattested

\end{exe}


Epenthesis can also apply in some codas with falling sonority, but only when they contain the possessive suffix -\textit{s} (1\textsc{sg}) or -\textit{t} (2\textsc{pl}).  In this case, as recognized by \citet{Baronian2017}, an analysis that would posit lexicalized -\textit{əs} and -\textit{ət} as allomorphs is possible, though he favors still considering the suffixes to trigger epenthesis.  We will return to this special but crucial case after analyzing the song.

Etymologically speaking, it can be shown that the schwas in \REF{baronian:ex7} and \REF{baronian:ex8} were epenthesized at some point, but one may still wonder how to prove the synchronic status of their epenthesis.  The fact that they are spelled without schwas in Armenian orthography should not automatically make us conclude that they are not part of the UR, even though Armenian orthography is closer to pronunciation than English or French is to their respective written forms.

Because one never hears the root [bədəd-] without its schwas, it is certainly possible that at least some speakers lexicalize it as /bədəd-/ instead of /bdd-/ suggested by the orthography, even though the schwas are entirely predictable from the way epenthesis works in this language.  In fact, in the case of onset C-liquid clusters, it is probably the case that they were forbidden historically, resulting in /ɡrɑɡ/ ‘fire’ being pronounced [ɡərɑɡ], but that this requirement was laxed for Modern Armenian in some traditional words, resulting in /krikor/ ‘Gregory’ being pronounced [krikor], and in new borrowings like \textit{Gloria} or \textit{iCloud}.  It is then probably simpler to consider the UR for ‘fire’ to be /ɡərɑɡ/ in Modern Armenian, and let epenthesis apply only in onsets where the second consonant is less sonorous than liquids.   A more radical approach might propose that onset epenthesis has disappeared altogether from the language.

For example, in a case like /nɡɑr/ pronounced [nəɡɑr], there is even a near-minimal pair with the word pronounced [ənɡer] meaning ‘friend’, where the schwa is placed differently.  In this case, orthography encodes the schwa in the latter (T/RS \armenian{ընկեր}, WT ënger), but not in the former (T/RS \armenian{նկար}, WT ngar), suggesting epenthesis is active in [nəɡɑr], but not in [ənɡer].  However, these orthographic choices do not prove anything: what is to prevent a speaker from lexicalizing both forms, each with a schwa in a different location?

What these examples show us is that there clearly was an active epenthesis in the language historically, but its conditions have been laxed over time, to a point where we may wonder whether epenthesis is still synchronically active at all.

\section{The song under study and its dialect}
The song we found where schwa epenthesis is used productively in the textsetting process is titled \textit{Ooska gukas} and was recorded by the Gomidas Band in Philadelphia in 1963 on Roulette Records.  The singer, Roger Mgrdichian, was born in 1930 in the US, the son of Ottoman Armenian immigrants from Peri, which is now known as Akpazar in Turkey.\footnote{An anonymous reviewer questions the fluency level of the US-born singer.  Knowing the demographics of Armenians in Philadelphia in those years, it sounds extremely unlikely to us that the singer would not have spoken Armenian fluently.  In a correspondence with his son (also Roger), Baronian was able to confirm that Mgrdichian spoke Armenian and some Turkish at home.  In fact, his son reports a family story, according to which his father was once sent home from the 1\textsuperscript{st} grade, because he could not speak enough English.}

We noticed that the dialect to which the song belongs has two defining isoglosses: 1) the merger of Proto-Indo-European (PIE) stop series II and III into a single voiced series (Type III voicing, as classified for example by \cite{Pisowicz1976}); 2) the use of \textit{gu}- as a present marker.  Both the voiced nature of PIE stop series II and the use of \textit{gu}- as a present marker marks the dialect as unambiguously part of the Western Armenian group of dialects.

The lyrics of the song include a few Turkish borrowings, but remain largely intelligible to a Standard Western Armenian speaker.  Because one of the main differences between Standard Eastern and Standard Western Armenian involves the voicing of stops and affricates, the voicing pattern of the stops in this dialect is probably the most challenging feature in terms of mutual intelligibility, as the merger, in effect, makes some words sound Eastern and some words sound Western.  To the best of our assessment, however, nothing in this dialect bears on the schwas in the language, nor on epenthesis.  For this reason and because the singer was part of a larger Western-Armenian speaking diasporan community in the Philadelphia area, it seemed to us that whatever conclusion we might draw about the dialect used in the song and its singer can be extended to Western Armenian speakers generally without any further assumptions.

The complexity of the situation, with the existence of two standards, multiple dialects and a spelling reform not universally used by all Armenians, made us hesitate on the question of how to best represent the words of the song.  Devising dialect-specific spellings to accommodate for the differences between the Western standard and the dialect under study would not easily be recognizable by someone who reads Armenian and would not have brought any additional information to readers unable to read Armenian.  Phonetic transcriptions with glosses seemed like the best avenue in such circumstances.  At the same time, Armenian orthography, even the traditional one, is close to being phonemic, thereby approximating the UR, so, at the suggestion of a reviewer, we decided that providing the spellings in the annex, where the entire song can be consulted, with a general Western Armenian transliteration (which makes distinctions in voicing/aspiration that the dialect does not make), along with our phonetic transcriptions was the best option.  This lessens the burden on the reader in the examples provided to make the argument, but the reader who wants to see how we determined the UR can consult the annex.  Readers should keep in mind these two points:
 
\begin{enumerate}
    \item The dialect under study merges two series of stops and affricates, which, in the Classical language, were unaspirated voiceless and voiced, into a voiced series.  (The Classical voiceless aspirates remain as such.)
    \item As we pointed out earlier, while schwa has its own letter in the Armenian alphabet, it is not always present in the written form, cf. \citet{Baronian2017} and similar distinctions in Dutch by \citet{Oostendorp2011}.  It is plausible that a good first approximation would state that schwa is written when it is part of the lexical form and not written when it is epenthesized synchronically, but, for the purpose of our analysis, it was important to transcribe every schwa that was present in the surface representation.
\end{enumerate}


\section{Data and analysis}

The rhythm used in the song, called \textit{djurdjuna}, is generally analyzed as a 10/16.\footnote{One reviewer questions how we determined the beat of the song.  We do not have the space to get into musical details, but Royer-Artuso is a professional musician who has studied and practiced Ottoman music for decades.  There is no doubt to either of us that this beat is recognizable as a \textit{djurdjuna}.}  In \REF{baronian:ex9}, the strong beats are marked with an upper case X, the weak beats are marked with a lower case x, and the non-obligatory strong beats are marked within parentheses (X):

\NumTabs{13}
\begin{exe}
    \ex \label{baronian:ex9} \tab Xx(X)\tab \tab	   	Xx\tab	Xx\tab     Xx(X)\tab	|\tab	Xx(X)\tab	\tab  	 Xx\tab	Xx\tab     Xx(X)\\
SR\tab	dɛr[ə]dəd\tab	jɛm\tab     jɛː\tab	\tab	|\tab	ɣɛr[ə]\tab \tab	  	 di\tab	vɑ\tab      nɑ\\
\\
SR\tab	dɛr[ə]d+əd\tab     	jɛm\tab     	jɛːɣɛr[ə]\tab \tab \tab	\tab 			divɑnɑ\\
\tab \tab worry+2SG \tab   	am\tab   	become+\textsc{Inf}\tab   \tab  	\tab 	crazy\\
‘I am your problem, I go crazy’

\end{exe}

Except in the chorus, the singer seems to structure each line with two measures as follows: 

\NumTabs{11}
\begin{exe}
    \ex \label{baronian:ex10} XxX\tab	   Xx\tab	Xx\tab     Xxx\tab		\tab|\tab	XxX\tab	   Xx\tab	Xx\tab     Xxx
\end{exe}


The textsetting rules we have identified are the following, and they were always followed:

\begin{exe}
    \ex \label{baronian:ex11}
    \begin{xlist}
        \ex XxX must be filled by two syllables,
        \ex Xx must be filled by a single syllable,
        \ex Xxx is filled by a heavy syllable (with coda or long vowel) in the middle of a line, but can be filled by a light syllable at the end of a line.
    \end{xlist}
\end{exe}

In the sample in \REF{baronian:ex9}, as well as in the full text of the song provided in the Appendix in Section \ref{baronian:append}, the schwas indicated in [square brackets] are the interesting ones not expected in the pronunciation of Armenian because: 1) they are not part of the historical forms of those words; 2) they do not represent the definiteness\slash specificity suffix; and 3) they do not break up unattested clusters of Armenian.  In fact, some very similar clusters sometimes appear elsewhere in the same song: /dɛrdd/ in \REF{baronian:ex9} and line [2] of the Appendix becomes [dɛr\textbf{ə}dəd], but /ɑrnim/ (line [7] of the Appendix) remains schwaless.  Our explanation for the insertion of those non-standard schwas is that the singer or composer syllabifies a consonantal mora in order to occupy a strong metrical position.  The singer always inserts these “new” schwas after a coda, never before a word-initial onset or never to break up a cluster that is not already broken up by regular syllabification or epenthesis:

\NumTabs{4}
\begin{exe}
    \ex \label{baronian:ex12} Textsetting schwas inserted by the singer in his performance\\(lines in [square brackets]):\smallskip\\
    \relax
    [1] us[ə]ɡɛ\tab		[2] dɛr[ə]dəd\tab		[2] jɛːɣɛr[ə]\tab		[3] bɑtʃiɡ[ə]\\
        ‘from where’\tab		‘your worry’\tab		‘it seems’\tab		‘kiss’\\
    \relax
    [4] jɑr[ə]\tab		[5] jɛs[ə]\tab		[8] uʃiɡ[ə]	\tab	[9] vod[ə]ɡəd	\\
    ‘soul’\tab			‘I’	\tab		‘late-ish’	\tab	‘your foot’\\
    \relax
    [9] vɑr[ə]di\tab		[10] dur[ə]	\tab	[12] dun[ə]	\tab	[13] dur[ə]\\
	‘rose-\textsc{dat}’	\tab	‘give-\textsc{imp}’	\tab	‘you’\tab		 	‘door’
\end{exe}



In most words in \REF{baronian:ex12}, schwa appears at the end of the word, after a single consonant.  Because most of these words are not nouns, the schwa in them cannot be interpreted as a definiteness\slash specificity suffix.  Even in \textit{bɑtʃiɡ[ə]} ‘kiss’ and in \textit{dur[ə]} ‘door’, the presence of the indefinite marker mə/mi immediately after lifts the ambiguity.  In the words \textit{us[ə]ɡɛ} and \textit{vɑr[ə]di}, schwa breaks up two consonants already belonging to two separate syllables and then does not even fall under traditional schwa epenthesis.  Therefore, the schwa epenthesis studied here is most similar to the cases of epenthesis studied by \textcitetv{chapters/09.HamannMiatto}, as well as \textcitetv{chapters/13.Nelson}, as it does not break up a consonant cluster within an onset or within a coda.  Contrary to Krämer’s excrescent or intrusive vowel cases, however, the epenthesis here is clearly available to prosodic computation, albeit musical, not linguistic.  Interestingly, we could say that the Armenian textsetting epenthesis is the mirror image of the -\textit{a} omission mentioned in section 5.2 of the paper by \textcitetv{chapters/03.Mansfieldetal}, in that both occur at the edge of a prosodic domain (the syllable for Armenian).


In the word \textit{dɛr[ə]dəd}, the second schwa is also epenthetic, but it is expected in the regular pronunciation, because it precedes the 2\textsc{sg} possessive suffix -\textit{d} (-\textit{t} in Standard Western Armenian as discussed earlier).  The unexpected epenthesis, on the other hand, breaks up the -\textit{rd}- cluster in the sense that the two consonants now belong to different syllables.  The exact same situation happens in the word \textit{vod[ə]ɡəd}.  

  This situation is interesting, because it recalls \citet{Baronian2017}'s analysis of the possessive suffix as being affixed with a coda status, thus revising \citet{Vaux1998}'s analysis that posited a prosodic word boundary.  The motivation for this analysis was that the suffix triggered epentheses that were not observed elsewhere in the language, thus:

\begin{exe}
    \ex \label{baronian:ex13} T/RS\tab 	\armenian{դուրս}\tab 		\armenian{դուռս}\tab 		\armenian{դուռդ}\\
	WT\tab 	turs\tab 		tuṙs\tab 		tuṙt\\
    UR\tab 	turs\tab 		turs\tab 		turt\\
    SR\tab 	turs\tab 		turəs\tab 		turət\\
    \tab \tab ‘outside’\tab 	‘my door’\tab 	‘your door’
\end{exe}

In /dɛrd{}-d/→[dɛrdəd] and /vodɡ-d/ →[vodɡəd]\footnote{In example \REF{baronian:ex13}, we used the Standard Western Armenian form of the 2\textsc{sg} possessive suffix (-t), but in \textit{dɛrdəd} and \textit{vodɡəd}, which are words taken from the song under analysis, we used the dialectal form -\textit{d}.}, epenthesis is necessary anyway, because -\textit{rdd} or -\textit{dgd} codas are not pronounceable in Armenian without it.  However, adopting \citet{Baronian2017}'s analysis of the possessive suffix as a consonant specified for moraicity (with the assumption that codas are moraic) offers a potential bridge to understanding the singer’s strategy in the song.  If epenthesis allows one to preserve the moraic nature of the consonant in the context of the possessive suffix, the singer may have simply generalized this epenthesis in order to facilitate the textsetting.  More precisely, the singer syllabifies a consonantal mora by epenthesizing after the moraic (coda) consonant, whereas in the regular language, epenthesis precedes a moraic (coda) suffix within a cluster.\footnote{\citet{Baronian2017} lists two other cases of morphological operations that need to refer to syllable structure in Western Armenian.}  We assume that the coda position is moraic for all consonants and therefore the derivation is the following:

\NumTabs{12}
\begin{exe}
    \ex UR\tab \tab	\tab	\tab		/dɛrd-d/\tab \tab	\tab		/vodɡ-d/\\
    Regular epenthesis\tab \tab		dɛrdəd	\tab \tab	\tab		vodɡəd\\
    Mapping to beat	\tab \tab	X\tab  	x\tab	x\tab	\tab	X\tab	x\tab	x\\
    \tab \tab \tab \tab \tab dɛ\tab  r\$\tab  dəd\tab \tab    vo\tab  d\$\tab  ɡəd\\
    Textsetting epenthesis\tab  dɛ\$\tab  rə\$\tab  dəd \tab \tab   vo\$\tab  də\$\tab  ɡəd
\end{exe}


A legitimate question to ask at this point is whether other linguistic phenomena active in the language are used by the singer to fit the meter of the song.  We believe this to be the case.  For example, reduplication is active in Armenian \citep{Vaux1998} and the singer reduplicated the last syllable of \textit{bəzdiɡ} ‘small’ on line [10] (see Appendix in Section \ref{baronian:append}).  On line [2] and in the last two lines of the song, the singer also lengthens a vowel in \textit{jɛːɣɛr[ə]}, \textit{divɑnɑː} and \textit{mɛziː}.  While vowel lengthening is not a phonological process reported for Armenian, it certainly exists as an emphasis strategy.  In both these cases (reduplication and lengthening), it is interesting to note that epenthesis would not have been quite applicable because, in \textit{bəzdiɡ}, epenthesis after [ɡ] would not have been enough to fit the Xxx position (the final syllable in resulting \textit{bəzdiɡə} would have been light, thereby violating textsetting principle C) and, in \textit{jɛɣɛr}, the [ɣ] is not moraic, because it is located in an onset, thereby preventing textsetting epenthesis.  It’s true, however, that, in the case of \textit{bəzdiɡ}, the singer could have opted to epenthesize word\hyp medially, yielding a metrically acceptable \textit{bəzədiɡ}.

\section{Conclusion}
The performance by Mgrdichian allows us to deepen our understanding of the synchronic status of schwas in Armenian.  While we cannot prove on the basis of this song that the schwas illustrated in \REF{baronian:ex7} are not lexicalized, the productivity of the generalized schwa epenthesis employed to fill what would otherwise remain as empty strong metrical positions suggests that schwa epenthesis was at least available to this speaker at the phonological level and reinforces the opinion defended by \citet{Vaux1998}, and more recently by \citet{Baronian2017} and \citet{Dolatian2021}, that schwa epenthesis is a synchronically active process of Armenian.  In particular, it is striking that the conditions for epenthesis used for textsetting purposes resemble those that occur in the context of possessive suffixation and reinforces \citet{Vaux1998}'s opinion that epenthesis was active in this case, while it favors, however, \citet{Baronian2017}'s analysis of the suffix as occupying a (moraic) coda position.  While the challenge of explaining the textsetting mechanism in a language other than English took up too much space for us to broaden the scope of the paper beyond this case study, we hope that readers will also be convinced of the usefulness of textsetting as a tool to understand the nature of phonological processes.  Finally, we also hope that readers will agree that the interplay of schwa epenthesis with textsetting in order to fill empty beat positions strengthens the view that schwa epenthesis can also be considered a prosodic process, rather than a strictly segmental one.  Like the -\textit{a} omission discussed by \textcitetv{chapters/03.Mansfieldetal}, it occurs at the edge of a prosodic domain (the syllable).  However, unlike the regular phonological epenthesis of Armenian (and unlike the cases studied in this volume by \citetv{chapters/07.Bellik, chapters/04.Bokhari, chapters/08.Hall, chapters/10.Kramer, chapters/03.Mansfieldetal, chapters/05.RubinKaplan}, and \citetv{chapters/02.Sande}), the textsetting epenthesis does not break up a cluster within an onset or a coda, but follows a coda, a case similar to those under study by \textcitetv{chapters/09.HamannMiatto}, as well as \textcitetv{chapters/13.Nelson}).

\section{Abbreviations}
\begin{tabularx}{.45\textwidth}{lQ}
TS & Traditional spelling \\
RS & Reformed spelling \\
WT & Western transliteration \\
\end{tabularx}
\begin{tabularx}{.45\textwidth}{lQ}
UR & Underlying representation \\
SR & Surface representation\\
\\
\end{tabularx}


{\sloppy\printbibliography[heading=subbibliography,notkeyword=this]}

\begin{paperappendix}
\section*{\textit{Ooska gukas}} \label{baronian:append}

\NumTabs{11}
Line [1]\\
TS\tab  \armenian{Ուսկէ  \tab  \tab  կու\tab  գաս \tab   \tab Վերիէ \tab     \tab \tab Վանայ}\tab \\
WT\tab  Usgē\tab  \tab    gu\tab  kas\tab   \tab veriē\tab     \tab  \tab vanay\\
\tab \tab XxX\tab    Xx\tab  Xx\tab  Xxx\tab   \tab  XxX\tab    Xx\tab \tab  Xx \tab Xxx \\
UR\tab  us\tab     ɡɛ \#\tab  ɡu \#\tab  ɡɑs \#\tab   \tab  vɛ ri\tab    ɛ\#\tab  \tab vɑ\tab  nɑ \#\\
SR\tab  us[ə]\tab     ɡɛ \#\tab  ɡu \#\tab  ɡɑs \#\tab  \tab   vɛ ri\tab    ɛ\#\tab  \tab vɑ\tab  nɑ \#\\
\tab \tab from where \tab  \textsc{prog}\tab  come-2\textsc{sg}\tab  upper-\textsc{dat-abl}  \tab Van-\textsc{abl}\\
‘From where do you come, from Upper Van?'\medskip

\noindent Line [2]\\
TS\tab  \armenian{Տերտդ\tab \tab \tab   եմ\tab  եղեր \tab   \tab \tab    տիվանա}\\
WT\tab  Derdt  \tab \tab  \tab   em\tab  e\.ger\tab \tab \tab        divana\\
\tab \tab XxX \tab   Xx\tab \tab  Xx\tab  Xxx\tab    XxX\tab  \tab   Xx \tab   Xx\tab  Xxx\\
UR\tab  dɛr\tab    dd \#\tab \tab  jɛm \#\tab  jɛ\tab    ɣɛr \# \tab \tab  di\tab  vɑ\tab  nɑ \#\\
SR\tab  dɛr[ə]\tab    dəd \# \tab \tab jɛm \# \tab jɛː\tab    ɣɛr[ə] \# \tab di\tab  vɑ\tab  nɑ \#\\
\tab \tab worry-\textsc{poss-2sg}\tab  am \tab it seems \tab \tab     going crazy\\
‘I am your worry, I go crazy’\medskip

\newpage
\noindent Line [3]\\
TS\tab  \armenian{Պաչիկ \tab    մը \tab տուր\tab   \tab Մայրը \tab  \tab չի\tab  մանա }  \\ 
WT\tab  Bačig \tab \tab     më\tab  dur \tab \tab  mayrë\tab \tab   či\tab  mana\\
\tab \tab XxX \tab   Xx\tab  Xx\tab  Xxx \tab \tab   XxX \tab \tab   Xx \tab Xx\tab  Xxx\\
UR\tab  bɑ tʃiɡ \tab   \# \tab mə \# \tab dur \#\tab  \tab  mɑ(j)rə \# \tab tʃi \# \tab mɑ \tab nɑ \#\\
SR\tab  bɑ tʃiɡ\tab    [ə] \# \tab mə \#\tab  dur \#\tab  \tab  mɑrə \# \tab tʃi \# \tab mɑ\tab  nɑ \#\\
\tab \tab kiss \tab \tab      a\tab  give-\textsc{imp-2sg} \tab mother-\textsc{def} \tab \textsc{neg} \tab watch-3\textsc{sg}\\
‘Give me a kiss, the mother is not watching’\medskip

\noindent Chorus: Line [4]\\
TS\tab    \armenian{Յար\tab \tab     կիւլիւմ\tab \tab   ճան \tab    է… }  \\      
WT\tab   Yar\tab    \tab  giwliwm\tab   ǰan\tab     ē\\
\tab \tab XxX\tab   \tab   Xx \tab  Xx\tab   Xxx \tab    XxX\tab     Xx\tab   Xx\tab   Xxx\\
UR\tab   jɑr \# \tab  \tab   ɡy\tab   lym \# \tab  dʒɑn \#\tab    ɛ... \#\\
SR\tab   jɑr[ə] \#\tab   ɡy\tab   lym \#\tab   dʒɑn \# \tab   ɛ... \#\\
\tab \tab Friend \tab \tab    rose  \tab \tab   soul  \tab   is\\
‘The friendly rose is my soul’\medskip

\noindent Line [5]\\
TS \tab   \armenian{Սիրեր\tab   \tab   եմ\tab   ես \tab  \tab   գեզ… } \\      
WT \tab  Sirer\tab  \tab      em\tab   es\tab    \tab  kez\\
\tab \tab XxX\tab     Xx\tab   Xx\tab   Xxx\tab  \tab    XxX\tab     Xx \tab  Xx \tab  Xxx\\
UR\tab   sirɛ \tab    r \# \tab  jɛm \#  \tab jɛs \#   \tab \tab   kɛz… \#\\
SR\tab   sirɛ\tab     ːr \# \tab  jɛm \#  \tab jɛs[ə] \# \tab   kɛz… \#\\
\tab \tab love  \tab  \tab    am\tab   I  \tab  \tab  you\\
‘I love you’\medskip

\noindent Line [6]\\
TS \tab  \armenian{Յար\tab  \tab    կիւլիւմ \tab  ճան \tab    է… } \\ 
WT\tab   Yar \tab   \tab  giwliwm \tab  ǰan  \tab   ē\\
\tab \tab XxX \tab   \tab  Xx \tab  Xx \tab  Xxx\tab     XxX \tab    Xx\tab   Xx  \tab Xxx \\    
UR  \tab jɑr \#   \tab\tab   ɡy\tab   lym \# \tab  dʒɑn \# \tab   ɛ… \#\\
SR \tab  jɑr[ə] \#\tab   ɡy \tab  lym \#  \tab dʒɑn \# \tab   ɛ… \#\\
\tab \tab Friend  \tab \tab   rose \tab \tab    soul \tab    is\\
‘The friendly rose is my soul’\medskip

\newpage
\noindent Line [7]\\
TS\tab   \armenian{ Պիտի \tab  \tab   առնեմ \tab  ես  \tab  \tab  գեզ… }  \\ 
WT\tab   Bidi  \tab \tab   aṙnem \tab  \tab   es  \tab  \tab  kez\\
\tab \tab XxX\tab  \tab    Xx \tab  Xx\tab   Xxx   \tab \tab  XxX \tab    Xx \tab  Xx \tab Xxx   \\  
UR \tab  bi di \#  \tab \tab   ɑr\tab   nim \# \tab  jɛs \#\tab  \tab   kɛz... \#\\
SR \tab  bi di \#  \tab \tab   ɑr \tab  nim \# \tab  jɛs[ə] \# \tab  kɛz... \#\\
\tab \tab \textsc{fut} \tab  \tab   take-1\textsc{sg}\tab   I \tab  \tab   you\\
‘I will take you away’\medskip

\noindent Line [8]\\
TS \tab   \armenian{Ուսկէ  \tab   \tab  \tab  կու\tab   գաս \tab   \tab  ուշիկ  \tab  \tab    մուշիկ}\\
WT \tab  Usgē \tab    \tab \tab   gu\tab   kas  \tab  \tab  ušig   \tab \tab    mušig\\
\tab \tab XxX \tab  \tab   Xx\tab   Xx \tab  Xxx \tab  \tab  XxX  \tab  Xx\tab   Xx \tab  Xxx \\
UR  \tab us\tab    \tab   ɡɛ \# \tab  ɡu \# \tab  ɡɑs \#  \tab \tab    uʃi  \tab   ɡ \#  \tab  mu \tab  ʃiɡ \#\\
SR \tab  us[ə] \tab   \tab   ɡɛ \# \tab  ɡu \#\tab   ɡɑs \# \tab  \tab    uʃi \tab    ɡ[ə] \#  \tab  mu\tab   ʃiɡ \#\\
\tab \tab from where \tab   \tab  \textsc{prog}  come-2\textsc{sg} \tab  late-\textsc{dim} \tab    \textsc{redup}\\
‘From where do you come so late?’\medskip

\noindent Line [9]\\
TS\tab    \armenian{Ոտքդ  \tab   \tab \tab   մտեր\tab  \tab   վարդի   \tab  բուշիկ }\\
WT\tab   odk’t  \tab  \tab \tab    mder \tab     \tab  varti \tab     \tab  pušig\\
\tab \tab XxX  \tab \tab   Xx \tab  Xx \tab  Xxx \tab   XxX\tab     Xx \tab  Xx\tab   Xxx\\
UR \tab  vod\tab   \tab   ɡd \# \tab  md \tab   ɛr \#  \tab    vɑr  \tab   di \# \tab  pu \tab  ʃiɡ \#\\
SR \tab  vod[ə]  \tab \tab   ɡəd \#\tab   məd \tab   ɛr \#  \tab    vɑr[ə]  \tab   di \#\tab   pu \tab  ʃiɡ \#\\
\tab \tab foot-\textsc{poss-2sg} \tab  enter \tab \tab      rose-\textsc{dat}   \tab   thorn\\
‘The rose’s thorn entered your foot’\medskip

\noindent Line [10]\\
TS\tab   \armenian{Տուր\tab  \tab    պզտիկ \tab  տիկ \tab    պաչիկ \tab  անուշիկ  }\\
WT\tab   Dur \tab  \tab   bzdig\tab   \tab   dig \tab    bač̣ig  \tab   \tab anušig\\
\tab \tab XxX \tab \tab     Xx\tab   Xx \tab  Xxx\tab     XxX  \tab \tab   Xx \tab  Xx \tab  Xxx\\
UR\tab   dur \#\tab  \tab    bz \tab  diɡ \tab   diɡ \#   \tab  bɑtʃiɡ \# \tab  ɑ  \tab nu \tab  ʃiɡ \#\\
SR\tab   dur[ə] \#\tab   bəz \tab  diɡ \tab  diɡ \#  \tab   bɑtʃiɡ \# \tab  ɑ\tab   nu\tab   ʃiɡ \#\\
\tab \tab give-\textsc{imp-2sg}\tab   small  \tab  \tab  \textsc{redup}\tab   kiss \tab  \tab   sweet\\
‘Give a little sweet kiss’\medskip

\newpage
\noindent Line [11]\\
TS\tab   \armenian{Ուսկէ  \tab   \tab   կու \tab  գաս \tab   \tab  Վերիէ  \tab   \tab   Վանայ } \\
WT\tab   Usgē \tab  \tab     gu \tab  kas \tab  \tab   veriē \tab \tab      vanay\\
\tab \tab XxX   \tab  Xx \tab  Xx \tab  Xxx\tab  \tab    XxX   \tab   Xx  \tab Xx\tab   Xxx\\ 
UR \tab  us \tab     ɡɛ \#\tab   ɡu \#\tab   ɡɑs \#  \tab   \tab  vɛ ri \tab    ɛ\# \tab  vɑ\tab   nɑ \#\\
SR \tab  us[ə] \tab     ɡɛ \# \tab  ɡu \# \tab  ɡɑs \# \tab  \tab    vɛ ri \tab    ɛ\# \tab  vɑ \tab  nɑ \#\\
\tab \tab from where \tab    \textsc{prog} \tab  come-2\textsc{sg} \tab  upper-\textsc{dat-abl} \tab  Van-\textsc{abl}\\
‘From where do you come, from Upper Van?'\medskip

\noindent Line [12]\\
TS\tab    \armenian{Ես  \tab  \tab  տիվանա \tab  \tab   դուն   \tab  \tab տիվանա} \\  
WT\tab   Es \tab\tab     divana \tab      \tab   tun\tab \tab     divana\\
\tab \tab XxX \tab \tab    Xx \tab  Xx \tab  Xxx  \tab   XxX \tab    \tab Xx\tab   Xx  \tab Xxx\\
UR  \tab jɛs[ə] \# \tab   di\tab   vɑ \tab  nɑː \#  \tab   dun[ə] \# \tab  di \tab  vɑ \tab  nɑ \#\\
SR \tab  jɛs[ə] \#\tab  \tab  di\tab   vɑ \tab  nɑː \# \tab    dun[ə] \# \tab  di\tab   vɑ\tab   nɑ \#\tab 
\tab I  \tab\tab    go crazy  \tab   \tab   you   \tab \tab  go crazy\\
‘I go crazy, you go crazy’\medskip

\noindent Line [13]\\
TS \tab   \armenian{Աստուածը \tab   \tab  մեզի\tab  \tab      դուռ\tab   \tab   մի  \tab բանա } \\ 
WT\tab   Asduaçë\tab   \tab   mezi \tab    \tab   tuṙ \tab   \tab  mi \tab  pana\\
\tab \tab XxX \tab   \tab  Xx\tab   Xx \tab  Xxx \tab    XxX \tab  \tab   Xx \tab  Xx \tab  Xxx\\
UR\tab   ɑsdvɑdz\tab   ə \# \tab  mɛ \tab  ziː \# \tab     dur[ə] \# \tab  mi \# \tab  bɑ \tab  nɑ \#\\
SR \tab  ɑsdvɑdz \tab  ə \# \tab  mɛ \tab  ziː \# \tab     dur[ə] \# \tab  mi \# \tab  bɑ \tab  nɑ \#\\
\tab \tab God-\textsc{def} \tab   \tab  us-\textsc{dat}  \tab   door \tab  \tab   a \tab  open-2\textsc{sg}\\
‘May God open a door for us’\\
\end{paperappendix}

\end{document}
