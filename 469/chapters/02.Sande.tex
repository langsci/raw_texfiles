\documentclass[output=paper,colorlinks,citecolor=brown]{langscibook}
\ChapterDOI{10.5281/zenodo.14264530}
\author{Hannah Sande\orcid{0000-0003-1335-8717}\affiliation{UC Berkeley}}
\title{Insertion or deletion? CVCV/CCV alternations in Kru languages} 

\abstract{Kru languages, spoken in Liberia and Côte d'Ivoire, show an alternation between CVCV and CCV in a subset of lexical items. The alternation is variable, where the same word can appear as either CVCV or CCV in a single environment, and lexically specific, where only a subset of morphemes alternate in this way. This paper describes the CVCV\slash CCV alternation in Kru, showing that whether this pattern is best categorized as an instance of vowel insertion or deletion differs by language. Two representative Kru case studies are presented in detail: The CVCV\slash CCV alternation in Dida is shown to be best analyzed as underlying /CCV/ with epenthesis or vowel intrusion, whereas the CVCV\slash CCV alternation in Guébie is shown to involve underlying /CVCV/ sequences with optional deletion of the initial vowel. Diagnostics are proposed for determining when a vowel/$\emptyset$ alternation is synchronically best analyzed as insertion or deletion. CVCV\slash CCV alternations found in nearby Kru, Mande, and Kwa languages are examined, leading to conclusions about the diachronic development of the CVCV\slash CCV alternation in West Africa.}

\IfFileExists{../localcommands.tex}{
   \addbibresource{../localbibliography.bib}
   %%\addbibresource{../collection_tmp.bib}
   \usepackage{langsci-optional}
\usepackage{langsci-gb4e}
\usepackage{langsci-lgr}

\usepackage{listings}
\lstset{basicstyle=\ttfamily,tabsize=2,breaklines=true}

%added by author
% \usepackage{tipa}
\usepackage{multirow}
\graphicspath{{figures/}}
\usepackage{langsci-branding}

   
\newcommand{\sent}{\enumsentence}
\newcommand{\sents}{\eenumsentence}
\let\citeasnoun\citet

\renewcommand{\lsCoverTitleFont}[1]{\sffamily\addfontfeatures{Scale=MatchUppercase}\fontsize{44pt}{16mm}\selectfont #1}
  
   %% hyphenation points for line breaks
%% Normally, automatic hyphenation in LaTeX is very good
%% If a word is mis-hyphenated, add it to this file
%%
%% add information to TeX file before \begin{document} with:
%% %% hyphenation points for line breaks
%% Normally, automatic hyphenation in LaTeX is very good
%% If a word is mis-hyphenated, add it to this file
%%
%% add information to TeX file before \begin{document} with:
%% %% hyphenation points for line breaks
%% Normally, automatic hyphenation in LaTeX is very good
%% If a word is mis-hyphenated, add it to this file
%%
%% add information to TeX file before \begin{document} with:
%% \include{localhyphenation}
\hyphenation{
affri-ca-te
affri-ca-tes
an-no-tated
com-ple-ments
com-po-si-tio-na-li-ty
non-com-po-si-tio-na-li-ty
Gon-zá-lez
out-side
Ri-chárd
se-man-tics
STREU-SLE
Tie-de-mann
}
\hyphenation{
affri-ca-te
affri-ca-tes
an-no-tated
com-ple-ments
com-po-si-tio-na-li-ty
non-com-po-si-tio-na-li-ty
Gon-zá-lez
out-side
Ri-chárd
se-man-tics
STREU-SLE
Tie-de-mann
}
\hyphenation{
affri-ca-te
affri-ca-tes
an-no-tated
com-ple-ments
com-po-si-tio-na-li-ty
non-com-po-si-tio-na-li-ty
Gon-zá-lez
out-side
Ri-chárd
se-man-tics
STREU-SLE
Tie-de-mann
}´
   \boolfalse{bookcompile}
   \togglepaper[2]%%chapternumber

}

\begin{document}
\maketitle 
\label{ch2}

\section{Introduction}\label{sec:intro} 
Kru languages, spoken in Liberia and Côte d'Ivoire, show a CVCV\slash CCV alternation, as in the Guébie example in \REF{gex}.

\ea \label{gex}
\gll \ipa{bala$^{3.3}$}/\ipa{bla$^{3}$}\\
hit\\
\glt `hit'  (Guébie, Eastern Kru)
\z

In some Kru languages this alternation has been analyzed as vowel insertion, while in others it has been called vowel deletion. This paper aims to describe the CVCV\slash CCV alternation in multiple Kru languages, and to diagnose for each language whether the alternation is best analyzed as synchronic deletion or insertion. As will be shown in Sections \ref{sec:dida} and \ref{sec:guebie}, the alternation seems to be best analyzed as V-deletion in some Kru languages but V-insertion in others. The two case studies examined in detail come from Dida, presented in \sectref{sec:dida}, and Guébie, in \sectref{sec:guebie}, both part of the Neyo-Dida group of Eastern Kru languages. \sectref{sec:areal} examines the wider picture of the CVCV\slash CCV alternation in Kru, Mande, and Kwa languages, drawing historical and areal conclusions about the diachronic path of this alternation. \sectref{sec:conclusion} summarizes the diagnostics proposed throughout the paper for determining when a vowel/$\emptyset$ alternation is best analyzed as synchronic insertion versus deletion.

\subsection{Background on CVCV/CCV alternations}\label{sec:cvcvintro}
\subsubsection{V/$\emptyset$ as insertion}\label{sec:epenthesis}

/CCV/ $\rightarrow$ [CVCV] alternations are often analyzed as \textit{copy epenthesis}, also called Dorsey's Law: A vowel is inserted in a CCV word to break up the CC cluster \citep{Miner:1979, Miner:1989, Hale&WhiteEagle, Hayes:1995, Clements:1986, Clements:1991, Halle:2000, Kawahara:2007, StantonZukoff2018}. For a specific implementation of Dorsey's law involving feature spreading, see \citet{Kawahara:2007}, and for an analysis involving correspondence and identity, see \citet{StantonZukoff2018}.  In such cases, the quality of the inserted vowel tends to match the quality of the underlying vowel, or surface as the predictable epenthetic vowel in the language. The inserted vowel is often shorter than an underlying vowel, and lacks its own prosody (e.g. no independent stress or tone). 

\ea
Winnebago /prás/ $\rightarrow$ [parás] \citep[27]{Miner:1979}
\z

CCV/CVCV alternations have also been analyzed as due to retiming of articulatory gestures, where an underlying CCV sequence is pronounced as CVCV due to gestural overlap (\cite{Hall:2003, Hall2006, Hall2011},  \citetv{chapters/08.Hall}, \citetv{chapters/07.Bellik}). For \citeauthor{Hall2006}, there is a distinction between phonologically epenthesized vowels repairing a marked structure (i.e., vowel epenthesis to break up a cluster as per Dorsey's Law), and intrusive vowels which are phonologically invisible and involve a rearrangement of articulatory targets. \citet[391]{Hall2006} provides diagnostics for distinguishing between phonologically visible epenthetic vowels and phonologically invisible intrusive vowels (see also \citetv{chapters/08.Hall}), repeated in (\ref{HallDiagnostics1}) and (\ref{HallDiagnostics2}), below.

\ea Properties of phonologically invisible inserted vowels (intrusive vowels) \citep[391]{Hall2006}\label{HallDiagnostics1}
	\ea The vowel's quality is either schwa, a copy of a nearby vowel, or influenced by the place of the surrounding consonants.
	\ex If the vowel copies the quality of another vowel over an intervening consonant, that consonant is a sonorant or guttural.
	\ex The vowel generally occurs in heterorganic clusters.
	\ex The vowel is likely to be optional, have a highly variable duration or disappear at fast speech rates.
	\ex The vowel does not seem to have the function of repairing illicit
	structures. The consonant clusters in which the vowel occurs may
	be less marked, in terms of sonority sequencing, than clusters which
	surface without vowel insertion in the same language.
\z
\ex Properties of phonologically visible inserted vowels (epenthetic vowels) \citep[391]{Hall2006}\label{HallDiagnostics2}
	\ea The vowel's quality may be fixed or copied from a neighboring vowel. A fixed-quality epenthetic vowel does not have to be schwa.
	\ex If the vowel’s quality is copied, there are no restrictions as to which consonants may be copied over.
	\ex The vowel’s presence is not dependent on speech rate.
	\ex The vowel repairs a structure that is marked, in the sense of being cross-linguistically rare. The same structure is also likely to be avoided by means of other processes within the same language.
\z
\z

\noindent \citeauthor{Hall2006}'s diagnostics are used throughout this paper to distinguish between types of inserted vowels. Note that these diagnostics do not necessarily distinguish between vowel insertion versus deletion, but rather between multiple types of vowel insertion. A separate set of diagnostics for determining whether a given vowel is best analyzed as inserted, or underlying and subject to deletion, is developed throughout the paper and presented in \sectref{sec:conclusion}.

\subsubsection{V/$\emptyset$ as deletion}\label{sec:deletion}
Vowel/$\emptyset$ alternations are also sometimes analyzed as deletion, primarily when the quality of the vowel that participates in the alternation is not predictable given the phonological context. Instances of vowel deletion tend to, but do not necessarily, interact with stress and syllable weight \citep{Hawkins:1950, Anderson:1965, Al:1981, Willett:1982, Willett:1991, Fitzgerald:1997, Fitzgerald:2002, Kager:1997, Yu:2000stress,  Jacobs:2004, Riggle:2006, McCarthy:Syncope}. 

One example of a language with a Vowel/$\emptyset$ alternation that has been analyzed as deletion comes from Southeastern Tepehuan (ST), \citep{Willett:1982, Willett:1991, Kager:1997}. In ST, all words have stress on the first or second syllable, whichever is heaviest (CVV $\gg$ CVC $\gg$ CV). If they are both equally heavy, the first syllable is stressed \citep[176]{Willett:1982}. Long vowels shorten and short vowels delete in unstressed syllables, which is particularly visible in reduplication contexts where the vowel shortens or deletes in one copy (the base) but not the other (the reduplicant). In reduplication the initial syllable is copied and prefixed to the base. If the initial syllable is underlyingly heavy as in (\ref{tepehuan}a), the first syllable of the base is shortened in reduplication, resulting in a light unstressed syllable. If the initial syllable is light as in (\ref{tepehuan}b), then the first vowel of the base, which is a short unstressed vowel, is deleted. We know that the vowel is underlyingly present because it is present in non-reduplicated forms, as well as in the reduplicant itself, but not in the base.

\NumTabs{6}
\begin{exe}
    \ex Southeastern Tepehuan syncope\label{tepehuan}\\
    \tab \tab \hspace{4mm} Underlying \hspace{3mm} Surface
    \begin{xlist}
        \ex  Singular\tab /kooʔ/ \tab (kóoʔ) \tab \tab `snake' \\
			 Plural \tab  /koo-kooʔ/ \tab (kóo).koʔ \tab \tab `snakes'\\
		 \ex Singular \tab /topaa/ \tab (to.páa) \tab \tab `pestle'\\
			 Plural \tab /to-topaa/ \tab (tót.pa, *tó.to.pa) \tab `pestles'
    \end{xlist}
\end{exe}

\noindent The quality of the vowel in the V/$\emptyset$ alternation in ST is not predictable from the phonological context. Additionally, the result of deletion is phonologically optimizing: stressed syllables are heavy but unstressed syllables are light. These two facts lead to the conclusion that this alternation involves deletion, and that it is phonologically motivated and derived. 

\subsubsection{Diagnosing deletion versus insertion}\label{sec:diagnostics}
Based on previous work on CVCV\slash CCV alternations, the following two lines of investigation can serve as a starting point in determining whether a CVCV\slash CCV alternation involves deletion or insertion. First, if the first vowel in CVCV forms that alternate with CCV is unpredictable given the phonological context, it must be analyzed as underlyingly present and deleted: CVCV$\rightarrow$CCV. On the other hand, a purely phonologically predictable initial vowel in alternating CVCV\slash CCV forms points to underlying /CCV/ plus vowel insertion. Second, we can determine whether the first vowel in an alternating CVCV form participates independently in the phonology of the language from the second vowel: Can the first vowel host its own tone or stress? Can it participate in phonological processes independently from the second vowel? If the first vowel is prosodically and phonologically independent of the second vowel, it is likely not intrusive, and may involve deletion or, if its quality is phonologically predictable, it may be analyzable a phonologically visible epenthetic vowel (see \sectref{sec:epenthesis} for \citeauthor{Hall2006}'s diagnostics for intrusive versus epenthetic vowels).

\subsection{Background on Kru languages}\label{sec:background}
Kru languages are spoken in Liberia and Côte d'Ivoire (\figref{KruMap}).

\begin{figure}
\caption{Kru languages \citep{Marchese:1979}}
\includegraphics[width=\textwidth]{figures/KruMap.pdf}
\label{KruMap}
\end{figure}
 
The Kru language family tree in \figref{KruFamilyTree} shows a split between Eastern and Western Kru, for which there is very clear historical evidence \citep{Marchese:1986, Zogbo:2012, Zogbo:2019}, as well as widely agreed-upon groups within Eastern and Western Kru. Not included are Aizi, Kuwaa, and Seme, which are sometimes called Kru isolates, but whose classification as Kru at all is debated \citep{Marchese:1979, Herault:1971, Vogler:2020}. The languages examined in depth in this paper come from the Neyo-Dida group, in bold.

\begin{figure}
\begin{forest}
forked edges
[Kru
	[Western
		[Bassa]
		[Grebo]
		[Klao]
		[Wee
			[Guere-Krahn]
			[Konobo]
			[Nyabwa]
			[Wobe-Wee]
		]
	]
	[Eastern
		[Bakwe-Wane]
		[Bete]
		[\textbf{Neyo-Dida}]
		[Kodia]
	]
]
\end{forest}
\caption{Kru language family tree}
\label{KruFamilyTree}
\end{figure}

There are several canonical phonological properties of Kru languages, including contrastive labiovelar and labialized velar stops in addition to voiced and voiceless stops at other places of articulation. There is a bilabial implosive in all Eastern Kru languages and many Western Kru languages (not so in so-called ``Kru isolates", \citealt[41]{Marchese:1979}), as well historical evidence for a Proto-Kru alveolar implosive \citep{Zogbo:2012}. In many Western Kru languages with contrastive nasal vowels, nasal consonants are analyzed as non-contrastive. Rather, they are allophones of oral consonants that surface adjacent to a nasal vowel (e.g. /l/ $\rightarrow$ [n] / \_[+nasal]), \citep{Zogbo:2019}. Many Eastern Kru languages show synchronic alternations between sonorant consonants and nasal consonants in the environment of a nasal consonant, though there are also demonstrably contrastive nasal consonants and a lack of nasal vowels in Eastern Kru. All Kru languages show vowel harmony, often ATR harmony, and sometimes also height harmony \citep{Marchese:1979}.

Kru languages have multiple contrastive tone heights, tones that differentiate lexical items, and grammatical tone patterns. No Kru language has fewer than three contrastive tone heights, and many have four. There are also contour tones on single tone-bearing units that can be analyzed as made up of sequences of level tones. Grammatical tone marks aspect distinctions, negation, and nominative versus accusative versus genitive case. Throughout, tone is marked as in the original data source, sometimes with numeral superscripts where higher numbers represent higher tones, and sometimes with diacritics on vowels. 

Syllables in Kru languages are almost exclusively CV, though V-initial syllables are possible in pronouns and loan words. There are complex segments such as [\ipa{kp, gb, kʷ, gʷ}], though there are no consonant clusters other than the alternating CVCV\slash CCV sequences that are the topic of this paper.


\section{Insertion in Dida}\label{sec:dida}
\subsection{Language background}
Dida is a dialect cluster spoken in south-central Côte d'Ivoire. This section focuses on the Lakota cluster of Dida languages, which itself is made up of three varieties: Lakota, Abou, and Vata. Dida Lakota is spoken in the town of Lakota and surrounding village communities by about 93,800 people as of 1993, according to Ethnologue \citep{Ethnologue20}.

The data presented here comes from existing descriptions of Dida in the literature: \citet{Guehoun:1993} on Dida Lakota, \citet{Vogler:1976,Kaye:1981,Kaye&Charette,Kaye:1982} on Dida Lakota and Vata, and \citet{Masson:1992} on Dida Yocoboué.

The vowel inventory of Dida is provided in \figref{didavowels}.

\begin{figure}
\caption{Dida Vowel inventory}
\begin{center}
		\begin{vowel}
			%    \putcvowel[l]{i}{1}
    		\putvowel[l]{i}{0pt}{0pt}
   			%\putcvowel[r]{y}{1}
   			\putcvowel[l]{e}{2}
   			%\putcvowel[r]{\o}{2}
   			\putcvowel[l]{\textepsilon}{3}
  			%\putcvowel[r]{\oe}{3}
    		%\putcvowel[l]{a}{4}
    		%\putcvowel[r]{\textscoelig}{4}
   			%\putcvowel[l]{\textscripta}{5}
    	    %\putcvowel[r]{\textturnscripta}{5}
    		%\putcvowel[l]{\textturnv}{6}
    		\putcvowel[r]{\textopeno}{6}
    		%\putcvowel[l]{\textramshorns}{7}
    		\putcvowel[r]{o}{7}
    		%\putcvowel[l]{\textturnm}{8}
    		\putcvowel[r]{u}{8}
    		%\putcvowel[l]{\textbari}{9}
    		%\putcvowel[r]{\textbaru}{9}
    		%\putcvowel[l]{\textreve}{10}
    		%\putcvowel[r]{\textbaro}{10}
    		\putcvowel{ə/ʌ}{11}
    		%\putcvowel[l]{\textrevepsilon}{12}
    		%\putcvowel[r]{\textcloserevepsilon}{12}
    		\putcvowel{ɪ%\textscy
    		}{13}
    		\putcvowel{ʊ}{14}
    		\putcvowel{%\textturna
    		a}{15}
    		%\putcvowel{\ae}{16}
		\end{vowel}
	\end{center}
\label{didavowels}
\end{figure}


The consonant inventory is provided in Table \ref{didaconsonants}.


\begin{table}
\small
\caption{Dida consonant inventory \citep[38]{Guehoun:1993}} \label{didaconsonants}
\begin{tabular}{l *6{c@{~}c}}
			\lsptoprule
			& 
				\multicolumn{2}{c}{Bilabial} &					% Bilabial
				\multicolumn{2}{c}{Lab. dent.} & 			% Labiodental
				\multicolumn{2}{c}{Alveo-pal.} & 					% Dental
				\multicolumn{2}{c}{Palatal} & 					% Palatal
				\multicolumn{2}{c}{Velar} & 					% Velar
				% Labialized
				\multicolumn{2}{c}{Lab. vel.}  					% labiovelar
				\\\midrule

Plosive &  							% Plosive
				\ipa{p} & \ipa{b} &													% Bilabial
				&&														% Labiodental
				\ipa{t} & \ipa{d} &						%
				\ipa{c} & ɟ &														% Palatal
				\ipa{k} & \ipa{g} &													% Velar
			
				%labialized
				\ipa{kp} & \ipa{g}b
				%labiovelar
					      \\					

Nasal & 							% Nasal
				& \ipa{m} &													% Bilabial
				& &											% Labiodental
				& \ipa{n} &							% Dental
				
				& \ipa{ɲ} &														% Palatal
				& \ipa{ŋ} &		&												% Velar
		
			    %labiovelar
						\\

Fricative & 						% Fricative
				%\ipa{F} & \ipa{B} &
				&&								% Bilabial
				\ipa{f} & \ipa{v} &													% Labio
				
				%\ipa{T} & \ipa{D} &
				
				\ipa{s} & \ipa{z} &																	%\ipa{S} & \ipa{Z} &
				%%							% 
				
				%Post-alveolar
								&&						% Palatal
				%x & \ipa{G} &
				&&		
												% Velar
			&
				%labialized	%labiovelar
				 \\ 

Approx & 							% Approx.
				& ɓ &														% Bilabial
				&& % \ipa{V} &											% Labiodental
	& \ipa{l} &
																		% Post-alveolar
								& \ipa{j} &														% Palatal
				&& % \textturnmrleg &									% Velar
				%labialized
				& \ipa{w}
				%labiovelar
							  \\	

			\lspbottomrule
		\end{tabular}
\end{table}

The Vata variety is analyzed as having four distinct tone heights \citep{Kaye&Charette}, while other Dida varieties have three contrastive tone heights \citep[68]{Guehoun:1993}, written with diacritics on vowels (the diacritic on mid-toned vowels is often left off). Grammatical tone differentiates aspect, negation, and case, as in other Kru languages.

\citet{Guehoun:1993} analyzes /gʷ, kʷ, ŋʷ/ as sequences of two consonants, but they distributionally function as singletons, and are analyzed as single segments in all other sources on Dida and other Kru languages (cf. \citealt[39]{Marchese:1979}). They are treated as singletons here. Otherwise, the only surface clusters in the language are found in alternating CVCV\slash CCV forms.

\subsection{Insertion or deletion?}
In Dida, all alternating CVCV\slash CCV forms share the property of having /l/ as their second consonant, CVLV/CLV (though note that /l/ surfaces as [r] after certain consonants in Dida, as in \tabref{tab:didacvcv}c). Additionally, the first vowel in alternating CVCV forms always matches the vowel quality of the following vowel, and the tone on the first vowel always exactly matches that of the second vowel.

In \tabref{tab:didacvcv} the alternating sequences are underlined. Note that the alternating CVCV sequence can be the first or last CVCV sequence of a surface CVCVCV form (\tabref{tab:didacvcv}a,d,e), or it can be the entirety of a surface CVCV form (\tabref{tab:didacvcv}b,c).

\begin{table}
\caption{Dida Lakota CVCV $\sim$ CCV alternations \citep[56]{Guehoun:1993} \label{tab:didacvcv}}
\begin{tabularx}{.8\textwidth}{lXXl}
\lsptoprule
& CVCV & CCV & \\
\midrule
a. & \ipa{\uline{wʊ̀lʊ̀}lɪ} & \ipa{\uline{wlʊ̀}lɪ} & `to leave'\\
b. & \ipa{\uline{ŋɛlɛ}} & \ipa{\uline{ŋlɛ}} & `smell'\\
c. & \ipa{\uline{ɟulu}} & \ipa{\uline{ɟru}} & `salt'\\
d. & \ipa{kpo\uline{kele}} & \ipa{kpo\uline{kle}} & `stool/chair'\\
e. & \ipa{du\uline{gbulu}} & \ipa{du\uline{gblu}} & `village center'\\
\lspbottomrule
\end{tabularx}
\end{table}

The synchronic facts in Dida point to the first vowel (V1) in alternating CVCV sequences as being inserted, rather than deleted in CCV sequences. All cases of alternating CVCV $\sim$ CCV sequences in Dida exhibit a V1 identical in features to V2, which means the initial vowel is entirely phonologically predictable. Additionally, all cases exhibit identical tone on V1 and V2, with no evidence that the first vowel displays independent prosodic or phonological behavior from the second vowel. Outside of these alternating forms, sequences of distinct tones on non-alternating CVCV sequences are quite common (e.g. \ipa{pal\`ɛli} `enter' with MLM), as are sequences of CVCV with non-identical vowels where the second consonant is /l/ (e.g. \ipa{wɔlʊ} `granary').

Additional evidence for a vowel-insertion analysis of Dida CVCV\slash CCV sequences comes from the surface tone patterns in words longer than two syllables, where the alternating CVCV\slash CCV is initial. Specifically, in Dida, contour tones and sequences of identical level tones only surface at the right edge of a word. Tones in Dida can be analyzed as associating to vowels (or tone-bearing units) one-to-one from left-to-right, such that any one-to-many (adjacent identical tone sequences) or many-to-one (contour tones) tone-to-vowel mappings occur at the right edge of a word, and not the left \citep{Guehoun:1993}. The L.L.M (low, low, mid) tonal pattern in \tabref{tab:didacvcv}a is not a possible lexical tone melody in Dida; it is unattested outside of alternating forms with an initial CVCV\slash CCV sequence. This initial one-to-many mapping, where two adjacent syllables surface with the same level tone, is unexpected given the regular tone association process in Dida; typically, on a three-syllable word with a L.M melody the Dida, the tone-to-vowel mapping results in a surface L.M.M pattern. The tonal pattern in \tabref{tab:didacvcv}a would thus need to be analyzed as an exception to the otherwise robust left-to-right tone-to-vowel association patterns in Dida, unless the initial vowel in [\ipa{\uline{wʊ̀lʊ̀}lɪ}] is inserted after tonal association takes place. The tonal patterns provide an additional piece of evidence that alternating CVCV\slash CCV sequences in Dida are best analyzed as /CCV/ with vowel insertion.

We can now consider whether the inserted vowels in Dida are phonologically visible epenthetic vowels, or phonologically invisible intrusive vowels, based on the diagnostics from \citet{Hall2006} given in (\ref{HallDiagnostics1}) and (\ref{HallDiagnostics2}). The initial vowel in alternating CVCV sequences in Dida has the same features as the following vowel. This type of vowel feature copying does not, on its own, distinguish between being epenthetic (\ref{HallDiagnostics2}a) or intrusive (\ref{HallDiagnostics1}a). However, the intervening consonant is always a sonorant (/l/), as predicted for cases of vowel intrusion (\ref{HallDiagnostics1}b), and the vowel is optional and dependent on speech rate, as predicted for cases of vowel intrusion (\ref{HallDiagnostics1}d) but not epenthesis (\ref{HallDiagnostics2}c). Thus, the CCV/CVCV alternation in Dida seems to be best analyzed as a case of underlying /CCV/ with intrusive vowels. On this account, non-alternating CVCV roots are analyzed as underlyingly /CVCV/ whereas alternating roots are analyzed as /CCV/. The representational difference results in distinct surface patterns, where /CVCV/ roots always surface as [CVCV], while /CCV/ roots surface either as [CCV] or with an optional intrusive vowel as [CVCV].

\section{Deletion in Guébie}\label{sec:guebie}
\subsection{Language background}
Guébie is an endangered Kru language spoken in the Gagnoa prefecture of southwest Côte d'Ivoire by about 7000 people. There is a related variety spoken across the border in Lakota that is also often called Guébie; however, the variety described here is the Gagnoa-Guébie variety, and more specifically, the Guébie of the village of Gnagbodougnoa. Guébie is sometimes classified as part of the Bété subgroup because Bété is the dominant indigenous language of the Gagnoa prefecture. However, Guébie is much more closely related to Dida than to Bété, as argued by \citet{Sande:classification} and as articulated by Guébie speakers, who report that while they can understand some Bété, Guébie is much more similar to Dida.

The data presented here comes from field work in collaboration with the Guébie community of Gnagbodougnoa, Côte d'Ivoire between 2013--2022, including seven in-situ fieldtrips, two years of remote elicitation, and two years of work with native speakers in the US and Canada. An initial study of the CVCV\slash CCV alternation in Guébie is available in \citet{Sande:diss}. All Guébie examples are labeled with a three-letter code that corresponds to the speaker who provided the examples, as well as the date in YYYYMMDD format. These labels correspond to file bundles of audio files and field notes in the Guébie archival collection in the California Language Archive \citep{BodjiSande:2014}.

The vowel inventory of Guébie is quite similar to that of Dida, and is given in \figref{guebieinventory}.

\begin{figure}
\caption{Guébie Vowel inventory} \label{guebieinventory}
\begin{vowel}
	%    \putcvowel[l]{i}{1}
	\putvowel[l]{i}{0pt}{0pt}
	%\putcvowel[r]{y}{1}
	\putcvowel[l]{e}{2}
	%\putcvowel[r]{\o}{2}
	\putcvowel[l]{\textepsilon}{3}
	%\putcvowel[r]{\oe}{3}
	%\putcvowel[l]{a}{4}
	%\putcvowel[r]{\textscoelig}{4}
	%\putcvowel[l]{\textscripta}{5}
    %\putcvowel[r]{\textturnscripta}{5}
	%\putcvowel[l]{\textturnv}{6}
	\putcvowel[r]{\textopeno}{6}
	%\putcvowel[l]{\textramshorns}{7}
	\putcvowel[r]{o}{7}
	%\putcvowel[l]{\textturnm}{8}
	\putcvowel[r]{u}{8}
	%\putcvowel[l]{\textbari}{9}
	%\putcvowel[r]{\textbaru}{9}
	%\putcvowel[l]{\textreve}{10}
	%\putcvowel[r]{\textbaro}{10}
	\putcvowel{\textschwa}{11}
	%\putcvowel[l]{\textrevepsilon}{12}
	%\putcvowel[r]{\textcloserevepsilon}{12}
	\putcvowel{\textsci\ %\textscy
	}{13}
	\putcvowel{ʊ}{14}
	\putcvowel{%\textturna
	a}{15}
	%\putcvowel{\ae}{16}
\end{vowel}
\end{figure}


 The consonant inventory is provided in Table \ref{guecons}.

\begin{table}%{\vspace{-3ex}}
\caption{Guébie consonant inventory} \label{guecons}
\resizebox{\textwidth}{!}{%
		\begin{tabular}{l *7{c@{~}c}}
%\begin{tabular}{lcc}
			\lsptoprule
			 &  \multicolumn{2}{c}{{Bilabial}} &					% Bilabial
				\multicolumn{2}{c}{{Lab. dent.}} & 			% Labiodental
				\multicolumn{2}{c}{{Alveo-pal.}} & 					% Dental
				\multicolumn{2}{c}{{Palatal}} & 					% Palatal
				\multicolumn{2}{c}{{Velar}} & 					% Velar
				\multicolumn{2}{c}{{Labialized}} &
				% Labialized
				\multicolumn{2}{c}{{Lab. vel.}}  					% labiovelar
				\\

			\midrule Plosive &  							% Plosive
				\ipa{p} & \ipa{b} &													% Bilabial
				&&														% Labiodental
				\ipa{t} & \ipa{d} &						%
				\ipa{c} & \ipa{ɟ} &														% Palatal
				\ipa{k} & \ipa{g} &													% Velar
				\ipa{kʷ} &
				\ipa{g}ʷ &
				%labialized
				\ipa{kp} & \ipa{g}b
				%labiovelar
					      \\					

			 Nasal & 							% Nasal
				& \ipa{m} &													% Bilabial
				& &											% Labiodental
				& \ipa{n} &							% Dental
				
				& \ipa{ɲ} &														% Palatal
				& ŋ &														% Velar
				& ŋʷ&
				%labialized
			    & ŋm
			    %labiovelar
						\\
			 Fricative & 						% Fricative
				%\ipa{F} & \ipa{B} &
				&&								% Bilabial
				\ipa{f} & \ipa{v} &													% Labio
				
				%\ipa{T} & \ipa{D} &
				
				\ipa{s} &  &																	%\ipa{S} & \ipa{Z} &
				%%							% 
				
				%Post-alveolar
								&&						% Palatal
				%x & \ipa{G} &
				&&		
												% Velar
			&&
				%labialized
					&
				%labiovelar
				 \\ 

				 Approx & 							% Approx.
				& ɓ &														% Bilabial
				&& % \ipa{V} &											% Labiodental
	& \ipa{l} &
																		% Post-alveolar
								& \ipa{j} &														% Palatal
				&& % \textturnmrleg &									% Velar
				&&
				%labialized
				& \ipa{w}
				%labiovelar
							  \\	
\lspbottomrule
		\end{tabular}}
\end{table}	

Guébie has four distinct tone heights marked here with numerals 1--4, where 4 is high. Multiple level tones can surface on a single short vowel, resulting in a surface contour tone. Each morpheme is associated with a tone melody underlyingly (except the definite enclitic, whose tone is determined by the tone of the word it attaches to), and the tone melody associates one-to-one with vowels in the word, from left-to-right \citep{Sande:diss}. See \citet{Sande:LLC} for a description of the phonology of Guébie.

Syllables in Guébie typically have the shape CV, though pronouns and loanwords can be vowel-initial, and in certain derived contexts coda nasals appear. Specifically, the vowel of a phrase-final /NV/ sequence can be unpronounced, resulting in a phrase-final nasal coda, in which case the tone of the unpronounced vowel is produced on the preceding vowel. As in Dida, there is a set of words that display CVCV\slash CCV alternations. Also as in Dida, this alternation is variable and lexically specific.

\subsection{Insertion or deletion?}
Certain roots can surface as either CVCV or CCV in Guébie. The forms in \tabref{tab:guebiealternations} show that while some alternating CVCV sequences have the same vowel and tone on both syllables, and in some cases the second consonant is /l/, there are also alternating forms where the vowels in the two syllables differ (\tabref{tab:guebiealternations}f,g,h), those where the tones on the two syllables differ (\tabref{tab:guebiealternations}f,g,h,i), and those where the second consonant is not /l/ (\tabref{tab:guebiealternations}d,e,g,h,i).\footnote{There is an alternation between [l] and [r] in Guébie, where [r] is typically used in onset clusters (CCV), and [l] is used elsewhere. While all surface [l] and [r] consonants come from underlying /l/, I use [r] in clusters and [l] elsewhere to reflect production patterns.}

\begin{table}
\caption{CVCV reduced to CCV \label{tab:guebiealternations} (syl\_20161207)}
\begin{tabularx}{.8\textwidth}{lXXl}
	\lsptoprule
	& CVCV & CCV & \\
	\midrule
	a. & \ipa{\uline{bala}$^{3.3}$} & \ipa{\uline{bra}$^{3}$} & `hit'\\
	b. & \ipa{\uline{tulu}$^{4.4}$} & \ipa{\uline{tru}$^{4}$} & `chase'\\
	c. & \ipa{\uline{wʊlʊ}$^{3.3}$} & \ipa{\uline{wrʊ}$^{3}$} & `granary'\\
	d. & \ipa{\uline{munu}$^{3.3}$} & \ipa{\uline{mnu}$^{3}$} & `bite/sting'\\
	e. & \ipa{\uline{mana}$^{3.3}$} & \ipa{\uline{mna}$^{3}$} & `meat'\\
	f. & \ipa{\uline{jɪla}$^{2.3}$} & \ipa{\uline{jra}$^{23}$} & `ask'\\
	g. & \ipa{\uline{sija}$^{2.3}$} & \ipa{\uline{sja}$^{23}$} & `be defeated' \\
	h. & \ipa{\uline{kuɓə}$^{3.1}$} & \ipa{\uline{kɓə}$^{31}$} & `yesterday'\\
	i. & \ipa{\uline{duɓuɓili}$^{3.1.1.2.2}$} & \ipa{\uline{dɓuɓri}$^{3.1.2}$} & `mourning'\\
	\lspbottomrule
\end{tabularx}
\end{table}

\noindent Note that in \tabref{tab:guebiealternations}i, both the first CVCV and the final CVCV sequence can surface as CCV. The second consonant in CCV forms can be /l/ (with the surface form [n] when the preceding consonant is nasal, or [r] immediately after a non-nasal consonant), a glide, or the implosive /\ipa{ɓ}/.

All roots that can surface as CCV have a corresponding CVCV form. Not all CVCV sequences can surface as CCV (\tabref{tab:guebienonalternation}).

\begin{table}
\caption{Non-alternating roots (syl\_20161207, syl\_20170315)\label{tab:guebienonalternation}}
\begin{tabularx}{.8\textwidth}{lXXl}
\lsptoprule
& CVCV & CCV & \\
\midrule
a. & \ipa{ɟʊla$^{3.2}$} & *\ipa{ɟra$^{32}$} & `take/borrow'\\
b. & \ipa{tɛlɪ$^{3.3}$} & *\ipa{trɪ$^{3}$} & `carve' \\
c. & \ipa{sijo$^{2.3}$} & *\ipa{sjo$^{23}$} & `wipe' \\
d. & \ipa{ɲɛpɛ$^{3.1}$} & *\ipa{ɲpɛ$^{31}$} & `sweep' \\
\lspbottomrule
\end{tabularx}
\end{table}

Unlike the Dida case, it is not immediately clear given the Guébie facts whether the CVCV\slash CCV alternation in Guébie is best analyzed as insertion or deletion. V1 of the alternating CVCV sequences need not have the same features as V2 (\ref{tab:guebiealternations}f-h). V1 can host its own tone, independent of V2, and in line with regular tonotactics of the language, and [23] contour tones are common in alternating CCV contexts (\ref{tab:guebiealternations}g) and non-alternating monosyllabic contexts (/\ipa{ɲa$^{23}$}/ `rubber'). V1 in alternating CVCV sequences is not significantly shorter in duration than V1 in non-alternating CVCV sequences.\footnote{I do not have access to duration data from Dida to compare with the Guébie duration facts.} These facts suggest that the initial vowel in alternating CVCV sequences is not intrusive. However, it may still be analyzable as epenthetic (/CCV/ $\rightarrow$ [CVCV]) or as deleted (/CVCV/ $\rightarrow$ [CCV]). To determine whether the initial vowels in Guébie alternating CVCV forms can be analyzed as epenthetic, we must determine whether the initial vowels are phonologically predictable.

There is one crucial piece of evidence that leads to the conclusion that no set of phonotactic traits or phonological conditioning factors can predict the initial vowel in alternating CVCV forms.  Namely, there are sets of words with the same CCV form, but distinct vowels in their CVCV forms (\tabref{tab:guebienotpred}).

\begin{table}
\caption{C1 is not predictable in Guébie \label{tab:guebienotpred} (bor\_20150602)}
\begin{tabularx}{.8\textwidth}{lXXl}
\lsptoprule
& CCV & CVCV & \\
\midrule
a. & \ipa{jra$^{23}$}  & \ipa{jɛla$^{2.3}$} & `appear'\\
b. & \ipa{jra$^{23}$} & \ipa{jɪla$^{2.3}$}& `ask'\\
\lspbottomrule
\end{tabularx}
\end{table}

Given the surface form [{\ipa{jra$^{23}$}}], the CVCV form is not predictable; it could be either [{\ipa{jɛla$^{2.3}$}}] or [{\ipa{jɪla$^{2.3}$}}] (\tabref{tab:guebienotpred}). So, in Guébie, the first vowel in a CVCV word is not predictable given the phonological form of its CCV counterpart. This evidence, combined with the fact that the two vowels in alternating CVCV sequences can show independent prosody, lead me to posit that Guébie alternating forms are underlyingly /CVCV/, with optional deletion of the first vowel.

Additional phonological evidence supports the conclusion that the alternating CVCV words in Guébie are not underlyingly /CCV/. The same set of roots that show a CVCV$\sim$CCV alternation in Guébie show another phonological alternation: Vowel replacement determined by a subset of suffixes and enclitics. When a third-person object enclitic attaches to an alternating root, both vowels in the alternating root surface with the vowel quality of the enclitic. Alternating roots are shown followed by three distinct third-person object enclitics in \tabref{tab:object}.

\begin{table}
\caption{Object enclitics trigger vowel replacement on alternating roots (syl\_20170315)}\label{tab:object}

\begin{tabular}{llllll}
\lsptoprule
    & Bare verb          & \textsc{3sg.hum} =\ipa{ɔ$^{2}$} & \textsc{3sg} =\ipa{ɛ$^{2}$} & \textsc{3pl} =\ipa{ɪ$^{2}$} & \\\midrule
a.	& \ipa{jili$^{2.3}$} & \ipa{jɔl=ɔ$^{2.32}$} & \ipa{jɛl=ɛ$^{2.32}$} & \ipa{jɪl=ɪ$^{2.32}$} & `steal'\\
b.  & \ipa{jɪla$^{2.3}$} & \ipa{jɔl=ɔ$^{23.2}$} & \ipa{jɛl=ɛ$^{23.2}$} & \ipa{jɪl=ɪ$^{23.2}$} & `ask'\\
c.  & \ipa{bala$^{3.3}$} & \ipa{bɔl=ɔ$^{3.2}$} &  \ipa{bɛl=ɛ$^{3.2}$}  & \ipa{bɪl=ɪ$^{3.2}$}  & `hit'\\
d.  & \ipa{wʊla$^{3.1}$} & \ipa{wɔl=ɔ$^{3.12}$} & \ipa{wɛl=ɛ$^{3.12}$} & \ipa{wɪl=ɪ$^{3.12}$} & `look at'\\
\lspbottomrule
\end{tabular}
\end{table}

The initial vowel in non-alternating CVCV roots does not alternate in this way (\tabref{tab:strongobj}). The final vowel in any CVCV root fails to surface before a V-initial suffix or enclitic due to a regular vowel hiatus resolution strategy, so we are examining alternations in the quality of the initial vowel of the root. Non-alternating roots are shown with the human third-person singular object enclitic in \tabref{tab:strongobj}.

\begin{table}
\caption{Non-alternating roots in object contexts (syl\_20161207, syl\_20170315) \label{tab:strongobj}}
\begin{tabularx}{.8\textwidth}{lXXl}
\lsptoprule
& Root & Root=\ipa{ɔ$^{2}$} &\\
\midrule
a. & \ipa{sumu$^{2.2}$} & \ipa{sum=ɔ$^{2.2}$}, *\ipa{sɔmɔ$^{2.2}$} & `boil him'\\
b. & \ipa{ɟʊla$^{3.2}$} & \ipa{ɟʊl=ɔ$^{3.2}$}, *\ipa{ɟɔlɔ$^{3.2}$} & `take him'\\
c. & \ipa{tɛlɪ$^{3.3}$} & \ipa{tɛl=ɔ$^{3.2}$} & `carve him' \\
d. & \ipa{sijo$^{2.3}$} & \ipa{sij=ɔ$^{2.32}$} & `wipe him' \\
e. & \ipa{ɲɛpɛ$^{3.1}$} & \ipa{ɲɛp=ɔ$^{3.12}$} & `sweep him' \\
\lspbottomrule
\end{tabularx}
\end{table}

Most third-person pronouns have the shape of a single vowel, but the human 3\textsc{pl} pronoun is /=\ipa{ʊa$^{2}$}/, produced [=wa$^{2}$], [{\ipa{ɓatɛ$^{3.1}$}}] $\rightarrow$ [{\ipa{ɓat=wa$^{3.12}$}}],  `search for them'. When this enclitic attaches to an alternating root, the first vowel of the root surfaces as [ʊ] and the second vowel is pronounced [a] (\tabref{tab:vreplace}).

\begin{table}[t]
\caption{3\textsc{pl} human object pronouns on alternating roots (syl\_20170315, syl\_20210817, oli\_20210727) \label{tab:vreplace}}
\begin{tabularx}{.8\textwidth}{lXXl}
\lsptoprule
& Root & Root=\ipa{wa$^{2}$} & \\
\midrule
a. & \ipa{bala$^{3.3}$}/ & \ipa{bʊla$^{3.2}$}/ & `hit them'\\
b. & \ipa{jɪla$^{2.3}$} & \ipa{jʊla$^{2.32}$} & `ask them'\\
c. & \ipa{wʊla$^{3.1}$} & \ipa{wʊla$^{3.12}$} & `look at them'\\
\lspbottomrule
\end{tabularx}
\end{table}

Previous work has analyzed this alternation in third-person object contexts as \textit{vowel harmony} \citep{Sande:Language}; however, it may be better analyzed as vowel replacement, and not vowel harmony, due to the alternations in the context of alternating roots and third-plural human object pronouns as in \tabref{tab:vreplace}. We do not see the same features on both vowels in \tabref{tab:vreplace}, as would be expected if this were vowel harmony. Rather, we see the two underlying vowels of the pronoun surfacing sequentially in the two available vowel slots.

This phonological evidence suggests that there is a vowel slot in alternating CVCV words, which can be filled in with the first vowel of the /=\ipa{ʊa}/ \textsc{3pl.acc} object enclitic in vowel replacement contexts, much like Semitic non-concatenative morphology. If alternating words were underlyingly /CCV/, there would be no vowel slot for the first vowel of the \textsc{3pl.acc} marker to associate to, and we would expect a surface form such as *[\ipa{bʊlʊa$^{3.3.2}$}] (with harmony) or [\ipa{balʊa$^{3.3.2}$}] (without harmony) rather than [\ipa{bʊla$^{3.2}$}] in \tabref{tab:vreplace}a. However, if all alternating morphemes are underlyingly /CVCV/, the vowel replacement facts can be straightforwardly accounted for as non-concatenative association of vowels or vocalic features to a CVCV template.

\subsection{Distinguishing alternating from non-alternating CVCV roots}

In an analysis of Guébie where alternating roots are underlyingly /CVCV/ with optional V-deletion, the question becomes how to differentiate alternating /CVCV/ roots from non-alternating /CVCV/ roots. Are alternating roots phonotactically distinct from alternating ones? Or is the availability of alternation lexically specific? First, I show that no set of phonotactic features picks out all and only the alternating set of roots.

There are a number of phonotactic traits that alternating roots tend to display, specifically those listed in (\ref{traits}).%\footnote{See the appendix for an attempt at modeling the choice of alternation versus non-alternation phonologically. This attempt does not work because it incorrectly assumes that all lexical items with the same phonotactic properties will show the same surface probabily of surfacing as CVCV or CCV; however, that is not the case in the data set.}
 
\ea Common phonotactic traits of CVCV sequences\label{traits}
\begin{itemize}
	\item C2 is a sonorant (/l, \ipa{ɓ}, w, j/ (or a nasal in nasal roots))
	\item V1 and V2 are identical
	\item T1 and T2 are identical
\end{itemize}
\z 

\noindent However, not every root with these features alternates, and not every alternating root has (some subset of) these features. V1 and V2 have identical vowel quality in 329 of the 617 distinct alternating CVCV roots in a Guébie corpus of over 12,000 utterances. We find the same prosody (identical tone) on both syllables in 270 of the 617 alternating forms. If either of these phonotactic traits were diagnostic of CVCV alternation with CCV, we would expect all, or nearly all, alternating roots to show these properties. What we find, though, is that only about half of alternating roots have the same vowel on both syllables in their CVCV form, and fewer than half of alternating roots have the same tone on both syllables.

Perhaps the most crucial piece of evidence that this alternation is lexically specific, and not determined by the phonotactics of a given CVCV form, is that there are (near) minimal pairs of CVCV forms where one alternates and the other does not (\tabref{tab:pairs}).

\begin{table}
\caption{(Near) minimal pairs of alternating and non-alternating roots (syl\_20161207)\label{tab:pairs}}
\begin{tabularx}{.8\textwidth}{@{}lXXl@{}}
\lsptoprule
	& CVCV & CCV &\\
	\midrule
		a. & \ipa{jili$^{2.2}$} & \ipa{jri$^{2}$} & `be fat'\\
		b. & \ipa{jili$^{2.2}$} & *\ipa{jri$^{2}$} &`fish'\\
		c. & \ipa{gɔlɔ$^{3.3}$} &  \ipa{grɔ$^{3}$} & `pain'\\
		d. & \ipa{gɔlɔ$^{2.3}$} &  *\ipa{grɔ$^{23}$} & `canoe'\\
		e. & \ipa{kpolo$^{3.1}$} &  \ipa{kpro$^{31}$} & `be clean'\\
		f. & \ipa{kpoke$^{2.4}$} & *\ipa{kpke$^{24}$} &  `crocodile'\\
		g. & \ipa{ɟulu$^{3.3}$} &   \ipa{ɟru$^{3}$}& `salt'\\
		h. & \ipa{ɟʊla$^{3.2}$} &   *\ipa{ɟra$^{32}$}& `take/borrow'\\
		\lspbottomrule
\end{tabularx}
\end{table}

\noindent The existence of minimal pairs means that at least some information about the availability of /CVCV/ alternation with [CCV] must be lexically specified. No combination of phonotactic traits exclusively and exhaustively predicts whether a root falls into the alternating or non-alternating class. Thus, the Guébie CVCV\slash CCV alternation is best analyzed as involving underlying /CVCV/ forms that optionally surface as [CCV], where alternating and non-alternating forms are distinguished from each other not based on phonotactics, but lexically.

Any analysis of this deletion phenomenon must account for the fact that this alternation is lexically specific and optional. Deletion is lexically specific in that it does not apply across the board to all lexical items equally, but only applies to a subset of lexically specified CVCV sequences. Deletion of the vowel applies optionally in that in any given morphosyntactic and phonological environment, an alternating sequence can surface as either CVCV or CCV.

There are two major classes of analyses that could equally well differentiate alternating from non-alternating CVCV roots in Guébie. One of these types of analysis involves a different underlying representation for alternating and non-alternating forms. One such analysis could involve gradient symbolic representations, where the initial vowel in alternating roots is only partially activated, whereas the initial vowel in non-alternating roots is fully activated \citep{Goldrick&Smolensky:2016}. Another such analysis may stipulate that the vowel in alternating roots is represented as defective, lacking the same amount of prosodic structure as the vowels in non-alternating roots (cf. \cite{Zimmermann:2013, Zimmermann:2016}). The second plausible type of analysis for modeling lexically specific CVCV$\sim$CCV alternations involves multiple lexically sensitive phonological grammars or cophonologies \citep{Orgun:1996, Inkelas:1997, Anttila:2002, Inkelas&Zoll, InkelasZoll:2007}. Cophonology Theory assumes that different morphosyntactic constructions can be associated with distinct phonological constraint rankings or weightings. In Guébie, the class of alternating roots could be associated with a different constraint ranking than the class of non-alternating roots; see \citet{Sande:Language} for such an analysis of Guébie vowel alternations. The variability in whether a root surfaces as CVCV or CCV in a given utterance could be  modeled in a constraint-based account using Stochastic OT \citep{Boersma:1998, BoersmaHayes2001} or Maximum Entropy Harmonic Grammar \citep{GoldwaterJohnson:2003}, or through a gestural model \citep{Hall:2003, Hall2006}.

Interestingly, [CCV] forms in Guébie are more common in fast casual speech than in careful speech. \citeauthor{Hall2006}'s diagnostics (\citeyear{Hall2006}) suggest that if speech rate plays a role, vowel intrusion is more likely than vowel epenthesis, but these diagnostics do not make predictions about speech rate and vowel \textit{deletion}. In Guébie, we have seen that the diagnostics for vowel intrusion are not met, and that due to the unpredictabilty of V1, Guébie CVCV\slash CCV alternations are likely synchronically best analyzed as deletion. Guébie, then, presents a case of speech rate partially determining the output form in a vowel deletion rather than vowel intrusion process, contributing to our understanding of the typological characteristics associated with vowel deletion.

\section{Areal findings}\label{sec:areal}
The two Kru languages discussed in detail in Sections \ref{sec:dida} and \ref{sec:guebie} were chosen as representative Kru case studies because there is available data on CVCV\slash CCV alternations, the lexicon, and the phonology of each, and because they serve as representative examples of two common patterns of CVCV\slash CCV alternation found across Kru languages. This section discusses the synchronic CVCV\slash CCV alternations across Kru languages, as well as in nearby Mande and Kwa languages, which have been discussed in the literature for decades (cf. \citealt{LeSaout:1974}), providing a picture of the synchronic patterns of CVCV\slash CCV alternations in West Africa, and drawing conclusions about the diachronic development of these patterns. These areal and historical findings bear on our understanding of the broader typology and development of vowel/$\emptyset$ alternations across languages (cf. \cite[40]{Blevins&Pawley}, who present a typology of pathways to vowel loss and vowel insertion, but who do not consider tonal languages or West African languages). Any conclusions made here about proto-forms in Kru, Mande, and Kwa languages should be taken as tentative hypotheses that can be tested in the future with additional careful studies of individual languages and sub-groups like those for Dida and Guébie presented in Sections \ref{sec:dida} and \ref{sec:guebie}.

\subsection{Kru}\label{sec:kru}
In Table \ref{krucvcv}, a number of Kru languages are listed, along with their classification as Eastern or Western Kru. Each language listed is specified as showing a synchronic CVCV\slash CCV alternation or not. If there is such an alternation, they are marked as having a phonologically predictable initial vowel quality or not (taken as a diagnostic of whether the alternation involves vowel insertion or deletion), and the possible second position consonants (C2) are provided.

\begin{table}
\caption{CVCV$\sim$CCV across Kru}\label{krucvcv}
\begin{tabularx}{\textwidth}{Xccl}
	\lsptoprule
	Language & CVCV $\sim$ CCV? & V1 predictable? & Possible C2s\\
	\midrule
	Dida/Vata (Eastern) & Yes & Yes & /l/ \\
	Nyabwa (Western) & Yes & Yes & /l, \ipa{ɓ}, w/\\
	Neyo/Neouolé (Eastern) & Yes & Yes & /l/\\
	Guébie (Eastern) & Yes & -- & /l, \ipa{ɓ}, j, w/\\
	Grebo (Western) & Yes & -- & /l/ \\
	Bété (Eastern) & Yes & -- & /l/ \\
	Godié (Eastern) & Yes & -- & /l/ \\
	Déwoin (isolate) & -- & -- & \\
	Kuwaa (isolate) & -- & -- & \\
	\lspbottomrule
\end{tabularx}
\end{table}
	

The sources of the generalization, or the data leading to the generalizations in Table \ref{krucvcv} include \citet{Delafosse:1904,Thomann:1905,Innes:1966,Marchese:1979,Zogbo:1981}, \citeauthor{Kaye:1981}
(\citeyear{Kaye:1981}, \citeyear{Kaye:1982}), \citet{Masson:1992,Guehoun:1993,Egner:1989,Saunders:2009,Allou:2017,Sande:diss}.

Interestingly, the Eastern/Western Kru split does not seem to correspond with a split in behavior of CVCV\slash CCV forms. There are Eastern and Western Kru languages with predictable V1s in CVCV alternations, which could be analyzed as underlying /CCV/ forms with vowel epenthesis or intrusion, as in Dida. Additionally, there are both Eastern and Western Kru languages with CVCV\slash CCV alternations but where the quality of the first vowel is not phonologically predictable, as with Guébie. The latter group must be analyzed as having underlying /CVCV/ forms, with optional deletion of the first vowel.

From a diachronic perspective, the data point to synchronically alternating forms across Kru developing from Proto-Kru /CVCV/ forms that alternated with [CCV]. This was reinterpreted as insertion (underlying /CCV/ with predictable [CVCV] variants) in some languages, likely independently in each of the languages where it occurred (Dida, Nyabwa, Neyo). A single, systematic change can result in proto-CVCV forms being reanalyzed as CCV (vowel syncope or deletion), but not the other way around, since the quality of the initial vowel in alternating CVCV\slash CCV forms is unpredictable in Guébie, Grebo, Bété, and Godié.
	
Kru isolates like Kuwaa and Déwoin\footnote{Déwoin is sometimes classified as Western Kru, and sometimes as a Kru isolate \citep{Marchese:1979, Zogbo:2012}.}, do not show any CVCV$\sim$CCV alternation, but have non-alternating CVCV forms that correspond with alternating forms in Eastern and Western Kru, suggesting that Kuwaa and Déwoin may have split off from the rest of Kru before the CVCV$\sim$CCV alternation arose. This alternation is thus proposed to have existed in Kru since before the Eastern/Western split, but it must have arisen after isolates like Kuwaa and Déwoin diverged from the rest of Kru languages.

\subsection{Mande}\label{sec:mande}
Mande languages are spoken across West Africa, including in Liberia and Côte d'Ivoire where they are in contact with Kru languages. The major sub-division within Mande languages is between Eastern and Western Mande. Many Mande languages, like Kru languages, show synchronic alternations between CVCV and CCV surface forms. In Table \ref{mandecvcv}, Mande languages are listed, along with their classification as (South)eastern or Western Mande. Each language is marked as showing a synchronic CVCV\slash CCV alternation or not. For languages with synchronic CVCV alternations, I have marked whether the quality of the initial vowel in the CVCV form is predictable given the phonological context, and listed the possible consonants in C2 position. The second consonant, while underlyingly typically only including liquids, can surface as [l], [r], or [n] in most Mande languages, depending on the quality of the first consonant \citep{Vydrine:2004}. The sources of the generalizations in Table \ref{mandecvcv} include \citet{LeSaout:1974,Morse:1976,Diallo:1988,Grossmann:1982,Dumestre:2003,Vydrine:2004,Babaev:2011,Sadler:2015,Khachaturyan:2015,Vydrina:2017, McPherson:2020}.

\begin{table}
\caption{CVCV$\sim$CCV across Mande}
\label{mandecvcv}
\begin{tabularx}{\textwidth}{Xccl}
	\lsptoprule
	Language & CVCV $\sim$ CCV? & V1 predictable? & Possible C2s\\
	\midrule
    Toura (Southeastern) & Yes & Yes, identical & /l/ \\
	Gban (Southeastern) & Yes & Yes, identical & /l/\\
	Dan (Southeastern) & Yes & Yes, identical & /l/\\
	Mano (Southeastern) & Yes & Yes, identical & /l/\\
	Gouro (Southeastern) & Yes & Yes, predictable & /l/\\
	Seenku (Western) & Yes & Yes, schwa & /m,n,ŋ,l/\\
	Dioula (Western) & Yes & -- & /l,r/\\
	Zialo (Western) & Yes & -- & /l, j, w/ \\
	Jalkunan (Western) & Yes & -- & /l/\\
	Bambara (Western) & Yes & -- & /l,r,n/\\
	Kakabe (Western) & -- & -- & -- \\
	Lɔɔma (Western) & -- & -- & -- \\
	Bandi (Western) & -- & -- & --\\
	Bobo (Western) & -- & -- & --\\
	\lspbottomrule
	\end{tabularx}
\end{table}


For Southeastern Mande languages, \citet{Vydrine:2004} analyzes the CVCV\slash CCV alternations as involving underlying /CLV/ forms, which can be realized as CvLV. Realization as CLV or CvLV is variable and not dependent on phonological context. Following a long descriptive tradition, he writes the initial vowel with a lowercase ``v" because it is typically pronounced as shorter than other vowels. Similarly, \citet[54--56]{Bearth:1971} says the V1 of CVCV forms in Toura (Eastern Mande) is very short and always identical to V2. The first vowel in alternating CVCV forms in Southeastern Mande always matches the features of the second, or is otherwise predictable given phonological context. In Gouro, when the first consonant is velar or palatal, the initial vowel predictably surfaces as round; otherwise, the initial vowel matches the features of the second vowel. From the available examples, it seems that the tone is always identical on both syllables in alternating CVCV sequences. Given all available evidence, it seems quite straightforward to analyze the alternating forms in Southeastern Mande as underlyingly /CCV/, with the option of a phonologically predictable inserted vowel.\footnote{See also \citet{LeSaout:1979} and \citet{Vydrine:2010} on \textit{le syllabème} in Mande.} Given the fact that the initial vowel is always predictable, its presence is optional rather than phonologically determined, it is short or gradiently realized, and it often lacks its own prosody, the best analysis of the inserted vowels in Southeast Mande is that they are intrusive rather than epenthetic (cf. the diagnostics in \REF{HallDiagnostics1} and \REF{HallDiagnostics2}). 

In some Western Mande languages, some morphemes analyzed as disyllabic, often with a single tone level, act phonologically similar to the CLV morphemes of Southeasterm Mande \citep{Dumestre:2003, Vydrine:2004, Diakite:2006, Green:2010, Green:2015, Green:2018}. In Bambara, /CVLV/ can be pronounced as [CLV], which, according to \citet{Green:2010, Green:2018}, can be analyzed as due to a drive for monosyllabic words (minimality) (recall the discussion in \sectref{sec:cvcvintro} of previous work on vowel/$\emptyset$ alternations analyzed as syncope or deletion motivated by stress and prosody). On Green's analysis, underlying /CVCV/ forms surface as [CCV] when phonotactics allow. Reduction of CVCV to CCV is especially common when the initial vowel of the CVCV sequence is high (/i,u/) (\tabref{tab:bambara}a-f,j), or if two consecutive vowels have the same vowel quality (\tabref{tab:bambara}g-i,k). Tones on the two syllables do not need to be identical (\tabref{tab:bambara}d,j,k). In trisyllabic words, vowel syncope in Bambara can lead to surface [CCV] (\tabref{tab:bambara}a-c,g-k) or [CVC] syllables (\tabref{tab:bambara}d-i), depending on the phonotactics \citep[57--62]{Green:2010}.

\begin{table}
\caption{Bambara vowel syncope \label{tab:bambara}}
\begin{tabularx}{.8\textwidth}{lXXl}
\lsptoprule
& CVCV & CCV/CVC & \\\midrule
a. & \ipa{ká.bi.la} & \ipa{ká.blá} & `tribute'\\
b. & \ipa{sà.fi.nɛ} & \ipa{sà.fn\'ɛ} & `soap'\\
c. & \ipa{s\`ã.ku.ra} & \ipa{s\`ã.krá} & `New Year'\\
d. & \ipa{dù.lo.ki} & \ipa{dlò.kí} & `shirt'\\\addlinespace
e. & \ipa{mò.ri.ba} & \ipa{mòr.bá} & `man's name'\\
f. & \ipa{sá.nu.ma} & \ipa{sán.má} & `holy'\\\addlinespace
g. & \ipa{sá.ra.ma} & \ipa{sár.má/srá.má} & `famous'\\
h. & \ipa{sú.ru.ku} & \ipa{súr.kú/srú.kú} & `hyena'\\
i. & \ipa{kè.le.ku} & \ipa{kèl.kú/klè.kú} & `to stumble'\\\addlinespace
j. & \ipa{sí.rã} & \ipa{sr\'ã} & `to scar'\\
k. & \ipa{d\`ɔ.lɔ} & \ipa{dl\v{ɔ}} & `beer'\\
\lspbottomrule
\end{tabularx}
\end{table}

\newpage
The initial vowel in Bambara alternating CVCV forms is not predictable given the phonological context; it could be a high vowel or a vowel of identical quality to the following vowel. Similarly, the tone of the initial vowel in CVCV forms is not predictable given the CCV form. Thus, in Bambara, /CVCV/ forms can be analyzed as underlying, with the option of surfacing as [CCV]. Other Western Mande languages can be analyzed in the same way as Bambara, with underlying /CVCV/ words that can surface as [CCV] depending on the phonotactic result (Zialo, Jalkunan). Seenku is an interesting case because it seems to be in an intermediate stage between /CVCV/ and /CCV/, where formerly /CVCV/ forms underwent vowel reduction and reanalysis to /CəCV/, and are perhaps in route to being reanalyzed as /CCV/ when the second consonant is a sonorant \citep{McPherson:2020}. In other Western Mande languages, surface CVCV forms cannot surface as CCV. This suggests that some Western Mande languages have not yet begun a shift from CVCV to CCV. \citet{Vydrine:2004} points out that the Western Mande languages spoken nearby to Southeastern Mande languages, such as Marka/Dafin dialects in Côte d'Ivoire, are more advanced in their shift towards CCV than other Western Mande languages, suggesting an areal or contact effect.

\largerpage[-1]
Summarizing the patterns and existing analyses of CVCV\slash CCV alternations in Mande, Southeastern Mande languages display synchronic CVCV\slash CCV alternations where the initial vowel of the CVCV form is predictable. They are analyzed as having underlying /CCV/ forms, which are sometimes realized with a short initial vowel [CvCV]. In some Western Mande languages, some underlying /CVCV/ forms can optionally be pronounced [CCV], but there are no underlying /CCV/ forms. In other Western Mande languages there are no surface [CCV] forms at all. These facts suggest that Proto-Mande had CVCV forms, some of which have been reanalyzed as underlyingly /CCV/ in Southeastern Mande, consistent with the conclusions of \citet{Vydrine:2004}. This reanalysis has not yet occurred in Western Mande, but it seems to be moving in that direction, at least in languages like Seenku, Dioula, Ziallo, Jalkunan, and Bambara.

\subsection{Kwa}\label{sec:kwa}

Some Kwa languages, spoken in the southeast of Côte d'Ivoire just to the east of Kru languages, in southern Ghana, and in central Togo, also show sequences that alternate between CVCV and CCV. Other Kwa languages have surface CCV forms that do not alternate with CVCV. Still others show no surface CCV forms. In Table \ref{kwacvcv}, Kwa language names are listed along with their sub-group. Each language is marked as showing CVCV\slash CCV alternations or not. If there are CVCV\slash CCV alternations, Table \ref{kwacvcv} indicates whether the quality of the first vowel in the CVCV form is phonologically predictable. If there are surface CCV forms at all, the final column indicates which consonants can surface as the second consonant in an alternating CVCV sequence. The second consonant in surface CCV in Kwa is always underlyingly /l/, which can surface as [l], [r], or [n] in many Kwa languages, depending on the phonological environment. The sources of the generalizations in Table \ref{kwacvcv} come from \citet{Westermann:1930, Hyman:1972, Allan:1973, LeSaout:1974, Bergman:1981, BoleRichard:1983, Dolphyne:1988, Schang:1995, Lenaka:1999, Leben:2002, Leben:2003, Ahua:2004, Bobuafor:2004, Hager:2014, VanPutten:2014, Delalorm:2016, Asante:2017, Paster:2010, Agbetsoamedo:2014, Abunya:2018}.

\begin{table}
\caption{CVCV$\sim$CCV across Kwa}
\label{kwacvcv}
\begin{tabularx}{\textwidth}{Xccl}
	\lsptoprule
	Language & CVCV $\sim$ CCV? & V1 predictable? & Possible C2s\\
	\midrule
    Gã (Ga-Dangme) & Yes & Yes, identical & /l/\\
    Baoulé  (Potou-Tano) & Yes & Yes, identical & /l/ \\
	Agni (Potou-Tano) & Yes & Yes, identical & /l/\\
	Akan (Potou-Tano)  & Yes & Yes, identical & /l/\\
	Atchan (Potou-Tano) & (always CCV) & -- & /l/\\
	Abidji (Agnegby) & (always CCV) & -- & /l/\\
	Avatime (Ka-Togo) & (always CCV) & -- & /l/\\
	Tafi (Ka-Togo) & (always CCV) & -- & /l/\\
%	Ega (possible separate branch of NC) & (always CCV) & -- & /l/\\
	Leleme (Na-Togo) & (always CCV) & -- & /l/\\
	Sekpele (Na-Togo) & (always CCV) & -- & /l/\\
	Selee (Na-Togo) & -- & -- & --\\
	Nkami (Potou-Tano) & -- & -- & --\\
	Twi (Potou-Tano) & -- & -- & --\\
	\lspbottomrule
	\end{tabularx}
\end{table}


\largerpage[-1]
In Selee \citep{Agbetsoamedo:2014}, Nkami \citep{Asante:2017}, and Twi \citep{Paster:2010} there are stated to be no surface consonant clusters. In Twi, CGV, where G is a glide, is possible on the surface, but is analyzed as underlying CVV \citep{Paster:2010}, and does not alternate with CVCV. The other Kwa languages examined here all have surface CCV forms, where underlying /l/ can surface as the second consonant in a surface cluster \citep{LeSaout:1974}. When these CCV forms alternate with surface CVCV, the initial vowel is always identical to the second, and is typically very short \citep{Dolphyne:1988, Ahua:2004, Leben:2002}. For most Kwa languages in the sample above, there is no attested context where these CCV forms surface as [CVCV]. In Baoulé the alternating words surface as CVCV in slow, careful speech, and in language games, but not in natural, casual speech \citep{Leben:2002, Leben:2003}. Even when Baoulé alternating words are pronounced as CVCV in careful speech, the initial vowel is very short, or its duration variable, as expected of intrusive vowels as discussed in \sectref{sec:cvcvintro}. Similarly, in Gã, \citet[41--42]{Berry:1969} report that surface CLV syllables, which are quite common, are sometimes produced as CVCV in slow, careful speech, but not in more natural speech.


\largerpage[-1]
Across the board in Kwa languages that have surface CCV forms, it is much more common for these words to surface as CCV than as CVCV; whereas in some Kru languages, for example, it is reportedly equally as common for an alternating root to be pronounced as CVCV or CCV in casual speech. For Kwa languages whose CCV forms lack a CVCV alternant, the simplest analysis is that these forms are underlyingly /CCV/, and always surface as such. The question is then how to analyze the alternating forms in languages like Baoulé and Akan, which do show alternations. Due to their behavior in careful speech and language games, \citet{Leben:2002, Leben:2003} treats alternating CVCV\slash CCV words in Baoulé as underlyingly /CVCV/, where the first vowel is reduced or completely elided in most contexts. However, given the fact that the initial vowel is always predictable given the phonological context, and it is always very short unlike other vowels in the language, I propose that it is preferable to analyze CVCV\slash CCV forms in Baoulé as synchronically underlyingly /CCV/, with intrusive vowels as per \citet{Hall:2003, Hall2006}.\footnote{One argument Leben makes for /CVCV/ as underlying in Baoulé is that disyllabic verbs show LH tone, which surface CCV verbs also display. CV verbs, on the other hand, have H tone. Thus, in terms of tonal behavior CCV forms are more like CVCV forms than CV forms. This could be due to historical reasons, since CCV forms likely are derived from former CVCV forms, or perhaps the presence of additional segmental structure in CCV forms than CV forms allows them to host an additional tone.}

In Akan, the identical vowels in alternating CVCV forms need not carry the same tone. When the two vowels of the CVCV form bear distinct tones, the tone typically associated with the first vowel surfaces on C2 in the corresponding CCV form \citep{Dolphyne:1988}. For example, the form [{\ipa{ò-fìrí}]} `he buys credit' has a low tone on the first syllable of the verb and a high tone on the second. In its CCV form, the low tone surfaces on the /r/, [{\ipa{ò-fr̀í}}]. Consonants do not typically bear tone in Akan. Additionally, if we said that there is an underlying CCV /fri/ that has a LH tone melody, and that in the case of a contour tone on a short vowel the first tone level is realized on the consonant, we might expect that we would also see CV roots with a tone on the onset consonant. However, no such examples exist in Akan. The simplest explanation of why a tone might surface on the [r] in a CrV syllable, then, is that it is underlyingly a /CVCV/, subject to regular tone association rules followed by optional deletion of the initial vowel segment in casual speech. Additional evidence for underlying /CVCV/ in Akan comes from forms with an initial high +ATR vowel, which show consonant palatalization in their CCV forms: [pira] $\sim$ [pʲra] `injure' \citep{Dolphyne:1988}. There are otherwise no surface palatalized labials in Akan; so, positing that /pʲra/ is underlying would require proposing additional consonants in the Akan inventory that coincidentally only appear in alternating CVCV\slash CCV forms; whereas, if the CVCV form is underlying, no such requirement is necessary. Thus, based on tonal patterns and the distribution of palatalized labial consonants, it is simpler in Akan to analyze alternating CVCV\slash CCV forms as underlyingly /CVCV/ with optional deletion. Like in Kru and Mande languages, CVCV\slash CCV alternations in some Kwa languages are best analyzed as involving underlying /CCV/ plus epenthesis or vowel intrusion (Baoulé), while others are best analyzed as having /CVCV/ forms that undergo optional deletion (Akan).

\sloppy
Of the Kwa languages sampled here, only Gã and Potou-Tano languages, specifically Potou-Tano languages spoken in Côte d'Ivoire, show synchronic CVCV\slash CCV alternations. (Though not all Potou-Tano languages of Côte d'Ivoire show synchronic CVCV\slash CCV alternations; see Atchan.) The Potou-Tano languages that show synchronic alternations are all in the same sub-group within Potou-Tano, namely, Central Tano, while Atchan and Nkami fall outside the Central Tano group. Nkami and Twi are spoken in Ghana rather than Côte d'Ivoire, so it seems likely that country borders and language contact have played a role in the historical development and stabilization of CVCV\slash CCV alternations in Kwa. The Potou-Tano languages of Côte d'Ivoire are spoken nearby to Kru and Mande languages which show synchronic CVCV\slash CCV alternations, supporting an areal story. It seems likely that the CVCV\slash CCV alternations of non-Kwa languages in Côte d'Ivoire have helped to stabilize the synchronic CVCV\slash CCV alternations in Ivoirian Potou-Tano languages, whereas CVCV\slash CCV alternations further away, in Ghanaian Potou-Tano languages, have shifted fully to non-alternating CCV forms. Agnegby and Ka-Togo languages have synchronic CCV forms that do not alternate with CVCV. The Na-Togo languages are mixed between having no surface CCV forms at all on the one hand, and having non-alternating CCV forms on the other.
\fussy

\citet[192]{Hyman:1972} discusses how proto-CVCV forms have become CCV in many Kwa languages over time. He hypothesizes that the existence of nasal vowels in some Kwa languages is due to a further step in the diachronic picture: CVNV words (a subset of alternating CVCV forms) became CNV, which became C{\~V}. He provides evidence for the intermediate stage from Gwari and Ewe \citep[175--179]{Hyman:1972}. In Atchan, there are synchronic nasal vowels and no synchronic CNV syllables (Katherine Russell, p.c.), so it seems to have reached the endpoint of the diachronic chain proposed by Hyman. Hyman's hypothesis raises an interesting question for the Kwa languages that do not have surface CCV sequences at all: Have CVCV words in such languages undergone syncope to CCV followed by consonant fusion or cluster simplification to CV? Or, have CVCV words in such languages never undergone syncope? There seems to be synchronic evidence favoring the former hypothesis. In some Kwa languages, such as Ewe \citep{Westermann:1930}, there are both CV and CCV forms of the same word: [ku, klu] `to scoop', [gbaa, gblaa] `broad' (p. 20). This variation between CCV and CV suggests that Ewe, and other Kwa languages with no synchronic CVCV\slash CCV alternation, likely used to have a CVCV\slash CCV alternation, which then progressed to non-alternating CCV, and then to CCV alternating with CV. Now the proto-CVCV forms in such languages are simply CV: CVCV$\gg$CCV$\gg$CV. For each step in the diachronic process, there is evidence that languages go through a stage of variation. Languages like Akan show synchronic variation between CVCV and CCV, and can be analyzed as currently undergoing the CVCV$\gg$CCV stage, while languages like Ewe show synchronic variation between CCV and CV and can be analyzed as currently undergoing the CCV$\gg$CV stage.

\subsection{Areal summary}
CVCV\slash CCV alternations are undoubtedly an areal phenomenon of West Africa, present in at least Mande, Kwa, and Kru languages, especially in the area around Liberia, Côte d'Ivoire, and Burkina Faso. Within these three language families, languages spoken further from the central contact zone in Côte d'Ivoire are more likely to lack a synchronic CVCV\slash CCV alternation.

In each of the three language families examined here, there are both individual languages that are synchronically best analyzed as having underlying /CVCV/ forms that can optionally delete the first vowel, and there are other languages that are best analyzed as having underlying /CCV/ forms that optionally insert a vowel. In all three language families, as has long been recognized \citep{Hyman:1972, LeSaout:1974, Leben:2002, Leben:2003, Vydrine:2010}, there is evidence for proto-CVCV forms, which synchronically correspond to alternating CVCV\slash CCV forms. These proto-CVCV forms best analyzed as synchronically /CVCV/ with optional vowel deletion in some languages, but have been reanalyzed as underlyingly /CCV/ with optional vowel epenthesis or intrusion in others. Perhaps this areal shift can be explained by a shift towards monosyllabicity, as proposed by \citet{Green:2010} for Bambara (Mande). In some Kwa languages, the diachronic shift has progressed even further, with CCV forms undergoing consonant cluster reduction to become CV.

In previous historical work on CVCV\slash CCV alternations outside of West Africa, \citet{Harms:1976} and \citet{Hall2006} show that intrusive vowels have become phonologized over time in languages such as Irish Gaelic, Sardinian, and Finnish: CCV $\gg$ CvCV $\gg$ CVCV. The opposite, where phonologically visible vowels are reanalyzed as intrusive over time, seems to have occurred in Dida, but not the closely related Guébie. A similar shift has occurred in certain Mande and Kwa languages. Thus, while there are previous examples of intrusive vowels being phonologized, the Guébie and Dida data show that the opposite is also possible.

\citet{Blevins&Pawley} argue that a synchronic CVCV\slash CCV alternation in Kalam (Trans New Guinea) is the remnant of historical vowel reduction and deletion: (V)CVCV $\gg$ (V)CvCV $\gg$ (V)CCV. They summarize the process in this way: ``Regular vowel loss yields vowel-zero alternations, which can be reinterpreted as insertions via rule inversion.'' This cycle seems to be very active in West Africa, where different languages fall synchronically at different states of this diachronic shift. In general, the trend in West Africa seems to be towards monosyllabic words, possibly motivated by a drive towards minimality as suggested by \citet{Green:2010, Green:2018}.

\section{Conclusions}\label{sec:conclusion}
This paper contributes to the typology of V/$\emptyset$ alternations by examining two closely related West African tonal languages, and considering both the best synchronic analysis, as well as how they fit into the broader areal and historical picture of V/$\emptyset$ alternations in West Africa. 

I have shown that there is not a single, unified analysis of synchronic CVCV$\sim$ CCV alternations in Kru languages. In Dida, alternating roots are best analyzed as /CCV/ and subject to vowel intrusion (\sectref{sec:dida}). In Guébie, on the other hand, alternating roots are best analyzed as /CVCV/ (\sectref{sec:guebie}), where one class of /CVCV/ roots is  lexically specified as alternating (V1 can optionally be deleted), and another class of /CVCV/ forms is not subject to alternation. The alternation cannot go the other direction in Guébie because the initial vowel is not predictable given the phonological environment. These two patterns are shown to be representative of the two most common CVCV\slash CCV synchronic patterns found across Kru languages (\sectref{sec:kru}).

Throughout this paper, evidence for CVCV\slash CCV alternations as involving deletion or insertion has come from the diagnostics introduced in \sectref{sec:diagnostics}, summarized in (\ref{diagnostics}). These diagnostics may be useful in distinguishing insertion from deletion in future work on CVCV\slash CCV alternations in other languages. Note that these are distinct from Hall's diagnostics of whether vowel insertion involves phonologically visible versus invisible vowels (\sectref{sec:epenthesis}).

\eanoraggedright Diagnostics for insertion versus deletion in CVCV\slash CCV alternations\label{diagnostics}
\begin{description}
	\item[Predictability of V1's quality:] If the quality of V1 is predictable given the phonological environment, it can be derived through vowel insertion (cf. Dida). If the vowel quality is not predictable, it must be underlying (cf. Guébie).
	\item[Tonal behavior:] If the first vowel in an alternating CVCV form can host its own prosody, it is likely an underlying vowel (cf. Akan tonal behavior in CVCV\slash CCV alternations, discussed in \sectref{sec:kwa}). On the other hand, if non-regular tonal patterns arise in CVCV forms, it may be because they are derived from underlying CCV forms as in Dida (discussed in \sectref{sec:dida}).
	\item[Independence in phonological alternations:] If the first vowel in alternating CVCV forms participates independently from the second vowel in regular phonological processes in the language, it is likely an underlying vowel (cf. vowel replacement in Guébie, described in \sectref{sec:guebie}).
\end{description}
\z

We have seen that evidence for insertion versus deletion may come from phonological processes seemingly unrelated to vowel presence or absence, such as vowel replacement in Guébie or tonotactics in Dida and Akan. Thus, it is useful to understand the full phonological picture of a language when diagnosing whether a vowel/$\emptyset$ alternation is best analyzed as deletion versus insertion.

The Guébie data in \sectref{sec:guebie} is dependent on speech rate but was shown to involve vowel deletion rather than insertion. We know from \citet{Hall:2003, Hall2006} that speech rate may play a role in determining whether intrusive vowels are realized or not, but the Guébie facts show that speech rate can also play a role in whether vowels are deleted or not.

From a Kru-internal diachronic perspective, this study has shown that before Eastern and Western Kru split, there was likely a CVCV$\sim$CCV alternation best analyzed as deletion. This alternation was reinterpreted as involving underlying /CCV/ forms in some languages (cf. Dida, Nyabwa). Kuwaa and Déwoin, Kru isolates which show no CVCV\slash CCV alternation, likely split off from the rest of Kru before the CVCV$\sim$ CCV alternation arose. In order to confirm this historical hypothesis, additional evidence from other areas of the grammar should be considered.

The Dida case study shows that phonologically visible vowels can be reanalyzed as intrusive over time. This is the opposite of what has previously been shown for a number of non-African languages \citep{Harms:1976, Hall2006}, showing that there is a possible diachronic cycle from /CVCV/ being reanalyzed as /CCV/ and then possibly again being reanalyzed as /CVCV/.

Initial evidence suggests that in nearby Mande languages, proto-/CVCV/ has been reanalyzed as /CCV/ in Southeastern Mande, and is moving towards a reanalysis in many Western Mande languages. One possible explanation is an areal drive towards monosyllabicity (cf. \citealt{Green:2010, Green:2015, Green:2018}). In most Kwa languages, also spoken in West Africa, there seems to have been a reanalysis of proto-CVCV forms as CCV, some of which still allow optional epenthesis or intrusion resulting in surface CVCV forms. In some Kwa languages, this change has progressed further, resulting in CCV forms reducing to CV.


% \begin{frame}{Phonotactic traits of alternating roots}
% The more of the relevant features a given root shows, the more likely it is to be in the alternating class.

% \begin{exe}
% \ex \textbf{Factors influencing alternation} \citep{Sande:diss}\label{reducibilitychart}
% \end{exe}
% \vspace{-3ex}
% \begin{flushleft}
% \begin{tiny}
% \begin{tabular}{|l||l||l|l|l||l|l|l||l|}
% \hline
% &  \textbf{None} & \textbf{T1=T2} & \textbf{C2=l} & \textbf{V1=V2} & \textbf{T\&C2} &\textbf{T\&V} & \textbf{C2\&V} &  \textbf{All} \\
% \hline
% \hline
% \textbf{Alternating}  & 157 & 269 & 287 & 328 & 145 & 208 & 199 &  127 \\
% \hline
% \textbf{Total} & 751 & 614 & 536 & 611 &  244 & 339 & 244 & 154 \\
% \hline
% \hline
% \textbf{Percent} & 20.9 & 43.8 & 53.5 & 53.7 & 59.4 & 61.4 & 81.6 &  82.5 \\
% %\hline
% %\textbf{O/E} & 1.31 & 1.60 & 1.60 & 1.77 & 1.83 & 2.44 & 2.46 & .624 \\
% \hline
% \end{tabular}
% \end{tiny}
% \end{flushleft}
% Though no combination of phonotactic traits exclusively and exhaustively predicts whether a root falls into the alternating or non-alternating class.
% \end{frame}

% \subsubsection*{Modeling the distribution of alternating roots in the lexicon}
% \begin{frame}{A MaxEnt-HG model}
% 	A MaxEnt-HG \cite{Goldwater&Johnson:2003} analysis confirms that these three properties are the most relevant in determining whether a given root falls into the alternating class.
% 	\begin{itemize}
% 		\item MaxEnt-HG is a weighted-constraint model that produces a probability distribution over output candidates.
% 		\item It can correctly predict the proportion of each type of root that falls into the alternating class.
% 	\end{itemize}

% \end{frame}

% \begin{frame}{Constraints}

% \begin{exe}
% \ex \textbf{Reduce(T1=T2)}\\
% Assign one violation if the tone on two consecutive syllables is identical (reduce if T1=T2).	
% \ex \textbf{Reduce(C2=l)}\\
% Assign one violation if a vowel intervenes between [l] and a preceding consonant (reduce if C2=l).	
% \ex \textbf{Reduce(V1=V2)}\\
% Assign one violation if vowels in two consecutive syllables are identical (reduce if V1=V2).	
% \ex \textbf{Max}\\
% Assign one violation for every input segment that lacks a corresponding output segment.
% \end{exe}
% \begin{itemize}
% \item A candidate violates one of the \textsc{Reduce} constraints if it has not deleted V1, and shows the specified surface property, T1=T2, C2=l, or V1=V2.
% 	\item  All CCV candidates violate \textsc{Max}, because the initial input vowel fails to surface.
% 	\item  	Other constraints such as \textsc{Max} specific to certain vowel qualities and constraints penalizing different height, backness, and rounding values of vowels within the same word were considered but did not improve the model.
% \end{itemize}
% \end{frame}

% \begin{frame}{Constraint weights}
% \begin{itemize}
% 	\item Constraint weights were determined by the MaxEnt Grammar Tool \citep{Hayesetal:2009}, providing a model of the distribution of alternation in the Guébie lexicon.
% 	\item In this model, candidates are groups of roots that share phonotactic properties.
% 	\end{itemize}
% \end{frame}


% \begin{frame}{A MaxEnt-HG model}
% 	\begin{exe}
% \ex \textbf{MaxEnt HG weights: Vowel deletion} \label{hgtabdist}
% \end{exe}
% \begin{tiny}
% \hspace{1mm}\begin{tabular}{|ll||c|c|c|c||c||c|c|}
% \hline
% & & \textsc{R(T)} & \textsc{R(C2)} & \textsc{R(V)} & \textsc{Max} & & &\\
% \hline
% & & .662 & 1.02 & 1.23 & 1.15 & \textbf{H} & \textbf{Obs (\%)} & \textbf{Pred (\%)}\\
% \hline
% \hline
% T1=T2 & & & & & & & &\\
% & CVCV & 1 & & &  & .662 & 57.2& 61.9\\
% \hdashline
% & CVCV\slash CCV & & & & 1 & 1.15 & 43.8 & 38.1\\
% \hline
% C2=l & & & & & & & &\\
% & CVCV & & 1 & & & 1.02 & 46.5 & 53.1\\
% \hdashline
% & CVCV\slash CCV & & & & 1 & 1.15 & 53.5 & 46.9\\
% \hline
% V1=V2 & & & & & & & &\\
% & CVCV & & & 1 & & 1.23 & 46.3 & 48.0\\
% \hdashline
% & CVCV\slash CCV & & & & 1 & 1.15 & 53.7 & 52.0\\
% \hline
% T, C2 & & & & & & & &\\
% & CVCV & 1 & 1 & & & 1.682 & 40.6 & 36.9 \\
% \hdashline
% & CVCV\slash CCV & & & & 1 & 1.15 & 59.4 & 63.1\\
% \hline
% T, V  & & & & & & & &\\
% & CVCV & 1 & & 1 & & 1.892 & 38.6 & 32.3\\
% \hdashline
% & CVCV\slash CCV & & & & 1 & 1.15 & 61.4 & 67.7 \\
% \hline
% C2, V & & & & & & & &\\
% & CVCV & & 1 & 1 & & 2.25 & 18.4 & 24.9\\
% \hdashline
% & CVCV\slash CCV & & & & 1 & 1.15 & 81.6 & 75.1\\
% \hline
% T, C2, V & & & & & & & &\\
% & CVCV & 1 & 1 & 1 & & 2.912 & 17.5 & 14.6\\
% \hdashline
% & CVCV\slash CCV & & & & 1 & 1.15 & 82.5 & 85.4\\
% \hline
% None & & & & & & & &\\
% & CVCV & & & & & 0 & 79.1 & 75.9 \\
% \hdashline
% & CVCV\slash CCV & & & & 1 & 1.15 & 20.9 & 24.1\\
% \hline
% \end{tabular}\\
% \end{tiny}
% \end{frame}

% \begin{frame}{A MaxEnt-HG model}
% 	The fact that the predicted amount of reduction for each type of root in the MaxEnt analysis in (\ref{hgtabdist}) so closely mirrors the observed pattern supports the analysis of the proposed parameters (T1=T2, C2=l, V1=V2) as those most relevant in determining whether a given root alternates. 
% \begin{itemize}
% \pause \item Note that this model \textit{cannot} be used to make predictions about how a particular root will surface.
% \begin{itemize}
% \item 38.1\% of /CVCV/ roots with the same tone on both syllables are predicted to always optionally be able to surface as CCV. 
% \item Nothing about this model predicts which T1=T2 roots will alternate and which will not, nor the frequency with which a specific alternating root will surface as CCV.
% \end{itemize}
% \item The existence of minimal pairs and the inability to predict the V1 in an alternating CVCV$\sim$CCV form from V2 means that some lexical specification is needed to differentiate alternating from non-alternating roots.
% \end{itemize}
% \end{frame}
% \begin{frame}{Psycholinguistic evidence}
% Another type of evidence to consider is whether speakers extend the option of CVCV$\sim$CCV or vowel replacement alternations productively to new words.
% \begin{itemize}
% 	\item If so, do they do so at the frequency predicted by the MaxEnt model?
% \end{itemize}
% \noindent \textbf{Preview:} The current results are inconclusive, but subtly point towards arbitrary lexical specificity over phonological determinedness.
% \end{frame}


% \begin{frame}{Psycholinguistic evidence}
% 	I ran a psycholinguistic experiment in the Guébie community in Summer 2019 to test whether speakers use phonotactic information about nonce words to determine whether they alternate or not.
% 	\begin{itemize}
% 		\item Worked with 22 speakers, each introduced to 44 nonce words and asked to produce them in a vowel replacement context.
% 		\item 9 of the participants' data were usable.
% 		\item Of the 9, 4 never extended alternation to nonce words, suggesting that alternation is lexically specified and learned separately for each given morpheme.
% 		\item 3 produced the alternation in 1/44 words, 1 in 2/44 words, and 1 in 4/44 words.
% 	\end{itemize}
% \end{frame}

% \begin{frame}{Psycholinguistic evidence, cont.}
% 	Of the 9 total words across all 9 participants that showed an alternation (2\% of the data), 8 had the same vowel in both syllables, and in 7 the second consonant was /l/.
% 	\begin{itemize}
% 		\item All test words had the same level tone melody, so we cannot determine whether tone had an effect on speaker behavior.
% 		\item Two of the words were produced as alternating by more than one speaker: \ipa{jOlO} (2 speakers), \ipa{wElE} (3 speakers).
% 	\end{itemize}
% 	There is too little data on any given word type to run stats, but it seems that some speakers extend alternation at low rates to nonce words, suggesting a small amount of phonological determinedness, while other speakers do not extend the alternation, suggesting lexical determinedness.
% 	\begin{itemize}
% 		\item When travel is again possible, I hope to run additional experiments in the Guébie community.
% 	\end{itemize}
% \end{frame}


\section*{Abbreviations}
\begin{tabularx}{.45\textwidth}{@{}lQ@{}}
C & Consonant\\
V & Vowel\\
L & Liquid\\
\end{tabularx}
\begin{tabularx}{.45\textwidth}{@{}lQ@{}}
N & Nasal\\
G & Glide\\
T & Tone\\
\end{tabularx}

\section*{Acknowledgments}
Thanks to Katherine Russell for being an excellent research assistant and compiling comparative Kru data relevant to this project. Thanks to Nancy Hall, Will Leben, and Lynell Zogbo for helpful discussion of related topics. Thanks also to audiences at the LSA 2020 Annual Meeting, WOCAL 9, and the Leipzig Strength in Grammar workshop for comments on related work, and to the audience of the Stonybrook Epenthesis Workshop in September 2021 for their comments on this work. Thanks especially to the Guébie community of Gnagbodgounoa, Côte d'Ivoire for their time and willingness to share the Guébie data presented in this paper. This work was funded by NSF grant \#2236768.


{\sloppy\printbibliography[heading=subbibliography,notkeyword=this]}
\end{document}
