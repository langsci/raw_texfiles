\documentclass[output=paper,colorlinks,citecolor=brown]{langscibook}
\ChapterDOI{10.5281/zenodo.14264534}

\author{Hassan Bokhari}\affiliation{Indiana University} 

\title{The patterning of epenthesis in Urban Hijazi Arabic} 
\abstract{This paper accounts for the different types and motivations of epenthesis in Urban Hijazi Arabic, such as syllable structure-driven epenthesis (SylE) and sonority-driven epenthesis (SonE). It also analyzes the default quality of the epenthetic vowel in SonE and SylE related to the type of the prosodic unit, syllable, or foot, specifically whether the epenthetic vowel is in the head syllable of the foot or the weak syllable. Beyond that, it analyzes the correlation between the quality of the nondefault SonE epenthetic vowel in final rising sonority coda clusters and the place features of the stem vowel or coda consonants. }

\IfFileExists{../localcommands.tex}{
   \addbibresource{../localbibliography.bib}
   % add all extra packages you need to load to this file

\usepackage{tabularx,multicol}
\usepackage{url}
\urlstyle{same}

\usepackage{listings}
\lstset{basicstyle=\ttfamily,tabsize=2,breaklines=true}

\usepackage{langsci-basic}
\usepackage{langsci-optional}
\usepackage{langsci-lgr}
\usepackage{langsci-osl}
% \usepackage{./langsci/styles/langsci-lgr}
% \usepackage{./langsci/styles/langsci-osl}
% \usepackage{langsci-gb4e}

\usepackage{tikz}
\usetikzlibrary{patterns,calc}
\pgfdeclarepatternformonly{south east lines}{\pgfqpoint{-0pt}{-0pt}}{\pgfqpoint{3pt}{3pt}}{\pgfqpoint{3pt}{3pt}}{
    \pgfsetlinewidth{0.6pt}
    \pgfpathmoveto{\pgfqpoint{0pt}{3pt}}
    \pgfpathlineto{\pgfqpoint{3pt}{0pt}}
    \pgfpathmoveto{\pgfqpoint{.2pt}{-.2pt}}
    \pgfpathlineto{\pgfqpoint{-.2pt}{.2pt}}
    \pgfpathmoveto{\pgfqpoint{3.2pt}{2.8pt}}
    \pgfpathlineto{\pgfqpoint{2.8pt}{3.2pt}}
    \pgfusepath{stroke}}
    
\usepackage{stmaryrd}
\usepackage{wasysym}
\usepackage{multirow}
\usepackage{caption}
\usepackage{subcaption}
\usepackage{mathrsfs}
\usepackage{qtree}

\usepackage{linguex}


   %pminos do not split footnotes
% \interfootnotelinepenalty=10000 %Footnote in Laporte chapters has to be split SN


%\DeclareIndexNameFormat{default}{%
%\nameparts{#1}%
%\usebibmacro{index:name}%
%{\index[names]}%
%{\namepartfamily}%
%{\namepartgiveni}%
% {}% L1
% {}% L2
%{\namepartprefix}% generates spurious space L3
%{\namepartsuffix}% generates spurious space L4
%}

%  {\DeclareIndexNameFormat{default}{%
%     \usebibmacro{index:name}{\index[names]}{#1}{#3}{#5}{#7}}}

%\DeclareIndexNameFormat{default}{%
%  \usebibmacro{index:name}{\sindex[nom]}{#1}{#3}{#5}{#7}}

%\DeclareIndexNameFormat{default}{%
%  \usebibmacro{index:name}{\sindex[person]}{#1}{#3}{#5}{#7}}
%\DeclareIndexNameFormat{default}{%
%\nameparts{#1} \usebibmacro{index:name}{\sindex[person]]}{\namepartfamily}{‌​\namepartgiven}{\nam‌​epartprefix}{\namepa‌​rtsuffix}}

%\newcommand{\smiley}{:)}

%\renewbibmacro*{index:name}[5]{%
%\usebibmacro{index:entry}{#1}%
%{\iffieldundef{usera}{}{\thefield{usera}\actualoperator}\mkbibindexname{#2}{#3}{#4}{#5}}}

% \newcommand{\noop}[1]{}

%remove for final
%\overfullrule=1mm

\newcommand{\tobi}[2]}}
\renewcommand{\S}[1]{\tobi{#1}{\textsc{*}}}

% this volume references
% puts: [this volume]
% already defined: \citetv
%\newcommand{\citepv}[1]{(\citeauthor{#1} \citeyear*{#1} [this volume])}
\newcommand{\citealtv}[1]{\citeauthor{#1} \citeyear*{#1} [this volume]}

%parentheses around example number
\newcommand{\pref}[1]{(\ref{#1})}

% in-text examples

\newcommand{\lnex}[1]{\textit{#1}} %target lang word
\newcommand{\lnlit}[1]{(lit.: `#1')} %literal reading
\newcommand{\lnlat}[1]{(#1)} % latinization
\newcommand{\lntrans}[1]{`#1'} %translation
\newcommand{\lnexl}[2]%
{\lnex{#1}{} \lnlat{#2}} % ex with latinization
\newcommand{\lnexlat}[3]{\lnex{#1}{} \lnlat{#2}{} \lntrans{#3}} % ex with latinization and tranl.

%ch01
\newcommand{\co}[1]{\mbox{\textbf{#1}}}

%ch09

\newcommand{\cyrbulg}[1]{\begin{otherlanguage*}{bulgarian}#1\end{otherlanguage*}}


%ch10
\newcommand{\nlp}{{\small NLP}}
\newcommand{\mwe}{{\small MWE}}
\newcommand{\rae}{{\small RAE}}
\newcommand{\lvc}{{\small LVC}}
\newcommand{\pos}{{\small P}o{\small S}}
%\newcommand{\todo}[1]{ \textcolor{red}{#1} }

%\renewcommand{\labelenumi}{\theenumi}
%\ainamefmt{{vv}{ll}{, ff}{, jj}} % fullname

\newcommand{\biberror}[1]{{\color{red}#1}}

\newcommand{\osenovaitem}{--~}
   %% hyphenation points for line breaks
%% Normally, automatic hyphenation in LaTeX is very good
%% If a word is mis-hyphenated, add it to this file
%%
%% add information to TeX file before \begin{document} with:
%% %% hyphenation points for line breaks
%% Normally, automatic hyphenation in LaTeX is very good
%% If a word is mis-hyphenated, add it to this file
%%
%% add information to TeX file before \begin{document} with:
%% %% hyphenation points for line breaks
%% Normally, automatic hyphenation in LaTeX is very good
%% If a word is mis-hyphenated, add it to this file
%%
%% add information to TeX file before \begin{document} with:
%% \include{localhyphenation}
\hyphenation{
    Beck-man
    Ngu-yen
    back-chan-nel
    back-chan-nels
    mo-not-o-nous
    ste-reo-typ-i-cal
}

\hyphenation{
    Beck-man
    Ngu-yen
    back-chan-nel
    back-chan-nels
    mo-not-o-nous
    ste-reo-typ-i-cal
}

\hyphenation{
    Beck-man
    Ngu-yen
    back-chan-nel
    back-chan-nels
    mo-not-o-nous
    ste-reo-typ-i-cal
}

   \boolfalse{bookcompile}
   \togglepaper[4]%%chapternumber
}{}

\begin{document}
\maketitle \label{ch4}

\section{Introduction} 
Urban Hijazi Arabic (UHA), also known as Makkan Arabic, is a prominent dialect spoken in the western region of Saudi Arabia, including the cities of Makkah, Jeddah, Madinah, to a lesser extent, Taif, and in some cities along the northwestern coast of Saudi Arabia.

UHA recognizes two major patterns of vowel insertion, both of which are incorporated into the syllabic structure of the word, making them epenthetic, as opposed to intrusive (\citetv{chapters/08.Hall}, \citetv{chapters/07.Bellik}). These two types of epenthesis appear to be unique to this particular dialect since most other Arabic dialects typically insert the default vowel [i] (\cite[102]{Farwaneh2017}). These two patterns are syllable structure-driven epenthesis (SylE) where the default epenthetic vowel is [a], and sonority-driven epenthesis (SonE) where the default epenthetic vowel is [i]. In both SylE and default SonE there is a relationship between the type of prosodic unit where epenthesis occurs and the quality of the default epenthetic vowel which will be discussed in more depth in Sections~\ref{both} and \ref{sone}. Furthermore, the study analyzes the relationship between the stem vowel or one of the last two consonants of a CVCC word-final syllable, which dictates the quality of the nondefault SonE vowel in word-final rising sonority coda clusters in terms of the features CORONAL, DORSAL, and PHARYNGEAL.

Within the typology of syllable structure of Arabic dialects (\cite{Kiparsky2003, Watson2007, Broselow2018}), UHA is considered one of the CV dialects (or ‘onset dialects’ in Broselow’s terminology) in which epenthesis occurs to the right of the unsyllabified consonant. For example, underlying /katab{}-t-l-u/ becomes [ka.tab.ta.lu] ‘I wrote to him’, with an epenthetic vowel to the right of /t/, forming the onset of the vowel.\footnote{{The final consonant of the CVCC syllable is extrametrical and does not bear any moraic weight. For example, underlying /katab{}-t/ surfaces as [ka}{\textsuperscript{μ}}{.ta}{\textsuperscript{μ}}{b}{\textsuperscript{μ}}{t] with an extrametrical [t]. Therefore, UHA does not allow mora sharing.} }

While \citet{Abu-Mansour1987} and \citet{Kabrah2004} have also studied epenthesis in the dialect, this study provides a novel, detailed analysis that differentiates two major types of epenthesis in UHA. The study also accounts for several systematic cases of consonant-vowel harmony in which the epenthetic vowel is not the default vowel but harmonizes with the [\textsc{place}] feature of the consonant of a potential CVCC. The source of data analyzed in this paper is from \citet{Bokhari2020}.

This paper is organized as follows. Section \ref{both} analyzes SylE and default SonE. Section \ref{sone} analyzes several types of non-default SonE, and Section \ref{disc} discusses some of the findings and concludes the paper.

 \section{SylE and default SonE}\label{both}
 \citet{DeLacy2006} provides several universal sonority-based constraint hierarchies and relates them to different positions of prosodic constituents. These sonority-based constraint hierarchies provide the right tool to set up the analyses of both types of  default epenthesis. He argues that variation in the quality of the epenthetic vowel across languages can be analyzed as the result of competing constraints, imposed by different Designated Terminal Elements (DTEs). DTEs refer to the head of a given prosodic unit, such as a mora, syllable, or foot. He differentiates between the head and non-head positions of these constituents, which are the DTEs (also abbreviated as Δ) and non-Designated Terminal Elements (non-DTEs, abbreviated as –Δ), respectively. According to \citeauthor{DeLacy2006}, universally, low vowels, which are higher in sonority than other vowels, tend to be favored as epenthetic vowels in DTE positions, i.e., the head position of the prosodic constituent, whereas high peripheral vowels [i, u], tend to be epenthesized in non-DTE positions~-- that is, in   unstressed syllables, moras, or feet.


According to \citet[305]{DeLacy2006}, vowel epenthesis in Arabic dialects cannot be accounted for in terms of DTE constraints alone. In UHA, however, the DTE of the prosodic constituents appears to be in some way connected with the sonority of the epenthetic segments, in which the low sonority vowel [i] is epenthesized  in the non-head position of the foot in sonority-driven epenthesis, whereas the high-sonority [a] is epenthesized in the DTE position of the syllable in syllable-driven epenthesis. Since UHA has two major types of epenthesis, SylE epenthesis and default SonE epenthesis, it is worth elaborating and differentiating between the two types of epenthesis. I start by elaborating on SylE epenthesis then explain SonE epenthesis and summarize some issues related to it.



\subsection{Syllable Structure-Driven Epenthesis (SylE)} \label{syl}

Regarding the syllable-driven epenthetic vowel [a], this vowel, which is higher in sonority than any other vowel in UHA, is inserted as a way of strengthening a weak degenerate syllable, which consists of only a single consonant as a syllable onset. This also can be accounted for with the constraints proposed by \citet[68]{DeLacy2006}, in which this vowel ([a]) represents the DTE of the syllable. This epenthetic vowel could fall in DTE or non-DTE position of the foot.\footnote{See the tableau in \REF{tab:pha}.} In other words, the DTE of the syllable takes priority, in which the high sonority vowel fills the nucleus of the degenerate syllable. The trigger of this epenthetic vowel is the syllabic structure, in which an onset with an empty nucleus is not allowed  in the dialect. Consider /ka.tabt.lu/ ‘I wrote for him’, which becomes [ka.tab.ta.lu], not *[ka.tab.t.lu], and /baːb.na/ ‘our door’, which becomes [ba:.ba.na], not *[ba:.b. na]. The vowel [a] is inserted after the [t] in the first example, which forces resyllabification of the [t] into the onset of the new syllable. The same is true in the second example, when [a] forms a new syllable with the preceding [b] in the process of resyllabification. Thus, the constraint ranking for this type of epenthesis is *Δσ${\leq}$\{i, u\}  $\gg$  *Δσ${\leq}$a, where *Δσ${\leq}$\{i, u\} is defined as the Designated Terminal Elements of a syllable may not be less sonorant than or equal to [i] or [u], and where *Δσ${\leq}$a is defined as the Designated Terminal Elements of a syllable may not be less sonorant than or equal to [a].

A potential OT analysis for the SylE epenthesis would rank *Δσ${\leq}$\{i, u\}, \textsc{Nucleus}, which requires each syllable to have a vowel nucleus, and *\textsc{ComplexOn\-set}, which bans onset clusters, above *Δσ${\leq}$a. This analysis is beyond the scope of this paper and is left for future research.

\citet{DeLacy2006} relates the sonority of the vowel to the head of the syllabic constituent; however, his proposal has a wider scope, in which he considers several universal sonority hierarchies of segments and how they interact with the head or non-head position of different constituents.



\subsection{Sonority-driven default epenthesis (SonE)} 
In UHA, the sonority-driven default epenthetic vowel [i] is inserted to break up a potential word-final CC cluster of rising sonority when the quality of the vowel  is not determined by high vowel spreading, a pharyngeal/laryngeal, coronal,  or dorsal segment. Consider the data in \tabref{table:defson}, in which [i] is inserted to break up the rising-sonority cluster.


\begin{table}
\caption{Default Sonority-Driven Epenthesis}
\label{table:defson}
\begin{tabular}{lllll}
\lsptoprule
Underlying & Gloss & Surface & Possessive.3\textsc{sg}.\textsc{masc} \\ \midrule
a. /lakm/   &   ‘punching’          &	[la.kim]    & [lak.mu]  \\
b. /tˤagm/  &	‘set (of things)’   &	[tˤa.gim]   & [tˤag.mu] \\
c. /ʃamʕ/   &	‘wax’               &	[ʃa.miʕ]	& [ʃam.ʕu]   \\
d. /fagʕ/	&   ‘type of mushroom’  &	[fa.giʕ]    & [fag.ʕu] \\
e. /ʃatm/	&   ‘cursing’	        &   [ʃa.tim]    & [ʃat.mu] \\
f. /nad͡ʒm/	&   ‘star’	            &   [na.d͡ʒim]   & [nad͡ʒ.mu] \\
g. /waʃm/	&   ‘tattoo’	        &    [wa.ʃim]    &	[waʃ.mu] \\
\lspbottomrule
\end{tabular}
\end{table}


As can be noted from the data, epenthetic [i] is not determined by the nature of the surrounding consonants. Regarding (1e–f), the [t] and [d\textsuperscript{͡}ʒ] are not the trigger of [i]-epenthesis, even though they agree with the vowel [i] in the feature [coronal].\footnote{{The coronality of [i] is still a topic of discussion. See \citet{Clements1991} and \citet{Hume1992} for arguments that support the coronality of [i].} } This is because the epenthetic vowel is required to harmonize in coronality with a following consonant if it is coronal as will be analyzed in section \ref{sone}, and not with the preceding consonant. Therefore, [i] is epenthesized when there is no harmony requirement between it and the segments in the coda cluster. In this dialect, the only coda cluster permitted by  the syllabic structure is the coda in a CVCC final syllable that does not have rising sonority. This monosyllabic word form also consists of a trochaic foot, in which the stem vowel is the head of the foot, even if the coda cluster of this syllable violates the sonority requirement and receives an epenthetic vowel. This epenthetic vowel is never stressed, and it falls in the non-head position of the foot~-- that is, the unstressed part. Therefore, adopting \citeauthor{DeLacy2006}'s  DTE model mentioned above, I can determine the type of default epenthetic vowel in this position, following de Lacy’s constraint ranking in the non-DTE  position of the foot, by having *–ΔFt${\geq}$a outrank *–ΔFt${\geq}$\{i,u\}, where *–ΔFt${\geq}$a is defined as the head of the Non-designated Terminal Element may not be less than or equal in sonority to the low vowel [a], and where  *–ΔFt${\geq}$\{i,u\} means that the head of the non-Designated Terminal Element may not be less than or equal in sonority to the high vowels [i] and [u]. This leaves us with the two high vowels [i, u] as potential candidates for a default vowel. On the basis of the Place of Articulation hierarchy, [u] is universally more marked than [i] \citep{Lombardi1995}. Consider Example \REF{univ}, which presents the universal Place of Articulation hierarchy.


\begin{exe} 
\ex \label{univ} Universal Place of Articulation Hierarchy (\cite{DeLacy2006})\\\relax
                 *[DORSAL]  $\gg$  *[LABIAL]  $\gg$  *[CORONAL]  $\gg$  [PHARYNGEAL]
\end{exe}

Therefore, I can say that *[\textsc{Dors}] outranks *[\textsc{Cor}]. By these two different rankings by \citeauthor{DeLacy2006} and \citeauthor{Lombardi1995}, I reach the conclusion that the most appropriate default epenthetic vowel is [i] in sonority-driven epenthesis in UHA. Before providing the OT analysis of default SonE, it is worth elaborating on the theoretical framework used in the OT analysis of the word-final coda clusters. 

This study employs the Split-Margin Theory to analyze coda clusters in UHA \citep{Baertsch2002, BaertschDavis2009}. This approach to the syllable provides a framework for analyzing the behavior of coda clusters in terms of sonority.  Under this theory, the onset and coda positions in the syllable (i.e., the syllable margins) are each optionally split into two positions, M\textsubscript{1} and M\textsubscript{2}, where M\textsubscript{2} is the position closest to the nucleus of the syllable in each margin, and M\textsubscript{1} is the position farthest from the nucleus.  The Sonority Sequencing Principle (SSP) states that sonority is highest at the nucleus and lowest at the edges of a syllable, making M\textsubscript{1} a low-sonority position and M\textsubscript{2} a high-sonority position.  Because the SSP also states that cross-linguistically, syllables prefer to begin with low-sonority segments and end with high-sonority segments, a singleton onset is M\textsubscript{1}, while a singleton coda is M\textsubscript{2}.

The M\textsubscript{1} position gives preference to low-sonority segments. When a *M\textsubscript{1} constraint is aligned with the sonority hierarchy, constraints avoiding high sonority will be universally highly ranked, and constraints avoiding low sonority will be universally lowly ranked. In UHA, *M\textsubscript{1} and *M\textsubscript{2} OT constraint hierarchies are inherently ranked following the UHA sonority scale. The M\textsubscript{1} position gives preference to low-sonority segments, while the M\textsubscript{2} position gives preference to high-sonority segments.

\begin{exe} 
		\ex \label{m1} *M\textsubscript{1}/Vowel $\gg$ *M\textsubscript{1}/Glide $\gg$ *M\textsubscript{1}/ʕ $\gg$ *M\textsubscript{1}/Liquid $\gg$ *M\textsubscript{1}/Nasal $\gg$ *M\textsubscript{1}/VcdFri $\gg$ *M\textsubscript{1}/Obs\\
  *M\textsubscript{2}/Obs $\gg$ *M\textsubscript{2}/VcdFri $\gg$ *M\textsubscript{2}/Nasal $\gg$ *M\textsubscript{2}/Liquid $\gg$ *M\textsubscript{2}/ʕ $\gg$    *M\textsubscript{2}/Glide $\gg$ *M\textsubscript{2}/ Vowel
\end{exe}

Now, let us consider the tableau below in \REF{defsonep}. 

\begin{table}
\caption{Default SonE [i]-Epenthesis}
\label{defsonep}
\begin{center}
\ShadingOn
\begin{tableau}{c:c|s:s|s} 
\inp{/lakm/ ‘punching’}      \const*{*–ΔFt≥a}  \const*{*[\textsc{Dors}]}  \const*{*–ΔFt≥{i,u}} \const*{*[\textsc{Cor}]} \const*{*[\textsc{Phar}]}
\cand[\Optimal]{[la.kim]}   \vio{}     \vio{}  \vio{*} \vio{*} \vio{}
\cand{[la.kam]}             \vio{*!}   \vio{}  \vio{*} \vio{}  \vio{*}
\cand{[la.kum]}             \vio{}    \vio{*!} \vio{*}  \vio{}  \vio{}
\end{tableau}
\end{center}
\end{table}

\noindent In this tableau \REF{defsonep}, candidate (a) wins because it respects both *–ΔFt${\geq}$a and *[\textsc{Dors}] by epenthesizing a coronal vowel, even though it violates low-ranked *–ΔFt${\geq}$\{i,u\} and *[\textsc{Cor}]. Candidate (b) loses because   the epenthetic vowel [a] violates *–ΔFt${\geq}$a; in addition, it violates *–ΔFt${\geq}$\{i,u\} and *[\textsc{Phar}], because [a] as a pharyngeal vowel is greater in sonority than [i] and [u]. Candidate (c) loses because it epenthesizes a dorsal vowel, violating the high-ranked *[\textsc{Dors}]; in addition, it violates *–ΔFt${\geq}$\{i,u\}. This tableau shows that whenever there is no harmony requirement between vowels or consonants and vowels, default [i] is epenthesized to break up a potential rising-sonority coda cluster.

In contrast, the motivation for SylE is the tendency of the dialect to avoid word-internal superheavy syllables by epenthesis and    resyllabification, in which the last stray consonant of the word-internal CVVC/CVCC is resyllabified to form the onset of the default [a] epenthetic vowel. The outcome of this epenthesis   is the new syllable, which is formed by the unsyllabified last consonant of the superheavy syllable and the epenthetic vowel, and this syllable is preceded by a heavy syllable (e.g., /baːb{}- na/ → [(ˈbaː).(ba̠.na)] ‘our door;’ /katab{}-t-l-u/ → [ka.(ˈtab).(ta̠.lu)] ‘I wrote for him’). In both types of epenthesis, stress location is preserved, even after epenthesis. Thus, I can conclude from the discussion above that the SylE epenthetic [a] is higher in sonority than the default SonE epenthetic [i], because [a], as the highest  sonority vowel, forms the nucleus of the syllable with the stray consonant, i.e. the DTE of the syllable that is not the weak part of a foot, whereas the lowest sonority high front vowel [i] is epenthesized in the default SonE in the non-DTE position of the foot. Note that word-final foot extrametricality prevents penultimate stress in these words.

\section{Non-default Sonority-Driven Epenthesis (non-default SonE)}\label{sone}
After differentiating between two major types of epenthesis in UHA, this section analyzes non-default SonE, which operates whenever the coda cluster would exhibit a rising sonority profile. The quality of both stem vowels and the consonant in the coda cluster play a role in determining the quality of the epenthetic vowel, which breaks up a potential rising coda cluster. An underlying high vowel spreads its features to the epenthetic vowel on the surface in the high vowel spreading operation. [i] is epenthesized agreeing with final coronal consonants, except pharyngealized [rˤ]. [a], which is a pharyngeal vowel, is epenthesized agreeing in the feature [pharyngeal] with a preceding pharyngeal/laryngeal consonant and a following pharyngeal rhotic. In words with no medial laryngeal/pharyngeal consonant, [u] is epenthesized agreeing with the final pharyngealized rhotic [rˤ] for old generation speakers of the dialect. Several OT constraints are used in the analysis of non-default SonE, that includes constraints related to high vowel spreading, and consonant-to-vowel harmony. When both \textsc{Dep,} which militates against epenthesis, and \textsc{Contig,} which militates against separating two adjacent underlying segments in the surface form, are dominated by any of the Split-Margin constraints, the non-default SonE constraint ranking operates.

\subsection{High Vowel Spreading}
In UHA, when the stem contains a high vowel [i] or [u], the epenthetic vowel, which breaks up the potential rising-sonority word-final coda cluster, is the result of the autosegmental spreading  of the underlying high vowel in the stem. In simpler terms, if the stem has the high vowel [i], the epenthetic vowel is [i], and if the stem vowel is [u], the epenthetic vowel is [u], as shown in the data in \tabref{table:highV}.\footnote{It is worth mentioning that the vowel inventory of UHA includes the vowels /i/, /i:/, /u/, /u:/, /a/, /a:/, and the mid vowels [ee] and [oo]. The mid vowels [ee] and [oo] are not underlying in the dialect and are phonologically derived.}

\begin{table}
\caption{High vowel spreading in potential rising coda clusters in UHA}
\label{table:highV}
\begin{tabular}{llllll}
\lsptoprule
Underlying & Gloss & Surface & Possessive.3\textsc{sg}.\textsc{masc} \\
\midrule
a. /gidr/   &   ‘pot’   &           [gi.dir]    &   [gid.rˤu] \\
b. /ʔism/   &   ‘name’  &           [ʔi.sim]    &   [ʔis.mu] \\
c. /fiʕl/	&   ‘verb, action’	&  [fi.ʕil]	      &  [fiʕ.lu] \\
d. /ħukm/   &	‘verdict, ruling’ &	[ħu.kum]	&    [ħuk.mu] \\
e. /χuʃm/   &	‘nose’	&          [χu.ʃum]	   &   [χuʃ.mu] \\
f. /duɦn/   &	‘fat’	&           [du.ɦun]    &	[duɦ.nu]\\
\lspbottomrule                
\end{tabular}
\end{table}


As illustrated in \REF{table:highV}, the sonority-driven epenthetic vowel has the same quality as the stem vowel, because the stem vowel is high; however, when the stem vowel is low, consonant-to-vowel harmony can take effect. Otherwise, the default epenthetic vowel is [i] as discussed in the preceding section. I analyze the high vowel spreading epenthesis~-- loosely following the logic of \citet{Walker2001} in her analysis of Altaic rounding harmony~-- as a process of autosegmental spreading: the [front] feature spreads to the epenthetic vowel if the stem contains a [high] [front] vowel, and the [back] feature spreads to the epenthetic vowel if the stem contains a [high] [back] vowel. In order to motivate the spreading of the [front] or [back] feature to the epenthetic vowel, the constraints in \REF{constr} are necessary. 

\begin{exe} 
		\ex \label{constr}
			\begin{xlist}
				\ex \textbf{\textsc{Spread}-[front] (adapted from \cite{Walker2001}):} \\
                    ‘For any vowel in a word linked to a [front] autosegment, that same [front] autosegment must also be associated to all other vowels in the word. Assign a  violation for any [front] autosegment that is not associated to all vowels in the word.'

                    \ex  \textbf{\textsc{Spread}-[back] (adapted from \cite{Walker2001}):}\\
                    ‘For any vowel in a word linked to a [back] autosegment, that same [back] autosegment must also be associated to all other vowels in the word. Assign a violation for any [back] autosegment that is not associated to all vowels in the  word.’

                    \ex \textbf{\textsc{Ident}-IO(Vowel):}\\
                    ‘Let α be a vowel in the input and β be a correspondent of α in the output; then α and β have identical featural specifications. Assign a violation for any discrepant featural specification between α and β.’

                    \ex \textbf{\textsc{Uniform}-[front]/[back] (adapted from \cite{Walker2001}):}\\
                    ‘A [front] or [back] autosegment may not be multiply-linked to vowels that are distinctly  specified for height.’
			\end{xlist}
	\end{exe}


\begin{table}
\caption{High back vowel spreading}
\label{highback}
\begin{center}
\ShadingOn
\TipaOn
\begin{tableau}{c:c|c:c:s:s} 
\inp{/ʃukr/ ‘thank’}  \const*{\rotatebox{90}{*O2L1]σ}}  \const*{\rotatebox{90}{\textsc{Ident}IO(Vowel)}}  \const*{\rotatebox{90}{\textsc{Spread}-[back], \textsc{Spread}-[front]}} \const*{\rotatebox{90}{\textsc{Uniform}-[front]/[back]}} \const{\rotatebox{90}{Dep}} \const{\rotatebox{90}{Contig}}
\cand{[ʃukr]}              \vio{*!}  \vio{}  \vio{}    \vio{}   \vio{}  \vio{}
\cand [\Optimal]{[ʃu.kurˤ]} \vio{}   \vio{}   \vio{}   \vio{}  \vio{*}  \vio{*}
\cand{[ʃu.kir]}             \vio{}    \vio{} \vio{*!}  \vio{}  \vio{*}  \vio{*}
\cand{[ʃi.kir]}             \vio{}    \vio{*!} \vio{}  \vio{}  \vio{*}  \vio{*}
\cand{[ʃu.kor]}             \vio{}    \vio{}   \vio{}  \vio{*!}  \vio{*}  \vio{*}
\end{tableau}
\end{center}

\end{table}

The faithful candidate (a) loses, because it violates the high-ranked Split Margin constraint by exhibiting a rising-sonority coda cluster. Candidate (b) wins, because the high back specification  spreads to the epenthetic vowel; therefore, both vowels are identical. Candidates (c) and (d) both  fatally violate \textsc{Spread}-[back] in addition to \textsc{Dep} and \textsc{Contig}. For (c), the epenthetic vowel is a high front vowel, which does not harmonize with the stem vowel. Candidate (d) loses, because it violates \textsc{Ident}-IO(Vowel), even though the vowel features are shared by both syllables. Candidate (e) loses, even though the [back] feature spreads to the epenthetic vowel, because the height specifications of the vowels are different; therefore, it violates \textsc{Uniform}-[back]. 

Having provided the analysis for stem high vowel spreading in the previous tableau, now I turn to explain consonant-to-vowel harmony in stems with an underlying low vowel.


\subsection{Coronal consonant-to-vowel harmony}
In UHA, in underlying CaCC words, in which the stem vowel is a low vowel and the last consonant is coronal, the vowel [i] is epenthesized if the last two consonants would form a rising-sonority coda cluster. Consider the examples in \tabref{corC} below.

\begin{table}
\caption{Coronal consonant-to-vowel harmony}
\label{corC}
\begin{tabular}{lllll}
\lsptoprule
Underlying & Gloss & Surface & Possessive.3\textsc{sg}.\textsc{masc} \\\midrule
a. /makr/   &   ‘cunning’   &   [ma.kir]    &   [mak.rˤu] \\
b. /saɦl/   &   ‘valley’    &   [sa.ɦil]    &   [saɦ.lu] \\
c. /laħn/   &   ‘melody’    &   [la.ħin]    &   [laħ.nu] \\
d. /χabz/   &   ‘baking’    &   [χa.biz]    &   [χab.zu] \\
e. /lafzˤ/  &   ‘word’      &   [la.fizˤ]   &   [laf.zˤu] \\
\lspbottomrule                
\end{tabular}
\end{table}

As can be noted in \REF{corC}, all words ending with a coronal segment require the preceding epenthetic vowel to be [i] in order to match the coronality (frontness) between the consonant and the epenthetic vowel. In addition, according to \citet{Padgett2011}, universally, there is a harmonic effect between coronal consonants and front vowels. Note that all intermediate consonants in these words are noncoronal consonants. Therefore, this confirms that the trigger of coronal harmony is the last consonant in the word, since coronal consonant-to-vowel harmony is regressive, i.e. from the last consonant of the word to the preceding epenthetic vowel. This will be clear if we compare the data above with the default SonE in words such as [lakim], in which the second and third consonants of the word are noncoronal consonants. Thus, there is a need for an additional constraint to regulate the relationship between the last coronal segment of the word and the epenthetic vowel in the environment of sonority-driven epenthesis in a rising-sonority coda cluster. 

\begin{exe} 
		\ex \label{agree}
				 \textbf{AGREE-FEAT-CORONAL (\textsc{Agree-F-Cor}):} \\
                    Segments immediately preceding and tautosyllabic with a coronal consonant must agree  with it in the feature [coronal]. Assign a violation for any segment in the output which immediately precedes and is tautosyllabic with a coronal consonant and does not share the feature [coronal].
	\end{exe}



The tableau in \REF{tab:agree} provides the analysis of CaCC words in which the last consonant is a coronal and requires the immediately preceding epenthetic vowel to be the coronal [i].


\begin{table}
\caption{\textsc{Agree-F-Cor} in coronal-final CaCC Words}
\label{tab:agree}
\ShadingOn
\begin{tableau}{c|c:s:s} 
\inp{/ʃakl/ ‘shape, appearance’}      \const*{*O2L1]σ}  \const{Agree-F-Cor} \const{Dep} \const{Contig}
\cand{[ʃakl]}              \vio{*!}  \vio{}  \vio{}    \vio{}   
\cand [\Optimal]{[ʃa.kil]} \vio{}   \vio{}   \vio{*}   \vio{*}  
\cand{[ʃa.kul]}             \vio{}    \vio{*!} \vio{*}  \vio{*}  
\cand{[ʃa.kal]}             \vio{}    \vio{*!} \vio{*}  \vio{*}  

\end{tableau}
\end{table}

Candidate (a) loses because it violates the Split-Margin constraint by exhibiting a rising-sonority coda cluster in the output form. Candidate (b) wins because [i] harmonizes with [l] by agreeing in the feature [coronal]. Candidate (c) loses because [u], which is a dorsal vowel, does not agree with the following coronal consonant. In the same way, candidate (d) loses because the epenthetic pharyngeal [a] does not agree with the [l].

Having provided the analysis in which a final coronal consonant dictates the quality of the preceding epenthetic vowel in the process of non-default SonE, now I turn to pharyngeal and laryngeal consonant-to-vowel harmony, in which these consonants dictate that [a] will be the surface form of the epenthetic vowel. This vowel can be followed by a pharyngealized [rˤ], which also contains a pharyngeal feature in its segmental representation.

\subsection{Pharyngeal consonant-to-vowel harmony}
Before explaining the reason why the words in \REF{potC} below receive the low vowel [a], it is worth explaining the status of /r/ in Arabic. In Arabic, /tˤ ðˤ dˤ sˤ/ are the main emphatic consonants, yet /r/ also has an emphatic allophone based on the dialect and some phonological conditions. \citet[121]{Younes1993} argues that the status of the emphatic [rˤ] is not fully established. According to him, emphatic [rˤ] causes lowering in adjacent vowels.

Additionally, \citeauthor{Herzallah1990}'s \citeyear{Herzallah1990} representation of emphatics in Palestinian Arabic, including underlyingly emphatic /rˤ/, shows that these consonants have a secondary place of articulation, which itself has two components: pharyngeal and dorsal. However, in contrast to \citeauthor{Herzallah1990}'s proposal that emphatic /rˤ/ is underlying in Palestinian Arabic, data demonstrate that for the majority of  UHA speakers /r/ is underlyingly only plain (coronal) and it is pharyngealized next to emphatic, pharyngeal, or laryngeal segments;  with regard to underlying pharyngealized /rˤ/, it is associated with some speakers of the older generation, as will be analyzed in Section \ref{dorscv}.

In UHA stems with a low vowel in which a potential coda cluster contains a pharyngeal or laryngeal consonant followed by pharyngealized [rˤ], the epenthetic vowel that is inserted to avoid the surfacing of such a sonority-rising cluster is the vowel [a], which also has a [pharyngeal]\footnote{{According to \citet{Herzallah1990} the low vowel [a] has a pharyngeal feature.}} component. The words in \REF{potC} include medial pharyngeal or laryngeal consonants followed by pharyngealized [rˤ]. These words receive a low epenthetic vowel in the output form in order to avoid a potential rising-sonority coda cluster.


\begin{table}
\caption{Potential coda clusters with pharyngeal consonant-to-vowel harmony}
\label{potC}
\begin{tabular}{lllll}
\lsptoprule
Underlying & Gloss & Surface & Possessive.3\textsc{sg}.\textsc{masc} \\\midrule 
a. /ʃaɦr/   &   ‘month’ &   [ʃa.ɦarˤ]   &   [ʃaɦ.rˤu] \\
b. /naɦr/   &   ‘river’ &   [na.ɦarˤ]   &   [naɦ.rˤu] \\
c. /maɦr/   &   ‘dowry’ &   [ma.ɦarˤ]   &   [maɦ.rˤu] \\
d. /baħr/   &   ‘sea’   &   [ba.ħarˤ]   &   [baħ.rˤu]\\
\lspbottomrule                
\end{tabular}
\end{table}

In UHA, /r/ is pharyngealized next to emphatics, pharyngeals, laryngeals, and low and back vowels; otherwise, it is only coronal (plain), as shown in \REF{potcc}.


\begin{table}
\caption{Potential coda clusters with plain [r] and pharyngealized [rˤ]}
\label{potcc}
\resizebox{\textwidth}{!}{%
\begin{tabular}{lllll}
\lsptoprule
Underlying & Gloss & Surface & \textit{Nisba} Adjective &  \textit{Nisba} Gloss\\\midrule
a. /fikr/   &   ‘thought’   &   [fi.kir]    &   [fik.ri]    &   ‘intellectual’ \\
b. /ʃukr/   &   ‘thanking’  &   [ʃu.kurˤ]		\\
c. /baħr/   &   ‘sea’       &   [ba.ħarʕ]   &   [baħ.ri]    &   ‘naval, nautical, marine’\\
\lspbottomrule 
\end{tabular}}
\end{table}


In data set \REF{potcc}a, [r] is only coronal (i.e., not emphatic) because it is not preceded by a low or back vowel nor by a laryngeal or pharyngeal consonant. Therefore, the epenthetic vowel to break this rising-sonority coda cluster is the default [i]. In \REF{potcc}b, [rˤ] is pharyngealized because it is adjacent to the high back vowel [u], which itself is the result of high vowel spreading from the stem vowel to the epenthetic vowel. In \REF{potcc}c, the underlying coronal (plain) /r/ becomes pharyngealized [rˤ], because it is preceded by a laryngeal or pharyngeal consonant. Therefore, the epenthetic vowel to break up such a rising-sonority coda cluster is the low vowel [a], which itself agrees with the surrounding consonants in the feature [pharyngeal]. I can conclude from \REF{potcc} that the trigger of the low vowel insertion in rising-sonority coda clusters ending with pharyngealized [rˤ] is the pharyngeal and laryngeal consonants. In derived forms, when the \textit{nisba} (adjectival) suffix /-i/ or the first-person possessive suffix /-i/ is  attached to the /r/-final stem, the /r/ resyllabifies to form an onset for the syllable containing /-i/; therefore, it surfaces faithfully as the coronal (plain) [r].

Before starting the OT analysis of pharyngeal and laryngeal consonant-to-vowel harmony, it is necessary to present the definition for the constraint needed in this analysis. 

\begin{exe} 
\ex \label{connec}
The Constraint Necessary for Pharyngeal Consonant-to-Vowel Harmony\\
AGREE-FEAT-PHARYNGEAL (\textsc{Agree-F-Phar}): Segments immediately following  and tautosyllabic with pharyngeal and laryngeal segments must agree with them  in the feature [pharyngeal]. Assign a violation for any segment in the output which immediately follows and is tautosyllabic with a pharyngeal or laryngeal segment and does not agree with it in the feature [pharyngeal].
\end{exe}
 

The new constraint will be used for the analysis of pharyngeal and laryngeal harmony in coda clusters including [rˤ].

\begin{table}
\caption{Pharyngeal and laryngeal consonant-to-vowel harmony}
\label{tab:pha}
\ShadingOn
\begin{tableau}{c|c:s:s} 
\inp{/baħr/ ‘sea’}      \const*{*O2L1]σ}  \const{Dep} \const{Contig} \const{Agree-F-Phar}
\cand{[baħr]}              \vio{*!}  \vio{*}  \vio{}    \vio{}
\cand{[baħrˤ]}            \vio{*!}   \vio{}   \vio{}   \vio{}
\cand[\Optimal]{[ba.ħarˤ]}  \vio{}    \vio{} \vio{*}  \vio{*}
\cand{[ba.ħar]}             \vio{}    \vio{*!} \vio{*}  \vio{*}
\end{tableau}
\end{table}


The faithful candidate (a) loses because it fatally violates the Split-Margin constraint; in addition, it violates \textsc{Agree-F-Phar} by preserving the plain [r] next to the pharyngeal consonant. This is because the underlying rhotic /r/ is [rˤ] if it is immediately adjacent to laryngeal and pharyngeal segments. Candidate (b) also loses, because it violates the Split-Margin constraint, even though the pharyngealized [rˤ] agrees with the preceding consonant in the feature [pharyngeal]. Candidate (c) is the winner, because it satisfies the undominated Split-Margin constraint; in  addition, it satisfies low-ranked \textsc{Agree-F-Phar}, even though it violates \textsc{Dep} and \textsc{Contig}. Candidate (d) is like the winner, except the [r] is not pharyngealized, which creates a violation of the \textsc{Agree-F-Phar} constraint.\footnote{Recall
  in section \REF{syl} that the vowel [a] falls in the independent position of the foot as in the word [ka.(ˈtab).(ta̠.lu)] ‘I wrote for him’. However, in the word [(ˈba.ħa̠rˤ)], [a] is epenthesized in the non-head position of the foot for a different reason. Here, it is inserted to agree with the [\textsc{Phar}] feature of the surrounding consonants.
  }

This section has provided the analysis of those situations wherein a potential coda cluster  includes a pharyngeal or laryngeal consonant followed by /r/, and how such a cluster is broken up by the appropriate [a] vowel, which also has a pharyngeal component.


\subsection{Dorsal consonant-to-vowel harmony with /rˤ/}\label{dorscv}

According to \citet{Youssef2019}, complex /r/ (i.e. pharyngealized [rˤ]) could have V-place [dorsal] as well as C-place [coronal]. He also indicates that pharyngealized [rˤ] shows pharyngeal constriction alongside dorsal lowering. Youssef’s description of the pharyngealized /rˤ/ matches the feature representation of pharyngealized /rˤ/ and other pharyngeal obstruents given by \citet{Herzallah1990}. Therefore, I adopt the feature representation given by \citet{Herzallah1990}  in which /rˤ/ has primary [coronal] place, which is dominated by a C node, and secondary [dorsal] and [pharyngeal] places, which are dominated by a V node. The dataset in \REF{rphar} below represents the uncommon realization of the rhotic in UHA, as produced by the older generation of UHA speakers, in which the rhotic is underlyingly pharyngealized.

Consistent with the assumptions of OT, I propose that the use of pharyngealized [rˤ] and agreement between this consonant  and the preceding epenthetic [u] happens simultaneously in the output form rather than positing an intermediate stage in which /r/ becomes [rˤ] followed by insertion of [u] to agree with this segment, as illustrated in \REF{rphar}.


\begin{table}
\caption{[rˤ] consonant-to-vowel harmony}
\label{rphar}
\resizebox{\textwidth}{!}{%
\begin{tabular}{lllll}
\lsptoprule
Underlying & Gloss & Surface & Possessive.3\textsc{p.sg.masc} &  Possessive.1\textsc{p.sg}\\\midrule
a. /d͡ʒadrʕ/    &   ‘wall’  &   [d͡ʒa.durʕ] &   [d͡ʒad.rʕu] ‘his wall’  &   [d͡ʒad.ri] ‘my wall’ \\
b. /sˤagrʕ/ &   ‘falcon’    &   [sˤa.gurʕ]  &   [sˤag.rʕu] ‘his falcon’ &   [sˤag.ri] ‘my falcon’ \\
c. /badrʕ/  &   ‘full moon’ &   [ba.durʕ]   &   [bad.rʕu] ‘his full moon’   &   [bad.ri] ‘my full moon’\\
\lspbottomrule 
\end{tabular}}
\end{table}


Compare the data above with the parallel data point for the speakers of the younger generation for which the underlying form of the rhotic is plain /r/ in \REF{underl}.


\begin{table}
\caption{Underlying rhotic for speakers of the younger generation of UHA}
\label{underl}
\resizebox{\textwidth}{!}{%
\begin{tabular}{lllll}
\lsptoprule
Underlying & Gloss & Surface & Possessive.3\textsc{p.sg.masc} &  Possessive.1\textsc{p.sg}\\\midrule
a. /badr/   &   ‘full moon’ &   [ba.dir]    &   [bad.rˤu] ‘his full moon’   &   [bad.ri] ‘my full moon’ \\
b. /sˤagr/  &   ‘falcon’    &   [sˤa.gir]   &   [sˤag.rˤu] ‘his falcon’ &   [sˤag.ri] ‘my falcon’ \\
c. /d͡ʒadr/ &   ‘wall’  &   [d͡ʒa.dir]  &   [d͡ʒad.rˤu] ‘his wall’	&[d͡ʒad.ri] ‘my wall’\\
\lspbottomrule 
\end{tabular}}
\end{table}


The data in \REF{rphar} show words ending with /rˤ/ for the older generation, which receive [u] as the epenthetic vowel. This epenthetic vowel, regardless of its quality, breaks up the rising-sonority cluster, as is expected. When adding the 1\textsuperscript{st} person possessive suffix [-i], [r] is not pharyngealized, because it forms the onset of the syllable which contains the vowel [i]. In the same way, when adding the 3\textsuperscript{rd} person masculine possessive suffix [-u], the allophonic [rˤ] surfaces, because it falls   in the onset position of the syllable containing the vowel [u]. From these data, when a speaker from the older generation has an underlying /rˤ/, it necessitates the insertion of the vowel [u] in the process of sonority-driven epenthesis, as both segments agree in the feature [dorsal].\footnote{I propose that the disagreement between the generations’ underlying forms of the rhotic can be understood as lexicon optimization, in which some output forms of the older generation with allophonic de-emphaticized [r] (e.g., the first person possessive suffix)  are reinterpreted as the underlying form of the rhotic by the younger generation (\cite[547--548]{Holt2015}). This reconstructed underlying form of the rhotic has an emphatic allophonic variant when it meets the condition of being next to a dorsal vowel. This proposal is compatible with the Synchronic Base Hypothesis \citep{Hutton1996}. See \citet[section 5.6]{Bokhari2020} for more details.}

This leads us to propose a constraint that requires agreement between pharyngealized [rˤ]  and the segment next to it.

\largerpage
\begin{exe} 
\ex \label{constr2} \textbf{Constraints motivating Dorsal Consonant-to-Vowel Harmony}
    \begin{xlist}
        \ex AGREE-FEAT-DORSAL-[rˤ] (\textsc{Agree-F-Dors}-[rˤ]): \\
            Segments adjacent to and tautosyllabic with [rˤ] must agree with it in the feature [dorsal]. Assign a violation for any segment in the output which is adjacent to and tautosyllabic with [rˤ] and does not agree with it in the feature [dorsal].

          \ex  \textsc{Max-V-Place}:\\
            The V-Place feature (secondary place of articulation) associated with a consonant in the input must have a correspondent in the output. Do not delete a V-Place feature (secondary place of articulation) from a consonant. Assign a violation mark for any secondary place of articulation feature which is deleted from a consonant in the output.
    \end{xlist}
\end{exe}
 
The tableau in \tabref{tab:dor} provides the analysis in which pharyngealized [rˤ] necessitates the epenthesis of [u] to  match its dorsal feature.


\begin{table}
\caption{Dorsal harmony with pharyngealized [rˤ]}
\label{tab:dor}
\ShadingOn
\begin{tableau}{c|c:c:c|c|s} 
\inp{/d͡ʒadrˤ/ ‘wall’}      \const*{\rotatebox{90}{*O2L1]σ}}  \const{\rotatebox{90}{Max-V-Place}} \const{\rotatebox{90}{Dep}} \const{\rotatebox{90}{Contig}} \const{\rotatebox{90}{Agree-F-Dors-[rˤ]}} \const{\rotatebox{90}{Agree-F-Cor}}

\cand{[ʤadrˤ]}
\vio{*!}  \vio{}  \vio{}    \vio{}     \vio{*}    \vio{}

\cand{[ʤadr]}
\vio{*!}  \vio{*}  \vio{}    \vio{}     \vio{}    \vio{}

\cand[\Optimal]{[ʤa.durˤ]}
\vio{}  \vio{}  \vio{*}    \vio{*}     \vio{}    \vio{*}  

\cand{[ʤa.dur]}
\vio{}  \vio{*!}  \vio{*}    \vio{*}     \vio{}    \vio{*}  

\cand{[ʤa.dir]}
\vio{}  \vio{*!}  \vio{*}    \vio{*}     \vio{}    \vio{}  

\cand{[ʤa.dirˤ]}
\vio{}  \vio{}  \vio{*}    \vio{*}     \vio{*!}    \vio{}  

\end{tableau}
\end{table}

The faithful candidate (a) loses because it violates the Split-Margin constraint. It also violates \textsc{Agree-F-Dors-[}rˤ]. This is because the underlying form for the rhotic is /rˤ/ for older speakers of UHA. In the same way, candidate (b) loses because it also violates the Split-Margin constraint, in addition to \textsc{Max-V-Place}. Candidate (c), with [rˤ], wins because it respects the Split-Margin constraint as well as\textsc{Max-V-Place} and \textsc{Agree-F-Dors}-[rˤ]. This is because the dorsal [u] agrees with the following pharyngealized [rˤ], which also has a dorsal component. The following candidates (d) and (e) lose, because they violate \textsc{Max-V-Place} in addition to \textsc{Dep} and \textsc{Contig} even though (e) respects lowest-ranked \textsc{Agree-F-Cor}. Candidate (f) loses because the pharyngealized [rˤ], which has a dorsal component, is preceded by [i], which is a coronal vowel. The tableau in \REF{tab:dor} demonstrates the OT analysis of [u]-epenthesis before the pharyngealized [rˤ] in potential rising-sonority coda clusters for the speakers of the older generation of UHA.

Having provided the analysis for rhotic consonant-to-vowel harmony, the next section discusses and concludes the paper.

\section{Discussion and conclusion}\label{disc}
This study is one of relatively few studies to discuss sonority- vs. syllable-driven epenthesis in UHA. Other studies do not discuss this in as much detail and it is the only study to provide a detailed OT analysis for the default sonority-driven epenthesis. Furthermore, the study utilized the universal DTE approach in distinguishing between the sonority-driven epenthesis as [i] and syllable-driven epenthesis as [a], based on the prosodic domain of the epenthesis and the quality/sonority of the epenthetic vowel. The result of this is compatible with the universal prediction that [a] is higher in sonority than [i]. 

The study elaborated on two types of epenthesis in UHA: SylE and default SonE. In UHA, in order to avoid word-internal superheavy syllables after suffixation, the vowel [a] is epenthesized in the process of resyllabification. The vowel [a], which has higher sonority than any other vowel, strengthens the weak degenerate syllable by forming the nucleus of the newly created syllable, which contains the previously unsyllabified consonant. Therefore, the constraint ranking for this type of epenthesis is *$\Delta \sigma $${\leq}$\{i, u\}  $\gg$  *$\Delta \sigma {\leq}$a. With regard to default SonE, the high peripheral vowel, [i], which is less in sonority than the low vowel, is epenthesized in the non-head position. The ranking of the DTE constraints is the opposite of that found in syllable-driven epenthesis. In addition, the universal Place of Articulation Hierarchy plays a role in eliminating the high back vowel, [u], from getting epenthesized in the process of default sonority-driven epenthesis. 

The default SonE epenthetic vowel in UHA is /i/. However, all vowels, /a/, /i/, and /u/, can be epenthetic to break up a potential rising sonority coda cluster. This is of course due to the harmony between the epenthetic vowel and the neighboring consonants and stem vowels. If the stem contains a high vowel, /i/ or /u/, this vowel spreads its feature [front] or [back] to the sonority-driven epenthetic vowel. This is because the high vowel   spreads its feature to the epenthetic vowel to break up such a rising-sonority coda cluster, by inserting [i] in the newly created syllable after the process of epenthesis.  

Further, coda consonants play an important role in determining the quality of the vowel in sonority-driven epenthesis. If the potential rising sonority coda cluster ends in a coronal consonant, the epenthetic vowel is the coronal [i] in the process of what is known as coronal consonant-to-vowel harmony.

In a stem that ends with /r/ when the medial consonant is pharyngeal or laryngeal, [a] is epenthesized in the process of pharyngeal harmony. Medial pharyngeal and laryngeal consonants in such words are the trigger of the epenthesis of the low vowel, [a], which has a pharyngeal component, and which itself is the trigger of the pharyngealization of the [rˤ]. 

The study also elaborated on the dorsal consonant-to-vowel harmony in words ending with /rˤ/ for older speakers. /rˤ/ is  the underlying form and is the uncommon realization of the rhotic. It is de-pharyngealized when it is adjacent to a coronal segment. In a potential rising-sonority coda cluster ending with a rhotic, [u] is epenthesized to prevent such fatal coda clusters from surfacing for the speakers of the older generation. The feature [dorsal] for the pharyngealized /rˤ/ dictates the quality of the epenthetic vowel to be [u], matching in the feature [dorsal] between the pharyngealized rhotic and the epenthetic [u]. In this word, [u] is epenthesized, rather than [a], because the secondary [dorsal] feature of the pharyngealized [rˤ] outranks its secondary pharyngeal feature. While this study has provided a phonological account of epenthesis in UHA, more research needs to be done in the future, specifically, to look at the phonetic properties of epenthetic vowels in comparison to lexical vowels in the dialect, similar to Lebanese Arabic \parencitetv{chapters/08.Hall}.

The Hasse diagram in \figref{UHA} summarizes Consonant-to-Vowel Harmony effect on coda clusters in UHA.

\begin{figure} 
\caption{Consonant-to-Vowel Harmony effect on coda clusters in UHA}
% % % % \includegraphics[width=1\textwidth]{figures/UHA.png}\bigskip\\
\fittable{\begin{forest}
for tree = {calign=child, l+=30}
[,phantom
[{*O$_{2}$ L$_{1}$]\textsubscript{σ}}, tier = 1, calign child = 1
    [Max-V-\\Place, align=center, tier=2
        [{A\textsc{gree}-F-\\
          \textsc{Dors}-[rˤ]}, align=center, tier=3
            [{\textsc{Agree}-F-\\
              {}C\textsc{or}{}}, align=center, tier=4]
        ]
    ]
    [{A\textsc{gree}-F-\\\textsc{Phar}}, align=center, tier=2]
    [*D\textsc{ors}, tier=2
        [*C\textsc{or}, tier=3]
    ]
    [{*$-\Delta_{\text{Ft}}\geq$ a}, tier=2
        [{*$-\Delta_{\text{Ft}}\geq$ \{i,u\}}, tier=3]
    ]
    [{\textsc{Spread}-\\\relax[back]/\\\relax[front]}, name=spread, tier=2, align=center]
    [{\textsc{Uniform}-\\\relax[front]/\\\relax[back]}, name=uniform, tier=2, align=center]
  ]
  [IDENT-IO(Vowel), name=id, tier=1, calign child = 2
    [\textsc{Dep}, name=dep, tier=2
        [{*O$_{2}$O$_{1}\text{]}_{\text{σ}}$}, name=OO, tier=3,
         before drawing tree={x+=2em}
        ]
    ]
    [C\textsc{ontig}, name=contig, tier=2]
]
]
\draw (id.south) -- (spread.north);
\draw (id.south) -- (uniform.north);
\draw (dep.south) -- (OO.north);
\draw (contig.south) -- (OO.north);
\end{forest}}
% \todo[inline]{Please check and confirm this newly generated structure}
\label{UHA}
\end{figure}


{\sloppy\printbibliography[heading=subbibliography,notkeyword=this]}
\end{document}
