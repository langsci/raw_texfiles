\chapter{Concluding summary} \label{sec:16}

Andakí is an extinct language formerly spoken in southern Colombia, with no known surviving speakers. While its genealogical affiliation remains uncertain, Andakí exhibits several lexical, morphological, and typological features that it shares with neighboring languages. These include vowel nasality and the use of prefixes or proclitics for verbal person marking. The strongest lexical parallels are found in Nasa Yuwe, suggesting some degree of areal influence or contact.

There are two primary sources of data on Andakí: a late 18th-century list (in two versions) containing sentences and isolated lexical items, and a word list from the early 19th century. These sources show significant orthographic — and likely allophonic — variation, complicating consistent interpretation. An additional challenge is that many of the sentence translations are not strictly literal, limiting the reliability of grammatical analysis.

The available material is heavily skewed toward imperatives and prohibitions — approximately half of the recorded phrases are commands — likely reflecting the missionary context in which the data were collected. Little to no information is available on other grammatical features such as tense marking. It remains unclear whether such features were absent from Andakí itself or simply not attested in the surviving data. As such, this book does not attempt a comprehensive grammatical description, but instead outlines tendencies based on a limited and fragmentary dataset.

The Andakí vowel inventory consists of three oral and three nasal vowels: /a/, /i/, /u/, and their nasal counterparts. The Andakí inventory of stop consonants is asymmetrical in that it contains a series of four voiceless stops /p/, /t/, /k/, /kʷ/, but only one voiced stop, /b/. Besides a voiceless affricate /ʧ/ and voiceless fricatives (/s/, /h/, and possibly /x/), there are two nasal consonants, /m/ and /n/, and one rhotic /r/. Stress appears to have fallen on the final vowel of a word. Root structure is generally straightforward, typically following a CVCV pattern.

Andakí demonstratives distinguish at least a two-way contrast. Interrogative constructions involving ‘who’ take the form of cleft sentences, though such structures are not attested with other interrogative pronouns. Personal pronouns appear to be optional, and may occur either before or after the verb.

Nouns in Andakí may carry classifiers related to shape and liquid consistency, as well as derivational gender markers. The numeral system appears to be etymologically transparent: the term for ‘two’ seems to derive from ‘eye’; ‘three’ builds on ‘one’ and ‘two’; and ‘five’ is related to ‘hand’.

Andakí features an oblique case marker with a wide range of functions, including marking the direct object, comitative, and topic of conversation. Genitive markers vary depending on whether they are used with pronouns or full nouns. The allative and purposive marker may attach to both nouns and verbs.

In Andakí, expressions corresponding to adjectives and adverbs in Spanish are commonly realized as headless relative clauses or are embedded in cleft constructions.

Verbal morphology in Andakí is complex, involving prefixes or proclitics as well as suffixes. Prefixes may indicate direction or associated motion. Evidence for verbal person marking is limited to the 2\textsuperscript{nd} person, with no reliable data for the 1\textsuperscript{st} or 3\textsuperscript{rd} persons. The language shows multiple strategies for expressing imperative meaning, including various morphemes (prefixes, proclitics, and suffixes). There are marginal instances of suppletion and reduplication in the verbal domain. Copular constructions involve a free morpheme or a prefix; the use of the copula prefix seems to be restricted to cleft constructions. In the verbal domain, negation is expressed through a combination of a proclitic and a suffix, or via a free morpheme placed after the verb.

The available Andakí data include examples of predicative possession. Andakí also appears to feature verbal ellipsis, a typologically rare phenomenon. In intransitive constructions with pronominal subjects, SV order is the most common, though VS order also occurs. Imperative constructions in Andakí allow both VO and OV word orders, with VO appearing to be the more common pattern.