\chapter{Cultural background} \label{sec:2}

This chapter provides a short background of Andakí culture. An overview of Andakí history can be found, for instance, in \citet{BuenaventuraAmézquita2019}. From the late 18\textsuperscript{th} to the mid-19\textsuperscript{th} century, the Andakí area was localized in the rainforest near the Upper Caquetá area (\citealt[121]{Friede1948a}; \citealt[32–33]{Friede1953}), in the present-day departments of Putumayo and Caquetá in southern Colombia (see \figref{fig:2.1}). The area was called \textit{Provincia de los Andaquíes} or \textit{Provincia del Andakí} by the end of the 18\textsuperscript{th} century \citep[190]{Friede1952}. According to \citet[59]{Albis1860--1861}, Andakí immaterial culture (religion, ceremonies) does not seem to have differed a lot from that of its neighbors such as the Guaque, a people speaking a Cariban language.

\begin{figure}
\includegraphics[width=1\textwidth]{figures/fig1.png}
\caption{Map of the approximate distribution area of Andakí, created by Arjan Mossel, based on \citet{EberhardSimonsFennig2025}}
\label{fig:2.1}
\end{figure}

The Andakí were sedentary farmers \citep[17]{VargasMotta1962} and cultivated corn, manioc, plantains, pineapple, squash, and sugarcane (\citealt[109--113]{Friede1948b}; \citealt[56]{Albis1860--1861}; \citealt[17--18]{VargasMotta1962}). In addition to farming or horticulture, the lifestyle of the Andakí was based on hunting and fishing. For hunting, they used blowpipes with darts dipped in vegetal poison \citep[17--18]{VargasMotta1962}. 

Among the trade good specialties of the Andakí were honey, and black and white wax found in the woods \citep[20]{VargasMotta1962}. They also traded iron tools and silver, of which they manufactured triangular ear accessories, for animals and their skins \citep[56--61]{Albis1860--1861}. The Andakí had access to gold provided by the Mazamorras River and mined minerals, such as amethyst \citep[78]{FernandezdePiedrahita1881}. 

The following observations have been made with respect to Andakí material culture. They manufactured hammocks and bags from agave fibers. Their flatstones used in grinding resembled tools used by certain Amazonian groups \citep[16]{VargasMotta1962}. Considering the existence of Andakí words for cotton, thread, and clothing, the Andakí were likely dressed in woven cotton clothing \citep[15]{VargasMotta1962}. On festive occasions, they wore necklaces, crowns of colored feathers, and used red and black plant-based makeup \citep[15]{VargasMotta1962}. Data from Andakí word lists also suggest that they knew pottery – compare, for instance, <guajizi> / <guajixi> ‘pot’, “olla” (M\_110) or <batonafi> ‘dish’, “plato” (\mbox{A\_008}).\footnote{The original Spanish translations from 18\textsuperscript{th}- and 19\textsuperscript{th}-century sources are provided wherever appropriate. For verbs, these are often cited in inflected rather than infinitive forms. However, this convention is not followed when verbs appear as inflected forms in the sources but only the root is relevant. We generally preserve the spelling found in the original sources, although in some cases it is adapted to modern Peninsular Spanish; the original spelling is always provided in the appendix. To ensure typographical consistency, individual Andakí forms in the running text are mostly cited with a lowercase initial, even though both Spanish and Andakí entries in the manuscripts consistently begin with an uppercase letter. Original capitalization is retained only in glossed examples. The use of voseo in the Spanish translations (e.g., \textit{vení} instead of \textit{ven} ‘come (\textsc{sg})!’) has been retained. Pointed brackets indicate graphemic representations.} It is interesting to see that Andakí terms for metals and items made of metal do not seem to be borrowings from Spanish: <bininto> / <binintó> ‘iron’, “hierro” (M\_476), <yrajaró> ‘steel’, “acero” (M\_477), <benije> (A\_009), <binixi> (M\_480) ‘machete’, <mifi> ‘needle’, “aguja” (M\_482) – the term for ‘needle’, for instance, is often a borrowing from Spanish or Portuguese in Lowland South American languages (e.g., Pache \& Urban, forthc.). Andakí houses had a circular shape and were distributed around what would have been the residence of the cacique or the chief of the tribe \citep[15]{VargasMotta1962}.

In their battles, the Andakí made use of spears, darts, clubs, and slings; their spears and clubs were made from the wood of \textit{chonta} palms (\citealt[16--17]{VargasMotta1962}; see \citealt[107--108]{Friede1953} for a description of the Andakí inventory of arms). Like their neighbors, the Andakí had the reputation of being “exceedingly wild and warlike” \citep[59]{Albis1860--1861} and they regularly united with them to fight the non-indigenous invaders (\citealt[91]{Friede1953}; \citealt[36]{VargasMotta1962}), resisting them for almost 250 years \citep[11]{Friede1953}. According to Colombia’s 2018 National Population and Housing Census \citep{CNPV2018}, there were still some 248 persons identifying themselves as Andakí at that time. Although there is no recent evidence that Andakí is still spoken – \textit{Glottolog} \citep{HammarströmForkelHaspelmathBank2025} and \textit{Ethnologue} \citep{EberhardSimonsFennig2025} consider the language extinct (Glottolog) or endangered/dormant, that is, “no longer used as a first language by any remaining members of the ethnic community” (Ethnologue) – the possibility that there are still some Andakí groups or individuals speaking Andakí cannot totally be excluded. 