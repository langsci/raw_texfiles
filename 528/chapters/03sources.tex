
\chapter{Sources and previous works on Andakí} \label{sec:3}

There are three primary sources on Andakí. First, there are two nearly identical vocabulary lists which are part of a larger collection of indigenous vocabularies sent to Madrid by the scholar José Celestino Mutis (1732–1808) by order of the archbishop-viceroy of Santafé de Bogotá, Don Antonio Caballero y Góngora. The word lists were a request to Carlos III by Catherine II of Russia who wanted to document all the languages of the world. These Andakí materials sent to Madrid in 1787 by Mutis never reached the empress (\citealt[382]{Ramos1959}; \citealt[102]{OrtegaRicaurte1978}; \citealt[54]{AdelaarMuysken2004}) and were eventually published by the Royal Library of Madrid in 1928 (collection \textit{Lenguas de América}). Scans of the two copies of the \textit{Vocabulario andaqui–español}, Ms. II/2911 and Ms. II/2912, are available online at \url{https://rbdigital.realbiblioteca.es/s/realbiblioteca/item/15059} and \url{https://rbdigital.realbiblioteca.es/s/realbiblioteca/item/15084}. Ms. II/2911 has 13 numbered pages, while Ms. II/2912 has 18.  These two manuscript lists are nearly identical: they contain mostly the same concepts, but regularly differ in spelling. They offer 772 different Andakí words and phrases, accompanied by their Spanish translations. For spelling differences between both lists, compare, for instance, ‘How is your (\textsc{sg}) wife?’, “Cómo está tu mujer?” (M\_288), transcribed as <Karupegua hszazini> in Ms. II/2911 and as <Karupegua ñszazini?> in Ms. II/2912. For further examples of different spelling conventions used in the two 18\textsuperscript{th}{}-century lists, see (\ref{ex:3:1}--\ref{ex:3:2}).

\begin{exe}
\ex\label{ex:3:1}
    \begin{xlist}
    \ex\label{ex:3:1a}
    <Ychuyzi kaquani?> \\
    \gll ychuyzi ka-qua ni? \\
    what 2-say \textsc{q} \\
    \glt ‘What do you (\textsc{sg}) say?’, “Qué dices?”\footnote{In the second line, we use a slightly simplified transcription of Andakí, which does not reproduce diacritics.} (M\_008)
\newpage
    \ex\label{ex:3:1b}
    <Ychuhizi caquani?> \\
    \gll ychuhi-zi ca-qua ni? \\
    what 2-say \textsc{q} \\
    \glt ‘What do you (\textsc{sg}) say?’, “Qué dices?” (M\_751)
    \end{xlist}
\ex\label{ex:3:2}
    \begin{xlist}
    \ex\label{ex:3:2a}
    <Ychuyzi kazini?> \\
    \gll ychuyzi ka-zi ni? \\
    what 2-do \textsc{q} \\
    \glt ‘What do you (\textsc{sg}) do?’, “Qué haces?” (M\_009)
    \ex\label{ex:3:2b}
    <Ychuhizi cazini?> \\
    \gll ychuhizi   ca-zi  ni? \\
    what 2-do \textsc{q} \\
    \glt ‘What do you (\textsc{sg}) do?’, “Qué haces?” (M\_752)
    \end{xlist}
\end{exe}

More information on spelling conventions in the different Andakí sources is provided in \sectref{sec:5}. 

In some cases, two consecutive entries in the 18\textsuperscript{th}{}-century lists seem to relate to each other in terms of question and answer, as in the case of (\ref{ex:3:3}--\ref{ex:3:4}).

\begin{exe}
\ex\label{ex:3:3}
<Bizejazini?> / <Bizefazini> \\
\gll bize-jazi  ni? \\
what-\textsc{purp} \textsc{q} \\
\glt ‘What for?’, “Para qué?” (M\_185)
\ex\label{ex:3:4}
<Choyaxazé> / <Choya xaré> \\
\gll choya-xaze \\
eat-\textsc{purp} \\
\glt ‘In order to eat’, “Para comer” (M\_186)
\end{exe}

Note the variation in the rendering of the same grammatical morpheme <fazi>, <jazi>, <xaré>, <xazé> ‘purposive’ in the two consecutive examples. 

Given that there are more corrections in Ms. II/2911 than in Ms. II/2912, it seems that the latter is of slightly later origin and a copy of Ms. II/2911. In single cases, either of both copies may contain a form that is not present in the other one: seven entries that are found in Ms. II/2911 have no counterpart in Ms. II/2912, whereas only two entries in Ms. II/2912 have no counterpart in Ms. II/2911. Therefore, it is important to consider both sources. Overall, the handwriting is quite different between the two manuscripts, and the documents may have been written by different hands. Some 400 of the 772 entries in these two 18\textsuperscript{th}{}-century sources are at least short phrases.  \citet{GómezTorres2012--2013a, GómezTorres2012--2013b} provide a transcription of the anonymous 18\textsuperscript{th}{}-century manuscripts, with additional comments on the sources and on some of the Spanish translations.

A second, shorter vocabulary list was composed by the Presbyter Manuel Maria Albis in 1854. This word list was part of his notes of a trip to the \textit{Provincia de los Andaquíes} \citep[53]{Albis1860--1861}. According to the introduction by the editors, Albis’ notes were delivered by a certain Dr. Manuel Antonio Bueno \citep[53]{Albis1860--1861}. These data were first published in Spanish in 1855 in a work with the title \textit{Los indios del Andaquí: memorias de un viajero} \citep{VergarayVergaraDelgado1855}, and translated into English and published in the \textit{Bulletin of the American Ethnological Society} (1860–1861). In his work, Albis describes the habits and language not only of the Andakí, but also of the Ingano (speakers of a Quechua variety), Koreguaje (Tucanoan language family), Guaque (Cariban language family), and Macaguaje (Tucanoan language family), who also inhabit this area. The Andakí vocabulary section of Albis contains 159 words and four phrases. Since the present work is mainly concerned with the grammar of Andakí, it is largely based on the Mutis materials from the 18\textsuperscript{th} century; data from \citet {Albis1860--1861} will be discussed primarily in the phonology chapter.  

Works based on these three primary Andakí sources have been relatively few so far. The first one is a presentation of 21 Andakí lexical items from Albis’ lists and a comparison with their counterparts in Koreguaje, a Tucanoan language \citep[481]{Latham1862}. Besides presenting data in a table, however, Latham does not further comment on Andakí, or on coinciding forms in Andakí and Koreguaje. Indeed, there is no evidence for a genealogical connection between Andakí and Tucanoan languages, although there may be some single loanwords such as Andakí <paga> / <pagá> (M\_499), <pagà> (A\_112) ‘manioc’, “yuca”, which is reminiscent of Proto-Tucanoan *poʔɡa ‘manioc flour’ \citep[144]{WaltzWheeler1972}. 

\citet{Rivet1924}, likewise based on Albis’ data, is the first study to discuss Andakí grammar: the author identifies a few grammatical morphemes among which are <-guae> ‘diminutive’ and a prefix <ma->, <man->, <min->, attested, for instance in <min-gosoa> ‘arm’ (cf. the entry in A\_100) and <ma-esegua> ‘tail’ (cf. the entry in \mbox{A\_097}); also, \citet[100--101]{Rivet1924} states that subject person is marked on the verb by prefixation in this language. A few grammatical morphemes are also mentioned by \citet[190]{Friede1952}, although he does not comment on their meaning.

More recently, papers on Andakí phonology \citep{CoronasUrzúa1994} and lexicon \citep{CoronasUrzúa1995} have been published, based on the two 18\textsuperscript{th}{}-century lists \citep{Anonymous-a, Anonymous-b} and on \citet{Albis1860--1861}. \citeauthor{CoronasUrzúa1994}'s (\citeyear{CoronasUrzúa1994}) observations on Andakí phonology will be addressed in more detail in this book, in \sectref{sec:5} on Andakí phonology. Apart from lexical roots, \citet {CoronasUrzúa1994, CoronasUrzúa1995} also mentions a small number of grammatical morphemes: for instance, the author argues that /-kʷa/ is a negation marker \citep[85]{CoronasUrzúa1994}; Andakí /haa/ is interpreted as an accusative or dative marker \citep[82]{CoronasUrzúa1995}, the suffix /-ra/ as a locative marker \citep[91]{CoronasUrzúa1995}, /-sa/ as an imperative marker \citep[92]{CoronasUrzúa1995}, and /-si/ as a suffix forming adjectives \citep[94]{CoronasUrzúa1995}.

Some further insights into Andakí grammar are provided by \citet[140]{AdelaarMuysken2004}: <ca{}->, for instance, marks the 2\textsuperscript{nd}-person subject, as shown in (\ref{ex:3:5}).

\begin{exe}
\ex\label{ex:3:5}
<Ninga camimi> \\
\gll ninga ca-mimi \\
I \textsc{2}{}-love \\
\glt ‘Do you love me?’ (M\_232; adapted from \citealt[140]{AdelaarMuysken2004})
\end{exe}

Modal categories, by contrast, are encoded by suffixes, as shown in (\ref{ex:3:6}); the same is true for nominalization and case (\citealt[140]{AdelaarMuysken2004}).

\begin{exe}
\ex\label{ex:3:6}
<Fsrrajanozá> \\
\gll fsrrajano-za \\
lie.down-\textsc{imp} \\
\glt ‘Lie down!’ (M\_366; adapted from \citealt[140]{AdelaarMuysken2004})
\end{exe}
 
Finally, \citet[525--526]{Jolkesky2016} also discusses a few aspects of Andakí grammar, for instance the 2\textsuperscript{nd}-person singular \textit{ka}{}-, or the 2\textsuperscript{nd}-person plural pronoun \textit{rika-kʷa}. 

Besides publications addressing Andakí itself, there have also been several attempts to connect this language to other languages or language families in the Americas. Some (e.g., \citealt{Rivet1924}; \citealt{Greenberg1956,Greenberg1987}; \citealt{Swadesh1962}; \citealt{Loukotka1968}) have considered Andakí to be a Chibchan or Páezan language. \citet{Rivet1924}, for instance, compares Andakí to Nasa Yuwe (also known as Páez), but also to Barbacoan languages and Chibchan languages, and considers them all to be genealogically related. The last decades, by contrast, have seen a rise of linguists proposing a different view, namely, of Andakí as a language isolate (\citealt{ConstenlaUmaña1981}; \citealt{ConstenlaUmaña1991}; \citealt{CoronasUrzúa1994}; \citealt{CoronasUrzúa1995}; \citealt{KaufmanBerlin2007}; \citealt{Campbell2012}; \citealt{SeifartHammarström2018}: \citealt{Campbell2024}). Nevertheless, Adelaar and Muysken suggest that Andakí is “possibly related to \textit{Páez}” (italics in original, \citealt[611]{AdelaarMuysken2004}), and the 2025 edition of \textit{Ethnologue} classifies Andakí as Páezan \citep{EberhardSimonsFennig2025}. Similarities with Nasa Yuwe (Páez) do exist above all in the lexicon (e.g., \citealt[572--574]{Pache2018}; \citealt{Pache2024}). 

Andakí has also been investigated in terms of language contact and areal typology. For instance, the Andakí vowel system is characteristic of Amazonian languages in that it distinguishes between nasalized and oral vowels \citep[80]{CoronasUrzúa1994}. The weakly developed numeral system of Andakí has likewise been argued to indicate a possible Amazonian origin of this language (\citealt{AdelaarMuysken2004}: 140). In terms of loanwords, \citet{Jolkesky2016} notes several lexical parallels with Chibchan languages, Nasa Yuwe, and Tinigua (Tinigua–Pamiguan).
