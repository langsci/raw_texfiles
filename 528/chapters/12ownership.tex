\chapter{Ownership: <inca>} \label{sec:12}

The morpheme <inca> with variants <enca>, <ka>, <naa>, and <nca> is used in predicative possession. This is among the few grammatical morphemes in Andakí with an initial vowel (see \sectref{sec:5.1}). Andakí <inca> may best be interpreted as a morpheme that derives a noun phrase referring to a person who has a relation of ownership with the referent of the base, similar to Quechuan {}-\textit{juq} (see \citealt[226--227]{Adelaar1977}). The resulting construction is shown in (\ref{ex:12:1}).

\begin{exe} 
\ex\label{ex:12:1}
<Jahá chanca enca> / <Jahá chanca encá> \\
\gll jaha    chanca {enca} \\
yes    husband  \textsc{prop} \\
\glt ‘Yes, I have a husband’, “Sí, tengo marido” (M\_173)
\end{exe}

Note that in (\ref{ex:12:1}) and elsewhere in the available Andakí examples, the possessor is not explicitly expressed. 

The morpheme <inca> occurs in both statements and questions. Specifically, we have only examples of interrogative clauses involving a 2\textsuperscript{nd}-person possessor (‘Do you have…?’) and declarative clauses involving a 1\textsuperscript{st}-person possessor (‘I have…’) in the available Andakí materials. The morpheme in question, <inca> and related forms, always follows the noun referring to the possessed entity. In questions, it precedes the copula, as in (\ref{ex:12:2}--\ref{ex:12:5}). 

\begin{exe} 
\ex\label{ex:12:2}
<Nanqui zincaque> / <Nañquizincaque>  \\
\gll nanquiz {inca} que  \\
meat    \textsc{prop}  \textsc{cop}   \\
\glt ‘Do you have meat?’, “Tenéis carne?” (M\_284)
\ex\label{ex:12:3}
<Néngui en caque?> / <Neñgui encaqué?> \\
\gll nengui {enca} que?  \\
fish  \textsc{prop}  \textsc{cop} \\
\glt ‘Do you have fish?’, “Tenéis pescado?”\footnote{Note the formal similarity between the Andakí terms for ‘meat’ in (\ref{ex:12:2}) and for ‘fish’ in (\ref{ex:12:3}).} (M\_079)
\ex\label{ex:12:4}
<Chigua cancaqué?> \\
\gll chiguaca {nca} que? \\
children  \textsc{prop}  \textsc{cop}  \\
\glt 
‘Do you have children?’, “Tenéis hijos?” (M\_174)
\ex\label{ex:12:5}
<Chanca encaquehé?> \\
\gll chanca {enca} que-he? \\
husband  \textsc{prop}  \textsc{cop}-\textsc{rea}\\
\glt ‘Do you have a husband?’, “Tenéis marido?” (M\_172)
\end{exe}

Note that in (\ref{ex:12:2}--\ref{ex:12:5}) there is no interrogative marker <ni> (see \sectref{sec:15.3}).