\chapter{Adjectives and adverbs} \label{sec:13}
\largerpage
The present chapter discusses those Andakí expressions that are translated as adjectives and adverbs in the available sources. There are more examples of adverbs in attributive function than adjectives. In attributive function, adverbials are attested as modifiers of verbs in Andakí. They may precede or follow the verb, as shown in (\ref{ex:13:1}--\ref{ex:13:2}).

\begin{exe} 
\ex\label{ex:13:1}
<Andagu buxiza> \\
\gll {andagu} bu-xi-za  \\
quickly  \textsc{transl}{}-go-\textsc{imp} \\
\glt ‘Leave (\textsc{sg}) soon!’, “Vete breve” (M\_023)
\ex\label{ex:13:2}
<Bujiza andagu> / <Bujizá andagu> \\
\gll bu-ji-za {andagu} \\
\textsc{transl-}go-\textsc{imp}  quickly \\
\glt ‘Go (\textsc{sg}) quickly!’, “Anda presto” (M\_190)
\end{exe}

Andakí attributive expressions often carry morphemes such as <na{}-> or \mbox{<nan->}, <-gu>, or <-zi>. The morphemes <na-> {\textasciitilde} <nan->, and <-gu> will be discussed further below. We find final <-zi> or <-ze>, for instance, in <dacoze> ‘quickly’, “presto” (compare <andagu> ‘quickly’, “presto” in \ref{ex:13:1} above), or in color terms such as <guachuarazi> ‘yellow’, “amarillo” (M\_713), <jiszimizi> ‘blue’, “azul” (M\_714), or <fitizi> ‘red’, “colorado” (M\_715), but also in <jinszizi> ‘dirty’, “sucio” (M\_123) and other forms translated with Spanish adjectives and interrogative pronouns (see \sectref{sec:10}). The use of <-zi>, but also of <na-> {\textasciitilde} <nan->, and <-gu>, seems to follow specific rules. For instance, the Andakí modifier <qua> ‘good’ carries <-zi> when in attributive function, as shown in (\ref{ex:13:3}--\ref{ex:13:4}), and we tentatively interpret <-zi> as forming (headless) relative clauses (cf. \citealt{Gil2013}). 

\begin{exe} 
\ex\label{ex:13:3}
<Quazi mimi> \\
\gll qua-{zi} mimi  \\
good-\textsc{rel}  love \\
\glt ‘I love you (\textsc{pl}) very much’, “Mucho os quiero.”\footnote{The translation, in Ms. II/2911, but not in Ms. II/2912, is “mucho os quiere” ‘she/he/it loves you very much’ with a 3\textsuperscript{rd}-person subject. Also, in Ms. II/2911, <quazi mimi> seems to be followed by a sequence which is difficult to decipher and is probably <rr>.} (M\_222)
\ex\label{ex:13:4}
<Quazira chiyaya> \\
\gll qua-{zi} ra-chiya-ya \\
good-\textsc{rel}  \textsc{aor}{}-eat\textsc{{}-pl} \\
\glt ‘They have eaten everything’, “Todo han comido” (M\_168)
\end{exe}

By contrast, in (\ref{ex:13:5}--\ref{ex:13:7}), Andakí <qua> or <quan> ‘good’ is used predicatively, as an argument in copula complement function, and does not carry <-zi> ‘relative’. 

\begin{exe} 
\ex\label{ex:13:5}
<Quancaquehé> \\
\gll {quan} ca-que-he  \\
good  2-\textsc{cop}--\textsc{rea} \\
\glt ‘Are you (\textsc{sg}, \textsc{m}) good?’, “Estás bueno?” (M\_258; M\_267)
\ex\label{ex:13:6}
<Quahini mijinahé?> \\
\gll {qua}{}-hi     ni   mijinahe? \\
good-\textsc{rea} \textsc{q}  dog \\
\glt ‘Is your (\textsc{sg}) dog good?’, “Es bueno tu perro?” (M\_271)
\ex\label{ex:13:7}
<Quahi> \\
\gll {qua}{}-hi \\
good\textsc{{}-rea} \\
\glt ‘It is good’, “Bueno es” (M\_272)
\end{exe}

Attributively used adjectives and relative clauses are not collapsed in all instances in Andakí, however. The morpheme <qua> ‘good’, shown in (\ref{ex:13:3}--\ref{ex:13:7}) above, also appears to be present in <nanquazi> ‘beautiful’, “bonito” (M\_724); yet, the form with final <-zi> appears only when listed as a separate entry in the 18\textsuperscript{th}-century Andakí materials. As illustrated below, unlike <qua> ‘good’, <nan-qua> ‘\textsc{cop}-good’ and its variant <nin-qua> appear without final <{}-zi> not only in those examples where they have a predicative function (\ref{ex:13:8}), but also where they have an attributive function (\ref{ex:13:9}). Andakí <na-> {\textasciitilde} <nan-> {\textasciitilde} <nin-> is tentatively interpreted as a copular prefix here. It is also found in certain Andakí interrogative cleft-constructions (see \sectref{sec:10}). First, an example of <nanqua> in predicative function is shown in (\ref{ex:13:8}).

\begin{exe} 
\ex\label{ex:13:8}
<Nanqua gaqui> \\
\gll {nan{}-qua} ga-qui   \\
\textsc{cop{}-}good  2-\textsc{cop}  \\
\glt ‘You (\textsc{sg,} \textsc{fem}) are beautiful’, “Eres bonita” (M\_122) 
\end{exe}

Second, (\ref{ex:13:9}), likewise of a cleft construction, illustrates <nin-qua> ‘\textsc{cop{}-}good’ in attributive function.

\begin{exe} 
\ex\label{ex:13:9}
<Ninquaca guabi>\footnote{In Ms. II/2911, the original form is <ninquaqua guabi>, final <qua> is crossed out and <ca> is placed above it.} \\
\gll {nin{}-qua} ca-gua-bi \\
\textsc{cop{}-}good  2-say-\textsc{rea} \\
\glt ‘You (\textsc{sg}) say very well’, “Muy bien dice” (M\_209)
\end{exe}

Andakí <-cu>, by contrast, attested in the isolated entry <yazicu> ‘ugly’, “feo” (M\_650) or <yaseco> ‘bad’, “malo” (A\_162), seems to have a distribution that can be compared to that of Andakí <-zi> ‘relativizer’ in <nanquazi> ‘beautiful’. As an isolated entry in the word lists, the form in question appears with final \mbox{<-gu>} {\textasciitilde} <-co>, and it occurs without final <-gu> {\textasciitilde} <-co> both in predicative and in attributive function, as shown in (\ref{ex:13:10}) and (\ref{ex:13:11}), respectively. 

\begin{exe} 
\ex\label{ex:13:10}
<Yazin Kaqui> \\
\gll {yazin} ka-qui \\
bad  2-\textsc{cop}  \\
\glt ‘You (\textsc{sg,} \textsc{fem}) are ugly’, “Eres fea” (M\_133)
\ex\label{ex:13:11}
<Andagu quagua yazi fi-rajichi>\footnote{Hyphenation is due to a line break in the manuscript.}  / <Andagu quagua yazi firajichi> \\
\gll andagu   qua-gua {yazi} firajichi \\
quickly  \textsc{2.imp-}cook  bad  hungry  \\
\glt ‘Cook (\textsc{sg}) quickly, I am hungry’, “Cocina presto, que tengo hambre” (M\_151)
\end{exe}

Thus, it seems that <yazicu> / <yazeco> ‘ugly’, ‘bad’ and <nanquazi> ‘beautiful’ only carry the relativizing suffixes <-cu> and <-zi> if they are elicited in isolation, and occur without them in predicative and in attributive function in the available Andakí examples.

A few Andakí adverbs do not seem to occur with a final suffix <-zi> or <-cu>. This is the case for <juatuxa> ‘straight’, “derecho” (M\_670). In attributive function, this modifier appears without any further morphology, as shown in (\ref{ex:13:12}). There is no corresponding predicative construction in the available Andakí data.

\begin{exe} 
\ex\label{ex:13:12}
<Cohagea juatuxa> / <Cohajea juatuxá> \\
\gll coha-ge-a {juatuxa}  \\
\textsc{2.imp-}go-\textsc{imp}  straight \\
\glt ‘Walk (\textsc{sg}) straight!’, “Anda derecho” (M\_295)
\end{exe}

To what extent the uses of the copula prefix and of the relativizing suffixes are also governed by semantic and pragmatic principles is difficult to glean from the available materials. In the neighboring Nasa Yuwe language, there is a suffix -\textit{sa}, possibly a counterpart of Andakí <-zi> ‘relative’, which derives headless relative clauses and can be attached to color terms or verbs in this language (e.g., \citealt[144--145]{Jung2008}; \citealt[261]{DiazMontenegro2019}). The use of Nasa Yuwe \textit{{}-sa}, for instance with a color term, has been argued to add emphasis, compared with the use of the color term without -\textit{sa} \citep[148]{Jung2008}.
