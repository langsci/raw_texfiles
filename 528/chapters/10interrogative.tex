\chapter{Interrogative pronouns} \label{sec:10}

Several interrogative pronouns can be found in the available Andakí materials. In order to convey an interrogative meaning, the respective pronoun is combined with <ni> or <ne>, an apparently unbound interrogative morpheme, which will be discussed in \sectref{sec:15.3}. The formal similarities between most interrogation words suggests that they are etymologically related and derive from a common base. 

The Andakí forms <biza> / <bizá> (M\_063) and <fizi> (M\_072) are translated as ‘what?’, “qué”. Other forms translated with ‘what?', “qué" are: <ychuhizi> (M\_751), <ychuyzi> (M\_007), <yffize> (M\_164), <ypchize> / <ypchizé> (M\_303), <ychuize> / <ypchize> (M\_304); <ychuhi> / <ychuchi> (M\_269). The use of some of the forms mentioned above is illustrated in (\ref{ex:10:1}--\ref{ex:10:3}). 

\begin{exe}
\ex\label{ex:10:1}
<Fizini?> \\
\gll fizi  ni  \\
what \textsc{q} \\
\glt ‘What is it?’, “Qué es?” (M\_072)
\ex\label{ex:10:2}
\begin{xlist}
\ex\label{ex:10:2a}
<Ychuyzi Kaquani> \\
\gll {ychuyzi} ka-qua  ni \\
what    2-say  \textsc{q} \\
\glt ‘What do you (\textsc{sg}) say?’, “Qué dices?” (M\_008)
\ex\label{ex:10:2b}
<Ychuhizi caquani?> \\
\gll {ychuhizi} ca-qua  ni? \\
what    2-say  \textsc{q} \\
\glt ‘What do you (\textsc{sg}) say?’, “Qué dices?” (M\_751)
\end{xlist}
\ex\label{ex:10:3}
<Ypchize canahá quine?> / <Ypchizé canahá quiné?> \\
\gll {ypchize} ca-naha{}-qui  ne?  \\
what    2-\textsc{caus}{}-go  \textsc{q} \\
\glt ‘What have you (\textsc{sg}) brought?’, “Qué has traído?” (M\_303)
\end{exe} 

\newpage
There is no dedicated object marker identifiable in (\ref{ex:10:2}--\ref{ex:10:3}). Andakí <biza> and <fizi> ‘what?’ recur in <fizajaqui=\footnote{The symbol <=> appears at times in the manuscript without an evident function. It occurs exclusively after an Andakí form, never before. It is more frequently found in the earlier portions of the list, especially in Ms. II/2911 where it appears only in the first half, and is found only twice in Ms. II/2912.}> / <fizajaquini> (M\_035) and in <bizefazi>, <bizejazi> (M\_185) ‘what for?’, “para qué”, which is illustrated in (\ref{ex:10:4}). 

\begin{exe} 
\ex\label{ex:10:4}
<Bizefazini?> / <Bizejazini?> \\
\gll {bize-fazi} ni? \\
what-\textsc{purp}  \textsc{q} \\
\glt 'What for?’, “Para qué?” (M\_185)
\end{exe} 

The second morpheme in Andakí <bizefazi>, <bizejazi> is the purposive ending <{}-fazi> or <{}-jazi>, which is discussed in \sectref{sec:11.5}. The interpretation of <bizejazi> and related elements as compound forms is also in line with the original orthography in (\ref{ex:10:5}), where the two morphemes are separated by a blank space.

\begin{exe} 
\ex\label{ex:10:5}
<Yza jazini?> \\
\gll {yza-jazi} ni? \\
what-\textsc{purp}  \textsc{q} \\
\glt ‘What for?’, “Para qué?” (M\_146)
\end{exe} 

The interrogative <bizazi> ‘how much?’, “qué tanto?” (M\_183), is attested only once in the data, and likewise seems to contain the morpheme <biza> ‘what?’ plus a morpheme <-zi>;\footnote{The Andakí compound interrogative word <bizazi> ‘how much?’ relates to <biza> ‘what?’ in a way which is reminiscent of how Nasa Yuwe <maz> ‘how many?’ relates to the Nasa Yuwe interrogative element <m> (cf. \citealt[284--286]{DiazMontenegro2019}).} it is shown in (\ref{ex:10:6}). 

\begin{exe} 
\ex\label{ex:10:6}
<Bizazini?> \\
\gll {biza{}-zi} ni? \\
what-\textsc{ls}  \textsc{q} \\
\glt ‘How much?’, “Qué tanto?” (M\_183)
\end{exe} 

A question word <bisci> / <biszi> ‘how many?’, “cuántos” seems to occur in two further constructions (M\_175; M\_243), which contain, however, further material that is not analyzable. 

Finally, Andakí <biza> / <bizá> (M\_063) ‘what?’ is formally reminiscent of <hiza> (M\_037), <yhiza> (M\_169), and <hihiza> (M\_297) ‘why?’, “por qué”. The use of <hiza> is shown in (\ref{ex:10:7}). 

\begin{exe} 
\ex\label{ex:10:7}
<Hizani> \\
\gll {hiza} ni \\
why  \textsc{q} \\
\glt ‘Why?’, “Por qué?” (M\_037)
\end{exe} 

The interrogative pronoun <nszaji> / <nszãji> (M\_176) and its variants – \mbox{<sxaxi>} (M\_067), <nszazi> (M\_286), <hszazi> / <ñszazi> (M\_288), and <sazi> (\mbox{M\_308}) – are either translated as ‘where?’, as in (\ref{ex:10:8}--\ref{ex:10:9}), or as ‘how?’, as in (\ref{ex:10:10}).

\begin{exe} 
\ex\label{ex:10:8}
<Sxaxini?> \\
\gll {sxaxi} ni? \\
how/where \textsc{q} \\
\glt ‘Where is she/he/it?’, “Dónde está?” (M\_067)
\ex\label{ex:10:9}
<Nszajini?> \\
\gll {nszaji} ni? \\
how/where \textsc{q} \\
\glt ‘Where are they?’, “Dónde están?” (M\_176)
\ex\label{ex:10:10}
<Sazi cayuni?> \\
\gll {sazi} ca-yu ni?\\
how/where  2-come  \textsc{q} \\
\glt ‘How have you (\textsc{sg}) come?’, “Cómo has venido?” (M\_308)
\end{exe}

(\ref{ex:10:8}--\ref{ex:10:10}) seem to illustrate the colexification of ‘where?' and ‘how?' in Andakí, which is not an unusual phenomenon in South American indigenous languages. It is also found, for instance, in Guahibo (Guahiboan), Itonama (isolate), Lengua-Mascoy languages, Trinitario (Arawakan), and in Waimaja (Tucanoan) \citep{Rzymskietal.2019}. 

The interrogative root <szaja> (M\_178, M\_188) or <szafa> (M\_551), by contrast, is always translated as ‘where?’ in the available Andakí data; its use is illustrated in (\ref{ex:10:11}). 

\begin{exe} 
\ex\label{ex:10:11}
<Szajani?> \\
\gll {szaja} ni? \\
where  \textsc{q} \\
\glt ‘Where?’, “Dónde?” (M\_188)
\end{exe}

A recurrent element in several Andakí interrogatives is final <ze> or <zi>, for instance, in <ypchize> ‘what?’ (\ref{ex:10:3}) and <sazi> ‘how/where?’ (\ref{ex:10:10}). It seems to be fossilized and it remains to be established whether it can be related to <-zi> as it occurs in <rihizi> ‘this’ (see \sectref{sec:8}) and to the productive relativizing suffix <-zi> discussed in \sectref{sec:13}, which deals with Andakí expressions translated as adjectives and adverbs. 

Lastly, Andakí <qua> (M\_003) is translated as ‘who?’; its variant <qui> is illustrated in (\ref{ex:10:12}). 

\begin{exe} 
\ex\label{ex:10:12}
<Naquini> \\
\gll {na{}-qui} ni \\
\textsc{cop{}-}who  \textsc{q} \\
\glt ‘Who?’, “Quién” (M\_002)
\end{exe}

Interrogative constructions comprising Andakí <qua> or <qui> ‘who?' are quite different from the interrogative constructions discussed above: Andakí <naqua> or <naqui>, as illustrated in (\ref{ex:10:12}), is morphologically complex; we tentatively interpret <na-> as a copular prefix. This construction occurs in cleft sentences like those shown in (\ref{ex:10:13}--\ref{ex:10:14}), where <na-qua> ‘\textsc{cop{}-}who' is not followed by the interrogative marker <ni>. 

\begin{exe} 
\ex\label{ex:10:13}
<Naqua fifiquani?> \\
\gll {na-qua} fifi {qua} {ni}? \\
\textsc{cop}{}-who  call  who  \textsc{q} \\
\glt ‘Who calls you (\textsc{sg})?’, “Quién te llama?” (M\_003)
\ex\label{ex:10:14}
<Naqua naquiquani> / <Náqua náquiguani?>  \\
\gll {na-qua} na-qui {qua}  {ni}? \\
\textsc{cop}{}-who  \textsc{caus}{}-come  who  \textsc{q} \\
\glt ‘Who brought you (\textsc{sg})?’, “Quién te trajó?” (M\_202)
\end{exe} 

In a number of cases, the interrogative suffix <ni> follows <na{}-qua> ‘\textsc{cop}{}-who’ and occurs in both clauses of the cleft sentence. This is illustrated in (\ref{ex:10:15}--\ref{ex:10:17}).

\begin{exe} 
\ex\label{ex:10:15}
<Naqua niuquani?> \\
\gll {na-qua}  {ni} u {qua}  {ni}? \\
\textsc{cop{}-}who  \textsc{q}  cook  who  \textsc{q} \\
\glt ‘Who cooks?’, “Quién cocina?” (M\_004)
\ex\label{ex:10:16}
<Naqua nimaquani?>  \\
\gll {na-qua} {ni} ma {qua}  {ni}? \\
\textsc{cop{}-}who  \textsc{q}  die  who  \textsc{q} \\
\glt ‘Who dies?’, “Quién muere?” (M\_005)
\ex\label{ex:10:17}
<Naqua niquaquani?>  \\
\gll {na-qua} {ni} qua {qua}  {ni}?\\
\textsc{cop{}-}who  \textsc{q}  kill  who  \textsc{q}\\
\glt ‘Who kills?’, “Quién mata?” (M\_006)
\end{exe}

It is difficult to know to what extent the use of <ni> expresses information structure and whether the focus of the question is different in (\ref{ex:10:13}--\ref{ex:10:14}) and (\ref{ex:10:15}--\ref{ex:10:17}), directed at the action in (\ref{ex:10:13}--\ref{ex:10:14}) and at the person in (\ref{ex:10:15}--\ref{ex:10:17}). Andakí <na-qua> and <na-qui> ‘\textsc{cop{}-}who’ is found only in subject position and never in object position in the available Andakí data.
