\chapter{Further grammatical morphemes and features} \label{sec:15}

In this chapter, we will consider grammatical elements that cannot clearly be linked to the parts of speech discussed so far. They either occur with elements pertaining to different word classes, such as the plural clitic <-ya>, which can be attached to verbs and nouns, but not to pronouns (\sectref{sec:15.1}), or they are unbound forms such as the negation marker <para> (\sectref{sec:15.2}), the interrogative marker <ni> (\sectref{sec:15.3}), and the hortative expression <ynsci> (\sectref{sec:15.4}). Augmentative, diminutive, and related notions are discussed in \sectref{sec:15.5}. The absence of dedicated tense markers in Andakí is discussed in \sectref{sec:15.6}. 

\section{Plural: <{}-ya>} \label{sec:15.1}

Plural number is marked by a clitic <{}-ya>. This morpheme can be attached to verbs and nouns alike, yet not to pronouns. In verbs, there is no evidence that Andakí \mbox{<-ya>} also pluralizes 1\textsuperscript{st}- and 2\textsuperscript{nd}-person subjects, and it seems that plural number is overtly marked by <-ya> only with a 3\textsuperscript{rd}-person subject. The morpheme <-ya> marks plural number of human referents, as in (\ref{ex:15:1}).

\begin{exe}
\ex\label{ex:15:1}
<Chinta nehayaya> / <Chinta nchayaya>  \\
\gll chinta{}-ne  haya-{ya}  \\
there-\textsc{loc}  live\textsc{{}-pl} \\
\glt ‘There they live’, “Allá viven” (M\_331)
\end{exe}

Non-human, animate plural subjects also receive <{}-ya>. In (\ref{ex:15:2}), plural number is marked on both the subject and the verb.

\begin{exe}
\ex\label{ex:15:2}
<Cabiyara chiyaya quique?> / <Cabiyara chiyayá quique?> \\
\gll cabi-{ya} ra-chiya-{ya} quique?  \\
parrot-\textsc{pl}  \textsc{aor}{}-eat-\textsc{pl}  corn \\
\glt ‘Have the parrots eaten the corn?’, “Han comido el maíz los loros?” (M\_166)
\end{exe}

The morpheme <{}-ya> also attaches to <juanti> ‘big, a lot’, which in (\ref{ex:15:3}) has a human referent.

\begin{exe}
\ex\label{ex:15:3}
<Juantiya neayaya?> / <Juantiya neayayá?> \\
\gll juanti-{ya} ne  aya-{ya}? \\
big/a.lot-\textsc{pl}  \textsc{q}  live-\textsc{pl} \\
\glt ‘Are there many people in your land?’, “Hay mucha gente en tu tierra?” (M\_764)
\end{exe}

It is difficult to determine whether the marking of plurality with <{}-ya> depends on the degree of animacy or of agency of the 3\textsuperscript{rd}-person plural referent. Inanimate referents are rarely agents of an action expressed by a lexical verb, which is in line with the observation that there are no examples, in the available Andakí data, of pluralized nouns referring to inanimate entities. 

\section{Negation: <para>, <ra-…-nca>} \label{sec:15.2}

There are several ways to express negation in Andakí (see also \sectref{sec:14.5.5} on the prohibitive and \sectref{sec:14.6.2} on the non-desiderative). The most widespread negative morpheme used with both nouns and verbs is a preposed element <para> that is always represented as an unbound form in the manuscripts. In several cases, it seems to have grammaticalized and to have given rise to a negation-marking prefix or proclitic which only seems to occur with verbs, however. 

The unbound negation marker <para> has several variants, among which are <paxa> / <paxá> (M\_143) and <paga> / <pagá> (M\_310). Its use with nouns in negative nominal predicates is illustrated in (\ref{ex:15:4}) and (\ref{ex:15:5}), respectively.

\begin{exe}
\ex\label{ex:15:4}
<Paxa jizi> / <Paxá fizi> \\
\gll {paxa} jizi \\
\textsc{neg}  firewood \\
\glt ‘There is no firewood’, “No hay leña” (M\_143)
\ex\label{ex:15:5}
<Sisza pagaha>\\
\gll sisza {paga}{}-ha \\
name  \textsc{neg{}-rea} \\
\glt ‘I have no name’, “No tengo nombre” (M\_307)
\end{exe}

Although in most cases, <para> or one of its variants precedes the negated entity, as in (\ref{ex:15:4}), in a few instances it follows it, as illustrated in (\ref{ex:15:5}). The negation marker <para> and variants may also occur alone, and constitute a single utterance, as shown in (\ref{ex:15:6}). 

\begin{exe}
\ex\label{ex:15:6}
<paja>; <paga> / <pajá>; <pagá> \\
\gll paja \\
\textsc{neg} \\
\glt ‘I do not have; there is not’, “No tengo; no hay” (M\_149)
\end{exe}

In verbal negation, Andakí <para> and variants occur together with other negative morphemes, as shown in (\ref{ex:15:7}--\ref{ex:15:8}).

\begin{exe}
\ex\label{ex:15:7}
<Pará nanqueanca> / <Pará nanquetanca>\footnote{We have no explanation for <t> in this form; it may have an antihiatic function here or be an error.} \\
\gll {para} nan-que-{anca} \\
\textsc{neg}  \textsc{caus}{}-go-\textsc{neg} \\
\glt ‘I bring nothing’, “Nada traigo” (M\_737)
\ex\label{ex:15:8}
<Para riszifigua> \\
\gll {para} riszi {figua} \\
\textsc{neg}  drink  \textsc{ndes} \\
\glt ‘I do not want to drink’, “No quiero beber” (M\_049)
\end{exe}

In most cases, however, a truncated form, <ra->, occurs as a negative prefix or proclitic. Its use is shown in (\ref{ex:15:9}--\ref{ex:15:10}).

\begin{exe}
\ex\label{ex:15:9}
<Ragua chanca> \\
\gll {ra}{}-guacha-{nca} \\
\textsc{neg}{}-find-\textsc{neg} \\
\glt ‘I have not found anything’, “Nada he topado” (M\_165)
\ex\label{ex:15:10}
<Ninga rahasza ficoha> \\
\gll ninga {ra}{}-hasza {ficoha} \\
I  \textsc{neg}{}-leave  \textsc{ndes} \\
\glt ‘I do not want to leave you (\textsc{sg})’, “Yo no quiero dejarte” (M\_251)
\end{exe}

As in the case of <para> and its variants, negation of a verb with <ra-> entails the use of a negation marker <ficoa> or <-anca>. The use of the negative suffix <-anca> or <-nca> is shown, for instance, in (\ref{ex:15:7}) and (\ref{ex:15:9}) above, and in (\ref{ex:15:11}) below.

\begin{exe}
\ex\label{ex:15:11}
<Ninga januzira guanca> / <Ninga janûzira guanca> \\
\gll ninga  janu-zi {ra}{}-gua-{nca} \\
I  big/a.lot-\textsc{rel}  \textsc{neg}{}-say-\textsc{neg} \\
\glt ‘I do not ask you (\textsc{sg}) for much’, “Yo no te pido harto” (M\_181)
\end{exe}

The use of <-anca> and variants appears to be mutually exclusive with the use of non-desiderative <ficoa> and related forms, the use of which is illustrated, for instance, in (\ref{ex:15:12}). 

\begin{exe}
\ex\label{ex:15:12}
<Ramaficora> \\
\gll {ra}{}-ma {ficora}\\
\textsc{neg}{}-die  \textsc{ndes} \\
\glt ‘Do not die’, “No muráis” (M\_323)\footnote{Attested in Ms. II/2912 only. This form follows the form <ramahi> ‘she/he/it died’, “murió”.}
\end{exe}

The prefix <ra-> has a variant <za->, illustrated in (\ref{ex:15:13}); note also the variant \mbox{<-yanca>} of the negative suffix, with <y> introduced possibly for antihiatic reasons, like <t> in (\ref{ex:15:7}) above. 

\begin{exe}
\ex\label{ex:15:13}
<Buxibi firajichiza chiya yanca> / <Buxibi firajichiza chiyayanca>  \\
\gll bu-xi-bi     firajichi {za}{}-chiya-{yanca}\\
\textsc{transl}{}-go-\textsc{rea}  hungry    \textsc{neg-}eat-\textsc{neg} \\
\glt ‘I am going to eat, I am hungry’, “Voy a comer, que tengo hambre” (M\_028)
\end{exe}

In (\ref{ex:15:14}), <ca-> is probably a negation marker and a variant of <ra->.

\begin{exe}
\ex\label{ex:15:14}
<Cachiyanca> / <Cachiyancá> \\
\gll {ca}{}-chiya-{nca} \\
\textsc{neg}{}-eat-\textsc{neg} \\
\glt ‘I have not eaten’, “No he comido” (M\_045)
\end{exe}

The interaction between negative <ra-> and the 2\textsuperscript{nd}-person subject marker <ca-> are not yet fully understood; <ca-> and <ra-> do not seem to co-occur on the same word as separate prefixes, as shown in (\ref{ex:15:15}--\ref{ex:15:16}). 

\begin{exe}
\ex\label{ex:15:15}
<Rafianzanca?> \\
\gll {ra}{}-fianz-{anca}?\\
\textsc{neg}{}-like-\textsc{neg} \\
\glt ‘Why do you (\textsc{sg}) not want to?’, “Por qué no quieres?” (M\_260)\footnote{Note that there is no interrogative pronoun ‘why?’ in the Andakí example.} 
\ex\label{ex:15:16}
<Yhiza raqueanquini?> \\
\gll yhiza {ra}{}-que-{anqui} ni? \\
why  \textsc{neg}{}-come\textsc{{}-neg} \textsc{q}\\
\glt ‘Why do you (\textsc{sg}) not come?’, “Por qué no vienes?” (M\_325)
\end{exe}

In the last example, <-anqui> is interpreted as a variant of <-anca>; final <i> remains unexplained but might reflect assimilation to the high front vowel in the following interrogative morpheme <ni>. 

Finally, if it is attached to a noun, the negation prefix seems to occur without any further negation suffix, as in (\ref{ex:15:17}).

\begin{exe}
\ex\label{ex:15:17}
<Rashunguahé> \\
\gll {ra}{}-shunguahe \\
\textsc{neg}{}-ear \\
\glt ‘Stupid (\textsc{m})’, “Tonto” (M\_649)
\end{exe}

\section{Interrogative: <ni>} \label{sec:15.3}

Questions in Andakí often contain an unbound interrogative morpheme <ni>. It occurs both in polar questions and in content questions. Its use in polar questions is illustrated in (\ref{ex:15:18}). 

\begin{exe}
\ex\label{ex:15:18}
<Quahini mijinahé?> \\
\gll qua-hi {ni} mijinahe?\\
good-\textsc{rea} \textsc{q}  dog \\
\glt ‘Is your (\textsc{sg}) dog good?’, “Es bueno tu perro?” (M\_271)
\end{exe}

It is also used in content questions, as shown in (\ref{ex:15:19}).

\begin{exe}
\ex\label{ex:15:19}
<Fizini?> / <Fizini.=> \\
\gll fizi {ni}? \\
what  \textsc{q} \\
\glt ‘What is she/he/it?’, “Qué es?” (M\_072)
\end{exe}

In (\ref{ex:15:18}--\ref{ex:15:19}), the interrogative marker follows the predicate. If the construction contains only a verb, <ni> seems to occur only in initial position, as shown in (\ref{ex:15:20a}--\ref{ex:15:20b}). 

\begin{exe}
\ex\label{ex:15:20}
\begin{xlist}
\ex\label{ex:15:20a} 
<Niyuhe?> \\
\gll {ni} yu-he?\\
\textsc{q}  come-\textsc{rea} \\
\glt ‘Does she/he/it come?; Has she/he/it come?’, “Viene?; Ha venido?” (M\_156)
\ex\label{ex:15:20b}
<Niyuhe?> / <Niyuche?>  \\
\gll {ni} yu-he?\\
\textsc{q}  come\textsc{{}-rea} \\
\glt ‘Does she/he/it come?; Has she/he/it come?’, “Viene?; Ha venido?” (M\_024)
\end{xlist}
\end{exe}

In constructions that contain both a question word and a verb, <ni> seems to occur only in final position, as shown in (\ref{ex:15:21}--\ref{ex:15:23}). 

\begin{exe}
\ex\label{ex:15:21}
<Ychuize camine?> / <Ypchize caminé?> \\
\gll ychuize  ca-mi {ne}?\\
what    2-want    \textsc{q} \\
\glt ‘What do you (\textsc{sg}) want?’, “Qué quieres?” (M\_304) 
\ex\label{ex:15:22}
<Ychuyzi Kazini?> \\
\gll ychuyzi  ka-zi {ni}?\\
what    2-do  \textsc{q} \\
\glt ‘What do you (\textsc{sg}) do?’, “Qué haces?” (M\_009)
\ex\label{ex:15:23}
<Sazi ca-yu-ni?> \\
\gll sazi     ca-yu {ni}?\\
how/where  2-come  \textsc{q} \\
\glt ‘How have you (\textsc{sg}) come?’, “Cómo has venido?” (M\_308)
\end{exe}

In a number of questions, however, we find no use of <ni> at all, as illustrated in (\ref{ex:15:24}--\ref{ex:15:25}). 

\begin{exe}
\ex\label{ex:15:24}
<Rafianzanca?> \\
\gll ra-fianz-anca? \\
\textsc{neg}{}-like-\textsc{neg} \\
\glt ‘Why do you (\textsc{sg}) not want to?’, “Por qué no quieres?” (M\_260)
\ex\label{ex:15:25}
<Cafsrriaxi?> \\
\gll ca-fsrriaxi? \\
2-forget \\
\glt ‘Did you (\textsc{sg}) forget?’, “Te olvidaste?” (M\_117)
\end{exe}

Note that in (\ref{ex:15:24}), which is a content question, there is no interrogative pronoun.

\section{Hortative: <ynszi>} \label{sec:15.4}

An Andakí hortative expression is <ynsci> / <ynszi> (M\_313), <inci> (A\_067), <inszi> (M\_129). These forms are translated as ‘let us go!’, “vamos” and ‘let us go away’, “vámonos”. They often occur without further morphemes, as in (\ref{ex:15:26a}--\ref{ex:15:26b}).

\begin{exe}
\ex\label{ex:15:26}
\begin{xlist}
\ex\label{ex:15:26a} 
<Ynszi> / <Ynsci> \\
\gll ynszi \\
let.us.go \\
\glt ‘Let us go away!’, “Vámonos” (M\_313)
\ex\label{ex:15:26b}
<Inci> \\
\gll inci \\
let.us.go \\
\glt ‘Let us go!’, “Vamos” (A\_067)
\end{xlist}
\end{exe}

In a few cases, the imperative marker <-za> is attached to <ynszi> ‘let us go!’, as shown in (\ref{ex:15:27}). 

\begin{exe}
\ex\label{ex:15:27}
<Ynsziza> \\
\gll {ynszi}{}-za \\
let.us.go-\textsc{imp} \\
\glt ‘Let us go!’, “Vamos” (M\_022)
\end{exe}

The most frequent construction in the data is <ynszi> ‘let us go!’ + verb/noun + <-ra> ‘allative’. The form glossed as ‘let us go!’ always occupies the first position in the sentence. In (\ref{ex:15:28}), the complement is a verb, while in (\ref{ex:15:29}), the complement is a noun.

\begin{exe}
\ex\label{ex:15:28}
<Ynsci fichara> / <Ynsci fichará> \\
\gll {ynsci} ficha-ra \\
let.us.go  brush-\textsc{all} \\
\glt ‘Let us go brush!’, “Vamos a rozar” (M\_328)
\ex\label{ex:15:29}
<Ynszi zota jera> / <Ynszi zotajera> \\
\gll {ynszi} zotaje-ra \\
let.us.go  mountain\textsc{{}-all} \\
\glt ‘Let us go to the forest!’, “Vamos al monte” (M\_253)
\end{exe}

<Ynszi> (and variants) and the imperative with \textit{{}-(a)ba} may also occur in a single (complex) sentence, as shown in (\ref{ex:15:30}), where they are found in two separate syntactic units.

\begin{exe}
\ex\label{ex:15:30}
<Ynszi yahara nâjubahá> \\
\gll {ynszi} yaha-ra  naju-{baha}\\
let.us.go  river-\textsc{all}  wash-\textsc{hort}   \\
\glt ‘Let us go to the river to wash ourselves!’ / ‘Let us go to the river, let us wash!’, “Vamos al río a lavarnos” (M\_254)
\end{exe}

\section{Augmentative, diminutive, and related notions} \label{sec:15.5}

Among the grammatical morphemes that may tentatively be identified in Andakí adjectives and nouns are those expressing augmentative, diminutive, and related notions. A suffix <-pi>, <-pihi>, or <-nipihi> and variants may have an augmentative meaning, and occur in forms such as <jushuampihi> / <jushuanipihi> ‘big’, “gordo” (M\_700), <cupihi> ‘major’, “mayor” (M\_698), <acozinipihi> ‘wicked’, “bellaco” (M\_582), <quasinimpihi> ‘good’, “bueno” (M\_583),\footnote{This form contains an element <qua> or <quasi> ‘good’, discussed further above.}  but also in nouns such as <chupihi> ‘grandfather’, “abuelo” (M\_488). 

The diminutive counterpart <-bi> or <-bihi> may be attested in forms such as <chibihi> ‘minor’, “menor” (M\_699)\footnote{Attested in Ms. II/2912 only.}  and in nouns such as <cabihi> ‘parrot’, “loro” (M\_600) and <rimbihi> ‘turtledove’, “tortola” (M\_607). 

\section{Tense} \label{sec:15.6}

There is no evidence for either verbal tense or nominal tense marking in the available Andakí data, and it is difficult to know how notions of nominal or verbal tense were indicated at all. For instance, (\ref{ex:15:31}) is translated with past tense (\textit{indefinido}) in Spanish, although there is no dedicated past tense marker; the only identifiable grammatical morphemes in (\ref{ex:15:31}) are <ca-> ‘2\textsuperscript{nd}-person subject’ and <-he> ‘realis mood’.  

\begin{exe}
\ex\label{ex:15:31}
<Caxihe?> / <Caxihé?> \\
\gll ca-xi-he? \\
\textsc{2}{}-go-\textsc{rea} \\
\glt ‘Did you go?’, “Fuisteis?” (M\_040)
\end{exe}

A form which is unmarked for tense, however, does not necessarily imply a past action or event, as shown in (\ref{ex:15:32}). 

\begin{exe}
\ex\label{ex:15:32}
<Ynszaxa caxini> \\
\gll ynszaxa  ca-xi  ni \\
where    2-go  \textsc{q} \\
\glt ‘Where do you (\textsc{sg}) go?’, “Dónde vas” (M\_059)
\end{exe}

In some cases, time reference may be expressed by adverbials. For instance, (\ref{ex:15:33}), which contains the term for ‘tomorrow’, is translated with future tense in Spanish.

\begin{exe}
\ex\label{ex:15:33}
<Ninga gozeha> / <Ninga gohezá> \\
\gll ninga  gozeha \\
I  tomorrow\\
\glt ‘I will go’, “Yo iré” (M\_276)\footnote{Compare the element <gozeha> ‘tomorrow’, “mañana” in (\ref{ex:15:33}) with the single entry \mbox{<nagosexa>} ‘tomorrow’, “mañana” (M\_472). An unidentified prefix is also attested in the single entry \mbox{<canajisexa>} ‘yesterday’, “ayer” (M\_470), as opposed to <najize> ‘yesterday’, “ayer” (M\_108).}
\end{exe}

Note that it is the verb ‘to go' which is omitted in (\ref{ex:15:33}), similar to what happens in (\ref{ex:14:93}) in \sectref{sec:14} above, repeated here as (\ref{ex:15:34}).

\begin{exe}
\ex\label{ex:15:34}
<Quixarani> \\
\gll quixara-ni \\
far-\textsc{imp} \\
\glt ‘Let us go far away!', "Vamos lejos" (M\_736)
\end{exe}

We suggest that verb ellipsis is pragmatically licensed in (\ref{ex:15:33}) and (\ref{ex:15:34}), and that the missing element could originally be inferred from the context. Verb root ellipsis is an infrequent phenomenon in the languages of the world. In South America, it has been observed to occur in Kwaza, a language isolate of Brazil (see \citealt{ComrieZamponi2019}). 