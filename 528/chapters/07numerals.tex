\chapter{Numerals} \label{sec:7}

Only a few numerals have been documented for Andakí. They are presented and discussed in the present chapter. The numerals are shown in \tabref{tab:7.1}.

\begin{table}
\begin{tabularx}{\textwidth}{Ql}
\lsptoprule
{\bfseries Andakí numeral} & {\bfseries English, Spanish}\\
\midrule
{ <guhigo>} & { ‘one’, “uno” (M\_624)}\\
{ <nanszihisze>} & { ‘two’, “dos” (M\_625)}\\
{ <nanszihisze haniguhe> / <nanszihisze haniguhé>}  & { ‘three’, “tres” (M\_626)}\\
{ <saquan cuacho aqua> / <saquan cuachoaqua>} & { ‘five’, “cinco” (M\_627)}\\
\lspbottomrule
\end{tabularx}
\caption{Andakí numerals}
\label{tab:7.1}
\end{table}

These numerals are all morphologically complex: <guhigo> ‘one’ contains a morpheme that is probably related to <guuhe> / <guuhé> ‘other’, “otro” \mbox{(M\_016)} and an unknown ending <-go>. A similar case of colexification of ‘one’ and ‘other’ occurs in Totoro (Barbacoan) and Ika (Chibchan) \citep{Rzymskietal.2019}, two languages that are likewise spoken in northwestern South America. 

The element <szihi> in the term for ‘two’, <nanszihisze>, may be compared to <szifi> ‘eye’, “ojos” (M\_423). A similar case of ‘two’ derived from ‘eye’ is found in some Nadahup languages (cf. \citealt{Epps2006}), likewise from northwestern South America. 

The numeral <nanszihisze haniguhe> ‘three’ is built on <nanszihisze> ‘two’ and an element <haniguhe>, which is probably analyzable as <hani-guhe> and may contain the root for ‘other’ or ‘one’, <guuhe>. 

Note the absence of a numeral ‘four' in the available Andakí materials. The numeral ‘four' can have a particular position in South American counting systems (for details, see, e.g., \citealt{Epps2006}; \citealt{Pache2018b}).

Finally, <saquan cuacho aqua> or <saquan cuachoaqua> ‘five’ may contain a morpheme that is related to or identical with <sacahá> ‘hands’, “manos” (M\_431), <sacaà> ‘hand’, “mano” (A\_121). 
