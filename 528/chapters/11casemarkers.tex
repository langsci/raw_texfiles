\chapter{Case markers and postpositions} \label{sec:11}

The present chapter is dedicated to Andakí case markers and postpositions. Although we transcribe all these grammatical morphemes with a hyphen, there is not always sufficient information that would allow us to clearly determine their status as suffixes, enclitics, or free forms. In {\sectref{sec:11.1}}, we suggest an interpretation of \mbox{<-xa>} {\textasciitilde} <{}-ja> as an oblique case marker. {\sectref{sec:11.2}} will argue in favor of \mbox{<-na>}, \mbox{<-uza>}, and <-ahé> as genitive markers. In {\sectref{sec:11.3}}, we discuss locative \mbox{<-ni>}, in {\sectref{sec:11.4}}, allative <-ra>, and in {\sectref{sec:11.5}}, purposive <-xaré> {\textasciitilde} <-jazi>. The character of these elements as bound or unbound morphemes is not always easy to determine.

\section{Oblique: <-xa> {\textasciitilde} <-ja>} \label{sec:11.1}

The oblique marker <-xa> {\textasciitilde} <-ja> can be attached to personal pronouns and nouns alike. Its use with personal pronouns is illustrated in (\ref{ex:11:1}--\ref{ex:11:3}). Its use might add emphasis, as suggested by the Spanish translations in (\ref{ex:11:2}) and (\ref{ex:11:3}). 

\begin{exe} 
\ex\label{ex:11:1}
<Ricaxa> \\
\gll rica-{xa} \\
you-\textsc{obl} \\
\glt ‘To you (\textsc{sg})’, “A vos, o a ti” (M\_017)
\ex\label{ex:11:2}
<Ningaxa riguantoza>  \\
\gll ninga-{xa} riguanto-za \\
I-\textsc{obl} believe-\textsc{imp} \\
\glt ‘Believe (\textsc{pl}) me!/It is me you (\textsc{pl}) should believe’, “Creedme a mí” (M\_348)
\ex\label{ex:11:3}
<Ricaxa fifihe> / <Ricaxa fifihé>  \\
\gll rica-{xa} fifi-he\\
you-\textsc{obl} call-\textsc{rea}\\
\glt ‘It is you (\textsc{sg}) she/he/it calls’, “A vos te llama” (M\_193) 
\end{exe}

There seems to be differential object marking depending on the animacy of the referent: <-xa> attaches only to animate patients. In (\ref{ex:11:4}--\ref{ex:11:5}), the animate patients <verayue> ‘rabbit’ and <nunszue> ‘fish’ are marked as direct objects.

\begin{exe} 
\ex\label{ex:11:4}
<Nunquiza verayueja> / <Nunquizá veraýueja> \\
\gll nunqui-za  verayue-{ja} \\
look-\textsc{imp} rabbit{{}-}\textsc{obl} \\
\glt ‘Look (\textsc{sg}) at the rabbit!’, “Mira el conejo” (M\_191) 
\ex\label{ex:11:5}
<Nunquiza nunszueja>  \\
\gll nunqui-za  nunszue-{ja}\\
look-\textsc{imp} fish{{}-}\textsc{obl} \\
\glt ‘Look (\textsc{sg}) at the fish!’, “Mira el pez” (M\_192) 
\end{exe}

Interrogative pronouns referring to inanimate entities (\ref{ex:11:6}) and nouns referring to inanimate entities such as food (\ref{ex:11:7}), drinks (\ref{ex:11:8}), and clothes (\ref{ex:11:9}), do not appear to receive the morpheme <-xa> {\textasciitilde} <-ja> in object position. 

\begin{exe} 
\ex\label{ex:11:6}
<Ychuize camine?> / <Ypchize caminé?> \\
\gll ypchize  ca-mi    ne? \\
what 2-want \textsc{q} \\
\glt ‘What do you (\textsc{sg}) want?’, “Qué quieres?” (M\_304) 
\ex\label{ex:11:7}
<Chiya nenguihi=> / <Chiya nênguihi> \\
\gll chiya   nenguihi \\
eat  fish \\
\glt ‘I want to eat fish’, “Quiero comer pescado” (M\_082) 
\ex\label{ex:11:8}
<Andagu biguaza bacoxe> / <Andagu biquaza bacoxé>  \\
\gll andagu  bi-gua-za  bacoxe\\
quickly  \textsc{rap}{}-make{}-\textsc{imp}  chicha \\
\glt ‘Make (\textsc{sg}) \textit{chicha} quickly!’, “Haz presto chicha” (M\_150)
\ex\label{ex:11:9}
<Quanca jehá sasza> / <Quanajeha sasza> \\
\gll quan{}-caje-ha    sasza \\
\textsc{2.imp-}wash-\textsc{imp}  clothes \\
\glt ‘Wash (\textsc{sg}) the clothes!’, “Lava la ropa” (M\_210) 
\end{exe}

Compare the direct object ‘fish’ in (\ref{ex:11:5}) and (\ref{ex:11:7}) above: in (\ref{ex:11:5}), the term for fish refers to a living entity and receives object marking, whereas in (\ref{ex:11:7}), it refers to fish in terms of food and the noun in object position does not carry any object marking. In some cases, however, <-xa> seems to be missing with animate direct objects, too; that is, the use of this direct object marker does not seem to be obligatory, at least with pronouns. This is illustrated in (\ref{ex:11:10}).

\begin{exe} 
\ex\label{ex:11:10}
<Ninga ca-mimi?> \\
\gll ninga   ca-mimi? \\
I  2-like \\
\glt ‘Do you love me?’, “Me queréis?” (M\_232) 
\end{exe}

It seems that <-xa> {\textasciitilde} <-ja> also has other functions than marking the direct object: with the verb <caya> ‘to sit; to live’ and its variants, for instance, it appears to express a comitative meaning (‘with’), as in (\ref{ex:11:11}--\ref{ex:11:12}).

\begin{exe} 
\ex\label{ex:11:11}
<Cafianzuhi ningaxa ca{}-yasesza?>\footnote{Hyphenation is due to a line break in the manuscript.} / <Cafianzuhi ningaxá cayasesza?> \\
\gll ca-fianzu-hi  ninga-{xa} caya-sesza? \\
2-like-\textsc{rea}  I-\textsc{obl} live-\textsc{des} \\
\glt ‘Do you (\textsc{sg}) want to live with me?’, “Quieres vivir conmigo?” (M\_353)
\ex\label{ex:11:12}
<Nincaxa coaya> \\
\gll ninca-{xa} coaya \\
I-\textsc{obl} live.\textsc{imp} \\
\glt ‘Live (\textsc{sg}) with me!’, “Vive conmigo”\footnote{(\ref{ex:11:12}) is tentatively interpreted as an imperative construction, given that we also find the single form of <coaya> / <coayá> ‘sit down (\textsc{sg})!’, “siéntate” (M\_088). Yet, neither the Spanish translation nor the Andakí forms allow it to clearly distinguish this construction from a construction involving a 3\textsuperscript{rd}-person subject and indicative mood.} (M\_223) 
\end{exe}
  
The morpheme <-xa> is translated as ‘about’, “de” in (\ref{ex:11:13}), and may also have been used to convey meanings such as conversation topic. 

\begin{exe} 
\ex\label{ex:11:13}
<Nszazi raguayani xin{}-gaxá?>\footnote{The hyphen is due to a line break in the manuscript.}  / <Nszâzi raguayani xingaxá?> \\
\gll nszazi    ra{}-gua-ya  ni  xinga-{xa}? \\
how/where  \textsc{aor}{}-say-\textsc{pl}  \textsc{q}  I{}-\textsc{obl} \\
\glt ‘What do they say about me?’, “Qué dicen de mí?” (M\_758) 
\end{exe}

Usually, terms that refer to feelings seem to take an unmarked subject; note that there are only examples of such constructions with a 1\textsuperscript{st}-person subject. Two cases are illustrated in (\ref{ex:11:14}--\ref{ex:11:15}).

\begin{exe} 
\ex\label{ex:11:14}
<Ninga firajichi> \\
\gll ninga  firajichi \\
I  hungry \\
\glt ‘I am hungry’, “Yo tengo hambre” (M\_029) 
\ex\label{ex:11:15}
<Ninga finajuche> \\
\gll ninga  finajuche \\
I  thirsty \\
\glt ‘I am thirsty’, “Yo tengo sed”\footnote{The terms for ‘thirsty’ and ‘hungry’ are almost certainly polymorphemic. For instance, the element <juche> in <finajuche> ‘thirsty’ may have a counterpart <xuchi> in <rúxuchi-hi> / <ruxuhi-hi> ‘I am cold’, “tengo frío” (M\_140; hyphenation added by the authors). The ending <-che> or <-chi> recurs in forms referring to emotions and to hunger, thirst, cold and similar feelings and is not further analyzed here.} (M\_030)
\end{exe}

There is also one instance in the data, however, where the experiencer, <ninga> ‘I’ once more carries the oblique marker <-xa> {\textasciitilde} <-ja> in such a construction (\ref{ex:11:16}).

\begin{exe} 
\ex\label{ex:11:16}
<Najanszichi ningaxa> \\
\gll najanszichi  ninga-{xa} \\
angry    I-\textsc{obl}\\
\glt ‘I (\textsc{m}) am furious’, “Yo estoy rabioso” (M\_333) 
\end{exe}

The use of oblique <-xa> in (\ref{ex:11:16}) remains unexplained; as a difference from (\ref{ex:11:14}) and (\ref{ex:11:15}), the pronoun referring to the experiencer follows the verb in (\ref{ex:11:16}), whereas it precedes it in (\ref{ex:11:14}) and (\ref{ex:11:15}); the use of <-xa> in (\ref{ex:11:16}) might be related to this different word order. 

\section{Genitive: <-na>, <-azu>, <-ahe>} \label{sec:11.2}

Used with pronouns, the Andakí morpheme <-na> functions as a genitive or possessive marker. This is shown in (\ref{ex:11:17}). 

\begin{exe} 
\ex\label{ex:11:17}
<Nszajini jixena ricaná?> \\
\gll nszaji    ni  jixena  rica-{na}? \\
how/where  \textsc{q}  land  you-\textsc{pronposs}\\
\glt ‘What is your (\textsc{sg}) land?’, “Cuál es tu tierra?” (M\_339)
\end{exe}

In (\ref{ex:11:18}), the 1\textsuperscript{st}-person pronoun is followed both by the genitive marker <-na> and the allative morpheme <-ra>.

\begin{exe} 
\ex\label{ex:11:18}
<Ynszi jixena ninganara> \\
\gll ynszi    jixena  ninga-{na}{}-{ra} \\
let.us.go  land  I-\textsc{pronposs{}-all} \\
\glt ‘Let us go to my land!’, “Vamos a mi tierra” (M\_338)
\end{exe}

The genitive marker <-na> seems to occur in possessive pronouns only. To what extent it is related to the locative nominalizer <-na> (see the introduction of \sectref{sec:14}) remains to be established. 

We found two examples in the data with a regular noun referring to the possessor. Here as well, the form referring to the possessor follows the noun referring to the possessed entity — yet, the possessor is marked by <-azu> in (\ref{ex:11:19}).  

\begin{exe} 
\ex\label{ex:11:19}
<Guasu nosehazu>  \\
\gll guasu  noseh-{azu} \\
egg  louse-\textsc{gen} \\
\glt ‘Nits’, “Liendres” (M\_689) 
\end{exe}

The term for ‘chicken egg’, “huevo de gallina” is given elsewhere as <huasho> (M\_674), obviously the same term as <guasu> in (\ref{ex:11:19}), the term for ‘lice’, “piojos” is <nozihi> (M\_688). In (\ref{ex:11:20}), no genitive marker can be securely identified. The separate entry for ‘cow’ is <guacaré> (M\_096), from Spanish “vaca” plus an unknown, possibly lexicalized suffix <-re>; glossing is tentative in (\ref{ex:11:20}). 

\begin{exe} 
\ex\label{ex:11:20}
<Chiguahé guacarahe> / <Chiguahé guacarahé>  \\
\gll chiguahe  guaca{}-r{}-{ahe} \\
offspring  cow{}-\textsc{ls}{}-\textsc{gen} \\
\glt ‘Calf’, “Becerrito” (M\_655) 
\end{exe}

We tentatively interpret the final element <{}-ahe> in <guacarahé> ‘of the cow’ as a genitive marker. In all these possessive constructions, the noun referring to the possessed entity precedes the noun or pronoun that refers to the possessor. 

\section{Locative: <-ni>} \label{sec:11.3}

In \sectref{sec:8}, it was suggested that <-ni> in <rini> ‘here’, “aquí” (M\_553) and <chini> ‘there’, “allí” (M\_554) has a locative meaning. The use of <rini> and <chini> is illustrated in (\ref{ex:11:21}--\ref{ex:11:22}). To what extent the use of <-ni> ‘locative’ is restricted to these demonstratives or was productive in 18\textsuperscript{th}{}- and 19\textsuperscript{th}{}-century Andakí is impossible to determine for lack of data. Note the different position of <rini> and <chini> in the two examples.

\begin{exe} 
\ex\label{ex:11:21}
<Coayaza rini> / <Coyazarini> \\
\gll coaya-za   ri{}-{ni} \\
sit\textsc{{}-imp}  \textsc{prox{}-loc} \\
\glt ‘Sit (\textsc{sg}) here!’, “Siéntate aquí” (M\_189) 
\ex\label{ex:11:22}
<Chini caxini>  \\
\gll chi{}-{ni} ca-xi  ni \\
\textsc{dist{}-loc}  2\textsc{{}-}go  \textsc{q} \\
\glt ‘Which way are you (\textsc{sg}) going?’, “Por dónde vas?” (M\_092) 
\end{exe}

In (\ref{ex:11:23}), the adverb <chinta> / <chintá> ‘there’, “allá” (M\_177; M\_552) seems to take a variant <-ne> of this locative marker.

\begin{exe} 
\ex\label{ex:11:23}
<Chinta nechayaya> / <Chinta nhayaya> \\
\gll chinta{}-{ne} haya-ya  \\
that-\textsc{loc}  live\textsc{{}-pl}  \\
\glt ‘There they live’, “Allá viven”\footnote{We have no explanation for the form <haya> as a variant of <caya> ‘to live’. In entry \mbox{(M\_764)}, we find the variant <aya> ‘to live’ (\ref{ex:15:3}, \sectref{sec:15.1}), and in entry (M\_754), we find <zaya> ‘to remain’.} (M\_331)
\end{exe}

\section{Allative: <-ra>} \label{sec:11.4}

A frequently attested Andakí grammatical morpheme is <{}-ra> ‘allative’. It is formally similar to Nasa Yuwe allative -\textit{na} (see \citealt[496]{DiazMontenegro2019}).\footnote{\label{footnote:11:6}For a correspondence of Andakí <r>: Nasa Yuwe /n/, compare also Andakí <ri-> ‘proximate demonstrative’ and Nasa Yuwe \textit{na} ‘proximate demonstrative’; see \sectref{sec:8}.} This morpheme marks the complement of the verb ‘to go’ and may be attached to nouns and verbs alike, as shown in (\ref{ex:11:24}--\ref{ex:11:26}) , which suggests a clitic status of this morpheme.

\begin{exe} 
\ex\label{ex:11:24}
<Jibi yahara> / <Jibi yahàra> \\
\gll ji-bi    yaha{}-{ra} \\
go-\textsc{rea}    river{}-\textsc{all} \\
\glt ‘I am going to the river’, “Me voy al río” (M\_292) 
\ex\label{ex:11:25}
<Jihiza chiyara> / <Jihiza chiyará>  \\
\gll jihi-za  chiya{}-{ra} \\
go-\textsc{imp}  eat{}-\textsc{all} \\
\glt ‘Go (\textsc{sg}) eat!’, “Anda come” (M\_317) 
\ex\label{ex:11:26}
<Jibi chihizerá> / <Jibi chihizera>  \\
\gll ji-bi  chihize{}-{ra} \\
go-\textsc{rea}  urinate{}-\textsc{all} \\
\glt ‘I am going to urinate’, “Voy a orinar” (M\_753)
\end{exe}

Since grammaticalization from allative to complementizer is quite widespread \citep[37]{HeineKuteva2002}, Andakí <{}-ra> is discussed here, in the chapter dealing with case markers and postpositions, even if it also occurs with verbs, as in (\ref{ex:11:25}--\ref{ex:11:26}). Besides to the complements of <ji> {\textasciitilde} <jihi> ‘to go’, as in (\ref{ex:11:24}--\ref{ex:11:26}), <{}-ra> is also attached to the complements of other motion verbs such as <buji> ‘to go away’ and <raje> ‘to bring’, both containing the root <ji> ‘to go’. These cases are shown in (\ref{ex:11:27}--\ref{ex:11:28}).

\begin{exe} 
\ex\label{ex:11:27}
<Ninga bujibi cogora> \\
\gll ninga   bu-ji-bi     cogo{}-{ra} \\
I  \textsc{transl}{}-go-\textsc{rea}  house{}-\textsc{all} \\
\glt ‘I want to go home’, “Yo quiero irme a casa” (M\_291)
\ex\label{ex:11:28}
<Quarajea cogora>  \\
\gll qua-ra-je-a    cogo{}-{ra} \\
\textsc{2.imp-caus}{}-go-\textsc{imp}  house{}-\textsc{all} \\
\glt ‘Bring (\textsc{sg)} me to your (\textsc{sg}) house!’, “Llévame a tu casa” (M\_771) 
\end{exe}

Allative <{}-ra> also follows the complement of <ynszi> ‘let us go’, as shown in (\ref{ex:11:29}--\ref{ex:11:30}). 

\begin{exe} 
\ex\label{ex:11:29}
<Inszijicora> \\
\gll inszi    jico-{ra} \\
let.us.go  church-\textsc{all} \\
\glt ‘Let us go to the church!’, “Vamos a la iglesia” (M\_128)
\ex\label{ex:11:30}
<Inszi rezara> / <Inszirezara>  \\
\gll inszi    reza-{ra} \\
let.us.go  pray-\textsc{all} \\
\glt ‘Let us pray!’, “Vamos a rezar” (M\_129) 
\end{exe}

Allative <-ra> might also be present in forms like <junquera> ‘upwards’, “arriba” (M\_568) and <canara> ‘downwards’, “abajo” (M\_570). In the previous examples illustrating the uses of allative <-ra>, the verb always precedes the allative complement. This order is not fixed, however, as shown in (\ref{ex:11:31}). 

\begin{exe} 
\ex\label{ex:11:31}
<Fichanara rajihi> / <Fichanará rajichi> \\
\gll ficha-na{}-{ra} ra-ji-hi \\
brush-\textsc{loc{}-all}  \textsc{aor{}-}go-\textsc{rea} \\
\glt ‘She/he/it left for the field’, “Se fue a la chagra” (M\_161) 
\end{exe}

\section{Purpose: <-xare> {\textasciitilde}} \label{sec:11.5}

Andakí has a purpose marker <-xare> or <-jazi>, which is attested, in one case, after an interrogative pronoun (\ref{ex:11:32}). 

\begin{exe} 
\ex\label{ex:11:32}
<Bizejazini?> / <Bizefazini?>  \\
\gll bize-{jazi} ni? \\
what-\textsc{purp} \textsc{q} \\
\glt ‘What for?’, “Para qué?” (M\_185)
\end{exe}

In all other cases discussed here, the purpose marker is attached to a verb and it may therefore have a clitic status. It is notwithstanding discussed in the chapter on case markers and postpositions, since purpose markers often derive from them (for instance, from benefactive markers, \citealt[335]{HeineKuteva2002}). Andakí <-jazi> has several variants, for instance, <-fanxe> / <-janxe> (M\_728), <-jachi> (M\_077), <-jahe> / <-jahé> \mbox{(M\_204)}, <-jaxi> (M\_078), <jaze> (M\_147), <-jazo> (M\_114), <-xaré> / <-xazé> (M\_186). The underlying phonological form is difficult to determine; it may have been /-hare/ or /-xare/. Some of the variants mentioned above are shown in (\ref{ex:11:33}--\ref{ex:11:36}). 

\begin{exe} 
\ex\label{ex:11:33}
\begin{xlist}
\ex\label{ex:10:33a}
<Chiya jaze> \\
\gll chiya-{jaze} \\
eat{}-\textsc{purp} \\
\glt ‘In order to eat’, “Para comer” (M\_147)
\ex\label{ex:11:33b}
<Choya xaré> / <Choyaxazé> \\
\gll choya-{xare} \\
eat{}-\textsc{purp} \\
\glt ‘In order to eat’, “Para comer” (M\_186)
\end{xlist}
\ex\label{ex:11:34}
<Riszifanxe> / <Riszijanxe> \\
\gll riszi-{fanxe} \\
drink-\textsc{purp} \\
\glt ‘In order to drink’, “Para beber” (M\_728)
\ex\label{ex:11:35}
<Quananqueha bacuchiriszijachi> / <Quananquehá Bacuchiriszijachi>  \\
\gll qua-nan{}-que-ha   bacuchi  riszi-{jachi} \\
\textsc{2.imp-caus}{}-come-\textsc{imp}  \textit{mazato}    drink-\textsc{purp}\\
\glt 
‘Bring (\textsc{sg}) \textit{mazato} in order to drink!’, “Trae mazato para beber” (M\_077)
\protectedex{
\ex\label{ex:11:36}
<Bacoxe quananquehá riszijaxi> / <Bacoxe quananqueha riszijaxi.=> \\
\gll bacoxe    qua-nan{}-que-ha   riszi-{jaxi}\\
\textit{chicha}    \textsc{2.imp-caus}{}-come-\textsc{imp}  drink-\textsc{purp} \\
\glt ‘Bring (\textsc{sg}) \textit{chicha} in order to drink!’, “Trae chicha para beber” (M\_078)
}
\end{exe}

The purposive construction stands at the end of the utterance in (\ref{ex:11:35}--\ref{ex:11:36}), yet the order of verb and object varies in the main clause: VO in (\ref{ex:11:35}) and OV in (\ref{ex:11:36}). 

Finally, the purposive marker does not only indicate purpose, as in (\ref{ex:11:33}--\ref{ex:11:36}) but also has a nominalizing function in some contexts, deriving the word ‘beverage’, as shown in (\ref{ex:11:37}).

\begin{exe} 
\ex\label{ex:11:37}
<Riszifaxi> / <Riszijaxi>\\
\gll riszi-{faxi} \\
drink-\textsc{purp}\\
\glt ‘Beverage’, “Bebida” (M\_116)
\end{exe}

In Andakí <chiyazi> ‘food’, “comida” (M\_041), derived from <chiya> ‘to eat’, “comer” (M\_082), <-yazi> is reduced to <-zi>, probably in order to avoid a sequence \textit{iaia} in this case.