\chapter{Introduction} \label{sec:1}

This book provides a first description of the grammar of Andakí (Glottocode: anda1286, \textsc{iso} 639-3 code: ana), also known as Andaki, Andaqui, or Andaquí, a little-known, extinct language of southern Colombia, which is possibly an isolate. The description presented here is based on the three primary linguistic sources available for this language: two anonymous word lists from the late 18\textsuperscript{th} and one from the mid-19\textsuperscript{th} century. This book is organized as follows: we first provide some cultural and historical background information (\chapref{sec:2}), discuss the sources and previous works on Andakí (\chapref{sec:3}), and methodological issues in the investigation of Andakí grammar (\chapref{sec:4}). This is followed by a chapter on the phonology and spelling of Andakí (\chapref{sec:5}) and, as the main contribution of this volume, a discussion of Andakí morphosyntax: we address the grammar of nouns (\chapref{sec:6}), numerals (\chapref{sec:7}), demonstratives (\chapref{sec:8}), personal pronouns (\chapref{sec:9}), interrogative pronouns (\chapref{sec:10}), case markers and postpositions (\chapref{sec:11}), the expression of ownership/predicative possession (\chapref{sec:12}), Andakí attributive expressions, translated as adjectives and adverbs in Spanish (\chapref{sec:13}), the morphosyntax of verbs (\chapref{sec:14}) and grammatical morphemes that cannot be connected with any specific word class (\chapref{sec:15}). Since a most relevant question with respect to Andakí is whether a relationship with another language or language family of the Americas can be demonstrated, we will primarily concentrate on the discussion of individual grammatical morphemes, which in a second step await systematic comparison with their equivalents in other languages of South and Central America. The main findings of this book are discussed in the concluding summary (\chapref{sec:16}).
