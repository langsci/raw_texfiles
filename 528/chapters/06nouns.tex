\chapter{Nouns} \label{sec:6}
\largerpage
Most single-word entries in the Andakí language materials are nouns, that is, forms that belong to an open class of items. They may refer to concrete entities, function as the argument of a predicate and as the head of a noun phrase, and take specific derivational morphemes, case markers, or specific plural markers (see, e.g., \citealt[39, 50]{Dixon2010}). Nouns are particularly prominent in \citeapo{Albis1860--1861} list, which almost exclusively contains nouns. In the available Andakí language materials, nouns include basic vocabulary items, such as <ynquifi> ‘nose’, “nariz” (M\_447), <sacaà> ‘hand’, “mano” (A\_121), or <sacanaji> ‘knee’, “rodilla” (A\_122). Others belong to cultural vocabulary, for instance, <batonafi> ‘plate’, “plato” (A\_008) or <bujucahá> (M\_479), <bojoca> (A\_010), both ‘ax’, “hacha”. 

Many Andakí nouns contain a final element <hi>, <he>, or <e> (cf. \citealt[140]{AdelaarMuysken2004}), which we tentatively interpret as a stem formative. It occurs, for instance, in <mixinehi> ‘dog’, “perro” (M\_099), <shuntahé> ‘deer’, “venado” (M\_657), <batihi> ‘gourd’, “mate” (M\_111), <muizihi> / <miszihi> ‘man, human being’, “hombre” (M\_459), <kaquihi> ‘sun’, “sol” (M\_463), <chiguae> ‘son’, “hijo” (A\_032), but not in forms with female referents (see \tabref{tab:6.1} below). This final element has the same shape as the realis mood morpheme <-hi> and its variants (see \sectref{sec:14.5.1}), but is probably not identical with it, as shown in (\ref{ex:6:1}), where we find <-hi> ‘realis mood’ co-occurring with the presumed stem formative \mbox{<-he>} in the same sentence (albeit in distinct syntactic units).\footnote{An anonymous reviewer suspects that the nominal element <-he> and its allomorphs have structural significance and observes that in Tinigua a similar element -\textit{he} functions as a verbal suffix, though its precise role remains unclear.} Given the uncertain status of this element, we prefer not to gloss it separately.


\begin{exe}
\ex\label{ex:6:1}
<Quahini mijinahé?> \\
\gll qua-hi ni mijinahe? \\
good\textsc{{}-rea}  \textsc{q} dog \\
\glt ‘Is your (\textsc{sg}) dog good?’, “Es bueno tu perro?” (M\_271)
\end{exe}

In other nouns, there seems to be an element <-na>, which may have referred to longish shape originally.\footnote{An anonymous reviewer points out the similarity between the Andakí shape-related element -\textit{na} and many Amazonian languages which also use -\textit{na} as a classifier for elongated referents.} It occurs in forms such as <sana> ‘finger’, “dedo” (A\_127) (compare the slightly different form <safsrí> / <safsri> ‘finger’, “dedo” in the 18\textsuperscript{th}{}-century materials, M\_448), <chipina> ‘face’, “rostro” (M\_425, possibly related to Andakí <szifi> ‘eyes’, “ojos”, M\_423, and Nasa Yuwe \textit{dʲiʔp} ‘face’, \citealt{Gerdel2023}), <szixaná> ‘chest’, “pecho” (M\_437). Note that the actual shape of the entity referred to by the root and the shape originally referred to by the classifying morpheme do not necessarily always coincide in a language (see, e.g.,  \citealt{Pache2016}). A stem-formative originally referring to roundish entities may be <-fi>, as attested in <szifi> ‘eye’, “ojos” (M\_423), or <chimbufi> ‘navel’, “ombligo” (M\_446), and possibly <szitafi> ‘hill’, “cerro” (M\_467) – a root related to <szita> occurs in <chitachini> ‘above’, “arriba” (M\_569). Finally, <-xe> occurs in several nouns referring to liquids, for instance, <bacoxe> ‘\textit{chicha}’ (M\_032), <vnanszaxi>, <unanszaxi> ‘spirits’, “aguardiente” (M\_563), or <jixe> ‘water’, “agua” (M\_031). None of these stem formatives/classifying morphemes is glossed in this book, given that the meaning of the root is often unknown (for instance, the meaning of <baco> in <bacoxe> ‘\textit{chicha}’). 

Gender marking does not occur in Andakí pronouns, and it is marginal in Andakí nouns. Certain female kinship terms or person designations take a derivational suffix <-gua> (or a variant of it), whereas their male counterparts take
<-he> or <-che>, as shown in \tabref{tab:6.1}.

\begin{table}
\begin{tabularx}{\textwidth}{QQ}
\lsptoprule
{\bfseries Female} & {\bfseries Male}\\
\midrule
{ <cachinigua> ‘granddaughter’, “nieta" (M\_491)} & { <cachinehe> / <cachineche> ‘grandson’, “nieto" (M\_490)}\\
{ <szasejagua> ‘cousin (female)’, “prima" (M\_493)} & { <szasejahe> / <szasejahé> ‘cousin (male)’, “primo" (M\_492)}\\
{ <chiguacoa> ‘girl’, “muchacha” (M\_702)} & { <chiguahe> ‘boy’, “muchacho” (M\_701)}\\
\lspbottomrule
\end{tabularx}
\caption{Derivational gender marking in Andakí nouns with male and female referents.}
\label{tab:6.1}
\end{table}

A suffix <-ni>, <-ne>, or <-re> with an unknown meaning occurs in Spanish loanwords in Andakí. Examples are <buytre-ni> ‘vulture’ (M\_606, from Spanish \textit{buitre} ‘vulture’), <oveja-ni> ‘sheep’ (M\_097, from Spanish \textit{oveja} ‘sheep’), and, possibly, <pari-ni> ‘priest’ (M\_560, possibly from Spanish \textit{padre} ‘priest’). A variant of <-ni> might be <-re>, attested in <guacaré> ‘cow’ (M\_096, from Spanish \textit{vaca} ‘cow’). Andakí nouns of non-Spanish origin do not seem to carry this morpheme. If Andakí <{}-ni> (and related forms) is indeed a classifying morpheme, it is remarkable in that its use does not seem to be linked to semantics but to the loanword status of the nominal root to which it is attached. It does not occur with borrowed verbal roots, as illustrated by (\ref{ex:6:2}). The root <reza> is borrowed from Spanish (\textit{rezar} ‘to pray’).

\begin{exe}
\ex\label{ex:6:2}
<Inszi rezara> / <Inszirezara> \\
\gll inszi reza-{ra} \\
let.us.go pray-\textsc{all} \\
\glt ‘Let us pray!’, “Vamos a rezar” (M\_129) 
\end{exe}

There are also several other, apparently derivational morphemes in Andakí nouns which have  an unknown meaning – compare, for instance, <sacaà> ‘hand’, “mano” (A\_121) versus \mbox{<sacanifi>} ‘forked stick’, “horqueta (palo bifurcado)” \mbox{(A\_123)}; the meaning of <nifi> remains unknown for the moment. Another case is <-jo>, illustrated in \tabref{tab:6.2}. Note the different position of <-jo> with respect to the presumed stem formative/classifying morphemes in Sets 1 and 2 (<-jo> plus <-hé>), Set 3 (<-fi> plus <-jo>), and Set 4 (<-jo> replaces <-he> / <{}-hé>). 

\begin{table}
\begin{tabularx}{\textwidth}{lQQ}
\lsptoprule
 & {\bfseries Form 1} & {\bfseries Form 2}\\
\midrule
{ 1} & { <shuntahé> ‘deer’, “venado” (M\_657)} & { <szuntijohé> ‘tapir’, “danta" (M\_656)}\\
{ 2} & { <szahé> ‘worm’, “gusano” (M\_645)} & { <szajohé> ‘dragonfly’, “chapul” (M\_644)}\\
{ 3} & { <szifi> ‘eyes’, “ojos" (A\_138)} & { <sifijo> ‘eyebrows’, “cejas" (A\_139)}\\
{ 4} & { <chunguahe> / <chunguahé> ‘ears’, “orejas" (M\_424)} & { <sunguajo> ‘ear’, “oreja" (A\_155)}\\
\lspbottomrule
\end{tabularx}
\caption{The derivational suffix <-jo>}
\label{tab:6.2}
\end{table}

Finally, in some cases, there are specific word families in Andakí, for instance, nouns derived from a root <chigua> by a range of unknown suffixes: <chiguae> ‘son, offspring’, “hijo” (A\_032), <chiguaco> ‘child’, “niño” (A\_028), <chiguana> ‘chest’ (A\_029), <chiguagua> ‘cousin’, “primo” (A\_030), <chiguagus> ‘grandchild’, “nieto” (A\_031). 

\largerpage
In a few cases, the lefthand morpheme of a compound or derived form remains unknown, such as in <jorapahi> ‘caiman’, “caimán” (M\_634), which is obviously related to <rapae> ‘caiman’, “caimán” (A\_118),\footnote{Compare, in Chicham languages, the formally similar terms Aguaruna \textit{dapi} ‘snake’, \textit{nampiʧ} ‘worm’ \citep{Deicat2023}, Huambisa \textit{napi} ‘snake’, \textit{nampiʧ} ‘worm’ \citep[66--67]{Jakway2008}.} or in the case of <szajihi> ‘lake’, “laguna” (M\_710), composed of <jixe> ‘water’, “agua” (M\_031), and an unknown element <sza->. As first observed by \citet[101]{Rivet1924}, some Andakí nouns may carry a prefix <m(V)-> of unknown meaning, illustrated in \tabref{tab:6.3}. 

\begin{table}
\begin{tabularx}{\textwidth}{QQ}
\lsptoprule
{ {Andakí} {noun} {with} {\textit{m(V)-}}{prefix}} & { {Andakí} {noun} {without {\textit{m(V)-}}}{prefix}}\\
\midrule
{ <manduguaso> ‘banana’, “plátano” (A\_089)} & { <duazo> (M\_054) and <anduazo> ‘banana’, “plátano” (M\_113)}\\
{ <mijina> ‘earth’, “tierra” (A\_097)} & { <jixena> ‘earth’, “tierra” (M\_338)}\\
{ <micaffi> ‘roasted corn’, “maíz tostado” (M\_539)} & { <cahixi> ‘corn porridge’, “mazamorra”, (M\_541), <finticahe> ‘empty corn cob’, “tusa de maíz” (M\_ 540)}\\
\lspbottomrule
\end{tabularx}
\caption{Andakí nouns with a prefix <m(V)->}
\label{tab:6.3}
\end{table}

Like most other derivational morphemes discussed above, the prefix \textit{m(V)}- is not frequently attested in the available Andakí materials. It may have a variant \textit{(V)m}- or \textit{(V)n}-: compare <mitae> (A\_101) versus <vntahe> / <untahé> \mbox{(M\_464)} ‘moon’, “luna”, <quifi> (A\_115) versus <ynquifi> (M\_447) ‘nose’, “nariz”, and <bihina> (M\_693) versus <imbina> (A\_066), ‘silver’, “plata” (cf. \citealt[77]{CoronasUrzúa1994}).
