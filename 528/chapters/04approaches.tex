\chapter{Approach and challenges} \label{sec:4}

For this book, we analyzed the language materials provided in 18\textsuperscript{th}{}-century lists \citep{Anonymous-a, Anonymous-b}, sent to Spain by Mutis, and the slightly more recent word lists published in \citet{Albis1860--1861}. The first author of this book, Jelien Moens, built a database combining the available Andakí materials. The reader can consult it here: \url{https://sites.google.com/view/Andaki-database}.

\begin{figure}
\includegraphics[height=.93\textheight]{figures/fig2.png}
\caption{A fragment from Ms. II/2911 (Patrimonio Nacional. Real Biblioteca de Palacio, RBPR2911).}
\label{fig:4.1}
\end{figure}

\begin{figure}
\includegraphics[width=\textwidth]{figures/fig3.png}
\caption{A fragment from Ms. II/2912 (Patrimonio Nacional. Real Biblioteca de Palacio, RBPR2912).}
\label{fig:4.2}
\end{figure}

The two anonymous 18\textsuperscript{th}{}-century lists, Ms. II/2911 and Ms. II/2912, are both provided in the database. Their transcriptions by \citet{GómezTorres2012--2013a,GómezTorres2012--2013b} have also been added, in order to illustrate differences with our own transcriptions. A synthesis column illustrates the sometimes divergent information and transcriptions. Also, modern equivalents of the more archaic spelling of the Spanish translations were added wherever deemed helpful. Finally, every entry is assigned an ID ranging from M\_001 to M\_772 (M for Mutis). 

As to the Albis sources (\citealt{VergarayVergaraDelgado1855}; \citealt{Albis1860--1861}), we consulted Albis’ travel notes, published in Spanish in \textit{Los indios del Andaquí: memorias de un viajero} in 1855, which is available online,\footnote{\url{https://bdh-rd.bne.es/viewer.vm?id=0000123147}} but is sometimes difficult to read. To solve this problem, and to stay as close to the original source as possible, the English translation of \citet{Albis1860--1861} was consulted whenever needed. 

\begin{figure}
\includegraphics[height=.93\textheight]{figures/fig4.png}
\caption{English translation of Albis’ original notes translated to English in \textit{American Ethnological Society} published in 1860--1861.}
\label{fig:4.3}
\end{figure}

\begin{figure}
\includegraphics[height=.93\textheight]{figures/fig5.png}
\caption{Albis’ original notes as published in \textit{Los indios del Andaquí: memorias de un viajero} in 1855.}
\label{fig:4.4}
\end{figure}

As with the Mutis lists, a modern-day Peninsular Spanish column has been added to facilitate the process; the resulting entries range from A\_001 to \mbox{A\_163} (A for Albis). The Albis source only contains four short sentences, while the remaining 159 entries are lexical (predominantly nouns).

In sum, the \textsc{excel} sheet in the database contains 935 rows of linguistic material to work with. In this book and in the data shown below, all data are presented together with their IDs. The reader can thus consult the corresponding original spellings by looking up the relevant ID in the database.

As with other scarcely documented, extinct languages (see, e.g., \citealt{Urban2019}), working with the Andakí materials turns out to be quite complex in several respects. For instance, there is a lot of orthographic and allophonic variation, discussed in \sectref{sec:5} on phonology, and in \citet{CoronasUrzúa1994}. Also, it is not always easy to identify word and morpheme boundaries in the 18\textsuperscript{th}{}-century data, and spacing may appear inconsistent. For instance, the original orthography in (\ref{ex:4:1}) suggests that the element \textit{zi} is interpreted as a prefix by the anonymous author.

{
\ea\label{ex:4:1}
<Nanqui zincaque?> / <Nañquizincaque> \\
\gll nanqui z-inca que  \\
meat ?-\textsc{prop} be  \\
\glt ‘Do you have meat?’, “Tenéis carne?” (M\_284)
\z
}

Yet, elsewhere in the 18\textsuperscript{th}{}-century lists, we find <nanguize> ‘meat’, “carne” (M\_094) or <ñanquise> ‘meat’, “carne” (A\_108), suggesting that \textit{z(i)} in (\ref{ex:4:1}) belongs in fact to the preceding root. Due to such orthographic irregularities, the interpretation of Andakí morphemes as suffixes, prefixes, clitics, or unbound forms must sometimes remain tentative. As a solution, we have interpreted such elements, which in the majority of cases are orthographically represented as prefixes or suffixes in the sources (that is, preceding the root or following the root and forming a word together with it), as prefixes or suffixes, respectively. In glossing, both affix and clitic boundaries are indicated by a hyphen.

(\ref{ex:4:1}) illustrates another challenge in the context of Andakí: the Spanish variety used in the manuscript seems to make ample use of \textit{voseo}. This can lead to ambiguities in certain cases and it is impossible to decide whether a form such as Spanish \textit{tenéis} in (\ref{ex:4:1}) refers to a 2\textsuperscript{nd}-person singular or plural. In these instances, we intentionally left the English pronoun \textit{you} ambiguous with respect to number as well and did not specify whether it refers to a 2\textsuperscript{nd}-person singular or plural. Andakí itself hardly seems to differentiate between singular and plural with 1\textsuperscript{st}- and 2\textsuperscript{nd}-person subjects or objects anyway (for discussion, see \sectref{sec:15.1}).

The translations provided in the word lists pose another challenge. In a number of cases, they appear partially inaccurate from the perspective of gloss-based translations. An example of this is given in (\ref{ex:4:2}--\ref{ex:4:6}). In (\ref{ex:4:2}--\ref{ex:4:3}), there is no identifiable predicative possessive construction, yet, the examples are translated as such; additionally, in (\ref{ex:4:2}), ‘I do not have a house’, there is no term for ‘house’ identifiable. 

\begin{exe}
\ex\label{ex:4:2}
<Ringa pacahá> \\
\gll ringa paca-ha  \\
I \textsc{neg-rea} \\
\glt ‘I do not have a house’, “Yo no tengo casa” (M\_770)
\newpage
\ex\label{ex:4:3}
<Ninga pajaha> / <Ninga pajahá> \\
\gll ninga paja-ha \\
I  \textsc{neg{}-rea} \\
\glt ‘I do not have’, “Yo no tengo” (M\_228)
\end{exe}

In (\ref{ex:4:4}--\ref{ex:4:5}), the translation implies a 2\textsuperscript{nd}-person possessor which is not overtly expressed in the original Andakí sentence.

\begin{exe}
\ex\label{ex:4:4}
<Quarajea cogora> \\
\gll qua-ra-je-a cogo-ra  \\
\textsc{2.imp-}\textsc{caus}{}-go-\textsc{imp}  house{}-\textsc{all} \\
\glt ‘Bring (\textsc{sg}) me to your house!’, “Llévame a tu casa”  (M\_771)
\ex\label{ex:4:5}
<Nszazini chiguaca?> \\
\gll nszazi ni chiguaca? \\
how/where \textsc{q} children \\
\glt ‘How are your \textsc{(sg)} children?’, “Cómo están tus hijos?” (M\_286) 
\end{exe}

In such cases, the missing element must have been recoverable from the original context. Finally, (\ref{ex:4:6}) is translated as a prohibition, but upon closer inspection, the literal translation of the Andakí sentence is a question, namely ‘Why are you (\textsc{sg}) mean?’.

\begin{exe}
\ex\label{ex:4:6}
<Yhiza vmaniquini> \\
\gll yhiza vmani qui ni \\
why  mean  \textsc{cop}  \textsc{q} \\
\glt ‘Do not be (\textsc{sg}, \textsc{m}) mean/stingy!’, “No seas mezquino” (M\_748)
\end{exe}

That said, in such a case, the translation is probably accurate in the sense that the original author wanted to provide a good translation, not necessarily a transparent one (Jean-Christophe Verstraete, p.c.). Also, contextual information is missing in the available language data, which makes the adequacy of the translations difficult to assess. In sum, the translations used in this book are made by the hand of Albis and the anonymous author or authors of the 18\textsuperscript{th}{}-century word lists, and they are not necessarily literal translations.

Another obstacle in analyzing the morphosyntax of Andakí is the scarcity of the data and the limited variation of constructions. For instance, as the reader can see in the database, there are many examples of orders and prohibitions in the word lists: roughly half of the phrases in the data are orders. This is probably due to the missionary context in which the data were collected and one cannot know with certainty whether a particular phenomenon is absent from Andakí itself, or merely absent from the available language data. This book therefore by no means aims to provide an exhaustive description of Andakí grammar: linguistic examples and claims presented do not cover all aspects of Andakí grammar and are representative of a tendency in the limited dataset only. 
