\addchap{Preface}
 
The history of the Andakí is that of one of the most enigmatic indigenous peoples of South America. From the beginning, their origins were wrapped in mystery. They have been described as remnants of the Precolumbian people who built the wonderful stone monuments surrounding the village of San Agustín in the upper Magdalena highlands, but also as Post-San Agustín invaders from the eastern lowland who might have welcomed the first Spanish conquistadors upon their arrival in the region. A more realistic and well-informed account of their origins is provided by the historian Juan Friede in his book \textit{Los andakí 1538--1947: Historia de la aculturación de una tribu selvática} (México D.F., 1953). He describes the Andakí as an invasive nation who penetrated the Alto Magdalena region from the eastern Andean foothills around 1600. For two consecutive centuries they sowed fear and terror among the local Spanish settlers and their indigenous associates. The latter had been ruthlessly exploited since their incorporation in the colonial encomienda system. This treatment eventually resulted in the near disappearance of the local population, some of whom escaped to the eastern foothills and lowland, where they may have joined the ranks of Andakí fighters. From 1600 onward attacks of the Andakí obtained the character of targeted revenge actions, including nighttime abductions and the forced disappearance of entire families. Punitive expeditions organised by the colonial authorities were helpless against the guerrilla tactics of the Andakí, who had learned to use effective weaponry and refined intelligence against the \textit{encomenderos} and their subjects. Their attacks continued throughout the 18th century, whereafter they diminished with the withdrawal of the Andakí to jungle bases on the Caquetá and Fragua rivers. Eventually, the memory of their feats slipped into oblivion, and only place names such as \textit{Belén de los Andaquíes} in the Caquetá department are reminiscent of their former existence. 
The language of the Andakí has been poorly documented so far and remains unclassified. There have been no recent reports of possible surviving speakers. Consequently, the Andakí language must be considered extinct, unless evidence of the contrary is brought forward. Jelien Moens and Matthias Pache have taken up the challenge of analysing and systematising the scarce language data of Andakí that have been preserved. The present book exhibits the result of their pioneering efforts and will stand as an indispensable contribution to our knowledge of the indigenous languages of Colombia.

\hfill  Willem Adelaar\\
\hfill Leiden, 13 June 2025