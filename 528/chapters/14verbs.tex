\chapter{Verbs} \label{sec:14}

In this chapter, we will discuss the grammar of the Andakí verb and particular phenomena of the verbal domain. First, we will discuss those features that are expressed by prefixes and proclitics: causative marking (\sectref{sec:14.1}), the marking of spatial notions (\sectref{sec:14.2}), aspect/aspectual notions (\sectref{sec:14.3}), and verbal person marking (\sectref{sec:14.4}). This is followed by a discussion of those bound and unbound grammatical elements that follow the verbal root and mark mood (\sectref{sec:14.5}) and modality (\sectref{sec:14.6}). This chapter also comprises a discussion of verbal root suppletion (\sectref{sec:14.7}) and the copula in Andakí (\sectref{sec:14.8}). 

The Andakí word lists – in particular those from the 18\textsuperscript{th} century – contain several verbs, that is, forms that typically occur as heads of a predicate, refer to actions, and take specific morphology such as markers of tense, aspect, and modality (see, e.g., \citealt[39, 52--53]{Dixon2010}). In the anonymous 18\textsuperscript{th}{}-century Andakí materials, individually listed verbs are easily identifiable in that they carry <{}-za>, which marks imperative mood. In a few cases, the verbal root and its nominal counterpart are nearly identical and seem to be etymologically related. This is the case with <shungua> ‘to hear’ (cf. M\_731) and the corresponding noun <chunguahe> / <chunguahé> ‘ears’, “orejas” (M\_424), which carries a stem formative <-he> {\textasciitilde} <-hé> (see \sectref{sec:6}). In other cases, the verb is fundamentally distinct from and probably etymologically unrelated to the corresponding noun, as is the case with <zega> ‘urine’, “orines” (M\_021) and <chisi> ‘to urinate’ (cf. M\_389). Overall, verbs are less frequently found in the available Andakí lists than nouns. 

The verb tends to occur at the end of an Andakí utterance in transitive and intransitive constructions alike. All intransitive and most transitive constructions have S or A expressed by a pronoun in the available Andakí materials, not by a noun. In intransitive constructions, in which S is expressed by a pronoun, SV is the most frequently attested constituent order, shown in (\ref{ex:14:1}), although VS also occurs, shown in (\ref{ex:14:2}).

\newpage
\begin{exe} 
\ex\label{ex:14:1}
<Ninga buxibi> \\
\gll ninga {bu-xi-bi} \\
I  \textsc{transl}{}-go-\textsc{rea} \\
\glt ‘I go away’, “Yo me voy” (M\_320)
\ex\label{ex:14:2}
<Yubi ninga> \\
\gll {yu-bi} ninga  \\
come-\textsc{rea}  I \\
\glt ‘I come’, “Yo vengo” (M\_256)
\end{exe}

There are only a few transitive constructions. A construction with AOV constituent order in which the transitive subject (A) is expressed by a personal pronoun is shown in (\ref{ex:14:3}).\footnote{An anonymous reviewer suggests that, alternatively, <chatize> in (\ref{ex:14:3}) functions as an adverbial modifier, not as an object.}

\begin{exe}
\ex\label{ex:14:3}
<Ninga chatize guabi> \\
\gll ninga  chatize {gua-bi} \\
I  truth  say-\textsc{rea} \\
\glt ‘I tell the truth’, “Yo digo verdad” (M\_349)
\end{exe}

In (\ref{ex:14:4}), a question in which the agent is expressed by a regular noun, the object follows the verbal predicate, and the constituent order is AVO. 

\begin{exe}
\ex\label{ex:14:4}
<Cabiyara chiyaya quique?> / <Cabiyara chiyayá quique?>  \\
\gll cabi-ya {ra-chiya-ya} quique?\\
parrot-\textsc{pl}  \textsc{aor}{}-eat-\textsc{pl}  corn \\
\glt ‘Have the parrots eaten the corn?’, “Han comido el maíz los loros?” (M\_166)
\end{exe}

Given the very limited data available and given the unusual constituent order in the Spanish translation, it is difficult to state how representative (\ref{ex:14:4}) is for Andakí constituent order in transitive clauses. In imperative constructions such as those shown in (\ref{ex:14:5}--\ref{ex:14:8}), the verbal predicate is likewise followed by the object and the order is VO. 

\begin{exe}
\ex\label{ex:14:5}
<Quananqueha Xifi=> / <Quananqueha Xifi>  \\
\gll {qua-nan-que-ha} xifi\\
\textsc{2.imp-caus}{}-come-\textsc{imp}    candle \\
\glt ‘Bring (\textsc{sg}) a candle!’, “Trae candela” (M\_075)
\ex\label{ex:14:6}
\begin{xlist}
\ex\label{ex:14:6a}
<Quananqueha Xizi. => /  <Quananqueha Xizi>  \\
\gll {qua-nan-que-ha} xizi \\
\textsc{2.imp-caus}{}-come-\textsc{imp}  firewood \\
\glt ‘Bring (\textsc{sg}) firewood!’, “Trae leña”\footnote{It remains to be established whether or not the terms for candle <xifi> and firewood <xizi> are etymologically related.} (M\_076)
\ex\label{ex:14:6b}
<Quananquea jizi> / <Guananquea fizi>  \\
\gll {qua-nan-que-a} jizi \\
\textsc{2.imp-caus}{}-come-\textsc{imp}    firewood \\
\glt ‘Bring (\textsc{sg}) firewood!’, “Trae leña” (M\_142)
\end{xlist}
\ex\label{ex:14:7}
<Quananqueha Jexe. => / <Quananqueha Jexé>  \\
\gll {qua{}-nan-que-ha} jexe \\
\textsc{2.imp-caus}{}-come-\textsc{imp}    water \\
\glt ‘Bring (\textsc{sg}) water!’, “Trae agua” (M\_074)
\ex\label{ex:14:8}
<Andagu biguaza bacoxe> / <Andagu biquaza bacoxé> \\
\gll andagu {bi{}-gua-za} bacoxe \\
quickly  \textsc{rap-}make-\textsc{imp}  \textit{chicha} \\
\glt ‘Make (\textsc{sg}) \textit{chicha} quickly!’, “Haz presto chicha” (M\_150)
\end{exe}

Cases where the object precedes the verbal predicate in imperative constructions (OV order) seem to occur somewhat less frequently; two such instances are illustrated in (\ref{ex:14:9}--\ref{ex:14:10}). The translation of (\ref{ex:14:9}) is the same as that of (\ref{ex:14:7}) above.

\begin{exe}
\ex\label{ex:14:9}
<Jixe nanqueza> \\
\gll jixe {nan-que-za} \\
water  \textsc{caus}{}-come-\textsc{imp} \\
\glt ‘Bring (\textsc{sg}) water!’, “Trae agua” (M\_031)
\ex\label{ex:14:10}
<Bacoxe quananqueha riszijaxi.=> / <Bacoxe quananquehá riszijaxi> \\
\gll bacoxe {qua-nan-que-ha} riszi-jaxi \\
\textit{chicha}   \textsc{2.imp-caus}{}-come-\textsc{imp}  drink-\textsc{purp}\\
\glt ‘Bring (\textsc{sg}) \textit{chicha} in order to drink’, “Trae chicha para beber” (M\_078)
\end{exe}

Constituent order in the Andakí imperative constructions (\ref{ex:14:5}--\ref{ex:14:10}) may be governed by information structure. 

As to the arrangement and ordering of grammatical morphemes in the verb or verb phrase, valency changes, spatial and aspectual notions, and mostly also verbal person (2\textsuperscript{nd}-person subject) are marked by prefixes or proclitics, whereas mood and modality are marked by bound or unbound grammatical elements that follow the verb. 

Given the incomplete nature of the available Andakí materials, several domains of the grammar of the verb must remain open, because evidence for them is found only in single examples. Among these topics are subordination and nominalization. Besides the relativizer <-ze> discussed above, in \sectref{sec:13}, a possible subordinating ending in Andakí is <-chaza>. Its use is illustrated in (\ref{ex:14:11}).

\begin{exe}
\ex\label{ex:14:11}
<Firajiquichaza cobaquea rincaxahá> \\
\gll firaji  qui-{chaza} coba-que-a     rinca-xaha \\
hungry  \textsc{cop}{}-\textsc{sub}  2.\textsc{imp}{}-come-\textsc{imp}  I-\textsc{obl} \\
\glt ‘When hungry, come (\textsc{sg}) to me!’, “En teniendo hambre ven a mí” (M\_762)
\end{exe}

A suffix <-na> seems to function as a locative nominalizer, as shown in (\ref{ex:14:12}), where it derives the term for ‘field’ <fichana> (M\_161) from the verb ‘to brush’ <ficha> (cf. M\_162).

\begin{exe}
\ex\label{ex:14:12}
<Fichanara rajihi> / <Fichanará rajichi> \\
\gll ficha-{na}{}-ra    ra-ji-hi \\
brush-\textsc{loc-all}  \textsc{aor}{}-go-\textsc{rea} \\
\glt ‘She/he/it went to the field’, “Se fue a la chagra” (M\_161)
\end{exe}

Besides suffixation and prefixation, reduplication may be a marginal morphological device in Andakí; the only example of this is the case of <mi> ‘to want’ (cf. M\_304) and <mimi> ‘to love’ (cf. M\_232), where reduplication seems to indicate intensity.

In what follows, we give an overview of those grammatical morphemes in the verbal domain for which we could find more substantial evidence.

\section{Causative: <ra->} \label{sec:14.1}

There is little information on verbal derivational morphology in Andakí. A derivational prefix which is attested with a range of verbs is the causative marker <ra-> (M\_194), <za-> (M\_757), <naha-> (M\_303), <na-> [na-] (M\_202), or <nan-> [nã-] (M\_289). For instance, it derives the verb ‘to bring, to carry’ from the verb ‘to go’, as illustrated in (\ref{ex:14:13}--\ref{ex:14:14}).

\begin{exe}
\ex\label{ex:14:13}
<Rica Kazaxihi>\footnote{It is not clear whether or not final <-hi> belongs to the root or is a realis mood suffix.} \\
\gll rica   ka-{za}{}-xihi \\
you  \textsc{2-caus}{}-go \\
\glt ‘You (\textsc{sg}) brought’, “Tú llevaste” (M\_757)\footnote{This example is attested in Ms. II/2911 only.}
\protectedex{
\ex\label{ex:14:14}
<Rajiza rihizi> \\
\gll {ra}{}-ji-za   rihizi \\
\textsc{caus}{}-go-\textsc{imp}  this \\
\glt  ‘Take (\textsc{sg}) this!’, “Lleva a este” (M\_194)
}
\end{exe}

Causative <ra-> has the variant <na-> and <naha->, illustrated in (\ref{ex:14:15}) and (\ref{ex:14:16}), respectively. 

\begin{exe}
\ex\label{ex:14:15}
<Naqua naquiquani?> / <Náqua náquiguani?> \\
\gll na-qua {na}{}-qui    qua  ni? \\
\textsc{cop{}-}who  \textsc{caus}{}-come  who  \textsc{q} \\
\glt ‘Who brought you (\textsc{sg})?’, “Quién te trajó?” (M\_202)
\ex\label{ex:14:16}
<Ypchize canahá quine?> / <Ypchizé canahá quiné?> \\
\gll ypchize   ca-{naha}{}-qui  ne? \\
what    2-\textsc{caus}{}-come  \textsc{q} \\
\glt  ‘What did you (\textsc{sg}) bring?’, “Qué has traído?” (M\_303)
\end{exe}

In other cases, changes in verbal valency are lexically encoded in Andakí, for instance, in <qua> ‘to kill’ (cf. M\_006) versus <ma> ‘to die’ (cf. M\_005).

\section{Spatial notions: <bu-> ‘translocative’, <chi-> ‘inward’, <fiqui-> ‘upward’} \label{sec:14.2}

Andakí has some grammatical morphemes expressing spatial notions or associated motion on verbs. The prefix <bu->, for instance, may encode a translocative meaning, as suggested by the pairs <ji> ‘to go’ (cf. M\_039, M\_753) versus <buji> ‘to go away’ (cf. M\_190). The prefix <bo-> or <bu-> may also express associated motion with non-motion verbs, as suggested by <daza> ‘sleep! (\textsc{sg})’, “dormí” \mbox{(M\_356)} versus <bondaza> ‘(go and) sleep! (\textsc{sg})’, “dormí” (M\_355) and <bunta> ‘go and sleep’ illustrated in (\ref{ex:14:17}).\footnote{As pointed out by an anonymous reviewer, associated motion is a common category among languages of western South America; it occurs, for instance, in Mosetén (isolate), Pano-Tacanan and Quechuan (e.g., \citealt{Ross2021}).}

\begin{exe}
\ex\label{ex:14:17}
<Quabuntahá> \\
\gll qua-{bu}{}-nta-ha \\
2.\textsc{imp-am}{}-sleep{}-\textsc{imp} \\
\glt ‘Go and sleep (\textsc{sg})!’, “Anda duerme” (M\_318)
\end{exe}

Inward motion is probably expressed by the prefix <chi->, illustrated in (\ref{ex:14:18}).

\begin{exe}
\ex\label{ex:14:18}
<Guachijiza andagu> \\
\gll gua-{chi}{}-ji-za    andagu \\
\textsc{2.imp-iw}{}-go-\textsc{imp}  quickly \\
\glt ‘Enter (\textsc{sg}) quickly!’, “Entra presto” (M\_141)
\end{exe}

A prefix <fiqui> or <fiji>, shown in (\ref{ex:14:19}--\ref{ex:14:20}), is tentatively glossed as ‘upward motion’.

\begin{exe}
\ex\label{ex:14:19}
<Fiquijizá> \\
\gll {fiqui}{}-ji-za \\
\textsc{uw-}go-\textsc{imp} \\
\glt ‘Stand up (\textsc{sg})!’, “Levántate” (M\_372)
\ex\label{ex:14:20}
<Quafixi Jea> / <Quafixijea>  \\
\gll qua-{fixi}{}-je-a \\
\textsc{2.imp-uw-}go-\textsc{imp} \\
\glt ‘Stand up (\textsc{sg})!’, “Levántate” (M\_089)
\end{exe}

\section{Aspectual notions: <ra-> ‘aorist’, <bi-> ‘rapid’} \label{sec:14.3}

We found no specific markers for imperfective aspect and related notions in the available Andakí data. There are two examples with 3\textsuperscript{rd}-person subjects translated with a progressive meaning. These examples do not bear any obvious aspectual morphology, as illustrated in (\ref{ex:14:21}--\ref{ex:14:22}). 

\begin{exe}
\ex\label{ex:14:21}
<Fichahe> / <Fichahé> \\
\gll ficha-he \\
brush-\textsc{rea}  \\
\glt ‘She/he/it is brushing’, “Está rozando” (M\_162)
\ex\label{ex:14:22}
<Zancahaya>  \\
\gll zancaha-ya \\
talk-\textsc{pl} \\
\glt ‘They are talking’, “Están hablando” (M\_125)
\end{exe}

We tentatively suggest that the aorist (past tense with perfective aspect) is marked by a prefix <ra->. We found instances of this prefix only in clauses with a 3\textsuperscript{rd}-person subject, as illustrated by (\ref{ex:14:23}--\ref{ex:14:25}); that is, this prefix may have been used exclusively with non-speech act participants. 

\begin{exe}
\ex\label{ex:14:23}
<Canchihi raquahi> \\
\gll canchihi {ra}{}-qua-hi \\
lightning  \textsc{aor}{}-kill-\textsc{rea}  \\
\glt ‘The lightning killed him/it’, “Lo mató el rayo” (M\_220)
\ex\label{ex:14:24}
<Ramahi> \\
\gll {ra}{}-ma-hi \\
\textsc{aor}{}-die-\textsc{rea} \\
\glt ‘She/he/it died’, “Murió” (M\_322)
\ex\label{ex:14:25}
<Naqua Raguaquani?> / <Naqua raguaquani?> \\
\gll na-qua {ra}{}-gua    qua  ni? \\
\textsc{cop{}-}who  \textsc{aor}{}-say  who  \textsc{q}\\
\glt ‘Who told you (\textsc{sg})?’, “Quién te dijo?” (M\_730)
\end{exe}

In a few instances, <ra-> occurs in examples that are translated with the Spanish perfect, as in (\ref{ex:14:26}).

\begin{exe}
\ex\label{ex:14:26}
<Cabiyara chiyaya quique?> / <Cabiyara chiyayá quique?> \\
\gll cabi-ya {ra}{}-chiya-ya  quique? \\
parrot-\textsc{pl}  \textsc{aor}{}-eat-\textsc{pl}  corn  \\
\glt ‘Have the parrots eaten the corn?’, “Han comido el maíz los loros?” (M\_166)
\end{exe}

There is also one sentence containing <ra-> which is translated, however, with the Spanish present tense (\ref{ex:14:27}).

\begin{exe}
\ex\label{ex:14:27}
<Nszazi raguayani xin-gaxá?>\footnote{The hyphen is due to a line break in the manuscript.}  / <Nszâzi raguayani xingaxá?> \\
\gll nszazi {ra}{}-gua-ya  ni  xinga-xa? \\
how/where  \textsc{aor}{}-say-\textsc{pl}  \textsc{q}  I{}-\textsc{obl} \\
\glt ‘What do they say about me?’, “Qué dicen de mí?” (M\_758) 
\end{exe}

The aorist-marking prefix <ra-> seems to have a variant <na->, illustrated in (\ref{ex:14:28}).

\begin{exe}
\ex\label{ex:14:28}
<Ninquihi chazajahi na-ranquihi> / <Ninquihí chazajahi naranquihi> \\
\gll ninqui-hi chazajahi {na}-ran-quihi \\
I-\textsc{rea} brother	\textsc{aor}-\textsc{caus}-come \\
\glt ‘My brother brought me', “Mi hermano me trajó." (M\_203)
\end{exe}

Another aspectual notion is likewise expressed by the prefix <bi->, which expresses rapidity, as in (\ref{ex:14:29}--\ref{ex:14:30}).

\begin{exe}
\ex\label{ex:14:29}
<Bijiza> \\
\gll {bi}{}-ji-za \\
\textsc{rap-}go-\textsc{imp} \\
\glt ‘Run (\textsc{sg})!’, “Corre” (M\_120)
\ex\label{ex:14:30}
<Andagu biguaza bacoxe> / <Andagu biquaza bacoxé> \\
\gll andagu {bi}{}-gua-za   bacoxe \\
quickly  \textsc{rap-}make-\textsc{imp} \textit{chicha} \\
\glt ‘Make (\textsc{sg}) \textit{chicha} quickly!’, “Haz presto chicha” (M\_150)
\end{exe}

\section{Verbal person marking} \label{sec:14.4}

Only a few person-marking morphemes can be identified on Andakí verbs. There is no evidence in the data for any dedicated 1\textsuperscript{st}-person subject affix.\footnote{1\textsuperscript{st}-person subject marking in Andakí requires further investigation. Prefixes occurring in constructions with a 1\textsuperscript{st}-person agent (transitive subject) are <ha-> in (M\_245) or <cho-> in \mbox{(M\_081)}; however, this information is not sufficient to identify any 1\textsuperscript{st}-person marking prefix in Andakí.}  The 1\textsuperscript{st}-person subject/agent usually seems to be indicated by a zero morpheme or the absence of any verbal prefix, as shown in (\ref{ex:14:31}--\ref{ex:14:32}). 

\begin{exe}
\ex\label{ex:14:31}
<Jahá najizequi> / <Jaha najizequi> \\
\gll jaha  najize    qui \\
yes  yesterday  come \\
\glt ‘Yes, I came yesterday’, “Sí, ayer vine” (M\_108)
\ex\label{ex:14:32}
<Quihi>\\
\gll quihi \\
come \\
\glt ‘I came’, “Vine” (M\_733)
\end{exe}

In some cases, a 1\textsuperscript{st}-person pronoun is used to refer to the participant involved in the event referred to, as shown in (\ref{ex:14:33}--\ref{ex:14:34}). 

\begin{exe}
\ex\label{ex:14:33}
<Ringa shungua>\\
\gll ringa  shungua \\
I  hear/understand \\
\glt ‘I heard it/him’, “Yo lo oí” (M\_731)
\ex\label{ex:14:34}
<Ninga buxibi> \\
\gll ninga   bu-xi-bi \\
I  \textsc{transl}{}-go-\textsc{rea}  \\
\glt ‘I go away’, “Yo me voy” (M\_320)
\end{exe}

In imperative (hortative) mood, there is no dedicated verbal person marker for the 1\textsuperscript{st}-person singular or plural subject either, as shown in (\ref{ex:14:35}) (1\textsuperscript{st}-person singular) and (\ref{ex:14:36}) (1\textsuperscript{st}-person plural).

\begin{exe}
\ex\label{ex:14:35}
<Quananquea chiguaca nunqueaha> / <Quananquea chiguaca nunqueahá> \\
\gll qua-nan-que-a   chiguaca   nunque-aha \\
\textsc{2.imp-caus}{}-come-\textsc{imp}  children  look-\textsc{imp} \\
\glt ‘Bring me your (\textsc{sg}) children, I want to see them’, “Tráeme a tus hijos, que quiero verlos” (M\_289)
\ex\label{ex:14:36}
<Chiyaba gunfigo> \\
\gll chiya{}-ba  gunfigo \\
eat-\textsc{imp}    one\\
\glt ‘Let us eat together!’, “Comamos juntos” (M\_293)\footnote{The grammaticalization path from ‘one’ to ‘together’, suggested by the translation of (\ref{ex:14:36}), is common in languages around the world \citep[225]{HeineKuteva2002}.}  
\end{exe}

In the indicative mood, the morpheme <ca-> marks the 2\textsuperscript{nd}-person subject; this element recurs in <rica>, the 2\textsuperscript{nd}-person pronoun (see \sectref{sec:9}). Andakí <ca{}-> is used to refer exclusively to a 2\textsuperscript{nd}-person transitive or intransitive subject, not to a 2\textsuperscript{nd}-person object. Two examples illustrating the use of <ca-> are shown in (\ref{ex:14:37}--\ref{ex:14:38}). 

\begin{exe}
\ex\label{ex:14:37}
<Kacó?> / <Kacô?> \\
\gll {ka}{}-co? \\
2-blow \\
\glt 
‘Did you \textsc{(sg)} blow?’, “Soplaste?” (M\_196)
\newpage
\ex\label{ex:14:38}
<Kariszi?> \\
\gll {ka}{}-riszi?\\
2-drink \\
\glt ‘Did you drink?’, “Bebisteis?” (M\_727)
\end{exe}

Note that we do not have examples of 2\textsuperscript{nd}-person declarative constructions with lexical verbs. With lexical verbs, we found only interrogative constructions. In declarative constructions with a 2\textsuperscript{nd}-person subject, we found only the copula, as shown in (\ref{ex:14:39}).

\begin{exe}
\ex\label{ex:14:39}
<Rica vchanagaquehi> / <Rica vcha naga quehi> \\
\gll rica  vchana {ga}{}-que-hi \\
you  brave    2\textsc{{}-cop-rea} \\
\glt ‘You (\textsc{sg}) are brave’, “Tú eres valiente” (M\_738)
\end{exe}

Whereas the use of a personal pronoun is optional with the 2\textsuperscript{nd} person (cf. \sectref{sec:9}), the use of <ca-> ‘2\textsuperscript{nd}-person subject’ (or one of its variants) seems obligatory, as illustrated in (\ref{ex:14:40}--\ref{ex:14:41}).

\begin{exe}
\ex\label{ex:14:40}
\begin{xlist}
\ex\label{ex:14:40a}
<Rica bayagaquihi> / <Rica bayaga quihi> \\
\gll {rica} baya {ga}{}-qui{}-hi \\
you  lazy  2-\textsc{cop{}-rea} \\
\glt ‘You (\textsc{sg}, \textsc{m}) are lazy’, “Tú eres flojo” (M\_739)
\ex\label{ex:14:40b}
<Bajoa-gaquihi>\footnote{Hyphenation is due to a line break in the manuscript.}  / <Bajoagaquihi> \\
\gll bajoa {ga}{}-qui{}-hi \\
lazy  2-\textsc{cop{}-rea} \\
\glt‘You (\textsc{sg}, \textsc{m}) are lazy’, “Tú eres flojo” (M\_740)
\end{xlist}
\ex\label{ex:14:41}
\begin{xlist}
\ex\label{ex:14:41a}
<Rica cachiya?> / <Rica cachiyá?> \\
\gll {rica} {ca}{}-chiya? \\
you  2-eat \\
\glt ‘Have you (\textsc{sg}) eaten?’, “Vos habéis comido?” (M\_044)
\ex\label{ex:14:41b}
<Cachiya?> / <Cachiyá?> \\
\gll {ca}{}-chiya? \\
2-eat \\
\glt ‘Have you eaten?’, “Habéis comido?” (M\_043)
\end{xlist}
\end{exe}

We tentatively consider <ca-> to be a verbal prefix or proclitic, since it occurs in preverbal position. In one single case, however, we found evidence that might challenge this interpretation. This case is shown in (\ref{ex:14:42}), where the 2\textsuperscript{nd}-person subject marker follows the verb. 

\begin{exe}
\ex\label{ex:14:42}
<Chiyaga?> / <Chiyagâ?> \\
\gll chiya {ga}?\\
eat 2 \\
\glt ‘Do you want to eat?’, “Queréis comer?” (M\_760)
\end{exe}

Alternatively, final <ga> / <gâ> in (\ref{ex:14:42}) is a truncated, postposed 2\textsuperscript{nd}-person pronoun – compare <rica> / <ricá> ‘you’, “tú” (M\_013). 

In what we call ‘imperative\textsubscript{1}' of Andakí (discussed in \sectref{sec:14.5.2}), the 2\textsuperscript{nd}-person marker is <coba->, <coha->, <coa->, or a related form, the underlying form of which is /kʷa-/. An example of the 2\textsuperscript{nd}-person imperative prefix is given in (\ref{ex:14:43}).

\begin{exe}
\ex\label{ex:14:43}
<Quabuntahá> \\
\gll {qua}{}-bu-nta-ha \\
\textsc{2.imp}{}-\textsc{am-}sleep-\textsc{imp} \\
\glt ‘Go and sleep (\textsc{sg})!’, “Anda duerme” (M\_318)
\end{exe}

Finally, there is no dedicated 3\textsuperscript{rd}-person subject marking affix on Andakí verbs, as illustrated in (\ref{ex:14:44}--\ref{ex:14:45}).

\begin{exe}
\ex\label{ex:14:44}
<Naqua fifiquani?> \\
\gll na-qua    fifi  qua  ni? \\
\textsc{cop{}-}who  call  who  \textsc{q} \\
\glt ‘Who calls you (\textsc{sg})?’, “Quién te llama?” (M\_003)
\ex\label{ex:14:45}
<Naqua naquiquani?> / <Náqua náquiguani?> \\
\gll na-qua    na-qui    qua  ni? \\
\textsc{cop{}-}who  \textsc{caus}{}-come  who  \textsc{q} \\
\glt ‘Who brought you (\textsc{sg})?’ “Quién te trajó?” (M\_202)
\end{exe}

There is no example of a 3\textsuperscript{rd}-person intransitive subject with a lexical verb in the available Andakí materials. However, since the aorist marker <ra->, discussed in \sectref{sec:14.3}, has been found only in constructions with a 3\textsuperscript{rd}-person subject, this morpheme might be interpreted as a portmanteau morpheme which expresses both aorist and 3\textsuperscript{rd}-person subject.

\section{Mood} \label{sec:14.5}

We found evidence for different morphemes marking realis mood (\sectref{sec:14.5.1}), imperative mood (\sectref{sec:14.5.2}–\sectref{sec:14.5.4}), and related notions such as the prohibitive (\sectref{sec:14.5.5}). The hortative expression <ynsci> is discussed in \sectref{sec:15.4}. In sum, there are several ways to mark directive speech acts on the Andakí verb. The morphemes in question are relatively well documented in the available language materials. The existence of several different imperatives in Andakí might be an areal feature: in Tucanoan languages, a neighboring language family, up to eleven imperatives have been identified \citep[7]{Aikhenvald2010}. 

\subsection{Realis mood: <-hi>} \label{sec:14.5.1}

The morpheme <-hi> is tentatively interpreted here as a realis mood marker; it has the variants <-bi> (mostly used with speech-act-participant subjects), <-xi>, and <-ha>. The realis mood marker occurs in statements and questions. Its use is shown in (\ref{ex:14:46}--\ref{ex:14:47}). 

\begin{exe}
\ex\label{ex:14:46}
<Yubi ninga> \\
\gll yu-{bi} ninga  \\
come-\textsc{rea}  I \\
\glt ‘I come’, “Yo vengo” (M\_256)
\ex\label{ex:14:47}
<Ricaxa fifihe> / <Ricaxa fifihé> \\
\gll rica-xa   fifi-{he} \\
you-\textsc{obl}  call-\textsc{rea}\\
\glt ‘It is you (\textsc{sg}) she/he/it calls’, “A vos te llama” (M\_193)
\end{exe}

(\ref{ex:14:48}) and possibly also (\ref{ex:14:49}) suggest that realis mood <-hi> may combine with the copula. 

\begin{exe}
\ex\label{ex:14:48}
<Rica vchanagaquehi> / <Rica vcha naga quehi> \\
\gll rica  vchana    ga-que-{hi} \\
you  brave    2-\textsc{cop-rea} \\
\glt ‘You (\textsc{sg}) are brave’, “Tú eres valiente” (M\_738)
\ex\label{ex:14:49}
<Quancaquehe?> / <Quancaquehé?>  \\
\gll quan  ca{}-que-{he}? \\
good  2-\textsc{cop-rea} \\
\glt ‘How have you (\textsc{sg}) been?’, “Cómo te ha ido?” (M\_026)
\end{exe}

When there is no copula, the realis mood marker directly attaches to the predicatively used adjective in constructions such as (\ref{ex:14:50}--\ref{ex:14:51}).

\begin{exe}
\ex\label{ex:14:50}
<Quahini mijinahé?> \\
\gll qua-{hi} ni   mijinahe? \\
good-\textsc{rea}  \textsc{q}  dog \\
\glt ‘Is your (\textsc{sg}) dog good?’, “Es bueno tu perro?” (M\_271)
\ex\label{ex:14:51}
<Quahi>   \\
\gll qua-{hi} \\
good\textsc{{}-rea} \\
\glt ‘He/it is good’, “Bueno es” (M\_272)
\end{exe}

The same phenomenon is attested in (\ref{ex:14:52}--\ref{ex:14:56}) below. 
Realis mood <-hi> seems to have a variant <-bi>, which is mostly used when the subject person is a speech-act participant, specifically with 1\textsuperscript{st}-person subjects in declarative clauses and with 2\textsuperscript{nd}-person subjects in interrogative clauses.

\begin{exe}
\ex\label{ex:14:52}
<Ninga chatize guabi> \\
\gll ninga  chatize  gua-{bi} \\
I  truth  say-\textsc{rea} \\
\glt ‘I tell the truth’, “Yo digo verdad” (M\_349)
\ex\label{ex:14:53}
<Ninga buxibi>  \\
\gll ninga   bu-xi-{bi} \\
I  \textsc{transl}{}-go-\textsc{rea}  \\
\glt ‘I go away’, “Yo me voy” (M\_320)
\ex\label{ex:14:54}
<Ninga buji-bi cogora> \\
\gll ninga   bu{}-ji-{bi} cogo{}-ra \\
I  \textsc{transl}{}-go-\textsc{rea}  house{}-\textsc{all} \\
\glt ‘I want to go home’, “Yo quiero irme a casa” (M\_291)
\ex\label{ex:14:55}
<Ynchua guabi> \\
\gll ynchua    gua-{bi}\\
lie    say-\textsc{rea}\\
\glt ‘I tell a lie’, “Digo mentira/hablo yanga” (M\_352) 
\ex\label{ex:14:56}
<Kabujibi?>  \\
\gll ka-bu-ji-{bi}? \\
2-\textsc{transl}{}-go-\textsc{rea}\\
\glt ‘Are you (\textsc{sg}) leaving already?’, “Ya te vas?” (M\_735)
\end{exe}

To what extent this distribution of <-bi> – in 1\textsuperscript{st}-person subject statements and 2\textsuperscript{nd}-person subject questions – can be linked to egophoric person marking, a relatively widespread feature in several languages of northwestern South America (e.g., \citealt{Knuchel2015}; \citealt{SanRoqueetal2018}), is not easy to establish, given the lack of data and the variability of transcriptions. Also, in a few cases, <-hi> occurs with a 2\textsuperscript{nd}-person subject in an interrogative clause (\ref{ex:14:57}), and  in a few instances, <-bi> occurs with a 3\textsuperscript{rd}-person subject (\ref{ex:14:58}). These cases, however, are exceptions in the available Andakí materials.

\begin{exe}
\ex\label{ex:14:57}
<Cafianzuhi ningaxa ca-yasesza?>\footnote{Hyphenation is due to a line break in the manuscript.} / <Cafianzuhi ningaxá cayasesza?> \\
\gll ca-fianzu-{hi} ninga-xa  caya-sesza? \\
2-like-\textsc{rea}  I-\textsc{obl}    live-\textsc{des} \\
\glt ‘Do you (\textsc{sg}) want to live with me?’, “Quieres vivir conmigo?” (M\_353) 
\ex\label{ex:14:58}
<Raxinachibi> / <Raxinachihic>  \\
\gll ra{}-xi    nachi-{bi} \\
\textsc{caus}{}-go   night{}-\textsc{rea} \\
\glt ‘Night fell’, “Anocheció” (M\_124)\footnote{This interpretation of <raxi> in (\ref{ex:14:58}) as ‘to cause to go, to bring’ is preliminary.}
\end{exe}

Realis mood <-hi> seems to have the variants <-ha>, <-fa>, and <-a>. The use of <-ha> and <-fa> with nouns in predicative function is illustrated in (\ref{ex:14:59}--\ref{ex:14:60}). 

\begin{exe}
\ex\label{ex:14:59}
<Bacoxe hane?> \\
\gll bacoxe{}-{ha} ne \\
\textit{chicha}-\textsc{rea}  \textsc{q}\\
\glt ‘Is there \textit{chicha}?’, “Hay chicha?” (M\_046)
\ex\label{ex:14:60}
<Noszuefane?> / <Noszuefané?> \\
\gll noszue-{fa} ne \\
 fish-\textsc{rea}  \textsc{q} \\
\glt ‘Is there fish?’, “Hay pescado?” (M\_324)
\end{exe}

In (\ref{ex:14:61}--\ref{ex:14:65}), we only tentatively interpret final <ha> as a realis mood marker. In (\ref{ex:14:61}--\ref{ex:14:62}), we present some examples containing the negation marker <para> and related forms.

\begin{exe}
\ex\label{ex:14:61}
<Ninga pajaha> / <Ninga pajahá> \\
\gll ninga  paja{}-{ha}  \\
I  \textsc{neg}{}-\textsc{rea} \\
\glt ‘I do not have’, “Yo no tengo” (M\_228)
\ex\label{ex:14:62}
\begin{xlist}
\ex\label{ex:14:62a}
<Psajaha> / <Pajaá> \\
\gll psaja{}-{ha}  \\
\textsc{neg}{}-\textsc{rea} \\
\glt ‘I do not have; there is not’, “Yo no tengo; no hay” (M\_229)
\ex\label{ex:14:62b}
<Pacahá> \\
\gll paca{}-{ha} \\
\textsc{neg}{}-\textsc{rea} \\
\glt ‘I do not have; there is not’, “No tengo; no hay” (M\_231)
\end{xlist}
\end{exe}

The negation marker may also occur without the realis mood marker, as illustrated by <pagá> ‘I do not have; there is not’, “no tengo; no hay” (M\_230). (\ref{ex:14:63}--\ref{ex:14:65}) show further instances of the presumed realis mood marker <-ha> {\textasciitilde} <-há>. 

\begin{exe}
\ex\label{ex:14:63}
<Kachiyaha> / <Kachiyahá> \\
\gll ka-chiya{}-{ha} \\
2-eat{}-\textsc{rea} \\
\glt ‘Did you (\textsc{sg}) already eat?’, “Ya comiste?” (M\_734)
\ex\label{ex:14:64}
<Raguayahá> \\
\gll ra-gua-ya{}-{ha} \\
\textsc{aor}{}-say-\textsc{pl}{}-\textsc{rea} \\
\glt ‘They said’, “Dijeron” (M\_729)
\ex\label{ex:14:65}
<Nanquaha> / <Nanquahá> \\
\gll nan{}-qua{}-{ha} \\
\textsc{cop{}-}good{}-\textsc{rea} \\
\glt ‘He/it is good’, “Bueno está” (M\_725)
\end{exe}

\subsection{Imperative\textsubscript{1}: <-(a)ba>} \label{sec:14.5.2}

There are different markers of the imperative mood in Andakí which, for practical purposes, will be called imperative\textsubscript{1}, imperative\textsubscript{2}, and imperative\textsubscript{3} here. The pragmatics of these different markers cannot be determined due to the nature of the available Andakí data. In the case of imperative\textsubscript{1} we find different forms for the 1\textsuperscript{st} and 2\textsuperscript{nd} person, discussed in \sectref{sec:14.5.2.1} and \sectref{sec:14.5.2.2}, respectively.  

\subsubsection{First person imperative\textsubscript{1}: <Ø{}-…-(a)ba>} \label{sec:14.5.2.1}

What we tentatively label imperative\textsubscript{1} in this work is marked by a morpheme <-ba>. Whereas in imperative mood, the 2\textsuperscript{nd} person is marked by a dedicated prefix, the 1\textsuperscript{st} person (hortative) is not. This is illustrated in (\ref{ex:14:66}). 

\begin{exe}
\ex\label{ex:14:66}
<Chiyaba gunfigo> \\
\gll chiya-{ba} gunfigo \\
eat-\textsc{imp}    one \\
\glt ‘Let us eat together!’, “Comamos juntos” (M\_293)
\end{exe}

As in the indicative mood, a zero morpheme (\textit{Ø-chiy}\textit{a{}-ba} ‘let us eat!’) may be postulated as a 1\textsuperscript{st}-person subject marker. 

Imperative <-ba> has a variant represented by <-aba>, which seems to occur only with the verb <buxi> (and variants) ‘to go away’, as illustrated in (\ref{ex:14:67}--\ref{ex:14:68}).

\begin{exe}
\ex\label{ex:14:67}
<Nahachi boxeaba => / <Nahachi boxeaba> \\
\gll nahachi   bo-xe-{aba} \\
night    \textsc{transl}{}-go-\textsc{imp} \\
\glt ‘Let us go away, it is late!’, “Vámonos, que es tarde” (M\_087)
\protectedex{
\ex\label{ex:14:68}
\begin{xlist}
\ex\label{ex:14:68a}
<Boxeaba => / <Boxeaba>  \\
\gll bo-xe-{aba}\\
\textsc{transl}{}-go-\textsc{imp} \\
\glt ‘Let us go away!’, “Vámonos” (M\_085)
\ex\label{ex:14:68b}
<Bujeaba> \\
\gll bu-je-{aba}\\
\textsc{transl}{}-go-\textsc{imp}\\
\glt ‘Let us go away!’, “Vámonos” (M\_281)
\end{xlist}
}
\end{exe}

Another variant of the imperative suffix is represented by <-aha>; its use with a 1\textsuperscript{st}-person singular subject being illustrated in (\ref{ex:14:69}).

\begin{exe}
\ex\label{ex:14:69}
<Quananquea chiguaca nunqueaha> / Quananquea chiguaca nunqueahá> \\
\gll qua-nan-que-a     chiguaca   nunque-{aha} \\
\textsc{2.imp-caus}{}-come-\textsc{imp}    children    look-\textsc{imp} \\
\glt ‘Bring me your (\textsc{sg}) children, I want to see them’, “Tráeme a tus hijos, que quiero verlos” (M\_289)
\end{exe}

\subsubsection{Second person imperative\textsubscript{1}: <qua{}-…-ba>} \label{sec:14.5.2.2}

With the imperative marked by <-ba> or a variant of it, 2\textsuperscript{nd}-person subjects are marked by a prefix <qua->, <coba->, <coha->, or <coa->. Some examples are shown in (\ref{ex:14:70}--\ref{ex:14:75}). 

\begin{exe}
\ex\label{ex:14:70}
<Firajiquichaza cobaquea rincaxahá> \\
\gll firaji    qui-chaza {coba}{}-que-{a} rinca-xaha \\
hungry    \textsc{cop}{}-\textsc{sub}  2.\textsc{imp}{}-come-\textsc{imp}  I-\textsc{obl} \\
\glt ‘When hungry, come (\textsc{sg}) to me!’, “En teniendo hambre ven a mí” (M\_762)
\ex\label{ex:14:71}
<Cohagea andagu>  \\
\gll {coha}{}-ge-{a} andagu\\
\textsc{2.imp-}go-\textsc{imp}  quickly \\
\glt ‘Go (\textsc{sg}) quickly!’, “Anda presto” (M\_296)
\ex\label{ex:14:72}
<Coagua anduazo> \\
\gll {coa}{}-gua  anduazo\\
\textsc{2.imp-}cook  banana\\
\glt ‘Cook (\textsc{sg}) bananas!’, “Cocina plátanos” (M\_227)
\protectedex{
\ex\label{ex:14:73}
<Quanca jehá sasza> / <Quanajeha sasza>  \\
\gll {quan}{}-caje-{ha} sasza \\
\textsc{2.imp-}wash-\textsc{imp}  clothes \\
\glt ‘Wash (\textsc{sg}) the clothes!’, “Lava la ropa” (M\_210)
}
\ex\label{ex:14:74}
<Quarichá>  \\
\gll {qua}{}-ri-{cha} \\
\textsc{2.imp-}drink-\textsc{imp} \\
\glt ‘Drink (\textsc{sg})!’, “Bebe”\footnote{That <ch> can refer to an allophone of the phoneme referred to by <h> is in line with pairs such as <cachinehe> / <cachineche> ‘grandson’, "nieto" (M\_490).} (M\_047)
\ex\label{ex:14:75}
<Quaxeba => / <Quaxeba> \\
\gll {qua}{}-xe-{ba} \\
2\textsc{.imp}{}-go-\textsc{imp}  \\
\glt ‘Go (\textsc{sg})!’, “Anda” (M\_083)
\end{exe}

The following case (\ref{ex:14:76}) is the only example where a personal pronoun occurs in an imperative construction of the kind discussed here.

\begin{exe}
\ex\label{ex:14:76}
<Quaxiha Rica> / <Quaxihá Ricá> \\
\gll qua-xi-ha {rica} \\
\textsc{2.imp}{}-go-\textsc{imp}  you \\
\glt ‘You (\textsc{sg}) go!’, “Anda vos” (M\_051)
\end{exe} 

In this kind of construction, the personal pronoun might be used for emphasis (see \sectref{sec:9}). Similar constructions without the personal pronoun are shown in (\ref{ex:14:75}) above and in (\ref{ex:14:77}). 

\begin{exe}
\ex\label{ex:14:77}
<Quexiha => / <Quexiha> \\
\gll {que}{}-xi-{ha} \\
\textsc{2.imp-}go-\textsc{imp} \\
\glt ‘Go (\textsc{sg})!’, “Anda” (M\_039)
\end{exe} 

\subsection{Imperative\textsubscript{2}: <-ra> {\textasciitilde} <{}-za>} \label{sec:14.5.3}

In single entries, verbs are presented with a suffix <-ra> or <-za> in the 18\textsuperscript{th}{}-century lists; these forms are translated as imperatives in Spanish. This is illustrated in (\ref{ex:14:78}--\ref{ex:14:80}).

\begin{exe}
\ex\label{ex:14:78}
<Chiyazá>\\
\gll chiya-{za}\\
eat-\textsc{imp}\\
\glt ‘Eat (\textsc{sg})!’, “Come” (M\_354)
\ex\label{ex:14:79}
<Finszuzá> \\
\gll finszu-{za}\\
clean-\textsc{imp}\\
\glt ‘Clean (\textsc{sg})!’, “Limpia” (M\_370)
\ex\label{ex:14:80}
<Cayaza> \\
\gll caya-{za}  \\
sit-\textsc{imp}\\
\glt ‘Sit down (\textsc{sg})!’, “Siéntate” (M\_376)
\end{exe} 

Andakí <-za> is glossed here as an imperative marker. It may also occur with a 2\textsuperscript{nd}-person plural subject, as illustrated in (\ref{ex:14:81}).

\begin{exe}
\ex\label{ex:14:81}
<Chuhuaza> / <Chuhuazá> \\
\gll chuhua-{za}  \\
hear{}-\textsc{imp} \\
\glt ‘Hear (\textsc{pl})!’, “Oíd” (M\_362)
\end{exe}

Andakí <-za> is also found with a 1\textsuperscript{st}-person subject in (\ref{ex:14:82}). Note that the subject person is expressed by a pronoun in this case.

\newpage
\begin{exe}
\ex\label{ex:14:82}
<Ninga nunquiza chin{}-queca riguacu>\footnote{The hyphen is due to a line break in the manuscript. The form <riguacu> ‘you.\textsc{pl}’ is unattested elsewhere in the available Andakí data.} / <Ninga nunquiza chinqueca riguacu> \\
\gll ninga  nunqui-{za} chinqueca  riguacu \\
I  look-\textsc{imp}  children  you.\textsc{pl} \\
\glt ‘I have to look upon you (\textsc{pl}) as children’, “Yo os he de mirar como a hijos” (M\_241)
\end{exe}

A possible use of <-za> with a 3\textsuperscript{rd}-person subject is shown in (\ref{ex:14:83}), which, however, may also address a 2\textsuperscript{nd}-person subject: given that in polite commands Spanish uses the 3\textsuperscript{rd}-person subjunctive, this form is ambiguous.

\begin{exe}
\ex\label{ex:14:83}
<Quihizá> \\
\gll quihi-{za} \\
come-\textsc{imp}\\
\glt ‘Let him/her/it come; may she/he/it come; come (\textsc{sg})!’, “Que venga” (M\_410)
\end{exe}

It is unclear to what extent the different imperatives, called imperative\textsubscript{1} and imperative\textsubscript{2} here for practical purposes, differ in illocutionary force and politeness. Judging from the Spanish translations provided in the available language materials, the imperative constructions in the present section (imperative\textsubscript{2}) seem to have the same meaning as those with <-(a)ba> and variants (imperative\textsubscript{1}), illustrated in the previous section. This is also suggested by the translations shown in (\ref{ex:14:84}--\ref{ex:14:87}), illustrating the two different imperative markers with the verb ‘to come’ (\ref{ex:14:84}--\ref{ex:14:85}) and with the verb ‘to bring’ (\ref{ex:14:86}--\ref{ex:14:87}).

\begin{exe}
\ex\label{ex:14:84}
<Quaqͧiha => / <Quaquiha> \\
\gll {qua}{}-qui-{ha}\\
\textsc{2.imp-}come-\textsc{imp} \\
\glt ‘Come (\textsc{sg})!’ “Vení” (M\_038)
\ex\label{ex:14:85}
<Quihizá> \\
\gll quihi-{za} \\
come-\textsc{imp}\\
\glt ‘Come (\textsc{sg})!’, “Ven” (M\_419)
\newpage
\ex\label{ex:14:86}
<Quananqueha Jexe. => / <Quananqueha Jexé> \\
\gll {qua}{}-nan-que-{ha} jexe \\
\textsc{2.imp-caus}{}-come-\textsc{imp}    water\\
\glt ‘Bring (\textsc{sg}) water!’, “Trae agua” (M\_074)
\ex\label{ex:14:87}
<Jixe nanqueza> \\
\gll jixe   nan-que-{za}  \\
water  \textsc{caus}{}-come-\textsc{imp} \\
\glt ‘Bring (\textsc{sg}) water!’, “Trae agua” (M\_031)
\end{exe} 

Note that the object (‘water’) follows the imperative verb in (\ref{ex:14:86}) and precedes it in (\ref{ex:14:87}). In some instances, there is a variant <-ra> of the imperative suffix, illustrated in (\ref{ex:14:88}--\ref{ex:14:89}); once more, it is not entirely clear whether the utterances address a 2\textsuperscript{nd} or a 3\textsuperscript{rd} person.

\begin{exe}
\ex\label{ex:14:88}
<Quijra> \\
\gll quij-{ra} \\
come-\textsc{imp}\\
\glt ‘Let him/her/it come; may she/he/it come; come (\textsc{sg})!’, “Que venga” (M\_160) 
\protectedex{
\ex\label{ex:14:89}
<Andagu quira> / <Andaguquira> \\
\gll andagu   qui-{ra} \\
quickly  come-\textsc{imp} \\
\glt ‘Let him/her/it come; may she/he/it come quickly; come quickly (\textsc{sg})!’, “Que venga presto” (M\_155)
}
\end{exe} 

In one instance, we found <-ca> as a variant  of imperative <-za>; this is illustrated in (\ref{ex:14:90}). 

\begin{exe}
\ex\label{ex:14:90}
<Raxicá> / <Racizá> \\
\gll ra-xi-{ca} \\
\textsc{caus-}go-\textsc{imp} \\
\glt ‘Carry (\textsc{sg})!’, “Lleva” (M\_379)
\end{exe} 

The graphemic variation in the context of imperative\textsubscript{2} needs further investigation; since the underlying rules are not yet understood. Despite its more frequent occurrence as <{}-za>, we suggest in \sectref{sec:5.3} that the underlying, phonemic form of the Andakí imperative\textsubscript{1} suffix is /{}-ra/.

\subsection{Imperative\textsubscript{3}: <-ni>} \label{sec:14.5.4}

The word lists also provide us with imperative constructions that seem to contain a suffix or enclitic <-ni>, such as <sani> ‘wait (\textsc{sg})!’, “espera” (M\_066), <szuzini> ‘stay (\textsc{pl}) here!’, “estaos quedo” (M\_236), <fsatani> / <fsâtani> ‘shut up, do not talk (\textsc{sg})!’, “calla, no hables” (M\_238). We have not been able to identify the verbal roots in question in other entries, which makes it difficult to interpret <-ni> as an imperative marker, except in the cases of (\ref{ex:14:91}--\ref{ex:14:92}); Andakí <qua> ‘to make’ is also attested elsewhere, for instance, in (\ref{ex:14:8}) and (\ref{ex:14:30}) above.

\begin{exe}
\ex\label{ex:14:91}
<Quani nanqua cogo>  \\
\gll qua{}-{ni} nan{}-qua  cogo \\
make-\textsc{imp}  \textsc{cop{}-}good  house \\
\glt ‘Make (\textsc{pl}) a good ranch; they should make a good ranch!’, “Hagan buen rancho” (M\_215)
\ex\label{ex:14:92}
<Cogo quani naqua> \\
\gll cogo   qua{}-{ni} na{}-qua \\
house  make-\textsc{imp}  \textsc{cop{}-}good \\
\glt ‘You have to make me a house!’, “Me habéis de hacer casa” (M\_249)
\end{exe} 

Note the different word order and translations in (\ref{ex:14:91}) and (\ref{ex:14:92}).\footnote{An anonymous reviewer observes that the translation does not reflect all elements of the original text in  (\ref{ex:14:92}). As mentioned in \sectref{sec:4} above, this happens in several cases in the available Andakí data.} The element <-ni> occurs not only with verbs, but also with adverbials, as illustrated in (\ref{ex:14:93}).\footnote{An anonymous reviewer observes that (\ref{ex:14:93}) looks like a rare case of verb root ellipsis.}

\begin{exe}
\ex\label{ex:14:93}
<Quixarani> \\
\gll quixara-{ni}\\
far-\textsc{imp} \\
\glt ‘Let us go far away!’, “Vamos lejos” (M\_736)
\end{exe} 

The adverbial expression ‘far’, “lejos” is documented elsewhere as <quejara> (M\_557). Finally, a suffix <-no> may express imperative mood in <yquano> / <yguano> ‘listen (\textsc{sg})!’, “oiga” (M\_264). The root <yqua> / <ygua> may be related to <shungua> ‘to hear, understand’ (cf. M\_731), but it is unclear how <-no> relates to <-ni>.

\subsection{Prohibitive: <ni-…-qua>} \label{sec:14.5.5}

The prohibitive is marked by a circumfix <ni-…-qua>. For illustrative purposes, we juxtapose an imperative and a prohibitive form in (\ref{ex:14:94}--\ref{ex:14:95}). The imperative form is shown in (\ref{ex:14:94}), the prohibitive form in (\ref{ex:14:95}).

\begin{exe}
\ex\label{ex:14:94}
<Quaquaha> / <Quaquahá>  \\
\gll {qua}{}-qua-{ha}\\
\textsc{2.imp-}fight\textsc{{}-imp} \\
\glt ‘Go fight (\textsc{sg})!’, “Anda pelea” (M\_300) 
\ex\label{ex:14:95}
<Niquaquaha> / <Niquaquahá> \\
\gll {ni}{}-qua{}-{quaha}\\
\textsc{proh-}fight{}-\textsc{proh}\\
\glt ‘Do not fight!’, “No peleéis” (M\_301)
\end{exe} 

The prohibitive suffix has several variants, among which are also <-qua> and <-coha>, illustrated in (\ref{ex:14:96}) and (\ref{ex:14:97}), respectively.\footnote{An anonymous reviewer suggests that the suffix <-qua> and its variants may also be analyzed as the 2\textsuperscript{nd}-person imperative marker. The fact that all provided examples involve 2\textsuperscript{nd}-person referents suggests this as a possible interpretation.}

\begin{exe}
\ex\label{ex:14:96}
<Ninaquá> (M\_197), <Ninaqua> (M\_237) \\
\gll {ni}{}-na-{qua} \\
\textsc{proh}{}-cry-\textsc{proh}\\
\glt ‘Do not cry (\textsc{sg})!’, “No llores” (M\_197; M\_237)\footnote{In another context, the root ‘to cry’ appears in a slightly different form: <naha> – compare \textit{naha-za} ‘cry!’, "llora" (M\_384).} 
\ex\label{ex:14:97}
<Nibugicoha> / <Nibujicohá>\\
\gll {ni}{}-bu-ji-{coha} \\
\textsc{proh}{}-\textsc{transl}{}-go-\textsc{proh} \\
\glt ‘Do not go away (\textsc{sg})!’, “No te vayas” (M\_314)
\end{exe} 

A remarkable phenomenon in Andakí is that the verb ‘to give’, illustrated in (\ref{ex:14:98}--\ref{ex:14:101}), looks like a prohibitive construction; the respective parts are highlighted in bold. However, the Spanish translations suggest otherwise and it has not been possible to analyze the constructions in question in more detail. The remarkable combination of letters <fsrr> probably refers to a sibilant (\citealt[90--91]{CoronasUrzúa1994}; see also \sectref{sec:5.3.5}). 

\begin{exe}
\ex\label{ex:14:98}
<Nifsrraquahá> \\
\gll \textbf{ni}fsrra\textbf{quaha}  \\
give \\
\glt ‘You will give’, “Daréis” (M\_726)
\ex\label{ex:14:99}
<Nifsrranquazá> \\
\gll \textbf{ni}fsrran\textbf{qua}{}-za \\
give-\textsc{imp} \\
\glt ‘Give (\textsc{sg}) me!’, “Dame” (M\_415)
\ex\label{ex:14:100}
<Bacoza nifsrranquaza>\\
\gll bacoza \textbf{ni}fsrran\textbf{qua}{}-za   \\
\textit{mazato}   give-\textsc{imp} \\
\glt ‘Give (\textsc{sg}) me \textit{mazato}!’, “Dame mazato” (M\_309)
\protectedex{
\ex\label{ex:14:101}
<Bacoxe nifsrran => / <Bacoxe nifssrân{qua}za>\\
\gll bacoxe   \textbf{ni}fssran\textbf{qua}-za \\
\textit{chicha}    give-\textsc{imp} \\
\glt ‘Give (\textsc{sg}) me \textit{chicha}!’, “Dame chicha” (M\_032)
}
\end{exe} 

The two forms shown in (\ref{ex:14:101}), <nifsrran> and <nifssrânquaza>, each attested in a different version of Mutis’ Andakí materials, suggest that final <qua> in <nifsrranqua> is indeed a separate morpheme.\footnote{The formal similarity of <nifsrran> / <nifsrraqua> ‘to give’ with the construction <nifsrracaquahá> ‘do not stumble!’, “no tropieces” (M\_750) is remarkable.} 

\section{Modality} \label{sec:14.6}

There are several markers of modality in Andakí, that is, morphemes that express inner states and attitudes of a speaker and indicate irrealis/counterfactual notions of will, obligation, necessity, or ability. This section will discuss the expression of desiderative (\sectref{sec:14.6.1}) and non-desiderative (\sectref{sec:14.6.2}) modality in Andakí.

\subsection{Desiderative: <-zea>} \label{sec:14.6.1}

The morpheme <-zea> and its variants <-cea> and <-seza> encode a desiderative or volitive meaning. Their use is illustrated in (\ref{ex:14:102}--\ref{ex:14:103}).

\begin{exe}
\ex\label{ex:14:102}
<Ningua coayazea rica co-axa>\footnote{The hyphenation of <ricaco-axa> is due to a line break in the source.} / <Ningua coayazea ricacoaxa>  \\
\gll ningua    coaya-{zea} rica-coa-xa \\
I    live-\textsc{des}  you-\textsc{pronpl-obl} \\
\glt ‘I want to live with you (\textsc{pl})’ / “Yo quiero vivir con vosotros” (M\_239)
\ex\label{ex:14:103}
<Cafianzuhi ningaxa ca-yasesza?>\footnote{The hyphenation of <ca-yasesza> is due to a line break in the manuscript.} / <Cafianzuhi ningaxá cayasesza?> \\
\gll ca-fianzu-hi  ninga-xa  caya-{sesza}?\\
2-like-\textsc{rea}  I-\textsc{obl}    live{}-\textsc{des}\\
\glt ‘Do you (\textsc{sg}) want to live with me?’, “Quieres vivir conmigo?” (M\_353) 
\end{exe} 

Note that unlike (\ref{ex:14:102}), (\ref{ex:14:103}) additionally contains the element <fianzu> ‘like'. Yet not every construction translated as a wish contains a desiderative marker, as shown in (\ref{ex:14:104}), where desiderative modality is not explicitly marked and found only in the Spanish translation.

\begin{exe}
\ex\label{ex:14:104}
<Chiya nenguihi => / <Chiya nênguihi> \\
\gll chiya   nenguihi \\
eat  fish \\
\glt ‘I want to eat fish’, “Quiero comer pescado” (M\_082)
\end{exe} 

Another case is shown in (\ref{ex:14:105}).

\begin{exe}
\ex\label{ex:14:105}
<Quananquea chiguaca nunqueaha> / Quananquea chiguaca nunqueahá> \\
\gll qua-nan-que-a   chiguaca   nunque-{aha}\\
\textsc{2.imp-caus}{}-come-\textsc{imp}  children  look-\textsc{hort}\\
\glt ‘Bring your (\textsc{sg}) children to me, I want to see them’, “Tráeme a tus hijos, que quiero verlos” (M\_289)
\end{exe} 

Here, a 1\textsuperscript{st}-person imperative/hortative construction is translated with a desiderative meaning in Spanish.

\subsection{Non-desiderative: <ficoa>} \label{sec:14.6.2}

In Andakí, non-desiderative or non-volitive meaning is expressed by an element <ficoa> or one of its variants <fica> (M\_206), <ficoha> (M\_251), <figua> (M\_049), or <ficora> (M\_323). This grammatical element is frequently but not always combined with a negation marker. Although its length suggests that it consists of several morphemes, so far it has been impossible to analyze it in more detail. The non-desiderative marker is frequently attested in the 18\textsuperscript{th}{}-century Andakí materials. In (\ref{ex:14:106}--\ref{ex:14:107}), <ficoha> and <ficora> occur together with the negative prefix <ra->. 

\begin{exe}
\ex\label{ex:14:106}
<Ninga rahasza ficoha>  \\
\gll ninga {ra}{}-hasza {ficoha}\\
I  \textsc{neg}{}-leave   \textsc{ndes} \\
\glt ‘I do not want to leave you (\textsc{sg})’, “Yo no quiero dejarte” (M\_251)
\ex\label{ex:14:107}
<Ramaficora>\footnote{Attested in Ms. II/2912 only. This entry follows the form <ramahi> ‘she/he/it died’, “murió”.} \\
\gll {ra}{}-ma {ficora} \\
\textsc{neg}{}-die  \textsc{ndes} \\
\glt ‘Do not die!’, “No muráis” (M\_323)
\end{exe}

In (\ref{ex:14:108}), the non-desiderative marker occurs in combination with the unbound negation marker <para>.

\begin{exe}
\ex\label{ex:14:108}
<Pará riszifigua> \\
\gll {para} riszi {figua} \\
\textsc{neg}  drink   \textsc{ndes}\\
\glt ‘I do not want do drink’, “No quiero beber” (M\_049)
\end{exe}

In (\ref{ex:14:109}), it occurs alone, without further negation markers, in a construction that has the same translation as (\ref{ex:14:108}).

\begin{exe}
\ex\label{ex:14:109}
<Riscificoa> \\
\gll risci {ficoa} \\
drink  \textsc{ndes} \\
\glt ‘I do not want do drink’, “No quiero beber” (M\_259)
\end{exe}

The non-desiderative morpheme can also occur in preverbal position, as shown in (\ref{ex:14:110}). Note that unlike in the other examples discussed in this subsection, the verb in (\ref{ex:14:110}) is in the prohibitive mood.  

\begin{exe}
\ex\label{ex:14:110}
<Ficaca neaszacoha> / <Ficaca neaszacohá> \\
\gll {fica} ca   ne-asza-coha  \\
\textsc{ndes}   you  \textsc{proh}{}-leave{}-\textsc{proh} \\
\glt ‘Do not leave \textsc{(sg)} me alone!’, “No me dejes solo” (M\_206)
\end{exe}

Whether or not the prohibitive marker <-coha> as attested in <neaszacoha> is etymologically related to <coa> or <coha> as in <ficoa>, <ficoha>, and the related forms illustrated above, remains to be established. In (\ref{ex:14:110}) we interpret <ca> as a truncated form of the 2\textsuperscript{nd}-person singular pronoun; prohibitive <ni-> does not seem to occur together with <ca->; compare \sectref{sec:14.5.5}. 

\section{Suppletion: <yu> and <qui> ‘to come’} \label{sec:14.7}

Andakí has two suppletive roots with the meaning ‘to come’: <yu> and <qui>. The root <qui> ‘to come’ is used in forms translated with the Spanish \textit{indefinido} (preterite/aorist), as well as with the imperative; <yu> ‘to come’ is used in all other contexts. The use of <qui> with forms translated with the Spanish \textit{indefinido} is shown in (\ref{ex:14:111}--\ref{ex:14:114}). 

\begin{exe}
\ex\label{ex:14:111}
<Canafizecaque?> / <Canafizecaquí?> \\
\gll canafize  ca-{que}? \\
yesterday  2-come \\
\glt ‘Did you (\textsc{sg}) come yesterday?’; “Ayer viniste?” (M\_107)\footnote{The function and meaning of <ca> in <canafize> are unknown; however, it is probably not a truncated version of <rica> ‘you’. Compare <najize> ‘yesterday’ in (\ref{ex:14:112}) and in the isolated entry <canajisexa> ‘yesterday’, “ayer” (M\_470).}
\ex\label{ex:14:112}
<Jahá najizequi> / <Jaha najizequi> \\
\gll jaha  najize {qui} \\
yes  yesterday  come \\
\glt ‘Yes, I came yesterday’, “Sí, ayer vine” (M\_108)
\ex\label{ex:14:113}
<Ningaqui> \\
\gll ninga {qui} \\
I  come \\
\glt ‘I came’, “Vine” (M\_311)
\ex\label{ex:14:114}
<Quihi> \\
\gll quihi \\
come \\
\glt ‘I came’, “Vine” (M\_733)
\end{exe}

In some cases, <qui> is used in constructions that are translated with the Spanish perfect tense, as in (\ref{ex:14:115}).\footnote{Another example seems to be (M\_240), which contains, however, language material that cannot be analyzed yet.} 

\begin{exe}
\ex\label{ex:14:115}
<Ypchize canahá quine?> / <Ypchizé canahá quiné?> \\
\gll ypchize  ca-naha{}-{qui} ne?  \\
what    2-\textsc{caus}{}-come  \textsc{q} \\
\glt ‘What have you brought?’, “Qué has traído?” (M\_303)
\end{exe}

In some exceptional cases, <yu> ‘to come’ is also used in constructions translated with the Spanish perfect, as shown in (\ref{ex:14:116}).

\begin{exe}
\ex\label{ex:14:116}
<Sazi cayuni?> \\
\gll sazi     ca-{yu} ni? \\
how/where  2-come  \textsc{q} \\
\glt ‘How have you (\textsc{sg}) come?’, “Cómo has venido?” (M\_308)
\end{exe}

The root <qui> is also used in the context of imperatives, as shown in (\ref{ex:14:117}--\ref{ex:14:118}). 

\begin{exe}
\ex\label{ex:14:117}
\begin{xlist}
\ex\label{ex:14:117a}
<Quaquiha> / <Quaqͧiha =>  \\
\gll qua{}-{qui}{}-ha \\
\textsc{2.imp-}come-\textsc{imp} \\
\glt ‘Come (\textsc{sg})!’ “Vení” (M\_038)
\ex\label{ex:14:117b}
<Quaque hà> / <Quacuoha =>\footnote{There is a stroke on the <o> in the manuscript, which is difficult to interpret.}  \\
\gll qua-{que}{}-ha \\
\textsc{2.imp-}come-\textsc{imp} \\
\glt ‘Come (\textsc{sg})!’, “Ven” (Ms. II/2911) / “Vení” (Ms. II/2912) (M\_084)
\end{xlist}
\ex\label{ex:14:118}
<Quihiza> \\
\gll {quihi}{}-za \\
come-\textsc{imp} \\
\glt ‘Come (\textsc{sg})!’, “Vení” (M\_127)
\end{exe}

Andakí <qui> ‘to come’ has a counterpart in Nasa Yuwe. The Nasa Yuwe form in question is /kĩh/ ‘to reach from above’, “llegar desde arriba” \citep[392]{DiazMontenegro2019}, ‘to go down’, “descender” \citep[401]{DiazMontenegro2019}.

In all other contexts, that is, except with the aorist and the imperative, Andakí <yu> ‘to come’ is used. The root <yu> resembles Nasa Yuwe /ju/- ‘to come’ (cf. \citealt[136]{DiazMontenegro2019}; \citealt{Pache2024}). The Andakí root in question is illustrated in (\ref{ex:14:119}--\ref{ex:14:121}). 

\begin{exe}
\ex\label{ex:14:119}
<Yubi ninga> \\
\gll {yu}{}-bi     ninga  \\
come-\textsc{rea}  I\\
\glt ‘I come’, “Yo vengo” (M\_256)
\ex\label{ex:14:120}
<Cayubi?> \\
\gll ca-{yu}{}-bi?\\
2-come-\textsc{rea} \\
\glt ‘Do you (\textsc{sg}) come?’, “Vienes?” (M\_257)
\ex\label{ex:14:121}
<Niyuhe?> / <Niyuche?> \\
\gll ni {yu}{}-he? \\
\textsc{q}  come\textsc{{}-rea} \\
\glt ‘Does she/he/it come?’; ‘Has she/he/it come?’, “Viene?”; “Ha venido?” (M\_024)
\end{exe}

As an exception, in (\ref{ex:14:122}), we also found <yu> to be used in an aoristic context.

\begin{exe}
\ex\label{ex:14:122}
<Rica cayuhe> \\
\gll rica  ca-{yu}{}-he  \\
you  2-come-\textsc{rea} \\
\glt ‘Did you (\textsc{sg}) already come?’, “Tú ya viniste?” (M\_312)
\end{exe}

Notwithstanding single cases that are not yet fully understood, it seems that Andakí has at least one case of verb suppletion depending on aspectual/tense notions (aoristic or not) and mood (imperative or not). 

\section{Copula: <qui>} \label{sec:14.8}

While the copular prefix <na-> or <nan-> appears to be limited to cleft constructions (see \sectref{sec:10} and \sectref{sec:13}), Andakí also makes use of a copula <qui> or <que>, which is found across a broader range of constructions, including statements (\ref{ex:14:123}), questions (\ref{ex:14:124}), and subordinate constructions (\ref{ex:14:125}).\footnote{An anonymous reviewer notes the formal similarity with the verb ‘to come'.} Its use is illustrated in (\ref{ex:14:123}--\ref{ex:14:125}). Note that in (\ref{ex:14:123}) the use of the copula seems to be optional, given that it is found only in Ms. II/2912, which seems to be by and large a copy of Ms. II/2911. 

\begin{exe}
\ex\label{ex:14:123}
<Rica chajañuca> / <Rica chajanûca quihi> \\
\gll rica  chajanuca  {qui}{}-hi  \\
you  relative  \textsc{cop{}-rea} \\
\glt ‘You (\textsc{sg}) are my relative’, “Tú eres mi pariente” (M\_745)\footnote{The prefix <ca-> ‘2\textsuperscript{nd}-person subject' may have been omitted on <quihi> in order to avoid a sequence <caca>.}
\ex\label{ex:14:124}
<Vnajagaquehé?> \\
\gll vnaja  ga-{que}{}-he? \\
rich  2-\textsc{cop{}-rea} \\
\glt ‘Are you (\textsc{sg}, \textsc{m}) rich?’, “Eres rico?” (M\_741)
\ex\label{ex:14:125}
<Firajiquichaza cobaquea rincaxahá> \\
\gll firaji {qui}{}-chaza   coba-que-a     rinca-xaha \\
hungry  \textsc{cop}{}-\textsc{sub}  2.\textsc{imp}{}-come-\textsc{imp}  I-\textsc{obl} \\
\glt ‘When hungry, come \textsc{(sg)} to me!’, “En teniendo hambre ven a mí” (M\_762)
\end{exe}

The copula also occurs in predicative possessive constructions, as in (\ref{ex:14:126}--\ref{ex:14:130}). The number of the addressee is impossible to determine in (\ref{ex:14:126}--\ref{ex:14:130}), given the existence of verbal \textit{voseo} in the Spanish variety used in the 18\textsuperscript{th}{}-century Andakí materials. The addressee in (\ref{ex:14:130}) may possibly be a 2\textsuperscript{nd}-person singular, given that ‘Yes, I have a husband’, “Sí, tengo marido” (\ref{ex:12:1}, \sectref{sec:12}), seems to be the answer to ‘Do you have a husband?’, “Tenéis marido?” shown in (\ref{ex:14:130}).

\begin{exe}
\ex\label{ex:14:126}
<Chigua cancaqué?> \\
\gll chiguaca  nca {que}? \\
children  \textsc{prop}  be \\
\glt ‘Do you have children?’, “Tenéis hijos?” (M\_174)
\ex\label{ex:14:127}
<Taranguehé Kaque?> \\
\gll taranguehe   ka {que}? \\
chicken  \textsc{prop}  \textsc{cop} \\
\glt ‘Do you have chickens?’, “Tenéis gallinas?” (M\_226)
\ex\label{ex:14:128}
<Anduozo Kaquij> \\
\gll anduozo  ka {qui}{}-j? \\
banana    \textsc{prop}  \textsc{cop{}-rea} \\
\glt ‘Do you have bananas?’, “Tenéis plátanos?” (M\_144)
\newpage
\ex\label{ex:14:129}
<Guaca juncaquehe?> / <Guaca juncaquehé> \\
\gll guacaju  nca {que}{}-he? \\
canoe    \textsc{prop}  \textsc{cop{}-rea} \\
\glt ‘Do you have canoes?’, “Tenéis canoas?”\footnote{In Ms. II/2911, the Spanish translation says: "Tenéis cañas?".} (M\_246)
\ex\label{ex:14:130}
<Chanca encaquehé?> / <Chanca encaquehé?> \\
\gll chanca    enca {que}{}-he?\\
husband  \textsc{prop}  \textsc{cop{}-rea} \\
\glt ‘Do you have a husband?’, “Tenéis marido?” (M\_172)
\end{exe}   

Note that we only have access to examples with 2\textsuperscript{nd}-person subjects in the context of the Andakí copula, which limits the possible generalizations that can be made. 
