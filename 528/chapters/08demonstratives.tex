\chapter{Demonstratives: <ri-> ‘proximate’, <chi-> ‘distal’} \label{sec:8}

Andakí demonstratives encode at least a two-way contrast. The demonstrative adverb <rini> (M\_553) or <rimi> (M\_261) is translated as ‘here’, “aquí”. The pronominal equivalent is <rihizi> ‘this’, “este” (M\_014). Its use is illustrated in (\ref{ex:8:1}).

\begin{exe}
\ex\label{ex:8:1}
<Rajiza rihizi> \\
\gll ra-ji-za {rihizi} \\
\textsc{caus}{}-go-\textsc{imp} this\\
\glt ‘Take (\textsc{sg}) this!’, “Lleva a este” (M\_194)
\end{exe}   

The distant demonstrative adverbs are <chinta> / <chintá> ‘there’, “allá” (\mbox{M\_177}; M\_552) and <chini> ‘there’, “allí” (M\_554). Their pronominal equivalent is <chihi> or <chichi> / <chihi> ‘that’, “aquel” (M\_015). This suggest that <-ni> in <rini> ‘here’ and <chini> ‘there’ has a locative meaning (see \sectref{sec:11.3}).\footnote{The demonstrative morphemes <ri-> ‘proximate’ and <chi-> ‘distal’ are reminiscent of Nasa Yuwe \textit{na} ‘\textsc{dem.prox}’ and \textit{txã}, \textit{txa} ‘\textsc{dem.dist}’ (see \citealt[276, 295, 764]{DiazMontenegro2019}). For a further possible instance of Andakí \textit{r} and \textit{i} corresponding with Nasa Yuwe \textit{n} and \textit{a}, see \fnpageref{footnote:11:6}.} There are no examples that would illustrate the use of Andakí demonstratives as pronominal modifiers.

