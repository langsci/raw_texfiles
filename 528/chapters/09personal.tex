\chapter{Personal pronouns: <ninga> ‘I’, <rica> ‘you (\textsc{sg})’, <ricacua> ‘you (\textsc{pl})’} \label{sec:9}

The data in the word lists contain Andakí 1\textsuperscript{st}- and 2\textsuperscript{nd}-person pronouns, but no 3\textsuperscript{rd}-person pronouns. The 1\textsuperscript{st}-person singular pronoun has several variants, among which are <ninga> (M\_010), <ningua> (M\_239), <ringa> (M\_011), or <dingá> / <dinga> (M\_012). There are no examples of 1\textsuperscript{st}-person plural pronouns. Accordingly, we do not know whether Andakí made a distinction between the 1\textsuperscript{st}-person plural inclusive and exclusive. 

The 2\textsuperscript{nd}-person singular personal pronoun is <rica> / <ricá> (M\_013) and has a variant <recá> (M\_050). The 2\textsuperscript{nd}-person plural pronoun is derived from the singular form by the plural marker <{}-coa> {\textasciitilde} <{}-qua>, as shown in (\ref{ex:9:1}).

\begin{exe}
\ex\label{ex:9:1}
<Ninga yubi ricacoaxa fsrrane janora joazajuexa riguanto ajaze> / <Ninga yubi rica coaxa fsr{}-rane janora joazajuexa ringuanto ajaze>\footnote{The hyphen is due to a line break in the manuscript.}  / \\
\gll ninga  yu{}-bi {rica}{}-{coa}{}-xa    fsrranejano{}-ra joazajue-xa  riguanto  ajaze\\
I  come{}-\textsc{rea}  you{}-\textsc{pronpl}{}-\textsc{obl}  teach{}-\textsc{all} God-\textsc{obl}  believe    \textsc{purp} \\
\glt ‘I come to teach you (\textsc{pl}) the law of God so that you may believe in him’, “Vengo a enseñaros la ley de Dios para que creáis en él” (M\_135)
\end{exe}  

The plural morpheme <-coa> {\textasciitilde} <-qua> occurs only with pronouns, not with nouns. The element <ri> as attested in <ringa> ‘I’, “yo” (M\_011) and <rica> / <ricá> ‘you’, “tú” (M\_013) is probably identical with the element <ri> that occurs in <rini> ‘here’, “aquí” (M\_553) or <rihizi> ‘this’, “este” (M\_014) (see \sectref{sec:8}).

In several instances, the use of personal pronouns appears to be optional in Andakí. In (\ref{ex:9:2}) the 1\textsuperscript{st}-person subject is not overtly expressed at all. 

\begin{exe}
\ex\label{ex:9:2}
<Quihi> \\
\gll quihi \\
come \\
\glt ‘I came’, “vine” (M\_733)
\end{exe}  

In (\ref{ex:9:3}), the 2\textsuperscript{nd}-person subject is expressed by a prefix or proclitic only. 

\begin{exe}
\ex\label{ex:9:3}
<Cachiya?> / <Cachiyá?> \\
\gll {ca}{}-chiya? \\
2-eat \\
\glt ‘Have you eaten?’, “Habéis comido?” (M\_043)
\end{exe} 

With imperative constructions, there is usually no pronoun (see \ref{ex:9:4}). 

\begin{exe}
\ex\label{ex:9:4}
<Quexiha> / <Quexihá> \\
\gll que-xi-ha  \\
\textsc{2.imp-}go-\textsc{imp}\\
\glt ‘Go (\textsc{sg})!’, “Anda” (M\_039)
\end{exe} 

Yet, we found one instance in the data of an imperative construction with a 2\textsuperscript{nd}-person pronoun, which follows the verb (\ref{ex:9:5}).

\begin{exe}
\ex\label{ex:9:5}
<Quaxiha Rica> / <Quaxihá Ricá> \\
\gll qua-xi-ha {rica} \\
\textsc{2.imp}{}-go-\textsc{imp}  you \\
\glt ‘You (\textsc{sg}) go!’, “Anda vos” (M\_051)
\end{exe} 

The Spanish translation suggests that in (\ref{ex:9:5}), the pronoun is added for emphasis. As a general rule, personal pronouns may precede or follow a verb in Andakí. In (\ref{ex:9:6}--\ref{ex:9:7}), subject person is expressed by a pronoun that precedes the verb.

\begin{exe}
\ex\label{ex:9:6}
<Ningaqui> \\
\gll {ninga} qui \\
I  come \\
\glt ‘I have already come’, “Ya yo vine” (M\_311)
\ex\label{ex:9:7}
<Rica cachiya?> / <Rica cachiyá?>  \\
\gll {rica} ca-chiya? \\
you  2-eat \\
\glt ‘Have you (\textsc{sg}) eaten?’, “Vos habéis comido?” (M\_044)
\end{exe} 

In (\ref{ex:9:8}--\ref{ex:9:9}), subject person is expressed by a pronoun that follows the verb.

\begin{exe}
\ex\label{ex:9:8}
<Yubi ninga> \\
\gll yu-bi {ninga}  \\
come-\textsc{rea}  I \\
\glt ‘I come’, “Yo vengo” (M\_256)
\ex\label{ex:9:9}
<Careszerecá?> \\
\gll ca-resze {reca}? \\
2-drink you \\
\glt ‘Did you (\textsc{sg}) drink yet?’, “Ya bebiste?”” (M\_050)
\end{exe} 

While the personal pronoun refers to an intransitive subject in, for instance, (\ref{ex:9:6}), it expresses object person in (\ref{ex:9:10}). 

\begin{exe}
\ex\label{ex:9:10}
<Ninga ca-mimi?> \\
\gll {ninga} ca-mimi?  \\
I  2-love \\
\glt ‘Do you love me?’, “Me queréis?” (M\_232)
\end{exe} 

Note that there is no oblique case marker attached to the pronoun in (\ref{ex:9:10}). In (\ref{ex:9:11}), by contrast, the personal pronoun referring to the direct object carries an oblique case marker <-xa>. The oblique case marker is discussed in detail in \sectref{sec:11.1}.

\begin{exe}
\ex\label{ex:9:11}
<Ricaxa fifihe> / <Ricaxa fifihé> \\
\gll {rica-xa} fifi-he \\
you-\textsc{obl}  call-\textsc{rea} \\
\glt ‘It is you (\textsc{sg}) she/he/it calls’, “A vos te llama” (M\_193)
\end{exe} 

Data are too scarce to determine the reasons underlying the pattern of object marking illustrated in (\ref{ex:9:10}--\ref{ex:9:11}). Note that the word order in the Spanish translation of (\ref{ex:9:11}) is highly marked.