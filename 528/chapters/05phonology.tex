\chapter{Phonology} \label{sec:5}

This chapter provides a sketch of Andakí phonology, based largely on a paper dedicated to this topic \citep{CoronasUrzúa1994}. An overview of Andakí phonology must take into account the limited availability of data and hence of minimal pairs, and the fact that we rely on data that are some 180 to 230 years old. The graphemes and grapheme combinations used in these sources do not necessarily represent phonemes, and their use is often quite variable. Variation in the orthography may have its origin, for instance, in spelling inconsistencies; in other instances, it may reflect alternation or free variation of sounds. An example of this variation is given in (\ref{ex:5:1}), where the root ‘to eat’ is given as <chiya> and <choya> and the purposive marker as <-jazi>, <-jaze>, <-xaré>, and <-xazé>.

\begin{exe}
\ex\label{ex:5:1}
\begin{xlist}
\ex\label{ex:5:1a}
<Chiya jaze> \\
\gll chiya-jaze \\
eat-\textsc{purp} \\
\glt ‘In order to eat’, “Para comer” (M\_147)
\ex\label{ex:5:1b}
<Choyaxazé> / <Choya xaré> \\
\gll choya-xaze \\
eat-\textsc{purp} \\
\glt ‘In order to eat’, “Para comer” (M\_186)
\end{xlist}
\end{exe}

In a number of cases, forms provided in \citet{Albis1860--1861} can be quite different from those found in the 18\textsuperscript{th}{}-century lists, as shown in the case of <condefui> \mbox{(A\_039)} in the 19\textsuperscript{th}{}-century list and the related <szuntijohé> (M\_656) in the 18\textsuperscript{th}{}-century list, both ‘tapir’, “danta”. Given that these differences in spelling are not always systematic, it is not easy to determine to what extent such differences between the 18\textsuperscript{th} and 19\textsuperscript{th}{}-century sources reflect dialectal differences, distinct ways to represent Andakí sounds or, in the case of the ‘tapir’ term, a reference to different species. Neither for the 18\textsuperscript{th}{}-century nor for the 19\textsuperscript{th}{}-century materials, is there explicit information on how Andakí sounds were pronounced and on how exactly the data were gathered. Finally, there are several homographs in the data, suggesting that Andakí may have had many homophonous forms, unless tone, which is not represented in the spelling, played a role. According to  \citegen[87]{CoronasUrzúa1995}  analysis of Andakí phonology, /kʷa/, for instance, can have the meanings ‘to kill’, ‘to cook’, ‘to do’, and ‘to say’, among others. Another example of Andakí homophony is <minzará> ‘sand’, “arena” (M\_695) versus <minzará> a tree, \textit{Genipa americana}, “jagua” (M\_696).

\section{Phonotactics, syllable and root structure} \label{sec:5.1}

The Andakí basic syllable structure is CV, as in <rica> / <ricá> ‘you’, “tú” (\mbox{M\_013}). A syllable structure V or VC is attested only in exceptional cases such as in <acai> ‘lemon’, “limón” (A\_001) or in <ypchize> ‘what?’, “qué” (M\_303; \mbox{M\_304}). If a word starts with a vowel, it is mostly \textit{a} followed by <n>, as in <antazá> ‘up (\textsc{sg})!, raise (\textsc{sg}) it!’, “alza” (M\_380); <andagu> ‘fast’, “presto” (M\_550) – compare <dacoze> ‘fast’, “presto” (M\_549), which seems to contain a related root *dako – or <anduazo> ‘banana’, “plátano” (M\_113) – compare <duazo> ‘banana’, “plátano” (M\_054). There are different interpretations possible for this phenomenon, and we tentatively propose here that the \textit{a} is epenthetic before a prenazalized stop in word-initial position (pace \citealt{CoronasUrzúa1994}). If nasal consonants occur in the syllable coda in other environments, they indicate nasality of the preceding vowel, for instance, in <ringa> ‘I’, “yo” (M\_011), /rĩka/ (\citealt[74--75]{CoronasUrzúa1994}). CVC syllables do not occur in Andakí. Consonant clusters in intervocalic position do not seem to exist either, except in single cases such as <ypchize> ‘what?’, “qué” (M\_303; M\_304). Consonant clusters in the onset of a syllable are exceedingly rare if they exist. We found a few potential examples which include a nasal consonant /n/ and a sibilant, as in <nszajini?> / <nszãjini> ‘where are they?’, “dónde están?” (M\_176); elsewhere, we find the respective forms without initial <n>: <sxaxini> ‘where is she/he/it?’, “dónde está?” (M\_067). Similar clusters involving <n> and grapheme combinations referring to sibilants are attested in <nszazincaha> / <nszâzincahá> ‘because I (\textsc{m}) was sick’, “porque estuve enfermo” (M\_326) or <nszazincahe?> / <nszazincaque?> ‘are you (\textsc{sg}, (\textsc{m}) sick?’, “estás enfermo?” (M\_321). The possibility cannot be excluded, however, that we are not dealing with consonant clusters in onset position in these cases, but that <n> in these examples refers to a syllabic nasal consonant, as proposed by \citet[95]{CoronasUrzúa1994}. In other cases, grapheme combinations – like <sz>, <fsrr>, or <jx>, as in <jxiizá> / <jxizá> ‘walk (\textsc{sg})!’, “camina” (M\_365) – seem to refer to single sounds such as sibilants (in the case of <sz> and <fsrr>) or to a laryngeal fricative (in the case of <jx>), not to consonant clusters (for discussion, see \sectref{sec:5.2}). 

Sequences of same and distinct vowels both occur. Sequences of same vowels are discussed below, in {\sectref{sec:5.2.6}}. Among the sequences of distinct vowels, <ea> or <ae> are among the most frequently attested; the sequence <oa> is less common in Andakí, <ao> does not exist at all in the available sources; an overview of the exact frequencies of particular combinations is provided in \citet[79–80]{CoronasUrzúa1994}. Some of these sequences, such as <oa>, may also be interpreted as a sequence of a glide plus a syllabic vowel. In single instances, sequences of up to three vowels are attested: compare <janszeaezá> ‘get brave (\textsc{sg}, \textsc{m})!’, “hazte bravo” (M\_412).

Most Andakí roots have the structure CVCV, as, for instance, <chisi> ‘to urinate’ (cf. M\_389). Monosyllabic roots are infrequent — for instance, <bi> ‘to carry’ (cf. M\_378) or <ko> ‘to blow’ (cf. M\_381) (cf. \citealt{CoronasUrzúa1995}). Grammatical morphemes often have the structure CV, such as <-ni> ‘locative’ or <ra-> ‘causative’ (see \sectref{sec:11} and \sectref{sec:14}); only a few polysyllabic grammatical elements have been identified here, such as non-desiderative <ficoa> (see \sectref{sec:14.6.2}). 

\section{Vowels and suprasegmentals} \label{sec:5.2}

The Andakí vowel inventory has been argued to contain six vowels: /a/, /i/, /u/, and their nasal counterparts /ã/, /ĩ/, /ũ/ \citep[85]{CoronasUrzúa1994}; it is shown in \tabref{tab:5.1}.

\begin{table}
\begin{tabular}{l ccc}
\lsptoprule
 & { Front} & { Central} & { Back}\\
\midrule
{ High} & { i, ĩ} &  & { u, ũ} \\
{ Non-high} &  & { a, ã} & \\
\lspbottomrule
\end{tabular}
\caption{Vowel phoneme inventory of Andakí as established by \citet[80]{CoronasUrzúa1994} and translated into English}
\label{tab:5.1}
\end{table}

There is little evidence for the existence of a high central vowel \textit{ɨ}, \textit{ɨ̃}, which is frequently found in Lowland South American languages (cf. \citealt[8]{DixonAikhenvald1999}); if there was one in Andakí, it would probably be represented by <y>, which also seems to substitute <i> or <u> in some instances, as in <s\textbf{y}mquarahá> / <s\textbf{u}mquarahá> ‘smoke’,\footnote{The boldfacing in Andakí examples discussed in this and other sections is not in the original, but has been added by the authors.} “humo” (M\_531) and <\textbf{y}nsci> / <\textbf{y}nszi> ‘let us go away!’, “vámonos” (M\_313), <\textbf{i}nszi> ‘hortative; let us go!’, “vamos” (M\_129).

Mid vowels from loanwords are represented as such in the orthography of the Andakí sources: <e> is found in <buytr\textbf{e}ni> ‘vulture’ (M\_606, from Spanish \textit{buitre} ‘vulture’), <e> and <o> occur in <\textbf{o}v\textbf{e}jani> ‘sheep’ (M\_097, from Spanish \textit{oveja} ‘sheep’) where they may have been pronounced as mid vowels by Andakí speakers. By contrast, Spanish /e/ seems to have been adapted to <i> [i] in <par\textbf{i}ni> ‘priest’ (M\_560), which might be a loan from Spanish (\textit{padre} ‘priest’). 

\subsection{Suprasegmentals} \label{sec:5.2.1}

Little is known about Andakí suprasegmentals. There is no direct evidence for tone, although the high frequency of homographs in the Andakí sources suggests that lexical tone may have existed in this language. Acute or grave accent is nearly always found in word-final position and may indicate stress, for instance, in <Ypchize canahá quine?> / <Ypchizé canahá quiné?> ‘What did you (\textsc{sg}) bring?’, “Qué has traído?” (M\_303); there are no indications for stress being a contrastive feature in Andakí \citep[97]{CoronasUrzúa1994}. 

\subsection{Vowel nasality} \label{sec:5.2.2}

The sequence of a vowel and <n> or <m> has been interpreted as representing a nasal vowel \citep[74--75]{CoronasUrzúa1994}; in a few instances, nasality has also been argued to be represented by a circumflex on the vowel or by a postposed <ñ> \citep[74--75]{CoronasUrzúa1994}. Among the few sets suggesting a phonemic distinction between a nasal and an oral vowel in Andakí is <r\textbf{i}caxa> ‘to you (\textsc{sg})’, “a vos, a ti” (M\_017) versus <r\textbf{in}caxa> ‘to me’ (cf. M\_105). The orthography used in the different sources is not always consistent, and sometimes, a vowel seems to have been nasal in one case and oral in another case. This is illustrated by <sh\textbf{un}tahé> ‘deer’, “venado” (M\_657) with a nasal \textit{ũ} versus <sh\textbf{o}tajihi> ‘deer’ with non-nasal \textit{o}, as it occurs in <mixinchi sh\textbf{o}tajihi> / <mixinehi sh\textbf{o}tajihi> ‘jaguar’, “tigre” (M\_659) – compare <mixenehi> / <mixinehi> ‘dog’, “perro” (M\_065). 

\subsection{/i/, /ĩ/} \label{sec:5.2.3}

In the available Andakí data, <i> seems to appear in free variation with <e> when these letters refer to a syllabic vowel, as in <nanquiz\textbf{i}> (M\_451) versus <nanguiz\textbf{e}> / <nañguiz\textbf{e}> (M\_094), both ‘meat’, “carne”, in <cant\textbf{i}juch\textbf{i}> (M\_504) versus <cant\textbf{e}joch\textbf{e}> / <cant\textbf{e}joch\textbf{é}> (M\_505), both ‘pineapple’, “piña” \citep[73]{CoronasUrzúa1994}, or in <\textbf{i}nca> (M\_284) versus <\textbf{e}nca> / <\textbf{e}ncá> (M\_173), both ‘ownership’ (see \sectref{sec:12}). There is no minimal pair suggesting a phonemic distinction between \textit{i} and \textit{e}; the contrast between <guajij\textbf{e}> ‘pot’, “olla” (A\_056) versus <guajij\textbf{i}> ‘small pot’, “olleta” (A\_057) rather reflects sound symbolism. In some instances, the Andakí forms in the 18\textsuperscript{th} century lists have <i> where they have <e> (sometimes <a>) in the 19\textsuperscript{th} century list, as in <guahiz\textbf{i}> / <guajix\textbf{i}> (M\_110) versus <guajij\textbf{e}> (A\_056), both ‘pot’, “olla”, or in <bis\textbf{i}nguay> / <bint\textbf{i}guay> (M\_647) versus <mand\textbf{e}guae> (A\_084), both ‘bat’, “murciélago”. Both in the 18\textsuperscript{th}{}-century and 19\textsuperscript{th}{}-century lists, <i> is more frequent than <e> \citep[74]{CoronasUrzúa1994}. The factors conditioning the choice of <i> or <e> are not yet understood; in vowel sequences, if preceded by <a>, <e> occurs more frequently than <i>, both in the 18\textsuperscript{th}{}- and 19\textsuperscript{th}{}-century sources \citep[79]{CoronasUrzúa1994}. 

Among the few minimal pairs opposing <i> to <a> is <nanqu\textbf{i}zi> ‘meat’, “carne” (M\_451) and <nãnqu\textbf{a}zi> / <nanqu\textbf{a}zi> ‘beautiful’, “bonito” (M\_724) \citep[74]{CoronasUrzúa1994}, and <quinaj\textbf{i}> ‘head’, “cabeza” (A\_117) versus <quinaj\textbf{a}> ‘hair of the head’, “cabello” (A\_116). A near-minimal pair opposing Andakí /i/ to /u/ is <chincah\textbf{i}> ‘brother-in-law’, “cuñado” (M\_573) versus <chuncah\textbf{á}> ‘all’, “todo” (M\_263). Non-syllabic /i/ is represented by <y> in intervocalic position, as in <chi\textbf{y}azi> ‘food’, “comida” (M\_041) and in <guati\textbf{y}e> ‘stone’, “piedra” (A\_062), or in final position, as in <bisingua\textbf{y}> / <bintigua\textbf{y}> ‘bat’, “murciélago” (M\_647). 

\subsection{/u/, /ũ/} \label{sec:5.2.4}

In the available Andakí sources, <u> seems to appear in free variation with <o>, as in <b\textbf{u}jeaba> (M\_281) versus <b\textbf{o}xeaba>, both ‘let us go away!’, “vámonos” (M\_085). Possible factors that determine the choice of <u> or <o> in the available language materials remain unknown. In several cases, <u> in the 18\textsuperscript{th}{}-century list corresponds to <o> in the 19\textsuperscript{th}{}-century lists, as in <f\textbf{u}chigua> (\mbox{M\_580}) versus <\textbf{o}chegua> (A\_110), both ‘hook’, “anzuelo”. Indeed, the grapheme <u> occurs more frequently than <o> in the 18\textsuperscript{th}{}-century sources, whereas <o> is slightly more frequent than <u> in the 19\textsuperscript{th}{}-century list \citep[73]{CoronasUrzúa1994}. There are some instances where /u/ is represented by <v>, as in <\textbf{v}ntahe> / <\textbf{u}ntahé> ‘moon’, “luna” (M\_464) or in <tara\textbf{u}nhé> / <tara\textbf{v}nhé> ‘chicken’, “gallina” (\mbox{M\_098}). We found no minimal pair suggesting an opposition between \textit{u} and \textit{o} in this language.

\subsection{/a/, /ã/} \label{sec:5.2.5}

The third vowel to be discussed here is /a/ <a>. Among the (near-)minimal pairs opposing /a/ <a> to /i/ <e> are <s\textbf{a}cahá> ‘hands’, “manos” (M\_431) versus <sz\textbf{e}cahá> ‘foot, leg’, “pie, pierna” (M\_428), <ant\textbf{a}zá> ‘up (\textsc{sg})!, raise (\textsc{sg}) it!’, “alza” (M\_380) versus <ant\textbf{e}zá> ‘shout (\textsc{sg})!’, “grita” (M\_375), and, possibly, <m\textbf{a}nsesai> ‘large turtledove’, “tórtola grande” (A\_090) versus <m\textbf{e}nsesai> ‘small turtledove’, “tórtola pequeña” (A\_094).\footnote{This pair, however, like the pair <guajije> ‘pot’, “olla” (A\_056) versus <guajiji> ‘small pot’, “olleta” (A\_057), discussed above, may rather reflect a widespread pattern of sound symbolism, with the high front vowel \textit{i} associated with smallness (e.g., \citealt{Jespersen1933}).} A minimal pair opposing /a/ <a> to /u/ <o> is <k\textbf{a}g\textbf{á}> ‘sweet potato’, “batata” (M\_500) versus <c\textbf{o}g\textbf{o}> ‘farm’, “casa, rancho” (M\_512). Additionally, it has been observed that Andakí /a/ is realized as a back vowel in some cases, for instance, in <andu\textbf{o}zo> ‘banana’, “plátano” (M\_144) versus <andu\textbf{a}zo> ‘bananas’, “plátanos” (M\_227) \citep[74]{CoronasUrzúa1994}. In other cases, /a/ may have been realized as a high front vowel, for instance, in the case of <naqu\textbf{i}> (M\_002) versus <naqu\textbf{a}> (M\_003), both ‘who?’, “quién”. The determining factors remain unknown.

\subsection{Sequences of identical vowels} \label{sec:5.2.6}

Sequences of identical vowels are widely found in the available Andakí data \citep[79]{CoronasUrzúa1994}. That we are dealing with vowel sequences, not with long vowels, is supported by the observation that in several of these cases, one of the vowels carries an acute accent and the other does not, as in <Kariszina aqu\textbf{eé}?> ‘Do you use to drink?’, “Sabéis beber?” (M\_152). Sequences of identical vowels occur much less frequently in the \citet{Albis1860--1861} list than in the two Mutis lists \citep{Anonymous-a, Anonymous-b}, and they are split by <h> in most cases in the latter source, although there are exceptions such as <g\textbf{uu}he> / <g\textbf{uu}hé> ‘other’, “otro” (M\_016) \citep[77--78]{CoronasUrzúa1994}. It is not yet clear what intervocalic <h> indicates if it occurs between two identical vowels, and whether it may refer, for instance, to voicelessness or aspiration.\footnote{An anonymous reviewer suggests that <h> may have referred to a glottal stop.} 

Sequences of identical vowels that are split by <h> must be carefully distinguished from cases where a suffix \textit{-hV} is attached to a word or morpheme that ends in the same vowel as the suffix vowel. In Andakí <quihi> ‘I came’, “vine” (M\_733), for instance, final <-hi> must not be confused with the realis mood marker <-hi> (see \sectref{sec:14.5.1}), because <hi> is also present in Andakí <quihiza> ‘come (\textsc{sg})!’, “vení” (M\_127), which contains an imperative mood suffix \mbox{<-za>}. Although in this particular example the matter is clear, this is not always the case. 

Most frequently, same-vowel sequences in one source do not correspond to same-vowel sequences in the other source – as in <sonj\textbf{uhu}> ‘netbag’, “mochila de red” (M\_628) versus <suj\textbf{u}> ‘netbag’, “mochila” (A\_154). Only in a few instances does a vowel sequence of the 18\textsuperscript{th}{}-century source correspond to a vowel sequence in the 19\textsuperscript{th}{}-century source, as in <sac\textbf{ahá}> (M\_431) ‘hands’, “manos” versus <sac\textbf{aà}> ‘hand’, “mano” (A\_121) \citep[78]{CoronasUrzúa1994}, or in <j\textbf{ihi}> ‘spirit’, “demonio” (M\_461) versus <g\textbf{ii}> ‘spirit’, “diablo” (A\_050). A minimal pair that seems to oppose <ihi> to <i> is <qu\textbf{ihi}> ‘to come’ (cf. M\_410) versus <qu\textbf{i}> copula ‘to be’ (cf. M\_762). 

In most but not all cases, sequences of identical vowels occur in word-final position \citep[78]{CoronasUrzúa1994}. There are hardly any minimal pairs opposing a sequence of identical vowels to a singleton vowel – an exception is the \mbox{(near-)}minimal pair <sac\textbf{aà}> ‘hand’, “mano” (A\_121) versus <sacc\textbf{a}> ‘reed’, “caña” (A\_124).  

\section{Consonants} \label{sec:5.3}

The Andakí consonant inventory proposed by \citet[96]{CoronasUrzúa1994} and adapted in the present work is shown in \tabref{tab:5.2}. 

\begin{table}
\resizebox{\textwidth}{!}{%
\begin{tabular}{l@{~~}c@{~~}cc@{~~}cc@{~~}c}
\lsptoprule
 & Bilabial & Alveolar & Prepalatal & Velar & Labiovelar & Laryngeal\\
\midrule
Voiceless stops & p & t [t, d, ⁿd] &  & k [k, ɡ, ᵑk] & kʷ [kʷ, ɡʷ, w] & \\
Voiced stop & b [b, ɸ, β] &  & (ɟ)\footnotemark{} &  &  & \\
Affricate &  &  & ʧ &  &  & \\
Voiceless fricatives &  & s [s, ʃ] &  & x [x, h]\footnotemark{} &  & h [h, x, ɸ]\\
Nasals & m [m, b] & n &  &  &  & \\
Trill  &  & r [ɾ, n, h] &  &  &  & \\
\lspbottomrule
\end{tabular}%
}
\caption{Consonant phoneme inventory of Andakí as adapted from \citet[96]{CoronasUrzúa1994}}
\label{tab:5.2}
\end{table}

\addtocounter{footnote}{-2}
\stepcounter{footnote}\footnotetext{ Proposed as represented by <y> in \citet[88]{CoronasUrzúa1994}; there are no minimal pairs suggesting that <y> refers to a phoneme that is distinct from /i/.}
\stepcounter{footnote}\footnotetext{ This phoneme is tentatively proposed in the present contribution; it is not proposed by \citet{CoronasUrzúa1994}.}

The historical phonology of Andakí still leaves many open questions, yet is beyond the scope of this book. In several instances, forms seem to be etymologically related, yet the sound changes leading to the distinct forms are difficult to trace; one of these cases is <nanszihisze> ‘two’, “dos” (M\_625), which seems to contain an element <szihi>, related to <szifi> ‘eyes’, “ojos” (M\_423) (see \sectref{sec:7}). Andakí <szifi> ‘eye’, in turn, may be related to <chipi> as attested in <chipina> ‘face’, “rostro” (M\_425). As illustrated below, there is considerable variation in the graphemes used in the manuscript; variation seems to be particularly frequent before <i>. 

\subsection{/p/} \label{sec:5.3.1}

Three bilabial consonant phonemes have been postulated for Andakí by \citet{CoronasUrzúa1994}: /p/, /b/, and /m/.  The voiceless bilabial stop /p/ is represented by <p> in the orthography of the available sources and frequently occurs before /a/ or /ã/, less frequently before front vowels and hardly before back vowels. It does not occur in complementary distribution with other bilabial consonants but in the same environments \citep[81]{CoronasUrzúa1994}. 

(Near-)minimal pairs suggesting a phonemic distinction between Andakí /p/ and /b/ or /m/ are <\textbf{p}iszihi> ‘rooster’, “gallo” (M\_595) versus <\textbf{b}iszica> ‘man’, “hombre” (M\_732) and <\textbf{m}uizihi> / <\textbf{m}iszihi> ‘man’, “hombre” (M\_459).\footnote{It is impossible to know whether the underlying phoneme is /b/ or /m/ in the Andakí term for ‘man’ (cf. also \citealt[94]{CoronasUrzúa1994}).} Andakí /p/ <p> is opposed to /b/ <b> in augmentative <-\textbf{p}i> {\textasciitilde} <-\textbf{p}ihi> versus diminutive <-\textbf{b}i> {\textasciitilde} <-\textbf{b}ihi> (see \sectref{sec:15.5}; sound symbolism may play a role in this context). A minimal pair opposing /p/ <p> to /k/ <k> is <\textbf{p}aga> / <\textbf{p}agá> ‘manioc’, “yuca" (M\_499) versus <\textbf{k}agá> ‘sweet potato’, “batata” (M\_500). 

\subsection{/b/} \label{sec:5.3.2}

The phoneme /b/ has been argued to be represented by <b> and <v> \citep[87--88]{CoronasUrzúa1994}. The grapheme <v>, probably representing a voiced bilabial fricative allophone of /b/ – mostly in 19\textsuperscript{th}{}-century Andakí – occurs only before <e> or <i> – compare <\textbf{v}egaè> (A\_163) and <\textbf{b}icahi> (M\_630), both ‘bream’, “dorada”. Additionally, we suggest that there may have been a voiceless bilabial fricative variant of /b/ in 18\textsuperscript{th}{}- and 19\textsuperscript{th}{}-century Andakí alike, represented by <f> – this is suggested by the occurrence of forms such as <\textbf{b}iza> / <\textbf{b}izá> ‘what?’, “qué” (M\_063), and its variant <\textbf{f}iza> (cf. M\_035). We tentatively propose that <v> and <f> refer to bilabial fricatives rather than to labiodental fricatives. Another fricative variant of /b/ may have been represented by <h> or zero, as suggested, for instance, by the variants <-(a)\textbf{b}a> and <-(a)\textbf{h}a> of the imperative\textsubscript{1} marker (see \sectref{sec:14.5.2}) and by <\textbf{b}iza> / <\textbf{b}izá> (M\_063), alternating with <yza> ‘what?’ (cf. M\_146). In some single instances, we found evidence for variation between <b> and <m>, as this is possibly the case in <\textbf{b}iszica> (cf. M\_732) and <\textbf{m}uizihi> / <\textbf{m}iszihi> (M\_459), both ‘man’, “hombre”. In two further cases, there are some \textit{b} : \textit{m} correspondences between the data provided by Mutis \citep{Anonymous-a, Anonymous-b} and \citet{Albis1860--1861}: compare <\textbf{m}ajihi> ‘macao’, “guacamayo” (M\_599) versus <\textbf{b}afé> ‘macao’, “guacamaya” (A\_004) and <\textbf{b}isinguay> / <\textbf{b}intiguay> (M\_647) versus <\textbf{m}andeguae> (A\_084), both ‘bat’, “murciélago”. The ‘bat’ term suggests that there may have been nasalization of \textit{b} if it is followed by a nasal vowel, similar to what can incidentally be observed for \textit{r} \citep[87--88]{CoronasUrzúa1994}.

\subsection{/m/} \label{sec:5.3.3}

An Andakí phoneme /m/ has been argued to be represented by <m> \citep[94--95]{CoronasUrzúa1994}. The documentation of the two forms <muizihi> / <miszihi> ‘man’ (M\_459), suggests that \textit{w} is inserted between \textit{m} and \textit{i} in some cases.\footnote{A similar phenomenon is attested in Muisca, an extinct Chibchan language of Central Colombia, where a non-syllabic element \textit{w} is inserted if a labial consonant is followed by \textit{ɨ}, as in /mɨska/ [mʷɨska] ‘human being’ \citep[84]{AdelaarMuysken2004}. Whether the terms for ‘human being’ in Andakí and Muisca are related or not remains to be established. Borrowing of a term for ‘man’ or ‘human being’ is also attested in other cases in South America, for instance, in Quechua \textit{runa} \citep[949]{RosatPontacti2004} versus Cholón \textit{nun} ‘man, male person’ \citep[363, 525]{Alexander-Bakkerus2005}.} A possible variation between \textit{b} and \textit{m} in Andakí has been briefly discussed in the previous paragraph and by \citet[87--88]{CoronasUrzúa1994}. 

In other cases, /b/ <f> and /m/ <m> are distinct phonemes: a minimal pair identified in the available Andakí data, opposing /b/ to /m/ is <\textbf{m}ifi> /mibi/ ‘needle’, “aguja” (M\_482) versus <\textbf{f}ifi> / <\textbf{f}ifi> /bibi/ ‘to call’ (cf. M\_357).

\subsection{/t/} \label{sec:5.3.4}

Only one oral alveolar stop is postulated for Andakí by \citet[82]{CoronasUrzúa1994}: /t/. The non-existence of /d/ in \citegen{CoronasUrzúa1994} consonant inventory of Andakí is a remarkable gap; yet, there are no minimal pairs suggesting an opposition of Andakí \textit{t} and \textit{d}. In most cases where we find <d>, it occurs in the 19\textsuperscript{th}{}-cenutry word lists after a nasal vowel and corresponds to a <t> in the 18\textsuperscript{th}{}-century word lists: compare <bisinguay> / <bin\textbf{t}iguay> (M\_647) versus <man\textbf{d}eguae> (A\_084), both ‘bat’, “murciélago”, <szun\textbf{t}ijohé> (M\_656) versus <con\textbf{d}efui> (A\_039), both ‘tapir’, “danta” or <shun\textbf{t}ahé> (M\_657) versus <son\textbf{d}ai> both ‘deer’, “venado” (A\_152). That is, /t/ was probably realized as [d] in 19\textsuperscript{th}{}-century Andakí after nasal vowels. In a few cases, a similar variation is found within the 18\textsuperscript{th}{}-century materials, as in <an\textbf{t}agoni> ‘quickly’, “aprisa” (M\_555) versus <an\textbf{d}agu> ‘quickly’, “presto” (M\_550). That is, after a nasal vowel, and in a few cases, Andakí /t/ could facultatively be realized as [d] in the Andakí variety documented in the 18\textsuperscript{th} century as well \citep[82]{CoronasUrzúa1994}.

Elsewhere, <d> appears only in word-initial position before <i>; in the case of <\textbf{d}ingá> / <\textbf{d}inga> ‘I’ (M\_012), it represents an allophone of /r/ (see \sectref{sec:9}); in the cases of <\textbf{d}ifiacai> ‘sling’, “honda” (A\_041) and <\textbf{d}iscaza> / <discazá> ‘ribbon’, “cinta” (M\_614), the phoneme represented by <d> remains unclear; it is probably not /t/, since <t> also occurs in word-initial position followed by <i>, for instance, in <\textbf{t}iffi> ‘a plant, \textit{Macleania rupestris}’, “uva camarona” (M\_719) and in <\textbf{t}ijitiana> ‘forehead’, “frente” (A\_157) \citep[82]{CoronasUrzúa1994}. A possible explanation is that <difiacai> ‘sling’, “honda” (A\_041) and <discaza> / <dizcazá> ‘ribbon’, “cinta” (M\_614) are borrowings. Alternatively, the distribution of <t> and <d> is just not perfectly complementary and we are dealing with unknown reasons of variation. 

We found a (near-)minimal pair opposing /t/ <t> to /ʧ/ <ch>: <\textbf{t}iffi> ‘a plant, \textit{Macleania rupestris}’, “uva camarona” (M\_719) versus <\textbf{ch}ifi> ‘penis’, “\textit{membrum v.}” (M\_454). 

\subsection{/s/} \label{sec:5.3.5}

A voiceless alveolar fricative /s/ has been proposed by  \citet[89--92]{CoronasUrzúa1994}. It is represented by <s>, <z>, <sh>, <sz>, <ch> before <i>, and also by <sc>, and occasionally by <ch> and <x> in the 18\textsuperscript{th}{}-century materials; <ch> and <x> may refer to a retroflex and/or palatalized allophone of /s/ (cf. \citealt[89--91]{CoronasUrzúa1994}). There is no minimal pair opposing \textit{s} and \textit{ʃ} in Andakí. The use of different graphemes referring to a sibilant phoneme is illustrated in <\textbf{s}iyozá> ‘grind (\textsc{sg})!’, “muele” (M\_414), <\textbf{z}iyozá quifi> ‘grind (\textsc{sg}) corn!’, “muele maíz” (M\_214), <ni\textbf{s}inxé> / <ni\textbf{z}inxé> ‘it is already raining’, “ya llueve” (M\_218), <yn\textbf{sz}i> / <yn\textbf{sc}i> ‘let us go away!’, “vámonos” (M\_313), <min\textbf{sz}iguaxo> ‘weapon, fork’ “arma, cuchara” (M\_584), <min\textbf{ch}inaxo> ‘fork’, “cuchara” (M\_112), and possibly also <juatu\textbf{x}a> / <juatu\textbf{x}á> (M\_295), <juatu\textbf{s-z}a> / <juatu\textbf{sz}á> (M\_345), both ‘straight’, “derecho”. The sequences <fs>, <fsr>, and <fsrr> are attested only in the 18\textsuperscript{th}{}-century materials and have been argued to represent another allophone of /s/, probably with a retroflex realization \citep[90--91]{CoronasUrzúa1994}. The sequence <fsrr> is illustrated in (\ref{ex:5:2}--\ref{ex:5:3}). 

\begin{exe}
\ex\label{ex:5:2}
<Cafsrriaxi?> \\
\gll ca-\textbf{fsrr}iaxi? \\
2-forget \\
\glt ‘Did you (\textsc{sg}) forget?’, “Te olvidaste?” (M\_117)
\ex\label{ex:5:3}
<Nifsrrajiqua> / <Nifsrrajigua> \\
\gll ni-\textbf{fsrr}aji-qua  \\
\textsc{proh}{}-forget-\textsc{proh} \\
\glt ‘Do not forget (\textsc{sg})!’, “No te olvides” (M\_118)
\end{exe}

By contrast, in the \citet{Albis1860--1861} lists, only <s> is found as a symbol representing a sibilant, no further letters or letter combinations are used \citep[81]{CoronasUrzúa1994}. 

A minimal pair distinguishing between /s/ [s] <s> or [ʃ] <sz> on the one hand and /ʧ/ <ch> on the other hand is <\textbf{s}ifi> (A\_138), <\textbf{sz}ifi> (M\_423), both ‘eyes’, “ojos”, versus <\textbf{ch}ifi> ‘penis’, “\textit{membrum v.}” (M\_454), another such minimal pair is <\textbf{sz}inasza> ‘eyelash’, “pestaña” (M\_436) versus <\textbf{ch}inaszá> ‘mouth’, “boca” (M\_439) \citep[83]{CoronasUrzúa1994}. A (near-)minimal pair opposing /s/ [ʃ] <sh> to /ʧ/ <ch> is <\textbf{sh}unguaxe> ‘broth’, “caldo” (M\_138) versus <\textbf{ch}unguahe> / <chunguahé> ‘ears’, “orejas” (M\_424). Finally, a (near-)minimal pair opposing /s/ <s> to /r/ <r> is <sa\textbf{s}aguana> ‘bobbin’, “canilla” (A\_133) versus <sa\textbf{r}aguañae> ‘chicken’, “gallina” (A\_132). 

\subsection{/n/} \label{sec:5.3.6}

Another alveolar consonant phoneme proposed by \citet[94--95]{CoronasUrzúa1994} is the nasal stop /n/ <n>. The grapheme <n> may also represent an allophone of /r/ before a nasal vowel, as illustrated in \sectref{sec:5.3.7}. In syllable coda, the grapheme <n> is used to indicate nasality of the preceding vowels \citep[74--75]{CoronasUrzúa1994}.

Among the near-minimal pairs that have been identified in this context are <\textbf{n}isinxé> / <\textbf{n}izinxé> ‘it is already raining’, “ya llueve” (M\_218) versus <\textbf{r}iszizá> ‘drink (\textsc{sg})!’, “bebe” (M\_364), <\textbf{n}aszuhe> / <\textbf{n}aszuhé> ‘rheumatism’, “romadizo” (\mbox{M\_638}) versus <\textbf{m}ashu> / <\textbf{m}ashú> ‘ripe banana’, “plátano maduro” (M\_495) \citep[95]{CoronasUrzúa1994}. 

\subsection{/r/} \label{sec:5.3.7}

One liquid consonant has been postulated for Andakí: the rhotic /r/ (\citealt{CoronasUrzúa1994}: 95). In some cases, there seems to be free variation between <r>, <d>, and <n>, as in <\textbf{r}inga> (M\_011), <\textbf{d}ingá> / <\textbf{d}inga> (M\_012), and <\textbf{n}inga> (M\_010) ‘I’, “yo”. The fact that we find both <r> and <d> in this form suggests that <r> may have represented a tap [ɾ] here, rather than a trill \citep[82, 95]{CoronasUrzúa1994}. Note that the vowel that follows the rhotic phoneme is nasal in these examples, which may account for the occurrence of <n> in <\textbf{n}inga> (M\_010) ‘I’, “yo” \citep[95]{CoronasUrzúa1994}. This may also be compared with the observation of <r> in <\textbf{r}umpaguaza> (M\_619) and <n> in <\textbf{n}ampaguana> (A\_106),\footnote{The apparent lowering of <um> to <am> in this form and of <in> to <an> in the term for ‘bat’, <bisinguay> / <b\textbf{in}tiguay> (M\_647) versus <m\textbf{an}deguae> (A\_084), is discussed by \citet[77]{CoronasUrzúa1994}.}  both ‘poison’, “veneno”, that is, of <r> before a nasal vowel in the 18\textsuperscript{th}{}-century list item and of <n> in the 19\textsuperscript{th}{}-century list item. There are further examples of variation in the case of Andakí <r>: compare, for instance, the different forms of the causative marker, <\textbf{r}a-> (M\_194), <\textbf{z}a-> (M\_757), <\textbf{n}aha-> (M\_303), <\textbf{n}a-> / \mbox{<\textbf{n}á->} /\textbf{n}a-/ (M\_202), or <\textbf{n}an-> [\textbf{n}ã-] (M\_289). Variation of <r> and <z> occurs in the context of the allative marker <-\textbf{r}a> or <{}-\textbf{z}a> in <Ynszi yaha\textbf{r}a nâjubahá> versus <Ynszi yaha\textbf{z}a nâjubahá>  ‘Let us go to the river to wash ourselves!’, “Vamos al río a lavarnos” \mbox{(M\_254)}, in the imperative\textsubscript{2} marker <-\textbf{r}a> or <{}-\textbf{z}a> (\sectref{sec:14.5.3}), and in the truncated negation marker <\textbf{r}a-> / <\textbf{z}a->, illustrated in \sectref{sec:15}. In these cases, variation may have occurred between the rhotic and a voiced sibilant [ʒ] or [z]; the conditioning factors are yet unknown. In the unbound negation marker /pa\textbf{r}a/ <pa\textbf{r}á> (M\_049), variation is attested between <r>, <x>, and <g>: compare the negation markers <pa\textbf{x}a> / <pa\textbf{x}á> (M\_143) and <pa\textbf{g}a> / <pa\textbf{g}á> (M\_310). Another possible case of variation occurs in the case of <r> and <h>, as in <fico\textbf{r}a> (M\_323) versus <fico\textbf{h}a> (M\_251), both ‘non-desiderative’; the last examples of variation to be mentioned here occur in the different forms of the purpose marker /-ha\textbf{r}i/ (see \sectref{sec:11.5}): <xa\textbf{r}é> / <{}-xa\textbf{z}é> (M\_186), <-fan\textbf{x}e> / <{}-jan\textbf{x}e> (M\_728), \mbox{<-ja\textbf{ch}i>} \mbox{(M\_077)}, <-ja\textbf{h}e> / <{}-ja\textbf{h}é> (M\_204), <-ja\textbf{x}i> (M\_078), <ja\textbf{z}e> (M\_147), <-ja\textbf{z}o> \mbox{(M\_114)}.

A minimal pair opposing /r/ and /ʧ/ is <\textbf{r}ini> ‘here’, “aquí” (M\_553) versus <\textbf{ch}ini> ‘there’, “allí” (M\_554); a minimal pair opposing /r/ and /n/ is <-\textbf{r}a> ‘allative’ (see \sectref{sec:11.4}) versus <-\textbf{n}a> ‘pronominal genitive’ (see \sectref{sec:11.2}). 

\subsection{/ʧ/} \label{sec:5.3.8}

\citet{CoronasUrzúa1994} postulates the existence of a prepalatal voiceless affricate /ʧ/ <ch>. We found no evidence for allophonic variation in the case of /ʧ/ or for a representation of /ʧ/ by a letter or letter combination other than <ch>. 

A near-minimal pair opposing /ʧ/ to /h/ is <gua\textbf{ch}ixicá> / <gua\textbf{ch}ixizá> ‘enter (\textsc{sg})!’, “entra” (M\_407) versus <gua\textbf{j}ixi> ‘pot’, “olla” (M\_518); a near-minimal pair opposing /ʧ/ to /t/ is <gua\textbf{ch}ixicá> / <gua\textbf{ch}ixizá> ‘enter (\textsc{sg})!’, “entra” (M\_407) versus <gua\textbf{t}ihi> ‘gourd cup’, “mate” (M\_519). For minimal pairs opposing /s/ to /ʧ/ see \sectref{sec:5.3.5}. 

\subsection{/k/} \label{sec:5.3.9}

There is also a voiceless velar stop /k/ in Andakí \citep[83--85]{CoronasUrzúa1994}. It is represented by <k> (mostly before <a>, and only in the 18\textsuperscript{th}{}-century lists), <qu> (before <i> and <e>) or <c> (elsewhere) and often appears in variation with <g> word-internally and after nasal vowel, as in <nan\textbf{c}ohe> / <nan\textbf{c}ohé> (M\_601) versus <nan\textbf{g}ohe> / <nan\textbf{g}ohé> (M\_102), both ‘turkey’, “pava”. The distribution of <c>, <k>, and <g> is similar to that of <t> and <d> in that <g> is more frequent after nasal vowels (cf. \citealt[84]{CoronasUrzúa1994}). Although it is less common for <g> to occur word-initially \citep[84]{CoronasUrzúa1994}, this occurs with the 2\textsuperscript{nd}-person subject/agent marker <ka->, <ga->, illustrated in <\textbf{g}aqui> (M\_122) and <\textbf{k}aqui> (M\_133), both ‘you are’, “eres”. 

As to minimal pairs, we found a set opposing /k/ <qu> to /h/ or /x/ <x>, <j> in <\textbf{qu}ifi> ‘corn’, “maíz” (M\_214) versus <\textbf{x}ifi> ‘candle’, "candela" (M\_075), and, in the 19\textsuperscript{th}{}-century materials, in <\textbf{qu}ifi> ‘nose’, "nariz" (A\_115) versus <\textbf{j}ifi> ‘candle’, "candela" (A\_070). 

\subsection{/kʷ/} \label{sec:5.3.10}

A labialized velar /kʷ/ has been postulated as an Andakí phoneme \citep[85--86]{CoronasUrzúa1994}. It can be represented by sequences such as <hu>, <gu>, <qu>, <co>, and is most frequently followed by <a> \citep[85--86]{CoronasUrzúa1994}. The sequence <qu> or <gu> may indeed have referred to a labialized velar /kʷ/ rather than a sequence /ku/, since <qua> occurs quite frequently, but <pu>, <bu>, <tu>, followed by a vowel are not attested at all; <dua> exceptionally occurs in <\textbf{dua}zo> (M\_054), <an\textbf{dua}zo> (M\_113) and related forms for ‘banana’, “plátano” \citep[86]{CoronasUrzúa1994}. The sequence /kʷa/ is also transcribed as <coba> and possibly realized as [koba] in a few cases:\footnote{A similar development occurred in Buglere, a Chibchan language from Panama, see \citet[275]{Pache2018}.} compare the 2\textsuperscript{nd}-person imperative marker illustrated in (\ref{ex:5:4}), as opposed to (\ref{ex:5:5}--\ref{ex:5:6}). 

\begin{exe}
\ex\label{ex:5:4}
<Firajiquichaza cobaquea rincaxahá> \\
\gll firaji qui-chaza \textbf{coba}{}-que{}-a rinca-xaha \\
hungry \textsc{cop}{}-\textsc{sub} 2.\textsc{imp}{}-come-\textsc{imp} I-\textsc{obl} \\
\glt ‘When hungry, come (\textsc{sg}) to me!’, “En teniendo hambre ven a mí” (M\_762)
\ex\label{ex:5:5}
<Cohagea juatuxa> / <Cohajea juatuxá> \\
\gll \textbf{coha}{}-je-a   juatuxa \\
\textsc{2.imp-}go-\textsc{imp}  straight \\
\glt ‘Walk (\textsc{sg}) straight!’, “Anda derecho” (M\_295)
\ex\label{ex:5:6}
<Cohagea andagu> \\
\gll \textbf{coha}{}-ge-a  andagu \\
\textsc{2.imp-}go-\textsc{imp}  quickly \\
\glt ‘Go (\textsc{sg}) quickly!’, “Anda presto” (M\_296)
\end{exe} 

The Andakí sequence <gua> [ɡʷa] seems to alternate with <ba> [ba] in <\textbf{gua}tihi> (M\_204) versus <\textbf{ba}tihi> (M\_111), both ‘gourd cup’, “mate”; [kʷa] or [ɡʷa] may also alternate with [ka] in several cases, as illustrated by <\textbf{coa}ya> / <\textbf{coa}yá> (M\_088) versus <\textbf{ca}yazá> (M\_376), both ‘sit down (\textsc{sg})!’, “siéntate” or in the case of <fi\textbf{gua}> (M\_049) versus <fi\textbf{ca}> (M\_206) ‘non-desiderative’ (see \sectref{sec:14.6.2}).

The only minimal pair available in the context of Andakí /kʷ/ is <chata\textbf{gua}> ‘hill’, “loma” (M\_508) versus <chata\textbf{ya}> ‘big river’, “río grande" (M\_543). 

\subsection{/x/} \label{sec:5.3.11}

In contrast with \citet{CoronasUrzúa1994}, we tentatively propose the existence of a voiceless velar fricative /x/ in Andakí. It can be represented by <x>, rarely <h>, and is attested in the noun class marker <-xe> {\textasciitilde} <{}-xi> {\textasciitilde} <{}-he> {\textasciitilde} <{}-hi> ‘liquid’, which occurs in <je\textbf{xe}> / <je\textbf{xé}> ‘water’, “agua” (M\_074), <baco\textbf{xe}>, ‘corn-based beverage’, “chicha” (M\_032), <cahi\textbf{xi}> ‘corn-based beverage’, “mazamorra” (\mbox{M\_541}), <vnansza\textbf{xi}> / <unansza\textbf{xi}> ‘spirits’, “aguardiente” (M\_563); this noun class marker is represented by <-hi> only in a few cases, for instance, in <szaji\textbf{hi}> ‘lake’, “laguna” (M\_710). By contrast, Andakí has a realis mood marker /-hi/, discussed in \sectref{sec:14.5.1}, which is represented only by <-he> or <-hi>, but never by *<-xe> or *<-xi>. This suggests that <x> and <h> may, in some cases, have referred to different phonemes, /x/ and /h/. This distinction may have existed only in the Andakí variety documented in the 18\textsuperscript{th}{}-century materials; \citet{Albis1860--1861} writes <j> throughout. 

\subsection{/h/} \label{sec:5.3.12}

The voiceless laryngeal fricative /h/ is represented by <j>, <g>, <x>, <f>, and <h>. The symbols <h> and <x> occur in the 18\textsuperscript{th}{}-century materials only. Representing allophones or variants of /h/, the symbols <g>, <x>, and <f> exclusively or almost exclusively occur before <i> \citep[81, 92--94]{CoronasUrzúa1994}. Some of the orthographic variation in the context of Andakí /h/ is illustrated by the examples <\textbf{h}ijya> (M\_159) versus <\textbf{j}ijya> / <\textbf{j}ijyâ> (M\_019), both ‘who knows?’, “quién sabe”, or by <bu\textbf{j}eaba> (\mbox{M\_ 281}) versus <bo\textbf{x}eaba> (M\_085), both ‘let us go away!’, “vámonos”. The grapheme <h> hardly occurs in word-initial position, but it does occur in morpheme-initial position in the 18\textsuperscript{th}{}-century lists. The use of <g> as representing Andakí /h/ is illustrated by <ma\textbf{j}ixi> versus <ma\textbf{g}ixi> ‘food’, “comida” (M\_115), the use of <x>, <j>, <f>, and <ff> for /h/ is illustrated by <\textbf{x}izi> (M\_076), <\textbf{j}izi> / <\textbf{f}izi> (M\_142) ‘firewood’, “leña”, and by <mica\textbf{ff}i> ‘roasted corn’, “maíz tostado” (\mbox{M\_539}), <fintica\textbf{h}é> ‘empty corn cob’, “tusa de maíz” (M\_540), and <ca\textbf{h}ixi> a corn-based beverage, “mazamorra” (M\_541). The labial allophone or variant of Andakí /h/, represented by <f>, also occurs before /a/ in a few cases, as in <sza\textbf{f}ani> / <sza\textbf{j}ani> ‘where?’, “dónde?” (M\_551), or in <bize\textbf{f}azini?> / <bize\textbf{j}azini?> ‘what for?’, “para qué?” (M\_185).\footnote{A realization of /h/ as a voiceless bilabial fricative [ɸ] also occurs in other languages of the Isthmo-Colombian Area such as Cabécar (Chibchan) of Costa Rica \citep[xxxix]{MargeryPeña1989}.} Another symbol representing /h/ seems to be <ch> in <niyu\textbf{h}e?> / <niyu\textbf{ch}e?> ‘does she/he/it come?; has she/he/it come?', “viene?; ha venido?” (\mbox{M\_024}). In the 19\textsuperscript{th}{}-century variant of Andakí documented by \citet{Albis1860--1861}, /h/ may have been realized as zero in some cases, as suggested by the forms <\textbf{f}uchigua> (M\_580), from the 18\textsuperscript{th} century, versus <ochegua> (\mbox{A\_110}), both ‘hook’ “anzuelo” \citep[94]{CoronasUrzúa1994}. These forms may also reflect dialectal differences between the two Andakí varieties documented in the 18\textsuperscript{th} and 19\textsuperscript{th} centuries, or a diachronic development. 
