%%%%%%%%%%%%%%%%%%%%%%%%%%%%%%%%%%%%%%%%%%%%%%%%%%%%
%%%                                              %%%
%%%     Language Science Press Master File       %%%
%%%         follow the instructions below        %%%
%%%                                              %%%
%%%%%%%%%%%%%%%%%%%%%%%%%%%%%%%%%%%%%%%%%%%%%%%%%%%%
% Everything following a % is ignored% Some lines start with %. Remove the % to include them

\documentclass[output=book,
 nonflat,
 modfonts,
 nobabel
                 ]{langsci/langscibook}
  
   
%%%%%%%%%%%%%%%%%%%%%%%%%%%%%%%%%%%%%%%%%%%%%%%%%%%%
%%%                                              %%%
%%%          additional packages                 %%%
%%%                                              %%%
%%%%%%%%%%%%%%%%%%%%%%%%%%%%%%%%%%%%%%%%%%%%%%%%%%%%

% put all additional commands you need in the 
% following files. {I}f you do not know what this might 
% mean, you can safely ignore this section

\usepackage[ngerman]{babel}
\usepackage{qtree}
\usepackage{amsmath}
\usepackage{pst-jtree}
\usepackage{array} 
\usepackage{mdsymbol}
\usepackage{diagbox}
\usepackage{pst-3d}
\usepackage{graphicx}
\usepackage{hyperref}
\usepackage{arydshln}
\usepackage{tabularx}
\newcolumntype{x}[1]{!{\centering\arraybackslash\vrule width #1}}
\usepackage[demo]{graphicx}
\usepackage{booktabs}
\usepackage{setspace}\usepackage{threeparttable}
\usepackage{multirow}
\usepackage{makecell}


\title{Methods in prosody:\, \newlineCover A Romance language perspective}  
\author{Ingo Feldhausen\and Jan Fliessbach\lastand Maria del Mar Vanrell} 
\renewcommand{\lsSeriesNumber}{6}  
% \renewcommand{\lsCoverTitleFont}[1]{\sffamily\addfontfeatures{Scale=MatchUppercase}\fontsize{38pt}{12.75mm}\selectfont #1}

\renewcommand{\lsISBNdigital}{978-3-96110-104-7}
\renewcommand{\lsISBNhardcover}{978-3-96110-105-4}

\renewcommand{\lsSeries}{silp}          
\renewcommand{\lsSeriesNumber}{6}
\renewcommand{\lsURL}{http://langsci-press.org/catalog/book/183}
\renewcommand{\lsID}{183}
\renewcommand{\lsBookDOI}{10.5281/zenodo.1471564}

\typesetter{Jan Fliessbach\lastand Felix Kopecky}
\proofreader{Adrien Barbaresi, Amir Ghorbanpour, Aysel Saricaoglu, Brett Reynolds, Conor Pyle, Daniela Kolbe-Hanna, Jeroen van de Weijer, Sebastian Nordhoff\lastand Varun deCastro-Arrazola}

\BackBody{This book presents a collection of pioneering papers reflecting current methods in prosody research with a focus on Romance languages. The rapid expansion of the field of prosody research in the last decades has given rise to a proliferation of methods that has left little room for the critical assessment of these methods. The aim of this volume is to bridge this gap by embracing original contributions, in which experts in the field assess, reflect, and discuss different methods of data gathering and analysis. The book might thus be of interest to scholars and established researchers as well as to students and young academics who wish to explore the topic of prosody, an expanding and promising area of study.}

\usepackage{langsci-optional}
\usepackage{langsci-gb4e}
\usepackage{langsci-lgr}

\usepackage{listings}
\lstset{basicstyle=\ttfamily,tabsize=2,breaklines=true}

%added by author
% \usepackage{tipa}
\usepackage{multirow}
\graphicspath{{figures/}}
\usepackage{langsci-branding}

%% hyphenation points for line breaks
%% Normally, automatic hyphenation in LaTeX is very good
%% If a word is mis-hyphenated, add it to this file
%%
%% add information to TeX file before \begin{document} with:
%% %% hyphenation points for line breaks
%% Normally, automatic hyphenation in LaTeX is very good
%% If a word is mis-hyphenated, add it to this file
%%
%% add information to TeX file before \begin{document} with:
%% %% hyphenation points for line breaks
%% Normally, automatic hyphenation in LaTeX is very good
%% If a word is mis-hyphenated, add it to this file
%%
%% add information to TeX file before \begin{document} with:
%% \include{localhyphenation}
\hyphenation{
affri-ca-te
affri-ca-tes
an-no-tated
com-ple-ments
com-po-si-tio-na-li-ty
non-com-po-si-tio-na-li-ty
Gon-zá-lez
out-side
Ri-chárd
se-man-tics
STREU-SLE
Tie-de-mann
}
\hyphenation{
affri-ca-te
affri-ca-tes
an-no-tated
com-ple-ments
com-po-si-tio-na-li-ty
non-com-po-si-tio-na-li-ty
Gon-zá-lez
out-side
Ri-chárd
se-man-tics
STREU-SLE
Tie-de-mann
}
\hyphenation{
affri-ca-te
affri-ca-tes
an-no-tated
com-ple-ments
com-po-si-tio-na-li-ty
non-com-po-si-tio-na-li-ty
Gon-zá-lez
out-side
Ri-chárd
se-man-tics
STREU-SLE
Tie-de-mann
}
\bibliography{localbibliography} 

%%%%%%%%%%%%%%%%%%%%%%%%%%%%%%%%%%%%%%%%%%%%%%%%%%%%
%%%                                              %%%
%%%             Frontmatter                      %%%
%%%                                              %%%
%%%%%%%%%%%%%%%%%%%%%%%%%%%%%%%%%%%%%%%%%%%%%%%%%%%% 
\begin{document}     

\newcommand{\sent}{\enumsentence}
\newcommand{\sents}{\eenumsentence}
\let\citeasnoun\citet

\renewcommand{\lsCoverTitleFont}[1]{\sffamily\addfontfeatures{Scale=MatchUppercase}\fontsize{44pt}{16mm}\selectfont #1}
   


\maketitle                
\frontmatter
% %% uncomment if you have preface and/or acknowledgements

\currentpdfbookmark{Contents}{name} % adds a PDF bookmark
\tableofcontents
\addchap{Preface}
\begin{refsection}

%content goes here
 
% \printbibliography[heading=subbibliography]
\end{refsection}


% \addchap{Acknowledgments} 
%content goes here
The help and support of Martin Haspelmath and Sebastian Nordhoff in the preparation of this volume is gratefully acknowledged. 

We would also like to thank the authors of the chapters in this volume for their cooperation during the editing process and especially for their input to the reviewing of chapters by their peers. 

We especially thank the following additional external reviewers, %individuals, 
who contributed their time and expertise to provide independent peer review for the papers in this collection: Lisa Bonnici, Jason Brown, Elisabet Engdahl, Marieke Hoetjes, Beth Hume, Anne O'Keefe, Adam Schembri, Thomas Stolz, Andy Wedel and Shuly Wintner.
 

% \addchap{Abbreviations}
\begin{tabular}{ll}
CR & Common Room     \\
NCR & non-Common Room        \\
SGH & Selwyn Girls' High     \\
The & BBs: The Blazer Brigade \\
\isi{The PCs} & The Palms Crew  \\
\end{tabular} 
\mainmatter         
  

%%%%%%%%%%%%%%%%%%%%%%%%%%%%%%%%%%%%%%%%%%%%%%%%%%%%
%%%                                              %%%
%%%             Chapters                         %%%
%%%                                              %%%
%%%%%%%%%%%%%%%%%%%%%%%%%%%%%%%%%%%%%%%%%%%%%%%%%%%%
 
\chapter{Introduction} \label{ch:1}
\section{Aims and scope} \label{sec:1aims}
The core problem to be dealt with in this book is the syntax of functional left peripheries in West Germanic. In particular, I will concentrate on how sentence types are marked at the leftmost edge of the clause and how the presence of multiple visible markers can be accounted for. Regarding syntactic structure, I adopt a minimalist framework (as proposed by \citealt{chomsky2001, chomsky2004, chomsky2008}, among others), according to which syntactic structures are derived by merge (external or internal). Further, in line with the principles of mainstream generative grammar, I assume that the derivation of structures is constrained by economy, and hence the number of projections, as well as of syntactic processes, is as minimal as possible.

The study of various issues associated with the left periphery of the clause has always been central in generative grammar and it continues to be one of the most well-researched areas of syntax. Among other functions, left peripheries are associated with defining the type of the clause, and they are also responsible for establishing connections between clauses that make them into complex sentences. Apart from purely syntactic concerns, left peripheries raise a number of questions that make this domain extremely relevant for the interfaces of syntax, referred to as PF (Perceptible Form or, more traditionally, Phonological Form\footnote{Since generative theory was initially limited to the study of oral languages, the term ``Phonological Form'' was established, and many properties of this interface reflect the properties of oral languages, even though sign languages also evidently have an interface connected to their perceptible form. In this sense, as proposed by \citet{sigurdsson2004}, the term ``Perceptible Form'' is more appropriate as it does not treat sign languages as secondary. See also \citet{vanderhulst2015} for the distinction of the two. In this book, I will restrict myself to examining selected oral languages, mostly from Germanic.}) and LF (Logical Form, indicating the semantic component) in standard generative grammar. The interaction with the interfaces becomes evident when considering issues related to the left periphery beyond clause typing proper: certain phrases appear to be located in the left periphery due to their specific information structural status. Apart from that, clausal ellipsis is also related to various functional heads (see \citealt{merchant2001}).

It is most probably this diversity of problems that led to a significant interest in the left periphery of the clause in generative grammar already in the 1970s, most notably in \citet{chomskylasnik1977}, followed by the well-known cartographic enterprise from the 1990s onwards, especially by \citet{rizzi1997, rizzi2004} and various analyses with more or less shared concerns: for example, \citet{sobin2002}, \citet{poletto2006}, \citet{bayerbrandner2008}, \citet{brandnerbraeuning2013}. I will both rely on these previous findings and critically evaluate them. In addition, while many questions have indeed been answered by previous accounts, there are several others that have remained unresolved and have not received an adequate explanation which would hold both cross-linguistically and specifically for West Germanic as well. In addition, I assume that any proposal should follow from general principles of the grammar rather than by applying construction-specific mechanisms. In other words, the specific configuration of the left periphery of one construction should be comparable to the left periphery of other clause types within a single model by applying predictable properties of the grammar. The aim of this work is to provide such an analysis and to enable a better understanding of functional left peripheries.

In the following, I will briefly provide an overview of the most important issues concerning functional left peripheries and clause typing in West Germanic, and then I will provide a concise outline of the problems to be dealt with in this book.

\section{Functional left peripheries} \label{sec:1functional}
Clauses can fulfil various functions in discourse; in canonical cases, the form of the clause is indicative of its discourse function. Consider the following examples:

\ea \label{clauses}
\ea Ralph is interested in poetry. \label{declarative}
\ex Is Ralph interested in poetry? \label{interrogative}
\z
\z

In (\ref{declarative}), we have a statement and the type of the clause is declarative. By contrast, (\ref{interrogative}) is a question and the type of the clause is interrogative. In the first case, a proposition (\textit{p}) is true; in the second case, the truth of the proposition is asked (\textit{p} or $\neg$\textit{p}). The two utterances differ in their form. The declarative sentence represents the neutral, unmarked word order in English, which is SVO: crucially, the subject (\textit{Ralph}) precedes the aspectual auxiliary (\textit{is}). In the interrogative clause, these two elements have exactly the opposite order: the aspectual auxiliary has been moved to the front of the clause.

In many cases, the form of an utterance is not indicative of its discourse function in a straightforward way. Consider the example in (\ref{window}):

\ea Could you open the window? \label{window}
\z

In this case, the speaker does not ask the addressee about the truth of the proposition but expresses a request: a simple \textit{yes} answer, which is satisfactory in (\ref{interrogative}), would not be pragmatically appropriate in (\ref{window}) if it is not accompanied by the speaker also opening the window. The pragmatic function of sentences is thus not in a one-to-one correspondence with the observed grammatical form; these issues are examined extensively in speech act theory, going back to the work of \citet{austin1962}. As the present book is concerned with the formal properties, especially the syntax of functional left peripheries and clause typing, these issues will not be addressed here.  

The two clauses in (\ref{clauses}) differ not only in their word order but also regarding their intonation: declarative clauses have falling intonation, while interrogative clauses have rising intonation. However, there are discrepancies in this respect as well; consider:

\ea Ralph is interested in poetry? \label{declquest}
\z

The example in (\ref{declquest}) is a declarative question: formally the clause is declarative but it has a rising (interrogative) intonation; regarding its function, it constitutes a special type of question which does not ask about the truth of a proposition but rather asks for confirmation or expresses surprise. Again, these cases will not be discussed in the present thesis as they are not immediately relevant to the specific syntactic issues to be examined.

The clauses in (\ref{clauses}) are main clauses. Clause types are identified in slightly different ways in embedded clauses such as (\ref{embedded}):

\ea \label{embedded}
\ea I think [\textbf{that} Ralph is interested in poetry]. \label{that}
\ex I wonder [\textbf{if} Ralph is interested in poetry]. \label{if}
\ex It is important [\textbf{for} Ralph to study Byron]. \label{for}
\z
\z

The highlighted complementisers determine the type of the embedded clause: (\ref{that}) and (\ref{for}) are declarative, while (\ref{if}) is interrogative. Apart from clause type, complementisers can also determine whether the clause is finite, as in (\ref{that}) and (\ref{if}), or non-finite, as in (\ref{for}). Finiteness, as determined by the C head, has an effect on whether the clause contains a tensed element (e.g. \textit{is} in (\ref{that}) and (\ref{if}) above) above or not (in which case, as in (\ref{for}), English uses the element \textit{to} and the infinitival form of the verb). The incompatibility of finite complementisers with a non-finite clause, and vice versa, is illustrated in (\ref{finnonfin}) below:

\ea \label{finnonfin}
\ea[*]{I think [for Ralph is interested in poetry].}
\ex[*]{It is important [that Ralph to study Byron].}
\z
\z

Likewise, the type of a complement clause must also be compatible with the lexical properties of the matrix verb: verbs like \textit{think} select for declarative complements, while verbs like \textit{wonder} select for interrogative complements. If these sectional restrictions are violated, the result is ungrammatical:

\ea
\ea[*]{I think [if Ralph is interested in poetry].}
\ex[*]{I wonder [that Ralph is interested in poetry].} 
\z
\z

In other words, it is evident that the left periphery of the clause has a dual function. On the one hand, it connects the clause to the matrix clause (in the case of embedded clauses) or to the discourse (in the case of root clauses). On the other hand, it has an impact on the internal properties of the clause itself.

Besides complementisers, the CP is known to host other elements as well, such as \textit{wh}-phrases in interrogative clauses:

\ea \label{wh}
\ea I wonder [\textbf{who} Mary will invite]. \label{who}
\ex I asked Louisa [\textbf{which city} she was travelling to]. \label{whichcity}
\z
\z

In (\ref{who}), the \textit{wh}-element consists of a single operator (\textit{who}), while in (\ref{whichcity}) the \textit{wh}-phrase is visibly phrase-sized, containing not only the operator \textit{which} but also a lexical element, the NP \textit{city}. This indicates that \textit{wh}-phrases can occupy only a phrase position, namely [Spec,CP], and not C. Further, since they also fulfil a role in the TP, that is, they are arguments, it is assumed in generative grammar that they undergo movement from a clause-internal position to the CP-domain. This is illustrated in (\ref{wonderasked}) below:

\ea \label{wonderasked}
\ea I wonder [\textbf{who} Mary will invite \sout{\textbf{who}}].
\ex I asked Louisa [\textbf{which city} she was travelling to \sout{\textbf{which city}}].
\z
\z

In line with current minimalist theory, I assume that movement involves the copying of the moved constituent: by default, the higher copy is realised phonologically at the PF interface, while PF eliminates lower copies of a movement chain. In English, \textit{wh}-elements move to the left periphery in interrogatives, leaving the higher copy in the CP overt. Operators moving to the left periphery thus differ from complementisers not only with respect to their relative position in the CP but also in that they land there via movement, while complementisers are base-generated in the left periphery.

Relative clauses also contain operator movement:

\ea
\ea This is the linguist [\textbf{who} Mary will invite].
\ex The candidate [\textbf{who} we voted for] has already left the city.
\z
\z

Relative clauses differ from interrogative clauses in that they modify a nominal head, referred to as the head noun, while embedded interrogatives are complements of a matrix predicate (and interrogative clauses can also be root clauses). Again, relative operators undergo leftward movement:

\ea
\ea This is the linguist [\textbf{who} Mary will invite \sout{\textbf{who}}].
\ex The candidate [\textbf{who} we voted for \sout{\textbf{who}}] has already left the city.
\z
\z

Such operators (both in interrogative and relative clauses, and beyond) move to the left periphery because they have a function regarding clause typing: cases like (\ref{wh}) are identifiable as interrogative clauses precisely because there are overt interrogative elements in the left periphery, there being no distinctive interrogative intonation or word order changes (such as subject--auxiliary inversion) in embedded clauses.

\section{The problems to be discussed} \label{sec:1problems}
\subsection{The model} \label{sec:1model}
In current minimalist theory, the Complementiser Phrase (CP) is responsible for typing clauses and for encoding finiteness in finite clauses. Apart from complementisers, as pointed out in \sectref{sec:1functional} above, various operators can appear in this domain. Consider:

\ea
\ea I wonder \textbf{if} Ralph has arrived. \label{englishifch1}
\ex I wonder \textbf{whether} Ralph has arrived. \label{englishwhetherch1}
\z
\z

In (\ref{englishifch1}), \textit{if} is a complementiser and it types the subordinate clause as interrogative. In (\ref{englishwhetherch1}), there is no overt complementiser but the operator \textit{whether} is present. In such cases, it is assumed that a zero complementiser types the clause (since the CP can be projected only by a C head, which in this case is not visible, though; see \citealt[137--138]{bacskaiatkari2020jcgl} for discussion), yet a sound model of the CP-periphery must also clarify the role of the overt operator  in (\ref{englishwhetherch1}).

On the other hand, the CP is not restricted to hosting a single overt element: depending on the particular construction and the dialect, multiple elements may appear in the CP-domain. This is illustrated by (\ref{englishdfcch1}) for non-standard English and by (\ref{norwegiandfcch1}) for Norwegian (\citealt[175]{bacskaiatkaribaudisch2018}):

\ea \label{dfcch1}
\ea[\%]{I wonder \textbf{which book that} Ralph is reading. \label{englishdfcch1}}
\ex[]{\gll Peter spurte \textbf{hvem} \textbf{som} likte bøker. \label{norwegiandfcch1}\\
           Peter asked.\textsc{3sg} who that liked books\\
\glt `Peter asked who liked books.'}
\z
\z

A proper formal account of the CP-domain must be able to condition when multiple overt elements are allowed and when not. Further, it must be clarified whether the appearance of several overt elements requires multiple CP projections, and in cases where it does, how word order restrictions can be modelled. The generation of multiple functional layers is in principle possible, yet it should be appropriately restricted to exclude the generation of superfluous layers that are empirically not motivated. This question is likewise relevant in cases involving a single overt C-element, since then the question arises whether and to what extent covert elements and phonologically not visible projections are present. 

Apart from the exact position of various elements in the CP, their function(s) must also be addressed. For instance, interrogative complementisers regularly mark finiteness as well. Consider:

\ea \label{ifwhetherch1}
\ea[]{I don't know \textbf{if} I should call Ralph. \label{iffinitech1}}
\ex[]{I don't know \textbf{whether} I should call Ralph. \label{whetherfinitech1}}
\ex[*]{I don't know \textbf{if} to call Ralph. \label{ifnonfinitech1}}
\ex[]{I don't know \textbf{whether} to call Ralph.  \label{whethernonfinitech1}}
\z
\z

In (\ref{iffinitech1}), the complementiser \textit{if} introduces a finite embedded interrogative clause, and as the ungrammaticality of (\ref{ifnonfinitech1}) shows, it is incompatible with a non-finite clause, suggesting that it encodes finiteness apart from the interrogative property, too. By contrast, the operator \textit{whether} is compatible with both a finite clause, see (\ref{whetherfinitech1}), and with a non-finite clause, see (\ref{whethernonfinitech1}), indicating that the overt marking of the interrogative property is not incompatible with a non-finite clause in English. Since \textit{whether} is not specified for finiteness, it should be clear that finiteness is specified by some other element in (\ref{whetherfinitech1}); the question is whether there is a separate element encoding finiteness in (\ref{iffinitech1}) as well and, if so, how the restriction of \textit{if} to finite clauses can be explained.

Finally, the function(s) of various left-peripheral elements must be clarified also because there are some non-trivial combinations in which elements seem to be largely similar, as in the non-standard German example in (\ref{alswiech1}) below:

\ea
\ea[\%]{\gll Ralf ist größer \textbf{als} \textbf{wie} Maria. \label{alswiech1}\\
Ralph is taller than as Mary\\
\glt `Ralph is taller than Mary.'}
\ex[]{\gll Ralf ist größer \textbf{als} Maria. \label{alsch1}\\
Ralph is taller than Mary\\
\glt `Ralph is taller than Mary.'}
\ex[\%]{\gll Ralf ist größer \textbf{wie} Maria. \label{wiech1}\\
Ralph is taller as Mary\\
\glt `Ralph is taller than Mary.'}
\ex[]{\gll Ralf ist so groß \textbf{wie} Maria. \label{wieequatch1}\\
Ralph is so tall as Mary\\
\glt `Ralph is as tall as Mary.'}
\z
\z

In (\ref{alswiech1}), the elements \textit{als} and \textit{wie} both seem to mark the comparative nature of the clause, whereby single \textit{als} is the comparative particle in Standard German comparatives, as shown in (\ref{alsch1}), and single \textit{wie} is the comparative particle in equatives, see (\ref{wiech1}), and in certain dialects also in comparatives, see (\ref{wieequatch1}). In such cases, the question is to what extent there is genuine doubling at hand and how it can be modelled.

A central issue for the theory regarding the above-mentioned constructions is how the various properties associated with clause typing are encoded in the syntax. The occurrence of multiple overt elements in the left periphery indicates some complexity and raises the question whether a single CP projection is sufficient or whether multiple projections are necessary. In this respect, cartographic approaches (starting from \citealt{rizzi1997}) have a relatively clear answer, inasmuch as they assume a designated projection (generated in narrow syntax) for each feature, which necessarily leads to multiple projections in the above cases. In turn, this kind of approach is prone to reducing analysis to description, as the observed surface patterns are restated as syntactic projections; the question in this regard is whether such models are tenable or at least favourable to more minimalist approaches. These questions will be addressed in \chapref{ch:2}.

\subsection{Embedded interrogative clauses} \label{sec:1interrogative}
In Standard English, Standard German and Standard Dutch, there is no overt complementiser with an overt interrogative operator. This is illustrated in (\ref{whothatch1}) for English embedded interrogatives:

\ea	I don't know \textbf{who (*that)} has arrived. \label{whothatch1}
\z

As can be seen, the complementiser \textit{that} is not permitted in Standard English in embedded constituent questions. This phenomenon is traditionally termed as the ``Doubly Filled COMP Filter'' (going back to the work of \citealt{chomskylasnik1977}). By contrast, there are languages and also many West Germanic varieties that allow such patterns, as in (\ref{dfcch1}) above. Further examples are given in (\ref{dfcintch1}) below from non-standard English (\citealt[331, ex. 1]{baltin2010}) and from non-standard Dutch (\citealt[32]{bacskaiatkaribaudisch2018}):

\ea \label{dfcintch1}
\ea[\%]{They discussed a certain model, but they didn't know \textbf{which model that} they discussed.}
\ex[\%]{\gll Peter vroeg \textbf{wie} \textbf{dat} er boeken leuk vindt. \label{dutchdfcch1}\\
Peter asked.\textsc{3sg} who that of.them books likeable finds\\
\glt `Peter asked who liked books.'}
\z
\z

Such patterns are often referred to as doubling patterns, indicating that there are two overt elements in a single CP: the \textit{wh}-phrase in the specifier and the complementiser in C. Note that this is not exceptional: the specifier of the CP and the C head can be both lexicalised overtly in main clauses, as in T-to-C movement in English interrogatives, and in V2 clauses in German and Dutch main clauses. Consider the examples for main clause interrogatives in Standard English:

\ea \label{ttocch1}
\ea	\textbf{Who saw} Ralph? \label{whosawch1}
\ex	\textbf{Who did} Ralph see? \label{whodidch1}
\z
\z

In this case, doubling in the CP involves a \textit{wh}-operator in [Spec,CP] and a verb in C. T-to-C movement is visible by way of \textit{do}-insertion in (\ref{whodidch1}), though not in (\ref{whosawch1}): in principle, one might analyse (\ref{whosawch1}) as not involving the movement of the verb to C, but the CP is clearly doubly filled in (\ref{whodidch1}).

Similarly, in German (and Dutch) V2 declarative clauses a verb moves to C, while another constituent moves to [Spec,CP] due to an [edge] feature (see \citealt{thiersch1978diss}, \citealt{fanselow2002, fanselow2004isis, fanselow2004}, \citealt{frey2005}, \citealt{denbesten1989}). Consider:

\ea \label{v2ch1}
\ea \gll \textbf{Ralf} \textbf{hat} morgen Geburtstag.\\
Ralph has tomorrow birthday\\
\glt `Ralph has his birthday tomorrow.'
\ex \gll \textbf{Morgen} \textbf{hat} Ralf Geburtstag.\\
tomorrow has Ralph birthday\\
\glt `Ralph has his birthday tomorrow.'
\z
\z

As can be seen, the fronted finite verb is preceded by a single constituent in each case, and since the first constituent is not a clause-typing operator in either case, it is evident that doubling in the CP in V2 clauses is independent of the interrogative property.

It is therefore clear that the ``Doubly Filled COMP Filter'' should be more restricted in its application domain. In principle, one could say that an operator and a complementiser with largely overlapping functions are not permitted to co-occur in standard West Germanic languages, or that the Doubly Filled COMP Filter should be seen as some kind of an economy principle. Still, the problem remains that the notion of the Doubly Filled COMP Filter implies that the C head and [Spec,CP] position would be filled without the Filter, and the Filter is responsible for ``deleting'' the content of C. 

Regarding this, at least two major questions arise. First, it should be clarified what requirement is responsible for filling C even in the presence of an overt operator in [Spec,CP], as in (\ref{dfcintch1}). Second, the question is what kinds of elements may appear in C: in particular, if elements other than complementisers can satisfy the requirement of filling C, then the deletion approach is probably mistaken.

In addition, there is a theoretical problem with the notion of the Filter, which arises from a merge-based, minimalist perspective, while it is less problematic in X-bar theoretic terms. X-bar theoretic notions can at best taken to be descriptive designators that are derived from more elementary principles, in the vein of \citet{kayne1994} and \citet{chomsky1995}.\footnote{Note that I will also use X-bar structures for representational purposes in this book.} Under this view, the position of an element (specifier, head, complement) is a result of its relative position when it is merged with another element, and which element is chosen to be the label. By contrast, the notion of the Doubly Filled COMP Filter, as applied to a CP (as in \citealt{baltin2010}), implies that a phrase is generated with designated, pre-given head and specifier positions, and that there are additional rules on whether and to what extent they can be actually ``filled'' by overt elements. In a merge-based account, there are no literally empty positions, as no positions are created independent of merge: zero heads and specifiers reflect elements that are either lexically zero or have been eliminated by some deletion process (e.g. as lower copies of a movement chain or via ellipsis). In other words, Doubly Filled COMP effects should be accounted for in a way other than referring to a pre-given XP. These questions will be addressed in \chapref{ch:3}.

\subsection{Relative clauses} \label{sec:1relative}
West Germanic languages show considerable variation in terms of elements introducing relative clauses. There are two major strategies: the relative pronoun strategy and the relative complementiser strategy. In present-day Standard English, both of these strategies are attested. Relative pronouns are illustrated in (\ref{whrelativebasic}) below:

\ea \label{whrelativebasic}
\ea I saw the woman \textbf{who} lives next door in the park. \label{whosubjectch1}
\ex The woman \textbf{who/whom} I saw in the park lives next door. \label{whoobjectch1}
\ex I saw the cat \textbf{which} lives next door in the park. \label{whichsubjectch1}
\ex The cat \textbf{which} I saw in the park lives next door. \label{whichobjectch1}
\z
\z

As can be seen, relative pronouns show partial case distinction and distinction with respect to whether the referent is human or non-human. In particular, \textit{who}/\textit{whom} is used with human antecedents, as with \textit{the woman} in (\ref{whosubjectch1}) and (\ref{whoobjectch1}); the form \textit{who} can appear both as nominative and as accusative, while the form \textit{whom} used for the accusative is restricted in its actual appearance (formal/marked). With non-human antecedents, such as \textit{the cat} in (\ref{whichsubjectch1}) and (\ref{whichobjectch1}), the pronoun \textit{which} is used, which shows no case distinction. Note that apart from human referents, \textit{who(m)} is possible with certain animals: these are the ``sanctioned borderline cases'' (see \citealt[41]{herrmann2005}, quoting \citealt{quirkgreenbaumleechsvartvik1985}). On the other hand, non-standard dialects allow \textit{which} with human referents, as illustrated in (\ref{boywhichch1}) below (\citealt[42, ex. 4a]{herrmann2005}):

\ea {[}\ldots] And the boy \textbf{which} I was at school with [\ldots] \label{boywhichch1}\\
(\textit{Freiburg English Dialect Corpus} Wes\_019)
\z

At any rate, English relative pronouns are formed on the \textit{wh}-base and no longer on the demonstrative base: note that this is historically not so, and the present-day complementiser \textit{that} was reanalysed from a pronoun, while the \textit{wh}-based relative operators appeared only in Middle English (\citealt{vangelderen2009}).

Accordingly, the complementiser \textit{that} constitutes the second major strategy:

\ea
\ea I saw the woman \textbf{that} lives next door in the park.
\ex The woman \textbf{that} I saw in the park lives next door.
\ex I saw the cat \textbf{that} lives next door in the park.
\ex The cat \textbf{that} I saw in the park lives next door.
\z
\z

The complementiser \textit{that} is not sensitive to case and to the human/non-human distinction, which follows from its status as a C head. 

Given the availability of two strategies, a number of questions arise regarding their distribution. First, while it seems logical that the two strategies can be combined, doubling, as mentioned above, is less likely to appear in relative clauses than in embedded interrogatives, which raises the question what restrictions apply here. Second, as also mentioned above, there seems to be a preference for the complementiser strategy in West Germanic varieties that have a choice in the first place: it should be investigated why this should be so and why relative operators still exist even in dialects that have the complementiser strategy. Third, apart from their syntagmatic distribution (combinability), the paradigmatic distribution of the two strategies must likewise be examined, that is, whether the individual strategies can relativise all functions and how potential differences correlate with the featural properties of the respective items. These questions will be addressed in \chapref{ch:4}.

\subsection{Embedded degree clauses} \label{sec:1degree}
Embedded degree clauses fall into two major types: degree equatives, also called comparatives expressing equality, as given in (\ref{astallch1}), and comparatives expressing inequality, as given in (\ref{tallerthanch1}):

\ea \label{comparisonch1}
\ea Ralph is as tall \textbf{as} Mary is.\label{astallch1}
\ex Ralph is taller \textbf{than} Mary is.\label{tallerthanch1}
\z
\z

In (\ref{astallch1}), the subclause introduced by \textit{as} expresses that the degree to which Mary is tall is the same as to which Ralph is tall, while in (\ref{tallerthanch1}) the subclause introduced by \textit{than} expresses that the degree to which Mary is tall is lower than the degree to which Ralph is tall.

The comparison constructions presented in (\ref{comparisonch1}) above are instances of degree comparison: there is one degree expressed in the matrix clause and another one expressed in the subclause. The matrix degree morpheme is \textit{as} in degree equatives and it selects an \textit{as}-clause, while the matrix degree morpheme in degree comparatives is -\textit{er} (or \textit{more}, which is actually a composite of -\textit{er} and \textit{much}, see \citealt{bresnan1973}, \citealt{bacskaiatkari2014diss, bacskaiatkari2018langsci}). However, it is possible to have comparison without degree; consider:

\ea \label{nondegreecomparisonch1}
\ea[]{Mary is tall, \textbf{as} is her mother. \label{tallasch1}}
\ex[]{Mary is glamorous \textbf{like} a film-star. \label{glamorouslikech1}}
\ex[]{Farmers have other concerns \textbf{than} the farm bill. \label{otherthanch1}}
\ex[\%]{Life in Italy is different \textbf{than} I expected. \label{differentthanch1}}
\z
\z

In these cases, there is obviously no matrix degree element. The sentences in (\ref{tallasch1}) and (\ref{glamorouslikech1}) express merely similarity with respect to the property denoted by the adjective; in (\ref{glamorouslikech1}), the subclause is introduced by \textit{like} and not by \textit{as}, a further difference from degree equatives. Given the availability of non-degree equatives, \citet[35]{jaeger2018} suggests that comparison constructions can be grouped into three major categories: non-degree equatives, degree equatives, and comparatives; these constitute a markedness hierarchy in this order (non-degree equatives being the least marked). However, constructions like (\ref{otherthanch1}) and (\ref{differentthanch1}) indicate that there is in fact a fourth category as well: these are non-degree comparatives expressing difference. This category seems not to be productive as the availability of the \textit{than}-clause is dependent on the presence of a particular element expressing difference in the matrix clause: the word \textit{other} or, at least in American English, the adjective \textit{different} are potential candidates.

While the patterns in (\ref{comparisonch1}) suggest a relatively simple left periphery consisting of a single CP at first sight, further data indicate that comparatives regularly demonstrate doubling, similarly to the German pattern given in (\ref{alswiech1}) above, which seems to be present at least underlyingly in comparatives proper in all cases, while equatives may indeed have a single CP in the subclause. Further, the left periphery of degree clauses is also relevant in terms of polarity marking. In English, both degree equatives and comparatives are negative polarity environments, as illustrated by the following examples containing the negative polarity items \textit{any} and \textit{ever}:

\ea \label{englishch1}
\ea Sophia is as nice as \textbf{any} other teacher in the school. \label{asanych1}
\ex Sophia is nicer than \textbf{any} other teacher in the school. \label{thananych1}
\ex Museums are as popular as \textbf{ever} before. \label{aseverch1}
\ex Museums are more popular than \textbf{ever} before. \label{thaneverch1}
\z
\z

Negative polarity items are licensed in other negative polarity contexts (cf. \citealt{klima1964}) such as interrogatives, clausal negation and conditionals, but not in affirmative clauses (\citealt[531, ex. 11]{seuren1973}):

\ea
\ea[*]{\textbf{Any} of my friends could \textbf{ever} solve those problems.}
\ex[]{Could \textbf{any} of my friends \textbf{ever} solve those problems?}
\ex[]{At no time could \textbf{any} of my friends \textbf{ever} solve those problems.}
\ex[]{If \textbf{any} of my friends \textbf{ever} solve those problems, I'll buy you a drink.}
\z
\z

While the data in (\ref{englishch1}) suggest that English is symmetrical regarding negative polarity across the two major types of comparison clauses, German shows an asymmetric pattern: comparatives but not equatives have negative polarity:

\ea \label{germanch1}
\ea[*]{\gll Museen sind so beliebt wie \textbf{jemals} zuvor. \label{wiejemalsch1}\\
museums are so popular how ever before\\
\glt `Museums are as popular as ever before.'}
\ex[]{\gll Museen sind beliebter als \textbf{jemals} zuvor. \label{alsjemalsch1}\\
museums are more.popular as ever before\\
\glt `Museums are more popular than ever before.'}
\z
\z

The data point to the conclusion that the role of the left periphery in comparatives extends to marking polarity, not in terms of designated projections but as part of the featural makeup of the individual projections that are present in the derivation anyway due to independent clause-typing and semantic properties. These issues will be investigated in \chapref{ch:5}.

\subsection{Information structure and ellipsis} \label{sec:1information}
Certain constituents may undergo topicalisation or focalisation involving movement to the left periphery of the clause. Consider the following examples taken from \citet[285, ex. 1 and 2]{rizzi1997}:

\ea \label{englishrizzich1}
\ea {[}Your book]\textsubscript{i}, you should give \textit{t}\textsubscript{i} to Paul (not to Bill). \label{topiccommentch1}
\ex {[}YOUR BOOK]\textsubscript{i} you should give \textit{t}\textsubscript{i} to Paul (not mine). \label{focuspresuppch1}
\z
\z

The construction in (\ref{topiccommentch1}) illustrates topicalisation, and the one in (\ref{focuspresuppch1}) focalisation. Apart from interpretive differences, they crucially differ in their intonation patterns: a topic is separated by a so-called ``comma intonation'' from the remaining part of the clause (the comment), while a focus bears focal stress and is thus prominent with respect to presupposed information (see \citealt[258]{rizzi1997}).

Such movement operations are clearly instances of A-bar movement, and since they are apparently not driven by clause-typing features either, they raise the question what triggers movement in the first place. The cartographic model proposed by \citet{rizzi1997}, adopted by others such as \citet{poletto2006}, proposes that leftward movement in these cases targets designated left-peripheral positions: TopP and FocP. Movement is driven by specific features making reference to information-structural properties: this operator-like feature agrees with the functional head (Top or Foc). In essence, this kind of movement is supposed to be similar to ordinary operator movement involving \textit{wh}-operators or relative operators. Such an assumption is problematic, though: while [wh] and [rel] features are lexically determined, [topic] and [focus] features are obviously not. Taking the examples in (\ref{englishrizzich1}) above, in both cases the entire phrase \textit{your book} is topicalised or focussed, and the phrase as such, being compositional, is not part of the lexicon. This indicates that features like [topic] and [focus] would have to be added during the derivation. In addition, even if one were to assume that a lexical element like \textit{Mary} can be equipped with information-structural features in the lexicon (contrary to generally accepted views about the lexicon and lexical features, cf. \citealt{neelemanszendroei2004} and \citealt{dendikken2006}), this would leave us with various lexical entries for \textit{Mary}: a neutral entry (not specified for any information-structural category), a focussed one, a topicalised one, not to mention possible fine-grained categories such as contrastive topic or aboutness topic. 

Moreover, foci (and topics) can occur in non-fronted positions. This is illustrated by the following examples taken from \citet[172, ex. 6c and 6d]{fanselowlenertova2011}, both answering the question \textit{What happened?}:

\ea
\ea \gll \textbf{Eine} \textbf{LAWINE} haben wir gesehen!\\
a.\textsc{f.acc} avalanche have.\textsc{1pl} we seen\\
\glt `We saw an AVALANCHE!'
\ex \gll Wir haben \textbf{eine} \textbf{LAWINE} gesehen!\\
we have.\textsc{1pl} a.\textsc{f.acc} avalanche seen\\
\glt `We saw an AVALANCHE!'
\z
\z

This kind of optionality obviously contrasts with the behaviour of ordinary \textit{wh}-movement (and relative operator movement) in German, which always targets the CP-domain. Note also that there are certain fronted elements in the German CP (occupying the ``first position'') that clearly do not correspond to information structural categories such as topic and focus. Consider the following examples from \citet[173, ex. 7a]{fanselowlenertova2011}:

\ea \gll \textbf{Wahrscheinlich} hat ein Kind einen Hasen gefangen.\\
probably has a.\textsc{n.nom} child a.\textsc{m.acc} rabbit caught.\textsc{ptcp}\\
\glt `A child has probably caught a rabbit.'
\z

In this case, the adverb \textit{wahrscheinlich} `probably' is a sentential adverb that evidently lacks a discourse function such as topic or focus.

These considerations indicate that movement is not always driven by lexical features; if so, this has consequences regarding the way functional left peripheries are organised.

As mentioned above, clausal ellipsis is also closely connected to the issue of functional left peripheries. The prototypical case for this is sluicing, demonstrated in (\ref{sluicech1}) below:

\ea\label{sluicech1} Someone phoned grandma but I don't remember \textbf{WHO} \sout{phoned grandma}.\z

The elliptical clause is embedded in a clause conjoined with another main clause: this clause (\textit{someone phoned grandma}) contains the antecedents for the elided elements in the elliptical clause. The elliptical clause contains a single remnant, the subject \textit{who}, which bears main stress: it contains non-given information. Ellipsis is licensed as all elided information is recoverable. The assumption regarding the implementation of ellipsis in grammar (\citealt[55--61]{merchant2001} and \citealt[670--673]{merchant2004}) is that there is an ellipsis feature, [E]. This is merged with a functional head (such as C) and the complement of this head is elided. The [E] feature is specified as having either an uninterpretable [wh] or an uninterpretable [Q] feature, ensuring that it occurs only in (embedded) questions. As shown by \citet{vancraenenbroeckliptak2006} and \citet{hoytteodorescu2012}, this particular syntactic condition is highly unsatisfactory as many languages allow canonical ellipsis processes such as sluicing also from non-interrogative projections, including relative clauses and projections hosting foci. Rather, it seems that the [E] feature is not tied to a specific projection or features; indeed, \citet{merchant2004} also proposes that a functional projection, FP, can be headed by [E] in fragment answers, illustrated in (\ref{phonegrandma}) below:

\begin{exe}
\ex \label{phonegrandma}
\begin{xlist} 
\exi{A:} Who phoned grandma?
\exi{B:} \textbf{Liz} \sout{phoned grandma}.
\end{xlist}
\end{exe}

In this case, the remnant (\textit{Liz}) is the subject and the rest of the clause is elided. Since in English the subject DP in declarative clauses is located in [Spec,TP] and not in [Spec,CP], the ellipsis mechanism assumed for sluicing (the [E] feature located in C) does not automatically carry over. As \citet{merchant2004} assumes, there is an unspecified FP projection hosting the remnant in its specifier, landing there by movement. In this vein, it seems that leftward movement can target functional projections due to reasons other than clause-typing. This raises the question whether such functional projections may not ultimately have a more substantial role in the architecture of a clause than merely enabling ellipsis.

Questions related to information structure and ellipsis, particularly regarding their relevance for the proposed model, will be addressed in \chapref{ch:6}.

\section{Methodology}
This book aims at examining the syntax of functional left peripheries in West Germanic from a generative perspective, applying the minimalist framework in the analysis of syntactic structure. The main focus lies on the analysis of English and German, and to a lesser extent on Dutch. As language variation is a central issue, other Germanic languages will also occasionally be considered, as well as other European languages (mostly Romance and Slavic, and to some extent also Greek and Uralic). Language comparison can help to understand the cross-linguistic status of the West Germanic patterns: beyond that, however, the present investigation cannot carry out a more detailed analysis of these languages. 

Since the clausal left periphery is a well-studied area of linguistics (see \sectref{sec:1aims}), part of the investigation is dedicated to the analysis of already known patterns, also taken to be a basis for further inquiries. In addition, however, the book presents novel empirical data gained via corpus studies, questionnaires, and grammaticality judgement experiments. Regarding this, it must be kept in mind that the individual West Germanic languages (and their varieties) under scrutiny differ considerably in terms of how accessible the relevant data are and to what extent they have been discussed in the literature. 

As far as historical data are concerned, the present investigation relies on parsed corpora to identify which patterns were used in the given periods and what their frequency is. Regarding English, the Michigan Corpus of Middle English Prose and Verse was used; in addition, I compiled a database on relative clauses in the King James Bible and its modernised version. Regarding German, the DDD Referenzkorpus Altdeutsch was used. For present-day dialect data, the SyHD atlas on Hessian dialects and the SynAlm database on Alemannic dialects have been used.

As part of my project (BA 5201/1) on functional left peripheries, I obtained data from various Germanic languages (Dutch, Swedish, Danish, Norwegian, Icelandic) via an online questionnaire; this allows for a direct comparison of the languages involved. For each language, two informants were gathered who translated sentences from English as well as answered specific questions about the combinability of certain elements. The questionnaire contains 147 questions altogether. The results have been published in an open access database under \citet{bacskaiatkaribaudisch2018} and will be referenced throughout this work.

Finally, the book also presents the results of a grammaticality judgement experiment (see \citealt{schuetze2016} on the methodology) on elliptical comparative clauses in German. This allows for a more fine-grained analysis than the mere grammaticality judgements available thus far in the literature.

\section{Previous work}
The present book builds on results gained in my research projects and partly published in earlier papers; these works will be referenced in the relevant chapters as well. In this section, I would like to point out how the present investigation relates to and differs from these articles, to provide better orientation for the reader in this respect.

\chapref{ch:2} summarises the most important principles regarding the proposed non-cartographic model. The basic ideas were spelt out in \citet{bacskaiatkari2018sardis} regarding data from South German dialects and some major concerns regarding the cartographic model were also expressed in terms of the proposal made by \citet{baltin2010}. In the present book, the scope of the investigation is naturally larger; in addition, this chapter contains a detailed critical review of the literature, pointing out additional problems that were not discussed before, in particular regarding the original cartographic proposal by \citet{rizzi1997, rizzi2004}.

\chapref{ch:3} discusses embedded interrogative clauses. The core part of this chapter was published in \citet{bacskaiatkari2020jcgl}, with a particular emphasis on the relation between Doubly Filled COMP patterns in German and V2 syntax. The present investigation has a wider empirical and theoretical scope. In the original study, results from a corpus study on Middle English \textit{whether} were discussed: this was based on a smaller sample from the two versions of the Wycliffe Bible. The present study includes the results from the entire text (for both versions). Regarding the theoretical scope, the present study includes a detailed critical study of alternative analyses of Doubly Filled COMP effects, in particular that of the original proposal made by \citet{chomskylasnik1977}, which was not discussed before. In addition, the present book contains a section on long movement.

\chapref{ch:4} examines relative clauses. A core part of the discussion is centred on a corpus study carried out on the King James Bible. Some implications regarding the subject/object asymmetries observed in the choice of relativisation strategies were discussed in \citet{bacskaiatkari2020nordlyd}. This previous study was based on a smaller data set: for the present study, the entire King James Bible was taken into account, using the parallel loci to relative clause introduced by \textit{who(m)}, \textit{which} and \textit{that} in the modernised version. The present book also discusses some statistical findings that were not included in the previous investigations at all. In addition, the present study connects the findings to the general non-cartographic approach, as well as to the dialectal German data and it also presents a detailed account of equative relative clauses, also connecting the findings to the proposal made by \citet{brandnerbraeuning2013}.

\chapref{ch:5} is dedicated to embedded degree clauses. Some of the findings regarding German historical data and their diachronic development were discussed in \citet{bacskaiatkari2021oup}. The present study is more extensive in this respect and it also places the discussion of these data into a cross-linguistic setting, showing that the polarity differences between equative and comparative clauses hold across languages. The analysis is also connected to the model proposed in this book, showing the importance of analysing multiple left-peripheral projections in a non-cartographic model. The proposed account relies on many insights of \citet{jaeger2018}, yet there are some important differences in the syntactic structure between the two models: this issue is also discussed in detail.

\chapref{ch:6} analyses ellipsis processes in embedded clauses, concentrating on elliptical comparatives in German. The key idea behind the proposed analysis for German was expressed in \citet{bacskaiatkari2017atoh}; however, that study was entirely based on classical, introspective grammaticality judgements in very specific context, explicitly targeted at measuring ambiguity. The present study includes the results of a grammaticality judgement experiment and it also relates the findings to the general theory of ellipsis and information structure.

\section{Roadmap}
This book is structured as follows. In \chapref{ch:2}, I will introduce the basic assumptions regarding the proposed model. Following this, the book offers in-depth analyses of the three major constructions that will be examined here: \chapref{ch:3} addresses embedded interrogatives, \chapref{ch:4} addresses relative clauses, and \chapref{ch:5} addresses embedded degree clauses. In \chapref{ch:6}, I show that the analysis can be extended beyond the scope of clause typing proper, connecting it to issues related to information structure and ellipsis.
  %add a percentage sign in front of the line to exclude this chapter from book
\chapter{An evolving landscape}\label{chap2}

Around a decade ago, an important review article entitled ``Language evolution in the laboratory'' \citep{scott2010language} was published in \textit{Trends in Cognitive Sciences}. Its central message, in my opinion, was that it was becoming possible, at last, to approach at least certain aspects of language evolution in a scientific manner. This was a sharp departure from over a century of statements declaring that language evolution was a mystery.

Remnants of this old attitude still exist \citep{hauser2014mystery}; they typically invoke in a tedious fashion the 1866 ban on all discussion of the evolution of language imposed by the Linguistic Society of Paris; they also frequently cite Lewontin's pronouncement that we will never know why cognition evolved the way it did \citep{lewontin1998evolution}. But things have changed quite dramatically over the past two decades, so much so that it has become possible to contemplate ``controlled hypothesis-testing through experimentation'' \citep{motamedi2019evolving} in the domain of language evolution.

I still recall being told as a graduate student that the topic of language evolution was more a matter of science fiction than science, and that this was best left as a domain of study for after retirement. Today, some of the brightest students I know are actively engaged in this field, illustrating the massive progress made over the past 20 years, well attested in the Proceedings of the Evolang conference series, as well as in the creation of centers for the study of language evolution in Edinburgh and more recently Z\''{u}rich. The main change (still ongoing), to my mind, is the resistance to exploring hypotheses until they can be formulated in a way that can be put to the test. A change from `I think \textit{x}' to `I think \textit{x} and I can test \textit{x} doing \textit{y}.'

The efforts of members of the Centre for Language Evolution at the University of Edinburgh, led by Simon Kirby, have shown how combining the development of artificial languages (mini-grammars) in a laboratory setting \citep{kirby2008cumulative,kirby2015compression}, as well as agent-based modelling approaches controlling for biases that language users in the lab bring to the task in an unconscious manner \citep{thompson2016culture}, reveals how learnability and expressivity pressures shape grammars. Subsequent work from other centers (e.g. \cite{raviv2019larger,raviv2020language,raviv2021makes}) also experimentally demonstrates how communicative contexts impact grammar formation and the emergence of new languages. While it is often said that such work only addresses language change (`glossogeny'), and not language evolution proper (language phylogeny, the emergence of the modern language capacity),\footnote{Terminology introduced in \cite{hurford1990nativist}.} I do not find this dichotomy particularly useful, and believe that a continuum of cognitive biases that interact with changing communicative conditions from which language-readiness emerges, shaping the range of grammars acquired, is a more adequate stance (more on this in chapter \ref{chap4}).

The same year the review article by \cite{scott2010language} appeared, the first draft of the Neanderthal genome was published \citep{green2010draft}, starting a revolution that continues unabated to this day \citep{reich2018we}. As we will see later on, the successful retrieval of ancient DNA, from a few skeletal remains and now even cave sediments, and of ancient proteins, allows us to ask questions at an unprecedented level of resolution and dramatically changes what we mean by ``fossil record''. The debt we owe to Svante Pääbo and his collaborators is hard to overstate \citep{paabo2014neanderthal,meyer2012high,prufer2014complete,prufer2017high,mafessoni2020high,slon2017neandertal,vernot2021unearthing,zavala2021pleistocene,welker2016palaeoproteomic,welker2020dental}.

Yet this massive amount of data that is now accessible would be ``empty'' if it were not for the progress made in linking the genotype and the phenotype. In the domain of language, the work pioneered by Simon Fisher on \textit{FOXP2} is the gold standard \citep{lai2001forkhead}, and arguably one of the most significant achievements in the language sciences in the past twenty five years \citep{fisher2009foxp2,fisher2015genetics,fisher2019human,den2021molecular}. It has taught us that for all the intricacies and levels of analyses separating genes and behavior, careful work can illuminate central issues that Lenneberg could only dream of when he wrote his classic book, \textit{Biological Foundations of Language}, over fifty years ago \citep{Lenneberg1967biological}.

Equally important for the success of what is sometimes called ``evolinguistics'' is the dramatic shift of perspective that took place in the domain of comparative psychology. This is well-captured in the following passage from \cite{de2010towards}:

\largerpage
\begin{quote}	
Over the last few decades, comparative cognitive research has focused on the pinnacles of mental evolution, asking all-or-nothing questions such as which animals (if any) possess a theory of mind, culture, linguistic abilities, future planning, and so on. Research programs adopting this top-down perspective have often pitted one taxon against another, resulting in sharp dividing lines. Insight into the underlying mechanisms has lagged behind \ldots \clearpage


A dramatic change in focus now seems to be under way, however, with increased appreciation that the basic building blocks of cognition might be shared across a wide range of species. We argue that this bottom-up perspective, which focuses on the constituent capacities underlying larger cognitive phenomena, is more in line with both neuroscience and evolutionary biology.
\end{quote}

In the domain of language, calls for recognizing an ever broader ``community of descent'', to borrow a phrase from \cite{darwindescent}, are more and more frequent \citep{lattenkamp2018vocal}. Far from being rhetorical, these calls demonstrate how much one can learn about our kind by studying behavior in numerous species in accordance with Tinbergen's multi-level approach.

As Ernst Mayr was fond of saying, ``evolutionary biology [unlike physics] is a historical science, [where] one constructs a historical narrative, consisting of a tentative reconstruction of the particular scenario that led to the events one is trying to explain'' \citep{mayr2000darwin}. Narratives will continue to dominate evolutionary investigations into language, but crucially, thanks to the progress made in key areas that I singled out above, these narratives are enriched with, and constrained by, ``numbers''. Hypotheses can now be put to the test.

It becomes very apparent in this context that simple narratives, appealing as they may appear, are hopelessly misguided. Recalling the words of H. L. Mencken, ``For every complex problem there is an answer that is clear, simple, and wrong''. What more complex problem is there than the problem of language evolution?

Accordingly, the simple, clear, ``minimalist'', and influential evolutionary scenario advocated by Berwick and Chomsky in their book \textit{Why Only Us} \citep{berwick2016only} must be wrong.\footnote{If I am right, this has non-trivial ramifications for the minimalist program. Over the years, talk of optimization, efficiency, etc., which occupied center stage in the early days of the program, has been replaced by a focus on evolutionary considerations. If such considerations lead to an impasse, the program as a whole may indeed have been (at best) premature.} I have tried to say so on several occasions \citep{boeckx2017not,martins2019language,de2020evolutionary}. Very briefly: it is wrong because it disregards the comparative evidence (`only us'), it fails to appreciate the multi-level approach required to link genotype and phenotype (claiming that a single mutation yields the simple, atomic operation ``merge''), it keeps the discussion at the logical level, without attempting to even sketch a plausible path to testing it, and does not engage with the many lessons coming from the great discoveries in paleo-sciences over the past decade.

The reason I have spent time arguing against Berwick and Chomsky's narrative is not only because it was proposed by influential linguists, but because it is representative of a family of approaches that linguists remain attracted to: it presupposes that other animals don't have much to teach us about the core of our language faculty, because essentially they are non-linguistic creatures. The gap between them and us is a chasm. It also takes for granted that our language capacity is very recent in evolutionary terms, going back maybe 150 000 years. As such, so the claim goes, there was very little time to evolve a ``kludgy'' language organ (cf. \cite{marcus2009kluge}). Accordingly, a narrative must be developed that keeps the core language faculty essentially free of evolutionary tinkering.

Such a narrative (in many ways, the culmination of the minimalist program envisaged by Chomsky) clashes with recent attempts to attribute a significant portion of our ``modern'' language faculty to the last common ancestor shared with our closest extinct relatives \citep{dediu2013antiquity,dediu2018neanderthal}. It also clashes with mounting evidence for a complex, temporally very extended, mosaic-like evolution of our lineage \citep{scerri2018did,Bergstrom}. Also, it makes certain assumptions about how many changes can be favored by natural selection within a relatively short window of time which are not obviously true---indeed, very implausible \citep{de2020evolutionary}. Last, but not least, it grants too much power to linguistic theorizing. As argued in \cite{martins2019language}, it is fallacious to draw a direct correspondence between the formal structure of a computational operation and the biological changes that would lead to it.\footnote{In their reply to \cite{martins2019language}, \cite{berwick2019all} completely---and surprisingly---miss this point; see \cite{martinsboeckx_clar} for illustration.} It is what theoretical linguists would love to be able to do: it would make their theoretical work immediately relevant for evolutionary claims. But it is logically incorrect. This is precisely why, in my opinion, evolutionary considerations impact how we do theoretical linguistics, or how we see the import of that work. If there is no such direct correspondence, if the link between genotype and phenotype is very complex indeed, I do not see any alternative to painstakingly developing linking hypotheses that, we hope, progressively spell out what it means to say that our linguistic condition is part of our human (biological) condition.

I want to insist once more on the importance of debunking simple accounts like Berwick and Chomsky's. It may well be that there will be certain behaviors or artifacts or anatomical traits that we can confidently ascribe exclusively to members of our species that ``emerged'' recently. Right now this is being questioned, but I would not be surprised if we are left with a small set of recent ``\textit{sapiens}-exclusive'' properties (brain changes giving rise to our globular skull, use of complex symbiotic tools like the bow and arrow, and some aspects of figurative art are fairly good bets in my current opinion), but crucially, even if the evidence settles along these lines, it should not be used to argue for a recent cognitive revolution that matches a minimalist vision of the language faculty. Rather, such evidence will have to be integrated into the complex mosaic of language that evolution has constructed over an extended period of time.

This is certainly a major lesson I learned from thinking about Darwin's problem: Evolutionary considerations invalidate certain theoretical frameworks that fail to come to grips with the ``complex dynamical system'' nature of language. The next two chapters deal with other lessons that pertain to a broader range of approaches, and implicate a larger number of researchers: even those linguists that readily accept that the evolutionary trajectory of our language capacity was long and complex still subscribe to certain views that I think we would do well to abandon. I'll focus on three such views here. One is that somehow, there is at least one aspect of language (typically, some aspect of syntax) that makes our language capacity special, and that as a result forms some sort of barrier in a comparative setting. Another is the belief that linguistic theory matters and that one's theory of language evolution depends on one's theory of language. And third, the claim that because languages don’t leave fossils, the evidence for studying the evolution of language is too sparse. These three claims are incorrect.

\documentclass[output=paper,colorlinks,citecolor=brown]{langscibook}
\ChapterDOI{10.5281/zenodo.14282810}
\author{Carolin Ulmer\orcid{}\affiliation{Freie Universität Berlin}}
%\ORCIDs{}

\title{The expression of motion events in Haitian Creole}

\abstract{This paper investigates the expression of motion events in Haitian Creole. A bipartite typology has been proposed by \citet{Talmy_1991}, sorting languages into verb-framed and satellite-framed languages, depending on where they express the Path component of motion events. \citet{Slobin_2004} expanded the typology by a third type, equipollently-framed languages, to include verb-serializing languages which can express the Path as well as the Manner component in a serial verb construction. Creole languages have so far received little to no attention in regard to this typology. Creole languages are especially interesting because they were formed in a situation of language contact. The investigation of their morphosyntactic features can shed light on the question of which features of the languages involved are passed on and which are not. This can in turn offer clues for the study of the markedness of these features. The languages which were relevant to the formation of Haitian Creole, French and Kwa languages, present different patterns here. In French, verb-framed patterns are predominantly used, but in some cases Manner verbs constitute the main verb of the sentence \citep{Pourcel_Kopecka_2005}. In contrast, Kwa languages can use verb serializations to encode motion events (\citealt{Ameka_Essegbey_2013}; \citealt{LambertBrtire_2009}), a pattern not found in French. In this paper, I describe a small study conducted in Berlin, Germany, in 2017, investigating the expression of Motion events by four native speakers of Haitian Creole. They narrated a picture story and described drawings depicting different combinations of Manner and Path components. A wide range of different morphosyntactic structures encoding motion events was elicited. Verb-framed patterns were frequently used, as well as different Manner-Path verb serializations. Only a few satellite-framed constructions were elicited, but using different Manner verbs and Path-PPs. Further research will need to test the acceptability of different Manner and Path elements in the particular structures.}


\IfFileExists{../localcommands.tex}{
   \addbibresource{../localbibliography.bib}
   \usepackage{langsci-optional}
\usepackage{langsci-gb4e}
\usepackage{langsci-lgr}

\usepackage{listings}
\lstset{basicstyle=\ttfamily,tabsize=2,breaklines=true}

%added by author
% \usepackage{tipa}
\usepackage{multirow}
\graphicspath{{figures/}}
\usepackage{langsci-branding}

   
\newcommand{\sent}{\enumsentence}
\newcommand{\sents}{\eenumsentence}
\let\citeasnoun\citet

\renewcommand{\lsCoverTitleFont}[1]{\sffamily\addfontfeatures{Scale=MatchUppercase}\fontsize{44pt}{16mm}\selectfont #1}
  
   %% hyphenation points for line breaks
%% Normally, automatic hyphenation in LaTeX is very good
%% If a word is mis-hyphenated, add it to this file
%%
%% add information to TeX file before \begin{document} with:
%% %% hyphenation points for line breaks
%% Normally, automatic hyphenation in LaTeX is very good
%% If a word is mis-hyphenated, add it to this file
%%
%% add information to TeX file before \begin{document} with:
%% \include{localhyphenation}
\hyphenation{
affri-ca-te
affri-ca-tes
an-no-tated
com-ple-ments
com-po-si-tio-na-li-ty
non-com-po-si-tio-na-li-ty
Gon-zá-lez
out-side
Ri-chárd
se-man-tics
STREU-SLE
Tie-de-mann
}
\hyphenation{
affri-ca-te
affri-ca-tes
an-no-tated
com-ple-ments
com-po-si-tio-na-li-ty
non-com-po-si-tio-na-li-ty
Gon-zá-lez
out-side
Ri-chárd
se-man-tics
STREU-SLE
Tie-de-mann
}
   \boolfalse{bookcompile}
   \togglepaper[3]%%chapternumber
}{}

\begin{document}
\maketitle

\section{Introduction}

The expression of motion events in different languages has been of great interest to many linguists since \citet{Talmy_1991} proposed his typology of them, sorting languages into two types depending on whether they typically express the Path of motion in the main verb or a so-called satellite, some other element that is closely associated with the main verb. Romance languages are often cited as typical members of the first group, called verb-framed languages, whereas Germanic languages represent the second group, named satellite-framed languages. Later, \citet{Slobin_2004} proposed a third type that he calls equipollently-framed languages to describe languages expressing both the Path and the Manner component in a verb serialization, a structure that is found e.g. in Mandarin Chinese. Even though many languages have been investigated with regard to the morphosyntactic structures used to encode the different components of motion events, there is still no research on this question for Romance-based Creole languages. As Creole languages were formed in a situation of language contact, their investigation can show which features of the languages involved were passed on. The present paper looks at the morphosyntactic expression of Motion events in Haitian Creole. A small pilot study was conducted with four native speakers of Haitian Creole in Berlin, Germany. After a short sociolinguistic interview to determine their language ideologies and habits of language use (which were deemed necessary as the speakers all lived away from their home country and in a multilingual environment), the speakers completed two different tasks. First, they narrated a picture story about a little bird flying out of its cage and house to explore the outside world. After that, they provided descriptions for single pictures which were assembled in order to control for several combinations of Manner and Path components of motion events. The results show that Haitian Creole possesses a rich inventory of morphosyntactic structures to express motion events. A preference exists for the use of verb-framed constructions, but Manner-Path verb serializations were also used frequently. Satellite-framed constructions were rare, but do not seem to be totally ungrammatical.

The structure of the paper is as follows. \sectref{sec:3:2} first gives a definition of motion events and then describes the three types of motion event encoding mentioned above. Then, the concept of Manner salience, which describes the frequency with which Manner elements are used in different languages, is introduced. Sections 2.3 and 2.4 provide an insight into the expression of motion events in French and some African languages which were relevant to the formation of Haitian Creole. After that, a first light is shed on motion event encoding in Haitian Creole from previous works on the Haitian Creole language. \sectref{sec:3:3} then outlines the study conducted by the author and explains how the data was classified. \sectref{sec:3:4} gives a broad overview of the different structures that were elicited, which are then discussed in \sectref{sec:3:5}.


\section{Motion events}\label{sec:3:2}

In this study, a motion event is understood as defined by \citet[60--61]{Talmy_1991}:

\begin{quote}
    The basic motion event consists of one object (the “Figure”) moving or located with respect to another object (the reference object or “Ground”). It is analyzed as having four components: besides “Figure” and “Ground”, there are “Path” and “Motion”. The “Path” [...] is the course followed or site occupied by the Figure object with respect to the Ground object. […] In addition to these internal components a Motion event can have a “Manner” or a “Cause”, which we analyze as constituting a distinct external event.
\end{quote}


In sentence \REF{ex:3:1}, the components of the motion event are distributed as follows:

\ea\label{ex:3:1}
\gll Tom is running {down the stairs}.\\
     {}  {} Manner  Path\\
\z
		
\textit{Tom} represents the Figure, \emph{running} the Manner, \emph{down} the Path, and \emph{the stairs} the Ground of the motion event described.

Different lexicalization patterns are found in the languages of the world concerning the motion event component expressed in the verb. A first type encodes Motion and Manner in the verb, which is typical e.g. in English, as seen in sentence \REF{ex:3:1}. A second type, which is typically found in Romance languages like French or Spanish, encodes Motion and Path in verbs, such as  \emph{descendre/bajar} (‘go.down’) \citep[62--68]{Talmy_1985}. These differences lead to different patterns of encoding motion events, described in the following.

\subsection{Motion event typology}

In his 1991 paper, Talmy develops a typology of motion event encodings, sorting the languages of the world into two types depending on the morphosyntactic element in which they express the Path component: verb-framed and satellite-framed languages \citep[486--487]{Talmy_1991}. Verb-framed languages, such as Romance languages, express the Path component in the main verb of the sentence.

\ea Spanish \citep[69]{Talmy_1985} \\\label{ex:3:2}
\glll La botella salió de la cueva flotando.\\
      The bottle went.out of the cave floating \\
      {}  {}     Path     {} {} {} Manner\\
\z
	                     	
Other languages, such as English or German, express the Path component in a so-called satellite, a term defined by \citet[102]{Talmy_1985} as “immediate constituents of a verb root other than inflections, auxiliaries, or nominal arguments, [related] to the verb root as periphery (or modifiers) to a head”. These languages are therefore named satellite-framed languages. The English counterpart to \REF{ex:3:2} can be seen in \REF{ex:3:3}.

\ea\label{ex:3:3}
\gll  The bottle floated {out of the cave}.\\
      {}  {}      Manner Path\\
\z
						
\citet{Slobin_2004} revises this binary typology by adding a third type, equipollently-framed languages, suited to describe languages with serial verb constructions where both Path and Manner can be expressed in a verb, as illustrated by the Mandarin Chinese example in \REF{ex:3:4}.

\ea\label{ex:3:4}Mandarin Chinese\\
\glll  Hǎiōu cóng dòng l\u{i} fēi chū.\\
       seagull from hole in fly exit\\
       {}      {}   {}   {}  Manner Path\\
\glt `The seagull flew out of the hole.'
\z

Much work has followed the papers of \citet{Talmy_1991} and \citet{Slobin_2004}, classifying many different languages into the different patterns. Many of these works have used the so-called “Frog Stories” also used in \citet{Slobin_2004}. In sections 2.3 and 2.4, a short overview will be given of the work on motion event encoding in Kwa languages as well as French, which have been relevant to the formation of Haitian Creole.

\subsection{Manner salience}

\citet{Slobin_2004} takes a more detailed look at the expression of Manner, that is to say the frequency with which it is expressed in different languages. To this end, he compares the encoding of one certain event in the Frog Stories, namely an owl flying out of a knothole \citep[224--225]{Slobin_2004}. In the verb-framed languages Spanish, French, Italian, Turkish and Hebrew, virtually no Manner verbs are used, see \REF{ex:3:5} for French.

\ea\label{ex:3:5}French \citep[224]{Slobin_2004}\\
\glll D'un trou de l'arbre sort un hibou.\\
      from.a hole in the.tree  exits  an owl\\
      {}   {}  {} {}     Path {} {} \\ 
\z

Between different satellite-framed languages, more variation can be found regarding the expression of Manner. In German and English, Manner verbs are not used very frequently (only by about 17--32\% of the speakers) to express the motion event in question. This is due to the fact that often deictic verbs are used with Path satellites, as in the German example in \REF{ex:3:6}.

\ea\label{ex:3:6}German\\
\glll  Aus dem Astloch kommt eine Eule raus.\\
      from \textsc{art.def.dat} knothole comes  \textsc{art.indef} owl out \\
      {}   {}  {}     {}    {}   {}  Path\\
\z

In the equipollently-framed languages Mandarin Chinese and Thai, Manner is expressed more frequently (by 40\% of the Mandarin and 59\% of the Thai speakers). In the SF language Russian, the Manner component of this event is expressed by 100\% of the speakers. In all these cases, either a deictic (\emph{pri-letet} `fly here') or a Path-prefix (\emph{vy-letet} `fly out') is added to the Manner verb \emph{letet} `to fly'.

Slobin comes to the conclusion that the frequency with which the Manner component is expressed depends on the language type as well as the morphosyntactic possibilities to encode Manner. He proposes to align languages along a scale of Manner salience, where languages expressing Manner in the main verb typically have a high Manner salience, whereas languages where Manner is subordinate to Path typically have low Manner salience \citep[250]{Slobin_2004}.


\subsection{Motion events in French}

As already mentioned above, French is classified as a verb-framed language, encoding Path in the main verb and Manner in a gerund.

\ea\label{ex:3:7}French\\
\glll  Elle entrait à la  maison en.courant.\\
       \textsc{3sg.f} entered \textsc{prep} \textsc{art.def} house run.\textsc{ger} \\
       {} Path {} {} {} Manner\\
\z


An extensive study of motion event encoding in French can be found in \citet{Pourcel_Kopecka_2005}. They analyze a total of 1800 written and 594 oral descriptions of motion events and, on this basis, describe five different patterns frequently found in French \citep[145--149]{Pourcel_Kopecka_2005}. The most frequent is the verb-framed type, as already shown in \REF{ex:3:7}. Another frequent pattern is the coordination of two verb phrases, one containing a Manner verb and the other containing a Path verb:\footnote{Motion events expressed in a single phrase are the main interest of the study, but because the coordinated pattern is so frequent in the data, it is nevertheless listed here.}

\ea\label{ex:3:8}French \citep[145]{Pourcel_Kopecka_2005}\\
\glll  Il court dans une rue puis rentre dans une maison.\\
       He runs on a street then enters into a house \\
       {} Manner {} {}  Path  \\
\z

The authors find a third pattern which they call “reverse verb-framed pattern” because it is structurally identical with a verb-framed pattern but Manner and Path “switch places”, so that the Manner component is expressed in the main verb and the Path component in a gerund, see \REF{ex:3:9}.

\ea\label{ex:3:9}French \citep[145]{Pourcel_Kopecka_2005}\\
\glll Il court {en traversant} la rue.\\
      He runs crossing the street\\
      {} Manner {}  Path\\  
\z

The fourth type, in which Manner is expressed in the verb and Path in a PP, is also called reverse verb-framed pattern by the authors. This fourth type can also be described as a satellite-framed construction, see \REF{ex:3:10}.

\ea\label{ex:3:10}French \citep[145]{Pourcel_Kopecka_2005}\\
\glll    Il court dans le jardin.\\
         He runs into  the garden\\
         {} Manner {}  Path \\
\z

The fifth pattern is a hybrid type because the verbs here encode Path as well as Manner. There are two subtypes to this pattern: In the first, both elements are expressed in the verb, as in \REF{ex:3:11}; in the second, Path is expressed in an incorporated prefix of the verb, as in \REF{ex:3:12}. 

\ea\label{ex:3:11}French \citep[146]{Pourcel_Kopecka_2005}\\
\glll    Marc {a plongé dans} le lac.\\
         Marc {dived into} the lake\\
         {}    Manner.Path\\
\ex\label{ex:3:12}French \citep[149]{Pourcel_Kopecka_2005}\\
\glll   L'oiseau   s'est en-volé du nid.  \\
        {The bird} has away-flown from the nest  \\
        {}         {}  Path-Manner\\
\z

Besides the description of these five patterns, the authors show by using acceptability judgments that in French it is dispreferred to express the Manner component of a motion event as long as it is inferable from the context. Only when the Manner of motion is not typical for the Figure or Ground of the event, it is acceptable to express it \citep[148]{Pourcel_Kopecka_2005}. This finding is in line with the observation of \citet{Berthele_2013} that Manner is seldomly expressed in French motion event encodings.

\subsection{Motion events in Kwa languages}

Kwa languages form part of the Niger-Congo language family, members of which show a general tendency to lexicalize Path in verbs, such as \emph{enter}, \emph{pass}, or \emph{ascend} \citep[200--202]{Schaefer_Gaines_1997}. As for the expression of Manner, much variation is found between the members of this language family \citep[209]{Schaefer_Gaines_1997}.

A more detailed study on the expression of motion events has been carried out for two different Kwa languages, viz. Ewe \citep{Ameka_Essegbey_2013} and Fon \citep{LambertBrtire_2009}.

In Ewe, serial verb constructions combining a Path verb and a Manner verb can be used to express motion events, see \REF{ex:3:13}.

\ea\label{ex:3:13}
Ewe \citep[24]{Ameka_Essegbey_2013} \\
\gll    Devi-a tá yi xɔ-a me.\\
        child-\textsc{def} crawl go room-\textsc{def} in \\
\glt ‘The child crawled into the room.’
\z

It is possible to combine a Manner verb with more than one Path verb, each indicating movement in respect to a different ground object, see \REF{ex:3:14}.

\ea\label{ex:3:14}Ewe \citep[30--31]{Ameka_Essegbey_2013}\\
\gll  Kofi tá tó ve-a me do yi kpó-á dzí.\\
      Kofi crawl pass ditch-\textsc{def} in exit go hill-\textsc{def} top\\
\glt ‘Kofi crawled through the ditch and emerged  at the top of the hill.’
\z

In Fon, motion events can also be expressed using verb serialization, as in \REF{ex:3:15}.

\ea\label{ex:3:15}Fon \citep[14]{LambertBrtire_2009}\\
\gll xɛ̀ví ɔ̀ zɔ̀n gbɔ̀ tá nǔ é\\
     bird \textsc{def} fly pass head for \textsc{3SG} \\
\glt ‘The bird flew over his head.’
\z

As in Ewe, a Manner verb can be combined with more than one Path verb, as in \REF{ex:3:16}.

\ea Fon \citep[22]{LambertBrtire_2009}\label{ex:3:16}\\
\gll Cùkú ɔ́ lɔ̌n tɔ́n sín xɔ̀ mɛ̀ gbɔ̀n flɛ́tɛ́ ɔ́ nù.\\
     dog \textsc{def} jump exit from room in pass window \textsc{def} edge \\
\glt ‘The dog jumped out of the room through the   window.’
\z


Available for motion event verb serialization is a closed class of ten Path verbs \citep[9]{LambertBrtire_2009}. All of these can also be used outside of verb serializations, but not all Path verbs are available for serialization, like e.g. \emph{xá} ‘go.up’ \citep[16]{LambertBrtire_2009}. Similarly, not all Manner verbs are available for serialization, see \REF{ex:3:17}.

\ea\label{ex:3:17}Fon \citep[15]{LambertBrtire_2009}\\
\gll * yě dǔ-wè tɔ́n sìn xwé ɔ́ mɛ̀  \\
     {} \textsc{3pl} move-dance exit from house \textsc{def} in \\
\glt \phantom{*}‘They danced out of the house.’
\z

Whereas \citet[36]{Ameka_Essegbey_2013} classify Manner-Path verb serializations in Ewe as equipollently-framed constructions, \citet[19]{LambertBrtire_2009} argues that the Fon Manner-Path verb serializations are satellite-framed constructions with the Path verbs acting as satellites. She reaches that conclusion because certain inflectional elements can only appear in front of the Manner verb, which marks them, in her point of view, as the main verb of the sentence.

In fact, the question how Manner-Path verb serializations should be classified in the typology described above is controversial. It depends mainly on the question whether the verbs are co- or subordinated. The discussion of this problem is outside of the scope of the present study. More details on the topic can be found in \citet{Talmy_2009}.

\subsection{Motion events in Haitian Creole}
 
To my knowledge, no study has aimed at investigating the expression of motion events in Haitian Creole\footnote{In the following, all examples are from Haitian Creole, so this will not be indicated in the rest of the paper.} until now. Nonetheless, some insights can be obtained from the literature on Haitian Creole. The language possesses an inventory of Path verbs, many of French origin, like in the example in \REF{ex:3:18}.

\ea\label{ex:3:18} \citep[203]{Fattier_2013}\\
\gll Dlo antre anndan kay. \\
     water enter \textsc{loc} house \\
\glt ‘Water came into the house.’
\z

Besides that, of all French-based Creole languages, Haitian Creole is the one that exhibits the most serial verb constructions \citep[44]{Mutz_2017}. Many of those constructions found in the literature do not express motion events, but a few examples of Manner-Path verb serializations can be found, such as the one in \REF{ex:3:19}.

\ea\label{ex:3:19} \citep[244]{Valdman_2015}\\
\gll  Tidjo kouri ale lakay li. \\
      Tidjo run go home \textsc{poss.pron} \\
\glt ‘Tidjo ran over to his house.’
\z

The dissertation on Haitian Creole verb serialization by \citet{BucheliBerger_2009} does not offer examples of Manner-Path verb serialization, but lists the possible combinations of Manner and Path verbs, a shortened version of which is reproduced here in \tabref{tab:tab1_03} (on the following page).\footnote{Her results are derived from online research. Marked as possible are those combinations for which she could find examples online. If a combination is not marked as possible, this does not necessarily mean that it is impossible but simply that the author could not find an example for it during her research. No acceptability study was carried out. An anonymous reviewer of this paper notes that some combinations, especially the ones with \emph{tonbe}, sound strange to them.}

% Table 1:
\begin{table}
\resizebox{\linewidth}{!}{%
\begin{tabular}{lllllllll}
    \lsptoprule
    & \textit{al(e)} & \textit{vin(i)} & \textit{sòt(i)/} & \textit{antre} & \textit{rive} & \textit{monte/} & \textit{desann} & \textit{(re-) tounen} \\
    & ‘go’ & ‘come’ & \textit{sot(i)} & ‘go in’ & ‘arrive’ & \textit{moute} & ‘go down’ & ‘come back’ \\
     &  &  & ‘go out’ &  &  & ‘go up’ &  &  \\ \midrule
    \textit{kouri ‘run’} & + & + & + & + & + & + & + & + \\
    \textit{mache}  ‘march’ & {+} & {+} & {+} & {+} & {+} & {} & {} & {} \\
    \textit{naje}  ‘swim’& {+} & {} & {+} & {} & {} & {} & {} & {} \\
    \textit{woule} ‘roll’ & + & + & + &  &  &  & + &  \\
    \textit{koule} ‘flow’ &  &  & + &  &  &  & + &  \\
    \textit{vole} ‘fly’ & + & + & + &  &  &  & + &  \\
    \textit{glise} ‘glide’ &  &  & + &  &  &  & + &  \\
    \textit{tonbè} ‘fall’ &  &  & + &  & + &  &  &  \\ 
    \lspbottomrule
\end{tabular}}
\caption{Manner of Motion V1 + Path of Motion V2 in Haitian Creole after \citet[202]{BucheliBerger_2009}}
\label{tab:tab1_03}
\end{table}


\section{Study design}\label{sec:3:3}

The present study investigates the expression of motion events as presented above in Haitian Creole. The main purpose is to describe the morphosyntactic elements used to express the components Manner and Path and the preferences of their use. For this purpose, four Haitian Creole speakers living in Berlin, Germany, took part in interviews that consisted of three parts: an interview on their habits of language use and attitudes towards all their languages, a narration of a picture story, and descriptions of single pictures representing different motion events that were drawn by the author of this study. The entire interviews were held in Creole. One of the participants, P1, helped realize the other three interviews as well as transcribe and translate the language data recorded. She will henceforth be referred to as the main participant. More information on participants and tasks is given in the following sections.


\subsection{Participants}

The four participants were aged between 34 and 56. P1 is female; P2, P3 and P4 are male. P1 is a B.A. student, P2 is a mechanical engineer, P3 is a salesperson and photographer and P4 is a political scientist and educator in development cooperation. They were all born in Haiti and completed most of their education there. P1, P2 and P3 come from the area of Port-au-Prince, P4 moved there from the North of the country when he was ten years old. All four emigrated between 20 and 30 years of age. P2 and P4 regularly work in Haiti. The four participants all speak Haitian Creole, French, German, Spanish, and English. They learned Haitian Creole as their first language from their parents and later learned French in school. They received education almost entirely in French; only P1 had Creole language classes for one year. All four report they are able to converse fluently in Creole but have problems with writing, as they have never learned a norm. As the four participants all live in Germany, they speak German on a daily basis.\footnote{Of course, the fact that the four participants all live in a non-Creole-speaking country and use other languages on a daily basis could influence the Creole they speak causing it to be different from the Creole spoken in Haiti. Because the present study was carried out as an MA thesis, getting fieldwork data was not possible. Possible contact phenomena will not be investigated in the present study, but this has to be kept in mind when interpreting the results.} P1, P3 and P4 report that they speak French often, mostly with their family, especially with their children. P4 also speaks French (as well as Creole) at work. Creole is spoken with friends and family in Haiti and abroad, e.g. with their parents and siblings. P2 is the only participant that reports that he speaks Creole often, mainly with his children but also the rest of the family, as well as when working in Haiti. He is also the only one to name Creole as the language he finds most elegant; for the three others that language is French. When asked what the Creole language means to them, all four replied that it is an important part of their identity and their origin. P3 and P4 also say that they feel that Creole is the most important one of their languages.

\subsection{Tasks}

There were two tasks aiming at eliciting motion events, the narration of a picture story and the description of single pictures drawn for the purpose of this study. The picture story selected was \emph{Die Geschichte vom Vogel} (‘The bird story’) \citep[from][]{RettichRettich1972}.\footnote{Unfortunately, the image of the picture story cannot be reproduced here due to copyright reasons.} Even though the Frog Stories have been used to elicit motion events in many previous studies, they were not used here, first because they were considered difficult to narrate by the author of this study and her supervisor, and second because many of the Frog Story pictures do not contain motion events. The bird story is about a bird that flies out of its cage and then out of its house. Outside, he meets different animals that all chase him away. Finally, he flies back to his house and into his cage. The story was chosen because it contains many different motion events which could help determine how frequent the Manner component would be expressed in order to investigate the Manner salience of Haitian Creole.

The description of single pictures aimed at exploring the morphosyntactic elements that could be used to express different Manner-Path combinations. Therefore, seven Manner elements (\emph{run, swim, fly, jump, crawl, dance, roll}) and ten Path elements (\emph{out, away, to, into, up, down, along, past, after somebody, through}) were used to create a total of 48 motion events, as in \REF{ex:3:20}, which were then portrayed in simple pictures by the author of this study.

\ea\label{ex:3:20}
\gll    He runs out {of the burning house.}\\
        {} Manner Path\\
\z

% Figure 1:
\begin{figure}
    \includegraphics[width=0.6\linewidth]{figures/fig1_03.png}
    \caption{Depiction of the motion event ‘to run out of’}
    \label{fig:fig1_03}
\end{figure}

The combination of the 17 motion event components would have yielded more than 48 events, but it was decided not to overwhelm the participants with too many pictures. The 48 drawings were divided into two groups of 24, which were presented to two participants each. When dividing the pictures, the different Manner and Path components were divided as equally as possible between the two groups. Within the two groups, the pictures were arranged in such a way that two following pictures never contained a component already depicted in the previous picture. During the interview, the participants were told to describe what the person in the picture was doing.

\subsection{Data analysis}

The interviews were transcribed and translated into German by the main participant. Transcriptions and translations were later checked by the author of this paper and revised together.

At the beginning of the data analysis, the number of sentences was counted for the picture story narrations. Every unit containing a subject and (at least) one verb was counted as one sentence. Coordinated clauses were counted as two sentences, but subordinated clauses like complement clauses, relative clauses, causal clauses, temporal clauses and the like were counted as part of the matrix clause.

P1’s narration contains 26 sentences, P2’s contains 46 sentences, P3’s contains 55, and P4’s narration contains 42 sentences. In total, 169 sentences were elicited.

After counting the number of sentences, the number of motion events encoded was determined. Every sentence expressing directional motion was analyzed as a motion event encoding.

P1’s narration contains 18, P2’s 20, P3’s 29, and P4’s 27 motion event encodings. Altogether, the four participants encoded 94 motion events in their picture story narrations. With a total of 169 sentences, more than 50\% of the sentences contained a motion event encoding.

The motion events were then sorted by means of the morphosyntactic structure they used to encode different motion event components. They were sorted into six different categories: Path verbs only, see \REF{ex:3:21}, Path verbs with Ground-PP/NP, see \REF{ex:3:22}, Manner verbs only, see \REF{ex:3:23}, Manner verbs with path elements, see \REF{ex:3:24}, serial verb constructions, see \REF{ex:3:25}, and motion events without a motion verb, see \REF{ex:3:26}. The remaining cases were classified as “Other”.

\ea\label{ex:3:21}
\gll Li rantre.\\
     \textsc{3sg} enter.again \\
\glt ‘He goes back in.’
\ex \label{ex:3:22}
\gll Epi l antre nan kay la. \\
and \textsc{3sg} enter \textsc{loc} house \textsc{def} \\
\glt ‘He enters the house.’
\ex E zwazo a kouri. \\\label{ex:3:23}
    and bird \textsc{def} \textsc{run} \\
\glt ‘And the bird runs/flies fast.’
\ex \label{ex:3:24}
\gll {{\ob}…{\cb}} zwazo a vole sou do yon erison.\\
     {} bird \textsc{def} fly \textsc{loc} back \textsc{indef} hedgehog \\
\glt ‘The bird flies onto the back of a hedgehog.’
\ex \label{ex:3:25}
   \gll Li kouri retounen nan kay {[}kote li te ye a.{]}\\
    \textsc{3sg} run return \textsc{loc} house {[}\textsc{rel.pron} \textsc{3sg} \textsc{pst} \textsc{cop} \textsc{def}{]} \\
\glt `He goes back into the house where he was before.'
\ex \label{ex:3:26}
\gll Epi li kraze rak.\\
     then \textsc{3sg} destroy forest \\
\glt ‘Then he beats loose.’
\z

The results of the analysis will be given in the following section.

The first step of the picture description analysis was to determine the number of descriptions. As 48 pictures were described by two participants each, 96 descriptions should have been elicited, but as one of the participants failed to describe two of the pictures, only 94 descriptions were elicited. Some of the descriptions consist of a simple sentence, whereas others consist of a complex sentence or even more than one sentence. A total of 119 sentences were elicited in both tasks.

If more than one sentence was used for the description of a picture, they were counted and analyzed separately. The same holds for complex sentences if they contained more than one motion event, e.g. a sequence of two relative clauses, as in \REF{ex:3:27}, or sentences with \emph{pou} `in order to', as in  \REF{ex:3:28}.

\ea\label{ex:3:27}
\gll Yon zwazo k ap vole k ap pase bò kot yon pyebwa.\\
\textsc{indef} bird \textsc{rel.pron} \textsc{prog} fly \textsc{rel.pron} \textsc{prog} \textsc{pass} beside side \textsc{indef} tree\\
\glt ‘A bird which is   flying, who is passing next to a tree.’\footnote{Mostly P2, but also P4, described several of the pictures with utterances of the form NP + relative clause. Even though these do not constitute regular sentences, it is possible to analyze them as elliptic versions of sentences like \emph{This is [NP] who is moving} which are also found in some descriptions. They were therefore included in the analysis.}
\ex\label{ex:3:28}
\gll Yon mesye k ap naje sòti nan plaj pou l ale bò rivaj. \\
\textsc{indef} man \textsc{rel.pron} \textsc{prog} swim exit \textsc{loc} beach for \textsc{3sg} go beside coast\\
\glt ‘A man who is swimming away from the beach in   order to swim to the coast.’
\z

Sentences which did not express motion (13 of 119) were not analyzed.

In a few cases, modal verbs were used, see \REF{ex:3:29} and \REF{ex:3:30}. These were ignored for the analysis and the event encodings of the motions were treated as if they did not contain a modal verb.

\ea\label{ex:3:29}
\gll Yon gason ki vle monte sou yon tab.\\
      \textsc{indef} boy \textsc{rel.pron} want ascend \textsc{loc} \textsc{indef} table \\
\glt ‘A boy who wants to go up onto a table.’
\ex\label{ex:3:30} 
\gll Yon mesye ki dwe travèse dyagonal yon chanm. \\
     \textsc{indef} man \textsc{rel.pron} must cross diagonal \textsc{indef} room \\
\glt `A man who has to cross a room diagonally.'
\z

The motion events expressed in the picture descriptions were then sorted into eight categories, seven of which are equivalent to those for the picture story. A new category was established for this part of the data: Manner verbs with Ground elements, as exemplified in \REF{ex:3:31}.

\ea \label{ex:3:31}
\gll  Yon moun k ap rale kote yon mi.\\
     \textsc{indef} person \textsc{rel.pron} \textsc{prog} crawl beside \textsc{indef} wall \\
\glt `A person who is crawling next to a wall.'
\z

The results of the analysis are given in the following section.


\section{Results}\label{sec:3:4}

An overview of the results is given in \tabref{tab:tab2_03}. In the following subsections, the results are discussed in detail.

% Table 2:
\begin{table}
\begin{tabular}{l rr rr rr}
\lsptoprule
                 & \multicolumn{2}{c}{Picture} & \multicolumn{2}{c}{Single} & \\
                 & \multicolumn{2}{c}{story}   & \multicolumn{2}{c}{pictures} &\multicolumn{2}{c}{Total} \\\midrule
{Path verb only} &  10     & 10.6\% & 4      & 3.4\%  & 14     & 6.6\% \\
{Path verb + ground PP/NP} &  25       & 26.6\%   & 29       & 24.4\%   & 54       & 25.4\%  \\
{Manner verb only} & 18      & 19.1\%  & 19      & 16.0\%    & 37      & 17.4\% \\
{Manner verb + ground} &  0      & 0.0\%      & 13     & 10.9\% & 13     & 6.1\% \\
{manner verb + path element} &    2     &  2.1\% &  5     &  4.2\% &  7     &  3.2\%\\
{SVC} &  16     & 17.0\%   & 36     & 30.3\% & 52     & 24.4\%\\
{Motion event without motion verb} &   7      &  7.4\%  &  1      &  0.8\%  &  8      &  3.8\% \\
{Other} &   16      & 17.0\%    & 12      & 10.1\%  & 28      & 13.1\% \\
\midrule
{Total} &  94 & & 119 & & 213 & \\
\lspbottomrule
\end{tabular}
\caption{Motion event expressions in picture story narrations and single picture descriptions}
\label{tab:tab2_03}
\end{table}

Path verbs only were used ten times in the picture story narrations (10.6\% of all occurrences). In most cases, the Ground was mentioned in the preceding or following context but not in the same clause, see \REF{ex:3:32} and \REF{ex:3:33}.

\ea \label{ex:3:32}
\gll E li ouvri pòt kalòj la pou li kapab sòti.\\
     and \textsc{3sg} open door cage \textsc{def} for \textsc{3sg} able.to exit\\
\glt ‘And she opens the door of the cage so that he can go out.’
\ex \label{ex:3:33}
\gll Epi zwazo a tounen. L al nan menm kay la {[}...{]}\\
     and bird \textsc{def} return \textsc{3sg} go \textsc{loc} same house \textsc{def}\\
\glt ‘And he returns. He goes into the same house.’
\z

In one case, no ground is mentioned at all, see \REF{ex:3:34}.

\ea\label{ex:3:34}
\gll Papiyon an ale.\\
     butterfly \textsc{def} go\\
\glt ‘The butterfly goes/flies away.’
\z

At this point, the picture story shows a butterfly flying away from the bird. Therefore, \textit{ale} seems to express not simply `go' but `go away' here.

In the single picture descriptions, Path verbs only occurred in four motion events (3.4\%). As in the narrations, the Ground was usually mentioned in the context, see \REF{ex:3:35}.

\ea\label{ex:3:35}
\gll    Yon moun k ap rale kote yon mi.  L ap pase {[}…{]} \\
        \textsc{indef} person beside \textsc{rel.pron} \textsc{prog} crawl beside \textsc{indef} wall  \textsc{3sg} \textsc{prog} pass \\
\glt ‘A person is crawling next to a wall. He is passing {[}it{]}…’
\z

Again, there was one case where no Ground was mentioned at all, again with the verb \textit{ale}, which seems to mean `go away' \REF{ex:3:36}.

\ea\label{ex:3:36}
\gll Yon zwazo ki sòti nan kalòj pou ale.\\
     \textsc{indef} bird \textsc{rel.pron} exit \textsc{loc} cage for go \\
\glt ‘A bird who leaves the cage in order to go/fly away.’
\z

\subsection{Path verb + Ground NP/PP}

The most frequent pattern used to express motion events in the picture story is a Path verb with a Ground NP or PP, which was used in 25 cases (26.6\%). The verbs \emph{antre} ‘enter’, \emph{pase} ‘pass’, \emph{atèri} ‘land’, \emph{tonbe} ‘fall’, \emph{ale} ‘go’, \emph{poze} ‘sit down’, and \emph{sòti} ‘go out’ were used with PPs \REF{ex:3:37}.

\ea
\label{ex:3:37}
\gll Li atèri sou flè a.\\
     \textsc{3sg} land \textsc{loc} flower \textsc{def}\\
\glt ‘He lands on the flower.’
\z

The verbs \emph{jwenn} ‘reach’, \emph{suiv} ‘follow’, \emph{kite} ‘leave’, \emph{tounen} ‘come back’ were used with object NPs \REF{ex:3:38}.

\ea\label{ex:3:38}
\gll  Li kite do erison an.\\
      \textsc{3sg} leave back hedgehog \textsc{def}\\
\glt ‘He leaves the back of the hedgehog.’
\z

The verb \emph{rive} ‘arrive’ was used with a PP three times (by P2 and P3) and with an object NP once (by P1), see \REF{ex:3:39} and \REF{ex:3:40}. Because of the small number of occurrences, nothing can be said about whether this is simply due to individual preferences.

\ea\label{ex:3:39}
\gll  Lè l rive sou pyebwa {[}…{]} \\
      when \textsc{3sg} arrive on tree  \\
\glt ‘When he arrives on the tree…’
\ex\label{ex:3:40}
\gll    Zwazo a rive lakay li.\\
        bird \textsc{def} arrive home \textsc{poss.pron}\\
\glt ‘The bird arrives at his house.’
\z

In the single picture descriptions, Path verbs with Ground NPs or PPs present the second most frequent pattern with 29 occurrences (24.4\%).

Used with an object NP were the verbs \emph{depase, desann}, and \emph{kite} \REF{ex:3:41}.

\ea\label{ex:3:41}
\gll  Tidjo kite lekòl la. \\
      Tidjo leave school \textsc{def}\\
\glt ‘Tidjo leaves the school.’
\z

The verbs \emph{antre, rantre, pase} and \emph{al(e)} were used with PPs. \emph{Antre} and \emph{rantre} were used with  \emph{nan} ‘into’, \emph{al(e)} with \emph{bò} ‘next to’ and \emph{nan direksyon} ‘in the direction of’, and \emph{pase} also with \emph{bò}, see \REF{ex:3:42}.

\ea\label{ex:3:42}
\gll  L ap pase bò yon kay. \\
      \textsc{3sg} \textsc{prog} pass beside \textsc{indef} house \\
\glt ‘He is passing a house.’
\z

The verbs \emph{monte} and \emph{sòti} were used with both NPs and PPs. \emph{Sòti} was used with three different prepositions, \emph{nan, sou} and \emph{a travè}. See \REF{ex:3:43} for an example with \emph{a travè}, and \REF{ex:3:44} for the use with an NP.

\ea\label{ex:3:43}
\gll Sa se   yon   moun   ki       sòti {a travè} yon fenèt {[}…{]}\\
     \textsc{dem} \textsc{cop} \textsc{indef} person \textsc{rel.pron} exit through \textsc{indef} window\\
\glt ‘That is a person who leaves through a window.’ 
\ex\label{ex:3:44}
\gll    Yon timoun ki sòti lekòl. \\
        \textsc{indef} child \textsc{rel.pron} exit school \\
\glt ‘A child that leaves school.’
\z

As there are only a few occurrences for each verb, often just one but six at the most, it remains unclear whether the use with NP or PP attested here is a general preference of the verb or whether all verbs can appear with both.

\subsection{Manner verb only}

A Manner verb alone cannot, strictly speaking, express a motion event as it is defined above, but because they occur so frequently in both picture story narrations and single picture descriptions, they are taken into account here.

In the picture story task, in the 18 cases counted for this category (19.1\%), only three different Manner verbs were used, \emph{vole} ‘fly’, \emph{kouri} ‘run‘, and \emph{mache} ‘walk’. The last of the three is used only once where a hedgehog continues walking after the bird has landed on his back. The most frequently used of these verbs is \emph{vole}. This is not surprising when taking into account that the story is about a bird and also features other flying animals like owls or butterflies. \emph{Kouri} was used six times to describe the motion of the bird, see \REF{ex:3:45}.

\ea\label{ex:3:45}
\gll Lè chwèt la kouri dèyè zwazo a, sa k pase, zwazo a kouri.\\
     when owl \textsc{def} run behind bird \textsc{def} \textsc{def} \textsc{rel.pron} happen bird \textsc{def} run \\        
\glt ‘When the owl flies behind/after the bird, the bird runs/flies away fast.’
\z

As the story shows several instances of an animal chasing another animal (mostly the little bird) away, the cases where \emph{vole} and \emph{kouri} are used alone always describe a situation where the animal flees. Apparently, in these cases, a directed motion away from the place of action seems to be described. The Path ‘away’ seems to be inferable from the context and is therefore left out. \emph{Kouri} obviously does not express the Manner ‘run’ in these cases, but rather an accelerated manner of movement.

In the single picture descriptions, Manner verbs were used alone in 19 cases (16\% of all occurrences). In seven of these, no further information on the motion event was given, see \REF{ex:3:46}.

% (46)	Yon 	mesye  ki 	     ap 	     naje.
% 		INDEF	man	REL.PRON PROG  swim
% 		‘A man that is swimming.’
\ea\label{ex:3:46}
\gll  Yon mesye ki ap naje. \\
      \textsc{indef} man \textsc{rel.pron} \textsc{prog} swim \\
\glt ‘A man that is swimming.’
\z

These cases do not express a motion event as it is understood here, but an activity.

In the remaining twelve cases, more information on the motion event is given in the preceding or the following context. In five cases, the manner verb is followed by a construction with \emph{pou} ‘to’ in which Path is expressed, see \REF{ex:3:47}.

\ea\label{ex:3:47}
\gll  Tidjo ap naje pou l depase lòt la.\\
      Tidjo \textsc{prog} swim for \textsc{3sg} pass other \textsc{def} \\
\glt ‘Tidjo is swimming in order   to pass the other.'
\z

In three cases, information on the Path is given in the preceding or following sentence, see \REF{ex:3:48} and \REF{ex:3:49}. As the translations show, it is possible to interpret these cases as single complex motion events.

\ea\label{ex:3:48}
\gll Tidjo ap naje. Li kite il la.  \\
     Tidjo \textsc{prog} swim \textsc{3sg} leave island \textsc{def}  \\
\glt ‘Tidjo is swimming. He leaves the island./Tidjo is swimming away from the island.’

\ex\label{ex:3:49}
\gll Tidjo antre nan kay. {[}…{]} L ap danse.\\
     Tidjo enter \textsc{loc} house {} \textsc{3sg} \textsc{prog} dance  \\
\glt ‘Tidjo enters the house. He is dancing./Tidjo is dancing into the house.’
\z

In two cases, Path and Manner are expressed in two precedent relative clauses, see \REF{ex:3:50}. Again, it is possible to interpret this as a single complex motion event.

\ea\label{ex:3:50} 
\gll Yon zwazo k ap vole k ap pase bò yon mi. \\
     \textsc{indef} bird \textsc{rel.pron} \textsc{prog} fly \textsc{rel.pron} \textsc{prog} pass beside \textsc{indef} wall  \\
\glt ‘A bird who is flying, who is passing beside a wall.\slash A bird who is flying past a wall.'
\z

In one case, two main clauses, one containing a Manner and the other a Path verb, are combined with the conjunction \emph{pandan} ‘while’, see \REF{ex:3:51}.

\ea\label{ex:3:51}
\gll L ap danse pandan l ap monte mach eskalye a.\\
     \textsc{3sg} \textsc{prog} dance while \textsc{3sg} \textsc{prog} ascend  step stairs \textsc{def}\\
\glt ‘He dances while he goes up the stairs./He dances up the stairs.’
\z

All these examples seem to represent complex motion events whose components were not expressed in a single clause, meaning that they were not conflated into one event.

\subsection{Manner verb + Ground element}

Similar to the previous category, Manner verbs combined with a Ground element in the same clause do not constitute motion events as defined by \citet{Talmy_1985}, because the Path element obligatory for motion events is not encoded. Such a pattern does not occur in the picture story narrations, but there are 13 such occurrences in the single picture descriptions (10.9\%). Six of the seven manner verbs tested in this task were used in this pattern, see \REF{ex:3:52} for an example with \emph{naje} ‘swim’.

\ea\label{ex:3:52}
\gll    Yon moun k ap naje nan lanmè.\\
        \textsc{indef} person \textsc{rel.pron} \textsc{prog} swim \textsc{loc} sea\\
\glt ‘A person who is swimming in the sea.’
\z

\subsection{Manner verb + Path element}

The least frequent of the patterns is the use of a Manner verb in combination with a Path element. Such cases constitute instances of the satellite pattern described above.

In the picture story descriptions, two such cases (2.1\%) occurred with the verb \emph{vole}, once with the preposition \emph{sou} ‘onto’, see \REF{ex:3:53}, and once with \emph{deyò} ‘out of’, see \REF{ex:3:54}.

\ea\label{ex:3:53}
\gll {[}…{]} zwazo a vole sou do yon erison. \\
      {}  bird \textsc{def} fly \textsc{loc} back \textsc{indef} hedgehog \\
\glt ‘A bird flies onto the back of a hedgehog.’
\ex\label{ex:3:54}
\gll    Li vole deyò. \\
        \textsc{3sg} fly outside  \\
\glt ‘He flies outside.’
\z

In the single picture descriptions, this satellite-framed pattern is used in five cases (4.2\%), three of them with the verb \emph{vole}, see \REF{ex:3:55}, one with \emph{naje}, see \REF{ex:3:56}, and one with \emph{woule} ‘roll’.

\ea\label{ex:3:55}
\gll Yon zwazo k ap vole a travè nyaj yo. \\
     \textsc{indef} bird \textsc{rel.pron} \textsc{prog} fly through cloud \textsc{pl} \\
\glt ‘A bird that is flying through the clouds.’
\ex\label{ex:3:56}
\gll    Tidjo ap naje {[}…{]} sou lòt bò lak la. \\
        Tidjo \textsc{prog} swim  {} \textsc{loc} other side lake  \textsc{def}\\
\glt ‘Tidjo is swimming to the other side of the lake.’
\z

\subsection{Serial verb constructions}

Serial verb constructions are used frequently in both the picture story narrations and the single picture descriptions. As for the story narrations, they present the second most frequent pattern with 16 occurrences (17\%). Most of these consist of a serialization of two Path verbs: \emph{sòti kite} ‘leave go away’, \emph{al(e) poze} `go sit down', \emph{vin poze} `come sit down', \emph{al tonbe} ‘go fall’ and \emph{al antre} ‘go enter’. In two cases, a Manner verb is followed by a Path verb: \emph{vole poze} ‘fly sit down’ and \emph{kouri retounen} ‘run return’. One single verb serialization consists of three verbs: \emph{leve pran kouri} ‘get up take run’. \emph{Pran} most probably acts as an aspectual marker here, encoding inchoativity (see also \cite{Valdman_2015}: 231). Besides the fact that this serial verb construction consists of three verbs, it is also different from the others because the order of the Path and Manner verb is inverted here in regard to the other cases: the Path verb is the first, the Manner the last verb of the serialization.

\hspace*{-2.8pt}The different serial verb constructions occur either with a Ground NP, a Ground PP, or with no Ground element in the same clause. In the last case, Ground is mentioned in the surrounding context.

In four serial verb constructions, \emph{vole poze, vin poze, al(e) poze} and \emph{al tonbe}, the same kind of action is expressed: the bird flying to a certain place and coming to rest. It is not clear whether these are sequential or simultaneous verb serializations, the first meaning ‘he flies and then sits down’ and the latter meaning ‘he flies onto [the tree]’. The same problem exists with \emph{al antre} 'go enter', where it is unclear whether it is said that the bird first goes and then enters or whether he ‘goes into’. 

The case of \emph{sòti kite} is described by \citet[79--81]{BucheliBerger_2009} as a serialization of two verbs that are close to synonyms and which she interprets as a simultaneous serial verb construction expressing one simple motion. It is also possible that the verbs have different meanings here, expressing the Paths ‘out of’ and ‘away’, which would make this a sequential serial verb construction.

In \emph{kouri retounen}, which once again describes the motion of the verb, \emph{kouri} cannot be interpreted as expressing the Manner ‘to run’, but rather a fast way of moving.

\emph{Leve pran kouri} is probably a sequential serial verb construction, expressing that the bird first gets up and then flies away fast, where the Path `away' is left unexpressed and to be inferred from the context.

In the single picture descriptions, serial verb constructions were the pattern most frequently used by the participants to encode a motion event. With 36 occurrences in total (30.3\% of all occurrences), 31 different verb combinations can be described. The different internal structures of these serial verb constructions are summarized in \tabref{tab:tab3_03}.

% Table 3:
\begin{table}[!ht]
\centering
\resizebox{\linewidth}{!}{%
\begin{tabular}{lccccccc}
    \lsptoprule
    \multirow{2}{*}{Type of SVC} & \multirow{2}{*}{manner-path} & \multirow{2}{*}{manner-manner} & \multirow{2}{*}{path-path} & \multirow{2}{*}{path-Other} & \multicolumn{3}{c}{3 verbs} \\ \cmidrule{6-8}
    &  &  &  &  & MMP & PPP & PMP \\ \midrule
    Frequency & 19 & 2 & 1 & 6 & 1 & 1 & 1 \\ \lspbottomrule
\end{tabular}%
}
\caption{Frequency of different serial verb constructions in the single picture descriptions}
\label{tab:tab3_03}
\end{table}

By far the most common verb serializations are those consisting of a Manner and a Path verb. Six different Manner verbs were used in such constructions: \emph{kouri} ‘run’, \emph{naje} ‘swim’, \emph{vole} ‘fly’, \emph{rale} ‘crawl’, \emph{woule} ‘roll’, and \emph{glise} ‘glide/slide’ (used once instead of `roll'). Two of the Manner verbs investigated were not used in Manner-Path serializations, \emph{sote} ‘jump’ and \emph{danse} ‘dance’. Table \ref{tab:tab4_03} shows the different Path verbs that these Manner verbs were used with. Nothing can be said about the possibility to form serial verb constructions other than the ones attested in the data of this study.\footnote{An anonymous reviewer of this paper notes that more combinations than the ones attested in this study should be possible, especially with \emph{rale}. Also, all manner verbs should be able to combine with \emph{ale}.}

% Table 4:
\begin{table}[!ht]
\centering
\resizebox{\linewidth}{!}{%
\begin{tabular}{@{}lllllllllll}
    \lsptoprule
    & \textit{sòti} & \textit{kite} & \textit{antre} & \textit{monte} & \textit{desann} & \textit{pase} & \textit{travèse} & \textit{ale} & \textit{poze} & \textit{suiv} \\ \cmidrule{2-11}
    \textit{kouri} & + & + & + & + & + & + & + &  &  &  \\
    \textit{naje} & + &  &  &  &  &  &  &  &  &  \\
    \textit{vole} & + &  & + &  &  &  &  & + & + & + \\
    \textit{rale} & + &  &  & + &  &  &  &  &  &  \\
    \textit{woule} &  &  &  &  & + & + &  &  &  &  \\
    \textit{glise} & + &  &  &  & + &  &  &  &  & \\ \lspbottomrule
\end{tabular}%
}
\caption{MANNER-PATH-SVC in the single picture descriptions}
\label{tab:tab4_03}
\end{table}

These Manner-Path serial verbs also occur both with NPs and PPs, see \REF{ex:3:57} and \REF{ex:3:58}.

\ea\label{ex:3:57}
\gll Tipyè rale monte mòn nan. \\
     Tipyè crawl ascend mountain \textsc{def} \\
\glt ‘Tipyè crawls up the mountain.’
\ex\label{ex:3:58}
\gll Zwazo  a vole antre nan kizin nan. \\
     bird \textsc{def} fly enter \textsc{loc} kitchen \textsc{def} \\
\glt ‘The bird flies into the kitchen.’
\z

In one case, no ground element is given in the same clause, see \REF{ex:3:59}. As the Path verb is \emph{ale} in this case, the Path can probably again be interpreted as ‘away’ in this case.

\ea\label{ex:3:59}
\gll  Yon zwazo {[}…{]} k ap vole ale. \\
      \textsc{indef} bird   {} \textsc{rel.pron} \textsc{prog} fly go \\
\glt ‘A bird that is flying away.’
\z

In two cases, two Manner verbs were serialized, see \REF{ex:3:60} and \REF{ex:3:61}. In \REF{ex:3:60}, it is obvious that again \emph{kouri} cannot be interpreted as 'run', but most probably means that the motion takes place fast. In \REF{ex:3:61}, it is unclear what the exact meaning of \emph{vole} is. It could possibly be interpreted as having a figurative meaning expressing something like jumping into the water in a high arc. This hypothesis cannot be tested here.\footnote{An anonymous reviewer suggests that the two verbs act like synonyms here.}

\ea\label{ex:3:60}
\gll Tipyè kouri naje dèyè yon lòt.\\
     Tipyè run swim behind \textsc{indef} other \\
\glt ‘Tipye swims fast behind/after another.’
\ex\label{ex:3:61}
\gll Li vole sote nan dlo a.\\
     \textsc{3sg} fly jump \textsc{loc} water \textsc{def} \\
\glt ‘He jumps into the water (in a high arc).’
\z

The only Path-Path serialization in the single picture descriptions is \emph{sòti kite}, which also occurred in the picture story narrations.

In six cases, the participants used serial verb constructions consisting of a Path and a non-motion verb, like in \REF{ex:3:62}.

\ea\label{ex:3:62}
\gll   Yon moun ki ap antre kache nan yon gwòt. \\
      \textsc{indef} person \textsc{rel.pron} \textsc{prog} enter hide \textsc{indef} \textsc{loc} cave \\
\glt ‘A person who is going into a cave in order to hide.’
\z

In three of these six cases, the verb \emph{ale} was used, but cannot be interpreted as a Path verb, see \REF{ex:3:63}. It could be interpreted as an analytical future, but such a structure with this function has not been described for Haitian Creole (see for example, \cite{Valdman_2015,DeGraff_2007}). This structure cannot be further analyzed at this point.

\ea\label{ex:3:63}
    \gll {[Sa se Johana k ap kouri]} pou l al pran bis la  nan stasyon an.\\
          {}                        for \textsc{3sg} go take bus \textsc{def} \textsc{loc} station \textsc{def}\\
\glt ‘{[}That’s Johanna who is running{]} in order to take a bus at the station.’
\z

In three cases, three motion verbs are serialized. The first of them is a Manner-Manner-Path serialization, see \REF{ex:3:64}. As this sentence is about a swimming person, \emph{kouri} probably once again express a fast motion.

\ea\label{ex:3:64}
\gll  Li kouri naje kite il la. \\
      \textsc{3sg} run swim leave island \textsc{def} \\
\glt ‘He swims away from the island fast.’
\z

The second of these serializations consists of three Path verbs \REF{ex:3:65}. \emph{Rive} and \emph{jwenn} are close to synonyms and are therefore interpreted as expressing the same meaning here. Together with \emph{avanse} ‘move forward’ they probably present a sequential serial verb construction.

\ea\label{ex:3:65}
\gll  Li avanse rive jwenn demwazèl la. \\
      \textsc{3sg} advance arrive reach lady \textsc{def} \\
\glt ‘He advances towards and reaches the lady.’
\z

In the last of the three cases, we find a Path-Manner-Path verb serialization, see \REF{ex:3:66}. Interestingly, V1 and V3 are the same verb, \emph{sòti}. As this is the only case where we find such a structure in the data, nothing can be said about whether this is a common pattern of serial verb constructions in Haitian Creole.

\ea\label{ex:3:66}
\gll   Boul la sòti woule sòti nan bwàt katon.\\
       ball \textsc{def} exit roll exit \textsc{loc} box cardboard\\
\glt ‘The ball rolls out of the cardboard box.’
\z

Even though many of the serial verb constructions elicited in this study are single cases that need to be described separately, some patterns can also be found. One frequent strategy to express complex motion events in Haitian Creole is the use of a Manner-Path verb serialization, which in a few cases also occurred in serializations of three verbs. In other cases, two Path verbs or a Path and a non-motion verb were serialized to encode a motion event depicted in one of the drawings. 

\subsection{Motion events without a motion verb}

In the picture story narrations, seven cases occurred where a motion event was expressed without a motion verb (7.4\%). Five of these were uttered by P3, when he used the idiom \emph{kraze rak}, which has the meaning ‘to beat loose’. The two other cases were uttered by P2 using \emph{jwenn direksyon} and \emph{pran direksyon} to say that the bird was going into a certain direction, see \REF{ex:3:67}.

\ea\label{ex:3:67}
\gll  {{[}…{]}} kounye a la      li jwenn direksyon fenèt la \\
      {}        moment \textsc{def} \textsc{dem} \textsc{3sg} reach direction window \textsc{def}\\
\glt ‘Now he takes the direction of the window.’
\z

In the single picture descriptions, there is only one case where a motion event is expressed without a motion verb (0.8\%), also using \emph{direksyon}, see \REF{ex:3:68}.

\ea\label{ex:3:68}
\gll Li pran nan direksyon machin nan. \\
     \textsc{3sg} take \textsc{loc} direction car \textsc{def}\\
\glt ‘He takes the direction of the car.’
\z

\subsection{Other}

The remaining encodings of motion events had to be sorted into a separate category because it was not possible to analyze them as any of the other categories.

For the picture story narrations, all of the 16 cases sorted into this category (17\%) use the preposition \emph{dèyè} ‘behind’ or ‘after’, 13 with \emph{kouri}, see \REF{ex:3:69}, and 3 with \emph{pati}, see \REF{ex:3:70}. In all of these cases, one animal is chasing another.

\ea\label{ex:3:69}
\gll Chwèt la kouri dèyè zwazo  a pou l pran l.\\
     owl \textsc{def} run behind bird \textsc{def} \textsc{3sg} take \textsc{3sg}\\
\glt ‘The owl flies behind/after the bird in order to catch him.’
\ex\label{ex:3:70}
\gll Koukou a pati dèyè zwazo a. \\
     cuckoo \textsc{def} leave behind bird \textsc{def}\\
\glt ‘The cuckoo flies behind/after the bird.’
\z

The problem here is the preposition \emph{dèyè}: If it expresses ‘behind’, the PP can be analyzed to encode the Ground and locate the place where the motion is taking place. If it expresses ‘after’, it can be analyzed as expressing the Path of motion.

The main participant translated all of these cases with \emph{chase away}. This could be the implication of flying fast behind someone. In one case, however, P4 uses \emph{kouri dèyè} where the subject isn’t moving at all. The picture shows a group of birds sitting in a tree chasing away the little bird by screaming at him \REF{ex:3:71}.

\ea\label{ex:3:71}
\gll Zwazo sa yo genlè pa renmen  li. Yo kouri dèyè  li. E lè sa li vole.\\
     bird \textsc{dem} \textsc{pl} seem \textsc{neg} like  \textsc{3sg} \textsc{pl} run behind  \textsc{3sg} and when \textsc{dem} \textsc{3sg} fly\\        
\glt ‘These birds seem to not like him. They chase him away. And then he flies away.’
\z

Another problem is the semantics of \emph{pati}. P1 uses \emph{kouri dèyè} and \emph{pati dèyè} in a similar way. When the owl is chasing the little bird by flying after him, she uses \emph{pati dèyè}. The semantics of \emph{dèyè} are obscure here, because the Path ‘away from X’ is not relevant here. It is neither shown in the pictures nor expressed in the narration.

For the single picture descriptions, twelve descriptions had to be sorted into the category “Other” (10.1\%). Three cases were equivalent to those in the picture story narrations where \emph{dèyè} was used. In two other cases, hybrid verbs were used which could not be clearly identified as Manner or Path verbs as they contain both components: \emph{plonje} ‘dive into’ and \emph{tonbe} ‘fall’.

In \REF{ex:3:72}, a Path verb is combined with a further description of the Path.

\ea\label{ex:3:72}
\gll Yon mesye ki dwe travèse dyagonal {[}…{]} yon chanm.\\
     \textsc{indef} man \textsc{rel.pron} must cross diagonal {} \textsc{indef} room \\
\glt ‘A man who must cross a room diagonally.’
\z

In \REF{ex:3:73}, the Manner component is expressed in a PP, which is the only occurrence of this type.

\ea\label{ex:3:73}
\gll Tidjo ap monte mòn ak kat pat.\\
     Tidjo \textsc{prog} ascend mountain with four paws\\
\glt ‘Tidjo is crawling up the mountain.’
\z

The four remaining occurrences contain a gerund construction, all used by the same participant, P1. In two of these cases, a structure equivalent to the French structure V \emph{en} V.GER is used, see \REF{ex:3:74}. This structure has not been described for Haitian Creole.

\ea\label{ex:3:74}
\gll Li desann eskalye a an dansan. \\
     \textsc{3sg} descend stairs \textsc{def} \textsc{prep} dance.\textsc{ger}\\
\glt ‘He/She goes down the stairs dancing.’
\z

In the other two cases, the Path verb is followed by a gerund form of `to be', \emph{etan}, and then a full sentence consisting of subject, aspect marker and manner verb, see \REF{ex:3:75}. This structure has also not been described for Haitian Creole.\footnote{Both a native speaker present at the talk at the SPCL Meeting in Tampere as well as an anonymous reviewer of this
paper noted that this structure is very uncommon in Haitian Creole and most probably due to other language influence.} 

\ea\label{ex:3:75}
\gll Tipyè antre etan l ap danse nan chanmnam. \\
     Tipyè enter be.\textsc{ger} \textsc{3sg} \textsc{prog} dance \textsc{loc} room \textsc{def} \\
\glt ‘Tipyè is dancing into the room.’
\z


\section{Discussion}\label{sec:3:5}

In the previous section, the morphosyntactic patterns used to express motion events in the present study were described. Most commonly used were three different structures: a Path verb with a Ground NP or DP, a serial verb construction, or a Manner verb only. Path verbs with Ground elements are verb-framed structures in the sense of \citet{Talmy_1991}. In all of those cases, the Manner component of the event was not expressed. Serial verbs often consisted of a Manner and a Path verb, which could be labelled an equipollently-framed construction in the sense of \citet{Slobin_2004}. Other verb serializations were also found, mostly of two Path verbs, but also combinations of Path and non-motion verbs. These are also verb-framed constructions. The third most frequent pattern is the use of a Manner verb only, with no Path element encoded in the same clause. According to \citeauthor{Talmy_1985}'s (\citeyear{Talmy_1985}) definition, the latter is not a motion event. These cases are nevertheless taken into account here, primarily because of their relatively high frequency. Besides that, it is possible that at least some of them do, contrary to Talmy’s definition, express complex motion events, because the Path component is possibly left to be inferred in these cases but implicitly present. This was the case in some examples from the picture story narrations, where the participants uttered sentences like \emph{Zwazo a kouri/vole} ‘The bird flies (fast)’, meaning that the bird is flying away. Most of the uses of a sole Manner verb in the single picture descriptions are probably due to the fact that the depicted motion event was too difficult to recognize as such, as in \emph{swimming along the coast}. In these cases, the participants simply expressed a motion activity instead of the event.

Besides these three main patterns, three further but marginal patterns were found in the data: a Manner verb with a Path element, a Path verb alone and a motion event without any motion verb. Manner verbs with Path elements constitute satellite-framed constructions in the sense of \citet{Talmy_1991}, which were rare but are still attested in the present data. They occurred with few, but different Manner verbs and also different Path elements. When Path verbs were used alone, information on the Ground was usually given in the context. The expression of motion events with idioms or constructions like \emph{pran nan direksyon} ‘take a certain direction’ is most probably not typical for Haitian Creole but occurs in many languages.

The different patterns described above show that Haitian Creole possesses a rich inventory of morphosyntactic structures available to express motion events. It is therefore not classified here as a language of one of the three types described above, VF, SF and EF languages. All of these three patterns are found in the Haitian Creole data, VF and EF patterns being more frequent than SF patterns.

Some problematic cases were also described above which need further investigation. In the cases where \emph{dèyè} was used, it was not clear whether it expresses ‘behind’ or ‘after’ and it could therefore not be decided whether it constitutes a Ground or a Path element. This shows that clear semantic criteria to identify the components of motion events are needed. Such criteria could also help to further investigate hybrid verbs like \emph{plonje} ‘dive’ which are said to express both Manner and Path. Two other cases were also problematic as they presented completely different structures from the ones described earlier. The structures where a gerund of a Manner verb or of the verb `to be' were used, neither of which has been described for Haitian Creole. As the participants of this study all lived outside of Haiti and were using other languages such as German or French on a daily basis, it is possible that these structures are due to language contact, most probably with French. More research needs to be done in this area.

The second aim of this study was to investigate the Manner salience of Haitian Creole, that is the frequency with which the Manner component is encoded in motion event expressions in comparison with other languages. In the data described above, Manner was expressed either in a Manner-Path verb serialization or in a satellite construction.\footnote{As Manner verbs only and Manner verbs with ground expressions are not considered motion events as defined by \citet{Talmy_1985}, they are not included here. This leaves us with 76 motions event expressions in the picture story narrations and 85 motion event expressions in the single picture descriptions.} In the picture story, both Manner-Path verb serializations and satellite constructions were very rare, which means that Manner was often left unexpressed. As the story was about a bird, whose Manner of motion typically is to fly, it is not necessary to encode Manner in every motion event, as it can easily be inferred. In the single picture descriptions, Manner-Path verb serializations are used in 24.7\% (21 of 85) and satellite-framed constructions in 5.9\% (5 of 85) of the motion event expressions. With a total of 30.6\%, the frequency of Manner encodings is much higher here than in the picture story narrations. Considering the fact that all of the pictures showed a specific Manner component, this number is nevertheless rather small. Both the picture story as well as the single picture descriptions therefore indicate that Haitian Creole has low Manner salience.

In comparison to the French motion verb expressions described in the first part of the paper, some similarities and some differences can be observed. Just like French, Haitian Creole possesses a rather large inventory of Path verbs, most of which probably go back to their French counterparts. This is why VF constructions are common in both languages. However, their percentage is larger in French than in Haitian Creole, as the latter possesses another structure not available in French: verb serialization, particularly the serialization of a Manner and a Path verb. This is a feature that Haitian Creole shares with various African languages, Kwa languages in particular, which are said to have played a significant role in the formation of Haitian Creole. Just as in the Kwa languages described above, the Manner verb precedes the Path verb in the Haitian Creole motion verb serializations. Another interesting observation is the fact that in the present data, no serializations with the Manner verb \emph{danse} ‘to dance’ are attested, a Manner-Path serialization which is ungrammatical in Fongbe according to \citet{LambertBrtire_2009}. The (un)grammaticality of such serializations in Haitian Creole needs to be tested in a subsequent study. The preliminary result of the ongoing research on this question is that the morphosyntactic patterns used in Haitian Creole to express motion events seem to be a mixture of the patterns found in the languages that were relevant to its formation.

In further research, more data will be elicited. As some of the drawings used for the single picture descriptions proved to be difficult to interpret, these representations of motion events need to be revised. If possible, videos showing motion events would be preferred for data elicitation. In addition, acceptability judgments will be elicited to investigate which Manner and Path elements are possible in which pattern, mostly in serial verb constructions and satellite-framed constructions. 


\section*{Acknowledgements}
I am grateful for the ongoing support of the supervisor of my thesis, Prof. Judith Meinschaefer. Further thanks go to the native speaker who helped me transcribe and translate the Haitian Creole data. I would also like to thank two anonymous reviewers who have provided very helpful comments on this paper. All remaining errors are my own.

{\printbibliography[heading=subbibliography,notkeyword=this]}

\end{document}

\documentclass[output=paper,
modfonts
]{langscibook} 
\bibliography{localbibliography}

\usepackage{langsci-optional}
\usepackage{langsci-gb4e}
\usepackage{langsci-lgr}

\usepackage{listings}
\lstset{basicstyle=\ttfamily,tabsize=2,breaklines=true}

%added by author
% \usepackage{tipa}
\usepackage{multirow}
\graphicspath{{figures/}}
\usepackage{langsci-branding}



\newcommand{\sent}{\enumsentence}
\newcommand{\sents}{\eenumsentence}
\let\citeasnoun\citet

\renewcommand{\lsCoverTitleFont}[1]{\sffamily\addfontfeatures{Scale=MatchUppercase}\fontsize{44pt}{16mm}\selectfont #1}
  
 
\title{Language Shift}  

\author{%
 Andreia Caroline Karnopp\affiliation{University of Zurich}
}

% \chapterDOI{} %will be filled in at production
% \epigram{}

\abstract{
Language Shift has always existed. Conquests were the first cause of language shifts, then migrations prompted these types of changes, and today it is mainly language diffusion that triggers this language contact phenomenon. There are some promoting and/or retarding factors for shift, but not a single condition evokes the same patterns of language use in all language contact situations. For this reason, and because each language community should thus be considered and analyzed in isolation, this chapter discusses the most significant approaches, models, methods and examples of possible language choice patterns and trends, and finally, also addresses possible factors that may or may not boost language shift within a determined linguistic community.
}

\begin{document}
\maketitle

\section{Introduction}

Since \emph{Language Shift} (LS) is always preceded by language contact or collective multilingualism \parencite[320]{Ostler2011}, this social phenomenon is important to include in the discussions addressed within this book. Even if LS can happen at an individual level (\emph{Language Attrition} (LA)\footnote{LA is about ‘forgetting' an educational or extracurricularly learned L1, L2 or foreign language. It thus describes the loss of language skills by an individual and can in a way be considered as a reversal of language acquisition \parencite{Lambert1982}.}), it usually refers to the change in usage of a given language community from a language A to a language B in all situations and domains. This change in norm is usually observable as a bi- or multilingual period within one or across several generations.

LS is often described as a kind of `transitional phenomenon' \parencite[33]{Bohm2010} of changing language contact situations. It refers to a shift away from a `healthy' language state due to a `disorder' or a range of `disorders'\footnote{The term ‘disorder(s)' here refers to the fact that a previously rather stable speech contact situation -- above described as ‘healthy' -- can become unstable and cause a ‘disorder' as a result of various mostly related factors, and therefore change the habitual language use (clearly visible in stage B of Figure \ref{figure1} ).} of the affected languages. \emph{Language maintenance} (LM), in contrast, describes a relatively stable\footnote{LM is described here as ‘relatively stable', since long-lasting and intensive language contact can lead to interferences (e.g. borrowing-scale by \citealt{thomasonkaufman1988}), and further language contact phenomena.} language contact situation in which bilingual speakers, speaker groups or an entire linguistic community continue using the minority or heritage language\footnote{Alternative terms used in the literature are community language, (im)migrant language, ethnic language, and home language. Heritage language, however, is probably the most widely used term \parencite[23]{Pauwels2016}.}, despite the pressure of the majority and socially dominant language and other influencing factors. Consequently, the mentioned `disorders' can provoke different patterns of language use, which is why each language contact situation must be considered separately. What all situations have in common, however, is that LS affects only groups and communities which are in contact with a more dominant and more powerful social group. This is why LS is generally understood as ‘a barometer of inequality between linguistic minorities and the majority’ \parencite[613]{Heinrich2015}.

LS is not a recent phenomenon, but has been occurring throughout the history in different societies and in diverse places \parencite[4]{Puthuval2017}. \cite[326-328]{Ostler2011} supposes that LS started happening with the Neolithic revolution and the related establishment and settlement of humankind -- however, it can be assumed that these processes were already taking place before this period. Between 3000 BC and 1500 AD, dominant languages spread mainly through \emph{wars} and subsequent \emph{conquests} of rural societies. Since then, the languages of European conquest have prevailed -- e.g. Spanish in Latin America, Portuguese in Brazil or English in the USA. Dominant languages are therefore often associated with oversea explorations, invasions and migrations. Before the 20th century, \emph{migration} was the biggest risk for a group to be affected by LS. Nowadays, in contrast, the physical \emph{diffusion}\footnote{\emph{Language diffusion} (LD) often happens on an individual level and is promoted by e.g. \emph{cohabitation} (founding bilingual families) or by \emph{recruitment} (new employment, military, etc.) \parencite[323-324]{Ostler2011}.  In certain domains a LS can then progress quickly and widely. \emph{Linguistic diffusion}, on the other hand, refers to a shift of individual linguistic variants within a language -- which can also be caused by language contact -- on an individual or social level over a longer period of time. Both speakers and listeners can give different preferences to individual variants or even generate new variants \parencite{Gong2012}.} of so-called world languages plays an increasing role because young people tend to learn one of these rather than maintain their parents' minority language. 
\begin{figure}
\includegraphics[width=10cm]{Karnopp_pic1}
\caption{\emph{Language Shift Model}Abb.: L1 = first
  language; L2 = second language; HL = heritage language; ML = majority
  language; L1-HL = abandoned/heritage language; L1-ML* =
  target/majority language (can contain phonetic, morphologic,
  syntactic, semantic and prosodic traces from the HL.}
\label{figure1}
\end{figure}

%Both figures (Fig 1 and 2) were created by me especially for this chapter. How can I quote that?

As most of the LS literature, this chapter will mainly discuss the language use of minority groups -- migrant communities and territorial linguistic minorities, with a special focus on the former. Therefore the model in Figure \ref{figure1}, which is based on \cite{Fishman1964}'s three-generational model, illustrates the different phases typically leading to LS in migrant groups. Likewise \cite{Weinreich1953} assumes that at least three generations are necessary for LS to happen. \cite[6]{Ortman2008}, in contrast, point out that in many \emph{inter}generational analyses, the so-called `mother tongue shift' occurs mainly in the second and third generation. In this regard, LS can be understood here as a gradual and progressive process ① or as a reversible process ③ of the dynamics of a natural multilingual language community. On the other hand, LS is sometimes also analyzed as an outcome ② or a consequence of a language contact situation: The use of the languages changes across the three stages in Figure \ref{figure1}. A monolingual stage A is followed by a situation of language contact, caused, for instance, by migration. A bilingual transition period then follows, which is often \emph{diglossic}\footnote{\emph{Diglossia} is a special form of bilingualism of a language community in which a high and a low variety coexist. While \cite{Ferguson1959} distinguished between two variants of the same language (e.g. the case of German and Swiss German in the German-speaking part of Switzerland), \cite{Fishman1967} extended this definition to language contact situations of unrelated varieties (e.g. Hindi and Tamil in India).} and in which the collective language choice is \emph{variable} \parencite{Fasold1984}. This middle stage, which is at the heart of the progressive LS process ①, can last for one or more generations, and may affect an entire language community as follows. Three different types of bilingualism are distinguished for stage B: ⓐ supplementary, ⓑ complementary  (see LM), and ⓒ replacive bilingualism \parencite{Haugen1972}. Given that the preferred language can influence the language skills of every individual speaker (in LS situations the L1-HL, and in RLS situations the L1-ML), the three types of bilingualism can also co-occur within the same community. The bilingualism phase is then followed by another not necessarily purely monolingual stage C, as Fishman \parencite*{Fishman1964} idealistically showed. The target language can still contain traces of the L1-HL in the form of code mixing or code switching (see \emph{shift variety} in Section \ref{patterns}, and Chapter 3), 
%Crossreference
or even adopt new features from the heritage language and thus end up in a new variety or language (see Chapter 4).

\subsection*{LS as outcome ②}

Languages are social entities that need an associated society in order that their memory not be lost. This means that if speaker groups or societies, for example migrants, do not live in their home countries and lose contact with them, there is a good chance that they will shift more quickly to the L1-ML. Therefore, a language's survival depends on who speaks what, to whom and when \parencite{Fishman1964}.

The \emph{transmission} of an L1-HL to the following generations can be disturbed or inhibited during the three LS phases (see Figure \ref{figure1}) for various reasons, e.g. if a language community dies out, if it is conquered by another group who speaks a different language, or if it is eradicated (see Chapter 4)\footnote{It is important that we take into account here that LS does not end with a person's life or the life of a group, but rather represents a shift or a change from generation to generation \parencite[195]{Jagodic2011}.}. The speakers make or are forced to make a so-called ‘social choice' in order to better integrate themselves (or not) in the new society \parencite[325]{Ostler2011}. In other words, the preference for one of two or more contact languages automatically generates social closeness or distance. So if a bilingual speaker chooses the L1-ML, he automatically selects social proximity to the out-group and social distance to the in-group -- and vice versa. At this point in time the corresponding L1-HL is threatened if it is not spoken by another group or if it is not used for a specific purpose there is danger of becoming an \emph{endangered language}\footnote{Without adequate documentation, frequent use between L1-HL speakers in different situations and domains, and without transmission to the next generation, a language is \emph{endangered} and thus threatened with extinction (see EGIDS, the 13 levels of language endangerment/vitality proposed by \cite[2]{Brenzinger2003} based on Fishman's (\citeyear{Fishman1991}) 8-level GIDS -- see Section \ref{approaches}.}. The former mother tongue is then gradually replaced from generation to generation by the L1-ML (\emph{obsolescence} or \emph{language dead} at group level and \emph{attrition} at speaker level, e.g. \citealt{Crystal2000}).
According to \cite[7]{Nettle2000}, more than half of the more than 6000 languages spoken in the world are currently at such a stage. The process mainly affects small minority languages in Australia, the Pacific and in North and South America. Normally, such an advanced state of LS is irreversible, and has therefore achieved its \emph{morbid endpoint}\footnote{The difference between a \emph{morbid endpoint} of a language and \emph{language death} is that in the former case a language can still be spoken by other language communities.  On the other hand, the term \emph{language death} can be understood literally, because in this case a language is not spoken anymore, because it has been forgotten or simply not learned or passed on, and therefore no longer exists.} -- and not \emph{death} \parencite[18]{Pauwels2016}.


\subsection*{Reversing LS ③}

If a speech community sees a reason to take active steps to preserve an endangered (heritage) language, and if the language policy of a region or country supports these actions, an \emph{ongoing}\footnote{LS is to be viewed as a process in which different factors come together. This can be extremely dynamic. For this reason, and so that a model can be predictable at all, it must be flexible and able to take account of changing circumstances. Therefore I use the term \emph{ongoing}, like \cite[112]{Pauwels2016}.} LS-process can change direction and be reversed (\emph{Reversing Language Shift}, RLS), if it has not yet reached the \emph{morbid endpoint}. The heritage language can be documented by linguists and stored in archives, or get actively preserved and maintained through revitalization \parencite[315]{Ostler2011}. This reversal requires a new distribution of power between the language communities, which may lead to a different language policy issue. The idea of many supporters of minority languages is to teach it to the younger generation in school so as to enable them to use it regularly and pass it on to subsequent generation(s) \parencite[4]{Puthuval2017}. If an L1-HL plays a part in defining a sense of identity (see \emph{core-value}), if it hosts the community's culture and traditions, and if it is the basis of knowledge and experience, nowadays people or institutions often eagerly try to preserve that language. In this respect, language diversity, even though LS is the social norm \parencite[84]{Pauwels2016}, is still a universal phenomenon.

\subsection*{LS-process ①}

On the other hand, LS can also be understood as a process, process in the sense that  the dominant language spreads at the cost of the minority language \parencite[31]{Bohm2010}. Language `lives' and is associated with an \emph{ongoing} learning process that can lead to changes such as
variant formation (see Chapter 2, short- or long-term-accommodation),
%Crossreference
speaker-related language mixing (see Chapter 3, code-switching), 
%Crossreferencenew
new languages or varieties (see Chapter 4, Creoles and Pidgins) 
%Crossreference
and thus to long-term change (language shift or language change). Due to differences in individual settings, situations, speakers, etc., there is still \emph{no uniform and general definition} of the phenomenon LS. As \cite[19]{Pauwels2016} explains:

\begin{quote}
it may take one or more generations of speakers before the language is entirely abandoned. It also implies that the shifting away from the L1 does not occur simultaneously across all its users or functions and settings. The rate and speed of the shift process will vary from community to community. In some cases the process is relatively swift, within one or two generations, and in other contexts it will take much longer.
\end{quote}

The duration of the shift process therefore depends on various influencing factors (see Section \ref{factors}): While some language communities change their main language within only one generation, for instance, Dutch migrants in New Zealand (e.g. \citealt{VanRijk2017}), other migrant groups manage to maintain their L1-HL over several decades or centuries as the Amish  in the USA (e.g. \citealt{Sağlamel2013}) or the Swiss in Brazil (e.g. \citealt{Karnopp}).

Figure \ref{figure1} represents an overview of the phases typically involved in LS in a bilingual language community with language contact. However, this model does not hold for all settings and all contact situations with LS as an outcome. The transition period between a monolingual setting with language A and a monolingual setting with language B can be more multifaceted than depicted in Figure \ref{figure1} . The transition phase is discussed in more detail in the following subsections. Section \ref{approaches} discusses the main approaches to LS, presents theoretical models, and summarizes the methods typically used in LS research. Language choice patterns, which are key to the process of LS, will be discussed in section \ref{patterns}. Section \ref{factors} gives an overview of possible factors promoting LS. And finally, section \ref{discussion} summarizes the most important findings of the chapter and points out promising routes for future research.

\section{Approaches, models, and methods}
\label{approaches}

\noindent The various approaches to the study of language shift are best understood when we observe the transition period from the initial monolingual setting preceding the shift to the final ‘monolingual' setting following it. As Figure \ref{figure2} illustrates, the bilingual transition period of an LS involves not only factors regarding an individual speaker, but also often produces a situation where wider social, even societal phenomena become relevant. Furthermore, a finer differentiation between social levels (micro, meso, macro) is helpful in assessing the approaches, models, and methods presented in this section.

\begin{figure}
\includegraphics[width=10cm]{Karnopp_pic2}
\caption{Social model for LS-processes, based on Sasse \parencite*[63]{Sasse1992}.}
  \label{figure2}
\end{figure}

\subsection{Approaches and models}
\label{approachesmodels}
%(See Section \ref{approaches}).

LS and its counterpart, LM, both have a multidisciplinary nature. Since the beginning of the 20th century, they have attracted the attention of a number of scholars from a variety of disciplines -- including sociology, language sociology, anthropology, social psychology, sociolinguistics, contact linguistics, applied linguistics, demography, politics and history. If a language is to be considered in connection with its speakers and an entire society, this can sometimes lead to an interdisciplinary challenge -- because each discipline has its own questions and methods, it can sometimes be difficult to make them compatible with each other. Since LMLS studies are characterized by a wealth of approaches, models, and research methods, only a portion of the most influential ones are introduced in what follows (however, in Section \ref{factors} some of these will be taken up with regard to factors that can promote or retard LS).

Kloss initiated LM's systematic study for ethnic minorities in Germany. His key text (\citeyear{Kloss1966}) on language choice in correlation with a wide range of individual and group factors led to the generation of a quantitative taxonomic-typological model\footnote{In a taxonomic-typological model, language is named on the basis of types and systematically classified with regard to its structural and functional features.}. This was the first attempt to demonstrate the dynamics of LM and LS.
Based on Kloss' work, Haugen  (\citeyear{Haugen1972}) developed his concept of ‘language ecology'\footnote{In his model, Haugen used the ecosystem as a metaphor to show how languages behave in different language contact situations and how endangered languages can be preserved, similar to endangered species.} in migrant settings and expanded the field from Europe to North America. His \textit{descriptive} and explanatory model was the first to take into account the interaction between languages, their speakers, and the social environment.
Fishman's (\citeyear{Fishman1972}) study on ‘language use patterns' is one of the most important approaches in LMLS research. Using his famous question ‘who speaks what language to whom, and when', shift processes can since then be analyzed across a range of (originally) five main domains: \emph{family, education, employment, friendship,} and \emph{religion} -- although these may vary depending on each specific language contact situation. He assumes an ideal language contact situation in which all members are multilingual, regardless of the language competence of its individual speakers. If the analysis of an \emph{intra}group within domains and further factors is extended to an \emph{inter}group situation, language contact can be analyzed not only at the micro level but also at the macro level (\cite[335-336]{Werlen2004}, see also Figure \ref{figure2}).
Giles' et al. \parencite*{Giles1977} ‘ethnolinguistic vitality model'\footnote{ \cite[308]{Giles1977} understand ‘ethnolinguistic vitality' to be the distinctive and active collective behaviour of a minority group in \emph{inter}group relations.} includes socio-psychological factors as \emph{status, demography}, and \emph{institutional support}. To analyze the vitality perceptions of languages in contact within and between minority and dominant groups, language identity and language attitudes play a decisive role (see Section \ref{factors}).
The fourth milestone is Gal's \parencite*{Gal1979} study on language use in bilingual communities in the Austrian-Hungarian border. She was the first to take into account \textit{social and communicative networks}\footnote{Gal \parencite*{Gal1979} understands ‘social and communicative network' as the environment in which a speaker normally interacts in a given unit of time.} (see also \citealt{Dorian1980}). Language choice plays a very important role in this. However, if both contact languages are equally appropriate in a network, it is not possible to predict which language bilingual speakers will choose in which communicative situation. With her new qualitative approach in this field she was able to show that for certain language groups, at a specific historical moment, \emph{language choice can be variable}. 

Gal's pioneering study was followed by further research and enhanced concepts such as Bourdieu's \parencite*{Bourdieu1977, Bourdieu1982} \emph{linguistic markets} - stands as a metaphor for the place where linguistic exchange occurs and linguistic ‘capital' can be exchanged; Lieberson's \parencite*{Lieberson1980} distinction between \emph{age-grading} and \emph{age-cohort} analyses - linguistic changes on an individual level (former) or within an age group (latter); Smolicz's \parencite*{Smolicz1980} \emph{core-values} and their relationship to LM, with regard to the most important cultural and social values of a linguistic community; Tajfel's \parencite*{Tajfel1981} \emph{social identity theory} - which is intended to explain \emph{inter}group behaviour; Fishman's \parencite*{Fishman1991} discussion about RLS and the necessary redistribution of power within a community, as well as the promotion of the 8-level \textit{Graded Intergenerational Disruption scale} (GIDS) - an evaluative framework to identify endangered languages; and Edwards' \parencite*{Edwards1992} \emph{typology of language endangerment} - which includes factors for the viability of endangered languages.
%verylongandcomplexsentence

Although the approaches and models listed here are fundamental for a better understanding of the dynamics of LS processes, each of them also has individual weaknesses. In addition, they can only shed light on a specific part of the whole phenomenon. For example, Kloss' \parencite*{Kloss1966} clear-cut factors are not necessarily unequivocal indicators of LM for all migrant contexts. On the other hand, additional factors - not taken into account by Kloss - may also lead to the preservation of a heritage language (e.g. \citealt{Clyne1991}  in his research on migrants in Australia). Fishman's \parencite*{Fishman1972} model is based on a clear domain-by-domain shift, which is nowadays extended to further domains, as each language contact situation is unique and can therefore generate additional ‘exchange locations'. On the other hand, Fishman's domains can be inhibited by other language contact phenomena such as code-mixing and code switching
%Crossreference Chapter 2
. His proposal is thus better placed within an expanded \emph{domain continuum} -- from public to private domains -- by taking into account both the LS of a single speaker and the LS within the language community. Smolicz' \parencite*{Smolicz1980} core-value theory was also criticized by Clyne \parencite*{Clyne1991} due to its relative simplicity: the definition of ‘group' is problematic, the model is inapplicable to several group affiliations, and language attitudes can change, even if they normally are considered to be stable over a longer period of time (e.g. RLS). Newer approaches and models aim at \emph{hybridity} and being \emph{ongoing}, whereby abstract and episodic-concrete language material is to be made comparable and tested using various factor combinations. One of the first hybrid models for LS was published by Wei \parencite*{Wei2002} with his concept of ‘market, hierarchy, and network' that makes interaction strategies of individual speakers combinable with the community wide norms and values.

In summary, while in the initial phase of LS research the focus was on universal and abstract variables and systems ⑥, which were based on top-down approaches and aimed at the development of traditional-generative models at the macro level, Haugen's \parencite*{Haugen1972} descriptive approach was the trigger to catch the macro world by defining the micro level ④ of a social system. Since then, research approaches within the LS study show a preference for user/agent-based bottom-up models\footnote{For the differentiation of generative and usage-based models, see Langacker \parencite*{Langacker2000} and Prochazka \parencite*{Prochazka2017}.}. Even if the meso level ⑤ was not directly addressed here, it is crucial especially for the preservation of a minority language, since it deals with \emph{inter}generational \emph{transmission}. If the language is not transmitted to the next generation, it is forgotten and lost within the language community (e.g. \citealt{Gal1979}, \citealt{Brenzinger2003}). Starting from the meso level, LS can be viewed in two different ways: By default, an intergenerational LS is normally assumed, that is, a change between generations or age groups within a language community (see Figure \ref{figure2} ⑤⑥, \citealt{Lieberson1980}). The Lagged Generation Model by Myers et al. (2006 in \citealt{Ortman2008}) serves this purpose. However, since LS can also happen within a single speaker (see LA), the \emph{intra}generational change must also be taken into account -- e.g. analyzable  by the Period Cross Section Approach, discussed in Myers et al. (2006 in \citealt{Ortman2008}, see Figure \ref{figure2} ⑤④). 
In this regard, \cite[1423-1424]{Lutz2006} stated in her study on Latino youth in the USA that ‘the shift from Spanish to English as a usual language appears to occur as children progress through the school system’. 

\subsection{Data gathering methods}

Language data for LS research can be collected in various ways: through large-scale surveys and census data ⑥, or through observing language use by individuals through participatory observation, interviews, tests, and experiments ④\&⑤ \parencite[48]{Pauwels2016}. A distinction is also made between real-time and apparent-time methods. In a real-time study, the language of different age groups is observed over a longer period of time. Longitudinal studies can show a possible language change of a community as progress through time. Apparent time methods, often implemented as a one-shot case study, focus on the speech patterns of different age groups -- younger and older speakers -- in a specific moment in time and can indicate a language change in progress.

Using a questionnaire is the most commonly applied method for data collection in LMLS investigations. It can be applied to large-field studies, which tend to target quantitative data, or to smaller-field qualitative analyses. A challenge regarding all methods that employ interviews, in addition to the choice of informants, is the interviewer's role. In certain cases a bilingual interviewer or a member of the group under study may be preferred. This helps to ensure the authenticity of the linguistic data, and that trust and solidarity with the informant can be built up. The questionnaires themselves vary from closed-ended questions to multiple-choice questions, point scales to open questions \parencite[53-61]{Pauwels2016}. 

Surveys and census data are often used in longitudinal studies, providing objective data for comparison. Within the field of LS these data can be used to assess, for instance, number of speakers, geographical distribution, and sociodemographic profiles \parencite{Clyne1991}. However, surveys can be expensive and they address only a portion of the targeted group at a time. Censuses, on the other hand, are more exhaustive, but data regarding language use is often inaccurate or even subjective and therefore not especially valuable. Regarding this concern \cite[para. 16]{Buda1992} adds that:

\begin{quote}
respondents may not be fully conscious of their own language usage patterns, or may wish to portray them in a socially or culturally favorable light. Very often the respondent's assessment of his or her own language ability and usage represents more of what he or she would wish them to be, and less of what they really are.
\end{quote}

In the same vain, Pauwels \parencite*[66]{Pauwels2016} notes that a self-assessment of language skills is not comparable with accurately measured linguistic proficiency in reliability.

\section{Language choice patterns and trends}
\label{patterns}

The Fishman question -- who speaks what language to whom, and when -- can be further expanded with the question of \textit{how well} a language is spoken. Pivotal for answering these questions is \emph{language choice}, the selection of a language in a given communicative situation. As language choice patterns are variable and often difficult to generalize, LS can be seen as a long-term consequence of language choice \parencite[53]{Holmes2013}. \cite[68]{Fishman1965} suggests that language choice must first be analyzed in individual face-to-face meetings before approaching the ‘problem of the broader, underlying choice determinants on the level of larger group or cultural settings’. Concerning this, language choice patterns within a stable bilingual setting can be further applied to interpret less stable contact situations (see Figure \ref{figure1}). A domain analysis concerning the three social levels (see Figure \ref{figure2}) is useful to observe some general language choice patterns and trends. In what follows, the expanded Fishman question will be used as a framework for this more compassing analysis.

\subsection*{Who?}

The question \emph{who} speaks a specific contact language can be viewed, for example, in relation to \emph{age-related} patterns. Many studies (e.g. Gal \citeyear{Gal1979}, Wei \citeyear{Wei2002}, Karnopp \citeyear{Karnopp}) show that older speakers would rather maintain an L1-HL, while younger people often shift much faster to an L1-ML. This has to do with the fact that older migrants are usually more dependent on their heritage language. Learning the majority language is often more difficult for them -- if they learn it at all -- and their \emph{social networks} tend to the in-group. However, the use of L1-ML can also increase for older generations, for instance when they spend a lot of time within the out-group during their working years (see subsection ‘When?’). Likewise, younger generations may grow up in an L1-HL environment, by the latest within school or already before -- or by older siblings or through media -- they come into contact with the majority language (\emph{speech capital} ⑤). \cite[84-85]{Pauwels2016} notes in this connection, that if the second generation does not speak the heritage language as well as the first generation, and, additionally, if their language displays more contact phenomena, such as for instance code switching (see Chapter 3),
%Crossreference
LS progresses faster. 

On the other hand, \emph{gender-related patterns} can influence language choice, even if researchers do not agree on this. \cite[213-215]{Labov1990} therefore proposed the ‘gender paradox', which states that women can be both conservative and innovative in language use. But whether the female gender inspires or retards LS depends on her \emph{role relationship} and \emph{status} in a determined minority \parencite[86-88]{Pauwels2016}. Hence, a monolingual housewife who never had to or could learn the majority language, and also cares for her (old) parents, rather tends toward LM. In contrast, a bilingual woman who no longer lives in a migrant context may prefer L1-ML (LA), although a ‘healthy' bilingualism (LM) cannot be excluded here either.


\subsection*{To whom?}

In addition to the individual circumstances of each bilingual speaker, it is equally important to consider \emph{to whom} someone speaks one of two or more contact languages. Again this is closely related to a speaker's \emph{social network}, \emph{role relationship} and the conversational \emph{topics}\footnote{Along with domain analysis, these are further factors \cite{Fishman1964} considers in order to best define the language choice within a speech community.}: If a bilingual works in the countryside and only has contact with members of the in-group, an increased use of the L1-HL and thus LM is more likely. In contrast, a small in-group network and a low \emph{common routine} can boost LS. 

The \emph{home domain} also helps show how the role relationship within a family can change over generations, as the home is usually the last location to be affected by LS \parencite[616]{Heinrich2015}. If the \emph{speech capital}\footnote{\cite[18]{Bourdieu1977} defined \emph{speech capital} as the mastery of a language. Speech capital is closely related to cultural capital, since it is not  important to learn grammar and vocabulary, but equally vital for the speaker to identify with the culture’s \emph{language attitude} and \emph{prestige}.} ⑤ in families is low, or parents consider it `unfavorable' to transmit a language (\emph{transmission pattern}), this can have a negative effect on the language setting and use. \cite[84-89]{Pauwels2016} emphasizes that in migrant communities it is often the case that parents of the first generation use the L1-HL regularly among themselves, and with others the same age and older. In parent-child conversations there is a continuum of reciprocal to non-reciprocal use of the heritage language. The second generation therefore can learn the heritage language, but uses it less and are not as likely to \emph{transmit} it, especially in exogamy families. Nevertheless, the L1-HL can be used again more frequently when children have to look after their parents in old age.


\subsection*{What language and how well?}

A distinction between \emph{inter}- and \emph{intra}individual variation is useful at this point, since nobody speaks the same way all the time, and the speaker's choice among varieties -- languages or speech styles (\emph{language choice pattern} -- is usually linked to the corresponding social context in some way (\cite[12-17]{Gal1979}, see also Chapter 2
%Crossreference
). In any case, bi- or multilingual speakers normally know which of the two or more languages in contact to use with whom, and when (\emph{linguistic competence} and \emph{linguistic performance}\footnote{According to \cite[3]{Chomsky1965} every person has an unconscious grammatical knowledge of a language which is innate and allows them to understand and speak. Within this concept, he makes a fundamental distinction between \emph{competence} -- which includes knowledge of the speaker and listener of a language -- and \emph{performance} -- which describes the actual use of a language in specific conversational situations.}). Depending on the \emph{speech capital} ⑤ of the parents or older siblings, speakers in minority settings can unconsciously learn several languages simultaneously (\emph{bi- or multilingualism}).  In doing so, the \emph{competence} of each speaker may differ as follows: If both parents are bilingual -- or even speak only L1-HL --, there is an increased tendency toward LM in the home domain. The same applies to endogamy (\emph{endogamy pattern}). However, if both parents come from a different ethnic minority or one of them belongs to the majority society (\emph{exogamy pattern}), the probability of LS is considerably higher \parencite[89]{Pauwels2016}. \cite[65]{Holmes2013} thus affirms that ‘{[}m{]}arriage to a majority group member is the quickest way of ensuring shift to the majority group language for the children’. In contrast, an L1-HL can be sparsely spoken and transmitted to the next generation due to, for instance, negative \emph{attitude}, negative \emph{prestige}, lack of \emph{institutional support}, or  the support of other (world) languages of more \emph{economic interest} (\emph{diffusion pattern}). Often the heritage language is then only hesitantly used due to \emph{uncertainty}. In an environment with such a low \emph{linguistic starting point} ④, the children may not acquire the full competence of a minority language and thus the performance can contain inaccuracies (see \emph{semi-speaker} in Dorian \citeyear[87]{Dorian1980}). Concerning \cite[116]{Wei2002}, such speakers tend to use ‘linguistic innovations, structural changes, and new varieties of language’. 

On this account recent LS-studies not only focus on \emph{language choice patterns}, but also on how languages influence each other within their \emph{linguistic levels}. This can happen on a lexical as well as on a grammatical level, as \cite[35]{thomasonkaufman1988} showed with their ‘five level scale of borrowing'\footnote{\cite[37]{thomasonkaufman1988} define \emph{borrowing} as the ‘incorporation of foreign features into a group's native language: the native language is maintained but is changed by the addition of the incorporated features’.}. In their opinion, the \emph{contact-intensity}, and not the \emph{language structures}, determine possible outcomes of language contact. However, later studies also confirm the latter (e.g. Treffers \citeyear{Treffers1999}), because 

\begin{quote}
speakers in general are able to construct new word formation devices, new syntactic forms and generally are linguistically creative, in the well-documented Chomskyan sense, even if input of a specific structure is slight or lacking. \parencite[591]{Gal2008}
\end{quote}

During an \emph{ongoing} shift process, language contact effects usually happen between the middle and the last stage (see Figure \ref{figure1}). In my own research on the Swiss colony called Helvetia in São Paulo \parencite{Karnopp}, where the speakers today are strongly assimilated to the L1-ML, I was still able to identify some \emph{salient patterns}, as some informants showed a different, and for the region rather atypical, pronunciation of  some consonants in both contact languages (\emph{pronunciation pattern}). For instance, the common retroflex [ɻ] of the region is hardly used by the older bilingual generation, while the youngest generation uses it more than the surveyed young generation of the out-group, in order to differentiate themselves linguistically. Another finding are neologisms (\emph{word-formation pattern}), such as \emph{xeníssimo}, composed of the Swiss German adjective ‘scheen' and the Portuguese suffix ‘-issimo'. 

Sometimes even in the third and last LS phase (see Figure \ref{figure1}) there can still be some ‘remnants' of the former language contact situation. This is the case, for example, when bilinguals ‘create' new \emph{shift varieties}, recognized and adopted subsequently by the entire language community (see \emph{shift-induced change} in Thomason and Kaufman \citeyear[38]{thomasonkaufman1988}). In Ireland, for instance, comparatives are double marked in non-standard Irish English such as ‘working more harder’ \parencite[153]{Hickey2010}. According to the author, this shift can have two causes: either the form was taken from the Irish comparatives, formed by the particle \emph{níos} 'more' as well as the inflection of the adjective; or it comes from an older form of English, where this \emph{doubling pattern} also occurred.

\subsection*{When?}

For a better understanding of LMLS processes the study of interactional settings are central and imperative. A fundamental distinction is made between public and private domains (\emph{domain pattern}), which -- as I have already mentioned -- should be treated as a \emph{continuum}, as they are not always clearly delimitable. For example, if someone teaches at home, this domain is becoming both private and public. The \emph{labor market}, on the other hand, is considered to be a public space, although acquaintances and friendships between colleagues or business partners can also be cultivated here, which in turn produces more of a private character. The labor market thus has many facets, of which four possible language contact situations are shown below.

Minority members who work in a family business, for example farmers with a little village shop, may have a smaller \emph{social network} often limited to the in-group. If an L1-HL enjoys positive \emph{prestige} in such an environment, LM can be expected. However, if a migrant no longer lives within his language community because they moved to the next larger city for professional reasons, the tendency to use L1-ML in everyday life increases drastically. If the use of the heritage language furthermore decreases within the family domain, \emph{attrition} can be the consequence. It becomes even more challenging when a heritage language speaker works in a multi- or international company, where the linguistic exchange takes place exclusively in a \emph{lingua franca}, such as English or Spanish (\emph{diffusion pattern}). The probability that a minority language will ‘survive' under such circumstances is at this point rather low within the labor market -- but not impossible. Another complex language contact situation occurs at construction sites, where members of different ethnic groups work together. Language contact phenomena such as accommodation (see Chapter 2)
%Crossreference
or code switching (see Chapter 3)
%Crossreference
are common here, since many construction workers have often not (yet) properly learned the majority language. In order to still be able to communicate with each other, the L1-ML is drastically simplified and usually pronounced with a noticeable accent -- which leads to this variety being strongly stigmatized (negative \emph{prestige}). For mutual understanding to be possible, only common knowledge of the meaning and application of the words referring the construction are important. Today this ‘primitive language' is considered a \emph{learner variety} -- and not a pidgin, even if there are simplified structures in both of them (see Chapter 4)
%Crossreference
-- of migrant workers \parencite[129-135]{Riehl2014}.

More (rather) public domains proposed by \cite{Fishman1972} are \emph{education} and \emph{religion}. The school is not only a possible pivot point for learning (heritage) languages, but is also crucial for their revitalization and preservation (see RLS). Consequently, it is important to have, for instance, a supportive \emph{language policy} as well as for the minority group(s) to be \emph{interested in preserving and cultivating} these languages. For example, the Swiss descendants in Helvetia \parencite{Karnopp} built their own private school shortly after the founding of the colony, where their children could learn High German -- since this is the official standard language in German-speaking Switzerland -- as well as Swiss history and culture. After the nationwide ban on learning and using foreign languages, official German lessons were discontinued\footnote{High German courses were offered in the 1990s and have again been introduced since 2015. However, these efforts are only moderately fruitful and, in my opinion, do not lead to a language revival in Helvetia.}. Since the school was nationalized, in the 1980s, and had to open its doors to non-Swiss descendants, LM  was not possible anymore within this domain. Either way, \cite[95-96]{Pauwels2016} points out that when private schools consider the L1-HL of a language community, the programs usually focus only temporarily on bilingualism. Their aim is to prepare the students for the \emph{linguistic assimilation} toward L1-ML. However, in-group children often go to mainstream schools, where they only communicate in the majority language anyway.

On the other hand, the church is an important meeting place for believing (minority) groups, because faith unites and strengthens. The Helvetians have always been very Catholic, therefore they built their own church, and some of them still believe their harvest depends on the goodness of Saint Nicholas. Until the ban on foreign languages, the mass was said in High German. To this day, the Helvetians have maintained their tradition of exchanging greetings in front of the church after the official part of the service. However, what has changed is the language: the discussions have shifted from Swiss German to ‘regional' Portuguese -- with very little interferences like \emph{Giotä Sunnti} (trad. Have a good Sunday).

\section{Factors that can promote LS}
\label{factors}

\begin{quote}
Why is it that one minority group assimilates and its language dies, while another one maintains its linguistic and cultural identity? (Bradley 2002: 1, \emph{apud} Pauwels \citeyear[58]{Pauwels2016})
\end{quote}

Most studies on LS have repeatedly focused on identifying causes and factors which can promote or slow down the LS  process. On this basis, scholars have tried to generate a \emph{unique set of factors} that make LS predictable within every language community. However, certain factors may achieve differential effects, even in very similar contact situations. Kloss (\citeyear{Kloss1966})  noticed this early and suggested a typology in which he not only offered a set of \emph{clear-cut} factors (which clearly promote LM), but also \emph{ambivalent} factors (which can promote LM \emph{and} LS). The latter are \emph{linguistic attitude}\footnote{\emph{Linguistic attitudes} describe a positive or negative evaluation through social status of a language or variety in contact.} (speakers with negative feelings toward their L1-HL tend to LS), \emph{educational level} (speakers with little or no education tend to LM because their \emph{social network} is often smaller and more limited to in-group contacts, while a speaker with a higher degree may work outside the community and therefore has more  frequent contact to the majority), \emph{linguistic and cultural similarity} (contact languages from the same language family do or do not tend toward LS -- it depends on whether the desire for assimilation or differentiation is greater), \emph{numerical strength of the group} (language communities with a smaller number of L1-HL speaker tend to be more LS oriented, because they have little \emph{common routine}), and other sociocultural characteristics such as \emph{role of the family} (if an L1-HL is not used anymore for communication within a family -- in other words, low \emph{speech capital} -- and is no longer \emph{transmitted} to the next generation, the tendency is LS).

Thereupon Fishman  (\citeyear{Fishman1972}) presented his domain analysis for speech communities. Each of them contains domain-specific factors with regard to \emph{addressee} (to whom a specific language is spoken), \emph{setting} (in which environment a language in contact is used), and \emph{topic} (which subjects promote the language choice) -- discussed in greater detail earlier in this section. 

Giles' et. al. (\citeyear{Giles1977}) suggested three factors to define a minority group's vitality: \emph{status} (economic, social, sociohistorial, and language status), \emph{demography} (distribution and numbers of speakers), and \emph{institutional support} (formal and informal facilities). He explains that minority groups with a higher vitality (\emph{high attitude, high prestige, common routine}, etc.) tend to differentiate themselves from the dominant group, while those with a lower vitality show faster assimilation, and thus a faster LS.

In Gal's (\citeyear{Gal1979}) pioneering study, she considered social causes ⑥ such as \emph{urbanization} (LS often lasts longer in rural areas than in cities), \emph{industrialization} (new and better qualified jobs, achievable e.g. through higher education can also lead to LS), \emph{loss of isolation} (once rural regions have been taken over by political power, LS progresses), and different \emph{social and communicative networks} that can influence language use and language choice. \cite{Dorian1980} adds \emph{migration}, \emph{mobility of people} (the progress of means of transport and communication makes people much more flexible and enables them to move within a short time to another linguistic environment), and \emph{community size} (see \emph{numerical strength of the group} in Kloss \citeyear{Kloss1966}) to the list of factors pushing LS.

Smolicz's (\citeyear{Smolicz1980}) core-value theory states that \emph{symbolic group values} -- for instance language attitude and prestige, family cohesion, and religious- and cultural unity -- have a massive influence on LMLS and can thus convey a different identity\footnote{\emph{Identity} is a term very difficult to define because it is dynamic and changeable. It stands for, among other things, a correlation between ‘being me' and ‘belonging to the group'. Every human being has different ‘identities' which predominate depending on the situation or with whom one is interacting. Within language contact research ethnic/national (group membership, e.g. based on physical, religious or social factors), social (e.g. social stratification), geographical (e.g. language and dialect) and contextual (e.g. secret languages) identities can be relevant \parencite[172-173]{Riehl2014}.}. For example, if the L1-HL is handled as a \emph{core-value} within a language community, LM is more likely. In contrast, a negatively assessed heritage language of a single speaker (see \emph{attitude}) or the majority (see \emph{prestige}), makes LS more likely.

In more recent case and theoretical studies, further factors such as \emph{age} (in minority communities older people tend to be bilingual, while younger people sometimes hardly understand or speak the L1-HL), \emph{gender} (gender roles in the society -- see Section \ref{patterns}), \emph{language transmission} (if an L1-HL is not passed on to the next generation, younger people no longer speak the heritage language, which provokes LS), \emph{religion} (if in a bilingual colony the sermon is delivered in L1-ML, it is more likely that the majority language will be maintained in conversations after the church service -- see Section 3), \emph{marital status} (exogamy usually leads to LS), \emph{linguistic, social and ethnic identity} (identification with a group often supports assimilation, which can promote LS or LM, depending on the situation), \emph{language prestige} (if a bilingual language community appreciates more the L1-ML, LS is foreseeable), \emph{literacy} (if a contact language is not read or written, there is also a tendency toward LS) and \emph{media}\footnote{By the term \emph{media} I do not only mean written access (newspapers, magazines, etc.), but also digital media such as televisions and, above all, the Internet, computers, smartphones and tablets, which nowadays make contact with the home country easier and more accessible.} (low medial contact with the L1-HL can cause LS) are proposed to analyze LS stages (see e.g. Lutz \citeyear{Lutz2006}, Ortman \citeyear{Ortman2008}, Böhm \citeyear{Bohm2010}, Jagodic \citeyear{Jagodic2011}, Ostler \citeyear{Ostler2011}, Sağlamel \citeyear{Sağlamel2013}, Heinrich \citeyear{Heinrich2015}, Pauwels \citeyear{Pauwels2016}, Perez \citeyear{Perez2016}, Puthuval \citeyear{Puthuval2017}, Rijk \citeyear{VanRijk2017}, and Karnopp \citeyear{Karnopp}). \cite*{Holmes2013} therefore proposes a classification into economic, political, institutional, demographic, attitudinal, educational and socio-cultural factors. For her, these categories are the ones that can be held responsible for the speed of LS within a bilingual community.

As already introduced before, my current research \parencite{Karnopp} looks at the language contact situation related to the Swiss colony Helvetia in São Paulo, Brazil. Since its foundation in 1888, the \emph{language usage patterns} within the colony have undergone some fundamental changes (see also Section \ref{patterns}). Initially Helvetia was a \emph{language island} and therefore linguistically quite well isolated and shielded. The everyday language was the L1-HL -- Swiss German dialect from the Canton of Obwalden -- and when communication with the out-group was required, a translator was called in. After the First World War, the colony's own school had to hire Portuguese teachers and introduce the Portuguese language and other subjects related to Brazil like history and geography. Most of the Helvetians slowly became bilingual and could then communicate with the out-group (\emph{outside diglossia}). With the Second World War all foreign languages were banned in Brazil and everyone who continued to use them was fined or even arrested. These circumstances then led to \emph{inner diglossia}, which henceforth favored LS in all domains within the colony. Today only a few of the oldest generation surveyed still speak and understand the old Swiss German dialect -- many times with a light accent or interferences from Portuguese (see Section \ref{patterns}) -- and with this advanced linguistic assimilation to the Brazilian out-group LS reached its \emph{morbid endpoint} there.

In order to illustrate more precisely which major factors lead to this outcome within the Swiss colony in São Paulo, I defined fourteen main social and individual factors on the basis of the proposed social model (see Figure \ref{figure2}):
\begin{quote}
\textbf{At the micro level ④:} (1) rapid decrease of L1-HL usage in \emph{all domains} -- today the old Swiss German dialect has, even in the \emph{home domain}, a very low \emph{common routine}, (2) \emph{growing language diffusion} among young Helvetians, who would rather learn Spanish or English than High German or the dialect of their ancestors, (3) \emph{low linguistic attitudes and values} toward their heritage language -- because it is no longer needed for communication within the community and therefore considered useless.\end{quote}

\begin{quote}
\textbf{At the meso level ⑤:} (4) \emph{lack in transmitting} the L1-HL after the Second World War, (5) \emph{small group size} which is still decreasing today, (6) \emph{increased exogamy}, among other things to avoid hereditary diseases, (7) \emph{little contact with the homeland} because the Swiss relatives rarely speak Portuguese and less than 30 Helvetians speak Swiss German or High German.\end{quote}

\begin{quote}
\textbf{At the macro level ⑥:} (8) \emph{length of stay since arrival}, because the attachment to Switzerland tended to diminish due to (7) and \emph{low medial contact}, (9) \emph{low geographical concentration} due to resettlement to neighboring bigger cities, offering them more economic opportunities and security, (10) \emph{industrialization} and \emph{career change} away from the peasant lifestyle, (11) lack of \emph{isolation}, especially after the Second World War, due to \emph{political pressure}, (12) \emph{no institutional support} of the L1-HL, (13) \emph{no official literacy}\footnote{Since the standard language in German-speaking Switzerland is High German, it is the language taught in school and used in official contexts (\emph{medial diglossia}, see Glaser \citeyear{Glaser2014}). Moreover, a universal grammar for Swiss German dialects does not exist, because there is a continuum of dialect varieties (\emph{Dialektkontinuum}).} is available for the heritage dialect to date -- neither in Switzerland nor in Brazil, (14) \emph{low language prestige} -- because older bilinguals often have a (light) accent probably caused through language contact, and this is often criticized by younger Helvetians and the out-group.\end{quote}

To conclude, I would like to discuss in more detail the factor that is currently considered one of the biggest risks for LS: \emph{language diffusion}. With regard to globalization and the resulting convergence of languages has been increasingly discussed in recent years. Although heritage languages with a larger population can be supported by the language policy of a given region/country, their use in many migrant settings is diminishing. As I have shown above, mostly older people continue to speak a heritage language or are at least \emph{semi-speakers} \parencite{Dorian1980}, while younger people often have no possibility to learn it due to lack of preservation, neglected transmission, missing institutional support or insufficient revival will. Consequently, what can happen is that younger migrants learn other languages -- apart from the L1-ML-- which, for instance, can help them in \emph{economic terms} \parencite{Holmes2013}. In this regard, India, Pakistan and China show the growing importance of English as a world language. In these countries financial security is defined by speaking English. Only with the competence of this \emph{lingua franca} it is possible to obtain a high rank in the business world, where English determines all financial activities. In contrast to this, \cite[74]{Nawaz2012} explain that in India the less prestigious Punjabi does not guarantee any financial security and is
associated with the low and uneducated majority. Lastly, the high bilingualism rate, even if applied in clearly different \emph{domains}, can cause language contact phenomena such as accommodation (see Chapter 2)
%Crossreference
or code switching (see Chapter 3).
%Crossreference
If learning second languages other than the heritage language becomes more important within a bilingual group, individual factors such as \emph{attitude}, \emph{identity}, \emph{language loyalty}\footnote{\emph{Language loyalty} is a term used to describe a speaker's (conscious or unconscious) relationship with her or his mother tongue.}, and consequently \emph{language prestige} also come into play. For example:

\begin{quote}
They may feel shame when other people hear their language. They may believe that they can only know one language at a time. They may feel that the national language is the best language for expressing patriotism, the best way to get a job, the best chance at improving their children's future. \parencite{SIL}
\end{quote}

Lastly, and to return to my current research, even if the Helvetians appreciate their ancestors and their efforts, their L1-HL will probably not experience a revival there because it has lost its \emph{vitality} and its \emph{social network}. The Swiss dialect is  still considered important for cultural events such as yodeling, but useless in terms of language use, and therefore largely irrelevant within the colony. Consequently, the L1-HL no longer possesses importance as a \emph{core-value} in Helvetia, which is why the LS process will be completed soon.

\section{Discussion}
\label{discussion}

From what we have seen so far, it emerges that not only are languages ‘alive', but also that every language contact situation is \emph{dynamic} and thus \emph{different}. In the scenario of \emph{ongoing} LS, an individual speaker, a group or a whole language community \emph{can} \emph{choose} between an L1-HL and an L1-ML, although this usually happens unconsciously.  \cite[para. 8]{Buda1992} confirms this by arguing that ‘{[}t{]}he phenomenon of language shift takes place out of sight and out of mind’. RLS, on the other hand, certainly happens much more consciously, since it relies on the \emph{will} of  individuals, and of the whole language group, to reintegrate the heritage language into their \emph{social network} for specific purposes (see Figure \ref{figure1}).

LS can also happen when more than just two languages are in contact. In similar settings, \cite{Perez2016} observed that the shift commonly goes toward one of the more prestigious languages. Consequently the \emph{prestige}-factor is certainly one of the important determinants with regard do LMLS. However, in her study of the language contact situation in the Anglo-Paraguayan community New Australia, different circumstances led to the fact that the population did not shift toward one of the global languages (English or Spanish) -- which in turn contradicts the widespread tendency toward LD -- but rather chose to adopt the indigenous language Guarani. 

As this chapter shows, approaches, models and methods must be applicable to exceptional and constantly changing settings that take into consideration the \emph{inter}- and \emph{intra}group variation, but at the same time these methodological choices often compromise different research goals within all three social levels (see Figure \ref{figure2}). With the inclusion of my current research, I wish to reaffirm the fact that the dynamics influencing individual and social changes can be very different in each language contact situation. Therefore the tools that need to be developed for the study of LMLS have to be \emph{hybridized}. Ideally, this would be done by designing a framework that includes a universally applicable \emph{continuum}, from which every researcher would take only what they need for their research goal. The right path to designing this framework has already been taken by recognizing that there are \emph{no specific factors} that can be applied to all LMLS situations, because some factors may or may not promote different language-choice patterns. Now it is only a matter of implementation.

\printbibliography[heading=subbibliography,notkeyword=this]

\end{document}

\section{Introduction} 
Phasellus maximus erat ligula, accumsan rutrum augue facilisis in. Proin sit amet pharetra nunc, sed maximus erat. Duis egestas mi eget purus venenatis vulputate vel quis nunc. Nullam volutpat facilisis tortor, vitae semper ligula dapibus sit amet. Suspendisse fringilla, quam sed laoreet maximus, ex ex placerat ipsum, porta ultrices mi risus et lectus. Maecenas vitae mauris condimentum justo fringilla sollicitudin. Fusce nec interdum ante. Curabitur tempus dui et orci convallis molestie \citep{Chomsky1957}.

\begin{table}
\caption{Frequencies of word classes}
\label{tab:1:frequencies}
 \begin{tabular}{lllll} 
  \lsptoprule
            & nouns & verbs & adjectives & adverbs\\ 
  \midrule
  absolute  &   12 &    34  &    23     & 13\\
  relative  &   3.1 &   8.9 &    5.7    & 3.2\\
  \lspbottomrule
 \end{tabular}
\end{table}

Sed nisi urna, dignissim sit amet posuere ut, luctus ac lectus. Fusce vel ornare nibh. Nullam non sapien in tortor hendrerit suscipit. Etiam sollicitudin nibh ligula. Praesent dictum gravida est eget maximus. Integer in felis id diam sodales accumsan at at turpis. Maecenas dignissim purus non libero scelerisque porttitor. Integer porttitor mauris ac nisi iaculis molestie. Sed nec imperdiet orci. Suspendisse sed fringilla elit, non varius elit. Sed varius nisi magna, at efficitur orci consectetur a. Cras consequat mi dui, et cursus lacus vehicula vitae. Pellentesque sit amet justo sed lectus luctus vehicula. Suspendisse placerat augue eget felis sagittis placerat. 

\ea
\gll cogito                           ergo      sum\\  
     think.\textsc{1sg}.\textsc{pres} therefore \textsc{cop}.\textsc{1sg}.\textsc{pres}\\ 
\glt `I think therefore I am.'
\z

Sed cursus eros condimentum mi consectetur, ac consectetur sapien pulvinar. Sed consequat, magna eu scelerisque laoreet, ante erat tristique justo, nec cursus eros diam eu nisl. Vestibulum non arcu tellus. Nunc dignissim tristique massa ut gravida. Nullam auctor orci gravida tellus egestas, vitae pharetra nisl porttitor. Pellentesque turpis nulla, venenatis id porttitor non, volutpat ut leo. Etiam hendrerit scelerisque luctus. Nam sed egestas est. Suspendisse potenti. Nunc vestibulum nec odio non laoreet. Proin lacinia nulla lectus, eu vehicula erat vehicula sed. 


\section*{Abbreviations}
\section*{Acknowledgements}



\ChapterAndMark{Special and Nexal Negation} 
\label{ch:5}
\is{nexal negation!and special negation|(}
\is{scope of negation|(}
\is{special negation!and nexal negation|(}

The negative notion may belong logically either to one definite idea or to the combination of two ideas (what is here called the nexus).

\is{special negation!defined}
The first, or special, negation may be expressed either by some modification of the word, generally a \is{prefixes!negative|(}prefix, as in\label{sec:neg-prefix}
\is{grammaticalization}  

\phantom{a}

\begin{tabular}{@{}l@{}}
\textit{\emph{n}ever} (etc., see p.~\pageref{para:neveretc})\\

\textit{\emph{un}happy}\\

\textit{\emph{im}possible}, \textit{\emph{in}human}, \textit{\emph{in}competent}\\

\textit{\emph{dis}order}\\

\textit{\emph{non}-belligerent}\\
\end{tabular}

\phantom{a}

\is{adverbs!negative|(}
\noindent (see on these prefixes \chapref{ch:13})\is{prefixes!negative|)}---or else by the addition of \il{English!not@\textit{not}|(}\textit{not} (\textit{not happy}) or \il{English!no@\textit{no}}\textit{no} (\il{English!no longer@\textit{no longer}}\textit{no longer}). Besides there seem to be some words with inherent negative meaning though positive in form: compare pairs like \is{inherently negative meaning}

\phantom{a}

\begin{tabular}{@{}ll@{}}
\textit{absent}& \textit{present}\\

\textit{fail}& \textit{succeed}\\

\textit{lack}& \textit{have}\\

\textit{forget}& \textit{remember}\\

\textit{exclude}& \textit{include}\\
\end{tabular}

\phantom{a}

But though we naturally look upon the former in each of these pairs as the negative (\textit{fail} = `not succeed'), nothing hinders us from logically inverting the order (\textit{succeed} = `not fail'). These words, therefore, cannot properly be classed with such formally negative words as \textit{unhappy}, etc.

A simple example of negatived nexus is \textit{he doesn't come}: it is the combination of the two positive ideas \textit{he} and \textit{coming} which is negatived. If we say \textit{he doesn't come today}, we negative the combination of the two ideas \textit{he} and \textit{coming today}; compare, on the other hand, \textit{he comes, but not today}, where it is only the temporal idea \textit{today} that is negatived.
\is{nexal negation!defined}

Though the distinction between special and nexal negation is clear enough in principle, it is not always easy in practice to distinguish the two kinds, which accounts for some phenomena to be discussed in detail below. In the sentence \textit{he doesn't smoke cigars} it seems natural to speak of a negative nexus, but if we add \textit{only cigarettes}, we see that it is possible to understand it as `he smokes, but not cigars, only cigarettes'.

Similarly, it seems to be of no importance whether we look upon one notion only or the whole nexus as being negatived in \textit{she is not happy} (`she is [positive] not-happy' or `she is not [negative nexus] happy'); %  PE: The (outer) parentheses are ours; the (inner) brackets are OJ's. He used ( ) parentheses. Didn't we agree that parentheses within parentheses would remain ( ) and not become [ ]? ... Though come to think of it, this is an editorial interpolation (by OJ), so [ ] would be more appropriate.
thus also \textit{it is not possible to see it}, etc. In these cases, there is a tendency to attract \textit{not} to the verb: \textit{she isn't happy}, \textit{it isn't possible to see it}, but there is scarcely any difference between these expressions and \textit{she is unhappy}, \textit{it is impossible to see it}, though the latter are somewhat stronger. If, however, we add a subjunct like \textit{very}, we see a great difference between \textit{she isn't very happy} and \textit{she is very unhappy}.

\is{quantifiers!negatived|(}
The nexus is negatived in \refp{ex:05-01}.

\ea \label{ex:05-01}
\textit{Many of us didn't want the war}, but many others did\hfill(news 1917)
\z

\noindent which rejects the combination of the two ideas \textit{many of us} and \textit{want the war} and thus predicates something (though something negative) about \textit{many of us}. But in \textit{Not many of us wanted the war} we have a special negative belonging to \textit{many of us} and making that into \textit{few of us}; and about these it is predicated that they wanted the war. Cf. p.~\pageref{08-not-all}ff (in \chapref{ch:8}) below % PE: OJ simply specifies chapter VIII; no page number
on \textit{not all}, \textit{all {\dots} not}.

Note also the difference between \textit{the disorder was perfect} (\textit{order} negatived) and \textit{the order was not perfect} (nexus negatived, which amounts to the same thing as: \textit{perfect} negatived).

In a sentence like \textit{he won't kill me} it is the nexus (between the subject \textit{he} and the predicate \textit{will kill me}) that is negatived, even though it is possible by laying extra emphasis on one of the words seemingly to negative the corresponding notion; for \textit{\textsc{he} won't kill me} is not `not-he will kill me', nor is \textit{he won't \textsc{kill} me} `he will do the reverse of killing me', etc.\footnote{Jespersen notes in his Addenda that on this page ``or in some other place combinations like \textit{he regretted that \textsc{more} Englishmen did \textsc{not} come here} (news 1917) should have been mentioned''. \eds} % PE: ??? It's not necessary to have both this footnote and the single-sentence paragraph starting "Combinations" that now appears below. Or if simply deleting this footnote is somehow unsatisfactory, then lets simplify it a little and attach it to the single-sentence paragraph.

Cf. also the following passage from \citet[\href{https://archive.org/details/elementarylesson00jevo/page/174/mode/2up?q=\%22curious+to+observe\%22&view=theater}{175}]{jevons1893elementary}:

\begin{quote}
It is curious to observe how many and various may be the meanings attributable to the same sentence according as emphasis is thrown upon one word or another. Thus the sentence ``The study of Logic is not supposed to communicate a knowledge of many useful facts,'' may be made to imply that the study of Logic \textit{does} communicate such a knowledge although it is not supposed to; or that it communicates a knowledge of a \textit{few} useful facts; or that it communicates a knowledge of many \textit{useless} facts. 
\end{quote}

\is{nexal negation!tendency towards}
\is{position of negative}
\label{para:nexal-negation-tendency}There is a general tendency to use nexal negation wherever it is possible (though we shall later on see another tendency that in many cases counteracts this one); and as the (finite) verb is the linguistic bearer of a nexus, at any rate in all complete sentences, we therefore always find a strong tendency to attract the negative to the verb. We see this in the prefixed \textit{ne} in French as well as in Old English, and also in the suffixed \is{grammaticalization}\is{negation, suffixes}\il{English!n't@\textit{-n't}}\textit{-n't} in Modern English, which will be dealt with in \chapref{ch:11}, and in the suffixed \textit{ikke} in modern Norwegian, as in \textit{Er ikke (erke) det fint?} (`Is not (archaic \textit{not}) that nice?') and \textit{Vil-ikke De komme?} (`Will-not you come?'), where Danish has the older word-order \textit{Er det ikke fint?} (`Is it not nice?') and \textit{Vil De ikke komme?} (`Will you not come?').

\is{auxiliary verbs}
In Modern English the use or non-use of the auxiliary \textit{do} serves in many, but not of course in all, cases to distinguish between nexal and special negation; thus we have special negation in \refp{ex:05-02}.\label{p:nexal-sp}

\ea \label{ex:05-02}
He seems \textit{not certain} of his way\hfill(\href{https://archive.org/details/mrswarrensprofes00shawuoft/page/160/mode/2up?q=%22seems+not+certain%22}{Shaw, \textit{Profession} 160})
\z

Combinations like \refp{ex:05-03} should also be mentioned.

\ea \label{ex:05-03}
He regretted that \textit{more} Englishmen \textit{did not} come here\hfill(news 1917) % PE: ??? It's not necessary to have both this little paragraph and the footnote "Jespersen notes in his Addenda that....").
\z\il{English!not@\textit{not}|)}
\is{quantifiers!negatived|)}

\is{partitive constructions|(}
In French we have a distinction which is somewhat analogous to that between nexal and special negation, namely that between \is{grammaticalization}\textit{pas de} (`not any') and \textit{pas du} (`not some'): \textit{je ne bois pas de vin} (`I do not drink (any) wine'); \textit{ceci n'est pas du vin, c'est du vinaigre} (`this is not (some) wine, it's (some) vinegar'), see the full treatment in \citet[\href{https://www.nb.no/items/b6e32092abfe229e8854840d79878e30?page=113}{p.~87ff}]{storm1911storre}. Good examples are found in \refp{ex:05-04}, but see % PE: "but" (by itself) is straight from OJ; "see" is my (feeble) addition, just to make it read less oddly. 
\refp{ex:05-05}. 

\ea \label{ex:05-04}
 \gll ce n' était \textit{plus de} la poésie, ce n' était \textit{pas} \textit{de la} prose, ce était de la poésie, mise en prose\\
 this not was {more of} the poetry this not was not {of the} prose this was of the poetry put into prose\\
 \glt `it was no longer poetry, nor was it prose, but poetry put into prose'\\\hfill(\href{https://www.gutenberg.org/cache/epub/62021/pg62021-images.html}{Rolland, \textit{Buisson} 192})
\z

\ea \label{ex:05-05}
 \gll Il n' y a \textit{pas} \textit{d'} amour, \textit{pas} \textit{de} haine, \textit{pas} \textit{d'} amis, \textit{pas} \textit{d'} ennemis, \textit{pas} \textit{de} foi, \textit{pas} \textit{de} passion, \textit{pas} \textit{de} bien, \textit{pas} \textit{de} mal.\\
 it not there has not of love not of hate not of friends not of enemies not of faith not of passion not of good not of evil\\
 \glt `There is no love, no hate, no friends, no enemies, no faith, no passion, no good, no evil.'\hfill(\href{https://www.gutenberg.org/cache/epub/62021/pg62021-images.html}{ibid 197})
\z

With the partitive force of \textit{pas} with \textit{de} should be compared the well-known use of the genitive for the object in Russian negative sentences and with \textit{nět} (`there is not'), etc., also the use of the partitive case for the subject of a negative sentence in Finnish. % ??? PE: I've moved this down from its previous position, in front of the last two quotations. 
\is{partitive constructions|)}

In the case of a contrast we have a special negation; hence the separation of \textit{is} (with comparatively strong stress) and \il{English!not@\textit{not}|(}\textit{not} in \refp{ex:05-06}. \il{English!do@\textit{do}|(}\textit{Do} is not used in such sentences as \refp{ex:05-07}.\is{auxiliary verbs|(}

\ea \label{ex:05-06}
the remedy is, not to remand him into his dungeon, but to accustom him to the rays of the sun\hfill(\href{https://archive.org/details/essaysonmiltona05macagoog/page/n128/mode/2up?view=theater&q=remand}{Macaulay, \textit{Milton} 1.41})
\z

\ea \label{ex:05-07}
\ea
I came not to send peace, but a sword\hfill(\href{https://www.kingjamesbibleonline.org/1611_Matthew-10-34/}{AV \textit{Matthew} 10.34})
\ex
my ruin came not from too great individualism of life, but from too little\hfill(\href{https://archive.org/details/deprofundiswilde00wildiala/page/104/mode/2up?q=%22great+individualism+of+life%22&view=theater}{Wilde, \textit{Profundis} 135})
\ex
We meet not in drawing-rooms, but in the hunting-field\\\hfill(\href{https://www.gutenberg.org/files/30432/30432-h/30432-h.htm}{Dickinson, \textit{Symposium} 14})
\z
\z

Even in such contrasted statements, however, the negative is very often attracted to the verb, which then takes \textit{do}, the latter part being then equivalent to \textit{but we meet in the hunting-field} \refp{ex:05-08}.\is{auxiliary verbs|)}

\ea \label{ex:05-08}
we do not meet in the drawing-room, but in the hunting-field 
\z

\ea \label{ex:05-09}
\ea
I do not complain of your words, but of the tone in which they were uttered
\ex
I do not admire her face, but [I do admire] her voice
\ex
He didn't say that it was a shame, but that it was a pity
\ex
I did not come to curse thee, Guinevere\\\hfill(\href{https://en.wikisource.org/wiki/Idylls_of_the_King/Guinevere}{Tennyson, \textit{Guinevere}}; contrast not expressed)
\z
\z\il{English!do@\textit{do}|)}\il{English!not@\textit{not}|)}

In such cases, the Old English verb naturally had no \il{English!Old English!ne@\textit{ne}|(}\textit{ne} before it, see e.g. \refp{ex:05-15}. The exception in \refp{ex:05-18} may be accounted for by the Latin word-order \textit{non veni pacem mittere, sed gladium}. But in Ælfric we have \refp{ex:05-19}, where the meaning is `it happened not-unprovidentially', as shown by the indicative \textit{wæs} and by the necessity of the repetition \textit{hit getimode}. Cf. also the Middle English version: % PE: As is, this reads strangely. Can we simply skip "edited by Paues 56:"?
%Brett: sure % PE Done.
\refp{ex:05-20}.

\ea \label{ex:05-15}
\ea\il{English!Old English!nalles@\textit{nalles}}
\gll wen ic þæt ge for wlenco nalles for wræcsiðum ac for higeþrymmum Hroðgar \textit{sohton}\\
 expect I that you for pride {not at all} for {misery} but for {high spirits} Hrothgar {sought}\\
\glt `I expect that you did not seek Hrothgar out of dire straits but out of boldness and strength of heart'\hfill(\href{http://ebeowulf.uky.edu/ebeo4.0/CD/main.html}{\textit{Beowulf} 338})
\ex
\gll ðæt he nalæs to idelnesse, swa sume oðre, ac to gewinne, in ðæt mynster \textit{eode}\\
 that he not to idleness as some others but to labour into that monastery went\\
\glt `that he did not go into the monastery for idleness, as some others, but to labour'\hfill(\href{https://archive.org/details/oldenglishversio02bede/page/264/mode/2up?q=%22+pcet+he+nales+to+idelnesse%2C+swa+aume+o%27Sre%22&view=theater}{\textit{Bede} 4.3}) % Linked-to version uses þ rather than ð, and also uses macrons (or what look like macrons). What to do?
%Brett: The old link was https://archive.org/details/anglosaxonreader00wyatuoft/page/52/mode/2up?q=%22to+idelnesse%22&view=theater. The new one still has þ, but at least it doesn't have the diacritics. % Peter: Thank you. This is good enough, I think.
\ex
\gll ðe ic lufode na for galnesse ac for wisdome\\
 whom I loved not for wantonness but for wisdom\\
\glt `whom I did not love for lust but for wisdom'\hfill(\href{https://archive.org/details/anglosaxonversi00thorgoog/page/n34/mode/2up?q=%22for+galnesse%22&view=theater}{\textit{Apollonius} 255})
\z
\z

\ea \label{ex:05-18}
\gll ne com ic sybbe to sendanne, ac swurd\\
 not come I peace to send but sword\\
\glt `I did not come to send peace, but a sword'\hfill(\href{https://books.google.co.jp/books?id=twINAAAAIAAJ&newbks=1&newbks_redir=0&printsec=frontcover&pg=PA48&dq=%22ic+sybbe+to+sendanne%22&hl=en&redir_esc=y#v=onepage&q=%22ic%20sybbe%20to%20sendanne%22&f=false}{WG \textit{Matthew} 10.34})
\z

\ea \label{ex:05-19}
\gll Ne getimode þam apostole Thome unforsceawodlice, þæt he ungleafful wæs Cristes æristes, ac hit getimode þurh Godes forsceawunge\\ % PE: Restored "Cristes æristes", whose meaning I'd GUESS is "Christ's resurrection" (see https://bosworthtoller.com/790 )
%Brett: looks good
 not happened {to the} apostle Thomas {by chance} that he unbelieving was Christ's resurrection but it happened through God's providence\\
\glt `It did not happen to the apostle Thomas by chance that he doubted Christ's resurrection, but it occurred through God's providence'\\\hfill(\href{https://archive.org/details/homiliesanglosa00thorgoog/page/234/mode/2up?q=%22apostole+Thome+unforsceawodlice%22&view=theater}{Ælfric, \textit{Homilies} 1.234}) % Peter: Can we change "was unbelieving in" to "doubted"?
%Brett: done
\z\il{English!Old English!ne@\textit{ne}|)}

\ea \label{ex:05-20}\il{English!Middle English!ne@\textit{ne}}\il{English!Middle English!noȝt@\textit{noȝt}}
\gll For Crist ne sende noȝt me {for to} baptyze, bote {for-to} preche þe gospel\\
 for Christ not sent not me to baptize but to preach the gospel\\
\glt `For Christ sent me not to baptize, but to preach the Gospel'\\\hfill(\href{https://archive.org/details/fourteenthcentur00pauerich/fourteenthcentur00pauerich/page/56/mode/2up?q=%22me+for+to+baptyze%22&view=theater}{MEV \textit{1 Corinthians} 1.17})
\z

\il{English!not@\textit{not}|(}Other examples of constructions in which \textit{not} is referred to the verb instead of some other word \refp{ex:05-21}.%Brett: OJ has single quotes

\ea \label{ex:05-21}
\ea
I did not step into the well-known boat Without a cordial greeting (`I stepped {\dots} not without')\\\hfill(\href{https://en.wikisource.org/wiki/The_Prelude_(Wordsworth)/Book_IV}{Wordsworth, \textit{Prelude} 4.16})
\ex
Don't pay only the arrears, pay all you can. (`Pay, not only')\hfill(\href{https://archive.org/details/quisantanovel00hopegoog/page/n144/mode/2up?q=%22pay+only+the+arrears%22&view=theater}{Hope, \textit{Quisanté} 132})
\ex
it doesn't only concern myself\hfill(\href{https://archive.org/details/freelands00galsrich/page/288/mode/2up?q=%22concern+myself%22&view=theater}{Galsworthy, \textit{Freelands} 332})
\z
\z

Note also \refp{ex:05-24}, where the sentence \textit{we aren't here} in itself is a contradiction in terms. (Differently in \refp{ex:05-25}, where \textit{not} belongs more closely to what follows.)

\ea 
\ea \label{ex:05-24}
We aren't here to talk nonsense, but to act
\ex \label{ex:05-25}
We are here, not to retire till compelled to do so 
\z
\z

\is{auxiliary verbs}
When the negation is attracted to the verb (in the form \textit{n't}), it occasions a cleaving of \il{English!never@\textit{never}}\textit{never}, \textit{ever} thus standing by itself. In writing the verbal form is sometimes separated in an unnatural way: (\ref{ex:05-26}, representing the spoken \textit{Can't she ever~{\dots}}); and thus we get seemingly \il{English!not ever@\textit{not ever}|(}\textit{not ever} (`never', \ref{ex:05-27}, different from the old \textit{not ever} as in \href{https://archive.org/details/utopiasirthomas00robigoog/page/n355/mode/2up?q=%22not+euer%22&view=theater}{More, \textit{Utopia} 244}, which meant `not always'). Compare the rare \il{English!not any@\textit{not any}}\textit{not any} as in \refp{ex:05-36}.

\ea \label{ex:05-26}
\textit{Can she not ever} write herself?\hfill(\href{https://archive.org/details/alfredlordtenny05tenngoog/page/n250/mode/2up?q=%22can+she+not+ever%22&view=theater}{Hallam, letter})
\z

\ea \label{ex:05-27}
\ea
You shan't \textit{touch} those hostels ever again. Ever.\hfill(\href{https://archive.org/details/wifeofsirisaacha00well/page/422/mode/2up?view=theater&q=%22touch+those+hostels%22}{Wells, \textit{Wife} 422})
\ex
I suppose you don't ever write to him?\hfill(\href{https://archive.org/details/dollydialogues00hope_0/page/62/mode/2up?view=theater&q=%22ever+write+to+him%22}{Hope, \textit{Dialogues} 40})
\ex
I can't ever see that man again.\hfill(\href{https://archive.org/details/marriageofwillia0000mrsh_i0u5/page/284/mode/2up?q=%22can%27t+ever+see+that+man+again%22&view=theater}{Ward, \textit{Marriage} 242})
\ex
Don't you ever go down beneath the surface of things?\\\hfill(\href{https://archive.org/details/septimus00unkngoog/page/n253/mode/2up?q=%22you+ever+go+down%22&view=theater}{Locke, \textit{Septimus} 26})
\ex
so don't you ever be troubled about that\hfill(\href{https://archive.org/details/prodigalson00caingoog/page/n222/mode/2up?view=theater&q=%22don%27t+you+ever+be+troubled%22}{Caine, \textit{Prodigal} 219})
\z
\z

\ea \label{ex:05-27a}
\ea
let not euer The soule of Nero enter this firme bosome\\\hfill(\href{https://internetshakespeare.uvic.ca/doc/Ham_F1/scene/3.2/index.html#tln-2260}{Shakespeare, \textit{Hml} 3.2.411})
\ex
A light around my steps which would not ever fade\\\hfill(\href{https://archive.org/details/completepoeticalshel/page/78/mode/2up?view=theater&q=%22light+around+my+steps%22}{Shelley, \textit{Revolt} 4.34})
\ex
Do you not ever go?\hfill(\href{https://archive.org/details/dukeschildrennov00troluoft/page/172/mode/2up?q=%22do+you+not+ever%22&view=theater}{Trollope, \textit{Children} 2.40})
\ex
you shall not---not ever\hfill(\href{https://archive.org/details/widowershousesun00shaw/page/42/mode/2up?q=%22you+shall+not%22&view=theater}{Shaw, \textit{Houses} 40})
\z
\z\il{English!not ever@\textit{not ever}|)}

\ea \label{ex:05-36}\il{English!no@\textit{no}}
``Had any gentleman heard of a dauphin killed by small-pox?'' No; \il{English!not any@\textit{not any}}\textit{not any} gentleman \textit{had} heard of such a case.\hfill(\href{https://archive.org/details/miscellaneousess00dequuoft/page/78/mode/2up?q=%22heard+of+a+dauphin%22&view=theater}{Quincey, \textit{Murder}}) % PE: Quincey italicizes "had"; OJ (among his Addenda) italicizes both "not any" and "had"
\z

\is{scope of negation!ambiguity of}
\is{scope of negation!cause or reason and|(}
A special case of frequent occurrence is the rejection of something as the cause of or reason for something real, expressed in a negative form: `he is happy, not on account of his riches, but on account of his good health' expressed in this form \textit{he is not (isn't) happy on account of his riches, but on account of his good health}. It will easily be seen that \textit{I didn't go because I was afraid} is ambiguous (`I went and was not afraid', or, `I did not go, and was afraid'), and sentences like this are generally avoided by good stylists. In \refp{ex:05-37}, the clause gives the reason for the speaker not wanting to be patronized. Similarly \refp{ex:05-38}.\is{ambiguity!related to reason}

\ea \label{ex:05-37}
Don't patronize \textit{me}, Ma, because I can take care of myself \\\hfill(\href{https://archive.org/details/ourmutualfriendc0000char/page/258/mode/2up?q=%22patronise+me%22&view=theater}{Dickens, \textit{Friend} 348}) % In this edition at least, "patronise" has an "s".
\ex \label{ex:05-38}
I have not drunk deep of life because I have been unathirst\\\hfill(\href{https://archive.org/details/bwb_P8-BLX-259/page/150/mode/2up?view=theater&q=unathirst}{Locke, \textit{Morals} 151})
\z

\is{scope of negation!intonation and}
In the spoken language a distinction will usually be made between the two kinds of sentences by the tone, which rises on \textit{call} in \textit{I didn't call because I wanted to see her} (but for some other reason), while it falls on \textit{call} in \textit{I didn't call because I wanted to avoid her} (the reason for not calling). In (\ref{ex:05-39} \& \ref{ex:05-40}), the clause indicates the reason for the prohibition.

\ea \label{ex:05-39}
You mustn't come whining back to me, because I won't have you\\\hfill(\href{https://archive.org/details/runningwater00masouoft/page/104/mode/2up?q=%22whining+back+to+me%22&view=theater}{Mason, \textit{Water} 95})
\ex\label{ex:05-40}
We have not gagged our Press because we disliked our freedom, but because to this extent the Prussian has triumphed\hfill(\textit{Parable}) % From Addenda.
\z\il{English!not@\textit{not}|)}
\is{adverbs!negative|)}
\is{scope of negation!cause or reason and|)}

In other languages we have corresponding phenomena. G. Brandes's \refp{ex:05-41} is ambiguous; and when Ernst Møller writes \refp{ex:05-42}, I suppose that most readers will misunderstand it as if \textit{opløses} were to be taken in a positive sense; it would have been made clearer by a transposition: \refp{ex:05-43}. Also, \refp{ex:05-44}.

\ea \label{ex:05-41}
\gll [Napoleon] handlede ikke saadan, fordi han trængte til sine generaler\\
[Napoleon] acted not thus because he needed to his generals\\
\glt `Napoleon did not act like this, as he needed his generals'\\
`Napoleon did not act thus because he needed his generals [but for some other reason]'\hfill(\href{https://archive.org/details/napoleonoggarib00brangoog/page/n33/mode/2up?q=%22handlede+ikke+saadan%22&view=theater}{\textit{Napoleon} 52}) % PE: OJ cites Brandes writing in Tilskueren. This is a journal; in March '24, the issue was not online at kb.dk. In the linked-to book, Brandes writes not "Napoleon" but "Han" (i.e. "He").
%% SG: The second alternative is not a possible translation of the Danish sentence. The two possible readings are: (a) 'Napoleon did not act like this, as he needed his generals'; and (b) 'Napoleon did not act thus because he needed his generals [but for some other reason]'

\ex \label{ex:05-42}
\gll Men retningens magt opløses, som alt fremhævet, ikke fordi dens argumenter og læresætninger eftergås og optrævles; dens magt vil blive stående\\
 but movement.\DEF.\POSS{} power {is dissolved} as already emphasized not because its arguments and doctrines {are examined} and {are unraveled} its power will remain standing\\
\glt 
`But, as already emphasized, the power of the movement will not be dissolved because its arguments and doctrines are examined and unraveled; its power will continue to stand'\\\hfill(\textit{Inderstyre} 249, in speaking of ``Christian Science'') % ??S When checked in March '24 and again in September '24, not online at kb.dk or archive.org. Is there a copy somewhere that we can link to?
\ex \label{ex:05-43}
\gll Men som alt fremhævet opløses retningens magt ikke~{\dots}\\
 but as already emphasized {is dissolved} movement.\DEF.\POSS{} power not\\
\glt `But, as already emphasized, the power of the movement is not dissolved~{\dots}'
\ex \label{ex:05-44}
\gll Jeg elsker ikke mit sprog, fordi det er eller har været herligt og {skjønt {\dots}} jeg elsker det, fordi det er mine fædres og mit folks sprog\\
 I love not my language because it is or has been glorious and beautiful I love it because it is my fathers.\POSS{} and my people.\POSS{} language\\
\glt `I do not love my language because it is or has been glorious and beautiful {\dots} I love it because it is the language of my ancestors and my people'
\hfill(\href{https://books.google.com/books?id=XAJJAQAAMAAJ&pg=RA3-PA90&lpg=RA3-PA90&dq=madvig+%22Jeg+elsker+ikke+mit+sprog%22&source=bl&ots=rZlO8Lq4of&sig=ACfU3U0BK7Im87JxQP6To6D2Ey22Jl04eg&hl=en&sa=X&ved=2ahUKEwjCn5X5gZOFAxV4nK8BHWEgCx8Q6AF6BAgIEAM#v=onepage&q=madvig%20%22Jeg%20elsker%20ikke%20mit%20sprog%22&f=false}{Madvig, \textit{Kjönnet} 90})% From Addenda. % OJ attributes this to "Madvig Program 1857. 90." Perhaps a mistake? Anyway, it's on p. 90 of Om Kjönnet i Sprogene.
\z

\phantomsection \label{p:48}\il{English!not@\textit{not}|(}
\is{auxiliary verbs|(}
\is{adverbs!negative|(}
Not unfrequently \textit{not} is attracted to the verb in such a way that an adverb, which belongs to the whole proposition, is more or less awkwardly placed between words which should not properly be separated, as in \refp{ex:05-45}. The tendency to draw the auxiliary and \textit{not} together has, on the other hand, been resisted in % ??? PE: Originally: "on the other hand, been resisted in the following passages"; immediately followed by six quotations. (Yes, "passages" is a bit of a stretch.) But in view of the new arrangement, "the following passages" seems superfluous at best. If we keep it, then let's move "The tendency to draw ... the following word" downwards, so that it immediately precedes "You will of course not meet him..." 
\refp{ex:05-49}. In most of these, \textit{not} evidently is a special negative, belonging to the following word.

\ea \label{ex:05-45}
\ea
you \textit{are not probably} aware {\dots}\\(`probably you are not aware', or: `you are probably not aware')\\\hfill(\href{https://archive.org/details/dukeschildrennov00troluoft/page/38/mode/2up?q=%22are+not+probably+aware%22&view=theater}{Trollope, \textit{Children} 1.76})
\ex
were he at that moment Home Secretary and in the cabinet, he \textit{would not probably} be reading it\hfill(\href{https://archive.org/details/marriageofwillia0000mrsh_i0u5/page/268/mode/2up?q=%22were+he+at+that+moment+Home+Secretary%22&view=theater}{Ward, \textit{Marriage} 228})
\ex
Edward Manisty, however, \textit{was not apparently} consoled by her remarks\hfill(\href{https://archive.org/details/cu31924013567130/page/2/mode/2up?q=%22not+apparently+consoled%22&view=theater}{Ward, \textit{Eleanor} 2}) % OJ has "M."; this is restored to "Manisty"
\ex
This is a strong expression. Yet it \textit{is not perhaps} exaggerated.\\\hfill(news 1917)
\z
\z
\is{adverbs!negative|)}

\ea \label{ex:05-49}
\ea
You \textit{will of course not} meet him until he has spoken to me\\\hfill(\href{https://archive.org/details/widowershousesun00shaw/page/28/mode/2up?q=%22you+will+of+course%22&view=theater}{Shaw, \textit{Houses} 27})
\ex
he \textit{is clearly not} a prosperous man\hfill(\href{https://archive.org/details/doctorsdilemmatr00shawuoft/page/20/mode/2up?q=%22prosperous+man%22&view=theater}{Shaw, \textit{Dilemma} 21})
\ex
they \textit{had clearly not} been unfavourable to him\hfill(\href{https://archive.org/details/strangeadventure00blac/page/268/mode/2up?view=theater&q=%22unfavorable+to+him%22}{Black, \textit{Phaeton} 280}) % The edition linked to actually spells it "unfavorable", but presumably British editions spell it "unfavourable".
\ex
a fashionable music-master, whose blood \textit{was certainly not} Christian\\\hfill(\href{https://archive.org/details/marriageofwillia0000mrsh_i0u5/page/154/mode/2up?q=%22whose+blood+was%22&view=theater}{Ward, \textit{Marriage} 133}) % "fashionable" restored
\ex
It'\textit{s simply not} fair to other people \phantom{x} (`is simply unfair')\\\hfill(\href{https://archive.org/details/silverboxcomedyi00gals/page/54/mode/2up?q=%22simply+not+fair+to+other+people%22&view=theater}{Galsworthy, \textit{Box} 55})
\ex
the smashing up of the Burnet family {\dots} \textit{was disagreeably not} in the picture of these suppositions\hfill(\href{https://archive.org/details/wifeofsirisaacha00well/page/120/mode/2up?view=theater&q=%22smashing+up+of%22}{Wells, \textit{Wife} 120}) % OJ had removed "by the International Stores"
\z
\z

\is{redundancy of expression|(}
It has sometimes been said that the combination \textit{he cannot possibly come} is illogical; \textit{not} is here taken to the verb \textit{can}, while in Danish and German the negative is referred to \textit{possibly}: \refp{ex:05-55}. There is nothing illogical in either expression, but only redundance: % PE: OJ wrote "redundance", not "redundancy". Uncommon a century ago and rare now, "redundance" seems a legit word.
the notion of possibility is expressed twice, in the verb and in the adverb, and it is immaterial to which of these the negative notion is attached.

\ea \label{ex:05-55}
\ea
\gll han kan umuligt komme [Danish]\\
 he can impossibly come\\
\glt `he can't possibly come'
\ex
\gll er kann unmöglich kommen [German]\\
 he can impossibly come\\
\glt `he can't possibly come'
\z
\z
 % ??? PE Pretty sure that this pair should either (A) be glossed, etc, and labelled as Danish and German respectively, or (B) moved back into the paragraph. The latter is the less cumbersome.
%% SG: Shall I add glosses?
\is{redundancy of expression|)}

\is{adverbs!negative|(}
When \textit{not} is taken with some special word, it becomes possible to use the adverb \textit{still}, which is only found in positive sentences: \refp{ex:05-56} is different from \textit{the officers were not yet friendly} (\textit{not yet} nexal negative) insofar as the latter presupposes a change having occurred after that time, which the former does not. Cf. also \refp{ex:05-57}.

\ea \label{ex:05-56}
The officers were still not friendly\hfill(news 1917) 
\z

\ea \label{ex:05-57}
\ea
Although I wrote to him a fortnight ago, I have still not heard from him\hfill(letter 1899)
\ex \il{English!no@\textit{no}}my head is still in no good order \phantom{x}(`is still bad', slightly different from `is not yet well')\hfill(\href{https://archive.org/details/journaltostellae00swifuoft/page/502/mode/2up?q=%22no+good+order%22&view=theater}{J. Swift, \textit{Journal} 503})
\z
\z

\textit{Yet not} is rare: \refp{ex:05-59}.

\ea \label{ex:05-59}
Pekuah was yet not satisfied {\dots}\hfill(\href{https://archive.org/details/historyrasselas01johngoog/page/n117/mode/2up?q=%22was+yet+not+satisfied%22&view=theater}{Johnson, \textit{Rasselas} 112}) % "P." expanded to "Pekuah"
\z
\is{auxiliary verbs|)}

\il{English!not a/one@\textit{not a}/\textit{one}|(}
\label{para:kind-of-stronger-no}\textit{Not a} or \textit{not one} before a substantive (very often \textit{word}) is a kind of stronger \il{English!no@\textit{no}}\textit{no}; at any rate, the two words may be treated as belonging closely together, i.e. as an instance of special negative, the verb consequently taking no auxiliary \textit{do}; cf. \citet[\href{https://archive.org/details/jespersen-1954-a-modern-english-grammar-on-historical-principles-part-ii-syntax-first-volume/page/426/mode/2up?q=\%22not+one+word\%22&view=theater}{16.73}]{jespersenMEG2}, where many examples are given; see further \refp{ex:05-60}.

\ea \label{ex:05-60}
\ea
Say not a word of it\hfill(\href{https://archive.org/details/mansfieldpark00aust_1/page/364/mode/2up?q=%22say+not+a+word%22&view=theater}{Austen, \textit{Mansfield} 395})
\ex
The Face seemed to smile, but answered not a word\\\hfill(\href{https://archive.org/details/snowimageothertw0000hawt/page/44/mode/2up?q=%22face+seemed+to+smile%22&view=theater}{Hawthorne, \textit{Image} 46}) % "The" and "Face" both capitalized, as in the book
\ex
he mentioned not a word\hfill(\href{https://archive.org/details/returnofthenativ00harduoft/page/270/mode/2up?q=%22he+mentioned+not%22&view=theater}{Hardy, \textit{Return} 356})
\ex
she said not a word about their interview\hfill(\href{https://archive.org/details/grandbabylonhote00bennuoft/page/82/mode/2up?view=theater&q=%22said+not+a+word%22}{Bennett, \textit{Babylon} 66}) % OJ has "that", but the book linked to has "their"
\ex
he lost not an hour in breaking entirely with the murderer\\\hfill(\href{https://archive.org/details/returnofsherlock0000acon/page/152/mode/2up?view=theater&q=%22lost+not+an+hour%22}{Doyle, \textit{Return} 5.230}) % OJ omits "entirely"
\z
\z\il{English!not a/one@\textit{not a}/\textit{one}|)}

\il{English!not the least\textit{/}the slightest@\textit{not the least}/\textit{the slightest}|(}
In a similar way \textit{not} is attracted to \textit{the least}, \textit{the slightest}, and in recent usage \textit{at all}, as shown by the absence of the auxiliary \textit{do} \refp{ex:05-65}. Cf. \refp{ex:05-71}.

\ea \label{ex:05-65}
\ea
his Majesty took not the least notice of us\hfill(\href{https://archive.org/details/bim_eighteenth-century_the-works-of-j-s-dd-_swift-jonathan_1735_3/page/200/mode/2up?view=theater&q=%22Maje%C5%BFty+took+not+the+lea%C5%BFt+Notice%22}{J. Swift, \textit{Travels} 200}) % "Notice" capitalized in the original
\ex
my resignation of the wardenship need offer not the slightest bar to its occupation by another person\hfill(\href{https://archive.org/details/warden0000anth_w6p5/page/228/mode/2up?q=%22need+offer+not%22&view=theater}{Trollope, \textit{Warden} 243})
\ex
He rested but two hours and slept not at all\hfill(\href{https://archive.org/details/themother00phil/page/350/mode/2up?q=%22but+two+hours%22&view=theater}{Phillpotts, \textit{Mother} 350}) % Starts the sentence
\ex
an urgency that helped him not at all\hfill(\href{https://archive.org/details/loveandmrlewisha00welluoft/page/64/mode/2up?view=theater&q=%22urgency+that+helped%22}{Wells, \textit{Love} 65})

\ex
this explanation enlightened the Commandant not at all\\\hfill(\href{https://archive.org/details/majorvigoureux00quil/page/58/mode/2up?q=%22enlightened%22&view=theater}{Quiller-Couch, \textit{Major} 59})
\il{English!not the least\textit{/}the slightest@\textit{not the least}/\textit{the slightest}|)}
\ex
they talked not at all for a long time\hfill(\href{https://archive.org/details/freelands00galsrich/page/184/mode/2up?q=%22not+at+all+for+a+long+time%22&view=theater}{Galsworthy, \textit{Freelands} 209})
\z
\z

\ea \label{ex:05-71}
he {\dots} cared not the snap of one of his thin, yellow fingers\hfill(\href{https://archive.org/details/freelands00galsrich/page/362/mode/2up?q=%22cared+not+the+snap%22&view=theater}{ibid 415}) % OJ deleted " had done it hundreds of times before and"
\z

\is{auxiliary verbs|(}
Where we have a verb connected with an infinitive, it is often of great importance whether the negation refers to the nexus (main verb) or to the infinitive. In the earlier stages of the language, this was not always clear: \textit{he tried not to look that way} was ambiguous; now the introduction of \il{English!do@\textit{do}|(}\textit{do} as the auxiliary of a negative nexus has rendered a differentiation possible: \textit{he did not try to look that way}; \textit{he tried not to look that way}; and the (not yet recognized) placing of \textit{not} after \textit{to} serves to make the latter sentence even more unambiguous: \textit{he tried to not look that way}. The distinction is clear in \refp{ex:05-72}.

\ea \label{ex:05-72}
She \textit{did not wish} to reflect; she strongly \textit{wished not to} reflect\\\hfill(\href{https://archive.org/details/cu31924013586940/page/470/mode/2up?q=%22did+not+wish+to+reflect%22&view=theater}{Bennett, \textit{Wives} 2.187})
\z\il{English!do@\textit{do}|)}\is{auxiliary verbs|)}

Other examples with \textit{not} belonging to an infinitive: \refp{ex:05-73}--\refp{ex:05-73c}.

\ea \label{ex:05-73}
\ea
\textit{Try not to do} it again\hfill(\href{https://archive.org/details/personalhistory05dickgoog/page/n53/mode/2up?q=%22try+not+to+do+it%22&view=theater}{Dickens, \textit{David} 112})
\ex
\textit{Try not to associate} bodily defects with mental\hfill(\href{https://archive.org/details/personalhistory05dickgoog/page/n191/mode/2up?q=%22bodily%22&view=theater}{ibid 432})
\ex
the more he \textit{endeavoured not to think}, the more he thought\\\hfill(\href{https://archive.org/details/christmascarol0000char_h5c8/page/40/mode/2up?q=%22endeavored%22&view=theater}{Dickens, \textit{Carol} 20}) % Edition linked to has "endeavored"; I (PE) presume that this was exclusively for the US market.
\ex
the fool {\dots} who \textit{resolved not to go} into the water till he had learnt to swim\hfill(\href{https://archive.org/details/essaysonmiltona05macagoog/page/n128/mode/2up?view=theater&q=resolved}{Macaulay, \textit{Milton} 1.41}) % "in the old story" has been cut
\ex
Tommy \textit{deserved not to be} hated\hfill(\href{https://archive.org/details/intrusionspeggy02hopegoog/page/n46/mode/2up?q=%22deserved+not+to+be%22&view=theater}{Hope, \textit{Intrusions} 38})
\ex
if one were to live always among those bright colours, one would \textit{get not to see} them\hfill(\href{https://archive.org/details/strangeadventure00blac/page/58/mode/2up?view=theater&q=%22among+those+bright+colors%22}{Black, \textit{Phaeton} 61}) % The edition linked to actually spells it "colors", but presumably British editions spell it "colours".
\ex
I soon \textit{got not to care}\hfill(\href{https://archive.org/details/in.ernet.dli.2015.260707/page/n79/mode/2up?q=%22soon+got+not+to+care%22&view=theater}{Galsworthy, \textit{Justice} 91})
\ex
I may \textit{come not to feel} such unbearable shame as I do now\\\hfill(\href{https://archive.org/details/lovescrosscurren00swinuoft/page/142/mode/2up?q=%22may+come+not+to+feel%22&view=theater}{Swinburne, \textit{Cross-currents} 158})
\ex
I knew he'd \textit{come not to care} about the book-selling\\\hfill(\href{https://archive.org/details/historydavidgri02wardgoog/page/458/mode/2up?q=%22come+not+to+care%22&view=theater}{Ward, \textit{David} 3.132})
\z
\z

\ea \label{ex:05-73a}
\ea
I beseech you before you go, not perhaps to return, once more to let me press the hand\hfill(\href{https://archive.org/details/vanityfairanove03thacgoog/page/n129/mode/2up?q=%22beseech+you+before%22&view=theater}{Thackeray, \textit{Vanity} 200})
\ex
the Prime Minister himself was personally too much absorbed in the zeal of his cause not sometimes to run counter to the feelings {\dots} of men less earnest\hfill(\href{https://books.google.co.jp/books?id=9PjnkXA5jOkC&pg=PP5&dq=%22justin+mccarthy%22+%22our+own+times%22&hl=en&newbks=1&newbks_redir=0&sa=X&ved=2ahUKEwiE2YnrmaCFAxVsoa8BHcesDZwQ6AF6BAgKEAI#v=onepage&q=%22much%20absorbed%22&f=false}{McCarthy, \textit{History} 2.521}) % Replaced an omission with dots, restored "personally" and "less earnest", etc. According to OJ, MacCarthy [sic] writes: "the Prime-minister was too much absorbed in the zeal of his cause not sometimes to run counter to the feelings of men".
\z
\z

\ea \label{ex:05-73b}
I wished to not treat you to more tears\hfill(\href{https://digital.library.upenn.edu/women/carlyle/jwclam/lam301.html#24}{J. Carlyle, \textit{Letters} 3.24})
\z

\ea \label{ex:05-73c}
``I might not have gone,'' I mused. ``I might easily not have gone.''\\\hfill(\href{https://archive.org/details/dollydialogues00hope_0/page/152/mode/2up?view=theater&q=%22might+not+have+gone%22}{Hope, \textit{Dialogues} 94}; cf. p.~\pageref{p:48} \hyperref[p:48]{above} % PE: Let's add page number
%Brett: go for it  % PE: Done
and p.~\pageref{must-and-may}ff (\chapref{ch:8}) below) % PE: Restoring "I mused". OJ merely says chapter VIII, not the page number(s) within this.
\z

When \textit{do} cannot be used, it is not always easy to see whether \textit{not} belongs to the main verb or the infinitive, as in \refp{ex:05-86},\footnote{Jespersen's Addenda include the example \textit{Sylvia was determined \textsc{not to be} disappointed} (\href{https://archive.org/details/runningwater00masouoft/page/114/mode/2up?view=theater}{Mason, \textit{Water} 104}). \eds} % Want to insert \href{https://archive.org/details/runningwater00masouoft/page/114/mode/2up?view=theater&q=%22Sylvia+was+determined%22}{Mason, \textit{Water} 104} within that pair of parentheses, but doing so triggers errors
%Brett: fixed by removing text search/highlighting. % Peter: Strange! I must try to remember this. 
where, however, the next line shows that what is meant is `it was not my purpose to have seen you here', and not `it was my purpose not to have {\dots}'. This paraphrase further serves to show that in some cases word-order may remove any doubt as to the belonging of the negative, thus very often with a predicative; cf. also such frequent cases as \refp{ex:05-87}. And in the spoken language the use of \textit{wasn't} [wɔznt] in one case, and unstressed \textit{was} [wəz] followed by a strongly stressed \textit{not} in the other, will at once make the meaning clear of such sentences as the one first quoted here.\is{stress!effect on negation of}

\ea \label{ex:05-86}
My purpose was not to haue seene you heere\\\hfill(\href{https://internetshakespeare.uvic.ca/doc/MV_F1/scene/3.2/index.html#tln-1575}{Shakespeare, \textit{Merch} 3.2.230}) % "seene" with three E s
\z

\ea \label{ex:05-87}
He was beginning not to despise the day of small things\\\hfill(\href{https://archive.org/details/septimus00unkngoog/page/n221/mode/2up?q=%22was+beginning+not%22&view=theater}{Locke, \textit{Septimus} 232})
\z

\textit{Don't let us} is the idiomatic expression, where logically it would be preferable to say \textit{let us} with \textit{not} to the infinitive (an injunction not to {\dots}): \refp{ex:05-88}.

\ea \label{ex:05-88}
Do not let us, however, be too prodigal of our pity upon Pegasus\\\hfill(\href{https://archive.org/details/dli.ministry.14127/page/343/mode/2up?q=%22be+too+prodigal%22&view=theater}{Thackeray, \textit{Pendennis} 2.213}) % "upon Pegasus" restored
\z

In the old construction without \textit{do} we see the same attraction of \textit{not} to \textit{let}: (\ref{ex:05-89}, though the last two quotations show \textit{not} placed with the infinitive).

\ea \label{ex:05-89}
\ea
let not vs rent it\hfill(\href{https://www.kingjamesbibleonline.org/1611_John-19-24/}{AV \textit{John} 19.24})
\ex
let not my behaviour seem rude\hfill(\href{https://archive.org/details/bim_eighteenth-century_epicne-or-the-silent-_jonson-ben_1776/page/n29/mode/2up?q=%22behaviour%22&view=theater}{Jonson, \textit{Epicœne} 3.183})
\ex
let not the prospect of worldly lucre carry us beyond your judgment\\\hfill(\href{https://archive.org/details/in.ernet.dli.2015.219151/page/n195/mode/2up?q=%22worldly+lucre%22&view=theater}{Congreve, \textit{Love} 255}) % Original is bristling with capitals
\ex

And let not those Londoners whose eyes have been accustomed to {\dots} suppose that {\dots}\hfill(\href{https://archive.org/details/lifeadventuresofdickrich/page/466/mode/2up?q=%22those+Londoners+whose%22&view=theater}{Dickens, \textit{Nicholas} 443}) % "Londoners" restored
\ex
let not another dare suspect it\hfill(\href{https://archive.org/details/evanharringtonno00mererich/page/194/mode/2up?q=%22dare+suspect%22&view=theater}{Meredith, \textit{Harrington} 219})
\ex
let us not add guilt to our misfortunes\hfill(\href{https://quod.lib.umich.edu/cgi/t/text/pageviewer-idx?c=ecco;cc=ecco;idno=004771299.0001.000;node=004771299.0001.000:6.1;seq=189;page=root;view=text}{Goldsmith, \textit{Good-natur'd} 5})
\ex
let us not imagine evils which we do not feel\hfill(\href{https://archive.org/details/historyrasselas01johngoog/page/n105/mode/2up?q=%22let+us+not+imagine%22&view=theater}{Johnson, \textit{Rasselas} 101}) % "Evil" corrected to "evils"
\z
\z

While now \textit{not} is always in natural language placed before the infinitive it belongs to, there is a poetic or archaic way of placing it after the infinitive, as in \refp{ex:05-96}.
\pagebreak

\ea \label{ex:05-96}
\ea
one object which you might pass by, Might see and \textit{notice not}\\\hfill(\href{https://archive.org/details/poemsofwilliamwo00wor/page/50/mode/2up?q=%22one+object+which+you+might+pass+by%22&view=theater}{Wordsworth, \textit{Michael}})
\ex
a continuance of enduring thought. Which then I can \textit{resist not}\\\hfill(\href{https://archive.org/details/manfreddramaticp06byro/page/n11/mode/2up?q=continuance&view=theater}{Byron, \textit{Manfred} 1.1})
\ex
God bless you, my son, {\dots} and when he smiles on you, may the frown of man \textit{affect you not}!\hfill(\href{https://archive.org/details/christianstory00cainrich/page/70/mode/2up?q=%22frown%22&view=theater}{Caine, \textit{Christian} 69}) % Little changes to accord with the printed book
\z
\z\il{English!not@\textit{not}|)}

\is{scope of negation!resolving ambiguity of|(}
In other languages, difficulties like those mentioned in English are obviated in different ways. Thus in Greek \textit{mē} is used to negative an infinitive, while \textit{ou} is used with a finite verb. In Danish, a certain number of combinations like \textit{jeg beklager ikke at kunne hjælpe Dem} (`I am sorry I cannot help you') may be ambiguous, though less so in the spoken than in the printed form; but in some instances the colloquial use of a preposition shows where \textit{ikke} belongs; instead of the literary \textit{prøv ikke at se derhen} (`try not to look over there') it is usual to say either \textit{prøv ikke på at se derhen} (`don't try to look over there') or \textit{prøv på ikke at se derhen}. There is another colloquial way out of the difficulty, by means of the verbal phrase \textit{lade være} or rather \textit{la vær} (literally `leave be'): % ??? PE: What's in single quotes is expected to be a more or less idiomatic translation; but a problem for me is that I don't understand "leave be" in these examples. "Leave her be", OK; ?"Leave her be to look", pretty strange; *"Leave be to look", ungrammatical (for me).
%% SG: It's a conventionalized expression: lad være at se derhen, lit. 'leave be to look over there' (= 'do not look over there')
\textit{prøv at} (\textit{å}) \textit{la vær at} (\textit{å}) \textit{se derhen} (`Avoid looking over there'). Thus also \textit{du skal lade vær at se derhen} (`You must refrain from looking over there'), different from \textit{du skal ikke se derhen} (`You must not look over there').

In Latin, the place of \textit{non} before the main verb or before the infinitive will generally suffice to make the meaning clear. Similarly in French: \textit{il ne tâche pas de regarder} (`he doesn't try to look'); \textit{il tâche de ne pas regarder} (`he tries not to look'); \textit{il ne peut pas entendre} (`he cannot hear'); \textit{il peut ne pas entendre} (`he can not hear (as he chooses)') --- whence the possibility of saying \textit{non potest non amare} (`he cannot not love'); \textit{il ne peut pas ne pas aimer} corresponding to Danish \textit{han kan ikke lade være at elske}, English \textit{he cannot but love}, \textit{cannot help loving} (\textit{cannot choose but love}). Cf. \chapref{ch:8} below. % PE: OJ writes "Cf. below ch. VIII.". We could insert a comma before "chapter 8" to show that it's in apposition, but I've switched the order instead. Feel free to revert.
\is{scope of negation!resolving ambiguity of|)}

\label{para:not-to-sing}In this connexion, I must mention an interesting phenomenon frequent in Russian; I take my examples from Holger \citet[12]{pedersen1916russisk}; \textit{a pět' už ne stal} `but sing now he not began' % Peter: The bunch of English translations in this paragraph are OJ's own, and putting them into ( ) would complicate matters.
% PE: I have now (Sept '04) rather changed my mind about this. I'm putting ‘must/should’ in ( ).
which is explained as standing for the logical `not-to-sing he began', i.e. `he ceased to sing'; \textit{ne vélěno étogo dělat'} `order is not given to do this' instead of the logical `order is given not to do this', i.e. `it is prohibited to do this'. Similarly with \textit{dolžen} (`must/should'). But how comes it that the negative \textit{ne} is in such expressions attached to the wrong word? There is another way of viewing these sentences, if we take the negative to mean not the contradictory, but the contrary term: \textit{ne stal} `did the opposite of beginning', i.e. `ceased'; \textit{ne velmo} `the opposite of order, i.e. prohibition, is given'. And in \citet[\href{https://archive.org/details/vergleichendesl00vondgoog/page/399/mode/2up?q=\%22mitunter+wird+der+begriff\%22&view=theater}{400}]{vondrak1908vergleichende}, I find:

\begin{quote}
mitunter wird der Begriff des Verbs nicht durch \textit{ne} aufgehoben, sondern in sein Gegenteil verwandelt: [altkirchenslawisch] nenaviděti ‘hassen’ ([böhmisch] náviděti ‘lieben’), [serbisch] nèstati ‘verschwinden’
 % The abbreviations are Vondrák’s; should we have somebody proficient in German expand them? Also, OJ writes "ins gegenteil" but V writes "in sein Gegenteil" (of course we're following OJ's non-capitalizing)
%Brett: The abbreviations in the text refer to different Slavic languages: aksl. - Altkirchenslavisch (Old Church Slavonic) b. - Bulgarisch (Bulgarian) s. - Serbisch (Serbian) % Peter: Ah yes, obvious now that I think of it. I've inserted "Bulgarian". But hang on -- isn't it a little odd to bring up Bulgarian? If "b." were also the abbreviation of a word meaning roughly "derived from", it would be a lot less surprising. 
%Brett: The abbreviation "b." does indeed appear to stand for "bei" or possibly "beziehungsweise" rather than "Bulgarisch" (Bulgarian). This is evident from the context: "aksl. nenaviditi 'hassen' (b. náviditi 'lieben')" Here, it's showing the contrast between the negated form (nenaviditi - to hate) and its positive counterpart (náviditi - to love).
% Peter: Accordingly, I've deleted "Bulgarian" from "Bulgarian náviděti". (However, I haven't provided any alternative rendering of "b.".)

`sometimes the concept of the verb is not negatived by \textit{ne}, but transformed into its opposite: in Old Church Slavonic, \textit{nenaviděti} means ``to hate'' (Czech náviděti ``to love''),, and in Serbian, \textit{nèstati} means ``to disappear''' % PE: I have (unenthusiastically) changed "negated" to "negatived".
\end{quote}

This closely resembles a Greek idiom, see:

\begin{quote}
Einzelne Begriffe werden besonders durch \textit{ou} aufgehoben, ja zuweilen ins Gegenteil verwandelt, wie \textit{oú phēmi} nego, verneine {\dots} \textit{ouk axiô} verlange dass nicht, \textit{ouk eô} veto, verwehre, widerrate (auch erlaube nicht)

`Individual concepts are especially negatived by \textit{ou}, and sometimes even transformed into the opposite, as in \textit{oú phēmi} (\textit{nego} in Latin, `I deny') {\dots} \textit{ouk axiô} (`I do not deem worthy'), \textit{ouk eô} (\textit{veto} in Latin, `I forbid, I refuse, I advise against'---also `I do not permit')'\hfill\citep[\href{https://archive.org/details/griechischesprac00kr/page/296/mode/2up?view=theater}{§67 1.a.2}]{kruger1875griechische}
% PE: How about: `Individual concepts are especially negated by \textit{ou}, and sometimes even transformed into the opposite, as in \textit{oú phēmi} (\textit{nego} (Latin), I deny) {\dots} \textit{ouk axiô} (I demand not), \textit{ouk eô} (\textit{veto} (Latin), I forbid, I refuse, I advise against (also I do not permit))'  How we deal with the Latin here should probably affect how we deal with it in the next quotation.
%Brett: Agreed. Done
%PE: I have (unenthusiastically) changed "negated" to "negatived".
\end{quote}
\is{litotes}
\begin{quote}
Eine ähnliche Litotes liegt vor, wenn \textit{phēmí} die Negation an sich zieht, die logisch richtiger beim abhängigen Infinitive stehen würde: \textit{oú phēmi toûto kalôs ékhein} nego hoc bene se habere % restored "Eine ähnliche"

\emergencystretch=3em
`A similar litotes occurs when \textit{phēmí} attracts the negation to itself, which logically would more correctly stand with the dependent infinitive: \textit{oú phēmi toûto kalôs ékhein} (\textit{nego hoc bene se habere} in Latin, `I deny that this is in a good state')'\phantom{.}\hfill\citep[\href{https://archive.org/details/p2ausfhrlichegra02khuoft/page/180/mode/2up?q=litotes&view=theater}{180}]{kuhner1904ausfuhrliche}
\end{quote}

\noindent This is explained as change into the contrary:

\begin{quote}
\textit{\emph{ouk eô} prohibeo {\dots} \emph{ou stérgō} odi {\dots} \emph{ou sumbouleúō} dissuadeo}

\emergencystretch=3em
`I prohibit, I hate, I dissuade'\hfill \citep[\href{https://archive.org/details/p2ausfhrlichegra02khuoft/page/182/mode/2up?q=litotes&view=theater}{182}]{kuhner1904ausfuhrliche}
\end{quote}

\is{position of negative|(}
\is{raising of negation|(}
As an ``accusative with an infinitive'' can be considered as a kind of dependent clause, the mention of Latin \textit{nego Gaium venisse} (`I say that Gaius has not come') naturally leads us to the strong tendency found in many languages to attract to the main verb a negative which should logically belong to the dependent nexus. In many cases, \textit{I don't think he has come} and similar sentences really mean `I think he has not come'; though \textit{I hope} (\textit{expect})\textit{ he won't come} is more usual than the less logical \textit{I do not hope} (\textit{expect})\textit{ he will come}, which is usual in Danish and German, and also, according to \citet[\href{https://archive.org/details/englishaswespeak00joycuoft/page/19/mode/2up?view=theater&q=\%22So+in+our+modern+speech\%22}{20}]{joyce1910english} among the Irish, who will say, e.g. \textit{It is not my wish that you should go to America at all}, by which is meant the positive assertion: `It is my wish that you should not go', --- as well as \textit{I didn't pretend to understand what he said} for `I pretended not to understand'.
\is{attraction!of negative to verb}

A few Scandinavian examples may be given of this tendency to insert the negative in the main sentence: \refp{ex:05-99}.

\ea \label{ex:05-99}
\ea
\gll saa vil jeg \textit{aldrig} ønske, at du maa blive gift\\
 then will I never wish that you may become married\\
\glt `{\dots} then I wish that you will never get married'\\\hfill(\href{https://tekster.kb.dk/text/adl-texts-hostrup01-shoot-workid54989#idm140699400034704}{Hostrup, \textit{Gjenboerne} 3.6}) % &&
%Brett: What's this? % Peter: I've no idea. Just a slip of the fingers, perhaps.
\ex
\gll Jeg tror \textit{ikke}, at mange har læst Brand og at færre har forstaaet den\\
 I believe not that many have read Brand and that fewer have understood it\\
\glt `I believe that not many have read Brand and that fewer have understood it'\hfill(Schandorff, news 1897; note here the continuation,\\\hfill which shows that what is meant is: \textit{tror at ikke mange {\dots}}) % It's not even clear to me (PE) which Schandorff this is.
\is{quantifiers!negatived}

\ex
\gll Men det lot {'o [= hun]} ikke, som 'o hørte\\
 but that pretended she not {as if} she heard\\
\glt `But she acted as if she hadn't heard that'\hfill(\href{https://www.nb.no/items/URN:NBN:no-nb_digibok_2008072810004?page=31&searchText=%22Men%20det%20lot%22}{Bjørnson, \textit{Guds} 21})
\ex
\gll Han trodde \textit{icke} presterna voro annat än examinerade studenter \textit{och} att deras besvärjelseord bara var mytologi\\
 he believed not priests.\DEF{} were other than examined students and that their incantations just were mythology\\
\glt `He believed that priests were nothing but graduated students, and that their incantations were mere mythology'\\\hfill(\href{https://litteraturbanken.se/f%C3%B6rfattare/StrindbergA/titlar/GiftasII1886/sida/134/faksimil}{Strindberg, \textit{Giftas} 2.134}; note also here the positive continuation) 
\z
\z

Compare from French \refp{ex:05-103}.

\ea \label{ex:05-103}
\gll il ne faut pas que tu meures\\
 it not need not that you die\\
\glt `you must not die'\hfill(\href{https://archive.org/details/vermischtebeitr04toblgoog/page/n197/mode/2up?view=theater&q=%22faut+pas+que+tu+meures%22}{Tobler, \textit{Beiträge} 1.164})
\z
\is{raising of negation|)}

\il{English!not@\textit{not}|(}
In English, we must note the distinction between \textit{I don't suppose} (\textit{I am not afraid}), where the main nexus is negatived, and \textit{I suppose not} (\textit{I am afraid not}) where the nexus is positive, but the object (a whole sentence understood) is negative; how old is this use of \textit{not} for a whole sentence? Examples: \refp{ex:05-104}.

\ea \label{ex:05-104}
\ea I'm afraid not\hfill(\href{https://archive.org/details/in.ernet.dli.2015.219151/page/n171/mode/2up?q=%22afraid+not%22&view=theater}{Congreve, \textit{Love} 121})
\ex `` {\dots} whether it ever came to my knowledge until this moment?'' --- ``I believe not directly'' {\dots}. --- ``Why, you know not''\hfill(\href{https://archive.org/details/personalhistory05dickgoog/page/n47/mode/2up?q=%22whether+it+ever%22&view=theater}{Dickens, \textit{David} 93}) % "until this moment" restored; CD's "Why" instead of OJ's "Well". (Incidentally, the original three paragraphs have other grammatical interest too.)
\ex ``I am afraid you can't learn it, my poor fellow.'' --- ``I am afraid not''\\\hfill(\href{https://archive.org/details/lifeadventuresofdickrich/page/330/mode/2up?q=%22you+can%27t+learn+it%22&view=theater}{Dickens, \textit{Nicholas} 311}) % "my poor fellow" restored
\ex ``can you bear the thought of that?'' --- {\dots} ``I should imagine not, indeed!''\hfill(\href{https://archive.org/details/lifeadventuresofdickrich/page/616/mode/2up?q=%22bear+the+thought%22&view=theater}{ibid 590}) % OJ mangles three paragraphs into one; I (PE) have replaced the second with dots and separated the other two.
\ex ``I should not mind'' {\dots}. --- ``I daresay not, because you have nothing particular to say.'' {\dots} --- ``But I have something particular to say.'' --- ``I hope not.'' --- ``Why should you hope not?''\hfill(\href{https://archive.org/details/dukeschildrennov00troluoft/page/194/mode/2up?q=%22I+should+not+mind%22&view=theater}{Trollope, \textit{Children} 2.81}) % repunctuated
\ex ``I'll tell the boys and they'll draw you like a badger.'' --- ``Please not, old man.''\hfill(\href{https://archive.org/details/lightthatfailed0000rudy_q6g8/page/228/mode/2up?q=%22please+not%2C+old+man%22&view=theater}{Kipling, \textit{Light} 217}) % "and they'll draw you like a badger" restored
\ex I believe I asked him to hold his tongue about them---he says not.\\\hfill(\href{https://babel.hathitrust.org/cgi/pt?id=hvd.hnpei8&seq=9&q1=hold+his+tongue}{Conway, \textit{Called} 1}) % "about them" restored. (This is a single sentence, spoken by one person.) 
\z
\z
\is{position of negative|)}

\is{adverbs!negative}
\is{proforms|(}
\is{subordinator, negative|(}
\label{not_that}Inversely, we have a negative adverb standing for a whole main sentence, \il{English!not that@\textit{not that}|(}\textit{not that} meaning `I do not say that' or `the reason is not that' as in \refp{ex:05-111}. We shall see in \chapref{ch:12} the use of \il{English!not but@\textit{not but}}\textit{not but} (\textit{that}) and \textit{not but what} in the same sense.

\ea \label{ex:05-111}
\ea Not that I lou'd Cæsar lesse, but that I lou'd Rome more\\\hfill(\href{https://internetshakespeare.uvic.ca/doc/JC_F1/scene/3.2/index.html#tln-1550}{Shakespeare, \textit{Cæs} 3.2.22}) % The uvic.ca page has not "Cæsar" but instead "Caesar". Is it simplifying, or is OJ indulging another minor eccentricity?
%Brett: the first folio has "Cæsar". We can link to it, but it's hard to search and hard to read https://archive.org/details/mrvvilliamshakes00shak/page/120/mode/2up?q=%22that+Friend+demand%2C+why+Brutus+role+againft+Ca%7E+far%2C+this+is+my+anfwer+%3A%22&view=theater % Peter: ⟨æ⟩ is a pain, but as it's spattered across the Danish that's so prominent in this book, and as changing those examples back to ⟨aa⟩ would be most anachronistic, let's tolerate it in English as well.
\ex Not that the heart can be good without knowledge\\\hfill(\href{https://archive.org/details/bunyanspilgrims00moffgoog/page/108/mode/2up?q=%22heart+can+be+good%22&view=theater}{Bunyan, \textit{Progress} 113})
\ex Not that I agree with everything I have said in this essay\\\hfill(\href{https://archive.org/details/intentions01wild/page/256/mode/2up?q=%22agree%22&view=theater}{Wilde, \textit{Intentions} 212})
\ex Not that he had forgotten them\hfill(\href{https://archive.org/details/wonderfulyear00lockuoft/page/326/mode/2up?q=%22not+that+he+had+forgotten%22&view=theater}{Locke, \textit{Year} 309})
\z
\z\il{English!not that@\textit{not that}|)}\il{English!not@\textit{not}|)}

In other languages correspondingly: \refp{ex:05-115}.

\ea \label{ex:05-115}
\ea
\gll Ikke at han havde (or: skulde ha) glemt dem\\
 not that he had {} should have forgotten them\\
\glt `Not that he had forgotten them'
\ex
\gll nicht dass er sie vergessen hätte\\
 not that he them forgotten would have\\
\glt `not that he would have forgotten them'
\ex
\gll Non pas qu'il parlât à personne\\
 not not {that he} spoke to anyone\\
\glt `Not that he spoke to anyone'\hfill(\href{https://www.gutenberg.org/cache/epub/61876/pg61876-images.html}{Rolland, \textit{Foire} 306})
\z
\z
\is{subordinator, negative|)}

\il{English!not@\textit{not}|(}
When we say (``He'll come back'') \textit{Not he!} % PE: I think what OJ is saying is "When we say (in response to 'He'll come back') 'Not he!', it is not really...." And so he deliberately puts "He'll come back" in quotation marks, which I've restored.
it is not really \textit{he} that is negatived, but the nexus, although the predicative part of it is unexpressed; the exclamation is a complete equivalent of \textit{He won't!} (with stress on \textit{won't}): \refp{ex:05-118}.

\ea \label{ex:05-118}
\ea Who, I rob? I a theefe? Not I.\hfill(\href{https://internetshakespeare.uvic.ca/doc/1H4_F1/scene/1.2/index.html#tln-240}{Shakespeare, \textit{H4A} 1.2.153}) 
\ex Please not, old man.\hfill(\href{https://archive.org/details/lightthatfailed0000rudy_q6g8/page/228/mode/2up?q=%22please+not%2C+old+man%22&view=theater}{Kipling, \textit{Light} 217})
\ex Were I a Steam-engine, wouldst thou take the trouble to tell lies of me? Not thou!\hfill(\href{https://archive.org/details/sartorresartus02unkngoog/page/222/mode/2up?view=theater&q=%22were+I+a+steam-engine%22}{T. Carlyle, \textit{Sartor} 169})
\ex Meg don't know what he likes. Not she!\hfill(\href{https://archive.org/details/chimes00dick/page/58/mode/2up?q=%22meg+don%27t+know+what%22&view=theater}{Dickens, \textit{Chimes} 30})
\ex They wouldn't have touched \textit{us} {\dots} Not they!\\\hfill(\href{https://archive.org/details/freelands00galsrich/page/222/mode/2up?q=%22not+they%21%22&view=theater}{Galsworthy, \textit{Freelands} 255}) % OJ has "wouldn't touch", but JG writes "wouldn't have touched"
\ex ``it'll perhaps rain cats and dogs to-morrow {\dots}'' {\dots} --- ``Not \textit{it}''\\\hfill(\href{https://archive.org/details/silasmarnerbygeo00elio/page/30/mode/2up?q=%22cats+and+dogs%22&view=theater}{Eliot, \textit{Silas} 44})
\ex ``Do you think it will last long?'' --- ``Not it!''\hfill(\href{https://archive.org/details/cu31924013586940/page/238/mode/2up?q=%22will+last+long%22&view=theater}{Bennett, \textit{Wives} 1.263})
\ex ``Bit late now, isn't it?'' --- ``Not it.''\hfill(\href{https://archive.org/details/cardstoryofadven00bennuoft/page/244/mode/2up?view=theater&q=%22bit+late+now%22}{Bennett, \textit{Card} 244}) % OJ merely says "Bennett Cd. 244": no example, no indication of what "Cd." is, etc.
\ex All sorts of things accumulate, sir.{\dots} Not \textit{you}, of course, in particular.\\\hfill(\href{https://archive.org/details/in.ernet.dli.2015.475865/page/n129/mode/2up?q=%22Not+you%2C+of+course%2C+in+particular.%22&view=theater}{Wells, \textit{Stories} 49}) % (i) Let's try to find a better copy than this one. (ii) OJ doesn't specify the example, only the page number.
%Brett: Is this a better copy? % Peter: It's not as bad a copy. I've resuscitated the emphasis on "you".
\z
\z
\is{proforms|)}

The following examples \refp{ex:05-127} show the accusative used as a modern (vulgar or half-vulgar) ``disjointed'' nominative.

\ea \label{ex:05-127}
\ea
We sha'n't hang up on any misunderstanding. Not us.\\\hfill(\href{https://archive.org/details/annveronicamoder0000hgwe/page/360/mode/2up?q=%22any+misunderstanding%22&view=theater}{Wells, \textit{Veronica} 338}) % OJ has "shan't hang upon", but that isn't what the Harper & Brothers 1909 edition says.
\ex ``you were all in the same room together, were not you?''\\\il{English!no@\textit{no}}``No, indeed, not us.''\hfill(\href{https://archive.org/details/sensesensibility00austrich/page/242/mode/2up?q=%22all+in+the+same+room%22&view=theater}{Austen, \textit{Sense} 269}) % PE: Adjusted JA's punctuation to accord with the edition linked to.
\z
\z\il{English!not@\textit{not}|)}

In Old English we have the corresponding \il{English!Old English!nic@\textit{nic}}\textit{nic} \refp{ex:05-129}. \textit{Nic} is spelt \textit{nîc} and \textit{nyc} (\href{https://archive.org/details/holygospelsinan01skeagoog/page/n75/mode/2up?view=theater&q=nyc}{\textit{John}, ed. Skeat, 1.21}), spelt \textit{nicc} and \textit{nicht} (\href{https://archive.org/details/holygospelsinan01skeagoog/page/n365/mode/2up?view=theater&q=nicht}{ibid 18.17}). This (with the positive counterpart \textit{I}, which is probably the origin of \textit{ay} (`yes'), and \textit{ye we} in \refp{ex:05-130}) closely resembles the French \textit{naje} `not I' (in the third person \textit{nenil}) and the positive \textit{oje} `hoc ego' (in the third person \textit{oïl, oui}), see Tobler (\citeyear[\href{https://www.jstor.org/stable/pdf/40845615.pdf}{423}]{tobler1877franzosische}; \citeyear[\href{https://archive.org/details/vermischtebeitr04toblgoog/page/n17/mode/2up?view=theater}{1}]{tobler1886vermischte}); \citet[\href{https://www.jstor.org/stable/45042774?seq=3}{465}]{paris1878periodiques}.

\ea \label{ex:05-129}
\gll Wilt þu fon sumne hwæl? --- Nic\\
 want you catch some whale {} not I\\
\glt `Do you want to catch a whale? --- No.'\hfill(\href{https://archive.org/details/anglosaxonoldeng01wriguoft/page/51/mode/2up?view=theater&q=%22wilt%22}{Wright \& Wülcker 1.94})
\z\il{English!Old English!nic@\textit{nic}|)}
%% SG: More idiomatically, the reply just translates to "no". It's a dedicated first-person negation
%BR: done

\ea \label{ex:05-130}
wille ye doo this {\dots} --- {\dots} ye we, lorde\hfill(\href{https://archive.org/details/TheHistoryOfReynardTheFoxArber/page/n87/mode/2up?q=%22wille+ye+doo+this%22&view=theater}{Caxton, \textit{Reynard} 58})
\z
\is{nexal negation!and special negation|)}
\is{scope of negation|)}
\is{special negation!and nexal negation|)}

% CHAPTER 6 ENCLITICS
\chapter{Enclitics}\label{ch:enclitics}
This chapter covers the enclitic\index[sub]{enclitic} suffixes of Southern Yauyos Quechua. In \SYQ, as in other Quechuan languages, enclitics attach to both nouns and verbs as well as to adverbs and negators. Enclitics always follow all inflectional suffixes, verbal and nominal; and, with the exception of restrictive \phono{-lla}, all follow all case suffixes, as well. \SYQ{} counts sixteen enclitics. \phono{-Yá} (emphatic) indicates emphasis. Consistently translated in Spanish by \textit{pues}.\footnote{An anonymous reviewer points out that \it{pues} is used in Andean Spanish “to negotiate common ground, shared knowledge. As such, it is possible that \phono{-ya} is also an interactional or stance marker,” a way a participant in a conversation may negotiate what other participants know or should know.} \phono{-chu} (interrogation, negation, disjunction) indicates absolute and disjunctive questions, negation, and disjunction. \phono{-lla} (restrictive) generally indicates exclusivity or limitation in number; it is generally translated as ‘just’ or ‘only’. \phono{-lla} may express an affective or familiar attitude. \phono{-ña} (discontinuitive) indicates transition, change of state or quality. In affirmative statements, it is generally translated as ‘already’; in negative statements, as ‘no more’ or ‘no longer’; in questions, as ‘yet’. \phono{-pis} (inclusion) indicates the inclusion of an item or event into a series of similar items or events; it is generally translated as ‘too’ or ‘also’ or, when negated, ‘neither’. \phono{-puni} (certainty, precision); it is generally translated ‘necessarily’, ‘definitely’, ‘precisely’. This last is attested only in the \QII{} dialects, where it is infrequently employed. \phono{-qa} (topic marker) indicates the topic of the clause; it is generally left untranslated.\footnote{\phono{-qa} may nevertheless be indicated in Spanish translations by intonation, gesture, and various circumlocutions of speech, as an anonymous reviewer points out.}\\
\phono{-raq} (continuative) indicates continuity of action, state or quality. Translated ‘still’ or, negated, ‘yet’. \phono{-taq} (sequential) indicates the sequence of events. In this capacity, translated ‘then’ or ‘so’. \phono{-taq} also marks content questions. \phono{-mI} (evidential~--~direct experience) indicates that the speaker has personal-experience evidence for the proposition under the scope of the evidential. Usually left untranslated.\\
\phono{-shI} (evidential~--~reportative/quotative) indicates that the speaker has non-perso\-nal-experience evidence for the proposition under the scope of the evidential. \phono{-shI} appears systematically in stories. Often translated as ‘they say.’ \phono{-trI} (evidential~--~conjectural) indicates that the speaker is making a conjecture to the proposition under the scope of the evidential from a set of propositions for which she has either direct or not-direct evidence. Generally translated in Spanish as \spanish{seguro} ‘for sure’, indicating possibility or probability. \phono{-ari} (assertive force) indicates conviction on the part of the speaker. Translated as ‘certainly’ or ‘of course’.\footnote{An anonymous reviewer writes that in other varieties of Quechuan, “\phono{-ari} is interpersonal. It expresses solidarity, affirming what someone else says, thinks or believes to be true.”} \phono{-ik} and \phono{-iki} (evidential modifiers) indicate increasing evidence strength (and increased assertive force or conjectural certainty, in the case of the direct and conjectural modifiers, \phono{-mI} and \phono{-trI}, respectively). Generally translated in Spanish as \spanish{pues} and \spanish{seguro}, respectively. Examples in Table~\ref{Tab30} are fully glossed in the corresponding sections.

% TABLE 30
% \newcommand{\tabexefour}[4]{\Qyell{\phono{#1}}&#2&\Qyell{\textit{#3}}&#4\\}%
\begin{table}[!ht]
\renewcommand*\arraystretch{1.3}
\small\centering
\caption{Enclitic suffixes, with examples}\label{Tab30}
\begin{tabularx}{\textwidth}{p{6ex}@{~}p{13ex}@{~}L@{~}L}
\lsptoprule
\tabexefour{-Yá}{emphasis}{¡Mana-\pb{yá} rupa-chi-nchik-chu! ¡Ari-yá!}{‘We do \pb{\emph{not}} set on fire!’ \mbox{‘Yes, indeed!’}}
\tabexefour{-chu\tss{1}}{interrogation}{¿Iskwila-man trura-shu-rqa-nki-\pb{chu} mama-yki?}{‘\pb{Did} your mother put you in school?’}
\tabexefour{-chu\tss{2}}{negation}{Chay-tri \pb{mana} suya-wa-rqa-\pb{chu}.}{‘That must be why she would\pb{n’t} have waited for me.’}
\tabexefour{-chu\tss{3}}{disjunction}{¿Qari-\pb{chu} ka-nki warmi-\pb{chu} ka-nki?}{‘Are you a man \pb{or} a woman?’}
\tabexefour{-lla}{restriction}{Uma-\pb{lla}-ña traki-\pb{lla}-ña ka-ya-sa.}{‘There was \pb{only} the head \pb{only} the hand.’}
\tabexefour{-ña}{discontuity}{Chay-shi ni-n kundinadaw-\pb{ña}-m wak-qa ka-ya-n.}{‘That one, they say, is \pb{already} condemned.’}
\tabexefour{-pis}{inclusion}{Tukuy tuta tusha-n qaynintin-ta-\pb{pis}.}{‘They dance all night and the next day, \pb{too}.’}
\tabexefour{-puni}{certainty}{Mana-\pb{puni}-m.}{‘By no means’, ‘Not on your life’}
\tabexefour{-qa}{topic}{Mana yatra-q-ni-n-\pb{qa}.}{‘Those of them who didn’t know’}
\tabexefour{-raq}{continuity}{Kama-n-pi puñu-ku-ya-pti-n-\pb{raq} tari-ru-n.}{‘He found him \pb{still} sleeping in his bed.’}
\tabexefour{-taq}{sequence}{hinaptin-ña-\pb{taq}-shi}{‘\pb{then}’ ‘so’}
\tabexefour{-mI}{evidential-direct}{Yanga-ña-\pb{m} qipi-ku-sa puri-ni.}{‘In vain, I walk around carrying it.’}
\tabexefour{-shI}{evidential-reportative}{Qari-n-ta-\pb{sh} wañu-ra-chi-n.}{‘She killed her husband, \pb{they say}.’}
\tabexefour{-trI}{evidential-conjecture}{Awa-ya-n-\pb{tr-iki} kama-ta.}{‘He \pb{must} be weaving a blanket.’}
\tabexefour{-ari}{assertive force}{Chay-\pb{sh-ari} kanan avansa-ru-nqa.}{‘That one \pb{definitely} will advance now, \pb{they say}.’}
\tabexefour{-ikI}{evidential \-modification}{Kay-na-lla-\pb{m-iki} kay urqu-pa-qa yatra-nchik.}{‘Just like this we live on this mountain.’}
\lspbottomrule
\end{tabularx}
\end{table}

\section{Sequence}
Combinations of individual enclitics\index[sub]{enclitic!sequence} generally occur in the order indicated in the table below. In complementary distribution are: \phono{-raq} with \phono{-ña}; the evidentials with each other as well as with \phono{-qa}; \phono{-ari} with \phono{-ikI;} and \phono{-Yá} with \phono{-ikI}.

\begin{center}
\small
\begin{tabular}{*{9}{c}}
\lsptoprule
	&	&	&				&	&	& \phono{-qa}	&	&				\\
	&	&	&				&	&	& \phono{-mI}	&	&				\\
	&	&	& \phono{-Raq}	&	&	& \phono{-shI}	&	& \phono{-ikI}	\\
\phono{-lla} & \phono{-puni} & \phono{-pis} & \phono{-ña} & \phono{-taq} & \phono{-chu} & \phono{-trI} & \phono{-Yá} & \phono{-aRi}\\
\lspbottomrule
\end{tabular}
\end{center}

\section{Individual enclitics}\label{sec:indenc}
In \SYQ, as in other Quechuan languages, the enclitics can be divided into two classes: (a)~those which position the utterance with regard to others salient in the discourse (restrictive/limitative \phono{-lla}, discontinuative \phono{-ña}, additive \phono{-pis}, topic marking \phono{-qa}, continuative \phono{-Raq}, sequential \phono{-taq}, and interrogative/negative/disjunctive \phono{-chu}); and (b)~those that position the speaker with regard to the utterance (emphatic \phono{-YÁ}, certainty marker \phono{-puni}, and the evidentials \phono{-mi}, \phono{-shi}, and \phono{-tri} along with their modifiers \phono{-ik}, \phono{-iki}, and \phono{-aRi}.). §~\ref{ssec:emphatic}--\ref{ssec:emotive} cover all enclitics except the evidentials and their modifiers, in alphabetical order. The evidentials and their modifiers are the subject of §~\ref{ssec:evidence}.

\subsection{Emphatic \phono{-Yá}}\label{ssec:emphatic}\index[sub]{emphatic}
Realized as \phono{-yá} in all environments~(\ref{Glo6:Ari}--\ref{Glo6:Sirbisatatr}) except following an evidential, in which case both the \phono{I} of the evidential and the \phono{Y} of the emphatic are elided and \phono{Yá} is realized as \phono{á}~(\ref{Glo6:Balikushatr}--\ref{Glo6:Unayqa}).\\ 

% 1
\gloexe{Glo6:Ari}{}{amv}%
{¡Ari\pb{yá}!}%amv que first line
{\morglo{ari-yá}{yes-\lsc{emph}}}%morpheme+gloss
\glotran{Yes \pb{indeed}.}{}%eng+spa trans
{}{}%rec - time

% 2
\gloexe{Glo6:Mana}{}{amv}%
{¡Mana-\pb{yá} rupa-chi-nchik-chu!}%amv que first line
{\morglo{mana-yá}{no-\lsc{emph}}\morglo{rupa-chi-nchik-chu}{burn-\lsc{caus}-\lsc{1pl}-\lsc{neg}}}%morpheme+gloss
\glotran{We do \emph{\pb{not}} set on fire!}{}%eng+spa trans
{}{}%rec - time

% 3
\gloexe{Glo6:Pantyunpa}{}{amv}%
{Pantyunpa\pb{yá}. ¡Ima wasiypitr pampamushaq!}%amv que first line
{\morglo{pantyun-pa-\pb{yá}}{cemetery-\lsc{loc}-\lsc{emph}}\morglo{ima}{what}\morglo{wasi-y-pi-tr}{house-\lsc{1}-\lsc{loc}-\lsc{evc}}\morglo{pampa-mu-shaq}{bury-\lsc{cisl}-\lsc{1.fut}}}%morpheme+gloss
\glotran{In the cemetery\pb{!} I doubt I’m going to bury someone in my house.}{}%eng+spa trans
{}{}%rec - time

% 4
\gloexe{Glo6:Imayna}{}{amv}%
{¿Imayna\pb{yá} piru paykuna yatran warmi u qari?}%amv que first line
{\morglo{imayna-yá}{how-\lsc{emph}}\morglo{piru}{but}\morglo{pay-kuna}{they-\lsc{pl}}\morglo{yatra-n}{know-\lsc{3}}\morglo{warmi}{woman}\morglo{u}{or}\morglo{qari}{man}}%morpheme+gloss
\glotran{How \pb{ever} can they know if it will be a woman or a man?}{}%eng+spa trans
{}{}%rec - time

% 5
\gloexe{Glo6:Sirbisatatr}{}{amv}%
{Sirbisatatr mas mastaqa rantikurun. Sirbisatayá.}%amv que first line
{\morglo{sirbisa-ta-tr}{beer-\lsc{acc}-\lsc{evc}}\morglo{mas}{more}\morglo{mas-ta-qa}{more-\lsc{acc}-\lsc{top}}\morglo{ranti-ku-ru-n}{buy-\lsc{refl}-\lsc{urgt}-\lsc{3}}\morglo{sirbisa-ta-yá}{beer-\lsc{acc}-\lsc{emph}}}%morpheme+gloss
\glotran{\spkr~1: “They must have sold a lot more beer.” \spkr~2: “Beer, \pb{all right}!”}{}%eng+spa trans
{}{}%rec - time

% 6
\gloexe{Glo6:Balikushatr}{}{lt}%
{Balikushatr kara. Payta\pb{má} rikarani.}%lt que first line
{\morglo{baliku-sha-tr}{request.a.service-\lsc{prf}-\lsc{evc}}\morglo{ka-ra}{be-\lsc{pst}}\morglo{pay-ta-m-á}{he-\lsc{acc}-\lsc{evd}-\lsc{emph}}\morglo{rika-ra-ni}{see-\lsc{pst}-\lsc{1}}}%morpheme+gloss
\glotran{He must have been requested. I saw him.}{}%eng+spa trans
{}{}%rec - time

% 7
\gloexe{Glo6:Trabahayta}{}{ch}%
{Trabahayta kanan kumunalta trulala\pb{má}.}%ch que first line
{\morglo{trabaha-y-ta}{work-\lsc{inf}-\lsc{acc}}\morglo{kanan}{now}\morglo{kumunal-ta}{community-\lsc{acc}}\morglo{trula-la-m-á}{put-\lsc{pst}-\lsc{evd}-\lsc{emph}}}%morpheme+gloss
\glotran{Now he’s put the community to work.}{}%eng+spa trans
{}{}%rec - time

% 8
\gloexe{Glo6:Unayqa}{}{sp}%
{Unayqa Awkichanka inkantakura\pb{shá} wak altupa yantaman riptiki.}%sp que first line
{\morglo{unay-qa}{before-\lsc{top}}\morglo{Awkichanka}{Awkichanka}\morglo{inkanta-ku-ra-sh-á}{enchant-\lsc{refl}-\lsc{pst}-\lsc{evr}-\lsc{emph}}\morglo{wak}{\lsc{dem.d}}\morglo{altu-pa}{high-\lsc{loc}}\morglo{yanta-man}{firewood-\lsc{all}}\morglo{ri-pti-ki}{go-\lsc{subds}-\lsc{2}}}%morpheme+gloss
\glotran{In olden times, Awkichanka, too, bewitched, \pb{they say}, up hill if you went for firewood.}{}%eng+spa trans
{}{}%rec - time

\subsection{Interrogation, negation, disjunction \phono{-chu}}\label{ssec:innedi}
\phono{-chu} indicates absolute~(\ref{Glo6:Iskwilaman}) and disjunctive questions~(\ref{Glo6:Qari}), (\ref{Glo6:Don}), negation~(\ref{Glo6:Chaytri}), and disjunction~(\ref{Glo6:Kandilaryapa}).\footnote{An anonymous reviewer points out that in Huaylas Q, negative \phono{-tsu} is distinguished from polar question \phono{-ku}. Huaylas is not unique among Quechuan languages in making this distinction.}\\

% 1
\gloexe{Glo6:Iskwilaman}{}{amv}%
{¿Iskwilaman trurashurqanki\pb{chu} mamayki?}%amv que first line
{\morglo{iskwila-man}{school-\lsc{all}}\morglo{trura-shu-rqa-nki-chu}{put-\lsc{2.obj}-\lsc{pst}-\lsc{2}-\lsc{q}}\morglo{mama-yki}{mother-\lsc{3}}}%morpheme+gloss
\glotran{\pb{Did} your mother put you in school?}{}%eng+spa trans
{}{}%rec - time

% 2
\gloexe{Glo6:Qari}{}{amv}%
{¿Qari\pb{chu} kanki warmi\pb{chu} kanki?}%amv que first line
{\morglo{¿qari-chu}{man-\lsc{q}}\morglo{ka-nki}{be-\lsc{2}}\morglo{warmi-chu}{woman-\lsc{q}}\morglo{ka-nki}{be-\lsc{2}}}%morpheme+gloss
\glotran{Are you a man \pb{or} a woman?}{}%eng+spa trans
{}{}%rec - time

% 3
\gloexe{Glo6:Don}{}{amv}%
{¿Don Juan\pb{chu} icha alman\pb{chu} hamuyan?}%amv que first line
{\morglo{Don}{Don}\morglo{Juan-chu}{Juan-\lsc{q}}\morglo{icha}{or}\morglo{alma-n-chu}{soul-\lsc{3}-\lsc{q}}\morglo{hamu-ya-n}{come-\lsc{prog}-\lsc{3}}}%morpheme+gloss
\glotran{Is it Don Juan, \pb{or} is his spirit coming?}{}%eng+spa trans
{}{}%rec - time

% 4
\gloexe{Glo6:Chaytri}{}{amv}%
{Chaytri \pb{mana} suyawarqa\pb{chu}.}%amv que first line
{\morglo{chay-tri}{\lsc{dem.d}-\lsc{evc}}\morglo{mana}{no}\morglo{suya-wa-rqa-chu}{wait-\lsc{1.obj}-\lsc{pst}-\lsc{neg}}}%morpheme+gloss
\glotran{That’s why she would\pb{n’t} have waited for me.}{}%eng+spa trans
{}{}%rec - time

% 5
\gloexe{Glo6:Kandilaryapa}{}{amv}%
{Kandilaryapa\pb{chu} bintisinkupa\pb{chu}.}%amv que first line
{\morglo{kandilarya-pa-chu}{Candelaria-\lsc{loc}-\lsc{disj}}\morglo{binti-sinku-pa-chu}{twenty-five-\lsc{loc}-\lsc{disj}}}%morpheme+gloss
\glotran{\pb{Either} on Candelaria \pb{or} on the twenty-fifth.}{}%eng+spa trans
{}{}%rec - time

\noindent
Where it functions to indicate interrogation\index[sub]{interrogation!\phono{-chu}} or negation\index[sub]{negation!\phono{-chu}}, \phono{-chu} attaches to the sentence fragment that is the focus of the interrogation or negation~(\ref{Glo6:Chaypachu}).\\

% 6
\gloexe{Glo6:Chaypachu}{}{amv}%
{¿Chaypa\pb{chu} tumarqanki?}%amv que first line
{\morglo{chay-pa-chu}{\lsc{dem.d}-\lsc{loc}-\lsc{q}}\morglo{tuma-rqa-nki}{take-\lsc{pst}-\lsc{2}}}%morpheme+gloss
\glotran{Did you take [pictures] \pb{there}?}{}%eng+spa trans
{}{}%rec - time

\noindent
Where it functions to indicate disjunction\index[sub]{disjunction} --~in either disjunctive questions or disjunctive statements~-- \phono{-chu} generally attaches to each of the disjuncts~(\ref{Glo6:Mario}).\\

% 7
\gloexe{Glo6:Mario}{}{amv}%
{Mario\pb{chu} karqa Julián\pb{chu} karqa.}%amv que first line
{\morglo{Mario-chu}{Mario-\lsc{disj}}\morglo{ka-rqa}{be-\lsc{pst}}\morglo{Julián-chu}{Julián-\lsc{disj}}\morglo{ka-rqa}{be-\lsc{pst}}}%morpheme+gloss
\glotran{It was \pb{either} Mario \pb{or} Julián.}{}%eng+spa trans
{}{}%rec - time

\noindent
Questions that anticipate a negative answer are indicated by \phono{mana-chu}~(\ref{Glo6:Manachu}).\\

% 8
\gloexe{Glo6:Manachu}{}{ch}%
{¿\pb{Manachu} kuska linman?}%ch que first line
{\morglo{mana-chu}{no-\lsc{q}}\morglo{kuska}{together}\morglo{li-n-man}{go-\lsc{3}-\lsc{cond}}}%morpheme+gloss
\glotran{\pb{Couldn’t} they go together?}{}%eng+spa trans
{}{}%rec - time

\noindent
\phono{mana-chu} may also “soften” questions~(\ref{Glo6:Paysanu}).\\

% 9
\gloexe{Glo6:Paysanu}{}{amv}%
{Paysanu, ¿\pb{manachu} vakata rantiyta munanki?}%amv que first line
{\morglo{paysanu}{countryman}\morglo{mana-chu}{no-\lsc{q}}\morglo{vaka-ta}{cow-\lsc{acc}}\morglo{ranti-y-ta}{buy-\lsc{inf}-\lsc{acc}}\morglo{muna-nki}{want-\lsc{2}}}%morpheme+gloss
\glotran{My countryman, \pb{do you not} want to buy a cow?}{}%eng+spa trans
{}{}%rec - time

\noindent
It may also be used, like \phono{aw} ‘yes’, in the formation of tag questions~(\ref{Glo6:Lliw}).\\

% 10
\gloexe{Glo6:Lliw}{}{ach}%
{Lliw lliwtriki wañukushun, puchukashun entonces, ¿\pb{manachu}?}%ach que first line
{\morglo{lliw}{all}\morglo{lliw-tr-iki}{all-\lsc{evc}-\lsc{iki}}\morglo{wañu-ku-shun}{die-\lsc{refl}-\lsc{1pl.fut}}\morglo{puchuka-shun}{finish.off-\lsc{1pl.fut}}\morglo{intunsis}{therefore}\morglo{mana-chu}{no-\lsc{q}}}%morpheme+gloss
\glotran{We’ll all have to die, to finish off then, \pb{isn’t that so}?}{}%eng+spa trans
{}{}%rec - time

\noindent
In negative sentences, \phono{-chu} generally co-occurs with \phono{mana} ‘not’~(\ref{Glo6:mana}); \phono{-chu} is also licensed by additive enclitic \phono{-pis}~(\ref{Glo6:Kaspin}), (\ref{Glo6:Manchakushpa}) and \phono{ni} ‘nor’~(\ref{Glo6:Apuraw}), (\ref{Glo6:wayta}).\\

% 11
\gloexe{Glo6:mana}{}{lt}%
{Aa, \pb{mana}yá kan\pb{chu}. \pb{Mana}yá bula kan\pb{chu}.}%lt que first line
{\morglo{aa}{ah}\morglo{mana-yá}{no-\lsc{emph}}\morglo{ka-n-chu}{be-\lsc{3}-\lsc{neg}}\morglo{mana-yá}{no-\lsc{emph}}\morglo{bula}{ball}\morglo{ka-n-chu}{be-\lsc{3}-\lsc{neg}}}%morpheme+gloss
\glotran{Ah, there are\pb{n’t} any. There are\pb{n’t} any balls.}{}%eng+spa trans
{}{}%rec - time

% 12
\gloexe{Glo6:Kaspin}{}{amv}%
{Kaspin\pb{pis} kan\pb{chu}.}%amv que first line
{\morglo{kaspi-n-pis}{stick-\lsc{3}-\lsc{add}}\morglo{ka-n-chu}{be-\lsc{3}-\lsc{neg}}}%morpheme+gloss
\glotran{She does\pb{n’t} have a stick.}{}%eng+spa trans
{}{}%rec - time

% 13
\gloexe{Glo6:Manchakushpa}{}{ach}%
{Manchakushpa tuta\pb{s} puñu:\pb{chu}.}%ach que first line
{\morglo{mancha-ku-shpa}{scare-\lsc{refl}-\lsc{subis}}\morglo{tuta-s}{night-\lsc{add}}\morglo{puñu-:-chu}{sleep-\lsc{1}-\lsc{neg}}}%morpheme+gloss
\glotran{Being scared, I \pb{don’t} sleep at night.}{}%eng+spa trans
{}{}%rec - time

% 14
\gloexe{Glo6:Apuraw}{}{amv}%
{Apuraw wañururqariki. \pb{Ni} apanña\pb{chu}.}%amv que first line
{\morglo{apuraw}{quick}\morglo{wañu-ru-rqa-r-iki}{die-\lsc{urgt}-\lsc{pst}-\lsc{r}-\lsc{iki}}\morglo{ni}{nor}\morglo{apa-n-ña-chu}{bring-\lsc{3}-\lsc{disc}-\lsc{neg}}}%morpheme+gloss
\glotran{He died quickly. They \pb{didn’t even} bring him [to the hospital].}{}%eng+spa trans
{}{}%rec - time

% 15
\gloexe{Glo6:wayta}{}{amv}%
{\pb{Manam} wayta\pb{chu} \pb{ni} pishqu\pb{chu}.}%amv que first line
{\morglo{mana-m}{no-\lsc{evd}}\morglo{wayta-chu}{flower-\lsc{neg}}\morglo{ni}{nor}\morglo{pishqu-chu}{bird-\lsc{neg}}}%morpheme+gloss
\glotran{\pb{Neither} a flower \pb{nor} a bird.}{}%eng+spa trans
{}{}%rec - time

\noindent
In prohibitions, \phono{-chu} co-occurs with \phono{ama} ‘don’t’~(\ref{Glo6:wawqi}).\\

% 16
\gloexe{Glo6:wawqi}{}{ach}%
{“¡\pb{Ama} wawqi:taqa wañuchiy\pb{chu}!” niptinshi wañurachin paywantapis.}%ach que first line
{\morglo{ama}{\lsc{proh}}\morglo{wawqi-:-ta-qa}{brother-\lsc{1}-\lsc{acc}-\lsc{top}}\morglo{wañu-chi-y-chu}{die-\lsc{caus}-\lsc{imp}-\lsc{neg}}\morglo{ni-pti-n-shi}{say-\lsc{subds}-\lsc{3}-\lsc{evr}}\morglo{wañu-ra-chi-n}{die-\lsc{urgt}-\lsc{caus}-\lsc{3}}\morglo{pay-wan-ta-pis}{he-\lsc{instr}-\lsc{acc}-\lsc{add}}}%morpheme+gloss
\glotran{When he said, “\pb{Don’t} kill my brother!” they killed him with him, too.}{}%eng+spa trans
{}{}%rec - time

\noindent
\phono{-chu} does not appear in subordinate clauses, where negation is indicated with a negative particle alone~(\ref{Glo6:qali}), (\ref{Glo6:qatrachakunanpaq}).\footnote{An anonymous reviewer points out that elsewhere in Quechua, the correlates of negative \phono{-chu} typically can appear in subordinate clauses. There are no naturally-occurring examples of this in the Yauyos corpus.}\\

% 17
\gloexe{Glo6:qali}{}{ch}%
{\pb{Mana} qali kaptinqa ñuqanchikpis taqllakta hapishpa qaluwanchik.}%ch que first line
{\morglo{mana}{no}\morglo{qali}{man}\morglo{ka-pti-n-qa}{be-\lsc{subds}-\lsc{3}-\lsc{top}}\morglo{ñuqanchik-pis}{we-\lsc{add}}\morglo{taqlla-kta}{plow-\lsc{acc}}\morglo{hapi-shpa}{grab-\lsc{subis}}\morglo{qaluwa-nchik}{turn.earth-\lsc{1pl}}}%morpheme+gloss
\glotran{When there are \pb{no} \pb{men}, we grab the plow and turn the earth.}{}%eng+spa trans
{}{}%rec - time

% 18
\gloexe{Glo6:qatrachakunanpaq}{}{amv}%
{\pb{Mana} qatrachakunanpaq mandilchanta watachakun.}%amv que first line
{\morglo{mana}{no}\morglo{qatra-cha-ku-na-n-paq}{dirty-\lsc{fact}-\lsc{refl}-\lsc{nmlz}-\lsc{3}-\lsc{purp}}\morglo{mandil-cha-n-ta}{apron-\lsc{dim}-\lsc{3}-\lsc{acc}}\morglo{wata-cha-ku-n}{tie-\lsc{dim}-\lsc{refl}-\lsc{3}}}%morpheme+gloss
\glotran{She’s tying on an apron \pb{so} she \pb{doesn’t} get dirty.}{}%eng+spa trans
{}{}%rec - time

% 19
\gloexe{Glo6:lluqsiptiyki}{}{amv}%
{Manam lluqsiptiyki(qa *\pb{chu}), waqashaqmi.}%amv que first line
{\morglo{mana-m}{no-\lsc{evd}}\morglo{lluqsi-pti-yki-qa}{go.out-\lsc{subds}-\lsc{2}-\lsc{top}}\morglo{chu}{neg}\morglo{waqa-shaq-mi}{cry-\lsc{1.fut}-\lsc{evd}}}%morpheme+gloss
\glotran{\pb{If} you \pb{don’t} go, I’ll cry.}{}%eng+spa trans
{}{}%rec - time

\noindent
In negative sentences, \phono{-chu} never occurs on the same segment as does an evidential enclitic~(\ref{Glo6:lluqsirqanki}).\\

% 20
\gloexe{Glo6:lluqsirqanki}{}{amv}%
{Mana lluqsirqanki(*mi)\pb{chu}.}% que first line
{\morglo{mana}{no}\morglo{lluqsi-rqa-nki-mi-chu}{go.out-\lsc{pst}-\lsc{2}-\lsc{evd}-\lsc{neg}}}%morpheme+gloss
\glotran{You \pb{didn’t} leave.}{}%eng+spa trans
{}{}%rec - time

\noindent
Interrogative \phono{-chu} does not appear in questions using interrogative pronouns~(\ref{Glo6:hamurqa}).\footnote{\phono{¿*Pi-taq} \phono{hamu-n-chu?} \phono{¿*Pi-taq-chu} \phono{hamu-n?} ‘Who is coming?’}\\

% 21
\gloexe{Glo6:hamurqa}{}{amv}%
{*¿Pi hamurqa\pb{chu}?}% que first line
{\morglo{pi}{who}\morglo{hamu-rqa-chu}{come-\lsc{pst}-\lsc{neg}}}%morpheme+gloss
\glotran{\pb{Who} came?}{}%eng+spa trans
{}{}%rec - time

\subsection{Restrictive, limitative \phono{-lla}}
\phono{-lla} indicates exclusivity or limitation in number\index[sub]{restrictive}: the individual~(\ref{Glo6:Iskwilapam}--\ref{Glo6:Kichwa}) or event/event type~(\ref{Glo6:Fwirti}), (\ref{Glo6:lliwtam}) remains limited to itself and is accompanied by no other.\\

% 1
\gloexe{Glo6:Iskwilapam}{}{sp}%
{Iskwilapam niytu:kunaqa wawa:kunaqa rinmi ñuqa\pb{lla}m ka: analfabitu.}%sp que first line
{\morglo{iskwila-pa-m}{school-\lsc{loc}-\lsc{evd}}\morglo{niytu-:-kuna-qa}{nephew-\lsc{1}-\lsc{pl}-\lsc{top}}\morglo{wawa-:-kuna-qa}{baby-\lsc{1}-\lsc{pl}-\lsc{top}}\morglo{ri-n-mi}{go-\lsc{3}-\lsc{evd}}\morglo{ñuqa-lla-m}{I-\lsc{rstr}-\lsc{evd}}\morglo{ka-:}{be-\lsc{1}}\morglo{analfabitu}{illiterate}}%morpheme+gloss
\glotran{My grandchildren are in school. My children went. I’m the \pb{only} illiterate one.}{}%eng+spa trans
{}{}%rec - time

% 2
\gloexe{Glo6:Runapi}{}{amv}%
{Runapi uma\pb{lla}ña traki\pb{lla}ña kayasa.}%amv que first line
{\morglo{runa-pi}{person-\lsc{gen}}\morglo{uma-lla-ña}{head-\lsc{rstr}-\lsc{disc}}\morglo{traki-lla-ña}{foot-\lsc{rstr}-\lsc{disc}}\morglo{ka-ya-sa}{be-\lsc{prog}-\lsc{npst}}}%morpheme+gloss
\glotran{\pb{Just} the head and the hand remained of the person.}{}%eng+spa trans
{}{}%rec - time

% 3
\gloexe{Glo6:Kichwa}{}{ch}%
{Kichwa\pb{lla}ktam limakuya: kaytrawlaq manam kastillanukta lima:chu.}%ch que first line
{\morglo{kichwa-lla-kta-m}{Quechua-\lsc{rstr}-\lsc{acc}-\lsc{evd}}\morglo{lima-ku-ya-:}{speak-\lsc{refl}-\lsc{prog}-\lsc{1}}\morglo{kay-traw-laq}{\lsc{dem.p}-\lsc{loc}-\lsc{cont}}\morglo{mana-m}{no-\lsc{evd}}\morglo{kastillanu-kta}{Spanish-\lsc{acc}}\morglo{lima-:-chu}{speak-\lsc{1}-\lsc{neg}}}%morpheme+gloss
\glotran{I’m talking \pb{just} Quechua. Here, still, we don’t speak Spanish.}{}%eng+spa trans
{}{}%rec - time

% 4
\gloexe{Glo6:Fwirti}{}{ch}%
{Fwirti kashpa\pb{lla}má linchik pustaman.}%ch que first line
{\morglo{fwirti}{strong}\morglo{ka-shpa-lla-m-á}{be-\lsc{subis}-\lsc{rstr}-\lsc{evd}-\lsc{emph}}\morglo{li-nchik}{go-\lsc{1pl}}\morglo{pusta-man}{clinic-\lsc{all}}}%morpheme+gloss
\glotran{\pb{Only} if it’s bad will we go to the health clinic.}{}%eng+spa trans
{}{}%rec - time

% 5
\gloexe{Glo6:lliwtam}{}{ach}%
{Lliw lliwtam rantishpa\pb{lla}ñam kanan kamatapis chay polarkunatapis.}%ach que first line
{\morglo{lliw}{all}\morglo{lliw-ta-m}{all-\lsc{acc}-\lsc{evd}}\morglo{ranti-shpa-lla-ña-m}{buy-\lsc{subis}-\lsc{rstr}-\lsc{disc}-\lsc{evd}}\morglo{kanan}{now}\morglo{kama-ta-pis}{blanket-\lsc{acc}-\lsc{add}}\morglo{chay}{\lsc{dem.d}}\morglo{polar-kuna-ta-pis}{fleece-\lsc{pl}-\lsc{acc}-\lsc{add}}}%morpheme+gloss
\glotran{Now \pb{just} buying everything -- blankets, [polyester] fleece.}{}%eng+spa trans
{}{}%rec - time

\noindent
\phono{-lla} can generally be translated as ‘just’~(\ref{Glo6:Chayna}), (\ref{Glo6:Sirka}) or ‘only’~(\ref{Glo6:Chay}); it sometimes has an ‘exactly’ interpretation~(\ref{Glo6:Iskinanpi}).\\

% 6
\gloexe{Glo6:Chayna}{}{amv}%
{Chayna\pb{lla}m mikuchin~\dots{} pachachin.}%amv que first line
{\morglo{chayna-\pb{lla}-m}{thus-\lsc{rstr}-\lsc{evd}}\morglo{miku-chi-n}{eat-\lsc{caus}-\lsc{3}}\morglo{pacha-chi-n}{dress-\lsc{caus}-\lsc{3}}}%morpheme+gloss
\glotran{\pb{Just} like that, she feeds him, she clothes him.}{}%eng+spa trans
{}{}%rec - time

% 7
\gloexe{Glo6:Sirka}{}{sp}%
{Sirka\pb{lla}tam riya: manam karutachu.}%sp que first line
{\morglo{sirka-lla-ta-m}{close-\lsc{rstr}-\lsc{acc}-\lsc{evd}}\morglo{ri-ya-:}{go-\lsc{prog}-\lsc{1}}\morglo{mana-m}{no-\lsc{evd}}\morglo{karu-ta-chu}{far-\lsc{acc}-\lsc{neg}}}%morpheme+gloss
\glotran{I \pb{just} go close; I don’t go far.}{}%eng+spa trans
{}{}%rec - time

% 8
\gloexe{Glo6:Chay}{}{amv}%
{Chay\pb{lla}tam yatrani. Masta yatranichu.}%amv que first line
{\morglo{chay-lla-ta-m}{\lsc{dem.d}-\lsc{lim}-\lsc{acc}-\lsc{evd}}\morglo{yatra-ni}{know-\lsc{1}}\morglo{mas-ta}{more-\lsc{acc}}\morglo{yatra-ni-chu}{know-\lsc{1}-\lsc{neg}}}%morpheme+gloss
\glotran{I \pb{only} know that. I don’t know more.}{}%eng+spa trans
{}{}%rec - time

% 9
\gloexe{Glo6:Iskinanpi}{}{lt}%
{Iskinanpi sikya tuna\pb{lla}npi wallpay watrakunraq.}%lt que first line
{\morglo{iskina-n-pi}{corner-\lsc{3}-\lsc{loc}}\morglo{sikya}{aqueduct}\morglo{tuna-lla-n-pi}{corner-\lsc{rstr}-\lsc{3}-\lsc{loc}}\morglo{wallpa-y}{chicken-\lsc{1}}\morglo{watra-ku-n-raq}{give.birth-\lsc{refl}-\lsc{3}-\lsc{cont}}}%morpheme+gloss
\glotran{My hen lays eggs in the corner, \pb{right} in the corner of the canal.}{}%eng+spa trans
{}{}%rec - time

\noindent
It is very, very widely employed~(\ref{Glo6:abaskuna}--\ref{Glo6:Chaytam}).\\

% 10
\gloexe{Glo6:abaskuna}{}{amv}%
{Lliwta abaskuna albirhakuna ayvis\pb{lla} rantikuni apani llaqtatam.}%amv que first line
{\morglo{lliw-ta}{all-\lsc{acc}}\morglo{abas-kuna}{broad.beans-\lsc{pl}}\morglo{albirha-kuna}{peas-\lsc{pl}}\morglo{ayvis-lla}{sometimes-\lsc{rstr}}\morglo{ranti-ku-ni}{buy-\lsc{refl}-\lsc{1}}\morglo{apa-ni}{bring-\lsc{1}}\morglo{llaqta-ta-m}{town-\lsc{acc}-\lsc{evd}}}%morpheme+gloss
\glotran{Everything -- broad beans, peas -- \pb{once in while} I sell stuff -- I bring it into town.}{}%eng+spa trans
{}{}%rec - time

% 11
\gloexe{Glo6:kwintuqa}{}{sp}%
{Chayna\pb{lla}m. Chay\pb{lla}m kwintuqa. Mas kanchu manam.}%sp que first line
{\morglo{chayna-lla-m}{thus-\lsc{rstr}-\lsc{evd}}\morglo{chay-lla-m}{\lsc{dem.d}-\lsc{rstr}-\lsc{evd}}\morglo{kwintu-qa}{story-\lsc{top}}\morglo{mas}{more}\morglo{ka-n-chu}{be-\lsc{3}-\lsc{neg}}\morglo{mana-m}{no-\lsc{evd}}}%morpheme+gloss
\glotran{That’s the way it goes. That’s \pb{all} there is to the story. There’s no more.}{}%eng+spa trans
{}{}%rec - time

% 12
\gloexe{Glo6:Chaytam}{}{amv}%
{Chaytam aysashpa\pb{lla} pasachiwaq.}%amv que first line
{\morglo{chay-ta-m}{\lsc{dem.d}-\lsc{acc}-\lsc{evd}}\morglo{aysa-shpa-lla}{pull-\lsc{subis}-\lsc{rstr}}\morglo{pasa-chi-wa-q}{pass-\lsc{caus}-\lsc{1.obj}-\lsc{ag}}}%morpheme+gloss
\glotran{They had me cross the river pulling [me by the hand].}{}%eng+spa trans
{}{}%rec - time

\subsection{Discontinuative \phono{-ña}}
Discontinuitive. \phono{-ña}\index[sub]{discontinuitive} indicates transition --~change of state or quality. In affirmative statements, it can generally be translated as ‘already’~(\ref{Glo6:Kundinadaw}--\ref{Glo6:Paqwayanchik}); in negative statements, as ‘no more’ or ‘no longer’~(\ref{Glo6:Unaytrik}), (\ref{Glo6:Manana}); and in questions, as ‘yet’~(\ref{Glo6:Pasarun}), (\ref{Glo6:Rimaya}).\\

% 1
\gloexe{Glo6:Kundinadaw}{}{amv}%
{Kundinadaw\pb{ña}m wakqa kayan.}%amv que first line
{\morglo{kundinadaw-ña-m}{condemned-\lsc{disc}-\lsc{evd}}\morglo{wak-qa}{\lsc{dem.d}-\lsc{top}}\morglo{ka-ya-n}{be-\lsc{prog}-\lsc{3}}}%morpheme+gloss
\glotran{That one is \pb{already} condemned.}{}%eng+spa trans
{}{}%rec - time

% 2
\gloexe{Glo6:kukaywan}{}{amv}%
{Ñuqaqa kukaywan\pb{ña}m qawaruni.}%amv que first line
{\morglo{ñuqa-qa}{I-\lsc{top}}\morglo{kuka-y-wan-ña-m}{coca-\lsc{1}-\lsc{instr}-\lsc{disc}-\lsc{evd}}\morglo{qawa-ru-ni}{see-\lsc{urgt}-\lsc{1}}}%morpheme+gloss
\glotran{I saw it with my coca \pb{already}.}{}%eng+spa trans
{}{}%rec - time

% 3
\gloexe{Glo6:Paqwayanchik}{}{ch}%
{Paqwayanchik\pb{ña}m talpuyta, ¿aw? Papaktapis talpulalu:\pb{ña}m, kanan halakta, ¿aw?}%ch que first line
{\morglo{paqwa-ya-nchik-ña-m}{finish-\lsc{prog}-\lsc{1pl}-\lsc{disc}-\lsc{evd}}\morglo{talpu-y-ta}{plant-\lsc{inf}-\lsc{acc}}\morglo{aw}{yes}\morglo{papa-kta-pis}{potato-\lsc{acc}-\lsc{add}}\morglo{talpu-la-lu-:-ña-m}{plant-\lsc{unint}-\lsc{urgt}-\lsc{1}-\lsc{disc}-\lsc{evd}}\morglo{kanan}{now}\morglo{hala-kta}{corn-\lsc{acc}}\morglo{aw}{yes}}%morpheme+gloss
\glotran{We’re finishing the planting \pb{already}, no? We’ve \pb{already} planted the potatoes, now the corn, no?}{}%eng+spa trans
{}{}%rec - time

% 4
\gloexe{Glo6:Unaytrik}{}{sp}%
{Unaytrik. Kananqa kan\pb{ña}chu imapis.}%sp que first line
{\morglo{unay-tri-k}{before-\lsc{evc}-\lsc{ik}}\morglo{kanan-qa}{now-\lsc{top}}\morglo{ka-n-ña-chu}{be-\lsc{3}-\lsc{disc}-\lsc{neg}}\morglo{ima-pis}{what-\lsc{add}}}%morpheme+gloss
\glotran{That would be a long time ago. Now there isn’t anything \pb{any more}.}{}%eng+spa trans
{}{}%rec - time

% 5
\gloexe{Glo6:Manana}{}{amv}%
{\pb{Manaña} ni santu ni imapis.}%amv que first line
{\morglo{mana-ña}{no-\lsc{disc}}\morglo{ni}{nor}\morglo{santu}{saint}\morglo{ni}{nor}\morglo{ima-pis}{what-\lsc{add}}}%morpheme+gloss
\glotran{There are \pb{no longer} saints or anything.}{}%eng+spa trans
{}{}%rec - time

% 6
\gloexe{Glo6:Pasarun}{}{amv}%
{¿Pasarun\pb{ñachu}? Tapushun.}%amv que first line
{\morglo{pasa-ru-n-ña-chu}{pass-\lsc{urgt}-\lsc{3}-\lsc{disc}-\lsc{q}}\morglo{tapu-shun}{ask-\lsc{1pl.fut}}}%morpheme+gloss
\glotran{Did she go by \pb{yet}? Let’s ask.}{}%eng+spa trans
{}{}%rec - time

% 7
\gloexe{Glo6:Rimaya}{}{lt}%
{¿Rimaya\pb{nña}\pb{chu} kanan wakpi?}%lt que first line
{\morglo{rima-ya-n-ña-chu}{talk-\lsc{prog}-\lsc{3}-\lsc{disc}-\lsc{q}}\morglo{kanan}{now}\morglo{wak-pi}{\lsc{dem.d}-\lsc{loc}}}%morpheme+gloss
\glotran{Are they talking \pb{yet} there now?}{}%eng+spa trans
{}{}%rec - time

\noindent
It can appear freely but never unaccompanied, redundantly, by \phono{ña}~(\ref{Glo6:tukuchkani}), (\ref{Glo6:riqsiyan}).\\

% 8
\gloexe{Glo6:tukuchkani}{}{amv}%
{“¡\pb{Ñam} tukuchkani\pb{ña}!” ¡Puk! ¡Puk! ¡Puk! sikisapa sapu.}%amv que first line
{\morglo{ña-m}{\lsc{disc}-\lsc{evd}}\morglo{tuku-chka-ni-ña}{finish-\lsc{dur}-\lsc{1}-\lsc{disc}}\morglo{puk}{puk}\morglo{puk}{puk}\morglo{puk}{puk}\morglo{siki-sapa}{behind-\lsc{mult.poss}}\morglo{sapu}{frog}}%morpheme+gloss
\glotran{“I’m \pb{already} finishing up!” Puk! Puk! Puk! said the frog with the behind bigger than usual.}{}%eng+spa trans
{}{}%rec - time

% 9
\gloexe{Glo6:riqsiyan}{}{lt}%
{\pb{Ñam} riqsiyan\pb{ña} hukya yaykun.}%lt que first line
{\morglo{ña-m}{\lsc{disc}-\lsc{evd}}\morglo{riqsi-ya-n-ña}{know-\lsc{prog}-\lsc{3}-\lsc{disc}}\morglo{huk-ya}{one-\lsc{emph}}\morglo{yayku-n}{enter-\lsc{3}}}%morpheme+gloss
\glotran{They’re getting to know it \pb{already} and another comes in.}{}%eng+spa trans
{}{}%rec - time

\subsection{Inclusion \phono{-pis}}
\phono{-pis}\index[sub]{inclusion} indicates the inclusion of an item or event into a series of similar items or events. Translated as ‘and’, ‘too’, ‘also’, and ‘even’~(\ref{Glo6:Turnuchawan}--\ref{Glo6:Mamanwa}) or, when negated, ‘neither’ or ‘not even’~(\ref{Glo6:Imapaqtaq}--\ref{Glo6:Pata}).\\

% 1
\gloexe{Glo6:Turnuchawan}{}{ch}%
{Turnuchawan ñuqakunaqa trabaha: walmi\pb{pis} qali\pb{pis}.}%ch que first line
{\morglo{turnu-cha-wan}{turn-\lsc{dim}-\lsc{instr}}\morglo{ñuqa-kuna-qa}{I-\lsc{pl}-\lsc{top}}\morglo{trabaha-:}{work-\lsc{1}}\morglo{walmi-pis}{woman-\lsc{add}}\morglo{qali-pis}{man-\lsc{add}}}%morpheme+gloss
\glotran{We work in turns, the women \pb{and} the men.}{}%eng+spa trans
{}{}%rec - time

% 2
\gloexe{Glo6:Tukuy}{}{amv}%
{Tukuy tuta tushun qaynintinta\pb{pis}.}%amv que first line
{\morglo{tukuy}{all}\morglo{tuta}{night}\morglo{tushu-n}{dance-\lsc{3}}\morglo{qaynintin-ta-pis}{next.day-\lsc{acc}-\lsc{add}}}%morpheme+gloss
\glotran{They dance all night and the next day, \pb{too}.}{}%eng+spa trans
{}{}%rec - time

% 3
\gloexe{Glo6:subrinu}{}{amv}%
{Pay\pb{pis} chay subrinu wañukuptinñamik payqa tumarun.}%amv que first line
{\morglo{pay-pis}{he-\lsc{add}}\morglo{chay}{\lsc{dem.d}}\morglo{subrinu}{nephew}\morglo{wañu-ku-pti-n-ña-mi-k}{die-\lsc{refl}-\lsc{subds}-\lsc{3}-\lsc{disc}-\lsc{evd-\lsc{ik}}}\morglo{pay-qa}{he-\lsc{top}}\morglo{tuma-ru-n}{take-\lsc{urgt}-\lsc{3}}}%morpheme+gloss
\glotran{He, \pb{too}, when his nephew died, took it [poison].}{}%eng+spa trans
{}{}%rec - time

% 4
\gloexe{Glo6:Salchipullu}{}{amv}%
{Salchipullu rantikuqta\pb{pis} tumarun.}%amv que first line
{\morglo{salchipullu}{fried.chicken}\morglo{ranti-ku-q-ta-pis}{buy-\lsc{refl}-\lsc{ag}-\lsc{acc}-\lsc{add}}\morglo{tuma-ru-n}{take-\lsc{urgt}-\lsc{3}}}%morpheme+gloss
\glotran{She took [pictures] of the people selling fried chicken \pb{also}.}{}%eng+spa trans
{}{}%rec - time

% 5
\gloexe{Glo6:Mamanwa}{}{amv}%
{Maman wañukuptin\pb{pis} manam waqanchu.}%amv que first line
{\morglo{mama-n}{mother-\lsc{3}}\morglo{wañu-ku-pti-n-pis}{die-\lsc{refl}-\lsc{subds}-\lsc{3}-\lsc{add}}\morglo{mana-m}{no-\lsc{evd}}\morglo{waqa-n-chu}{cry-\lsc{3}-\lsc{neg}}}%morpheme+gloss
\glotran{\pb{Even} when his mother died, he didn’t cry.}{}%eng+spa trans
{}{}%rec - time

% 6
\gloexe{Glo6:Imapaqtaq}{}{amv}%
{“¿Imapaqtaq ñuqa waqashaq?” nin. “Warmiypaq\pb{pis} waqarqani\pb{chu}.”}%amv que first line
{\morglo{ima-paq-taq}{what-\lsc{purp}-\lsc{seq}}\morglo{ñuqa}{I}\morglo{waqa-shaq}{cry-\lsc{1.fut}}\morglo{nin}{say-\lsc{3}}\morglo{warmi-y-paq-pis}{woman-\lsc{1}-\lsc{ben}-\lsc{add}}\morglo{waqa-rqa-ni-chu}{cry-\lsc{pst}-\lsc{1}-\lsc{neg}}}%morpheme+gloss
\glotran{“Why am I going to cry?” he said. “I did\pb{n’t} cry for my wife, \pb{either}.”}{}%eng+spa trans
{}{}%rec - time

% 7
\gloexe{Glo6:Paykunaqa}{}{amv}%
{Paykunaqa \pb{manam} qawarqa\pb{pischu}.}%amv que first line
{\morglo{pay-kuna-qa}{he-\lsc{pl}-\lsc{top}}\morglo{mana-m}{no-\lsc{evd}}\morglo{qawa-rqa-pis-chu}{see-\lsc{pst}-\lsc{add}-\lsc{neg}}}%morpheme+gloss
\glotran{\pb{Neither} did they see us.}{}%eng+spa trans
{}{}%rec - time

% 8
\gloexe{Glo6:Pata}{}{amv}%
{Pata saqayta\pb{pis} atipan\pb{chu}.}%amv que first line
{\morglo{pata}{terrace}\morglo{saqa-y-ta-pis}{go.up-\lsc{inf}-\lsc{acc}-\lsc{add}}\morglo{atipa-n-chu}{be.able-\lsc{3}-\lsc{neg}}}%morpheme+gloss
\glotran{They ca\pb{n’t} \pb{even} go up one terrace.}{}%eng+spa trans
{}{}%rec - time

\noindent
\phono{-pis} may --~or, even, may generally~-- imply contrast with some preceding element. Where it scopes over subordinate clauses, it can often be translated ‘although’ or ‘even’~(\ref{Glo6:Uratam}), (\ref{Glo6:Hinaptin}).\\

% 9
\gloexe{Glo6:Uratam}{}{amv}%
{Uratam muna\pb{shpapis}.}%amv que first line
{\morglo{ura-ta-m}{hour-\lsc{acc}-\lsc{evd}}\morglo{muna-shpa-pis}{want-\lsc{subis}-\lsc{add}}}%morpheme+gloss
\glotran{\pb{Although} I want to know the time.}{}%eng+spa trans
{}{}%rec - time

% 10
\gloexe{Glo6:Hinaptin}{}{sp}%
{Hinaptin wasipiña rumiwan takaptin\pb{pis} uyan\pb{chu}.}%sp que first line
{\morglo{hinaptin}{then}\morglo{wasi-pi-ña}{house-\lsc{loc}-\lsc{disc}}\morglo{rumi-wan}{stone-\lsc{instr}}\morglo{taka-pti-n-pis}{hit-\lsc{subds}-\lsc{3}-\lsc{add}}\morglo{uya-n-chu}{be.able-\lsc{3}-\lsc{neg}}}%morpheme+gloss
\glotran{Later, at home, \pb{even when} they hit it with a rock, it couldn’t.}{}%eng+spa trans
{}{}%rec - time

\noindent
Attaching to interrogative-indefinite stems, it forms indefinites and, with \phono{mana}, negative indefinites~(\ref{Glo6:Chaynam}--\ref{Glo6:chambyakuqpaq}).\\

% 11
\gloexe{Glo6:Chaynam}{}{amv}%
{Chaynam \pb{imallatapis} wasiman apamun.}%amv que first line
{\morglo{chayna-m}{thus-\lsc{evd}}\morglo{ima-lla-ta-pis}{what-\lsc{rstr}-\lsc{acc}-\lsc{add}}\morglo{wasi-man}{house-\lsc{all}}\morglo{apa-mu-n}{bring-\lsc{cisl}-\lsc{3}}}%morpheme+gloss
\glotran{That way he brings a little \pb{something} to his house.}{}%eng+spa trans
{}{}%rec - time

% 12
\gloexe{Glo6:tiyndaman}{}{ach}%
{Llapa tiyndaman yaykushpaqa lliw lliwshi \pb{imantapis} apakun.}%ach que first line
{\morglo{llapa}{all}\morglo{tiynda-man}{store-\lsc{all}}\morglo{yayku-shpa-qa}{enter-\lsc{subis}-\lsc{top}}\morglo{lliw}{all}\morglo{lliw-shi}{all-\lsc{evr}}\morglo{ima-n-ta-pis}{what-\lsc{3}-\lsc{acc}-\lsc{add}}\morglo{apa-ku-n}{bring-\lsc{refl}-\lsc{3}}}%morpheme+gloss
\glotran{They entered all the stores and took everything and \pb{anything} they had.}{}%eng+spa trans
{}{}%rec - time

% 13
\gloexe{Glo6:chambyakuqpaq}{}{amv}%
{Alli chambyakuqpaq \pb{manam imapis} faltanmanchu.}%amv que first line
{\morglo{alli}{good}\morglo{chambya-ku-q-paq}{work-\lsc{refl}-\lsc{ag}-\lsc{ben}}\morglo{mana}{no}\morglo{ima-pis}{what-\lsc{add}}\morglo{falta-n-man-chu}{be.missing-\lsc{3}-\lsc{cond}-\lsc{neg}}}%morpheme+gloss
\glotran{\pb{Nothing} can be lacking for a good worker.}{}%eng+spa trans
{}{}%rec - time

\noindent
It is in free variation with \phono{-pas}, and, after a vowel, with \phono{-s}~(\ref{Glo6:Diskansakamuy}--\ref{Glo6:harquruwara}), the latter particularly common in the \ACH{} dialect.\\

% 14
\gloexe{Glo6:Diskansakamuy}{}{lt}%
{“¡Diskansakamuy wasikipa!” niwan kikin\pb{pas} diskansuman ripun.}%lt que first line
{\morglo{diskansa-ka-mu-y}{rest-\lsc{refl}-\lsc{cisl}-\lsc{imp}}\morglo{wasi-ki-pa}{house-\lsc{2}-\lsc{loc}}\morglo{ni-wa-n}{say-\lsc{1.obj}-\lsc{3}}\morglo{kiki-n-pas}{self-\lsc{3}-\lsc{add}}\morglo{diskansu-man}{rest-\lsc{all}}\morglo{ripu-n}{go-\lsc{3}}}%morpheme+gloss
\glotran{“Go rest in your house,” he said to me and he, himself, \pb{too}, went to rest.}{}%eng+spa trans
{}{}%rec - time

% 15
\gloexe{Glo6:Hinaptinqa}{}{sp}%
{Hinaptinqa yutu pawaptinqa chay, “¡Aaaapship ship ship!” Yutu\pb{pas} “¡Wwaaaayyy!”}%sp que first line
{\morglo{hinaptin-qa}{then-\lsc{top}}\morglo{yutu}{partridge}\morglo{pawa-pti-n-qa}{fly-\lsc{subds}-\lsc{3}-\lsc{top}}\morglo{chay}{\lsc{dem.d}}\morglo{aaaapship}{aaaapship}\morglo{ship}{ship}\morglo{ship}{ship}\morglo{yutu-pas}{partridge-\lsc{add}}\morglo{wwaaaayyy}{wwaaaayyy}}%morpheme+gloss
\glotran{Then, when the partridge jumped, he [cried], “Aaaap-ship-ship-ship!” The partridge, \pb{too}, [cried] “Wwaaaayyy!”}{}%eng+spa trans
{}{}%rec - time

% 16
\gloexe{Glo6:harquruwara}{}{lt}%
{Ñuqata\pb{s} harquruwara Kashapataman riranim.}%lt que first line
{\morglo{ñuqa-ta-s}{I-\lsc{acc}-\lsc{add}}\morglo{harqu-ru-wa-ra}{toss.out-\lsc{urgt}-\lsc{1.obj}-\lsc{pst}}\morglo{Kashapata-man}{Kashapata-\lsc{all}}\morglo{ri-ra-ni-m}{go-\lsc{pst}-\lsc{1}-\lsc{evd}}}%morpheme+gloss
\glotran{They threw me out, \pb{too}, and I went to Kashapata.}{}%eng+spa trans
{}{}%rec - time

\subsection{Precision, certainty \phono{-puni}}
\phono{-puni} indicates certainty\index[sub]{certainty} or precision\index[sub]{precision}. It can be translated as ‘necessarily’, ‘definitely’, ‘precisely’. It is attested only in the \AMV{} dialect, where, still, it is not widely employed.\\

% 1
\gloexe{Glo6:Paqarin}{}{amv}%
{Paqarin\pb{puni}m rishaq.~\updag}%amv que first line
{\morglo{paqarin-puni-m}{tomorrow-\lsc{cert}-\lsc{evd}}\morglo{ri-shaq}{go-\lsc{1.fut}}}%morpheme+gloss
\glotran{I’m going to go \pb{precisely} tomorrow.}{}%eng+spa trans
{}{}%rec - time

% 2
\gloexe{Glo6:puni}{}{amv}%
{Mana\pb{puni}m.~\updag}%amv que first line
{\morglo{mana-puni-m}{no-\lsc{cert}-\lsc{evd}}}%morpheme+gloss
\glotran{By no means.}{}%eng+spa trans
{}{}%rec - time

% 3
\gloexe{Glo6:wiqawninchikman}{}{amv}%
{Chay wiqawninchikman\pb{puni} chiri yakuta truranchik.}%amv que first line
{\morglo{chay}{\lsc{dem.d}}\morglo{wiqaw-ni-nchik-man-puni}{waist-\lsc{euph}-\lsc{1pl}-\lsc{all}-\lsc{cert}}\morglo{chiri}{cold}\morglo{yaku-ta}{water-\lsc{acc}}\morglo{trura-nchik}{put-\lsc{1pl}}}%morpheme+gloss
\glotran{We put cold water \pb{right} on our lower backs.}{}%eng+spa trans
{}{}%rec - time

\subsection{Topic-marking \phono{-qa}}\label{ssec:topic}
\phono{-qa}\index[sub]{topic marker} indicates the topic of a clause~(\ref{Glo6:sultiram}--\ref{Glo6:Difindiwanchik}), including in those cases where it attaches to subordinate clauses~(\ref{Glo6:pasiyuman}), (\ref{Glo6:Qipiruptinqa}).\\

% 1
\gloexe{Glo6:sultiram}{}{ch}%
{Madri sultiram kaya: ñuqalla\pb{qa}.}%ch que first line
{\morglo{madri}{mother}\morglo{sultira-m}{alone-\lsc{evd}}\morglo{ka-ya-:}{be-\lsc{prog}-\lsc{1}}\morglo{ñuqa-lla-qa}{I-\lsc{rstr}-\lsc{top}}}%morpheme+gloss
\glotran{\pb{I}’m a single mother.}{}%eng+spa trans
{}{}%rec - time

% 2
\gloexe{Glo6:Ganawniyki}{}{lt}%
{Ganawniyki\pb{qa} achkam miranqa.}%lt que first line
{\morglo{ganaw-ni-yki-qa}{cattle-\lsc{euph}-\lsc{2}-\lsc{top}}\morglo{achka-m}{a.lot-\lsc{evd}}\morglo{mira-nqa}{increase-\lsc{3.fut}}}%morpheme+gloss
\glotran{Your \pb{cattle} are going to multiply a lot.}{}%eng+spa trans
{}{}%rec - time

% 3
\gloexe{Glo6:waqakunki}{}{sp}%
{Qam\pb{qa} waqakunki sumaqllatam. Ñuqa\pb{qa} quyu quyuta waqayani.}%sp que first line
{\morglo{qam-qa}{you-\lsc{top}}\morglo{waqa-ku-nki}{cry-\lsc{refl}-\lsc{2}}\morglo{sumaq-lla-ta-m}{pretty-\lsc{rstr}-\lsc{acc}-\lsc{evd}}\morglo{ñuqa-\pb{qa}}{I-\lsc{top}}\morglo{quyu}{ugly}\morglo{quyu-ta}{ugly-\lsc{acc}}\morglo{waqa-ya-ni}{cry-\lsc{prog}-\lsc{1}}}%morpheme+gloss
\glotran{\pb{You} sing nicely. \pb{I}’m singing awfully.}{}%eng+spa trans
{}{}%rec - time

% 4
\gloexe{Glo6:Yatraqnin}{}{amv}%
{Yatraqnin\pb{qa}; mana yatraqnin\pb{qa} manayá.}%amv que first line
{\morglo{yatra-q-ni-n-qa}{know-\lsc{ag}-\lsc{euph}-\lsc{3}-\lsc{top}}\morglo{mana}{no}\morglo{yatra-q-ni-n-qa}{know	-\lsc{ag}-\lsc{euph}-\lsc{top}}\morglo{mana-yá}{no-\lsc{emph}}}%morpheme+gloss
\glotran{\pb{Those} of them who knew; not \pb{those} of them who didn’t know.}{}%eng+spa trans
{}{}%rec - time

% 5
\gloexe{Glo6:mikunchik}{}{amv}%
{Kanan\pb{qa} mikunchik munasanchik[ta] qullqi kaptin\pb{qa}.}%amv que first line
{\morglo{kanan-qa}{now-\lsc{top}}\morglo{miku-nchik}{eat-\lsc{1pl}}\morglo{muna-sa-nchik[-ta]}{want-\lsc{prf}-\lsc{1}-\lsc{acc}}\morglo{qullqi}{money}\morglo{ka-pti-n-qa}{be-\lsc{subds}-\lsc{3}-\lsc{top}}}%morpheme+gloss
\glotran{\pb{Now} we eat whatever we want when there’s money.}{}%eng+spa trans
{}{}%rec - time

% 6
\gloexe{Glo6:Llaqtaykipa}{}{amv}%
{Llaqtaykipa\pb{qa} ¿tarpunkichu sibadata?}%amv que first line
{\morglo{llaqta-yki-pa-qa}{town-\lsc{2}-\lsc{loc}-\lsc{top}}\morglo{tarpu-nki-chu}{plant-\lsc{2}-\lsc{q}}\morglo{sibada-ta}{barley-\lsc{acc}}}%morpheme+gloss
\glotran{In \pb{your town}, do you plant barley?}{}%eng+spa trans
{}{}%rec - time

% 7
\gloexe{Glo6:puriq}{}{amv}%
{Uray\pb{qa} puriq kani trakillawan trakinchikpis nananankama.}%amv que first line
{\morglo{uray-qa}{down.hill-\lsc{top}}\morglo{puri-q}{walk-\lsc{ag}}\morglo{ka-ni}{be-\lsc{1}}\morglo{traki-lla-wan}{foot-\lsc{rstr}-\lsc{instr}}\morglo{traki-nchik-pis}{foot-\lsc{1pl}-\lsc{add}}\morglo{nana-na-n-kama}{hurt-\lsc{nmlz-}\lsc{3}-\lsc{lim}}}%morpheme+gloss
\glotran{I would walk \pb{down hill} just on foot until our feet hurt.}{}%eng+spa trans
{}{}%rec - time

% 8
\gloexe{Glo6:Difindiwanchik}{}{amv}%
{Difindiwanchik malichukunapaq\pb{qa}.}%amv que first line
{\morglo{difindi-wa-nchik}{defend-\lsc{1.obj}-\lsc{1pl}}\morglo{malichu-kuna-paq-qa}{curse-\lsc{pl}-\lsc{abl}-\lsc{top}}}%morpheme+gloss
\glotran{It protects us against \pb{curses}.}{}%eng+spa trans
{}{}%rec - time

% 9
\gloexe{Glo6:pasiyuman}{}{ch}%
{Lluqsila pasiyuman yaykushpa\pb{qa} manaña puydila\uo{}chu piru.}%ch que first line
{\morglo{lluqsi-la}{go.out-\lsc{pst}}\morglo{pasiyu-man}{walk-\lsc{all}}\morglo{yayku-shpa-qa}{enter-\lsc{subis}-\lsc{top}}\morglo{mana-ña}{no-\lsc{disc}}\morglo{puydi-la-chu}{be.able-\lsc{pst}-\lsc{neg}}\morglo{piru}{but}}%morpheme+gloss
\glotran{They went out for a walk but \pb{when they went in}, they couldn’t.}{}%eng+spa trans
{}{}%rec - time

% 10
\gloexe{Glo6:Qipiruptinqa}{}{sp}%
{Qipiruptinqa~\dots{} chay kundur\pb{qa} qipiptin huk turuta pagaykun.}%sp que first line
{\morglo{qipi-ru-pti-n-qa}{carry-\lsc{urgt}-\lsc{subds}-\lsc{3}-\lsc{top}}\morglo{chay}{\lsc{dem.d}}\morglo{kundur-qa}{condor-\lsc{top}}\morglo{qipi-pti-n}{carry-\lsc{subds}-\lsc{3}}\morglo{huk}{one}\morglo{turu-ta}{bull-\lsc{acc}}\morglo{paga-yku-n}{pay-\lsc{excep}-\lsc{3}}}%morpheme+gloss
\glotran{\pb{When he carried her}, after the condor carried her, she payed him a bull.}{}%eng+spa trans
{}{}%rec - time

\subsection{Continuative \phono{-Raq}}
\phono{-Raq}\index[sub]{continuitive} --~realized in \CH{} as \phono{-laq}~(\ref{Glo6:Kichwallaktam}) and in all other dialects as \phono{-raq}~-- indicates continuity of action, state or quality.\\

% 1
\gloexe{Glo6:Kichwallaktam}{}{ch}%
{Kichwallaktam limakuya: kaytraw\pb{laq} manam kastillanukta lima:chu.}%ch que first line
{\morglo{kichwa-lla-kta-m}{Quechua-\lsc{rstr}-\lsc{acc}-\lsc{evd}}\morglo{lima-ku-ya-:}{talk-\lsc{refl}-\lsc{prog}-\lsc{1}}\morglo{kay-traw-laq}{\lsc{dem.p}-\lsc{loc}-\lsc{cont}}\morglo{mana-m}{no-\lsc{evd}}\morglo{kastillanu-kta}{Spanish-\lsc{acc}}\morglo{lima-:-chu}{talk-\lsc{1}-\lsc{neg}}}%morpheme+gloss
\glotran{I’m just talking Quechua. Here, \pb{still}, we don’t speak Spanish.}{}%eng+spa trans
{}{}%rec - time

\noindent
It can generally be translated ‘still’~(\ref{Glo6:Qamqa}--\ref{Glo6:Kamanpi}) or, negated, ‘yet’~(\ref{Glo6:Runtuwanmi}), (\ref{Glo6:mayqinniypis}).\\

% 2
\gloexe{Glo6:Qamqa}{}{ach}%
{Qamqa flaku\pb{raq}mi. Hawlapam qamtaqa wirayachisayki.}%ach que first line
{\morglo{qam-qa}{you-\lsc{top}}\morglo{flaku-raq-mi}{skinny-\lsc{cont}-\lsc{evd}}\morglo{hawla-pa-m}{cage-\lsc{loc}-\lsc{evd}}\morglo{qam-ta-qa}{you-\lsc{acc}-\lsc{top}}\morglo{wira-ya-chi-sayki}{fat-\lsc{inch}-\lsc{caus}-\lsc{1>2.fut}}}%morpheme+gloss
\glotran{You’re \pb{still} skinny. I’m going to fatten you up in a cage.}{}%eng+spa trans
{}{}%rec - time

% 3
\gloexe{Glo6:Taqsana}{}{amv}%
{Taqsana\pb{raq}tri. Millwata taqsashun.}%amv que first line
{\morglo{taqsa-na-raq-tri}{wash-\lsc{nmlz}-\lsc{cont}-\lsc{evc}}\morglo{millwa-ta}{wool-\lsc{acc}}\morglo{taqsa-shun}{wash-\lsc{1pl.fut}}}%morpheme+gloss
\glotran{It has to be cleaned \pb{still}. We have to clean the wool.}{}%eng+spa trans
{}{}%rec - time

% 4
\gloexe{Glo6:Kamanpi}{}{lt}%
{Kamanpi puñukuyaptin\pb{raq} tarirun.}%lt que first line
{\morglo{kama-n-pi}{bed-\lsc{3}-\lsc{loc}}\morglo{puñu-ku-ya-pti-n-raq}{sleep-\lsc{refl}-\lsc{prog}-\lsc{subds}-\lsc{3}-\lsc{cont}}\morglo{tari-ru-n}{find-\lsc{urgt}-\lsc{3}}}%morpheme+gloss
\glotran{He found him when he was sleeping \pb{still} in his bed.}{}%eng+spa trans
{}{}%rec - time

% 5
\gloexe{Glo6:Runtuwanmi}{}{amv}%
{Runtuwanmi qaquyanmi chaypa \pb{mana}\pb{raq}mi shakashwan.}%amv que first line
{\morglo{runtu-wan-mi}{egg-\lsc{instr}-\lsc{evd}}\morglo{qaqu-ya-n-mi}{massage-\lsc{prog}-\lsc{3}-\lsc{evd}}\morglo{chay-pa}{\lsc{dem.d}-\lsc{loc}}\morglo{mana-raq-mi}{no-\lsc{cont}-\lsc{evd}}\morglo{shakash-wan}{guinea.pig-\lsc{instr}}}%morpheme+gloss
\glotran{He’s massaging with an egg -- \pb{not yet} with the guinea pig.}{}%eng+spa trans
{}{}%rec - time

% 6
\gloexe{Glo6:mayqinniypis}{}{amv}%
{\pb{Mana}m mayqinniypis wañuni\pb{raq}chu.}%amv que first line
{\morglo{mana-m}{no-\lsc{evd}}\morglo{mayqin-ni-y-pis}{which-\lsc{euph}-\lsc{1}-\lsc{add}}\morglo{wañu-ni-raq-chu}{die-\lsc{1}-\lsc{cont}-\lsc{neg}}}%morpheme+gloss
\glotran{\pb{None} of us has died yet.}{}%eng+spa trans
{}{}%rec - time

\noindent
Marking rhetorical questions, it can indicate a kind of despair~(\ref{Glo6:Yawarnintachu}), (\ref{Glo6:gringukunaqa}).\\

% 7
\gloexe{Glo6:Yawarnintachu}{}{ach}%
{¿Yawarnintachu? ¿Imata\pb{raq} hurqura chay dimunyukuna?}%ach que first line
{\morglo{yawar-ni-n-ta-chu}{blood-\lsc{euph}-\lsc{3}-\lsc{acc}-\lsc{q}}\morglo{ima-ta-raq}{what-\lsc{acc}-\lsc{cont}}\morglo{hurqu-ra}{take.out-\lsc{pst}}\morglo{chay}{\lsc{dem.d}}\morglo{dimunyu-kuna}{Devil-\lsc{pl}}}%morpheme+gloss
\glotran{His blood? \pb{What in the world} did the devil suck out of him?}{}%eng+spa trans
{}{}%rec - time

% 8
\gloexe{Glo6:gringukunaqa}{}{ach}%
{Chay gringukunaqa altukunatash rin. ¿Imayna\pb{raq} chay runata wañuchin?}%ach que first line
{\morglo{chay}{\lsc{dem.d}}\morglo{gringu-kuna-qa}{gringo-\lsc{pl}-\lsc{top}}\morglo{altu-kuna-ta-sh}{high-\lsc{pl}-\lsc{acc}-\lsc{evr}}\morglo{ri-n}{go-\lsc{3}}\morglo{imayna-raq}{how-\lsc{cont}}\morglo{chay}{\lsc{dem.d}}\morglo{runa-ta}{\lsc{person}-\lsc{acc}}\morglo{wañu-chi-n}{die-\lsc{caus}-\lsc{3}}}%morpheme+gloss
\glotran{The gringos go to the heights, they say. \pb{How on earth} could they kill those people?}{}%eng+spa trans
{}{}%rec - time

\noindent
With subordinate clauses, it may indicate a prerequisite or a necessary condition for the event to take place, translating in English as ‘first’ or ‘not until’~(\ref{Glo6:ruwashpa}).\\

% 9
\gloexe{Glo6:ruwashpa}{}{amv}%
{Kisuta ruwashpa\pb{raq} trayamuyan.}%amv que first line
{\morglo{kisu-ta}{cheese-\lsc{acc}}\morglo{ruwa-shpa-raq}{make-\lsc{subis}-\lsc{cont}}\morglo{traya-mu-ya-n}{arrive-\lsc{cisl}-\lsc{prog}-\lsc{3}}}%morpheme+gloss
\glotran{\pb{Once} she makes the cheese, she’s coming.}{}%eng+spa trans
{}{}%rec - time

\noindent
\phono{Chay-raq} indicates an imminent future, translating in Andean Spanish \spanish{recién}~(\ref{Glo6:tapayan}). Employed as a coordinator, it implies a contrast between the coordinated elements (see~§~\ref{sec:coord}).\\

% 10
\gloexe{Glo6:tapayan}{}{amv}%
{Chay\pb{raq}mi tapayan. Qallaykuyani chay\pb{raq}.}%amv que first line
{\morglo{chay-raq-mi}{\lsc{dem.d}-\lsc{cont}-\lsc{evd}}\morglo{tapa-ya-n}{cover-\lsc{prog}-\lsc{3}}\morglo{qalla-yku-ya-ni}{begin-\lsc{excep}-\lsc{prog}-\lsc{1}}\morglo{chay-raq}{\lsc{dem.d}-\lsc{cont}}}%morpheme+gloss
\glotran{He’s \pb{just now going to} cap it. I’m \pb{just now} going to start.}{}%eng+spa trans
{}{}%rec - time

\subsection{Sequential \phono{-taq}}
\phono{-taq}\index[sub]{sequential} indicates the sequence of events~(\ref{Glo6:Tardiqa}).\\

% 1
\gloexe{Glo6:Tardiqa}{}{amv}%
{Tardiqa yapa listu suyan; yapa\pb{taq}shi trayarun.}%amv que first line
{\morglo{tardi-qa}{afternoon-\lsc{top}}\morglo{yapa}{again}\morglo{listu}{ready}\morglo{suya-n}{wait-\lsc{3}}\morglo{yapa-taq-shi}{again-\lsc{seq}-\lsc{evr}}\morglo{traya-ru-n}{arrive-\lsc{urgt}-\lsc{3}}}%morpheme+gloss
\glotran{In the afternoon, \pb{again}, ready, he waits. \pb{Then, again}, [the zombie] arrived.}{}%eng+spa trans
{}{}%rec - time

\noindent
Adelaar~(p.c.)\index[aut]{Adelaar, Willem F. H.} points out that in Ayacucho Quechua \phono{-ña-taq} is a fixed combination. It appears that may be the case here too~(\ref{Glo6:pikarushpa}--\ref{Glo6:makiywan}). In these examples \phono{-taq} seems to continue to indicate a sequence of events.\\

% 2
\gloexe{Glo6:pikarushpa}{}{amv}%
{Lliwta pikarushpa, kaymanñataq quturini trurani wakmanña\pb{taq}.}%amv que first line
{\morglo{lliw-ta}{all-\lsc{acc}}\morglo{pika-ru-shpa}{pick-\lsc{urgt}-\lsc{subds}}\morglo{kay-man-ña-taq}{\lsc{dem.d}-\lsc{all}-\lsc{disc}-\lsc{seq}}\morglo{qutu-ri-ni}{gather-\lsc{incep}-\lsc{1}}\morglo{trura-ni}{put-\lsc{1}}\morglo{wak-man-ña-taq}{\lsc{dem.p}-\lsc{all}-\lsc{disc}-\lsc{seq}}}%morpheme+gloss
\glotran{When I have all these sorted, \pb{then} I gather everything here and \pb{then} store it there.}{}%eng+spa trans
{}{}%rec - time

% 3
\gloexe{Glo6:takllawanmi}{}{ch}%
{Qaliqa takllawanmi halun. Qipantaña\pb{taq} kulpakta maqanchik pikuwan.}%ch que first line
{\morglo{qali-qa}{man-\lsc{top}}\morglo{taklla-wan-mi}{plow-\lsc{instr}-\lsc{evd}}\morglo{halu-n}{turn.earth-\lsc{3}}\morglo{qipa-n-ta-ña-taq}{behind-\lsc{3}-\lsc{acc}-\lsc{disc}-\lsc{seq}}\morglo{kulpa-kta}{clod-\lsc{acc}}\morglo{maqa-nchik}{hit-\lsc{1pl}}\morglo{piku-wan}{pick-\lsc{instr}}}%morpheme+gloss
\glotran{Men turn over the earth with a foot plow. Behind them, \pb{then}, we break up the clods with a pick.}{}%eng+spa trans
{}{}%rec - time

% 4
\gloexe{Glo6:makiywan}{}{amv}%
{Ñuqapa makiywan aytrichiyanmi. Kanan trakillaña\pb{taq}. Huknin makiwanña\pb{taq} kananmi.}%amv que first line
{\morglo{ñuqa-pa}{I-\lsc{gen}}\morglo{maki-y-wan}{hand-\lsc{1}-\lsc{instr}}\morglo{aytri-chi-ya-n-mi}{stir-\lsc{caus}-\lsc{prog}-\lsc{3}-\lsc{evd}}\morglo{kanan}{now}\morglo{traki-lla-ña-taq}{foot-\lsc{rstr}-\lsc{disc}-\lsc{seq}}\morglo{huk-ni-n}{one-\lsc{euph}-\lsc{3}}\morglo{maki-wan-ña-taq}{hand-\lsc{instr}-\lsc{disc}-\lsc{seq}}\morglo{kanan-mi}{now-\lsc{evd}}}%morpheme+gloss
\glotran{He’s stirring it with my hand. Now, the foot. Now with the other hand.}{}%eng+spa trans
{}{}%rec - time

\noindent
In a question introduced by an interrogative (\phono{pi-}, \phono{ima-}~\dots) \phono{-taq} attaches to the interrogative in case it is the only word in the phrase or, in case the phrase includes two or more words, to the final word in the phrase (\ref{Glo6:Ishpaykuruwan}--\ref{Glo6:Imanashaq}).\\

% 5
\gloexe{Glo6:Ishpaykuruwan}{}{amv}%
{¡Ishpaykuruwan! ¿Imapaq\pb{taq} ishpan?}%amv que first line
{\morglo{ishpa-yku-ru-wa-n}{urinate-\lsc{excep}-\lsc{urgt}-\lsc{1.obj}-\lsc{3}}\morglo{ima-paq-taq}{what-\lsc{purp}-\lsc{seq}}\morglo{ishpa-n}{urinate-\lsc{3}}}%morpheme+gloss
\glotran{It urinated on me! \pb{Why} does it urinate?}{}%eng+spa trans
{}{}%rec - time

% 6
\gloexe{Glo6:rikuq}{}{amv}%
{¿Ima rikuq\pb{taq} karqa sapatillayki?}%amv que first line
{\morglo{ima}{what}\morglo{rikuq-taq}{color-\lsc{seq}}\morglo{ka-rqa}{be-\lsc{pst}}\morglo{sapatilla-yki}{shoe-\lsc{2}}}%morpheme+gloss
\glotran{\pb{What color} were your shoes?}{}%eng+spa trans
{}{}%rec - time

% 7
\gloexe{Glo6:Imanashaq}{}{lt}%
{¿Imanashaq\pb{taq}? Diosllatañatriki.}%lt que first line
{\morglo{ima-na-shaq-taq}{what-\lsc{vrbz}-\lsc{1.fut}-\lsc{seq}}\morglo{Dios-lla-ta-ña-tr-iki}{God-\lsc{rstr}-\lsc{acc}-\lsc{disc}-\lsc{evc}-\lsc{iki}}}%morpheme+gloss
\glotran{\pb{What am I going to do}? It’s for God already.}{}%eng+spa trans
{}{}%rec - time

\noindent
In this capacity, \phono{-taq} may be the most transparent of the enclitics attaching to \phono{q}-phrases. In a clause with a conditional or in a subordinate clause, \phono{-taq} can indicate a warning~(\ref{Glo6:Kurasunniyman}).\\

% 8
\gloexe{Glo6:Kurasunniyman}{}{amv}%
{Kurasunniyman shakashta trurayan. Ñuqa niyani “¡Kaniruwaptinña\pb{taq}!”}%amv que first line
{\morglo{kurasun-ni-y-man}{heart-\lsc{euph}-\lsc{1}-\lsc{all}}\morglo{shakash-ta}{guinea.pig-\lsc{acc}}\morglo{trura-ya-n}{put-\lsc{prog}-\lsc{3}}\morglo{ñuqa}{I}\morglo{ni-ya-ni}{say-\lsc{prog}-\lsc{1}}\morglo{kani-ru-wa-pti-n-ña-taq}{bite-\lsc{urgt}-\lsc{1.obj}-\lsc{subds}-\lsc{3}-\lsc{disc}-\lsc{seq}}}%morpheme+gloss
\glotran{He’s putting the guinea pig over my heart. I’m saying, “\pb{Be careful} it doesn’t bite me!”}{}%eng+spa trans
{}{}%rec - time

\noindent
\phono{-taq} also functions as a conjunction~(\ref{Glo6:puchkawan}) (see~§~\ref{sec:coord}).\\

% 9
\gloexe{Glo6:puchkawan}{}{amv}%
{Warmiña\pb{taq} puchkawan qariña\pb{taq} tihiduwan.}%amv que first line
{\morglo{warmi-ña-taq}{women-\lsc{disc}-\lsc{seq}}\morglo{puchka-wan}{spinning-\lsc{instr}}\morglo{qari-ña-taq}{man-\lsc{disc}-\lsc{seq}}\morglo{tihidu-wan}{weaving-\lsc{instr}}}%morpheme+gloss
\glotran{Women with spinning \pb{and} men with weaving.}{}%eng+spa trans
{}{}%rec - time

\subsection{Emotive \phono{-ya}}\label{ssec:emotive}
\phono{-ya} indicates regret or resignation\index[sub]{emotive}. It can be translated ‘alas’ or ‘regretfully’ or with a sigh. Not very widely employed.\\

% 1
\gloexe{Glo6:Hinashpaqa}{}{amv}%
{Hinashpaqa\pb{ya}, “Wañurachishaqña wakchachaytaqa dimasllam sufriyan.”}%amv que first line
{\morglo{hinashpa-qa-ya}{then-\lsc{top}-\lsc{emo}}\morglo{wañu-ra-chi-shaq-ña}{die-\lsc{urgt}-\lsc{caus}-\lsc{1.fut}-\lsc{disc}}\morglo{wakcha-cha-y-ta-qa}{lamb-\lsc{dim}-\lsc{1}-\lsc{acc}-\lsc{top}}\morglo{dimas-lla-m}{too.much-\lsc{rstr}-\lsc{evd}}\morglo{sufri-ya-n}{suffer-\lsc{prog}-\lsc{3}}}%morpheme+gloss
\glotran{Then, \pb{alas}, “I’m going to kill my little lamb already -- he’s suffering too much,” [I said].}{}%eng+spa trans
{}{}%rec - time

% 2
\gloexe{Glo6:runakunaqa}{}{amv}%
{Unay runakunaqa yatrayan masta, masta\pb{ya}, lliwta~\dots{} aaaa.}%amv que first line
{\morglo{unay}{before}\morglo{runa-kuna-qa}{person-\lsc{pl}-\lsc{top}}\morglo{yatra-ya-n}{know-\lsc{prog}-\lsc{3}}\morglo{mas-ta}{more-\lsc{acc}}\morglo{mas-ta-ya}{more-\lsc{acc}-\lsc{emo}}\morglo{lliw-ta}{all-\lsc{acc}}\morglo{aaaa}{ahhh}}%morpheme+gloss
\glotran{In the old days, people knew more, more, everything, \pb{ahhh}.}{}%eng+spa trans
{}{}%rec - time

\subsection{Evidence}\label{ssec:evidence}
Evidentials\index[sub]{evidentials} indicate the type of the speaker’s source of information. \SYQ, like most\footnote{Note, though, that Huallaga Q counts four evidentials, (\phono{-mi}, \phono{-shi}, \phono{-chi}, snd \phono{-chaq}) (Weber 1989:76). South Conchucos Q counts six, (\phono{-mi}, \phono{-shi}, \phono{-chi}, \phono{-cha:}, and \phono{-cher}); Sihuas, too, counts six (Hintz and Hintz 2014).} other Quechuan languages, counts three evidential suffixes: direct \phono{-mi}~(\ref{Glo6:Taytacha}--\ref{Glo6:puntraw}), reportative \phono{-shi}~(\ref{Glo6:Radyukunapa}--\ref{Glo6:Qarinta}), and conjectural \phono{-tri}~(\ref{Glo6:trayarachiptiki}--\ref{Glo6:Wasiy}) (\ie~the speaker has her own evidence for P (generally visual); the speaker learned P from someone else; or the speaker infers P based on some other evidence). Following a short vowel, these are realized as \phono{-m}, \phono{sh}, and \phono{-tr}, respectively~(\ref{Glo6:puntraw}), (\ref{Glo6:Qarinta}), (\ref{Glo6:Wasiy}).\\

% 1
\gloexe{Glo6:Taytacha}{}{amv}%
{Taytacha José irransakurqa chaypa\pb{m}.}%amv que first line
{\morglo{tayta-cha}{father-\lsc{dim}}\morglo{José}{José}\morglo{irransa-ku-rqa}{herranza-\lsc{refl}-\lsc{pst}}\morglo{chay-pa-m}{\lsc{dem.d}-\lsc{loc}-\lsc{evd}}}%morpheme+gloss
\glotran{My grandfather José held herranzas \pb{there}.}{}%eng+spa trans
{}{}%rec - time

% 2
\gloexe{Glo6:Trurawarqaya}{}{amv}%
{Trurawarqaya huk ratu. Manayá puchukachiwarqachu. Trurawarqa\pb{m}.}%amv que first line
{\morglo{trura-wa-rqa-yá}{put-\lsc{1.obj}-\lsc{pst}-\lsc{emph}}\morglo{huk}{one}\morglo{ratu}{moment}\morglo{mana-yá}{no-\lsc{emph}}\morglo{puchuka-chi-wa-rqa-chu}{finish-\lsc{caus}-\lsc{1.obj}-\lsc{pst}-\lsc{neg}}\morglo{trura-wa-rqa-m}{put-\lsc{1.obj}-\lsc{pst}-\lsc{evd}}}%morpheme+gloss
\glotran{They put me in [school] a short while. They didn’t have me finish, but they did \pb{put me in}.}{}%eng+spa trans
{}{}%rec - time

% 3
\gloexe{Glo6:puntraw}{}{ach}%
{Qayna puntraw qanin puntrawlla\pb{m} trayamura:.}%ach que first line
{\morglo{qayna}{previous}\morglo{puntraw}{day}\morglo{qanin}{day.before.yesterday}\morglo{puntraw-lla-m}{day-\lsc{rstr}-\lsc{evd}}\morglo{traya-mu-ra-:}{arrive-\lsc{cisl}-\lsc{pst}-\lsc{1}}}%morpheme+gloss
\glotran{I arrived yesterday, \pb{just the day} before yesterday.}{}%eng+spa trans
{}{}%rec - time

% 4
\gloexe{Glo6:Radyukunapa}{}{sp}%
{Radyukunapa rimayta rimayan. Lluqsiyamun\pb{shi} tirrurista. Tirrurista rikariyamun\pb{shi}.}%sp que first line
{\morglo{radyu-kuna-pa}{radio-\lsc{pl}-\lsc{loc}}\morglo{rima-y-ta}{talk-\lsc{inf}-\lsc{acc}}\morglo{rima-ya-n}{talk-\lsc{prog}-\lsc{3}}\morglo{lluqsi-ya-mu-n-shi}{go.out-\lsc{prog}-\lsc{cisl}-\lsc{3}-\lsc{evr}}\morglo{tirrurista}{terrorist}\morglo{tirrurista}{terrorist}\morglo{rikari-ya-mu-n-shi}{appear-\lsc{prog}-\lsc{cisl}-\lsc{3}-\lsc{evr}}}%morpheme+gloss
\glotran{On the radio they talk for the sake of talking. Terrorists \pb{are coming out, they say}. Terrorists \pb{are appearing, they say}.}{}%eng+spa trans
{}{}%rec - time

% 5
\gloexe{Glo6:uchukllapa}{}{amv}%
{Chay uchukllapa pashñataq uywakuptinñataq\pb{shi} maqtaqa aparqa mikunanta.}%amv que first line
{\morglo{chay}{\lsc{dem.d}}\morglo{uchuk-lla-pa}{small-\lsc{rstr}-\lsc{loc}}\morglo{pashña-taq}{girl-\lsc{acc}}\morglo{uywa-ku-pti-n-ña-taq-shi}{raise-\lsc{refl}-\lsc{subds}-\lsc{3}-\lsc{disc}-\lsc{seq}-\lsc{evr}}\morglo{maqta-qa}{young.man-\lsc{top}}\morglo{apa-rqa}{bring-\lsc{pst}}\morglo{miku-na-n-ta}{eat-\lsc{nmlz}-\lsc{3}-\lsc{acc}}}%morpheme+gloss
\glotran{When \pb{he raised} the girl in that cave, the man brought her his food, \pb{they say}.}{}%eng+spa trans
{}{}%rec - time

% 6
\gloexe{Glo6:Qarinta}{}{amv}%
{Qarinta\pb{sh} wañurachin mashanta\pb{sh} wañurachin.}%amv que first line
{\morglo{qari-n-ta-sh}{man-\lsc{3}-\lsc{acc}-\lsc{evr}}\morglo{wañu-ra-chi-n}{die-\lsc{urgt}-\lsc{caus}-\lsc{3}}\morglo{masha-n-ta-sh}{son.in.law-\lsc{3}-\lsc{acc}-\lsc{evr}}\morglo{wañu-ra-chi-n}{die-\lsc{urgt}-\lsc{caus}-\lsc{3}}}%morpheme+gloss
\glotran{She killed her \pb{husband, they say}; she killed her \pb{son-in-law, they say}.}{}%eng+spa trans
{}{}%rec - time

% 7
\gloexe{Glo6:trayarachiptiki}{}{amv}%
{Qiñwalman trayarachiptiki wañukunman\pb{tri}.}%amv que first line
{\morglo{qiñwal-man}{quingual.grove-\lsc{all}}\morglo{traya-ra-chi-pti-ki}{arrive-\lsc{urgt}-\lsc{caus}-\lsc{subds}-\lsc{2}}\morglo{wañu-ku-n-man-tri}{die-\lsc{refl}-\lsc{3}-\lsc{cond}-\lsc{evc}}}%morpheme+gloss
\glotran{If you make her go all the way to the quingual grove, she might die.}{}%eng+spa trans
{}{}%rec - time

% 8
\gloexe{Glo6:Suwawan}{}{lt}%
{Suwawan\pb{tri}. Durasnuy kara mansanay kara qanin puntraw.}%lt que first line
{\morglo{suwa-wa-n-tri}{rob-\lsc{1.obj}-\lsc{3}-\lsc{evr}}\morglo{durasnu-y}{peach-\lsc{1}}\morglo{ka-ra}{be-\lsc{pst}}\morglo{mansana-y}{apple-\lsc{1}}\morglo{ka-ra}{be-\lsc{pst}}\morglo{qanin}{previous}\morglo{puntraw}{day}}%morpheme+gloss
\glotran{They \pb{may have robbed} me. The day before yesterday I had peaches and apples.}{}%eng+spa trans
{}{}%rec - time

% 9
\gloexe{Glo6:Wasiy}{}{amv}%
{Wasiy rahasa kayan. Saqaykurunqa\pb{tr}.}%amv que first line
{\morglo{wasi-y}{house-\lsc{1}}\morglo{raha-sa}{crack-\lsc{prf}}\morglo{ka-ya-n}{be-\lsc{prog}-\lsc{3}}\morglo{saqa-yku-ru-nqa-tr}{go.down-\lsc{excep}-\lsc{urgt}-\lsc{3.fut}-\lsc{evc}}}%morpheme+gloss
\glotran{My house is cracked. \pb{It’s going to fall down}.}{}%eng+spa trans
{}{}%rec - time

The evidential system of \SYQ{} is unusual among Quechuan languages, however, in that it overlays the three-way distinction standard to Quechua with a second three-way distinction. The set of evidentials in \SYQ{} thus counts nine members: \phono{-mI}, \phono{-m-ik}, and \phono{-m-iki}; \phono{-shI}, \phono{-sh-ik}, and \phono{-sh-iki}; and \phono{-trI}, \phono{-tr-ik}, and \phono{-tr-iki}. The \phono{-I}, \phono{-ik}, and \phono{-iki} forms are not allomorphs: they receive different interpretations, generally indicating increasing degrees of evidence strength or, in the case of modalized verbs, increasing modal force. §~\ref{ssec:evidence} describes this system in some detail. For further formal analysis, see \citet{Shimelman12}.\index[aut]{Shimelman, Aviva}

In addition to indicating the speaker’s information type, evidentials also function to indicate focus or comment and to complete copular predicates (for further discussion and examples, see §~\ref{sec:emphasis} and~\ref{sec:equative} on emphasis and equatives).

Evidentials are subject to the following distributional restrictions. They never attach to the topic or subject; these are, rather, marked with \phono{-qa}. In content questions, the evidential attaches to the question word or to the last word of the questioned phrase~(\ref{Glo6:chay}) (see~§~\ref{sec:interr} on interrogation).\\

% 10
\gloexe{Glo6:chay}{}{amv}%
{¿May\pb{mi} chay warmi?}%amv que first line
{\morglo{may-mi}{where-\lsc{evd}}\morglo{chay}{\lsc{dem.d}}\morglo{warmi}{woman}}%morpheme+gloss
\glotran{\pb{Where} is that woman?}{}%eng+spa trans
{}{}%rec - time

\noindent
Evidentials do not appear in commands or injunctions~(\ref{Glo6:Ruwaruchun}); finally, only one evidential may occur per clause~(\ref{Glo6:Vakay}).\\

% 11
\gloexe{Glo6:Ruwaruchun}{}{amv}%
{¡Ruwaruchun*mi/shi/tri!}% que first line
{\morglo{ruwa-ru-chun-*mi/shi/tri}{make-\lsc{urgt}-\lsc{injunc}-\lsc{evd}-\lsc{evr}-\lsc{evc}}}%morpheme+gloss
\glotran{\pb{Let} him do it!}{}%eng+spa trans
{}{}%rec - time

% 12
\gloexe{Glo6:Vakay}{}{amv}%
{¡Vakay wira wira\pb{m}, matraypi puñushpa, allin pastuta mikushpa\pb{m}.}% que first line
{\morglo{vaka-y}{cow-\lsc{1}}\morglo{wira}{fat}\morglo{wira-m}{fat-\lsc{evd}}\morglo{matray-pi}{cave-\lsc{loc}}\morglo{puñu-shpa}{sleep-\lsc{subis}}\morglo{allin}{good}\morglo{pastu-ta}{pasture.grass-\lsc{acc}}\morglo{miku-shpa-m}{eat-\lsc{refl}-\lsc{evd}}}%morpheme+gloss
\glotran{My cow is really fat, sleeping in a cave and eating good pasture grass.}{}%eng+spa trans
{}{}%rec - time

All three evidentials are interpreted as assertions. The first, \phono{-mI}, is generally left untranslated in Spanish; the second, \phono{-shI}, is often rendered \phono{dice} ‘they say’; the third is reflected in a change in verb tense or mode (see~§~\ref{ssec:conjectural}). The difference between the three is a matter, first, of whether or not evidence is from personal experience, and, second, whether that evidence supports the proposition, \phono{p}, immediately under the scope of the evidential or another set of propositions, \phono{P’}, that are evidence for \phono{p}, as represented in Table \ref{Tab31}.

% Table 31
\begin{table}
\small\centering
\caption{Evidential schema: “evidence from” by “evidence for”}\label{Tab31}
\begin{tabular}{lll}
\lsptoprule
	& Supports scope 			& Supports \phono{P’}		\\
	& proposition \phono{p} 	& evidence for \phono{p}		\\
\midrule
Direct 	&\lsc{direct} &\lsc{conjectural}\\
(personal experience) evidence 			& \phono{-mI} & \phono{-trI}		\\[1ex]
Reportative 	&\lsc{reportative}&\lsc{conjectural}\\
(non-personal experience) evidence 	& \phono{-shI} & \phono{-trI}	\\
\lspbottomrule
\end{tabular}
\end{table}

So, employing \phono{-mI}(\phono{p}), the speaker asserts predicate \phono{p} and represents that she has personal-experience evidence for \phono{p}; employing \phono{-shI}(\phono{p}), the speaker asserts \phono{p} and refers the hearer to another source for evidence for \phono{p}; and employing \phono{-trI}(\phono{p}), the speaker asserts \phono{p} and represents that \phono{p} is a conjecture from \phono{P’}, propositions for which she has either \phono{-mI}-type or \phono{-shI}-type evidence or both. That is, although \SYQ{} counts three evidential suffixes, it counts only two evidence types, direct and reportative; these two are jointly exhaustive. §~\ref{ssec:direct}--\ref{ssec:conjectural} cover \phono{-mI}, \phono{-shI}, and \phono{trI}, in turn. §~\ref{ssec:evidmodifi} covers the evidential modifiers, \phono{-ari} and \phono{-ik/iki}.

\subsubsection{Direct \phono{-mI}}\label{ssec:direct}
\phono{-mI}\index[sub]{evidentials!direct} indicates that the speaker speaks from direct experience. Unlike \phono{-shI} and \phono{-trI}, it is generally left untranslated. Note that in the examples below, with the exception of~(\ref{Glo6:Vakaqa}), the speaker’s knowledge is \emph{not} the product of visual experience.\\

% 1 (3)
\gloexe{Glo6:Vakaqa}{}{amv}%
{Vakaqa kaypa waqrayuq\pb{mi}ki kayan.}%amv que first line
{\morglo{vaka-qa}{cow-\lsc{top}}\morglo{kay-pa}{\lsc{dem.p}-\lsc{loc}}\morglo{waqra-yuq-m-iki}{horn-\lsc{poss}-\lsc{evd}-\lsc{iki}}\morglo{ka-ya-n}{be-\lsc{prog}-\lsc{3}}}%morpheme+gloss
\glotran{The cows here \pb{have horns}.}{}%eng+spa trans
{}{}%rec - time

% 2 (1)
\gloexe{Glo6:pakarayan}{}{amv}%
{Piñiy\pb{mi} pakarayan wasiypa wak ichuypa ukunpa.}%amv que first line
{\morglo{piñi-y-mi}{necklace-\lsc{1}-\lsc{evd}}\morglo{paka-ra-ya-n}{hide-\lsc{unint}-\lsc{intens}-\lsc{3}}\morglo{wasi-y-pa}{house-\lsc{1}-\lsc{loc}}\morglo{wak}{\lsc{dem.d}}\morglo{ichuy-pa}{straw-\lsc{gen}}\morglo{uku-n-pa}{inside-\lsc{3}-\lsc{loc}}}%morpheme+gloss
\glotran{\pb{My necklace} is hidden in my house under the straw.}{}%eng+spa trans
{}{}%rec - time

% 3 (2)
\gloexe{Glo6:Chaywan}{}{amv}%
{Chaywan\pb{mi} pwirtata ruwayani. Mana\pb{m} achkataq ruwanichu.}%amv que first line
{\morglo{chay-wan-mi}{\lsc{dem.d}-\lsc{instr}-\lsc{evd}}\morglo{pwirta-ta}{door-\lsc{acc}}\morglo{ruwa-ya-ni}{make-\lsc{prog}-\lsc{1}}\morglo{mana-m}{no-\lsc{evd}}\morglo{achka-taq}{a.lot-\lsc{acc}}\morglo{ruwa-ni-chu}{make.\lsc{1}-\lsc{neg}}}%morpheme+gloss
\glotran{I make doors with this. I don’t make a lot.}{}%eng+spa trans
{}{}%rec - time

% 4
\gloexe{Glo6:Karrupis}{}{ach}%
{Karrupis ashnakuyan\pb{mi}.}%ach que first line
{\morglo{karru-pis}{car-\lsc{add}}\morglo{ashna-ku-ya-n-mi}{smell-\lsc{refl}-\lsc{prog}-\lsc{3}-\lsc{evd}}}%morpheme+gloss
\glotran{The buses, too, \pb{stink}.}{}%eng+spa trans
{}{}%rec - time

% 5
\gloexe{Glo6:Qunirirachishunki}{}{ach}%
{Qunirirachishunki. Kaliyntamanchik\pb{mi}.}%ach que first line
{\morglo{quni-ri-ra-chi-shu-nki}{warm-\lsc{incep}-\lsc{caus}-\lsc{2.obj}-\lsc{2}}\morglo{kaliynta-ma-nchik-mi}{warm-\lsc{1.obj}-\lsc{1pl}-\lsc{evd}}}%morpheme+gloss
\glotran{It warms you up. \pb{It warms us up}.}{}%eng+spa trans
{}{}%rec - time

\subsubsection{Reportative \phono{-shI}}
\phono{-shI}\index[sub]{evidentials!reportative} indicates that the speaker’s evidence does not come from personal experience~(\ref{Glo6:Awkichanka}--\ref{Glo6:Lata}).\\

% 1
\gloexe{Glo6:Awkichanka}{}{amv}%
{Awkichanka urqupaqa inkantu\pb{sh} -- karru\pb{sh} chinkarurqa qutrapa.}%amv que first line
{\morglo{Awkichanka}{Awkichanka}\morglo{urqu-pa-qa}{hill-\lsc{loc}-\lsc{top}}\morglo{inkantu-sh}{spirit-\lsc{evr}}\morglo{karru-sh}{car-\lsc{evr}}\morglo{chinka-ru-rqa}{lose-\lsc{urgt}-\lsc{pst}}\morglo{qutra-pa}{lake-\lsc{loc}}}%morpheme+gloss
\glotran{In the hill Okichanka, there is \pb{a spirit, they say} -- a car was lost in a reservoir.}{}%eng+spa trans
{}{}%rec - time

% 2
\gloexe{Glo6:Mashwaqa}{}{ch}%
{Mashwaqa prustatapaq\pb{shi} allin.}%ch que first line
{\morglo{mashwa-qa}{mashua-\lsc{top}}\morglo{prustata-paq-shi}{prostate-\lsc{ben}-\lsc{evr}}\morglo{allin}{good}}%morpheme+gloss
\glotran{Mashua is good for the \pb{prostate}, \pb{they say}.}{}%eng+spa trans
{}{}%rec - time

% 3
\gloexe{Glo6:Chaypash}{}{amv}%
{Chaypa\pb{sh} runtuta mikuchishunki.}%amv que first line
{\morglo{chay-pa-sh}{\lsc{dem.d}-\lsc{loc}-\lsc{evr}}\morglo{runtu-ta}{egg-\lsc{acc}}\morglo{miku-chi-shu-nki}{eat-\lsc{caus}-\lsc{2.obj}-\lsc{2}}}%morpheme+gloss
\glotran{They’ll feed you eggs \pb{there}, \pb{they say}.}{}%eng+spa trans
{}{}%rec - time

% 4
\gloexe{Glo6:Lata}{}{ach}%
{Lata-wan yanu-shpa-taq-\pb{shi} runa-ta-pis miku-ru-ra.}%ach que first line
{\morglo{lata-wan}{can-\lsc{instr}}\morglo{yanu-shpa-taq-shi}{cook-\lsc{subis}-\lsc{seq}-\lsc{evr}}\morglo{runa-ta-pis}{person-\lsc{acc}-\lsc{add}}\morglo{miku-ru-ra}{eat-\lsc{urgt}-\lsc{pst}}}%morpheme+gloss
\glotran{They [the Shining Path] even \pb{cooked} people in metal pots and ate them, \pb{they say}.}{}%eng+spa trans
{}{}%rec - time

\noindent
It is used systematically in stories~(\ref{Glo6:Unay}), (\ref{Glo6:Chaypaqshi}).\\

% 5
\gloexe{Glo6:Unay}{}{sp}%
{Unay\pb{shi} kara huk asnu.}%sp que first line
{\morglo{unay-shi}{before-\lsc{evr}}\morglo{ka-ra}{be-\lsc{pst}}\morglo{huk}{one}\morglo{asnu}{donkey}}%morpheme+gloss
\glotran{\pb{Once upon a time, they say} there was a mule.}{}%eng+spa trans
{}{}%rec - time

% 6
\gloexe{Glo6:Chaypaqshi}{}{lt}%
{Chaypaq\pb{shi} kutirun maman kaqta papanin kaqta.}%lt que first line
{\morglo{chay-paq-shi}{\lsc{dem.d}-\lsc{abl}-\lsc{evr}}\morglo{kuti-ru-n}{return-\lsc{urgt}-\lsc{3}}\morglo{mama-n}{mother-\lsc{3}}\morglo{ka-q-ta}{be-\lsc{ag}-\lsc{acc}}\morglo{papa-ni-n}{father-\lsc{euph}-\lsc{3}}\morglo{ka-q-ta}{be-\lsc{ag}-\lsc{acc}}}%morpheme+gloss
\glotran{He returned \pb{from there, they say}, to his mother’s place, to his father’s place.}{}%eng+spa trans
{}{}%rec - time

\subsubsection{Conjectural \phono{-trI}}\label{ssec:conjectural}
\phono{-trI}\index[sub]{evidentials!conjectural} indicates that the speaker does not have evidence for the proposition directly under the scope of the evidential, but is, rather, conjecturing to that proposition from others for which she does have evidence~(\ref{Glo6:Awayan}--\ref{Glo6:hamuyan}).\\

% 1
\gloexe{Glo6:Awayan}{}{amv}%
{Awayan\pb{tr}iki kamata.}%amv que first line
{\morglo{awa-ya-n-tr-iki}{weave-\lsc{prog}-\lsc{evr}-\lsc{iki}}\morglo{kama-ta}{blanket-\lsc{acc}}}%morpheme+gloss
\glotran{\pb{He must be weaving} a blanket.}{}%eng+spa trans
{}{}%rec - time

% 2
\gloexe{Glo6:kayachuwan}{}{amv}%
{Wañuypaqpis kayachuwan\pb{tr}iki.}%amv que first line
{\morglo{wañu-y-paq-pis}{die-\lsc{inf}-\lsc{abl}-\lsc{add}}\morglo{ka-ya-chuwan-tr-iki}{be-\lsc{prog}-\lsc{1pl.cond}-\lsc{evc}-\lsc{iki}}}%morpheme+gloss
\glotran{\pb{We could be} also about to die.}{}%eng+spa trans
{}{}%rec - time

% 3
\gloexe{Glo6:Kukachankunata}{}{amv}%
{Kukachankunata aparuptiyqa tiyaparuwanqa\pb{tr}ik.}%amv que first line
{\morglo{kuka-cha-n-kuna-ta}{coca-\lsc{dim}-\lsc{3}-\lsc{pl}-\lsc{acc}}\morglo{apa-ru-pti-y-qa}{bring-\lsc{urgt}-\lsc{subds}-\lsc{1}-\lsc{top}}\morglo{tiya-pa-ru-wa-nqa-tr-ik}{sit-\lsc{ben}-\lsc{urgt}-\lsc{1.obj}-\lsc{evc}-\lsc{ik}}}%morpheme+gloss
\glotran{If I bring them their coca, \pb{they’ll accompany me sitting}.}{}%eng+spa trans
{}{}%rec - time

% 4
\gloexe{Glo6:Chayman}{}{ach}%
{Chayman\pb{tr}ik ayarikura.}%ach que first line
{\morglo{chay-man-tr-ik}{\lsc{dem.d}-\lsc{all}-\lsc{evc}-\lsc{ik}}\morglo{aya-ri-ku-ra}{cadaver-\lsc{incep}-\lsc{refl}-\lsc{pst}}}%morpheme+gloss
\glotran{She \pb{must} have become a cadaver.}{}%eng+spa trans
{}{}%rec - time

% 5
\gloexe{Glo6:Upyachinman}{}{ch}%
{Upyachinman\pb{tri}.}%ch que first line
{\morglo{upya-chi-ma-n-tri}{drink-\lsc{caus}-\lsc{1.obj}-\lsc{3}-\lsc{evc}}}%morpheme+gloss
\glotran{She \pb{might} make me drink.}{}%eng+spa trans
{}{}%rec - time

% 6
\gloexe{Glo6:rikuyan}{}{ach}%
{Yakuña\pb{tr} rikuyan pampantaqa.}%ach que first line
{\morglo{yaku-ña-tr}{water-\lsc{disc}-\lsc{evc}}\morglo{ri-ku-ya-n}{go-\lsc{refl}-\lsc{prog}-\lsc{3}}\morglo{pampa-n-ta-qa}{ground-\lsc{3}-\lsc{acc}-\lsc{top}}}%morpheme+gloss
\glotran{\pb{Water should} already be running along the ground.}{}%eng+spa trans
{}{}%rec - time

% 7
\gloexe{Glo6:Allintaqa}{}{sp}%
{Allintaqa. Kapas\pb{tr}iki palabrata kichwapa apakunqa kananpis.}%sp que first line
{\morglo{allin-ta-qa}{good-\lsc{acc}-\lsc{top}}\morglo{kapas-tr-iki}{possible-\lsc{evc}-\lsc{iki}}\morglo{palabra-ta}{word-\lsc{acc}}\morglo{kichwa-pa}{Quechua-\lsc{gen}}\morglo{apa-ku-nqa}{\lsc{bring}-\lsc{refl}-\lsc{3.fut}}\morglo{kanan-pis}{now-\lsc{add}}}%morpheme+gloss
\glotran{Good. \pb{Maybe} they’ll bring Quechua now, too.}{}%eng+spa trans
{}{}%rec - time

% 8
\gloexe{Glo6:hamuyan}{}{amv}%
{Ayvis kumpañaw hamuyan -- wañuypaqpis kayachuwantriki.}%
{\morglo{ayvis}{sometimes}\morglo{kumpañaw}{accompanied}\morglo{hamu-ya-n}{come-\lsc{prog}-\lsc{3}}\morglo{wañu-y-paq-pis}{die-\lsc{1}-\lsc{purp}-\lsc{add}}\morglo{ka-ya-chuwan-tr-iki}{be-\lsc{prog}-\lsc{1pl}.\lsc{cond}-\lsc{evc}-\lsc{iki}}}%morpheme+gloss
\glotran{Sometimes someone comes accompanied -- we might be also about to die.}{}%eng+spa trans
{}{}%rec - time

\subsubsection{Evidential modification}\label{ssec:evidmodifi}
\SYQ{} counts four evidential modifiers\index[sub]{evidentials!modification}, \phono{-ari} and the set \uo, \phono{-ik} and \phono{-iki}. §~\ref{par:assertive} and~\ref{par:evistre} cover \phono{-ari} and \phono{-\uo/-ik/iki}, respectively. The latter largely repeats~\citet{Shimelman12}.

\paragraph{Assertive force \phono{-aRi}}\label{par:assertive}
\phono{-aRi}\index[sub]{evidentials!assertive force} --~realized \phono{-ali} in \CH{}~(\ref{Glo6:Wayrakuyan}) and \phono{-ari} in all other dialects~-- indicates conviction on the part of the speaker.\footnote{The Quechuas of (at least) Ancash-Huailas \citet[151]{Parker76gram},\index[aut]{Parker, Gary J.} Cajamarca-Canaris \citet[158]{Quesada76}\index[aut]{Quesada Castillo, Félix} and Junin-Huanca \citet[238--9]{CerroP76a}\index[aut]{Cerrón-Palomino, Rodolfo M.} have suffixes \phono{-rI}, \phono{-rI} and \phono{-ari}, respectively, which, like the \SYQ{} \phono{-k} succeed evidentials and are most often translated \spanish{pues} ‘then’. It seems unlikely that the \lsc{ahq}, \lsc{ccq} and \lsc{jhq} forms correspond to the \phono{-k} or \phono{-ki} of \SYQ. First, unlike \phono{-ik} or \phono{-iki}, \phono{-rI} and \phono{-ari} may appear independent of any evidential and they may function as general emphatics. Second, \SYQ, too, has a suffix \phono{-ari} which, like \phono{-rI} and \phono{-ari}, functions as a general emphatic, also translating as \spanish{pues}. Third, the \SYQ{} \phono{-ari} is in complementary distribution with \phono{-k} and \phono{-ki}. Finally, unlike the \lsc{ahq}, \lsc{ccq} and \lsc{jhq} forms, the \SYQ{} \phono{-ari} cannot appear independently of the evidentials \phono{-mI} or \phono{-shI} or else of \phono{-y}, and, further, always forms an independent word with these.} \\

% 1
\gloexe{Glo6:Wayrakuyan}{}{amv}%
{Wayrakuyan\pb{mari}.}%amv que first line
{\morglo{wayra-ku-ya-n-m-ari}{wind-\lsc{refl}-\lsc{prog}-\lsc{3}-\lsc{evd}-\lsc{ari}}}%morpheme+gloss
\glotran{\pb{It’s windy}.}{}%eng+spa trans
{}{}%rec - time

\noindent
It can often be translated as ‘surely’ or ‘certainly’ or ‘of course’. \phono{-aRi} generally occurs only in combination with \phono{-mI}~(\ref{Glo6:llapa}), (\ref{Glo6:firmachiwan}), \phono{-shI}~(\ref{Glo6:shali}), (\ref{Glo6:shari}) and \phono{-Yá}~(\ref{Glo6:Qillakuyanki}--\ref{Glo6:Yatraqninqa}).\\

% 2
\gloexe{Glo6:llapa}{}{amv}%
{Mana\pb{mari} llapa ruwayaqhina kayani.}%amv que first line
{\morglo{mana-m-ari}{no-\lsc{evd}-\lsc{ari}}\morglo{llapa}{all}\morglo{ruwa-ya-q-hina}{make-\lsc{prog}-\lsc{ag}-\lsc{comp}}\morglo{ka-ya-ni}{be-\lsc{prog}-\lsc{1}}}%morpheme+gloss
\glotran{\pb{No, of course}, it seems like I’m making it all up.}{}%eng+spa trans
{}{}%rec - time

% 3
\gloexe{Glo6:firmachiwan}{}{lt}%
{Ñuqa[ta]s firmachiwan\pb{mari}. Piru mana\pb{shari} chay wawi warmiytapis firmachinraqchu.}%lt que first line
{\morglo{ñuqa[-ta]-s}{I-\lsc{acc}-\lsc{add}}\morglo{firma-chi-wa-n-m-ari}{sign-\lsc{caus}-\lsc{1.obj}-\lsc{3}-\lsc{evd}-\lsc{ari}}\morglo{piru}{but}\morglo{mana-sh-ari}{no-\lsc{evr}-\lsc{ari}}\morglo{chay}{\lsc{dem.d}}\morglo{wawi}{baby}\morglo{warmi-y-ta-pis}{woman-\lsc{1}-\lsc{acc}-\lsc{add}}\morglo{firma-chi-n-raq-chu}{sign-\lsc{caus}-\lsc{3}-\lsc{cont}-\lsc{neg}}}%morpheme+gloss
\glotran{\pb{They made me sign}, too. But they \pb{didn’t} make my daughter sign yet, \pb{they say}.}{}%eng+spa trans
{}{}%rec - time

% 4
\gloexe{Glo6:shali}{}{ch}%
{Viñacpaq\pb{shali}.}%ch que first line
{\morglo{Viñac-paq-\pb{sh-ali}}{Viñac-\lsc{abl}-\lsc{evr}-\lsc{ari}}}%morpheme+gloss
\glotran{\pb{From Viñac, she says, then}.}{}%eng+spa trans
{}{}%rec - time

% 5
\gloexe{Glo6:shari}{}{amv}%
{Ripun\pb{shari} umaqa kunkanman.}%amv que first line
{\morglo{ripu-n-sh-ari}{go-\lsc{3}-\lsc{evr}-\lsc{ari}}\morglo{uma-qa}{head-\lsc{top}}\morglo{kunka-n-man}{neck-\lsc{3}-\lsc{all}}}%morpheme+gloss
\glotran{The head \pb{went} [flying back] towards his neck, \pb{they say}.}{}%eng+spa trans
{}{}%rec - time

% 6
\gloexe{Glo6:Qillakuyanki}{}{lt}%
{¡Kurriy! Qillakuyanki\pb{trari}.}%lt que first line
{\morglo{kurri-y}{run-\lsc{imp}}\morglo{qilla-ku-ya-nki-tr-ari}{lazy-\lsc{refl}-\lsc{prog}-\lsc{2}-\lsc{evc}-\lsc{ari}}}%morpheme+gloss
\glotran{Run!~\dots{} \pb{You must be being lazy}.}{}%eng+spa trans
{}{}%rec - time

% 7
\gloexe{Glo6:Kidakushun}{}{ach}%
{Kidakushun kaypa\pb{yari}.}%ach que first line
{\morglo{kida-ku-shun}{stay-\lsc{refl}-\lsc{1pl.fut}}\morglo{kay-pa-y-ari}{\lsc{dem.p}-\lsc{loc}-\lsc{emph}-\lsc{ari}}}%morpheme+gloss
\glotran{We’re going to stay \pb{here}.}{}%eng+spa trans
{}{}%rec - time

% 8 (10)
\gloexe{Glo6:Yatraqninqa}{}{amv}%
{Yatraqninqa mana yatraqninqa mana\pb{yari}.}%amv que first line
{\morglo{yatra-q-ni-n-qa}{know-\lsc{ag}-\lsc{euph}-\lsc{3}-\lsc{top}}\morglo{mana}{no}\morglo{yatra-q-ni-n-qa}{know-\lsc{ag}-\lsc{euph}-\lsc{3}-\lsc{top}}\morglo{mana-y-ari}{no-\lsc{emph}-\lsc{ari}}}%morpheme+gloss
\glotran{The ones who knew how. The ones who didn’t know how, \pb{no, of course}.}{}%eng+spa trans
{}{}%rec - time

\noindent
It is far less often employed than \phono{-ik} and \phono{-iki.} It is, however, prevalent in the LT dialect\phono, which supplied the single instance of \phono{tr-ari} in the corpus~(\ref{Glo6:itana}).\\

% 9 (11)
\gloexe{Glo6:itana}{}{amv}%
{Chay wayra itana piru rimidyum Hilda. ¡Piru wachikun\pb{yari}!}%amv que first line
{\morglo{chay}{\lsc{dem.d}}\morglo{wayra}{wind}\morglo{itana}{thorn}\morglo{piru}{but}\morglo{rimidyu-m}{remedy-\lsc{evd}}\morglo{Hilda}{Hilda}\morglo{piru}{but}\morglo{wachi-ku-n-y-ari}{sting-\lsc{refl}-\lsc{3}-\lsc{emph}-\lsc{ari}}}%morpheme+gloss
\glotran{The wind thorns are medicinal, Hilda. But \pb{do they ever sting}!}{}%eng+spa trans
{}{}%rec - time

\paragraph{Evidence strength \phono{-ik} and \phono{-iki}}\label{par:evistre}\index[sub]{evidentials!evidence strength}
\SYQ{} is unusual\footnote{Ayacucho Q also makes use of \phono{-ki}.} in that each of its three evidentials counts three variants, formed by the suffixation of \phono{-\uo}, \phono{-ik} or \phono{-iki}. The resulting nine forms are direct \phono{-mI-\uo}, \phono{-m-ik} and\phono{-m-iki}~(\ref{Glo6:trayamunchu}--\ref{Glo6:Wanuchinakun}); reportative \phono{-shI-\uo}, \phono{-sh-ik} and \phono{-sh-iki}~(\ref{Glo6:susyukuna}--\ref{Glo6:nisha}); and conjectural \phono{-trI-\uo}, \phono{-tr-ik} and\phono{-tr-iki}~(\ref{Glo6:Imapaqraq}--\ref{Glo6:Alkansachin}).\footnote{In Lincha, \phono{-iki} may modify both \phono{-mI} and \phono{-shI} but not \phono{-trI}; in Tana, \phono{-iki} may modify all three evidentials.}\\

% 1
\gloexe{Glo6:trayamunchu}{}{ach}%
{Manam trayamunchu mana\pb{mik} rikarinchu.}%ach que first line
{\morglo{mana-m}{no-\lsc{evd}}\morglo{traya-mu-n-chu}{arrive-\lsc{cisl}-\lsc{3}-\lsc{neg}}\morglo{mana-m-ik}{no-\lsc{evd}-\lsc{ik}}\morglo{rikari-n-chu}{appear-\lsc{3}-\lsc{neg}}}%morpheme+gloss
\glotran{He \pb{hasn’t} arrived. He \pb{hasn’t} showed up.}{}%eng+spa trans
{}{}%rec - time

% 2
\gloexe{Glo6:rishaq}{}{lt}%
{Limatam rishaq. Limapaqa buskaq kan\pb{miki}. Sutintapis rimayan\pb{miki}. ¿Ichu manachu?}%lt que first line
{\morglo{Lima-ta-m}{Lima-\lsc{acc}-\lsc{evd}}\morglo{ri-shaq}{go-\lsc{1.fut}}\morglo{Lima-pa-qa}{Lima-\lsc{loc}-\lsc{top}}\morglo{buska-q}{look.for-\lsc{ag}}\morglo{ka-n-m-iki}{be-\lsc{3}-\lsc{evd}-\lsc{iki}}\morglo{suti-n-ta-pis}{name-\lsc{3}-\lsc{acc}-\lsc{add}}\morglo{rima-ya-n-m-iki}{talk-\lsc{prog}-\lsc{3}-\lsc{evd}-\lsc{iki}}\morglo{ichu}{or}\morglo{mana-chu}{no-\lsc{q}}}%morpheme+gloss
\glotran{I’m going to go to Lima. In Lima, \pb{there are} people who read cards, \pb{then}. They’re \pb{saying} his name, \pb{then}, yes or no?}{}%eng+spa trans
{}{}%rec - time

% 3
\gloexe{Glo6:Wanuchinakun}{}{sp}%
{Wañuchinakun ima\pb{miki} chaytaqa muna:chu.}%sp que first line
{\morglo{wañu-chi-naku-n}{die-\lsc{caus}-\lsc{recip}-\lsc{3}}\morglo{ima-m-iki}{what-\lsc{evd}-\lsc{iki}}\morglo{chay-ta-qa}{\lsc{dem.d}-\lsc{acc}-\lsc{top}}\morglo{muna-:-chu}{want-\lsc{1}-\lsc{neg}}}%morpheme+gloss
\glotran{They kill each other and \pb{what-not, then}. I don’t want that.}{}%eng+spa trans
{}{}%rec - time

% 4
\gloexe{Glo6:susyukuna}{}{amv}%
{Chay\pb{shik} chay susyukuna ruwapakurqa chay nichuchanta wañushpa chayman pampakunanpaq.}%amv que first line
{\morglo{chay-sh-ik}{\lsc{dem.d}-\lsc{evr}-\lsc{ik}}\morglo{chay}{\lsc{dem.d}}\morglo{susyu-kuna}{associates-\lsc{pl}}\morglo{ruwa-paku-rqa}{make-\lsc{jtacc}-\lsc{pst}}\morglo{chay}{\lsc{dem.d}}\morglo{nichu-cha-n-ta}{crypt-\lsc{dim}-\lsc{3}-\lsc{acc}}\morglo{wañu-shpa}{die-\lsc{subis}}\morglo{chay-man}{\lsc{dem.d}-\lsc{all}}\morglo{pampa-ku-na-n-paq}{bury-\lsc{refl}-\lsc{nmlz}-\lsc{3}-\lsc{purp}}}%morpheme+gloss
\glotran{\pb{That’s why, they say}, before, the members made each other the small crypts, to bury them when they died.}{}%eng+spa trans
{}{}%rec - time

% 5
\gloexe{Glo6:Llutanshiki}{}{lt}%
{Llutanshiki. Llutan runa\pb{shik} kan.}%lt que first line
{\morglo{llutan-sh-iki}{ugly-\lsc{evr}-\lsc{iki}}\morglo{llutan}{ugly}\morglo{runa-sh-ik}{person-\lsc{evr}-\lsc{ik}}\morglo{ka-n}{be-\lsc{3}}}%morpheme+gloss
\glotran{\pb{They’re messed up, they say}. There are messed up \pb{people, they say}.}{}%eng+spa trans
{}{}%rec - time

% 6
\gloexe{Glo6:nisha}{}{ch}%
{“¡Mátalo!” nisha\pb{shiki}.}%ch que first line
{\morglo{mátalo}{{}[Spanish]}\morglo{ni-sha-sh-iki}{say-\lsc{npst}-\lsc{evr}-\lsc{iki}}}%morpheme+gloss
\glotran{“Kill him!” \pb{she’s said, they say}.}{}%eng+spa trans
{}{}%rec - time

% 7
\gloexe{Glo6:Imapaqraq}{}{ach}%
{¿Imapaqraq chayta ruwara paytaqa? Yanqaña\pb{trik} chayta wañuchira.}%ach que first line
{\morglo{ima-paq-raq}{what-\lsc{purp}-\lsc{cont}}\morglo{chay-ta}{\lsc{dem.d}-\lsc{acc}}\morglo{ruwa-ra}{make-\lsc{pst}}\morglo{pay-ta-qa}{he-\lsc{acc}-\lsc{top}}\morglo{yanqa-ña-tr-ik}{lie-\lsc{disc}-\lsc{evc}-\lsc{ik}}\morglo{chay-ta}{\lsc{dem.d}-\lsc{acc}}\morglo{wañu-chi-ra}{die-\lsc{caus}-\lsc{pst}}}%morpheme+gloss
\glotran{What did they do that to him for? They \pb{must have} killed him \pb{just for the sake of it}.}{}%eng+spa trans
{}{}%rec - time

% 8
\gloexe{Glo6:Ablan}{}{sp}%
{Ablan\pb{shiki}. “Tragu, vino”, nishpa\pb{triki} ablayamun.}%sp que first line
{\morglo{abla-n-sh-iki}{talk-\lsc{3}-\lsc{evr}-\lsc{iki}}\morglo{tragu}{drink}\morglo{vino}{wine}\morglo{ni-shpa-tr-iki}{say-\lsc{subis}-\lsc{evc}-\lsc{iki}}\morglo{abla-ya-mu-n}{talk-\lsc{prog}-\lsc{cisl}-\lsc{3}}}%morpheme+gloss
\glotran{\pb{They talk, they say, for sure}. “Pay me liquor, wine,” \pb{they must be saying}, talking.}{}%eng+spa trans
{}{}%rec - time

% 9
\gloexe{Glo6:Alkansachin}{}{amv}%
{Alkansachin warkawan\pb{tri}. Kabrapis kasusam, piru. Riqsiyan\pb{triki} runantaqa.}%amv que first line
{\morglo{alkansa-chi-n}{reach-\lsc{caus}-\lsc{3}}\morglo{warka-wan-tri}{sling-\lsc{instr}-\lsc{evc}}\morglo{kabra-pis}{goat-\lsc{add}}\morglo{kasu-sa-m}{attention-\lsc{prf}-\lsc{evd}}\morglo{piru}{but}\morglo{riqsi-ya-n-tr-iki}{know-\lsc{prog}-\lsc{3}-\lsc{evc}-\lsc{iki}}\morglo{runa-n-ta-qa}{person-\lsc{3}-\lsc{acc}-\lsc{top}}}%morpheme+gloss
\glotran{She \pb{must make [the stones] reach} with the sling, \pb{for sure}. The goats obey her. They \pb{must know} their master, \pb{for sure}.}{}%eng+spa trans
{}{}%rec - time

Evidentials obligatorily take evidentional modifier (hereafter “\lsc{em}”) arguments; \lsc{em}s are enclitics and attach exclusively to evidentials. So, for example, \phono{*mishi-m} [cat-\lsc{evd}] and \phono{*mishi-ki} (cat-\lsc{iki}) are both ungrammatical. The corresponding grammatical forms would be \phono{mishi-m-\pb{\uo}} [cat-\lsc{evd}-\uo] and \phono{*mishi\pb{-mi}-ki} (cat-\lsc{evd}-\lsc{iki}), respectively. With all three sets of evidentials, the \phono{-ik} form is associated with some variety of increase over the \phono{-\uo} form; the \phono{-iki} form, with greater increase still. With all three evidentials, \phono{-ik} and \phono{-iki} --~except in those cases in which they take scope over universal-deontic-modal or future-tense verbs~-- indicate an increase in strength of evidence. With the direct \phono{-mI}, \phono{-ik} and \phono{-iki} generally also affect the interpretation of strength of assertion; with the conjectural \phono{-trI}, the interpretation of certainty of conjecture. In the case of universal-deontic modal and future-tense verbs, with both \phono{-mI} and \phono{trI}, \phono{-ik} and \phono{-iki} indicate increasingly strong obligation and increasingly imminent/certain futures, respectively.

\subsubsection{Evidentials in questions}
In questions, the evidentials\index[sub]{evidentials!questions} generally indicate that the speaker expects a response with the same evidential (\ie,~an answer based on direct evidence, reportative evidence or conjecture, in the cases of \phono{-mI}, \phono{-shI}, and \phono{-trI}, respectively)~(\ref{Glo6:Amador}--\ref{Glo6:Kutiramunman}).\\

% 1
\gloexe{Glo6:Amador}{}{ach}%
{¿Amador Garaychu? ¿\pb{Imam} sutin kara?}%ach que first line
{\morglo{Amador}{Amador}\morglo{Garay-chu}{Garay-\lsc{q}}\morglo{ima-m}{what-\lsc{evd}}\morglo{suti-n}{name-\lsc{3}}\morglo{ka-ra}{be-\lsc{pst}}}%morpheme+gloss
\glotran{Amador Garay? \pb{What} was his name?}{}%eng+spa trans
{}{}%rec - time

% 2
\gloexe{Glo6:Maypish}{}{ch}%
{¿\pb{Maypish} wasinta lulayan?}%ch que first line
{\morglo{may-pi-sh}{where-\lsc{loc}-\lsc{evr}}\morglo{wasi-n-ta}{house-\lsc{3}-\lsc{acc}}\morglo{lula-ya-n}{make-\lsc{prog}-\lsc{3}}}%morpheme+gloss
\glotran{\pb{Where did she} say she’s making her house?}{}%eng+spa trans
{}{}%rec - time

% 3
\gloexe{Glo6:Kutiramunman}{}{ach}%
{¿Kutiramunman\pb{chutr}? ¿\pb{Imatrik} pasan?}%ach que first line
{\morglo{kuti-ra-mu-n-man-chu-tr}{return-\lsc{urgt}-\lsc{cisl}-\lsc{q}-\lsc{evc}}\morglo{ima-tr-ik}{what-\lsc{evc}-\lsc{ik}}\morglo{pasan}{pass-\lsc{3}}}%morpheme+gloss
\glotran{\pb{Could} he come back? \pb{What would have} happened?}{}%eng+spa trans
{}{}%rec - time

\noindent
The use of \phono{-trI} in a question may, additionally, indicate that the speaker doesn’t actually expect any response at all~(\ref{Glo6:Kawsan}).\\

% 4
\gloexe{Glo6:Kawsan}{}{ach}%
{¿Kawsan\pb{chutr} mana\pb{chutr}? No se sabe.}%ach que first line
{\morglo{kawsa-n-chu-tr}{live-\lsc{3}-\lsc{q}-\lsc{evc}}\morglo{mana-chu-tr?}{no-\lsc{q}-\lsc{evc}}\morglo{No se sabe.}{{}[Spanish]}}%morpheme+gloss
\glotran{\pb{Would} he be alive or dead? We don’t know.}{}%eng+spa trans
{}{}%rec - time

\noindent
And the use of \phono{-shI} may indicate not that the speaker is expecting an answer based on reported evidence, but that the speaker is reporting the question~(\ref{Glo6:5}).



% % copy the lines above and adapt as necessary

%%%%%%%%%%%%%%%%%%%%%%%%%%%%%%%%%%%%%%%%%%%%%%%%%%%%
%%%                                              %%%
%%%             Backmatter                       %%%
%%%                                              %%%
%%%%%%%%%%%%%%%%%%%%%%%%%%%%%%%%%%%%%%%%%%%%%%%%%%%%

\is{some term| see {some other term}}
\il{some language| see {some other language}}
\issa{some term with pages}{some other term also of interest}
\ilsa{some language with pages}{some other lect also of interest}
 
% There is normally no need to change the backmatter section
\backmatter 
\phantomsection 
\addcontentsline{toc}{chapter}{\lsIndexTitle} 
\addcontentsline{toc}{section}{\lsNameIndexTitle}
\ohead{\lsNameIndexTitle} 
\printindex 
\cleardoublepage
  
\phantomsection 
\addcontentsline{toc}{section}{\lsLanguageIndexTitle}
\ohead{\lsLanguageIndexTitle} 
\printindex[lan] 
\cleardoublepage
  
\phantomsection 
\addcontentsline{toc}{section}{\lsSubjectIndexTitle}
\ohead{\lsSubjectIndexTitle} 
\printindex[sbj]
\ohead{} 
 
\end{document} 

% you can create your book by running
% xelatex main.tex
%
% you can also try a simple 
% make
% on the commandline

