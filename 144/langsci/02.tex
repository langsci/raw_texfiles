\chapter{Hintergründe}
\label{chapter:hintergrund}
In diesem Kapitel führe ich einige Hintergrundaspekte in Isolation ein. Da ich in Bezug auf diese Aspekte im weiteren Verlauf der Arbeit meine eigene Sicht vertrete, halte ich es für wichtig, aufzuzeigen, wo diese Perspektiven verankert sind und welche denkbaren Alternativen es gibt.

Zunächst stelle ich in Abschnitt~\ref{sec:forschung} deskriptive und theoretische Arbeiten vor, die sich z.T. am Rande, z.T. als Hauptgegenstand, mit der Kombination von MPn beschäftigen. Es geht mir weniger um eine chronologische Aufarbeitung der Forschung zu diesem Thema, sondern vielmehr um eine Herausarbeitung der Kriterien, die die Ansätze erarbeitet haben. Es wird sich zeigen, dass das Phänomen der Sequenzierung von MPn in Kombinationen bisher aus ganz verschiedenen Perspektiven betrachtet worden ist. Nahezu alle Teilbereiche sprachwissenschaft\-licher Beschäftigung sind ebenso vertreten wie diachrone Aspekte. Die von mir verfolgte Überlegung, die MP-Abfolgen aus der Interpretation der Strukturen abzuleiten, ist folglich ebenfalls nur ein Blick auf die Dinge. Abschnitt~\ref{sec:transparenz} diskutiert Interpretationsmöglichkeiten, die im Zuge der Bedeutungszuschreibung von MP-Kombina\-tionen vorgeschlagen worden sind. Ich werde hier auch meine im weiteren Verlauf der Arbeit vertretene Position bereits deutlich machen, dass die MPn koordinativ miteinander verknüpft werden. In Abschnitt~\ref{sec:diskursmodell} führe ich das Diskursmodell ein, im Rahmen dessen ich die Einzelpartikeln sowie deren Kombinationen untersuche. Abschnitt~\ref{sec:zugang} führt meine Sicht auf den Beitrag der MPn ein, Anforderungen an Vorgangskontexte zu stellen. Hier wird ebenfalls erstmals deutlich, wie diese Auffassung mit der Modellierung im Diskursmodell einhergeht. Abschnitt~\ref{sec:ikonizität} beschäftigt sich mit dem Konzept der \textit{Ikonizität}. Zweck dieses Abschnitts ist es, einzuordnen, von welcher Art von Form-Funktionszusam\-menhang ich bei der Korrespondenz zwischen Abfolge und Diskursbeitrag, für die ich argumentieren werde, ausgehe.

\section{Modalpartikel-Kombinationen in der Forschung}
\label{sec:forschung}
In \citet[1542-1545]{Zifonun1997} werden in der Behandlung von MP-Kombinationen – differenziert nach verschiedenen satzmodalen Kontexten – die jeweils zulässigen MP-Abfolgen angegeben (vgl. (\ref{14}) bis (\ref{16})).\footnote{Die \citet[594]{Duden2009} begnügt sich mit dem Hinweis auf die Möglichkeit von MP-Kombinationen unter festen Abfolgen und gibt die folgende Reihung an: \\
\textit{ja} > \textit{halt} > \textit{doch} > \textit{einfach} > \textit{auch} > \textit{mal}}

\begin{exe}
	\ex\label{14} 
		Partikelfolge im Aussage-Modus \\
		ja > denn > eben > halt > doch > eben > halt > wohl > einfach > auch > schon > auch > mal 
	\begin{xlist}	
		\ex\label{14a}(\ldots) die Sache etwa kann \textbf{ja eben} nur funktionieren, wenn Rußland eben eine Macht ist wie die andere, nicht? (YAR, 22) 
		\ex\label{14b}Das stimmt \textbf{doch einfach} nicht. (XED, 62)
	\end{xlist}	
\end{exe}
\begin{exe}
	\ex\label{15} 
		Partikelfolge im Frage-Modus \\
		denn > nicht > wohl > etwa > schon > auch > schon > nur > bloß
	\begin{xlist}	
		\ex\label{15a} Was soll ich Ihnen \textbf{denn schon} groß erzählen? (Bild, 22.3.1967, 7)
		\ex\label{15b} \glqq Was ist \textbf{denn bloß} geschehen?\grqq{} fragte ein nicht eben großer Mann, der an der Tür stand, (\ldots). (TPM, 8)	
	\end{xlist}	
\end{exe}
	 	
\begin{exe}
	\ex\label{16} 
		Partikelfolge im Aufforderungs-Modus (einschließlich Wunsch-Modus) \\
		doch > halt > eben > einfach > schon > auch > nur > bloß > ruhig > mal > ja	
	\begin{xlist}	
		\ex\label{16a} (Und jetzt wollte ich natürlich den Pastor NN bitte, daß er 'n gutes Wort für uns einlegt, nicht?) – 				Machen Sie \textbf{doch bloß}. (XEZ, 24)
		\ex\label{16b} Schließ \textbf{auch JA} die Tür ab, wenn du weggehst!
	\end{xlist}			
\hfill\hbox{\citet[1542/1543]{Zifonun1997}}
\end{exe}
Wenngleich die Formulierung derartiger Partikelabfolgen innerhalb der verschie\-denen Satzmodi deskriptiv zutreffen mag, so gibt sie zum einen wenig genera\-lisierende Informationen, da der Grad der Abstraktheit sich auf den satzmodalen Kontext beschränkt. Zum anderen (und das ist der entscheidendere Punkt) bieten sie keinerlei Angaben im Hinblick auf eine \uline{Erklärung} der relativ festen Abfolgesequenzen in MP-Kombinationen.

Im Vergleich zur sehr reichhaltigen, geradezu unüberblickbaren Forschung zum Auftreten der Einzelpartikeln ist die Forschungslage zu MP-Kombinationen als recht übersichtlich einzustufen. In den folgenden Abschnitten werden bestehende Ansätze, die sich mit den Abfolgebeschränkungen in MP-Kombinationen befassen, in ihren Grundannahmen (wenn möglich gruppiert nach geteilten Zugangsarten) vorgestellt.
	 														 
\subsection{Klassenbildung}
Die erste Vorgehensweise, die sich in der Literatur zur Erfassung der Abfolgebeschränkungen von MPn in Kombinationen ausmachen lässt, ist die Gruppierung von (M)Pn zu einander geordneten Klassen. Bei gleichzeitigem Auf\-treten verschiedener Partikeln gibt die Ordnung dieser Klassen vor, in welcher Abfolge die Mitglieder der verschiedenen Klassen auftreten können. Dieses prin\-zipielle Vorgehen findet sich in verschiedenen Arbeiten. Die Ansätze unterscheiden sich einerseits in den konkreten Klassenbildungen voneinander. Andererseits wird die Klassenzugehörigkeit der jeweiligen Partikeln mehr oder weniger motiviert. Den (im Folgenden vorgestellten) Ansätzen von \citet{Engel1968}, \citet{Helbig1981}, \citet{Helbig1990} sowie \citet{Helbig1999} ist eine solche Motiviertheit der Klassenbildung abzusprechen. \citet{Thurmair1991} nennt Gründe für die Zuordnung bestimmter MPn zu gleichen Klassen. 

In \citet{Engel1968} findet sich meines Wissens der erste Vorschlag einer Klassenbildung \is{Klassenbildung}, die die Abfolge in MP-Kombinationen abzubilden beabsichtigt. Die heute als MPn eingestuften Ausdrücke zählt Engel zur Teilgruppe der Adverbiale (\textit{adjungierte Adverbialia}). Er behandelt sie im Rahmen seiner Untersuchung der Positionierung der Negationspartikel \textit{nicht} sowie der \glqq sogenannten Adverbien\grqq{} (\textit{aber}, \textit{doch}, \textit{noch}, \textit{nur}, \textit{wohl} u.a.) (\citealt[85]{Engel1968}). Der Autor nimmt die sieben Gruppen in (\ref{17}) an.

\begin{exe}
	\ex\label{17} 
		Klassen nach \citet[91-94]{Engel1968}
	\begin{xlist}	
		\ex\label{17a} \textit{denn}, \textit{doch}, \textit{ja} (alle unbetont)
		\ex\label{17b} \textit{nun}, \textit{wohl}, \textit{aber}, \textit{also}, \textit{eben}/\textit{halt} (alle unbetont)
		\ex\label{17c} \textit{doch} (betont), \textit{schon} (betont/unbetont)
		\ex\label{17d} \textit{auch} (betont, unbetont)
		\ex\label{17e} \textit{nur}, \textit{bloß}
		\ex\label{17f} \textit{noch}
		\ex\label{17f} \textit{nicht} (betont, unbetont)
	\end{xlist}	
\end{exe}
Treten in einem Satz Partikeln verschiedener Gruppen auf, gilt die Reihenfolge a-b-c-d-e-f-g. Engel führt selbst keine Beispiele an, die Daten in (\ref{17}) aus \citet{Helbig1981} (deren Klassifikation unten angeführt wird) bestätigen die Abfolge der Klassen aus (\ref{18}). 

\begin{exe}
	\ex\label{18} 
		\begin{xlist}	
			\ex\label{18a} Das ist \textbf{denn} (a) \textbf{doch} (a) \textbf{auch} (d) \textbf{nur} (e) ein kleiner Erfolg.
			\ex\label{18b} Damit ist \textbf{doch} (a) \textbf{wohl} (b) \textbf{schon} (c) der Abstieg entschieden.
			\ex\label{18c} Geh \textbf{doch} (a) \textbf{schon} (c) \textbf{mal} (?) nach Hause!
			\hfill\hbox {\citet[42]{Helbig1981}}
		\end{xlist}
\end{exe}
Die Sätze in (\ref{19}) aus Abraham (1995) sind (ebenfalls im Einklang mit der vorhergesagten Partikelabfolge nach (\ref{17}) nicht wohlgeformt (s. zum direkten Vergleich die akzeptablen Sätze in (\ref{20}).

\begin{exe}
	\ex\label{19} 
		\begin{xlist}	
			\ex\label{19a} *Er hat \textbf{doch} (a) \textbf{auch} (d) \textbf{denn} (a) nur wenig verloren.
			\ex\label{19b} *Ich kann \textbf{auch} (d) \textbf{eben} (b) \textbf{aber} (b) \textbf{bloß} 24 Stunden arbeite[n] 					[sic!].
		\end{xlist}
\end{exe}

\begin{exe}
	\ex\label{20} 
		\begin{xlist}	
			\ex\label{20a} Er hat \textbf{denn} (a) \textbf{doc}h (a) \textbf{auch} (d) nur wenig verloren.
			\ex\label{20b} Ich kann aber (b) \textbf{eben} (b) \textbf{auch} (d) \textbf{bloß} 24 Stunden im Tag arbeiten.\footnote{Da \citet{Abraham1995} (im Gegensatz zu \citealt{Engel1968} [s.u.]) korrekt zwischen MPn und anderen Wortarten unterscheidet, sind hier nur die tatsächlichen MP-Vorkommen berücksichtigt. Zu dieser generelleren Problematik mit derartigen Ansätzen s.u..}	
		\end{xlist}
		\hfill\hbox{\citet[248/249]{Abraham1995}}	
\end{exe}
Schon \citet{Engel1968} verweist auf zwei Ausnahmen: Wie (\ref{21}) illustriert, ist die Abfolge von \textit{doch} und \textit{aber} vertauschbar. Und auch \textit{nicht} folgt auf Elemente der Klassen a bis c (vgl. (\ref{22})), kann allerdings vor oder hinter Partikeln der Klassen d bis f stehen (vgl. (\ref{23})).

\begin{exe}
	\ex\label{21} 
		\begin{xlist}	
			\ex\label{21a} Das ist \textbf{aber} (b) \textbf{doch} (a) bedenklich.
			\ex\label{21b} Du hast \textbf{doch} (a) \textbf{aber} (b) keine Zeit.
		\end{xlist}
\end{exe}

\begin{exe}
	\ex\label{22} 
		\begin{xlist}	
			\ex\label{22a} die wir \textbf{nun} (b) \textbf{nicht} (g) kennen
			\ex\label{22b} Warum möchtest du \textbf{denn} (a) \textbf{nicht} (g) bestraft werden?
		\end{xlist}
\end{exe}

\begin{exe}
	\ex\label{23} 
		\begin{xlist}	
			\ex\label{23a} denn das möcht ich \textbf{auch} (d) \textbf{nicht} (g) immer
			\ex\label{23b} Is es \textbf{net} (g) \textbf{auch} (d) so?	
			\hfill\hbox {\citet[94]{Engel1968}}
		\end{xlist}
\end{exe}
Engel ist zwar bemüht, die (aus heutiger Sicht als MPn eingestuften Ausdrücke) über Interpretationsunterschiede (durch Ersatzprobe bzw. Positionierung) abzugrenzen von ihren betonten Varianten (obwohl diese z.T. in anderen Klassen auftreten) sowie von ihren gleichlautenden Formen in anderen Wortarten (in diesem Fall z.B. von Konjunktionen, (temporalen) Adverbien). Trotzdem sind in seine Betrachtung nicht nur MPn eingegangen. Er verwendet diesen Terminus gar nicht, sondern spricht von \glqq Partikeln\grqq{} bzw. \glqq Adverbien\grqq{} (\citealt[91/85]{Engel1968}). 

Das Vorgehen von \citet{Engel1968}, die Partikeln in Klassen zu gruppieren und damit bei gehäuftem Auftreten von Partikeln aus verschiedenen dieser Klassen die Abfolge entlang der Ordnung der Klassen vorherzusagen, wurde in der Folgezeit von verschiedenen Autoren aufgegriffen. Die Ansätze unterscheiden sich voneinander und von Engels ursprünglicher Formulierung in der Anzahl der angenommenen Klassen sowie der Zuordnung ihrer Mitglieder.

\citet{Helbig1981} schlagen die Klassen in (\ref{24}) vor. (\ref{25}) illustriert die Abfolge der Partikeln in Kombinationen entlang der Klassen a bis e. (\ref{26}) führt inakzep\-table Kombinationen an, die der Abfolge in (\ref{24}) nicht entsprechen. Den direkten Vergleich bietet (\ref{27}).

\begin{exe}
	\ex\label{24} 
	 Klassen nach \citet[41/42]{Helbig1981}
		\begin{xlist}	
			\ex\label{24a} \textit{denn}, \textit{doch} (unbetont), \textit{eigentlich}, \textit{etwa}, \textit{ja}
			\ex\label{24b} \textit{aber}, \textit{eben}, \textit{halt}, \textit{vielleicht}, \textit{wohl}
			\ex\label{24c} \textit{doch} (betont), \textit{schon}
			\ex\label{24d} \textit{auch}, \textit{mal}
			\ex\label{24e} \textit{bloß}, \textit{nur}
		\end{xlist}
\end{exe}

\begin{exe}
	\ex\label{25} 
		\begin{xlist}	
			\ex\label{25a} Damit ist \textbf{doc}h (a) \textbf{wohl} (b) \textbf{schon} (c) der Abstieg entschieden.
			\ex\label{25b} Er hat \textbf{eigentlich} (a) \textbf{wohl} (b) \textbf{doch} (c) \textbf{schon} (c) seine Arbeit beendet.
			\ex\label{25c} Er hat \textbf{vielleicht} (b) \textbf{doch} (c) \textbf{mal} (d) \textbf{nur} (e) zugeschaut.
		\end{xlist}
	\hfill\hbox{\citet[42]{Helbig1981}}	
\end{exe}

\begin{exe}
	\ex\label{26} 
		\begin{xlist}	
			\ex\label{26a} ??Sie haben \textbf{doch} (a) \textbf{wohl} (b) \textbf{eigentlich} (a) \textbf{schon} alles getan.
			\ex\label{26b} *Sie haben \textbf{wohl} (b) \textbf{eigentlich} (a) \textbf{doch} (a) \textbf{schon} alles getan.
		\end{xlist}
\end{exe}

\begin{exe}
	\ex\label{27} 
	Sie haben \textbf{eigentlich} (a) \textbf{doch} (a) \textbf{wohl} (b) schon alles getan.
	\newline
	\hbox{}\hfill\hbox{\citet[249]{Abraham1995}}
\end{exe}
Wie auch für die Ausführungen von \citet{Engel1968} gilt, dass die Autoren nicht zwischen MPn und dem Vorkommen der Ausdrücke in anderen Wortarten unterscheiden (vgl. z.B. das temporale Adverb \textit{schon} in (\ref{25b}) oder die Fokuspartikel \textit{nur} in (\ref{25c})). 

Einen weiteren Vorschlag der Klassenbildung \is{Klassenbildung}, die für die Abfolge von MPn in Kombinationen aufkommen soll, macht \citet{Thurmair1991}. Die Problematik der gemeinsamen Betrachtung verschiedener Wortarten scheint mir auf die hier vor\-geschlagene Gruppierung der Partikeln nicht zuzutreffen. In den angeführten Beispielen treten nur MPn auf. \citet[31]{Thurmair1991} nimmt die folgenden fünf Gruppen an:

\begin{exe}
\ex\label{199xy}
\begin{tabular}[t]{lllllllll}
  	\textit{ja} & & \textit{halt} & & \textit{wohl} & & \textit{auch} & & \textit{einfach}\\
  	\textit{denn} & > & \textit{eben} & > & \textit{eigentlich} & > & \textit{nur} & > & \textit{ruhig}\\
  	\textit{doch} & & & & \textit{vielleicht} & & \textit{bloß} & & \textit{schon}\\
  	\textit{aber} & & & & & & \textit{etwa} & & \textit{mal}\\
\end{tabular}
\end{exe}

\noindent	
In (\ref{28}) finden sich Beispiele mit akzeptablen MP-Kombinationen, die den von Thurmair vorhergesagten Abfolgen entsprechen. (\ref{29}) zeigt ungrammatische Sätze, in denen sich die Abfolgen der MPn innerhalb der Kombinationen nicht nach der von Thurmair angenommenen Ordnung richten.

\begin{exe}
	\ex\label{28} 
		\begin{xlist}	
			\ex\label{28a} War er \textbf{denn} (1) \textbf{etwa} (4)\footnote{Die Nummern beziehen sich von links nach rechts auf die Klassen von \citet{Thurmair1991} in (\ref{199xy}).} ein Engel in Kriegszeiten?
			\ex\label{28b} Du könntest \textbf{ja} (1) \textbf{ruhig} (5) \textbf{mal} (5) etwas freundlicher sein.
		\end{xlist}
\end{exe}

\begin{exe}
	\ex\label{29} 
		\begin{xlist}	
			\ex\label{29a} Du könntest *\textbf{ruhig} (5) \textbf{ja} (1) die Sachen wegräumen.
			\ex\label{29b} Die könnten *\textbf{wohl} (3) \textbf{ja} (1) \textbf{auch} (4)/*\textbf{wohl} (3) \textbf{auch} (4) \textbf{ja} (1) ihr Auto 				richten.		
	\hfill\hbox{\citet[29]{Thurmair1991}}	
	\end{xlist}
\end{exe}
Anders als die zuvor angeführten Ansätze gibt \citet{Thurmair1991} eine gewisse Motivation für die gemeinsame Klassenzugehörigkeit bestimmter Partikeln an. Sie macht die Beobachtung, dass die MPn in den jeweiligen Klassen die Wortartenzugehörigkeit ihrer \glq Dubletten\grq {} teilen. Die MPn am Anfang einer Kombination (Klasse 1/2) weisen gleichlautende Formen in den Klassen der Konjunktionen (\textit{aber}, \textit{denn}, \textit{doch}) und Diskurspartikeln (\textit{ja}, \textit{doch}, \textit{ebe}n) auf. Am rechten Rand einer Kombination (Klasse 5) treten MPn auf, die \glq Dubletten\grq {} in der Klasse der Adverbien haben (\textit{einfach}, \textit{schon}, \textit{mal}). Davor werden MPn mit gleichlautenden Formen in der Klasse der Fokuspartikeln positioniert (\textit{auch}, \textit{nur}, \textit{bloß} [Klasse 4]). Und Partikeln mit \glq Dubletten\grq {} bei Satzadverbien (\textit{wohl}, \textit{eigentlich}, \textit{vielleich}t in Klasse 3) treten in der Mitte einer Kombination auf. \citet{Thurmair1991} führt folg\-lich ein Kriterium an, das einen Hinweis auf die Gemeinsamkeiten der MPn in den jeweiligen Klassen beisteuert. Basierend auf anderen Klassenbildungen leitet \citet{Abraham1995} ganz ähnlich unter Bezug auf die Wortartenzugehörigkeit der Vorgängerlexeme der MPn zusätzlich syntaktische Stellungsklassen der MPn ab. Sein Ansatz geht deshalb in dem Sinne über Thurmairs hinaus, als dass er nicht nur die Zusammengehörigkeit der MPn zu einer Klasse motiviert, sondern darüber hinaus Gründe für die (aus der jeweiligen Wortartenzuordnung resultierenden) Abfolgen zwischen den Klassen angibt. Da \citet{Abraham1995} explizit Bezug auf die Vorgängerlexeme nimmt, wird der Ansatz in Abschnitt~\ref{sec:vormp} behandelt.\footnote{Ein weiterer Ansatz, dem man die Annahme aufeinander folgender Klassen zuschreiben kann, deren Ordnung dann die Ordnung der MPn vorgibt, ist \citet[203-206]{Verschueren2003}. Der Autor führt die feste Abfolge bestimmter Partikeln im Niederländischen auf eine generelle, aus der menschlichen Informationsverarbeitungsfähigkeit abgeleitete Tendenz zurück, dass linguistische Elemente, die konzeptuell zusammengehören, auch in der linguistischen Struktur nahe beieinander stehen. Er buchstabiert diese Idee der konzeptuellen Zusammengehörigkeit der Partikeln innerhalb der von ihm angenommenen Klassen unter Bezug auf ihre Funktion in der kontextuellen Verankerung der Äußerung aus. Die Ordnung der Klassen ergibt sich dann aus den jeweiligen Bezugselementen der Mitglieder der Klassen. Die Zuordnung der Partikeln zu den Klassen sowie die Ordnung der Klassen lässt sich folglich ebenfalls als motiviert einstufen. }
\noindent
Die in diesem Abschnitt vorgestellten Ansätze teilen die Vorgehensweise, die Partikeln in verschiedene Klassen zu gruppieren, so dass sich die MP-Abfolge in Kombinationen entlang der Ordnung der Klassen ergibt.

Neben der Problematik, dass insbesondere in älteren Ansätzen nicht präzise zwischen MPn und ihren gleichlautenden Formen unterschieden wird, geraten die Ansätze an ihre Grenzen, wenn einerseits Variabilität in der Abfolge von MPn verschiedener Klassen und andererseits feste Abfolgen von MPn innerhalb einer Klasse auftreten. Im Einklang mit den Überlegungen derartiger Ansätze finden sich Fälle, in denen Partikeln innerhalb einer Klasse umgestellt werden können (vgl. z.B. (\ref{30})). 

\begin{exe}
	\ex\label{30} 
		\begin{xlist}	
			\ex\label{30a} Hilf deiner Schwester \textbf{auch mal}!
			\ex\label{30b} Hilf deiner Schwester \textbf{mal auch}!	
			\hfill\hbox {\citet[42]{Helbig1981}}
		\end{xlist}
\end{exe}
Die Ordnung sollte zwischen den Klassen bestehen, so dass eine freiere Anordnung innerhalb der Klassen (in (\ref{30}) wäre aus \citealt{Helbig1981} Klasse d) betroffen [vgl. (\ref{24})]) eigentlich Teil der Vorhersage wäre. Eine derartige Vertauschung sollte allerdings nicht zwischen MPn verschiedener Klassen eintreten. Fälle dieser Art führt schon Engel (1968) an (vgl. (\ref{21})).

Nach der Klassenbildung aus \citet{Thurmair1991} (vgl. (\ref{199xy})) liegt ein solcher Fall auch in (\ref{30}) vor. Insgesamt scheinen derartige Fälle aber selten.

Problematischer ist, dass sich recht leicht und frequenter Abfolgen zwischen Mitgliedern derselben Klasse finden lassen, die ausgeschlossen werden müssen. Hierunter fallen Daten wie in (\ref{31}) und (\ref{32})).

\begin{exe}
	\ex\label{31} 
		\begin{xlist}	
			\ex\label{31a} *Ich kann \textbf{eben} (b) \textbf{aber} (b) \textbf{auch} (d) \textbf{bloß} 24 Stunde[n] [sic!] arbeiten.
			\ex\label{31b} *Er hat \textbf{doch} (a) \textbf{denn} (a) \textbf{auch} (d) \textbf{nur} wenig verloren.
		\end{xlist}
\end{exe}
	
\begin{exe}
	\ex\label{32} 
		\begin{xlist}	
			\ex\label{32a} Er hat \textbf{denn} (a) \textbf{doch} (a) \textbf{auch} (d) \textbf{nur} wenig verloren.
			\ex\label{32b} Ich kann \textbf{aber} (b) \textbf{eben} (b) \textbf{auch} (d) \textbf{bloß} 24 Stunden im Tag 						arbeiten.
		\end{xlist}
		\hfill\hbox {\citet[248-249]{Abraham1995}}
\end{exe}
Die Sätze in (\ref{31}) sind problematisch für \citet{Engel1968} sowie \citet{Helbig1981}. (\ref{32a}) wird auch durch \citet{Thurmair1991} nicht herausgefiltert. Alle hier bespro\-chenen Ansätze bräuchten folglich weitere Kriterien für Restriktionen innerhalb einer Klasse. 
\noindent
Folgearbeiten haben sich genauer mit den Eigenschaften beschäftigt, die die MPn in Kombinationen jeweils aufweisen, um auf diesem Wege ein Mehr an Erklärung für die Ordnung der Klassen bzw. die Reihungsbeschränkungen allgemein zu leisten. \citet{Thurmair1989, Thurmair1991} formuliert in diesem Sinne eine Reihe von Generalisie\-rungen hinsichtlich derartiger Eigenschaften.

\subsection{Katalog von Bedingungen}
\label{sec:katalog}
In \citet{Thurmair1989, Thurmair1991} ist die Absicht der Autorin in Bezug auf ihre Behandlung von MP-Kombinationen die Formulierung von Generalisierungen hinsichtlich der Eigenschaften von MPn, die steuern, in welchen Abfolgen MPn in Reihungen auftreten. 

Vorbereitend für die Bildung von fünf Hypothesen ist eine Untersuchung, an welchen Positionen bestimmte MPn in Kombinationen relativ zu anderen MPn, mit denen sie prinzipiell kombiniert werden können, stehen. (\ref{33}) bis (\ref{35}) illustrieren beispielhaft das Vorgehen der Autorin. Die Beispielsätze zeigen jeweils, wie bei Einhalten der formulierten Reihenfolgen akzeptable, bei Nicht-Einhalten inakzeptable Strukturen entstehen.
	
\begin{exe}
	\ex\label{33} \textit{ja}: \framebox{ja} – denn – doch – eben – wohl – einfach – sowieso – vielleicht – schon/auch – ruhig – mal
		\begin{xlist}	
			\ex\label{33a} Das ist \textbf{ja vielleicht} auch eine Frechheit!
			\ex\label{33b} *Er kommt \textbf{sowieso ja} morgen.
		\end{xlist}
\end{exe}
	
\begin{exe}
	\ex\label{34} \textit{mal}: ja – doch – eben – halt – einfach – wohl – auch – nur – ruhig – \framebox{mal} 
		\begin{xlist}	
			\ex\label{34a} Geh \textbf{halt einfach }mal hin!
			\ex\label{34b} *Komm \textbf{mal doch} her!
		\end{xlist}
\end{exe}
	
\begin{exe}
	\ex\label{35} \textit{auch}: ja – denn – doch – aber – eben – halt – wohl – schon – einfach – \framebox{auch} – schon
 			– einfach – nur – bloß – JA	 
		\begin{xlist}	
			\ex\label{35a} Und machen Sie \textbf{auch JA} viele Fotos – wir haben die Exklusivrechte.
			\ex\label{35b} *Mach \textbf{auch aber }das Fenster zu! 
			\hfill\hbox {\citet[286/288/287]{Thurmair1989}}
		\end{xlist}
\end{exe}															  
Auf der Basis derartiger Stellungsuntersuchungen einzelner MPn (zu weiteren Fällen vgl. \citealt[285-288]{Thurmair1989}) formuliert die Autorin die folgenden fünf Hypothesen, um abstraktere Gesetzmäßigkeiten aufzudecken (wobei anzumerken ist, dass die Hypothesen 1, 3 und 4 bereits von \citealt{Dahl1988} $[$auf weniger systematische Art$]$ formuliert wurden):

\begin{exe}
	\ex\label{36}
\begin{itemize}
	\item[H1] unspezifisch > spezifisch
	\item[H2] Bezug auf momentane Äußerung > qualitative Bewertung des Vorgän\-gerbeitrags	
	\item[H3] Flexibilität des Illokutionstyps > Festlegung des Illokutionstyps
	\item[H4] keine Modifizierung des Stärkegrades der Illokution > Modifizierung des Stärkegrades der Illokution
	\item[H5] keine besondere Beeinflussung des Gesprächspartners in seinem \linebreak (nicht-)
sprachlichen Handeln > besondere Beeinflussung des Ge\-sprächspartners in seinem (nicht-)sprachlichen Handeln  
\end{itemize}
\end{exe}
Um die Hypothesen zu illustrieren, ist es nötig, auf die Ausführungen von \citet{Thurmair1989} zum Beitrag der Einzelpartikeln einzugehen. Hypothese 1 sagt aus, dass die MP mit der unspezifischsten Bedeutung in einer Kombination an erster Stelle steht (vgl. dazu auch schon \citealt[225, 238, 242, 255, 265]{Dahl1988}). Diese These findet ihre Bestätigung beispielsweise darin, dass \textit{denn} stets vor allen anderen MPn auftritt.

\begin{exe}
	\ex\label{36} 
		\begin{xlist}	
			\ex\label{36a} Wer macht \textbf{denn nur}/*\textbf{nur denn} so einen Lärm?
			\ex\label{36b} Was wird’s \textbf{denn auch schon}/*\textbf{auch schon denn} groß sein?
		\end{xlist}
	\hfill\hbox{\citet[29]{Thurmair1991}}	
\end{exe}
Die MP \textit{denn} drückt in Fragen nach Ansicht der Autorin aus, dass sich der Anlass für die Frage im aktuellen Äußerungskontext befindet $($<KONNEX>$)$. Dazu tritt ggf. (vornehmlich in E-Fragen) das Erstaunen des Sprechers, d.h. eine (nicht-) \linebreak sprachliche Handlung des Diskurspartners war für den Sprecher unerwar\-tet \linebreak $($<UNERWARTET>$_{\textrm{V}})$\footnote{\textit{V} steht für \textit{Vorgänger}.} (vgl. (\ref{37})).

\begin{exe}
	\ex\label{37} 
			Karl: Also Fredi hat sich nicht geäußert darüber, un Kai auch nich.\\
			Iris: Hör ma, was is \textbf{denn} das für 'n komisches Geräusch im Hintergrund?\\
			Karl: Ja, ich sag dir, die Ellen näht da in ihrem Zimmer. (BA, 152)
			\newline
			\hbox{}\hfill\hbox{\citet[166]{Thurmair1989}}	
\end{exe}
Die Annahme einer unspezifischen Bedeutung lässt sich für \textit{denn} plausiblerweise nachweisen, wenn man annimmt, dass es sich bei seiner Funktion der Markierung der Anknüpfung an den Kontext um einen Beitrag handelt, der in kohärenter Kommunikation unter den Standardverlauf eines Gespräches fällt und ansonsten auch nicht weiter durch sprachliches Material kodiert wird. \citet[170]{Thurmair1989} zieht sogar in Erwägung, \textit{denn} in w-Fragen \is{w-Frage} als reine Fragepartikel, d.h. (unspezifischen) Anzeiger dieses Funktionstyps einzustufen. Vergleicht man den Beitrag von \textit{denn} z.B. mit dem, den die Autorin für \textit{nur} ansetzt (vgl. (\ref{38})), wird der Nachweis der Rolle von zunehmender Spezifizität (von links nach rechts) in einer Kombination umso nachvollziehbarer.

\begin{exe}
	\ex\label{38} 
		\glqq Wirklich ausgezeichnet, dieser Martini. Echter britischer Gordon Gin. Wo haben Sie den im vierten Kriegsjahr \textbf{nur} immer noch herbekommen, Monsieur Ferroud (\ldots)\grqq{} (Si, 334) 			
	\hfill\hbox{\citet[179]{Thurmair1989}}	
\end{exe}
Für die MP \textit{nur} nimmt die Autorin an, dass sie in w-Fragen die Illokution der Äußerung verstärkt $($<VERST\"ARKUNG>$)$. Das ganze Interesse des Sprechers sei auf den Frageakt gerichtet. Die Partikel \textit{nur} leistet dann in Fragen \is{Frage} plausiblerweise einen spezifischeren Beitrag als \textit{denn}, da sie der Frage eine spezifischere Färbung verleiht. Es handelt sich nicht mehr nur um eine Frage, sondern um eine Frage mit starker Frageillokution.

Hypothese 2 von \citet{Thurmair1989} besagt, dass Partikeln, die sich auf die momentane Äußerung beziehen, Partikeln, die eine qualitative Bewertung des Vor\-gängerbeitrags vornehmen, vorangehen. Unter diese These fällt laut Thurmair beispielsweise, dass \textit{sowieso} auf \textit{halt} folgt (vgl. (\ref{39})).

\begin{exe}
	\ex\label{39} 
	A: Nett, und jetzt habt ihr Peter nicht mal gefragt, ob er mit möchte?\\
	B: Peter ist \textbf{halt sowieso}/*\textbf{sowieso halt} immer bereits verplant.
\end{exe}
Den Beitrag von\textit{ halt} sieht die Autorin darin, die Verbindung zum Vorgängerbeitrag $($<KONNEX>$)$ anzuzeigen, sowie den ausgedrückten Sachverhalt als plausibel auszuzeichnen $($<PLAUSIBEL>$_{\textrm{H}})$:

\begin{exe}
	\ex\label{40} 
			Es war nur ein Rappel, meint der Doktor, nicht das Herz. Sie ist \textbf{halt} wetterfühlig, und die senile Demenz 					wird auch schuld sein. (= Weil sie wetterfühlig ist, war es nur ein Rappel.)
			\hbox{}\hfill\hbox{\citet[125]{Thurmair1989}}	
\end{exe}
Verwendet ein Sprecher die Partikel \textit{sowieso}, drückt er Thurmair zufolge aus, dass ihm der Sachverhalt vor dem Gespräch bereits bekannt war $($<BEKANNT$>_{\textrm{S}})$. 

Dazu schränkt er den Vorgängerbeitrag in seiner Relevanz ein \linebreak $($<RELEVANZEINSCHRÄNKUNG>$_{\textrm{V}})$ und gibt eine Korrekturanweisung an den Hörer, den Sachverhalt zu berücksichtigen $($<KORREKTUR>$)$ (vgl. \citealt[136-137]{Thurmair1989} und das Beispiel in (\ref{41})).

\begin{exe}
	\ex\label{41} 
			Max: Das Bier war leider nicht im Kühlschrank.\\
			Rolf: Macht nichts ich hab \textbf{sowieso} nen empfindlichen Magen.
			\newline
			\hbox{}\hfill\hbox{\citet[138]{Thurmair1989}}	
\end{exe}
Thurmairs Hypothese 2 bestätigt sich im Falle der Abfolge von \textit{halt} und \textit{sowieso}, wenn man annimmt, dass sich die Einschätzung von Plausibilität auf die \textit{halt}-Äußerung selbst (und damit die momentane Äußerung) bezieht, während die eingeschränkte Relevanz dem Vorgängerbeitrag zugeschrieben wird.

Hypothese 3 besagt, dass MPn, die den Illokutionstyp \is{Illokutionstyp} festlegen, in einer Kombination an letzter Stelle stehen (vgl. hierzu schon \citealt[222, 225]{Dahl1988}). Die MP \textit{mal} tritt z.B. stets an der letzten Stelle einer Kombination auf (vgl. (\ref{42})). 

\begin{exe}
	\ex\label{42} 
		\begin{xlist}	
			\ex\label{42a} Geh \textbf{doch mal}/*\textbf{mal doch} zur Seite.
			\ex\label{42b} Geh \textbf{halt mal}/*\textbf{mal halt} hin.
			\end{xlist}
\end{exe}
Für diese Partikel gilt, dass sie nur in Sätzen auftritt, die als Aufforderungen \is{Aufforderung} zu interpretieren sind (vgl. (\ref{43})).

\begin{exe}
	\ex\label{43} 
		\begin{xlist}	
			\ex\label{43a} Ich richte den Reissalat her. Du könntest \textbf{mal} nach den Getränken schauen.
			\ex\label{43b} Gehst du \textbf{mal} ans Telefon?
			\hfill\hbox {\citet[184/185]{Thurmair1989}}
		\end{xlist}
\end{exe}
Den Beitrag von \textit{mal} sieht Thurmair darin, die Aufforderungsillokution abzu-schwächen $($<ABSCHW\"ACHUNG>$)$. Eine Festlegung auf einen bestimmten Illokutionstyp wie im Falle von \textit{mal} auf die Aufforderung ist bei den MPn, die \textit{mal} in Kombinationen vorangehen, nicht nachzuweisen (\textit{doch} in Aussagen, Wünschen, Aufforderungen, Fragen, Ausrufen, \textit{halt} in Aussagen, Aufforderungen). In \citet[30]{Thurmair1991} differenziert sie diese Hypothese weiter. Sie nimmt an, dass eine MP umso weiter rechts auftritt, je spezifischer sie den Illokutionstyp bestimmt. Diese Formulierung erlaubt die Positionierung einer MP, die den Illokutionstyp festlegt, am linken Rand einer MP-Kombination. Beispielsweise steht \textit{denn} (wie oben gesehen) stets vorne in Kombinationen, legt aber den Illokutionstyp der Frage fest. Die differenziertere Version von Hypothese 3 erfasst nun, dass weitere MPn, die zusammen mit \textit{denn} auftreten, auf \textit{denn} folgen, weil sie illokutionär spezifischer sind in dem Sinne, dass sie speziellere Fragetypen kodieren. Wie oben bereits erläutert, ist \textit{denn} eine Partikel, die mit dem Anzeigen der Frageillokution und der Anbindung an den Kontext einen relativ unspezifi\-schen Beitrag leistet. Andere MPn, die zusammen mit \textit{denn} auftreten können, bestimmen genauer, um welche Art von Frage es sich handelt. Wie (\ref{44}) illustriert, legen \textit{auch} und \textit{etwa} beispielsweise Antworterwartungen \is{Antworterwartung} konkreter nahe, wie sie für (\ref{45}) nicht anzunehmen sind.
\begin{exe}
	\ex\label{44} 
		\begin{xlist}	
			\ex\label{44a} Ist das Kleid \textbf{auch} durchsichtig?
			\ex\label{44b} Ist das Kleid \textbf{etwa} durchsichtig?
			\hfill\hbox {\citet[27]{Thurmair1991}}
		\end{xlist}
\end{exe}

\begin{exe}
	\ex\label{45} 
	Ist das Kleid \textbf{denn} durchsichtig?
\end{exe}						          
In (\ref{44a}) ist die bevorzugte Antwort \textit{ja}, in (\ref{44b}) \textit{nein}. Da \textit{auch} und\textit{ etwa} den Illokutionstyp \is{Illokutionstyp} \textit{Frage} \is{Frage} (anders als \textit{denn}) weiter einschränken auf Fragen mit bestimmten Antworterwartungen, folgen sie im Einklang mit Thurmairs Differenzierung der These 3 der MP \textit{denn} (vgl. auch \citealt[238]{Dahl1988}).

\begin{exe}
	\ex\label{46} 
		\begin{xlist}	
			\ex\label{46a} Ist das Kleid \textbf{denn auch}/*\textbf{auch denn} durchsichtig?
			\ex\label{46b} Ist das Kleid \textbf{denn etwa}/*\textbf{etwa denn} durchsichtig?
		\end{xlist}
\end{exe}
Hypothese 4 formuliert die von Thurmair beobachtete Verteilung, dass MPn, die die Illokution abschwächen bzw. verstärken, am rechten Rand der Kombination auftreten. Dies trifft z.B. auf das betonte \textit{JA} zu (vgl. (\ref{47})).

\begin{exe}
	\ex\label{47} 
		\begin{xlist}	
			\ex\label{47a} Mach mir \textbf{auch JA}/*\textbf{JA auch} immer deine Aufgabe ordentlich!
			\ex\label{47b} Ich darf \textbf{doch JA}/*\textbf{JA doch} meine Autoschlüssel nicht vergessen.
		\end{xlist}
	\hfill\hbox{\citet[286]{Thurmair1989}, (\citeyear[109]{Thurmair1991})}	
\end{exe}
Bei der Einzelbetrachtung von \textit{JA} nimmt Thurmair an, dass diese Partikel in der Regel in Imperativsätzen \is{Imperativsatz} auftritt. In diesem sprachlichen Kontext verstärke sie den Sprecherwillen $($<VERSTÄRKUNG>$)$, so dass die Aufforderung als Warnung oder Drohung verstanden werde (vgl. (\ref{48})).

\begin{exe}
	\ex\label{48} 
	Mutter zu Tochter: Komm \textbf{JA} nicht zu spät heim!
	\hfill\hbox {\citet[109]{Thurmair1989}}
\end{exe}
Für die der MP \textit{JA} in (\ref{47}) vorangehenden Partikeln \textit{auch} bzw. \textit{doch} ist hingegen nicht anzunehmen, dass sie einen Beitrag zur Verstärkung oder Abschwächung des Illokutionstyps leisten (vgl. \citealt[118/119]{Thurmair1989} bzw. 158 zu \textit{doch} bzw. \textit{auch} in Imperativsätzen).

In ihrer fünften These hält Thurmair fest, dass MPn, die eine besondere Be\-einflussung des Gesprächspartners in seinem (nicht-)sprachlichen Handeln bewirken, am rechten Rand einer MP-Kombination auftreten. Dies lässt sich erneut anhand des Auftretens von MPn in Fragen illustrieren. Wie oben ausgeführt, legt eine Frage \is{Frage} mit \textit{denn} keine spezielle Antworterwartung nahe. Anderes gilt hier für Fragen mit beispielsweise \textit{etwa} oder \textit{schon}. Fragen mit \textit{etwa} weisen eine ne\-gative Antworterwartung auf, d.h. die bevorzugte Antwort ist \textit{nein} (s.o.). Tritt \textit{schon} in w-Fragen \is{w-Frage} auf, wirkt diese MP als Indikator für rhetorische Fragen \is{rhetorische Frage}. Die Lücke, die der w-Ausdruck eröffnet, wird entweder durch einen negierenden Ausdruck (vgl. (\ref{49})) oder genau \underline{eine} mögliche lexikalische Füllung geschlossen (vgl. (\ref{50})) – wobei in beiden Fällen (wie generell in rhetorischen Fragen) vom Sprecher angenommen wird, dass die Antwort für die Beteiligten klar/bekannt ist.

\begin{exe}
	\ex\label{49} 
	Keine Frage: Kinder müssen gesund essen. Aber wer kommt \textbf{schon} gegen Pommes
 	und Schokolade, Limonade und Naschereien an? (SZ)
\end{exe}
	
\begin{exe}
	\ex\label{50} 
		Uwe: Was hast du denn heute gemacht? \\
		Mara: Na, was werd ich \textbf{schon} gemacht haben? (Gearbeitet natürlich.)
			\newline
			\hbox{}\hfill\hbox{\citet[154]{Thurmair1989}}	
\end{exe}
Anders als \textit{denn}-Fragen bringen \textit{etwa}- und \textit{schon}-Fragen folglich Antworterwar\-tungen \is{Antworterwartung}mit sich. In diesem Sinne beeinflussen diese beiden MPn den Gesprächs\-partner mehr, weshalb sie im Einklang mit Thurmairs fünfter These in der Kombination mit \textit{denn} am rechten Rand stehen (vgl. (\ref{51}), (\ref{52})).

\begin{exe}
	\ex\label{51} 
	Ist das Kleid \textbf{denn etwa}/*\textbf{etwa denn} durchsichtig?
\end{exe}
\vspace{-0.65cm}
\begin{exe}
	\ex\label{52} 
	Na, was werd ich \textbf{denn schon}/*\textbf{schon denn} gemacht haben?
\end{exe}	
Neben diesen fünf Hypothesen, die auf verschiedene interpretatorische Aspekte Bezug nehmen, formuliert \citet[289]{Thurmair1989}, \citealt[31]{Thurmair1991} schließlich die übergeordnete These, dass die letzte Partikel in einer Kombination die wichtigere sei. Diese Annahme spiegelt sich zumindest in vier der fünf konkreten Thesen wider: Am rechten Rand einer Kombination steht die spezifischere Partikel (H1), wird der Illokutionstyp festgelegt (H3), entscheidet sich der Stärkegrad der Illokution (H4) und findet die Beeinflussung des Kommunikationspartners statt (H5).

\citet{Abraham1991a} buchstabiert diese Generalisierung von Thurmair im Rahmen eines generativ-syntaktischen Zugangs aus. Er nimmt eine Satzstruktur wie in (\ref{53}) an. Wie dem Strukturbaum zu entnehmen ist, gibt es drei Positions\-möglichkeiten für MPn: Sie können inkorporiert ins Verb, an einer $\textrm{V}^{\prime}$-Ebene oder adjungiert an IP auftreten.

\begin{exe}
	\ex\label{53}   
\begin{jtree}
\! = {CP}
:({Spec CP} <vert>{Er}) \jtwide {$\textrm{C}^{\prime}$}
:({COMP}<vert>{hat}) {IP}
:({MP}<vert>{\textbf{eben}}) \jtwide {IP}
:{Spec IP} {$\textrm{I}^{\prime}$}
:{$\textrm{I}^{0}$} {VP}
:{MP} {VP}
:({IO}<vert>{Vater}) {$\textrm{V}^{\prime}$}
:({MP}<vert>{\textbf{eben}}){$\textrm{V}^{\prime}$}
:({DO}<vert>{ein Auto}) \jtwide {$\textrm{V}^{0}$}
:({MP}<vert>{\textbf{eben}}) ({$\textrm{V}^{0}$}<vert>{gegeben})
.
\end{jtree}
\end{exe}
\citet{Abraham1991a} baut auf der heutzutage zum Standard gewordenen Annahme auf, dass MPn die ganze Proposition des Satzes in ihren Skopus \is{Skopus}nehmen, d.h. auf Ebene der \textit{Logischen Form} \is{Logische Form} weisen MPn Satzskopus \is{Satzskopus} auf. Ferner nimmt er an, dass der Skopus in diesem Fall dieselbe Rektionsrichtung \is{Rektionsrichtung} nimmt wie das Verb in der VP, d.h. der Skopus verläuft nach links. Unter Bezug auf diese Konzepte sieht \citet[118]{Abraham1991a} auf folgende Weise die Möglichkeit, die fünf Beschränkungen Thurmairs zu vereinen:

\begin{quotation}
[...] it is no less than plausible that the more specific MP constraints \\ govern (in its formal, syntactic sense) the less specific ones, since they must be available for the interpretation of the sentence from the very first compositional step onward. Only when the specificity constraints are equal is there linear symmetry, that is, linear interchangeability.
\end{quotation}
Wenngleich sich Abraham hier nur auf die Eigenschaft der Spezifizität bezieht (und damit so erstmal nur für Thurmairs Beschränkung 1 $[$bestenfalls noch 3 und 4$]$ aufkommt) (vgl. auch die Kritik von \citealt[229]{Rinas2006}), so lässt sich seine Überlegung dennoch auf Thurmairs eigene Generalisierung ausweiten, dass das wichtigste Element in einer Kombination am Endrand auftritt. \citet[229]{Rinas2006} stellt hier in Frage, warum die spezifischere MP in der Derivation zuerst verfügbar sein sollte. Geht man aber hier allgemeiner von der frühen Verfügbarkeit der wichtigeren MP (was auf die spezifischere in diesem Fall ebenfalls zu\-treffen würde) aus, ist Abrahams Überlegung nachvollziehbar, wenn man sie so auslegt, dass die wichtigere MP früh im Strukturaufbau eingefügt wird und die andere MP (im syntaktischen Sinne) regiert. D.h. beispielsweise unter Bezug auf Thurmairs Beschränkung 3 würde früh in der Derivation festgelegt, welcher Illokutionstyp vorliegt, da die letzte MP einer Kombination in einer bottom-up-Derivation zuerst in die Struktur gelangen würde. Die MP, die diese Information einbringt, kann dann eine weitere MP zu sich nehmen, die mit dieser Eigenschaft (des Satzes) kompatibel ist. Oder in Bezug auf Beschränkung 4 ist es durchaus plausibel, anzunehmen, dass der Stärkegrad der Illokution früh im Satzbau festgelegt ist und die MP, die diese Eigenschaft festlegt, anschließend weitere (hinsichtlich dieser Information weniger konkrete, aber mit ihr kompatible) MPn zu sich neh\-men kann.

\citet[118]{Abraham1991a} schätzt seinen Ansatz Thurmairs Generalisierungen gegen\-über als überlegen ein:
\begin{quotation}
If we accept that a formal (not 'formalized'!) account is to be preferred over a pragmatic one (disregarding its vagueness) and any account is to be preferred that connects to other established accounts, the explanation of the observational facts would have preferences over Thurmair’s.
\end{quotation}
Die von \citet{Thurmair1989, Thurmair1991} formulierten Beschränkungen stellen einen ersten Vorschlag für eine Erklärung der (relativ festen) Abfolgen in MP-Kombinationen dar. Wie Thurmairs eigene Generalisierung über ihre fünf Prinzipien sowie die Uniformierungsabsichten der einzelnen Beschränkungen von \citet{Abraham1991a} zeigen, stellt sich ausgehend von Einzelbeobachtungen wie Thurmair sie macht, weiter die Frage nach (abstrakteren) Kriterien, über die sich die Reihungen erfassen lassen. Dies geht in der Forschung zum Thema der MP-Kombinationen mit theoretischeren Zugängen einher. Ein Ansatz, der mit der \textit{assertiven Kraft} ein solches Kriterium vorschlägt, ist \citet{Doherty1985, Doherty1987}.

\subsection{Assertive Kraft}
\label{sec:ass}
\subsubsection{Doherty (1985)}
\label{sec:doh85}
\citet{Doherty1985} beschäftigt sich mit der Abfolge der MPn \textit{ja}, \textit{doch} und \textit{wohl} in Kombination. Das Kriterium, das die Abfolge dieser MPn in MP-Sequenzen vorgibt, ist in (\ref{54}) formuliert. \is{assertive Kraft}

\begin{exe}
	\ex\label{54} 
		Die Partikel mit der größeren assertiven Stärke muss der Partikel mit der geringeren assertiven Stärke übergeordnet sein.   
			\hfill\hbox {\citet[83]{Doherty1985}}
\end{exe}
Voraussetzung für die Illustration der Wirkungsweise dieses Kriteriums ist die Einführung einiger Grundannahmen der sehr komplexen Theorie Dohertys über die sprachliche Kodierung und Interaktion verschiedener Typen von \textit{Einstellungen}. \is{Einstellung}\\

\noindent
\textbf{Hintergrundannahmen}\\
Generell geht es in Dohertys Arbeit um den Ausdruck von Einstellungen durch sprachliche Mittel wie z.B. \is{Matrixverb} Matrixverben, Satzadverbien, MPn \is{Satzadverb} oder \is{Modalverb} Modalverben. Sie spricht hier von \textit{positionaler Bedeutung} \is{positionale Bedeutung}. Die Mittel, die des Ausdrucks dieser Bedeutung dienen, sind \textit{positionale Ausdrucksmittel} \is{positionale Ausdrucksmittel}. In (\ref{55}) wird die Einstellung (E) der Vermutung ausgedrückt – in  (\ref{55a}) satzartig, in (\ref{55b}) nicht satzartig.

\begin{exe}
	\ex\label{55} 
		\begin{xlist}	
			\ex\label{55a} Miriam \textbf{\textit{vermutet}}, dass der Gasableser vormittags kommt. 
			\ex\label{55b} Der Gasableser kommt \textbf{\textit{vermutlich}} vormittags. \hfill\hbox{(E: Vermutung)}
		\end{xlist}
\end{exe}
Neben solchen spezifischen Einstellungen nimmt Doherty auch \glq basalere\grq {} Einstellungen  an: eine positive Haltung (Bestätigung) bzw. negative Haltung (Ableh\-nung) gegenüber einer Proposition oder einer weiteren Einstellung. Diese Arten von positionalen Einstellungen kommen in der affirmativen bzw. negativen Variante eines Satzes (vgl. (\ref{56})) zum Ausdruck.

\begin{exe}
	\ex\label{56} 
		\begin{xlist}	
			\ex\label{56a} Ulrich ist müde. $(\textrm{E: pos}_{\textrm{S}}(\textrm{p}))$
			\ex\label{56b} Ulrich ist nicht müde. $(\textrm{E: neg}_{\textrm{S}}(\textrm{p}))$
		\end{xlist}
\end{exe}
Von derartigen Einstellungen unterscheidet Doherty den \textit{Einstellungsmodus} \is{Einstellungsmodus} (EM), der die Werte \textit{assertiv} und \textit{nicht-assertiv} annehmen kann. Im assertiven Fall bestätigt der Sprecher die Richtigkeit einer Einstellung E zu einer Proposition p. Im nicht-assertiven Fall bestätigt der Sprecher die Richtigkeit einer Einstellung E zu einer Proposition p nicht und lässt die gegenteilige Einstellung zu. Der Einstellungsmodus kann lexikalisch (vgl. (\ref{57})), syntaktisch (vgl. (\ref{58})) und phonologisch (vgl. (\ref{59})) realisiert werden.

\begin{exe}
	\ex\label{57} 
		\begin{xlist}	
			\ex\label{57a} Stefan vermutet, dass der Trockner geliefert wurde. (EM: assertiv)
			\ex\label{57b} Guste fragt, ob der Trockner geliefert wurde. (EM: nicht-assertiv)
		\end{xlist}
\end{exe}

\begin{exe}
	\ex\label{58} 
		\begin{xlist}	
			\ex\label{58a} Nummer 5 ist das schönste Haus in der Straße. (EM: assertiv)
			\ex\label{58b} Ist Nummer 5 das schönste Haus in der Straße? (EM: nicht-assertiv)
		\end{xlist}
\end{exe}	 
			
\begin{exe}
	\ex\label{59} 
		\begin{xlist}	
			\ex\label{59a} In Tallinn sind minus 19 Grad. (EM: assertiv)
			\ex\label{59b} In Tallinn sind minus 19 Grad? (EM: nicht-assertiv)	
		\end{xlist}
\end{exe}
Das Beispiel in (\ref{60}) dient der weiteren Illustration der bis hierhin eingeführten Komponenten und Notationsweisen. Wie die Notation zeigt, fasst Doherty Einstellungen und Einstellungsmodi als Prädikate auf, die Argumente zu sich nehmen. Die Hierarchisierung entspricht stets: EM > E > p. Die Einstellung E benötigt einen Einstellungsträger. In komplexen Sätzen ist dies das Matrixsubjekt, in selbständigen Sätzen ist es in der Regel der Sprecher. 

\begin{exe}
	\ex\label{60} 
		\begin{xlist}	
			\ex\label{60a} Stefan ist müde.
			\ex\label{60b} Ass$(\textrm{pos}_{\textrm{S}}$(\textrm{p})$)$	
			\ex\label{60c} Der Sprecher bestätigt (Ass), dass er eine positive Einstellung $(\textrm{pos}_{\textrm{S}}$) zum durch die Proposition 				p (= Stefan ist müde) ausgedrückten Sachverhalt hat.
		\end{xlist}
\end{exe}
Innerhalb der positionalen Bedeutung unterscheidet Doherty weiter zwischen \textit{propositionaler} \is{propositionale Bedeutung} und \textit{nicht-propositionaler} Bedeutung \is{nicht-propositionale Bedeutung} sowie zwischen \textit{wörtlicher} \is{wörtliche Bedeutung} und \is{implizite Bedeutung} \textit{impliziter} Bedeutung. Unter die propositionale Bedeutung fasst sie die positionale Bedeutung, die durch satzwertige positionale Ausdrucksmittel wie Matrixverben + Komplementsatz kodiert wird (vgl. z.B. (\ref{55a})). Zur nicht-propositio\-nalen Bedeutung gehört die durch Satzadverbien (vgl. (\ref{55b}), MPn, affirmative/ne\-gierte Satzform (vgl. (\ref{56})) und die Satzintonation vermittelte positionale Bedeutung. Im Rahmen (eines Aspektes) von Dohertys späterer Ableitung der Beschrän\-kungen von MPn in Kombinationen ist der folgende Unterschied zwischen der propositionalen und nicht-propositionalen Bedeutung relevant: Wie (\ref{61}) illustriert, kann nur einer propositional positionalen Einstellung eine weitere Einstellung entgegengebracht werden.
\begin{exe}
	\ex\label{61} 
		\begin{xlist}	
			\ex\label{61a} Paul \textbf{\textit{vermutet}} nicht, dass Maria gekocht hat. 
			\ex\label{61b} *Maria hat nicht \textbf{\textit{vermutlich}} gekocht.	
		\end{xlist}
\end{exe}
Neben den bis hierher angeführten Einstellungen, die wörtliche Bedeutungsanteile der Sätze darstellen, nimmt sie auch indirekte, d.h. implizierte Einstellungen an, denen sie den Status von \textit{konventionellen Implikaturen} \is{konventionelle Implikatur} zuschreibt. Eine implizierte Einstellung liegt beispielsweise in (\ref{62}) vor.

\begin{exe}
	\ex\label{62} 
		\begin{xlist}	
			\ex\label{62a} \{-ade\} ist ein Morphem?
			\ex\label{62b} Nicht-Ass$(\textrm{pos}_{\textrm{X}}(\textrm{p})) \textrm{und} \ \textrm{IM(neg}_{\textrm{S}}(\textrm{p}))$
		\end{xlist}
\end{exe}
Syntaktisch liegt ein affirmativer Deklarativsatz vor, der normalerweise asserti\-ven EM aufweist. Phonologisch wird durch die steigende Satzintonation der nicht-assertive EM zugeordnet. Die Bedeutungszuschreibung in (\ref{62b}) beinhaltet eine wört\-liche Einstellung (Der Sprecher bestätigt nicht, dass jemand eine positive Haltung gegenüber p einnimmt.) und eine implizierte Einstellung (Der Sprecher nimmt in Bezug auf p eine negative Haltung ein.). 

Bevor Dohertys Ableitung der Abfolgebeschränkungen von \textit{ja}, \textit{doch} und \textit{wohl} schließlich skizziert werden kann, ist – wie in allen Ansätzen, die auf der Interpretation der MPn basieren, – ein Blick auf ihre Modellierung der Einzelbedeutungen nötig.\\

\noindent
\textbf{Einzelbeschreibungen der MPn \textit{ja}, \textit{doch} und \textit{wohl}}\\
Dohertys Bedeutungszuschreibung an die MP \textit{ja} findet sich in (\ref{63}).

\begin{exe}
	\ex\label{63} 
		\textit{ja}: Ass$(\textrm{E}_{\textrm{S}}$(\textrm{p})$)$ und I\textrm{M}$(\textrm{E}_{\textrm{X}}$(\textrm{p})$)$
		\hfill\hbox {\citet[80]{Doherty1985}}
\end{exe}
(\ref{64b}) illustriert eine Ausbuchstabierung dieses allgemeinen Formats unter Bezug auf die Äußerung in (\ref{64a}).

\begin{exe}
	\ex\label{64} 
		\begin{xlist}	
			\ex\label{64a} Im März sind \textbf{ja} noch Semesterferien.
			\ex\label{64b} Ass$\textrm{(pos}_{\textrm{S}}$(\textrm{p})$)$ und IM$(\textrm{pos}_{\textrm{X}}$(\textrm{p})$)$
		\end{xlist}
\end{exe}
Wörtlich bestätigt der Sprecher (Ass) seine positive Einstellung $(\textrm{pos}_{\textrm{S}})$ hinsichtlich der ausgedrückten Proposition p (= Im März sind noch Semesterferien). Zu\-sätzlich wird durch eine konventionelle Implikatur die indirekte Einstellung ausgedrückt, dass eine andere Person p ebenfalls die vom Sprecher vertretene wörtliche Einstellung entgegenbringt $(\textrm{pos}_{\textrm{X}}$(\textrm{p})$)$. Festzuhalten ist an dieser Stelle, dass die Partikel den Sprecher nicht nur auf E $\textrm{(pos}_{\textrm{S}}$(\textrm{p})$)$ festlegt, sondern auch auf den assertiven EM (Ass). Diese Tatsache trägt entscheidend zu Dohertys Beschränkung über Kombinationen bei (s.u.).

Für \textit{doch} formuliert Doherty die Bedeutung in (\ref{65}). Der Sprecher legt sich \underline{potenziell} auf die Einstellung E fest, womit die Autorin die Tatsache auffängt, dass eine \textit{doch}-Äußerung assertiv interpretiert werden \underline{kann}.

\begin{exe}
	\ex\label{65} 
	\textit{doch}: $\textrm{Ass}^{\prime} (\textrm{E}_{\textrm{S}}(\textrm{p})) \textrm{und IM(neg}_{\textrm{X}}(\textrm{p}))$	
	\hfill\hbox {\citet[71]{Doherty1985}}
\end{exe}
Im affirmativen Deklarativsatz in (\ref{66a}) ist dies auch der Fall: Der Sprecher bestätigt seine positive Einstellung zu p. Zusätzlich wird implizit ausgedrückt, dass jemand anders p negativ bewertet. 

\begin{exe}
	\ex\label{66} 
		\begin{xlist}	
			\ex\label{66a} Das Semester beginnt \textbf{doch} vor Ostern.
			\ex\label{66b} Ass$\textrm{(pos}_{\textrm{S}}(\textrm{p})) \textrm{und IM(neg}_{\textrm{X}}\textrm{(p))}$
		\end{xlist}
\end{exe}
Die Option des möglichen nicht-assertiven Einstellungsmodus ist relevant, weil \textit{doch} auch in einer \textit{Sekundärfrage} \is{Sekundärfrage} (vgl. \citealt[68]{Doherty1985}) wie in (\ref{67a}) auftreten kann. Doherty nimmt die Bedeutungszuschreibung in (\ref{67b}) an.
\begin{exe}
	\ex\label{67} 
		\begin{xlist}	
			\ex\label{67a} Das Semester beginnt \textbf{doch} vor Ostern?
			\ex\label{67b} Nicht-Ass$\textrm{(pos}_{\textrm{X}}$(\textrm{p})$)$ \textrm{und IM}$(\textrm{pos}_{\textrm{S}}$(\textrm{p})$)$
		\end{xlist}
\end{exe}
Wörtlich wird der nicht-assertive EM kodiert, d.h. der Sprecher bestätigt nicht, dass jemand p eine positive Einstellung entgegen bringt. Die Sprecher\-einstellung (hier die positive Haltung gegenüber p) wird in dieser Konstruktion impliziert. 

Im Gegensatz zur Bedeutungsbeschreibung bei der MP \textit{ja} wird der Sprecher im Fall von \textit{doch} nicht auf die Assertion von E festgelegt, sondern schwächer nur auf E (vgl. (\ref{65})). Der EM kann aufgrund der nur potenziellen Assertion assertiv (vgl. (\ref{66})) oder nicht-assertiv (vgl. (\ref{67})) sein.

Für \textit{wohl} setzt Doherty die Bedeutung in (\ref{68}) an.

\begin{exe}
	\ex\label{68} 
	$\textrm{Pres}(\alpha (\textrm{p})) \ \textrm{und}\ \textrm{Ass}^{\prime} \textrm{(VERMUTUNG}_{\textrm{S}}(\alpha = \textrm{E}))$
	\hfill\hbox{\citet[82]{Doherty1985}}
\end{exe}
Die potenzielle Assertion in der Bedeutungszuschreibung erklärt sich in Analogie zu dieser Annahme bei \textit{doch}: \textit{wohl} kann sowohl in Aussagen mit assertivem EM wie in (\ref{69a}) als auch in Sekundärfragen wie in (\ref{70a}) auftreten.

\begin{exe}
	\ex\label{69} 
		\begin{xlist}	
			\ex\label{69a} Das Semester beginnt \textbf{wohl} vor Ostern.
			\ex\label{69b} $\textrm{Pres}(\alpha(\textrm{p})) \ \textrm{und Ass (VERMUTUNG}_{\textrm{S}}(\alpha = \textrm{pos}))$
		\end{xlist}
\end{exe}

\begin{exe}
	\ex\label{70} 
		\begin{xlist}	
			\ex\label{70a} Das Semester beginnt \textbf{wohl} vor Ostern?
			\ex\label{70b} $\textrm{Pres}(\alpha(\textrm{p})) \ \textrm{und nicht-Ass(pos}_{\textrm{X}}(\textrm{p})) \  \textrm{und IM(VERMUTUNG}_{\textrm{S}}(\alpha = \textrm{pos}))$
		\end{xlist}
\end{exe}
Da sich auch Daten wie (\ref{71}) finden, in denen die Vermutung offensichtlich(er) (als in (\ref{69}) und (\ref{70})) der durch \textit{wirklich} kodierten Einstellung gilt, geht auch in die Bedeutungsbeschreibungen in (\ref{69b}) und (\ref{70b}) ein, dass die Vermutung der positiven Einstellung gilt ($\alpha$ = \textrm{pos}).

\begin{exe}
	\ex\label{71} 
	Das Semester beginnt \textbf{wohl} \textbf{\textit{wirklich}} vor Ostern.\\
	Ass > Vermutung > (Wirklich > p)	
\end{exe}
Die Einstellung der Vermutung lässt sich noch genauer fassen, indem man sie als eine \textit{opake} \is{opake Einstellung} (im Gegensatz zur \textit{transparenten}) \is{transparente Einstellung} Einstellung auffasst. Der Unterschied zwischen diesen beiden Typen von Einstellungen lässt sich illustrieren unter Bezug auf die Beispiele in (\ref{72}), in denen mit \textit{wirklich} ein Satzadverb vorliegt, für das man annimmt, dass es eine transparente Einstellung zum Ausdruck bringt, und mit \textit{vermutlich} ein Satzadverb, das eine opake Einstellung kodiert.

\begin{exe}
	\ex\label{72} 
		\begin{xlist}	
			\ex\label{72a} Bielefeld ist \textbf{\textit{wirklich}} besser als sein Ruf.
			\ex\label{72b} Bielefeld ist \textbf{\textit{vermutlich}} besser als sein Ruf.
		\end{xlist}
\end{exe}	
Ob transparente und opake Einstellungen vorliegen, zeigt sich daran, ob der Sachverhalt, dem die Einstellung entgegengebracht wird, im Fall der Bestätigung der Einstellung (es ist Wirklichkeit, dass Bielefeld besser als sein Ruf ist vs. es ist eine Vermutung, dass Bielefeld besser als sein Ruf ist) ebenfalls bestätigt wird oder nicht. In (\ref{72}) gilt, dass das Bestehen des Sachverhalts ausgedrückt wird, wenn bestätigt wird, dass der Sachverhalt als wirklich ausgegeben wird. Der Sachverhalt wird jedoch nur in einem eingeschränkten Sinn bestätigt, wenn die Einstellung der Vermutung bestätigt wird. 

Diese typischerweise bei Satzadverbien auftretende Unterscheidung lässt sich auch auf MPn übertragen: Die MP \textit{doch} kann in diesem Sinne als transparent, \textit{wohl} als opak eingestuft werden. Für (\ref{73a}) gilt, dass ein Sprecher mit der Bestätigung der positiven Einstellung gegenüber p auch p selbst bestätigt. Letzteres tritt in (\ref{73b}) nur eingeschränkt ein.
\begin{exe}
	\ex\label{73} 
		\begin{xlist}	
			\ex\label{73a} Astrid kommt \textbf{doch} mit dem Zug.
			\ex\label{73b} Astrid kommt \textbf{wohl} mit dem Zug.
		\end{xlist}
\end{exe}
Den Beitrag der MP \textit{wohl} fasst Doherty deshalb als opake Sprechereinstellung auf, die der Richtigkeit der Einstellung im Skopus der Partikel gilt. In (\ref{73b}) ist dies die positive Haltung gegenüber p (vgl. (\ref{74a})), in (\ref{71}) die Wirklichkeit von p (vgl. (\ref{74b})).
\begin{exe}
	\ex\label{74} 
		\begin{xlist}	
			\ex\label{74a} $\textrm{Pres}(\alpha(\textrm{p})) \ \textrm{und Ass(VERMUTUNG}_{\textrm{S}}(\alpha = \textrm{pos}))$	(s. (\ref{73b}))
			\ex\label{74b} $\textrm{Pres}(\alpha(\textrm{p})) \ \textrm{und Ass(VERMUTUNG}_{\textrm{S}}(\alpha = \textrm{WIRKLICH}))$\footnote{Die Erläuterungen zu der von \citet{Doherty1985} angenommenen Bedeutungsbeschreibung von \textit{wohl} haben bis hierhin den linken Ausdruck $\textrm{Pres}(\alpha(\textrm{p}))$ (für \textit{Präsupposition}) sowie die Variable $\alpha$ im rechten Teil der Beschreibung ausgeklammert. Doherty fängt über diese Komponenten die besondere Eigenschaft des opaken \textit{wohl} auf, andere transparente Einstellungen ohne kontrastive Lesart in seinen Skopus zu nehmen (vgl. S. 40-54). Bei Dohertys Modell handelt es sich um ein komplexes System, dem man im Rahmen einer kurzen Skizze nicht gerecht werden kann. Da die Rolle dieser Bedeutungsanteile und ihre Modellierung als \textit{Propositionalisierungseffekt} für Dohertys Ableitung der Beschränkungen der MP-Kombinationen nicht entscheidend sind, soll an dieser Stelle ein Verweis auf die Ausführungen der Autorin zu diesen Aspekten ausreichen (vgl. \citealt[81-82]{Doherty1985}). Relevant ist für die Diskussion der MP-Kombinationen die Annahme, dass mit \textit{wohl} ggf. (d.h. unter Vorliegen des assertiven EM) die opake Einstellung gegenüber einer anderen Einstellung bestätigt wird.} (s. (\ref{71}))
\end{xlist}
\end{exe}

\noindent
\textbf{Modalpartikel-Abfolgen: Grade assertiver Stärke}\\
Nach der Einführung einiger Grundkonzepte aus Dohertys Modell und ihrer Auffassung zu den Einzelbeschreibungen der MPn (die auch für die Ausführungen in Abschnitt~\ref{subsec:input} Voraussetzung sind) ist es nun möglich, ihre Ableitung der Abfolgebeschränkung der Partikeln \textit{ja}, \textit{doch} und \textit{wohl} (vgl. (\ref{75}) bis (\ref{77})) zu erläutern.

\begin{exe}
	\ex\label{75} 
	Konrad ist \textbf{ja doch}/\textbf{*doch ja} verreist.		
\end{exe}
\vspace{-0.65cm}
\begin{exe}
	\ex\label{76} 
	Konrad ist \textbf{ja wohl}/\textbf{*wohl ja} verreist.	
\end{exe}
\vspace{-0.65cm}
\begin{exe}
	\ex\label{77} 
	Konrad ist \textbf{doch wohl}/\textbf{*wohl doch} verreist.
	\hfill\hbox {\citet[83]{Doherty1985}}
\end{exe}
Die Generalisierung der Autorin zur Erfassung der (un)zulässigen Abfolgen ist in (\ref{78}) erneut notiert.

\begin{exe}
	\ex\label{78} 
	Die Partikel mit der größeren assertiven Stärke muss der Partikel mit der geringeren assertiven Stärke übergeordnet sein.	
	\hfill\hbox {\citet[83]{Doherty1985}}
\end{exe}
Dass die in (\ref{75}) bis (\ref{77}) zu beobachtende (relative) Ordnung von \textit{ja} > \textit{doch} > \textit{wohl} der Generalisierung aus (\ref{78}) entspricht, zeigt sich, wenn man erneut einen Blick auf die von Doherty angenommenen Einzelbeschreibungen wirft:

\begin{exe}
	\ex\label{79} 
	\textit{ja}: $\textrm{Ass(E}_{\textrm{S}}(\textrm{p})) \ \textrm{und IM(E}_{\textrm{X}}(\textrm{p}))$
\end{exe}
\vspace{-0.65cm}	
\begin{exe}
	\ex\label{80} 
	\textit{doch}: $\textrm{Ass}^{\prime}(\textrm{E}_{\textrm{S}}(\textrm{p})) \  \textrm{und IM(neg}_{\textrm{X}}(\textrm{p}))$
\end{exe}	
\vspace{-0.65cm}	
\begin{exe}
	\ex\label{81} 
	\textit{wohl}: $\textrm{Pres}(\alpha(\textrm{p})) \  \textrm{und Ass}^{\prime}(\textrm{VERMUTUNG}_{\textrm{S}}(\alpha=\textrm{E}))$
\end{exe}	

Die Sprecherhaltungen, die durch die MPn jeweils bestimmt werden, weisen unterschiedliche Grade der Verbindlichkeit auf: Mit der Verwendung von \textit{ja} wird der Sprecher auf die assertive Haltung zur Einstellung im Skopus der MP festgelegt (Ass(E)). Im Falle von \textit{doch} legt er sich auf die Einstellung im Skopus der Partikel fest (E) (weil der EM je nach Auftretensweise assertiv oder nicht-assertiv sein kann). Durch die Verwendung von \textit{wohl} wird der Sprecher auf eine opake Einstellung zur Einstellung im Skopus der Partikel festgelegt $(\textrm{VERMU\-TUNG}_{\textrm{S}}$(\textrm{E})$)$. D.h. von \textit{ja} zu \textit{doch} zu \textit{wohl} nimmt der Sprecher entlang dieser Bedeutungsmodellierungen eine immer weniger verbindliche Haltung ein. Diese unterschiedlichen Grade der Verbindlichkeit der Sprecherhaltung wie sie durch MPn vermittelt werden, bezeichnet Doherty als \textit{assertive Stärke} \is{assertive Stärke} (\citeyear[83]{Doherty1985}) der MPn. Die Skala in (\ref{81}), die die relativen Ordnungen aus (\ref{75}) bis (\ref{77})) abbildet, drückt gleichzeitig (absteigend) die Verhältnisse hinsichtlich des den MPn zuzu\-schreibenden Grades assertiver Stärke aus.

\begin{exe}
	\ex\label{82} 
	$\underrightarrow{\text{\textit{ja} > \textit{doch} > \textit{wohl}}}$\\
	abnehmende assertive Stärke
\end{exe}	
Die Relevanz der assertiven Stärke der an Kombinationen beteiligten MPn für ihre Anordnung vertritt die Autorin auch in \citet{Doherty1987}.

\subsubsection{Doherty (1987)}
Diese Arbeit ist die überarbeitete und erweiterte Version von \citet{Doherty1985}. Da die semantischen Beschreibungen der MPn von den Bedeutungszuschreibungen in \citet{Doherty1985} abweichen, ergeben sich die unterschiedlichen Grade assertiver Stärke auf etwas andere Art als in der Vorgängerversion.\\

\noindent
\textbf{Hintergrundannahmen}\\
Wie in der Skizzierung der Ableitung von \citet{Doherty1985} sind vorweg einige Grundannahmen der Modellierung von Einstellungen sowie ihre Annahmen zur Bedeutung der Einzelpartikeln anzuführen. Entsprechend der Dreiteilung aus \citet{Doherty1985} in \textit{Einstellungsmodus} (EM), \textit{Einstellung} (E) und \textit{Proposition} (p) unterscheidet die Autorin zwischen \textit{attitudinal mood} \is{attitudinal mood} (AM), \textit{attitude} \is{attitude} (A) und \textit{proposition} (p). Die propositionale Bedeutung ist der Teil der Satzbedeutung, der \textit{evaluierbar} \is{evaluierbar} ist, d.h. z.B. eine Bewertung hinsichtlich wahr und falsch zulässt. Wenn Sätze sich hinsichtlich ihrer propositionalen Bedeutung voneinander unterscheiden, werden verschiedene Sachverhalte ausgedrückt. Die anderen beiden Typen von Bedeutung sind non-propositional und evaluieren selbst. AM und A sind zwei verschiedene Arten von Einstellungsbedeutungen. AM wird durch Wortstellung, Intonation oder Konnektoren realisiert. A wird durch Adverbiale oder Hauptsatzverben kodiert. Entlang dieser Dreiteilung unterscheiden sich die Sätze in (\ref{83}) in ihrer propositionalen Bedeutung, die Sätze in (\ref{84}) und (\ref{85}) hinsichtlich ihrer Einstellungsbedeutung. In (\ref{84}) bleibt AM konstant bei wechselndem A. In (\ref{85}) variiert AM bei gleich bleibendem A.

\begin{exe}
	\ex\label{83} 
		\begin{xlist}	
			\ex\label{83a} Conrad is \textbf{\textit{really}} abroad.
			\ex\label{83b} Lily is \textbf{\textit{really}} abroad.
			\ex\label{83c} Conrad is \textbf{\textit{really}} at home.
		\end{xlist}
\end{exe}

\begin{exe}
	\ex\label{84} 
		\begin{xlist}	
			\ex\label{84a} Conrad is \textbf{\textit{really}} at home. (AM: Ass, A: Wirklichkeit)
			\ex\label{84b} Conrad is \textbf{\textit{probably}} at home. (AM: Ass, A: Wahrscheinlichkeit)
			\ex\label{84c} I \textbf{\textit{suppose}} Conrad is at home. (AM: Ass, A: Vermutung)
		\end{xlist}
\end{exe}

\begin{exe}
	\ex\label{85} 
		\begin{xlist}	
			\ex\label{85a} Conrad is \textbf{\textit{really}} abroad. (AM: Ass, A: Wirklichkeit)
			\ex\label{85b} Conrad is \textbf{\textit{really}} abroad? (AM: O, A: Wirklichkeit)
			\hfill\hbox {\citet[14]{Doherty1987}}
		\end{xlist}
\end{exe}				           
Diese drei Dimensionen sind in jedem Satz obligatorisch vorhanden. Sie sind hie\-rarchisch geordnet entlang der Ordnung AM > A > p. A bildet dabei die \textit{propositionale Bedeutung} \is{propositionale Bedeutung} auf eine \textit{semievaluier\-te propositionale Bedeutung} \is{semievaluiert} ab. Dieser Schritt produziert nach Doherty einen \textit{potenziellen Gedanken} und involviert, dass eine Einstellungsbeziehung zwischen einem unspezifizierten Subjekt k und der propositionalen Bedeutung p hergestellt wird (A(k,p)). AM bildet eine \underline{semi}evalu\-ierte propositionale Bedeutung auf eine \underline{komplett} evaluierte propositionale Bedeutung ab. Der potenzielle Gedanke wird vervollständigt, indem er an ein Subjekt gebunden wird. Der Unterschied zwischen semievaluierter und komplett evaluierter propositionaler Bedeutung lässt sich anhand des Unterschieds zwischen (83) und (84) illustrieren.

\begin{exe}
	\ex\label{86} 
		(\ldots) Conrad has \textbf{\textit{really}} left.
\end{exe}
\vspace{-0.65cm}
\begin{exe}
	\ex\label{87} 
		\begin{xlist}	
			\ex\label{87a} Conrad has \textbf{\textit{really}} left.
			\ex\label{87b} Conrad has \textbf{\textit{really}} left?
		\end{xlist}
\end{exe}
In allen Sätzen liegt eine Proposition vor (Konrad ist abgereist.) und auch eine Einstellung (Wirklichkeit). D.h. eine Bewertung von p liegt zwar sowohl in (\ref{86}) als auch in (\ref{87}) vor, der bewertende Teil kann in (\ref{86}) aber noch auf verschiedene Art komplettiert werden, z.B. durch Konnektoren (z.B.\textit{ that}, \textit{whether}, \textit{because}, \textit{if}) im Falle eines abhängigen Satzes oder indem die Lücke frei bleibt und ein selbständiger Satz (wie in (\ref{87})) resultiert. In (\ref{86}) liegt zwar eine bestimmte Einstellung vor, es ist jedoch offen, wer der Träger der Einstellung ist. Je nach Füllung kann es der Sprecher oder das Matrixsubjekt sein. In diesem Sinne liegt in (\ref{86}) eine semievaluierte propositionale Bedeutung vor (A hat auf p bereits appliziert), auf die AM angewendet wird, und dann eine komplett evaluierte propositionale Bedeutung ausgibt, die in (\ref{87}) bereits vorliegt.

Die Unterscheidung zwischen diesen drei Ebenen und die unterschiedliche Natur der Strukturen, die die Applikationsgrundlage der nächst höheren Ebene darstellen, ist für die Betrachtung von MPn von daher relevant, als dass MPn semievaluierte Propositionen und nicht \glq reine\grq {} Propositionen als ihre Argumente nehmen. Semievaluierte Propositionen sind hinsichtlich des Evaluierungsprozes\-ses \glq komplettere\grq {} Propositionen als \glq einfache\grq {} Propositionen (s.o.). Dieser Status der Argumente von MPn zeigt sich daran, dass Satzadverbien in ihrem Skopus stehen können. MPn können folglich Ausdrücke in ihren Skopus nehmen, denen schon eine Einstellung entgegengebracht wird (vgl. (\ref{88})). D.h. es liegen bereits evaluierte Propositionen im Skopus der Partikeln vor, die noch nicht von AM komplettierend bewertet wurden.

\begin{exe}
	\ex\label{88} 
	Konrad ist \textbf{ja}/\textbf{doch} \textbf{\textit{wahrscheinlich}} verreist.
\end{exe}
\noindent
\textbf{Einzelbeschreibungen der Modalpartikeln \textit{ja}, \textit{doch} und \textit{wohl}}\\
Die Einzelbedeutungen der MPn modelliert Doherty je derart, dass diese eine implizite Evaluation in Bezug auf A vornehmen.

Tritt \textit{ja} in einem Deklarativsatz \is{Deklarativsatz} auf wie in (\ref{89a}), bezieht der Satz eine explizite Evaluierung (hier die positive Evaluierung von p) auf eine implizite: Der Sprecher impliziert, dass es möglich ist, dass der Hörer um den vom Sprecher assertierten Inhalt bereits weiß.

\begin{exe}
	\ex\label{89} 
		\begin{xlist}	
			\ex\label{89a} Konrad ist \textbf{ja} verreist.	
			\ex\label{89b} \textit{ja}:	$\textrm{IM}(\Diamond \textrm{KNOW}_{\textrm{h}}(\textrm{POS(p)}))$	
			\hfill\hbox {\citet[101]{Doherty1987}}
		\end{xlist}
\end{exe}
Tritt ein Satzadverb \is{Satzadverb} auf, das eine Einstellung ausdrückt wie in (\ref{90a}), gilt das dem Hörer potenziell zugeschriebene Wissen der Wahrscheinlichkeit des Sachverhalts (vgl. (\ref{90b}). Die explizite Evaluierung sieht in diesem Falle so aus, dass der Sprecher p für wahrscheinlich hält.

\begin{exe}
	\ex\label{90} 
		\begin{xlist}	
			\ex\label{90a} Konrad ist \textbf{ja \textit{wahrscheinlic}h} verreist.	
			\ex\label{90b} $\textrm{IM}(\Diamond \textrm{KNOW}_{\textrm{h}}(\textrm{PROB(p)}))$
			\hfill\hbox {\citet[101]{Doherty1987}}
		\end{xlist}
\end{exe}
Die Partikel \textit{doch} führt nach Dohertys Modellierung zu der expliziten Bewertung eine alternative Bewertung ein. Tritt \textit{doch} in einem assertiven Deklarativsatz \is{Deklarativsatz} auf wie in (\ref{91a}), bewertet der Sprecher p selbst explizit positiv ($\textrm{Ass(POS}_{\textrm{S}})$). 

Weiter nimmt er an, dass die Bewertung des Hörers seiner eigenen möglicherweise entgegengesetzt ist $(\textrm{IM(POS}_{\textrm{h}}(\textrm{p}) \lor \sim\textrm{POS}_{\textrm{h}}(\textrm{p})))$.

\begin{exe}
	\ex\label{91} 
		\begin{xlist}	
			\ex\label{91a} Konrad ist \textbf{doch} verreist.
			\ex\label{91b} \textit{doch}: $\textrm{IM(POS}_{\textrm{h}}(\textrm{p})  \lor \sim\textrm{POS}_{\textrm{h}}(\textrm{p}))$
			\hfill\hbox {\citet[106]{Doherty1987}}
		\end{xlist}
\end{exe}
In Sekundärfragen \is{Sekundärfrage} wie in (\ref{92a}) liegt eine explizite Bewertung durch den Sprecher vor ($\textrm{O(POS}_{\textrm{S}})$) sowie eine implizite Bewertung der gleichen semievaluierten propositionalen Bedeutung (in (\ref{92a}) $\textrm{IM(POS}_{\textrm{h}}(\textrm{p}))$).

\begin{exe}
	\ex\label{92} 
		\begin{xlist}	
			\ex\label{92a} Konrad ist \textbf{doch} verreist?
			\ex\label{92b} $\textrm{IM(POS}_{\textrm{h}}(\textrm{p}))$
			\hfill\hbox {\citet[107]{Doherty1987}}
		\end{xlist}
\end{exe}
Für \textit{wohl} nimmt Doherty an, dass es aus einer semievaluierten propositionalen Bedeutung eine Hypothese macht. Mit (\ref{93a}) assertiert der Sprecher bei\-spielsweise, dass er es für eine Hypothese hält, dass der Sachverhalt, dass Konrad verreist ist, Realität ist (vgl. (\ref{93b})).
\begin{exe}
	\ex\label{93} 
		\begin{xlist}	
			\ex\label{93a} Konrad ist \textbf{wohl \textit{wirklich}} verreist.
			\ex\label{93b} $\textrm{ASS(H} \ \textrm{REAL}_{\textrm{S}}(\textrm{p}))$
			\hfill\hbox {\citet[112]{Doherty1987}}
		\end{xlist}
\end{exe}
In einer Sekundärfrage \is{Sekundärfrage} (vgl. (\ref{94a})) ist dieser Bedeutungsanteil impliziert (vgl. (\ref{94b})).
\begin{exe}
	\ex\label{94} 
		\begin{xlist}	
			\ex\label{94a} Konrad ist \textbf{wohl} verreist?
			\ex\label{94b} $\textrm{O(REAL}_{\textrm{S}}(\textrm{p}))  \textrm{und IM(H} \ \textrm{REAL}_{\textrm{S}}(\textrm{p}))$
			\hfill\hbox {\citet[112]{Doherty1987}}
		\end{xlist}
\end{exe}
\noindent
\textbf{Modalpartikel-Abfolgen: Grade assertiver Stärke}\\
Wie schon in \citet{Doherty1985} beabsichtigt die Autorin eine Ableitung der Abfolgebeschränkungen in (\ref{95}) bis (\ref{97}).

\begin{exe}
	\ex\label{95} 
	Konrad ist \textbf{ja doch}/\textbf{*doch ja} verreist.
\end{exe}
\vspace{-0.65cm}
\begin{exe}
	\ex\label{96} 
	Konrad ist \textbf{ja wohl}/\textbf{*wohl ja} verreist.
\end{exe}
\vspace{-0.65cm}
\begin{exe}
	\ex\label{97} 
	Konrad ist \textbf{doch wohl}/\textbf{*wohl doch }verreist.
			\hfill\hbox {\citet[114]{Doherty1987}}
\end{exe}
Sie macht dasselbe Kriterium für die (In)akzeptabilität der Reihungen verantwortlich wie in \citet{Doherty1985}. Die Partikeln sind in Kombinationen nach ihrer \textit{assertiven Kraft} \is{assertive Kraft} angeordnet: Eine Partikel mit höherer assertiver Kraft geht einer Partikel mit geringerer assertiver Kraft voran. \textit{Assertive Kraft} deutet \citet{Doherty1987} als Entfernung einer Bewertung vom Status des Wissens aus, d.h. je näher die Bewertung dem Wissen kommt, desto größer ist die assertive Kraft.

Dohertys Einzelbeschreibungen sind wiederholt in (\ref{98}).
\begin{exe}
	\ex\label{98} 
		\begin{xlist}	
			\ex\label{98a} \textit{ja}: $\textrm{IM}(\Diamond \textrm{KNOW}_{\textrm{h}}(\textrm{p}))$
			\ex\label{98b} \textit{doch}: $\textrm{IM}(\varepsilon_{\textrm{h}} \lor \sim\varepsilon_{\textrm{h}})$ (im Deklarativsatz mit Aussageintonation), IM$(\varepsilon_{\textrm{s}})$ (im Deklarativsatz mit Frageintonation)
			\ex\label{98c} \textit{wohl}: $\textrm{AM(H}\varepsilon_{\textrm{s}}$)\footnote{$\varepsilon$ steht für eine semievaluierte propositionale Bedeutung, d.h. 																			letztlich eine Einheit des Typs A(p).}			
		\end{xlist}
\end{exe}
Für die Ordnung \textit{ja} > \textit{doch} > \textit{wohl} motiviert sie nun folgendermaßen abnehmende assertive Kraft: Die höchste assertive Kraft ist \textit{ja} zuzuschreiben. Teil der Bedeutung von \textit{ja} ist, dass der Sprecher implizit eine Annahme über das Wissen des Hörers macht. Der Inhalt des angenommenen Hörerwissens gehört folglich auch zum Wissen des Sprechers (vgl. \citealt[103]{Doherty1987}). Sie nimmt konkret an, dass \textit{es ist möglich für den Hörer p} (bzw. genauer die Einstellung zu p) \textit{zu wissen} genauso wie \textit{p} (bzw. die Einstellung zu p) \textit{zu wissen} p präsupponiert. Wenn \textit{ja} also das Wissen um (die Einstellung) zur Proposition im Skopus der Partikel durch den Sprecher impliziert, liegt hier der denkbar engste Bezug zum Wissen vor. 

In einem Satz, in dem \textit{doch} auftritt, wird A – je nach AM – assertiert oder impliziert. Es besteht folglich kein Bezug zum Wissen um A, weshalb es plausibel ist, \textit{doch} im Vergleich zu \textit{ja} geringere assertive Stärke zuzuschreiben. Diese Abgrenzung zwischen Wissen um die Wahrheit von A und der Assertion von A ist  vor dem Hintergrund zu sehen, wie Doherty assertive AM verstanden wissen möchte. In Dohertys Modell handelt es sich hierbei nicht um ein illokutives Konzept im sprechakttheoretischen Sinne. Sie vertritt eine abstraktere Auffassung, die sich mit \glq Entscheidung\grq {} des Sprechers (assertive AM) vs. \glq Unentschiedenheit\grq {} des Sprechers (offene AM) umschreiben lässt (vgl. \citealt[19]{Doherty1987}). Sich als Sprecher für A zu entscheiden im Zuge einer expliziten bzw. impliziten Evaluie\-rung von A lässt plausiblerweise Distanz zu einer Evaluierung, die (die Wahrheit um) A als Wissensinhalt einstuft. A in diesem Sinne zu assertieren bzw. implizieren ist auf einer Skala, deren oberstes Ende die Einstufung einer Einstellung als Wissen ist, jedoch immer noch höher einzuordnen, als eine Hypothese (d.h. eine opake Einstellung) in Bezug auf A auszudrücken. 

Da die Autorin\textit{ wohl} genau diesen Beitrag zuschreibt, ist die Einordnung von \textit{doch} zwischen \textit{ja} und \textit{wohl} auf einer Skala der assertiven Stärke \is{assertive Stärke} plausibel nachzuvollziehen (vgl. (\ref{99})).

\begin{exe}
	\ex\label{99} 
	$\underrightarrow{\text{\textit{ja} > \textit{doch} > \textit{wohl}}}$\\
	abnehmende assertive Stärke
\end{exe}
Wie durch den von Doherty angenommenen Zusammenhang zwischen Abfolgen von MPn in Kombinationen und dem Grad der assertiven Stärke der beteiligten Einzelpartikeln vorhergesagt, ist der Verlauf der assertiven Stärke innerhalb der wohlgeformten Kombinationen in (\ref{95}) bis (\ref{97}) jeweils absteigend. Die ungrammatischen Kombinationen folgen hingegen nicht dem Prinzip abnehmender assertiver Stärke vom linken zum rechten Sequenzrand.

Mit dem Kriterium der \textit{assertiven Stärke} geht Doherty auf Bedeutungsanteile der einzelnen MPn einer Kombination ein. In der Form, wie ihr Ansatz hier präsentiert wurde, spielt die Bedeutung der MP-Kombinationen, d.h. insbesondere welche Art von Bezug zwischen den MPn in einer Sequenz besteht, in ihrer Ableitung der Abfolgebeschränkungen zunächst einmal keine Rolle. Die Ansätze, die im folgenden Abschnitt angeführt werden, leiten die akzeptablen Reihungen in Sequenzen von MPn aus der (ihrer Ansicht nach zuzuschreibenden) Interpretation der MP-Kombinationen ab.\footnote{Dies soll nicht bedeuten, dass \citet{Doherty1985, Doherty1987} zur Frage der Interpretation der MP-Kombinationen keinen Beitrag leistet. Ganz im Gegenteil macht sie hier sehr konkrete Annahmen, die dazu einen deutlichen Gegenpol zu den hier betrachteten Ansätzen von \citet{Ormelius-Sandblom1997} und \citet{Rinas2006, Rinas2007} darstellen. Dohertys Ausführungen zur Frage der Bedeutung von MP-Kombinationen werden in Abschnitt~\ref{subsec:input} erläutert.}

\subsection{Korrelation von Oberflächenabfolgen und Skopusverhältnissen}
\label{sec:skopus}
Sowohl \citet{Ormelius-Sandblom1997} als auch \citet{Rinas2006, Rinas2007} gehen dabei davon aus, dass zwischen den einzelnen MPn in einer Kombination interpretatorisch ein Skopusverhältnis \is{Skopus} besteht, d.h. eine weiter links stehende Partikel nimmt Skopus über eine weiter rechts stehende Partikel. Die Oberflächenabfolgen in einer MP-Sequenz spiegeln diese Interpretationsverhältnisse syntaktisch wider. Die Ansätze unterscheiden sich in den jeweils zugrundegelegten Bedeutungsbeschreibungen der Einzelpartikeln, vor allem in der Natur der Bedeutung, die die Autoren jeweils für den MP-Beitrag ansetzen. Es wird sich zeigen, dass diese Annahmen für eine Argumentation, die auf Skopusverhältnisse Bezug nimmt, von besonderer Relevanz sind. 

\subsubsection{\citet{Ormelius-Sandblom1997}}
\label{sec:os}
\citet{Ormelius-Sandblom1997} beschäftigt sich mit den MPn \textit{ja}, \textit{doch} und \textit{schon}\footnote{Ich beschränke mich in der Skizzierung ihres Ansatzes auf die MPn \textit{ja} und \textit{doch}.} und versucht, Abfolgebeschränkungen (wie in (\ref{100})) auf die Interpretation, die sie den MP-Sequenzen zuschreibt, zurückzuführen. Wie einleitend bereits angeführt, nimmt sie dabei konkret an, dass die Partikel am linken Rand Skopus \is{Skopus} über die Partikel am rechten Rand der Kombination nimmt.	

\begin{exe}
	\ex\label{100} 
	(Was sind denn die finanziellen Konsequenzen eines solchen Schrittes?) wir können \textbf{ja doch}/\textbf{*doch ja} nicht übersehen, daß \ldots
	\newline
	\hbox{}\hfill\hbox{\citet[93]{Ormelius-Sandblom1997}}	
\end{exe}
Um die Argumentation Ormelius-Sandbloms darzulegen, ist (wie bei allen Ansätzen) zunächst ein Blick auf die Modellierung der Bedeutung der Einzelpartikeln zu werfen.

Die MPn \textit{ja} und \textit{doch} ähneln sich der Autorin zufolge darin, dass sie (wenn auch zu unterschiedlichen Graden) affirmativ wirken. Aufgrund dieser Gemeinsamkeit nimmt sie als Bedeutungskern dieser MPn \textit{Faktizität} \is{Faktizität} an. Unter Verwendung von \textit{ja} und \textit{doch} weise der Sprecher auf den \glqq Faktizitätsaspekt\grqq{} der Instantiierung der Proposition durch einen Sachverhalt (\citeyear[80]{Ormelius-Sandblom1997}) hin, wobei es sich bei einem Sachverhalt nicht um das objektive Geschehen selbst, sondern um den Sachverhalt wie vom Sprecher wahrgenommen handelt (vgl. \citealt[99]{Rehbock1992}). Eine Proposition versteht Ormelius-Sandblom als Beschreibung eines Sachverhalts. Wenn eine Proposition (p) durch einen Sachverhalt (e) instantiiert wird (INST) ([e INST p]) (vgl. \citealt[23]{Bierwisch1988}, \citealt[35-36]{Brandt1992b} zum Operator INST), bedeutet dies, dass ein (wie vom Sprecher wahrgenommener) Sachverhalt der Beschreibung entspricht, die die Proposition abgibt. Mit den MPn charakterisiert ein Sprecher dann die Beziehung zwischen Propositionen (d.h. Beschreibungen von Sachverhalten) und seiner Vorstellung über die Welt. Der Hinweis auf die Faktizität der Instantiierung der Proposition durch einen Sachverhalt (s.o.) (\glq p ist FAKT\grq {}) meint unter Berücksichtigung der skizzierten Auffassungen, dass der Sprecher ausdrückt, dass es Tatsache ist, dass die Situation, wie er sie wahrnimmt, mit der Beschreibung der Proposition übereinstimmt (FAKT [e INST p]), d.h. der Sprecher von der Existenz des Sachverhalts ausgeht.

In ihrer Modellierung des semantischen Beitrags von \textit{ja} und \textit{doch} fasst Ormelius-Sandblom die MPn als Operatorenausdrücke auf und setzt als Basis ihrer Bedeutung einen Operator \is{Operator} FAKT an. Dieser Operator ist Teil einer Dualitätsgruppe, so dass aus \textit{FAKT} durch die Operationen der inneren (vgl. (\ref{101a})), äußeren (vgl. (\ref{101b})) und dualen Negation (vgl. (\ref{101c})) ein neuer Operator geformt werden kann. Aus einigen dieser Operatoren bildet sich die Bedeutung der MPn (s.u.).

\begin{exe}
	\ex\label{101} 
		\begin{xlist}	
			\ex\label{101a} FAKT$\neg$p (innere Negation)
			\ex\label{101b} $\neg$FAKTp (äußere Negation)
			\ex\label{101c} $\neg$FAKT$\neg$p (duale Negation)
		\end{xlist}
\end{exe}
Standardmäßig treten \textit{ja} und \textit{doch} in Assertionen auf (aber vgl. \citealt[87-92]{Ormelius-Sandblom1997} auch zum Auftreten dieser MPn in anderen Äußerungstypen). 

(\ref{102}) gibt Ormelius-Sandbloms Modellierung der Bedeutung von \textit{ja} sowie die Paraphrase derer wider.

\begin{exe}
	\ex\label{102} 
		\begin{xlist}	
			\ex\label{102a} $\lambda \textrm{p[FAKTp]}$
			\ex\label{102b} Hinweis auf die Faktizität der Instantiierung der Proposition durch einen Sachverhalt 
			\hfill\hbox {\citet[82]{Ormelius-Sandblom1997}}
		\end{xlist}
\end{exe}
Wendet man diese Bedeutungsbeschreibung auf das Auftreten von \textit{ja} in der konkre\-ten Assertion in (\ref{103a}) an, ergibt sich als Gesamtbedeutung der Äußerung (\ref{103b}), die sich wie in (\ref{103c}) umschreiben lässt.
\begin{exe}
	\ex\label{103} 
		\begin{xlist}	
			\ex\label{103a} Er wird \textbf{ja} kommen.  
			\ex\label{103b} $\exists \textrm{e [FAKT[e INST[KOMM ER}]]]$
			\ex\label{103c} \glq Es gibt einen Sachverhalt e derart, daß es ein FAKT ist, daß e die Proposition \glq er wird kommen\grq {} instantiiert\grq 			{}.                                    
			\hfill\hbox {\citet[89]{Ormelius-Sandblom1997}}
		\end{xlist}		
\end{exe}
(\ref{104}) illustriert Ormelius-Sandbloms Modellierung der Semantik von \textit{doch}.

\begin{exe}
	\ex\label{104} 
		$\lambda \textrm{p[FAKTp}]$\\
		\textsc{Implikatur}$[\exists \textrm{q[q} \rightarrow \neg \textrm{p}]]$
			\hfill\hbox {\citet[83]{Ormelius-Sandblom1997}}
\end{exe}
Die MP \textit{doch} wirkt ebenfalls affirmativ. Hinzu tritt eine adversative Komponente (von Ormelius-Sandblom modelliert \is{konventionelle Implikatur} als \textit{konventionelle Implikatur}), die darauf verweist, dass die in der MP-Äußerung ausgedrückte Proposition p im Kontrast zu einer (im Kontext auftretenden/aus dem Kontext abzuleitenden) anderen Proposition q steht, aus der die Negation von p folgt. Im Dialog in (\ref{105a}) führen diese Bedeutungsannahmen dazu, dass B ausdrückt, dass es Fakt ist, dass p gilt, d.h. die Proposition p \textit{dass Patricks Auto da ist} gibt die Vorstellung von B korrekt wider. Weiter verweist B auf den Kontrast zwischen dieser Proposition p und der Proposition q aus As Äußerung \textit{dass Patrick nicht zu Hause ist}: Wenn Patrick nicht zu Hause ist, sollte sein Auto nicht da sein (vgl. (\ref{105b}) und (\ref{105c})).

\begin{exe}
	\ex\label{105} 
		\begin{xlist}	
			\ex\label{105a} A: Patrick ist nicht zu Hause.\\
			 				B: Aber sein Auto ist \textbf{doch} da.                                   
			\hfill\hbox {\citet[83]{Ormelius-Sandblom1997}}
			\ex\label{105b} $\exists \textrm{e[FAKT [e INST [DASEIN AUTO}]]]$ \\
							\textsc{Implikatur} $[\exists \textrm{q [q} \rightarrow \neg [\textrm{e INST [DASEIN AUTO}]]]]$			
			\ex\label{105c} \glq Es gibt einen Sachverhalt e derart, dass es ein FAKT ist, dass e die Proposition \glq sein Auto ist da\grq {} instantiiert 							und es gibt eine Proposition q im Kontext, die impliziert, dass es nicht der Fall ist, dass der Sachverhalt e die Proposition 								\glq sein Auto ist da\grq {} instantiiert.\grq {}
		\end{xlist}
\end{exe}
\textbf{Reihenfolgebeschränkungen: Skopus in Modalpartikel-Kombinationen}\\
\citet[92-93]{Ormelius-Sandblom1997} vertritt explizit die Annahme, dass zwischen den von ihr untersuchten MPn in Kombinationen ein Skopusverhältnis \is{Skopus} besteht. Aus diesem Skopusverhältnis ist die feste Abfolge der MPn in Reihung abzuleiten, indem die Oberflächenabfolge als direkter Reflex der Skopusrelationen aufgefasst wird: 

\begin{quotation}
Was das Verhältnis zwischen zwei MPn im selben Satz betrifft, so will ich davon ausgehen, dass eine MP $\alpha$, die eine MP $\beta$ \is{c-Kommando} c-kommandiert, diese semantisch in ihren Skopus nimmt und entsprechend auf sie einwirkt. Für diese Annahme spricht, dass beim Auftreten mehrerer MPn in einem Satz gewisse Abfolgen entweder unakzeptabel erscheinen oder wenigstens deutlich schlechter zu sein scheinen als andere.
\end{quotation}
Die Tatsache, dass \textit{ja} (Ormelius-Sandbloms Ansicht nach) in Kombination mit \textit{doch} stets an erster Stelle stehen muss (vgl. (\ref{106})), führt die Autorin darauf zurück, dass die korrekte Bedeutungszuschreibung an das kombinierte Auftreten dieser Partikeln diejenige sei, in der \textit{ja} Skopus über \textit{doch} nimmt (vgl. (\ref{107}) und (\ref{108})).

\begin{exe}
	\ex\label{106}
	(Was sind denn die finanziellen Konsequenzen eines solchen Schrittes?) 
		\begin{xlist}	
			\ex\label{106a} wir können \textbf{ja doch} nicht übersehen, daß \ldots                                  
			\ex\label{106b} *wir können \textbf{doch ja} nicht übersehen, daß \ldots
		\end{xlist}
\end{exe}

\begin{exe}
	\ex\label{107}														 
	ja(doch(p))
\end{exe}

\begin{exe}
	\ex\label{108}														 
	$[\textrm{FAKT[FAKTr]]}$\\
	\textsc{Implikatur}$ [\exists \textrm{q [q} \rightarrow \neg \textrm{r}]]$
	\hfill\hbox {\citet[93]{Ormelius-Sandblom1997}}
\end{exe}
Die Skopusinterpretation aus (\ref{108}) setzt sich auf die folgende Weise aus den Einzelbedeutungen von \textit{ja} und \textit{doch} (vgl. zur Wiederholung (\ref{109})) zusammen.

\begin{exe}
	\ex\label{109} 
		\begin{xlist}	
			\ex\label{109a} \textit{ja}: $[\textrm{FAKTr}]$
			\ex\label{109b}  \textit{doch}: $[\textrm{FAKTr}]$ \\
							\textsc{Implikatur} $[\exists \textrm{q [q} \rightarrow \neg \textrm{r}]]$			
		\end{xlist}
\end{exe}
Aufgrund der Annahme des Skopusverhältnisses zwischen \textit{ja }und \textit{doch} wird für die Variable r hier der Ausdruck $\textrm{[FAKTr]}$ eingesetzt. Beim Einzelauftreten von \textit{ja} entspricht r in (\ref{109a}) der Proposition p. Dieser Ausdruck entspricht der Proposition r, die durch \textit{doch} bereits modifiziert ist. Aus dieser Schachtelung der Bedeutungsanteile ergibt sich die komplexere Bedeutung \glq Es ist Fakt, dass es Fakt ist, dass r.\grq {}. Die Implikatur behält die Gestalt, die sie auch aufweist, wenn \textit{doch} alleine auftritt.
 
Ormelius-Sandblom argumentiert, der umgekehrte Skopus von \textit{ja} und \textit{doch} würde nicht der Interpretation von (\ref{106}) entsprechen. Die Bedeutung dieser Interpretation ist in (\ref{111}) angegeben.

\begin{exe}
	\ex\label{110}														 
	doch(ja(p))
\end{exe}
\vspace{-0.65cm}
\begin{exe}
	\ex\label{111}														 
	$\textrm{[FAKT[FAKTr}]]$\\
	\textsc{Implikatur} $[\exists \textrm{q [q} \rightarrow \neg \textrm{FAKTr}]]$
	\hfill\hbox {\citet[93]{Ormelius-Sandblom1997}}
\end{exe}
Der erste Teil der Bedeutungszuschreibung ist identisch mit der Bedeutung in (\ref{108}), die Implikatur ist jedoch eine andere, weil für r die bereits durch \textit{ja} modifizierte Proposition eingesetzt wird, d.h. $\textrm{[FAKTr}]$.

Die Argumentation für \glq \textit{ja} nimmt Skopus über \textit{doch}\grq {} und gegen \glq \textit{doch} nimmt Skopus über \textit{ja}\grq {} macht Ormelius-Sandblom von der Implikatur abhängig, die die adversative Relation kodiert. Unter der Interpretation in (\ref{111}) wendet sich der Sprecher gegen die Proposition \glq es ist nicht Fakt, dass r\grq {}. Die Autorin möchte allerdings annehmen, dass sich der ausgedrückte Widerspruch in (\ref{106}) auf die Proposition \glq wir können übersehen, dass \ldots \grq {} , d.h. auf $\neg$r bezieht. Dieser Kontrast ergebe sich jedoch nur aus dem Skopusverhältnis, in dem \textit{ja} Skopus über \textit{doch} nimmt, da die Implikatur in diesem Falle die passende Gestalt \glq aus q folgt $\neg$r \grq {} annimmt (vgl. (\ref{108})).\footnote{Mit derselben prinzipiellen Argumentation, dass die MP, die die andere MP in einer Kombination in ihren Skopus nimmt, dieser MP vorangeht, leitet Ormelius-Sandblom ab, wieso \textit{doch} in der Kombination aus \textit{doch} und \textit{schon} an erster Stelle auftreten muss (vgl. \citealt[93-94]{Ormelius-Sandblom1997}).}

Das Kriterium, das die Abfolge der MPn in Kombinationen steuert, ist nach Ormelius-Sandblom folglich der Verlauf des Skopus der beteiligten MPn: Diejenige MP, die eine andere MP in ihren Skopus nimmt, geht dieser MP an der syntakti\-schen Oberfläche linear voran bzw. ist ihr in der hierarchischen Dimension übergeordnet, so dass die Reihung der Partikeln (bzw. ihre c-Kommandorelationen) \is{c-Kommando} den direkten Reflex der semantischen Skopusrelation darstellt.\\

\noindent
Auch der im folgenden Abschnitt skizzierte Ansatz von \citet{Rinas2006, Rinas2007} bringt die Seria\-lisierungsbeschränkungen von MPn in einer Kombination mit den Skopusverhältnissen in Verbindung.

\subsubsection{\citet{Rinas2007}}
\label{sec:ri}
Ebenso wie Ormelius-Sandblom argumentiert der Autor für den denkbar direktesten Zusammenhang zwischen der Abfolge der Partikeln und ihrem Skopus: Die in (lineare Abfolge übersetzte) hierarchische syntaktische Struktur korreliert direkt mit dem Skopus, den syntaktisch hierarchisch höher positionierte MPn über strukturell tiefer verankerte MPn nehmen.\\

\noindent
\textbf{Hierarchische Verhältnisse zwischen Präsuppositionen von Modalpartikeln}\\
Rinas modelliert die Bedeutung von MPn, indem er annimmt, dass sie \textit{semanti\-sche Präsuppositionen} \is{Präsupposition} auslösen. Dem Autor zufolge setzt ein Sprecher durch die Verwendung von MPn beim Hörer gewisse Einstellungen oder Kenntnisse voraus, d.h. mit anderen Worten \textit{präsupponiert} er bestimmte Annahmen (vgl. \citealt[421-422]{Rinas2007}). (\ref{112}) und (\ref{113}) illustrieren exemplarisch zwei Bedeutungszuschreibungen wie von \citet[420 bzw. 425]{Rinas2007} vorgeschlagen.  

\begin{exe}
	\ex\label{112} 
			\textit{ja}: JA(p) $»$ NICHT-GLAUBT(H, NICHT-p)	
\end{exe}
\vspace{-0.65cm}
\begin{exe}
	\ex\label{113} 
			\textit{auch}: AUCH(p) $»$ NICHT-ÜBERRASCHEND (q) WEIL (p)	
\end{exe}
Da Rinas (wie eingangs erläutert) ebenso wie Ormelius-Sandblom von einem Skopusverhältnis \is{Skopus} zwischen den Partikeln in einer Kombination ausgehen möchte, ist für seine Ableitung das Verhalten von Präsuppositionen im Skopus anderer Präsuppositionen von Bedeutung. Anhand der Präsuppositionen, die die Sätze in (\ref{114a}) (ausgelöst durch die lexikalische Bedeutung von \textit{aufhören}) und (\ref{115a}) (ausgelöst durch das faktive Verb \textit{bedauern}) in Isolation jeweils aufweisen (vgl. (\ref{114b}) und (\ref{115b})), weist er nach, dass sich präsupponierte Inhalte schachteln lassen. 

\begin{exe}
	\ex\label{114} 
		\begin{xlist}	
			\ex\label{114a} Franz hat aufgehört zu rauchen. $»$ Franz hat vorher geraucht.
			\ex\label{114b} $\textrm{AUFHÖRT}_{\textrm{ti}}(\textrm{RAUCHEN(F))} \ » \ \textrm{RAUCHEN}_{\textrm{ta} < \textrm{ti}}(\textrm{F})$ \\
			$[\textrm{t}_{\textrm{i}} = \text{Zeitpunkt}, \textrm{t}_{\textrm{a}} < \textrm{t}_{\textrm{i}} = \text{vor} \ \textrm{t}_{\textrm{i}}  \text{liegender Zeitpunkt} \ \textrm{t}_{\textrm{a}}]$ \\
			\glq Wenn zu einem Zeitpunkt $\textrm{t}_{\textrm{i}}$ gilt, dass Franz aufhört zu rauchen, dann präsupponiert dies, dass es einen vor $\textrm{t}_{\textrm{i}}$ \ liegenden Zeitpunkt $\textrm{t}_{\textrm{a}}$ gibt, zu dem Fritz raucht.\grq {}
		\end{xlist}
\end{exe}

\begin{exe}
	\ex\label{115} 
		\begin{xlist}	
			\ex\label{115a} Paul bedauert, dass Otto erkältet ist. $»$ Otto ist erkältet.
			\ex\label{115b} BEDAUERT(P, ERKÄLTET(O)) $»$ WAHR(ERKÄLTET(O))\\
			\glq Wenn gilt, dass Paul bedauert, dass Otto erkältet ist, dann präsupponiert dies, dass Otto erkältet ist.\grq {}
			\hfill\hbox {\citet[421]{Rinas2007}}
		\end{xlist}
\end{exe}	  
Für den Satz in (\ref{116a}) setzt er die Bedeutung in (\ref{116b}) an und schreibt: \glqq Die beiden Präsuppositionen \is{Präsupposition} stehen demnach in einem hierarchischen Verhältnis zueinander. Die faktive Präsupposition \glq dominiert\grq {} gewissermaßen die andere.\grqq{} (\citealt[421]{Rinas2007})

\begin{exe}
	\ex\label{116} 
		\begin{xlist}	
			\ex\label{116a} Paul bedauert, dass Fritz aufgehört hat zu rauchen.
			\ex\label{116b} $\textrm{BEDAUERT}(\textrm{P}, \text{AUFHÖRT}_{\textrm{ti}}(\textrm{RAUCHEN(F))} \  » \  \textrm{RAUCHEN}_{\textrm{ta} < \textrm{ti}}(\textrm{F}))\\ 
			» \textrm{WAHR}(\textrm{AUFHÖRT}_{\textrm{ti}}(\textrm{RAUCHEN(F))} \ » \ \textrm{RAUCHEN}_{\textrm{ta} < \textrm{ti}}(\textrm{F}))$\\
			\hbox{}\hfill\hbox{\citet[421]{Rinas2007}}\\
			Paul bedauert, dass Fritz aufgehört hat zu rauchen. $»$ Fritz hat aufgehört zu rauchen. $»$ Es gibt einen Zeitpunkt vor dem Aufhören, zu dem Fritz geraucht hat.
		\end{xlist}
\end{exe}
\textbf{Interpretation von Modalpartikel-Kombinationen}\\
\noindent
Obwohl Rinas sich prinzipiell für ein Skopusverhältnis \is{Skopus} zwischen MPn in Kombinationen ausspricht, illustriert er die daraus resultierende Interpretation im Detail eigentlich nur am Beispiel in (\ref{117}).

\begin{exe}
	\ex\label{117} 
		A: Das Menü war ausgezeichnet!\\ 
		– B: Es war \textbf{ja auch} das teuerste Essen auf der Speisekarte.
		\hfill\hbox {\citet[242]{Rinas2007}}
\end{exe}
Er schließt sich \citet[155]{Thurmair1989} zum Beitrag der Einzelpartikeln an. Für \textit{auch} gelte demnach: \glqq Der Sprecher \ldots signalisiert, daß der dort geäußerte Sachverhalt für ihn durchaus erwartbar war \ldots; die Äußerung, in der \textit{auch} steht, liefert dafür die Begründung oder Erklärung.\grqq{} (\citealt[209]{Thurmair1989}) Der Beitrag von \textit{ja} ist es, anzuzeigen, dass der Sachverhalt bekannt ist (bzw. wie \citealt[243-244]{Rinas2007} Thurmairs Beschreibung erweitert, als bekannt unterstellt ist). Rinas nimmt an, dass die Bedeutung der MP-Kombination in (\ref{117}) durch die Paraphrase in (\ref{118}) erfasst wird. 

\begin{exe}
	\ex\label{118} 
		Es ist bekannt/evident, dass die Begründung für das gute Essen die Tatsache ist, dass das Essen das teuerste auf der Karte ist.
\end{exe}
Der Begründungszusammenhang zwischen der Qualität des Essens und seinem Preis wird als bekannt/evident ausgezeichnet. 

Unter Anwendung seiner Bedeutungsmodellierung für \textit{ja} und \textit{auch} (vgl. (\ref{112}) und (\ref{113})), weist Rinas der MP-Kombination \textit{ja auch} die Bedeutung in (\ref{120}) zu.

\begin{exe}
	\ex\label{120} 
		ja $>$ auch\\
		$\textrm{JA(AUCH(p)} \  » \ \textrm{NICHT-ÜBERRASCHEND(q) WEIL(p))} \\ » \ \textrm{NICHT-GLAUBT(H, NICHT(AUCH(p)} » \\ 					\textrm{NICHT-ÜBERRASCHEND(q)WEIL(p)))}$\\
		\glq Es ist unkontrovers (es ist nicht der Fall, dass der Hörer daran zweifelt), dass q nicht überraschend ist, weil p.\grq {}\\
		\glq Es ist bekannt/evident für den Hörer, dass q aufgrund von p erwartbar ist.\grq {} 
		\hbox{}\hfill\hbox{\citet[425]{Rinas2007}}
\end{exe}
Er schreibt (\citeyear[425]{Rinas2007}): 
\begin{quotation}
[...] \textit{ja} und \textit{auch} [stehen] in dieser Kombination also nicht jeweils in einer unmittelbaren Beziehung zur Proposition, sondern sind  \glq hierarchisch\grq {}  oder  \glq asymmetrisch\grq {}  aufeinander bezogen. Damit wird auch verständlich, wa\-rum die Abfolge dieser Abtönungspartikeln nicht umkehrbar ist [...]
\end{quotation}
Er verweist hier auf (\ref{121}).

\begin{exe}
	\ex\label{121} 
	*Es war \textbf{auch ja} das teuerste Essen auf der Speisekarte. 
			\hfill\hbox {\citet[425]{Rinas2007}}
\end{exe}
Genau wie Ormelius-Sandblom postuliert er folglich einen direkten Zusammenhang zwischen Oberflächenabfolge und Skopusverhältnis. Alle weiteren Fälle von MP-Kombinationen, die er betrachtet, werden auf die Interpretation der Sequenz von \textit{ja} und \textit{auch} rückbezogen. (Außer in wenigen Ausnahmen) argumentiert Rinas für ein Skopusverhältnis zwischen den Partikeln, das die feste Abfolge der Partikeln in Kombinationen auffangen soll (vgl. \citeyear[432-448]{Rinas2007}).\\

\noindent 
Sowohl \citet{Ormelius-Sandblom1997} als auch \citet{Rinas2006, Rinas2007} leisten zu der Frage nach der Erfassung und Ableitung der Abfolgebeschränkungen in MP-Kombinationen einen Beitrag, indem sie einen Zusammenhang herstellen zwi\-schen der Interpretation der MP-Sequenzen und ihrer syntaktischen Abfolge. Sie fassen letztere dabei als Reflex der Bedeutungszuschreibung auf bzw. nehmen an, dass sich die Interpretation aus dem hierarchischen syntaktischen Verhältnis ergibt. Rinas' Ausführungen entsprechen hierbei eher der Sicht, dass sich die syntaktische Abfolge aus der Interpretation ergibt (vgl. z.B. die folgende Aussage des Autors:  \glqq Es wird dafür argumentiert, dass die Berücksichtigung der Skopus-Verhältnisse \is{Skopus} auch eine Erklärung des syntaktischen Verhaltens der Abtönungspartikel-Kombinationen ermöglicht.\grqq{} (\citealt[408]{Rinas2007}), während Or\-melius-Sandblom aus dem syntaktischen c-Kommando-Verhältnis \is{c-Kommando} auf das Skopusverhältnis schließt (vgl. das folgende Zitat:  \glqq Was das Verhältnis zwischen zwei MPn im selben Satz betrifft, so will ich davon ausgehen, dass eine MP $\alpha$, die eine MP $\beta$ c-kommandiert, diese semantisch in ihren Skopus nimmt und entsprechend auf sie einwirkt.\grqq{} (\citealt[92-93]{Ormelius-Sandblom1997})). Beiden Ansätze gehen auf jeden Fall von einer Korrelation zwischen der Abfolge und der Interpretation (in) einer MP-Kombination aus.\\

\noindent
Eine solche Art der Argumentation lässt m.E. zwei Fragen offen. So scheint mir eine Annahme zu fehlen, die auf irgendeine Art begründet, \underline{warum} die MP-Abfolge mit der Interpretation der MP-Kombination überhaupt korrelieren soll. Sicherlich ist es eine weit akzeptierte Annahme, dass die c-Kommando-Verhältnis\-se zwischen sprachlichen Einheiten (auf Ebene der Logischen Form!) \is{Logische Form} den Skopusverhältnissen zwischen diesen Elementen entsprechen. Wenngleich diese Korrelation oftmals auch an der syntaktischen Oberfläche bereits gegeben ist, lassen sich dennoch relativ einfach verschiedenste Strukturen anführen, bei denen die Oberflächensyntax die Skopusverhältnisse im Satz nicht reflektiert.

Insbesondere unter gereihtem Auftreten von Ausdrücken, die Sprechereinstellungen kodieren, hat man es im Deutschen sicherlich nicht mit einer direkten Zuordnung von Oberflächensyntax (d.h. c-Kommandorelationen) und Interpretation (d.h. Skopusverhältnissen) zu tun. Der Satz in (\ref{122}) bedeutet nicht, dass es notwendig/zukünftig der Fall ist, dass es bekannt ist, dass Konrad verreist ist.

\begin{exe}
	\ex\label{122} 
	Konrad \textit{\textbf{muss}}/\textit{\textbf{wird}} \textbf{ja} verreist sein.
			\hfill\hbox {\citet[100]{Doherty1987}}
\end{exe}
Die Interpretation läuft hier den strukturellen Verhältnissen gerade zuwider: Es ist bekannt, dass es notwendigerweise/zukünftig der Fall ist, dass Konrad verreist ist (zu weiteren Beispielen dieser Art unter Auftreten von Modalverben, MPn und Satzadverbien vgl. \citealt[121-128]{Doherty1985}).				

Wie die Sätze in (\ref{123}) zeigen, können auch Sätze, in denen Satzadverbien \is{Satzadverb} gehäuft auftreten, entgegen der syntaktischen Abfolgen interpretiert werden.

\begin{exe}
	\ex\label{123} 
		\begin{xlist}	
			\ex\label{123a} Hans schläft \textit{\textbf{leider wahrscheinlich}} hier.
			\ex\label{123b} Hans schläft \textit{\textbf{wahrscheinlich leider}} hier.
			\ex\label{123c} \textit{\textbf{Wahrscheinlich}} schläft Hans \textit{\textbf{leider}} hier.
			\hfill\hbox {\citet[205]{Lang1979}}
		\end{xlist}
\end{exe}
So nimmt \citet[205]{Lang1979} für alle Sätze in (\ref{123}) (unabhängig der Abfolge der auftretenden Satzadverbien) dieselbe Skopusstruktur \is{Skopus} an:\\ $\textrm{LEIDER[WAHRSCHEINLICH[p]}]$.

Und auch im Bereich der Strukturposition und Skopusinterpretation von w-Phrasen und Fokuspartikeln finden sich durchaus Fälle, bei denen die c-Komman\-doverhältnisse an der syntaktischen Oberfläche und die Skopusinterpretation nicht einhergehen (vgl. z.B. \citealt[94]{Reis1992}, \citealt[29-30]{Brandt1992b}, \citealt[167]{Lohnstein2000}, \citealt[54]{Hoeksema1991}), wenngleich dieser Zusammenhang in der Regel vorliegt.

Wenn man einen (durchaus plausiblen) Zusammenhang zwischen Oberflächenabfolge (c-Kommando) und Interpretation (Skopus) für die MP-Abfolgen aufma\-chen möchte wie Ormelius-Sandblom und Rinas es vorschlagen, fehlt in den Ansätzen die Annahme eines Prinzips, das diese Form von Isomorphismus \is{Isomorphismus} zwischen Struktur und Interpretation motiviert. Wie oben erwähnt geht es den Autoren nicht darum, den Zusammenhang zwischen c-Kommando- und Skopusverhältnissen auf der Ebene der Logischen Form anzusiedeln, für die man tatsächlich annimmt, dass sie einhergehen. D.h. es wäre für die beiden Autoren nötig, im Bereich der Abfolge von MPn in Kombinationen im Deutschen auf ein Prinzip der Art der \textit{Scope Transparency} (ScoT) \is{Scope Transparency} (If the order of two elements at LF is A » B, the order at PF is A $»$ B.) aus \citet[92]{Wurmbrandt2008} bzw. \citet[373]{Wurmbrandt2012} oder ähnlichen Prinzipien wie \textit{Minimize (PF:LF) Mismatch} \is{Minimize (PF:LF) Mismatch} (\citealt{Bobaljik1995, Bobaljik2002}) oder dem \textit{Scope Principle} \is{Scope Principle} (\citealt{Diesing1997}) zu bauen.\\

\noindent
Ein weiterer Aspekt, der mir bei diesen beiden Versionen eines Skopusansatzes unberücksichtigt erscheint, ist, dass die Autoren allenfalls für die \underline{akzeptable} Reihung der MPn aufkommen. Sie bieten in diesem Sinne keine Erklärung für den Ausschluss der umgekehrten Abfolge. Bei \citet[425 bzw. 434]{Rinas2007} liest man dazu z.B.: \glqq Damit wird auch verständlich, warum die Abfolge dieser Abtönungspartikeln (\textit{ja} und \textit{auch} [S.M.]) nicht umkehrbar ist [...].\grqq{} Selbst wenn hier eine Korrelation von syntaktischer Struktur und Skopusinterpretation vorliegt, stellt sich mir die Frage, was ausschließt, dass die umgekehrte MP-Abfolge mit einem anderen Skopusverhältnis und somit einer anderen Interpretation auftreten kann. Keiner behauptet, dass die Kombinationen \textit{ja doch} und \textit{doch ja}, \textit{ja eben} und \textit{eben ja} etc. dieselbe Bedeutung haben. Für Ansätze, die aufgrund der Skopusinterpretation auf die syntaktische Struktur schließen bzw. andersherum aus der Abfolge in der Syntax die Bedeutung ableiten, wäre es ja geradezu ihre Vorhersage, dass je nach Abfolge eine andere Bedeutung vorliegt. Sowohl Ormelius-Sandblom als auch Rinas lösen vor diesem Hintergrund folglich letztlich (auf ihre Art) nur das halbe Puzzle der Abfolgebeschränkungen in MP-Kombinationen. Das Skopusverhältnis, \is{Skopus} das unter der inakzeptablen Abfolge eintritt, müsste im Rahmen der beiden Arbeiten plausiblerweise aus semantischen Gründen ausgeschlossen werden können. Dieser offene Punkt in den Ableitungen von Orme\-lius-Sandblom und Rinas wird umso offensichtlicher, nimmt man hinzu, dass sich die Interpretation der umgekehrten MP-Abfolgen in den Modellen der beiden Autoren problemlos abbilden ließe. In Analogie zur Bedeutung in (\ref{124}), die Rinas der Kombination \textit{ja doch} auf der Basis der Einzelbedeutungen in (\ref{125}) und (\ref{126}) zuschreibt, ließe sich die Interpretation der Abfolge \textit{doch ja} (entlang der von ihm verwendeten Notation) wie in (\ref{127}) modellieren.\footnote{Ormelius-Sandbloms Ausführungen zum Ausschluss der Abfolge \textit{schon doch} kommen dem Kritikpunkt der fehlenden Erklärung für die Inakzeptabilität der umgekehrten Abfolge dagegen nach. Sie verweist hier auf die prinzipielle Unsinnigkeit der Interpretation.}

\begin{exe}
	\ex\label{124} 
		\begin{xlist}	
			\ex\label{124a} JA(DOCH(p) $»$ WIDERSPRICHT(p,q) \& KENNT(H,p) $\lor$ KENNT(H,q))
				$»$ NICHT GLAUBT(H, NICHT(DOCH(p) $»$ WIDERSPRICHT(p,q) \& KENNT(H,p) $\lor$ KENNT(H,q)))
			\ex\label{124b} \glq Es ist unkontrovers (es ist nicht der Fall, dass der Hörer daran zweifelt), dass p im Widerspruch zu einer anderen Proposition q steht und dass dem Hörer p oder q bekannt ist.\grq {}
			\hfill\hbox {\citet[431]{Rinas2007}}
		\end{xlist}
\end{exe}

\begin{exe}
	\ex\label{125} 	
	\textit{ja}: JA(p) $»$ NICHT-GLAUBT(H, NICHT-p)	
\end{exe}
\vspace{-0.65cm}
\begin{exe}
	\ex\label{126} 	
	\textit{doch}: DOCH(p) $»$ WIDERSPRICHT(p,q) \& (KENNT(H,p) $\lor$ KENNT(H,q))
\end{exe}

\begin{exe}
	\ex\label{127}
		\begin{xlist}	
			\ex\label{127a} DOCH(JA(p) $»$ NICHT-GLAUBT(H, NICHT-p))\\
			$»$ WIDERSPRICHT((JA(p) $»$ NICHT-GLAUBT(H, NICHT-p)),q) \& \\(KENNT(H, JA(p) $»$ NICHT-GLAUBT(H, NICHT-p)) $\lor$ KENNT(H,q))
			\ex\label{1247}  \glq Dass p unkontrovers ist, steht im Widerspruch zu einer anderen Proposition q und dem Hörer ist bekannt, dass p unkontrovers ist oder q.\grq {} 
		\end{xlist}
\end{exe}
Die beiden hier diskutierten Aspekte ( a) die Annahme eines Prinzips zur Erfassung der nicht selbstverständlichen Korrelation von syntaktischer Abfolge und Skopusverhältnissen und b) der (semantische) Ausschluss des umgekehrten Skopusverhältnisses in den umgekehrten Abfolgen), die in den Arbeiten von Ormelius-Sandblom und Rinas weitestgehend offen bleiben, finden eine Berücksichtigung in der im nächsten Abschnitt dargestellten Arbeit von \citet{Vismans1994}.

\subsection{Ebenenzugehörigkeit}
\label{sec:ebenen}
Die beiden im Folgenden ausgeführten Ansätze zur Beschränkung der Reihung von MPn in Kombinationen basieren auf der Idee, die Abfolgen der MPn als Reflex ihrer Zugehörigkeit zu verschiedenen Ebenen abzuleiten. MPn wirken dieser Auffassung nach auf verschiedenen Schichten. Da die jeweiligen Schichten (unabhängig) einer Ordnung folgen, entspricht auch die Reihung der MPn der Hie\-rarchisierung dieser Ebenen. Die Ableitungen von \citet{Vismans1994} und \citet{Ickler1994} unterscheiden sich darin, welche Ebenen sie annehmen und welche Zuordnungen zwischen MPn und Schichten sie deshalb vornehmen. Bei Vismans handelt es sich hierbei um Schichten des Satzes, die im Paradigma der \textit{Funktionalen Grammatik} \is{Funktionale Grammatik} unabhängig Bestand haben. Ickler baut weniger auf grammatiktheoretisch begründeten Ebenen auf. Er nimmt Ebenen an, die er durch die Bedeutung und Funktion der jeweils zugeordneten Einzelpartikeln motiviert.

\subsubsection{Satz, Proposition, Prädikation, Illokution}
\label{sec:vismans}
\textbf{Reihung von Modalpartikeln in niederländischen Direktiven}\\
Untersuchungsgegenstand in \citet{Vismans1994} (vgl. auch \citealt{Vismans1992, Vismans1996}) sind MPn in Direktiven \is{Direktiv} im Niederländischen. Wenn MPn in direktiven Sprechakten in Kombinationen auftreten, entspricht ihre Abfolge – differenziert nach Satztypen, in denen die betrachteten MPn prinzipiell (d.h. in Isolation) lizensiert sind – der Reihenfolge in (\ref{128}) (vgl. \citealt[5]{Vismans1994}).\footnote{Ich gebe an dieser Stelle unter Bezug auf die Grammatik von \citet[160-165]{Verheyen2010} Übersetzungen der Einzelpartikeln an. Dies können jedoch nur ungefähre Übersetzungen sein, da sich MPn nur schwer in andere Sprachen übersetzen lassen. Ebenso wie im Deutschen kann die Bedeutung bzw. Funktion der Partikeln z.B. auch je nach Satztyp variieren. Zur Übersetzung von MPn vgl. für eine Vielzahl von Arbeiten beispielsweise \citet{Schubiger1965}, \citet{Burkhardt1995} oder \citet{Masi1996}.}

\begin{exe}
	\ex\label{128}
	\scriptsize
     \begin{tabular}[t]{|l|l|}
     	\hline
      	Typ & Reihenfolge in Kombinationen\\
                \hline
                Dekl & \textit{ook} (\glq auch\grq {}), \textit{maar} (\glq nur\grq {}), \textit{eens} (\glq mal\grq {}), \textit{even} (\glq mal (eben)\grq {})\\
                \hline
                Int & \textit{nou} (\glq nun\grq {}), \textit{misschien}/\textit{soms}* (\glq vielleicht\grq {}), \textit{ook} (\glq auch\grq {}), \textit{eens} (\glq mal\grq {}), \textit{even} ((\glq mal (eben)\grq {})\\
                \hline
                Imp & \textit{dan} (\glq auch\grq {})/\textit{nou}* (\glq nun\grq {}), \textit{toch} (\glq doch\grq {}), \textit{maar} (\glq nur\grq {}), \textit{eens} (\glq mal\grq {}), \textit{even} (\glq mal(eben)\grq {})\\
       		    \hline
                \end{tabular}\\
                * = austauschbar
\end{exe}
Der Imperativ in (\ref{129}) dient der Illustration der Serialisierungsbeschränkung über das gereihte Auftreten der niederländischen MPn aus (\ref{128}).
		
\begin{exe}
	\ex\label{129} 
	\gll Geef de boeken \textbf{dan} \textbf{nu} \textbf{toch} \textbf{maar} \textbf{'es} \textbf{even} hier.\\
	\textit{Gib} \textit{die} \textit{Bücher} MP MP MP MP MP MP \textit{her}\\
	\hfill\hbox{\citet[98]{Hoogvliet1903}}	
\end{exe}		
(\ref{129}) ist ein in Arbeiten zu MPn im Niederländischen gern angeführtes Beispiel, das (mit sechs MPn in Reihung) das maximale Cluster von MPn in niederländi\-schen Direktiven darstellt. Während \textit{dan} und \textit{nu} in ihrer Abfolge umkehrbar sind, ist die Abfolge der übrigen hier auftretenden MPn fest.\\
\newline
\noindent
\textbf{Funktionale Grammatik und das Prinzip der zentripetalen Orientierung}\\
Die Ableitung der Abfolgebeschränkung Vismans’ erfolgt im Rahmen der \textit{Funktionalen Grammatik} \is{Funktionale Grammatik} wie entwickelt in \citet{Dik1978}, \citet{Dik1989}, \citet{Dik1997} bzw. wie in Anlehnung an diese Darstellungen fortgeführt. (\ref{130} zeigt eines von in \citet[Kapitel 16]{Dik1997} formulierten neun so genannten \textit{Generellen Prinzipien}.

\begin{exe}
	\ex\label{130} 
			Generelles Prinzip 3: The Principle of Centripetal Orientation\\
	Constituents conform to (GP3) when their ordering is determined by their relative distance from the head, which may lead to \glqq mirror-image\grqq{}		ordering around the head.
			\hbox{}\hfill\hbox{\citet[401]{Dik1997}}	
\end{exe}
Das \textit{Prinzip der zentripetalen Orientierung} \is{Prinzip der zentripetalen Orientierung} spiegelt die Zusammengehörigkeit des Kopfes und der von ihm abhängigen Elemente wider. Ein weiteres Prinzip (Prinzip 9) aus der Menge von 18 angenommenen \textit{Spezifischen Prinzipien} \is{Spezifische Prinzipien}, das eng mit dem obigen Prinzip 3 aus (\ref{130}) verbunden ist, ist in (\ref{131}) formuliert.

\begin{exe}
	\ex\label{131} 
	$\pi$-operators prefer centripetal orientation according to the schema:\\
 	$\pi_{\textrm{4}}\pi_{\textrm{3}}\pi_{\textrm{2}}\pi_{\textrm{1}}\textrm{[stem]}\pi_{\textrm{1}}\pi_{\textrm{2}}\pi_{\textrm{3}}\pi_{\textrm{4}}$ 
 	\newline
 	\hbox{}\hfill\hbox{\citet[414]{Dik1997}, ursprünglich \citet[141]{Hengeveld1989}}
\end{exe}
Voraussetzung für die Illustration der Wirkungsweise des Prinzips in (\ref{131}) ist zum einen die Klärung des funktional-grammatischen Verständnisses von Ope\-ratoren ($\pi$ in (\ref{131})) \is{Operator} (die die Entitäten darstellen, für die (\ref{131}) die Präferenzregel formuliert) und zum anderen eine Erläuterung der im funktionalen Paradigma vertretenen Auffassung der Existenz verschiedener hierarchisch geordneter Satz\-\glq schichten\grq {}, auf die sich in (\ref{131}) die Indizes beziehen.\\
\newline
\noindent
\textbf{Die Schichten des Satzes}\\
\noindent
Die Annahme verschiedener Satzschichten, die in das Prinzip in (\ref{131}) eingeht, ist zu verstehen als Spezifikation einer älteren Annahme über die Zweiteilung des Satzes in einen Teil, der sich auf die Darstellung des \underline{Sachverhalts} bezieht und einen anderen, der sich auf die Darstellung des \underline{Sprechaktes} bezieht. \citet{Hengeveld1989} unterscheidet hier zwischen zwei Ebenen: \textit{representational level} \is{representational level} und \is{interpersonal level} \textit{interpersonal level}.
 
Im Rahmen eines mehrschichtigen Satzmodells tragen die Bestandteile zu der einen oder anderen (Sachverhalt vs. Sprechakt) übergeordneten Schicht bei. In \citet[4]{Hengeveld1990} sind die angenommenen Schichten \textit{Satz} \is{Satz}, \textit{Proposition} \is{Proposition}, \textit{Prädikation} \is{Prädikation} und \textit{Term} \is{Term}, die dadurch zu unterscheiden sind, welche Einheiten sie jeweils bezeichnen. \textit{Terme} bezeichnen \textit{Individuen} oder \textit{Entitäten}, \textit{Prädikate} (je nach Stelligkeit) \textit{Eigenschaften} von Individuen oder Entitäten bzw. \textit{Relationen} zwischen Individuen und/oder Entitäten. Im Zusammenspiel dienen sie der Konstitution der \textit{Prädikation}, deren Denotat \textit{Sachverhalte} sind. Diese beiden Teilschichten des Satzes dienen folglich der Darstellung eines Sachverhalts. Die \textit{Proposition} be\-zeichnet den \textit{propositionalen Inhalt}, der zusammen mit Sprecher (S) und Adressat (A) Argument des \textit{Sprechaktes} ist, der auf der höchsten Satzebene bezeichnet wird. Auf der \textit{interpersonellen Ebene} wird folglich der Sprechakt dargestellt, indem ein Satz mit Hilfe eines abstrakten illokutionären Prädikats (\textit{ILL}) gebildet wird, das die kommunikative Relation zwischen Sprecher, Adressat und der Proposition spezifiziert. (\ref{132}) bietet einen Überblick über die angenommenen Schichten, die Denotate und die Zuordnung zu den zwei Ebenen.

\begin{exe}
	\ex\label{132}
     \begin{tabular}[t]{|l|l|l|l|}
     		\hline
     		Ebene & Schicht & Denotat & Variable\\
            \hline
            interpersonell & Satz & Sprechakt & $\textrm{E}_{\textrm{1}}$\\
             \hline
             & Proposition & propositionaler Inhalt & $\textrm{X}_{\textrm{1}}$\\
             \hline
             repräsentationell & Prädikation & Sachverhalt & $\textrm{e}_{\textrm{1}}$\\
             \hline
             & Prädikat & Eigenschaft oder Relation & $\textrm{pred}_{\beta}$\\
             \hline
             & Term & Individuum oder Entität & $\textrm{x}_{\textrm{1}}..\textrm{x}_{\textrm{n}}$\\
       		 \hline
      \end{tabular}\\
      \hbox{}\hfill\hbox{\citet[46]{Dik1989} und \citet[130]{Hengeveld1989}}
\end{exe}
(\ref{133}) dient der Illustration der Schichten (vgl. \citealt[4]{Hengeveld1990}).

\begin{exe}
	\ex\label{133}
	Illustration der Schichten\\
	\begin{tabular}[t]{ll}
  	Satz & ($\textrm{E}_{\textrm{1}}$: Ist Stephan gegangen? ($\textrm{E}_{\textrm{1}}$))\\
  	Proposition & ($\textrm{X}_{\textrm{1}}$: Stephan ist gegangen ($\textrm{X}_{\textrm{1}}$))\\
  	Prädikation & ($\textrm{e}_{\textrm{1}}$: Stephans Gehen ($\textrm{e}_{\textrm{1}}$))\\
	Prädikat & (pred: gehen)\\
	Term & ($\textrm{x}_{\textrm{1}}$: Stephan ($\textrm{x}_{\textrm{1}}$))\\  	
\end{tabular}
\end{exe}
Die Proposition $\textrm{X}_{\textrm{1}}$ (\glq Stephan ist gegangen\grq {}) macht den Inhalt aus, der in der Entscheidungsfrage $\textrm{E}_{\textrm{1}}$ (\textit{Ist Stephan gegangen?}) erfragt wird. Innerhalb der Proposition $\textrm{X}_{\textrm{1}}$ wird Bezug genommen auf den Sachverhalt $\textrm{e}_{\textrm{1}}$ (\glq Stephans Gehen\grq {}), der sich durch das einstellige Prädikat (\textit{gehen}) und das Individuum \textit{Stephan} konstituiert.\\
\newline
\noindent
\textbf{(Die Ordnung von) Operatoren (auf den Schichten des Satzes)}\\
\noindent
Auf diesen einzelnen Schichten eines Satzes können nun Objekte zweier Naturen wirken: \textit{Operatoren} \is{Operator} und \textit{Satelliten} \is{Satellit}. Diese beiden Einheiten unterscheiden sich im Wesentlichen darin, dass erstere grammatische und letztere lexikalische Ausdrucksmittel darstellen. Wird das Tempus beispielsweise morphosyntaktisch durch die Flexion markiert, wird diese grammatische Kategorie im Satz durch einen Operator realisiert. Tritt ein Adverb zur Tempusmarkierung auf, liegt die Kodierung durch einen Satelliten vor.

Grammatische Ausdrucksmittel, die z.B. Aspektunterschiede kodieren, werden zu den Operatoren gezählt, die auf der Satzschicht des Prädikats wirken. In (\ref{134}) beeinflusst die Perfektiv/Imperfektiv-Unterscheidung die Momenthaftigkeit des Sachverhalts, die im Falle des imperfektiven Progressivs nicht gegeben ist (vgl. \citealt[134]{Hengeveld1989} unter Bezug auf \citealt[43]{Comrie1976}). 

\begin{exe}
	\ex\label{134} 
	Prädikat
		\begin{xlist}	
			\ex\label{134a} The soldiers reached the summit. $[\textrm{+ momentan}]$ ($\pi_{\textrm{1}}$)
			\ex\label{134b} The soldiers were reaching the summit. $[\textrm{−momentan}]$ ($\pi_{\textrm{1}}$) 
		\end{xlist}
	\hfill\hbox{\citet[134]{Hengeveld1989}}	
\end{exe}														
Auf der Ebene der \textit{Prädikation} \is{Prädikation} werden nicht interne Eigenschaften des Sachverhalts modifiziert und spezifiziert. Die ausgedrückten Unterscheidungen beziehen sich hier auf das Auftreten des Sachverhalts in der realen bzw. erdachten Welt. Zu den \textit{Prädikationsoperatoren} ($\pi_{\textrm{2}}$) \is{Prädikationsoperator} zählen deshalb z.B. die Tempusmarkierung (vgl. (\ref{135})), die Häufigkeit des Auftretens und Ausdrücke der \textit{objektiven Modalität} \is{objektive Modalität} (vgl. \citealt[135-138]{Hengeveld1989}).

\begin{exe}
	\ex\label{135} 
	Prädikation
		\begin{xlist}	
			\ex\label{135a} I crossed the street. ($\pi_{\textrm{2}}$)
			\ex\label{135b} I had crossed the street. ($\pi{\textrm{2}}$) 
	\hfill\hbox{\citet[136]{Hengeveld1989}}	
	\end{xlist}
\end{exe}
\textit{Propositionsoperatoren} ($\pi{\textrm{3}}$)) \is{Propositionsoperator} werden vom Sprecher verwendet, um anzugeben, zu welchem Grad er sich zur Wahrheit des vermittelten Inhalts bekennt. Unter Propositionsoperatoren fallen grammatische Ausdrucksmittel der \textit{subjektiven Mo\-dalität}, die anzeigen, wie \is{subjektive Modalität} sicher sich der Sprecher seines vermittelten Inhaltes ist (vgl. die Markierung \textit{ski} (\glq sicher\grq {} (\textsc{cert}) in (\ref{136})) sowie Evidenzialitätsmarker, die Angaben über die Quelle der Information machen (vgl. \citealt[138-140]{Hengeveld1989}). 

\begin{exe}
	\ex\label{136}
	Proposition\\
	\gll wac\'{e}o \'{u}ixi        a   \'{a}ciwi         \textbf{\textit{ski}} ($\pi_{\textrm{3}}$) (Hidatsa, \citealt{Matthews1964})\\
	Mann Antilope er aufgespürt \textsc{cert}\\
	\glt Der Mann hat gewiss eine Antilope aufgespürt.	
\hfill\hbox {\citet[139]{Hengeveld1989}}	
\end{exe}
Durch Illokutionsoperatoren ($\pi_{\textrm{4}}$) \is{Illokutionsoperator} kann der illokutionäre Wert eines Satzes modifiziert oder spezifiziert werden. Zwei Strategien der Verwendung von Operatoren auf dieser Ebene sind die Verstärkung und Abschwächung von Sprechakten (vgl. \citealt[140-141]{Hengeveld1989}). Das Beispiel aus dem Chinesischen in (\ref{137}) illustriert die letztere Art der Modifikation eines Sprechaktes. Die abschwächende Partikel \textit{a} bzw. \textit{ya} reduziert die Kraft des Sprechaktes. In diesem Fall geht dies mit einer Wirkung einher, die – im Vergleich mit den jeweils unmodifizierten Versionen – von \citet[140]{Hengeveld1989} als \glqq viel freundlicher\grqq{} und \glqq viel weicher\grqq{} beschrieben wird. 

\begin{exe}
	\ex\label{137}
	Illokution\\
	\gll Chi-fan          a/ya ($\pi_{\textrm{4}}$) (Mandarin Chinesisch, \citealt{Li1981})\\
	essen-Nahrung \textsc{mit}\\
	\glt Iss, OK?!	
\hfill\hbox {\citet[140-141]{Hengeveld1989}}	
\end{exe}
Treten Operatoren in einem Satz nun gereiht auf, folgt ihre Abfolge dem \textit{Prinzip der zentripetalen Orientierung} (vgl. (\ref{130})) \is{Prinzip der zentripetalen Orientierung} bzw. dem \textit{Spezifischen Prinzip} \is{Spezifische Prinzipien} aus (\ref{131}). So steuert die Präferenzregel beispielsweise, dass die Abfolge der Kodierung von Aktionsart, Tempus und Evidenzialität in Beispiel (\ref{138}) genau entlang dieser Reihenfolge erfolgt.
	
\begin{exe}
	\jamwidth=5cm\relax
	\ex\label{138}
	\gll W\'{i}ra   i   \'{a}p\'{a}ari   ki    stao          wareac. \\
	Baum es wächst \textsc{ingr} \textsc{rem.past} \textsc{quot}\\
	\glt Sie sagen, der Baum fing vor langem an zu wachsen.
	\newline
\hbox{}\hfill\hbox {\citet[141]{Hengeveld1989}}	
\end{exe}	
\textit{\'{A}p\'{a}ari} ist das Prädikat ($\textrm{pred}_{\beta}$), \textit{ki} kodiert die inchoative Aktionsart (= Beginn), \textit{stao} markiert das Tempus (= Vergangenheit) und \textit{wareac} ist ein Evidenzialitätsmarker, der anzeigt, dass es sich um ein Zitat handelt. Die vorliegenden Kodierer der jeweiligen Information werden alle als Operatoren aufgefasst. Entlang der obigen Ausführungen zur Wirkungsweise von Operatoren auf den verschiedenen Satzschichten entlang der jeweils kodierten Information handelt es sich bei dem Aspektmarker \textit{ki} um einen Prädikatsoperator ($\pi_{\textrm{1}}$), \textit{stao} ist als Tempuskodierer ein Prädikationsoperator ($\pi_{\textrm{2}}$) und \textit{wareac}, das Evidenzialität ausdrückt, zählt zu den Propositionsoperatoren ($\pi_{\textrm{3}}$). Das Prädikat (\textit{\'{a}p\'{a}ari}) entspricht in (\ref{138}) dem Stamm.

Vergleicht man die durch das Prinzip der zentripetalen Orientierung vorhergesagte Ordnung mit der tatsächlich in (\ref{138}) auftretenden Reihung der Operatoren, entspricht die Abfolge der Vorhersage.

\begin{exe}
	\ex\label{139} 
		$\textrm{Pred}_{\beta}\pi_{\textrm{1}} \pi_{\textrm{2}} \pi_{\textrm{3}}$	
\end{exe}
Die Abfolgebeschränkungen der MPn bringt Vismans mit dem Prinzip aus (\ref{130}) in Verbindung, indem er MPn den Status von Operatoren zuschreibt (zur Motivation dieser Annahme vgl. \citealt[129-139]{Vismans1994}) und weiter argumentiert, dass MPn als Operatoren nicht alle auf derselben Schicht anzusetzen sind. Es lassen sich verschiedene Typen von MPn ausmachen, die je einer von drei der angenommenen Satzschichten zugeordnet werden können. Die festen Abfolgen ergeben sich dann wiederum unter Bezug auf (\ref{130}), da die verschiedenen MP-Typen (wie alle Operatoren) hinsichtlich des Stammes zentripetal geordnet sind, wobei der Stamm hierbei der Fokuskonstituente des Satzes entspricht (vgl. \citealt[142]{Vismans1994}). In diesem Sinne entspricht die Abfolge in einer MP-Kombination der Zugehörigkeit zu den (ebenfalls) (hierarchisch) geordneten Satzschichten.\\
\newline
\noindent
\textbf{Zuordnung von Modalpartikeln und Satzschichten}\\
Wie bereits angeführt, nimmt Vismans an, dass sich verschiedene Typen von MPn ausmachen lassen, in dem Sinne, dass sie zu Klassen zusammengefasst werden können, die sich dadurch konstituieren, dass ihre Mitglieder auf verschiedenen Satzschichten wirken. Die konkrete Zuordnung der untersuchten MPn und ihrer Schichten ist in (\ref{140}) notiert.

\begin{exe}
	\ex\label{140}
	\begin{tabular}[t]{|l|l|}
	\hline
  	MP & Operatortyp\\
  	\hline
  	\textit{eens}, \textit{even} & Prädikation ($\pi_{\textrm{2}}$)\\
  	\hline
  	\textit{ook}, \textit{toch}, \textit{maar} & Proposition ($\pi_{\textrm{3}}$)\\
  	\hline
	\textit{dan}, \textit{nou}, \textit{misschien}, \textit{soms} & Illokution ($\pi_{\textrm{4}}$)\\  	
	\hline
\end{tabular}
\end{exe}
Vismans' Methode, die ihn zur Ordnung in (\ref{140}) führt und die im Folgenden anhand einiger weniger Beispiele illustriert wird, ist die Nebensatzeinbettung von MPn. Da angenommen wird, dass die Schichten des Satzes hierarchisch geordnet sind, enthalten sie sich in absteigender Reihenfolge (Illokution > Proposition > Prädikation > Prädikat). Wenn eine Schicht vorhanden ist, sind auch alle dieser Ebene untergeordneten Schichten vorhanden. Für eingebettete Nebensätze wird angenommen, dass je nach Nebensatztyp nicht stets alle Schichten des Satzmo\-dells auftreten. Prinzipiell gilt, dass keine Operatoren einer nicht vorhandenen Schicht auftreten können. D.h. ein der jeweils hierarchisch höchsten Schicht zugeordneter Operator, der im betrachteten Komplementsatz lizensiert ist, zeigt an, welche maximale Schicht vorliegt bzw. dass er dieser Schicht zuzuordnen ist. 

Die drei Komplemente, die für Vismans' weitere Argumentation relevant sind, sind Illokutions-, Propositions- und Prädikationskomplemente. Zu den Prädikationskomplementen \is{Illokutionskomplement} werden \is{Propositionskomplement} beispielsweise \textit{te}-Infinitive \is{Prädikationskomplement} gezählt (vgl. (\ref{141})).

\begin{exe}
	\ex\label{141}
	\gll Ik heb besloten [(om) er een boek over te schrijven].\\
	Ich habe beschlossen (für) es ein Buch über zu schreiben\\
	\glt Ich habe beschlossen, darüber ein Buch zu schreiben.
	\newline
	\hbox{}\hfill\hbox {\citet[152]{Vismans1994}}	
\end{exe}
Dass das Komplement hier eine reduzierte Satzschicht aufweist (die höchste vorliegende Schicht ist die der Prädikation) lässt sich darüber ableiten, dass zwar temporale Satelliten \is{Satellit} (Satelliten des Typs ($\phi_{\textrm{2}}$)) (vgl. (\ref{142a})), nicht aber propositionale Satelliten wie die Angabe der Quelle in (\ref{142b}) (Typ $\phi_{\textrm{3}}$-Satelliten) der hierarchisch höheren propositionalen Satzschicht auftreten können.

\begin{exe}
	\ex\label{142} 
	\begin{xlist}	
			\ex\label{142a} 
			\gll Ik heb besloten [(om) er op vakantje een boek over te schrijven].\\
			 Ich habe beschlossen (für) es im Urlaub ein Buch über zu schreiben\\
			\glt Ich habe beschlossen, darüber im Urlaub ein Buch zu schreiben.
			\ex\label{142b}
			\gll *Ik heb besloten [er volgens Piet een boek over te schrijven].\\
			 Ich habe beschlossen es nach Piet ein Buch über zu schreiben\\
			\glt Ich habe beschlossen, Piet zufolge, ein Buch darüber zu schreiben.		
	\end{xlist}
	\hfill\hbox{\citet[152]{Vismans1994}}
\end{exe}
Die maximal vertretene Schicht ist im Falle des \textit{te}-Infinitivsatzes folglich \is{Prädikation} die Prädikation. 

Mit dieser Methode (vgl. \citealt[148-155]{Vismans1994}) leitet Vismans ab, dass es sich bei infiniten Komplementen wie in (\ref{143}) um Propositionen handelt, d.h. die Schicht der Proposition die höchste vertretene Satzschicht ist, sowie dass in (\ref{144}) ein Illokutionskomplement vorliegt. (\ref{145}) ist ein weiteres Beispiel für ein Prädikationskomplement.

\begin{exe}
	\ex\label{143}
	Propositionskomplement\\
	\gll Ik heb je gezegd me te SCHRIJven.\\
	Ich habe dir gesagt  mir zu schreiben\\
	\glt Ich habe dir angeordnet, mir zu schreiben.	
\end{exe}

\begin{exe}
	\ex\label{144}
	Illokutionskomplement\\
	\gll Ik heb je gevraagd of je me wilt SCHRIJven.\\
	Ich habe dich gefragt ob du mir willst schreiben\\
	\glt Ich habe dich gefragt, ob du mir schreiben willst.	
\end{exe}

\begin{exe}
	\ex\label{145}
	Prädikationskomplement\\
	\gll Ik heb je gevraagd om me te SCHRIJven.\\
	Ich habe dich gefragt um mir zu schreiben\\
	\glt Ich habe dich gebeten, mir zu schreiben.	
	\hbox{}\hfill\hbox {\citet[156]{Vismans1994}}
\end{exe}		
Ausgehend von den Annahmen, dass sich verschiedene Typen von Komplement\-sätzen unterscheiden lassen sowie (abgeleitet über Auftretensmöglichkeiten von Operatoren und Satelliten) Schlüsse auf die Schichtenzugehörigkeit erlaubt sind, bestimmt Vismans, welche der von ihm betrachteten MPn auf der Ebene der Prädikation, der Proposition oder der Illokution anzusiedeln sind.

Wenn MPn auf der Ebene der Prädikation anzusiedeln sind, sollten sie in allen drei Typen von Komplementsätzen einzusetzen sein. Sie wirken dann auf der Prädikationsschicht \is{Prädikationsschicht}, die aber ebenfalls vorhanden ist, wenn zusätzlich die Proposi\-tions- und Illokutionsschicht auftreten. Wenn eine MP auf der Ebene der Proposition anzusiedeln ist, sollte sie in Propositions- und Illokutionskomplementen \is{Propositionsschicht} auftreten \is{Illokutionsschicht}können. Handelt es sich bei den jeweiligen MPn um Illokutionsope\-ratoren \is{Illokutionsoperator}, sollten sie nur dann einsetzbar sein, wenn ein Illokutionskomplement vorliegt.

Unter einer solchen Überlegung über den Zusammenhang zwischen in der Struktur vorhandenen Satzschichten und dem Auftreten verschiedener Typen von Operatoren gilt es letztlich herauszufinden, welche MP auf einer bestimmten Schicht nicht, wohl aber auf der (nächst) höhe\-ren Ebene auftreten kann. In Prädikationskomplementen \is{Prädikationskomplement} können beispielsweise \textit{eens} und \textit{even} vorkommen (vgl. (\ref{146})). Abgesehen von satzmodalen Inkompatibilitäten im Falle von \textit{dan}, \textit{toch} und \textit{maar}, die zum unabhängigen Ausschluss führen, folgert Vismans, dass die MPn, die im Prädikationskomplement \is{Prädikationskomplement} nicht auftreten können, auf einer höhe\-ren Schicht anzusiedeln sind, die in (\ref{146}) nicht auftritt.

\begin{exe}
	\ex\label{146} 
	Ik heb je gevraagd me eens te SCHRIJven.\\
	Ik heb je gevraagd me even te SCHRIJven.\\
	Ik heb je gevraagd me *dan te SCHRIJven.\\
	Ik heb je gevraagd me *nou te SCHRIJven.\\
	Ik heb je gevraagd me *ook te SCHRIJven.\\
	Ik heb je gevraagd me *toch te SCHRIJven.\\
	Ik heb je gevraagd me *maar te SCHRIJven.\\
	Ik heb je gevraagd me *misschien te SCHRIJven.\\
	Ik heb je gevraagd me *soms te SCHRIJven.
	\hfill\hbox {\citet[157]{Vismans1994}}
\end{exe}
In einem Propositionskomplement \is{Propositionskomplement} können (neben \textit{eens} und \textit{even}) auch \textit{maar}, \textit{ook} und \textit{toch} auftreten (vgl. (\ref{147})), woraus Vismans' Argumentation nach zu folgern ist, dass diese Propositionsoperatoren \is{Propositionsoperator} darstellen. In (\ref{146}) können sie nicht auftreten, weil die Sätze die Propositionsschicht nicht aufweisen. Die Prädikationsschicht ist in der Propositionsschicht aber enthalten, weshalb \textit{eens} und \textit{even} in (\ref{147}) ebenfalls auftreten können (vgl. (\ref{143})).

\begin{exe}
	\ex\label{147} 
	Ik   heb   je  gezegd me eens te SCHRIJven.\\
	Ik   heb   je  gezegd me even te SCHRIJven.\\
	Ik   heb   je  gezegd me toch te SCHRIJven.\\
	Ik   heb   je  gezegd me ook te SCHRIJven.\\
	Ik   heb   je  gezegd me maar te SCHRIJven.\\
	Ik   heb   je  gezegd me *dan te SCHRIJven.\\
	Ik   heb   je  gezegd me *nou te SCHRIJven.\\
	Ik   heb   je  gezegd me *misschien te SCHRIJven.\\
	Ik   heb   je  gezegd me *soms te SCHRIJven.
	\hfill\hbox {\citet[158]{Vismans1994}}
\end{exe}
Vismans weist mit derselben Argumentation ebenfalls nach, dass \textit{dan}, \textit{nou}, \textit{misschien} und \textit{soms} auf Ebene der Illokution anzusiedeln sind (vgl. \citealt[159]{Vismans1994}). Die MPn tieferer Ebenen (Proposition, Prädikation) können erneut auftreten, weil die tieferen Satzschichten in den höheren enthalten sind. 

Unter Anwendung der hier erläuterten Methode gelangt Vismans zur Zuordnung aus (\ref{140})).\\
\newline
\noindent
\textbf{Die Anordnung der Modalpartikeln in Kombinationen}\\
Unter der nun motivierten Annahme dass MPn Operatoren sind und entlang ihrer Schichtenzuordnung einem von drei Typen von Operatoren angehören, gelten auch die Prinzipien, die für Operatoren generell gelten. Insbesondere sind sie entsprechend dem \textit{Principle of Centripetal Orientation} \is{Prinzip der zentripetalen Orientierung} (s.o.) aus \citet{Dik1997} geordnet (vgl. (\ref{131})).

Die separat motivierten Annahmen zusammengeführt, sagt Vismans in Kombinationen die Ordnungen in (\ref{149}) voraus. Der Typ (Deklarativ, Interrogativ, Imperativ) unterscheidet hier verschiedene Strukturtypen, mit denen sich Direktive, die Vismans ausschließlich betrachtet, realisieren lassen.
					    
\begin{exe}
	\ex\label{149}
	\scriptsize
	\begin{tabular}[t]{|l|l|}
	\hline
  	Typ & Reihenfolge in Kombination\\
  	\hline
  	Deklarativsatz & \textit{ook} ($\pi_{\textrm{3}}$), \textit{maar} ($\pi_{\textrm{3}}$), \textit{eens} ($\pi_{\textrm{2}}$), \textit{even} ($\pi_{\textrm{2}}$)\\
  	\hline
  	Interrogativsatz & \textit{nou} ($\pi_{\textrm{4}}$), \textit{misschien}/\textit{soms}* ($\pi_{\textrm{4}}$), \textit{ook} ($\pi_{\textrm{3}}$), \textit{eens} ($\pi_{\textrm{2}}$), 			\textit{even} ($\pi_{\textrm{2}}$)\\
  	\hline 
  	Imperatisatz & \textit{dan}/\textit{nou}* ($\pi_{\textrm{4}}$), \textit{toch} ($\pi_{\textrm{3}}$), \textit{maar} ($\pi_{\textrm{3}}$), \textit{eens} ($\pi_{\textrm{2}}$), \textit{even} ($\pi_{\textrm{2}}$)\\
  	\hline
\end{tabular}\\
* = austauschbar
\end{exe}
Wenn MPn der verschiedenen Schichten (4 = Illokution, 3 = Proposition, 2 = Prädikation) auftreten, d.h. $\pi_{\textrm{4}}$-, $\pi_{\textrm{3}}$- oder $\pi_{\textrm{2}}$-Operatoren, tritt stets eine Ordnung entlang der drei Satzebenen ein. Dies ist z.B. der Fall im anfangs angeführten Beispielen in (\ref{150}) sowie in (\ref{151}).

\begin{exe}
	\ex\label{150} 
	Geef de boeken \textbf{dan} ($\textrm{MP}_{\pi\textrm{4}}$) \textbf{nu} ($\textrm{MP}_{\pi\textrm{4}}$) \textbf{toch} ($\textrm{MP}_{\pi\textrm{3}}$) \textbf{maar} ($\textrm{MP}_{\pi\textrm{3}}$) \textbf{'es} ($\textrm{MP}_{\pi\textrm{2}}$) \textbf{even} ($\textrm{MP}_{\pi\textrm{2}}$) hier.\\
	Gib die Bücher her!
	\hbox{}\hfill\hbox{\citet[98]{Hoogvliet1903}}
\end{exe}

\begin{exe}
	\ex\label{151} 
	Doe de deur \textbf{nou} ($\textrm{MP}_{\pi\textrm{4}}$) \textbf{toch} ($\textrm{MP}_{\pi\textrm{3}}$) \textbf{eens} ($\textrm{MP}_{\pi\textrm{2}}$) DICHT.\\
	Mach die Tür zu!
	\hbox{}\hfill\hbox{\citet[163]{Vismans1994}}	
\end{exe}

\subsubsection{Sprechakt, Thema, Argumentation, Inhalt}
\label{sec:stai}
Die in \citet{Vismans1994} ausgearbeitete und funktionsgrammatisch modellierte Idee, die Abfolge von MPn über das (in der Ordnung vorbestimmte) Aufeinandertreffen von Elementen verschiedener hierarchischer Ebenen abzuleiten, findet sich in ihrer Grundintuition auch in \citet{Ickler1994} wieder. 

Der Autor diskutiert in seiner Betrachtung der Bedeutung von Einzelpartikeln auch die seiner Meinung nach unmögliche Umkehrung von \textit{denn eigentlich} (vgl. \citealt[384]{Ickler1994}) und \textit{ja doch} (vgl. \citealt[404]{Ickler1994}) (vgl. (\ref{152}), (\ref{153})). 

\begin{exe}
	\ex\label{152} 
	Wie spät ist es \textbf{denn eigentlich}?
\end{exe}
\vspace{-0.65cm}	
\begin{exe}
	\ex\label{153} 
	Wir sind \textbf{ja doch} alte Bekannte.
\end{exe}		
Ickler vertritt die Annahme, dass nicht alle MPn von der gleichen interpretatorischen Natur sind in dem Sinne, dass sie nicht alle auf der gleichen Ebene wirken (vgl. \citealt[379]{Ickler1994}). Über die unterschiedlichen Ebenenzugehörigkeiten argumentiert er, unter welchen Abfolgen die Kombinationen interpretierbar sind bzw. eben nicht.

Für Fragen, in denen \textit{denn} auftritt (vgl. (\ref{154})), nimmt der Autor an, dass sie durch Bestandteile der Gesprächssituation veranlasst sind. 
	
\begin{exe}
	\ex\label{154} 
	Wie spät ist es \textbf{denn}?
\end{exe}	
Fragen mit \textit{eigentlich} (vgl. (\ref{155})) führen Ickler zufolge die Interpretation herbei, dass die Frage als Vertiefung eines Themas verstanden wird. Das Thema der Frage sei im Großen und Ganzen gegeben und werde nur modifiziert (vgl. \citealt[384]{Ickler1994}).

\begin{exe}
	\ex\label{155} 
	Wie spät ist es \textbf{eigentlich}?
\end{exe}
In beiden Fällen sind die Fragen demnach im Vorgängerkontext verankert. In diesem Sinne ist die Funktion von \textit{denn} und \textit{eigentlich} eine ähnliche. Trotz funktionaler Ähnlichkeit unter dieser Perspektive operieren die beiden MPn jedoch auf unterschiedlichen Ebenen: Die Partikel \textit{eigentlich} wirkt auf der Ebene des thematischen Zusammenhangs, d.h. der Inhalt der Frage motiviert sich aus dem Kontext. Tritt \textit{denn} in einer Frage auf, geht es um die Motivation der Frage an sich (d.h. der Frage als Zug im Diskurs) und die MP wirkt in diesem Sinne auf der Ebene des Sprechaktes. 

Die Kombination von \textit{denn} und \textit{eigentlich} in Fragen führt nach Ickler zu folgender Interpretation der Fragen: Eine Frage, die inhaltlich als Vertiefung eines gegebenen Themas verstanden wird (\textit{eigentlich}), ist als diskursiver Zug an sich durch den Vorgängerkontext motiviert (\textit{denn}). Icklers Grundannahme, dass MPn auf verschiedenen Ebenen wirken, spiegelt sich im Falle von \textit{denn} und \textit{eigentlich} folglich darin wider, dass \textit{eigentlich} auf der \textit{Inhaltsebene} \is{Inhaltsebene} und \textit{denn} auf der \textit{Sprech\-aktebene} \is{Sprechaktebene} wirkt. In Icklers Ableitung der festen Abfolge von \textit{denn} und \textit{eigentlich} geht neben der konkreten Einzelbeschreibung der beteiligten MPn und ihrer Zuordnung zu der jeweiligen Ebene die folgende grundsätzliche Annahme ein: \glqq Jedenfalls liegen die weiter rechts stehenden Modalpartikeln im Operationsbereich der weiter links stehenden.\grqq{} (\citealt[379]{Ickler1994}) Wenngleich Ickler keine präzise Ausbuchstabierung der Gestalt bzw. Ordnung verschiedener derartiger Schichten (\glqq Operationsbereich\grqq{}) vornimmt, ist es eine intuitiv plausible Annahme, dass die Sprechaktebene der thematischen Ebene übergeordnet ist und die MP \textit{eigentlich}, die auf Ebene des Frageinhalts operiert, im Wirkungsbereich von \textit{denn} liegt und deshalb wiederum die Sprechakt-MP der Thema-MP vorangeht. Die feste Reihenfolge ergibt sich nach Ickler schließlich aus der unmöglichen Interpretation der umgekehrten Abfolge (vgl. \citealt[384]{Ickler1994}). Die sich unter der umgekehrten Abfolge einstellende Interpretation führt er nicht weiter aus, aber unter der Annahme der Geordnetheit der Schichten, auf denen die MPn wirken, scheint plausibel, dass die untergeordnete themati\-sche Ebene nicht die übergeordnete Sprechaktebene in ihren Wirkungsbereich nehmen kann. 

Während die relevanten Ebenen im Falle von \textit{denn} und \textit{eigentlich} als Sprech\-aktebene und thematische Ebene angenommen werden, führt Ickler in der Ablei\-tung der Abfolge \textit{ja doch} zwar prinzipiell eine analoge Argumentation, er nimmt mit den Ebenen der \textit{Argumentation} \is{Argumentationsebene} und des \textit{Inhalts} \is{Inhaltsebene} allerdings Wirkungsbereiche anderer Formate an, die ebenso wie \textit{Sprechakt} \is{Sprechaktebene} und \textit{Thema} \is{Themaebene} in Interaktion treten. Für \textit{ja} nimmt Ickler an, dass es einer Aussage in der Argumentation ihren Stellenwert zuweist (vgl. \citeyear[404]{Ickler1994}). In einem argumentativen Zusammenhang werde eine Aussage bestätigt/bekräftigt und in diesem Sinne auf ihre Gültigkeit bestanden (vgl. \citeyear[399]{Ickler1994}). Der Wirkungsbereich von \textit{ja} ist dieser Auffassung zufolge die Ebene der Argumentation. 
	
Die MP \textit{doch} kennzeichnet nach Ickler einen inhaltlichen Gegensatz (vgl. \citeyear[493]{Ickler1994}). In den einzelnen Verwendungsfällen gilt es deshalb, den konkreten adversativen Bezugspunkt herauszufinden. In einem Dialog wie in (\ref{156}) lässt sich der Gegensatz so auffassen, dass der Adressat den Sachverhalt der \textit{doch}-Äußerung nicht berücksichtigt hat.

\begin{exe}
	\ex\label{156} 
	A: Wir müssen noch Peter abholen.\\
	B: Peter ist \textbf{doch} krank. Der kommt ganz sicherlich nicht mit.
\end{exe}
Da der Autor den Beitrag von \textit{doch} als Konstitution eines inhaltlichen Gegensatzes auffasst, lässt sich die Inhaltsebene als Wirkungsbereich dieser MP auffassen. Für die Kombination aus \textit{ja} und \textit{doch} argumentiert Ickler nun, dass sich die Abfolge \textit{ja doch} daraus ableiten lässt, dass die Inhaltsebene in den Wirkungsbereich der Argumentationsebene falle. Nach dieser Interpretation wird in der Argumentation auf die Gültigkeit der Aussage bestanden, die inhaltlich einen Gegensatz ausdrückt. Erst drückt das \textit{doch} einen inhaltlichen Gegensatz aus, anschließend wird der derart modifizierten Aussage ihr entsprechender Stellenwert im argumentativen Zusammenhang zugewiesen. Im gleichen Sinne, in dem die Sprechaktebene der Themaebene übergeordnet ist, ist die Inhaltsebene der Argumentationsebene untergeordnet. Mit diesem Verhältnis geht wiederum einher, dass das \textit{ja} dem \textit{doch} in der Kombination vorangeht. Die umgekehrte Abfolge \textit{doch ja} schließt Ickler aus, da er das umgekehrte Verhältnis und somit die umgekehrte Interaktion zwischen Inhalts- und Argumentationsebene \is{Inhaltsebene} für \is{Argumentationsebene} nicht möglich hält (vgl. \citeyear[404]{Ickler1994}).

\subsection{Inputbedingungen}
\label{subsec:input}
Während bei \citet{Vismans1994} und \citet{Ickler1994} die Zuordnung der MPn zu verschiedenen semantischen Ebenen für die Abfolgen in Kombinationen entscheidend ist, macht der Ansatz von \citet{Doherty1985} und (in einem bestimmten Einzelfall, auf den ich nur verweisen werde) auch \citet{Rinas2007}, die bereits in anderen Zusammenhängen in Abschnitt~\ref{sec:ass} und \ref{sec:skopus} behandelt wurden, die Art der jeweils verfügbaren Bedeutung für die Sequenzierung verantwortlich. Die MPn erfordern für ihre erfolgreiche Applikation im Satz die Verfügbarkeit bestimmter Bedeutungsanteile. Je nach Anordnung der Abfolge des Wirkens der MPn, sind die nötigen semantischen Objekte/Interpretationen verfügbar oder nicht. Im ersten Fall resultieren akzeptable Abfolgen, im zweiten inakzeptable Reihungen. In diesem Sinne spiegeln die zulässigen Sequenzen akzeptable Anwendungsordnungen der Partikeln wider.

In Abschnitt~\ref{sec:ass} wurden die Ableitungen aus \citet{Doherty1985, Doherty1987} zu den Abfolgebeschränkungen bei der Kombination von \textit{ja}, \textit{doch} und \textit{wohl} ausgeführt. Das entscheidende Kriterium ist in diesen Ansätzen die assertive Stärke \is{assertive Stärke} der ge\-reihten MPn. Innerhalb einer Kombination muss der Grad der assertiven Stärke abnehmen. Im Falle der drei MPn \textit{ja}, \textit{doch} und \textit{wohl} verläuft die Skala abnehmender Assertivität entlang der Ordnung in (\ref{157}). Diese Ordnung entspricht wiederum der relativen Abfolge dieser MPn in Kombinationen.

\begin{exe}
	\ex\label{157} 
	Relative Ordnung\\
	$\underrightarrow{\text{\textit{ja} > \textit{doch} > \textit{wohl}}}$\\
	abnehmende assertive Stärke
\end{exe}	
Die Zuweisung der unterschiedlichen Grade assertiver Stärke nimmt \citet{Doherty1985} unter Bezug auf die von ihr vorgeschlagenen Einzelbeschreibungen der MPn in (\ref{158}) bis (\ref{160}) vor.

\begin{exe}
	\ex\label{158} 
		\textit{ja}: Ass$(\textrm{E}_{\textrm{S}}$(\textrm{p})$)$ und IM$(\textrm{E}_{\textrm{X}}$(\textrm{p})$)$
			\hfill\hbox {\citet[80]{Doherty1985}}
\end{exe}
\vspace{-0.65cm}
\begin{exe}
	\ex\label{159} 
	\textit{doch}: $\textrm{Ass}^{\prime} (\textrm{E}_{\textrm{S}}(\textrm{p})) \textrm{und IM(neg}_{\textrm{X}}(\textrm{p}))$	
	\hfill\hbox {\citet[71]{Doherty1985}}
\end{exe}
\vspace{-0.65cm}
\begin{exe}
	\ex\label{160} 
	\textit{wohl}: $\textrm{Pres}(\alpha (\textrm{p})) \ \textrm{und}\ \textrm{Ass}^{\prime} (\textrm{VERMUTUNG}_{\textrm{S}}(\alpha = \textrm{E}))$
	\hfill\hbox{\citet[82]{Doherty1985}}
\end{exe}
Mit dem Kriterium der \textit{assertiven Stärke} \is{assertive Stärke} fängt sie die jeweils zu beobachtenden unterschiedlichen Grade der Verbindlichkeit der Sprecherhaltungen auf. Von (\ref{158}) nach (\ref{160}) legt sich der Sprecher zunehmend weniger fest: In (\ref{158}) ist er festgelegt auf die assertive Haltung zur Einstellung im Skopus von \textit{ja} (Ass$(\textrm{E}_{\textrm{S}}$(\textrm{p})$)$), in (\ref{159}) aufgrund des \underline{potenziellen} assertiven EM schwächer auf die Einstellung E und in (\ref{160}) erneut abgeschwächt lediglich auf die Vermutungshaltung zur Einstellung im Skopus der Partikel. 

Eine derartige Reihung entlang des Kriteriums der assertiven Stärke wirkt auf den ersten Blick wie ein oberflächliches Kriterium, das allenfalls eine deskriptive Generalisierung darstellt. Betrachtet man jedoch die konkrete Anwendung des Kriteriums in \citet{Doherty1985}, wenn sie komplexe Beispiele durchrechnet, also genaue Bedeutungsangaben macht und aufschlüsselt, welche Bedeutungskomponenten durch die MPn sowie andere sprachliche Ausdrucksmittel in die Struktur gelangen, lässt sich ihr Kriterium der assertiven Stärke abstrakter als generellere Beschränkung über die Interaktion von Einstellungen auffassen. Bei diesen Interaktionen spielt vor allem eine Rolle, auf welcher Art von Einstellungsausdruck die MPn applizieren. Gilt es, den Bedeutungsbeitrag mehrerer Partikeln nacheinander in die Gesamtstruktur zu integrieren, ist entscheidend, welcher Typ von Einstellung bereits in der Struktur vorhanden ist, wenn die jeweilige MP zur Wirkung kommt.

Als Gesamtbedeutung der Struktur in (\ref{161a}), die in umgekehrter Abfolge von \textit{ja} und \textit{doch} Doherty zufolge ungrammatisch ist (vgl. (\ref{162})), nimmt sie (\ref{161b}) an.

\begin{exe}
	\ex\label{161} 
		\begin{xlist}	
			\ex\label{161a} Konrad ist \textbf{ja}/\textbf{doch} \textbf{\textit{wahrscheinlich}} verreist.
			\ex\label{161b} $\textrm{Ass(WAHRSCHEINLICH(p)})$ und $\textrm{IM(neg}_{\textrm{x}}(\textrm{p})$ \\ und $\textrm{WAHRSCHEINLICH}_{\textrm{y}}(\textrm{p}))$
			\hfill\hbox {\citet[85]{Doherty1985}}
		\end{xlist}
\end{exe}

\begin{exe}
	\ex\label{162} 
		*Konrad ist \textbf{doch ja \textit{wahrscheinlich}} verreist.
\end{exe}      
Im Einvernehmen mit ihren bisherigen Annahmen zur Bedeutung von \textit{doch} (vgl. (\ref{159})), nimmt die Autorin an, dass der Sprecher in diesem assertiven Kontext mit $\textrm{WAHRSCHEINLICH}_{\textrm{S}}(\textrm{p})$ auf die Einstellung im Skopus des \textit{doch} festgelegt wird. Die MP \textit{ja} leistet den Beitrag, die Einstellung in ihrem Skopus zu bestätigen (vgl. $\textrm{Ass(E}_{\textrm{S}}(\textrm{p}))$ in (\ref{158})). Prinzipiell würde sich als diese Einstellung hier der durch \textit{doch} modifizierte Ausdruck anbieten. Da \textit{doch} jedoch der Autorin zufolge keine \underline{Einstellung} ausdrückt, sondern eine \underline{Haltung} gegenüber einer Einstellung und \textit{ja} als Argument aber ein Objekt des Typs E benötigt, lege \textit{ja} den Sprecher (ebenso wie \textit{doch}) auf die Einstellung \textit{WAHRSCHEINLICH(p)} fest. Sowohl \textit{ja} als auch \textit{doch} assertieren in (\ref{161a}) demnach wörtlich, dass p wahrscheinlich ist. Neben dieser wörtlichen Bedeutung implizieren die Partikeln zusätzlich verschie\-dene Inhalte: Die indirekte Einstellung, die \textit{doch} zugeschrieben wird, ist, dass eine andere Person dem durch p ausgedrückten Sachverhalt eine negative Haltung entgegenbringt (vgl. (\ref{159}) und (\ref{161b}))). Für \textit{ja} gilt in dieser Hinsicht, dass impliziert wird, dass die Einstellung im Skopus des \textit{ja} von jemand anderem geteilt wird (vgl. (\ref{159})). Für (\ref{161a}) bedeutet dies, dass auch eine andere Person p für wahrscheinlich hält. Dies sind die Bedeutungsanteile, die Doherty im Falle der akzeptablen Abfolge annimmt.\footnote{An dieser Stelle von Dohertys Argumentation wird deutlich, dass sie von einem inkrementellen bottom-to-top Struktur- und Bedeutungsaufbau ausgeht.}

Werden die beiden MPn entgegen ihrer assertiven Stärke geordnet, wirkt zu\-nächst das \textit{ja}. Es leistet den wörtlichen Beitrag, die Wahrscheinlichkeit von p zu assertieren ($\textrm{Ass(WAHRSCHEINLICH}_{\textrm{S}}(\textrm{p})$). Im nächsten Schritt gilt es dann, die Bedeutung von \textit{doch} in die Gesamtbedeutung zu integrieren. Unter Auftreten von \textit{doch} legt sich der Sprecher auf eine Einstellung E fest, die er in diesem Fall bestätigt (assertiver EM). D.h. \textit{doch} fordert eine Einstellung in seinem Skopus. Ein Einstellungs\underline{modus} (wie von \textit{ja} eingeführt) kann diese Forderung nicht erfüllen. 

Aus dieser Perspektive ist das Kriterium, das die Abfolge von \textit{ja} und \textit{doch} beschränkt, ein tiefer liegendes als das der Assertivität. Letztlich lassen sich Dohertys Ausführungen als eine Beschränkung über zulässigen und unzulässigen Input in der Anwendung von \textit{doch} und \textit{ja}, d.h. konkret erlaubten Argumenten der Prädikate, ausdeuten. In der akzeptablen Kombination \textit{ja doch} hat \textit{doch} mit \textit{WAHRSCHEINLICH(p)} [= E(p)] eine Inputform, auf der sie applizieren kann, d.h. der sie eine positive Haltung entgegenbringen kann. Gleiches gilt für \textit{ja}, das eine Einstellung benötigt, die dann im Skopus des assertiven EM auftritt ($\textrm{Ass(WAHRSCHEINLICH}_{\textrm{S}}(\textrm{p})$). In der inakzeptablen Abfolge in (\ref{162}) ist \textit{WAHRSCHEINLICH(p)} ein zulässiger Input für \textit{ja}; \textit{doch}, das sich nur auf eine Einstellung beziehen kann, hat deshalb keine Verwendung für einen EM (hier \textit{Ass}), der nach Applikation des \textit{ja} vorliegt.

Deutet man die Beschränkungen über die im Skopus der MPn erlaubten Objekte als Inputbedingungen, wie hier von mir angenommen, liegt eine Art von Regelordnung vor, die auf zulässigem Input der Wirkungsmöglichkeiten der MPn basiert. Die Argumentation zum Ausschluss mancher Abfolgen ist dann mit phonologischen Regelanordnungen vergleichbar. Im inakzeptablen Fall in (\ref{162}) liegt in dieser Analogie ein Fall von \textit{Bleeding} vor: Nach Wirken von \textit{ja} liegt (mit $\textrm{Ass(E}_{\textrm{S}}(\textrm{p})$)) Material vor, das nicht als Input des anschließend operierenden \textit{doch} dienen kann. Da \textit{doch} kein zulässiges Material finden kann, können die beiden MPn nicht in dieser Abfolge Anwendung finden. Dieses Problem ergibt sich bei der Abfolge in (\ref{161a}) nicht. Sowohl \textit{ja} als auch \textit{doch} finden mit $\textrm{E}_{\textrm{S}}(\textrm{p})$ (= $\textrm{WAHRSCHEINLICH}_{\textrm{S}}(\textrm{p})$) angemessenen Input. Beide Partikeln legen sich auf diese Vermutungseinstellung fest.

Unter solch einer Auffassung der Ableitung der Abfolgebeschränkungen in MP-Kombinationen spielt das Kriterium der Assertivität \is{Assertivität} letztlich nicht mehr die primäre Rolle. Im Vordergrund steht vielmehr die Natur \is{positionale Bedeutung} der positionalen Bedeutung. Dies gilt sowohl für die konkrete Bedeutungszuschreibung an die akzep\-table Abfolge als auch für die Ableitung der schlechten Abfolge in der umgekehrten Anordnung der Partikeln.\footnote{Wenn das Kriterium der assertiven Stärke in diesem Sinne abstrakter ausgelegt wird (weil die von den MPn semantisch geforderten Argumente jeweils andere sind), lassen sich unter allgemeinerer Fassung des Kriteriums (Natur von positionaler Bedeutung) auch weitere Fälle in die derart argumentierende Ableitung aufnehmen. Vgl. hierzu \citet[83-84]{Doherty1985} zu \textit{ja/doch nicht etwa}. Vgl. auch die Ableitung der Abfolge \textit{doch nicht etwa} in \citet[447]{Rinas2007}.}				
Bis hierhin wurden verschiedene Ansätze angeführt, die in ihren Erklärungen für die Abfolgebeschränkungen von MPn in Kombinationen auf die Interpretation der Einzelpartikeln bzw. der Kombinationen Bezug nehmen. Hierbei standen mit Aspekten wie assertiver Stärke, verschiedenen Arten von Einstellungsbedeutung, Skopusverhältnissen und Ebenen ihrer Wirkung im Satzmodell die wörtliche Bedeutung bzw. je nach Modellierung ggf. indirektere Bedeutungsbeiträge der MPn im Mittelpunkt, d.h. der Fokus lag eher auf dem inhärenten Bedeutungskern der MPn selbst. Auch in dem im folgenden Abschnitt ausgeführten Ansatz von \citet{Vriendt1991} argumentieren die Autoren unter Bezug auf interpretatorische Verhältnisse. Anders als die bisher angeführten Arbeiten nehmen sie allerdings eher pragmatische Wirkungsweisen der MPn in den Blick, indem sie sich die Rolle der Partikeln in der informationsstrukturellen Gliederung von Sätzen für ihre Ableitung der Abfolgebeschränkungen in MP-Sequenzen zu Nutze machen. Untersuchungsgegenstand der Arbeit sind MP-Kombinationen im Niederländischen.

\subsection{Thema-Rhema-Gliederung}
Es ist eine bekannte Annahme, dass sich die Bedingungen für die Serialisierung von Konstituenten im Mittelfeld nicht rein strukturell fassen lassen, sondern Aspekte der Organisation von Information Einfluss nehmen. Als ein solches informationsstrukturelles Kriterium führen \citet[44]{Vriendt1991} das \textit{Thematic Principle} \is{Thematic Principle} (vgl. (\ref{163})) an.

\begin{exe}
	\ex\label{163} 
		Thematic Principle\\
		The greater the communicative content of a constituent, the further to the right in the middle field: thematic information precedes rhematic 				information.
\end{exe} 		
Zur Illustration dieser Generalisierung führen die Autoren die Sätze in (\ref{164}) und (\ref{165}) an.

\begin{exe}
    \ex\label{164}
        \gll Ik  wou [zijn vriendin]$_{\textrm{definit}}$ [een keer]$_{\textrm{indefinit}}$ ontmoeten.\\
        Ich wollte seine Freundin ein Mal treffen\\
        \glt Ich wollte einmal seine Freundin treffen.
\end{exe}

\begin{exe}
    \ex\label{165}
        \gll Hij wou [die keer]$_{\textrm{definit}}$ [een vriendin]$_{\textrm{indefinit}}$ ontmoeten.\\
        Er wollte dieses Mal eine Freundin treffen\\
        \glt Er wollte dieses Mal eine Freundin treffen.
\end{exe}
\citet{Vriendt1991} entwickeln eine informationsstrukturelle Ableitung der festen Abfolgen von MPn in Kombinationen, die darauf fußt, dass die Abfolge von MPn in Kombinationen den Regularitäten gehorcht, die für die Serialisierung anderer Mittelfeldkonstituenten angenommen worden sind. Dies bedeutet, dass das thematische Prinzip greift. Konkret folgt aus diesen Überlegungen, dass sich für einzelne MPn nachweisen lässt, dass es thematische(re) und rhematische(re) Mitglieder dieser Klasse gibt.

Um die Abfolge in kleineren (vgl. (\ref{166})) und größeren Clustern (vgl. (\ref{167})) zu erfassen, ordnen sie die MPn in die drei Klassen in (\ref{168}).

\begin{exe}
	\ex\label{166} 
		\begin{xlist}	
			\ex\label{18a} \textit{dan maar}, \textit{toch 'es}, \textit{toch 'es even}
			\ex\label{18b} *\textit{maar toch}, *\textit{'es maar}, *\textit{even 'es}
		\end{xlist}
\end{exe}

\begin{exe}
	\ex\label{167} 
	\gll Geef de boeken dan nu toch maar eens een keertje hier.\\
         Gib die Bücher dann nun sicher nur einmal ein Mal hier\\
    \glt 
\hfill\hbox {\citet[47]{Vriendt1991}}
\end{exe}

\begin{exe}
	\ex\label{168} 
		\begin{xlist}	
			\ex\label{168a} Deiktisch-anaphorische Gruppe: \is{deiktisch-anaphorische Gruppe} \textit{dan}, \textit{nu}, \textit{toch}
			\ex\label{168b} Modale Abtönungspartikeln: \is{modale Abtönungspartikeln} \textit{maar}, \textit{wel}
			\ex\label{168c} Existenziell quantifizierende Gruppe: \is{existenziell quantifizierende Gruppe} \textit{eens}, \textit{even}
		\end{xlist}
\end{exe}
In Bezug auf die Zuordnungen in (\ref{168}) gilt, dass die Abfolge in Kombinationen der Ordnung (a) > (b) > (c) folgt. Die Klassenbildungen und entsprechenden Ka\-tegorisierungen aus (\ref{168}) motivieren die Autoren sowohl über den Bedeutungsbeitrag der MPn selbst als auch über den semantischen Beitrag ihrer jeweiligen gleichlautenden Pendants in anderen Wortarten. \citet{Vriendt1991} gehen von der Annahme aus, dass die MPn und gleichlautenden Formen im Verhältnis der Homonymie zueinander stehen und sich deshalb semantische Bezüge ausmachen lassen, die in diesem Fall ggf. in die Argumentation zur Klassenordnung eingehen. Die oben angeführte Grundidee des Ansatzes der Autoren vorausgesetzt, gilt es zu motivieren, dass die MPn der deiktisch-anaphorischen Gruppe (vgl. (\ref{168a})), die in Kombination stets am linken Rand auftreten, einen hohen Grad an Thematizität \is{Thematizität} aufweisen. Für die Klasse der e\-xistenziell quantifizierenden MPn aus (\ref{168c}) muss gezeigt werden, inwiefern sie informationsstrukturell als rhematisch einzustufen sind. Neben diesen beiden  \glq Extrema\grq {} hinsichtlich ihres Grades von Thematizität bzw. Rhematizität \is{Rhematizität} ist ebenfalls die Zwischenstellung der modalen Abtönungspartikeln aus (\ref{168b}) zu motivieren.\footnote{ Wenngleich mir die prinzipielle Idee der Autoren sehr plausibel erscheint, muss man doch sagen, dass sie keine besonders explizite Ausbuchstabierung der Motivation der Klassenbildung und Kategorisierung vornehmen. Die folgenden Ausführungen sind in dem Sinne weitestgehend als meine eigene Rekonstruktion dieser Motivation unter Festhalten an der übergeordneten Idee zur informationsstrukturellen Ableitung der festen Abfolgen der MPn in Kombinationen zu sehen.}

Die MPn der deiktisch-anaphorischen Gruppe (\textit{dan}, \textit{nu}, \textit{toch}) sind in dem Sinne thematisch, als dass sie Rahmenbedingungen setzen. Die MP \textit{dan} ist z.B. homonym \is{Homonymie} mit dem anaphorischen konditionalen Adverb \textit{dan} (vgl. (\ref{169})).

\begin{exe}
	\ex\label{169} 
	\gll Zijn we morgen vrij \textbf{\textit{dan}} gaan we uit.\\
	Wenn wir morgen frei dann gehen wir aus\\
    \glt Wenn wir morgen Zeit haben, gehen wir aus. 
\newline
\hbox{}\hfill\hbox {\citet[50]{Vriendt1991}}
\end{exe}
Als MP tritt \textit{dan} häufig in Fragen auf (vgl. (\ref{170})). Fürs Deutsche ist angenommen worden, dass Fragen mit \textit{denn} Fragen sind, die sich aus dem Vorgängerkontext motivieren (vgl. z.B. \citealt[164, 166]{Thurmair1989}, \citealt[420]{Diewald2006}). 

\begin{exe}
	\ex\label{170} 
	\gll Waarom \textbf{dan}?\\
	 warum   denn \\
    \glt Warum denn?
\hfill\hbox {\citet[50]{Vriendt1991}}
\end{exe}
Betrachtet man diese sprachlichen Kontexte, in denen die MP und ihr Homonym auftreten, lässt sich unter einer genügend abstrakten Perspektive zwischen den beiden Ausdrücken im folgenden Sinne eine Verbindung ziehen: Die MP zeichnet den Sprechakt der Frage, in der sie auftritt, als im Vorgängerkontext motiviert aus. Das konditionale Adverb \textit{dan} bedeutet so viel wie \glq unter diesen Umständen\grq {}, \glq in diesem Fall\grq {}. Unter beiden Verwendungen von \textit{dan} werden die sprachlichen Strukturen somit auf den Kontext bezogen, indem sie in Relation zur \glq Szene\grq {} gesetzt werden. \textit{Dan} lässt sich hier als Element der deiktisch-anaphorischen Klasse \is{deiktisch-anaphorische Gruppe} als  \glq rahmensetzend\grq {}  und somit thematisch einstufen.

Die MPn der existenziell quantifizierenden Gruppe \is{existenziell quantifizierende Gruppe} stehen in Kombinationen am rechten Rand. Sie sind nach der Idee der Autoren deshalb als rhematisch(er)e MPn einzustufen. Wird z.B. \textit{even} nicht als MP interpretiert, bedeutet es  \glq ein kurzes Mal\grq {} (vgl. (\ref{171})).

\begin{exe}
	\ex\label{171} 
	\gll Wil je \textbf{\textit{even}} in de spiegel kijken?\\
	 Willst du mal in den Spiegel gucken\\
    \glt Willst du einen kurzen Blick in den Spiegel werfen?
\newline
\hbox{}\hfill\hbox {\citet[53]{Vriendt1991}}
\end{exe}
Unter dieser Interpretation quantifiziert die Partikel direkt über den durch das Prädikat ausgedrückten Sachverhalt. Das Eintreten des Sachverhalts in (\ref{171}) ist einmalig und von kurzer Dauer. Unter der Deutung von \textit{even} als MP wird die illokutionäre Kraft (z.B. eines Direktivs) abgeschwächt (vgl. (\ref{172})). 

\begin{exe}
	\ex\label{172} 
	\gll Kom \textbf{even} hier.\\
	 Komm kurz her\\
	 \glt
\hfill\hbox {\citet[54]{Vriendt1991}}
\end{exe}
Als MP ist \textit{even} nicht rückverweisend. Möglicherweise kann es genau deshalb diskurseröffnend verwendet werden. Wird über einen Sachverhalt ausgesagt, dass er einmalig auftritt und/oder von kurzer Dauer ist, lässt er sich (in vielen Fällen) womöglich als neu im Diskurs annehmen. Er ist nicht schon zuvor bereits vorgekommen oder vermutlich handelt es sich nicht um einen Sachverhalt, der sich regelmäßig wiederholt. 

Für die Mitglieder der Klasse der modalen MPn (\textit{maar}, \textit{wel}) gilt es, zu motivieren, in welchem Sinne sie zwischen den (nach Einschätzung der Autoren) eindeutig thematischen und rhematischen MPn positioniert werden. Über diese Klasse von MPn schreiben die Autoren (\citeyear[58]{Vriendt1991}): \glqq The purely modal downtoners are to be interpreted against a presuppositional background.\grqq{}  Der in diesem Zitat angesprochene präsupponierende \is{Präsupposition} Charakter von \textit{maar} und \textit{wel} lässt sich auf die folgende Art ableiten: Der Partikel \textit{maar} schreiben \citet{Vriendt1991} die Funktion zu, Resignation anzuzeigen und damit eine negative Haltung zu kodieren. Der Partikel \textit{wel} wird eine versichernde Wirkung zugesprochen. Sie bringt eine positive Haltung zum Ausdruck. (\ref{173}) werde in diesem Sinne als eine neutrale Absicht verstanden, (\ref{174}) wirke selbst-ermahnend und potenziell resigniert, während der Sprecher sich in (\ref{175}) dazu bekenne, die Handlung auszuführen und damit ein versichernder Ton vorliege.

\begin{exe}
	\ex\label{173} 
	\gll Ik zal dat in orde brengen.\\
	 Ich werde das in Ordnung bringen \\
\end{exe}

\begin{exe}
	\ex\label{174} 
	\gll Ik zal dat (dan) \textbf{maar} (eens) in orde brengen.\\
	 Ich werde das MP MP MP in Ordnung bringen \\
	 \glt Ich werde das dann wohl mal in Ordnung bringen. 
\end{exe}

\begin{exe}
	\ex\label{175} 
	\gll Ik zal dat \textbf{wel} in orde brengen.\\
	 Ich werde das MP in Ordnung bringen \\
	 \glt Ich werde das schon in Ordnung bringen. 
	 \newline
	 \hbox{}\hfill\hbox{\citet[55]{Vriendt1991}}
\end{exe}
Die Präsupposition \is{Präsupposition} des ausgedrückten Sachverhalts ist nun derartig nachzuwei\-sen, dass eine Sache, die ein Sprecher positiv oder negativ bewertet, plausiblerweise zu einem gewissen Grad vorausgesetzt und somit Teil des Hintergrundes sein muss. Über die Eigenschaft, den Sachverhalt, über den sie eine Bewertung ausdrücken, zu präsupponieren, lassen sich die MPn der Klasse der modalen Abtönungspartikeln \is{modale Abtönungspartikeln} im folgenden Sinne zwischen thematischen und rhematischen MPn einordnen: Präsuppositionen, die in vielen Fällen anaphorischen Charakter haben, lassen sich problemlos als thematisch auslegen. Sie stellen in einer solchen Argumentation alte Information im Hintergrund \is{Hintergrund} dar, auf die sich die Diskursteilnehmer bereits geeinigt haben. Auf der Basis dieser Information werden weitere, neue Informationen verankert. Präsupponierte Inhalte gestalten in diesem Sinne den Rahmen mit, innerhalb dessen neue Informationen ausge\-wertet werden. Es ist ebenfalls ein bekanntes Phänomen, dass Präsuppositionen nicht notwendigerweise alte und bekannte Inhalte darstellen. Unter der informativen Verwendung von Präsuppositionen – bekannt als Phänomen der \textit{Akko\-mmodation} \is{Akkommodation} – handelt es sich bei den präsupponierten Inhalten um neue, d.h. rhematische Informationen. Ein Inhalt wird als gegeben vorgegeben, obwohl er es nicht ist. Unter Bezug auf diese zwei Verwendungen von Präsuppositionen würde sich folglich die Einordnung der Klasse der modalen Abtönungspartikeln zwischen den eindeutig thematischen (deiktisch-anaphorische MPn) und den eindeutig rhematischen MPn (existenziell quantifizierende MPn) erklären. 

Die Abfolgen von MPn in Kombinationen folgen (wie erläutert) der Ordnung der drei Klassen entlang (a) > (b) > (c). Die MPn gliedern das Mittelfeld unter dieser Perspektive nicht in einen thematischen und rhematischen Bereich, sondern sie haben selbst Teil an einer Ordnung der Mittelfeldelemente entlang des Kriteriums ihres Grades an Thematizität \is{Thematizität} bzw. Rhematizität \is{Rhematizität}: Thematische(re) E\-lemente gehen rhematische(re)n voran. 

Der Ansatz von \citet{Vriendt1991} fällt unter diejenigen Arbeiten, die die Abfolgebeschränkungen in MP-Kombinationen auf interpretatorische Gegebenheiten zurückführen – seien dies nun eher inhärente Bedeutungsaspekte der MPn oder eher funktionale Wirkungsweisen hinsichtlich infomationsstruktureller Gliederungen. Auf die eine oder andere Art beziehen sich alle diese Autoren auf Bedeutungs-/Funktionsaspekte der einzelnen MPn bzw. der MP-Kombinationen. Dies gilt auch für die Kriterien aus \citet{Thurmair1989, Thurmair1991}. 

Im folgenden Abschnitt wird mit \citet{Lindner1991} ein von dieser Argumentationsweise abweichender Ansatz vorgestellt. Wenngleich sie sich auch mit der Interpretation der \textit{ja doch}-Kombination beschäftigt, bezieht sie sich mit ihrem Kriterium zur Ableitung der (in)akzeptablen Abfolge der beiden MPn mit der Phonologie auf eine ganz andere Ebene des Grammatiksystems als alle bisher angeführten Arbeiten.

\subsection{Eine phonologische Beschränkung}
\label{sec:phon}
Der Ansatz von \citet{Lindner1991} ist prinzipiell zu den Skopusansätzen (vgl. Abschnitt~\ref{sec:skopus}) zu rechnen. Sie spielt verschiedene Skopusverhältnisse \is{Skopus} zwischen den MPn durch und sieht diese auch durchaus als Kriterium an, um die Abfolgen zu motivieren. Da sie letztlich jedoch keine der mit verschiedenen Skopoi modellierten Interpretationen für gänzlich passend hält und zudem gerade diejenige Skopusinterpretation als am adäquatesten einstuft, bei der die Oberflächenabfolge gerade nicht die Skopusrelationen widerspiegelt, motiviert sie die zu beobachtende Abfolge schließlich unter Annahme phonologischer Verhältnisse.\footnote{Ich beschränke mich hier auf das phonologische Kriterium. Lindners Skopusannahmen werden in Abschnitt~\ref{sec:skopusexp} in die Diskussion eingebunden.} Die Ausführungen Lindners beziehen sich auf die Kombination der Partikeln \textit{ja} und \textit{doch} wie in einer Äußerung in (\ref{176}).

\begin{exe}
	\ex\label{176} 
	Wir sind \textbf{ja doch} alte Bekannte.
\end{exe}
Lindner nimmt an, dass die Sequenz \textit{ja doch} in schneller Sprache gegenüber \textit{doch ja} präferiert ist, da in letzterer Abfolge zwei Frikative mit einem geringen Unterschied hinsichtlich der \textit{Stärke der Konsonanten} \is{konsonantische Stärke} aufeinander treffen. Um diese Erklärung nachvollziehen zu können, sind einige Erläuterungen zum Konzept der \textit{konsonantischen Stärke} und seiner Rolle im Silbenaufbau vonnöten.

Die \textit{Stärkehierarchie der Sprachlaute der deutschen Standardsprache} (vgl. (\ref{177}), \citealt[284]{Vennemann1982}) verläuft genau entgegengesetzt der übliche(re)n \is{Sonoritätshierarchie} \textit{Sonori\-tätshierarchie}.

\begin{exe}
	\ex\label{177} 
	Hierarchie konsonantischer Stärke \is{konsonantische Stärke}\\
	Vokale < Liquide < Nasale < stimmhafte Frikative < stimmhafte Plosive/ \\ stimmlose Frikative < stimmlose Plosive
\end{exe}
\citet[283]{Vennemann1982} formuliert sechs \textit{Präferenzgesetze} \is{Präferenzgesetz} für den Silbenbau und die Silbenverkettung, die auf dem Konzept der konsonantischen Stärke beruhen. Für den von Lindner betrachteten Fall ist das \textit{Silbenkontaktgesetz} \is{Silbenkontaktgesetz} das relevante Präferenzgesetz. Im Falle des Aufeinandertreffens zweier Silben sieht es vor, dass der letzte Laut der ersten Silbe und der erste Laut der zweiten Silbe eine große Differenz hinsichtlich ihrer konsonantischen Stärke aufweisen. Aus dem \textit{Endrandgesetz} \is{Endrandgesetz} und dem \textit{Anfangsrandgesetz} \is{Anfangsrandgesetz} folgt weiterhin, dass der letzte Laut der ersten Silbe wenig, der erste Laut der zweiten Silbe große konsonantische Stärke aufweisen sollte. 

Unter Bezug auf Vennemanns Präferenzgesetze sowie der Klassifikation von /x/ als stimmlosem velarem Frikativ und /j/ als stimmhaftem palatalem Frikativ (vgl. \citealt[284]{Vennemann1982}) liegt zwischen diesen beiden Lauten, wenn die Silben \textit{doch} und \textit{ja} aufeinander treffen, ein schlechter Silbenkontakt vor, da sich die beiden Laute hinsichtlich ihrer konsonantischen Stärke wenig voneinander unterscheiden. Der Silbenkontakt ist jedoch als nahezu optimal zu bewerten, wenn im Falle des Aufeinandertreffens zweier Silben der letzte Laut der ersten Silbe ein tiefer Vokal (hier /a:/ ist $[$der den geringsten Grad konsonantischer Stärke aufweist, der möglich ist$]$) und der erste Laut der zweiten Silbe ein stimmhafter Plosiv (/d/) ist, dem nur ein stimmloser Plosiv auf der Skala konsonantischer Stärke übergeordnet ist. 

Lindner macht folglich phonologische Gründe dafür verantwortlich, dass die Abfolge \textit{doch ja} an der syntaktischen Oberfläche nicht auftritt, begründet über das Kriterium der Differenz von konsonantischer Stärke von Lauten im End- bzw. Anfangsrand von Silben. Unter einer derartigen Begründung stellt die Oberflächenabfolge keinen Indikator für ein Skopusverhältnis \is{Skopus} dar. 

Eine derartige phonologische Motivation der Abfolge wäre für Lindner auch dann weiterhin denkbar, wenn eine andere Interpretation der MP-Kombinationen angenommen würde. 	

\subsection{Zusammenhang mit Vormodalpartikellexemen}
\label{sec:vormp}
Allen bis hierhin angeführten Ansätzen zur Ableitung der Abfolgebeschränkungen in MP-Kombinationen, die über eine rein deskriptive Erfassung hinausgehen (vgl. Abschnitt~\ref{sec:katalog} bis \ref{sec:phon}), ist gemein, dass ihre Erklärungen jeweils auf einer der linguistischen Beschreibungsebenen verankert sind. Unabhängig davon, ob eine phonologische Beschränkung formuliert wird, das Verhältnis zwischen Syntax und Semantik bemüht wird, semantische Argumentanforderungen oder semantische Eigenschaften der MPn verantwortlich gemacht werden, auf verschie\-dene Beschreibungsebenen von Sätzen Bezug genommen wird oder ob funktional-pragmatische Aspekte der Verwendung von MP-Äußerungen beleuchtet werden, so erfolgen die Erklärungen der beschränkten Sequenzierungen in allen Fällen aus dem Grammatiksystem selbst heraus. Aus diesem Zugang folgt insbesondere, dass die Perspektive auf den Gegenstandsbereich eine rein synchrone \is{Synchronie} ist. Von anderen Autoren sind ebenso diachrone \is{Diachronie} Faktoren benannt worden, die in einer Ableitung der beschränkten Sequenzierungen eine Rolle spielen sollen.

Die (Aspekte der) Ansätze von \citet{Vismans1994} und \citet{Abraham1995}, die dieser Idee folgen, sind Gegenstand des nächsten Abschnitts.

\subsubsection{Syntaktische Stellungsklassen}
\citet{Abraham1995} präsentiert einen Ansatz, der die MP-Abfolge in Kombinatio\-nen unter Bezug auf die entsprechenden Vormodalpartikellexeme \is{Vormodalpartikellexem} abzuleiten versucht.\footnote{Für die folgende Darstellung der Kernideen Abrahams muss angemerkt werden, dass es sich an manchen Stellen um meine eigene Rekonstruktion handelt, weil der Aufsatz die präsentierten Ideen wenig konkret und unzusammenhängend ausbuchstabiert. Wenn ich die Annahmen aufgrund dieses Umstands falsch auslege und einschätze, ist dies unbeabsichtigt und dieser Rekonstruktion geschuldet.} Er folgt dabei der von vielen Autoren vertretenen Annahme, dass sich die als MPn eingestuften Formen aus den (synchron vorhandenen) gleichlautenden Formen in anderen Wortarten im Zuge eines Grammatikalisierungsprozesses \is{Grammatikalisierung} herausgebildet haben. Er macht sich in seiner Ableitung der Abfolgebeschränkungen konkret semantische Zusammenhänge zwischen den MPn und ihren Vormodalpartikellexemen zu Nutze. 

Die Kernidee des Autors ist die Annahme, Abfolgebeschränkungen wie in (\ref{178}) bis (\ref{180}) ließen sich ableiten über eine Zuordnung der MPn zu drei syntaktischen Stellungsklassen, denen Strukturpositionen in der Satzhierarchie entsprechen. 

\begin{exe}
	\ex\label{178} 
	Was soll das \textbf{denn nur}?
\end{exe}
\vspace{-0.65cm}
\begin{exe}
	\ex\label{179} 
	Was ist \textbf{denn auch schon} dabei, wenn sie mit diesem Typen ins Kino geht?
\end{exe}
\vspace{-0.65cm}
\begin{exe}
	\ex\label{180} 
	*Mach \textbf{auch aber} das Fenster zu!
	\hfill\hbox {\citet[286-287]{Thurmair1989}}
\end{exe}
Die Stellungsklassen werden durch die Vormodalpartikellexeme \is{Vormodalpartikellexem} konstituiert und basieren auf syntaktischen Eigenschaften dieser. Die Zuordnung der MPn zu den Klassen erfolgt allein auf der Basis der jeweiligen Vormodalpartikellexeme, denen aufgrund ihrer Klassenzugehörigkeit bereits eine Strukturposition zugeordnet ist. Entlang der Abbildung der zueinander hierarchisch geordneten (Elemente der) drei Stellungsklassen auf die lineare Reihung dieser Klassenmitglieder ergeben sich die Abfolgen der MPn in Sequenzen in Parallelität zur Abfolge der ansonsten diese Positionen füllenden Elemente (d.h. mitunter auch der Vormodalpartikellexeme).

Abraham stuft die Vormodalpartikellexeme als MP-Homonyme \is{Homonymie} ein (vgl. \citeyear[95]{Abraham1995}) und geht davon aus, dass sich die spezifischen illokutiven MP-Funktionen aus den lexikalischen Bedeutungen der MP-Homonyme ergeben. Diese Auffassung lässt sich auf die MPn \textit{denn} und \textit{nur}/\textit{bloß} auf die folgende Weise anwenden: Als beiordnende Konjunktion ist \textit{denn} ein kausaler Verknüpfer. Als MP tritt \textit{denn} in Fragen auf (vgl. (\ref{181a})), in denen die Partikel nach Abraham ausdrückt, dass es einen Grund gibt, die Frage zu stellen (vgl. \citeyear[101]{Abraham1995}). Gemeinsamer Nenner der MP \textit{denn} und ihrem Homonym \textit{denn} ist folglich der Bedeutungsaspekt der Kausalität.

\begin{exe}
	\ex\label{181a} 
	Wie spät ist es \textbf{denn}?
	\hfill\hbox {\citet[99]{Abraham1995}}
\end{exe}
Als Vormodalpartikellexeme treten \textit{nur} und \textit{bloß} als Fokuspartikeln \is{Fokuspartikel} auf (vgl. (\ref{181})).

\begin{exe}
	\ex\label{181} 
	Du arbeitest \textit{\textbf{bloß}}/\textit{\textbf{nur}} als Nachtwächter.
	\hfill\hbox {\citet[101]{Abraham1995}}
\end{exe}	
Der Beitrag dieser Fokuspartikeln lässt sich derart fassen, dass sie anzeigen, dass es eine größere  Dimension mit geordneten Werten gibt, innerhalb derer die je\-weils thematisierte Sache bei einem unteren Grenzwert angesetzt wird. In (\ref{181}) ist etwa von einer Skala sozialer Arbeitsplatzwerte auszugehen (vgl. (\ref{182})), auf der es höhere Werte als die Auswahl des Nachtwächters gibt.

\begin{exe}
	\ex\label{182} 
	Pilot > Arzt > Lehrer > Nachtwächter > Schuhputzer
\end{exe}	
Treten \textit{nur} und \textit{bloß} als MPn auf wie in der Frage in (\ref{183}), geht Abraham davon aus, dass sie anzeigen, dass es einen Grund für das Stellen der Frage gibt. Diese Frage nach dem Grund weise aber – im Gegensatz zur \textit{denn}-Fragen – eine gestei\-gerte Qualität auf, deren Richtung durch die homonyme Fokuspartikel vorgezeichnet sei.

\begin{exe}
	\ex\label{183} 
	Wie spät IST es \textbf{nur}/\textbf{bloß}?
	\hfill\hbox {\citet[101]{Abraham1995}}
\end{exe}												        		       
Es handelt sich bei (\ref{183}) nicht nur um eine Frage nach der Zeit, sondern sie legt nahe, dass ein besonderer Anlass vorliegt, nach der Zeit zu fragen (wie z.B. dass der Fragende denkt, er sei zu spät oder habe vergessen, die Frage rechtzeitig zu stellen). Die Skala, die die größere Dimension (= die Angemessenheit des Zeitpunkts der Frage) aufspannt, könnte als Pole \textit{rechtzeitig} und \textit{zu spät} aufweisen und der Zeitpunkt des Stellens der Frage wäre dann in einem unteren Bereich (= zu spät) zu verankern. Der gemeinsame Nenner, über den der Zusammenhang zwischen Fokuspartikel \is{Fokuspartikel} und MP hergestellt wird, ist dann die (semantische) Eigenschaft, dass eine größere (skalar) organisierte Dimension eröffnet wird, in/auf der der Inhalt, auf den die Partikel sich bezieht, im unteren Bereich (d.h. unter Verweis auf höher geordnete Alternativen innerhalb der Dimension) verankert wird.

Wie eingangs angeführt, teilt Abraham die MPn – mit dem Ziel über die Abfolgebeschränkungen zu abstrahieren – in drei syntaktische Stellungsklassen ein: \textit{C1}, \textit{C2}, \textit{C3}. Die drei Klassen wiederum entsprechen drei unterschiedlichen Positionen im Strukturbaum. Diese Auffassung ist im Zusammenhang mit einer anderen Annahme von Abraham zu sehen: Er versteht eine MP als \textit{Satzoperator} \is{Satzoperator}, der in einer \textit{Topikphrase} \is{Topikphrase} (TopP) die Sprechaktfunktion des Satzes festlegt (vgl. \citealt[96-97]{Abraham1995}). Dies geschieht nicht an der Satzoberfläche, sondern auf der Ebene der LF \is{Logische Form}, denn MPn verbleiben sichtbar im Mittelfeld und können an der Satzoberfläche in der Regel nicht an der Satzspitze stehen. Konkret steuert eine MP Abraham zufolge eine $\textrm{Top}^{0}$-Position auf LF an, aus der heraus das Element unter c-Kommando \is{c-Kommando} Skopus über den Satz hat. Die Abfolgen der Einzelpartikeln in den Kombinationen gehorchen dann den Hierarchisierungen bzw. linearen Abfolgen dieser drei Klassen. Es gibt letztlich drei TopP-Domänen, die die CP dominieren.

\begin{exe}
	\ex\label{184}   
\begin{jtree}
\! = {$\textrm{TopP}_{1}$}
:({$\textrm{Top}^{0}_{1}$} {\textbf{C1}}) {$\textrm{TopP}_{2}$}
:({$\textrm{Top}^{0}_{2}$} {\textbf{C2}}) {$\textrm{TopP}_{3}$}
:({${\textrm{Top}^{0}_{3}}$} {\textbf{C3}}){CP}
:{$\textrm{C}^{0}$} {IP} {...}.
\end{jtree}
\end{exe}
(\ref{185}) fasst Abrahams konkreten Angaben zur Füllung der drei Topik-Kopfpositionen zusammen (vgl. \citeyear[103]{Abraham1995}).
	
\begin{exe}
	\ex\label{185}   
		\textbf{C1}: \textit{denn}\\
		\textbf{C2}: \textit{nur}, \textit{bloß}, \textit{JA}, \textit{etwa}, \textit{schon}, \textit{nicht}\\
		\textbf{C3}: \textit{mal}, \textit{ruhig}, \textit{schon}, \textit{auch}, \textit{JA}
\end{exe}	
Wie bereits angesprochen, ergibt sich die Klassenzugehörigkeit zu C1, C2 und C3 aus den syntaktischen Eigenschaften der \is{Homonymie} MP-Homonyme. $\textrm{Top}^{0}_{3}$ (= C3) sieht Abraham z.B. durch die subordinierende Konjunktion \textit{dass} gefüllt. Die Konjunktion \textit{dass} hat sich diachron \is{Diachronie} aus dem Demonstrativum bzw. Korrelat \textit{das} herausgebildet. Die MPn in dieser Position sind in ihrer Vormodalpartikelfunktion \is{Vormodalpartikellexem} Fokuspartikeln \is{Fokuspartikel}, die einen Wertebereich ausmessen (s.o.). Über genau diesen Aspekt zieht der Autor die Verbindung zur Konjunktion \textit{dass} bzw. genauer zu ihren Vorkonjunktionslexemen: Der lokale Bezug, wie er zwischen einer Fokuspartikel und ihrem Bezugswort gegeben ist, lässt sich auch hier annehmen, da sich das Kor\-relat  (unter Adjazenz) auf den folgenden Nebensatz, das Artikelwort auf eine folgende DP bezieht (vgl. \citealt[105-106]{Abraham1995}). 

In $\textrm{Top}^{0}_{2}$ (= C2) steht in w-Fragen \is{w-Frage} das finite Verb, in $\textrm{SpecTop}_{2}\textrm{P}$ eine w-Konstitu\-ente. w-Wörter stehen grundsätzlich für Satzglieder, d.h. Subjekte, Objekte, Adverbiale. Die MPn, die er C2 zuordnet, haben Vormodalpartikellexeme, die Adverbien \is{Adverb} sind. Der intendierte Zusammenhang könnte hier sein, dass Adverbien nun in Funktion von Adverbialen verwendet werden, so dass der gemeinsame Nenner über die parallele Satzgliedfunktion zustandekommt. 

Für die Besetzung von C1 gilt, dass in dieser linken Satzposition koordinierende Adverbien wie \textit{denn} oder \textit{aber} auftreten (vgl. (\ref{186})).

\begin{exe}
	\ex\label{186} 
		\begin{xlist}	
			\ex\label{186a} Und/denn/aber (C1) wie (C2) hat er das gesehen?
			\ex\label{186b} Und/denn/aber (C1) wie (C2) daß (C3) er sie gesehen hat!
		\end{xlist}
	\hfill\hbox{\citet[106]{Abraham1995}}	
\end{exe}
Um die Abfolgebeschränkungen in MP-Kombinationen abzuleiten (vgl. (\ref{187a}) bis (\ref{189})) ist es unter der Perspektive, die Abraham hier einschlägt, nicht nötig, semantische Kriterien der MPn einzubeziehen. 

\begin{exe}
	\ex\label{187a} 
	Was soll das \textbf{denn} (C1) \textbf{nur} (C2)?
\end{exe}
\vspace{-0.65cm}
\begin{exe}
	\ex\label{187} 
	Was ist \textbf{denn} (C1) \textbf{auch} (C3) \textbf{schon} (C3) dabei, wenn sie mit diesem Typen ins Kino geht?
\end{exe}
\vspace{-0.65cm}
\begin{exe}
	\ex\label{189} 
	*Mach \textbf{auch} (C3) \textbf{aber} (C1) das Fenster zu!
	\hfill\hbox {\citet[286-287]{Thurmair1989}}
\end{exe}												    		   

\noindent
\citet{Abraham1995} bezieht diachrone \is{Diachronie} Erkenntnisse in seine Ableitung der Abfolgebeschränkungen von MPn in Kombinationen ein, indem er sich den Zusammenhang zwischen den MPn und ihren jeweiligen Vormodalpartikellexemen \is{Vormodalpartikellexem} zu Nutze macht.\footnote{Die Zuordnung der MPn zu den drei Stellungsklassen basiert in \citet{Abraham1995} zum einen auf syntaktischen Eigenschaften der Vormodalpartikellexeme. Er motiviert die Klassifikation dazu ebenfalls synchron unter Bezug auf den funktionalen Beitrag der MPn. Er stützt sich hier auf die Kriterien in \citet{Thurmair1989, Thurmair1991} Kriterien und bildet ihre Vorhersagen zur Positionierung von MPn in Kombinationen auf die drei Stellungsklassen ab. Aus Thurmairs diskursiven Charakterisierungen der MPn mit Graden \textit{illokutiven Gewichts} leitet er zudem ein weiteres Kriterium ab, das seiner Argumentation nach bei der Sequenzierung von MPn in Kombinationen eine Rolle spielt. Die Bestimmung eines größeren oder geringeren illokutiven Gewichts einer MP erfolgt ebenfalls unter Einbezug von Eigenschaften der jeweiligen Vormodalpartikellexeme \is{Vormodalpartikellexem} (vgl. \citealt[104]{Abraham1995}).} Auf eine andere Art fließen auch in den Ansatz von \citet{Vismans1994}, der in Abschnitt~\ref{sec:ebenen} als eine Arbeit, die über die Zuordnung der MPn zu verschiedenen Ebenen des Satzes argumentiert, dargestellt wird, Aspekte der Ent\-wicklungsgeschichte der MPn ein. Er betrachtet jedoch nicht allein die Verbin\-dung zwischen den MPn und ihren Vormodalpartikellexemen, sondern sieht die unterschiedliche Entstehungszeit der Einzelpartikeln in einer Kombination als beteiligt bei der Frage nach dem Zustandekommen der festen Sequenzbildungen.

\subsubsection{Entstehungszeit}
Wie in Abschnitt~\ref{sec:vismans} ausgeführt, leitet Vismans die Abfolgen in MP-Kombinatio\-nen (vgl. (\ref{190})) aus der Zuordnung der (als Operatoren \is{Operator} aufgefassten) MPn zu verschiedenen Satzschichten und der festen Ordnung eben dieser Ebenen des Satzes ab.

\begin{exe}
	\ex\label{190} 
	\begin{xlist}	
			\ex\label{190a} 
				Doe de deur \textbf{nou} ($\textrm{MP}_{\pi4}$) \textbf{toch} ($\textrm{MP}_{\pi3}$) \textbf{eens} ($\textrm{MP}_{\pi2}$) DICHT.\\
				Mach die Tür zu!
			\ex\label{190b} 
				Kun je \textbf{misschien} ($\textrm{MP}_{\pi4}$) \textbf{eens} ($\textrm{MP}_{\pi2}$) lángs komen?\\
				Kannst du vorbeikommen?				
	\hfill\hbox{\citet[163/201]{Vismans1994}}
	\end{xlist}
\end{exe}
Da die Hierarchisierung der (hier relevanten) Satzschichten der Ordnung \is{Illokution} Illokution (Ebene 4) > Proposition \is{Proposition} (Ebene 3) > Prädikation \is{Prädikation} (Ebene 2) entspricht, gehen in (\ref{190}) auch $\pi_{4}$-Operatoren $\pi_{3}$- bzw. $\pi_{2}$-Operatoren linear voran. Wie für alle Ansätze, die die Abfolgen in MP-Kombinationen abstrakter über die Ordnung von Klassen, denen die einzelnen MPn zugeordnet sind, zu fassen versuchen, gilt auch für Vismans' Ansatz, dass sich die Frage stellt, wie die ebenfalls zu beobachtenden Abfolgebeschränkungen \underline{innerhalb} einer der Klassen aufzufangen sind. Die Partikeln \textit{eens} und \textit{even} werden von Vismans beispielsweise beide auf der Ebene der Prädikation \is{Prädikation} verankert. Dennoch handelt es sich nur bei (\ref{191a}) um eine akzeptable Reihung.

\begin{exe}
	\ex\label{191} 
	\begin{xlist}	
			\ex\label{191a} 
				Kom \textbf{eens} ($\textrm{MP}_{\pi2}$) \textbf{even} ($\textrm{MP}_{\pi2}$).\\
				Komm!
			\ex\label{191b} 
				*Kom \textbf{even} \textbf{eens}.			
	\hfill\hbox{\citet[163]{Vismans1994}}
	\end{xlist}
\end{exe}
Das Phänomen tritt auch auf den anderen Satzschichten auf. Die Abfolgemöglich\-keiten sind in (\ref{192}) (aus \citealt[5]{Vismans1994}) wiederholt unter Angabe des jeweiligen Typs von Operator. 

\begin{exe}
	\ex\label{192}
	\scriptsize
	\begin{tabular}[t]{|l|l|}
	\hline
  	Typ & Reihenfolge in Kombination\\
  	\hline
  	Dek. & \textit{ook} ($\pi_{3}$), \textit{maar} ($\pi_{3}$), \textit{eens} ($\pi_{2}$), \textit{even} ($\pi_{2}$)\\
  	\hline
  	Inter. & \textit{nou} ($\pi_{4}$), \textit{misschien}/\textit{soms}* ($\pi_{4}$), \textit{ook} ($\pi_{3}$), \textit{eens} ($\pi_{2}$), 			\textit{even} ($\pi_{2}$)\\
  	\hline 
  	Imp. & \textit{dan}/\textit{nou}* ($\pi_{4}$), \textit{toch} ($\pi_{3}$), \textit{maar} ($\pi_{3}$), \textit{eens} ($\pi_{2}$), \textit{even} ($\pi_{2}$)\\
  	\hline
\end{tabular}\\
* = austauschbar
\end{exe}
Der Datenbereich, für dessen Erfassung Vismans die von ihm angestellte dia\-chrone \is{Diachronie} Betrachtung nutzbar zu machen beabsichtigt, sind MP-Abfolgen, in denen beide MPn der gleichen Satzschicht zugeordnet sind. Eine grundsätzliche Unterscheidung, die Vismans in Anlehnung an \citet{Hengeveld1989} macht, ist die zwischen MPn, die der \underline{Verstärkung} des vorliegenden Sprechaktes \is{Sprechakt} dienen, und solchen, die entgegengesetzt eine \underline{Abschwächung} des Sprechaktes bewirken. Zur Funktion der Verstärkung gehöre der Ausdruck von Sicherheit, Entschlossenheit, Positivität, Bedeutsamkeit oder Spezifizität, mit Abschwächung gehe umgekehrt der Ausdruck von Zweifel, Unentschlossenheit, Generalität, Unbedeutsamkeit und Negativität einher. In (\ref{193}) und (\ref{194}) findet sich jeweils Vismans' Zuordnung der von ihm untersuchten MPn in Direktiven \is{Direktiv} zu diesen beiden Effekten auf Sprechakte.
		
\begin{exe}
	\ex\label{193} 
	Verstärker\\
	\textit{dan}, \textit{eens}, \textit{nou}, \textit{ook}, \textit{toch}
\end{exe}		
	
\begin{exe}
	\ex\label{194} 
	Abschwächer\\
	\textit{even}, \textit{maar}, \textit{misschien}, \textit{soms}
	\hfill\hbox {\citet[73]{Vismans1994}}
\end{exe}		
Die Klassifikation einer MP als Verstärker oder Abschwächer geht zurück auf die Bedeutung \is{Vormodalpartikellexem} der Vormodalpartikellexeme. 

Die modale Verwendung von \textit{dan} z.B. (vgl. (\ref{195})), die von \citet[605]{Dale1992} als Ausdruck von Ungeduld und Missfallen beschrieben wird, geht u.a. zurück auf die temporale Verwendung (vgl. (\ref{196})).

\begin{exe}
	\ex\label{195} 
	\gll Doe je werk \textbf{dan}!\\
	Tu deine Arbeit MP\\
\end{exe}

\begin{exe}
	\ex\label{196} 
	\gll Ik ga morgen naar Amsterdam. Ben jij daar \textbf{\textit{dan}} óók?\\
	Ich gehe morgen nach Amsterdam Bist du dort dann auch\\
	\glt Ich fahre morgen nach Amsterdam. Wirst du dann auch dort sein?
	\newline
	\hbox{}\hfill\hbox {\citet[62/61]{Vismans1994}}
\end{exe}	
Als temporales Adverb bezeichnet \textit{dan} einen spezifischen Zeitpunkt. Im Bedeutungsaspekt dieser Spezifizität sieht \citet[58-60]{Vismans1994} die Funktion der Verstärkung verankert. Wenngleich das Adverb in der Entwicklung zur MP seine temporale Bedeutung verliere, bleibe die zugrunde liegende verstärkende Bedeutung erhalten, weshalb die MP in einem Kontext auftreten könne, in dem die Funktion der Verstärkung gefordert ist.\footnote{Zur Motivation der Zuordnung der anderen MPn zur Klasse der Abschwächer oder Verstärker in (\ref{193}) und (\ref{194}) vgl. \citet[61-73]{Vismans1994}.}

Bei der Unterscheidung zwischen Abschwächern und Verstärkern handelt es sich um einen Faktor, der bei Vismans' Erfassung der Reihung von MPn, die derselben Satzschicht angehören, eine Rolle spielt. Ein anderer Aspekt, der dort beteiligt ist, ist das Alter einzelner MPn. Vismans nimmt in seiner Arbeit auch eine diachrone Untersuchung \is{Diachronie} vor, deren Ergebnis ist, dass nicht alle MPn, die heutzutage in Direktiven \is{Direktiv} auftreten, diese Funktion zur selben Zeit aufgenommen haben. Insbesondere stellt er fest, dass sich die MPn, die zu den Verstärkern zählen, vor den MPn, die unter die Abschwächer fallen, entwickelt haben (vgl. \citeyear[103]{Vismans1994} zu einer Übersicht). Gründe für diese Entwicklung sieht Vismans in sozialen und gesellschaftlichen Veränderungen im 16./17. Jahrhundert, die auch mit dem erstmaligen Aufkommen anderer linguistischer Ausdrücke (betroffen ist hier das Pronominalsystem) einhergehen (vgl. \citealt[102-106]{Vismans1994}).

Für die Ordnung von MPn, die derselben Satzschicht angehören, spielen die Unterscheidung zwischen Abschwächern und Verstärkern sowie die Erkenntnisse zum Alter der MPn nun insofern eine Rolle, als dass Vismans annimmt, dass in einer solchen Kombination stets die Verstärker den Abschwächern vo\-rangehen. Diese Voraussage bestätigt sich im Beispiel in (\ref{197}).
	
\begin{exe}
	\ex\label{197} 
	Geef de boeken \textbf{dan} ($MP_{\pi4}$) \textbf{nu} ($MP_{\pi4}$) \textbf{toch} ($MP_{\pi3}$) \textbf{maar} ($MP_{\pi3}$) \textbf{'es} ($MP_{\pi2}$) \textbf{even} ($MP_{\pi2}$) hier.\\
	Gib die Bücher her!
	\hbox{}\hfill\hbox{\citet[98]{Hoogvliet1903}}
\end{exe}	
Die Partikeln \textit{dan} und \textit{nou}, die beide auf der Illokutionsschicht \is{Illokutionsschicht} verankert sind und nach (\ref{193}) und (\ref{194}) zu den Verstärkern zählen, lassen sich laut (\ref{192}) austauschen. Für die beiden MPn der Ebene der Proposition \is{Propositionsschicht} (\textit{toch}, \textit{maar}) und der Prädikation \is{Prädikationsschicht} ('\textit{es} (\textit{eens}), \textit{even}) gilt Vismans' Generalisierung, da der Verstärker (\textit{toch} bzw. '\textit{es} (\textit{eens})) dem Abschwächer (\textit{maar} bzw. \textit{even}) vorangeht. 

Die Ordnung von MPn derselben Satzschicht entlang der Abfolge \textit{Verstärker} > \textit{Abschwächer} bringt Vismans mit den Ergebnissen seiner historischen Be\-trachtung in Verbindung. Da sich die Abschwächer erst entwickelt haben, nachdem sich das Verstärkersystem ausgebildet hatte, spiegelt die Ordnung der MPn innerhalb einer Beschreibungsebene des Satzes auch die historische Ordnung wider. Die älteren (verstärkenden) MPn stehen vor den neueren (abschwächenden) MPn.\\

\noindent
Dieser Überblick zeigt, dass Arbeiten, die Ableitungen der beschränkten MP-Abfolgen vorgeschlagen haben, auf verschiedene Beschreibungsebenen Bezug nehmen (Phonologie, Syntax, Semantik, Pragmatik) und auch diachrone Überlegungen eingeflossen sind. Ein Schwerpunkt liegt hier sicherlich im Bereich interpretatorischer Kriterien.

Mein eigener Zugang reiht sich in derartige Modellierungen ein, da ich vertre\-ten werde, dass die MP-Abfolgen gewünschte Diskursverläufe spiegeln. Ich gehe hier\-bei von einem inhärenten, diskursstrukturellen Beitrag der MPn aus und argumentiere im Zuge der m.E. vorliegenden Form-Funktions-Korrespondenz unter Bezug auf Gebrauchsbedingungen der Äußerungen. Die Perspektive ist somit eine pragmatische.

Der nächste Abschnitt dieses Teils der Arbeit, der einige grundsätzliche Aspekte anspricht und in diesem Sinne Hintergrundinformation liefert, beschäftigt sich mit Fragen zur Interpretation von MP-Kombinationen.

\section{(Non-)Transparenz der Interpretation}
\label{sec:transparenz}
\setcounter{equation}{0}
Unabhängig von der Tatsache, dass in der Literatur kein Konsens über die Modellierung der Bedeutung der Einzelpartikeln besteht, wird in Bezug auf MP-Sequen\-zen kontrovers diskutiert, von welchen Skopusverhältnissen \is{Skopus} auszugehen ist. Es steht zur Diskussion, ob die MPn jeweils den gleichen Skopusbereich aufweisen oder ob asymmetrische Verhältnisse bestehen, indem sie übereinander Skopus nehmen (s.u.). Da dieser Aspekt zum einen mit den beschränkten Abfolgemög\-lichkeiten in Zusammenhang gebracht wird und zum anderen für die Bedeutungszuschreibung der Kombinationen entscheidend ist, behandle ich diese Frage in einem se\-paraten Kapitel und somit unabhängig der von mir im Detail untersuchten MPn. Sie wird in den Kapiteln~\ref{chapter:jud}, \ref{chapter:hue} und \ref{chapter:dua} bei den Einzeluntersuchungen zu \textit{ja doch}, \textit{halt eben} und \textit{doch auch} jeweils aufgegriffen. 

Wenngleich Uneinigkeit besteht, wie genau der Skopus in den Strukturen verläuft, so teilen die Arbeiten i.d.R. eine kompositionelle Perspektive, d.h. sie gehen (wenn auch auf verschiedene Art) davon aus, dass sich die Bedeutung der Kombination transparent rekonstruieren lässt aus dem Beitrag der beteiligten Einzelpartikeln und ihrem Bezug auf die Proposition. Mit derartigen Ansätzen beschäftigen sich die Abschnitte~\ref{sec:skopusexp} und \ref{sec:skopusimp}. Es ist allerdings auch die gegensätzliche Sicht vertreten worden, dass bestimmte MP-Kombinationen holistische Einheiten sind, deren Gesamtbedeutung nicht auf ihre Bestandteile und die Relation zwi\-schen diesen zurückzuführen ist (vgl. z.B. \citealt[487]{Eroms2000}, \citealt{Lemnitzer2001}).

\subsection{Explizite Skopusannahmen}
\label{sec:skopusexp}
Explizite Annahmen zum Skopus in MP-Kombinationen werden z.B. in \citet{Thurmair1989, Thurmair1991} gemacht. Wie in Abschnitt~\ref{sec:katalog} bereits gesehen, bildet die Autorin die Bedeutung der MPn anhand von Merkmalen \is{semantisches Merkmal} ab: Sie nimmt eine begrenzte Menge relevanter Merkmale an (z.B. $<$BEKANNT$>$, $<$EVIDENT$>$, $<$ERWARTET$>$, \\ $<$KORREKTUR$>$) (vgl. \citealt[100-101, 200]{Thurmair1989}), die z.T. auf den Sprecher oder Hörer ausgerichtet werden. Die MP \textit{ja} erhält dann beispiels\-weise das Merkmal $<$BEKANNT$>_{\textrm{H}}$ (\glq bekannt für den Hörer\grq {}). In ihrem Beispiel (\ref{198a}) wissen die Kollegen, dass es \glqq hoher Besuch\grqq{} war und nehmen an, dass die Lehrerin es auch weiß.

\begin{exe}
	\ex\label{198a} 
	Eine bekannte Schauspielerin kommt zur Lehrerin ihres Sohnes in die Sprechstunde. Die Kollegen, die das beobachtet haben, sagen später zu der 				Lehrerin: \glqq Sie haben heute \textbf{ja} hohen Besuch gehabt.\grqq{}
	\hfill\hbox {\citet[104]{Thurmair1989}}
\end{exe}
Für \textit{auch} nimmt sie z.B. an, dass es eine Beziehung zur Vorgängeräußerung herstellt, indem der Sprecher ausdrückt, dass der Inhalt der Vorgängeräußerung erwartbar war. Für diese Erwartetheit gibt der Sprecher einen Grund/eine Er\-klärung.

\begin{exe}
	\ex\label{198} 
	Elke: Stell dir vor, der Peter hat eine Eins im Staatsexamen!\\
	Gisi: Der hat \textbf{auch} ziemlich viel dafür geschuftet.
	\hfill\hbox {\citet[155]{Thurmair1989}}
\end{exe}
Die Partikel \textit{auch} wird deshalb charakterisiert durch die Merkmale $<$KONNEX$>$ und $<$\textrm{ERWARTET}$>_{\textrm{V}}$.

Werden MPn kombiniert, führt dies i.E. dazu, dass ihre semantischen Merkmale \is{semantisches Merkmal} addiert werden. Treten \textit{ja} und \textit{auch} zusammen auf, addieren sich die Merkmale nach Thurmair deshalb zu: ja + auch = $<$KONNEX$>$, $<$\textrm{ERWARTET}$>_{\textrm{V}}$, \\$<$\textrm{BEKANNT}$>_{\textrm{H}}$. Die Autorin geht folglich von der Möglichkeit einer kompositionellen Ableitung der Bedeutung der MP-Kombinationen aus und nimmt konkret an, dass sich die Bedeutung der Kombination additiv aus der Bedeutung der Bestandteile ergibt. Sie thematisiert die Skopusfrage nicht und diskutiert auch keine Alternativen. In jedem Fall weisen die MPn in einer Kombination dieser Sicht nach aber den glei\-chen Bezugsbereich auf: Die Merkmale der einen MP beziehen sich auf genau den gleichen dargestellten Sachverhalt wie die Merkmale der anderen MP.\\

\noindent
Eine ganz andere Antwort auf die Skopusfrage \is{Skopus} geben die beiden Arbeiten, die ich in Abschnitt~\ref{sec:skopus} vorgestellt habe. \citet{Ormelius-Sandblom1997} und \citet{Rinas2007} gehen von einer Korrelation der syntaktischen Oberflächenabfolge und dem asymmetrischen Skopusverhältnis zwischen den beiden Partikeln aus. Für ein Beispiel wie in (\ref{199}) nimmt Thurmair an, dass sich der Beitrag von \textit{ja} derart in die Struktur einfügt, dass der Sachverhalt, der als Begründung fungiert, ebenfalls bekannt ist bzw. als solcher vom Sprecher ausgegeben wird. D.h. es ergibt sich eine Paraphrase wie in (\ref{200}).

\begin{exe}
	\ex\label{199} 
	A: Das Menü war ausgezeichnet!\\
	B: Es war \textbf{ja auch} das teuerste Essen auf der Speisekarte.
	\hfill\hbox {\citet[424]{Rinas2007}}
\end{exe}	

\begin{exe}
	\ex\label{200} 
	\glq Dass das Essen das teuerste auf der Karte war, ist bekannt/evident und ist die Begründung dafür, dass erwartbar war, dass das Essen ausgezeichnet 	war.\grq {}
\end{exe}
Rinas stellt in Frage, dass sich die Bekanntheit oder Evidenz tatsächlich auf den Sachverhalt bezieht, dass das Essen das teuerste auf der Karte war. Er geht anders von der Interpretation in (\ref{201}) aus.

\begin{exe}
	\ex\label{201} 
	\glq Es ist bekannt/evident, dass die Begründung für das gute Essen die Tatsache ist, dass das Essen das teuerste auf der Karte ist.\grq {}
\end{exe}				
Zwischen \textit{ja} und \textit{auch} besteht s.E. ein asymmetrisches Skopusverhältnis \is{Skopus}: auch(p) fällt in den Skopusbereich von \textit{ja}:

\begin{exe}
	\ex\label{202} 
	JA(AUCH(p))
\end{exe}
Das Pendat zu (\ref{202}) entlang von Thurmairs Analyse ist (\ref{203}). \textit{Ja} und \textit{auch} steuern ihre Merkmale jeweils in Bezug auf den ausgedrückten Sachverhalt bei.

\begin{exe}
\ex\label{203}
\begin{tabular}[t]{l@{}l@{}l@{}l@{}l@{}}
  	& & p & & \\
  	& $\nearrow$ & & $\nwarrow$ &\\
  	JA & & & & AUCH\\
\end{tabular}
\end{exe}
Konkret ergibt sich aus der Modellierung für \textit{auch} (vgl. (\ref{204})) und \textit{ja} (vgl. (\ref{205})) in \citet{Rinas2007} die Gesamtbedeutung in (\ref{206}) für die \textit{ja auch}-Sequenz.

\begin{exe}
	\ex\label{204} 
			\textit{auch}: AUCH(p) $»$ NICHT-ÜBERRASCHEND (q) WEIL (p)\\
			\glq Eine Proposition ist erwartbar, weil p gilt.\grq {}
\end{exe}

\begin{exe}
	\ex\label{205} 
			\textit{ja}: JA(p) $»$ NICHT-GLAUBT(H, NICHT-p)\\
			\glq Der Hörer glaubt nicht non-p.\grq {}
\hfill\hbox{\citet[425/420]{Rinas2007}}			
\end{exe}

\begin{exe}
\ex\label{206}
	ja > auch\\
	JA(AUCH (p) $»$ NICHT-ÜBERRASCHEND(q) $»$ WEIL(p)) $»$\\
	NICHT-GLAUBT(H, NICHT(AUCH(p) >> NICHT-ÜBERRASCHEND(q) \\ WEIL(p)))\\
	\glq Es ist unkontrovers (es ist nicht der Fall, dass der Hörer daran zweifelt), dass q nicht überraschend ist, weil p.\grq {}	
	\hfill\hbox {\citet[425]{Rinas2007}}
\end{exe}
Genauso wie Thurmair grundsätzlich davon ausgeht, dass sich die Bedeutung der Kombinationen additiv aus der Bedeutung der Einzelpartikeln ergibt, vertritt auch Rinas nahezu lückenlos, dass zwischen den Partikeln einer Zweiersequenz ein asymmetrisches Skopusverhältnis besteht. Für die von ihr untersuchten Kombinationen \textit{ja doch} und \textit{doch schon} argumentiert \citet{Ormelius-Sandblom1997} paral\-lel (vgl. Abschnitt~\ref{sec:os}). 

Der Ansatz von \citet{Lindner1991} (vgl. Abschnitt~\ref{sec:phon}) bringt nun noch eine weitere Interpretationsmöglichkeit ins Spiel. Sie beschäftigt sich mit der Kombination \textit{ja doch}, für deren Bestandteile sie die Bedeutungszuschreibungen in (\ref{207}) und (\ref{208}) zugrunde legt ($\alpha$ entspricht in Assertionen \is{Assertion} der ausgedrückten Proposition.).

\begin{exe}
	\ex\label{207} 
		($\textrm{P}_{\textrm{ja}}$) The speaker assumes that p is not controversial for the addressee (in the eyes
			of the speaker).	
		\hfill\hbox {gekürzte Variante aus \citet[178]{Lindner1991}}		
\end{exe}

\begin{exe}
	\ex\label{208} 
		($\textrm{P}_{\textrm{doch common core}}$) (It is necessary that) If the speaker uses MP \textit{doch} in an illocution type IT referring to $\alpha$ then s/he assumes at the time of speaking that it is not the case that $\alpha$ is being taken into consideration.
		\newline
		\hbox{}\hfill\hbox {\citet[190]{Lindner1991}}		
\end{exe}		
Für eine kompositionelle Ableitung schlägt sie die zwei Hypothesen in (\ref{209}) und (\ref{210}) vor.

\begin{exe}
	\ex\label{209} 
	H3 If the MPs \textit{ja} and \textit{doch} are both associated with one illocution type, then they make their contributions one after the other.
\end{exe}
	
\begin{exe}
	\ex\label{210} 
	H4 If the MPs \textit{ja} and \textit{doch} are both associated with one illocution type, then they make their contributions simultaneously.
	\hfill\hbox {\citet[194]{Lindner1991}}		
\end{exe}	
H4 lässt sich der Auffassung Thurmairs zuordnen (wenngleich Lindner selbst etwas ratlos in Bezug auf diese Interpretation ist) (vgl. \citeyear[196]{Lindner1991}): Die MPn leisten gleichwertig ihren Beitrag zur Proposition. Die Reihenfolge spielt keine Rolle, weil sich die Interpretation additiv ergibt. H3 entspricht bei Lindner der Skopusannahme: Die beiden MPn applizieren sequenziell und unterscheiden sich (deshalb) hinsichtlich ihres Skopus. 

Für die \textit{ja doch}-Äußerung im Kontext in (\ref{212}) nimmt sie an, dass die Interpretation korrekt erfasst wird, wenn \textit{doch} Skopus \is{Skopus} über \textit{ja} nimmt, wie in (\ref{213}) paraphrasiert.

\begin{exe}
	\ex\label{212} 
	Teil eines Gesprächs zwischen Graf Hans Karl Bühl und der Magd der Dame Agathe (Komödie \textit{Der Schwierige} von Hugo von Hofmannsthal, Akt 1, 			Szene 6)\\
	H.K.: Guten Abend, Agathe.\\
	A.:	Daß ich Sie sehe, Eure Gnaden Erlaucht! Ich zittre ja.\\
	H.K.: Wollen Sie sich nicht setzen?\\
	A.: (stehend) Oh, Euer Gnaden, seien nur nicht ungehalten darüber, daß ich gekommen bin statt dem Brandstätter.\\
	H.K.: Aber liebe Agathe, \textbf{wir sind \underline{ja doch} alte Bekannte}. Was bringt Sie denn zu mir?\\
	A.:	Mein Gott, das wissen doch Erlaucht. Ich komm wegen der Briefe.	
\end{exe}
		
\begin{exe}
	\ex\label{213} 
	The speaker assumes that the addressee is not taking into consideration that it is not 
	controversial (in the eyes of the speaker) that they are old friends. 
	\hfill\hbox{\citet[195]{Lindner1991}}
\end{exe}
Unter dieser Interpretation entspricht die Oberflächensyntax folglich nicht den Skopusverhältnissen \is{Skopus}, was Lindner dazu veranlasst, andere Gründe (in ihrem Fall phonologische) für die Oberflächenabfolge verantwortlich zu machen (vgl. Abschnitt~\ref{sec:skopus}). Für die Frage nach dem in MP-Kombinationen vorliegenden Skopusverhältnis liefert dieser Ansatz den Beitrag, dass genau umgekehrte Skopusverhältnisse vorliegen als durch die Syntax vorhergesagt. Im Moment haben wir es folglich mit den drei Möglichkeiten in (\ref{214}) zu tun.
	
\begin{exe}
	\ex\label{214} 
	Skopus der Abfolge $\textrm{MP}_{1}$, $\textrm{MP}_{2}$
		\begin{xlist}	
			\ex\label{214a} $\textrm{MP}_{1}(\textrm{p}) \ \& \ \textrm{MP}_{2}(\textrm{p})$ (gleicher Skopus) (\citealt{Thurmair1989, Thurmair1991})
			\ex\label{214b} $\textrm{MP}_{1}(\textrm{MP}_{2}(\textrm{p}))$ (verschiedener Skopus) $[\textrm{Korrelation mit Syntax}]$
			\newline 
			\hbox{}\hfill\hbox{\citet{Ormelius-Sandblom1997}, \citet{Rinas2007}}
			\ex\label{214c} $\textrm{MP}_{2}(\textrm{MP}_{1}(\textrm{p}))$ (verschiedener Skopus) $[\textrm{keine Korrelation mit Syntax}]$
			\newline
			\hbox{}\hfill\hbox{\citet{Lindner1991}}		
		\end{xlist}	
\end{exe}
Lindner assoziiert das in H3 beschriebene Verhältnis mit der Skopuslesart, für die es prinzipiell zwei Möglichkeiten gibt ((\ref{214b}) und (\ref{214c})), von denen sie (\ref{214c}) für zutreffender hält. Für meine Begriffe muss mit einer sequenziellen Applikation der MPn auf die Proposition aber gar nicht einhergehen, dass die Partikeln Skopus übereinander nehmen. Ich halte eine aufeinander folgende Applikation auf dem gleichen Input für ebenso denkbar. 

Für Lindner sind in (\ref{215hh}) nur (\ref{215s}) und (\ref{215t}) sequenziell. Ich möchte annehmen, dass sich $\textrm{MP}_{1}$ und $\textrm{MP}_{2}$ auch gleichermaßen auf p beziehen können, obwohl sie nacheinander und nicht gleichzeitig applizieren. Auch hier ist es wieder denkbar, dass $\textrm{MP}_{1}$ vor oder nach $\textrm{MP}_{2}$ ihren Beitrag leistet (vgl. (\ref{215v}) und (\ref{215w})).\footnote{Mit der Formulierung in (\ref{215b}) (vgl. auch schon (\ref{214a}) möchte ich ausdrücken, dass sich $\textrm{MP}_{1}$ und $\textrm{MP}_{2}$ beide auf p beziehen, aber dennoch erst die eine und dann die andere Partikel zur Wirkung kommt. Alternativ könnte man dies auch repräsentieren als $[\textrm{MP}_{1} \ \& \ \textrm{MP}_{2}]$(p). Dies suggeriert aber, dass die beiden MPn zusammen eine Einheit bilden, wovon ich nicht ausgehe. Ich halte die MP-Kombinationen nicht für ein komplexes Lexem o.ä. (s.u.). Die Bedeutungszuschreibung entspricht (\ref{203}).	
}

\begin{exe}
	\ex\label{215hh} 
	Kompositionelle Ableitung
		\begin{xlist}	
			\ex\label{215a} Skopus (Asymmetrie)
				\begin{xlist}
						\ex\label{215s} $\textrm{MP}_{1} > \textrm{MP}_{2} \ [\textrm{MP}_{1}(\textrm{MP}_{2}(\textrm{p}))]$
						\ex\label{215t} $\textrm{MP}_{2} > \textrm{MP}_{1} \ [\textrm{MP}_{2}(\textrm{MP}_{1}(\textrm{p}))]$
				\end{xlist}			
			\ex\label{215b} Additiv (Symmetrie)
				\begin{xlist} 		
						\ex\label{215u} $\textrm{MP}_{1} $+$ \textrm{MP}_{2} \ [\textrm{MP}_{1}(\textrm{p}) \ \& \ \textrm{MP}_{2}(\textrm{p}) = \textrm{MP}_{2}(\textrm{p}) \ \& \ \textrm{MP}_{1}(\textrm{p})]$
						\ex\label{215v} $\textrm{MP}_{1}, \textrm{MP}_{2} \ [1. \textrm{MP}_{1}(\textrm{p}), 2. \textrm{MP}_{2}(\textrm{p})]$
						\ex\label{215w} $\textrm{MP}_{2}, \textrm{MP}_{1} \ [1. \textrm{MP}_{2}(\textrm{p}), 2. \textrm{MP}_{1}(\textrm{p})]$
				\end{xlist}
		\end{xlist}	
\end{exe}
Die Annahme der Sequenzierung der beiden Partikeln muss folglich nicht zu unterschiedlichem (hierarchisiertem) Skopus \is{Skopus} führen. Betrachtet man den Aspekt aus Perspektive von H4, entspricht bei Lindner Simultanität gleichem Skopus. Gleicher Skopus kann aber nicht nur durch simultane Applikation zustande kommen, sondern kann auch durch aufeinander folgende Applikation unter gleichem Input eintreten. Die Zuordnung von \textit{simultan} und \textit{gleichem Skopus} sowie \textit{sequenziell} und \textit{verschiedenem Skopus} erschöpft in meinen Augen nicht die Möglichkei\-ten. Unter dieser meiner Sicht entstehen zwei weitere Fälle unter der nach Thurmair additiven Interpretation. Ich werde in meinen Einzelanalysen die Interpretation in (\ref{215v}) bzw. (\ref{215w}) vertreten. Ich bin der Meinung, dass zwischen den Partikeln kein Skopusverhältnis besteht, dass es aber dennoch Gründe gibt, die die Reihung motivieren. Je nachdem welchen Beitrag die Partikeln leisten, gibt es Gründe, die jeweilige Bedeutung früher oder später einzuführen.

Die Zuordnung von \textit{Skopus} und \textit{Sequenzierung} einerseits und \textit{Koordination} und \textit{Simultanität} andererseits steckt in der Darstellung nach \citet{Lindner1991}. \citet{Thurmair1989, Thurmair1991} kann eigentlich nicht davon ausgehen, dass sich die MPn glei\-chzeitig auf p beziehen. Sie formuliert mit ihren Thesen (vgl. Abschnitt~\ref{sec:katalog}) schließlich auch Generalisierungen darüber, welche Arten von Beiträgen vor welchen anderen beigesteuert werden, d.h. auch bei ihr gibt es – unter koordinativer Interpretation – Präferenzen für die Reihenfolge der Präsentation der MP-Beiträge und die Kriterien bauen darauf, dass die verschieden gearteten Einschätzungen des Sachverhalts nacheinander applizieren.\footnote{Auch bei der Reihung von Adjektiven zeigt sich, dass koordinative Interpretationen von Se\-rialisierungsbeschränkungen begleitet sein können. Eine Unterscheidung aus der Literatur zu diesem Thema ist die zwischen \textit{gestuften} und \textit{gereihten} Adjektiven (vgl. z.B. \citealt[348-350]{Trost2006}). Bei ersteren wird von einem Skopusverhältnis gesprochen. In (\ref{xyz}) hat \textit{neue} Skopus über das Nomen, das Adjektiv \textit{falsche} nimmt wiederum Skopus über dieses Nomen. Aus der Menge neuer Scheine wird der falsche Schein ausgewählt.

\begin{exe}
	\ex\label{xyz} 
	der falsche neue Fünfzigeuroschein
	\hfill\hbox{\citet[438]{Trost2006}}
\end{exe}
In (\ref{abc}) hingegen verändern sich die Skopusverhältnisse auf die Art, dass jetzt der neu hinzugekommene Schein der falschen Serie bezeichnet wird.

\begin{exe}
	\ex\label{abc} 
	der neue falsche Fünfzigeuroschein
	\hfill\hbox{\citet[349]{Trost2006}}
\end{exe}
Unter Reihung, wie in (\ref{ghj}) und (\ref{jkl}), haben sowohl \textit{falsche} als auch \textit{neue} Skopus über das Nomen. Gesucht wird der Schein, der falsch und neu ist.

\begin{exe}
	\ex\label{ghj} 
	der falsche, neue Fünfzigeuroschein
\end{exe}
\vspace{-0.65cm}
\begin{exe}
	\ex\label{jkl} 
	der falsche und neue Fünfzigeuroschein
	\hfill\hbox{\citet[349]{Trost2006}}
\end{exe}
In zahlreichen Arbeiten werden syntaktische, semantische und pragmatische Faktoren für Serialisierungsbeschränkungen in Adjektivclustern formuliert (vgl. z.B. \citealt{Posner1980}, \citealt{Eichinger1991}, \citealt{Trost2006}, \citealt{Eroms2011}). Meist werden dabei nur gestufte Adjektive betrachtet. 

\citet[381-383]{Trost2006} weist allerdings nach, dass auch die gereihten Adjektive in seinem Korpus den von ihm aufgestellten Serialisierungsprinzipien für die gestuften Adjektive folgen (wenngleich hier insgesamt eine größere Stellungsfreiheit vorzuliegen scheint). 

Auch wenn sich mehrere Adjektive gleichermaßen auf das Nomen beziehen, greifen scheinbar Abfolgebeschränkungen. Ich habe dazu den Eindruck, dass die Annahme verschiedener Skopusverhältnisse bei gestuften und gereihten Adjektiven in einem anderen Sinne gemeint ist als ich \textit{unterschiedlichen Skopus} oben im Text verstehe. Die Interpretation verschiedener Typen von Adjektiven ist mitunter ein schwieriger Gegenstand. Gestufte und gereihte Adjektive werden, wenn sie jeweils in Bezug auf das Nomen eine intersektive Interpretation zulassen, auch beide intersektiv interpretiert. \citet[60-61]{Posner1980} zeigt auf (und dem stimme ich zu), dass (\ref{bnm}), das ebenfalls nicht ohne Weiteres zu (\ref{cvb}) umgekehrt werden kann, interpretatorisch keinen Unterschied zu (\ref{cvb}) aufweist.

\begin{exe}
	\ex\label{bnm} 
	ein runder weißer Tisch
\end{exe}
\vspace{-0.65cm}
\begin{exe}
	\ex\label{cvb} 
	?ein weißer runder Tisch
	\hfill\hbox{\citet[63]{Posner1980}}
\end{exe}
Eine intersektive Interpretation der Adjektive scheint mir auch hier vorzuliegen. Es handelt sich folglich nicht um ein Skopusverhältnis zwischen den Adjektiven in dem Sinne, dass es um einen auf falsche Art neuen oder auf neue Art falschen Schein geht, genauso wenig wie der Tisch auf weiße Art rund oder auf runde Art weiß ist. Beide Adjektive schränken die durch das Nomen bezeichneten Objekte weiter ein auf Exemplare, die sowohl die vom einen Adjektiv als auch vom anderen Adjektiv bezeichnete Eigenschaft aufweisen.}
 
\subsection{Störfaktoren bei der Bedeutungszuschreibung}
Ich halte es für recht verwunderlich, dass bisher keine Einigung hergestellt werden konnte hinsichtlich der Skopusfrage. Welches Skopusverhältnis anzunehmen ist, ist schließlich eine empirische Frage in dem Sinne, dass zu entscheiden sein sollte, welche Interpretation einer bestimmten Kombination zukommt, insbesondere unter Betrachtung authentischer Beispiele. Ich schließe nicht aus, dass die Interpretation je nach konkreter Kombination eine andere ist (weshalb ich den breiten Generalisierungen von Thurmair $[$Koordination$]$ und Rinas $[$Skopus$]$ eher skeptisch gegenüberstehe). Für ein und dieselbe Sequenz sollten aber nicht verschiedene Alternativen verfügbar sein. Diese Situation ist meiner Meinung nach verschiedenen Aspekten geschuldet, die die Entscheidung über die zu\-treffende Bedeutungszuschreibung stören. Z.B. variieren die zugrundegelegten Mo\-dellierungen der Einzelpartikeln natürlich. Sollte dies der ausschlaggebende Punkt sein, verwundert aber auch dies. Welche Skopusverhältnisse vorliegen, sollte keine Frage bestimmter Modellierungen sein. 

Die Entscheidung ist nur zu fällen, wenn ein Ansatz es vermag, die drei Möglich\-keiten aus (\ref{214}) zu modellieren (zwei Skopusverhältnisse, Addition). Es darf auch nicht sein, dass die technische Ausstattung eines Ansatzes Lesarten prinzipiell ausschließt. Das gilt für \citet{Doherty1985} (vgl. Abschnitt~\ref{sec:doh85}): Weil die Partikeln in ihrem Modell unterschiedliche Anforderungen an die semantischen Objekte in ihrem Skopus stellen, können die Varianten in (\ref{215}) allein aus diesem Grund gar nicht alle formuliert, geschweige denn anhand von Beispielen getestet werden. Ähnlich verhindert eine Modellierung über Merkmale bei \citet{Thurmair1989, Thurmair1991} in gewissem Sinne das Vorliegen von asymmetrischem Skopus, weil sich bei ihnen eine additive Verknüpfung grundsätzlich plausibler anbietet. Auch in der Modellierung in \citet{Ormelius-Sandblom1997} nimmt die Natur ihrer semantischen Beschreibung Einfluss auf ihre konkrete Modellierung von Skopus. In Abschnitt~\ref{sec:skopus} habe ich ihren Ansatz dargestellt. Die Bedeutung von \textit{ja} findet sich in (\ref{215}), von \textit{doch} in (\ref{216}).

\begin{exe} 
	\ex\label{215} 
			\textit{ja}: $\textrm{[FAKTr]}$
\end{exe}
\vspace{-0.65cm}
\begin{exe}
	\ex\label{216} 
		\textit{doch}: $\textrm{[FAKTr}]$\\
		\textsc{Implikatur}$[\exists \textrm{q[q} \rightarrow \neg \textrm{r}]]$
\end{exe}
Für die Abfolge \textit{ja doch} leitet sie hieraus unter Skopus von \textit{ja} über \textit{doch} (\ref{217}) ab.

\begin{exe}
	\ex\label{217}
		\textit{ja} $>$ \textit{doch}\\										
		$[\textrm{FAKT[FAKTr]]}$\\
		\textsc{Implikatur}$[\exists \textrm{q[q} \rightarrow \neg \textrm{r}]]$
\end{exe}
Die Implikatur liegt hier nicht im Skopus von \textit{ja}. Es handelt sich um die gleiche Implikatur, die vorliegt, wenn das \textit{ja} nicht zum \textit{doch} hinzutritt. Im Skopus stünde die Implikatur dann, wenn sie in den Skopus des äußeren FAKT-Operators fiele.

Bei der Modellierung des umgekehrten Skopusverhältnisses, das sie für \textit{doch ja} ansetzt (vgl. (\ref{218})), steht der \textit{ja}-Beitrag anders tatsächlich auch im Skopus der Implikatur.

\begin{exe}
	\ex\label{218}									 
	$\textrm{[FAKT[FAKTr}]]$\\
	\textsc{Implikatur}$[\exists \textrm{q[q} \rightarrow \neg \textrm{FAKTr}]]$
\end{exe}
FAKT r (ja(p)) wird hier für $\neg$p auch in der Implikatur eingesetzt. Ohne es zu explizieren,  scheint Ormelius-Sandblom anzunehmen, dass die Implikatur einer MP \is{Implikatur} nicht in den Skopus einer über diese MP Skopus nehmenden MP fällt.

Das Verhältnis von (konventionellen) Implikaturen \is{konventionelle Implikatur} (und auch Präsuppositionen \is{Präsupposition}) in eingebetteten Kontexten ist sicherlich ein viel diskutiertes und komplexes Thema, dem ich an dieser Stelle nicht gerecht werden kann. Es soll al\-lerdings der Hinweis darauf gegeben werden, dass auch Bedeutungen, die als konventionelle Implikaturen behandelt werden, prinzipiell in den Wirkungsbereich anderer Skopus nehmender Elemente fallen können und dann auch lokal, d.h. im Skopus dieses Elementes, interpretiert werden können bzw. gar derart interpretiert werden müssen. In (\ref{219}) tritt ein non-restriktiver Relativsatz \is{non-restriktiver Relativsatz} auf, der von \citet{Potts2005} zu den konventionellen Implikaturen gezählt wird, und dessen Inhalt hier Gültigkeit innerhalb von Joans Glaubenssystem hat und nicht auf den glo\-balen Diskurskontext bezogen wird (zu weiteren Beispielen vgl. \citealt[733-739]{Amaral2007}, vgl. auch \citealt{Harris2009}).

\begin{exe}
	\ex\label{219}									 
	Joan is crazy. She's hallucinating that some geniuses in Silicon Valley have invented a new brain chip that's been installed in her left temporal lobe 	[...]. Joan believes that her chip, \textbf{which she had installed last month}, has a twelve year guarantee.
	\hfill\hbox{\citet[735-736]{Amaral2007}}	
\end{exe}
Durch den Kontext ist klar, dass der Sprecher gerade nicht davon ausgeht, dass Joan der Chip eingepflanzt wurde. Nur Joan geht davon aus.

Ormelius-Sandblom kann deshalb – wenn sie sich darauf festlegt, dass konventionell implikatierte Inhalte beteiligt sind – nicht grundsätzlich ausschließen, dass sie lokal, d.h. im Skopus des sie einbettenden Elements, interpretiert werden. Ich frage mich deshalb, wie motiviert ist, dass die Skopuslesart nicht ist: 

\begin{exe}
	\ex\label{220}									 
	\glq Es ist Fakt, dass r Fakt ist und dass es eine andere Proposition gibt, aus der das Gegenteil von r folgt.\grq {}
\end{exe}
Wenn solche Aspekte wie unabhängig (un)zulässige Einbettungsmöglichkeiten beteiligt sein können, spricht dies dafür, die MP-Bedeutung möglichst neutral hinsichtlich der Annahme der beteiligten Natur von Bedeutung anzusetzen bzw. andernfalls diese Arten von Bedeutung mit all ihren Konsequenzen zu behandeln.

Generell stellt sich mir bei manchen Skopusmodellierungen auch die Frage, wie die Entscheidung möglich sein soll, ob (eher unhandliche) Beschreibungen wie in (\ref{221}) zutreffen oder nicht.
	
\begin{exe}
	\ex\label{221} 
		\textit{ja ruhig}
		\begin{xlist}	
			\ex\label{221a} JA(RUHIG(p) $»$ KEINE-BEDENKEN (S,(REALISIERT (H,p)))\\ \& WILL(H,(REALISIERT(H,p))))
			$»$ NICHT-GLAUBT(H,\\ NICHT(RUHIG(p) $»$ KEINE-BEDENKEN (S,(REALISIERT (H,p))) \\ \& WILL(H,(REALISIERT(H,p)))))		
			\ex\label{221b} \glq Es ist unkontrovers (es ist nicht der Fall, dass der Hörer daran zweifelt), dass der Sprecher S keine Bedenken dagegen hat, dass der Hörer H p realisiert und dass H p realisieren will.\grq {}
			\hfill\hbox {\citet[436]{Rinas2007}}
		\end{xlist}
\end{exe}
Rinas weist (abgesehen von der Sequenz \textit{ja auch}) nicht nach, dass seine anderen Skopusmodellierungen zutreffen. Er vertritt diese Bedeutungszuschreibung viel\-mehr pauschal. Man sollte aber in Kontexte schauen und die verschiedenen Lesarten durchspielen.\\

\noindent
Neben diesen angeführten Arbeiten, die explizit Aussagen über die auftretenden Skopusverhältnisse machen und diese zumindest teilweise für ihre Ableitung (un)zulässiger Abfolgen zu Nutze machen, kann man die Skopusfrage auch in Bezug auf Ansätze stellen, die gar nicht direkt unter Bezug auf Skopus argumentieren.

\subsection{Implizite Skopusannahmen}
\label{sec:skopusimp}
Beispielsweise ist der Ansatz von \citet{Vismans1994} (vgl. Abschnitt~\ref{sec:vismans} .E. eine Arbeit, die nicht explizit unter Bezug auf Skopus argumentiert in dem Sinne, dass sie zwar nicht aus den Skopusverhältnissen selbst die Abfolgen ableitet, das zugrundegelegte Prinzip aber sehr deutliche Vorhersagen macht, weil an dieses bestimmte Skopusannahmen \is{Skopus} gebunden sind.

Vismans stützt sich in seiner Ableitung auf die Prinzipien in (\ref{222}) und (\ref{223}).

\begin{exe}
	\ex\label{222} 
		Generelles Prinzip 3: The Principle of Centripetal Orientation \is{Prinzip der zentripetalen Orientierung}\\
		Constituents conform to (GP3) when their ordering is determined by their relative distance from the head, which may lead to  mirror-image  					ordering around the head.
		\hfill\hbox {\citet[401]{Dik1997}}
\end{exe}

\begin{exe}
	\ex\label{223} 
	$\pi$-operators prefer centripetal orientation according to the schema:\\
 	$\pi_{\textrm{4}}\pi_{\textrm{3}}\pi_{\textrm{2}}\pi_{\textrm{1}}\textrm{[stem]}\pi_{\textrm{1}}\pi_{\textrm{2}}\pi_{\textrm{3}}\pi_{\textrm{4}}$ 
 	\newline
 	\hbox{}\hfill\hbox{\citet[414]{Dik1997}, ursprünglich \citet[141]{Hengeveld1989}}
\end{exe}
Wie in Abschnitt~\ref{sec:vismans} ausführlicher erläutert, beziehen sich die Nummerierungen auf verschiedene Satzschichten \is{Satzschichten} (4: Illokution \is{Illokutionsschicht}, 3: Proposition \is{Propositionsschicht}, 2: Prädikation \is{Prädikationsschicht}, 1: Prädikat \is{Prädikat}). Vismans behandelt MPn als Operatoren \is{Operator} und verankert sie auf verschiedenen dieser Ebenen.

Bei \citet[402]{Dik1997} heißt es, dass die zentripetale Orientierung der Konstituen\-ten die Nähe des Bundes zwischen den Abhängigen und dem Kopf widerspiegelt und die Skopusrelationen zwischen den Abhängigen: Die $\pi_{\textrm{4}}$-Operatoren haben den weitesten Skopus und nehmen die anderen Operatoren in ihren Wirkungsbereich. $\pi_{\textrm{3}}$-Operatoren nehmen Skopus über $\pi_{\textrm{2}}$ und $\pi_{\textrm{1}}$. Dies bedeutet, dass zwi\-schen MPn, die man verschiedenen Schichten zuordnet, Skopus besteht.

Vismans spielt nicht an Beispielen durch, dass/ob die Interpretation auf seine Beispiele zutrifft. Die Annahme um Skopus zwischen den Schichten folgt vielmehr aus dem Festhalten an dem Prinzip und ist in diesem Sinne motiviert aus der Architektur des Modells. Wie ich oben erläutert habe, ist das Prinzip nicht über MPn motiviert worden, sondern über Fälle, wie z.B. in (\ref{224}), für die tatsächlich von geschachtelten Skopoi auszugehen ist.

\begin{exe}
	\ex\label{224} 
	Sie sagen, der Baum fing vor langem an zu wachsen.
\end{exe}
\vspace{-0.65cm}
\begin{exe}
	\ex\label{225} 
	Evidenzialität ($\pi_{\textrm{3}}$) > Tempus ($\pi_{\textrm{2}}$) > Aktionsart ($\pi_{\textrm{1}}$)
\end{exe}
Zumindest zwischen den MP-Klassen müsste Vismans ein Skopusverhältnis \is{Skopus} ansetzen.\footnote{Analog müsste auch \citet{Ickler1994} von einem Skopusverhältnis ausgehen (vgl. Abschnitt~\ref{sec:stai}).}

Ähnlich kann man auch mit anderen Ansätzen vorgehen. \citet{Abraham1995} geht davon aus, dass sich die MPn auf LF in Top-Kopfpositionen bewegen. Motiviert über die Vormodalpartikellexeme \is{Vormodalpartikellexem} werden die MPn drei verschiedenen Klassen zugeordnet, denen je andere Domänen der TopP entsprechen, die wiederum in einem hierarchischen Verhältnis zueinander stehen. Ganz genau führt er nicht aus, wie er sich die Ordnung vorstellt. Er spricht aber auch von den Specs der drei Top-Domänen. Wenn jede Domäne einen Kopf und einen Spec hat, wäre es plausibel, anzunehmen, dass sich die Top-Phrasen gegenseitig selegieren (vgl. (\ref{226})).

\begin{exe}
	\ex\label{226}   
\begin{jtree}
\! = {$\textrm{\textbf{TopP}}_{1}$}
:{$\textrm{SpecTopP}_{1}$} {$\textrm{Top}^{\prime}_{1}$}
:({$\textrm{Top}^{0}_{1}$} {\textbf{C1}}) {$\textrm{\textbf{TopP}}_{\textbf{2}}$}
:{$\textrm{SpecTopP}_{2}$} {$\textrm{Top}^{\prime}_{2}$}
:({$\textrm{Top}^{0}_{2}$} {\textbf{\textbf{C2}}}) {$\textrm{\textbf{TopP}}_{\textbf{3}}$}
:{$\textrm{SpecTopP}_{3}$} {$\textrm{Top}^{\prime}_{3}$}
:({${\textrm{Top}^{0}_{3}}$} {\textbf{C3}}){CP}
:{$\textrm{C}^{0}$} {IP}.
\end{jtree}
\end{exe}
Dann stehen C1, C2 und C3 in einem hierarchischen Verhältnis zueinander. Und wenn er (wie er schreibt) der klassischen Sicht folgt, dass Skopus \is{Skopus} unter c-Komman\-do \is{c-Kommando} eintritt (er schreibt dies für Operatoren und sieht die MPn als Sprechaktope\-ratoren gerade in Parallelität zum Skopus von w-Phrasen), sollte zwischen MPn, die nicht der gleichen Klasse angehören, auf LF Skopus bestehen.

Auch aus Abrahams Architektur folgt somit, dass er letztlich von einem Skopusverhältnis zwischen den (drei Klassen von) MPn ausgehen muss. Welche Skopusannahmen für Partikeln der gleichen Klasse anzunehmen sind, ist eine Frage, die sich sowohl für Vismans' als auch Abrahams Ansatz stellt (dazu s.u. Vismans).\\

\noindent
Dieser Blick auf Ansätze, die explizite Skopusannahmen machen bzw. denen Vorstellungen zuzuschreiben sind, zeigt, dass in den meisten Fällen von einem Skopusverhältnis ausgegangen wird. Aus Bedeutungsperspektive (womit ich nicht die konkrete Betrachtung von Beispielen im Kontext meine – was die meisten Ansätze gerade nicht tun) scheint man sich schnell für ein asymmetrisches Verhältnis zu entscheiden (Ausnahme ist hier nur \citealt{Thurmair1989}) und auch im Zusammenhang mit syntaktischen Modellierungen scheint Skopus nahezuliegen. Dies ist aber darin begründet, dass in der Syntax in den meisten Modellen von hie\-rarchischem Strukturaufbau ausgegangen wird. Sobald MP-Sequenzen auf diese Strukturen abgebildet werden, \glq kaufen\grq {} die Arbeiten die syntaktischen Skopusverhältnisse (ob zwischen Schichten oder Phrasen), die sich dann auch semantisch spiegeln sollten.\footnote{Es gibt allerdings auch syntaktische Vorschläge, die gerade die additive Lesart aufzufangen beabsichtigen. \citet[98]{Coniglio2011} geht z.B. für offene und geschlossene Kombinationen davon aus, dass MPn nicht übereinander Skopus nehmen. Er fasst MPn als defekte maximale Projektionen auf, die in einer funktionalen IP-internen Projektion basisgeneriert werden. In Kombinationen adjungieren MPn aneinander und bilden dadurch komplexe Cluster (vgl. \citeyear[119]{Coniglio2011}). Diese sind als Ganzes in der IP basisgeneriert und können sich als Ganzes oder in Teilen bewegen. Das additive Verhältnis ergibt sich somit aus der Adjunktions\-relation. Für den Fall einer offenen Kombination scheint mir die Annahme gleichen Skopus im Rahmen von Coniglios Modellierung allerdings schon wieder unplausibler, wenn sich die Partikel aus dem Cluster in eine höhere Position bewegt. \citet[121]{Coniglio2011} schreibt: \glqq die höhere Partikel einer Kombination [modifiziert] die niedrigere Partikel durch Adjunktion\grqq{}. Mir ist nicht klar, wie syntaktischer und semantischer Skopus hier einhergehen würden. Vielleicht ist aber auch genau dies der Grund dafür, dass gerne auch im Bereich der Interpretation von Skopus ausgegangen wird, obwohl eigentlich das syntaktische Skopusverhältnis angedacht ist.

Zu einer anderen syntaktischen Repräsentation von MP-Kombinationen aus Perspektive von Dependenz und Valenz, deren Bedeutungszuschreibung sich mir jedoch nicht erschließt, vgl. \citet[1021]{Eroms2006}.
} 
Ob diese Interpretation wirklich zutrifft, wird dann in den meisten Fällen jedoch nicht überprüft oder diskutiert.\\

\noindent
Wenngleich die verschiedenen Arbeiten durchaus unterschiedliche Skopusannahmen machen, gehen sie alle von der Möglichkeit einer transparenten Bedeutungszuschreibung aus: Die Bedeutung der Kombination ergibt sich aus der Bedeutung der Einzelpartikeln und ihrer jeweils angenommenen Verknüpfung. Ich werde bei der Analyse der Kombinationen aus \textit{ja} \& \textit{doch} und \textit{halt} \& \textit{eben} sowie \textit{doch} \& \textit{auch} ebenfalls von einer kompositionellen Analyse ausgehen. Genauer werde ich in allen drei Fällen für die Lesart unter gleichem Skopus argumentieren.\\

\noindent
Meine Bedeutungszuschreibung an die drei betrachteten MP-Kombinationen neh\-me ich in den Kapiteln~\ref{chapter:jud}, \ref{chapter:hue} und \ref{chapter:dua} in Form einer Beschreibung des Diskursbeitrags der Äußerungen vor. Im nächsten Abschnitt wird das Diskursmodell nach \citet{Farkas2010} vorgestellt, im Rahmen dessen ich diese Beschreibung leisten werde.

\section{Das Diskursmodell}
\label{sec:diskursmodell}
Das Modell steht in der Tradition von Arbeiten, die eine \textit{dynamische Bedeutungstheorie} \is{dynamische Bedeutungstheorie} zur Bestimmung/Modellierung der Bedeutung von Sätzen/Äußerungen verfolgen (vgl. z.B. auch \citealt{Hamblin1971}, \citealt{Stalnaker1978}, \citealt{Ginzburg1996}, \citealt{Roberts1996}, \citealt{Giannakidou1998}, \citealt{Bartels1999}, \citealt{Stalnaker2002}, \citealt{Buering2003}, \citealt{Farkas2003}, \citealt{Gunlogson2003}, \citealt{Caponigro2007}, \citealt{Malamud2011}). Derartige Ansätze teilen die Perspektive, dass es sich bei einem Diskurs um eine Sequenz von Informationszuständen handelt. Jeder Äußerungskontext stellt einen bestimmten Informationszustand dar, der durch Äußerungen der Gesprächsteil\-nehmer verändert werden kann. Es handelt sich bei diesen Theorien folglich deshalb um dynamische Bedeutungstheorien, weil sich die Bedeutung von Sätzen/Äußerungen durch ihren Einfluss auf den Kontext einer Konversation bestimmt. Genauer werden die jeweils im Kontext hervorgerufenen Veränderungen betrachtet. Bestimmt wird das sogenannte \textit{Kontextwechselpotential} \is{Kontextwechselpotential} von sprachlichen Ausdrücken. Wenngleich die Arbeiten diese Grundannahme teilen, unterscheiden sie sich (je nach Absicht der Untersuchung) darin, wie viele und welche Komponenten das jeweilige Diskursmodell aufweist. \citet{Farkas2010} nehmen die vier Komponenten in (\ref{227}) an.

\begin{exe}
	\ex\label{227} 
		Komponenten des Diskursmodells
		\begin{xlist}	
			\ex\label{227a} Common Ground \is{Common Ground}
			\ex\label{227b} Diskursbekenntnismenge (discourse commitment set) \is{Diskursbekenntnismenge (discourse commmittment set)}
			\ex\label{227c} Tisch (table)	\is{Tisch (table)}
			\ex\label{227d} Projektionsmenge (projected set) \is{Projektionsmenge (projected set)}
		\end{xlist}
\end{exe}
Die Rolle dieser Komponenten in der Bestimmung des Kontextwechselpotentials einer Äußerung wird im folgenden Abschnitt erläutert.

\subsection{Die Komponenten des Diskursmodells}
Eine der Komponenten, die in jedem Diskursmodell im Mittelpunkt steht, ist der \textit{Common Ground} (cg), der über eine Menge von Propositionen modelliert wird. Die Inhalte des cg sind bei \citet{Farkas2010} die bewusst geteilten \underline{öffentlichen Diskursbekenntnisse} der Gesprächspartner. Die Propositionen des cg sind die Inhalte, zu denen sich die Konversationspartner öffentlich bekannt haben, hinsichtlich derer sie sich einig sind und von denen sie gegenseitig wissen, dass sie sich auf sie geeinigt haben. Hinzu kommt ebenfalls bewusst geteiltes Hintergrundwissen (allgemeine\-rer Natur).\\

\noindent 
Neben dieser Gesamtmenge an Diskursbekenntnissen nehmen die Autoren auch individuelle Systeme für die Diskursbekenntnisse der einzelnen Gesprächs\-teilnehmer an. Neben dem cg gibt es somit das \textit{discourse commitment set}, eine Diskursbekenntnismenge \is{Diskursbekenntnismenge} ($\textrm{DC}_{\textrm{X}}$) von jedem einzelnen Diskursteilnehmer X. Das DC beinhaltet die Propositionen, zu denen sich der jeweilige Teilnehmer im Verlauf des bisherigen Kontextes öffentlich bekannt hat und die (noch) nicht geteilte Bekenntnisse sind. Die kompletten Diskursbekenntnisse eines Teilnehmers X werden folglich erfasst durch: $\textrm{DC}_{\textrm{X}}$ $\cup$ cg.

Für alle im Modell auftretenden Diskursbekenntnisse gilt, dass sie nicht wirklich wahr sein müssen, sondern dass sie aus der Perspektive der Konversation als wahr angesehen werden. Genauso müssen Diskursbekenntnisse auch nicht wirklich von den Beteiligten für wahr gehalten werden, wenngleich die Menge von Diskursbekenntnissen eines Teilnehmers standardmäßig (d.h. unter Berufung auf die Grice'sche Qualitätsmaxime \is{Maxime der Qualität} $[\textit{\textrm{Sage nichts, was du für falsch hältst!}}]$ (\citealt{Grice1989}) der Teilmenge seines privaten Glaubenssystems entsprechen sollte.

Wenngleich Diskursbekenntnisse sowohl in cg als auch $\textrm{DC}_{\textrm{X}}$ enthalten sein können, ist allein cg die Ebene, die der Erfüllung von Präsuppositionen \is{Präsupposition} dient. Vorausgesetzt werden kann nur, was sich in cg befindet. Hat A sich in (\ref{228a}) beispielsweise dazu bekannt, dass Christoph früher einmal bei Melitta gearbeitet hat, und haben sich A und B darauf geeinigt, dass sie sich in dieser Sache nicht einig sind, d.h. p ist nicht cg geworden, kann A nicht (\ref{228b}) äußern.

\begin{exe}
	\ex\label{228} 
		Komponenten des Diskursmodells
		\begin{xlist}	
			\ex\label{228a} $\textrm{DC}_{\textrm{A}}$ = ${\textrm{Christoph hat früher einmal bei Melitta gearbeitet}}$\\
							$\textrm{DC}_{\textrm{B}}$ = ${\textrm{Christoph hat nicht früher einmal bei Melitta gearbeitet}}$	
			\ex\label{228b} A: Christoph arbeitet jetzt \textbf{\textit{wieder}} bei Melitta.  
		\end{xlist}
\end{exe}

\noindent
Eine weitere Diskurskomponente ist der \textit{Tisch} \is{Tisch} (\textit{table}). Dieser hält die offenen Themen des Gesprächs fest (vgl. die \textit{Question Under Discussion} \is{Question Under Discussion} $[\textrm{QUD}]$ z.B. in den Diskursmodellierungen in \citealt{Ginzburg1996}, \citealt{Roberts1996}, \citealt{Buering2003}). In dem Modell von \citet{Farkas2010} sind auf dem Tisch platzierte Elemente Paare von syntaktischen Objekten und ihren Denotaten. 

Wird eine Assertion \is{Assertion} getätigt, liegt nach ihrer Äußerung auf dem Tisch die syntaktische Information, dass es sich um einen Deklarativsatz handelt, sowie die semantische Information, dass das Denotat die Proposition p ist (vgl. (\ref{229})).

\begin{exe}
	\ex\label{229} 
			Assertion: Astrid ist zu Hause.\\
			Tisch: $\langle \textrm{Astrid ist zu Hause}[\textrm{D}]; \lbrace \textrm{p}\rbrace \rangle$
\end{exe}
Nachdem eine Entscheidungsfrage \is{E-Frage} gestellt wurde, ist auf dem Tisch sowohl die syntaktische Information, dass es sich um einen Interrogativsatz handelt, als auch die semantische Information, dass das Denotat der Frage der Menge der Propositionen aus p und $\neg$p entspricht, verfügbar (vgl. (\ref{230})).

\begin{exe}
	\ex\label{230} 
			Entscheidungsfrage: Ist Astrid zu Hause?\\
			Tisch: $\langle \textrm{Astrid ist zu Hause} [\textrm{I}];\lbrace \textrm{p},\neg \textrm{p}\rbrace\rangle$
\end{exe}
Dazu kommt die Annahme, dass die Elemente auf dem Tisch einen Stapel formen, innerhalb dessen das aktuellste \glq Problem\grq {} oben aufliegt. Der Tisch speichert folglich, was in der Konversation zur Debatte steht, d.h. aktuelles Gesprächsthema ist/offen ist/gerade ausgehandelt wird. Solange sich Elemente auf dem Tisch befinden, gibt es zu regelnde Themen.

Unter Bezug auf die Belegung des Tisches lassen sich zwei Kontextzustände unterscheiden: Eine Konversation befindet sich in einem \textit{stabilen} Kontextzustand \is{stabiler Kontextzustand (stable state)} (\textit{stable state}), wenn der Tisch leer ist. Sie befindet sich in einem \textit{instabilen} Kontextzustand, \is{instabiler Kontextzustand} wenn Objekte auf dem Tisch liegen. Damit geht einher, dass eine Konversation sich nur an einem natürlichen Endpunkt befindet, wenn der Kontextzustand stabil ist.\\

\noindent
Die vierte Komponente im Diskursmodell ist die \textit{Projektionsmenge} (\textit{projected set} \is{Projektionsmenge (projected set)} $[\textrm{ps}]$). Die Funktion dieser Komponente ist vor dem Hintergrund der Überlegung der Autoren zu sehen, dass es auf zwei Arten kanonische Verhaltensweisen in der Konversation gibt. Die eine gilt für Gespräche allgemein, die andere bezieht sich konkret auf bestimmte Sprechakte. 

Zum einen wird Konversation \citeauthor{Farkas2010} zufolge generell durch zwei Aspekte getrieben: Der erste ist, dass Teilnehmer dem Bedürfnis folgen, den cg zu erweitern. Weil sie danach streben, platzieren sie überhaupt Elemente auf dem Tisch. Der zweite ist, dass Teilnehmer danach streben, einen stabilen Kontextzustand zu erreichen, d.h. einen Zustand, in dem keine Frage offen ist oder (mit den Komponenten sprechend) nichts auf dem Tisch liegt. Aufgrund dieser Bestrebung entfernen die Teilnehmer die Elemente dann so vom Tisch, dass der cg erweitert wird. Dies sind kanonische Verhaltensweisen für Gespräche allgemein. \citet{Farkas2010} nehmen darüber hinaus konkreter an, dass es für einzelne Sprechakte kanonische Reaktionen gibt. Jeder Zug im Diskurs, der ein Element auf dem Tisch platziert, ist deshalb mit einem kanonischen Zug verbunden, um dieses  Element vom Tisch zu entfernen. Der kanonische Weg, ein Thema vom Tisch zu entfernen, ist, einen Diskurszustand zu erreichen, in dem das Thema \textit{entschieden} (\textit{decided}) \is{Entschiedenheit} ist. Eine Proposition p gilt relativ zum cg als entschieden, wenn p oder $\neg$p Teil des cg ist.  Züge, die Elemente auf dem Tisch platzieren, definieren Zielzustände der Konversation (oder ggf. Mengen solcher Zustände), die erreicht sind, wenn das betroffene Element auf die Art vom Tisch entfernt wird, die den cg erweitert. Diese Idee des kanonischen Zuges, um ein Element vom Tisch zu entfernen, wird mit der \textit{Projektionsmenge} (ps) \is{Projektionsmenge (projected set)} durch eine eigene Komponente aufgefangen. Ein Zug im Diskurs, der ein Element auf dem Tisch platziert, projiziert simultan eine Menge von zukünftigen cgs, relativ zu denen das Thema auf dem Tisch entschieden ist. Diese Mengen im ps sind Obermengen des aktuellen cg. 

\subsection{Illustration des Zusammenspiels der Komponenten: Der assertive Kontextwechsel}
Um die Rolle dieser Komponenten in der Modellierung von Diskurssituationen zu veranschaulichen, werden in diesem Abschnitt die Schritte des Kontextwechsels unter Äußerung einer Assertion \is{Assertion} durchgespielt. In Kapitel~\ref{chapter:hue} werde ich das Mo\-dell erweitern, um Direktive erfassen zu können. In \citet{Farkas2010} werden neben Assertionen auch Entscheidungsfragen behandelt.

Betrachtet wird im Folgenden der Effekt, den eine Standardassertion, die \citet{Farkas2010} durch einen V2-Deklarativsatz mit fallender Intonation realisiert sehen, auf den Kontext ausübt. 

Bevor die Assertion in (\ref{232}) getätigt wird, besteht der Kontextzustand $\textrm{K}_{1}$ (vgl. (\ref{231})).

\newcolumntype{C}[1]{>{\centering}p{#1}} 
\begin{exe}
\ex\label{231} K$_1$: initialer Kontextzustand\\[-1em]
\begin{tabular}[t]{| C{6em}|p{6em}|p{6em}|  C{6em}|}
\hline
 $\textrm{DC}_{\textrm{A}}$ & \multicolumn{2}{C{12em}|}{Table} &  $\textrm{DC}_{\textrm{B}}$ \tabularnewline
\hline
{} & \multicolumn{2}{p{12em}|}{} & {}  \tabularnewline
\hline
\multicolumn{2}{|p{12em}|}{Common Ground $\textrm{s}_{1}$}&\multicolumn{2}{|p{12em}|}{Projected Set $\textrm{ps}_{1} = \lbrace \textrm{s}_{1} \rbrace$} \tabularnewline
\hline
\end{tabular}
\end{exe}

\begin{exe}
	\ex\label{232}
	A: Astrid ist zu Hause.
\end{exe}
Der Effekt, den die Äußerung der Assertion aus (\ref{232}) auf den Kontext nimmt, ist, dass die ausgedrückte Proposition p $\textrm{DC}_{\textrm{A}}$ hinzugefügt wird, sowie, dass die syntaktische Struktur des Satzes und sein Denotat oben auf den Stapel auf dem Tisch abgelegt werden. Am Zustand des cg ändert sich nichts: Der neue Zustand $\textrm{s}_{2}$ ist identisch mit dem vorherigen (vgl. (\ref{233})).

\begin{exe}
\ex\label{233} $\textrm{K}_{2}$: A hat relativ zu $\textrm{K}_{1}$ assertiert: \textit{Astrid ist zu Hause}.\\[-1em]
\begin{tabular}[t]{| C{6em}|p{6em}|p{6em}|  C{6em}|}
\hline
 $\textrm{DC}_{\textrm{A}}$ & \multicolumn{2}{C{12em}|}{Table} &  $\textrm{DC}_{\textrm{B}}$ \tabularnewline
\hline
{p} & \multicolumn{2}{p{12em}|}{$\langle \textrm{Astrid ist zu Hause} [\textrm{D}];\lbrace\textrm{p}\rbrace\rangle$} & {}  \tabularnewline
\hline
\multicolumn{2}{|p{12em}|}{Common Ground $\textrm{s}_{2}$ = $\textrm{s}_{1}$}&\multicolumn{2}{|p{12em}|}{Projected Set $\textrm{ps}_{2} = \lbrace \textrm{s}_{1} \cup \lbrace \textrm{p} \rbrace \rbrace$} \tabularnewline
\hline
\end{tabular}
\end{exe}
Der Inhalt dieser Assertion kann jetzt cg-Inhalt werden, indem B ihn annimmt/\\bestätigt/akzeptiert, etwa durch Äußerungen bzw. non-verbale Handlungen wie in (\ref{234}).

\begin{exe}
	\ex\label{234}
	B: Ah ok./Ah./Alles klar./Ja, das stimmt./nicken/schweigen etc.
\end{exe}
Bevor B eine derartige Reaktion nicht zeigt, bleibt p allein As Beitrag, zu dem er sich öffentlich bekannt hat.

Aus dieser Situation, die sich nach der Äußerung von (\ref{232}) einstellt, leiten \citet{Farkas2010} ab, dass Assertionen ein Thema eröffnen können, indem sie ein Element auf den Tisch legen. Zu \is{Assertion} assertieren, \underline{dass} p bedeutet unter dieser Sicht, das Thema aufzumachen, \underline{ob} p gilt. Mit anderen Worten, wird $\langle[\textrm{D}]; \lbrace \textrm{p}\rbrace \rangle$ auf den Tisch gelegt, steht die Frage zur Debatte, ob p, d.h. letztlich p $\vee$ $\neg$p. In diesem Sinne hat eine Assertion die gleiche Funktion wie eine \is{E-Frage} Entscheidungsfrage. Der einzige Unterschied ist, dass eine Entscheidungsfrage kein Sprecherbekenntnis beinhaltet, sondern sich in dieser Hinsicht neutral verhält.

Akzeptiert B p durch eine Art von bestätigender Reaktion (vgl. (\ref{234})), weist auch B ein Diskursbekenntnis zu p auf, d.h. beide Gesprächspartner haben ein Diskursbekenntnis zu p (vgl. Teil 1 in (\ref{235a})). Anschließend gelangt p in den cg als bewusst geteiltes öffentliches Bekenntnis beider Gesprächspartner und das Thema wird vom Tisch entfernt (vgl. (\ref{235b})).

\begin{exe}
	\ex\label{235} 
		$\textrm{K}_{3}$: Reaktion durch B \& Verlauf
		\begin{xlist}	
			\ex\label{235a} Teil 1: B hat As Inhalt bestätigt\\[-1em]
				\begin{tabular}[t]{| C{6em}|p{6em}|p{6em}|  C{6em}|}
				\hline
 				$\textrm{DC}_{\textrm{A}}$ & \multicolumn{2}{C{12em}|}{Table} &  $\textrm{DC}_{\textrm{B}}$ \tabularnewline
				\hline
				{p} & \multicolumn{2}{p{12em}|}{$\langle \textrm{Astrid ist zu Hause} [\textrm{D}];\lbrace\textrm{p}\rbrace\rangle$} & {p}  								\tabularnewline
				\hline
				\multicolumn{2}{|p{12em}|}{Common Ground $\textrm{s}_{3}$ = $\textrm{s}_{2}$}&\multicolumn{2}{|p{12em}|}{Projected Set $\textrm{ps}_{3} = 					\lbrace \textrm{s}_{1} \cup \lbrace \textrm{p} \rbrace \rbrace$} \tabularnewline
				\hline
				\end{tabular}

			\ex\label{235b} Teil 2: p wird cg\\[-1em]
				\begin{tabular}[t]{| C{6em}|p{6em}|p{6em}|  C{6em}|}
				\hline
 				$\textrm{DC}_{\textrm{A}}$ & \multicolumn{2}{C{12em}|}{Table} &  $\textrm{DC}_{\textrm{B}}$ \tabularnewline
				\hline
				{} & \multicolumn{2}{p{12em}|}{} & {}  								\tabularnewline
				\hline
				\multicolumn{2}{|p{14em}|}{Common Ground $\textrm{s}_{4}$ = $\lbrace \textrm{s}_{1} \cup \lbrace \textrm{p} \rbrace \rbrace$}&\multicolumn{2}{|p{10em}|} {Projected Set $\textrm{ps}_{4} = \lbrace \textrm{s}_{4}\rbrace$} \tabularnewline
				\hline
				\end{tabular}			
		\end{xlist}
\end{exe}
Der Vergleich der Füllung von ps von (\ref{231}) über (\ref{233}) und (\ref{235a}) zu (\ref{235b}) illustriert auch konkret die Vorstellung der kanonischen Reaktion auf Züge im Diskurs (hier Assertionen) nach \citet{Farkas2010}. Da der Kontextzustand in $\textrm{K}_{1}$ (vgl. (\ref{231})) noch nicht verändert wurde, tritt auch weiter keine Veränderung beim ps ein. Die Menge entspricht ihrer Beschaffenheit, die Resultat des letzten Kontextwechsels ist. Wie bereits erläutert, machen Assertionen Vorschläge, indem ein Sprecher ein öffentliches Bekenntnis abgibt, wodurch er das Thema auf dem Tisch eröffnet. Ist es das Ziel eines Gesprächs, die offene(n) Frage(n) des Tisches zu entscheiden, tut man dies, wenn das Thema durch das öffentliche Bekenntnis von p eröffnet wurde, als Adressat möglichst direkt, indem man das Angebot des ersten Sprechers annimmt. Die kanonische Reaktion auf Assertionen ist nach \citet{Farkas2010} deshalb die Bestätigung der Assertion. Widerspruch und Ablehnung sind natürlich zulässige Reaktionen, doch es handelt sich den Autoren zufolge nicht um Standardzüge. Es gibt im Falle der Äußerung von Assertionen folglich so etwas wie einen konversationellen Druck, den cg anzureichern, indem veröffentlichte Bekenntnisse zu bewusst geteilten Bekenntnissen gemacht werden. Um diesen Zustand zu erreichen, muss der Gesprächspartner den Inhalt annehmen (zur weiteren Motivation dieser Annahme vgl. \citealt[85]{Farkas2010}). 

Eine Assertion projiziert folglich Bestätigung: Sie projiziert einen zukünftigen cg, in dem die assertierte Proposition enthalten ist. Dies sieht man an der Füllung in $\textrm{K}_{2}$. Eine Assertion weist in diesem Sinne eine Voreingenommenheit zugunsten der ausgedrückten Proposition auf. Vergleicht man die Füllung von cg und ps, sieht man, wie der vorweggenommene cg aus ps (vgl. $\textrm{K}_{2}$ in (\ref{233})) in $\textrm{K}_{3}$ Teil 2 (vgl. \ref{235b}) dem tatsächlichen neuen cg-Zustand entspricht. Das ps entspricht nach Aufnahme von p in den cg dem aktuellen cg (wenn der Tisch leer ist) bzw. projiziert im Einvernehmen mit den Elementen, die sich nach diesem Kontext\-wechsel noch immer auf dem Tisch befinden, die Menge denkbarer cgs, die eine Auflösung der offenen Themen auf dem Tisch relativ zum aktuellen cg darstellen (würden).

Die Autoren formalisieren die beschriebenen Kontextveränderungen wie in (\ref{236}) notiert. 

\begin{exe}
	\ex\label{236} 
		S($[\textrm{D}], \textrm{a}, \textrm{K}_{\textrm{i}}) = \textrm{K}_{\textrm{o}}$ such that
		\begin{xlist}	
			\ex\label{236a} $\textrm{DC}_{\textrm{a,o}} = \textrm{DC}_{\textrm{a, i}} \cup \lbrace\textrm{p}\rbrace$
			\ex\label{236b} $\textrm{T}_{\textrm{o}} = \textrm{push}(\langle \textrm{S}[\textrm{D}]; \lbrace \textrm{p} \rbrace \rangle, \textrm{T}						_{\textrm{i}})$
			\ex\label{236c} $\textrm{ps}_{\textrm{o}} = \textrm{ps}_{\textrm{i}} \ \overline\cup \ \lbrace \textrm{p} \rbrace$
			\hfill\hbox {\citet[92]{Farkas2010}}
		\end{xlist}
\end{exe}
Sie nehmen einen Sprechaktoperator A an, dessen Argument ein Satz ist ($\textrm{S}[\textrm{D}]$), der von Autor a in Kontext $\textrm{K}_{\textrm{i}}$ (i = input) geäußert wird und $\textrm{K}_{\textrm{i}}$ dadurch zu einem Kontext $\textrm{K}_{\textrm{o}}$ (o = output) verändert, für den gilt: Das neue Diskursbekenntnissystem ($\textrm{D}_{(\textrm{a,o})}$) entspricht dem Diskursbekenntnissystem des Vorzustandes ($\textrm{D}_{(\textrm{a,i})}$) plus Aufnahme der mit der Assertion ausgedrückten Proposition ($\textrm{DC}_{\textrm{a,i}} \cup \lbrace \textrm{p} \rbrace$). Die Assertion ($\textrm{S}[\textrm{D}]; \textrm{p}$) wird auf den alten Tisch ($\textrm{T}_{\textrm{i}}$) oben aufgelegt (durch die Operation \textit{push}), so dass sie auf dem neuen Tisch ($\textrm{T}_{\textrm{o}}$) oben auf dem Stapel liegt. Jede Teilmenge aus ps ($\textrm{ps}_{\textrm{i}}$) wird mit p vereint, wobei die inkonsistenten neu entstehenden Teilmengen entfernt werden. Die projizierte Zukunft des cg beinhaltet folglich p unter Bewahrung der Konsistenz eines jeden denkbaren cg.\\

\noindent
Im nächsten Abschnitt wird der von mir verfolgte Zugang zur Modellierung des MP-Beitrags erläutert. Hierbei handelt es sich um eine Auffassung, die von Diewald in verschiedenen Arbeiten vertreten wird (vgl. \citealt{Diewald1997, Diewald2006, Diewald2007}, \citealt{Diewald1998}, \citealt{Diewald2010}), und die ich in den Kapiteln~\ref{chapter:jud}, \ref{chapter:hue} und \ref{chapter:dua} in das Diskursmodell von \citet{Farkas2010} integrieren werde. In Abschnitt~\ref{sec:vorteile} werde ich aufzeigen, warum es sich anbietet, Diewalds Konzeption der MP-Bedeutung und Farkas \& Bruce'  Modellierung von Diskursbeiträgen zusammenzuführen und warum genau diese beiden Zugänge (aus der großen Menge prinzipiell zur Verfügung stehender Ansätze) als geeignet erscheinen.

\section{Der Modalpartikel-Zugang}
\label{sec:zugang}
Da meine Ableitung der Abfolgebeschränkungen von MPn in Kombinationen auf der Interpretation der Kombinationen basieren soll, ist die Festlegung auf eine Beschreibung der Einzelbedeutungen der MPn notwendig. Ich lege eine Auffassung zur Modellierung des Beitrags der MPn zugrunde, wie sie von Diewald in einer Reihe von (kooperativen) Arbeiten vertreten wird (vgl. \citealt{Diewald1997, Diewald1999a, Diewald2006, Diewald2007, Diewald1998, Diewald2010}). 

\subsection{Modalpartikeln als genuin grammatische Einheiten}
Die Grundüberlegung von Diewalds Ausführungen ist, dass MPn (wie andere grammatische Kategorien) eine relationale Komponente aufweisen, wie z.B. Tempus, Modus, Konjunktionen oder Pronomen, und dass sie damit gleichermaßen zu den grammatischen Einheiten zu rechnen sind. Das anaphorische Pronomen \is{anaphorisches Pronomen} \textit{er} in (\ref{237}) bezieht sich beispielsweise zurück auf einen Referenzpunkt (den Eigennamen \textit{Ulrich}).
	
\begin{exe}
	\ex\label{237} 
	Ulrich hatte Hunger. Er ging in die Küche und machte sich ein Brot.
\end{exe}	
In (\ref{238}) verortet die verbale Kategorie des Tempus \is{Tempus} das beschriebene Geschehnis relativ zur Äußerungszeit. 

\begin{exe}
	\ex\label{238} 
	Zu Ostern besuchte sie die Eltern.
\end{exe}
Und in (\ref{239}) verweist die Konjunktion \is{Konjunktion} \textit{aber} zurück auf den Vorgängersatz und bezieht ihn auf den folgenden Nebensatz.

\begin{exe}
	\ex\label{239} 
	Sie wollte länger bleiben, aber sie hatte keine Zeit.
\end{exe}
Für alle hier angeführten zweifelsohne grammatischen Zeichen lässt sich folglich relationales Potential nachweisen. Sie verbinden sprachliche Einheiten unter\-einander (wie im Falle des anaphorischen Pronomens oder der Konjunktion) oder beziehen sprachliche Einheiten auf non-sprachliche Entitäten (wie im Falle von Tempus und auch Modus).

Diese \textit{relationale} oder auch \textit{indexikalische} Komponente grammatischer Zeichen fasst Diewald wie in (\ref{240}) zusammen.

\begin{exe}
	\ex\label{240} 
	point of reference $\leftarrow$ (grammatical sign \& unit modified by grammatical sign)
	\hfill\hbox{\citet[415]{Diewald2006}}
\end{exe}
(\ref{241}) bis (\ref{243}) formulieren ihre konkreten Ausbuchstabierungen für die drei Fälle aus (\ref{237}) bis (\ref{239}). 
	
\begin{exe}
	\ex\label{241} 
	preceding noun phrase $\leftarrow$ (pronoun \& syntactic function)
\end{exe}	
\vspace{-0.65cm}
\begin{exe}
	\ex\label{242} 
	utterance time $\leftarrow$ (tense marker \& proposition)
\end{exe}		
\vspace{-0.65cm}		
\begin{exe}
	\ex\label{243} 
	proposition 1 $\leftarrow$ (conjunction \& proposition 2)
	\hfill\hbox {\citet[415]{Diewald2006}}
\end{exe}															
Das grammatische Element (Pronomen, Tempus, Konjunktion) bezieht jeweils die Einheit, die es modifiziert (syntaktische Funktion, Proposition, weitere Proposition) auf ein anderes Element (vorweggehende NP, Äußerungszeit, andere Pro\-position).

Da Diewald davon ausgehen möchte, dass MPn gleichermaßen wie Konjunktionen, Pronomen und Tempus zu grammatischen Elementen zu rechnen sind und sie als Kerncharakteristik dieser ihre relationale Komponente ansieht, gilt es, analog zu (\ref{237}) bis (\ref{239}) die modifizierte Einheit und die Einheit, auf die diese bezogen wird, auszumachen. Weder verorten MPn die Propositionen, auf die sie sich beziehen, in Bezug auf die Sprecherorigo (wie z.B. das Tempus) noch beziehen sie sich auf explizit versprachlichtes Material (wie Konjunktionen und Pronomen es tun). Sie geht deshalb davon aus, dass es sich hierbei um eine \textit{pragmatisch gegebene Einheit} handelt. Diese Annahme lässt sich nachvollziehen, vergleicht man die Sätze in (\ref{244}) und (\ref{245}) miteinander. Während (\ref{244}) nicht in einen bestimmten Kontext eingepasst ist, d.h. nicht Bezug nimmt auf andere sprachliche Einheiten (und sich in diesem Sinne zum Kontext neutral verhält), führt das Auftreten der MP \textit{eben} dazu, dass die Proposition p = \textit{dass Deutsch schwer ist} als kontextuell gegeben verstanden wird. Auf diese bereits vorgegebene Einheit nimmt der Sprecher mit der aktuellen Äußerung in (\ref{245}) Bezug (vgl. \citealt[416]{Diewald2006}).

\begin{exe}
	\ex\label{244} 
	Deutsch ist schwer.
\end{exe}
\vspace{-0.65cm}
\begin{exe}
	\ex\label{245} 
	Deutsch ist \textbf{eben} schwer.
\end{exe}
Diese relationale Funktion, auf eine pragmatisch, also nicht explizit gegebene Proposition zu verweisen, sieht Diewald als die Kernfunktion aller MPn an. Da es aber natürlich Bedeutungsunterschiede zwischen verschiedenen MPn gibt, geht die Autorin zusätzlich von einer je nach MP spezifischen Lexembedeutung aus, die die Art der Relation genauer charakterisiert. 
	
Für \textit{eben} ist das nach Diewald beispielsweise eine iterative Relation, d.h. die ausgedrückte Proposition stimmt mit einer Annahme überein, die der Sprecher bereits zu einem früheren Zeitpunkt vertreten hat. Im Falle von \textit{aber} (vgl. (\ref{246})) besteht die konkrete Lexembedeutung darin, dass der Sprecher der gegenteiligen Annahme des pragmatisch Gegebenen ist. Die Relation ist eine adversative.

\begin{exe}
	\ex\label{246} 
	Deutsch ist \textbf{aber} schwer.
\end{exe}
Durch den Verweis auf die pragmatische Gegebenheit derjenigen Proposition, die der in der aktuellen Äußerung ausgedrückten Proposition identisch ist, bezieht sich die MP stets zurück. Sie markiert die aktuelle Äußerung als non-initial, reaktiv. Die Proposition, auf die eine MP verweist, ist dabei nicht notwendigerweise explizit gegeben. \citet[417]{Diewald2006} geht davon aus, dass die Proposition, die in der MP-Äußerung enthalten ist und als im Kontext gegeben markiert wird, in der MP-Äußerung typischerweise tatsächlich erstmalig erwähnt wird. In diesem Sinne sind MPn nicht textverknüpfend wie Konjunktionen, die eine Verbindung zwischen zwei wirklich benannten und dazu unterschiedlichen Propositionen herstellen. In Analogie zu (\ref{241}) bis (\ref{243}) formuliert Diewald (\ref{247}).

\begin{exe}
	\ex\label{247} 
	pragmatically given unit $\leftarrow$ (MP \& utterance in the scope of the MP)
	\newline
	\hbox{}\hfill\hbox{\citet[417]{Diewald2006}}
\end{exe}
Das folgende Zitat aus \citet[130]{Diewald2007} fasst die Auffassung der Autorin sehr klar zusammen: 		
\begin{quotation}			
Durch diesen Verweis auf eine pragmatisch präsupponierte Einheit erscheint die partikelhaltige Äußerung als zweiter, d.h. reaktiver Gesprächszug in einer unterstellten dialogischen Sequenz. Dies muss nicht der tatsächlichen Situation entsprechen. Ganz im Gegenteil: durch das Setzen einer Abtönungspartikel kann der Sprecher einen nicht-initialen Zug simulieren und damit die unterschiedlichsten pragmatischen Wirkungen erzielen (höflicher, stärker partnerorientiert, unsicher, ungeduldig usw.).
\end{quotation}
Dass die Einheit, auf die die MP-Äußerung Bezug nimmt, bei Diewald nicht die ausgedrückte Proposition ist, sondern sie abstrakter von der \glqq pragmatisch gegebenen Einheit\grqq{} spricht, ist darauf zurückzuführen, dass es sich bei der gegebenen Einheit auch um eine Proposition + Sprechakt handeln kann (s.u.), d.h. das Bezugselement kann größer sein als eine Proposition – wenngleich es sich oftmals allein um die in der MP-Äußerung ausgedrückte Proposition handelt.

\subsection{Ein dreistufiges Beschreibungsmodell}
Um ihre Überlegungen zur relationalen Bedeutung von MPn zu operationalisieren, nimmt die Autorin ein dreistufiges Modell an. (\ref{248}) zeigt das Basisschema.

\begin{exe}
	\ex\label{248} Grundschema zur Bedeutungsbeschreibung der MPn\\[-0.5em]
     \begin{tabular}[t]{|l|p{7cm}|}
     	\hline
      	pragmatischer Prätext & im Raum steht: Proposition/ggf. Sprechakttyp\\
        \hline
        relevante Situation & Sprecherbewertung bezüglich der im Raum stehenden Proposition/ggf. Sprechakttyp\\
        \hline
        $\rightarrow$ Äußerung & Modifizierte Proposition mit Partikel\\
        \hline
     \end{tabular}\\
     \hbox{}\hfill\hbox{\citet[84]{Diewald1998}, ergänzt um \citet[418]{Diewald2006}}
\end{exe}
Der \textit{pragmatische Prätext} \is{pragmatischer Prätext} entspricht der \textit{pragmatisch gegebenen Einheit} \is{pragmatisch gegebene Einheit} aus\\ \citet{Diewald2006} wie oben erläutert. Die \textit{relevante Situation} \is{relevante Situation} steht für den Anlass der MP-Äußerung des Sprechers. Er beabsichtigt eine Bewertung der pragmatisch gegebenen Einheit, was zur MP-Äußerung führt.

Wie oben erläutert, nimmt die Autorin an, dass die relationale/indexikalische Komponente jeder MP zukommt, dass jede MP aber weiterhin einen lexemspe\-zifischen Beitrag einbringt. Für eine Darstellung entlang von (\ref{248}) bedeutet dies, dass das Grundschema hinsichtlich der je nach MP anzunehmenden konkreten Relationen angepasst wird. Im Folgenden sollen exemplarisch verschiedene Füllungen der Komponenten illustriert werden. Dabei wird sich auch zeigen, wie das Modell mit verschiedenen Bezugsgrößen im pragmatischen Prätext sowie mit unterschiedlichen Sprechakttypen umgeht.

Im Falle der \textit{aber}-Äußerung in (\ref{249}) entspricht die Haltung des Sprechers der ausgedrückten Proposition. Die MP bringt zum Ausdruck, dass der Sprecher mit seiner Position in Bezug auf die (bis auf die Polarität) identische Proposition im Prätext die entgegengesetzte Annahme vertritt. Die MP \textit{aber} markiert nach dieser Interpretation eine adversative Relation.

\begin{exe}
	\ex\label{249} 
	Das ist \textbf{aber} keine gute Konstruktion.
\end{exe}
\vspace{-0.65cm}	
\begin{exe}
	\ex\label{250} Bedeutungsschema der MP \textit{aber}\\[-0.5em]
     \begin{tabular}[t]{|l|p{7cm}|}
     	\hline
      	pragmatischer Prätext & im Raum steht: daß das eine gute Konstruktion ist\\
        \hline
        relevante Situation & ich denke: das ist keine gute Konstruktion\\
        \hline
        $\rightarrow$ Äußerung & Das ist \textbf{aber} keine gute Konstruktion\\
        \hline
     \end{tabular}\\
     \hbox{}\hfill\hbox{\citet[84]{Diewald1998}}
\end{exe}
Der Ausdruck \textit{im Raum steht} ist nach \citet[85]{Diewald1998} eine neutrale Formulierung, um auf die pragmatisch gegebene Einheit zu verweisen, ohne eine Festlegung zu treffen, wer die Quelle dieser Information ist und wie sie in den Kontext gelangt ist. Dies schließt nicht aus, dass sich ggf. konkreter angeben lässt, auf wen diese Angabe zurückgeht.

Anfänglich wurde bereits erwähnt, dass das pragmatisch Gegebene komplexer sein kann als Propositionen. Ein solcher Fall liegt \citet[135]{Diewald2007} zufolge z.B. bei \textit{denn} vor in einer Verwendung wie in (\ref{251}).

\begin{exe}
	\ex\label{251} 
	Zeuge:	Und wir fahren, nicht wahr, und dann mit der Geschwindigkeit achtzig bis hundert.\\
	Richter: Warum fahren Sie \textbf{denn} so schnell?
    \hbox{}\hfill\hbox{\citet[61]{Hoffmann1994}}
\end{exe}
Sie nimmt an, dass das \textit{denn} in der Frage anzeigt, dass diese aus dem Vorgängerbeitrag motiviert ist. Obwohl Fragen in der Regel Dialogsequenzen eröffnen, ist sie hier reaktiv, da an den pragmatischen Prätext angeknüpft. Pragmatisch gegeben ist, dass die Frage \textit{Warum fahren Sie so schnell?} aus dem vorangehenden Beitrag folgt (vgl. (\ref{252})). Die pragmatisch vorgegebene Einheit ist in diesem Fall somit keine Proposition, sondern ein Sprechakt. Die durch \textit{denn} angezeigte Relation ist eine konsekutive.

\begin{exe}
	\ex\label{252} Bedeutungsschema der MP \textit{denn}\\[-0.5em]
     \begin{tabular}[t]{|l|p{7cm}|}
     	\hline
      	pragmatischer Prätext & Aus dem Gesagten folgt: Ich frage: Warum fahren Sie so schnell?\\
        \hline
        relevante Situation & Ich frage: Warum fahren Sie so schnell?\\
        \hline
        $\rightarrow$ Äußerung & Warum fahren Sie \textbf{denn} so schnell?\\
        \hline
     \end{tabular}\\
     \hbox{}\hfill\hbox{\citet[136]{Diewald2007}}
\end{exe}
(\ref{252}) illustriert auch, dass Zeile 2 je nach ausgeführtem Sprechakt variieren kann. Der Ausdruck \textit{ich denke} (vgl. (\ref{250})) zeigt an, dass der Sprecher einen assertiven Sprechakt \is{Assertion} tätigt, die Formulierung \textit{ich frage} verweist in (\ref{252}) entsprechend auf \is{erotetischer Illokutionstyp} einen erotetischen Illokutionstyp.\\

\noindent
Ohne andere Ansätze in allen Belangen ausschließen zu müssen, halte ich Diewalds Zugang als Basis für eine Bearbeitung meiner Fragestellung aus verschiedenen Gründen für geeignet. 

\subsection{Vorteile der Modellierung im Rahmen von Diewalds Zugang und Farkas \& Bruce' Diskursmodell}
\label{sec:vorteile}
Diewalds Modellierung der MP-Bedeutung gehört zu den \textit{bedeutungsminimalisti\-schen} \is{Bedeutungsminimalismus/-maximalismus} Arbeiten (vs. \textit{-maximalistischen}). Eine Frage, die in alten wie neuen Beschreibungen von MPn auftritt, ist, wie konkret oder abstrakt die MP-Bedeutung gefasst werden muss. Beschränkt man sich auf MP-Verwendungen, können die meisten Partikeln in verschiedenen Äußerungstypen auftreten.\footnote{Erweitert man die Betrachtung um gleichlautende Formen anderer Wortarten, verschärft sich der hier beschriebene Aspekt zusätzlich.}\textit{Doch} tritt z.B. in Assertionen, aber auch Direktiven, Fragen, Wünschen, Ausrufen (vgl. (\ref{253})) auf.

\begin{exe}
	\ex\label{253} 
		\begin{xlist}	
			\ex\label{253a} A: Wir sind Mitte September auf den Azoren.\\
							B: Da hast du \textbf{doch} Geburtstag.
			\ex\label{253b} Fahr \textbf{doch} an die Nordsee!
			\ex\label{153c} Wie heißt der Ort \textbf{doch} (noch)?
			\ex\label{253d} Hätte ich \textbf{doch} mehr Ruhe!
			\ex\label{153e} Wie schön es \textbf{doch} am Meer ist!
		\end{xlist}
\end{exe}
Weitere Differenzierungen sind denkbar entlang von Dimensionen wie z.B., ob ein Sprecherwechsel beteiligt ist, eine monologische Verwendung vorliegt, die MP-Äußerung reaktiv oder initial erfolgt. Für eine MP ergibt sich eine gewisse Palette von Auftretenskontexten. Für jeden MP-Zugang stellt sich vor diesem Hintergrund die Frage, als wie selbständig diese Varianten zu behandeln sind. Eine (extreme) bedeutungsmaximalistische Position geht davon aus, dass pro Auftretensweise ein eigenes Lexem mit eigener Bedeutung anzusetzen ist und zwischen den Bedeutungen keine Verbindung besteht. Unter der bedeutungsminimalistischen Perspektive \is{Bedeutungsminimalismus/-maximalismus} wird eine invariante Grundbedeutung angesetzt. Die Variation ist auf andere Faktoren zurückzuführen. Aus theoretischer Sicht ist ein minimalistischer Ansatz sicherlich zu bevorzugen. Eine völlige Unabhängigkeit der Bedeutungsvarianten trifft nicht zu. Wer die Meinung vertritt, es gebe verschiedene Lexeme für jede Verwendung (z.B. $doch_{1}$ bis $doch_{7}$ in \citealt[33-34]{Helbig1981} oder gar 1 bis 42 in \citealt{Volmert1991}) mit je eigener Bedeutung, die nicht in Relation zueinander stehen, muss sich die Frage gefallen lassen, warum sich die Effekte auch überlappen. 

Es ist generell nicht wünschenswert, von vielen verschiedenen Einzelverwendungen auszugehen. Diese Vorstellung ist aber noch weniger wünschenswert, wenn man Kombinationen von MPn betrachtet: Die Verwendungen würden sich noch potenzieren. Eine solche Annahme scheint mir unplausibel und eher störend, außer, es lässt sich feststellen, dass diese Verwendungsweisen für diese Fragestellung relevant sind, weil die Einzelpartikeln in den Kombinationen jeweils nur in ganz bestimmten Verwendungen auftreten können.

Wenngleich ich eine abstrakte Zuschreibung bevorzuge, muss diese aber natürlich dennoch für die Einzelverwendungen aufkommen können. Ich glaube, dass Diewalds Ansatz (bzw. meine in das Diskursmodell integrierte Version) variabel genug ist, dies zu leisten. 

An den obigen Beispielen sieht man z.B. auch, dass die Autorin unterschiedliche Sprechakttypen \is{Sprechakt} auffangen kann, indem verschiedene Sprechereinstellungen \is{Einstellung} angesetzt werden (\textit{ich denke}, \textit{ich frage}, \textit{ich will} (vgl. (\ref{250}), (\ref{252})).\footnote{In meiner Modellierung von Assertionen und Direktiven werden sich diese verschiedenen Sprechereinstellungen in unterschiedlichen Komponenten des Diskursmodells wiederfinden.}

Der pragmatische Prätext \is{pragmatischer Prätext} kann für eine bestimmte Partikel aber gleich bleiben. Ich werde in den Einzelanalysen vor allem ausnutzen, dass undefiniert bleibt, warum etwas z.B. im Raum steht oder warum jemand anderes etwas denkt.

Ein weiterer Pluspunkt von Diewalds Modellierung der MP-Bedeutung ist, dass sie keine Skopusmöglichkeiten von vornherein ausschließt. Interessant ist erst die Betrachtung von konkreten Fällen, aber die Situationen im pragma\-tischen Prätext, die (neben dem Äußerungstyp in der relevanten Situation) die Unterschiede zwischen Partikeln auffangen, können im Prinzip additiv oder geschach\-telt aufeinander bezogen werden. Ich habe in Abschnitt~\ref{sec:skopusexp} gezeigt, dass die Model\-lierungen von Doherty und Thurmair es nicht erlauben, alle denkbaren Skopusrelationen abzubilden.

Anders als Diewald möchte ich mich allerdings nicht festlegen, dass es sich bei der Information im pragmatischen Prätext um eine pragmatische Präsupposition \is{Präsupposition} handelt. Wie in Abschnitt~\ref{sec:skopusexp} anhand des Ansatzes von Ormelius-Sandblom gesehen, stört eine solche Festlegung hinsichtlich der Natur des Bedeutungsbeitrags ggf. nur die Entscheidung über die Interpretation der MP-Kombination. Obwohl dies natürlich eine wichtige Frage ist, möchte ich die Beschreibung mög\-lichst neutral halten, um zu versuchen, unvoreingenommen eine Entscheidung über die geeignetste Bedeutungszuschreibung der MP-Kombinationen zu treffen. 

Diewalds Ansatz erlaubt darüber hinaus auch eine Betrachtung des Phänomens aus Diskurssicht. Der pragmatische Prätext stellt im Grunde den der MP-Äuße\-rung vorangehenden Kontextzustand dar. MPn ist aus dieser Sicht gemein, dass sie Anforderungen an die Kontexte stellen, in denen sie passend geäußert werden können. Diese theoretische Behandlung fängt den reaktiven Aspekt der Partikeln auf. Ich fasse MP-Äußerungen in meinen Einzelanalysen deshalb als \glq rückwärtsgerichtete Kontextwechsel\grq {} auf: Während man für MP-lose Äußerungen vorwärtsgerichtet ihre Auswirkungen auf den Kontext beschreibt, charakte\-risiert man für MP-Äußerungen die vorangehenden Kontextzustände, auf die die folgenden Äußerungen erst reagieren. 

Es gibt zwei Aspekte des in Abschnitt~\ref{sec:diskursmodell} skizzierten Diskursmodells, aufgrund derer es sich m.E. von anderen Modellen (s.o.) unterscheidet\footnote{Dies soll nicht heißen, dass die von \citet{Farkas2010} angenommenen Komponenten nicht unter gleichen/ähnlichen Funktionen und ggf. anderen Bezeichnungen auch in anderen Ansätzen auftreten. Ganz im Gegenteil: Bei \citet{Ginzburg1995a, Ginzburg1995b, Ginzburg1996}, \citet{Roberts1996}, \citet{Buering2003} findet sich mit der \textit{Question Under Discussion} eine dem \textit{Tisch} vergleichbare Komponente. Bei \citet{Hamblin1971}, \citet{Bartels1999}, \citet{Gunlogson2003} und \citet{Caponigro2007} werden neben dem cg ebenfalls individualisierte Systeme der einzelnen Diskursteilnehmer angenommen.}, und es für die Analyse des diskursiven Beitrags von MPn geeignet macht: Zum einen ist dies die Aufsplittung der Diskursbekenntnisse und zum anderen die Annahme kano\-nischer Reaktionen auf \is{Sprechakt} Sprechakte. Der erste Aspekt spiegelt sich in der Annahme der Komponenten $\textrm{DC}_{\textrm{A}}$, $\textrm{D}_{\textrm{CB}}$ und cg wider und erlaubt die Rückverfolgung sowie (im Falle der Uneinigkeit) das Festhalten der diskursiven Beiträge der Beteiligten. Da MPn auf Sprechereinstellungen und -erwartungen sowie den diskursiven Status von Information Bezug nehmen, ist es notwendig, dass in einer Modellierung zwischen Sprecher-, Hörer- und geteiltem Wissen unterschieden werden kann. Der zweite Aspekt bildet die Überlegung ab, dass in Sprechakten gewünschte zukünftige Gegenzüge angelegt sind, die den Zielzustand herbeiführen, das zur Diskussion stehende Thema aufzulösen, damit vom Tisch zu entfernen und möglichst direkt eine cg-Erweiterung zu erreichen. Diese Überlegung im Umgang mit Sprechakten ist für die Modellierung von MP-Bedeu\-tung sehr wichtig, da MPn Äußerungstypen auf die Art modifizieren können, dass sie bestimmte spezifischere Versionen dieser Typen herbeiführen. Es wird sich zeigen, dass MPn auf genau diese gewünschten Zielzustände verweisen können. Auch die Unterscheidung von Standardtypen eines Äußerungstyps und weniger prototy\-pischen Varianten desselben wird sich als relevant herausstellen. Ein weiterer Vorteil der beiden unabhängig entwickelten Zugänge ist, dass sich konkrete Komponenten m.E. leicht aufeinander beziehen lassen. Dies gilt für Diewalds etwas unspezifisches \textit{im Raum stehen} und den \textit{Tisch} sowie ihr \textit{ich denke/\\jemand denkt} und die Diskursbekenntnisse von Sprecher und Hörer.\\

\noindent 
Da ich anhand der (dis)präferierten Abfolgen von \textit{ja} \& \textit{doch}, \textit{halt} \& \textit{eben} sowie \textit{doch} \& \textit{auch} dafür argumentieren werde, dass es eine Korrespondenz zwischen der Form und der (Diskurs-)funktion der MP-Sequenzen gibt, möchte ich im letz\-ten Abschnitt dieses Kapitels eine Einordnung dieses Zusammenhangs vor dem Hintergrund des Konzepts der \textit{Ikoni\-zität} \is{Ikonizität} vornehmen.

\section{Ikonizität}
\label{sec:ikonizität}
\textit{Ikonizität} ist ein Konzept, mit dem sich verschiedene Disziplinen beschäftigen. Linguistisch berührt es systematische Aspekte auf allen Beschreibungsebenen. Genauso stellen sich aber auch Untersuchungsfragen in angewandte(re)n Gebie\-ten der linguistischen Forschung, wie in \is{Typologie} der Typologie, in \is{Schriftsystem} Schriftsystemen, bei \is{Sprachwandel} Sprachwandel, \is{Sprachevolution} -evolution, \is{Spracherwerb} -erwerb, -störungen \is{Sprachstörung} oder der \is{Pidgin-Sprache} Entstehung von Pidgin-Sprachen. Das Phänomen wird ebenfalls bearbeitet in der Gestenforschung, Psychologie, Philosophie und Literaturwissenschaft. Blickt man über das sprachliche Zeichen hinaus, eröffnen sich noch weitere Bereiche, in denen Ikonizität eine Rolle spielt, wie z.B. Kunst, Film, Musik, Synästhesie und Marketing. 

Meine Absicht ist es an dieser Stelle, einzuordnen, wo die Art von Ikonizität, für die ich in dieser Arbeit argumentiere, im Verhältnis zu anderen Typen der Ikonizität steht. 

\citet[129]{Croft1995} definiert Ikonizität in der Sprache wie folgt:
\begin{quotation} 
[...] the principle that the structure of language should, as closely as possible, reflect the structure of experience, that is, the structure of what is being expressed by language.
\end{quotation}
Es besteht dieser Auffassung nach somit eine Ähnlichkeitsbeziehung zwischen Sprache und Konzepten. Die Beschaffenheit sprachlicher Zeichen bzw. Strukturen ist extern motiviert. Der Zusammenhang zwischen Form und Funktion wird nicht als arbiträr \is{arbiträr} angesehen.

Eine grundsätzliche Unterscheidung, die in der Ikonizitätsliteratur getroffen wird und die auf \citet[2.277]{Peirce1960})) zurückgeht, ist die zwischen \textit{bildhafter Ikonizität} \is{bildhafte Ikonizität} und \is{diagrammatische Ikonizität} \textit{diagrammatischer Ikonizität}. Erstere spielt sich auf der Ebene eines einzelnen sprachlichen Zeichens ab, für das gilt, dass zwischen seiner Form und seinem Inhalt eine Ähnlichkeit besteht. B-Ikonizität wird klassischerweise durch lautsymbolische Phänomene \is{Lautsymbolik} wie Onomatopoetika \is{Onomatopoesie} (\textit{klatsch}, \textit{boing}, \textit{kikeriki}), Pho\-naesteme \is{Phonaestem} (im Deutschen z.B. die Assoziation von /gl/ \& Leuchten $[$\textit{glänzen}, \textit{glühen}, \textit{glimmen}$]$) oder Vokal-/Konsonantenqualität (z.B. hintere Vokale, stimmhafte Konsonanten \& große, schwere, runde Dinge vs. Vordervokale, stimmlose Konsonanten \& kleinere, gezackte Dinge (\citealt{Koehler1929}, \citealt{Ramachandran2001})) (vgl. die schematische \is{Signifikat} \is{Signifikant} Darstellung in (\ref{254})).

\begin{exe}
\ex\label{254}
\begin{tabular}[t]{ccccc}
  	Signifikant & & miau & &\\
 	$\updownarrow$ & & $\updownarrow$ & & \scriptsize(direkte Ähnlichkeit in Laut-/Wort-Form)\\
 	Signifikat & & \multicolumn{3}{l}{\glq von Katze hervorgerufener Laut\grq {}}\\
\end{tabular}\\
\hbox{}\hfill\hbox{\citet[xxii]{Fischer1999}}
\end{exe}
Bei der \textit{diagrammatischen Ikonizität} geht es nicht um das isolierte sprachliche Zeichen, sondern um komplexere Strukturen. Es liegt eine Motivation der Relationen zwischen Zeichen vor. Man hat es nicht mit einer direkten (vertikalen) Relation zwischen Signifikat und Signifikant zu tun. Die Verbindung besteht anders zwischen der (horizontalen) Relation auf Ebene des Signifikats und der horizontalen Ebene des Signifikants. Diese Verhältnisse können strukturell (oft syntaktisch, aber auch morphologisch) oder semantisch realisiert sein. In (\ref{255}) entspricht die Reihenfolge der drei Formen im bekannten Cäsar-Zitat der Abfolge der Ereignisse in der realen Welt.

\begin{exe}
\ex\label{255}
\begin{tabular}[t]{ccccccccc}
  	Signifikant & & veni & & $\rightarrow$ & vidi & & $\rightarrow$ & vici\\
  	$\nupdownarrow$ & & & & $\updownarrow$ & & & $\updownarrow$ & \\
  	Signifikat & & \glq Ereignis\grq {} & & $\rightarrow$  & \glq Ereignis\grq {} & & $\rightarrow$ & \glq Ereignis\grq {}\\
  	& & & \multicolumn{5}{c}{(in der realen Welt)} &\\	
\end{tabular}\\
\hbox{}\hfill\hbox{\citet[xxii]{Fischer1999}}
\end{exe}
Im Rahmen meiner Argumentation ist die \textit{strukturelle diagrammatische Ikoni\-zität} \is{strukturell diagrammatische Ikonizität} von Interesse. Auch hier werden – zurückgehend auf \citet{Haiman1980} – zwei Typen unterschieden: \textit{Isomorphie} \is{Isomorphie} und \is{ikonische Motivierung} \textit{ikonische Motivierung}. Der für meine Ausführungen relevante (und dazu in der Syntax generell entscheidende ikonische Zusammenhang) ist der zweite (zum ersten vgl. \citealt[516]{Haiman1980}, \citealt{Schachter1973}, \citealt{Bickel1995}). \citet[516]{Haiman1980} fasst ihn folgendermaßen: \glqq a gramma\-tical structure, like an onomatopoeic word, reflects its meaning directly\grqq{}. Klassischerweise werden unter diesen Typ struktureller diagrammatischer Ikonizität Sequenzierungsbeschrän\-kungen gefasst.

Unter diesen Ordnungen markiert die temporale Abfolge wiederum den Klassiker, neben konsekutiven, kausalen oder finalen Zusammenhängen. So besteht die Tendenz, Sätze im Diskurs entsprechend der zeitlichen Abfolge der beschriebenen Ereignisse anzuordnen, wie in (\ref{256}) bis (\ref{259}), wobei in (\ref{257}) zusätzlich Finalität, in (\ref{258}) Kausalität und in (\ref{259}) ein Bedingungs-Folge-Verhältnis kodiert werden.

\begin{exe}
	\ex\label{256} 
		\begin{xlist}	
			\ex\label{256a} He opened the door, came in, sat and ate.
			\ex\label{256b} *He sat, came in, ate and opened the door.
			\hfill\hbox {\citet[92]{Givon1991}}
		\end{xlist}
\end{exe}

\begin{exe}
	\ex\label{257} 
		\begin{xlist}	
			\ex\label{257a} 
			\gll Zh\={a}ngs\={a}n sh\`{a}ng-l\'{o}u shu\`{i}-ji\`{a}o\\
			{} {$\textrm{VP}_{1}$} {$\textrm{VP}_{2}$}\\
			\ex\label{257b}
			\gll *Zh\={a}ngs\={a}n sh\`{a}ng-l\'{o}u shu\`{i}-ji\`{a}o\\
			{} {$\textrm{VP}_{1}$} {$\textrm{VP}_{2}$}\\
			\glt John went upstairs to sleep.
			\hfill\hbox {\citet[51]{Tai1985}}
		\end{xlist}
\end{exe}

\begin{exe}
	\ex\label{258} 
		\begin{xlist}	
			\ex\label{258a} He shot and killed her.
			\ex\label{258b} *He killed and shot her.
			\hfill\hbox {\citet[92]{Givon1991}}
		\end{xlist}
\end{exe}

\begin{exe}
	\ex\label{259} 
		\begin{xlist}	
			\ex\label{259a} If he comes, we'll do it.
			\ex\label{259b} We'll do it if he comes. (wenig frequent)
			\hfill\hbox {\citet[93]{Givon1991}}
		\end{xlist}
\end{exe}
Zwei weitere Relationen und Strukturen, die unter diesen Typ von Ikonizität gefasst werden können, zeigen (\ref{260}) und (\ref{261a}).

\begin{exe}
	\ex\label{260} 
		\begin{xlist}	
			\ex\label{260a} Berg und Tal/*Tal und Berg
			\ex\label{260b} Hand und Fuß/*Fuß und Hand	
			\hfill\hbox {\citet[140-141]{Plank1979}}
		\end{xlist}
\end{exe}

\begin{exe}
	\ex\label{261a} 
		Nimm dieses Buch und bring es mir!
		\hfill\hbox {nach \citet[161]{Simone1995}}
\end{exe}
Es gibt z.B. \is{irreversible Binomiale} \textit{irreversible Binomiale}, wie in (\ref{260}), bei denen sich die Ordnung der Komponenten aus der Oben-Unten-Konzeptualisierung ergibt. Äußerungen wie in (\ref{261a}) sind nur zu verstehen, wenn die Teilsätze wie die Handlungen, die ausgeführt werden sollen, angeordnet sind.

\textit{Ikonische Motivierung} \is{ikonische Motivierung} ist der \is{strukturell diagrammatische Ikonizität} Typ \textit{strukturell diagrammatischer Ikonizität}, dem ich diskursstrukturell motivierte Ordnungen von MPn unterordnen werde. 

Neben Phänomenen wie den obigen, deren Gemeinsamkeit Sequenzierungen sind, sind hier auch andere Fälle angeführt worden. \citet[191]{Haiman1992} verweist z.B. auch auf \is{Alienation} \textit{Alienation}. Sprachliches Material, das eng zusammen steht, gehört auch konzeptuell eng zusammen. Andersherum gehört Material, das durch andere Elemente getrennt wird, auch konzeptuell weniger eng zusammen. 

Ein weiterer Typ ikonischer Motivierung \is{The unexpected} (\textit{The unexpected}) (vgl. \citealt[194]{Haiman1992}) fängt den Zusammenhang auf, dass markierter Inhalt auch mit markierten Formen \is{Markiertheit} einhergeht. Je vorhersagbarer ein Referent ist, umso weniger Aufwand wird für seine Realisierung aufgebracht.\footnote{Vgl. auch die Prinzipien in \citet{Givon1991}, die sich z.T. zuordnen lassen: \textit{The quantity principle} (S. 87), \textit{The proximity principle} (S. 89), \textit{Semantic principle of linear order} (S. 92), \textit{Pragmatic principle of linear order} (S. 93).}

(\ref{261}) zeigt die Ikonizitätstypen im Überblick.

\begin{exe}
\ex\label{261}
\begin{jtree}[xunit=2, yunit=1]
\! = {Ikonizität}
       :{Bildhafte Ikonizität}() [scaleby=2 1]{Diagrammatische Ikonizität}
       :{strukturell}!a {semantisch}.
\!a = : {Isomorphie} {ikonische Motivierung}.
\end{jtree}
\end{exe}
\hfill{\scriptsize(\textbf{Sequenzierung}, Alienation, Unerwartetheit (z.B.))}\\

\noindent
Es versteht sich von selbst, dass derartige Annahmen über ikonische Zusammenhänge in funktionalen Schulen und Modellen verankert sind, da sprachliche Strukturen dort unter Bezug auf Gebrauchsbedingungen motiviert werden. Systemextern verankerte Kriterien wie Erfahrungen und Erwartungen werden als Einfluss nehmend angesehen, so dass die non-linguistische Realität letztlich I\-mitation in linguistischer Struktur erfährt. Die angeführten Beispiele zeigen auch, dass in der Literatur sowohl Strukturen aus der Sicht von Ikonizität behandelt worden sind, bei denen man es mit deutlichen Akzeptabilitätsunterschieden zu tun hat, als auch Sätze, Äußerungen und Ausdrücke, für die \glq nur\grq {} ein weniger frequentes Auftreten bei völliger Akzeptabilität anzunehmen ist. Dieser Aspekt ist wichtig, da für zwei der von mir untersuchten MP-Kombinationen (\textit{ja doch} – \textit{doch ja}, \textit{doch auch} – \textit{auch doch}) gilt, dass zwischen den beiden Abfolgen ein deutlicher Markiertheitsunterschied \is{Markiertheitsunterschied} besteht, der im Falle der Reihungen von \textit{halt} und \textit{eben} nicht derart ausgeprägt ist. Die von mir untersuchten Strukturen fügen sich in dieser Hinsicht folglich in das Gesamtbild der Ikonizitätsforschung. Insbesondere ist es nicht der Fall, dass aus der Perspektive dieser zugrundeliegend funktionalen Sicht nur leichte Markiertheits- oder Frequenzunterschiede behandelt worden sind.

Die Annahme um ikonische Zusammenhänge ist natürlich nicht kritiklos (vgl. z.B. \citealt{Haiman1983}, \citealt{Haspelmath2008a, Haspelmath2008b}). Der Typ von Ikonizität, auf den ich mich in meiner Argumentation beziehe (strukturell diagrammatische Ikonizität \is{strukturell diagrammatische Ikonizität}, genauer ikonische Motivierung \is{ikonische Motivierung}), darf hier allerdings als etabliert gelten. Er zählt auch nicht zu den von \citet{Haspelmath2008a} angezweifelten Fällen. Im Gegenteil, diese Form der Ikonizität (wenn auch in allgemeinerer Form als von mir konkret vertre\-ten) fällt unter die sprachlichen Universalien \is{sprachliche Universalien} bei \citet[103]{Greenberg1963}: \glqq the order of elements in language parallels that in physical experience or the order of know\-ledge\grqq{}. Ich werde in den Kapiteln~\ref{chapter:jud}, \ref{chapter:hue} und \ref{chapter:dua} jeweils ausführen, inwiefern die markierte und unmarkierte Abfolge der behandelten MPn in Kombination für meine Begriffe diskursstrukturelle Verhältnisse spiegelt, von denen unabhängig auszugehen ist.  
 
Gegenstand von Kapitel~\ref{chapter:jud} sind zunächst Sequenzen aus \textit{ja} und \textit{doch}.









