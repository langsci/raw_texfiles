\chapter{Kombinationen aus \textit{doch} und \textit{auch}}
\label{chapter:dua} 
\section{Die Präferenz für \textit{doch auch} - Annahmen aus der Literatur und Korpusfrequenzen}
\label{sec:präferenz}
Für die Kombination von \textit{doch} und \textit{auch} ist zunächst einmal festzuhalten, dass die Abfolge \textit{doch auch} gegenüber der Reihung \textit{auch doch} klar die bevorzugte ist. (\ref{824}) bis (\ref{827}) zeigt einige Beispiele.

\begin{exe}
	\ex\label{824} 
	\scriptsize
	B: \glqq Sie wissen dass sie mir meinen Job nicht gerade leicht machen?\grqq{}\\
	A: \glqq \textbf{Na sie müssen sich ihr Geld \underline{doch auch} verdienen Lucius!} Wenn sie mich dann entschuldigen würden, 	ich muss noch einige Einkäufe tätigen und den organisatorischen Kram erledigen.\grqq{}
	\newline
	\hbox{}\hfill\hbox{(DECOW14AX01)}
	\newline
	\hbox{}\hfill\hbox{(http://www.tabletopwelt.de/index.php?/topic/92424-40k-rpg-20/)}
\end{exe}

\begin{exe}
	\ex\label{825} 
	\scriptsize
	Also das was Guido Knopp macht, sind bestimmt keine 100 prozentigen wissenschaftlich/historisch korrekten Dokumentationen. \textbf{Will er \underline{doch auch} gar nicht.} Sowas kann man auf Arte sehen. 
	\newline
	\hbox{}\hfill\hbox{(deWac: 1624)}
\end{exe}
	
\begin{exe}
	\ex\label{826} 
	\scriptsize
    \begin{tabular}[t]{ll}
	0956 BS	& $[$ähm$]$ schule hier her isst nur dann geht er zur therapie dann zum judo und um sechs\\
	{} & uhr kommt er heim und da muss er noch hausaufgaben machen\\
	{} & $^{o}$h hat er (.)$[$s so gesagt \emph{dass es ihm zu viel is}$]$\\
	0957 HM & $[$hm\_hm$]$\\
	0958 SZ & \hspace{1cm}\textbf{$[$(is \underline{doch auch}) (.) ä programm$]$}\\
	0959 NG & $[$hm$]$ \_hm\\
	0960 BS & und da hab ich gsagt	
	\hfill\hbox{(FOLK\_E\_00026\_SE\_01\_T\_01)} 					 
    \end{tabular}   
\end{exe}

\begin{exe}
	\ex\label{827} 
	\scriptsize
    \begin{tabular}[t]{ll}
	0089 S1 & Äh, hat, was haben Sie denn zu Weihnachten für Kuchen gebacken? Und \emph{haben Sie auch}\\
 	& \emph{Mohnklöße gehabt zu Weihnachten?}\\
	0090 S2 & \emph{Ja, selbstverständlich.}\\
	0091 S1	& Ja, erzählen Sie mal, wie Sie die\\
	0092 S2 & \textbf{Ich bin \underline{doch auch} Schlesier.}			
	\hfill\hbox{(OS--\_E\_00349\_SE\_01\_T\_01; DGD)}	 
    \end{tabular}       
\end{exe}	
Die Präferenz von \textit{doch auch} wird auch in der Literatur vertreten, wenn sich andere Autoren zu dieser Kombination äußern. Anders als bei \textit{ja} \& \textit{doch} und \textit{halt} \& \textit{eben} ist mir hier keine detaillier\-tere Untersuchung bekannt. \textit{Auch doch} gilt in anderen Arbeiten als ungrammatisch oder nicht belegt (vgl. \citealt[227, 230]{Dahl1988}, \citealt[356]{Lemnitzer2001}, \citealt[196]{Kwon2005}). Gleiches lässt sich aus Abfolgelisten (vgl. (\ref{828})) oder Klassenbildungen (vgl. (\ref{829})) ablesen (vgl. auch \citealt[91-94]{Engel1968}, \citealt[42]{Helbig1981}).

\begin{exe}
	\ex\label{828}Aussagemodus\\
	\scriptsize
	ja $>$ denn $>$ eben $>$ halt $>$ doch $>$ eben $>$ halt $>$ wohl $>$ einfach $>$ auch $>$ schon $>$ auch $>$ mal\\
	\newline
	\hbox{}\hfill\hbox{\citet[908, 1542]{Zifonun1997}}
\end{exe}
\vspace{-0.5cm}
\begin{exe}
	\ex\label{829} 
    \begin{tabular}[t]{lll}
	1. Gruppe & $>$ & 4. Gruppe\\
	\textit{doch} & {} & \textit{auch}\\
	(Konjunktionen, & {} & (Fokuspartikeln)\\
	Diskurspartikeln) & &			 
    \end{tabular}   
    \newline
	\hbox{}\hfill\hbox{\citet[31]{Thurmair1991}}   
\end{exe}
	
Modelle, die über eine reine Benennung der Partikelabfolgen hinausgehen, beabsichtigen ebenfalls, die Reihung \textit{doch auch} zu erfassen. 

Thurmair (1989: 221-222) erwähnt die Kombination nur im Zuge der Betrachtung der Kombinationen mit \textit{doch}, nicht unter den Kombinationen mit \textit{auch}. Sie argumentiert, dass ihre Hypothese 2 (vgl. Kapitel~\ref{chapter:hintergrund}, Abschnitt~\ref{sec:katalog} zu einer detaillier\-teren Darstellung von Thurmairs Ansatz) hier greife (vgl. \citeyear[288]{Thurmair1989}). Diese lautet, dass MPn, die Bezug auf die momentane Äußerung nehmen, vor MPn stehen, die eine qualitative Bewertung des Vorgängerbeitrags vornehmen. \textit{Doch} beschreibt sie durch die Merkmale BEKANNT$_{\textrm{H}}$, KORREKTUR, d.h. die Proposition ist dem Hörer aus Sicht des Sprechers bekannt und drückt eine Aufforderung aus, seine Ansichten zu ändern. \textit{Auch} zeigt an, dass die Vorgängeräuße\-rung aus Sprechersicht erwartet war (KONNEX, ERWARTET$_{\textrm{V/S}}$). Es liegt folglich sowohl der Bezug auf die aktuelle Äußerung vor (durch \textit{doch}) als auch auf die Vorgängeräußerung (durch \textit{auch}), die qualitativ bewertet wird. 

Wie in Kapitel~\ref{chapter:hintergrund}, Abschnitt~\ref{sec:ri} ausgeführt, vertritt \citet{Rinas2007}, dass MPn Skopus übereinander nehmen und die Abfolge die Skopusverhältnisse spiegelt. Unter Bezug auf das Beispiel in (\ref{830}) sieht er die Interpretation einer \textit{doch auch}-Assertion durch die Paraphrase in (\ref{831}) erfasst (vgl. auch (\ref{832})).

\begin{exe}
	\ex\label{830}
	Ein Vater beklagt sich, daß seine Tochter so frech und unverschämt ist. Die Großmutter: Was regst du dich denn auf? \textbf{Du hast ihr \underline{doch auch} immer alles durchgehen lassen.}
	\hfill\hbox{\citet[221]{Thurmair1989}}
\end{exe}

\begin{exe}
	\ex\label{831}
	\glq Die Einschätzung, dass der Sachverhalt q nicht überraschend ist angesichts von p, steht im Widerspruch zu einer Auffassung r und besagte 				Einschätzung oder die Auffassung r ist dem Hörer bekannt.\grq {}
\end{exe}

\begin{exe}
	\ex\label{832}
	DOCH(AUCH(p) $>>$ NICHT-ÜBERRASCHEND(q) WEIL(p))\\
	$>>$ WIDERSPRICHT((AUCH(p) $>>$ NICHT-ÜBERRASCHEND(q)\\
	 WEIL(p)), r) \& (KENNT(H, (AUCH(p) $>>$ NICHT-ÜBERRASCHEND(q)\\ WEIL(p))) $\vee$ KENNT(H,r))
\end{exe}	
Des besseren Verständnisses wegen führen (\ref{833}) und (\ref{834}) die Modellierung der Einzelpartikeln nach \citet{Rinas2007} an.
\begin{exe}
	\ex\label{833}
	DOCH(p) $>>$ WIDERSPRICHT(p, q) \& (KENNT(H, p) $\vee$ KENNT(H, q))
\end{exe}	
\vspace{-0.65cm}	
\begin{exe}
	\ex\label{834}
	AUCH(p) $>>$ NICHT-ÜBERRASCHEND(q) WEIL(p)
	\hfill\hbox{\citet[134/136]{Rinas2007}}	
\end{exe}
Die Propositionen aus (\ref{832}) entsprechen in (\ref{830}) p = der Vater hat das Verhalten der Tochter immer durchgehen lassen, q = die Tochter ist frech und unverschämt, r = der Vater ist über das Verhalten seiner Tochter empört. Die Interpretation von (\ref{830}) ist dann: Die Großmutter drückt aus, dass es nicht überraschend ist, dass die Tochter frech ist, angesichts der Tatsache, dass der Vater ihr Verhalten hat durchgehen lassen. Diese Bewertung steht im Widerspruch zur Einschätzung des Vaters, dass das Verhalten der Tochter empörend ist. Dem Vater ist entweder -- trivialerweise -- seine eigene Empörung bekannt oder der Inhalt der Bewertung der Großmutter.

Über die umgekehrte Abfolge schreibt \citet[149]{Rinas2007}:
\begin{quotation}
Angesichts der Skopus-Verhältnisse in dieser Kombination sollte eine Um\-kehrung der APn-Abfolge nicht möglich sein. Dies ist zutreffend: *Du hast ihr \textbf{auch doch} immer alles durchgehen lassen.
\end{quotation}
Und schließlich gibt es einige wenige Arbeiten, in denen authentische Daten angeführt werden. \citet[233]{Rath1975} findet unter 180 \textit{doch}-Äußerungen zwei Mal \textit{doch auch} und ein Mal \textit{doch aber auch}. \citet[53]{Rudolph1983} listet basierend auf einer Korpusuntersuchung die Mehrfachkombinationen in (\ref{835}) mit je einem Beleg.
			
\begin{exe}
	\ex\label{835}
	\textit{doch aber auch}, \textit{doch auch ganz}, \textit{doch auch nur}, \textit{doch eben auch}, \textit{eben doch auch}, \textit{ja doch auch}
\end{exe}					
\citet[254]{Hentschel1986} macht in ihren Daten ein Mal \textit{doch auch} und ein Mal \textit{auch doch} aus. In den gesprochenen Daten von \citet{Moellering2004} lässt sich bei allen von ihr hinsichtlich der MP-Funktion bereinigten \textit{doch}-Belegen ein \textit{doch auch}-Treffer finden (\citeyear[256]{Moellering2004}). Ihre \textit{auch}-Daten (\citeyear[450]{Moellering2004}) kann man nicht einsehen, weil sie nicht alle Belege im Kontext disambiguiert hat, um den MP-Gebrauch auszusondern. Die Korpusuntersuchung in \citet{Braber2010} fördert zwei \textit{doch auch} und einen \textit{auch doch}-Treffer zu Tage.

Die Annahme in allen deskriptiven oder theoretischen Arbeiten ist, dass die grammatische Abfolge \textit{doch auch} ist. Für mich bedeutet dies nicht, dass die E\-xistenz von \textit{auch doch} abzusprechen ist und nicht betrachtet werden sollte. Eine Frage, die es aber in jedem Fall zu klären gilt, ist, warum das \textit{doch} dem \textit{auch} präferiert vorangeht. Die Arbeiten, die Korpusdaten benutzt haben, erlauben keine Aussage zur (Nicht-)Existenz der umgekehrten Abfolge. Es werden zwar nur zwei \textit{auch doch}-Belege angeführt, man sieht aber, dass die Belegzahlen auch für \textit{doch auch} sehr niedrig sind.

Eine Problematik, die die Beschäftigung mit MPn zwar generell begleitet, die im Falle der MP-Kombination aus \textit{doch} und \textit{auch} aber verschärft auftritt, ist, dass es nötig ist, sich die MP-Äußerungen im Kontext anzuschauen, um das Risiko zu minimieren, es mit einer gleichlautenden Form einer anderen Wortart zu tun zu haben. Beide Partikeln sind hier für diese Fehlinterpretation anfällig. Die Angaben aus den angeführten Studien sind auch aus diesem Grund schwierig zu bewerten. In \citet{Braber2010} werden z.B. gar keine Belege angeführt. \citet{Rudolph1983} präsentiert einige Beispiele im Kontext, bei \citet{Moellering2004} ist der Kontext sehr knapp gehalten. Aus meinen eigenen Datenuntersuchungen (s.u.) weiß ich, wie heikel die Entscheidung sein kann, insbesondere, ob \textit{auch} als MP (oder nicht als Gradpartikel \is{Gradpartikel} oder \is{Konjunktionaladverb} Konjunktionaladverb) vorliegt. Dass die Betrachtung großer Datenmengen hier bisher ausgeblieben ist ($[$s.o.$]$ M{öllering passt beispielsweise aufgrund dieses Aspektes in ihrer \textit{auch}-Untersuchung), verwundert nicht, da ein entsprechender Aufwand zur Beantwortung dieser Frage erforderlich ist. Wenngleich sicherlich davon auszugehen ist, dass \textit{doch auch} deutlich überwiegen wird, bin ich dieser Frage in DeReKo und DGD2 nachgegangen, um herauszufinden, in welchem Verhältnis \textit{doch auch} und \textit{auch doch} in großen Datenmengen stehen. Da selbst in einem Teilkorpus von DECOW zu viele Treffer ausgegeben werden, um zu jedem Beispiel den Kontext zu suchen (anders als in DeReKo und DGD2 ist er nicht Teil der extrahierten Daten) und zu entscheiden, ob sowohl \textit{doch} als auch \textit{auch} als MP auftreten, kann ich hier nur eine sehr grobe Schätzung angeben. Es ist davon auszugehen, dass die Anzahl de facto niedriger ist. Höher ist sie in keinem Fall. Die Angabe ist unpräzise, vermittelt aber einen Eindruck der Größenordnung. Findet man die umgekehrte Abfolge, ermöglicht sich so auch ein Vergleich zum Verhältnis von \textit{halt eben} und \textit{eben halt}, bei dem die markierte Reihung gut belegt ist.

Da jeder Beleg im Kontext angeschaut werden muss, habe ich mir für die Daten in DeReKo eine Möglichkeit der Hochrechnung der Ergebnisse einer Stichprobe zu Nutze gemacht und \glq nur\grq {} ein Teilkorpus von DECOW betrachtet. Obwohl die Entscheidung für jeden Beleg einzeln vorgenommen wurde, gilt dennoch die Einschränkung, dass es sich sicherlich nicht um absolute Urteile handelt. Das Kriterium, das ich angesetzt habe, ist, ob \textit{doch} und \textit{auch} als MP interpretiert werden \underline{können}. Ich teste, ob die Partikeln jeweils alleine als MPn gebraucht werden \underline{könnten}, nicht ob sie es \underline{müssen} (wenn dies überhaupt zu entscheiden ist). Bei der Abfolge \textit{doch auch} gilt es, auszuschließen, dass \textit{auch} als Adverb \is{Adverb} engen Skopus nimmt; bei der Abfolge \textit{auch doch} dass \textit{doch} betont auftritt. Ein völliger Ausschluss des Vorkommens einer der \glq Dubletten\grq {} ist in manchen Äußerungen allerdings nahezu unmöglich. (\ref{836}) zeigt die Ergebnisse.

\begin{exe}
	\ex\label{836} Verteilung \textit{doch auch} - \textit{auch doch} in Korpora\\[-1em]
	\begin{tabular}[t]{|c|c|c|}
	\hline
	& \textit{doch auch} & \textit{auch doch}\\
	\hline
	DeReKo & 59 (\scriptsize{auf 500}) & 4\\
	& 6654 ... 11258 \scriptsize{(gesamt: 95\%-Konfidenzintervall}) & -\\
	& 8864 & \\
	\hline
	DGD2 & 60 & 2\\
	\hline
	DECOW14AX01 & 6552 \scriptsize{(Schätzung)} & 8\\
	\hline				 
    \end{tabular}    
\end{exe}	
In (\ref{837}) bis (\ref{842}) führe ich einige Belege an.

\begin{exe}
	\ex\label{837}
	\scriptsize
	Bei den Diskussionen über eine Ausbildungsplatzabgabe nennen Sie immer die Bauindustrie als positives Beispiel und behaupten, dass das dortige 				Umlagesystem wunderbar funktioniere.\\
	(Willi Brase $[$SPD$]$: \textbf{Das funktioniert \underline{doch auch}!})
	\newline
	\hbox{}\hfill\hbox{(PBT/W15.00074 Protokoll der Sitzung des Parlaments Deutscher Bundestag am 12.11.2003)}
\end{exe}

\begin{exe}
	\ex\label{838}
	\scriptsize
	Es wurde natürlich immer von den Heidelbeeren ziemlich äh Marmelade gekocht, die sofort weggegessen wurde aufs Brot, \textbf{weil sie \underline{doch 		auch} hinten und vor nicht gereicht hat}, beziehungsweise immer ge\-spart werden mußte, nicht!	         
	\hfill\hbox{(OS--\_E\_00065\_SE\_01\_T\_01)}
\end{exe}

\begin{exe}
	\ex\label{839}
	\scriptsize
	0350 S2	ich meine s$+$ $+$g$+$ das ist eben ne Schachtel Zigaretten im Monat im höchsten Falle $+$s . ich mein s$+$ die sollte doch wohl jeder über 		haben $+$s . $+$g$+$ \textbf{wird ihm \underline{doch auch} einiges für geboten.}/ und der z$+$ NN $+$z hat vor i$+$ so ne $+$g$+$ diese 					Kaminabende(nich?)	         
	\hfill\hbox{(FR--\_E\_00064\_SE\_01\_T\_01)}
\end{exe}			

\begin{exe}
	\ex\label{840}
	\scriptsize
	0142 S2 ... $\plus$p$\plus$ ( ja und?) $\plus$p$\plus$ was hätte da jetzt in diesem Augenblick anders sein können?.\\
	0143 S1	$\plus$p$\plus$ ... s is schon gut so.\\
	0144 S2	( f$\plus$ nei $\plus$f )( na) was is?. $\plus$g$\plus$ es is doch gar nichts gewesen./ \textbf{es war ja \underline{auch doch}} $\plus$g$			\plus$ war keine gar keine so Absicht i$\plus$ so irgendein $\plus$g$\plus$ durch irgendein Verhalten irgendwas zu bewirken $\plus$i ./ ich mußte 		ja hereinplatzen ,$\plus$ nachdem ich dich zehn Minuten gesucht hab $\plus$p$\plus$ und gar nicht damit gerechnet( doch) zum Schluß damit gerechnet 	hab $\plus$, ,$\plus$ daß du noch oben bist $\plus$, ,$\plus$ weil da wieder s Licht an war $\plus$, ./ aber da hattest du schon die...                              
	\hfill\hbox{(FR--\_E\_00106\_SE\_01\_T\_01)}
\end{exe}											     				

\begin{exe}
	\ex\label{841}
	\scriptsize
	Hallo Atze,\\
	$>$ '/bin/bash': No such file or directory\\
	gibt es die Datei /mnt/hdb2/knx/source/KNOPPIX/bin/bash\\
	Tschuess\\
	Karl\\
	
	\noindent
	Hey Karl,\\
	nee, \textbf{kann es \underline{auch doch} gar nicht geben}, oder?\\
	das ist doch der Inhalt der Knoppix CD den ich da rüberkopieren muss oder?\\
	BZW. oder hab ich hier was falsch verstanden? 
	\hfill\hbox{(DECOW14AX01)}
	\newline
	\hbox{}\hfill\hbox{(http://www.knoppixforum.de/knoppix-forum-deutsch/remastern/}
	\newline
	\hbox{}\hfill\hbox{thread2133/probleme-beim-remastern-von-knoppix-5-0.html)}
\end{exe}

\begin{exe}
	\ex\label{842}
	\scriptsize
	Da mein Lipobestand langsam immer größer wird \textbf{und man \underline{auch doch} immer wieder Horrorgeschich\-ten von Lipobränden hört} wollte ich mal 	fragen ob den Interesse besteht eine Sammelbestellung für Munitionskisten zu organisieren. Ich hab selber schon so eine Box und mir ist es mittlerweile echt wohler wenn meine Akkus da drin liegen. Aber die eine Kiste reicht nicht mehr daher brauch ich Nachschub.						         
	\hfill\hbox{(DECOW14AX01)}
	\newline
	\hbox{}\hfill\hbox{(http://www.modellbauvideos.de/board/wbb/sonstiges/sammelbestellungen}
	\newline
	\hbox{}\hfill\hbox{/2031-aufbewahrungsbox-fuer-lipos-munitionskiste/)}
\end{exe}								
Nach der Bereinigung einer Zufallsauswahl von 500 Treffern bleiben 59:4 Belege übrig. Die Hochrechnung auf die Gesamtdatenmenge (unter Berücksichtigung der Gesamttextwörterzahl im Korpus sowie der für die Anfragen ausgegebenen Gesamttrefferzahl) (zum genauen Vorgehen vgl. \citealt[Kapitel 5.6]{Perkuhn2012}) ergibt, dass mit 95\%iger Wahrscheinlichkeit mit 6654 bis 11258 \textit{doch auch}-Belegen zu rechnen ist. Für \textit{auch doch} lässt sich dieses Intervall nicht berechnen, weil die Bedingung nicht erfüllt wird, dass im Korpus mindestens neun Treffer vorliegen. Die Schätzung entlang des Anteils wäre, dass ca. 7 Treffer vorhanden sind.

Anhand der Verwendung der beiden MP-Kombinationen bestätigt sich folg\-lich (wie erwartet) die sehr klare Präferenz für \textit{doch auch}, die der Intuition entspricht, die aus den geringen Belegzahlen bestehender Arbeiten aber nicht geschlossen werden konnte. Wenngleich die Belegzahlen für \textit{auch doch} im Vergleich sehr gering aussehen und diese Abfolge natürlich unterrepräsentiert ist, ist dies für mich kein Grund, die \textit{auch doch}-Treffer nicht näher zu untersuchen und in ihnen nach Mustern zu suchen. Wählt man eine größere Datenmenge (das DECOW14-Gesamtkorpus), steht eine größere Datensammlung zur Verfügung. Ohne Zweifel ist die Lupe sehr groß. Es steht fest, dass es die zentrale Aufgabe der Analyse ist, die deutliche Präferenz von \textit{doch auch} zu erklären. Dennoch sollte es erlaubt sein und sollte die Analyse in der Lage sein -- sofern sich eine Systematik feststellen lässt -- Gründe zu benennen, warum die Umkehr gerade dort stattfindet. Auch in Kapitel~\ref6{chapter:jud} basiert meine Argumentation, dass man die Abfolge \textit{doch ja} nicht komplett ausschließen sollte, nicht auf Frequenzen, sondern der Tatsache, dass sie auf bestimmte Kontexte beschränkt zu sein scheint.

In den DeReKo-Daten machen die \textit{doch auch}-Treffer nur 0,1\% der Kombinationen aus \textit{doch} und \textit{auch} aus. Im Falle von \textit{halt eben} und \textit{eben halt} (715-117) nimmt das in meiner Argumentation markierte \textit{eben halt} immerhin 14\% der kombinierten Fälle ein. Als Vergleichswert habe ich in (\ref{843}) die Verteilung von \textit{ja doch} und \textit{doch ja} in den DeReKo-Daten kalkuliert. 
\pagebreak
\begin{exe}
	\ex\label{843} Verteilung \textit{ja doch} - \textit{doch ja} in DeReKo\\[-1em]
	\begin{tabular}[t]{|c|c|c|}
	\hline
	& \textit{ja doch} & \textit{doch ja}\\
	\hline
	DeReKo & 135 (\scriptsize{auf 500}) & 8 (auf 337)\\
	& 4049 ... 5715 \scriptsize{(gesamt: 95\%-Konfidenzintervall}) & -\\
	\hline
	& 4813 & \\
	\hline				 
    \end{tabular}    
\end{exe}	
Die \textit{doch ja}-Treffer machen 0,2\% der Kombinationen aus. Die Verteilung von \textit{doch auch} und \textit{auch doch} ähnelt in der Größenordnung somit der Verteilung von \textit{ja doch} und \textit{doch ja}. Wohlgemerkt stammt auch da die größte Menge von Belegen, die Anlass für meine Argumentation gegeben hat, aus Webdaten. 

\section{Distribution von \textit{doch}, \textit{auch} und \textit{doch auch}}
\label{sec:distributionda}
Verfolgt man die Absicht, eine Erklärung für die Präferenz von \textit{doch auch} gegenüber \textit{auch doch} zu finden, ist es entscheidend, zu wissen, welche Satzmodi \is{Satzmodus} bzw. Äuße\-rungstypen \is{Illokutionstyp} erfasst werden können müssen. Wie bereits in Kapitel~\ref{chapter:jud}, Abschnitt~\ref{sec:distributionjd} und Kapitel~\ref{chapter:hue}, Abschnitt~\ref{sec:bedhe} angeführt, können sich die gleichen MPn nicht in allen (Satz)kontexten kombinieren. Die etablierte Annahme, auf die ich mich hier verlasse, ist die Schnittmengenbedingung \is{Schnittmengenbedingung} aus \citet{Thurmair1989, Thurmair1991}: MPn können sich nur in den Satzmodi kombinieren, in denen sie auch in Isolation auftreten können.\\

\noindent
\textit{Doch} weist eine sehr weite Distribution auf (vgl. (\ref{844}) bis (\ref{848})).

\begin{exe}
	\ex\label{844} Deklarativsatz\\
	Die Strecke über HH-Harburg ist \textbf{doch} länger.
\end{exe}

\begin{exe}
	\ex\label{845} E-Interrogativsatz\\
	*Hast du \textbf{doch} am Wochenende Zeit?
\end{exe}	
	
\begin{exe}
	\ex\label{846} w-Interrogativsatz\\
	Wie heißt \textbf{doch} (gleich) der Platz in Nippes, wo täglich Markt ist?
\end{exe}		
	
\begin{exe}
	\ex\label{847} Imperativsatz\\
	Mach \textbf{doch} die Heizung an!
\end{exe}	
		
\begin{exe}
	\ex\label{848} Optativsatz\\
	Hätte ich \textbf{doch} am Gewinnspiel teilgenommen!
\end{exe}		
Die Auftretensmöglichkeiten in Exklamativsätzen unterscheiden sich je nach Typ von Exklamativsatz. Aus Satzexklamativsätzen mit $[\minus$w, V2$]$-Stellung ist \textit{doch} ausgeschlossen (vgl. (\ref{849})).
	
\begin{exe}
	\ex\label{849}Satzexklamativsätze\\[-1.25em]
		\begin{xlist}	
			\ex\label{849a} *DER hat \textbf{doch} einen Bart!
			\ex\label{849b} DER hat \textbf{aber}/\textbf{vielleicht} einen Bart!	
			\hfill\hbox {\citet[218]{Rinas2006}}
		\end{xlist}
\end{exe}	
	
\begin{exe}
	\ex\label{850}\textit{dass}-Exklamativsatz\\
	Daß der mir \textbf{doch} die Vorfahrt nimmt!
	\newline
	\hbox{}\hfill\hbox {\citet[152]{Zaefferer1988}}
\end{exe}
	
\begin{exe}
	\ex\label{851}w-Exklamativsatz\\[-1.25em] 
		\begin{xlist}	
			\ex\label{851a} Was BIST du \textbf{doch} bloß für ein Mensch!
			\ex\label{851b} Wie SCHÖN du \textbf{doch} bist!	
			\hfill\hbox {\citet[218-219]{Rinas2006}}
		\end{xlist}
\end{exe}		
Hat man diese \textit{doch}-Kontexte bestimmt, lässt sich überprüfen, in welcher dieser Umgebungen \textit{auch} ebenfalls auftreten kann. Ist nur eine oder keine Partikel akzeptabel, sollten auch beide nicht kombiniert möglich sein.

An (\ref{852}) sieht man, dass der Deklarativsatz ein zulässiger Kontext ist.

\begin{exe}
	\ex\label{852}
	Die Strecke über HH-Harburg ist \textbf{doch}/\textbf{auch}/\textbf{doch auch} länger.
\end{exe}
Da bei \textit{auch} die Unterscheidung zum Konjunktionaladverb \is{Konjunktionaladverb} und zur Gradpartikel \is{Gradpartikel} schwierig sein kann, benötigt man hier Kontext, um die MP-Lesart (eindeutig(er)) nahezulegen.\footnote{Es ist wichtig, die MPn in geeigneten Kontexten hinsichtlich ihrer strukturellen Auftretensmäglichkeiten zu überprüfen, da auch die ausgedrückten Sachverhalte bereits Möglichkeiten des Gebrauchs mitsteuern. Stellt man fest, dass eine Partikel in einer bestimmten strukturellen Umgebung in einer isolierten Äußerung nicht akzeptabel erscheint, sollte man erst überprüfen, ob ein geeigneter Kontext fehlt. Ich gebe deshalb meist typische Kontexte für beide MPn an, auch wenn sie dann nicht in ein und demselben Satz vorkommen.} (\ref{853}) zeigt einen typischen Verwendungskontext.

\begin{exe}
	\ex\label{853}
	A: (Eine alte Frau ist auf der Straße ausgerutscht und hat sich verletzt.)\\
	B: Es ist \textbf{auch} furchtbar glatt auf der Straße.
	\hfill\hbox {\citet[88]{Helbig1990}}
\end{exe}
Im Rahmen der Untersuchung dieser beiden MPn sind \is{V1-Deklarativsatz} der V1- und \textit{Wo}-VL-Deklara\-tivsatz \is{Wo-VL-Deklarativsatz} von besonderem Interesse (vgl. auch meine gesonderte Betrachtung in Abschnitt~\ref{sec:Rand}).

\begin{exe}
	\ex\label{854}
	\scriptsize
	Mit zwei Koffern in der Hand, ganz neu wollte ich anfangen in Berlin. Die erste Zeit schlief ich auf einem Feldbett, umringt von Bauschutt und Zement. 		Daß um mich herum saniert wurde, störte mich damals nicht. \textbf{\textit{Wollte} ich \underline{doch auch} mein eigenes Leben sanieren.}  	
	\hfill\hbox {(TAZ, 28.03.1991, 25)}
	\newline
	\hbox{}\hfill\hbox {\citet[74-75]{Kwon2005}}
\end{exe}
   
\begin{exe}
	\ex\label{855}
	\scriptsize
	Auf eine kostspielige Asbest-Voruntersuchung wurde gutgläubig verzichtet. \textbf{\textit{Wo} \underline{doch auch} die Bundesbahn Gegenteiliges 			versichert habe.} 
	\hfill\hbox {(TAZ, 14.09.1996, 34)}
	\newline
	\hbox{}\hfill\hbox {\citet[75]{Kwon2005}}
\end{exe}  
Aus Perspektive der Schnittmengenbedingung \is{Schnittmengenbedingung} stellen sie eine Besonderheit dar, da \textit{auch} in Isolation in ihnen nicht verwendet zu werden scheint. Sie enthalten \textit{doch} oder die Kombination \textit{doch auch}. Die Partikel \textit{doch} wiederum ist für diese Sätze sehr typisch (\textit{Wo}-VL) bzw. sogar obligatorisch (V1).\\

\noindent
Wie oben bereits gesehen, kann \textit{doch} im E-Interrogativsatz nicht stehen. Obwohl \textit{auch} zulässig ist, ist das kombinierte Auftreten ausgeschlossen (vgl. (\ref{856})). (\ref{857}) zeigt einen typischen Kontext für \textit{auch} im \is{E-Interrogativsatz} E-Interrogativsatz.

\begin{exe}
	\ex\label{856}E-Interrogativsatz\\
	Hast du \textbf{*doch/auch/*doch auch} am Wochenende Zeit?
\end{exe}

\begin{exe}
	\ex\label{857}
	(Der Nikolaus fragt die Kinder:) Wart ihr \textbf{auch} brav gewesen? 
	\newline
	\hbox{}\hfill\hbox {\citet[57]{Dahl1988}}
\end{exe}
Aus dem gleichen Grund (mit anderer (in)akzeptabler Verteilung) ist eine Kombination auch im Optativsatz \is{Optativsatz} nicht möglich.

\begin{exe}
	\ex\label{858}Optativsatz\\
	Hätte ich \textbf{doch/*auch/*doch auch} am Gewinnspiel teilgenommen!
\end{exe}
Im w-Interrogativsatz \is{w-Interrogativsatz} können beide Partikeln stehen (vgl. (\ref{859a}) und (\ref{859b})), das gemeinsame Auftreten führt aber zu einer inakzeptablen Struktur (s.u. für meine Erklärung).
	
\begin{exe}
	\ex\label{859}w-Interrogativsatz\\[-1.25em]
		\begin{xlist}	
			\ex\label{859a} A: Mir ist furchtbar kalt.\\
			B: Warum ziehst du dich \textbf{auch} so leicht an?
			\hfill\hbox {\citet[89]{Helbig1990}}
			\ex\label{859b} Warum ziehst du dich \textbf{doch} (gleich) so leicht an? (Du hast es mir schon mal erklärt.)	
	 		\ex\label{859c} *Warum ziehst du dich \textbf{doch auch} so leicht an?			
		\end{xlist}
\end{exe}		
Da sowohl \textit{doch}- als auch \textit{auch}-Imperativsätze möglich sind, steht im Einverneh\-men mit der syntaktischen Schnittmengenbedingung auch der Kombination nichts im Wege (vgl. (\ref{860})). In (\ref{861}) findet sich ein typischer Kontext für einen \textit{auch}-Imperativsatz.

\begin{exe}
	\ex\label{860}Imperativsatz\\
	Mach \textbf{doch/auch/doch auch} die Heizung an!
\end{exe}

\begin{exe}
	\ex\label{861}
	Schreibe \textbf{auch} ordentlich!
	\hfill\hbox {\citet[90]{Helbig1990}}
\end{exe}
In Exklamativsätzen \is{Exklamativsatz} beobachtet man Unterschiede je nach Exklamativsatztyp. Aus dem Satzexklamativsatz \is{Satzeklamativsatz} scheint \textit{doch} (den Beispielen aus der Literatur folgend) (zumindest in Isolation) ausgeschlossen (vgl. (\ref{862})).

\begin{exe}
	\ex\label{862}Satzexklamativsatz\\[-1.25em]
		\begin{xlist}	
			\ex\label{862a} *DER hat \textbf{doch} einen Bart! vs. DER hat \textbf{aber/vielleicht} einen Bart!
			\newline
			\hbox{}\hfill\hbox {\citet[218]{Rinas2006}}
			\ex\label{862b} *DER ist \textbf{doch} alt geworden!
	 		\ex\label{862c} *Ist DER \textbf{doc}h alt geworden!		
	 		\hfill\hbox {\citet[224]{Kwon2005}}	
		\end{xlist}
\end{exe}	
\textit{Auch} (in Kombination mit \textit{aber}) ist aber akzeptabel.

\begin{exe}
	\ex\label{863}
		\begin{xlist}	
			\ex\label{863a} Donnerwetter, DAS ist \textbf{aber auch} ein Busen!	
			\hfill\hbox {(TAZ, 30.11.1988, 14)}
	 		\ex\label{863b} Ist DAS \textbf{aber auch} ein Busen!		
	 		\hfill\hbox {\citet[224]{Kwon2005}}	
		\end{xlist}
\end{exe}	
Nach der syntaktischen Schnittmengenbedingung sollte die Kombination nicht auftreten. Äußerungen wie (\ref{863}) werden für mein Befinden aber nicht inakzep\-tabel, wenn das \textit{doch} hinzutritt. Eindeutige Belege habe ich für die Sequenz \textit{doch aber auch} oder \textit{aber doch auch} weder in DeReKo noch DECOW finden können. Exklamativsätze sind (in diesen Daten) aber generell nur sehr schwierig zu belegen. Die \textit{aber auch}-Treffer, für die sich die Interpretation als Satzexklamativsatz anbietet, stört das Hinzufügen von \textit{doch} allerdings nicht (vgl. z.B. (\ref{864})).

\begin{exe}
	\ex\label{864}
	\scriptsize
	Schön, dass alle wieder gut nach Hause gekommen sind! \textbf{Das war aber auch ein Sch...wetter!} Nur gabba störte das wenig, der hatte eine 				passende Mütze auf!   
	\newline
	\hbox{}\hfill\hbox {(www.tt-board.de/forum/archive/index.php/t-15413.html)}
	\newline
	\hbox{}\hfill\hbox {(DECOW14AX)}
\end{exe}   													  
\textit{Dass}- und w-Exklamativsätze zeigen ein eindeutiges Bild: Beide Partikeln können in Isolation und gemeinsam auftreten (vgl. (\ref{865}) bis (\ref{868})).

\begin{exe}
	\ex\label{865} \textit{dass}-Exklamativsatz\\
	Daß der mir \textbf{doch/auch/doch auch} die Vorfahrt nimmt!  
	\newline
	\hbox{}\hfill\hbox {nach \citet[152]{Zaefferer1988}}
\end{exe} 
\vspace{-0.5cm}
\begin{exe}
	\ex\label{866} 
	Daß der Zug \textbf{auch} gerade heute so viel Verspätung hat!
	\newline
	\hbox{}\hfill\hbox {nach \citet[90]{Helbig1990}}
\end{exe} 	

\begin{exe}
	\ex\label{867}w-Exklamativsatz\\[-1.25em]
		\begin{xlist}	
			\ex\label{867a} Was war das \textbf{doch} für ein Fußballspiel!	
			\hfill\hbox {\citet[116]{Helbig1990}}
	 		\ex\label{867b} Wie SCHÖN du \textbf{doch} bist!		
	 		\hfill\hbox {\citet[218-219]{Rinas2006}}
		\end{xlist}
\end{exe}

\begin{exe}
	\ex\label{868}
		\begin{xlist}	
			\ex\label{868a} Was war das \textbf{auch} für ein Erfolg!
			\hfill\hbox {\citet[90]{Helbig1990}}
	 		\ex\label{868b} Was der Kerl \textbf{auch} für Einfälle hat!		
	 		\hfill\hbox {\citet[177]{Schubiger1977}}
		\end{xlist}
\end{exe}	
Im Falle der w-Exklamativsätze ist es dabei unerheblich, ob ein V2- oder VE-Satz vorliegt und \textit{auch} muss auch nicht mit \textit{aber} kombiniert sein.

Die Betrachtung dieser Satzkontexte zeigt, dass man es in Deklarativ-, Impera\-tiv-, w-, \textit{dass}- und ggf. auch Satzexklamativsätzen mit der Sequenz \textit{doch auch} zu tun hat. Diese Feststellung ist für die intendierte Ableitung der Reihenfolgebe\-schränkung bzw. -präferenz hochrelevant, da sie mitbeeinflusst, welcher Art die Reihungsbeschränkung sein kann. In Kapitel~\ref{chapter:jud} musste die Analyse der Abfolge von \textit{ja} und \textit{doch} nur für Assertionen aufkommen. Die dort vorgeschlagene Beschränkung bezog sich deshalb auch auf ein Charakteristikum von Assertionen. Im Falle von \textit{halt} und \textit{eben} (vgl. Kapitel~\ref{chapter:hue}) ergab es sich bereits, dass die vorgeschlagene Ableitung einen weiteren Anwendungsbereich aufweisen musste, weil es notwen\-dig war, Assertionen und Direktive zu erfassen. In der aktuellen Betrachtung von \textit{doch} und \textit{auch} sollte die Restriktion nun so beschaffen sein, dass sie prinzipiell assertive, direktive und exklamative Äußerungen auffangen kann. Mit Ausnahme der Randtypen des \is{Wo-VL-Deklarativsatz} \textit{Wo}-VL- und V1-Deklarativsatzes \is{V1-Deklarativsatz} (dazu s. Abschnitt~\ref{sec:Rand}) lässt sich die Akzeptabilität der MP-Kombination in Deklarativ-, Impe\-rativ- und Ex\-klamativsätzen sowie die Inakzeptabilität derselben im E-Interrogativ- und Optativsatz recht einfach über die syntaktische Schnittmengenbedingung aus \citet{Thurmair1989, Thurmair1991} ableiten. Zum Auftreten von \textit{doch} und \textit{auch} in w-Interroga\-tivsätzen gibt es hingegen mehr zu sagen: Die syntaktische Schnittmengenbedingung, die auf Satzmodi Bezug nimmt, kann nicht entscheidend sein, da die beiden Partikeln hier prinzipiell stehen können (vgl. (\ref{869})).

\begin{exe}
\ex\label{869}
Warum ziehst du dich \textbf{doch} (gleich)/\textbf{auch}/\textbf{*doch auch} so leicht an?
\end{exe}	
\citet[281]{Thurmair1989}; (\citeyear[27]{Thurmair1991}) und auch schon \citet[218, 222, 224-225]{Dahl1988} führen andere Beispiele an, in denen die Satzmodusbedingung eigentlich erfüllt ist, die Kombination aber trotzdem nicht zulässig ist. Sie erklären diese Fälle über eine semantisch-pragmatische Schnittmengenbedingung. Die beiden MPn bzw. die Äußerungen, die sie enthalten, dürfen auch nicht inkompatible Interpretationen oder Verwendungsbedingungen aufweisen. 	
	
Zu \textit{auch}-w-Fragen heißt es in der deskriptiven Literatur, dass als Reaktion auf eine solche Frage vom Sprecher eine negative oder gar keine Antwort erwartet wird (vgl. (\ref{869}), (\ref{870})). 

\begin{exe}
	\ex\label{869}
	A: Ich bin heute sehr müde.\\
	B: Warum gehst du \textbf{auch} immer so spät ins Bett?		
	\hfill\hbox {\citet[89]{Helbig1990}}\\
	(= Du sollst nicht so spät ins Bett gehen \& es ist klar, dass du müde bist, wenn du so spät ins Bett gehst.)		
\end{exe}	

\begin{exe}
	\ex\label{870}
	A: Ich friere so.\\
	B: Warum ziehst du dich \textbf{auch} so leicht an bei so nem nasskalten Wetter?
	\newline	
	\hbox{}\hfill\hbox {\citet[218]{Franck1980}}\\
	(= Du sollst dich nicht so leicht anziehen \& es ist klar, dass du frierst, wenn du dich so anziehst.)		
\end{exe}	
Man hat es folglich weniger mit einer echten Frage, d.h. einer \is{Informationsfrage} Informationsfrage, sondern einer rhetorischen Frage \is{rhetorische Frage} zu tun. Es handelt sich eher um einen Kommentar/eine Begründung der Vorgängeräußerung bzw. eine Bewertung der Proposition der Frage (vgl. \citealt[218-219]{Franck1980}, \citealt[51-54]{Dahl1988}, \citealt[158-159]{Thurmair1989}, \citealt[89]{Helbig1990}, \citealt[231]{Karagjosova2004}, \citealt[77, 202]{Kwon2005}).

In \textit{auch}-w-Interrogativsätzen tritt oft ein kausales w-Pronomen auf, möglich sind aber auch andere w-Ausdrücke (vgl. (\ref{871}), (\ref{872})).

\begin{exe}
	\ex\label{871}
	Der Jochen muß 4.000 Mark Kaution bezahlen! Aber wer unterschreibt \textbf{auch} einen Mietvertrag, ohne ihn vorher genau durchzulesen?	
	\newline
	\hbox{}\hfill\hbox {\citet[159]{Thurmair1989}}
\end{exe}
\vspace{-0.65cm}	
\begin{exe}
	\ex\label{872}
	Was wollte sie \textbf{auch} hier?
	\hfill\hbox {\citet[51]{Dahl1988}}	
\end{exe}
\textit{Doch}-Fragen, wie in (\ref{873}) und (\ref{874}), werden so charakterisiert, dass der Sprecher nach einer Information fragt, die er eigentlich kennt, die er aber vergessen hat/an die er sich im Moment nicht erinnern kann. Der Sprecher will die Antwort vom Hörer deshalb erneut erfahren, wobei es keine Voraussetzung für eine solche Äußerung ist, dass der Hörer die Frage beantworten kann. Sie erfragt beispiels\-weise nicht allgemein Bekanntes (\citealt[88]{Dahl1988}, \citealt[117]{Thurmair1989}, \citealt[114]{Helbig1990}, \citealt[204]{Kwon2005}).

\begin{exe}
	\ex\label{873}
	Wie heißt \textbf{doch} euer Hund?
	\hfill\hbox {\citet[114]{Helbig1990}}
\end{exe}
\vspace{-0.65cm}	
\begin{exe}
	\ex\label{874}
	Wer war \textbf{doch} der berühmte Feuerfresser im Zirkus Krone?
	\hfill\hbox {\citet[88]{Dahl1988}}	
\end{exe}
Auf der Basis dieser Eindrücke und Charakterisierungen müsste eine w-Frage, die die Sequenz \textit{doch auch} beinhaltet, gleichzeitig eine rhetorische Frage sein, deren Antwort beiden Diskursteilnehmern als bekannt vorausgesetzt wird, und eine Frage, mit der der Sprecher sich an etwas erinnern möchte, das er eigentlich weiß. Der Sprecher weiß die Antwort folglich wirklich, nimmt an, dass er und der Hörer sie wissen (\textit{auch}) und er weiß sie nur eigentlich und fragt deshalb nach/möchte sich erinnern (\textit{doch}). Er würde somit ausdrücken, dass er in der konkreten Situation die Antwort weiß (\textit{auch}) und nicht weiß (\textit{doch}). Der Hörer muss die Frage beantworten können (\textit{auch}) und er muss sie nicht beantworten können (\textit{doch}). Diese Verwendungsbedingungen sind schlichtweg unvereinbar bzw. meine Vorhersage wäre, dass die Kombination möglich wird, wenn ein Kontext vorliegt, in dem dieser Konflikt nicht entsteht.

Wenngleich es bei einer Ableitung der Abfolgepräferenz von \textit{doch} und \textit{auch} folglich Deklarativ-, Imperativ- und Exklamativsätze zu berücksichtigen gilt, be\-schränke ich mich im Folgenden zunächst auf \is{Deklarativsatz} Deklarativsätze. Der Beitrag der MPn (besonders gilt dies für \textit{auch}) ist am besten für Deklarativsätze bzw. Assertionen \is{Assertion} untersucht worden. Viele Ansätze zu \textit{auch} berücksichtigen nicht alle drei Auftretenskontexte (vgl. z.B. \citealt{Franck1980}, \citealt[100-105]{Burkhardt1982} und \citealt{Ickler1994}, die Imperativ- und Exklamativsätze nicht behandeln, \citealt{Dahl1988}, der w-Exklamative nicht beücksichtigt, \citealt[222]{Karagjosova2004}, die Imperativsätze aus\-klammert). Dies hat vor allem damit zu tun, dass unklar ist, ob einer MP in verschiedenen Satzmodi dieselbe Interpretation zugeschrieben werden kann. Diese Frage ist Teil der in dieser Arbeit schon mehrfach angesprochenen Grundsatzdiskussion zur Konkret-/Abstraktheit von \is{Bedeutungsminimalismus/-maximalismus} MP-Beschreibungen (Bedeutungs\-minimalismus/-maximalismus): Ist für eine Form von verschiedenen, abweichenden Bedeutungen auszugehen oder gibt es eine invariante Grundbedeutung und die Variation ist auf andere Faktoren zurückzuführen? Dass MP-Beschreibungen i.d.R. anhand von Deklarativsätzen erfolgen, ist sicherlich auch darauf zurückzuführen, dass generell unterschiedlich viel Klarheit über die Satzmodi besteht. Zu Exklamativsätzen wird zudem sogar die Diskussion geführt, ob es sich hier überhaupt um einen eigenen Satzmodus \is{Satzmodus} handelt (vgl. \citealt{Naef1987}, \citealt{Rosengren1992, Rosengren1997}). Und auch im Rahmen der \is{Kontextwechseltheorie} Kontextwechseltheorie, die den Hintergrund fär meine Analyse des Beitrags des Satzkontextes ausmacht, sind vor allem Deklarativsätze bzw. Assertionen behandelt worden. Ich halte deshalb das Vorgehen für legitim, sich für die Formulierung einer Ableitung zunächst auf Deklarativsätze zu beschränken, und anschließend zu überlegen, wie sich auch andere Satzmodi in das entworfene Bild einfügen lassen. 

Der folgende Abschnitt~\ref{sec:V2} beschäftigt sich deshalb zunächst nur mit Standard-Deklarativsätzen. Wie schon erwähnt, verkompliziert sich auch im Bereich der Deklarativsätze mit den Randtypen (\textit{Wo}-VL- und V1-Sätze) das Bild. Diese assertiven Äußerungen sind Gegenstand von Abschnitt~\ref{sec:Rand}. In Abschnitt~\ref{sec:direktive} werde ich die Analyse auf Imperativsätze übertragen. 

\section{V2-Deklarativsätze}
\label{sec:V2}
\subsection{Das Einzelauftreten von \textit{doch} und \textit{auch}}
\subsubsection{\textit{doch}}
\label{sec:doch}
Für \textit{doch} habe ich in Kapitel~\ref{chapter:jud}, Abschnitt~\ref{sec:doch1} schon eine Analyse vorgeschlagen, die ich an dieser Stelle beibehalte. Im Folgenden wiederhole ich diese Modellierung der besseren Lesbarkeit halber kurz. Unter Berufung auf \citet{Diewald1998} gehe ich davon aus, dass \textit{doch} eine konzessive Relation indiziert. Der \textit{doch}-Äußerung geht eine Situation voran, in der die Frage offen ist, ob das, was die Assertion ausdrückt, gilt oder nicht gilt. Vor dem Hintergrund zweier im Kontext bestehender Alternativen vertritt der Sprecher eine der beiden Möglichkeiten. Der Sprecher entscheidet sich für die in seiner Äußerung ausgedrückte Proposition, obwohl die gegenteilige Annahme ebenfalls kontextuell aktiviert ist. Aus \textit{es steht die Frage im Raum, ob p oder non-p gilt}  aus \citet{Diewald1998} wird in meiner Modellierung, dass die Frage schon auf dem Tisch liegt, bevor die \textit{doch}-Assertion gemacht wird. Die Assertion steuert ferner völlig regulär das Bekenntnis des Sprechers zur Proposition bei.

Typischerweise beziehen sich \textit{doch}-Äußerungen auf Implikaturen \is{Implikatur} aus der Vor\-gängeräußerung. Für den Dialog in (\ref{875}) kann man annehmen, dass die erste Äußerung das Gegenteil des zweiten Beitrags implikatiert (vgl. (\ref{875})).

\begin{exe}
	\ex\label{875}
	A: Patrick ist nicht zu Hause.\\
	B: Aber sein Auto ist \textbf{doch} da.
	\hfill\hbox {\citet[83]{Ormelius-Sandblom1997}}
\end{exe}
\vspace{-0.65cm}	
\begin{exe}
	\ex\label{876}
	Patrick ist nicht zu Hause. $>$ Patricks Auto ist nicht da.\\
	(Wenn Patrick nicht zu Hause ist, ist normalerweise sein Auto auch nicht da.)
	\hfill\hbox {\citet[83]{Ormelius-Sandblom1997}}
\end{exe}
Nach As Äußerung besteht folglich der Kontextzustand in (\ref{877}), wobei es sich um den Ausgangskontext der folgenden MP-Äußerung handelt.
\pagebreak
\newcolumntype{C}[1]{>{\centering}p{#1}} 
\begin{exe}
	\ex\label{877} Kontext nach As Äußerung\\
	A: Patrick ist nicht zu Hause. (= $\neg$p) $>$ Patricks Auto ist nicht da. (= $\neg$q)\\[-1em]	
 \begin{tabular}[t]{|C{6em}|C{6em}|C{6em}|}
 \hline 	
   $\textrm{DC}_{\textrm{A}}$ & {Tisch} & $\textrm{DC}_{\textrm{B}}$ \tabularnewline
  \hline
    $\neg$p & p $\vee$ $\neg$p & \tabularnewline
    {} & q $\vee$ $\neg$q & \tabularnewline
  \hline      
   \multicolumn{3}{|l|}{cg s$_{1}$} \tabularnewline   
   \hline
 \end{tabular}
\end{exe}
A hat ein Bekenntnis zu $\neg$p, weshalb sich auf dem Tisch die Frage eröffnet, ob p oder $\neg$p gilt. $\neg$p weist die \is{Implikatur} Implikatur $\neg$q auf, so dass auch die Frage, ob q oder $\neg$q zutrifft, auf dem Tisch landet. Auf diese durch die Implikatur bedingte offene Frage reagiert B mit seinem öffentlichen Bekenntnis zu q (vgl. (\ref{878})).

\begin{exe}
	\ex\label{878} Kontext nach Bs Äußerung\\
	B: Aber sein Auto ist \textbf{doch} da.(= q)\\[-1em]	
 \begin{tabular}[t]{|C{6em}|C{6em}|C{6em}|} 
 \hline 	
   $\textrm{DC}_{\textrm{A}}$ & {Tisch} & $\textrm{DC}_{\textrm{B}}$ \tabularnewline
  \hline
    $\neg$p & p $\vee$ $\neg$p & \tabularnewline
    {} & q $\vee$ $\neg$q & q\tabularnewline
  \hline      
   \multicolumn{3}{|l|}{cg s$_{2}$ = s$_{1}$} \tabularnewline   
   \hline
 \end{tabular}
\end{exe}
Eine \textit{doch}-Äußerung setzt dieser Ansicht nach einen Kontextzustand voraus, in dem der durch sie ausgedrückte Sachverhalt bereits zur Diskussion steht. Man kann auch sagen, dass eine \textit{doch}-Assertion einen instabilen Kontextzustand \is{instabiler Kontextzustand} voraussetzt. Die Proposition ist unentschieden vor der Äußerung, und weil die Assertion nur den \glq halben Beitrag\grq {} leistet, um eine der beiden Propositionen in den cg zu befördern, bleibt die Proposition auch nach der Äußerung im Kontext unentschieden.

\subsubsection{\textit{auch}}
\label{sec:auch}
Für die MP \textit{auch} sind in verschiedenen Arbeiten eine Reihe von deskriptiven Generalisierungen festgehalten worden, die ich im Folgenden darstelle, um anschließend meine Modellierung von \textit{auch} im Diskursmodell nach \citet{Farkas2010} auszuführen, in der sich diese Beschreibungen und Eindrücke natürlich wiederfinden sollen.

Über \textit{auch}-Assertionen wird gesagt, dass die MP-Äußerung einen Zusammenhang mit der Vorgängeräußerung herstellt, in dem Sinne, dass der Inhalt der \textit{auch}-Äußerung Voraussetzung für den Inhalt der vorweggehenden Äußerung ist. Sie fungiert als die Erklärung der ersten Äußerung (vgl. \citealt[47]{Dahl1988}, \citealt[160]{Thurmair1989}, \citealt[1226]{Zifonun1997}, \citealt[343]{Karagjosova2004}, \citealt[222]{Moellering2004}). Die Beispiele in (\ref{879}) und (\ref{880}) illustrieren, wie sich zwi\-schen der \textit{auch}-Äußerung und dem vorherigen Beitrag ein Begründungszusammenhang einstellt.

\begin{exe}
	\ex\label{879}
	A: Das Essen war ausgezeichnet.\\
	B: Es war \textbf{auch} die teuerste Speise, die es in diesem Hotel gibt.	
\end{exe}

\begin{exe}
	\ex\label{880}
	A: Er ist zum Direktor ernannt worden.\\
	B: Er hat \textbf{auch} die meisten Erfahrungen auf unserem Gebiet.	 
	\newline
	\hbox{}\hfill\hbox {\citet[88]{Helbig1990}}
\end{exe}
In (\ref{879}) erklärt der hohe Preis, warum das Essen sehr gut war; in (\ref{880}) dient die Information, dass er die meiste Erfahrung hat, als Erklärung für seine Ernennung zum Direktor. Es handelt sich hierbei um plausible, aber keineswegs notwendige Zusammenhänge. Denn natürlich kann ein teures Essen auch schlecht sein oder jemand, obwohl er der erfahrenste Kandidat ist, einen Posten nicht bekommen. Evidenz für die Tatsache, dass \textit{auch} in Assertionen einen Begründungszusammenhang kodiert, liefern auch Beispiele wie in (\ref{881}), in denen sich diese Relation nicht anbietet. Dass der Wein billig ist, erklärt schlecht, warum er so gut ist.

\begin{exe}
	\ex\label{881}
	A: Der Wein ist ja ausgezeichnet!\\
	B: *Ja, das war \textbf{auch} der billigste Wein im Handel.	 
	\hfill\hbox {\citet[211]{Franck1980}}
\end{exe}
Eine andere Beobachtung ist, dass durch eine \textit{auch}-Assertion die vorweggehende Äußerung (implizit) bestätigt/anerkannt wird (vgl. \citealt[212]{Franck1980}, \citealt[160]{Thurmair1989}, \citealt[88]{Helbig1990}, \citealt[221-222]{Moellering2004}, \citealt[343]{Karagjosova2004}). Es ist plausibel, dass ein Sprecher den Sachverhalt, den er begründet, auch annimmt. Evidenz für dieses Verhältnis ist, dass einer \textit{auch}-Äußerung kein \textit{Nein}. vorangehen kann (vgl. \citealt[212]{Franck1980}) (vgl. (\ref{882})).

\begin{exe}
	\ex\label{882}
	A: Der Wein ist ziemlich dürftig im Geschmack.\\
	B: *Nein, das war \textbf{auch} der billigste Wein bei Lichdi.	 
	\hfill\hbox {\citet[211]{Franck1980}}
\end{exe}
Es lassen sich auch weitere Aussagen über die bestätigte Information machen. Sie geht der \textit{auch}-Äußerung präferiert vorweg. Für mich stellt sich zwischen (\ref{883}) und (\ref{883}) für \textit{auch} ein Akzeptabilitätsunterschied ein, der für \textit{halt} und \textit{eben}, die ebenfalls einen Begründungszusammenhang herstellen können (vgl. Kapitel~\ref{chapter:hue}, Abschnitt~\ref{sec:bedhe} und \ref{sec:kontexte}), nicht gleichermaßen gilt.

\begin{exe}
	\ex\label{883}
	Susanne war \textbf{halt}/\textbf{eben}/\#\textbf{auch} müde. (Nur) DEShalb hat sie so wenig gesagt.	 
	\newline
	\hbox{}\hfill\hbox {nach \citet[89]{Autenrieth2002}}
\end{exe}

\begin{exe}
	\ex\label{884}
	A: Susanne hat sehr wenig gesagt.\\
	B: Sie war \textbf{halt}/\textbf{eben}/\textbf{auch} müde.
\end{exe}
Dies liegt nicht etwa an der Tatsache, dass \textit{auch} nicht monologisch verwendet werden kann (s.u.). 

Die bestätigte Relation kann auch nicht präsupponiert \is{Präsupposition} sein (vgl. (\ref{885}) vs. (\ref{886})).

\begin{exe}
	\ex\label{885}
	A: Warum hast du denn so schlechte Laune?\\
	B: Ich hab \textbf{halt}/\textbf{eben}/\#\textbf{auch} Hunger.
\end{exe}

\begin{exe}
	\ex\label{886}
	A: Du hast heute echt schlechte Laune.\\
	B: Ich hab \textbf{halt}/\textbf{eben}/\textbf{auch} Hunger.
\end{exe}
(\ref{887}) zeigt, dass die bestätigte Information aber aus dem Kontext abgeleitet sein kann.

\begin{exe}
	\ex\label{887}
	Arzt: Na, was fehlt dir denn?\\
	Sabine: Mein Finger blutet und mein Fuß tut weh.\\
	Arzt: Die werden \textbf{auch} wieder heilen.	 
	\hfill\hbox {\citet[118]{Bublitz1978}}
\end{exe}
Begründet wird vom Arzt ein Beitrag wie \textit{Du musst dir keine Sorgen machen.}, was aber nicht explizit vom Arzt mitgeteilt wird.

Darüber hinaus wird der beteiligte Zusammenhang als allgemein gültig angesehen (vgl. \citealt[47]{Dahl1988}). \citet[103]{Burkhardt1982} formuliert diese Beobachtung folgendermaßen: 

\begin{quotation}
Es zeigt sich nach alledem insgesamt, daß es allen exemplifizierten Varianten von \textit{auch} als Abtönungspartikel gemeinsam ist, die \textbf{Erwartbarkeit} eines Sachverhalts aus der Sicht des Sprechers aufgrund einer \textbf{allgemeinen Norm}, einer \textbf{Gesetzmäßigkeit} oder von \textbf{Erfahrungswerten} zu präsupponie\-ren.
\hfill\hbox {(Hervorhebungen S.M.)}
\end{quotation}
Die Beispiele, die man in der Literatur findet, involvieren typischerweise allgemein bekannte Zusammenhänge. Dies trifft auf die oben angeführten Fälle genauso zu wie auf die weiteren Beispiele in (\ref{888}) bis (\ref{890}).

\begin{exe}
	\ex\label{888}
	A: Peter sieht sehr schlecht aus.\\	 
	B: Er ist \textbf{auch} sehr lange krank gewesen.
	\hfill\hbox {\citet[88]{Helbig1990}}\\
	\glq Wenn man lange krank war, sieht man schlecht aus.\grq {}
\end{exe}

\begin{exe}
	\ex\label{889}
	A: Der Wein ist aber ziemlich dürftig im Geschmack.\\	 
	B: (Ja), das war \textbf{auch} der billigste Weisswein bei Lichdi.
	\newline
	\hbox{}\hfill\hbox {\citet[211]{Franck1980}}\\
	\glq Wenn der Wein billig ist, schmeckt er nicht gut.\grq {}
\end{exe}

\begin{exe}
	\ex\label{890}
	A: Das Boot sieht ja wieder fast wie neu aus!\\	 
	B: Ich hab \textbf{auch} seit Ostern jedes Wochenende daran rumgebastelt.
	\newline
	\hbox{}\hfill\hbox {\citet[211]{Franck1980}}\\
	\glq Wenn man viel am Boot bastelt, sieht es aus wie neu.\grq {}
\end{exe}
Ein weiterer Punkt ist, dass der Relevanzwert der Äußerung des ersten Sprechers vom zweiten Sprecher weniger hoch eingeschätzt wird als vom ersten Sprecher (vgl. \citealt[47-48]{Dahl1988}). Anders gefasst, wird der Vorgängeräußerung das Erstaunliche oder Fragwürdige (vgl. \citealt[211-212]{Franck1980}, \citealt[88]{Helbig1990}, \citealt[74]{Kwon2005}) bzw. ihre Informativität (\citealt[223-224]{Karagjosova2004}) genommen. Diese Einschätzung kann auch expliziert werden, wie in (\ref{891}).

\begin{exe}
	\ex\label{891}
	(nach einer gemeinsamen Autofahrt:)\\	 
	A: Wir haben heute nur zwei Stunden gebraucht bis nach Hause.\\
	B: \textbf{Das wundert mich nicht.} Du bist \textbf{auch} gefahren wie ein Weltmeister.
	\newline
	\hbox{}\hfill\hbox {nach \citet[101]{Burkhardt1982}}
\end{exe}	
Im Falle eines Sprecherwechsels ist der Inhalt der A-Äußerung für B ableitbar. In (\ref{892}) ist allgemein bekannt, dass wenn man nicht arbeitet, man wahrscheinlich die Prüfung nicht schafft. Und wenn B darüber hinaus annimmt, dass Peter sich nicht vorbereitet hat, folgt für B, dass Peter die Prüfung vermutlich nicht schafft. Es ist für ihn deshalb nicht verwunderlich oder neue Information, dass er die Prüfung nicht geschafft hat.

\begin{exe}
	\ex\label{892}
	A: Peter hat die Prüfung nicht bestanden.\\	 
	B: Er hat sich \textbf{auch} nicht vorbereitet.
\end{exe}
Im monologischen Gebrauch schätzt der Sprecher die Relevanz der ersten Proposition nicht so hoch ein, weil er sie aus der zweiten ableiten kann.

\begin{exe}
	\ex\label{893}
	Peter hat die Prüfung nicht bestanden. Er hatte sich \textbf{auch} nicht vorbereitet.
	\newline
	\hbox{}\hfill\hbox {nach \citet[227]{Karagjosova2004}}
\end{exe}
Der letzte deskriptive Punkt, der sich in anderen Arbeiten findet, ist, dass der Inhalt der \textit{auch}-Assertion (im Gegensatz zur beteiligten kausalen Relation) neue Information ist, d.h. er ist nicht bekannt oder wird nicht als bekannt vorausgesetzt (vgl. \citealt[215]{Franck1980}, \citealt[156]{Thurmair1989}, \citealt[71]{Koenig1997}, \citealt[343]{Karagjosova2004}, \citealt[73]{Kwon2005}). Wenn der Sprecher von Bekanntheit ausgeht, verwendet er die Kombination \textit{ja auch}, die \citet[156]{Thurmair1989} zufolge häufiger auftritt als nur \textit{auch}. Als Beispiel, das diesen Punkt illustriert, kann (\ref{894}) fungieren.

\begin{exe}
	\ex\label{894}
	A: Ich bin in der Stadtbücherei und ... interessiere mich  n bisschen für Malerei.\\
	B: Malen Sie selbst?\\
	A: Nein, ich male nicht selbst, ich hab \textbf{auch} überhaupt kein Geschick dazu.
	\newline
	\hbox{}\hfill\hbox {(PFE/BRD cc008), \citet[69]{Kwon2005}}
\end{exe}
Der Gesprächspartner kann hier nicht wissen, dass A kein Geschick hat. Beiden sollte jedoch bekannt sein, dass wenn man kein Geschick hat fürs Malen, man es normalerweise nicht macht. Dass A kein Geschick hat, ist seine Begründung für das Nichtmalen.

Die vier relevanten Punkte, die auch durch die diskursstrukturelle Modellierung abgedeckt sein sollten, finden sich in (\ref{895}) (p = \textit{auch}-Proposition, q = vorweggehende Proposition).

\begin{exe}
	\ex\label{895} 
		\begin{xlist}	
			\ex\label{895a} kausale Relation: p $>$ q
			\ex\label{896b} Bekanntheit von p $>$ q
			\ex\label{896c} Relevanzeinschränkung von q
			\ex\label{896d} Unbekanntheit von p			
		\end{xlist}
\end{exe}
Diese Beobachtungen, die vornehmlich aus deskriptiven Arbeiten stammen, übersetze ich auf die folgende Art in das Diskursmodell aus \citet{Farkas2010}. (\ref{896}) dient der Illustration der Modellierung.

\begin{exe}
	\ex\label{896}
	B: Der Wein ist aber ziemlich dürftig im Geschmack. (= q)\\
	A: (Ja), das war \textbf{auch} der billigste Wein bei Lichdi. (= p)
	\hfill\hbox {\citet[211]{Franck1980}}
\end{exe}
Nach Bs Äußerung, die den Ausgangskontext für die \textit{auch}-Assertion darstellt, besteht der Kontextzustand in (\ref{897}). Fett markiert sind die diskursstrukturellen Gegebenheiten, die auf \textit{auch} zurückgehen bzw. von \textit{auch} gefordert werden.

\begin{exe}
	\ex\label{897} Kontextzustand vor der \textit{auch}-Assertion\\[-1em]	
 	\begin{tabular}[t]{|C{6em}|C{6em}|C{6em}|} 
 	\hline 	
   	$\textrm{DC}_{\textrm{A}}$ & {Tisch} & \textbf{$\textrm{DC}_{\textrm{B}}$} \tabularnewline
 	 \hline
     & q $\vee$ $\neg$q & \textbf{q}\tabularnewline
  	\hline      
   	\multicolumn{3}{|l|}{\textbf{cg s$_{1}$ = $\lbrace$p $>$ q$\rbrace$}} \tabularnewline   
   \hline
 \end{tabular}
\end{exe}
B teilt mit, dass er den Wein für nicht gut im Geschmack empfindet, d.h. er bekennt sich zu q. Dadurch stellt sich die Frage, ob beide Gesprächsteilnehmer dies so sehen: Auf dem Tisch liegt q $\vee$ $\neg$q. Im cg befindet sich der Zusammenhang p $>$ q, d.h. es ist allgemein bekannt, dass billiger Wein in der Regel nicht schmeckt. Vor diesem Hintergrund erfolgt die \textit{auch}-Assertion. Zunächst tritt die Kontextsituation wie in (\ref{898a}) ein. A bekennt sich durch die Assertion zu p. Dadurch wird das Thema eröffnet, ob p gilt. Die Proposition q wird implizit bestätigt, d.h. A bekennt sich auch zu q (denn für A und B gilt p $>$ q und für A gilt auch p). Da B schon sein Diskursbekenntnis zu q gegeben hat, wird q zu einem cg-Inhalt (vgl. (\ref{898b})). Dieses Thema ist folglich entschieden. Das Thema rund um p ist hingegen noch offen, denn B könnte den Inhalt der Begründung für sich auch ablehnen.

\begin{exe}
	\ex\label{898} Kontextzustand nach der \textit{auch}-Assertion\\[-1.25em]	
	\begin{xlist}	
		\ex\label{898a} Teil 1\\[-1em]
 		\begin{tabular}[t]{|C{6em}|C{6em}|C{6em}|} 
 		\hline 	
   		$\textrm{DC}_{\textrm{A}}$ & {Tisch} & $\textrm{DC}_{\textrm{B}}$ \tabularnewline
 	 	\hline
     	q & q $\vee$ $\neg$q & q\tabularnewline
     	p & p $\vee$ $\neg$p & \tabularnewline
  		\hline      
   		\multicolumn{3}{|l|}{cg s$_{2}$ = s$_{1}$} \tabularnewline   
   		\hline
 		\end{tabular}
 		
 		\ex\label{898b} Teil 2\\[-1em]
 		\begin{tabular}[t]{|C{6em}|C{6em}|C{6em}|} 
 		\hline 	
   		$\textrm{DC}_{\textrm{A}}$ & {Tisch} & $\textrm{DC}_{\textrm{B}}$ \tabularnewline
 	 	\hline
     	p & p $\vee$ $\neg$p & \tabularnewline
  		\hline      
   		\multicolumn{3}{|l|}{cg s$_{3}$ = $\lbrace$ s$_{2}$ $\cup$ $\lbrace$q$\rbrace\rbrace$} \tabularnewline   
   		\hline
 		\end{tabular} 		
 	\end{xlist}
	\end{exe}
Meine Modellierung von \textit{auch} fängt die deskriptiven Beiträge aus der Literatur auf: Der Begründungszusammenhang zwischen der MP-Äußerung und der vorweggehenden Äußerung findet Eingang durch die Inferenzrelation p $>$ q (vgl. auch \citealt[341-342]{Karagjosova2003}; \citeyear[220-235]{Karagjosova2004} sowie meine Ausführungen zum kausalen Bedeutungsmoment bei \textit{halt} und \textit{eben} in Kapitel~\ref{chapter:hue}, Abschnitt~\ref{sec:kontexte}). Die Bekanntheit dieses Zusammenhangs ist dadurch garantiert, dass p $>$ q im cg ent\-halten ist. A kommt unter der Verankerung von p $>$ q im cg und seiner Annahme von p selbst zum Schluss von q. q ist folglich keine neue Information für ihn. Er hält q nicht für mitteilungswürdig, was der Proposition den Status von Neuigkeit und in diesem Sinne auch Erstaunen nimmt: In seinen Augen ist klar, dass wenn der Wein billig war (p), er nicht schmeckt (q). Die Relevanzeinschränkung des Inhalts der Vorgängeräußerung wird durch meine Beschreibung folglich ebenfalls aufgefangen. Und schließlich wird auch der Inhalt p der MP-Äußerung nicht als bekannt vorausgesetzt: Es ist B nicht wirklich bekannt oder wird ihm als bekannt unterstellt, dass der Wein billig war.

Wie schon bei den Modellierungen der anderen Partikeln liegt der reguläre Beitrag der Assertion vor: Der Sprecher bekennt sich zu ihrem Inhalt. Indem durch die Assertion die Proposition eingeführt wird, eröffnet sich das Thema. Zusätzlich müssen aber bestimmte Verhältnisse vorliegen, damit die \textit{auch}-Asser\-tion angemessen geäußert werden kann. Die minimale Anforderung an den Vorgangskontext einer \textit{auch}-Assertion ist meiner Meinung nach, dass p $>$ q im cg enthalten ist, und dass B (im Dialog) bzw. A selbst (im Monolog) ein Bekenntnis zu q hat. Es muss nicht die Frage auf dem Tisch liegen, ob q gilt, wie in (\ref{896}). Es kann beispielsweise auch der Fall sein, dass die Diskursteilnehmer sich schon auf q geeinigt haben, wie in (\ref{899}). Auch unter diesen Umständen hat der Gesprächspartner aber trotzdem ein Bekenntnis zu q.

\begin{exe}
	\ex\label{899}
	A: Das Boot sieht \textbf{ja} wieder fast wie neu aus! (= q)\\
	B: Ich hab \textbf{auch} seit Ostern jedes Wochenende daran rumgebastelt. (= p)
	\newline
	\hbox{}\hfill\hbox {\citet[211]{Franck1980}}
\end{exe}
Minimal muss q in Bs Diskursbekenntnissen enthalten sein bzw. -- im Falle eines Monologs - A von q ausgehen.

Ich gehe deshalb von (\ref{900}) als dem kontextuellen Vorzustand aus, der gegeben sein muss, damit die \textit{auch}-Assertion angemessen geäußert werden kann.
\begin{exe}
	\ex\label{900} Kontextzustand vor der \textit{auch}-Assertion\\[-1em]	
 	\begin{tabular}[t]{|C{6em}|C{6em}|C{6em}|} 
 	\hline 	
   	$\textrm{DC}_{\textrm{A}}$ & {Tisch} & $\textrm{DC}_{\textrm{B}}$ \tabularnewline
 	 \hline
     (q) & & (q)\tabularnewline
  	\hline      
   	\multicolumn{3}{|l|}{cg s$_{1}$ = $\lbrace$p $>$ q$\rbrace$} \tabularnewline   
   \hline
 \end{tabular}
\end{exe}
Da es sich um eine minimalistische Bedeutungsmodellierung \is{Bedeutungsminimalismus/-maximalismus} handelt, ist natürlich nicht ausgeschlossen, dass weitere Felder besetzt sind (s.o.).
\subsubsection{Der Vergleich des Gebrauchs von \textit{halt}, \textit{eben} und \textit{auch}}
Vergleicht man die Kontextanforderungen aus (\ref{900}) mit denen, die ich für \textit{halt} und \textit{eben} formuliert habe (vgl. (\ref{901}) und (\ref{902})), zeigt sich, dass diese sich nur minimal voneinander unterscheiden.
\pagebreak
\begin{exe}
	\ex\label{901} Kontextzustand vor der \textit{halt}-Assertion\\[-1em]	
 	\begin{tabular}[t]{|C{6em}|C{6em}|C{6em}|} 
 	\hline 	
   	$\textrm{DC}_{\textrm{A}}$ & {Tisch} & $\textrm{DC}_{\textrm{B}}$ \tabularnewline
 	 \hline
     p $>$ q & & \tabularnewline
     (q) & & (q) \tabularnewline
  	\hline      
   	\multicolumn{3}{|l|}{cg s$_{1}$} \tabularnewline   
   \hline
 \end{tabular}
\end{exe}

\begin{exe}
	\ex\label{902} Kontextzustand vor der \textit{eben}-Assertion\\[-1em]	
 	\begin{tabular}[t]{|C{6em}|C{6em}|C{6em}|} 
 	\hline 	
   	$\textrm{DC}_{\textrm{A}}$ & {Tisch} & $\textrm{DC}_{\textrm{B}}$ \tabularnewline
 	 \hline
     & & p \tabularnewline
     (q) & & (q) \tabularnewline
  	\hline      
   	\multicolumn{3}{|l|}{cg s$_{1}$ = $\lbrace$p $>$ q$\rbrace$} \tabularnewline   
   \hline
 \end{tabular}
\end{exe}
Die kausale Komponente -- hier erfasst in Form der Inferenzrelation \is{Inferenzrelation} p $>$ q -- liegt in allen drei Beschreibungen vor: \textit{Auch} ähnelt \textit{eben} darin, dass die kausale Relation \is{kausale Relation} als gesetzt (und deshalb als Teil des cg) angenommen wird. Die Gründe für die Verankerung im cg sind dabei ggf. verschiedene: Bei \textit{auch} handelt es sich um Normen und Erfahrungswerte, bei \textit{eben} wird die kausale Relation als kategorisch/Fakt angesetzt. Mit \textit{halt} teilt es den Umstand, dass der Inhalt der MP-Äußerung nicht als bekannt angesehen/vorausgesetzt wird. Da die Beschreibungen in (\ref{900}) bis (\ref{902}) sehr präzise Aussagen zur Verankerung der Information in den verschiedenen Diskurskomponenten machen, bietet sich die Möglichkeit, die unterschiedliche Verteilung der drei Partikeln in Kontexten zu erklären. Es finden sich Kontexte, in denen alle drei MPn akzeptabel sind, aber auch solche, in denen ihr Gebrauch unterschiedlich angemessen ist. Durch die jeweils angelegten Szenarien lässt sich ableiten, welche/r Gemeinsamkeit/Unterschied für die Verteilung verantwortlich ist.

In dem Dialog in (\ref{903}) beispielsweise können alle drei Partikeln gut auftreten.

\begin{exe}
	\ex\label{903}
	A: Peter sieht sehr schlecht aus. (= q)\\
	B: Er war \textbf{halt}/\textbf{eben}/\textbf{auch} lange krank. (= p) 
	\hfill\hbox {nach \citet[340]{Karagjosova2003}}
\end{exe}
In allen drei Fällen wird As Beitrag besätigt. Unter der Verwendung von \textit{halt} hält B den Zusammenhang zwischen p und q für plausibel, teilt p mit und leitet q allein ab. Gebraucht B \textit{eben}, wird p $>$ q als evidente Relation ausgegeben, p wird ebenfalls als bekannt angenommen, weshalb q für beide abzuleiten ist. Beide Interpretationen bieten sich an, genauso wie die Einordnung von p $>$ q als Norm sowie die Mitteilung der neuen Information p und die Ableitbarkeit von q durch B (\textit{auch}). 

Neben Kontexten, in denen die Äußerung aller drei MPn angemessen ist, weil sich Interpretationen entlang der drei Füllungen der Komponenten in (\ref{900}) bis (\ref{902}) anbieten, gilt für andere Dialoge, dass nur manche Partikeln akzeptabel sind. In den Dialogen in (\ref{904}) und (\ref{905}), die Thurmair anführt, um nachzuweisen, dass \textit{halt} und \textit{eben} nicht identisch gebraucht werden, kann \textit{auch} ebenfalls nicht gut stehen.

\begin{exe}
	\ex\label{904}
	Du kannst deine Freunde schon mitbringen. (= q)\\
	Wir haben \textbf{halt}/\#\textbf{eben}/\#\textbf{auch} kein Bier mehr. (= p)
\end{exe}

\begin{exe}
	\ex\label{905}
	Er: Ich muß noch Rosinen kaufen. Für den Obstsalat.\\
	Sie: Brauchts das denn?\\
	Er: Ja, paßt dir das nicht?\\
	Sie: Naja, in einen Obstsalat gehören \textbf{halt}/\#\textbf{eben}/\#\textbf{auch} keine Rosinen, find ich.
	\hfill\hbox {nach \citet[124]{Thurmair1989}}
\end{exe}
Für (\ref{904}) habe ich schon in Kapitel~\ref{chapter:hue}, Abschnitt~\ref{sec:untersch} angenommen, dass denkbar ist, dass allein der Sprecher davon ausgeht, dass wenn kein Bier mehr vorhanden ist, keine weiteren Gäste eingeladen werden. Da dies nur seine Sicht ist und dazu im Diskurs auch noch nicht bekannt ist, dass es kein Bier gibt, ist das Zugeständnis möglich und auch die Anfrage, die vorausgegangen sein wird, sinnvoll. (\textit{halt}) Wäre der Zusammenhang evident, d.h. beide wären sich einig, dass ohne Bier keine weiteren Leute eingeladen werden, und wäre bekannt, dass kein Bier mehr vorhanden ist, könnte das Zugeständnis schwieriger gemacht werden bzw. wäre die Anfrage schon unangemessen.  Beide Diskursteilnehmer wären sich dann einig, dass keine weiteren Leute eingeladen werden. (\#\textit{eben}) Ein Zugeständnis zu machen, ist auch schwieriger, wenn der Zusammenhang eine Norm ist und man sich gegen sie entscheidet, obwohl man annimmt, dass die Diskursteilnehmer sie teilen. (\#\textit{auch}) Die Information, dass das Bier aus ist, wurde vom Fragenden allerdings in diesem Fall noch nicht gewusst. 

Eine Rolle in der Ableitung des inakzeptablen Gebrauchs von \textit{auch} kann auch spielen, dass hier mit dem Zugeständnis ein Sachverhalt bestätigt wird, der selbst nur eingeschränkt vertreten wird und bei dem es sich deshalb auch nicht eindeutig um eine positive Aussage handelt, die eine \textit{auch}-Äußerung im vorangehenden Kontext aber benötigt (vgl. Abschnitt~\ref{sec:auch}). 

Den Dialog in (\ref{905}) habe ich bereits in Kapitel~\ref{chapter:hue}, Abschnitt~\ref{sec:untersch} so analysiert, dass sich die beiden Gesprächspartner hier sicherlich nicht einig sind, wann der Salat als gut einzustufen ist. Sie nimmt an, dass es ein guter Obstsalat ist, wenn keine Rosinen enthalten sind; er vertritt die Ansicht, dass es ein guter Salat ist, wenn sie enthalten sind. Es bietet sich folglich keine Interpretation an, unter der p $>$ q im cg enthalten ist, was die Inakzeptabilität von \textit{eben} erklärbar macht. \textit{Halt} kann hingegen problemlos gebraucht werden, da sie ihre Version des Zusammenhangs vertreten kann. Da zwei unterschiedliche Positionen vertreten werden, kann der Zusammenhang zwischen einem (guten) Obstsalat und (keinen) Rosinen auch nicht als allgemeine Norm aufgefasst werden. \textit{Auch} ist somit aus dem gleichen Grund ausgeschlossen wie \textit{eben}.

Das zu den Verteilungen aus (\ref{904}) und (\ref{905}) gespiegelte Verhältnis liegt in (\ref{906}) vor.

\begin{exe}
	\ex\label{906}
	Tim: Mensch du bist ja ganz trocken!\\
	Hans: Das ist ?\textbf{halt}/\textbf{eben}/\textbf{auch} Goretex.	
	\hfill\hbox {nach \citet[125]{Thurmair1989}}
\end{exe}
Dass Goretex Wasserschutz bietet, darf als kategorisch, evident/Fakt (\textit{eben}) und auch als allgemeine Norm, erwarteter Erfahrungswert (\textit{auch}) angesehen werden und kann hier demzufolge sehr plausibel im cg verankert sein. Die fragwürdige Verwendung von \textit{halt} ist auf die Art abzuleiten, dass die Aussage, dass nur der Sprecher vom Zusammenhang zwischen Goretex und Trockensein ausgeht, zu schwach ist.

In den bisher angeführten Dialogen haben sich \textit{eben} und \textit{auch} parallel verhalten. Dies ist u.a. darauf zurückzuführen, dass beide Partikeln (wenn auch etwas unterschiedlich motiviert) die cg-Zugehörigkeit der kausalen Relation voraussetzen. Wenn sich diese Verankerung im Dialog nicht anbietet, sind beide Partikeln ausgeschlossen. Kontexte, für die sich argumentieren lässt, dass von einem faktischen Zusammenhang, aber keiner Norm auszugehen ist (eine Norm, die kein Faktum ist, ist m.E. auszuschließen), habe ich bisher nicht finden können. Da sich \textit{eben} und \textit{auch} (neben diesem subtilen Unterschied) auch hinsichtlich des nur bei \textit{eben} beim Hörer vorausgesetzten Bekenntnisses zu p unterscheiden, gibt es Kontexte, in denen diese beiden Partikeln nicht gleichermaßen angemessen gebraucht werden können. Dies trifft für meine Begriffe auf (\ref{907}) zu.

\begin{exe}
	\ex\label{907}
	Arzt: Na, was fehlt dir denn?\\
	Sabine: Mein Finger blutet und mein Fuß tut weg.\\
	Arzt: Die werden \textbf{auch}/\textbf{halt}/\#\textbf{eben} wieder heilen.	
	\hfill\hbox {nach \citet[118]{Bublitz1978}}
\end{exe}
Die Einschätzung der Unangemessenheit des Auftretens von \textit{eben} ändert sich auch nicht, wenn die begründete Aussage expliziert wird.

\begin{exe}
	\ex\label{908}
	Du musst dir keine Sorgen machen. (= q) Die werden \textbf{auch}/\#\textbf{eben}/\textbf{halt} wieder heilen. (= p)
\end{exe}	
Die Differenzierung der Akzeptabiliät ist hier nicht auf den in den obigen Beispielen auftretenden Kontrast der Bewertung der kausalen Relation als cg- vs. Sprecherannahme zurückzuführen. Der Zusammenhang p $>$ q kann als Fakt (\textit{eben}), Norm (\textit{auch}) oder auch nur als plausibler Zusammenhang in den Augen des Sprechers (\textit{halt}) angesehen werden. Die Inakzeptabilität von \textit{eben} ist vielmehr darauf zurückzuführen, dass es in diesem Szenario unangemessen scheint, p als bekannt anzunehmen/zu unterstellen. Dies führt dazu, dass bei bekanntem Zusammenhang p $>$ q und bekanntem p q als für beide Diskursteilnehmer ableitbar dargestellt wird. Folglich teilt der Arzt dem Patienten mit, dass sie beide wissen, dass seine Sorgen (und damit möglicherweise sein ganzes Erscheinen) unnötig sind. \textit{Halt} kann angemessen geäußert werden, weil der Arzt ausdrücken kann, dass er selbst die Sorgen für unbegründet hält (p $>$ q in DC$_{\textrm{Arzt}}$ und Assertion von p). \textit{Auch} ist ebenfalls akzeptabel, weil der Zusammenhang zwischen Heilen und unnötigen Sorgen gut (und wahrscheinlich auch plausibler als nicht) als allgemeine Norm eingeschätzt werden kann. Da p assertiert wird, wird q als für den Sprecher abgeleitet bewertet. In beiden Fällen wird (anders als bei \textit{eben}) keine Einschätzung hinsichtlich der Bekanntheit von p vorgenommen. Auch hier lässt sich unter Bezug auf die verschiedenen Füllungen der Diskurskomponenten im Rahmen meiner Modellierung der ggf. abweichende Gebrauch der drei sehr ähnlichen Partikeln ableiten.

\subsection{Das kombinierte Auftreten von \textit{doch} und \textit{auch}}
Wie schon zuvor in Kapitel~\ref{chapter:jud} und \ref{chapter:hue} stellt sich auch hier als nächstes die Frage, wie eine Äußerung, in der \textit{doch} und \textit{auch} auftreten, interpretiert wird. Und es wiederholt sich die zentrale Frage, ob und, wenn ja, wie, die Skopoi \is{Skopus} der Einzelpartikeln interagieren. Angenommen, die beiden MPn nehmen jeweils Skopus über die Proposition p (= dass Peter krank ist) wie in (\ref{909}), ergeben sich für die Sequenz in (\ref{910}) die vier möglichen Skopusverhältnisse in (\ref{911}) und (\ref{912}).

\begin{exe}
	\ex\label{909} 
		\begin{xlist}	
			\ex\label{909a} Peter ist \textbf{doch} krank. doch(p)
			\ex\label{909b} Peter ist \textbf{auch} krank. auch(p)
		\end{xlist}
\end{exe}

\begin{exe}
	\ex\label{910} 
		Peter ist \textbf{doch auch} krank.
\end{exe}
	
\begin{exe}
	\ex\label{911} Verschiedener Skopus\\[-1em]
		\begin{xlist}
			\ex\label{911a} doch(auch(p))
			\ex\label{911b} auch(doch(p))
		\end{xlist}
\end{exe}

\begin{exe}
	\ex\label{912} Gleicher Skopus\\[-1em]
		\begin{xlist} 	
			\ex\label{912a} doch(p) \& auch(p) ((doch \& auch)(p))
			\ex\label{912b} auch (p) \& doch(p)	((auch \& doch)(p)) 
		\end{xlist}
\end{exe}
Wenngleich eine gängige Erklärung für die (in den Augen der Literatur) feste Abfolge ist, dass sie das asymmetrische Skopusverhältnis \is{Skopus} widerspiegelt, meine ich, dass eine Äußerung, in der \textit{doch} und \textit{auch} gereiht vorkommen, die passendste Interpretation unter gleichem Skopus erhält. Ich gehe somit von einer Interpretation entlang von (\ref{912}) aus.

\subsubsection{Additive Bedeutungskonstitution}
Dass sich für eine \textit{doch auch}-Äußerung die passendste Interpretation ergibt, wenn die beiden MPn den gleichen Skopus nehmen, zeigt die Analyse von Korpusbelegen. Ein erstes Beispiel findet sich in (\ref{913}).

\begin{exe}
	\ex\label{913} 
	\scriptsize
	B: \glqq Sie wissen dass sie mir meinen Job nicht gerade leicht machen?\grqq{}\\
	A: \glqq \textbf{Na sie müssen sich ihr Geld \underline{doch auch} verdienen Lucius!} Wenn sie mich dann entschuldigen würden, ich muss noch einige Einkäufe tätigen und den organisatorischen Kram erledigen.\grqq{}
	\newline
	\hbox{}\hfill\hbox{(DECOW14AX01)}
	\newline
	\hbox{}\hfill\hbox{(http://www.tabletopwelt.de/index.php?/topic/92424-40k-rpg-20/)}
\end{exe}
In diesem Kontext können beide Partikeln sehr plausibel als MPn verwendet sein. (\ref{914}) und (\ref{915}) wiederholen die Zustände, die meiner Modellierung nach im Kontext vor den Partikeläußerungen vorliegen müssen.

\begin{exe}
	\ex\label{914} Kontext vor einer \textit{doch}-Assertion\\[-1em]
 	\begin{tabular}[t]{|C{6em}|C{6em}|C{6em}|}
 	\hline 	
 	$\textrm{DC}_{\textrm{A}}$ & {Tisch} & $\textrm{DC}_{\textrm{B}}$ \tabularnewline
  	\hline
    & p $\vee$ $\neg$p & \tabularnewline
 	\hline      
   	\multicolumn{3}{|l|}{cg s$_{1}$} \tabularnewline   
   	\hline
 	\end{tabular}
\end{exe}

\begin{exe}
	\ex\label{915} Kontextzustand vor der \textit{auch}-Assertion\\[-1em]	
 	\begin{tabular}[t]{|C{6em}|C{6em}|C{6em}|} 
 	\hline 	
   	$\textrm{DC}_{\textrm{A}}$ & {Tisch} & $\textrm{DC}_{\textrm{B}}$ \tabularnewline
 	 \hline
     (q) & & (q)\tabularnewline
  	\hline      
   	\multicolumn{3}{|l|}{cg s$_{1}$ = $\lbrace$p $>$ q$\rbrace$} \tabularnewline   
   \hline
 \end{tabular}
\end{exe}
Wenn meine Annahme zur additiven Bedeutungskonstitution korrekt ist, muss deshalb zwischen den beiden Äußerungen die kausale Verbindung \is{kausale Relation} aufzufinden sein. Darüber hinaus muss sich mindestens einer der Diskurspartner zu dem Sachverhalt, der in Folge begründet wird, bekennen (\textit{auch}). Und die Proposition, die als Begründung fungiert, muss zur Debatte stehen (\textit{doch}). Im konkreten Fall kann und wird das genaue Zustandekommen dieser Verhältnisse durchaus variieren. Für meine Analyse sind nur die Inhalte von Bedeutung, die ich an das Partikelvorkommen von \textit{doch} und \textit{auch} und somit auch \textit{doch auch} binde.

Die beiden relevanten Propositionen sind in (\ref{913}) p = B muss sein Geld wert sein und $\neg$q = A macht Bs Job für B nicht leicht. Die relevante Relation ist: \glq Wenn B sein Geld wert sein muss, macht A Bs Job für B nicht leicht.\grq {} Dieser Zusammenhang ist in (\ref{913}) im cg enthalten. Darüber hinaus präsupponiert \is{Präsupposition} Bs Frage (faktives \textit{wissen}), dass A B den Job nicht leicht macht. Aus diesem Grund ist auch $\neg$q im cg und deshalb ist $\neg$q ebenfalls unter As und Bs Diskursbekenntnissen. Die Anforderungen, die \textit{auch} an den Kontext stellt, sind somit erfüllt (vgl. (\ref{914})). Da die Frage m.E. vorwurfsvoll oder negativ erstaunt hinsichtlich der Tatsache $\neg$q wirkt, denke ich, dass sich aus der Tatsache, dass B diese Frage stellt, ableiten lässt, dass B sich zu $\neg$p bekennt, d.h. dazu, dass er sein Geld nicht wert sein muss. Wenn er annähme, dass er sein Geld wert sein muss, wäre er von seiner harten Zeit nicht überrascht. Auf diesem Wege eröffnet sich in diesem Dialog das Thema \textit{Muss B sein Geld wert sein?}.\footnote{Dass aus dem Überraschtsein von $\neg$q ein Bekenntnis zu $\neg$p ableitbar ist, ließe sich ggf. auch als Implikatur im cg verankern.}

\begin{exe}
	\ex\label{916} Kontext vor der \textit{doch auch}-Assertion\\[-1em]	
 	\begin{tabular}[t]{|C{6em}|C{6em}|C{6em}|} 
 	\hline 	
   	$\textrm{DC}_{\textrm{A}}$ & {Tisch} & $\textrm{DC}_{\textrm{B}}$ \tabularnewline
 	 \hline
     & p $\vee$ $\neg$p & $\neg$p\tabularnewline
  	\hline      
   	\multicolumn{3}{|l|}\textbf{cg s$_{1}$ = $\lbrace$p $>$ $\neg$q, $\neg\textrm{q}\rbrace$} \tabularnewline   
   \hline
 \end{tabular}
\end{exe}
Wenn die \textit{doch auch}-Assertion gemacht wird (vgl. (\ref{917})), führt A p ein. Diese Proposition dient der Erklärung von $\neg$q. Für A folgt $\neg$q, weil p $>$ $\neg$q Teil des cg ist. Dieser Bedeutungseffekt geht auf \textit{auch} zurück. Dazu kommt, dass A auf das offene Thema p $\vee$ $\neg$p reagiert, wofür \textit{doch} verantwortlich ist.

\begin{exe}
	\ex\label{917} Kontext nach der \textit{doch auch}-Assertion\\[-1em]	
 	\begin{tabular}[t]{|C{6em}|C{6em}|C{6em}|} 
 	\hline 	
   	$\textrm{DC}_{\textrm{A}}$ & Tisch & $\textrm{DC}_{\textrm{B}}$ \tabularnewline
 	\hline
    p & p $\vee$ $\neg$p & $\neg$p\tabularnewline
  	\hline      
   	\multicolumn{3}{|l|}{cg s$_{2}$ = s$_{1}$} \tabularnewline   
   \hline
 \end{tabular}
\end{exe}
Nach der MP-Äußerung weiß B, dass A p annimmt und dass dies in As Augen die Erklärung für $\neg$q ist. Je nach dem weiteren Kontextverlauf kann B seine eigene Ansicht gegenüber $\neg$p revidieren oder beibehalten. Es ist folglich möglich, zu motivieren, warum die Anforderungen, die \textit{doch} und \textit{auch} in Isolation benötigen, in diesem Dialog, in dem eine \textit{doch auch}-Assertion verwendet wird, erfüllt sind.

Bevor ich alternative Interpretationen einer \textit{doch auch}-Assertion durchspiele, sollen zwei weitere authentische Beispiele zeigen, dass die additive Bedeutungszu\-schreibung den Effekt der MP-Äußerung adäquat auffängt. Zunächst sei zu diesem Zweck (\ref{918}) betrachtet, in dem die Verhältnisse etwas komplexer sind und indirekter zustande kommen als in dem vorherigen Beispiel.

\begin{exe}
	\ex\label{918}
	\scriptsize
	Also das was Guido Knopp macht, sind bestimmt keine 100 prozentigen wissenschaftlich/historisch korrekten Dokumentationen. \textbf{Will er 					\underline{doch auch} gar nicht.} Sowas kann man auf Arte sehen. 
	\hfill\hbox {(deWac: 1624)}
\end{exe}
Um nachzuweisen, dass eine Analyse, in der \textit{doch} und \textit{auch} sich jeweils auf die gleiche Proposition beziehen, zutreffend ist, bietet es sich an, zunächst das Ein\-zelauftreten der beiden MPn zu motivieren. Die MP-Äußerung könnte auch lauten \textit{Will er \textbf{doch} gar nicht.} oder \textit{Will er \textbf{auch} gar nicht.} Die beiden relevanten Propositionen sind in (\ref{919}) festgehalten.
	
\begin{exe}
	\ex\label{919} 
		\begin{xlist}	
			\ex\label{919a} p = Guido Knopp (GK) will keine hundertprozentigen wissenschaftlich/ historisch korrekten Dokumentationen
			\ex\label{919b} q = GK macht keine hundertprozentigen wissenschaftlich/historisch korrekten Dokumentationen
			\hfill\hbox {\citet[42]{Helbig1981}}
		\end{xlist}
\end{exe}
Die für \textit{auch} benötigte Inferenzrelation ist: \glq Wenn GK keine hundertprozentigen wissenschaftlich/historisch korrekten Dokumentationen will (= p), sind seine Dokumentationen nicht hundertprozentig wissenschaftlich/historisch korrekt (= q).\grq {} D.h. die \textit{auch}-Äußerung bestätigt q, die Aussage, die der Sprecher selber macht. Da der Zusammenhang zwischen gewollter und tatsächlicher Gestaltung als allgemeingültig angesehen werden kann, kann hier p $>$ q sehr plausibel im cg enthalten sein. Dass er entsprechende Dokumentationen nicht beabsichtigt, kann in diesem Dialog darüber hinaus auch als neue Information gewertet werden. Im weiteren Vorgangskontext wird nicht darüber gesprochen, dass GK dies will - wenngleich von einer anderen Person beigetragen wird, dass die Produktionen an sich populärwissenschaftlich sind.

Nach meiner Analyse von \textit{doch} muss auf dem Tisch liegen \textit{Will GK derartige Dokus oder will er sie nicht?} D.h. p $\vee$ $\neg$p steht zur Debatte und der Sprecher entscheidet sich mit der Assertion für p. Dieses offene Thema kann in diesem Kontext ebenfalls nachgewiesen werden, über eine Inferenzrelation \is{Inferenzrelation} und eine Standardannahme, von der auszugehen ist. Letztere ist, dass normalerweise davon ausgegangen wird, dass Dokumentationen wissenschaftlich korrekt sind (t $>$ k $[$\glq Wenn eine TV-Sendung eine Doku ist, ist sie normalerweise hundertprozentig korrekt.\grq {}$]$). Wenn eine Abweichung von diesem Standard auftritt (r) (es gelten t und $\neg$k), stellt sich die Frage, ob dies gewollt ist (p $\vee$ $\neg$p) (r $>$ (p $\vee$ $\neg$p)).

Basierend auf diesen zwei Teilanalysen interpretiere ich die Szene in (\ref{918}) des\-halb folgendermaßen: Im Normalfall sind Dokumentationen wissenschaftlich/his\-torisch korrekt (t $>$ k). GK produziert eine nicht-korrekte Dokumentation ($\neg$k trotz t). Da die Doku von der Norm abweicht (r), stellt sich ein unerwarteter Umstand ein. Man wundert sich. Auch stellt sich ein potenzieller Vorwurf oder eine potenzielle Kritik ein. Aufgrund der Normabweichung (r), die erst einmal immer unerwartet ist, stellt sich die Frage, ob GK eine korrekte Doku (nicht) wollte (p $\vee$ $\neg$p). Diese Frage beantwortet der Sprecher derart, dass GK eine korrekte Doku nicht wollte ($\neg$p). Dadurch rechtfertigt er zwar nicht diese eher unübliche Art der Dokumentation. Er nimmt aber die Spannung/potenzielle Kritik, dass dies ein Fehler ist oder GK etwa zu keiner korrekten Dokumentation fähig sei. (\ref{920}) zeigt den Kontextzustand vor der \textit{doch auch}-Assertion.

\begin{exe}
	\ex\label{920} Kontext vor der \textit{doch auch}-Assertion\\[-1em]	
 	\begin{tabular}[t]{|C{6em}|C{6em}|C{6em}|} 
 	\hline 	
   	$\textrm{DC}_{\textrm{A}}$ & Tisch & $\textrm{DC}_{\textrm{B}}$ \tabularnewline
 	\hline
 	q & q $\vee$ $\neg$q & \tabularnewline
    & p $\vee$ $\neg$p & \tabularnewline
  	\hline      
   	\multicolumn{3}{|l|}{cg s$_{1}$ = $\lbrace$p $>$ q, t $>$ k, r $>$ (p $\vee$ $\neg$p$)\rbrace$} \tabularnewline   
   \hline
 \end{tabular}
\end{exe}
Der Sprecher hat ein Bekenntnis zu q (= GK macht nicht hundertprozentig wissenschaftlich/historisch korrekte Dokus). Dieses Thema gelangt standardmäßig auch auf den Tisch. Für die MP-Verwendung relevant ist aber vielmehr, dass aufgrund der Annahme, dass Standarddokumentationen normalerweise hundertprozentig wissenschaftlich/historisch korrekt sind (t $>$ k), der Zusammenhang r $>$ (p $\vee$ $\neg$p) greift, d.h. sich die Frage stellt, ob GK dies so will. Aus diesem Grund liegt auch p $\vee$ $\neg$p auf dem Tisch. Es gibt hier zwar keinen Diskurspartner, der q explizit bestätigt, im Vorkontext wurde aber schon angemerkt, dass GKs Produktionen populärwissenschaftlich sind, d.h. diese Annahme kann im Kontext als etabliert gelten (die Fragen \textit{ob q} und \textit{ob r} werden deshalb sofort zugunsten von q und r aufgelöst). Wenn dies nicht gegeben wäre, müsste man annehmen, dass nur der Sprecher die Offenheit von p annimmt (weil nur er die Abweichung von der Norm t $>$ k sieht). Da q aber schon eingeführt ist, kann p $\vee$ $\neg$p für alle Beteiligten zur Diskussion stehen. Dieses offene Thema adressiert die \textit{doch auch}-Assertion (\textit{doch}) und löst die offene Frage nach p auf. Da im cg zudem die Inferenzrelation \is{Inferenzrelation} enthalten ist, dass GKs Absicht die Erklärung für diese Art von Dokus ist (p $>$ q), begründet der Sprecher q (dass GK derartige Dokus macht) damit, dass GK solche Dokus machen will. Für den Sprecher ist deshalb nicht verwunderlich, dass GKs Filme so beschaffen sind (vgl. (\ref{921})).

\begin{exe}
	\ex\label{921} Kontext nach der \textit{doch auch}-Assertion\\[-1em]	
 	\begin{tabular}[t]{|C{6em}|C{6em}|C{6em}|} 
 	\hline 	
   	$\textrm{DC}_{\textrm{A}}$ & Tisch & $\textrm{DC}_{\textrm{B}}$ \tabularnewline
 	\hline
    p & p $\vee$ $\neg$p & \tabularnewline
  	\hline      
   	\multicolumn{3}{|l|}{cg s$_{2}$ = $\lbrace$p $>$ q, r, t $>$ k, r $>$ (p $\vee$ $\neg$p$)\rbrace$} \tabularnewline   
   \hline
 \end{tabular}
\end{exe}
Auch in diesem komplexeren Beispiel gehe ich davon aus, dass sich \textit{auch} und \textit{doch} auf die Proposition p beziehen und sich die Interpretation aus dem additiven Zusammenschluss beider MP-Beiträge ergibt.

Ein weiteres m.E. unkompliziertes Beispiel findet sich in (\ref{922}).

\begin{exe}
	\ex\label{922} 
	\scriptsize
    \begin{tabular}[t]{ll}
	0956 BS	& $[$ähm$]$ schule hier her isst nur dann geht er zur therapie dann zum judo und um sechs\\
	{} & uhr kommt er heim und da muss er noch hausaufgaben machen\\
	{} & $^{o}$h hat er (.)$[$s so gesagt \emph{dass es ihm zu viel is}$]$\\
	0957 HM & $[$hm\_hm$]$\\
	0958 SZ & \hspace{1cm} \textbf{$[$(is \underline{doch auch}) (.) ä programm$]$}\\
	0959 NG & $[$hm$]$ \_hm\\
	0960 BS & und da hab ich gsagt	
	\hfill\hbox{(FOLK\_E\_00026\_SE\_01\_T\_01)} 					 
    \end{tabular}   
\end{exe}
Die beteiligten Propositionen finden sich in (\ref{923}).

\begin{exe}
	\ex\label{923} 
		\begin{xlist}	
			\ex\label{923a} p = es (Schule, Therapie, Judo, Hausaufgaben) ist ein Programm
			\ex\label{923b} q = es ist zu viel
		\end{xlist}
\end{exe}
Die Inferenzrelation \is{Inferenzrelation} ist: \glq Wenn etwas als Programm bezeichnet wird, ist es zu viel.\grq {}. SZs Äußerung fungiert in diesem Dialog als Begründung für seine (unausgesprochene) Zustimmung zur Einschätzung des Sohnes, dass es ihm zu viel ist: p wird assertiert, weshalb es – vor dem Hintergrund der Relation im cg – in SZs Augen klar ist, dass das Pensum zu viel ist. Diese Verhältnisse legitimieren die Verwendung von \textit{auch}. Die Offenheit von p  lässt sich dadurch motivieren, dass unter der im cg enthaltenen Relation p $>$ q q thematisiert wird. Dadurch kommt die Frage auf, ob p tatsächlich der Grund ist. Auch die Kontextanforderung, die ich \textit{doch} zuschreibe (p $\vee$ $\neg$p liegt auf dem Tisch), kann somit als erfüllt angesehen werden.

Unter Bezug auf die Gebrauchsweise der \textit{doch auch}-Assertionen im Kontext in Dialogen wie (\ref{913}), (\ref{918}) und (\ref{922}) gehe ich davon aus, dass das Skopusverhältnis in (\ref{924}) vorliegt.

\begin{exe}
	\ex\label{924} 
	doch(p) \& auch(p)
\end{exe}	
Die beiden Partikeln beziehen sich auf dieselbe Proposition und weisen demnach einen identischen Skopusbereich auf. 

Es ist \underline{eine} Komponente der Argumentation, zu zeigen, dass die additive Bedeutungskonstitution für Äußerungen im Kontext passende Interpretationen leistet. Im Folgenden möchte ich aber ebenfalls durchspielen, welche Ergebnisse resultieren, wenn man von einer Skopuslesart ausgeht. Die entstehenden Lesarten sind zum einen nicht zutreffend und zum anderen auch aus generelleren Gründen nicht sinnvoll anzunehmen.

Eine Alternative zu einer Aufaddierung der beiden Partikelbeiträge ist, dass \textit{doch} Skopus über \textit{auch} nimmt. Das Ergebnis für die Verhältnisse im Kontextzustand vor der MP-Assertion zeigt (\ref{925}).

\begin{exe}
	\ex\label{925} Kontext vor der \textit{doch auch}-Assertion (doch(auch(p)))\\[-1em]	
 	\begin{tabular}[t]{|C{7em}|C{12em}|C{7em}|} 
 	\hline 	
   	$\textrm{DC}_{\textrm{A}}$ & Tisch & $\textrm{DC}_{\textrm{B}}$ \tabularnewline
 	\hline
     & (q $\in$ $\textrm{DC}_{\textrm{A/B}}$ \& cg = $\lbrace$p $>$ q$\rbrace$) $\vee$ & \tabularnewline
     & $\neg$(q $\in$ $\textrm{DC}_{\textrm{A/B}}$ \& cg = $\lbrace$p $>$ q$\rbrace$) \tabularnewline
  	\hline      
   	\multicolumn{3}{|l|}{cg s$_{1}$} \tabularnewline   
   \hline
 \end{tabular}
\end{exe}
Ich denke nicht, dass (\ref{925}) die Situation passend auffängt, die die Ausgangslage für die MP-Äußerung ist. Unter Bezug auf (\ref{913}) ginge es in dem Dialog um die Frage, ob A oder B annehmen, dass A Bs Job für B nicht einfach macht und ob der kausale Zusammenhang besteht zwischen dem Sachverhalt, dass A Bs Job für B nicht leicht macht und dass B sein Geld wert sein muss. Diese Interpretation trifft schlichtweg nicht zu: Dass A und B sich zu $\neg$q bekennen (A macht Bs Job nicht leicht für B) steht hier sicherlich nicht zur Diskussion, diese Diskursbekenntnisse liegen eindeutig vor (Sie wissen, dass $\neg$q?). Es ist auch nicht Thema des Gesprächs, ob p $\neg$q begründet; auf diese Relation bezieht A sich einfach. Genauso wenig wird thematisiert, ob die beiden Aspekte zusammen gelten oder nicht. In (\ref{922}) begründet SZ zwar nur eine implizite Zustimmung (es ist zu viel), es steht in diesem Dialog aber dennoch nicht zur Diskussion, ob er dies vertritt. Es wird auch keine Diskussion geführt, ob die Kennzeichnung eines Tagesablaufs als \glqq Programm\grqq{} die Aktivitäten als \glqq zu viel\grqq{} einschätzt. In (\ref{918}) ist offensichtlich, dass der Sprecher vertritt, dass Guido Knopp nicht wissenschaftlich/historisch hundertprozentig korrekte Dokumentationen macht. Dies steht nicht zur Frage. Die MP-Äußerung scheint mir auch nicht auf die Situation zu reagieren, dass unklar ist, ob von dem Zusammenhang auszugehen ist, dass aus Guido Knopps gewollter Art der Dokumentation auf die Beschaffenheit dieser geschlossen werden kann. 

Die Alternative ist, dass das umgekehrte Skopusverhältnis vorliegt: auch(doch(p)). Der Kontext vor der \textit{doch auch}-Assertion ist in diesem Fall beschaffen wie in (\ref{926}).

\begin{exe}
	\ex\label{926} Kontext vor der \textit{doch auch}-Assertion (auch(doch(p)))\\[-1em]	
 	\begin{tabular}[t]{|C{6em}|C{6em}|C{6em}|} 
 	\hline 	
   	$\textrm{DC}_{\textrm{A}}$ & Tisch & $\textrm{DC}_{\textrm{B}}$ \tabularnewline
 	\hline
 	(q) & & (q) \tabularnewline
  	\hline      
   	\multicolumn{3}{|l|}{cg s$_{1}$ = $\lbrace$((p $\vee$ $\neg$p) $\in$ T) $>$ q$\rbrace$} \tabularnewline   
   \hline
 \end{tabular}
\end{exe}
Hier ist der Bedeutungsaspekt beteiligt, dass es eine Hintergrundannahme gibt, der zufolge aus einem zur Debatte stehenden Thema q normalerweise folgt. Dies scheint mir auch ohne den Bezug auf konkrete Beispiele nicht besonders sinnvoll zu sein.

Für (\ref{913}) hieße dies, dass man sich einig ist, dass wenn im Diskurs unentschieden ist, ob B sein Geld wert sein muss oder nicht, normalerweise folgt, dass A Bs Job für B nicht leicht macht. In (\ref{925}) läge die Situation vor, dass bekannt ist, dass aus der Diskussion, ob das Tagespensum ein \glqq  Programm\grqq{} ist, plausiblerweise folgt, dass es für den Sohn zu viel ist. In (\ref{918}) wäre aus der Offenheit des Themas, ob Guido Knopp derartige Produktionen will, abzuleiten, dass die Dokus derart beschaffen sind. 

Wenngleich ich diese Verhältnisse nicht für völlig abwegig halte, glaube ich nicht, dass der nach dieser Lesart vorliegende Kontextzustand die Verwendung der MP-Äußerung motiviert. p $\vee$ $\neg$p müsste dann gar nicht wirklich zur Debatte stehen, was in meinen Augen aber in allen drei Dialogen der Fall ist. Prinzipiell dürfte es, wenn dies die zutreffende Interpretation ist, keine Kontexte geben, in denen hinsichtlich q gegensätzliche Annahmen bestehen, wenn gleichzeitig p $\vee$ $\neg$p auf dem Tisch liegt. Wenn das Thema zur Debatte steht, müssen beide die Proposition vertreten, die der Sprecher der MP-Äußerung begründet. Es wäre z.B. ausgeschlossen, dass der Gesprächspartner $\neg$q vertritt, während gleichzeitig  p $\vee$ $\neg$p verhandelt wird, bzw. es müsste dann eine Einigung hinsichtlich q klar sein.

Auf (\ref{913}) und (\ref{918}) scheint mir diese Konstellation zuzutreffen. $\neg$q bzw. q ist jeweils cg und p $\vee$ $\neg$p steht zur Diskussion. In (\ref{922}) ist q aber noch unentschieden, obwohl die Diskussion um p/$\neg$p zur Verhandlung auf dem Tisch liegt. Ob das Pensum zu viel ist, entscheidet sich im Dialog nicht. In (\ref{918}) muss es nicht der Fall sein, dass der Sprecher der Vorgängeräußerung q akzeptiert und die Proposition dadurch cg wird. Nur für SZ ist in jedem Fall klar, dass q gilt.

Auch die umgekehrte Skopusrelation scheint folglich nicht stets zu einer sinn\-vollen Interpretation zu führen, auch wenn sie – je nach Szenario – mehr oder weniger problematisch erscheint. In allen drei Fällen erfasst sie meiner Meinung nach aber nicht die Situation, die die MP-Äußerung motiviert.

\subsubsection{Erklärung für die unmarkierte Abfolge}
Obwohl die Annahme, dass die Präferenz für eine Partikelabfolge auf das Skopusverhältnis zurückgeführt werden kann, plausibel erscheint, denke ich, dass es wenig von Nutzen ist, wenn die resultierende Interpretation in Dialogen, in denen derartige Äußerungen angemessen verwendet werden können, nicht zu\-treffend ist. Meine Ableitung der präferierten Reihung baut deshalb auf der additiven Interpretation auf. Die Grundidee, die ich in meiner Arbeit verfolge, ist, dass man die Abfolge der MPn in Kombinationen über ihre Interpretation motivieren kann. Ich vertrete somit eine Form von Ikonizität: Die Form entspricht der MP-Sequenz (\textit{doch} vor \textit{auch}), die Funktions-/Bedeutungsseite ist dem Diskursbeitrag der MP-Äußerung zugeordnet. Die Überlegung ist, dass die unmarkierte Abfolge dem diskursiven Ziel direkter nachkommt als die markierte Reihung. Ferner gehe ich auch davon aus, dass die markierte Anordnung (wenn es sie denn gibt) zulässig ist, weil sie unter bestimmten Umständen auftritt, in denen im Vergleich zur unmarkierten Abfolge geänderte Bedingungen vorliegen. Inwieweit anzunehmen ist, dass diese Sicht auch auf Kombinationen aus \textit{doch} und \textit{auch} zutrifft, diskutiere ich in Abschnitt~\ref{sec:distributionad}.

(\ref{927}) und (\ref{928}) wiederholen erneut die Kontexte, die meiner Modellierung nach vorliegen müssen, damit eine \textit{doch}- bzw. \textit{auch}-Assertion angemessen verwendet werden kann.

\begin{exe}
	\ex\label{927} Kontext vor einer \textit{doch}-Assertion\\[-1em]
 	\begin{tabular}[t]{|C{6em}|C{6em}|C{6em}|}
 	\hline 	
 	$\textrm{DC}_{\textrm{A}}$ & {Tisch} & $\textrm{DC}_{\textrm{B}}$ \tabularnewline
  	\hline
    & p $\vee$ $\neg$p & \tabularnewline
 	\hline      
   	\multicolumn{3}{|l|}{cg s$_{1}$} \tabularnewline   
   	\hline
 	\end{tabular}
\end{exe}

\begin{exe}
	\ex\label{928} Kontextzustand vor der \textit{auch}-Assertion\\[-1em]	
 	\begin{tabular}[t]{|C{6em}|C{6em}|C{6em}|} 
 	\hline 	
   	$\textrm{DC}_{\textrm{A}}$ & {Tisch} & $\textrm{DC}_{\textrm{B}}$ \tabularnewline
 	 \hline
     (q) & & (q) \tabularnewline
  	\hline      
   	\multicolumn{3}{|l|}{cg s$_{1}$ = $\lbrace$p $>$ q$\rbrace$} \tabularnewline   
   \hline
 \end{tabular}
\end{exe}
\textit{Doch} verweist meiner Auffassung nach auf die Offenheit der Proposition, d.h. reagiert auf das Thema, das gerade ausgehandelt wird bzw. noch zur Verhandlung steht. \textit{Auch} begründet eine andere Annahme.

Neben diesen Beiträgen der MPn nimmt meine Erklärung zudem Bezug auf die zwei Aspekte in (\ref{929}), die auch an anderer Stelle bereits in die Analyse eingebunden waren (vgl. Kapitel~\ref{chapter:jud}, Abschnitt~\ref{sec:markiert}).

\begin{exe}
	\ex\label{929} Antriebe für Konversation\\[-1em]
		\begin{xlist}	
			\ex\label{929a} Erweiterung des cg
			\ex\label{929b} Herstellung eines stabilen Kontextzustands
			\newline
			\hbox{}\hfill\hbox {\citet[87]{Farkas2010}}
		\end{xlist}
\end{exe}
Zum einen folgen Teilnehmer dem Drang, den cg anzureichern. Aus diesem Grund legen sie Elemente auf den Tisch. Zum anderen streben sie danach, einen stabilen Kontextzustand zu erreichen, d.h. einen Zustand, in dem kein Thema offen ist. Aufgrund dieser Absichten entfernen sie die Elemente so vom Tisch, dass der cg angereichert wird.

Die Idee, die ich vertreten möchte, ist, dass die Abfolge \textit{doch auch} das Diskurs\-ziel direkter abbildet als \textit{auch doch}, weil es für den Gang der Konversation oder das Ziel von Kommunikation im Sinne von (\ref{929}) direkter relevant ist, das aktuelle Thema zu adressieren (was \textit{doch} leistet), als einen Grund für einen anderen Sachverhalt anzuführen (was für \textit{auch} gilt). Für den Diskurs ist es dringlicher, zu erfahren, dass die assertierte Proposition Teil des Themas der Diskussion ist, als dass der Sprecher annimmt, dass die Proposition eine andere Proposition begründet.

Den Diskurs vorwärts zu bringen, ist dieser Argumentation nach einer qualitativen Bewertung übergeordnet. Natürlich treten sowohl die Adressierung des Themas als auch das Anführen des Grundes ein, aus Perspektive der Diskursabsichten geht die Themaadressierung der Begründung aber voran.

Meine Erklärung bietet einen Anknüpfungspunkt zu Hypothese 2 aus \citet[288]{Thurmair1989}. Sie führt sehr wenig aus, welche MP-Reihungen welche in den Hypothesen 1–5 formulierten Verhältnisse spiegeln. Für die Abfolge von \textit{doch} und \textit{auch} kommt H2 i.E. in Frage (\citeyear[288]{Thurmair1989}). H2 lautet, dass MPn, die Bezug auf die momentane Äußerung nehmen, vor MPn stehen, die eine qualitative Bewertung des Vorgängerbeitrags vornehmen. 

\textit{Doch} vereint in ihrer Modellierung die Merkmale BEKANNT$_{\textrm{H}}$ und KORREKTUR. Die ausgedrückte Proposition ist dem Hörer bekannt und die Äußerung fordert den Hörer auf, seine Ansicht zu ändern.

\textit{Auch} wird durch die Merkmale KONNEX und ERWARTET$_{\textrm{V/S}}$ charakterisiert. Die Partikel zeigt an, dass die Vorgängeräußerung aus Sprechersicht erwartet war.

Mit BEKANNT$_{\textrm{H}}$, KORREKTUR liegt der Bezug auf die aktuelle Äußerung vor, mit KONNEX und ERWARTET$_{\textrm{V/S}}$ auf die Vorgängeräußerung, die dadurch, dass sie als erwartet ausgegeben wird, qualitativ bewertet wird.

Natürlich arbeitet Thurmair mit einer anderen Modellierung der Einzelbedeutungen. Die durch Hypothese 2 abstrakter gefasste Konstellation findet sich in meiner Ableitung aber dennoch wieder: \textit{Doch} leistet aus Diskurssicht den dringlicheren Beitrag, das offene Thema anzugehen, während \textit{auch} im Vergleich eine untergeordnetere Angabe macht, dass diese Proposition für den Sprecher als Begründung für einen anderen Sachverhalt herhält. Im Gegensatz zu Thurmairs Hypothese beschreibe ich nicht allein die Verhältnisse, die sich bei dieser Partikelabfolge einstellen, sondern biete einen Erklärungsvorschlag an. Bei ihrer Untersuchung der MP-Kombinationen bleibt im Grunde bei jeder der fünf Hypothesen die Frage offen, \underline{warum} die Partikelabfolgen die beschriebenen Verhältnisse spiegeln. Warum gehen Partikeln, die sich auf die momentane Äußerung beziehen, Partikeln, die eine qualitative Bewertung der Vorgängeräußerung vor\-nehmen, voran? Dies beantworte ich im vorliegenden Fall mit der direktesten Ableitung gewünschter Diskursziele. Das oberste Ziel von Kommunikation ist der Vorstellung entlang von (\ref{929}), Themen vom Tisch zu entfernen und den cg anzureichern. Die Voraussetzung dafür ist es, die zur Diskussion stehenden Themen zu adressieren, womit einhergeht, sich auf die momentane Äußerung zu beziehen. Diskursstrukturell sekundär ist die Einschätzung kausaler Zusammenhänge, wobei es sich um die qualitative Bewertung handelt. 

Meine Erklärung dient somit auch der Stützung der von Thurmairs recht allgemein formulierten und wenig an konkreten Partikelkombinationen durchgespielten Hypothese.

Im folgenden Abschnitt, der sich mit den assertiven Randtypen der \textit{Wo}-VL- und V1-Deklarativsätze beschäftigt, werde ich mir die Interaktion von zu begründender Proposition und Offenheit der Folgeassertion, die die Begründung darstellt, in meiner Argumentation zu Nutze machen.
\setcounter{equation}{0}
\section{\textit{Wo}-Verbletzt- und Verberst-Deklarativsätze}
\label{sec:Rand}
Die hier betrachteten \textit{Wo}-VL- und V1-Sätze sind Deklarativsätze, in denen \textit{doch auch} auftreten kann (vgl. (\ref{930}), (\ref{931})). Welche interpretatorischen Besonderheiten diese Satztypen aufweisen, wird in den Folgeabschnitten noch detailliert ausgeführt. Für den Moment soll der Hinweis genügen, dass sie kausal (bzw. konzessiv) gelesen werden.

\begin{exe}
	\ex\label{930}
	\scriptsize
	Henry befand sich indes in einem tiefen Schlaf in er von abstruden Dingen träumte.\\
	Er träumte unter anderem davon, wie er auf einem Tisch lag und eine Decke anstarrte.\\
	\emph{Seltsam!} \textbf{\textit{Wo} er sich \underline{doch auch} auf einem solchen befand.}\\
	Ihm gefiel dieser Traum nicht. Er wollte etwas anderes träumen. 				         
	\hfill\hbox{(DECOW14AX)}
	\newline
	\hbox{}\hfill\hbox{(http://www.kurzgeschichten.de/vb/archive/index.php?t-4329.html)}
\end{exe}

\begin{exe}
	\ex\label{931}
	\scriptsize
	Die Direktorin der Zentralmusikschule Eisenstadt, Renate Bedenik, \emph{war sichtlich stolz} über das gelungene Konzert ihrer Musikschüler. 				\textbf{\textit{Legten} \underline{doch auch} zwei davon}, Hans Peter Gradwohl am Klavier und Martin Gruber am Schlagwerk, \textbf{dabei ihre 				öffentliche Abschlussprüfung ab.}				         
	\newline
	\hbox{}\hfill\hbox{(BVZ09/NOV.01705 Burgenländische Volkszeitung, 25.11.2009)}
\end{exe}										      
In (\ref{930}) und (\ref{931}) wären auch \textit{denn}-Sätze denkbar. Um Verwechslungen mit anderen Strukturen zu vermeiden, ist es wichtig, dass die \textit{Wo}-Sätze nicht lokal interpretiert werden (was ggf. die erste Assoziation mit diesem Einleiter ist) und dass die V1-Sätze nicht vorfeldelliptisch sind, d.h. aus Sicht ihrer Argumentstruktur sind sie vollständig.

Die Frage, die diese Strukturen im Rahmen meiner Untersuchung zu einem re\-levanten Betrachtungsgegenstand macht, ist, ob sich die von mir vorgeschlagene Erklärung für die Präferenz der Abfolge \textit{doch auch} aufrechterhalten lässt, wenn man Beobachtungen, die für derartige Randtypen unabhängig gemacht worden sind, hinzunimmt. Sie weisen nämlich einige Besonderheiten auf.
																
\subsection{Obligatorizität, Konzessivität/Kausalität und Ausbleichung}
\label{sec:eig}
Beispielsweise heißt es, \textit{doch} sei in diesen Sätzen obligatorisch bzw. in V1-Sätzen obligatorisch (z.B. \citealt[36]{Winkler1992}, \citealt[1020]{Altmann1993}, \citealt[155/157]{Oennerfors1997}, \citealt[2299]{Zifonun1997}, \citealt[157]{Pittner2011}, \citealt[40/42]{Oppenrieder2013}, \citealt[640]{Thurmair2013}). 

Wie eingangs erwähnt, werden Sätze der Art in (\ref{930}) und (\ref{931}) kausal bzw. konzessiv interpretiert, wobei sich diese Interpretationen auf verschiedenen Ebenen einstellen.

Für beide Satztypen gilt, dass die kausale Interpretation auf epistemischer \is{epistemischer Kausalsatz}oder illokutionärer Ebene \is{illokutionärer Kausalsatz} wirkt (modale Lesart) \is{modaler Kausalsatz} und es sich nicht um Sachverhaltsbegründungen handelt (vgl. auch Kapitel~\ref{chapter:jud}, Abschnitt~\ref{sec:markiert}). Es werden Annahmen, Sprechakte oder Einstellungen begründet, wie z.B. in (\ref{930}), warum es seltsam ist, oder in (\ref{931}), wie der Eindruck beim Sprecher entsteht, dass Renate Bedenik stolz ist. Für die \textit{Wo}-Sätze ist angenommen worden, dass sie auch eine konzessive Komponente haben können (vgl. \citealt[2312-2313]{Zifonun1997}, \citealt{Pasch1999}, \citealt{Guenthner2002}). Was die Konzessivität betrifft, haben allerdings nicht alle Autoren erkannt, was \citealt{Pasch1999} am klarsten beschreibt: Die konzessive Lesart spielt sich nicht auf der modalen, sondern auf der \is{propositionaler Konzessivsatz} propositionalen \is{modaler Konzessivsatz} Ebene ab und betrifft in diesem Sinne eine tiefere Interpretationsschicht (vor allem contra \citealt[2312-2313]{Zifonun1997}, \citealt{Guenthner2002}).

\begin{exe}
	\ex\label{932}
	\scriptsize
	\emph{Es ist schon komisch.} Über Wochen kann der Bär nicht mit einem Betäubungsgewehr überrascht werden. \textbf{\textit{Wo} er \underline{doch} 			angeblich kaum scheu ist und der geneigte Wanderer von ihm als Appetithappen angesehen wird. 		}		         
	\hfill\hbox{(BRZ06/JUL.00738 Braunschweiger Zeitung, 03.07.2006)}
\end{exe}	       											   
Die in (\ref{932}) beteiligte konzessive Relation ist beispielsweise: \glq Obwohl der Bär kaum scheu ist, kann er nicht mit dem Gewehr überrascht werden.\grq {}, die kausale: \glq Weil der Bär kaum scheu ist, wundere ich mich.\grq {}. Die konzessive Lesart ergibt sich sehr klar nicht auf modaler Ebene. Unter Konzessivität auf modaler Ebene fallen Beispiele wie in (\ref{933}).

\begin{exe}
	\ex\label{933}
	Ich will diesen Rock kaufen. Obwohl: Er hat ein Loch.	     
	\newline    
	\hbox{}\hfill\hbox{\citet[436]{Antomo2013}}
\end{exe}
Es liegt eine Infragestellung, Einschränkung oder Zurücknahme der vorherigen Äußerung vor. Ggf. will der Sprecher den Rock auch gar nicht mehr kaufen. In (\ref{934}) hingegen (propositional konzessiv) \is{propositionaler Konzessivsatz} besteht der Wille klar entgegen der Erwartung, dass man Röcke, die Löcher haben, normalerweise nicht kauft (vgl. \citealt[436]{Antomo2013}).

\begin{exe}
	\ex\label{934}
	Ich will diesen Rock kaufen, obwohl er ein Loch hat.	   
	\newline      
	\hbox{}\hfill\hbox{\citet[436]{Antomo2013}}
\end{exe}
In (\ref{932}) wird der Sachverhalt, dass der Bär nicht mit einem Gewehr überrascht werden kann, eindeutig nicht in Frage gestellt, etwa wie in (\ref{935}).

\begin{exe}
	\ex\label{935}
	\scriptsize
	Es ist schon komisch. Über Wochen kann der Bär nicht mit einem Betäubungsgewehr überrascht werden. \#Obwohl: Angeblich ist er kaum scheu $[$...$]$.	         
\end{exe}
Das Pendant zu (\ref{933}) ist in diesem Fall sogar unsinnig, weil der Sachverhalt (Der Bär kann nicht mit einem Gewehr überrascht werden.) vom Sprecher als präsupponiert \is{Präsupposition} ausgegeben wird. Er wundert sich schließlich über ihn.

Für alle hier betrachteten \textit{Wo}-Sätze gilt, dass sie auf modaler Ebene kausal sind. Ob sie zusätzlich ein propositional-konzessives Bedeutungsmoment aufweisen, ist meiner Meinung nach allein von der Art der begründeten Einstellung abhängig. Konzessivität ist beteiligt, wenn z.B. Haltungen wie \textit{erstaunt sein}, \textit{wundern}, \textit{komisch finden} auftreten. Die Einstellungen werden begründet, und sie ergeben sich aber entscheidenderweise aus einem unerwarteten Zusammenstoß von Ereignissen. (\ref{936}) und (\ref{937}) zeigen zwei Beispiele für einen \textit{Wo}-Satz, bei dem keine konzessive Relation im Spiel ist. Begründet wird allerdings die vom Sprecher ausgedrückte Annahme der Möglichkeit, dem Gefallen nachzukommen und die saliente Sache einzurichten, bzw. die aus der rhetorischen Frage abzuleitende Annahme, dass außerhalb von Schlesien niemand Karpfen mit brauner Soße aß.	

\begin{exe}
	\ex\label{936}
	\scriptsize
	\emph{Vielleicht} tut man dir trotzdem den Gefallen und ermöglicht es. \textbf{\textit{Wo} du \underline{doch} so nett darum bittest.}       
	\newline  
	\hbox{}\hfill\hbox{(DECOW2014) (http://www.crystals-dsa-foren.de/archive/index.php/thread-2885.html)}
\end{exe}
	
\begin{exe}
	\ex\label{937}
	\scriptsize
	\emph{Wer} goutierte bis dahin diesseits von Schlesien \emph{schon} Karpfen mit brauner Soße? \textbf{\textit{Wo} es selbst Karpfen blau bei Umfragen 		gerade auf zwei Prozent Zustimmung bringt.	     }
	\newline  
	\hbox{}\hfill\hbox{(HA09/DEZ.02971 Hannoversche Allgemeine, 19.12.2009)}
\end{exe}	
Bei \citet[161]{Oennerfors1997} heißt es, der V1-Satz könne nur kausal gelesen werden, Konzessivität sei nie beteiligt. Als Evidenz führt er den Kontrast in (\ref{938}) und (\ref{939}) an.

\begin{exe}
	\ex\label{938}
	A: Max ist jetzt endgültig ans Bett gefesselt.\\
	B: \textbf{\textit{Wo}} er \textbf{doch} immer so gesund war.	
	\hfill\hbox{\citet[203]{Oppenrieder1989}}
\end{exe}	

\begin{exe}
	\ex\label{939}
	A: Max ist jetzt endgültig ans Bett gefesselt.\\
	B: *\textbf{\textit{War}} er \textbf{doch} immer so gesund.	
	\hfill\hbox{\citet[161]{Oennerfors1997}}
\end{exe}
Önnerfors Annahme zur Interpretation dieses Typs von V1-Satz ist m.E. nicht korrekt. Ich teile zwar sein Urteil in (\ref{939}), halte aber andere Gründe für verantwortlich. Auch meine ich, dass sich der generelle Eindruck, dass V1-Sätze nicht konzessiv gebraucht werden können, begründen lässt.

Wie (\ref{940}) und (\ref{941}) nachweisen, findet man in den Korpusbelegen durchaus V1-Sätze, die neben der modal-kausalen Lesart auch die propositional-konzessive aufweisen.

\begin{exe}
	\ex\label{940}
	\scriptsize
	Die Brand- und Verletzungsgefahr des Elta-Geräts \emph{erstaunte die Berliner Tester sehr}. \textbf{\textit{Trägt} der Elta \underline{doch} wie alle 		anderen Haartrockner das CE-Zeichen und dazu das GS-Zeichen (Geprüfte Sicherheit).   	}
	\hfill\hbox{(HAZ09/OKT.02744 Hannoversche Allgemeine, 19.10.2009)}
\end{exe}
In (\ref{940}) ist der kausale Zusammenhang: \glq Weil der Fön das Zeichen trägt, sind die Tester erstaunt, dass Brand- und Verletzungsgefahr des Gerätes besteht.\grq {} und die konzessive Relation: \glq Obwohl der Fön das Zeichen trägt, besteht Brand- und Verletzungsgefahr.\grq {}

\begin{exe}
	\ex\label{941}
	\scriptsize
	Ich bin klar enttäuscht über das Resultat der FDP. Das schlechte Abschneiden \emph{ist sehr überraschend}. \textbf{\textit{Führten} die Freisinnigen 		\underline{doch} einen super Wahlkampf} – ganz im Gegensatz zu den anderen Parteien. 
	\newline  
	\hbox{}\hfill\hbox{(A08/SEP.09380 St. Galler Tagblatt, 29.09.2008)}
\end{exe}
In (\ref{941}) sind die analogen Relationen: \glq Weil die Freisinnigen einen super Wahlkampf führten, ist das schlechte Abschneiden für den Sprecher sehr überraschend.\grq {} und \glq Obwohl die Freisinnigen einen super Wahlkampf führten, haben sie schlecht abgeschnitten.\grq {} Die konzessive Interpretation derartiger V1-Sätze ist ebenfalls möglich.

Allerdings gibt es durchaus Verwendungsunterschiede, die die Inadäquatheit von Bs Reaktion in (\ref{939}) ableiten: V1-Sätze treten sehr wenig im mündlichen Sprachgebrauch oder dialogisch auf (wenngleich auch dies nicht ausgeschlossen ist $[$vgl. (\ref{942}) bis (\ref{945})$]$), sondern sind dem Schriftmedium zugeordnet (vgl. auch \citealt[157]{Oennerfors1997}).
	
\begin{exe}
	\ex\label{942}
	\scriptsize
	Auf den Vorwurf, viele Schulabgänger seien heute nicht mehr ausbildungsfähig, konterte Frans Thön\-nes kompetent mit der These, dass die Unternehmer den 		Kontakt zu den Schulen suchen sollten. \glqq Holen Sie sich nicht nur Schüler, sondern auch die Lehrer als Praktikanten in die Betriebe. Bringen Sie 		ihnen Wirtschaft bei! Bei den rückläufigen Geburten können wir es uns nicht mehr leisten, dass nur ein einziger Schulabgänger auf der Strecke bleibt. 		\textbf{\textit{Sind} es \underline{doch} die Neugeborenen von heute, die morgen unser Sozialsystem bezahlen müssen.}\grqq{}, so der Staatssekretär 		$[$...$]$.                                                                                           
	\newline  
	\hbox{}\hfill\hbox{(HMP09/MAI.00413 Hamburger Morgenpost, 06.05.2009)}
\end{exe}	
	
\begin{exe}
	\ex\label{943}
	\scriptsize
	Danke für den link Thomas. \textbf{\textit{Räumt} er \underline{doch} mit der hier geäußerten Meinung auf, der Dollar wäre unterbewertet.                                                                                           	}
	\hfill\hbox{(DECOW2014)}
	\newline  
	\hbox{}\hfill\hbox{(http://www.computerbase.de/forum/archive/index.php/t-394561-p-2.html)}
\end{exe}		
	 
\begin{exe}
	\ex\label{944}
	\scriptsize
	Jeden Monat lassen wir eine prominente Person zu Wort kommen, diesen Monat Art Furrer (76), Bergführer, Skilehrer und Hotelier auf der Riederalp.\\
	\newline
	\noindent
	Wie häufig trifft man Sie am Postschalter?\\
	\newline
	\noindent
	Recht oft. \textbf{\textit{Liegt} \underline{doch} die Post in der Bergstation der Grosskabinenbahn}, die zu uns auf die Alp führt. Hat man etwas 			vergessen, helfen die Pöstler immer.			      
	\hfill\hbox{(DECOW2014)}
	\newline  
	\hbox{}\hfill\hbox{(http://personalzeitung.post.ch/de/leute/promis-ueber-die-post/art-furrer-201201)}
\end{exe}

\begin{exe}
	\ex\label{945}
	\scriptsize
	Zitat von: Hans Bergman\\
	18.06.2011 16:49 \#74622\\
	\newline
	\noindent
	Warum so bald? ich würde Ihnen etwas mehr Zeit für die Weiterentwicklung geben.\\
	\newline
	\noindent
	Nee, 500 Jahre reicht. \textbf{\textit{Kommen} diese Herrschaften \underline{doch} alle aus einer Wel}t, in der das Rad und auch der Computer bereits 		erfunden sind. Dementsprechend müssen bestimmnte Dinge nicht erst noch erarbeitet werden.  			      
	\hfill\hbox{(DECOW2014)}
	\newline  
	\hbox{}\hfill\hbox{(http://181209.homepagemodules.de/t29f2-Michael-Schnarch-31.html)}
\end{exe}																	           
Gegen (\ref{942}) könnte man noch einwenden, dass unklar ist, inwieweit es sich hier um eine geplante Rede handelt, die damit konzeptionell schriftlicher Sprache nahekommt, trotz ihrer medialen Mündlichkeit. Die Belege in (\ref{943}) bis (\ref{945}) sind hingegen als konzeptionell mündlich einzustufen.

Unter Bezug auf denselben Aspekt ist auch der Eindruck zu erklären, dass im Falle der V1-Sätze konzessive Interpretationen generell ausbleiben. Dieser Satztyp ist fast ausschließlich schriftsprachlich zu finden und damit auch viel in Zeitungstexten. Die DeReKo-Verteilungen zeigen in Abschnitt~\ref{sec:korp} auch, dass die \textit{Wo}-Sätze weit weniger auftreten als die V1-Sätze, während sich dieses Verhältnis in den DECOW-Daten deutlich annähert. In den DeReKo-Daten treten in meinen Augen schlicht weniger die Einstellungen auf, deren Begründung den konzessiven Aspekt einbringt, sondern eher Annahmen oder rhetorische Fragen, bei denen die Konzessivität keine Rolle spielt (s.o.). Im prinzipiellen Potenzial unterscheiden sich die \textit{Wo}-VL- und V1-Sätze hier aber nicht. Für ihre Interpretation gilt folglich, dass sie auf modaler Ebene kausal wirken (Begründung einer Einstellung) und ggf. (je nach Einstellung) auf propositionaler Ebene konzessiv.

Der Aspekt, der mich im Rahmen meiner Untersuchung zur Abfolge von \textit{doch} und \textit{auch} vor allem interessiert, ist, dass angenommen wurde, dass \textit{doch} in den obigen V1-Sätzen nicht transparent verwendet wird. Genauer vertritt \citet[167]{Oennerfors1997}, dass der Partikel das Element des Widerspruchs fehlt. Er beruft sich auf die \textit{doch}-Zuschreibung von \citet{Ormelius-Sandblom1997} (vgl. (\ref{946})), nach der die ausgedrückte Proposition ein	Fakt ist und sich als eine konventionelle Implikatur \is{konventionelle Implikatur} gegen (eventuelle) Einwände wendet.

\begin{exe}
	\ex\label{946} 
		$\lambda \textrm{p[FAKTp}]$\\
		\textsc{Implikatur}$[\exists \textrm{q[q} \rightarrow \neg \textrm{p}]]$
			\hfill\hbox {\citet[83]{Ormelius-Sandblom1997}}
\end{exe}
\citet[167]{Oennerfors1997} vertritt, die Implikatur liege in den V1-Sätzen nicht vor. Der Sprecher wende sich nicht gegen eine andere Proposition, die ggf. implizit ableitbar ist. Seine Lösung des Problems ist, zu sagen, dass die Implikatur im V1-Satz streichbar ist, weshalb er sie auch für eine \underline{konversationelle} Implikatur \is{konversationelle Implikatur} hält.

Ich denke, dass diese Erklärung aus dem Grund wenig attraktiv ist, da \textit{doch} auch durch \glq härtere\grq {} Bedeutungsaspekte lizensiert werden kann, wie konventionelle Implikaturen, Implikationen \is{Implikation} oder \is{Sprechaktbedingung} Sprechaktbedingungen (vgl. meine Ausführungen in Abschnitt~\ref{sec:doch} dieses Teils sowie in Kapitel~\ref{chapter:jud}, Abschnitt~\ref{sec:doch1}), die man i.d.R. nicht streichen kann. Darüber hinaus werden Implikaturen normalerweise auch eher durch kontextuelle Informationen in Dialogen gestrichen und nicht aufgrund grammatischer Gegebenheiten. Wenngleich ich die Ableitung von Önnerfors nicht teile, lässt sich seine Beobachtung, dass es schwierig scheint, in diesen Sätzen $\neg$p zu motivieren, aber dennoch zunächst einmal annehmen. 

Man sieht in diesem Kontext bereits, dass die Beantwortung dieser Fragen immer auch stark von der jeweils zugrundegelegten Bedeutung von \textit{doch} abhängt. In diesem Sinne ist dieser Satzkontext deshalb auch generellerer Testboden für die Geeignetheit einer \textit{doch}-Modellierung. Es werden Überlegungen dazu angestellt, warum \textit{doch} so wichtig für diese Satztypen ist. Mit der Klärung dieser Frage hängt zusammen, dass man besondere Eigenschaften dieser Sätze ausgemacht hat, die man mit \textit{doch} in Zusammenhang gebracht hat. Aus diesem Grund habe auch ich mich mit Eigenschaften dieser Sätze beschäftigt. Je nach \textit{doch}-Auffassung kommt man für diese Eigenschaften (nicht) auf bzw. ist mit ihnen ggf. auch gar nicht einverstanden. Ich verfolge darüber hinaus auch das Ziel, die \textit{Wo}-VL- und V1-Sätze in diesen Fragen parallel zu behandeln. In der Literatur werden sie zwar zusammen erwähnt (vgl. z.B. \citealt[161]{Oennerfors1997}), detaillier\-tere Untersuchungen beschäftigen sich aber immer nur mit einem von beiden. Ich halte es für lohnenswert, beide Sätze parallel zu betrachten, was nicht heißen soll, dass sie sich nicht in manchen Aspekten voneinander unterscheiden.

Die folgenden Abschnitte beleuchten einige der schon angeführten sowie unerwähnte Eigenschaften, mit denen die beiden Satztypen in Verbindung gebracht worden sind.
	
\subsection{Obligatorizität/Typizität von \textit{doch}}
\label{sec:korp}
Der erste Aspekt, den ich genauer untersucht habe, ist, wie deutlich \textit{doch} in \textit{Wo}-VL- und V1-Sätzen tatsächlich vertreten ist.\footnote{Wenn ich im Folgenden Verteilungen angebe, gilt die Einschränkung, dass ich nur Fälle in die Betrachtung einbezogen habe, in denen die Sätze auch orthografisch selbständige Sätze darstellen, wie in den bisher angeführten Beispielen. Die relevanten Strukturen kommen auch in Nebensatzform vor. Ich glaube nicht, dass für diese Vorkommensweisen anderes gilt, Aufschluss würde hier aber nur ihr Einbezug geben. Die Beschränkung auf die auch orthografisch als solche erkennbaren selbständigen Sätze hat allein den praktischen Grund, die Menge der zu betrachtenden Daten zu reduzieren.} (\ref{947}) zeigt, ob/welche MPn in \textit{Wo}-VL-Sätzen in DeReKo vorzufinden sind.

\begin{exe}
	\ex\label{947} \textit{Wo}-VL-Sätze (nachgestellt) in DeReKo (Tagged C) (exhaustiv)\footnote{Jede Suche ist ggf. auch durch ihre Suchanfrage beschränkt. In diesem Fall sind \textit{Wo}-VL-Sätze ausgeschlossen, die von einem Fragezeichen beendet werden (Anfrage: Wo /s0 MORPH(V IND -INF -PCP) /w0 $<$se$>$ \%$+$w0 \textbackslash?). Die DECOW-Daten zeigen, dass es solche Sätze gibt. Ich habe sie allein aus praktischen Gründen ausgeschlossen, um keine w-Interrogativsätze unter den Ergebnissen zu haben.}\\[-1em]
    \begin{tabular}[t]{|l|l|l|l|}
    \hline
    \textit{doch} & \textit{schon} & \textit{doch sowieso} & keine Partikel\\
    \hline
    129 & 1 & 2 & 7\\
    \hline	 
    \end{tabular}   
\end{exe}
(\ref{948}) bis (\ref{950}) zeigen die Ergebnisse der parallelen Betrachtung von funktional ähnlichen kausalen Nebensätzen.

\begin{exe}
	\ex\label{948} \textit{Denn}-Sätze in DeReKo (Tagged C) (Zufallsstichprobe 300 bereinigt)\footnote{reduziert aus 284702 Treffern}\\[-1em]
    \begin{tabular}[t]{|l|l|l|l|l|l|l|l|}
    \hline
    keine MP & MP & & & & & &  \\
    \hline
    287 & 12 & & & & & & \\
    \hline
    & \textit{ja} & \textit{doch} & \textit{auch} & \textit{wohl} & \textit{ja} & \textit{eben} & \textit{einfach}\\
    \hline
    & 4 & 2 & 1 & 1 & 1 & 1 & 2\\
    \hline	 
    \end{tabular}   
\end{exe}

\begin{exe}
	\ex\label{949} \textit{da}-Sätze (nachgestellt) in DeReKo (Tagged C)\\
	(Zufallsstichprobe 500 bereinigt)\footnote{reduziert aus 260671 Treffern}\\[-1em]
    \begin{tabular}[t]{|l|l|l|l|l|}
    \hline
    keine MP & MP & & & \\
    \hline
    300 & 9 & & & \\
    \hline
    & \textit{ja} & \textit{doch} & \textit{eben} & \textit{sowieso}\\
    \hline
    & 6 & 1 & 1 & 1\\
    \hline	 
    \end{tabular}   
\end{exe}

\begin{exe}
	\ex\label{950} \textit{Zumal}-Sätze in DeReKo (Tagged C) (exhaustiv) \\[-1em]
    \begin{tabular}[t]{|l|l|l|l|l|l|l|}
    \hline
    keine MP & MP & & & & & \\
    \hline
    & \textit{ja} & \textit{doch} & \textit{eben} & \textit{wohl} & \textit{ja auch} & \textit{auch}\\
    \hline
    & 3 & 1 & 2 & 2 & 1 & 22\\
    \hline	 
    \end{tabular}   
\end{exe}
Funktionale Ähnlichkeit meint hier, dass für diese Konnektoren angenommen wurde, dass sie modale Interpretationen \is{modaler Kausalsatz} aufweisen. \textit{Denn} kann (so \citealt[320]{Volodina2010}) nur modal gelesen werden bzw. favorisiert diese Interpretationsweise (\citealt[270]{Bluehdorn2006}; \citeyear[29]{Bluehdorn2008}). Ebenso wird das (nachgestellte) \textit{da} mit dieser Lesart assoziiert (vgl. z.B. \citealt[335]{Pasch1983}, \citealt[182]{Rosengren1987}, \citealt[2303]{Zifonun1997}, \citealt[397]{Pasch2003}; in \citealt[411, 415]{Frey2012} zählen \textit{da}-Sätze zu den \textit{peripheren Nebensätzen}, \is{peripherer Nebensatz} mit denen $[$wenn auch eher unausgesprochen$]$ die modalen Lesarten in Verbindung gebracht werden). Zu \textit{zumal}-Sätzen finden sich in der Literatur sehr wenige Äußerungen. In \citet[6]{Bluehdorn2014} zählen sie zu den \textit{peripheren Nebensätzen}. \citet[397]{Pasch2003} ordnen \textit{zumal} als nicht-propositionalen Konnektor ein, der folglich oberhalb der Sachverhalts\-ebene verknüpft. In \citet[81]{Heidolph1981} wird \textit{zumal}-Sätzen die gleiche Funktion zugeschrieben wie \textit{wo}+\textit{doch}-Sätzen (dagegen vgl. \citealt[78-79]{Borst1985}).

Im Vergleich zu (\ref{947}) sind die Verteilungen in (\ref{948}) bis (\ref{950}) deutlich verschieden. In \textit{denn}-,\textit{ da}- und \textit{zumal}-Sätzen hat man es hinsichtlich der Verteilung \textit{MP} vs. \textit{keine MP} quasi mit genau gespiegelten Verhältnissen zu tun. Partikeln scheinen generell eher wenig aufzutreten und diese Verteilungen spiegeln vermutlich genau die Verhältnisse, mit denen MPn überhaupt in Assertionen vorkommen. Vor diesem Hintergrund müssen die Verhältnisse im \textit{Wo}-Satz erst recht als besonders gelten.

Aufgrund anderer Abfragemöglichkeiten habe ich in den Webdaten in DECOW2014 verglichen, welche MPn auftreten (über exhaustive Suchen nach den Sätzen mit den Partikeln). Angaben für \textit{Wo}-Sätze ohne MPn liegen deshalb nicht vor.

\begin{exe}
	\ex\label{951} \textit{Wo}+VL-Sätze (nachgestellt) in DECOW14AX\footnote{ Anfrage: $[$word= \glqq Wo\grqq{} $][]\lbrace$0,4$\rbrace[$word= \glqq ja\grqq{}$][ ]\lbrace$0,4$\rbrace[$pos= \glqq VVFIN\grqq{}$]$} \\
	\scriptsize
    \begin{tabular}[t]{|l|l|l|l|l|l|l|l|l|l|l|}
    \hline
    \textit{doch} & \textit{halt} & \textit{eben} & \textit{auch} & \textit{doch einfach} & \textit{doch eh} & \textit{doch sowieso} & \textit{doch auch} & 	\textit{ja} & \textit{ja auch} & \textit{ja doch} \\
	\hline
    605 & - & 1 & 5 & 1 & 6 & 2 & 20 & 17 & 3 & 1\\
    \hline	 
    \end{tabular}   
\end{exe}
Die Webdaten geben hier folglich kein anderes Bild ab als die Daten aus DeReKo. Es ist davon auszugehen, dass \textit{doch} sowieso häufiger vorkommt als die Kombinationen aus (\ref{951}). Auf deren Unterrepräsentiertheit sollte man folglich nicht schließen. Auch vermute ich, dass \textit{ja} häufiger verwendet wird als \textit{doch}, was den Kontrast noch verstärkt. Genauso ist davon auszugehen, dass MP-lose Assertionen normalerweise überwiegen (was auch (\ref{948}) bis (\ref{950}) nahelegen). Für \textit{Wo}-VL-Sätze bestätigt sich in der Verwendung folglich, dass \textit{doch} zwar nicht obligatorisch ist, aber sehr typisch, z.T. in Kombination, auftritt.

Ein Vergleich der Trefferzahl einer exhaustiven Suche nach \textit{doch} und \textit{ja} in V1-Sätzen in DeReKo ergibt ein unmissverständliches Übergewicht von \textit{doch}.

\begin{exe}
	\ex\label{951a} V1-Sätze in DeReKo (Tagged C) (exhaustiv)\footnote{ Anfrage: (MORPH(V IND -INF -PCP) /w0 $<$sa$>$) /s0 doch}\\[-1em]
    \begin{tabular}[t]{|l|l|}
    \hline
    \textit{doch} & \textit{ja}\\
    \hline	 
    3685 (57 \textit{doch auch}) & 22 (6 x \textit{ja auch}, 2 x \textit{ja eh}, 1 x \textit{ja sowieso})\\    
    \hline
    \end{tabular}   
\end{exe}
Gleiches gilt für eine exhaustive Suche nach V1-Sätzen in einem Teilkorpus von DECOW2014 (vgl. (\ref{952})).

\begin{exe}
	\ex\label{952} V1-Sätze in DECOW2014AX (Teilkorpus) (exhaustiv)\footnote{Anfrage: $<$s$>[$pos= \glqq VVFIN\grqq{} $][]\lbrace$0,4$\rbrace[$word= \glqq doch\grqq{} 		$]$}\\[-1em]
    \begin{tabular}[t]{|l|l|}
    \hline
    \textit{doch} & \textit{ja}\\
    \hline	 
    698 (2 x \textit{doch wohl}, 2 x \textit{ja doch}, 15 x \textit{doch auch}) & 16 (3 x \textit{ja auch}, 2 x \textit{ja doch})\\
    \hline
    \end{tabular}   
\end{exe}
Vorausgesetzt, \textit{doch} und \textit{ja} stehen nicht sowieso in dem Frequenzverhältnis zuein\-ander, wie es sich hier für \textit{doch}- und \textit{ja}-V1-Sätze einstellt, scheint behauptbar, dass \textit{doch} in diesem Satztyp deutlich überwiegt. Die \textit{ja}-Treffer, die auftreten, können aber nicht durchweg als älteren Sprachstufen zugehörig angenommen werden (vgl. (\ref{953}) und (\ref{954})) (contra \citealt[158]{Oennerfors1997}) (vgl. auch schon meine Belege in Kapitel~\ref{chapter:jud}, Abschnitt~\ref{sec:nonkan}).

\begin{exe}
	\ex\label{953}
	\scriptsize
	Grundsätzlich könnt das durchaus Sinn machen. \textbf{\textit{Geht} es bei Monopoly \underline{ja} darum}, Gebäude und Orte zu kaufen, zu erhalten und 		dafür Miete zu kassieren. 			      
	\hfill\hbox{(DECOW2014AX)}
	\newline  
	\hbox{}\hfill\hbox{(http://locationmarketing.at/)}
\end{exe}
								          
\begin{exe}
	\ex\label{954}
	\scriptsize
	Neben den bereits genannten Lutzmannsburg, Großwarasdorf und Sieggraben würden mit Rattersdorf, Mannersdorf, Steinberg und Tschurndorf gleich weitere 		vier Teams mittendrin statt nur dabei sein. \textbf{\textit{Trennen} den Neunten, Rattersdorf, \underline{ja} lediglich sechs Punkte vom Vorletzten 		Großwarasdorf.} 			      
	\newline  
	\hbox{}\hfill\hbox{(BVZ08/DEZ.00744 Burgenländische Volkszeitung, 03.12.2008, S. 71}
\end{exe} 
Bei beiden Satztypen sprechen die Verteilungen folglich dafür, dass es berechtigt ist, die Frage zu stellen, warum die MP \textit{doch} für diese Sätze so wichtig ist und damit auch die nächste Frage aufzuwerfen, welchen Beitrag sie in ihnen leistet.
	
\subsection{Unkontroverse/Thematizität}
\label{sec:unkontr}
Für beide Satztypen ist angenommen worden, dass sie den Sachverhalt, auf den sie sich beziehen, als unkontrovers, bekannt, ein Faktum oder Hintergrundinformation markieren,  ihn m.a.W. präsupponieren \is{Präsupposition} (wenngleich nicht alle Arbeiten diesen konkreten Begriff verwenden): Bei \citet[90]{Kwon2005} heißt es z.B. über \textit{Wo}-VL- und V1-Sätze, der Sachverhalt sei unkontrovers. Über erstere schreibt \citet[43]{Winkler1992}, ihr Inhalt sei bekannt. \citet[145]{Pasch1999} schreibt, er sei eine Tatsache, ein Faktum, evident, er werde als dem Adressaten bekannt ausgegeben. Ähnlich liest man bei \citet[236]{Eroms2000} von einem \glqq diskursiv akzeptierten Tatbestand\grqq{} und bei \citet[315]{Guenthner2002} von Hintergrundinformation, die als evident ausgelegt wird, evidentem/präsupponiertem Inhalt (S. 325) und der Unmöglichkeit der Rhematizität des Begründungsinhalts (S. 325) (vgl. ähnliche Attribute auch bei \citealt[134, 135, 148]{Guenthner2007}). Über die V1-Sätze schreibt \citet[1020]{Altmann1993}, der mit ihnen ausgedrückte Sachverhalt sei unkontrovers und akzeptiert. Auch \citet[171]{Pittner2011} hält ihn für einen allgemein akzeptierten Grund.

Mein im Folgenden ausgeführter Punkt ist, dass ich mich für beide Satztypen gegen diese Einschätzungen aussprechen möchte. Ich werde meine Argumente für beide Typen getrennt darlegen. Meine Argumentation gegen den präsupponierten Status des Sachverhalts ist auch durch die Sicht bedingt, dass in meiner Modellierung der \textit{doch}-Bedeutung der Aspekt von Präsupponiertheit nicht vorhanden ist. Ich bin der Meinung, dass man ihn nicht zur Bedeutung von \textit{doch} machen muss. 

\subsubsection{\textit{Wo}-Verbletzt-Sätze}
Beispielsweise frage ich mich, ob die Eigenschaft, unkontrovers, bekannt, hintergrundierend, evident, präsupponiert zu sein, nicht zu einem gewissen Grad auf jeden kausalen Nebensatz zutrifft. Sicherlich sind Unterschiede in der Interpretation und Verwendung verschiedener Kausalsätze gemacht worden, vermutlich würde aber keiner Fragliches oder Spekulatives zum Inhalt eines Satzes machen, der als Begründung intendiert ist.

Ferner geht die Zuschreibung dieser Eigenschaft nicht gut mit der Annahme einher, dass die \textit{Wo}-Sätze modal gelesen werden und somit subjektiviert sind. Man geht davon aus, dass sich die Kausalität dieser Sätze immer auf epistemi\-scher oder illokutionärer Ebene abspielt. Sie können Annahmen, Einstellungen oder Sprechakte begründen. Sachverhaltsbegründungen sollten hingegen nicht möglich sein. Ein \textit{Wo}-Satz kann deshalb z.B. nicht die Antwort auf eine Ergänzungsfrage sein (vgl. (\ref{955})).

\begin{exe}
	\ex\label{955}
	A: Warum kommt du nicht mit essen?\\
	B: *\textbf{Wo} ich keinen Hunger hab.	 			      
	\hfill\hbox{\citet[325]{Guenthner2002}}\footnote{Günthner selbst vertritt eine andere Erklärung der Inakzeptabilität des \textit{Wo}-Satzes in diesem 		Kontext.}
\end{exe} 
Gibt man einen Sachverhalt stets als Faktum aus, würde es auch naheliegen, damit einen anderen Sachverhalt zu begründen. Ich sehe keinen Grund, warum mit einem als real ausgegebenen Sachverhalt ausschließlich subjektive Einschätzungen begründet werden sollten. Von den vier Möglichkeiten der Verteilung von Objektivität und Subjektivität auf den begründeten und begründenden Sachverhalt erscheint mir die objektive Begründung einer subjektiven Wahrnehmung am unplausibelsten, weshalb ich die Beschränkung des kausalen \textit{Wo}-Satzes auf dieses Verhältnis nicht akzeptieren kann.

\citet{Guenthner2002} leitet aus der Annahme, der Inhalt der Sätze sei stets präsupponiert und evident, ab, warum \textit{Wo}-Sätze sich nicht als Antwort auf eine w-Frage eignen (vgl. (\ref{955})). Der Kontext würde die Rhematizität des Inhalts nahelegen. Auf dieselbe Art erklärt sie die Ungrammatikalität von (\ref{956}) und (\ref{957}).

\begin{exe}
	\ex\label{956}
	*ich heirate ihn, wo ich ihn liebe und nicht, wo er Geld hat.
	\newline 	 			      
	\hbox{}\hfill\hbox{\citet[325]{Guenthner2002}}
\end{exe}
\vspace{-0.65cm}
\begin{exe}
	\ex\label{957}
	*Ich habe deshalb das Fenster geschlossen, \textbf{\textit{wo}} es \textbf{doch} so verbrannt gerochen hat. 			      
\end{exe}
Hier zeigt sich, dass \textit{wo}-Sätze nicht im Skopus der Negation stehen können und es auch keine \textit{deshalb-wo}-Konstruktion geben kann.

Der Grund für den Status der Daten in (\ref{955}) bis (\ref{957}) ist m.E. ein anderer als die unmögliche Kodierung von Rhematizität. (\ref{955}) habe ich schon derart beschrieben, dass der Kontext die propositional-kausale Lesart fordert, die \textit{wo} nicht aufweist. Und genauso lassen sich die beiden weiteren Beispiele erklären. Die \textit{deshalb}-Struktur wird immer propositional interpretiert und auch in (\ref{956}) liegt eine Sachverhaltsbegründung vor (vgl. \citealt[143]{Pasch1999} für weitere Evidenz der ausgeschlossenen propositionalen Lesart). Mit der Bekanntheit der Proposition muss/kann der Ausschluss auch nicht begründet werden, weil \textit{denn}, für das angenommen wird, dass es sich auf neue Information bezieht, hier ebenfalls nicht auftreten kann. Auch \textit{denn} scheint aber immer/favorisiert nicht-propositional interpretiert zu werden – was folglich die Datenlage erklärt.

Weitere Evidenz für die Annahme, dass der \textit{wo}-Satz Hintergrundinformation beisteuert, sieht \citet{Guenthner2002} im Dialog in (\ref{958}).

\begin{exe}
	\ex\label{958} HAUSRENOVATION (Bodensee)\\
	43Dora: mhm. des geht ECHT langsam. (–)\\
	44 \hspace{0.5cm}nobwohl wir VIE:L zeit rein$[$stecken.$]$\\
	45Ute:\hspace{0.5cm}			         $[$(           )$]$\\
	46Dora: JE.DES. WOCHENENDE machen wir dran rum! 
	\newline	 				      
	\hbox{}\hfill\hbox{\citet[331]{Guenthner2002}}
\end{exe}
Sie geht davon aus, dass \textit{wo}-Konstruktionen durch \textit{obwohl}-Konstruktionen zu ersetzen sind, aber die umgekehrte Ersetzbarkeit nicht gegeben ist. I.E. gibt es folg\-lich weitere Restriktionen für konzessive \textit{wo}-Strukturen, und zwar, wenn ihr Inhalt neue Information ist. (\ref{958}) ist Günthner zufolge ein Fall, in dem anstelle des \textit{obwohl}-Satzes kein \textit{wo}-Satz eingesetzt werden kann. Als Grund führt sie an, der \textit{obwohl}-Satz könne hier keine Hintergrundinformation liefern. Wiederum liegt die Inakzeptabilität des \textit{wo}-Satzes in meinen Augen aber nicht daran, dass er neue Information kodieren muss, sondern ist darauf zurückzuführen, dass sich die modale Begründung einer Einstellung/Annahme hier nicht gut anbietet. Der \textit{wo}-Satz wird aber akzeptabel, wenn man den ersten Satz so liest, dass sich die Person über die Langsamkeit z.B. aufregt oder wundert, und der \textit{wo}-Satz diese Haltung begründet. Die \glq konzessiven\grq {} \textit{wo}-Sätze können prinzipiell nicht verwendet werden, wenn man die modale Begründung nicht motivieren kann.

Betrachtet man \textit{Wo}-VL-Daten, finden sich keine Beispiele, die die Ansicht Günth\-ners zum Status des Sachverhalts im \textit{Wo}-Satz in dem Sinne widerlegen, dass man sagen müsste, ein ausgedrückter Sachverhalt kann kein Faktum/keine Hintergrundannahme etc. sein. Dies liegt aber daran, dass jede assertierte Information akkommodiert \is{Akkommodation} werden kann. Es ist ebenso schwierig, assertive Kontexte zu finden, in denen \textit{ja} nicht stehen kann. Auf der Basis von Belegen halte ich die Annahme deshalb für schwer widerlegbar. Es gibt aber andererseits auch keinen Grund, die Inhalte stets als faktisch zu lesen. In den \textit{Wo}-Sätzen treten nicht ausschließlich absolute Fakten oder unumstößliche Tatsachen auf. Es kann sich genauso plausibel nur um eine Ansicht oder Einschätzung des Sprechers handeln, die im gegebenen Kontext neu ist.

\begin{exe}
	\ex\label{959}
	\scriptsize
	Warum?\\
	Warum ist es Wichtig, das sich die Eheleute \glqq lieb haben\grqq{} und viele viele gemeinsame Kinder zeugen?\\
	Wo ist der Nutzen?\\
	Reicht es nicht, wenn se einfach nur nebeneinander in einem Haus leben, ohne im extrem mit einander zu Reden?\\
	Und Warum ist dafür die Liturgie nötig? \textbf{\textit{Wo} es \underline{doch} reicht, wenn man den beiden einfach ihre Freiheiten lässt (die ihnen 		die Liturgie nimmt)?} 	
	\hfill\hbox{(DECOW2014)}
	\newline
	\hbox{}\hfill\hbox{(http://www.ulisses-forum.de/archive/index.php/t-8220.html)}
\end{exe}
 
\begin{exe}
	\ex\label{960}
	\scriptsize
	Auf die Gefahr hin hier geschlagen zu werden: boah, sind die beide häßlich\\
	\textbf{\textit{Wo} es \underline{doch} so endlos geile Käfer gibt}\\
	*duckundwech*			
	\hfill\hbox{(DECOW2014)}
	\newline
	\hbox{}\hfill\hbox{(http://www.passat35i.de/archive/index.php/t-16223.html)}
\end{exe}					                 

\begin{exe}
	\ex\label{961}
	\scriptsize
	Anrufe!! Er hatte heute gar nicht angerufen...\\
	Warum hatte er mich nicht angerufen??? \textbf{\textit{Wo} ich \underline{doch} anfing ihn ganz nett zu finden!!!!!!!!!!} Ich konnte es mir schon 			denken warum:\\
	Nach den 2 Tagen wo wir miteinander geredet hatten, hatte er mich schon satt...\\
	War ja klar, dass er mich verarscht hatte... Wer würde mich schon lieben??\\
	NIEMAND!!!!!!! 			
	\hfill\hbox{(DECOW2014)}
	\newline
	\hbox{}\hfill\hbox{(http://www.rockundliebe.de/liebesgeschichten/liebesgeschichten\_1474\_m.php)}
\end{exe}										  
Ein anderer Punkt, der für mich gegen die vorgenommenen Eigenschaftszuschreibungen (Hintergrund, Bekanntheit, Faktum, Unkontroversität) spricht, ist auch, dass man im \textit{Wo}-Satz das Vorkommen von \textit{ja} erwarten würde, als beliebte Partikel an der Stelle, weil diese unmissverständlich Bekanntheit, Faktenorientierung etc. kodiert. Die Verteilungen in (\ref{947}) und (\ref{951}) zeigen aber, dass \textit{ja} hier nicht auffällig beliebt ist.

Zudem bin ich nicht der Meinung, dass \textit{doch} überhaupt inhärent diese Bedeutungsaspekte mitbringt. Nach meiner Modellierung setzt \textit{doch} einen Kontextzustand voraus, in dem der Sachverhalt, auf den sich die MP-Äußerung bezieht, schon zur Diskussion steht. Die von einigen Autoren ins Feld geführten Bedeutungsaspekte des \textit{Wo}-Satzes sind mit dieser Bedeutung zwar nicht inkompatibel, aber in ihr auch nicht explizit vertreten. 

In einer empirischen Studie, in der \citet{Doering2014} die Korrelation von be\-stimmten Diskursrelationen und MPn untersucht, findet sich keine Evidenz für eine Assoziation von \textit{doch} mit Relationen wie HINTERGRUND und EVIDENZ. 

In einer Korpusstudie (Protokolle von Parlamentsreden von Helmut Kohl) be\-stimmt sie die Häufigkeit des Auftretens einer MP in Verbindung mit einer be\-stimmten Diskursrelation relativ zu einem Erwartungswert für das unabhängige Auftreten der untersuchten \is{Diskursrelation} Diskursrelationen. 

Sie stellt fest, dass \textit{ja} häufiger als erwartet mit den Relationen HINTERGRUND, EVIDENZ und GRUND auftritt, und \textit{doch} in den Relationen JUSTIFY, EVALUATION, INTERPRETATION, MOTIVATION und EVIDENZ. Während die Korrelationen von \textit{ja} zu der dieser Partikel zugeschriebenen Bedeutung (Döring zufolge Bekanntheit und Unkontroverse) passen, hält sie die auftretenden Relationen bei \textit{doch} (Bedeutung: Bekanntheit, Unkontroverse, Kontrast) für unerwartet. Sie erklärt sie über manipulative Verwendungen, was sie aber nur für EVIDENZ ausführt. 

Das Ergebnis spricht für mich gerade dafür, dass \textit{doch} die Komponente von Bekanntheit/Unkontroverse nicht aufweisen muss: Während ein evidenter Sach\-verhalt plausiblerweise auch als bekannt/unkontrovers ausgegeben wird, ist eine Rechtfertigung stets sprechergebunden und muss nicht Einigkeit voraussetzen. Eine Evaluation muss erst recht keine Einigkeit voraussetzen: Es handelt sich um eine subjektive Einschätzung und es ist auch gar nicht beabsichtigt, dass der Adressat diese Ansicht teilt. Nimmt der Sprecher eine Ausdeutung des zuvor Gesagten vor (INTERPRETATION), muss diese Information auch keineswegs bekannt oder unkontrovers sein. Die Bedeutungskomponenten, die gerade die Partikel \textit{ja} auszeichnen, scheinen mir nicht das Auftreten von \textit{doch} mit den ge\-nannten Diskursrelationen zu motivieren. 

Die Frage ist, ob der Aspekt von Widerspruch/Kontrast, der in meiner Modellierung durch das geforderte offene Thema vertreten ist, mit diesem Ergebnis zu motivieren ist. Im Falle der Relation JUSTIFY ist diese Argumentation denkbar: Eine Rechtfertigung hat einen Auslöser, d.h. dass das Thema (aus irgendwelchen Gründen) schon im Raum steht, scheint sehr plausibel. Bei der Relation MOTIVATION soll der Adressat zu einer Handlung bewegt werden. In der Erklärung von \citet[89]{Doering2014} tritt auch hier der Aspekt von Bekanntheit/Unkontroverse nicht auf. Dies bietet sich auch wieder nicht an. Sie schreibt hier vor allem über die Verwendung in Imperativen, bei denen generell qua Sprechaktbedingung davon auszugehen ist, dass ihr Inhalt nicht bereits bekannt ist. Für die Imperative, die über die Relation der Motivation in den Diskurs eingebunden sind, lässt sich gut annehmen, dass die Handlungsaufforderung verstärkt wird, indem sie (expliziter als Direktive überhaupt) voraussetzen, dass fraglich ist, ob der Angesprochene die Aktion ausführen wird (vgl. auch meine Ausführungen in Abschnitt~\ref{sec:direktive}). Für die anderen drei Relationen, bei denen sie Korrelationen mit \textit{doch} feststellt (EVALUATION, INTERPRETATION, EVIDENZ), scheint mir ein vorausgesetztes offenes Thema auch nicht direkt zu motivieren. Kompatibel ist die Situation sicherlich, ausgeschlossen sind Bekanntheit/Unkontroverse aber auch nicht.

Zu ihrer eigenen Verwunderung stellt Döring keine Verbindung zwischen dem Vorkommen von \textit{doch} und den Relationen KONTRAST, CONCESSION und ANTITHESE fest (\citeyear[89]{Doering2014}). Sie erklärt dies tentativ über das vorliegende Genre: In den Parlamentsreden sei sowieso klar, dass der Hörer eine andere Meinung vertrete.\footnote{Als potenziell beeinflussenden Faktor führt sie auch an, dass die Daten zum einen auf einen einzigen Sprecher zurückgehen, und zum anderen, dass die Entscheidungen über die vorliegenden Diskursrelationen eine subjektive Einschätzung bleiben (vgl. \citeyear[90]{Doering2014}).}

In einer weiteren experimentellen Studie bestätigt sich der Aspekt der Korpusstudie, dass \textit{doch} keine Korrelation mit Hintergrundinformation eingeht. In als BACKGROUND- bzw. JUSTIFY-Kontexten angelegten Umgebungen wie in (\ref{962}) hatten die Testanten die Wahl zwischen \textit{ja}, \textit{doch} und (als Filler) betontem \textit{schon}.

\begin{exe}
	\ex\label{962}
	Wenn Ganztagsschulen eingeführt werden, verlieren Musikschulen und Sportvereine viele Mitglieder.\\
	\newline
	\noindent
	BACKGROUND: In Musikschulen machen Kinder \hrulefill \ die größte Gruppe \\
	der Mitglieder aus.\\
	JUSTIFY: Dieser Aspekt muss \hrulefill  \ mal in 
	den Vordergrund gerückt werden.\\
	\newline
	\noindent
	Ein solcher Mitgliederschwund ist für diese Einrichtungen verheerend!	
	\newline
	\hbox{}\hfill\hbox{\citet[90-91]{Doering2014}}
\end{exe}		
Das Ergebnis ist, dass \textit{ja} häufiger als erwartet im Hintergrundkontext und weniger häufig als erwartet im Rechtfertigungskontext gewählt wurde. \textit{Doch} wurde genau entgegengesetzt öfter als erwartet zusammen mit der Relation der Rechtfertigung und seltener als erwartet mit der Relation des Hintergrundes ausgewählt. Die beiden Partikeln werden folglich jeweils in einem der beiden Kontexte bevorzugt. Wie ich oben bereits ausgeführt habe, sehe ich keinen Grund, davon auszugehen, dass Rechtfertigungen stets mit den Bedeutungsaspekten Bekanntheit und Unkontroverse einhergehen.
 
Ein letztes Argument gegen die Annahme, der \textit{Wo}-Satz verweise stets auf bekannte Information, ist für mich die Tatsache, dass Strukturen der Art in (\ref{963}) bis (\ref{965}) ein sehr typisches Muster in den Belegen sind.

\begin{exe}
	\ex\label{963}
	\scriptsize
	Wieso müssen Agenturpartys immer am Donnerstag sein? \textbf{\textit{Wo} \underline{doch} jeder weiß, dass Donnerstags die neuen Filme anlaufen.}	
	\hfill\hbox{(DECOW2014AX)}
	\newline
	\hbox{}\hfill\hbox{(http://www.ankegroener.de/anke1/pasdeblog/blogarchiv/september2002.html)}
\end{exe}

\begin{exe}
	\ex\label{964}
	\scriptsize
	Vampire Hunter D:\\
	Das war jetzt aber sehr fies. \textbf{\textit{Wo} \underline{doch} jeder weiß, daß das gar nicht zu schaffen ist!}		
	\hfill\hbox{(DECOW2014AX)}
	\newline
	\hbox{}\hfill\hbox{(http://www.comicforum.de/archive/index.php/t-88976.html)}
\end{exe}				               

\begin{exe}
	\ex\label{965}
	\scriptsize
	Kein leichtes Thema: Paranoia ist ein Hirngespinst; und die Schwierigkeit des Autors besteht deshalb insbesondere darin, den Leser bei der Stange zu 		halten. \textbf{\textit{Wo} der \underline{doch} weiß, dass die Dinige, die geschehen, lediglich einen innerpsychischen Wahrheitswert für die 				Betroffenen haben.}
	\newline
	\hbox{}\hfill\hbox{(NUN08/JUL.00391 Nürnberger Nachrichten, 04.07.2008)}
\end{exe}				               		                            
In den DECOW-Daten machen diese Strukturen 15\% der \textit{Wo-doch}-Belege aus (93 x eine Form von \textit{wissen} + Nebensatz, davon 63 x eingeleitet durch \textit{jeder weiß}). In einer Stichprobe von 500 Sätzen mit finiten, lexikalischen Verben treten in DECOW2014AX sechs Vorkommen von \textit{wissen} auf. Es ist folglich nicht anzunehmen, dass der Anteil der Strukturen mit \textit{wissen} auf die analoge Häufigkeit dieses Verbs zurückzuführen ist. Handelt es sich um ein Muster, finde ich es auch in diesem Unterfall der Verwendung von \textit{Wo}-Sätzen schwierig, abzuleiten, dass \textit{doch} hier Hintergrund/Bekanntheit \is{Hintergrund} markiert. Der Inhalt der Sätze transportiert dann schließ\-lich genau die Bedeutung, die die MP beisteuern soll. Es gibt andere Beispiele von MP-Charakterisierungen, bei denen derart vorgegangen worden ist (zur Kritik vgl. \citealt[380]{Ickler1994}) (vgl. (\ref{966}) und (\ref{967})).

\begin{exe}
	\ex\label{966}
	Sie wissen \textbf{ja}, daß er nächste Woche operiert wird.\\
	\textit{ja}: Bekanntheit des Sachverhalts für den Hörer
\end{exe}	

\begin{exe}
	\ex\label{967}
	Da kann man \textbf{halt} nichts machen.\\
	\textit{halt}: Einsicht des Sprechers in die Unabänderlichkeit des geäußerten Sachverhalts
	\hfill\hbox{\citet[380]{Ickler1994}}
\end{exe}
Es darf zumindest als ungeschickt gelten, anhand derartiger Beispiele die Bedeutung der Partikeln motivieren zu wollen.\\
\newline
\noindent
Auch für die V1-Sätze gilt, dass einige Aspekte die Annahme in Frage stellen, dass ihr Inhalt stets unkontrovers, thematisch und präsupponiert ist.

\subsubsection{Verberst-Sätze}
Für diese Sätze gilt ebenfalls der Einwand, dass unklar ist, ob \textit{doch} die besagten Bedeutungsanteile tatsächlich kodiert und sein typisches Auftreten durch ihr Vorliegen deshalb motiviert ist.

Schon für die \textit{Wo}-Sätze habe ich festgestellt, dass es schwierig erscheint, explizit zu widerlegen, dass die Sätze nicht auf Fakten verweisen. Bei den V1-Sätzen ist dies noch weiter erschwert, da sie fast ausschließlich in schriftsprachlichen Kontexten vorkommen. In Zeitungskontexten beispielsweise werden sowieso meist Fakten berichtet. Ich sehe allerdings auch keinen Grund anzunehmen, dass die Inhalte deshalb stets als bekannt oder hintergründig eingestuft werden müssen. Für meine Begriffe handelt es sich bei (\ref{968}) und (\ref{969}) nicht um Inhalte, die der Leser in jedem Fall bereits weiß und deshalb unzweifelbar hinnimmt oder die längst akzeptiert sind. Es bietet sich ebenso die Interpretation an, dass die Inhalte neu sind ((\ref{968})) oder es sich lediglich um eine Annahme des Schreibers handelt ((\ref{969})).
	
\begin{exe}
	\ex\label{968}
	\scriptsize
	Rund um die Weihnachtsfeiertage gab es im Land um Hollabrunn Büro noch einen weiteren Grund zum Feiern. \textbf{\textit{Konnte} man \underline{doch} 		dem Regionalmanager Didi Jäger zum Geburtstag gratulieren.}
	\newline
	\hbox{}\hfill\hbox{(NON09/JAN.01979 Niederösterreichische Nachrichten)}
\end{exe}	

\begin{exe}
	\ex\label{969}
	\scriptsize
	Aber warum eigentlich eine Himbeere? Tut man ihr damit nicht Unrecht und kratzt an ihrem Image? \textbf{\textit{Schmeckt} sie \underline{doch} wirklich 	lecker} – und süß. Wie Erfolg. Walle, äh ... \glqq WALL-E\grqq{} weiß, wovon ich rede.
	\hfill\hbox{(BRZ09/FEB.12094 Braunschweiger Zeitung, 25.02.2009)}
\end{exe}
Ein anderer Einwand betrifft die vollkommentarische Natur \is{vollkommentarisch} von V1-Deklarativ\-sätzen. 

Es gibt Arbeiten, die sich mit den generellen Eigenschaften von V1-Deklara\-tivsätzen beschäftigen. Ich bin der Meinung, dass die Annahme, es sei die Natur der begründenden V1-Sätze, stets unkontrovers, bekannt etc. zu sein, einigen grundsätz\-lichen Überlegungen zu V1-Deklarativsätzen zuwiderläuft.

In einer Typologie deutscher V1-Deklarativsätze reiht sich dieser Typ in die Fälle in (\ref{970}) ein. 

\begin{exe}
	\ex\label{970} V1-Deklarativsatztypen im Deutschen\\[-1em]
		\begin{xlist}	
			\ex\label{970a} narrative V1-Deklarative \is{narrativer V1-Deklarativsatz}
				\begin{xlist}
					\ex\label{970x} Hab ich ihr ganz frech noch en Kuß gegeben.
					\ex\label{970y} $[$Die Tübinger mögen sowas$]$. Sprach die Künstlerin hinterher erfreut-verwundert: ...					
				\end{xlist}
			\ex\label{970b} aufzählend-reihende V1-Deklarative \is{aufzählend-reihender V1-Deklarativsatz}
				\begin{xlist}
					\ex\label{970s} Bleibt ein dritter Einwand, nicht weniger gravierend.
					\ex\label{970t} $[$Die Tendenz / geht ... / nach unten.$]$ Kommt noch hinzu, ...				
				\end{xlist}
			\ex\label{970c} inhaltlich-begründende V1-Deklarative \is{inhaltlich-begründender V1-Deklarativsatz}\\
				$[$Sein Tod bewegt viele.$]$ Hatte doch seine Ära den Wiederaufstieg / ... / begründet.   	
				\hfill\hbox {\citet[216]{Reis2000} [nach \citet[99-184]{Oennerfors1997}]}
		\end{xlist}
\end{exe}
In diversen Arbeiten ist angenommen worden, dass V1-Sätze (und zwar auch nicht-deklarative/assertive) vollrhematisch \is{vollrhematisch} bzw. vollfokussiert \is{vollfokussiert} sind (vgl. \citealt{Rosengren1992} (vgl. zu einem Überblick über derartige Erwähnungen \citealt[71-86]{Oennerfors1997}). Auch typologisch (vgl. \citealt{Sasse1995} und diachron (vgl. \citealt{Coniglio2012} zum Mittelhochdeutschen, vgl. auch die dort genannten Arbeiten zum Althochdeut\-schen (\citeyear[10]{Coniglio2012}) ist dies eine gängige Annahme. Deklarative V1-Sätze sind aufgrund dieser Charakteristik mit thetischen Sätzen assoziiert worden.

Önnerfors argumentiert, dass man nicht annehmen kann, dass die Sätze vollrhematisch sind in dem Sinne, dass sie kein Element beinhalten, das im Kontext nicht bekannt ist (z.B. Pronomen). Auch argumentiert er gegen Vollfokus (weil z.B. Scrambling \is{Scrambling} möglich ist) (\citeyear[76-82]{Oennerfors1997}). Er geht davon aus, dass V1-Deklarativsätze keine Topik-Kommentar-Gliederung \is{Topik-Kommentar-Gliederung} haben (\citeyear[84]{Oennerfors1997}) und des\-halb z.B. keine Topikalisierung \is{Topikalisierung} im Mittelfeld erlauben. Sie sind vollkommentarisch. Die klassische \is{Topikposition} Topikposition (das Vorfeld) ist in diesen Sätzen nicht vorhanden. In dieser Hinsicht ähneln sie thetischen Sätzen (vgl. (\ref{971})) und V2-Sätzen mit initialem \textit{es} (vgl. (\ref{972})).

\begin{exe}
	\ex\label{971}
	PEter hat angerufen.
\end{exe}
\vspace{-0.65cm}
\begin{exe}
	\ex\label{972}
	Es war einmal ein König.
	\hfill\hbox{\citet[305]{Oennerfors1997a}}
\end{exe}
(\ref{971}) weist keine Topik-Kommentar-Gliederung auf. Es liegt auch Vollfokus vor, d.h. \is{Satzfokus} Satzfokus. Der Satz kann out-of-the-blue \is{out-of-the-blue-Äußerung} geäußert werden und die Antwort auf eine sehr offene Frage darstellen. In vielen Sprachen liegt in Sätzen dieser Art Verb-Subjekt-Abfolge vor. Sätze wie (\ref{972}) führen ein Topik ein, \textit{es} kann aber kein Topik sein. Önnerfors unterscheidet zwar die drei V1-Satztypen in (\ref{970}), nimmt aber auch an, dass sie – trotz unterschiedlicher Funktionen – diesen \glqq informations\-strukturelle$[$n$]$ Kern\grqq{} teilen.

Wenn es so ist, dass die begründenden V1-Sätze die Eigenschaft aufweisen, vollkommentarisch zu sein, halte ich die Annahme für merkwürdig, dass \textit{doch} in diesen Sätzen anzeigen soll, dass der Sachverhalt präsupponiert \is{Präsupposition} ist. Önnerfors lehnt ab,  dass die Sätze auch vollfokussiert und rhematisch sind (s.o.). Der Kommentar, den die Sätze ausschließlich aufweisen, kann natürlich vollfokussiert und rhematisch sein, er muss es aber nicht. Derartige Sätze können z.B. erneut geäußert werden, um jemanden zu erinnern (vgl. (\ref{973})).

\begin{exe}
	\ex\label{973}
	Wie du bereits weißt: Es regnet.
\end{exe}
Folgte man den Annahmen aus der Literatur, läge ein Satztyp vor, in dem der Kommentar immer auf hintergrundierende, bekannte und akzeptierte Information verweist. Dies halte ich für unplausibel. \citet[223, Fn 22]{Reis2000} verweist darauf, dass \textit{doch} die vollkommentarische Eigenschaft außer Kraft setzen kann, weil I-Topikalisierung \is{I-Topikalisierung} möglich ist.

\begin{exe}
	\ex\label{974}
	$[$Das geht schon,$]$ weinen ALLe (/) Mädchen doch NICHT (\textbackslash) so leicht.
\end{exe}
Nimmt man diesen Aspekt hinzu, greift mein Argument gegen den präsupponier\-ten \is{Präsupposition} Status unter Bezug auf die vollkommentarische Natur dieses Subtyps nicht mehr. In (\ref{974}) sind \textit{alle Mädchen} das Topik \is{Topik} und der Rest des Satzes, auf den sich auch \textit{doch} bezieht, der Kommentar. Dann würde man allerdings nach wie vor davon ausgehen, dass der Kommentar \is{Kommentar} in diesen Sätzen immer alte Information beinhaltet. Ich glaube nicht, dass man dies verallgemeinern kann.

Neben Önnerfors Annahmen, die für mich gegen den stets präsupponierten Status der Sätze sprechen, halte ich auch Reis' Beispiel (vgl. (\ref{974})) für real. In meinen Daten gibt es ohne Zweifel Fälle, in denen Referenten aus dem Vorgänger\-kontext in den V1-Sätzen aufgegriffen werden und in diesen Sätzen eine Interpretation als Topik erfahren können (vgl. (\ref{975}) bis (\ref{977})).

\begin{exe}
	\ex\label{975}
	\scriptsize
	@Melvin: Danke für \emph{die Antwort}. \textbf{\textit{Zeigt} \textsc{diese} \underline{doch}, daß auch von dieser Seite noch in vernünftigen Bahnen 		gedacht wird.}	
	\hfill\hbox{(DECOW14AX)}
	\newline
	\hbox{}\hfill\hbox{(http://foren.waffen-online.de/lofiversion/index.php/t380704.html)}
\end{exe}
			 
\begin{exe}
	\ex\label{976}
	\scriptsize
	Aber \emph{Peter Sch.} hätte vor dem Auftritt über diese Besonderheit aufgeklärt werden sollen. \textbf{\textit{Lamentierte} \textsc{er} 					\underline{doch} darüber, dass seine Strumpfhose nur ein Bein habe}.....kicher......	
	\hfill\hbox{(DECOW14AX)}
	\newline
	\hbox{}\hfill\hbox{(http://home.arcor.de/abschlusshakemicke93/geschichten.html)}
\end{exe}						       

\begin{exe}	
	\ex\label{977}
	\scriptsize
	Es ist, so nehme ich mal an, eine \emph{der klassischen Armbanduhren oder Chronografen}, die zwar nicht mit ihrer Ganggenauigkeit, aber jederzeit mit 		ihrer Ausstrahlung der modernen Konkurrenz Paroli bieten können. \textbf{\textit{Signalisieren} \textsc{sie} \underline{doch}, dass ihr Besitzer mit 		einen individuellen Geschmack ausgestattet ist} und aller Wahrscheinlichkeit nach eine berufliche Position begleitet, die nicht mehr am sekundengenauen 	Erscheinen gemessen wird. 
	\hfill\hbox{(DECOW14AX)}
	\newline
	\hbox{}\hfill\hbox{(http://www.freunde-alter-wetterinstrumente.de/61news08.htm)}
\end{exe}						            						
Hier werden in den V1-Sätzen weitere Informationen über \textit{die Antwort}, \textit{Peter Sch.} und \textit{die Uhren} präsentiert. Es sind folglich keine auffälligen Strukturen wie die I-Topikalisierung \is{I-Topikalisierung} in Reis' Beispiel vonnöten. 

Nach Betrachtung einer großen Menge von \textit{doch}-V1-Sätzen (aus DECOW) bin ich der Meinung, dass Reis' Beispiel nicht gegen Önnerfors Auffassung spricht, sondern die Daten seine Einschätzung stützen: Fälle der Art in (\ref{975}) bis (\ref{977}) gibt es zwar, sie sind aber eher unüblich. Evidenz für diese Annahme liefert eine Untersuchung der auftretenden Subjekte in den 698 \textit{doch}-V1-Sätzen. Mit hohem Anteil (und zwar mit höherem als erwartet $[$s.u.$]$) kommen nicht referierende, semantisch leere Subjekte vor. Es treten folglich Subjekte auf, die sich nicht als Topik \is{Topik} eignen. Konkret handelt es sich hierbei um non-anaphorische \textit{es}-Subjekte (zu weiteren Details s.u.). Natürlich muss das Subjekt nicht das Topik im Satz sein (vgl. (\ref{978})), man geht aber davon aus, dass diese beiden Status in der Regel zusammenfallen. Das Subjekt gilt als das unmarkierte Topik (vgl. z.B. \citealt[132]{Lambrecht1994}).

\begin{exe}
	\ex\label{978}
	A: Erzähle mir was \textit{von Hans}!\\
	B: Maria hat \textit{ihn} verlassen.
\end{exe}
In den besagten V1-Sätzen mit \textit{es}-Subjekten scheint mir ebenfalls nicht das Verhältnis vorzuliegen, dass stets andere Elemente als Topik fungieren. Strukturen, um die es geht, sind solche der Art in (\ref{979}) bis (\ref{981}).

\begin{exe}
	\ex\label{979}
	\scriptsize
	Johnny Cash, der Mann in Schwarz, er bekam insgeheim den meisten Applaus. \textbf{\textit{Gilt} es \underline{doch} die Musiker zu ehren}, die sich um 		die Country Music verdient gemacht haben.
	\hfill\hbox{(DECOW14AX)}
	\newline
	\hbox{}\hfill\hbox{(http://www.musikansich.de/ausgaben/1203/\_cma.html)}
\end{exe}


\begin{exe}
	\ex\label{980}
	\scriptsize
	Als seine Frau Nancy sich von ihm scheiden lässt, erscheint ihm sein größtes Kinoidol Humphrey Bogart und gibt ihm gute Ratschläge für die Zukunft. 		Allan schöpft wieder Hoffnung. \textbf{\textit{Gibt} es \underline{doch} mehrere Millionen Frauen allein in New York} und eine davon wird er bestimmt 		erobern. 
	\newline
	\hbox{}\hfill\hbox{(DECOW14AX)}
	\newline
	\hbox{}\hfill\hbox{(http://www.volkstheater.at/home/spielplan/1494/Spiel\%5C\%27s$+$nochmal\%2C$+$Sam)}
\end{exe}							          
		   
\begin{exe}
	\ex\label{981}
	\scriptsize
	Für den neuen Handwerkskammerpräsidenten Hans Peter Wollseifer sind die Themen Aus- und Wei\-terbildung von besonderer Bedeutung.
	\textbf{\textit{Geht} es \underline{doch} darum, die jungen Menschen mit einer bestmöglichen Ausbildung auszustatten}, um Sie für das Berufsleben 			optimal auszurüsten.
	\hfill\hbox{(DECOW14AX)}
	\newline
	\hbox{}\hfill\hbox{(http://www.kunststoff-magazin.de/epaper/km091010/km\_091010/}
	\newline
	\hbox{}\hfill\hbox{blaetterkatalog/blaetterkatalog/html/wirtschaftsminister\_guntram\_schn.html)}
\end{exe}
Legt man eine \is{Aboutness-Topik} Aboutness-Topik-Auffassung (im Sinne von (\ref{982})) mit dem Topik als Satzgegenstand, über den eine Aussage gemacht wird, zugrunde, kommen die Subjekte hier nicht als Topiks in Frage.

\begin{exe}
	\ex\label{982} 
		\begin{xlist}	
			\ex\label{982a} What about x?
			\hfill\hbox {\citet[32]{Gundel1977}}
			\ex\label{982b} Ich sage dir über NP, daß S.
			\hfill\hbox {\citet[68-69]{Sgall1974}}
			\ex\label{982c} Er sagt über x, dass ...
			\hfill\hbox {\citet[65]{Reinhart1981}}
		\end{xlist}
\end{exe}
Die entscheidende Anforderung an einen potenziellen Topik-Referenten \is{Topik-Referent} ist, dass er referieren kann (vgl. z.B. \citealt{Reinhart1981}, \citealt[331]{Frey2007}). Aus diesem Grund eignen sich die Vorfeldkonstituenten in (\ref{983}) beispielsweise nicht als Topik.

\begin{exe}
	\ex\label{983} 
		\begin{xlist}	
			\ex\label{983a} \textbf{\textit{Keiner}} mag den Film.
			\ex\label{983b} \textbf{\textit{Leider}} hat Paul verschlafen.
			\ex\label{983c} \textbf{\textit{Wer}} hat Maria heute gesehen?
			\hfill\hbox {\citet[331]{Frey2007}}
		\end{xlist}
\end{exe}
Die Sätze können nicht als Aussage über \textit{keinen}, \textit{leider} oder \textit{wer} interpretiert werden. In Sätzen der Art in (\ref{979}) bis (\ref{981}) kommt \textit{es} folglich ohne Zweifel nicht als Topik in Frage. Für diese Sätze gilt somit, dass das Subjekt nie die Funktion des Topiks übernimmt.

Konkret befinden sich unter den 698 \textit{doch}-V1-Sätzen 116 Sätze mit strukturellem \textit{es}-Subjekt. Diese Strukturen machen auf den ersten Blick folglich 17\% der Sätze aus. Bewertungen solcher Verteilungen sind aber – wie ich schon an verschiedenen Stellen gezeigt habe – sehr von dem Erwartungswert für das Vorkommen des besagten Kontextes an sich abhängig. Im vorliegenden Fall ist es notwendig, zu wissen, mit wie vielen \textit{es}-Subjekten innerhalb von Subjekten überhaupt zu rechnen ist. Liegt dieser vor, kann man feststellen, wie das Auftreten des \textit{doch}-V1-Satzes zu diesem Erwartungswert steht.	

In einer Stichprobe von 500 Sätzen mit lexikalischen finiten Verben des glei\-chen Teilkorpus, aus dem die 698 \textit{doch}-V1-Sätze stammen, sind lediglich 35 derartige Subjekte enthalten. Zu erwarten sind non-anaphorische \textit{es}-Subjekte demzufolge nur mit 7\%. (\ref{984}) gibt auch das 95\%-Konfidenzintervall bei dieser Stichprobengröße an.

\begin{exe}
	\ex\label{984}Erwartungswert Verteilung \textit{es}-Subjekte in DECOW2014AX\\[-1em]		
 		\begin{tabular}[t]{|c|c|c|} 
 		\hline 	
   	 	& \textbf{\textit{es}-Subjekt} & \textbf{kein \textit{es}-Subjekt} \\
   	 	\hline 
  		absolute Zählung & 35 & 465\\ 
   		\hline
   		Anteil & \textbf{7\%} & \textbf{93\%}\\
   		\hline
   		95\%-Konfidenzintervall & $[$4,991\% ... 9,693\%$]$ & $[$90,31\% ... 95,01\%$]$ \\
   		\hline
 		\end{tabular}
\end{exe}
Der Unterschied zwischen 116 non-referenziellen \textit{es}-Subjekten und 582 anderen Subjekten stellt sich auf der Basis des ermittelten Erwartungswertes als signifikant heraus ($\chi^{2}$(1, n = 698) = 101,8247, p $<$ 0,001, V = 0,38). Vor dem Hintergrund des Erwartungswertes entspricht der Anteil der non-referenziellen \textit{es}-Subjekte quasi 73\%. Vor der Folie des niedrigeren Erwartungswertes als des Wertes, mit dem die Strukturen in den V1-Sätzen auftreten, kommen diese Subjekte folglich sehr häufig vor. Es tauchen somit überwiegend Subjekte auf, die sich nicht als Topik eignen.

Die non-referenziellen \textit{es}-Vorkommen in Subjektfunktion sind vor dem Hintergrund typologisch interessierter Arbeiten wie z.B. \citet{Askedal1990} und \citet{Zitterbart2002} nicht alle als gleich einzustufen. \citet{Speyer2009}, dessen Dreiteilung mir für meine Argumentation ausreichend erscheint, unterscheidet die drei Typen, die durch (\ref{985}) bis (\ref{987}) illustriert werden. \is{Vorfeld-es} \is{expletives es} \is{Korrelat-es}

\begin{exe}
	\ex\label{985} 
	\textbf{\textit{Es}} folgten mit weitem Abstand Uller und Karl Zaible.
	\hfill\hbox {(Vorfeld-\textit{es})}
\end{exe}
\vspace{-0.65cm}
\begin{exe}
	\ex\label{986} 
	\textit{\textbf{E}s} handelt sich um Ullers Verhältnis zu Rathkolb.
	\hfill\hbox {(Expletives \textit{es})}
\end{exe} 	
\vspace{-0.65cm}
\begin{exe}
	\ex\label{987} 
	\textbf{\textit{Es}} macht richtig Spaß, Kajak zu fahren.
	\hfill\hbox {(Korrelat-\textit{es})}
	\newline
	\hbox{}\hfill\hbox{\citet[324/325/326]{Speyer2009}}
\end{exe} 
Das expletive \textit{es} tritt als strukturelles Subjekt von Verben auf, die kein logisches Subjekt aufweisen. Im Gegensatz zum Vorfeld-\textit{es} ist es nicht auf diese Position beschränkt. Das Korrelat-\textit{es} ist in meinen Fällen mit einem Subjektsatz bzw. einer Verbativergänzung im Nachfeld koindiziert.

Unter den 116 non-referenziellen \textit{es}-Subjekten sind 65 expletive \textit{es} und 51 Korrelatauftreten. In dem Erwartungswert gibt es nur ein Vorkommen als Korrelat. Transparenter ist es m.E. deshalb, von einem Erwartungswert von 7\% für die expletiven \textit{es} auszugehen. Die Korrelate in struktureller Subjektposition/-funktion scheinen generell noch viel seltener zu sein. Nur 0,2\% der finiten Verben der 500er Stichprobe weisen ein solches Subjekt auf. Die 51 Vorkommen in den Daten sind somit extrem auffällig. Schließt man die Korrelate aus der Kalkulation aus, ergibt sich für die Verteilung des expletiven \textit{es} (65:582) nach wie vor ein signifikanter Unterschied ($\chi^{2}$(1, n = 647) = 9,2233, p $<$ 0,05, V = 0,12). Der Effekt ist allerdings schwach.

Die Darstellung konzentriert sich im Folgenden auf die expletiven \textit{es}. Die 65 Belege werden weitestgehend erfasst durch das Auftreten der Prädikate \textit{handeln um} (12x), \textit{geben} (30x) und \textit{gehen um} (16x) (vgl. (\ref{988}) bis (\ref{990})). Hierbei handelt es sich um ein \glqq  allgemeines Existentialprädikat\grqq{} und zwei \glqq allgemeine kommunikative Themaprädikate\grqq{} (\citealt[218]{Askedal1990} zu \textit{geben} und \textit{handeln um}).\footnote{Ferner treten auf: \textit{geben als}, \textit{ankommen auf}, \textit{brauchen}, \textit{stehen}, \textit{laufen}, \textit{fehlen} und \textit{gehen mit}.}

\begin{exe}
	\ex\label{988} 
	\scriptsize
	Auch die Nachricht, die wir an diesem Samstagabend auf die Ticker gegeben haben, ist rechtlich nicht zu beanstanden. \textbf{\textit{Handelt} es sich 		\underline{doch} um eine objektive Tatsache:} Die Staatsanwaltschaft Hamburg ermittelt gegen Gregor Gysi – und zwar mit ausdrücklicher Billigung des 		Immunitätsausschusses des Deutschen Bundestags, der am 31. Januar über den Fall beraten hat. 
	\hfill\hbox {(DECOW14AX)}
	\newline
	\hbox{}\hfill\hbox{(http://investigativ.welt.de/2013/02/10/gregor-gysi-und-die-wahrheit/)}
\end{exe} 

\begin{exe}
	\ex\label{989} 
	\scriptsize
	Auch stellen die Autoren zu wenig ihren eigenen Ansatz in Frage. \textbf{\textit{Gibt} es \underline{doch} schon lange Jahre die Diskussion}, ob eine 		Vorklassifizierung und –strukturierung von Information oder der effektive und intelligente Einsatz von Suchwerkzeugen erfolgversprechender sind.
	\hfill\hbox {(DECOW14AX)}
	\newline
	\hbox{}\hfill\hbox{(http://www.b-i-t-online.de/archiv/1999-04/rezen6.htm)}
\end{exe} 							 

\begin{exe}
	\ex\label{990} 
	\scriptsize
	Axel Köhler-Schnura vom Vorstand der CBG und einer ihrer Gründer ergänzt: \glqq Das Verfahren hat grundsätzliche Bedeutung und wird bundesweit mit 			Aufmerksamkeit verfolgt. \textbf{\textit{Geht} es \underline{doch} um die im Rahmen von Deregulierung und entfesseltem Kapitalismus überall zunehmende 		Unterwerfung von Forschung und Lehre unter wirtschaftliche Interessen und Konzernprofite.}\grqq{}
	\hfill\hbox {(DECOW14AX)}
	\newline
	\hbox{}\hfill\hbox{(http://www.nrhz.de/flyer/beitrag.php?id=18513)}
\end{exe}						                                      
Derartige Strukturen lassen sich alle als \textit{präsentative Konstruktionen} \is{Präsentativkonstruktion} einordnen. \textit{Es gibt x.} stellt als \textit{Existentialkonstruktion} \is{Existentialkonstruktion} einen Unterfall der Präsentativkonstruktion dar. Sie dient dem Einführen von Referenten (wenn auch nicht notwendigerweise – wie der Name suggeriert – um die Existenz an sich zu behaupten, sondern um einen Referenten in die Szene einzuführen $[$vgl. \citealt[179]{Lambrecht1994}$]$). Auch Strukturen, die durch \textit{es handelt sich um} oder \textit{es geht um} eingeleitet werden, ordne ich den präsentativen Konstruktionen zu. Sie führen auch gerade einen Gegenstandsbereich, ein Thema oder auch Referenten ein, über das/die in der Folge plausiblerweise weiter berichtet wird. Präsentative Strukturen gelten als Konstruktionen, die Topiks einführen. Dazu werden sie als thetisch \is{thetisch} behandelt, weisen also auch Vollfokus \is{vollfokussiert} auf (vgl. \citealt[144, 177]{Lambrecht1994}). Das Subjekt eignet sich folglich nicht als Topik (was nach \citealt[144-145]{Lambrecht1994} auch die definierende Eigenschaft thetischer Sätze ist). Dies bedeutet aber nicht, dass nicht andere Einheiten diese Funktion innehaben können. In Bezug auf die Sätze mit expletivem \textit{es} stellt sich somit die Frage, ob man andere Topiks in den Sätzen ausmachen kann. Meiner Meinung nach gibt es in ca. einem Drittel der Fälle Konstituenten, die sich prinzipiell eignen, wobei ich es auch in diesen Fällen für fraglich halte, dass sie wirklich als Topik fungieren.

In (\ref{991}) bis (\ref{995}) sind mit allquantifizierten NPs, Eigennamen, Pronomen und generischen NPs Einheiten in den Sätzen enthalten, die als Topik in Frage kommen. Man kann sich vorstellen, dass über diese Einheiten eine Aussage gemacht wird. Im Sinne der Vorstellung (wie in \citealt{Reinhart1981} vertreten), dass ein Topik einer Karteikarte entspricht, auf der durch den Kommentar Einträge gemacht werden, ist vertretbar, dass es für diese Einheiten eine Karte gibt.

\begin{exe}
	\ex\label{991} 
	\scriptsize
	Es stellt sich als schwierig heraus eine festgelegte Zielgruppe für dieses Buch zu definieren. \textbf{\textit{Gibt} es \underline{doch} \textsc{in 		allen Fachbereichen}, die sich mit dem Personenkreis von Menschen mit Behinderung befassen, solche die sich philosophischen Gedankengängen eher öffnen 		als andere.}
	\hfill\hbox {(DECOW14AX)}
	\newline
	\hbox{}\hfill\hbox{(http://www.socialnet.de/rezensionen/8130.php)}
\end{exe}

\begin{exe}
	\ex\label{992} 
	\scriptsize
	Sie versteht sich zugleich als Forum für eine geschlechtergerechtere Gesellschaft. \glqq Nicht gegen die Männer können wir uns emanzipieren, sondern 		nur in Auseinandersetzung mit ihnen. \textbf{\textit{Geht} es \textsc{uns} \underline{doch} um die Loslösung von den alten Geschlechterrollen}, um die 		menschliche Emanzipation überhaupt.	\grqq{} 	
	\hfill\hbox {(DECOW14AX)}
	\newline
	\hbox{}\hfill\hbox{(http://www.gemeinsamlernen.de/laufend/geschlechterrollen/}
	\newline
	\hbox{}\hfill\hbox{literatur70/gutenmorgen/margit\_kurz.htm)}
\end{exe}

\begin{exe}
	\ex\label{993} 
	\scriptsize
	Es gibt aber auch sehr viele Menschen, die diesem Hype sehr kritisch gegenüber stehen. \textbf{\textit{Handelt} es sich \underline{doch} \textsc{bei 		Schönheits-Operationen} um einen sehr persönlichen, medizinischen Eingriff}, der durchaus mit Gesundheitsrisiken behaftet ist.  		
	\hfill\hbox {(DECOW14AX)}
	\newline
	\hbox{}\hfill\hbox{(http://www.abazo-plastische-chirurgie.de/)}
\end{exe}							               
						                			                      
\begin{exe}
	\ex\label{995} 
	\scriptsize
	Hier wird Improvisieren zum Abenteuer jenseits von Konzepten und Klischees. \textbf{\textit{Geht} es \textsc{Thomas Ruëckert} doch keineswegs um 			vordergründige Virtuosität}, sondern um das Ausloten neuer (energetisch pulsierender) Klangdimensionen.		 		
	\hfill\hbox {(DECOW14AX)}
	\newline
	\hbox{}\hfill\hbox{(http://www.altes-pfandhaus.de/veranstaltungen/archiv.html)}
\end{exe}								                       						  	
In manchen Sätzen finden sich auch spezifische Adverbiale, die prinzipiell Topik sein können (vgl. (\ref{996}) und (\ref{997})), im Gegensatz zu unspezifischen Angaben wie in (\ref{998}) und (\ref{999}).

\begin{exe}
	\ex\label{996} 
	\scriptsize
	Wie schon beim Perikles wird auch hier wieder eine hochgesteigerte Schönheit verschmolzen mit persönlichen Zügen. \textbf{\textit{Gab} es 					\underline{doch} \textsc{in Griechenland} immer Kunstrichter}, die es rühmten, wenn die Künstler edle Männer noch edler darstellten.	 		
	\hfill\hbox {(DECOW14AX)}
	\newline
	\hbox{}\hfill\hbox{(http://www.lexikus.de/bibliothek/Antike-Portraets/Bemerkungen-}
	\newline
	\hbox{}\hfill\hbox{zu-den-Tafeln/Griechische-und-Roemische-Portraets)}
\end{exe}

\begin{exe}
	\ex\label{997} 
	\scriptsize
	Doch keinem der Indios war es im ersten Eifer der Eroberung und im kindlichen Besitzerglück eingefallen, einen Streifen Neuland als festen und 				dauernden Familienbesitz abzugrenzen. \textbf{\textit{Gab} es \underline{doch} \textsc{rings um das Geviert des ersten Rodungsbrandes} noch genug 			freies Land}, um auch im nächsten und übernächsten Jahr nochmals einen Flecken Urwald abzuholzen und urbar zu machen.	
	\newline 		
	\hbox{}\hfill\hbox {(DECOW14AX)}
	\newline
	\hbox{}\hfill\hbox{(http://www.payer.de/bolivien2/bolivien0222.htm)}
\end{exe}	
	
\begin{exe}
	\ex\label{998} 
	\scriptsize
	Auch stellen die Autoren zu wenig ihren eigenen Ansatz in Frage. \textbf{\textit{Gibt} es \underline{doch} \textsc{schon lange Jahre} die Diskussion}, 		ob eine Vorklassifizierung und 	–strukturierung von Information oder der effektive und intelligente Einsatz von Suchwerkzeugen erfolgversprechender 		sind.	 		
	\hfill\hbox {(DECOW14AX)}
	\newline
	\hbox{}\hfill\hbox{(http://www.b-i-t-online.de/archiv/1999-04/rezen6.htm)}
\end{exe}								                  

\begin{exe}
	\ex\label{999} 
	\scriptsize
	P.S. Den Bassdruck wirst du nur mit großen Resonanzkörper erzeugen können. \textbf{\textit{Gab} es \underline{doch} \textsc{vor kurzem} einen 				gehörlosen der zur 	Musik bei einer dieser Talentshows getanzt hat} ... Wir erinnern uns? 	 		
	\hfill\hbox {(DECOW14AX)}
	\newline
	\hbox{}\hfill\hbox{(http://www.mtb-news.de/forum/archive/index.php?t-484264.html)}
\end{exe}				                    
Für meine Begriffe interpretiert man die Adverbiale in (\ref{996}) und (\ref{997}) allerdings nicht als Satzgegenstand. Die Sätze beabsichtigen keine Aussage über Griechenland oder das genannte Gebiet. In 13 Sätzen kommen Ausdrücke vor, die prinzi\-piell als Topik \is{Topik} in Frage kommen. In 10 weiteren Fällen findet man adverbiale Situations- oder Umweltbezüge, die z.T. auch konkreter lokal oder temporal sind. (\ref{1000}) und (\ref{1001}) dienen der Illustration.

\begin{exe}
	\ex\label{1000} 
	\scriptsize
	So gesehen, könnte den beiden Lindenau-Grafen eigentlich der Titel \glqq Tierarzt ehrenhalber\grqq{} zugeordnet und ihr Wiegenort Machern als eine Art 		\glqq Wallfahrtsort\grqq{} für Tierärzte und Pferdezüchter gepriesen werden. \textbf{\textit{Gibt} es \underline{doch} \textsc{hier} viele Andenken an 		beide:} das Schloss mit der Ritterstube, die drei Grafen-Wappen und die Lindenau-Ausstellung wie auch den Landschaftsgarten mit seinen Parkbauten und 		Plastiken. 		
	\hfill\hbox {(DECOW14AX)}
	\newline
	\hbox{}\hfill\hbox{(http://www.uni-leipzig.de/\~{}mielke/MachernH/vetarzte.htm)}
\end{exe}

\begin{exe}
	\ex\label{1001} 
	\scriptsize
	Robbins-Fans werden sich über dieses neue Buch freuen. Nicht so unsere Wende-Politiker. \textbf{\textit{Handelt} es sich \underline{doch} hierbei um ein einziges 				enthusiastisches Plädoyer für den Terrorismus:} \glqq Die Terroristen sind die Büchsenöffner im Supermarkt des Lebens\grqq{}.		
	\hfill\hbox {(DECOW14AX)}
	\newline
	\hbox{}\hfill\hbox{(http://blogs.taz.de/hausmeisterblog/2007/01/page/5/)}
\end{exe}
Ich frage mich, ob diese Einheiten nicht eher die Funktion haben, einzuordnen, in welche Szene der Referent/das Thema eingeführt wird, als das Prädikat mit seinen Argumenten auf einen Satzgegenstand zu beziehen. Dies gilt deutlich für Fälle der Art in (\ref{1000}) und (\ref{1001}), aber auch für (\ref{996}) und (\ref{997}) und z.T. auch für Beispiele der Art in  (\ref{991}) bis (\ref{995}). Mit Ausnahme von fünf Fällen (wobei in zweien grammatische Gründe vorliegen) könnten die \glq Topikkonstituenten\grq {} auch weggelassen werden, ohne dass die Sätze anders interpretiert würden (vgl. z.B. (\ref{995}) und (\ref{1002}) vs. (\ref{1004})).

\begin{exe}
	\ex\label{1002} 
	\scriptsize
	Sogar Diesel durften jetzt sportlich sein, wie der Ibiza Cupra zum Modelljahr 2004 	zeigte. \textbf{\textit{Gab} es \textsc{ihn} \underline{doch} 			erstmals als 160 PS starken 1,9-Liter-TDI.} 		
	\hfill\hbox {(DECOW14AX)}
	\newline
	\hbox{}\hfill\hbox{(http://ww2.autoscout24.at/test/renault-zoe/so-wird-das-}
	\newline
	\hbox{}\hfill\hbox{nichts-frau-merkel/4319/418106/mrb-mz\_home?article=}
	\newline
	\hbox{}\hfill\hbox{420574\%26intcidm=mrb-mz\_home)}	
\end{exe}	
 		 						                            
\begin{exe}
	\ex\label{1004} 
	\scriptsize
	Aber so richtig neu ist Irish Tour '74 natürlich nicht. \textbf{\textit{Handelt} es sich (\textsc{hierbei}) \underline{doch} um den be\-kannten Tourfilm 		zum gleichnamigen klassischem Livealbum}, welches Rorys Ruf als außergewöhnlichen Musiker für immer zementierte.      		
	\hfill\hbox {(DECOW14AX)}
	\newline
	\hbox{}\hfill\hbox{(http://www.musikansich.de/ausgaben/0511/reviews/rory\_gallagher.html)}
\end{exe}								                      
Dieses Verhältnis ist nicht darauf zurückzuführen, dass die potenziellen Topiks weitestgehend Adverbiale \is{Adverbial} sind. Es gibt andere Adverbiale, die als Topik fungieren können und die nicht ohne Veränderung der Interpretation auslassbar sind. Dies gilt für rahmensetzende Ausdrücke wie in (\ref{1005}).

\begin{exe}
	\ex\label{1005} 
		\begin{xlist}	
			\ex\label{1005a} \textbf{\textit{Finanziell}} hat er keine Probleme.
			\ex\label{1005b} \textbf{\textit{Gesundheitlich}} hat er keine Probleme.
			\ex\label{1005c} \textbf{\textit{Privat}} hat er keine Probleme.
			\hfill\hbox {\citet[42]{Helbig1981}}
		\end{xlist}
\end{exe}
Wenngleich in ca. einem Drittel der expletiven \textit{es}-Subjekte Einheiten auftreten, die aufgrund ihrer referenziellen Eigenschaften als Topiks fungieren können, bezweifle ich, dass sie diese Aufgabe in den V1-Sätzen auch tatsächlich überneh\-men. Selbst mit den auftretenden referierenden Ausdrücken gelingen Tests, die dem Nachweis des Topikstatus dienen (vgl. \citealt[28-29]{Musan2010}), nur bedingt (vgl. (\ref{1006})).

\begin{exe}
	\ex\label{1006} 
		\begin{xlist}	
			\ex\label{1006a} ?Ich erzähl dir was über \textbf{\textit{Schönheits-OPs}}. Es handelt sich um einen persönlichen Eingriff.
			\ex\label{1006b} ?Hast du das über \textbf{\textit{Schönheits-OPs}} schon gehört? Es handelt sich um...
			\ex\label{1006c} ?Was \textbf{\textit{Schönheits-OPs}} betrifft, so handelt es sich um...
			\ex\label{1006d} ?Es sind \textbf{\textit{Schönheits-OPs}}, bei denen es sich um einen persönlichen Eingriff handelt.
		\end{xlist}
\end{exe}
Den Bezug auf einen Satzgegenstand halte ich auch daher für schwierig, weil die beteiligten Prädikate sehr abstrakte Inhalte vermitteln, d.h. lexikalisch recht blass sind. \citet[68]{Reinhart1981} weist unter Bezug auf \citet{Kuno1972} darauf hin, dass sich Entitäten besser als Topiks eignen, wenn sie deutlich affiziert werden.

Man täte besser daran, die Topiks in den V1-Sätzen als \textit{Diskurs}- und \is{Diskurstopik} nicht als \textit{Satztopiks} \is{Satztopik} aufzufassen. Die Sätze werden als Information über einen vorerwähn\-ten Film, Operation, einen Ort etc. gewertet, sie unterliegen selbst aber in der Regel keiner Topik-Kommentar-Gliederung. Diskurstopiks hat man auch in den Fällen, in denen weder das Subjekt Topik ist (weil es dies kategorisch hier nicht sein kann) noch eine (adverbiale) Einheit diese Funktion übernimmt. In (\ref{1007}) und (\ref{1008}) könnte man Ausdrücke einbauen wie \textit{dort}, \textit{dabei}, \textit{in dieser Diskussion}, \textit{hierbei} oder \textit{bei dieser Nachricht}. Man liest solche Einheiten mit, weil die Informationen, die die Sätze vermitteln, natürlich einen Bezug benötigen.

\begin{exe}
	\ex\label{1007} 
	\scriptsize
	Grau oder Sandsteinfarben? So lautete beim Meinungsaustausch der Immobilien- und Standortgemeinschaft (ISG) am Samstagmorgen in der Gaststätte \glqq 		Alt Ochtrup\grqq{} die Frage aller Fragen. \textbf{\textit{Ging} es \underline{doch} um das künftige Bild der Innenstadt} – genauer die Pflasterung.   		\hfill\hbox {(DECOW14AX)}
	\newline
	\hbox{}\hfill\hbox{(http://www.azonline.de/Muensterland/Kreis-Steinfurt/Ochtrup/(offset)/225)}
\end{exe}	

\begin{exe}
	\ex\label{1008} 
	\scriptsize
	Auch die Nachricht, die wir an diesem Samstagabend auf die Ticker gegeben haben, ist rechtlich nicht zu beanstanden. \textbf{\textit{Handelt} es sich 		\underline{doch} um eine objektive Tatsache:} Die Staatsanwaltschaft Hamburg ermittelt gegen Gregor Gysi – und zwar mit ausdrücklicher Billigung des 		Immunitätsaussschusses des Deutschen Bundestags, der am 31. Januar über den Fall beraten hat.    		
	\hfill\hbox {(DECOW14AX)}
	\newline
	\hbox{}\hfill\hbox{(http://investigativ.welt.de/2013/02/10/gregor-gysi-und-die-wahrheit/)}
\end{exe}	
Ich fasse diese Einheiten nicht als Satztopik \is{Satztopik} auf. Man hat es m.E. mit Diskurstopiks \is{Diskurstopik} zu tun. Ein solches müssen sie haben. Auch wenn die Sätze vollkommentarisch sind, steuern sie natürlich zu irgendeiner Sache eine Information bei. Da die \textit{doch}-V1-Sätze nie diskursinitial auftreten (Sie begründen schließlich stets eine Annahme.), kann man davon ausgehen, dass das Diskurstopik stets bereits etabliert ist. Aus meiner Annahme ergibt sich, dass das Diskurstopik im Satz repräsentiert werden kann, es aber nicht realisiert sein muss. Ich halte dies nicht für unplausibel, da es Sätze gibt, die ein anderes Satz- als Diskurstopik aufweisen. Der Paragraph in (\ref{1009}) kann als Ganzes als Information über die Stadt Kamp-Lintfort gewertet werden. Dennoch können (neben dem Diskurstopik) auch andere Einheiten als Satztopik fungieren.

\begin{exe}
	\ex\label{1009} 
	\scriptsize
	Ich erzähle dir mal was von meiner Heimatstadt: \textbf{\textit{Kamp-Lintfort}} liegt nördlich von Düsseldorf am linken Niederrhein. 						\textbf{\textit{Die Stadt}} liegt acht Kilometer von Moers entfernt und sechs Kilometer von Rheinberg. \textbf{\textit{Sie}} gliedert sich in elf 			Stadtteile. Über die Stadtgrenze hinaus ist \textbf{\textit{Kamp-Lintfort}} bekannt für das Kloster Kamp. \textbf{\textit{Dieses Zisterzienserkloster}} 	wurde 1123 gegründet. Bekannt ist \textbf{\textit{Kamp-Lintfort}} auch aus dem Radio. \textbf{\textit{Das Kreuz Kamp-Lintfort}} ist anfällig für 			Staus. Seit 2014 hat \textbf{\textit{die Stadt}} zudem eine Hochschule. \textbf{\textit{Die Hochschule Rhein-Waal}} ist mit der Fakultät \glqq 				Kommunikation und Umwelt\grqq{} ansässig.
\end{exe}
Wenn das Diskurstopik in den V1-Sätzen realisiert ist, ist eine vorerwähnte Konstituente enthalten. \citet{Oennerfors1997} (s.o.) hält gerade dies für möglich und er nimmt deshalb nicht an, dass die Sätze vollfokussiert oder rhematisch sind. Trotz vorerwähnten und damit bekannten Einheiten kodiert der Gesamtsatz den Kommentar. 

Eine Konstellation wie ich sie für die \textit{doch}-V1-Sätze mit expletivem \textit{es} als strukturellem Subjekt ansetze, liegt in (\ref{1010}) unter Beteiligung eines analogen V2-Satzes vor.

\begin{exe}
	\ex\label{1010} 
	\scriptsize
	\textbf{\textit{Bielefeld}} hat 330000 Einwohner. \textbf{\textit{Die Stadt}} liegt eine Stunde von Hannover entfernt an der ICE-Strecke Köln-Berlin. 		\textbf{\textit{Bielefeld}} ist die größte Stadt in der Region Ostwestfalen-Lippe. \textbf{Es gibt \underline{in Bielefeld} \textit{14 Gymnasien}.} 		\textbf{\textit{Sie}} verteilen sich auf die 10 Stadtteile.
\end{exe}	
(\ref{1010}) kann so interpretiert werden, dass der globale Diskurs Bielefeld als Thema hat. Der \textit{es gibt}-Satz ist thetisch \is{thetisch} und hat damit kein \is{Satztopik} Satztopik. Er dient der Einführung der Gymnasien. Dennoch ist das Diskurstopik auch in diesem Satz \textit{Bielefeld}. Es wird hier versprachlicht, könnte aber genauso gut ausgelassen werden. An der Interpretation, dass dies eine weitere Information über Bielefeld ist, würde sich dann nichts ändern.\footnote{Eine andere Strategie, die in der Literatur verfolgt wird, um thetischen Sätzen nicht jeglichen Bezugspunkt abzusprechen, ist, dass für diese Sätze eine Art von Situationstopik angenommen wird. \citet[16]{Erteschik-Shir2007} geht z.B. von so genannten \textit{stage topics} aus, die die spatio-temporalen Parameter (das Hier und Jetzt) des Diskurses bezeichnen (vgl. auch \citealt{Back1995} zur \textit{diffusen Deixis}). Diese Modellierungen \glqq retten\grqq{} die Ansicht, dass jeder Satz ein Topik aufweist. Über ein derartiges stage topic verfügt dann aber auch jeder Satz (ggf. zusätzlich zu seinem \glq eigentlichen\grq{} Topik), weil jeder Satz auch in Raum und Zeit verankert ist. Das Problem, dass die Sätze nicht wirklich einen Satzgegenstand beinhalten, löst man meiner Meinung nach auf diesem Wege nicht.}

Neben den 65 Sätzen mit expletivem \textit{es} treten 51 strukturelle \textit{es}-Subjekte auf, die als Korrelat \is{es-Korrelat} fungieren. Da unter den 500 lexikalischen finiten Verben der Stichprobe nur eines mit einem Subjekt-Korrelat auftritt, ist für diese Strukturen generell eine extrem niedrige Auftretenswahrscheinlichkeit anzunehmen. Umso bedeutsamer ist ihr Vorkommen in den \textit{doch}-V1-Strukturen. (\ref{1011}) bis (\ref{1013}) zeigen einige Beispiele.
	
\begin{exe}
	\ex\label{1011} 
	\scriptsize
	Vor mehr als 100 Jahren war man auch im Waldeck-Frankenberger Land mit dem Eisenbahnbau fieberhaft beschäftigt. \textbf{\textit{Galt} es 					\underline{doch}, mit dem Anschluss an die 1850 eröffnete Main-Weser-Bahn zwi\-schen Kassel, Marburg und Frankfurt/ Main dem schleichenden Niedergang der 	ohnehin spärlichen Wirtschaft und der damit verbundenen Verarmung der Bevölkerung in unserer Region entgegenzuwirken.  	}	  		
	\hfill\hbox {(DECOW14AX)}
	\newline
	\hbox{}\hfill\hbox{(http://regiowiki.hna.de/Frankenberg)}
\end{exe}	
	
\begin{exe}
	\ex\label{1012} 
	\scriptsize
	Sie sollten wissen, dass wir sie in das Verfahren mit einbeziehen werden, denn sie als Tarifpartner sind für uns Mitbeteiligter. \textbf{\textit{Geht} 		es \underline{doch} auch darum, nachzuweisen, dass wir richtige Verhandlungen geführt haben.  }	  		
	\hfill\hbox {(DECOW14AX)}
	\newline
	\hbox{}\hfill\hbox{(http://www.judicialis.de/Th\%C3\%BCringer-}
	\newline
	\hbox{}\hfill\hbox{Landesarbeitsgericht\_2-BV-3-00\_Beschluss\_17.10.2002.html)}
\end{exe}	
										           
\begin{exe}
	\ex\label{1013} 
	\scriptsize
	Bascha Mika wirft den Frauen in ihrem Buch \glqq Die Feigheit der Frauen\grqq{} vor, sie würden sich freiwillig den uralten Rollenbildern unterwerfen. 		\textbf{\textit{Scheint} es \underline{doch} so verlockend, das bequeme Leben zu wählen: einen Ernährer suchen, Kinder bekommen, daheim bleiben.} 		
	\hfill\hbox {(DECOW14AX)}
	\newline
	\hbox{}\hfill\hbox{(http://www.christundwelt.de/themen/detail/artikel/unsere-preussische-kinderstube/)}
\end{exe}	
Anders als in anderen Haupt-/Nebensatzstrukturen, die sich ebenfalls unter den V1-Sätzen befinden (vgl. \textit{der Vorwurf} und \textit{er} in (\ref{1014}) und (\ref{1015})), gibt es im Hauptsatz kein Subjekt, das als Topik herhalten könnte. 

\begin{exe}
	\ex\label{1014} 
	\scriptsize
	Nun hat Magnus-Essay-Enzensberger bereits 1957 dem deutschen Nachrichtenmagazin vorgeworfen, die Nachricht der Story zu opfern. Ein schrecklicher Befund, träfe er den Sigmaringer Volksboten oder den Zillertaler Almdudler. \textbf{\textit{Besagt} \textsc{der Vorwurf} \underline{doch} nichts anderes, als dass die Primärtugenden der Journalisten-Schule, \glqq wer, was, wann, wo, warum, wie\grqq{} zu fragen, darauf wahrheitsgemäß und konzise zu beantworten, der spannenden Story im Stahlnetzfieber des Lesers geopfert werden.		}
	\newline
	\hbox{}\hfill\hbox {(DECOW14AX)}
	\newline
	\hbox{}\hfill\hbox{(http://www.goedartpalm.de/spie.html)}
\end{exe}	

\begin{exe}
	\ex\label{1015} 
	\scriptsize
	Wer Paul Gerhard kennt, den überrascht das nicht. \textbf{\textit{Sagt} er \underline{doch} in dem Lied \glqq Befiehl du deine Wege\grqq{}:} Mit Sorgen 	und mit Grämen und mit selbsteigner Pein lässt Gott sich gar nichts nehmen, es muss erbeten sein. 		
	\hfill\hbox {(DECOW14AX)}
	\newline
	\hbox{}\hfill\hbox{(http://www.predigtpreis.de/predigtdatenbank/newsletter/}
	\newline
	\hbox{}\hfill\hbox{article/liedpredigt-zu-dem-lied-wie-soll-ich-dich-empfangen-eg-11.html)}	
\end{exe}									         
Ähnlich wie bei den expletiven \textit{es} weist ca. ein gutes Drittel der Sätze (19x) ein anderes potenzielles Topik auf, d.h. eine Phrase, die das \is{Topik} Topik sein könnte. Ob man diese Einheiten tatsächlich als das Topik der Sätze liest, ist wiederum eine andere Frage. In den meisten Fällen scheint mir auch hier eher eine situative Einordnung vorgenommen zu werden und weniger einem Referenten eine Eigenschaft zugeschrieben zu werden (vgl. (\ref{1016}) und (\ref{1017})).					                     
		
\begin{exe}
	\ex\label{1016} 
	\scriptsize
	Die scheibchensweise und nebulöse Informationspolitik der Bürgermeisterin Pfordt geht ungehindert für die Öffentlichkeit weiter. Von Transparenz kann weiter keine Rede sein. \textbf{\textit{Hieß} es \underline{doch} noch \textsc{im Artikel des KSTA v. 16.08.2006}:} Hamacher bestätigte, dass das Grundstück inzwischen der Ehefrau eines ehemaligen Rista-Gesellschafters gehöre, der gleichzeitig auch Liquidator des Grundstückes sei. 			
	\newline
	\hbox{}\hfill\hbox {(DECOW14AX)}
	\newline
	\hbox{}\hfill\hbox{(http://www.glessen-gazette.de/2009\_05\_05\_spielplatz.htm)}	
\end{exe}		

\begin{exe}
	\ex\label{1017} 
	\scriptsize
	Es ist eine der schönsten Aufgaben, die man als Vertreterin der Landesregierung wahrnehmen kann.
	\textbf{\textit{Gilt} es \underline{doch} \textsc{bei einer solchen Gelegenheit} Menschen zu würdigen, die sich in besonderem Maß für die Gemeinschaft 		eingesetzt haben.}				
	\hfill\hbox {(DECOW14AX)}
	\newline
	\hbox{}\hfill\hbox{(http://www.melinaev.de/blog/tags....Inzest/?view=30)}	
\end{exe}						   
Ob der Auftretensanteil dieser Einheiten wirklich bedeutet, dass eher selten überhaupt Einheiten auftreten, die Topiks sein können, lässt sich erst beurteilen, wenn ein Richtwert zum Auftreten solchen Materials in derartigen Sätzen, die keine begründenden V1-Sätze sind, vorliegt.

Dass auch in den Sätzen mit Korrelat-\textit{es} zusätzliche Elemente auftreten, die in der Regel eine situative Verankerung vornehmen, ist ebenfalls dem Umstand zuzuschreiben, dass die Aussagen verankert werden müssen. Ein durch \textit{es heißt} vermittelter Inhalt (vgl. (\ref{1018}) und (\ref{1019})) wird natürlich jemandem oder einem Schriftstück zugeschrieben bzw. in Bezug auf irgendetwas/einen Zeitpunkt verstanden. Der Inhalt kann nicht im luftleeren Raum stehen. Das, was hier ggf. angeführt wird, kann weggelassen werden, oder wird – wenn es nicht auftritt – mitgedacht.

\begin{exe}
	\ex\label{1018} 
	\scriptsize
	Zudem ist noch anzumerken, dass der Scheck ein Sparkassenscheck war – also von der Konkurrenz\-veranstaltung – was natuerlich die Nicht-Berechnung der 3 		DM Einloesungsgebuehr in einem noch strengeren Licht erscheinen laesst. \textbf{\textit{Heisst} es \underline{doch} \textsc{bei Raiffeisen}} – \glqq 		Einer fuer alle, alle fuer einen\grqq{}. Aber nicht fuer die Sparkasse! 					
	\hfill\hbox {(DECOW14AX)}
	\newline
	\hbox{}\hfill\hbox{(http://www.hoeflichepaparazzi.de/forum/archive/index.php/t-20765.html)}	
\end{exe}

\begin{exe}
	\ex\label{1019} 
	\scriptsize
	Trotz aller Fortschritte der Digitaltechnik, trotz neuer Daten- und Tonträgerformate ist die Schallplatte klanglich das Maß aller Dinge. 					\textbf{\textit{Heißt} es \underline{doch} \textsc{in Bezug auf DVD und SACD}:} \glqq Nie war digital so analog wie heute!\grqq{} 	 					
	\hfill\hbox {(DECOW14AX)}
	\newline
	\hbox{}\hfill\hbox{(http://verein365.de/firma/profil/212/analogue-audio-association-ev.html)}	
\end{exe}						         
Genauso gilt immer für irgendetwas, dass es um x geht (vgl. (\ref{1020})).

\begin{exe}
	\ex\label{1020} 
	\scriptsize
	Die Ernährung eines Sportlers mit intensivem oder regelmäßigem Trainingsaufwand ist grundsätzlich von den alltäglichen Essgewohnheiten eines 				Normalverbrauchers zu unterscheiden. \textbf{\textit{Geht} es \underline{doch} \textsc{bei einer gesunden, sinnvollen und vor allem sportlich 				orientierten Ernährung} darum, gewisse persönliche Ziele zu erreichen.}	
	\hfill\hbox {(DECOW14AX)}
	\newline
	\hbox{}\hfill\hbox{(http://www.bodybuilding-ironbody.de/sportlernahrung.html)}	
\end{exe}
Man interpretiert den Satz aber auch nicht anders, wenn diese Einheit nicht genannt wird (vgl. (\ref{1021})).

\begin{exe}
	\ex\label{1021} 
	\scriptsize
	Sie sollten wissen, dass wir sie in das Verfahren mit einbeziehen werden, denn sie als Tarifpartner sind für uns Mitbeteiligter. \textbf{\textit{Geht} 		es \underline{doch} auch darum, nachzuweisen, dass wir richtige Verhandlungen geführt haben.}
	\hfill\hbox {(DECOW14AX)}
	\newline
	\hbox{}\hfill\hbox{(http://www.judicialis.de/Th\%C3\%BCringer-Landesarbeitsgericht\_}
	\newline
	\hbox{}\hfill\hbox{2-BV-3-00\_Beschluss\_17.10.2002.html)}
\end{exe}
Auch für diese Sätze möchte ich deshalb in gleicher Argumentation wie bei den Sätzen mit expletivem \textit{es} vertreten, dass es sich nicht um das Satztopik \is{Satztopik} handelt, wenn eine sprachliche Realisierung des \glq Topiks\grq {} vorliegt, sondern um das Diskurstopik.

Wie bei den expletiven \textit{es} gilt auch für die Korrelatstrukturen, dass die auftretenden Verben eine eher abstrakte und lexikalisch arme Semantik aufweisen: Die Prädikate \textit{gelten} (18x), \textit{heißen} (17x) $[$1x im Sinne von \textit{gelten}$]$ und um \textit{etw. gehen} (7x) machen den Großteil der Verben aus. Dazu kommen \textit{scheinen} (2x), \textit{gelingen} (2x), \textit{liegen}, \textit{geschehen}, \textit{auf etw. hinauslaufen}, \textit{an etw. kratzen} und \textit{aussehen}. Dass diese semantisch blassen Verben vorkommen, spricht für mich dafür, dass große Teile des Satzes mitteilungswürdig sind. Hierzu passt auch die Vorstellung aus \citet[153]{Zitterbart2002}, die die Funktion des Korrelats als \glqq Progressionsindikator\grqq{} und \glqq Rhemaexponent für den extraponierten Nebensatz\grqq{} ansieht. Das Korrelat \glqq kündigt an, dass die im Nebensatz enthaltene Hauptinformation noch zu kommen hat\grqq{}. Da \textit{doch} sich auf das ganze Satzgefüge bezieht, würde man dann zum Ausdruck bringen, dass der Gesamtsatz bekannt ist, obwohl der Nebensatzinhalt hochrhematisch \is{Rhematizität} ist.

\subsection{Kausalität}
Nachdem ich gegen die von anderen Autoren vertretene Annahme argumentiert habe, dass \textit{doch} in V1-/\textit{Wo}-VL-Sätzen Bekanntheit/Unkontroverse/Hintergrund kodiert, schließt sich als nächste Frage an, ob/inwiefern die kausale Interpretation der Sätze mit der Bedeutung dieser MP zusammenhängt. 

\subsubsection{Ist \textit{doch} direkt für Kausalität verantwortlich?}	
Es gibt Arbeiten, in denen die kausale Interpretation der V1-Sätze direkt an das \textit{doch} gebunden wird. In \citet[59]{Koenig1990} wird \textit{doch} als \glqq kausale konjunktional gebrauchte Partikel\grqq{} eingestuft, in \textit{grammis 2.0}\footnote{http://hypermedia.ids-mannheim.de/call/public/gramwb.ansicht?v\_app=g\&v\_kat=Konnek\-tor\&v\_id=2058} wird es als kausa\-ler Adverbkonnektor gehandelt. \citet[168]{Oennerfors1997} und \citet[170]{Pittner2011} vertre\-ten demgegenüber, dass \textit{doch} nicht an sich kausal ist. Alle Äußerungen zu dieser Frage in der Literatur beziehen sich auf die V1-Sätze. Dies ist sicherlich darauf zurückzuführen, dass die Partikel in dieser Satzumgebung obligatorisch ist. Wie oben gezeigt, ist ihr Auftreten in den \textit{Wo}-Sätzen zwar nicht notwendig, aber dennoch sehr typisch. Seltsamerweise hat man sich bei der Betrachtung dieser Sätze nicht die Frage gestellt, ob das \textit{doch} mehr mit Kausalität zu tun hat. Dass das Auftreten von \textit{doch} nicht i.e.S. mit Kausalität verbunden ist, zeigen die Verteilungen aus Abschnitt~\ref{sec:korp}. Gäbe es diesen direkten Zusammenhang, sollte \textit{doch} schließlich auch in anderen kausalen Sätzen beliebt sein. Insbesondere sollte dies gelten für solche Kausalsätze, die ähnlich verwendet werden wie die kausal interpretierten V1- und \textit{Wo}-Sätze, d.h. die \is{modaler Kausalsatz} modalen Kausalsätze. Aus den Korpusdaten ist aber abzuleiten, dass der typische Kausalsatz gar keine Partikel aufweist und wenn eine Partikel auftritt, dann ist dies nicht auffällig häufig \textit{doch}. 

Die in Abschnitt~\ref{sec:unkontr} angeführte Korpusstudie von \citet{Doering2014} zur Interaktion von MPn und Diskursrelationen deckt keine Interaktion von \textit{doch} und der Relation \textit{CAUSE} auf. Die Anzahl der \textit{doch}-Äußerungen in dieser Diskursrelation entspricht hier in etwa der Häufigkeit, mit der die Relation überhaupt vorkommt (vgl. \citeyear[88]{Doering2014}). Einen schwachen Effekt stellt sie hier für \textit{ja} fest. Wenn die Kausali\-tät in den beiden Satztypen direkt auf die Partikel zurückzuführen sein soll, wäre folglich – wie schon beim Kriterium der Präsupposition – noch eher mit dem präferierten Auftreten von \textit{ja} zu rechnen. 

Natürlich können Sätze auch kausal aufeinander bezogen werden, wenn das \textit{doch} nicht in ihnen enthalten ist (vgl. \citealt[168]{Oennerfors1997}, \citealt[171]{Pittner2011}). In diesem Kontext ist die Erkenntnis über die Existenz von \is{konzeptuelles Deutungsmuster} grundlegenden konzeptu\-ellen Deutungsmustern (vgl. \citealt[228-229]{Linke2001}, \citealt[26-28]{Averintseva-Klisch2013}, die sich auf \citet{Hume1955[1748]} bezieht, vgl. auch \citealt[6]{Sanders1992}, \citealt[339-340]{Fabricius-Hansen2000}, \citealt{Kehler2002, Kehler2004}) von Bedeutung. Diese sind aus der Sicht zu sehen, dass alles Wahrgenommene nicht als chaotische Menge betrachtet wird, sondern – in einem abgesteckten Rahmen – gewisse Möglichkeiten von Bezügen bestehen. Für Sprache heißt dies, dass Sprecher auch dann Kohärenzrelationen \is{Kohärenzrelation} herstellen, wenn keine Kohäsionsmarker \is{Kohäsionsmarker} vorliegen. Die drei basalen Relationen sind koordinative, temporale und kausale Bezüge.

\begin{exe}
	\ex\label{1022} 
	koordinativ $<$ temporal $<$ kausal
\end{exe}
Die Stärke der Beziehungen nimmt in (\ref{1022}) von links nach rechts zu. Zusätz\-lich gilt, dass die linke Relation immer auch Voraussetzung für die rechte ist. Es ist auch angeführt worden, dass Sprecher – wenn möglich – in ihrer Interpretation von der engsten Verbindung ausgehen, d.h. den kausalen Zusammenhang auswählen (vgl. z.B. \citealt[58, Fn 6, 61-62]{Breindl2006}).

Unabhängig davon, ob in (\ref{1023}) und (\ref{1024}) \textit{doch} verwendet wird oder nicht, wird der zweite Satz als Begründung des ersten gedeutet.

\begin{exe}
	\ex\label{1023} 
	Hans kommt nicht. Er ist krank.
\end{exe}
\vspace{-0.65cm}
\begin{exe}
	\ex\label{1024} 
	Hans kommt nicht. Er ist \textbf{doch} krank.
	\hfill\hbox {\citet[168]{Oennerfors1997}}
\end{exe}
Bei den \textit{Wo}-Sätzen sieht man, dass die Sätze auch kausal gelesen werden können, wenn kein \textit{doch} auftritt, obwohl dies in meinen Daten wenig eintritt, wie man an der Verteilung in Abschnitt~\ref{sec:korp} sieht.

\begin{exe}
	\ex\label{1025}
	 \scriptsize 
	\glqq Wieso hat ein amischer Mann Lautsprecherkabel?\grqq{} denke ich laut. \glqq \textbf{\textit{Wo} er weder Radio noch Fernseher besitzt.} Er benutzt ja 			nicht mal eine Melkmaschine oder einen Generator zur Milchverarbeitung.\grqq{} 
	\newline
	\hbox{}\hfill\hbox {\citet[53]{Castillo2011}}
\end{exe}
Unter den nachgestellten \textit{doch}-Sätzen ist keiner, für den sich nicht eine kausale Interpretation anbietet. Prinzipiell ist eine nicht-kausale Interpretation für \textit{Wo}-\textit{doch}-Sätze aber nicht ausgeschlossen (vgl. den modifizierten Beleg in (\ref{1026}), in dem ursprünglich keine Partikel auftritt).
	
\begin{exe}
	\ex\label{1026}
	 \scriptsize 
	Was wäre der Sport ohne seine Funktionäre – und umgekehrt? Eine interessante Frage \emph{in Zeiten} \emph{wie diesen}. \textbf{\textit{Wo} \underline{doch} 		tagtäglich über neue Highlights aus der \glqq Königsklasse des Sports\grqq{} – der olympischen Bewegung samit ihren Funktionären – berichtet wird.}
	\newline
	\hbox{}\hfill\hbox {(BVZ09/SEP.00997 Burgenländische Volkszeitung, 09.09.2009) $[$verändert S.M.$]$}
\end{exe}	
Fehlt das \textit{doch}, ist man allerdings eher bereit, den Satz nicht kausal zu lesen. 

\begin{exe}
	\ex\label{1027}
	\scriptsize 
	Was wäre der Sport ohne seine Funktionäre – und umgekehrt? Eine interessante Frage \emph{in Zeiten} \emph{wie diesen}. \textbf{\textit{Wo} tagtäglich über 		neue Highlights aus der \glqq Königsklasse des Sports\grqq{} – der olympischen Bewegung samit ihren Funktionären – berichtet wird.}
	\newline
	\hbox{}\hfill\hbox {(BVZ09/SEP.00997 Burgenländische Volkszeitung, 09.09.2009)}
\end{exe}                                    
Für (\ref{1027}) bietet sich auch gut die Interpretation als Relativsatz an. Dies ist selbst dann möglich, wenn kein explizites Bezugselement vorhanden ist:

\begin{exe}
	\ex\label{1028}
	 \scriptsize 
	Und endlich mal Dusche, Badewanne und die Kloschüsseln mit der Bürste von unschönen Kalkrändern befreien. Und natürlich noch schnell den Garten 			entlauben, umgraben und mit Mulch bedecken. \textbf{\textit{Wo} es gerade so trocken ist.}    
	\hfill\hbox {(BRZ05/NOV.18215 Braunschweiger Zeitung, 19.11.2005)}
\end{exe}                                                                        
Beim \textit{Wo}-Satz vereindeutigt das \textit{doch} die Interpretation. Bei den V1-Sätzen sieht man, dass eine der drei Eigenschaften, die die drei deklarativen V1-Sätze aus\-zeichnet (vgl. Abschnitt~\ref{sec:unkontr}), bzw. ein segmental identischer anderer V1-Satz vorliegen muss, da die Struktur ansonsten nicht interpretierbar ist. Es ist aber auch bei den V1-Sätzen nicht so, dass sie kausal gelesen werden müssen, sobald \textit{doch} vorkommt. (\ref{1029}) kann beispielsweise auch adhortativ \is{Adhortativ} gelesen werden. In (\ref{1030}) liegt ein \is{emphatischer V1-Deklarativsatz} emphatischer V1-Deklarativsatz vor, den man (anders als (\ref{1029})) auch gar nicht kausal verknüpfen kann.

\begin{exe}
	\ex\label{1029}
	 \scriptsize 
	Eine positive Einstellung und Wertschätzung ihnen gegenüber lohnt sich auf jeden Fall. \textbf{Bauen wir \underline{doch} Brücken auf und Vorurteile 		ab.}                  		
	\hfill\hbox {(A08/NOV.01067 St. Galler Tagblatt, 05.11.2008)}
\end{exe}

\begin{exe}
	\ex\label{1030}
	 \scriptsize 
	Wenn die Maus sich darunter zum Schlemmern niederlässt, Deckel runter, peng. \textbf{Bricht \underline{doch} der Knauf der Dose ab!} Mitten in der 			Inszenierung.    
	\newline              		
	\hbox{}\hfill\hbox {(BRZ05/DEZ.05511 Braunschweiger Zeitung, 05.12.2005)}
\end{exe}
Ich glaube deshalb, dass es für \textit{Wo}- und V1-Sätze sehr wichtig ist, dass die Sachverhalte bzw. Annahmen, die kausal aufeinander bezogen werden, diese Verbindung überhaupt erlauben (vgl. auch \citealt[87-128]{Gohl2000} zur Relevanz von situativem und konzeptuellem Wissen bei der kausalen Interpretation \is{Asyndese} asyndetischer Strukturen). Ist dies gegeben und tritt dann \textit{doch} in diesen Sätzen auf, werden sie i.d.R. auch kausal gelesen. Auf die Annahme, dass die Konstellation der Sachverhalte in dem vorweggehenden Satz und dem V1-Satz relevant ist, bauen auch \citet{Oennerfors1997} und \citet{Pittner2011}. Die Erklärung des Zustandekommens der kausalen Interpretation wird in den beiden Arbeiten deshalb als Zusammenspiel des MP-Beitrags, der Verbstellung, des inhaltlichen Bezugs der beiden Sätze und der Nachstellung des V1-Satzes gesehen. Sie unterscheiden sich allerdings darin, dass Önnerfors (anders als Pittner) davon ausgeht, dass die eigentliche \textit{doch}-Bedeutung nicht vorhanden ist.

\subsubsection{Ist \textit{doch} indirekt für Kausalität verantwortlich?}
\label{sec:kausalind}
Den Beitrag von \textit{doch} fassen die beiden Autoren wie folgt:

\begin{quotation}
Durch die V1-Stellung im bV1-DS erfolgt eine besonders enge Anknüpfung an den unmittelbar vorausgehenden Kotext. Durch diese Anknüpfung, die durch die inhaltliche Nähe der beiden – in begründender Weise aufeinander bezogenen – Propositionen noch unterstrichen wird, wird gewissermaßen signalisiert, daß im Falle des bV1-DS die \glq Bezugsdomäne\grq {} des rückverweisenden \textit{doch} nicht, wie im Standardfall, der Kontext ist, sondern der Kotext, genauer: die Proposition des Bezugssatzes. Dieser obligatorische Bezug auf den unmittelbar vorausgehenden Kotext blockiert die Möglichkeit einer auf den Kontext zugreifenden Widerspruchsimplikatur des \textit{doch}.                            
\newline              		
\hbox{}\hfill\hbox {\citet[170]{Oennerfors1997}}
\end{quotation}

\begin{quotation}
Indem der Rezipient auf einen Sachverhalt hingewiesen wird, der als unkontrovers, aber in seinem momentan aktualisierten Wissen nicht präsent gekennzeichnet wird, kann in Zusammenhang mit der engen Anbindung an den Bezugssatz durch die V1-Stellung die Relation der stützenden Begründung erschlossen werden.
\hfill\hbox {\citet[170]{Pittner2011}}
\end{quotation}
Interessanterweise erachten beide die V1-Stellung als sehr relevant, sie sehen in ihr das Potenzial im Beitrag dieser Sätze. Die \textit{doch}-Bedeutung selbst (er geht von (\ref{1031}) aus) spielt bei Önnerfors keine Rolle. Sie ist getilgt (s.o.).

\begin{exe}
	\ex\label{1031} 
		$\lambda \textrm{p[FAKTp}]$\\
		\textsc{Implikatur}$[\exists \textrm{q[q} \rightarrow \neg \textrm{p}]]$
		\hfill\hbox {\citet[83]{Ormelius-Sandblom1997}}
\end{exe}
Es geht ihm nur um die generelle Funktion des Rückverweises im Kontext des sowieso verfügbaren kausalen Zusammenhangs und der durch die Verbstellung angezeigten Verbindung zwischen den Sätzen. Bei Pittner spielt die \textit{doch}-Bedeu\-tung hingegen eine Rolle: Sie modelliert sie über eine Anweisung an den Hörer, wie in (\ref{1032}), und unterscheidet zudem zwischen dem \textit{cg des Dialogs} \is{Dialog-cg} und einem \is{genereller cg} \textit{generellen cg}. Der generelle cg umfasst einen \is{persönlicher cg} persönlichen cg, der zwischen Individuen in der Interaktion zustande kommt (z.B. durch gemeinsame Erfahrungen, Handlungen) und \is{kultureller cg} einen \textit{kulturellen cg}, der zwischen Mitgliedern bestimmter Gruppen entsteht (z.B. Nation, Sprache).

\begin{exe}
	\ex\label{1032} 
	Ersetze $\neg$p durch p.	
	\hfill\hbox {\citet[167]{Pittner2011}}
\end{exe}
Eine \textit{doch}-Äußerung nimmt ihr zufolge Bezug auf eine Situation, in der der Hörer p eigentlich weiß, es aktuell aber nicht berücksichtigt, so dass die MP-Äußerung ihn auffordert, den aktuellen Wissensstand (Dialog-cg) aus dem generellen cg zu aktualisieren. 

Das Beispiel in (\ref{1033}) deutet Pittner so, dass der Sprecher der \textit{doch}-Äußerung den Adressaten an ihren Inhalt erinnern möchte, d.h. davon ausgeht, dass er ihn eigentlich weiß. Entlang von (\ref{1032}) müsste folglich gelten, dass der Angespro\-chene im aktuellen Diskurs von $\neg$p ausgeht, obwohl im generellen cg p gilt, und er wird aufgefordert, p anzunehmen (vgl. \citealt[167-168]{Pittner2011}).

\begin{exe}
	\ex\label{1033} 
	$[$Ein Junge will in Gegenwart eines Erwachsenen etwas aus einer Flasche trinken.$]$
	Du bist noch nicht groß genug. Du kannst \textbf{doch} nicht eine Flasche Wein allein austrinken.			
	\hfill\hbox {\citet[168]{Pittner2011}}
\end{exe}
Ich halte die von Pittner angesetzte \textit{doch}-Bedeutung unabhängig der Betrachtung von V1-Sätzen für problematisch, da sie in jeder Verwendung nachweisen können müsste, dass der Gesprächspartner die gegenteilige Annahme vertritt. Es ist sicherlich für jede Bedeutungszuschreibung eine Herausforderung, in jedem Kontext die formulierten Bedeutungsaspekte nachzuweisen. Auf dieses Bedeutungsmoment lässt sich aber gut verzichten. Hinzu kommt, dass sie immer annehmen muss, dass Gesprächsteilnehmer angesprochen werden. Sie gehen schließlich von $\neg$p aus und sollen diese Annahme revidieren. Für mich hat diese Vorstellung etwas sehr Aktives. Insbesondere die V1-Sätze, mit denen Pittner sich beschäftigt, treten so gut wie gar nicht dialogisch auf. Die gegenteilige Annahme müsste dann immer dem Leser unterstellt werden und er müsste angesprochen werden, $\neg$p durch p zu ersetzen, weil p im kulturellen cg enthalten ist. Es ist eher nicht davon auszugehen, dass ein Autor und seine anonyme Leserschaft einen persönlichen cg teilen bzw. dass die V1-Sätze nur unter diesen Umständen verwendet werden. Man müsste dann annehmen, dass dieses Verhältnis stets vorgegeben ist – was ich nicht für besonders wünschenswert halte.

Für (\ref{1034}) z.B. halte ich es nicht für passend, anzunehmen, dass der Leser vertritt, dass es nicht König Dagobert I. war und dass er durch die \textit{doch}-Äußerung angehalten wird, diese Ansicht zu ersetzen durch die ihm eigentlich bekannte Annahme, dass KD der Metzer Domkirche ein Weingut in Neef schenkte.

\begin{exe}
	\ex\label{1034}
	\scriptsize 
	Schon seit dem frühen Mittelalter hat der Wein dem Ort Bedeutung verliehen. \textbf{\textit{War} es \underline{doch} König Dagobert I., der der Metzer 		Domkirche ein Weingut in Neef schenkte.}    
	\newline              		
	\hbox{}\hfill\hbox {(RHZ09/OKT.24515 Rhein-Zeitung, 28.10.2009)}
\end{exe}
Ich verstehe an Pittners Ausführungen auch nicht, wie man aus ihrer \textit{doch}-Model\-lierung ableiten kann, dass die Sätze, die aufgrund der V1-Stellung eng aufeinander bezogen werden, kausal verknüpft werden. Sie führt diesen Aspekt nicht aus. Warum begünstigt \glq Ersetze $\neg$p durch p.\grq{} eine kausale Verbindung zwischen den Sätzen?

Önnerfors schreibt der Partikel nur rückverweisende Funktion zu. Ich halte die Annahme dieses allgemeinen Beitrags dieser speziellen Partikel für zu schwach, weil der Rückverweis der Grundbeitrag einer jeden MP ist. \textit{Doch} zählt nicht einmal zu den Partikeln, die den Rückverweis am deutlichsten vorweisen. Sie kann auch diskursinitial verwendet werden, anders als z.B. \textit{halt}, \textit{eben} oder \textit{auch}, für deren adäquate Verwendung ein Beitrag vorweggehen muss. Es gibt Partikeln, die sich besser eignen (würden), den Rückverweis anzuzeigen, wenn dies der Grund für das obligatorische Vorkommen der Partikel in diesen Sätzen sein soll.

Beide Ansätze lassen den Anteil der Partikel an der kausalen Relation offen. Da ich gegen die stets vorliegende Bekanntheit des Inhalts des V1-Satzes argumentiere und zudem nicht davon ausgehe, dass \textit{doch} Unkontroverse markiert, stellt sich für mich umso mehr die Frage, welchen Beitrag \textit{doch} in diesen und in den \textit{Wo}-VL-Sätzen leistet. 

An beiden Ansätzen stört mich darüber hinaus, dass sie so viel Potenzial in der Verberststellung sehen. Sie bewirkt den beiden Autoren zufolge die enge Kontextanbindung. Da in den \textit{Wo}-VL-Sätzen, die ich unter der gleichen Erklärung aufzufangen beabsichtige, keine V1-Stellung vorliegt, finde ich es schwierig, sie positiv in meine Analyse zu integrieren. Man bräuchte dann in jedem Fall eine andere Erklärung für die \textit{Wo}-Sätze.

Ich möchte deshalb die gegenteilige Perspektive einschlagen und den Blick auf die beiden Satztypen wählen, zu sagen, dass sie beide gewisse \glq Defekte\grq {} (und weniger Potenziale) aufweisen. Aus diesem Grund müssen sie auf ganz be\-stimmte Art eingebunden und sprachlich ausgestattet werden, um im Diskurs überhaupt Verwendung zu finden. Ich halte hier die recht einfache Aussage von \citet[250]{Scheutz2009} zum V1-Satz für entscheidend. Er schreibt, der V1-Satz drücke \glqq textuelle Unselbständigkeit\grqq{} aus. Dem schließe ich mich an und nehme an, dass es einen \glq neutralen\grq {} V1-Deklarativsatz nicht gibt. Jeder V1-Deklarativsatz \is{V1-Deklarativsatz} muss eine bestimmte Beschaffenheit haben, um grammatisch lizensiert und textuell/diskurs\-strukturell verwendbar zu sein. Er weist kein Vorfeld und eine besetzte linke Satzklammer auf, d.h. zwei Positionen, in denen sich sonst textuelle Anknüpfungen maßgeblich abspielen, scheiden für derartige Kodierungen aus. Der \textit{Wo}-Satz bringt nun ein ähnliches Problem mit sich, wenn auch weniger verschärft. Er ist prinzipiell ambig, weil \textit{wo} temporal, adversativ, lokal, kausal und konzessiv interpretiert werden kann. In diesem Sinne ist seine inhärente Bedeutung sehr arm/unspezifisch und könnte abstrakt etwa als \glq Gleichheit\grq {} gefasst werden. Die textuelle Verknüpfung ist folglich auch für diesen Satztyp nötig, aber ebenfalls schwierig, da auch hier das Vorfeld und die linke Satzklammer nicht verfügbar sind. Für beide Satztypen bleibt somit nur das Mittelfeld, um die textuelle Anbindung, die sie zur Interpretation benötigen, zu gewährleisten. Für die Kodierung kommen MPn und Adverbien im Mittelfeld in Frage.

Neben dieser \glq defekten\grq {} Ausgangslage geht in meine Analyse der \textit{Wo}-VL- und V1-Sätze auch die oben ausgeführte Annahme zu konzeptuellen Grundmustern \is{konzeptuelles Grundmuster} ein. Der V1/\textit{Wo}-Satz und der vorweggehende Satz sind prinzipiell kausal aufeinander zu beziehen. Ich habe gezeigt, dass dies ein wichtiger Aspekt ist. Das \glq rei\-chere\grq {} \textit{Wo} ermöglicht auch allein (d.h. ohne \textit{doch}) die kausale Interpretation. Wenn die kausale Relation nicht nahe liegt (aber alternative Lesarten zulässig sind), tritt sie auch bei den V1-Sätzen nicht zwingend ein. Die kausale Relation kann folglich stets als bestehend angesehen werden, entweder, weil sie tatsächlich gilt, oder, weil eine Default-Interpretation in Kraft tritt, der zufolge Sprecher sehr geneigt sind, da, wo es möglich ist, Kausalität hineinzulesen, wenn zwei Sätze aufeinander folgen.\footnote{Im Rahmen meiner Analyse sind folglich weder die \textit{Wo}-VL- noch die V1-Sätze inhärent kausal oder konzessiv. Ein Gutachter argumentiert, dass \textit{Wo}-VL-Sätze alleine, d.h. unabhängig der vorweggehenden Einstellung, konzessiv gelesen würden. Er vertritt, dass ein V1-Satz in einem Kontext wie in (\ref{1037}), in dem das Erstaunen ausgeschlossen werden muss und sich die Begründung einer Annahme nicht anbietet, unangemessen ist, während ein \textit{Wo}-Satz gut stehen könnte. 

\begin{exe}
	\ex\label{1037}
	Das Elta-Gerät hat einen hohe Brand- und Verletzungsgefahr. \#Trägt der Elta doch wie alle anderen Haartrockner das CE-Zeichen und dazu das GS-Zeichen 		(GeprüfteSicherheit).   
\end{exe}

\begin{exe}
	\ex\label{1038}
	Das Elta-Gerät hat einen hohe Brand- und Verletzungsgefahr. Wo der Elta doch wie alle anderen Haartrockner das CE-Zeichen und dazu das GS-Zeichen 		(GeprüfteSicherheit).  
\end{exe}
Es wird sich später zeigen, dass es tatsächlich entlang der Interpretationen \textit{konzessiv} vs. \textit{non-konzessiv} Verwendungsunterschiede zwischen den beiden Satztypen gibt. Ich bin aber weiterhin der Ansicht, dass die Sätze nicht inhärent konzessiv (und auch nicht inhärent kausal) sind.
}

Ferner meine ich, dass \textit{doch} diese enge Kontextanbindung, die prinzipiell vor\-handen ist, forciert. Die von mir angesetzte \textit{doch}-Bedeutung erlaubt es, zu motivieren, warum sich genau diese Partikel für die enge Kontextanbindung gut eignet. Nach meiner Modellierung zeigt \textit{doch} an, dass die Äußerung auf ein offenes Thema reagiert (vgl. (\ref{1035})).
\newcolumntype{C}[1]{>{\centering}p{#1}} 
\begin{exe}
	\ex\label{1035} Kontext vor einer \textit{doch}-Assertion\\[-1em]
 	\begin{tabular}[t]{|C{6em}|C{6em}|C{6em}|}
 	\hline 	
 	$\textrm{DC}_{\textrm{A}}$ & {Tisch} & $\textrm{DC}_{\textrm{B}}$ \tabularnewline
  	\hline
    & p $\vee$ $\neg$p & \tabularnewline
 	\hline      
   	\multicolumn{3}{|l|}{cg s$_{1}$} \tabularnewline   
   	\hline
 	\end{tabular}
\end{exe}
Auf ein offenes Thema zu reagieren, verweist auf eine enge Kontextanbindung bzw. schafft diese. Eine Vorstellung, die in Diskursmodellen steckt (vgl. z.B. \citealt{Roberts1996}, \citealt{Buering2003}), ist, dass sich ein Diskurs idealerweise durch Frage-Antwort-Sequenzen gestaltet, wobei die Fragen (wie in meiner Darstellung des Modells von \citealt{Farkas2010} gezeigt) auch durch Assertionen eröffnet werden können. Tatsächlich ist ein Diskurs i.d.R. nicht durch strikte Frage-(komplette) Antwort-Sequenzen strukturiert, sondern es werden Teilantworten gegeben, wes\-halb auch Modelle davon ausgehen, dass sich ein Diskurs in Unterfragen splittet. Diese Subfragen bilden aber entscheidenderweise mit ihren Antworten ebenfalls Frage-Antwort-Paare (vgl. (\ref{1035a}) und (\ref{1035b}), \citealt[515-516]{Buering2003}).

\begin{exe}
\ex \label{1035a}
        \begin{jtree}
        \! = {Diskurs}
                :[scaleby=3.75 1]{Frage}!a [scaleby=3.75 1]{Frage}
                <vert>{\ldots}.
        \!a = <wideleft>[scaleby=2.75 1]{Subfrage}(<vert>{Antwort}) ^<left>[scaleby=1.75 1]{Subfrage}(<vert>{Antwort}) ^<right>[scaleby=1.75 1]{Subfrage}!b ^<wideright>[scaleby=2.75 1]{Subfrage}(<vert>{Antwort}).
        \!b = :[scaleby=1.25 1]{Subfrage}(<vert>{Antwort}) [scaleby=1.25 1]{Subfrage}(<vert>{Antwort}).
        \end{jtree}
\end{exe}

\begin{exe}
        \ex \label{1035b}
        \begin{tabbing}
        Wie war \= War die \= Wie war der Schlagze \= Es war \= \kill
        Wie war das Konzert?\\
                \> War die Musik gut?                \>\> Nein, sie war schrecklich.\\
                \> Wie war das Publikum?        \>\> Es war enthusiastisch.\\       
                \> Wie war die Band?\\
                        \>\> Wie war der Schlagzeuger?        \>\> Fantastisch.\\
                        \>\> Wie war der Sänger?        \>\> Besser den je!\\
                \> Wurden alte Lieder gespielt? \>\> Nein, kein einziges.\\
        Was hast du nach dem Konzert gemacht? \ldots
        \end{tabbing}
\end{exe}
Jede Antwort auf eine \glq Frage\grq {} leistet somit eine enge Kontextanbindung. Sie bringt den Diskurs weiter, auch dann, wenn man dadurch nur zur nächsten Frage gelangt. Es ist unmöglich, zu sagen, dass ein Sprecher mit seiner Assertion auf eine offene Frage reagiert und gleichzeitig damit anzeigt, dass er keinen direkt relevanten Diskursbeitrag leistet.

Ich gehe davon aus, dass die Sätze aufgrund der ausgedrückten Sachverhalte bzw. einer generellen Erwartung zur kausalen Default-Interpretation zweier auf\-einander folgender Sätze kausal aufeinander bezogen werden. Wären die Sätze nicht unterspezifiziert und bräuchten deshalb eine Interpretationshilfe, könnte Kausa\-lität über Kohärenz auch ohne die Partikel hergestellt werden. In diesen Satzkontexten braucht man aber ein Element, das diese Lesart stützt. \textit{Wo} kann diese Interpretation prinzipiell auch alleine erreichen, \textit{doch} begünstigt sie allerdings. In meinen Daten finden sich sehr wenige \textit{Wo}-Sätze ohne \textit{doch}. V1-Sätze hingegen brauchen einen speziellen Marker, um ihre Interpretation sicherzustellen. In den kausalen V1-Deklarativsätzen ist dies das \textit{doch}. Wie in Abschnitt~\ref{sec:unkontr} gesehen, weisen auch die anderen deklarativen V1-Sätze sprachliche Besonderheiten auf. 

Wenn \textit{doch} in diesen Sätzen indirekt einen Beitrag zur kausalen Lesart leistet, schließt sich die Frage an, warum nicht eine andere Partikel verwendet wird, die Kausalität besser kodiert. \textit{Doch} wird inhärent nicht kausal interpretiert. Ginge es in diesen Sätzen nur darum, die sowieso schon vorhandene kausale Verknüpfung zwischen den Sätzen zu forcieren, würden sich durchaus andere MPn anbieten, die dies nicht indirekt und nur auf Umwegen leisten können. Schließlich gibt es MPn, die Kausalität direkt anzeigen, und zwar \textit{halt}, \textit{eben} und \textit{auch}, die in Abschnitt~\ref{sec:kontexte} in Kapitel~\ref{chapter:hue} und Abschnitt~\ref{sec:auch} behandelt werden. Den Ausschluss dieser Partikeln motiviere ich im Folgenden in den Abschnitten~\ref{sec:transdoch} und \ref{sec:litdoch} unter Bezug auf zwei weitere Aspekte dieser Sätze. Einen solchen Ausschluss leisten Önnerfors und Pittner nicht. Beide gehen vom Vorliegen von Unkontroverse/Bekanntheit/ Präsupponiertheit aus. Nach meiner Bedeutungsmodellierung schließen diese Bedeutungsaspekte die Verwendung von \textit{halt} und \textit{auch} aus, da sie die Proposition, auf die sie sich beziehen, assertieren. \textit{Eben} käme allerdings in Frage, da es sowohl Präsupponiertheit als auch Kausa\-lität kodiert. Da ich gegen den präsupponierten Status des Satzinhaltes argumentiere, kommen im Rahmen meiner Ableitung bisher auch \textit{halt} und \textit{auch} in Frage.

\subsection{Die Transparenz der \textit{doch}-Bedeutung}
\label{sec:transdoch}
Der erste Aspekt, der die Eignung von \textit{doch} unterstreicht, ist, dass ich nicht glaube, dass die Partikel ausschließlich dem Anzeigen von Kontextbezug dient. Geht man von der Modellierung aus, dass sie auf ein vorausgesetztes, offenes Thema reagiert, lässt sich von einer transparenten Verwendung ausgehen. Die Tatsache, dass Önnerfors vertritt, die \textit{doch}-Bedeutung sei in den V1-Sätzen ausgeblichen, ist folglich auf die Bedeutung zurückzuführen, auf die er sich stützt. Auch Pittner, die prinzipiell für das Vorliegen der regulären \textit{doch}-Bedeutung argumentiert, zeigt an keinem V1-Satz auf, inwiefern sie die Anweisung \glq Ersetze $\neg$p durch p!\grq {} vorliegen sieht.

Schaut man sich die Kontexte genauer an, scheint es nicht abwegig, die \textit{doch}-Bedeutung zu rekonstruieren, und zwar fern der indirekten Funktion im kausalen Zusammenhang. (\ref{1039}) zeigt einen \textit{Wo}-VL-Satz und es lässt sich motivieren, warum nach der vorweggehenden Frage das Thema offen ist, ob jeder weiß, dass donnerstags die neuen Filme anlaufen.

\begin{exe}
	\ex\label{1039} 
	\scriptsize
	Wieso müssen Agenturpartys immer am Donnerstag sein? \underline{\textbf{\textit{Wo} doch \textit{jeder weiß}}},\\ 
	\underline{\textbf{dass Donnerstags die neuen Filme anlaufen}}.                                      	
	\newline              		
	\hbox{}\hfill\hbox {(http://www.ankegroener.de/anke1/pasdeblog/blogarchiv/september2002.html)}
\end{exe}
Wenn die Leute, die die Agenturpartys planen, diese auf Donnerstag legen, kann man sich fragen, ob sie nicht wissen, dass an diesem Tag das Kino stattfindet, da sie die Partys ansonsten nicht auf diesen Termin legen würden. Die Ableitung der offenen Frage, ob sie nicht um das Kino wissen, geschieht also vor dem Hintergrund, dass man eine Agenturparty normalerweise nicht auf den Donnerstag legen würde, wenn man wüsste, dass dann Kino ist.  Die Leute, die die Partys planen, wissen es scheinbar nicht. Der Sprecher kann aber nicht so recht glauben, dass sie dies nicht wissen. Es kann somit auch als auf dem Tisch liegend ausgegeben werden, dass fraglich ist, ob sie p wissen. Diese Frage beantwortet der Sprecher dahingehend, dass sie es s.E. wissen müssen, indem er sagt, dass es jeder weiß (also auch diese Leute).

Es lässt sich aus den konzessiven Fällen immer die Offenheit von p ableiten. Die vorkommenden Einstellungen unter dieser Interpretation (Verwunderung, Erstauntsein, Sichfragen) kommen überhaupt zustande, weil der beteiligte Sachverhalt normalerweise anders zu erwarten wäre. Und in diesem Sinne kann der Sachverhalt, der der Grund für diese Einstellung ist, immer in Frage stehen, weil der Sprecher/Schreiber von der gegenteiligen Annahme ausgeht. Unter Bezug auf (\ref{1039}) bedeutet dies: Obwohl in den Augen des Sprechers jeder weiß, dass donnerstags die Filme anlaufen, finden dann die Partys statt. $\rightarrow$ Weiß es nicht jeder? $\rightarrow$ Wissen die Organisatoren es nicht? $\rightarrow$ Dass in den Augen des Sprechers jeder weiß, dass donnerstags die Filme anlaufen, begründet seine Verwunderung.

(\ref{1040}) zeigt einen V1-Satz, für den eine parallele Analyse möglich ist.

\begin{exe}
	\ex\label{1040} 
	\scriptsize
	Ich bin klar enttäuscht über das Resultat der FDP. Das schlechte Abschneiden ist sehr überraschend. \textbf{\textit{Führten} die Freisinnigen 				\underline{doch} einen super Wahlkampf} – ganz im Gegensatz zu den anderen Parteien. 
	\hfill\hbox {(A08/SEP.09380 St. Galler Tagblatt, 29.09.2008)}
\end{exe}
In (\ref{1040}) ist die konzessive Relation \glq Obwohl sie einen tollen Wahlkampf geführt haben, haben sie schlecht abgeschnitten.\grq {} und die kausale \glq Ich wundere mich über das schlechte Abschneiden, weil sie einen tollen Wahlkampf hatten.\grq {} Aufgrund des schlechten Abschneidens kann man sich fragen, ob sie keinen guten Wahlkampf geführt haben, weil ein guter Wahlkampf normalerweise zu gutem Abschneiden führt bzw. andersherum aus schlechtem Abschneiden ein schlechter Wahlkampf abzuleiten ist. Die Frage, ob der Wahlkampf schlecht war, wird negativ beantwortet und erklärt die Verwunderung: Sie hatten einen super Wahlkampf, aber sie haben schlecht abgeschnitten.

Und auch wenn Konzessivität nicht beteiligt ist, bin ich der Meinung, dass sich die Offenheit des Themas (anders als das Vorliegen der der \textit{doch}-Äußerung entgegengesetzten Proposition) motivieren lässt, so dass \textit{doch} auch in diesem Fall eine völlig transparente Verwendung zuzuschreiben ist.

In (\ref{1041ab}) begründet der Sprecher seine Einschätzung, dass Markus Haag stolz sein kann damit, dass er 99\% der Stimmen erhalten hat.

\begin{exe}
	\ex\label{1041ab} 
	\scriptsize
	Wattwils Gemeindammann Markus Haag hingegen darf auf sein Wahlresultat überaus stolz sein. \textbf{\textit{Sprachen} \underline{doch} 99 Prozent all 		derer, die an die Urne gingen, ihm ihr Vertrauen aus und gaben ihm die Stimme.}
	\hfill\hbox {(A00/SEP.65537 St. Galler Tagblatt, 25.09.2000)}
\end{exe}
Wie in Abschnitt~\ref{sec:kausalind} bereits angeführt, ist davon auszugehen, dass Sprecher – bemüht um Kohärenz – zu einer kausalen Default-Interpretation \is{kausale Default-Interpretation} tendieren. Wenn es sich anbie\-tet, beziehen sie aufeinander folgende Sätze kausal aufeinander. Der Adressat rechnet folglich damit, dass der Folgesatz als Begründung des vorweggehenden Satzes fungieren kann. Für (\ref{1041ab}) würde dies bedeuten, dass der Rezipient nach Abschluss der Aufnahme des ersten Satzes die Erwartungshaltung aufweist, dass eine der denkbaren Begründungen, derer es sicherlich verschiedene gibt (vgl. z.B. p, r und s in (\ref{1041})), die zutreffende Begründung ist und der Sprecher sie somit im Folgesatz vertreten könnte.

\begin{exe}
	\ex\label{1041} 
		\begin{xlist}	
			\ex\label{1041a} Wenn man 99\% der Stimmen bekommt (p), kann man stolz sein (q).
			\ex\label{1041b} Wenn man als sehr junger Mensch kandidiert (r), kann man stolz sein (q).
			\ex\label{1041c} Wenn man gegen einen Ortsansässigen antritt (s), kann man stolz sein (q).
		\end{xlist}
\end{exe}
Wenn q (die Folge) assertiert wird, stellt sich dieser Überlegung nach für p, r und s die Frage, ob dies der tatsächliche Grund ist. Prinzipiell angeführt werden könnten sie alle. Und da sie alle als Grund denkbar wären, stellt sich auch für alle diese in Frage kommenden Propositionen – als Voraussetzung, als Grund zu fungieren – ob sie im Diskurs Gültigkeit haben. Bevor der V1-Satz in (\ref{1041ab}) geäußert wird, steht in diesem Sinne bereits im Raum, ob p/r/s gelten. Diese Überlegung baut dann natürlich darauf, dass sich der Inhalt des V1/\textit{Wo}-Satzes plausibel als Begründung des ersten eignet. Von dieser Konstellation gehen auch schon \citet{Oennerfors1997} und \citet{Pittner2011} (für die V1-Sätze) aus. Diese Annahme, dass in diesem Sinne nach jeder Äußerung prinzipiell denkbare Begründungen in Frage stehen, muss nicht heißen, dass \textit{doch} deshalb präferiert in kausal interpretierten Sätzen auftritt (ob durch Konnektoren ausgezeichnet oder asyndetisch \is{Asyndese} verbunden). Das Aufdecken der kodierten Interpretation, die im Kontext vorliegt, ist schließlich kein Muss. \textit{Doch} sollte allerdings in derart interpretierten Sätzen stehen können. 

Eine Frage, die man an dieser Stelle berechtigterweise aufwerfen kann, ist, ob die Vorstellung, dass jede Äußerung Möglichkeiten ihrer Begründung evoziert, nicht zu redundant ist. Ich meine jedoch, dass in Verbindung mit dieser Art von kausalen Relation, die bei den V1- und \textit{Wo}-VL-Sätzen vorliegt, in größerem Maße von der Erwartungshaltung des Adressaten, eine Begründung zu erhalten, ausgegangen werden kann. Wie zu Beginn der Diskussion dieser Satztypen ausgeführt, begründen \textit{Wo}-VL- und V1-Sätze stets auf \is{epistemischer Kausalsatz} der epistemischen oder \is{illokutionärer Kausalsatz} illokutionären Ebene. 

\citet{Gohl2000} untersucht asyndetische kausal interpretierte Strukturen, die vor\-herige Konversationszüge erklären oder rechtfertigen, wie z.B. in (\ref{1042}), wo die unterstrichenen Beiträge Vorschläge bzw. dispräferierte Reaktionen auf solche (s.u.) begründen.
		
\begin{exe}
	\scriptsize
	\ex\label{1042} 
	\begin{tabular}[t]{ll}
	\multicolumn{2}{l}{MEAT (Schwab 8; 15:18)}\\
	1 Erik:	& tausche mer $<<$p$>$ \underline{deins isch kleiner}.$>$\\
	2 Kai: & gar net –\\
	3 Uwe: & noi nemm du \underline{i will bloße schtü$[$ck};=\\
	 & \hspace{5cm}$[$nein (2 Silben) esse.\\
	4 Kai: & i will doch net solche brocken.
	\hfill\hbox{\citet[100]{Gohl2000}} 						 
    \end{tabular}       
\end{exe}														      
Die untersuchten Strukturen entsprechen am ehesten epistemischen oder illokutionären Begründungen, da höhere Einheiten als Sachverhalte begründet werden. \citet[103]{Gohl2000} fragt sich, in Reaktion auf welche Beiträge derartige Begründungen angeführt werden und kommt zu dem Schluss, dass die vorherige Äußerung ihrer bedarf: 

\begin{quotation}
An utterance is much more likely to be interpreted as an account, i.e. as an utterance explaining and/or justifying a previous conversational move, if the preceding utterance, by virtue of its sequential and social implications, calls for accounting.                                                                                             
\end{quotation}
\citet[58, Fn 7]{Breindl2006} gehen davon aus, dass im Grunde jede nicht-rituelle Äußerung strittig sein kann. Gohl benennt allerdings einige konkrete Fälle, auf die erklärende oder rechtfertigende Begründungen in ihren Daten folgen. Sie nennt hier die unerwünschte Reaktion auf eine vorherige Aktion (z.B. ein abgelehnter Rat, die Unmöglichkeit, eine Frage zu beantworten), Bewertungen, Aufforderungen, Beschwerden und Vorwürfe. \citet[289-297]{Ford2000} führt als weiteren Fall einen vorweggehenden Kontrast an (vgl. z.B. (\ref{1043})), wobei das Konzept sehr weit gefasst wird (explizit gegensätzliche Propositionen, Negation präsupponierter Information, Uneinigkeit zwischen Sprechern):

\begin{exe}
	\ex\label{1043} 
    \begin{tabular}[t]{ll}
	& S: I have a check.\\
	& S: Eight fifty.\\
	contrast & J: Nah. I won t take – I don t take second-party checks.\\
	& S: eh huh huh huh huh\\
	explanation & J: I don't got no way of telecheking  em,						 
    \end{tabular} \\
    \hbox{}\hfill\hbox{\citet[294]{Ford2000}} 
\end{exe}	
Das typische Muster contrast + explanation erklärt \citet[289]{Ford2000} auf die Art, dass die Diskursteilnehmer Kontraste als Problem darstellen, die eine Erklärung erforderlich machen. \citet[103]{Gohl2000} nimmt an, dass im Falle der uner\-wünschten Reaktion auf die vorherige Aktion sowie bei Aufforderungen, Be\-schwerden und Vorwürfen gesichtsbedrohende, konversationell sensible Züge vorliegen, die deshalb einer Stützung bedürfen. Und diese fördere auch eine subjektive Sprechereinschätzung.

\citet{Gohl2000} und \citet{Ford2000} benennen folglich konkrete Anlässe für Erklärungen und Rechtfertigungen, die der kausalen Relation auf der epistemischen und illokutionären \is{epistemischer Kausalsatz} Ebene \is{illokutionärer Kausalsatz} entsprechen. Man könnte sagen, dass es sich um sprachliche Handlungen handelt, an die die Erwartung einer Erklärung (aus Adressaten\-perspektive) bzw. die Annahme des Bedarfs (aus Sprecherperspektive) geknüpft ist.

Für meine Betrachtung ist interessant, dass \citet{Ford1993, Ford2000} und \citet{Gohl2000} derartige Kontexte benennen. Ich gehe davon aus, dass sich aus der vorweggehenden Äußerung ableiten lässt, dass der Sachverhalt, der als Begründung angeführt wird, tatsächlich offen ist und nicht nur als solcher vorgegeben wird. Er ist dies meiner Argumentation nach deshalb, weil mit einer Erklärung zu rechnen ist. Die Inhalte der plausiblerweise in Frage kommenden Begründungen stehen zur Einigung im Raum, da ihre jeweilige Akzeptanz Voraussetzung dafür ist, dass sie auch die Erklärung ausmachen können. (\ref{1044}) zeigt, mit welchen Anteilen die Kontexte aus Gohl und Ford in einer Zufallsstichprobe der Größe 100 aus der exhaustiven Menge aller \textit{Wo}- bzw. V1-Sätze aus dem Archiv Tagged C in DeReKo vorkommen.

\begin{exe}
	\ex\label{1044} 
    \begin{tabular}[t]{|l|l|l|}
    \hline
	& V1 & \textit{Wo}-VL\\
	\hline
	Kontrast & 14 & 10\\
	\hline
	Bewertung & 35 & 32\\
	unerwartete Reaktion & - & -\\
	\hline
	Aufforderung & - & -\\
	\hline
	Beschwerde & - & 8\\
	\hline
	Vorwurf & - & 34\\
	\hline
	anderes & 51 & 16\\
	\hline	
    \end{tabular}   
\end{exe}
Schon in Bezug auf die Arbeiten von Gohl und Ford ist zu sagen, dass es schwierig ist, mehr festzustellen, als dass diese Kontexte mit Erklärungen  auftreten (können), weil dort keine Angaben über ihr prinzipielles Vorkommen gemacht werden. \citet[548]{Ford1993} schreibt, in 53\% ihrer Fälle handle es sich um den Kontrastkontext. Ich halte Kontrast \is{Kontrast} allerdings für eine derart unspezifische/typische Kategorie, dass man den Grad von Besonderheit erst einschätzen kann, wenn man weiß, welchen Anteil derartige Kontexte in ihrem Korpus haben. Das glei\-che gilt für Bewertungen, die \citet{Ford1993} auch schon anführt. Da in Gohl gar keine Zahlen genannt werden, sind die Annahmen in dieser Hinsicht auch schwer einzuschätzen. 

Bei den V1-Sätzen fällt ungefähr die Hälfte der Belege in die Kategorien \textit{Be\-wertung} \is{Bewertung} und \textit{Kontrast} \is{Kontrast} und diese Kategorien sind bei den \textit{Wo}-Sätzen ähnlich häufig vertreten. Nicht unproblematisch ist es, die Grenze zu anderen Kategorien zu ziehen. Unter den V1-Daten finden sich viele Annahmen, Vermutungen und Einschätzungen, die ich nicht im engen Sinne als Bewertungen aufgefasst habe. Tut man dies, erhöht sich die Zahl hier. Kontrast ist natürlich auch bei Einstellungen wie Erstaunen und Wundern beteiligt, bei denen es sich letztlich aber auch um Bewertungen handelt. Man könnte auch denken, dass Kontrast bei\-spielsweise anzeigt/begünstigt, dass etwas als Annahme verstanden wird. Begründet wird nämlich in keinem Fall die Kontrastrelation selbst. Wenn sich Kategorien überlappen, ist die Frage, welche Kategorie die wirklich relevante ist. Interessant ist, dass sich zwischen den beiden Satztypen eine Arbeitsteilung he\-rauskristallisiert: Beschwerden und Vorwürfe sind in den Vorgängeräußerungen der \textit{Wo}-Sätze dominant vertreten, in dieser Stichprobe aber absent in den V1-Daten. Dieser Tatbestand hat damit zu tun, dass die \textit{Wo}-Sätze häufiger als die V1-Sätze zusätzlich zu ihrer kausalen Interpretation das konzessive Bedeutungsmoment aufweisen (vgl. die Verteilung in (\ref{1045})).

\begin{exe}
	\ex\label{1045} 
    \begin{tabular}[t]{|l|l|l|}
    \hline
	& V1 & \textit{Wo}-VL\\
	\hline
	kausal & 87 & 16\\
	\hline
	kausal + konzessiv & 13 & 84\\
	\hline
    \end{tabular}   
\end{exe}
Wie ich oben ausgeführt habe, ist die Konzessivität auf die Begründung be\-stimmter Einstellungen zurückzuführen. Da Beschwerden/Vorwürfe durch die Beurteilung eines als abweichend empfundenen Verhaltens zustandekommen, ist nicht verwunderlich, dass in diesem Zusammenhang Konzessivität eine Rolle spielt (vgl. (\ref{1046}) bis (\ref{1049})).

\begin{exe}
	\ex\label{1046} 
	\scriptsize
	$[$...$]$ Warum denn ins Konzert gehn, wenn ich so schöne CDs zu Hause habe. Somit hört der Tag auf, wie er angefangen hat. Binär.\\
	Binär? Wissen sie, was das heisst? Nein? \textbf{\textit{Wo} \underline{doch} heute alles binär ist.} 0 oder 1, keine andere Zahl.
	\newline              		
	\hbox{}\hfill\hbox {(A98/APR.22936 St. Galler Tagblatt, 11.04.1998, Ressort: RT-PIA (Abk.); gast)}
\end{exe}
			                     
\begin{exe}
	\ex\label{1047} 
	\begin{tabular}[t]{ll}
	Zustandekommen des Vorwurfs: & Obwohl alles binär ist, wissen Sie \\
	& nicht, was \textit{binär} bedeutet.
	\end{tabular}
\end{exe}
	
\begin{exe}
	\ex\label{1048} 
	\scriptsize
	Der junge Arzt, der mich jetzt betreut, macht mir klar, dass Magen und Darm eingehend endoskopisch untersucht werden müssen. Und jetzt hält unser 			Gesundheitswesen eine faustdicke Überraschung für mich parat: Diese Untersuchungen laufen nicht mehr in der Klinik, dafür muss ich mir in einer 			Facharztpraxis einen Termin geben lassen. Das kann Wochen dauern. Ist das jetzt eine sinnvolle und kostengünstige Form der Diagnostik? 						\textbf{\textit{Wo} ich \underline{doch} schon im Krankenhaus bin.}
	\newline              		
	\hbox{}\hfill\hbox {(RHZ08/NOV.25632 Rhein-Zeitung, 29.11.2008)}
\end{exe}	

\begin{exe}
	\ex\label{1049} 
	\begin{tabular}[t]{ll}
	Zustandekommen der Beschwerde: & Obwohl ich schon im Krankenhaus \\
	& bin, muss ich zur Untersuchung \\
	& zum Facharzt in eine Praxis.
	\end{tabular}
\end{exe}
Die häufiger festzustellende konzessive Lesart der \textit{Wo}-Sätze geht somit einher mit der Begründung bestimmter Einstellungen und dem Vorliegen bestimmter Äußerungstypen. V1-Sätze werden eher verwendet, um Argumente und Annahmen zu stützen, \textit{Wo}-Sätze treten eher in Kontexten auf, in denen Ärger, Entsetzen, gegenteilige Erwartungen oder abweichendes Verhalten beteiligt sind. Letz\-teres macht den Anteil begründeter Beschwerden/Vorwürfe nachvollziehbar. Eine Beobachtung ist, dass für die Mehrheit der V1-Sätze in meiner Stichprobe gilt, dass sie nur schwer auszulassen sind (68\%). Dies spricht für die gesteigerte Erwartung ihres Vorkommens. Der Eindruck ist aber nicht nur auf das Auftreten der von Ford und Gohl angeführten Äußerungstypen zurückzuführen. Es gehen den V1-Sätzen mitunter auch Annahmen voraus, die aufgrund ihres Inhalts einer weiteren Erklärung bedürfen, weil sie z.B. Ungenauigkeiten oder Andeutungen beinhalten und meiner Mei\-nung nach schlecht ohne die folgenden Erklärungen stehen bleiben können. (\ref{1050}) bis (\ref{1053}) zeigt einige Beispiele.

\begin{exe}
	\ex\label{1050} 
	\scriptsize
	Engelburg. Die SVP und Bruno Stump, zwei Namen, die oft im gleichen Atemzug genannt werden. Zumindest in der Region St. Gallen–Gossau. Dort ist der 		66jährige Kantonsrat bekannt wie kaum ein anderer SVPler. \textbf{\textit{Gründete} er \underline{doch} 1995 die SVP des Bezirks Gossau}, danach die 		SVP Waldkirch und im Jahr 2000 die SVP Gaiserwald, die heute 60 Mitglieder zählt. Dieser stand er seither als Präsident vor.        
	\hfill\hbox {(A09/MAR.06564 St. Galler Tagblatt, 20.03.2009)}
\end{exe}

\begin{exe}
	\ex\label{1051} 
	\scriptsize
	Der Teufel steckt allerdings im Detail. \textbf{\textit{Hatten} \underline{doch} Union und SPD recht eigenwillige Einzelheiten auf der Latte}, zu denen 	noch eine Fülle von Expertenmeinungen außerhalb der Koalitionsrunde kamen.                 
	\hfill\hbox {(NUZ09/JAN.00384 Nürnberger Zeitung, 06.01.2009)}
\end{exe}                                              

\begin{exe}
	\ex\label{1052} 
	\scriptsize
	Den grössten Clou landete aber die Gattin des OK-Präsidenten Karin Hanselmann aus Oberriet. \textbf{\textit{Zeigte} sie \underline{doch} mit ihrem 			souveränen Ritt in der höchst dotierten Prüfung des Anlasses}, was in ihrem Pferd Canetta und in ihr steckt.                      
	\hfill\hbox {(A08/MAR.03654 St. Galler Tagblatt, 10.03.2008)}
\end{exe}	                                     

\begin{exe}
	\ex\label{1053} 
	\scriptsize
	Die Erzbahn, in der Ausstellung braun gestaltet, stellte die Ingenieure der schwedischen Eisenbahn vor besondere Herausforderungen. 						\textbf{\textit{Mussten} sie ihre Gleise und Bahnhöfe \underline{doch} ins \glqq buchstäbliche Nichts\grqq{} setzen}, erklärt Rainer Merten:                     
	\hfill\hbox {(NUZ08/DEZ.00406 Nürnberger Zeitung, 04.12.2008)}
\end{exe}
Mein Eindruck ist, dass dies weniger deutlich auf die \textit{Wo}-Sätze zutrifft (55\% sind schlecht weglassbar). Ich finde Gohls Überlegung, dass Beschwerden und Vorwürfe aus sozialen Gründen die Erwartung einer Begründung mit sich führen, sehr nachvollziehbar. Anscheinend führt diese Erwartungshaltung aber nicht so sehr zu dem Eindruck eines (bei Auslassung der Begründung) unvollständigen Diskurses, wie er sich meiner Meinung nach bei den V1-Sätzen in vielen Fällen einstellt.

Die Annahme der erwarteten Begründung ist in meiner Argumentation wichtig, um die Offenheit des Sachverhalts, der mit dem V1-/\textit{Wo}-VL-Satz ausgedrückt wird, in den Fällen zu motivieren, in denen kein konzessiver Bedeutungsanteil vorliegt. Dies betrifft – wie die Zahlen oben belegen – mehr die V1- als die \textit{Wo}-Sätze. Der Nachweis der Erwartung der Begründung ist im Rahmen meiner Argumentation allerdings in den \textit{Wo}-Sätzen auch weniger relevant als bei den V1-Sätzen. Ist Konzessivität beteiligt, ist die Offenheit über die nicht erfüllte/in Frage gestellte Erwartung zu motivieren. 

Insgesamt führen die Überlegungen und empirischen Untersuchungen zu den Gebrauchsweisen der beiden Satztypen dazu, dass ich unter der von mir vertretenen \textit{doch}-Bedeutung keinen Grund sehe, sagen zu müssen, dass nur eine uneigentliche Verwendung vorliegt, über die (durch den als solchen zu deutenden engen Kontextbezug) die kausale Verbindung zwischen den Sätzen forciert wird.

Wird in \textit{Wo}-VL- und V1-Sätzen tatsächlich die Offenheit des Themas ausgedrückt, erklärt sich auch der Ausschluss des (isolierten) Auftretens von \textit{eben}, \textit{halt} und \textit{auch}.

Der nächste Abschnitt untersucht einen letzten Aspekt, der \textit{Wo}-VL- und V1-Sätzen zugeschrieben worden ist und der leichter mit dem Beitrag von \textit{doch} als dem der hinsichtlich anderer Aspekte ggf. ebenfalls geeigneten Partikeln in Einklang zu bringen ist: eine gewisse emotionale Involviertheit.

\subsection{Expressivität/Emotionalität}
\label{sec:litdoch}
Über beide Satztypen ist gesagt worden, dass sie expressiv verstärkt sind (vgl. \citealt[204]{Oppenrieder1989} über \textit{wo}-VL- und V1-Sätze, \citeyear[42]{Oppenrieder2013} über \textit{wo}-VL-Sätze). Für V1-Sätze (insbesondere für den narrativen deklarativen Typ) wird dazu generell für Expressivität argumentiert (vgl. \citealt[218]{Reis2000}). Wenngleich man die Formulierung dieses Eindrucks findet, konnte ich weiter nicht aufdecken, was genau hinter dieser Auffassung steckt. Inwiefern diese Sätze expressiver \is{Expressivität} sind als andere und inwiefern dies möglicherweise mit ihrem Gebrauch einhergeht, wird nicht ausgeführt.

Obwohl der folgende Aspekt noch genauerer empirischer Beschäftigung bedarf, möchte ich die Annahme vertreten, dass sich sowohl in den Zeitungsdaten als auch in literarischen Beispielen Verwendungskontexte der beiden Satztypen ausmachen lassen, denen ohne Weiteres Expressivität zugeschrieben werden kann.

In DeReKo treten die \textit{Wo}-Sätze vornehmlich in Texten auf, die man dem Feuilleton zuordnen würde, wie z.B. Kolumnen. In diesem Sinne liegen recht \glq lockere\grq {} Kontexte vor. Typisch sind Fälle wie in (\ref{1048}) und (\ref{1054}).

\begin{exe}
	\ex\label{1054} 
	\scriptsize
	Was ich denn so von den Spielerfrauen auf der Tribüne halte, insbesondere von Frau Beckham? Unnatürlich und eindeutig fit gespritzt, lautet meine 			schlaue Antwort. Okay, die Klippe wäre umschifft. Warum ich denn kein Autogramm von Lukas Podolski mitgebracht habe, mischt sich unsere Tochter ein. 		Die hat es nötig, denke ich. \textbf{\textit{Wo} sie \underline{doch} sonst jeden oberpeinlich-uncool findet, der nur das Wort Fußball in den Mund 			nimmt.}
	\hfill\hbox {(BRZ06/JUL.01343 Braunschweiger Zeitung, 04.07.2006)}
\end{exe}
Ein Ich-Erzähler berichtet von einem Erlebnis und empört sich oftmals. Dies geht einher mit der Beobachtung aus dem letzten Abschnitt, dass \textit{Wo}-Sätze oftmals auf Beschwerden und Vorwürfe folgen. Diese Einstellung kann auch durchaus performativ zum Ausdruck gebracht werden (vgl. (\ref{1055}), (\ref{1056})).

\begin{exe}
	\ex\label{1055} 
	\scriptsize
	Und wenn wir ihr irgendetwas ganz Besonderes zum Valentinstag schenken möchten, dann ist das nicht weniger als die Bereitschaft unserer Arme, sie in 		diese zu nehmen und zu schützen – vor der bösen Männer-Welt mit ihrer herzlosen Hämoglobin-Häme.\\
	\emph{16,3!} \textbf{\textit{Wo} \underline{doch} jeder weiß, wie dieser Wert zustande kam} – durch den Überschuss an roten Valentinsherzchen, die so 		viel Liebe nicht mehr ertrugen und sich deshalb alle in rote Blutkörperchen verwandelten. Wie die Valentinsherzchen in Evis süßen Körper gelangten? 
	\newline              		
	\hbox{}\hfill\hbox {(M06/FEB.12331 Mannheimer Morgen, 14.02.2006; Blutherzchen)}
\end{exe}

\begin{exe}
	\ex\label{1056} 
	\scriptsize
	Ein Hockey-Trainer im DFB, \emph{igitigitt!}\\
	\textbf{\textit{Wo} dieser Verband \underline{doch} Koryphäen der Ausbildung wie den von Sportdirektor Matthias Sammer, welcher statt Peters das Rennen 	machte, aus dem Ruhestand reaktivierten Erich Rutemöller vorzuweisen hat.}   
	\hfill\hbox {(M06/SEP.76347 Mannheimer Morgen, 29.09.2006)}
\end{exe}
Es finden sich auch \textit{Wo}-Sätze, deren Interpunktion zusätzlich eine expressive Intonation nahelegt (vgl. (\ref{1057}), (\ref{1058})).

\begin{exe}
	\ex\label{1057} 
	\scriptsize
	Julia Fischer, gerade mal 25 Jahre alt, ist die bestgelaunte Geigerin der Welt und strahlt mit dem charmantesten Lächeln. Zu diesem positiven Befund 		kommen wir nach ihrem Gastspiel, das jüngst und angeblich im Heidelberger Frühling stattfand. \textbf{\textit{Wo} \underline{doch} noch tiefer Winter 		ist!} 	  
	\newline              		
	\hbox{}\hfill\hbox {(M09/JAN.07195 Mannheimer Morgen, 28.01.2009)}
\end{exe}

\begin{exe}
	\ex\label{1058} 
	\scriptsize
	Sein Pech: Zugleich setzt sich sein Sohn samt Braut per Ballon in den Westen ab, und der Vater hatte noch ahnungslos die Materialien beschafft. Jetzt 		steht er als Fluchthelfer da, im Westen gefeiert, im Osten verdammt. \textbf{\textit{Wo} er \underline{doch} so gern zurück zu seiner Braut will!} 	  
	\newline              		
	\hbox{}\hfill\hbox {(M07/OKT.00333 Mannheimer Morgen, 02.10.2007)}
\end{exe}	
Wenn die Einstellung, die durch den \textit{Wo}-Satz begründet wird, nicht performativ vorliegt, treten Ausdrücke auf, die \is{exklamatives Prädikat} unter \textit{exklamative} (\citealt[39-40]{Avis2001}) (z.B. \textit{überrascht}/\textit{fasziniert sein}, \textit{komisch}/\textit{nicht normal}/\textit{erschreckend finden}) oder auch \is{emotiv-faktives Prädikat} \textit{emotiv faktive} (\citealt[363]{Kiparsky1970}) (z.B. \textit{sich freuen}/\textit{ärgern}, \textit{ge\-nervt}/\textit{traurig sein}) Prädikate fallen. Letztere sind affektiv aufgeladen, weil eine subjektive Einschätzung der Proposition vorgenommen wird. Bei Exklamativsätzen kommt die angenommene Expressivität durch die Wahrnehmung des Unerwarte\-ten/der Normabweichung/nicht erfüllten Erwartung zustande. Da genau diese Einstellung in den konzessiven \textit{Wo}-Fällen beteiligt ist, treten auch entsprechende Prädikate häufig auf.

In anderen Fällen wird der Leser direkt und im Falle hier auch typischerweise auftretender rhetorischer Fragen dazu provokativ angesprochen (vgl. (\ref{1046}) und (\ref{1059})).

\begin{exe}
	\ex\label{1059} 
	\scriptsize
	Nun stellt sich die Frage, wie herzlos Seehund-Mamas sein müssen, um ihre erst wenige Tage alten Sprösslinge zurückzulassen? Klar, so ein Junges 			überfordert eine allein erziehende Seehund-Mama schon mal. Dauernd das Genörgel nach frischem Fisch, das fiese Kitzeln der kleinen Barthaare beim 			Säugen. Aber sind das Gründe, ein Baby auszusetzten? \textbf{\textit{Wo} es \underline{doch} (noch) keine Seehund-Babyklappen gibt.}  
	\hfill\hbox {(HMP06/JUN.00808 Hamburger Morgenpost, 09.06.2006)}
\end{exe}
Die vorweggehenden, zu begründenden Einstellungen sind im Falle der \textit{Wo}-Sätze folglich selbst deutlich emotional aufgeladen. Ist die begründete Einstellung affektiv aufgeladen, scheint es unnatürlich, dass eine rationale, neutrale oder indifferente Begründung folgt.

Wie in Abschnitt~\ref{sec:transdoch} gesehen, sind die Begründungskontexte der V1-Sätze nicht identisch mit denen der \textit{Wo}-Sätze. Exklamative Prädikate oder entsprechende performative Varianten treten eher wenig auf. Emotive Einstellungen sind auch vertreten. Das Gros der Fälle machen aber epistemische Einschätzungen aus (d.h. Annahmen, Vermutungen, Hypothesen, Überlegungen) (wie z.B. in (\ref{1060}) und(\ref{1061})).

\begin{exe}
	\ex\label{1060} 
	\scriptsize
	Der 73-Jährige, der über 19 Jahre lang an der Spitze des Olympischen Comités fungierte und auch Österreichs einziges Mitglied im Internationalen 			Olympischen Comité (IOC) ist, hätte sich seinen Abgang aber \emph{sicher} anders vorgestellt. \textbf{\textit{Galt} \underline{doch} gerade der 			ehemalige Casino-General seit Jahren als die graue Eminenz im heimischen Sport} – als ein Sir schlechthin. 
	\newline              		
	\hbox{}\hfill\hbox {(NON09/SEP.03193 Niederösterreichische Nachrichten, 07.09.2009)}
\end{exe}
				 
\begin{exe}
	\ex\label{1061} 
	\scriptsize
	Früher ein Duell mit unsicherem Ausgang, \emph{dürfte} die Sache für die Öko-Ritter aus Güssing auch auswärts kein Problem sein. \textbf{\textit{Liegt} 	die Vogler-Truppe \underline{doch} klar auf Platz eins der Tabelle in der Hauptrunde zwei.}
	\hfill\hbox {(BVZ07/APR.00677 Burgenländische Volkszeitung, 11.04.2007)}
\end{exe}	
Die begründeten Einstellungen, die in meiner obigen Argumentation generell einen emotionaleren Kontext schaffen, sind hier folglich weniger deutlich expressiv aufgeladen als bei den \textit{Wo}-Sätzen. Doch denke ich, dass der Sprecher bei jeder modalen Begründung \is{modaler Kausalsatz} spürbarer ist als bei einer propositionalen, da diese immer subjektiviert \is{Subjektivierung} ist. Und da auch mit den V1-Sätzen keine Sachverhalte, sondern Einstellungen/Ansichten begründet werden, ist Subjektivität und damit emotionale Involviertheit sowohl in der Begründung als auch in dem Begründeten ebenfalls beteiligt.

Der Eindruck der verstärkten Expressivität der beiden Satztypen lässt sich meiner Meinung nach folglich in dem Sinne bestätigen, dass der gesamte Kontext (begründete Einstellung + Begründung) aufgrund der Tatsache, dass Einstellungen motiviert werden, eine gewisse affektive Aufladung vorweist. Man kann sich schlecht vorstellen, dass sie vor der Begründung abbricht. Die Geeignetheit der Partikel \textit{doch} lässt sich vor diesem Hintergrund so erklären, dass das als offen vorausgesetzte Thema Involviertheit beim Sprecher/Schreiber schafft, da dieser in eine bereits bestehende Diskussion einsteigt. In den kolumnenartigen Texten scheint er mir in den Dialog mit dem Leser treten zu wollen und ihn somit involvieren zu wollen. Je nach begründeter Einstellung wird er auch schon in der vorangehenden Äußerung (oder im weiteren Kontext) direkt angesprochen. Durch die Verwendung von \textit{doch} ist der Leser sofort beteiligt, weil Schreiber und Leser ein gemeinsames Konversationsthema aufweisen, zu dessen Auflösung der Sprecher/Schreiber den \glq halben\grq {} Beitrag geleistet hat und die Reaktion des Lesers erwünscht ist (auch wenn hier natürlich kein direkter Dialog zustande kommt).

Betrachtet man das Auftreten der beiden Strukturen in literarischen Daten, fällt auf, dass sie häufig (und auch dies gilt es, empirisch stichhaltiger zu machen) in Kontexten auftreten, in denen die Erzählung Einblick in das Innenleben einer Figur gewährt. Generell begegnet man sowohl V1- als auch \textit{Wo}-VL-Sätzen sehr selten in Erzähltexten. Wenn sie vorkommen, findet man sie aber m.E. auffällig in diesen Kontexten.\footnote{ Aufgrund der Seltenheit ihres Vorkommens führe ich hier teilweise auch orthografische Varianten an, die ich in den anderen Datenmengen nicht beachtet habe.} (\ref{1062}) bis (\ref{1067}) zeigt einige Beispiele.

\begin{exe}
	\ex\label{1062} 
	\scriptsize
	Steiger beobachtete die Gruppe, und wie schon oft fiel ihm auf, dass man in den Männern, ohne sich groß bemühen zu müssen, noch die Jungen sah, die sie 	einmal gewesen waren, vor allem, wenn sie sich so hemmungslos ausschütteten wie jetzt. \emph{Erstaunlich}, \textbf{\textit{wo} \underline{doch} das 		Leben und die Nachtdienste nicht spurlos an ihren Gesichtern vorbeigegangen waren} und sie in Wahrheit fette, alte Kerle waren. Die meisten machte das 		sympathischer, fand Steiger, nur Benno Krone nicht.
	\newline              		
	\hbox{}\hfill\hbox {\citet[187-188]{Horst2011}}
\end{exe}

\begin{exe}
	\ex\label{1063} 
	\scriptsize
	\glqq Erst einmal vielen Dank, Herr Richter, dass Sie sich die Zeit genommen haben, heute hierher zu kommen\grqq{}, eröffnete Kluftinger das Gespräch 		und \emph{ärgerte sich} sofort darüber, dass er den Mann vor ihm \glqq Richter\grqq{} genannt hatte, \textbf{\textit{wo} dieser \underline{doch} schon 		längst aus dem Staatsdienst ausgeschieden war.}	
	\hfill\hbox {\citet[269]{Kluepfel2012}}
\end{exe}

\begin{exe}
	\ex\label{1064} 
	\scriptsize
	\glqq Ich wünschte, es wäre nicht so.\grqq{} Ich zeige auf das Haus. \glqq Da drin sind vier Kinder, die ihre Eltern nie wiedersehen werden.\grqq{}
	\glqq In so einem Moment fragt man sich doch, wie gütig Gott eigentlich ist.\grqq{}
	\glqq Ich \emph{frage mich} da noch eine Menge anderer Dinge.\grqq{} Wie zum Beispiel \emph{warum ich immer noch Polizistin bin}, \textbf{\textit{wo} 		\underline{doch} die beiden letzten Fälle mir so schwer zugesetzt haben}. Dazu muss man wissen, dass ich meine Arbeit liebe. Tief im Innersten bin ich 		Idealistin und mag die Vorstellung, etwas bewirken zu können.	
	\hfill\hbox {\citet[40]{Castillo2012}}
\end{exe}

\begin{exe}
	\ex\label{1065} 
	\scriptsize
	\glqq Schalalala, schalalala, heey DSC!\grqq{}, hallte das Echo ohrenbetäubend von den Wänden wider, als der Zug der Linie 4 einfuhr. Bröker schätzte, 		dass mehr als 100 Menschen versuchten, sich in die Waggons der Stadtbahn zu drängen. Er wusste selbst nicht, wie er unter diejenigen geriet, die es ins 	Wageninnere schafften. Die Freude darüber dauerte jedoch nicht lange an. \textbf{Wurde er \underline{doch} von der} \textbf{Masse der anderen Fahrgäste 	derart zusammengedrückt, dass er kaum Luft bekam.}	
	\newline              		
	\hbox{}\hfill\hbox {\citet[7]{Glauche2014}}
\end{exe}

\begin{exe}
	\ex\label{1066} 
	\scriptsize
	Paradol-Kammer? Was war das? Während Maupertuis weitersprach, \emph{prägte sich} Sherlock den seltsamen Begriff genau \emph{ein}. 							\textbf{\textit{Könnte} sich das \underline{doch} noch als wichtiger und verhängnisvoller Versprecher erweisen}, der Mycroft sicher brennend 				interessieren würde. 	
	\hfill\hbox {\citet[337]{Lane2014}}
\end{exe}

\begin{exe}
	\ex\label{1067} 
	\scriptsize
	Er hatte gegen die klaren Anweisungen seines Onkels verstoßen, und er hatte das dumpfe Gefühl, dass jeder Versuch, das Ganze mit einem Hinweis auf ein 		vermeintliches Treffen mit Amyus Crowe zu erklären, mit rigoroser Härte beantwortet werden würde. Schlimmer noch: Er war in einen ordinären Faustkampf 		verwickelt worden. Und sogar schlimmer noch als das: Er hatte verloren. Das würde zwar Sherrinford und Anna Holmes wahrscheinlich nicht sonderlich 			berühren, aber wenn Sherlocks Vater jemals etwas davon mitbekäme, \emph{würde er außer sich vor Zorn sein}. \textbf{\textit{War} \underline{doch} eine 		seiner beliebtesten Redensarten:} Ein Gentleman beginnt niemals einen Kampf, sondern beendet ihn stets. 	
	\hfill\hbox {\citet[220-221]{Lane2014}}
\end{exe}
Die \textit{Wo}-Sätze werden – anders als die V1-Sätze – allerdings auch in den fiktiven Kontexten ebenfalls in direkter Rede verwendet (vgl. (\ref{1068}) und (\ref{1069})).

\begin{exe}
	\ex\label{1068} 
	\scriptsize
	\glqq Und nach uns muss noch einer da gewesen sein\grqq{}, sagt er dann weiter und gar nicht mehr so motzig wie gerade. Nein, voller Begeisterung tut 		er das kund. \glqq Ein Riesendepp muss das gewesen sein. Weil: der hat nämlich das Kellerfenster eingeschlagen. \textbf{\textit{Wo} \underline{doch} 		die Tür auf war!}\grqq{} 	
	\hfill\hbox {\citet[68]{Falk2010}}
\end{exe}

\begin{exe}
	\ex\label{1069} 
	\scriptsize
	\glqq Ja. Und danke, Mütze. Du hast mir sehr geholfen! \grqq{}\\
	\glqq Wobei auch immer, \textbf{\textit{wo} du \underline{doch} gar nicht auf Detektivpfaden wandelst}.\grqq{}
	\hfill\hbox {\citet[56]{Glauche2014}}
\end{exe}
In Fällen wie in (\ref{1062}), (\ref{1063}) und (\ref{1065}) bis (\ref{1067}) hat man es mit einer \textit{personalen Erzählweise} \is{personaler Erzähler} zu tun (vgl. \citealt[89-102]{Eicher2001}, \citealt[264-283]{Spoerl2004}, \citealt[49-50]{Klarer2009}, \citealt[114-119]{Nuenning2015} zu Erzählsi\-tuationen). Die Distanz zwischen Erzähler und Figur wird dadurch aufgegeben, dass der Erzähler hinter die Figur tritt. Das Ergebnis ist, dass der Leser durch die Figur Zugriff auf die erzählte Welt hat. Somit hat er Teil an ihren Handlungen und ihrer Wahrnehmung. In der Darstellung der Figuren überwiegt die Innen\-perspektive. Je nach Ausgeprägtheit dieser Annäherung von Erzähler und Figur, ist der Erzähler noch mehr oder weniger präsent. In allen obigen Beispielen ((\ref{1062}), (\ref{1063}) und (\ref{1065}) bis (\ref{1067})) entsteht der Eindruck, dass das \glq Geschehen\grq {} aus der Perspektive von Bröker, Kluftinger, Steiger bzw. Holmes präsentiert wird. In (\ref{1062}), (\ref{1063}) und (\ref{1065}) ist der Erzähler aber dennoch als Mittlerinstanz deutlicher zu spüren als in (\ref{1066}) und (\ref{1067}). Die beiden Textstellen lassen sich als \textit{erlebte Rede} \is{erlebte Rede} oder \textit{Gedankenbericht} \is{Gedankebericht}  auffassen. Diese Techniken der Bewusstseinsdarstellung sind typisch für eine personale Erzählweise, da der Erzähler die Gedanken einer Figur in ihrer Sprache (erlebte Rede) oder in seiner Sprache (Gedankenbericht) präsentiert. Dadurch, dass das fiktive Geschehen bei der personalen Erzählsituation aus Sicht einer Reflektorfigur \is{Reflektorfigur} wahrgenommen und verarbeitet wird, ergibt sich sehr eindeutig eine Situation, in der sowohl diese Figur als auch der Leser hochgradig involviert sind: Durch den Fokus auf die Innenperspektive nimmt die Figur ihre Umwelt wahr, denkt und fühlt: In den obigen Beispielen ärgert sie sich, staunt, hat Befürchtungen, bringt sich einen Begriff ins Bewusstsein, um ihn nicht zu vergessen bzw. entwickelt eine schlechte Laune. Die sich anschließende Begründung dieser Haltungen wird m.E. verstanden als Begründung der Figur und nicht als Bewertung eines außenstehenden Erzählers, der hier ja gerade (anders als bei auktorialem \is{auktorialer Erzähler} oder \is{neutraler Erzähler} neutralem Erzählen) mit der Figur zusammenfällt/hinter sie tritt. Wie für das Auftreten der Begründungen in den Zeitungstexten ist es auch hier nicht verwunderlich, dass in diesen sowieso schon expressiven Kontexten auch expressive Erklärungen angeführt werden. Das Anzeigen des schon offenen Themas durch \textit{doch} fügt sich gut in diese vermittelte Stimmung ein, da dadurch ausgedrückt wird, dass die Figur bzw. Erzählinstanz bereits an einer bestehenden Diskussion Teil hat und somit in die Lösung/Entscheidung des Sachverhalts eingebunden ist. Gleiches gilt auch für den Leser, für den es in dieser Erzählform ein Leichtes ist, sich mit der Figur zu identifizieren. Durch das vorausgesetzte gemeinsame offene Thema wird er als beteiligt ausgegeben, da auch ihm Anteil an der Diskussion zugeschrieben wird.

In (\ref{1064}) tritt eine \textit{Ich-Erzählsituation} auf. Der \is{Ich-Erzähler} Ich-Erzähler nimmt als Figur (als Kommissarin Kate Burkholder) an der erzählten Welt teil. Wenngleich beschränkt auf diese eine Figur, sind natürlich auch in dieser Erzählsituation Bewusstseinsdarstellungen möglich, die – aufgrund der Tatsache, dass es sich um psychische Zustände handelt – als affektiv und damit expressiv eingestuft werden können. Der Leser erhält unvermittelten Einblick in die Gedanken und Gefühle der Figur, was seine Identifikation fördert. Die oben als mit der personalen Erzählweise verbunden angeführte Stimmung der Involviertheit von Figur und Leser gilt hier folglich gleichermaßen und wird durch das durch \textit{doch} vorausgesetzte offene Thema, an dem Erzähler/Figur und Leser beteiligt sind, gefördert.

Ich gehe folglich davon aus, dass diese beiden Satztypen auch gewisse stilistische Effekte mit sich bringen (Involviertheit, Expressivität, Affektivität), die sich in der Schriftsprache im Auftreten in Zeitungsdaten in bestimmten Genres bzw. im literarischen Bereich in bestimmten Erzählsituationen niederschlagen. Wie erläutert, eignet sich \textit{doch} – setzt man den Bedeutungsbeitrag des schon zur Diskussion stehenden Themas an – gut, um die vorliegende Involviertheit und affektive Atmosphäre sowohl aus Sprecher-/Schreiber- als auch Lesersicht zu motivieren. Die anderen Partikeln, die für die Kodierung von Kausalität sogar besser in Frage kämen, können diesen Aspekt nicht auffangen. \textit{Ja} setzt Zustimmung beim Gegenüber voraus, genauso wie \textit{eben}. Der Gesprächspartner wird übergangen, weil ihm keine Einspruchsmöglichkeit gewährt wird. Eine gesteigerte Beteiligung des Sprechers/Schreibers scheint mir auch nicht ableitbar. \textit{Auch} ko\-diert – unter Annahme einer Norm/eines gesetzten Zusammenhangs – allein Kausalität und ist demnach ebenfalls kein besserer Kandidat, um den expressiven Eindruck aufzufangen. \textit{Halt} ist zwar enger sprechergebunden (da der kausale Zusammenhang allein Annahme des Sprechers ist), die Involviertheit des Hörers folgt allerdings nicht. Auch aus der Perspektive einer vorliegenden gesteigerten Expressivität dieser Sätze scheint das \textit{doch} gut geeignet. Dieser Aspekt bietet somit einen weiteren Beitrag zu der Erklärung seines obligatorischen bzw. sehr typi\-schen Vorkommens in diesen Sätzen.
	
Der grundsätzliche Punkt, der aus meiner Untersuchung von \textit{Wo}-VL- und V1-Sätzen resultiert, ist folglich, dass \textit{doch} in diesen Satztypen transparent verwendet wird. Das Thema wird als offen angezeigt. Dies hat zwar indirekt die Funktion, die kausale Lesart zu forcieren (enger Kontextbezug durch Reaktion auf offene Frage). Man muss aber nicht annehmen, dass dies nur eine völlig uneigentliche Verwendung von \textit{doch} ist, die hier ausschließlich indirekt wirkt. Aus der Konzessivität und Kausalität, die aufgrund der beteiligten Sachverhalte/Ein\-stellungen vorliegen, lässt sich die Offenheit des Themas in der Begründung motivieren. Und schließlich lässt sich über das offene Thema, das zwischen Sprecher/Schreiber und Leser besteht, auch die nachzuweisende Expressivität kodieren.

Für meine Ableitung der Abfolge von \textit{doch} und \textit{auch} bedeutet dies, dass sie auch in diesen Satztypen gelten kann. Prinzipiell mache ich im Rahmen meiner Erklärung keinen Unterschied zwischen verschiedenen assertiven Satztypen. Auch wenn der Gesamtsatz wie hier ein Kausalsatz ist, gehe ich davon aus, dass wenn diese beiden MPn auftreten, ihr Gebrauch auch beabsichtigt ist, so dass meine Erklärung die gleiche sein kann wie in allen anderen Assertionen. Die Adressierung des Themas steht über einer qualitativen Bewertung. Welche weitere Interpretation die Äußerung hat, nimmt auf meine Ableitung zunächst keinen Einfluss.\footnote{Wie auch schon in Abschnitt~\ref{sec:markiert} in Kapitel~\ref{chapter:jud} argumentiert, gehe ich allerdings davon aus, dass spezielle interpretatorische Eigenschaften von Äußerungen Einfluss auf die Abfolgen nehmen können, da sich für bestimmte Bedeutungsanteile, auf denen meine Ableitung der Abfolgepräferenzen basiert, unterschiedliche Gewichtungen einstellen können und sich somit Verhältnisse verschieben können. Ich werde dies in Abschnitt~\ref{sec:distributionad} auch für Kausalsätze annehmen.}

Ausgangspunkt der Betrachtung dieser Deklarativsätze war in Abschnitt~\ref{sec:eig} gerade, dass – auf der Basis einer anderen Bedeutungszuschreibung an \textit{doch} – angenommen worden ist, dass die \textit{doch}-Bedeutung hier nicht transparent vorliegt. Es ist somit dieser Aspekt, der diese beiden Satztypen interessant macht, und nicht die Tatsache, dass sie kausal interpretiert werden. Da das Thema meiner Argumentation nach aber auch in diesen Satzkontexten tatsächlich offen ist, gehe ich auch hier davon aus, dass für die Partikeln untereinander gilt, dass das Anzeigen der Thema-Adressierung dem Anzeigen von Kausalität übergeordnet ist, und zwar auch, wenn sich diese Kodierungen innerhalb eines kausalen Satzes abspielen. Dies ließe sich schwieriger annehmen, wenn \textit{doch} seine Bedeutung hier gar nicht aufwiese. Die Adressierung des Themas hat in diesem speziellen Fall sogar mehr Relevanz als sonst, da man über diese auch eine andere Eigenschaft der Sätze ableiten kann (Expressivität). Den Sätzen ist es wichtig, dass das Thema adressiert wird, die Hinzufügung von \textit{auch} ist anschließend möglich, da die weitere Auszeichnung des Zusammenhangs zwischen den Sätzen als erwartet natürlich denkbar ist. Für die Relevanz des Anzeigens des offenen Themas spricht auch, dass \textit{auch} zwar hinzutreten kann, es aber sehr schlecht (wenn nicht gar nicht) allein auftreten kann. Wie erläutert, ist es nicht das einzige Ziel dieser Sätze, kausal interpretiert zu werden, was darüber hinaus eben auch schon gegeben ist, wenn \textit{doch} allein vorkommt. 

Vor dem Hintergrund meiner Ausführungen in Kapitel~\ref{chapter:hue} ist mir wichtig, dass \textit{doch} nur indirekt für die Kausalität verantwortlich ist. Andernfalls hätte man es auf der Ebene von durch MPn kodierter Kausalität mit einem Fall von Redundanz zu tun, den ich an anderer Stelle in MP-Kombinationen ausschließe. Ein Aspekt kann zunächst indirekt (\textit{doch}) und anschließend erneut direkt (\textit{auch}) kodiert werden. Die umgekehrte bzw. doppelt direkte Vermittlung wird hingegen als abwei\-chend eingeschätzt, da die Darstellung redundant erfolgt.\\

\noindent
Abschnitt~\ref{sec:distributionda} hat gezeigt, dass die Kombination \textit{doch auch} eine weite Verwendung in verschiedenen Äußerungstypen hat. Der folgende Abschnitt beschäftigt sich genauer mit Direktiven.
\setcounter{equation}{0}
\section{Direktive}
\label{sec:direktive}
Im Anschluss an die These um \is{Bedeutungsminimalismus/-maximalismus} Bedeutungsminimalismus (wie in Kapitel~\ref{chapter:hue} auch schon zu \textit{halt} und \textit{eben} vertreten) gehe ich – sofern keine dem widersprechenden Verhältnisse festzustellen sind – davon aus, dass in Direktiven der gleiche Beitrag für die MPn anzusetzen ist wie in Assertionen. Im Rahmen meiner Modellierung bedeutet dies, dass der geforderte vorweggehende Kontextzustand der gleiche ist. Unter Berücksichtigung von deskriptiven Eindrücken aus der Literatur und authentischen Belegen werde ich aufzeigen, wie eine diskursstrukturelle Modellierung die Interpretation von \textit{doch}-, \textit{auch}- und \textit{doch auch}-Direktiven abbilden kann. Wenngleich ich dafür argumentiere, dass der Partikelbeitrag beibehalten wird, ändert sich natürlich der mit dem Sprechakt ausgeführte Kontextwechsel. Dies übersetzt sich in meine Modellierung in die Füllung anderer Diskurskomponenten. 

In Abschnitt~\ref{sec:dirdm} in Kapitel~\ref{chapter:hue} habe ich die Diskursmodellierung aus \citet{Farkas2010} erweitert, um direktive Äußerungen auffangen zu können. Die entscheidende Komponente ist in dieser Erweiterung die To-Do-Liste (TDL) \is{To-Do-Liste} eines jeden Diskursteilnehmers, die die Propositionen enthält, zu deren Realisierung er sich bekannt hat. Notationell habe ich die unterschiedliche Qualität der Inhalte der Diskursbekenntnismengen (DC$_{\textrm{X}}$) \is{Diskursbekenntnismenge} bzw. des cg \is{Common Ground} und TDL$_{\textrm{X}}$ dadurch aufgefangen, dass !p in TDL enthalten ist. (\ref{1070}) bis (\ref{1072}) zeigt die Kontextveränderungen, die im Zuge der Äußerung eines Direktivs eintreten. 
\pagebreak
\newcolumntype{C}[1]{>{\centering}p{#1}}
\begin{exe}
\ex\label{1070} K$_1$: Kontextzustand K$_{1}$ nach Äußerung eines Direktivs\\[-0.6em]
\begin{tabular}[t]{|C{6em}|C{12em}|C{6em}|}
\hline
$\textrm{DC}_{\textrm{A}}$ & Tisch &  $\textrm{DC}_{\textrm{B}}$ \tabularnewline
\hline
!p $\in$ $\textrm{TDL}_{\textrm{B}}$ & p $\vee$ $\neg$p & {}  \tabularnewline
\cline{1-1}\cline{3-3}
$\textrm{TDL}_{\textrm{A}}$ & {} & $\textrm{TDL}_{\textrm{B}}$  \tabularnewline
\cline{1-1}\cline{3-3}
{} & !p $\in$ $\textrm{TDL}_{\textrm{B}}$ $\vee$ $\neg$(!p $\in$ $\textrm{TDL}_{\textrm{B}}$) & {}  \tabularnewline
\hline
\multicolumn{3}{|l|}{cg s$_{1}$} \tabularnewline
\hline
\end{tabular}
\end{exe}
				                   
\begin{exe}
\ex\label{1071} K$_1$: Kontextzustand K$_{2}$ nach Äußerung eines Direktivs\\[-0.6em]
\begin{tabular}[t]{|C{6em}|C{12em}|C{6em}|}
\hline
$\textrm{DC}_{\textrm{A}}$ & Tisch &  $\textrm{DC}_{\textrm{B}}$ \tabularnewline
\hline
!p $\in$ $\textrm{TDL}_{\textrm{B}}$ & p $\vee$ $\neg$p & !p $\in$ $\textrm{TDL}_{\textrm{B}}$  \tabularnewline
\cline{1-1}\cline{3-3}
$\textrm{TDL}_{\textrm{A}}$ & {} & $\textrm{TDL}_{\textrm{B}}$  \tabularnewline
\cline{1-1}\cline{3-3}
{} & !p $\in$ $\textrm{TDL}_{\textrm{B}}$ $\vee$ $\neg$(!p $\in$ $\textrm{TDL}_{\textrm{B}}$) & !p  \tabularnewline
\hline
\multicolumn{3}{|l|}{cg s$_{2}$ = s$_{1}$} \tabularnewline
\hline
\end{tabular}
\end{exe}		

\begin{exe}
\ex\label{1072} K$_1$: Kontextzustand K$_{3}$ nach Äußerung eines Direktivs\\[-0.6em]
\begin{tabular}[t]{|C{6em}|C{12em}|C{6em}|}
\hline
$\textrm{DC}_{\textrm{A}}$ & Tisch &  $\textrm{DC}_{\textrm{B}}$ \tabularnewline
\hline
{} & p $\vee$ $\neg$p & {}  \tabularnewline
\cline{1-1}\cline{3-3}
$\textrm{TDL}_{\textrm{A}}$ & {} & $\textrm{TDL}_{\textrm{B}}$  \tabularnewline
\cline{1-1}\cline{3-3}
{} & {} & !p  \tabularnewline
\hline
\multicolumn{3}{|l|}{cg s$_{3}$ = $\lbrace$s$_{2}$ $\cup \ \lbrace$!p $\in$ $\textrm{TDL}_{\textrm{B}}\rbrace\rbrace$} \tabularnewline
\hline
\end{tabular}
\end{exe}	
Äußert A einen Direktiv \is{Direktiv} adressiert an B, geht A davon aus, dass !p zum Element von Bs TDL wird. Da B ablehnen kann, !p in seine TDL aufzunehmen, steht zunächst die Frage im Raum, ob !p auf seine TDL gelangt.\footnote{Streng genommen geht es hier jeweils um den nächsten Zustand von TDL$_{\textrm{B}}$}.
Dies gilt parallel zur Einspruchsmöglichkeit bei der Äußerung einer Assertion, von deren Inhalt der Sprecher den Adressaten zu überzeugen beabsichtigt. Nach der Äußerung des Direktivs steht ebenfalls im Raum, ob p eintritt (vgl. (\ref{1070})). Da p das erfüllte !p ist, hängt die Entscheidung der Frage p $\vee$ $\neg$p davon ab, ob B p/$\neg$p realisiert. Die Auflösung dieser Unentschiedenheit ist somit unabhängig von Bs Reaktion zum Direktiv. D.h. p $\vee$ $\neg$p verbleibt auch auf dem Tisch, wenn B !p nicht in seine TDL aufnimmt. Dies wäre gleichbedeutend damit, dass er !$\neg$p in seine TDL einfügt. Akzeptiert B den Direktiv, gelangt !p auf seine TDL und er bekennt sich dazu, dass !p in seiner TDL enthalten ist (vgl. (\ref{1071})). Als Folge wird es ein cg-Inhalt, dass !p auf Bs TDL steht (vgl. (\ref{1072})). Die Frage, ob p zutrifft, entscheidet sich, wenn B !p realisiert und p damit wahr wird. Wenn sich eine Proposition in TDL befindet, steht folglich auch immer ihre Realisierung im Raum.\footnote{Für Assertionen habe ich nicht angenommen, dass immer das Thema zur Debatte steht, wenn in einem DC-System eine Proposition enthalten ist. Der Grund hierfür ist, dass man sich in diesem Fall auch einig sein kann, sich nicht einig zu sein. Solange TDL gefüllt ist, hat der Diskursteilnehmer diese Proposition noch nicht realisiert. Deshalb gehe ich davon aus, dass ihre Realisierung offen ist und demnach noch keine Einigkeit dahingehend besteht, ob p gilt oder nicht.} 

Ich denke, die Modellierung in (\ref{1070}) bis (\ref{1072}) kann Gemeinsamkeiten und Unterschiede zwischen Assertionen \is{Assertion} und Direktiven \is{Direktiv} abbilden. In beiden Fällen macht der Sprecher einen Vorschlag, von dem er ausgeht und der im Diskurs generell gelten soll, der aber von der Akzeptanz durch den Hörer abhängt. Bei Assertionen handelt es sich hierbei um eine Annahme p, von der der Sprecher möchte, dass der Hörer sie bestätigt, damit p cg wird. Das Ziel ist erfüllt, sobald der Hörer p akzeptiert. Bei Direktiven ist !p beteiligt, das der Sprecher vom Hörer angenommen sehen will, um das letztliche Ziel zu erreichen, dass !p realisiert und p damit wahr wird, so dass p schließlich auch in den cg gelangt. Da p eine noch zu rea\-lisierende Proposition ist, hat man es hier mit einem zusätzlichen Schritt zu tun: Der Hörer akzeptiert zunächst einmal nur, dass er !p auf seine TDL setzt. p selbst entscheidet sich erst, wenn er die Handlung ausführt und der Sprecher dies als Realisierung ansieht. Sowohl bei Assertionen als auch Direktiven besteht eine Einspruchsmöglichkeit: Bei ersteren ist der Einspruch direkt damit verbunden, p nicht zuzustimmen, bei letzteren entscheidet die Weigerung, die TDL mit !p anzureichern, noch nicht über die Zukunft von p im Diskurs. 

Treten MPn in Direktiven auf, liegt ein beschränkterer Kontextzustand vor. Wie alle MPn in meiner Modellierung fordern \textit{doch} und \textit{auch} in Direktiven einen bestimmten Zustand im Kontext, damit die \textit{doch}-, \textit{auch}- und \textit{doch auch}-Direktive angemessen geäußert werden können. 

\subsection{Das Einzelauftreten von \textit{doch}}
\subsubsection{Der Diskursbeitrag von \textit{doch}}
Die speziellen Verwendungen, Effekte und Nuancen von \textit{doch}-Direktiven (s.u.) können meiner Meinung nach darauf zurückgeführt werden, dass in dem Kontext, in dem der Direktiv geäußert wird, p $\vee$ $\neg$p bereits auf dem Tisch liegt. Die potenzielle Realisierung des Sachverhalts, zu dem B im Anschluss aufgefordert wird, steht in Frage und ist damit salient/steht aktuell im Raum (vgl. (\ref{1073})).
\pagebreak
\begin{exe}
\ex\label{1073} Kontextzustand vor der Äußerung eines \textit{doch}-Direktivs\\[-0.6em]
\begin{tabular}[t]{|C{6em}|C{12em}|C{6em}|}
\hline
$\textrm{DC}_{\textrm{A}}$ & Tisch &  $\textrm{DC}_{\textrm{B}}$ \tabularnewline
\hline
{} & p $\vee$ $\neg$p & {}  \tabularnewline
\cline{1-1}\cline{3-3}
$\textrm{TDL}_{\textrm{A}}$ & {} & $\textrm{TDL}_{\textrm{B}}$  \tabularnewline
\cline{1-1}\cline{3-3}
{} & {} & {}  \tabularnewline
\hline
\multicolumn{3}{|l|}{cg s$_{1}$} \tabularnewline
\hline
\end{tabular}
\end{exe}
\textit{Doch} steuert hier folglich den gleichen Beitrag bei wie in Assertionen (vgl. auch \citealt[92]{Diewald1998}, die für \textit{doch} in Direktiven ebenfalls prinzipiell das gleiche Schema ansetzen wie in Assertionen und die Einstellung ändern zu \textit{Ich will: p} (vs. \textit{Ich glaube: p})) (vgl. auch Kapitel~\ref{chapter:jud}, Abschnitt~\ref{sec:mpn}). Ich bin der Meinung, dass dies als Minimalanforderung \is{Bedeutungsminimalismus/-maximalismus} ausreicht. Wie bei den assertiven Kontextwechseln kann es aber wiederum diverse Szenarien geben, \underline{warum} das Thema rund um diese zur Realisierung ausstehende Proposition schon offen ist (und nicht erst durch die Äußerung des Direktivs eingeführt wird) und der Sprecher B in diesem Zusammenhang zur Realisierung eines der beiden Sachverhalte anhält.

Es ist schwieriger, Belege für \textit{doch}-Direktive in Korpora zu finden als für \textit{doch}-Assertionen. Ich präsentiere im Folgenden drei Arten der Verwendung, die ich in Korpora und literarischen Texten habe ausmachen können. Ich erhebe keinen Anspruch auf Vollständigkeit, es scheint sich aber um typische Verwendungen zu handeln. Entscheidenderweise lässt sich der in (\ref{1073}) beschriebene Kontextzu\-stand als minimale Anforderung nachweisen, da er in allen drei Verwendungen vorliegt.

In der ersten Gebrauchsweise wird die Aufforderung zum wiederholten Male erteilt, was involviert, dass B dem erteilten Auftrag bisher nicht nachgegangen ist und ihn beispielsweise ablehnt oder verzögert. In (\ref{1074}) beantwortet Adam die erste Frage nach seinem Aufenthalt zunächst nicht und weist in diesem Sinne die erste Aufforderung, die Angabe zu machen, zurück. 

\begin{exe}
	\ex\label{1074} 
	\scriptsize
	Adam muss uns kommen gehört haben, denn er rollt sich gerade unter dem Traktor heraus und steht auf. „Chief Burkholder.“ Sein Blick gleitet zu Pickles 		und wieder zurück zu mir.\\ 
	\glqq Ich hatte nicht erwartet, Sie so schnell wiederzusehen. Ist alles in Ordnung?\grqq{}\\
	Ich sehe ihn offen an. \glqq Wo waren Sie letzte Nacht und heute Morgen?\grqq{}\\
	Er tritt einen Schritt zurück, als wolle er von etwas Unschönem Abstand gewinnen. \glqq Warum fragen Sie das?\grqq{}\\
	\glqq \textbf{Beantworten Sie \underline{doch} bitte einfach die Frage.}\grqq{}\\
	\glqq Ich war hier auf der Farm.\grqq{}\\
	\glqq War jemand bei Ihnen?\grqq{}	
	\hfill\hbox {\citet[105]{Castillo2012}}
\end{exe}
Nach der ersten Frage von Burkholder steht im Raum p $\vee$ $\neg$p (Beantwortet er die Frage oder nicht?). Durch Adams Reaktion auf die Frage (entgegen der Vorstellung des Fragenden) wird !$\neg$p in seiner TDL verankert, denn (zumindest zeitweise) weigert er sich, die Antwort zu geben (vgl. (\ref{1075})).

\begin{exe}
\ex\label{1075} Burkholder: \glqq Wo waren sie letzte Nacht und heute Morgen?\grqq{}\\
				Adam: \glqq Warum fragen Sie das?\grqq{}\\[-0.6em]
\begin{tabular}[t]{|C{6em}|C{14em}|C{6em}|}
\hline
$\textrm{DC}_{\textrm{Burkholder}}$ & Tisch &  $\textrm{DC}_{\textrm{Adam}}$ \tabularnewline
\hline
!p $\in$ $\textrm{DC}_{\textrm{Adam}}$ & p $\vee$ $\neg$p & !$\neg$p $\in$ $\textrm{DC}_{\textrm{Adam}}$  \tabularnewline
\cline{1-1}\cline{3-3}
$\textrm{TDL}_{\textrm{A}}$ & {} & $\textrm{TDL}_{\textrm{B}}$  \tabularnewline
\cline{1-1}\cline{3-3}
{} & !p $\in$ $\textrm{DC}_{\textrm{Adam}}$ $\vee$ $\neg$(!p $\in$ $\textrm{DC}_{\textrm{Adam}}$) & !$\neg$p  \tabularnewline
\hline
\multicolumn{3}{|l|}{cg s$_{1}$} \tabularnewline
\hline
\end{tabular}
\end{exe}
Burkholder gibt sich mit diesem Zustand nicht zufrieden. Wenn er bestehen bleibt, wird p im Kontext wahrscheinlich nicht wahr. Aus diesem Grund reagiert sie erneut auf das ohnehin schon offene Thema, indem sie die Aufforderung zur Beantwortung der Frage explizit ausspricht, und beabsichtigt, !$\neg$p auf Adams TDL durch !p zu überschreiben.

Ein ähnlich gelagertes Beispiel ist (\ref{1076}).

\begin{exe}
	\ex\label{1076} 
	\scriptsize
	Mit freundlicher Zurückhaltung bat sie die beiden Polizisten ins Haus. Diese aber machten keine Anstalten, auszusteigen.\\
	\glqq Bitte treten Sie ein, meine Herren!\grqq{}, wiederholte sie ihre Aufforderung.\\
	Wieder aber erkannte sie keine Regung bei den Beamten, die nun demonstrativ in Richtung des Hundes sahen.\\
	Hiltrud Urban aber schien nicht zu verstehen und bat nun, schon etwas ungeduldig: \glqq \textbf{Ja kommen Sie \underline{doch} herein!}\grqq{} 		
	\hfill\hbox {\citet[221]{Kluepfel2012}}
\end{exe}
Auch hier ist die Aufforderung schon erteilt worden, so dass p $\vee$ $\neg$p im Raum steht. Der Adressat zeigt Anzeichen dafür, dass er die gegenteilige Handlung beabsichtigt, weshalb der Direktiv wiederholt wird.

Dass der Adressat Gegenteiliges vorzuhaben scheint, muss allerdings nicht dadurch offenbart werden, dass der Direktiv bereits (mehrfach) nicht angenommen wurde. Die Offenheit des zu realisierenden Sachverhalts kann auch dadurch zustandekommen, dass sich der Adressat aktuell entgegengesetzt zum vom Spre\-cher gewünschten Zustand verhält, so dass nicht damit zu rechnen ist, dass der vom Sprecher gewünschte Sachverhalt eintreten wird bzw. (schwächer) unklar ist, ob der Sachverhalt realisiert werden wird. Belege, die ich unter diesen Fall fasse, zeigen, dass das Ausleben des gegenteiligen Verhaltens unterschiedlich deutlich sein kann.

In (\ref{1077}) beispielsweise gilt im Kontext vor dem \textit{doch}-Direktiv der Sachverhalt, dass die Mutter jodelt (diese Proposition q ist im cg enthalten $[$vgl. (\ref{1078})$]$).
		
\begin{exe}
	\ex\label{1077} 
	\scriptsize
	Wie typische Jodlerinnen kommen die 38jährige Claudia Städler und die 36jährige Andrea Haffa nicht daher. Wenn sie aber ihre Leidenschaft zum Jodeln 		beschreiben, geraten die beiden ins Schwärmen. Eine Herzenssache sei es, etwas, das tief aus der Seele komme, \glqq voll einfahre\grqq{} und dem 			Publikum die Tränen in die Augen treibe. Claudia Städler sah das als Mädchen noch nicht so und tat sich mit dem Jodeln schwer. \glqq Hör doch auf			\grqq{}, habe sie ihre Mutter jeweils angefleht, wenn diese jodelnd in die Stube gekommen sei. Später habe sie sich einen Walkman gewünscht, um bei 		Autofahrten mit der Familie der ungeliebten Jodelmusik zu entkommen.		
	\hfill\hbox {(A10/APR.00228 St. Galler Tagblatt, 01.04.2010)}
\end{exe}		
 	    
\begin{exe}
\ex\label{1078} Kontextzustand vor dem \textit{doch}-Direktiv\\[-0.6em]
\begin{tabular}[t]{|C{6em}|C{12em}|C{6em}|}
\hline
$\textrm{DC}_{\textrm{A}}$ & Tisch &  $\textrm{DC}_{\textrm{B}}$ \tabularnewline
\hline
{} & p $\vee$ $\neg$p & !$\neg$p $\in$ $\textrm{DC}_{\textrm{B}}$  \tabularnewline
\cline{1-1}\cline{3-3}
$\textrm{TDL}_{\textrm{A}}$ & {} & $\textrm{TDL}_{\textrm{B}}$  \tabularnewline
\cline{1-1}\cline{3-3}
{} & {} & !$\neg$p  \tabularnewline
\hline
\multicolumn{3}{|l|}{cg s$_{1}$ = $\lbrace$q, q $>$ (!$\neg$p $\in$ $\textrm{TDL}_{\textrm{B}}$)$\rbrace$} \tabularnewline
\hline
\end{tabular}
\end{exe}
Es ist davon auszugehen, dass wenn sie in der aktuellen Situation jodelt, sie im naheliegenden zukünftigen Zustand nicht nicht jodeln wollen wird, d.h. sie wird vorhaben, weiterzujodeln ((q $>$ (!$\neg$p $\in$ $\textrm{DC}_{\textrm{B}}$)) ist cg). Deshalb ist !$\neg$p in $\textrm{TDL}_{\textrm{B}}$ und !$\neg$p $\in$ $\textrm{TDL}_{\textrm{B}}$ in DC$_{\textrm{B}}$ enthalten. Folglich steht im Raum, ob p oder $\neg$p realisiert wird (mit einer Voreingenommenheit in Richtung $\neg$p). Die Tochter möchte p, weshalb sie zu seiner Verwirklichung aufruft.\\

\noindent
In (\ref{1079}) arbeitet der Sprecher an seiner Antwort. Der Adressat tut nichts/zeigt keine Reaktion, das zu tun, wozu er anschließend aufgefordert wird und was auch naheliegend ist, wenn jemand etwas allein nicht zu schaffen scheint. Ihm kann deshalb auch zugeschrieben werden, dass er das Gegenteil vorhat, so dass aus diesem Grund offen ist, was realisiert wird.

\begin{exe}
	\ex\label{1079} 
	\scriptsize
	Dann könnte ich zum Beispiel. Wenn man zurückrechnet. Wenn ich um halb Zwölf dort bin, viereinhalb Stunden, heijeijei, wäre? Halb zwölf? \textbf{Helfen 	Sie mir \underline{doch} bitte.}\\
	\noindent		
	Ja, da sollten Sie um sieben Uhr drei auf jeden Fall fahren.		
	\newline              		
	\hbox{}\hfill\hbox {(Tübinger Baumbank des Deutschen/Spontansprache)}
\end{exe}
Die Situation ist in diesem Verwendungsfall von \textit{doch}-Direktiven im Grunde die gleiche wie bei denjenigen \textit{doch}-Assertionen, bei denen aus einer anderen Proposition bzw. einem Verhalten des Angesprochenen auf sein Bekenntnis zu $\neg$p geschlossen werden kann. Aufgrund dessen steht p $\vee$ $\neg$p im Raum (mit einer Voreingenommenheit zu $\neg$p) und der Sprecher vertritt die entgegengesetzte Proposition, die er durch Umstimmung des Adressaten zu einem cg-Inhalt machen möchte. Im Falle des Direktivs ist aus Bs Verhalten abzuleiten, dass er ¬p beabsichtigt und deshalb die Realisierung von p $\vee$ $\neg$p offen ist. Sofern sich nichts ändert, wird aber vermutlich $\neg$p realisiert werden. Da der Sprecher Gegenteiliges wünscht, fordert er zu !p auf, um den Adressaten umzustimmen und die Realisierung von p herbeizuführen.

Die Zuschreibung der gegenteiligen Absicht kann auch konventionalisiert sein. 
	
\begin{exe}
	\ex\label{1080} 
	\scriptsize
    \begin{tabular}[t]{lll}
	0001 & & (1.95)\\
	0002 &	XM & fritzsche\\
	0003 & MF & ja\\
	0004 & XM &	\textbf{nehmen sie \underline{doch} ? $[$platz} (.) ihre$]$ prüfer kennen sie\\
	0005 & MF &	$[$schuldigung h$^{0}$ $]$\\
	0006 & & (0.24)\\
	0007 & MF & ja\\
	0008 & XM & ((schmatzt)) ja (.) ich bin tamara hackel bin die prüfungsvorsitzende
	\end{tabular}
	\newline
	\hbox{}\hfill\hbox{(FOLK\_E\_00059\_SE\_01\_T\_01)} 					        
\end{exe}
Ein Direktiv wie in (\ref{1080}) kann geäußert werden, wenn jemand wirklich zögert, sich hinzusetzen, was in Kontexten, in denen eine solche Äußerung gemacht wird, auch durchaus vorkommen kann (weil jemand z.B. unschlüssig im Raum steht, sich ziert, sich hinzusetzen). In den meisten Fällen wird er aber als Floskel verwendet werden, womit meiner Meinung nach aber nicht einhergeht, dass die \textit{doch}-Bedeutung nicht transparent zu erfassen ist: Der Sprecher geht davon aus, dass der Hörer sich nicht von allein hinsetzen wird (qua Konvention), obwohl im Kontext klar ist, dass einmal der sitzende Zustand bestehen soll. Beratungsgespräche, Prüfungen (wie in (\ref{1080})) und Besuche finden i.d.R. nicht im Stehen statt. Man kann hier aber nicht unterstellen, dass der Adressat wirklich beabsichtigt, sich \underline{nicht} hinzusetzen, so dass man sagen müsste, dass auf seiner TDL steht, dass er sich nicht hinsetzen will oder dass man ihm unterstellen müsste, dass er ein Verhalten zeigt, das dem (späteren) Hinsetzen entgegengesetzt ist. Der Adressat ist vielmehr unsicher, ob er sich hinsetzen kann oder nicht oder ob er aufgefordert werden muss. In diesem Sinne kann plausibel im Raum stehen, ob er sich hinsetzt oder nicht, ohne ihm eine Absicht zuzuschreiben. Für meine Begriffe kann eine Äußerung wie \textit{Setz dich \textbf{doch}!} gleichermaßen geäußert werden, wenn der Angesprochene beabsichtigt oder nicht beabsichtigt, sich hinzusetzen.

Ähnlich verhält es sich in Interviews, in denen der Interviewer den Interviewten auffordert, zu reden (\textit{Erzählen Sie \textbf{doch} mal...}, \textit{Sagen Sie \textbf{doch} mal...}) (vgl. (\ref{1081})).

\begin{exe}
	\ex\label{1081} 
	\scriptsize
    \begin{tabular}[t]{lll}
	0001 & S1 &	\textbf{Herr Sch., erzählen Sie \underline{doch} ? mal etwas über Ihren .. Lebenslauf.}\\
	0002 & S2 & Ich wurde im Jahre neunzehnhundertneununddreißig in dem damaligen Sudetenland \\
	& & geboren... Im Jahre neunzehnhundertfünfundvierzig, bei Kriegsende, mußten \\
	& & wir die Heimat verlassen.
	\end{tabular}
	\hfill\hbox{(PF--\_E\_00249\_SE\_01\_T\_01)} 					       
\end{exe}
Hier sollte dem Interviewten klar sein, dass er zu bestimmten Dingen etwas sagen soll. Qua Konvention fängt man aber nicht einfach an zu reden. Der Interviewte wird quasi \glq angeschoben\grq {}, auch, weil unklar ist, ob er von allein reden soll oder nicht. Hier lässt sich ebenfalls nicht sagen, dass der Adressat beabsichtigt, nichts zu sagen, oder dass er aktuell das entgegengesetzte Verhalten zeigt, nicht zu reden, und dies so bliebe, wenn man ihn nicht ermunterte. Es ist aber sicherlich fraglich, ob er von allein erzählen würde. In der durch (\ref{1080}) und (\ref{1081}) illustrierten Situation liegt eine andere Lage vor, als wenn jemand freiwillig wirklich nichts sagte, oder er demjenigen, der die Aufforderung ausspricht, zu wenig sagt (vgl. (\ref{1082}) und (\ref{1083})).

\begin{exe}
	\ex\label{1082} 
	\scriptsize
    \glq Äh, ach so, ja, ja, ja äh wir haben Karl, weißt du, der hat sein, sein großes Messer abgebrochen.\grq {} \glq Messer abgebrochen, wieso, wo, 			wobei?\grq {} \glq Ja, er, er, haa/ , hat da ein Pferd mit tot gestochen, ja, das hat er, ja. Und da ist das dann, ist ihm das dann dabei abgebrochen.		\grq {} \glq Pferd, was für ein Pferd tot stechen, was für ein Pferd denn?\grq {} \glq Na, unseren schwarzen Heng/ Hengst, den hat er tot gestochen.		\grq {} \glq Jan, Mensch, bist du des Teufels, unseren schwarzen Hengst, warum das denn?\grq {} \glq Der hatte sich da ein Bein gebrochen.\grq {} \glq 		Der hat sich ein Bein gebrochen, wo das denn bei?\grq {} \glq Beim Wasser fahren.\grq {} \glq Was habt ihr dann für Wasser zu fahren?\grq {} \glq Na 		nach dem Feuer, wir mußten doch löschen.\grq {} \glq Feuer, was für ein Feuer! Menschenkind, Jan, \textbf{nun ? erzähl \underline{doch} mal}, nun sei 		doch nicht so maulfaul!\grq {} \glq Ja nun äh, äh äh die Scheune ist doch abgebrannt!\grq {} \glq Die Scheune ist abgebrannt, was für eine Scheu/ , 		unsere Scheune?\grq {} 		  
	\hfill\hbox{(ZW--\_E\_05156\_SE\_01\_T\_01)} 					     
\end{exe}
					     
\begin{exe}
	\ex\label{1083} 
	\scriptsize
    Mutters Klavier (Loriot)\\
    Zwei Möbelträger stellen ein Klavier vor einer Wohnungstür ab und klingeln.\\
	VATI (die Tür öffnend) Aha!\\
	TRÄGER (Lieferschein ablesend) Ist das hier richtig bei...Panislowski?\\
	VATI (in die Wohnung zurückrufend) Thomas! Also, Netzschalter auf \glqq On\grqq{}, gleichzeitig die Tasten \glqq Start\grqq{} und \glqq Aufnahme\grqq{} 	drücken, aber die Kamera erst auslösen, wenn die beiden Herren mit dem Klavier in der Wohnzimmertür erscheinen...\\
	TRÄGER Sind wir hier richtig bei...Panislowski?\\
	VATI Thomas!\\
	THOMAS (von innen) Ja...\\
	VATI Hast du verstanden?\\
	THOMAS Ja...\\
	VATI Na, \textbf{dann sag \underline{doch} was!}\\
	TRÄGER Wir kommen von der Firma... 
    \hfill\hbox{(http://privat.flachpass.net/html/mutters\_klavier.html)}		
    \newline
    \hbox{}\hfill\hbox{(eingesehen am 07.12.2015)}			     
\end{exe}
In den Beispielen, die ich bisher angeführt habe, konnte man (abgesehen von den konventionalisierten Varianten von Fall 2) dem Adressaten gut das Vorhaben um das Gegenteil des vom Sprecher gewünschten Zustands zuschreiben und die Offenheit der Frage um die Rea\-lisierung von p $\vee$ $\neg$p daraus ableiten. Generell argumentiere ich aber gerade dagegen, dass \textit{doch} stets i.e.S. auf einen Widerspruch verweist (vgl. Kapitel~\ref{chapter:jud}, Abschnitt~\ref{sec:doch1}). Vor allem auf die diskursinitialen Fälle und \textit{Wo}-VL-/V1-Sätze scheint dies nicht zuzutreffen. Die letzte Gebrauchsweise von \textit{doch}-Direktiven, die ich habe ausfindig machen können, ist diejenige, die mir diese Annahme auch bei den Direktiven zu bestätigen scheint. Es liegen Kontexte vor, in denen die Realisierung von  p $\vee$ $\neg$p zwar fraglich ist und es deshalb nötig ist, den Adressaten in die gewollte Richtung zu lenken, man dem Adressaten aber nicht das gegenteilige Verhalten oder entgegengesetzte Absichten zuschreiben kann. Ich halte es für eine zu starke Annahme, in diesen Fällen davon auszugehen, dass der Adressat das Gegenteil in Planung hat. Natürlich vollzieht er die Handlung, zu der er aufgefordert wird, aktuell noch nicht. Dies ist eine Bedingung, die an jeden Direktiv gestellt wird. In diesen Beispielen hat man aber keine Anzeichen dafür, dass die gegenteilige Handlung beabsichtigt ist. Es besteht eher eine gewisse Ratlosigkeit. Illokutionär handelt es sich hierbei um Anregungen, Nahelegungen, Vorschläge, Ratschläge, Ermunterungen oder Einladungen. Beispiele, die ich hier zuordne, findet man z.B. in Ratgeberzeitungen oder Tippsektionen in Zeitungen. Und sie können auch Teil von Horoskopen sein (vgl. z.B. (\ref{1084}) bis (\ref{1086})).

\begin{exe}
	\ex\label{1084} 
	\scriptsize
	\textbf{Nehmen Sie diesbezüglich Ihren Arbeitsalltag \underline{doch} einmal genau unter die Lupe}, um unnötige Belastungen zu vermeiden.	
	\hfill\hbox{(DECOW 2014)}  
	\newline
	\hbox{}\hfill\hbox{(http://www.lambertus-apotheke.de/html/verschleiss.html)} 				     
\end{exe}

\begin{exe}
	\ex\label{1085} 
	\scriptsize
	Nun ist Zeit zum Entspannen und Faulenzen. Vielleicht auch zum Träumen? \textbf{Schmieden Sie mit dem Partner \underline{doch} ein paar Reise- und/oder 	Zukunftspläne.} 	
	\newline
	\hbox{}\hfill\hbox{(RHZ09/JAN.06468 Rhein-Zeitung, 10.01.2009; Zwillinge 21.5.)} 				     
\end{exe}
				        
\begin{exe}
	\ex\label{1086} 
	\scriptsize
	Sie sehen den Wald vor lauter Bäumen nicht. Sie haben Mühe damit, im Beruf alles in den Griff zu bekommen. \textbf{Reduzieren Sie \underline{doch} Ihr 		Pensum.} Auch im Privatleben ist Ärger möglich. Seien Sie toleranter.	
	\hfill\hbox{(NON09/APR.15465 Niederösterreichische Nachrichten, 27.04.2009)} 				     
\end{exe}
Der Adressat hat ein Interesse, ein bestimmtes Ziel zu erreichen, z.B. hier Ausgeglichenheit, Entspannung, Wohlbefinden. Er hat in dieser Hinsicht wahrscheinlich Defizite, d.h. diese Aspekte liegen nicht vollends vor. Anderenfalls würde er nicht solche Texte lesen, in denen er hofft, eine Lösung für sein Problem zu finden, bzw. möchte er sich auf die angesprochenen Ziele (z.B. in den Horoskopen) einlassen. Da der Adressat als Hilfesuchender auftritt, stellen sich vor diesem Hintergrund im Kontext dann Fragen wie: Wird der Adressat den Alltag auf Belastungen untersuchen (um ausgeglichener zu sein)? Tritt die Situation ein, dass er sein Pensum reduziert (damit er beruflich weniger eingebunden ist)? Damit es dem Adressaten besser geht, müssen diese Punkte in die Richtung des Direktivs aufgelöst werden. Ich gehe davon aus, dass im cg die Relation p $>$ q ent\-halten ist. Da B nach einem Weg sucht, den Zustand q herzustellen, obwohl er weiß, dass das Vorliegen von p diesen normalerweise mit sich bringen würde, lässt sich ableiten, dass verstärkt unklar ist, ob er von alleine p realisieren würde. Man kann annehmen, dass in diesen Situationen fraglich ist, ob diese Dinge für den Adressaten gelten (weil er anderenfalls keinen Beratungsbedarf hätte). Man kann aber z.B. nicht sagen, dass er explizit beabsichtigt, nicht das Pensum zu reduzieren (zumindest nicht mehr als dies sowieso nicht bereits gegeben sein kann, damit man den Direktiv äußern kann). Die Offenheit von p kann auch in diesen Fällen als Kontextanforderung angesetzt werden. Denkbar ist auch, dass der Hörer um den Zusammenhang noch gar nicht weiß und sich die Fraglichkeit der Realisierung deshalb nur in den Augen des Sprechers ergibt. Diese Konstellation habe ich auch bei \textit{doch} in Assertionen bereits für möglich gehalten. Mit der Auslegung der Beispiele aus (\ref{1084}) bis (\ref{1086}) in (\ref{1087}) geht einher, dass !p dem Adressaten bekannt sein sollte. Die Frage ist hier, ob man dies für alle diese Typen von \textit{doch}-Direktiven vertreten möchte. \textit{Doch} \underline{verweist} hier aber m.E. nicht auf die Bekanntheit des Rates, sondern auf die Fraglichkeit der Realisierung. Hier wird deutlich, dass verschiedene Partikeln in gleichen Kontextsituationen unterschiedliche Aspekte thematisieren. Als evident und bekannt zeichnet \textit{eben} eine Aufforderung aus. Es wäre aber äußerst unangemessen, in den obigen Beispielen eben zu verwenden und dem Hilfesuchenden somit anzuzeigen, dass er die Antwort selber weiß (auch wenn es de facto so ist!).
	
\begin{exe}
\ex\label{1087} Kontextzustand vor einem \textit{doch}-Direktiv\\[-0.6em]
\begin{tabular}[t]{|C{6em}|C{12em}|C{6em}|}
\hline
$\textrm{DC}_{\textrm{A}}$ & Tisch &  $\textrm{DC}_{\textrm{B}}$ \tabularnewline
\hline
{} & p $\vee$ $\neg$p & {}  \tabularnewline
\cline{1-1}\cline{3-3}
$\textrm{TDL}_{\textrm{A}}$ & {} & $\textrm{TDL}_{\textrm{B}}$  \tabularnewline
\cline{1-1}\cline{3-3}
{} & {} & {}  \tabularnewline
\hline
\multicolumn{3}{|l|}{cg s$_{1}$ = $\lbrace\neg$q, p $>$ q$\rbrace$} \tabularnewline
\hline
\end{tabular}
\end{exe}		 
Ähnliches wie für derartige Ratschläge gilt auch für Einladungen der Art in (\ref{1088}) und (\ref{10888}).

\begin{exe}
	\ex\label{1088} 
	\scriptsize
	Diskutiert wird mit den Gästen im Studio natürlich auch über aktuelle niedersächsische Themen. Ein Beleg dafür, dass Talkshows im Radio weitaus 			spannender als die im Fernsehen sind. \textbf{Hören Sie \underline{doch} mal rein.} Wir treffen uns hier wieder, an dieser Stelle, heute in einer 			Woche. Bis dahin viel Spaß mit Ihrem Radio! 		
	\hbox{}\hfill\hbox{(HAZ09/MAI.02715 Hannoversche Allgemeine, 16.05.2009)} 				     
\end{exe}

\begin{exe}
	\ex\label{10888} 
	\scriptsize
	Spielst du seit zwei Jahren ein Blasinstrument oder Schlagzeug, dann bist du bei uns genau richtig. Nebst dem Musizieren legen wir auch Wert auf das 		gemütliche Zusammensein (Sommer- oder Winterplausch usw.). Der Dirigent heisst Christoph Diem und wohnt in Flawil. $[$...$]$ Die offenen Proben sind am 	17. März und 21. April. \textbf{Komm \underline{doch} einfach mal vorbei} und schau, ob es dir gefallen würde!
	\newline
	\hbox{}\hfill\hbox{(A10/MAR.05294 St. Galler Tagblatt, 16.03.2010)} 				     
\end{exe}	               			           				  
Man kann hier nicht davon ausgehen, dass der Adressat das Gegenteil beabsichtigt. Der Sprecher nimmt in (\ref{1088}) natürlich nicht an, dass der Adressat sowieso vorhat, zu kommen oder sowieso schon Mitglied des Orchesters ist. Er hat aber auch nicht vertreten, dass er plant, nicht zu kommen (oder nicht reinhören zu wollen). Wenn er dies vorhätte, würde er die Anzeige wahrscheinlich auch nicht lesen.

Auch bei Vorschlägen wie in (\ref{1089}) kann man nicht davon ausgehen, dass der Adressat Gegenteiliges zu tun plant. In einem Gespräch über Terminvereinbarungen, in dem der Tag dazu schon feststeht, lässt sich aber ableiten, dass das Thema zur Diskussion steht, ob der Adressat früh morgens kommen wird. 

\begin{exe}
	\ex\label{1089} 
	\scriptsize
	Prima und dann noch vielleicht übermorgen ja da bleibt uns ja nur der Nachmittag.\\
	
	Ja guten Tag Frau Müller. Samstag der fünfzehnte passt mir sehr gut ja.\\
	
	Ja okay \textbf{dann kommen Sie \underline{doch} gleich um neun Uhr zu mir.}
	\newline
	\hbox{}\hfill\hbox{(Tübinger Baumbank des Deutschen/Spontansprache)} 				     
\end{exe}
Ob sich die Gesprächsteilnehmer um neun Uhr treffen, ist einfach salient, da sie sich zu irgendeiner Zeit treffen müssen. 

Ich gehe folglich davon aus, dass die kontextuelle Voraussetzung für einen \textit{doch}-Direktiv ist, dass die Frage, \textit{ob p} (wobei es sich bei p um die realisierte Proposition handelt, zu deren Erfüllung der im Anschluss geäußerte Direktiv auffordert), auf dem Tisch liegt, d.h. ein offenes, salientes Thema ist. Wie oben ausgeführt, kann es verschiedene Szenarien geben, wie sich dieser Kontextzu\-stand ergibt.

Ich halte die Annahme, die davon ausgeht, dass \textit{doch} immer auf einen Gegensatz oder Widerspruch verweist, was bei Direktiven dann bedeuten würde, dass der Adressat Gegenteiliges vorhat oder bereits ausführt, für zu stark. Als schwä\-chere Variante vertrete ich deshalb, dass die Realisierungsfrage um p offen ist. Die Erfüllung von p ist somit fraglich. Es ist aber dazu zu sagen, dass den beiden Alternativen nicht die gleiche Wahrscheinlichkeit ihrer Realisierung zukommt. Für \textit{doch} in Assertionen gilt ebenfalls, dass es nicht rein bestätigend verwendet werden kann (vgl. (\ref{1090})).

\begin{exe}
	\ex\label{1090} 
	B: Peter hat Maria mit einem Mann in der Stadt gesehen.\\
	A: \#Es war \textbf{doch} nicht ihr Ehemann.				     
\end{exe}
Für die Direktive bedeutet dies, dass \textit{doch} nicht vorkommen kann, wenn die Realisierung von p in keiner Hinsicht fraglich ist, und zwar über das Maß hinaus, mit dem die tatsächliche Realisierung der Handlung sowieso noch aussteht (auch nach Zustimmung der Realisierung). Nach meiner Modellierung steht nach der Äußerung eines jeden Direktivs bis zur tatsächlichen Ausführung der beteiligten (Nicht-)Handlung schließlich die Frage im Raum, ob p oder $\neg$p realisiert wird. Dieser Aspekt wird eine Rolle spielen in Abschnitt~\ref{sec:kombida}, in dem ich die Kombinationen aus \textit{doch} und \textit{auch} betrachte.

\subsubsection{Deskriptive Eindrücke aus der Literatur}
Meine Modellierung fängt auch die deskriptiven Eindrücke aus der Literatur zu \textit{doch}-Direktiven auf. 

Ein Aspekt, der in verschiedenen Arbeiten angeführt wird, ist, dass ein Widerspruch zwischen der geforderten Handlung und der aktuellen Situation bestehe. Natürlich liegt dieser zu einem gewissen Grad in jedem Kontext, in dem ein Direktiv geäußert wird, vor, er werde durch die Partikel aber zusätzlich thematisiert (\citealt[139]{Hentschel1986}; auch \citealt[118]{Thurmair1989}, \citealt[214]{Kwon2005}). Dieser Aspekt findet sich in meiner Modellierung auf die Art wieder, dass die Realisierung von p (vs. $\neg$p) je nach Verwendung vor der Äußerung aus verschiedenen Gründen bereits salient ist. D.h. anstatt von einem Widerspruch auszugehen, vertrete ich die \underline{Fraglichkeit} der Realisierung von p.

Weiter hat \citet[139]{Hentschel1986} den Eindruck formuliert, \textit{doch}-Direktive würden eine enge Anbindung an den Kontext leisten. Da die Offenheit von p aus dem aktuellen Handeln/Verhalten des Adressaten oder der Kontextsituation folgt, spie\-gelt meine Analyse auch diesen Aspekt.

Einig sind sich existierende Arbeiten auch dahingehend, dass zwei Arten von \textit{doch}-Direktiven zu unterscheiden sind: a) ein verstärkender Typ, dem Eigenschaften wie \glq ungeduldig\grq {}, \glq dringend\grq {}, \glq drängend\grq {}, \glq ärgerlich\grq {}, \glq vorwurfsvoll\grq {}, \glq unwirsch\grq {} zugeordnet werden, und b) ein abschwächender Typ, belegt mit Charakteristika wie \glq höflich‘, \glq freundlich‘, \glq beiläufig‘, \glq beruhigend‘. \citet[191]{Rinas2006} paraphrasiert die erste Verwendungsweise durch \glq auch wenn DU/Sie dich/sich offenbar nicht traust/trauen\grq {} (vgl. auch \citealt[188]{Franck1980}, \citealt[139]{Hentschel1986}, \citealt[113]{Helbig1990}, \citealt[91-92, 214]{Kwon2005}) und die zweite durch \glq auch wenn Du/Sie das nicht willst/wollen\grq {}. Mit dieser Zweiteilung gehen auch die verschiedenen denkbaren Illokutionstypen einher: Aufforderung, Befehl, Bitte, Empfehlung, Rat (vgl. z.B. \citealt[27]{Volmert1991}, \citealt[113]{Helbig1990}, \citealt[91-92, 214]{Kwon2005}). Welcher Illokutionstyp realisiert wird, ist abhängig von Kontext, Inhalt und dem Auftreten weiteren sprachlichen Materials (z.B. \textit{endlich} vs. \textit{bitte}, \textit{mal}). Es gibt Autoren, die vertreten, dass \textit{doch}-Direktive i.d.R. Ratschläge \is{Ratschlag} sind (\citealt[402]{Ickler1994}) und Befehle \is{Befehl} zu Bitten \is{Bitte} und Ratschlägen abschwächen (\citealt[111]{Bublitz1978}). Der Nachweis dieser Annahme, dass innerhalb der \textit{doch}-Direktive bestimmte Illokutionen bzw. einer der zwei angeführten Typen überwiegen, steht aber noch aus.

Für die \textit{doch}-Modellierung bedeuten derartige illokutive Unterscheidungen, dass sie für die beiden in ihrem Ton entgegengesetzt wirkenden Direktivtypen aufkommen können sollte. Ich denke, dass dies möglich ist: Der Verstärkungsfall (mit den einhergehenden Eigenschaften) liegt sinnvollerweise vor, wenn ein Direktiv wiederholt wird, weil ihm noch nicht nachgekommen worden ist/keine Bereitschaft zum Nachkommen angezeigt wird (Fall 1) oder der Adressat aktuell das Gegenteil ausführt und man ihn dazu bewegen möchte, im zukünftigen Zustand ein anderes Verhalten zu realisieren (Fall 2). Dass die Realisierungsfrage schon im Raum steht, kann Aspekte wie Dringlichkeit, Ungeduld und Vorwurf zur Folge haben. Auch der Effekt der Abschwächung und damit einhergehender Höflichkeit lässt sich im Rahmen meiner Modellierung erfassen: Prinzipiell sind Direktive immer gesichtsbedrohend für den Adressaten, da in sein Gebiet eingegriffen wird (negative \is{Gesichtsbedrohung} Gesichtsbedrohung) (\citealt[127]{Held1995}). Im Grunde besteht auch eine Bedrohung des positiven Gesichts des Sprechers, weil er es ist, der den Adressaten in seiner Handlungsfreiheit einschränkt. Diese Gesichtsbedrohungen werden durch das Auftreten von \textit{doch} etwas genommen: Die Direktive sind weniger aufdringlich (und damit auch beruhigend, freundlich), wenn angezeigt wird, dass die Realisierungsfrage bereits im Raum steht. Dadurch, dass sie Thema ist, weisen die Diskurspartner schon diese Gemeinsamkeit auf, was die Gesichtsbedrohung auf beiden Seiten entlastet. Vor diesem Hintergrund wird der Adressat in eine Richtung beeinflusst, es ist aber nicht der Direktiv, der die Realisierungsfrage mitbringt. Dass der Sachverhalt, der ihm als Sprecherwunsch mitgeteilt wird, eine Möglichkeit ist, ist ebenfalls schon gegeben. Der Inhalt des Auftrags kann somit nicht völlig überraschend sein. Auch über die diskursinitialen \textit{doch}-Assertionen habe ich in Abschnitt~\ref{sec:doch1} in Kapitel~\ref{chapter:jud} gesagt, dass sie höflicher wirken, weil sie der Assertion die \glq Aufgabe\grq {} abnehmen, das Thema zu eröffnen. Hier hätte man es unter diesen Betrachtungen folglich mit parallelen Effekten zu tun, die gerade auf die grundsätzlich parallele Modellierung zurückgehen.

Im Falle einer Einladung \is{Einladung} kann eine Gesichtsverletzung dadurch zustandekommen, dass der Adressat gar nicht an der Aktivität teilnehmen möchte, oder auch dadurch, dass er denkt, dass er sich aufdrängt, wenn er dazugebeten wird. Auch diese beiden Gründe für die Gesichtsbedrohung können überdeckt werden von der schon zur Debatte stehenden Realisierung. Beide Alternativen sind dann schon verfügbar (unabhängig davon, was der Sprecher möchte) und der Adressat kann sich auf eine Art einfacher für die Ablehnung und Realisierung von $\neg$p entscheiden, weil diese Möglichkeit bereits gleichermaßen im Raum steht wie p. Ablehnen kann er natürlich auch einen MP-losen Direktiv, aber unter diesen Umständen eröffnen sich die Alternativen hinsichtlich der weiteren Entwicklung erst im Zuge der Äußerung des Direktivs und im Zuge dessen, dass der Sprecherwille deutlich wird. Stehen die beiden Möglichkeiten schon im Vorfeld zur Verfügung, scheint mir dies dem Adressaten mehr Raum zu geben. Die Gefahr des Aufdrängens wird ebenfalls abgeschwächt, weil das potenzielle Mitkommen auch schon vor der Äußerung des Direktivs eine verfügbare Alternative ist.

In Ratschlägen \is{Ratschlag} wird gesichtsbedrohend \is{Gesichtsbedrohung} eingeschätzt, dass suggeriert wird, dass der Adressat ohne das Eingreifen des Sprechers nicht zurecht kommt und der Sprecherbeitrag somit als unerwünschtes Einmischen gedeutet wird (vgl. \citealt[114]{Frank2011} und die dort zitierte Literatur). Auch diese Bedrohung wird mit einem \textit{doch}-Direktiv genommen, da sowieso bereits fraglich ist, was der Adressat tun wird.

Ich denke, die Betrachtung der \textit{doch}-Direktive zeigt auch, dass die Bedeutungs\-komponente von Bekanntheit bei dieser MP nicht zutreffend ist bzw. nicht als kategorisch beteiligt angenommen werden kann. \citet[118-119]{Thurmair1989} hält sie zwar auch in diesem Äußerungstyp hoch (dem Adressaten sei der Sprecherwille klar) (vgl. ähnlich auch \citealt[111]{Bublitz1978}, \citealt[168]{Karagjosova2004}), aber man kann in meinen Augen nicht für alle der obigen Beispiele von Bekanntheit, Ableitbarkeit, Offensichtlichkeit ausgehen (vgl. auch \citealt[401]{Ickler1994}, der darauf hinweist, dass neue Vorschläge und Aufforderungen möglich sind). Bekanntheit lässt sich generell vertreten für Fall 1 (weil die Aufforderung zum wiederholten Male getätigt wird) sowie die konventionalisierten Varianten von Fall 2. Genauso ist der Vorschlag in (\ref{1089}) ableitbar und offensichtlich. Nicht zutreffend scheint mir eine solche Einschätzung allerdings für die Ratschläge. Ausgenommen, man sagt, dass der Inhalt von Ratschlägen (wie in Ratgebersektionen) sowieso bekannt ist, kann man für derartige Vorschläge generell nicht annehmen, dass sie offensichtlich sind. Wäre dies der Fall, würde der Adressat sie nicht lesen, sondern direkt sein Leben entsprechend ändern. Wenn dem Hörer hier bekannt sein sollte, was zu tun ist, verweist \textit{doch} für meine Begriffe nicht auf diese Einschätzung. Wenn Ratschläge dem Hörer stets als bekannt ausgegeben würden, hätte man es dazu mit sehr unhöflichen Kontexten zu tun. Wie oben schon angeführt, muss man unterscheiden zwischen der Kontextsituation und den Aspekten, auf die die Partikeln verweisen. Im gleichen Kontext können natürlich oftmals verschiedene Partikeln verwendet werden, die eben auf verschiedene Verhältnisse Bezug nehmen.

Auch Einladungen kann man nicht grundsätzlich als bekannt einstufen. In manchen Kontexten bietet sich eine Interpretation als Stereotyp an, etwa, dass der Angesprochene in Ankündigungen wie in (\ref{1088}) zu den beteiligten Aktivitäten (testweise) dazugebeten wird. 
 
Will man die Bekanntheitshypothese aufrecht erhalten, gibt es natürlich immer die Möglichkeit, dieses Bedeutungsmoment als vorausgesetzt anzusehen. Für die Verwendung als Vorschlag, Rat, Empfehlung oder Einladung halte ich eine solche Bewertung des Direktivs nicht für plausibel. Dieser Eindruck bestätigt sich auch, wenn man genauer in die (Bedingungen der) Illokutionstypen schaut. Für Vorschläge \is{Vorschlag} heißt es z.B., der Hörer habe dem Sprecher ein \glqq vages Ziel\grqq{} angegeben (\citealt[319]{Rehbein1977}). Es handle sich dabei nur um \glqq ein Grobziel\grqq{}, \glqq das so vage ist, daß er nicht weiß, wie er es erreichen soll\grqq{} (S. 317) (vgl. \citealt[186]{Rolf1997}). Für Ratschläge \is{Ratschlag} gilt nach \citet[186-187]{Rolf1997}, dass der Hörer ein (technisch- oder moralisch-praktisches) Problem hat, das er auch explizit an den späteren Sprecher adressiert haben kann. Das Vorliegen eines Problems gilt auch für Empfehlungen (vgl. ebd.). Vor dem Hintergrund dieser Charakterisierungen wäre es meiner Meinung nach merkwürdig, wenn ein Vorschlag, Rat, eine Empfehlung stets bekannt wäre. Unter diesen Umständen müsste sich der Adressat auch gar nicht in den Kontext begeben, dass er den/die jeweilige(n) Vorschlag, Rat oder Empfehlung erhält. Diese Illokutions\-typen hätten dann immer die Lesart, dass der Adressat die Lösung sowieso weiß. Diese Verhältnisse entsprechen nicht meiner Intuition hinsichtlich der Gründe, aus denen derartige Äußerungen gemacht werden.  

An den Ausführungen zu \textit{doch} in Direktiven sieht man, dass diese Partikel mit verschiedenen direktiven Illokutionstypen \is{Illokutionstyp} kompatibel ist und in diesem Sinne recht unspezifisch ist (vgl. \citealt[119]{Thurmair1989}). Für \textit{auch}, dessen Vorkommen in direktiven Äußerungen Gegenstand des nächsten Abschnitts ist, liegen spezifi\-schere Auftretensweisen vor.

\subsection{Das Einzelauftreten von \textit{auch}} 
\label{sec:dadir}
\subsubsection{Der Diskursbeitrag von \textit{auch}}
Da ich grundsätzlich von der Möglichkeit einer minimalistischen Bedeutungszu\-schreibung \is{Bedeutungsminimalismus/-maximalismus} ausgehe, verfolge ich auch hier das Ziel, \textit{auch} in Direktiven parallel zu seinem Vorkommen in Assertionen zu behandeln. Für assertive Verwendungen (vgl. z.B. (\ref{1091})) habe ich in Abschnitt~\ref{sec:auch} angenommen, dass die Relation p $>$ q, die auf einer Norm, Erwartung oder Erfahrung basiert, im cg enthalten ist, und sich q mindestens unter den Diskursbekenntnissen von A oder B befindet. Durch das Äußern der \textit{auch}-Assertion wird p eingeführt und fungiert als Erklärung für q, das als für den Sprecher ableitbar ausgegeben wird, weil es aus der cg-Relation folgt.
	
\begin{exe}
	\ex\label{1091} 
	A: Es ist so kalt.\\
	B: Wir haben \textbf{auch} November.				     
\end{exe}	
Generell sind \textit{auch}-Direktive in der Literatur bisher sehr wenig berücksichtigt worden. Wenn überhaupt, werden sie bei der Diskussion von \textit{auch}-Assertionen am Rande erwähnt. (\ref{1092}) zeigt einige Beispiele.

\begin{exe}
	\ex\label{1092} 
		\begin{xlist}	
			\ex\label{1092a} Sei \textbf{auch} brav!	
			\hfill\hbox {\citet[97]{Diewald1998}}
			\ex\label{1092b} Nun iß \textbf{auch} schön!
			\hfill\hbox {\citet[158]{Thurmair1989}}
			\ex\label{1092c} Jetzt geh \textbf{auch} (schön)!	
			\hfill\hbox {\citet[60]{Dittmann1980}}
			\ex\label{1092d} Schreibe \textbf{auch} ordentlich!		
			\hfill\hbox {\citet[90]{Helbig1990}}
			\ex\label{1092e} Kommt \textbf{auch} nicht zu spät nach Haus!		
			\hfill\hbox {\citet[58]{Dahl1988}}
			\ex\label{1092f} (A sieht, wie B den Autoschlüssel nimmt, er sagt:)\\
			A: Vergiß \textbf{auch} das Tanken nicht!		
			\hfill\hbox {\citet[50]{Dahl1988}}
		\end{xlist}
\end{exe}
In deskriptiven Erfassungen des \textit{auch}-Beitrags ist aus verschiedenen Ausführungen abzulesen, dass die Handlung ableitbar bzw. bekannt ist (vgl. z.B. \citealt[109]{Burkhardt1982}). \citet[158]{Thurmair1989} schreibt, die Handlung, zu der aufgefordert wird, ist eine \glqq allgemein gültige Norm\grqq{}. \citet[50]{Dahl1988} und \citet[78]{Kwon2005} vertreten, die Handlungsausführung sei bei B bekannt (weil beispielsweise vorher darüber gesprochen wurde). Angenommen wurde darüber hinaus, ein \textit{auch}-Befehl könne nicht der erste Befehl sein (\citealt[50]{Dahl1988}, \citealt[78]{Kwon2005}) und der Hörer sollte den Sachverhalt schon ausgeführt haben (\citealt[90]{Helbig1990}, \citealt[215]{Kwon2005}). Diese Eindrücke finden sich auch in der Paraphrase in (\ref{1093}).

\begin{exe}
	\ex\label{1093} 
	\glq Ich, der Sprecher, habe erwartet, daß du p tust/getan hast; du tust p nicht/\\hast p nicht getan. Tue jetzt p.\grq {} 	
	\hfill\hbox {\citet[60]{Dittmann1980}}
\end{exe}
Ferner wird vertreten, bei \textit{auch}-Direktiven sei eine asymmetrische Beziehung beteiligt, die sich darin äußert, dass sie typischerweise von Erwachsenen an Kinder gerichtet würden (\citealt[158]{Thurmair1989}, \citealt[78]{Kwon2005}). 

Illokutionär sind \textit{auch}-Direktive Vorwürfe \is{Vorwurf} oder \is{Ermahnung} Ermahnungen (\citealt[58]{Dahl1988}, \citealt[78]{Kwon2005}). Mit diesem Punkt hängt auch zusammen, in welchen Formtypen die \textit{auch}-Direktive möglich sind. Neben V1-Imperativsätzen, die ich hier aus\-schließlich betrachte, tritt diese Partikel auch in \textit{dass}-VL-Sätzen auf (vgl. (\ref{1094})).

\begin{exe}
	\ex\label{1094} 
    \begin{tabular}[t]{ll}
	Enkeltochter: & Oma, hast du ma was Geld? Meine Strümpfe sind kaputt, \\
	& ich muß mich doch vorstelln gehn.\\
	Großmutter: & (Seufzend kramt sie in ihrer Schürzentasche) \\
	& Daß du \textbf{auch} Strümpfe kaufst! Nicht wieder bloß Süßkram!
	\end{tabular}
	\newline
	\hbox{}\hfill\hbox{(TAZ, 24.09.1994, 40)}
	\newline
	\hbox{}\hfill\hbox{\citet[78]{Kwon2005}}				       
\end{exe}
\textit{Das}s-VL-Aufforderungen gelten als starke Aufforderungen (vgl. \citealt[54]{Thurmair1989}). 

Ich meine, dass man bei den \textit{auch}-Direktiven zwei grundsätzliche Fälle trennen muss: In der einen Verwendung hat der Adressat sich schon dazu bekannt, dass er p tun wird. Dieser Gebrauch fängt mit auf, dass es sich nicht um den ersten Befehl handelt, wobei m.E. hinzu kommt, dass die Bereitschaft zur Übernahme bereits angezeigt wurde. Nach Ablehnung kann ein \textit{auch}-Direktiv nicht die nächste Aufforderung sein (vgl. (\ref{1095}) und die Belege weiter unten).

\begin{exe}
	\ex\label{1095} 
 	A: Trink deinen Kaffee!\\
	B: Ne, ich möchte nicht.\\
	A: \#Trink ihn \textbf{auch}! Der ist wirklich gut.\\
	vs.\\
	A : Trink ihn \textbf{doch}! Der ist wirklich gut.
\end{exe}
Die zweite Verwendungsweise umfasst Belege, die im Kontext zum ersten Mal geäußert werden und für die gilt, dass die Handlungsaufforderung aus dem Kontext, Weltwissen oder einer erfüllten salienten Bedingung (s.u.) ableitbar ist.

Die beiden Verwendungsweisen können sich überlappen, dies muss aber nicht der Fall sein.

Im Folgenden wird zunächst Gebrauch 2 genauer untersucht, an dem ich meine Modellierung einführe. Es handelt sich hierbei um Kontexte, die Beispielen der Art \textit{Sei \textbf{auch} artig/brav!}, \textit{Benimm dich \textbf{auch}!} entsprechen (vgl. (\ref{1096}) bis (\ref{1100})).

\begin{exe}
	\ex\label{1096} 
	\scriptsize
 	Trixis Vater freute sich und wünschte ihr eine gute Reise. \glqq Grüß' mir Tante Ottilie \textbf{und sei \underline{auch} artig!} Ich rufe heute Abend 		an.\grqq{}\\	
	\glqq Mach  ich, Papa. Aber jetzt müssen wir zum Bahnhof fahren, sonst verpasse ich wirklich noch meinen Zug.\grqq{}, verabschiedete sich Trixi von 		ihrem Vater.
	\hfill\hbox{(Liebrecht 2002 Der Sonderbare Schaukelstuhl)}				 
  	\newline
	\hbox{}\hfill\hbox{(https://books.google.de/books?id=in9y1c-FPdgC\&pg=PA5\&lpg=PA}		
   	\newline
	\hbox{}\hfill\hbox{5\&dq=\%22sei+auch+artig\%22\&source=bl\&ots=3Fkfw7Of0F\&sig=k}
	\newline
	\hbox{}\hfill\hbox{WTagKqIg\_5j80JCD8WPnLv6FoI\&hl=de\&sa=X\&ved=0CCsQ6AE}
	\newline
	\hbox{}\hfill\hbox{wAmoVChMIvcbX7PzGyAIVyPJyCh2qoQHB\#v=onepage\&q=\%22s}
	\newline
	\hbox{}\hfill\hbox{ei\%20auch\%20artig\%22\&f=false)}
\end{exe}
	
\begin{exe}
	\ex\label{1097} 
	\scriptsize
 	\glqq Den Schlüssel hast du?\grqq{} fragte Papa.\\
	\glqq Ja, Papa.\grqq{}\\
	\glqq Bitte, schließ die Tür gut ab, wenn du die Wohnung verläßt. Und laß nachmittags, wenn Du allein zu Hause bist, ja keinen rein!\grqq{} ermahnte 		Papa sie.\\
	\glqq Ja, Papa.\grqq{}\\
	\glqq \textbf{Mach \underline{auch} deine Hausaufgaben!}\grqq{}\\
	\glqq Ich weiß schon selbst, daß ich meine Hausaufgaben machen muß!\grqq{} entgegnete Tina.
	\newline
	\hbox{}\hfill\hbox{(Hogger 2000, Tina und der Teddybär)}				 
  	\newline
	\hbox{}\hfill\hbox{(https://books.google.de/books?id=tyNR-EWT9qMC\&pg=PA109\&lp}		
   	\newline
	\hbox{}\hfill\hbox{g=PA109\&dq=\%22Mach+auch+deine+Hausaufgaben\%22\&source=b}
	\newline
	\hbox{}\hfill\hbox{l\&ots=Dcwv-lmPUy\&sig=FWyk77CetirGeSeOEIx4vME0UaI\&hl=de}
	\newline
	\hbox{}\hfill\hbox{\&sa=X\&ved=0CCUQ6AEwAWoVChMIxMiR7J7HyAIV5yVyCh27}
	\newline
	\hbox{}\hfill\hbox{7AKq\#v=onepage\&q=\%22Mach\%20auch\%20deine\%20Hausaufgabe}
	\newline
	\hbox{}\hfill\hbox{n\%22\&f=false)}
\end{exe}						                              
						
\begin{exe}
	\ex\label{1098} 
	\scriptsize
 	Anruf von Frau Mutter: \glqq \textbf{Trink \underline{auch} genug bei dem Wetter!}\grqq{}
	\newline
	\hbox{}\hfill\hbox{(https://twitter.com/herrhellmute/status/618429887519526913)}				 
  	\newline
	\hbox{}\hfill\hbox{(Google-Suche, eingesehen am 12.12.2015)}		
\end{exe}	
								                    
\begin{exe}
	\ex\label{1099} 
	\scriptsize
 	Mit Rauschebart schien der finstere Gesell' schon ziemlich alt, jedenfalls musste er sich hinsetzen und Marktmeisterin Monika Haudel die Geschenke aus 		dem Sack holen. Irgenwie schien er aber alle Kinder zu kennen und kannte sich auch sonst in der Stadt gut aus: \glqq Räum scheen dei Zimmer uff, und 		\textbf{putz dir \underline{auch} die Zähne}\grqq{} gab er den Kleinsten mit. 
	\newline
	\hbox{}\hfill\hbox{(http://www.heylive.de/index.php?id=96\&tx\_ttnews$[$pointer$]$}				 
  	\newline
	\hbox{}\hfill\hbox{=13\&tx\_ttnews$[$tt\_news$]$=104\&tx\_ttnews$[$backPid$]$=35\&cHa}		
	\newline
	\hbox{}\hfill\hbox{sh=f2ddcfb43b7771418de993f3915d72c5)}
	\newline
	\hbox{}\hfill\hbox{(Google-Suche, eingesehen am 12.12.2015)}
\end{exe}
									
\begin{exe}
	\ex\label{1100} 
	\scriptsize
 	\glqq \textbf{Sei \underline{auch} pünktlich!}\grqq{} ermahnt mich Kaspar jedesmal, wenn wir uns verabreden. Dabei müßte er längst wissen, daß ich 			geradezu superpünktlich bin.
	\newline
	\hbox{}\hfill\hbox{(http://www.abendblatt.de/archiv/1965/article200886539/Sin}				 
  	\newline
	\hbox{}\hfill\hbox{d-wir-wirklich-unpuenktlich.html)}		
\end{exe}
(\ref{1101}) zeigt meine Modellierung des Kontextzustandes, der besteht, bevor ein \textit{auch}-Direktiv geäußert wird.

\begin{exe}
\ex\label{1101} Kontextzustand vor einem \textit{auch}-Direktiv\\[-0.6em]
\begin{tabular}[t]{|C{6em}|C{12em}|C{6em}|}
\hline
$\textrm{DC}_{\textrm{A}}$ & Tisch &  $\textrm{DC}_{\textrm{B}}$ \tabularnewline
\hline
{} & {} & q  \tabularnewline
\cline{1-1}\cline{3-3}
$\textrm{TDL}_{\textrm{A}}$ & {} & $\textrm{TDL}_{\textrm{B}}$  \tabularnewline
\cline{1-1}\cline{3-3}
{} & {} & {}  \tabularnewline
\hline
\multicolumn{3}{|l|}{cg s$_{1}$ = $\lbrace$q $>$ !p$\rbrace$} \tabularnewline
\hline
\end{tabular}
\end{exe}									
Die Relation q $>$ !p ist im cg enthalten. Aus dem Kontext ist ersichtlich, dass mindestens B (der Adressat des späteren Direktivs), meistens A und B, von q ausgehen. Dass nur A von q ausgeht, scheidet meiner Meinung nach aus, da der Direktiv immer an B gerichtet wird. Wenn der Direktiv geäußert wird, teilt A B etwas mit, das für diesen nicht überraschend ist, weil er es aus der Relation im cg ableiten können sollte. Er wird folglich zur Realisierung von p angehalten und diese Anweisung ist erwartet, weil sie aus der aktuellen Situation (Eintritt/Vorliegen von q) folgt. B hat im Grunde keine Möglichkeit, !p nicht auf seine TDL aufzunehmen, da B q $>$ !p ebenfalls teilt. Die Möglichkeit der Ablehnung bei der Frage \textit{Wird B p auf seine TDL nehmen oder nicht?} wird somit übersprungen und es entsteht ein ähnlicher Effekt, wie ihn \textit{ja} und \textit{eben} in Assertionen bewirken oder auch \textit{eben} ihn in Direktiven auslöst (vgl. (\ref{1102})).
\pagebreak		
\begin{exe}							
\ex\label{1102} Kontextzustand nach einem \textit{auch}-Direktiv\\[-0.6em]
\begin{tabular}[t]{|C{6em}|C{12em}|C{6em}|}
\hline
$\textrm{DC}_{\textrm{A}}$ & Tisch &  $\textrm{DC}_{\textrm{B}}$ \tabularnewline
\hline
{} & {} & q  \tabularnewline
\cline{1-1}\cline{3-3}
$\textrm{TDL}_{\textrm{A}}$ & {} & $\textrm{TDL}_{\textrm{B}}$  \tabularnewline
\cline{1-1}\cline{3-3}
{} & {} & !p  \tabularnewline
\hline
\multicolumn{3}{|l|}{cg s$_{2}$ = $\lbrace$s$_{1}$ $\cup$ $\lbrace$!p $\in$ $\textrm{TDL}_{\textrm{B}}\rbrace$} \tabularnewline
\hline
\end{tabular}
\end{exe}										 
In Beispiel (\ref{1096}) ist dann die Relation in (\ref{1103}) beteiligt.

\begin{exe}
\ex\label{1103} 
	Wenn du Tante Ottilie besuchst, musst du artig sein.
\end{exe}	
Aus dem Kontext ist klar, dass Vater und Tochter sich einig sind, dass die Tochter die Tante besucht. Die Ermahnung des Vaters ist deshalb erwartbar. Da die Tochter weiß, dass dies das angemessene Verhalten ist, besteht für sie keine Möglichkeit, die Realisierungsabsichten hinsichtlich dieses Sachverhalts abzuleh\-nen. Die gleichen Verhältnisse stellen sich in den anderen Fällen in (\ref{1096}) bis (\ref{1100}) ein (vgl. die beteiligten Relationen in (\ref{1104})).

\begin{exe}
	\ex\label{1104} 
		\begin{xlist}	
			\ex\label{1104a} Wenn in der Schulzeit Nachmittag ist, musst du deine Hausaufgaben machen.
			\ex\label{1104b} Wenn es warm ist wie derzeit, musst du genug trinken.\footnote{In dem Beleg wird mit dieser Ermahnung eigentlich gespielt, 				indem vom Angesprochenen ein Bier hochgeladen wird, an das seine Mutter sicherlich nicht dachte.}
			\ex\label{1104c} Wenn der Nikolaus zufrieden sein soll, musst du dir die Zähne putzen.
			\ex\label{1104d} Wenn sie sich mit ihm verabredet, soll sie pünktlich sein.
		\end{xlist}
\end{exe}
In dieser Verwendung wird zu Verhaltensweisen aufgefordert, die normiert in bestimmten Situationen (die durch den aktuellen Kontext vorgegeben sind) zu realisieren sind, so dass es nicht verwunderlich ist, dass der Sprecher zu ihnen auffordert. Weil eine Norm beteiligt ist, kann auch gar kein Zweifel bestehen, dass der Adressat die Bereitschaft zur Ausführung der Handlung nicht ablehnt. Nicht umsonst handelt es sich hier illokutionär \is{Ermahnung} um Ermahnungen. \citet[179]{Rolf1997} nimmt an, dass \glqq der Adressat hinsichtlich dessen, was er tun oder unterlassen soll, bereits aufgefordert worden ist\grqq{}. Eine bereits explizit erfolgte Aufforderung liegt hier sicherlich nicht vor, aber der Adressat muss mit der Aufforderung rechnen. In meinen Augen entsteht der Eindruck von vorbeugenden Ermahnungen, die dazu auch etwas Belehrendes haben können. Denkbare Reaktionen sind beispielsweise \textit{Ja. Ja. Ich weiß./Natürlich.} Meiner Modellierung nach liegen grundsätzlich zu Assertionen parallele Verhältnisse vor, die aller\-dings gespiegelt sind, weil eine \textit{auch}-Assertion den Grund für die gegebene Folge und ein \textit{auch}-Direktiv die Folge aus der gegebenen Bedingung angibt. Im Falle von Assertionen ist die Relation p $>$ q im cg, q ist im Kontext und q wird über p begründet. Wenn der Adressat p akzeptiert, sollte das Vorliegen von q für ihn klar sein. Bei \textit{auch}-Direktiven ist die Relation q $>$ !p im cg, q trifft im Kontext zu, weshalb !p folgt.

Auch für die oben erwähnte zweite Gebrauchsweise (Gebrauch 1) eines \textit{auch}-Direktivs setze ich die Kontextbeschreibung in (\ref{1101}) an. Die Gebrauchsunterschiede ergeben sich aus der Gestalt von q. 

In dieser Verwendung hat der Adressat sich schon dazu bekannt, die Handlung auszuführen, sie ist bisher aber ausgeblieben, so dass für den Sprecher ein Grund entsteht, darauf hinzuweisen, dass ihr nachzukommen ist. Im Dialog in (\ref{1105}) ist z.B. aus dem Kontext klar, dass das Kind essen soll, und es gibt ein Hin und Her darum, wie es die Nudeln haben möchte, um sie zu essen. Schließlich sind sie so hergerichtet wie gefordert, und dennoch sieht es so aus, als ob es sie nicht essen wird.

\begin{exe}
	\ex\label{1105} 
	\scriptsize
 	K: Kei:ne Soße da:zu.\\
	V zu K: Du willst $\uparrow$keine Soße dazu?\\
	K zu V: Ne:e.\\
	V zu K: Sag mal, so langsam\\
	M zu K: spinnst du. (.) Saskia, stell dich nicht so an! ((zweiter Teil deutlich lauter))\\
	V zu K: Was machst du denn da?\\
	K zu V: Will keine So:ße $\downarrow$haben.\\
	V zu K: Keine Soße?\\
	K zu V: $^{0}\textrm{mhmh}^{0}$ ((verneinend)) ((K schüttelt vorsichtig den Kopf))\\
	V zu K: $\uparrow$Komm her, dann geb ich dir meine. (ein paar Wörter unverständlich) lass mal sehen. $\uparrow$Dann kriege ich die da mit der Soße. (.) 	So und du kriegst die hier $\uparrow$ohne Soße. (.) Bitteschön $>$ von Papa$<$ Ohne Soße. Bitteschön. Ohne Soße. Da! ((V nimmt sich Teller von K und 		schüttet Nudeln vom K auf seinen Teller und gibt K Nudeln ohne Soße zurück))\\
	((K fängt an zu quengeln))\\
	K: Der hat mir meine abgenom:men. ((zeigt auf seinen Teller und fängt an weinerlich zu schreien))\\
	V zu K: Du wolltest doch keine Soße haben. ((laut))\\
	((V guckt kurz vom Kind zu Mutter))\\
	V zu M: Also sowas! ((wieder bisschen leiser))\\
	V zu K: Du wolltest doch ohne Soße haben. (.) Ja, willst du wieder deine Soßendinger haben, oder $\uparrow$was? ((wieder laut))\\
	K zu V: Ja::a ((weinerlicher Ton, aber K hört auf zu quengeln))\\
	V zu K: Also doch mit Soße! ((laut))\\
	M zu K: Mm, eben hast gesagt \glqq ohne Soße\grqq{}, \textbf{$\uparrow$jetzt iss \underline{auch} ohne Soße.} ((K fängt wieder an zu quengeln))(.) (Ein 	paar Wörter unverständlich) zurück. ((auch laut))\\
	((K zeigt mit Finger auf die Schüssel Nudeln))
	\newline
	\hbox{}\hfill\hbox{(Keller 2015: 32-33., Die Entwicklung der Generation Ich Eine}				 
  	\newline
	\hbox{}\hfill\hbox{psychologische Analyse aktueller Erziehungsleitbilder, Springer}		
	\newline
	\hbox{}\hfill\hbox{(http://link.springer.com/book/10.1007\%2F978-3-658-10392-7)}
\end{exe}
Dem Kind kann zugeschrieben werden, dass es die Aufnahme von !p in seine TDL bereits akzeptiert hat. Wenn es zustimmt, die Nudeln ohne Soße zu essen, kann es nicht länger das Essen ablehnen, wenn der Zustand, unter dem es bereit ist zu essen, hergestellt ist.

In Beispiel (\ref{1106}) ist trotz des Versprechens die Handlung zur Heirat bisher noch nicht erfolgt. Auf der TDL von B steht somit das Vorhaben zu heiraten, die Rea\-lisierung ist aber noch nicht erfolgt bzw. es sieht nicht danach aus.
	
\begin{exe}
	\ex\label{1106} 
	\scriptsize
 	\glqq [...] Gleich nach der Hochzeit geht's nach Europa. Heute Abend noch. In ein paar Stunden. Lass uns
 	zu mir fahren. Der Priester wartet und die Gäste, und ...\grqq{}\\
	\glqq Nein, Stephen, es geht nicht. Ich kann dich nicht heiraten. Ich hab's mir anders ...\grqq{}\\
	\glqq Halt! Sag es nicht, Lorraine. Du hast behauptet, du liebst mich! Du willst mich heiraten, hast du versprochen. \textbf{Jetzt tu es 					\underline{auch}!}\grqq{}	
	\hfill\hbox{(Morgen 2009, Bis der Tod uns eint)}				 
  	\newline
	\hbox{}\hfill\hbox{(https://books.google.de/books?id=y1QOre76SQwC\&pg=PA235\&lpg=PA235}		
	\newline
	\hbox{}\hfill\hbox{\&dq=\%22Jetzt+tu+es+auch\%22\&source=bl\&ots=m9nDWXTBus\&sig=F8HC}	
	\newline
	\hbox{}\hfill\hbox{enpQ19vhLiP\_rsX7B7i8ZEk\&hl=de\&sa=X\&ved=0CCkQ6AEwA2oVChMItu}
	\newline
	\hbox{}\hfill\hbox{vHmZbHyAIVxSVyCh3wZQkS\#v=onepage\&q=\%22Jetzt\%20tu\%20es\%20au}
	\newline
	\hbox{}\hfill\hbox{ch\%22\&f=false)}
\end{exe}	
Ähnlich liegt in (\ref{1107}) die Situation vor, dass der Adressat gesagt hat, dass sie p tun werden. !p ist somit auf seiner TDL. Da die Handlung noch nicht ausgeführt wurde, scheint trotz der Ankündigung zweifelhaft, ob sie tatsächlich realisiert wird.
	
\begin{exe}
	\ex\label{1107} 
	\scriptsize
 	Meine Damen und Herren, ich habe bereits mehrfach gesagt, dass wir die Stellungnahmen, die eingegangen sind, insbesondere natürlich auch die der beiden 	kommunalen Landesverbände, ernst nehmen und den Gesetzentwurf deutlich überarbeiten. (Heiterkeit bei einzelnen Abgeordneten der CDU - Dr. Armin Jäger, 		CDU: \textbf{Dann machen Sie das \underline{auch}!} Dann machen Sie das!) Er wird mit anderen Worten nicht so in den Landtag eingebracht, wie er im 		November 2004 zur Anhörung gebracht wurde.	 	
	\newline
	\hbox{}\hfill\hbox{(PMV/W04.00054 Protokoll der Sitzung des Parlaments Landtag}
	\newline
	\hbox{}\hfill\hbox{Mecklenburg-Vorpommern am 10.03.2005. 54. Sitzung der 4. Wahlperiode}
	\newline
	\hbox{}\hfill\hbox{2002-2006. Plenarprotokoll, Schwerin, 2005)}
\end{exe}	
In dieser Gebrauchsweise von \textit{auch} in Direktiven liegt ebenso die Relation q $>$ !p im cg vor. Die Proposition q entspricht hier allerdings der komplexen Aussage !p $\in$ TDL$_{\textrm{B}}$ (vgl. (\ref{1108})).

\begin{exe}
\ex\label{1108} Kontextzustand vor dem \textit{auch}-Direktiv in Verwendung 1\\[-0.6em]
\begin{tabular}[t]{|C{6em}|C{12em}|C{6em}|}
\hline
$\textrm{DC}_{\textrm{A}}$ & Tisch &  $\textrm{DC}_{\textrm{B}}$ \tabularnewline
\hline
{} & {} & q  \tabularnewline
\cline{1-1}\cline{3-3}
$\textrm{TDL}_{\textrm{A}}$ & {} & $\textrm{TDL}_{\textrm{B}}$  \tabularnewline
\cline{1-1}\cline{3-3}
{} & p $\vee$ $\neg$p & !p  \tabularnewline
\hline
\multicolumn{3}{|l|}{cg s$_{1}$ = $\lbrace$q $>$ !p$\rbrace$} \tabularnewline
\hline
\end{tabular}
\end{exe}
Dieser Gebrauch wirkt vorwurfsvoller als der zuvor beschriebene. Bei Kontext 2 gibt es m.E. keine Anzeichen dafür, dass der Adressat der geforderten Handlung nicht nachkommen wird. Z.T. handelt es sich hierbei auch um völlig überflüssige Hinweise, die Reaktionen wie \textit{Ja. Ja.} oder \textit{Ich weiß.} hervorrufen (s.o.). In Kontext 1 besteht (trotz Bestätigung von !p) $[$das deshalb auf der TDL steht$]$ Zweifel, ob B p nachkommt.

Ich bin der Meinung, dass man diese beiden Kontexte, die in der Literatur in Ausführungen zu \textit{auch} in Direktiven in einem Zug erwähnt werden, trennen sollte. Der minimale Bedeutungsbeitrag \is{Bedeutungsminimalismus/-maximalismus} von \textit{auch} ist zwar derselbe (q $>$ !p im cg, q mindestens in DC$_{\textrm{B}}$), q nimmt jedoch jeweils eine andere Gestalt an, womit in Kontext 1 wiederum zusätzliche Füllungen der Komponenten einhergehen.
				
Neben diesen beiden Gebrauchsweisen, die in der Literatur Erwähnung finden, lässt sich dazu noch eine weitere Verwendung ausmachen: Es gibt Fälle, in denen die Aufforderung ableitbar ist, der Adressat ihr bisher noch nicht nachgekommen ist, er sich aber noch nicht dazu bekannt hat, p zu tun. Es handelt sich um Kontexte, in denen die Bedingung offen thematisiert ist (in Kontext 2 ist diese nur situativ gegeben), was mit sich bringt, dass der Aspekt der Ableitbarkeit noch salienter ist als in Kontext 2. Äußerungen wirken dadurch vorwurfsvoller als in Kontext 2. Anders als in Kontext 1 hat der Adressat !p aber nicht schon zuge\-stimmt. Beispiele für diesen Gebrauch finden sich in (\ref{1109}) und (\ref{1110}).

\begin{exe}
	\ex\label{1109} 
	\scriptsize
 	Bezeichnend ist der Ruf von Barisits nach einem Spieler wie Lukas Kulovits, der sicher nicht die fußballerische Qualität der Siegls hat, dafür in 			anderen Bereichen seine Vorzüge genießt. Gegen Winden muss jetzt ein Sieg her und diese Ausnahmekicker sind jetzt um so mehr gefordert, denn dass sie 		mit zu den besten im Land zählen, bleibt unbestritten. \textbf{Zeigt es \underline{auch}!}	
	\newline
	\hbox{}\hfill\hbox{(BVZ12/APR.00864 Burgenländische Volkszeitung, 12.04.2012)}
\end{exe}	

\begin{exe}
	\ex\label{1110} 
	\scriptsize
 	\glqq Er ist unser Sohn! Oder sag mir einen guten Grund, warum er es nicht mehr sein kann!\grqq{} \glqq Du würdest das nicht verstehen. Es ist eine 		Sache unter Männern.\grqq{} Der kleine Wagen ächzte, als Mechthild neben ihren Mann auf den schmalen Holzsitz stieg. \glqq Wenigstens erkennst du, dass 	Gerald ein Mann geworden ist. Dann behandle ihn auch so!\grqq{} Er funkelte sie an. \glqq Er muss den ersten Schritt tun!\grqq{}	
	\newline
	\hbox{}\hfill\hbox{(DIV/ERB.00001 Erwin, Birgit ; Buchhorn, Ulrich: Die Herren}
	\newline
	\hbox{}\hfill\hbox{von Buchhorn, $[$Roman$]$. - Meßkirch, 25.03.2011)}
\end{exe}
Die Relationen, die jeweils im cg enthalten sind, sind:

\begin{exe}
	\ex\label{1111} 
		\begin{xlist}	
			\ex\label{1111a} Wenn man zu den besten Spielern gehört, muss man es zeigen.
			\ex\label{1111b} Wenn Gerald ein Mann ist, musst du ihn wie einen behandeln.
		\end{xlist}
\end{exe}	
Der Direktiv ist hier ableitbar, wenn man die Relation teilt, die Bedingung vorerwähnt ist und sie als geteilte Information ausgegeben wird.

Da für alle \textit{auch}-Direktive gilt, dass die Relation q $>$ !p im cg enthalten ist und q im Kontext salient ist, muss es natürlich immer einen Grund geben, einen Direktiv zu äußern, um den der Adressat eigentlich weiß. Seine Äußerung kann aus Gründen der Versicherung erfolgen (Kontext 2 $[$(vorbeugende, belehrende) Ermahnung$]$). Sie kann auch dadurch bedingt sein, dass nicht klar ist, ob der Adressat p tatsächlich tut, obwohl er schon zugestimmt hat (Kontext 1 $[$Vorwurf$]$). Oder die Bedingung ist salient und es sieht aber nicht so aus, als ob der Adressat von allein die Folge erfüllt (Kontext 3 $[$Ermahnung und Vorwurf$]$). Die invariante Kontextanforderung von \textit{auch} ist in allen drei Gebrauchsweisen durch (\ref{1112}) erfasst.
 	
\begin{exe}
\ex\label{1112} Kontextzustand vor einem \textit{auch}-Direktiv\\[-0.6em]
\begin{tabular}[t]{|C{6em}|C{12em}|C{6em}|}
\hline
$\textrm{DC}_{\textrm{A}}$ & Tisch &  $\textrm{DC}_{\textrm{B}}$ \tabularnewline
\hline
{} & {} & q  \tabularnewline
\cline{1-1}\cline{3-3}
$\textrm{TDL}_{\textrm{A}}$ & {} & $\textrm{TDL}_{\textrm{B}}$  \tabularnewline
\cline{1-1}\cline{3-3}
{} & {} & {}  \tabularnewline
\hline
\multicolumn{3}{|l|}{cg s$_{1}$ = $\lbrace$q $>$ !p$\rbrace$} \tabularnewline
\hline
\end{tabular}
\end{exe}		
Die Relation ist Teil des cg und die Voraussetzung wird mindestens von B vertreten.
					
\subsubsection{Der Diskursbeitrag von \textit{auch}- und \textit{eben}-Direktiven}
Meine Modellierung für \textit{auch}- und \textit{eben}-Direktive ist im Sinne der vorliegenden Diskurszustände/-effekte dieselbe. Für letztere habe ich in Abschnitt~\ref{sec:kontexte} in Kapitel~\ref{chapter:hue} ebenfalls angenommen, dass dem Adressaten ein Auftrag erteilt wird, den er ableiten könnte (vgl. z.B. (\ref{1113})).

\begin{exe}
	\ex\label{1113} 
	A: Ich bin immer so müde.\\
	B: Dann geh \textbf{eben} früher ins Bett!
\end{exe}
Wenngleich anhand des Kontextzustandes, wie modelliert in Anlehnung an das Modell von \citet{Farkas2010}, eine Differenzierung zwischen \textit{auch}- und \textit{eben}-Direktiven nicht möglich ist, gibt es dennoch weitere Kriterien der Unterscheidung. Die \textit{eben}-Direktive haben für meine Begriffe eine eingeschränktere Verwendung, weil q ein Problem darstellt. Illokutiv sind sie deshalb \is{Ratschlag} Ratschläge. Dadurch, dass q als Problem ausgegeben wird und der eine Diskursteilnehmer den anderen um Hilfe bittet, signalisiert er, dass er nicht von allein auf !p gekommen wäre. Da die Relation im cg ist, müsste er dies eigentlich. Aus diesem Verhältnis folgt die harsche Wirkung, die \textit{eben}-Direktive im Gegensatz zu \textit{auch}-Direktiven aufweisen. Die \textit{auch}-Direktive werden hingegen i.d.R. verwendet, wenn gar nicht Gegenteiliges signalisiert wird hinsichtlich der Bereitschaft, p zu erfüllen. 							
\textit{Auch} kann in dem \textit{eben}-Kontext aus (\ref{1113}) auftreten (vgl. (\ref{1114})), d.h. das Vorliegen eines Problems interveniert nicht negativ mit dem \textit{auch}-Beitrag.

\begin{exe}
	\ex\label{1114} 
	A: Ich bin immer so müde.\\
	B: Dann geh \textbf{auch} früher ins Bett!
\end{exe}
Es handelt sich um die Verwendung aus Kontext 3. !p ist ableitbar, weil q $>$ !p im cg ist. Die Bedingung q ist zudem salient und nach dem Beitrag von A geteilte Information zwischen den Diskurspartnern.

\textit{Eben} kann hingegen nicht verwendet werden, wenn kein Problem besteht. In den Beispielen für den \textit{auch}-Kontext 2 ist die Ersetzung von \textit{auch} durch \textit{eben} nicht möglich (vgl. (\ref{1115}) bis (\ref{1119})).
	
\begin{exe}
	\ex\label{1115} 
	Trixis Vater freute sich und wünschte ihr eine gute Reise. \glqq Grüß  mir Tante Ottilie und \textbf{\#sei \underline{eben} artig!} Ich rufe heute 			Abend an.\grqq{}
\end{exe}	
	
\begin{exe}
	\ex\label{1116} 
	\scriptsize
	\glqq Bitte, schließ die Tür gut ab, wenn du die Wohnung verläßt. Und laß nachmittags, wenn Du allein zu Hause bist, ja keinen rein!\grqq{} ermahnte 		Papa sie.
	\glqq Ja, Papa.\grqq{}
	\glqq \textbf{\#Mach \underline{eben} deine Hausaufgaben!}\grqq{}
	\glqq Ich weiß schon selbst, daß ich meine Hausaufgaben machen muß!\grqq{} entgegnete Tina.
\end{exe}		

\begin{exe}
	\ex\label{1117} 
	\scriptsize
	Irgenwie schien er aber alle Kinder zu kennen und kannte sich auch sonst in der Stadt gut aus: \glqq Räum scheen dei Zimmer uff, \textbf{\#und putz dir 	\underline{eben} die Zähne}\grqq{} gab er den Kleinsten mit. 
\end{exe}	
		
\begin{exe}
	\ex\label{1118} 
	\scriptsize
	\glqq \textbf{\#Sei \underline{eben} pünktlich!}\grqq{} ermahnt mich Kaspar jedesmal, wenn wir uns verabreden. Dabei müßte er längst wissen, daß ich 		geradezu superpünktlich bin. 
\end{exe}		

\begin{exe}
	\ex\label{1119} 
	Anruf von Frau Mutter: \glqq \textbf{\#Trink \underline{eben} genug bei dem Wetter!}\grqq{}
\end{exe}	
Damit die Ersetzung zulässig wird, muss im Kontext ein Problem vorliegen. Am ehesten ist dies hier zu konstruieren für Beispiel (\ref{1120}):

\begin{exe}
	\ex\label{1120} 
	A: Oh, mir ist total schwindelig bei dieser Hitze.\\
	B: Trink \textbf{eben} genug bei dem Wetter!
\end{exe}	
Gleiches gilt für Kontext 1 und 3 (vgl. (\ref{1121}) bis (\ref{1126})).

\begin{exe}
	\ex\label{1121} 
	\scriptsize
 	Meine Damen und Herren, ich habe bereits mehrfach gesagt, dass wir die Stellungnahmen, die eingegangen sind, insbesondere natürlich auch die der beiden 	kommunalen Landesverbände, ernst nehmen und den Gesetzentwurf deutlich überarbeiten. (Heiterkeit bei einzelnen Abgeordneten der CDU - Dr. Armin Jäger, 		CDU: \textbf{Dann machen Sie das \underline{eben}!}) 	
\end{exe}

\begin{exe}
	\ex\label{1122} 
	\scriptsize
 	 \glqq Nein, Stephen, es geht nicht. Ich kann dich nicht heiraten. Ich hab s mir anders ...\grqq{}\\
	\glqq Halt! Sag es nicht, Lorraine. Du hast behauptet, du liebst mich! Du willst mich heiraten, hast du versprochen. \textbf{\#Jetzt tu es 					\underline{eben}!}\grqq{} 	
\end{exe}	

\begin{exe}
	\ex\label{1123} 
	\scriptsize
 	V zu K: Also \textit{doch} mit Soße! ((laut))\\
	M zu K: Mm, eben hast gesagt \glqq ohne Soße\grqq{}, \textbf{\#$\uparrow$jetzt iss \underline{eben} ohne Soße}. ((K fängt wieder an zu quengeln))(.) 		(Ein paar Wörter unverständlich) zurück. ((auch laut))\\
	((K zeigt mit Finger auf die Schüssel Nudeln))	
\end{exe}

\begin{exe}
	\ex\label{1124} 
	\scriptsize
 	Bezeichnend ist der Ruf von Barisits nach einem Spieler wie Lukas Kulovits, der sicher nicht die fußballerische Qualität der Siegls hat, dafür in 			anderen Bereichen seine Vorzüge genießt. Gegen Winden muss jetzt ein Sieg her und diese Ausnahmekicker sind jetzt um so mehr gefordert, denn dass sie 		mit zu den besten im Land zählen, bleibt unbestritten. \textbf{\#Zeigt es \underline{eben}!}	
\end{exe}	

\begin{exe}
	\ex\label{1125} 
	\glqq Wenigstens erkennst du, dass Gerald ein Mann geworden ist. \textbf{\#Dann behandle ihn \underline{eben} so!}\grqq{} Er funkelte sie an.
\end{exe}	

\begin{exe}
	\ex\label{1126} 
	Na, na, du behauptest mir sei zu hoch, was du sagen willst? \textbf{Bitte, dann erkläre dich \underline{eben}!}
\end{exe}
In (\ref{1126}) ist der Vorgangsbeitrag einigermaßen passabel als Problem zu deuten (vgl. (\ref{1127})).

\begin{exe}
	\ex\label{1127} 
	A: Ihr versteht mich ja alle nicht.\\
	B: Dann erkläre dich \textbf{eben}!
\end{exe}
In (\ref{1121}) verliert die Äußerung ihre direktive Interpretation, wenn \textit{eben} auftritt. 
	
Ich halte in allen diesen Beispielen \textit{auch} für adäquater. Damit \textit{eben} auftreten kann, müsste man beim Adressaten eine problematische Ausgangslage schaffen (vgl. die veränderten Kontexte in (\ref{1128}) bis (\ref{1130})). Diese kann ebenfalls bei \textit{auch} vorliegen (obwohl dies normalerweise nicht der Fall ist), sie ist aber nicht notwendig wie im Falle von \textit{eben}.

\begin{exe}
	\ex\label{1128} 
	A: Ich komme mit Gerald gar nicht mehr klar. Er behauptet, jetzt ein Mann zu sein. \\
	B: Dann behandle ihn \textbf{eben} so.
\end{exe}

\begin{exe}
	\ex\label{1129} 
	A: Ich will nicht, dass wir uns bei jeder Verabredung streiten.\\
	B: Sei \textbf{eben} pünktlich.
\end{exe}
		
\begin{exe}
	\ex\label{1130} 
	A: Meine Eltern machen schon Witze, dass ich die Hochzeit versprochen habe und immer noch nichts passiert ist.\\
	B: Dann tu es \textbf{eben}!
\end{exe}
Ich halte die Beobachtung, die Anlass zu diesem Abschnitt gibt, für relevant, weil sie aufzeigt, dass sich nicht jegliche Unterschiede zwischen MPn mit den durch das Diskursmodell vorgegebenen Kategorien erfassen lassen.

\subsubsection{Deskriptive Eindrücke aus der Literatur}
Nach meiner Betrachtung authentischer Belege bewerte ich die Annahmen aus der Literatur (s.o.) folgendermaßen: Die Handlungsaufforderung ist in dem Sinne bekannt, dass sie abzuleiten ist. Damit ist verbunden, dass beim Adressaten wenig Einspruchsmöglichkeit besteht. Aus diesen Verhältnissen wird zudem er\-klärbar, warum die Partikel auch in \textit{dass}-Direktiven auftreten kann, die als starke Direktive gelten.

Ein \textit{auch}-Direktiv kann der erste Befehl sein im Gespräch. Wiederholt wird der \textit{auch}-Direktiv nur in Szenarien entlang von Kontext 1 und da aber auch nur in dem Sinne, dass der Adressat !p schon zugestimmt hat. Nur wiederholen, weil der Aufforderung noch nicht nachgekommen wurde, kann man die \textit{auch}-Direktive für meine Begriffe nicht.

Die Eltern-Kind-Interaktion ist für Kontext 2 ein typisches Interaktionsverhältnis, es muss aber nicht vorliegen. Ich glaube, dass es ein beliebtes Schema ist, das sich deshalb gut eignet, weil a) eine asymmetrische Relation benötigt wird, b) eine zu realisierende Handlung als erwartet ausgegeben werden muss und c) kein Problemlösungsszenario im Raum zu stehen hat wie bei den \textit{eben}-Direktiven. Als Umstände bieten sich hier typischerweise Verhaltskontexte an, die wiederum oft von Eltern an Kinder gerichtet sind. Die obigen Belege zeigen bereits, dass dies nicht so sein muss (vgl. Nikolaus in (\ref{1099}), Freundin und Freund in (\ref{1100})).
 							 
Denkbar ist beispielsweise auch die Anweisung eines Herrchens an seinen Hund (vgl. (\ref{1131})) oder der Rat einer Freundin (vgl. (\ref{1132})).
	
\begin{exe}
	\ex\label{1131} 
	Komm, mach \textbf{auch} Platz!
\end{exe}	
\vspace{-0.5cm}	
\begin{exe}
	\ex\label{1132} 
	\scriptsize
 	\textbf{Dann sei \underline{auch} artig}, Babsi und höre darauf, was das Dokterchen sagt. Sonst hast du nachher wieder Beschwerden, das wäre doch auch 		schlecht. Also Geduld, auch wenn es schwer fällt.
	\newline
	\hbox{}\hfill\hbox{(http://www.forum-garten.de/was-habt-ihr-heute-}
	\newline
	\hbox{}\hfill\hbox{alles-im-garten-gemacht-t243395,start,1130.htm)}
	\newline
	\hbox{}\hfill\hbox{(Google-Suche, eingesehen am 13.12.2015)}
\end{exe}
Man könnte aufgrund der Beispiele denken, dass naive Kontexte dominant vertreten sind. Ihr Vorkommen liegt m.E. daran, dass man nur in wenigen Interaktions\-zusammenhängen das typische \textit{Sei \textbf{auch} artig/brav.} sagt.

Auch Kontexte wie in (\ref{1133}) und (\ref{1134}) halte ich für denkbar.

\begin{exe}
	\ex\label{1133} 
	(Arzt in Menschenmenge)\\
	A: Zum Glück. Endlich ein Arzt!\\
	B: (Muss Leute beiseite schieben.) Lassen Sie mich \textbf{auch} durch! 
\end{exe}

\begin{exe}
	\ex\label{1134} 
	(Polizei zu Menschenmenge):\\
	Jetzt machen Sie \textbf{auch} Platz!, Bilden Sie \textbf{auch} eine Gasse!
\end{exe}	
Meiner Meinung nach sind die \textit{auch}-Direktive in genau den Kontexten zulässig, in denen \textit{auch}-E- bzw. -w-Fragen bzw. -Assertionen auftreten können. Die beteiligte Relation ist stets die gleiche. Sofern in diesen Kontexten Direktive überhaupt geäußert werden können, sollten \textit{auch}-Direktive möglich sein (vgl. (\ref{1135}) bis (\ref{1138})).
								
\begin{exe}
	\ex\label{1135} 
	A: Der Nikolaus war nett zu uns. (q)\\
	B: Ihr wart \textbf{auch} artig dieses Jahr. (p)\\
	$[$p $>$ q \glq Wenn ihr artig wart, ist der Nikolaus nett zu euch!\grq {}$]$
\end{exe}	

\begin{exe}
	\ex\label{1136} 
		\begin{xlist}
			\ex\label{1136a} Seid \textbf{auch} artig im Jahr! (!p) (Sonst gibt es Probleme mit dem Nikolaus) (Kontext 2)
			\ex\label{1136b} A: Wir wollen, dass der Nikolaus nett zu uns ist. (q)\\
							 B: Dann seid das Jahr über \textbf{auch} artig! (!p)\\
		$[$q $>$ !p \glq Wenn ihr wollt, dass der Nikolaus nett ist, müsst ihr das Jahr über artig sein.\grq$]$					 
		\end{xlist}
\end{exe}	

\begin{exe}
	\ex\label{1137} 
	A: Der Nikolaus war gar nicht nett zu uns.\\
	B: Warum wart ihr \textbf{auch} nicht artig dieses Jahr?\\
	$[$p $>$ q \glq Wenn ihr artig wart, ist der Nikolaus nett zu euch.\grq\\
	$\neg$q $>$ $\neg$p ($\neg$p ist präsupponiert in der \textit{Warum}-Frage)$]$
\end{exe}

\begin{exe}
	\ex\label{1138} 
	Nikolaus: Wart ihr \textbf{auch} artig?\\
	(mit Antwortpräferenz zu p, weil sie wollen, dass der Nikolaus nett zu ihnen ist)\\
	$[$p $>$ q \glq Wenn ihr artig wart, ist der Nikolaus nett zu euch.\grq$]$
\end{exe}
In den anderen beiden Verwendungskontexten von \textit{auch}-Direktiven spielt die Annahme zu den Interaktionspartnern aus der Literatur keine Rolle.

Die Analyse der Beispiele und Belege zeigt, dass \textit{auch}-Äußerungen immer einen situativen oder dialogischen Kontext benötigen. Wie \textit{halt}- und \textit{eben}-Äuße\-rungen sind sie im engeren Sinne reaktiv. Diese Erkenntnis, dass q kontextuell verfügbar sein muss, passt zum Eindruck der \textit{auch}-Direktive aus \citet[60]{Dittmann1980}, dass sie sich \glqq auf die Situation selbst und einen Zeitraum unmittelbar nach dem Sprechzeitraum\grqq{} beziehen. Im Gegensatz zu den assertiven Fällen ist der monologische Gebrauch (in dem q nur vom Sprecher vertreten wird) auszuschließen, weil der Direktiv immer an den Adressaten gerichtet ist. Monologische Fälle gibt es, wenn man auch Vorhaben des Sprechers miteinbezieht. Hierbei handelt es sich um Fälle, in denen der Sprecher eine Handlungsabsicht bekundet und sich (bzw. einer größeren Gruppe) somit selbst eine zur Realisierung ausstehende Proposition auf die TDL legt. Beispiele sind hier für alle drei Kontexte zu konstruieren (vgl. (\ref{1139}) bis (\ref{1141})).

\begin{exe}
	\ex\label{1139} 
	Ich besuche morgen Oma. Ich werde \textbf{auch} artig sein.
\end{exe}

\begin{exe}
	\ex\label{1140} 
		\begin{xlist}
			\ex\label{1140a} 
				A: Du hast gesagt, unter diesen Umständen isst du es.\\
				B: Ja. Ich werde es \textbf{auch} essen.
			\ex\label{1140b} 
				A: Du hast versprochen, mich zu heiraten.\\
				B: Ich werde es \textbf{auch} machen.
		\end{xlist}
\end{exe}	

\begin{exe}
	\ex\label{1141} 
	A: Ihr seid Spitzenspieler.\\
	B: Wir werden es \textbf{auch} zeigen.
\end{exe}	
	
\subsubsection{Gibt es eine assertive Folge-Verwendung von \textit{auch}?}
Im Zusammenhang mit derartigen Verwendungen von \textit{auch} tritt noch ein weiterer Aspekt zu Tage. Ich habe eingangs schon geschrieben, dass in Darstellungen zu \textit{auch} meist nur Assertionen untersucht werden. Es gibt Autoren, die Direktive nicht betrachten, weil unklar sei, ob es sich beim Vorkommen von \textit{auch} in diesem Satzkontext überhaupt um die MP-Verwendung handelt (\citealt[222]{Karagjosova2004}). Ich denke, dass dieser Punkt beim Auftreten von \textit{auch} in Direktiven tatsächlich eine Rolle spielt.

Eindeutig als MP einstufen kann man \textit{auch} in Kontext 2. Bei den Beispielen handelt es sich um die klassischen Beispiele in der Literatur zum Thema. Zudem gibt es Kontext 1, der diejenigen Fälle auffängt, zu denen es in der Literatur heißt, dass der Direktiv wiederholt wird. Auf der Basis der Betrachtung von Belegen habe ich zudem Kontext 3 hinzugenommen.

Ich bin der Meinung, dass man mit der Reihung \textit{Kontext 2 – 1 – 3} die wörtliche Bedeutung eines Adverbs \textit{auch} immer deutlicher spürt, so dass die Verwendungen von \textit{auch} sich in Kontext 1 und 3 in die Richtung des Adverbs bewegen. Der Beitrag lässt sich zwar nicht durch \textit{ebenfalls} ersetzen, diese Bedeutung ist aber zunehmend spürbar. Da die Direktive aber auch in diesen Kontexten die Folge anzeigen, fügen sie sich gut in meine Ableitung.

Für die meisten MPn gilt, dass die Verbindung zu den Vormodalpartikellexemen \is{Vormodalpartikellexeme} wahrnehmbar ist. Dies trifft auch auf die Additivlesart von \textit{auch} zu: Bei erfüllter Bedingung tritt ebenfalls die Folge ein (Direktive) bzw. die Folge hat Gültigkeit und zusätz\-lich greift die Bedingung, die sie begründet (Assertionen). Dieses additive Moment scheint mir innerhalb der Direktiv-Verwendungen unterschiedlich stark durchzudringen.

In Beispielen der Art in (\ref{1142}) ist der additive Beitrag am wenigsten wahrnehmbar: Er liegt vor in dem Sinne, dass der Besuch bei der Tante mit dem Artigsein einhergeht. Die Tante kann nicht nur besucht werden, man muss zusätzlich/ebenfalls artig sein.

\begin{exe}
	\ex\label{1142} 
	Trixis Vater freute sich und wünschte ihr eine gute Reise. \glqq Grüß' mir Tante Ottilie und \textbf{sei \underline{auch} artig}! Ich rufe heute Abend 		an.\grqq{}
\end{exe}
In Beispiel (\ref{1143}), das für Kontext 1 steht, ist der Wille des Kindes erfüllt, weshalb der Sprecher davon ausgeht, dass es der Handlung jetzt nachkommen wird. Das \textit{auch} im Direktiv drückt aus, dass man nicht nur \underline{sagen} kann, dass man etwas tut (hier die Nudeln ohne Soße zu essen), man muss es zusätzlich/ebenfalls tun.

\begin{exe}
	\ex\label{1143} 
	\scriptsize
 	V zu K: Also \textit{doch} mit Soße! ((laut))\\
	M zu K: Mm, eben hast gesagt \glqq ohne Soße\grqq{}, \textbf{\#$\uparrow$jetzt iss \underline{eben} ohne Soße}. ((K fängt wieder an zu quengeln))(.) 		(Ein paar Wörter unverständlich) zurück. ((auch laut))\\
	((K zeigt mit Finger auf die Schüssel Nudeln))	
\end{exe}
In (\ref{1144}) wird ausgesagt, dass man nicht nur bester Spieler sein kann, sondern dass damit ebenfalls einhergehen muss, dass man es sehen kann. Deshalb werden sie aufgefordert, es zu zeigen.

\begin{exe}
	\ex\label{1144} 
	\scriptsize
	Gegen Winden muss jetzt ein Sieg her und diese Ausnahmekicker sind jetzt um so mehr gefordert, denn dass sie mit zu den besten im Land zählen, bleibt 		unbestritten. \textbf{Zeigt es \underline{auch}}! 
\end{exe}
Für meine Begriffe nimmt der erkennbare additive Beitrag von (\ref{1142}) über (\ref{1143}) zu (\ref{1144}) zu.

Mir ist keine Arbeit bekannt, in der für Assertionen angenommen worden ist, dass sie auch die Folgelesart haben können. Dies ist eigentlich merkwürdig, da diese bei den ansonsten ähnlichen MPn \textit{halt} und \textit{eben} möglich ist. Zählt man die Kontexte 1 und 3 zu MP-Verwendungen, wäre zu überlegen, ob Beispiele der folgenden Art nicht auch MP-Verwendungen sind, die eben nicht kausal sind. 

Am nächsten kämen einer MP \textit{auch} in assertiver Folge Äußerungen mit performativ gebrauchten \is{performatives Modalverb} Modalverben, wie z.B. in (\ref{1145}).

\begin{exe}
	\ex\label{1145} 
	\scriptsize
 	Sie machen sie dem Volk zugänglich und gewöhnen es daran. Ohne Gemeindeeinrichtungen kann sich ein Volk eine freie Regierung geben, aber den Geist der 		Freiheit besitzt es nicht. Für uns geht es darum, daß man auch aus der Armut eine Antwort sucht im Geist der Freiheit. Das kann man nur, wenn man eine 		politische Handlungsmöglichkeit findet. (Jan Ehlers SPD: \textbf{Das heißt aber nicht nur reden, Sie müssen \underline{auch} handeln!} Sie müssen den 		Tocqueville auch richtig verstehen!) - Darauf gehe ich gern gleich ein, daß das auch handeln heißt. 
	\newline
	\hbox{}\hfill\hbox{(PHH/W16.00001 Protokoll der Sitzung des Parlaments Hamburgische Bürgerschaft am}
	\newline
	\hbox{}\hfill\hbox{ 08.10.1997. 1. Sitzung der 16. Wahlperiode 1997-2001. Plenarprotokoll, Hamburg, 1997)}
\end{exe}
Die assertive MP-Äußerung entspricht hier Kontext 1: \textit{Dann handeln Sie \textbf{auch} so!}.

Als nächstes wären auf einer \glq Skala\grq {} von MP-Adverb-Qualität unpersönliche Strukturen mit Modalverben einzuordnen (vgl. (\ref{1146})).

\begin{exe}
	\ex\label{1146} 
	\scriptsize
 	\glqq Genauso wenig wie sich Märkte fair von alleine regulieren, trifft Deutschland von allein die richtige Entscheidung. Nichtwählen würde deshalb nie 	für mich infrage kommen – wer eine Stimme hat, \textbf{sollte sie \underline{auch} nutzen!} In diesem Jahr setze ich mein Kreuz bei der SPD und hoffe 		auf eine rot-grüne Koalition. $[$...$]$\grqq{}	
	\hfill\hbox{(HMP13/SEP.01762 Hamburger Morgenpost, 20.09.2013)}
\end{exe}
Würde eine Person direkt angesprochen, wäre ein \textit{auch}-Direktiv denkbar:

\begin{exe}
	\ex\label{1147} 
	Du hast eine Stimme? Nutze sie \textbf{auch}!
\end{exe}
Und noch tiefer auf der Skala stufe ich Fälle der Art in (\ref{1148a}) mit unpersönlichen Subjekten und indikativischen Verben, die hier m.E. performativ verwendet werden, ein.

\begin{exe}
	\ex\label{1148a} 
	\scriptsize
	\glqq Was man angefangen hat, \textbf{macht man \underline{auch} fertig}!\grqq{}, und so singt er von Abbrechern in der Schule, beim Klavierunterricht 		und in der Ausbildung. 
	\newline
	\hbox{}\hfill\hbox{(RHZ09/NOV.01056 Rhein-Zeitung, 02.11.2009)}
\end{exe}

\begin{exe}
	\ex\label{1148} 
	Du hast es angefangen. Jetzt mach es \textbf{auch} fertig!
\end{exe}
Die Beispiele in (\ref{1146}) und (\ref{1148a}) entsprächen meinem dritten Kontext. 

Diese Verwendungen kontrastieren mit anderen Fällen, in denen \textit{auch} in Folgen auftritt, in denen für meine Begriffe eindeutig das Adverb (mit weitem Skopus) vorliegt (vgl. z.B. (\ref{1149}) bis (\ref{1151})).

\begin{exe}
	\scriptsize
	\ex\label{1149} 	
	\scriptsize
	{Der Lohn war die frühe Führung: Ein Freistoß von Andac Güleryüz klatschte ans Aluminium – Nico Granatowski setzte nach und traf zum 0:1 (15.). 				\textbf{Und wenn der VfL II erst einmal führte, dann hatte er in der laufenden Saison \underline{auch} noch nicht verloren} – diese Serie hielt bis 		gestern Abend. 
	\newline
	\hbox{}\hfill\hbox{(BRZ13/MAI.05814 Braunschweiger Zeitung, 16.05.2013)}}\\
	$[$Wenn sie geführt haben, galt ebenfalls: Sie haben noch nicht verloren.$]$
\end{exe}

\begin{exe}
	\ex\label{1150} 
	\scriptsize
	{\glqq $[$...$]$ Etwas schade ist, dass wir dieses Jahr keine Halbfinalbegegnungen austragen konnten. Aber nächstes Jahr, wenn die Kinder am Samstag 		schulfrei haben, beginnen wir früher \textbf{und sind dann \underline{auch} in der Lage, die im Rahmen der sportlichen Fairness notwendigen Halbfinals 		auszutragen}.\grqq{}
	\newline
	\hbox{}\hfill\hbox{(A97/JUN.09605 St. Galler Tagblatt, 16.06.1997)}}\\
	$[$Wenn wir früher beginnen, gilt ebenfalls: Wir können die Halbfinals austragen.$]$
\end{exe}

\begin{exe}
	\ex\label{1151} 
	\scriptsize
	{\glqq $[$...$]$ Ich mag Feldschlösschen, aber wenn ich im Appenzellerland bin, \textbf{dann will ich \underline{auch} Bier von dort trinken}.\grqq{}   
	\hfill\hbox{(A97/AUG.19304 St. Galler Tagblatt, 20.08.1997)}}\\
	$[$Wenn ich im AL bin, dann gilt ebenfalls: Ich will ein Bier von dort trinken.$]$
\end{exe}
Ich denke, dass die Aspekte der Performativität und Handlungsorientierung eine Rolle spielen, wenn sich die \textit{auch}-Verwendung in assertiven Folgen der MP \textit{auch} in Direktiven, die immer Folgen darstellen, annähert. Im Gegensatz zu (\ref{1145}), (\ref{1146}) und (\ref{1148a}) sind in (\ref{1149}) bis (\ref{1151}) direktive Pendants gar nicht denkbar.

Das Vorkommen von \textit{auch} in Fall 1 und 3 bei den Direktiven stellt in meinen Augen den Übergang zur Adverbverwendung dar. Man hat es hier mit einer gewissen Grauzone zu tun. Wenn die Performativität ausbleibt, liegt auch in Kontext 2 eher das Adverb vor. 

Von (\ref{1152}) zu (\ref{1154}) nimmt die MP-Qualität zu.
	
\begin{exe}
	\ex\label{1152} 
	Wenn Paula Tante Anne besucht, muss sie \textbf{auch} artig sein.
\end{exe}	
\vspace{-0.6cm}	
\begin{exe}
	\ex\label{1153} 
	Du besuchst Tante Anne? Dann musst du \textbf{auch} artig sein!
\end{exe}	
\vspace{-0.6cm}
\begin{exe}
	\ex\label{1154} 
	Du besuchst Tante Anne? Dann sei \textbf{auch} artig!
\end{exe}	
Nachdem nun das Einzelauftreten von \textit{doch} und \textit{auch} in Direktiven untersucht wurde, geht es im folgenden Abschnitt um die Sequenz \textit{doch auch}. Neben der Klärung der Frage, in welchen Arten direktiver Äußerungen die Kombination möglich ist, gilt es vor allem, festzustellen, ob sich die glei\-che Erklärung der präferierten Sequenz \textit{doch auch} anbietet, wie ich sie in Abschnitt~\ref{sec:kombida} für Assertionen vertrete. 

\subsection{Das kombinierte Auftreten von \textit{doch} und \textit{auch}}
\label{sec:kombida}
Die Kombination der beiden Partikeln ist immer dann möglich, wenn die Kontextanforderungen, die ich in Abschnitt~\ref{sec:dadir} formuliert habe, beide erfüllt sind: Zum einen folgt die Handlung, zu der aufgefordert wird, aus der Situation/Handlung/dem Vorgängerbeitrag und dieser Zusammenhang stellt eine allgemein gültige Norm dar (\textit{auch}). Zum anderen ist es fraglich, ob der Adressat p realisieren wird (\textit{doch}). Ich gehe folglich wiederum davon aus, dass sich die Bedeutung der MP-Kombination additiv ergibt und die Partikeln nicht Skopus \is{Skopus} übereinander nehmen. Weiter unten spiele ich die Interpretation, die unter Skopus resultiert, ebenfalls durch.

\subsubsection{Vorkommensweisen}
Einfach belegen lässt sich \textit{doch auch} in den Kontexten 1 und 3, d.h. bei wiederholter Anweisung bzw. erfüllter, salienter Vorbedingung.

In (\ref{1155}) vertreten die Parteien das Ziel der Verbesserung der Bildungschancen. Bisher sind aber keine Taten gefolgt, weshalb fraglich ist, ob es tatsächlich umgesetzt wird. !p befindet sich auf der TDL, die Auflösung von p $\vee$ $\neg$p in Richtung p ist aber noch nicht erfolgt, woraus sich der verstärkte Zweifel ergibt, ob p noch realisiert werden wird. Die beteiligte Relation ist \glq Wenn man das Ziel hat, die Bildungschancen zu verbessern, muss dieses Vorhaben umgesetzt werden.\grq {}.

\begin{exe}
	\ex\label{1155} 
	\scriptsize
	Bei so viel Schimpfen kann man leicht das Ziel aus den Augen verlieren. Bei der jüngsten Bundestagswahl sind nahezu alle Parteien mit dem Ziel 				angetreten, die Bildungschancen zu verbessern. \textbf{Nun setzt das \underline{doch auch} endlich um!} Wer Bildungschancen für unsere Kinder nicht 		einschränken will, muss für alle Kinder, egal wo sie zur Schule gehen, die Beförderungskosten übernehmen – und zwar bis zum Abitur. 		
	\hfill\hbox{(RHZ09/OKT.16499 Rhein-Zeitung, 19.10.2009)}
\end{exe}
In (\ref{1156}) ist die beteiligte Relation \glq Wenn man sagt, dass man etwas tut, macht man es.\grq {} und die Offenheit von p entsteht hier durch die Information, dass die angesprochene Person schon wiederholt gesagt hat, dass sie zum Arzt gehen wird, es aber nicht getan hat, weshalb verstärkt fraglich ist, ob sie es tut. !p ist somit schon in TDL$_{\textrm{B}}$, im cg ist enthalten, dass !p ein Element von TDL$_{\textrm{B}}$ ist, p $\vee$ $\neg$p liegt auf dem Tisch.

\begin{exe}
	\ex\label{1156} 
	\scriptsize
	\glqq Ich dachte, dir geht's wieder besser ...\grqq{}, meinte sie.\\
	\glqq Ja, mir ging's auch wieder besser ... es hat nur gerade wieder angefangen ...\grqq{}\\
	\glqq Du solltest vielleicht mal zum Osteopathen gehen.\grqq{}\\
	\glqq Ich werde zum Osteopathen gehen. Édouard hat mir schon einen empfohlen.\grqq{}\\
	\glqq Naja, aber \textbf{dann geh \underline{doch auch} hin}. Du redest immer nur und tust dann doch nichts.\grqq{}\\	
	\glqq Ich geh ja hin ...\grqq{}
	\hfill\hbox{(Foenkinos 2013, Zum Glück Pauline)}
	\newline
	\hbox{}\hfill\hbox{(https://books.google.de/books?id=tfotAAAAQBAJ\&pg=PT83\&lpg=}
	\newline
	\hbox{}\hfill\hbox{PT83\&dq=\%22Dann+geh+doch+auch\%22\&source=bl\&ots=GgydR-}
	\newline
	\hbox{}\hfill\hbox{XHZZ\&sig=dbVCZbobwlBNDmIKhNFHduYaEAM\&hl=de\&sa=X\&}
	\newline
	\hbox{}\hfill\hbox{ved=0CC4Q6AEwA2oVChMIiJL3yrrPyAIVg1osCh369Ae6\#v=onep}
	\newline
	\hbox{}\hfill\hbox{age\&q=\%22Dann\%20geh\%20doch\%20auch\%22\&f=false)}
\end{exe}
Die Beispiele zeigen, dass die Kombination \textit{doch auch} im \textit{auch}-Kontext 1 gut stehen kann. Der Adressat hat den Auftrag schon akzeptiert, die Realisierung ist aber bisher ausgeblieben. p $\vee$ $\neg$p steht folglich im Raum. \textit{Doch} kann hinzutreten, wodurch die Frage, ob der Adressat p nachkommen wird, zusätzlich hervorgehoben wird.

Es lassen sich ebenfalls Belege nachweisen, die meinem dritten Kontext entspre\-chen. Dies gilt z.B. für (\ref{1158}).

\begin{exe}
	\ex\label{1158} 
	\scriptsize
	Warum sagen Sie den Leuten sogar nach der Bundestagswahl noch, wo doch nun alles gelaufen ist, Dinge, für die Sie nicht eine müde Mark im Haushalt 			werden aufbringen können? Warum machen Sie das? – Sie schaden nicht nur Ihrem Ansehen, sondern Sie schaden auch dem Ansehen der gesamten Politik. Sie 		wollen doch Staatsmann werden. \textbf{Dann verhalten Sie sich \underline{doch auch} entsprechend!} 
	\newline
	\hbox{}\hfill\hbox{(PNI/W14.00013 Protokoll der Sitzung des Parlaments Landtag Nieder-}
	\newline
	\hbox{}\hfill\hbox{sachsen am 29.10.1998. 13. Sitzung der 14. Wahlperiode 1998-2003.}
	\newline
	\hbox{}\hfill\hbox{Plenarprotokoll, Hannover, 1998)}
\end{exe}
Beteiligt ist die Relation: \glq Wenn man Staatsmann sein will, benimmt man sich wie ein Staatsmann.\grq {}. Die Bedingung ist offen thematisiert. Der Angespro\-chene will Staatsmann werden. Momentan zeigt er aber nicht das entsprechende Verhalten, weshalb fraglich ist, ob er sich in Zukunft so benehmen wird. Die Handlungsanweisung ist klar, wenn die obige Relation anerkannt wird.

Ähnlich kann man in (\ref{1159}) als Relation, auf die \textit{auch} Bezug nimmt, ansetzen: \glq Wenn die Verhältnisse auf bestimmte Art beschaffen sind, dann teilt diese mit.\grq {}. Bis zum Sprechzeitpunkt sagen die Angesprochenen dies scheinbar nicht, woraus sich ergibt, dass fraglich ist, ob sie es sagen werden. Auf diese offene Frage bezieht sich \textit{doch}.
	
\begin{exe}
	\ex\label{1159} 
	\scriptsize
	Das ist dann aber keine rentenrechtliche Frage, sondern eine sozialpolitische Frage. Dann geht es letzten Endes darum, dass ihr nur die Grundsicherung 		im Alter von dem jetzigen Betrag von 680 Euro auf 1 050 Euro anheben wollt. \textbf{Das sagt dann \underline{doch auch}! }
	\newline
	\hbox{}\hfill\hbox{(PBT/W17.00198 Protokoll der Sitzung des Parlaments Deutscher Bundestag am}
	\newline
	\hbox{}\hfill\hbox{s18.10.2012. 198. Sitzung der 17. Wahlperiode 2009-. Plenarprotokoll, Berlin, 2012)}
\end{exe}	
Im \textit{auch}-Kontext 2 steht überhaupt nicht zur Diskussion, ob p realisieren werden wird oder nicht. Eine Gegenreaktion liegt nicht bereits vor und mit ihr ist auch nicht zu rechnen. Die Handlungsaufforderung wird mitgeteilt und der Adressat verpflichtet sich dazu, ihr nachzukommen. Er würde sich vermutlich aber von allein nicht anders verhalten. Dies ist der Grund, aus dem \textit{doch} nicht gut zum \textit{auch} hinzutreten kann. In Kontext 1 scheint mir dies stets möglich zu sein, weil trotz Ankündigung, p zu tun, dies noch nicht erfolgt ist. Bei Kontext 3 scheint die Ausgangslage auch so zu sein, dass die Bedingung gerade deshalb thematisiert wird, weil die damit einhergehende (bekannte) Folge bisher nicht realisiert wurde oder fraglich ist, ob sie es wird.

In Kontext 2 kann \textit{doch} nicht ohne Weiteres hinzutreten. Man kann dem bei der Tante abgesetzten Kind nicht (\ref{1160}) mit auf den Weg geben.

\begin{exe}
	\ex\label{1160} 
	\#Sei \textbf{doch}/\textbf{doch auch} artig!
\end{exe}	
Man kann jemandem, der etwas mit Präzision schreiben muss, über die Schulter (\ref{1161}) zukommen lassen.

\begin{exe}
	\ex\label{1161} 
	Schreib \textbf{auch} ordentlich!
\end{exe}	
Wenn kein weiterer Grund zur Annahme besteht, dass der Adressat dies viel\-leicht nicht tut, ist (\ref{1162}) allerdings keine denkbare Äußerung in diesem Kontext.

\begin{exe}
	\ex\label{1162} 
	\#Schreib \textbf{doch}/\textbf{doch auch} ordentlich!
\end{exe}
Damit \textit{doch} hinzutreten kann, muss die Fraglichkeit der Realisierung von p in den Kontext gelangen.

Wie schon in Abschnitt~\ref{sec:doch} erwähnt, kann \textit{doch} nicht rein bestätigend verwendet werden. Assertiert B p, kann A zum Zwecke der Einigung auch nicht \textit{doch(p)} äußern. Bei den Direktiven bedeutet dies, dass nicht klar sein darf, dass von der Realisierung von !p sowieso auszugehen ist. Wenngleich ich gegen einen stets vorliegenden Widerspruch argumentiere, darf die Entscheidung zugunsten der Proposition in der \textit{doch}-Äußerung nicht schon gefallen sein bzw. dürfen nicht alle Anzeichen so stehen, dass mit ihrer Annahme/Realisierung sowieso zu rechnen ist.

Belege, die diese Situation (eine zusätzliche Fraglichkeit im \textit{auch}-Kontext 2) widerspiegeln, sind sehr schwer aufzufinden. (\ref{1163}) eignet sich für die Illustration.

\begin{exe}
	\ex\label{1163} 
	\scriptsize
	\textit{Frei(n)} sind Kinder, \glqq die nicht fremden\grqq{} (sich gegenüber Fremden ablehnend und unfreundlich benehmen)$^{3}$ \textit{sind frei mit enand! 	Bis au(ch) frei und schrei nüd eisig} (\textbf{sei \underline{doch auch} brav} und schrei nicht so) sagt die Mutter zu ihrem laut schreienden Kinde.
	\newline
	\hbox{}\hfill\hbox{(Zeitschrift für vergleichende Sprachforschung auf dem Gebie-}
	\newline
	\hbox{}\hfill\hbox{te der Indogermanischen Sprachen: Ergänzungshefte, Aus-}
	\newline
	\hbox{}\hfill\hbox{gaben 15–19, Vandenhoeck and Ruprecht, 1957)}
\end{exe}
Hier weiß man, dass die Äußerung in einer Situation gemacht wird, in der das Kind schreit. Es ist im unmittelbaren Kontext vor bzw. während der Äußerung nicht brav, so dass fraglich ist, ob es dies in Zukunft ist. Auf diese Frage kann \textit{doch} Bezug nehmen. Zusätzlich ist klar, dass es in dieser Situation brav sein soll (weil Kinder immer brav sein sollen). Die Kontextbedingung für \textit{auch} ist somit ebenfalls gegeben.

Wenngleich Belege kaum zu finden sind, lassen sich derartige Kontexte leicht konstruieren (vgl. (\ref{1164})).

\begin{exe}
	\ex\label{1164} 
	Philipp schreibt eine Hochzeitskarte.\\
	Melanie ist unzufrieden: \textbf{Schreib \underline{doch auch} ordentlich!} Dieses Gekrakel kann keiner lesen.
\end{exe}
Die Sequenz \textit{doch auch} ist immer dann möglich, wenn Zweifel daran besteht, dass die angesprochene Person der Handlung nachkommt, die der Sprecher als klar ausgibt.

So ist auch in (\ref{1165}) aufgrund des aktuellen Verhaltens der Leute fraglich, ob sie den Arzt im nächsten Moment durchlassen werden, da sie es gerade noch nicht tun, obwohl unmissverständlich ist, dass sie einen Arzt durchlassen müssen.

\begin{exe}
	\ex\label{1165} 
	(Ein Arzt wühlt sich durch eine Menschenansammlung.)\\
	Lassen Sie mich \textbf{doch auch} durch. Ich bin Arzt!
\end{exe}
Wenn sich die Fraglichkeit der Realisierung motivieren lässt, kann \textit{doch auch} im \textit{auch}-Kontext 2 verwendet werden. Die Interpretation ist dann stets: Obwohl klar ist, dass p zu tun ist, gibt es Anzeichen dafür, dass es fraglich ist, ob p realisiert wird.
						  
\subsubsection{Gegen ein Skopusverhältnis}
Ich denke, dass die Interpretation der \textit{doch auch}-Direktive in diesen Beispielen zeigt, dass die Partikeln nicht Skopus \is{Skopus} übereinander nehmen: Für die Vorgangskontextzustände ist nachzuweisen, dass die Anforderungen beider Einzelpartikeln vorliegen und auch vorliegen müssen: Es ist fraglich, ob p oder $\neg$p realisiert wird, d.h. diese Disjunktion liegt auf dem Tisch. Zusätzlich gilt, dass es nicht ausreicht, dass dieses Verhältnis allein aus dem Grund eintritt, weil die Realisierung der Handlung, zu der aufgefordert wird, noch nicht erfolgt ist. Dieses Verhältnis (das beispielsweise vorliegt, wenn \textit{auch} in Kontext 2 verwendet wird oder auch wenn der Angesprochene der Realisierung zugestimmt hat) reicht nicht aus, um die Realisierung von p als fraglich auszuzeichnen. Zu\-sätzlich ist die Inferenzrelation \is{Inferenzrelation} q $>$ !p Teil des cgs und q gilt im Diskurs. Der \textit{doch auch}-Direktiv thematisiert deshalb das Thema auf dem Tisch, indem er dazu anhält, eine der zur Diskussion stehenden Handlungen zu erfüllen, und er drückt aus, dass die Aufforderung zu dieser Handlung für den Adressaten ableitbar ist.

Angenommen, man ginge davon aus, dass \textit{doch} über \textit{auch} Skopus nimmt, läge vor einem angemessenen \textit{doch auch}-Direktiv die Kontextsituation in (\ref{1166}) vor.

\begin{exe}
\ex\label{1166} Kontextzustand vor einem \textit{doch auch}-Direktiv $[$doch(auch(p))$]$\\[-0.6em]
\begin{tabular}[t]{|C{6em}|C{12em}|C{6em}|}
\hline
$\textrm{DC}_{\textrm{A}}$ & Tisch &  $\textrm{DC}_{\textrm{B}}$ \tabularnewline
\hline
{} & (cg = $\lbrace$q $>$ !p$\rbrace$ \& q $\in$ $\textrm{DC}_{\textrm{B}}$) $\vee$ $\neg$(cg = $\lbrace$q $>$ !p$\rbrace$ \& q $\in$ $\textrm{DC}_{\textrm{B}}$) & {} \tabularnewline
\cline{1-1}\cline{3-3}
$\textrm{TDL}_{\textrm{A}}$ & {} & $\textrm{TDL}_{\textrm{B}}$  \tabularnewline
\cline{1-1}\cline{3-3}
{} & {} & {}  \tabularnewline
\hline
\multicolumn{3}{|l|}{cg s$_{1}$} \tabularnewline
\hline
\end{tabular}
\end{exe}					                       
Da das p auf dem Tisch das realisierte !p ist, würde das Verhältnis in (\ref{1166}) bedeuten, dass die Frage im Raum steht, ob der Adressat realisiert, dass cg und DC$_{\textrm{B}}$ so aussehen oder ob er dies nicht tut. Auf der Basis der oben durchgespielten Belege scheint mir dies nicht der Kontext zu sein, auf den ein \textit{doch auch}-Direktiv reagiert. Dazu kommt, dass ich diese Interpretation auch für ziemlich abwegig halte, da sie voraussetzt, dass der Adressat auf die Herbeiführung dieser Füllung der Komponenten Einfluss nehmen kann. Übertragen auf die Beispiele von oben bedeutet dies, dass zur Diskussion steht, ob der Adressat realisiert, dass a) im cg enthalten ist, dass wenn der Angesprochene Staatsmann werden will, er sich wie ein Staatsmann zu benehmen hat, und b) dass der Angesprochene vertritt, dass er Staatsmann werden will oder ob er dies nicht realisiert. Führen die Angesprochenen herbei, dass es Teil des cg wird, dass wenn ein Arzt an einen Unfallort kommt, sie normalerweise Platz machen, und dass sie davon ausgehen, dass der Arzt an den Unfallort kommt oder tun sie dies nicht? 							

Ich denke, in den Diskurssituationen, in denen \textit{doch auch}-Direktive getätigt werden, ist eindeutig, dass unklar ist, ob p realisiert wird, und nicht, ob der Adressat den einen oder anderen cg-Zustand bewirkt. Das Enthaltensein von q in DC$_{\textrm{B}}$ liegt zudem jeweils vor und muss nicht mehr herbeigeführt werden.

Unter dem umgekehrten Skopusverhältnis, in dem \textit{auch} Skopus über \textit{doch} nimmt, ergibt sich der Kontextzustand in (\ref{1167}).

\begin{exe}
\ex\label{1167} Kontextzustand vor einem \textit{doch auch}-Direktiv $[$auch(doch(p))$]$\\[-0.6em]
\begin{tabular}[t]{|C{6em}|C{12em}|C{6em}|}
\hline
$\textrm{DC}_{\textrm{A}}$ & Tisch &  $\textrm{DC}_{\textrm{B}}$ \tabularnewline
\hline
{} & {} & q \tabularnewline
\cline{1-1}\cline{3-3}
$\textrm{TDL}_{\textrm{A}}$ & {} & $\textrm{TDL}_{\textrm{B}}$  \tabularnewline
\cline{1-1}\cline{3-3}
{} & {} & {}  \tabularnewline
\hline
\multicolumn{3}{|l|}{cg s$_{1}$ = $\lbrace$q $>$ !((p $\vee$ $\neg$p) $\in$ T)$\rbrace$} \tabularnewline
\hline
\end{tabular}
\end{exe}
Im cg wäre folglich enthalten, dass normalerweise aus der mindestens von B vertretenen Bedingung q folgt, dass der Adressat bewirken soll, dass p $\vee$ $\neg$p auf dem Tisch liegt. Aus q folgt somit, dass vom Adressaten realisiert werden soll, dass offen ist, ob er p oder $\neg$p realisiert.

Ich halte diese Interpretation für recht unangebracht. In Kontext 1 führt diese Interpretation generell zu einer redundanten Situation: Da q entspricht, dass !p ein Element der TDL von B ist, steht p $\vee$ $\neg$p sowieso zur Diskussion. Die aus q folgende Aufforderung, p zur Diskussion zu machen, ist völlig überflüssig. Auch in Kontext 2 und 3 scheint diese Auslegung wenig überzeugend: Im cg wären z.B. die Relationen enthalten: Wenn der Adressat Staatsmann werden will, soll er realisieren, dass zur Diskussion steht, ob er sich so benehmen wird. Wenn der Arzt an einer Unfallstelle ankommt, soll der Hörer dafür sorgen, dass zur Diskussion steht, ob er Platz machen soll oder nicht.

Ich gehe somit erneut davon aus, dass \textit{doch} und \textit{auch} bei ihrem kombinierten Auftreten gleichen Skopus nehmen. Beide Partikeln beziehen sich nacheinander auf die gleiche Proposition, so dass sich die Bedeutung der Sequenz additiv ergibt. \textit{Doch} verweist auf die saliente Frage, ob der Adressat p realisieren wird, und die Äußerung beeinflusst diese zugunsten von p, indem der Sprecher ihm aufträgt, p zu bewirken. \textit{Auch} drückt zusätzlich aus, dass die Handlungsaufforderung aus der Situation/Vorgängeräußerung (q gilt mindestens für den späteren Adressaten des Direktivs) ableitbar ist, und deshalb klar sein sollte. 

\subsubsection{Die Erklärung der (un)markierten Abfolge}
Da ich als reine Partikelbedeutung in den Direktiven die gleiche Bedeutung ansetze wie in den Assertionen und sich die Unterschiede dadurch ergeben, dass man es mit einer anderen Äußerungsart zu tun hat, möchte ich auch eine sehr ähnliche Erklärung für die präferierte Reihung \textit{doch auch} anführen, die in den Direktiven gleichermaßen wie in den Assertionen gilt. Im Falle der Assertionen habe ich gesagt, dass das \textit{doch} dem \textit{auch} vorangeht, weil gespiegelt wird, dass es ein übergeordnetes Diskursziel ist, das zur Debatte stehende Thema zu adressieren. Diesem Bedeutungsbeitrag ist eine qualitative Bewertung, wie hier, dass es sich um die Begründung der Vorgängeräußerung handelt, nachgeordnet. Vor dem Hintergrund der Überlegung, dass die Anreicherung des cg das Hauptziel von Kommunikation ist, halte ich es für natürlicher, auszudrücken \glq Die vertretene Proposition adressiert das aktuelle Thema und stellt darüber hinaus eine Begründung für einen anderen Sachverhalt dar.\grq {} als diese beiden diskursiven Beiträge in die entgegengesetzte Reihung zu bringen: \glq Was mit dieser Äußerung mitgeteilt wird, ist eine Begründung für den Vorgängerbeitrag und es adressiert ebenfalls die aktuelle Diskursfrage.\grq {}

Aus dem einfachen Grund, dass Direktive auf die Realisierung von Sachverhalten abheben (und nicht auf (geteilte) Annahmen), die von der Erfüllung durch den Adressaten abhängig ist, ist eine direkte Übertragung meiner Auslegung der Situation in Assertionen nicht möglich. 

Der Sprecher des Direktivs leistet sicherlich nicht auf die gleiche direkte Art wie in Assertionen seinen Beitrag, das Thema zu lösen. Dies kann in Reaktion auf Direktive gerade nur durch den Angesprochenen erfolgen. Im Rahmen der Möglichkeiten des Sprechers, p realisiert zu sehen und den Kontext in diesem Sinne hinsichtlich p aufzulösen, lässt sich aber durchaus für seinen Beitrag zur Entscheidung des Themas argumentieren. 

Mit \textit{doch} adressiert er das Thema/die im Raum stehende Frage, m.a.W. das, was den Diskurs aktuell bewegt und für dessen Lösung der Sprecher sich einsetzt. Mit \textit{auch} nimmt der Sprecher, wie in Assertionen, eine Wertung vor, indem er sagt, dass diese Aufforderung ableitbar und in diesem Sinne auch erwartbar ist. Bei beiden Einstufungen, die \textit{auch} vornehmen kann, handelt es sich um qualitative Bewertungen. Da \textit{auch} in Direktiven die Einspruchsmöglichkeiten beschränkt, beeinflusst es die Realisierung von p in größerem Ausmaß zugunsten von p als \textit{doch}. Die Akzeptanz von !p, die durch \textit{auch} evoziert wird, bringt p seiner Rea\-lisierung näher als der Verweis von \textit{doch} auf das offene Thema, dem in dieser Hinsicht kein Einfluss zugeschrieben werden kann. Die Verhältnisse weichen aber ab von denen, die im Falle der Sequenz \textit{ja doch} vorliegen. Diese Abfolge habe ich in Kapitel~\ref{chapter:jud}, Abschnitt~\ref{sec:markiert} auf die Art erklärt, dass \textit{ja} präferiert vorangeht, weil es (im Gegensatz zu \textit{doch}) unmittelbar cg herstellen kann bzw. auf cg-Zugehörigkeit verweisen kann. Dieser Beitrag entspricht genau dem, was man sich von Assertionen erwünscht. \textit{Auch} kann in den Direktiven aber nicht analog bewirken, dass p wirklich realisiert wird, so dass p cg werden würde.

Ich halte die Abfolge der Beiträge, erst zu sagen, dass die Aufforderung ableitbar ist, und anschließend auszudrücken, dass sie sich auf das im Raum stehende Thema bezieht, für unnatürlich und weniger der Vorstellung von Diskursabläufen entsprechend, als zu vermitteln, dass mit dem Beitrag das aktuelle Thema adressiert wird, und anschließend der Handlungsaufforderung die weitere Qualität zuzu\-schreiben, dass sie ableitbar, klar bzw. erwartbar ist. Wie sich p $\vee$ $\neg$p auf dem Tisch auflöst, ist dann von der Handlung des Hörers abhängig. Die Beeinflussung der Realisierung betreffend ist die Überzeugung des Adressaten von der vom Sprecher gewünschten Handlung, die die Frage auflöst, das, was der Sprecher tun kann. 

\subsubsection{Ausblick auf weitere Satzmodi/Äußerungstypen}
Meine Modellierung des Bedeutungsbeitrags der beiden Einzelpartikeln und der MP-Kombination in Assertionen \is{Assertion} und Direktiven \is{Direktiv} ist im Prinzip die gleiche. Unterschiede stellen sich aufgrund der verschiedenen Äußerungstypen ein. Abschnitt~\ref{sec:distributionda} hat gezeigt, dass die Kombinationsmöglichkeiten von \textit{doch} und \textit{auch} (im Vergleich zu anderen Partikelkombinationen) recht weit sind. Ein weiterer Satzkontext, in dem das kombinierte Vorkommen ebenfalls möglich ist, ist der \is{Exklamativsatz} Exklamativsatz. \textit{Dass}- und w-Exklamativsätze \is{dass-Exklamativsatz} \is{w-Exklamativsatz} erlauben die Kombination (und entsprechend das isolierte Auftreten) in jedem Fall. Für meine Fragestellung wäre es zweifellos interessant, zu schauen, inwiefern die angenommene Modellierung auch auf diese Äußerungstypen zutrifft. Diese Untersuchung kann ich an dieser Stelle nur als Desiderat ausmachen, weil zu viele Unbekannte (s.u.) vorliegen, um dieser Frage nachzugehen. Einige Voraussetzungen müssten zunächst geschaffen werden. 

Ich habe schon an verschiedenen Stellen darauf hingewiesen, dass Assertionen und das Auftreten von MPn in Assertionen am besten bearbeitet worden sind. Wie die Darstellung gezeigt hat, stellen sich für die Behandlung von Direktiven neue Fragen. Dies setzt sich bei den Exklamativsätzen weiter fort. Beispiels\-weise ist zu klären, wie sich Exklamativsätze im Diskursmodell erfassen lassen. Einen vielversprechenden Beitrag, mit dem man in diese Frage einsteigen könn\-te, leistet hier \citet{Chernilovskaya2014}, die w-Exklamative (mit einem Ausblick auf andere Exklamativsätze $[$\citeyear[131]{Chernilovskaya2014} $]$) in das Modell aus \citet{Farkas2010} integriert. Andere Autoren (\citealt{CastroviejoMiro2008}) vertreten, dass ein einheitliches Kontextwechselpotential für die verschiedenen Exklamativsätze gar nicht formuliert werden kann. Für die deutschen Beispiele müsste man dann auch klären, ob sich die V2-Variante von der VL-Variante unterscheidet. Chernilovskaya untersucht englische und niederländische w-Exklamative. Eine Beschreibung der Diskurssemantik der verschiedenen Exklamativsätze im Deutschen steht noch aus. Es gibt zahlreiche Arbeiten zu Exklamativsätzen, auf die man hier bauen kann (z.B. \citealt{Roncador1977}, \citealt{Zaefferer1983}, \citealt{Rosengren1992}, \citealt{Avis2001}), wobei natürlich insbesondere eine Untersuchung der Pragmatik der Sätze vonnöten ist, im Sinne ihrer Diskursfunktion, von Wissensverteilungen oder Annahmenzuweisungen. 

Neben der diskursstrukturellen Erfassung des Äußerungstyps steht auch die Untersuchung des Auftretens von Partikeln in diesem Satzkontext noch aus. Die Arbeit mit Belegen ist hier fast aussichtslos, da in Korpusdaten Exklamativsätze sehr wenig frequent sind. Die Korpusuntersuchung von \citet{Naef1996} weist darüber hinaus nach, dass nur in einem geringen Anteil von w-Exklamativsätzen \is{w-Exklamativsatz} überhaupt MPn vertreten sind (5\%). Belege anzuführen und zu analysieren scheint folg\-lich nicht vielversprechend, um Unterschiede zwischen \textit{auch}- und \textit{doch}-Exkla\-mativsätzen aufzudecken und um zu entscheiden, wie die Kombination interpretiert wird. Erschwerend kommt hinzu, dass der Partikelbeitrag umso unklarer ist, je randständiger die Äußerungstypen sind. \citet[637]{Thurmair2013} schreibt hierzu z.B., dass die Bedeutungsunterschiede zwischen den verschiedenen MPn in w-Exklamativsätzen gering sind. 

Diese kurzen Ausführungen geben Einblick in die Aspekte, die in Isolation zu untersuchen sind, bevor eine Beschäftigung mit den Partikelkombinationen möglich wird. Ich möchte eine solche Untersuchung an dieser Stelle nicht anstellen. (Meine) weitere Forschung muss zeigen, inwiefern sich auch die Verwendungen von \textit{doch}, \textit{auch} sowie ihren Kombinationen in diesem Kontext in meine Modellierung und Analyse der Abfolgen integrieren lassen.

Der Vollständigkeit halber sei auch eine noch randständigere Verwendungsweise der beiden MPn erwähnt: Es gibt auch einen Gebrauch in w-Interrogativsätzen, \is{w-Interrogatuvsatz} unter dem die beiden Partikeln sich kombinieren lassen. (\ref{1168}) illustriert diesen Typ, der in Betrachtungen von \textit{auch} nahezu gar nicht erwähnt wird (anders als w-Interrogativsätze der Art in (\ref{1169}), aus denen \textit{doch} ausgeschlossen ist) (vgl. meine Erklärung zum Ausschluss der Kombination hier in Abschnitt~\ref{sec:distributionda}).

\begin{exe}
	\ex\label{1168} 
	Wie heißt \textbf{doch}/\textbf{auch}/\textbf{doch auch}/\#\textbf{auch doch} Gretes Kaninchen? Ich hab's vergessen.
\end{exe}
\vspace{-0.6cm}	
\begin{exe}
	\ex\label{1169} 
	Warum gehst du *\textbf{doch}/\textbf{auch}/*\textbf{doch auch} im Regen vor die Tür?
\end{exe}	
Bei den relevanten w-Interrogativsätzen wie in (\ref{1168}) handelt es sich um Fragen, deren Antwort der Sprecher einmal gewusst hat und an die er sich aktuell nicht mehr erinnern kann. In den Darstellungen zum Gebrauch von \textit{doch} werden sie in der Regel angeführt.

Auch hier steht die Aufgabe noch aus, den Partikelbeitrag zu bestimmen. Es wäre zwar wünschenswert, wenn sich ein bedeutungsminimalistischer Ansatz \is{Bedeutungsminimalismus/-maximalismus} durch alle Äußerungstypen hindurch verfolgen ließe, der Nachweis müsste aber erst einmal erbracht werden, dass auch hier die Kontextzustände vorliegen, die ich für das Einzelauftreten von \textit{doch} und \textit{auch} angenommen habe. An diese Entscheidung schließt sich die Überlegung an, ob die Erklärung der präferierten Abfolge die gleiche sein kann. 	

Weitere Äußerungstypen betrachte ich folglich nicht. Im nächsten Abschnitt gehe ich abschließend zu meiner Untersuchung von \textit{doch} und \textit{auch} der Frage nach, ob die umgekehrte Abfolge bei dieser Kombination kategorisch auszuschlie\-ßen ist oder ob sich nicht auch hier Fälle nachweisen lassen, die zeigen, dass diese nicht völlig abzulehnen ist. Ich werde für die letztere Ansicht argumentieren.

\section{Die Distribution von \textit{auch doch}}
\label{sec:distributionad}
Die Frequenzen, die ich in Abschnitt~\ref{sec:präferenz} angegeben habe, zeigen, dass selten auch die Sequenz \textit{auch doch} zu finden ist. In den zwei Korpora, für die ich die Anzahl der \textit{doch auch}-Treffer (näherungsweise) bestimmt habe, liegen ohne Zweifel sehr wenige Belege für diese Ordnung vor (vier in DeReKo, zwei in DGD2 vs. ca. 8664 bzw. 60 \textit{doch auch}-Bespiele). Aufgrund der großen Anzahl von \textit{doch auch}-Treffern habe ich für DECOW keine Häufigkeiten gegenüberstellen können. Gerade in dieser Datenmenge findet sich aber eine größere Menge von \textit{auch doch}-Belegen (s.u.).

\subsection{Der Ausschluss der \glq Dubletten\grq {}}
Ich gehe deshalb auch im Falle dieser Kombination davon aus, dass aus den seltenen Treffern in DeReKo und DGD2 nicht auf die Non-Existenz und Ungrammatikalität dieser Reihung zu schließen ist. Bei der Beschäftigung mit MPn hat man stets mit dem Umstand umzugehen, dass es sich bei einer betrachteten \glq MP\grq {} um eine ihrer \glq Dubletten\grq {}  handeln könnte. Vor diesem Hintergrund ist die Untersuchung des kombinierten Auftretens der Partikeln \textit{doch} und \textit{auch} deshalb besonders problematisch, da prinzipiell beide nicht in ihrer MP-Verwendung vorliegen können.

Die konkurrierenden Formen bei \textit{auch} sind das \is{Adverb} Adverb (meist mit engem Skopus, ggf. betont) (vgl. (\ref{1170}), (\ref{1171})) und das Konjunktionaladverb (unbetont, weiter Skopus) (vgl. (\ref{1172})). 

\begin{exe}
	\ex\label{1170} 
	Das Wasser des Sees war \textbf{\textit{auch}} dem abgehärtetesten Schwimmer zu kalt.\\
	$[$= \textit{sogar}, \textit{selbst}, \textit{ebenfalls}$]$
	\hfill\hbox{\citet[92]{Helbig1990}}
\end{exe}

\begin{exe}
	\ex\label{1171} 
	Der andere Lehrer hat \textbf{\textit{AUCH}} recht.\\
	$[$= \textit{ebenfalls}, \textit{gleichfalls}$]$
	\hfill\hbox{\citet[22]{Mueller2014b}}
\end{exe}							
			
\begin{exe}
	\ex\label{1172} 
	Peter macht Weihnachtseinkäufe. Beim Metzger besorgt er die Gans, beim Konditor Marzipan. 
		\begin{xlist}
			\ex\label{1172a} \textbf{\textit{Auch}} kauft er einige neue Tannenbaumkugeln im Baumarkt.
			\ex\label{1172b} Er kauft \textbf{\textit{auch}} einige neue Tannenbaumkugeln im Baumarkt.
			$[$= \textit{außerdem}, \textit{zusätzlich}$]$
		\hfill\hbox{\citet[22]{Mueller2014b}}	
		\end{xlist}		
\end{exe}
\textit{Doch} kann betontes Adverb sein (vgl. (\ref{1173}))).

\begin{exe}
	\ex\label{1173} 
	Peter wollte erst an Silvester zu Hause bleiben und ist dann \textbf{\textit{DOCH}} auf die Party gekommen. $[$= \textit{dennoch}, \textit{trotzdem}$]$
\end{exe}	
\citet[84]{Kwon2005} zieht auch in Betracht, dass \textit{doch} im Mittelfeld als Konjunktio\-naladverb \is{Konjunktionaladverb} vorkommen kann (vgl. (\ref{1174}) und (\ref{1175})).
	
\begin{exe}
	\ex\label{1174} 
	\scriptsize
	Denn die Semester habe ich erst später gemacht. Ich habe zunächst eine Stelle in Bochum an der Stadtverwaltung angetreten. Das war \textbf{doch} sehr 		schwierig, damals unterzukommen überhaupt. 
	\newline
	\hbox{}\hfill\hbox{(PFE/BRD.cf008)}
\end{exe}	
\vspace{-0.5cm}			
\begin{exe}
	\ex\label{1175} 
	\textbf{Doch} war das sehr schwierig... $[$= \textit{aber}, \textit{jedoch}$]$ 
\end{exe}	
Er schreibt allerdings, dass unklar ist, ob \textit{doch} im Mittelfeld als Konjunktionaladverb \is{Konjunktionaladverb} auftreten kann. In (\ref{1174}) hält er den MP-Gebrauch für wahrscheinlicher.

M.E. kann man die Konjunktionaladverbien kaum ausmachen. Mein Hauptau\-genmerk liegt auf dem Schluss der Adverbien.

Man hat bei den \textit{doch auch}-Treffern insbesondere damit zu tun, die Adverbverwendung von \textit{auch} herauszufiltern (vgl. (\ref{1176}) und (\ref{1177})) – was sich bei unbetontem Auftreten und weitem Skopus als schwierig erweist.

\begin{exe}
	\ex\label{1176} 
	\scriptsize
	{so angenehm die Börsengeschäfte für die Banken auch sind, \textbf{so lauern hier \underline{doch auch} mancherlei Gefahren}.
	\hfill\hbox{(H87/BM5.11485 Mannheimer Morgen, 25.04.1987, S. 05; Die fetten Jahre)}}\\
	$[$Adverb/MP \textit{doch}, Adverb \textit{auch}$]$
\end{exe}

\begin{exe}
	\ex\label{1177} 
	\scriptsize
	{Aber, liebe Leute vom ORF, \textbf{da hätte \underline{doch auch} Kärnten einiges zu bieten}: Die blaue Küste von Reifnizza, die Nocky Mountains, die 		Gegend zwischen Saualm und Klein St. Paul, kurz Sao Paolo genannt. 
	\hfill\hbox{(K97/JUN.42849 Kleine Zeitung, 08.06.1997, Ressort: Lokal)}}\\	
	$[$MP \textit{doch}, Adverb \textit{auch}$]$
\end{exe}	
In den \textit{auch doch}-Belegen gilt es andererseits insbesondere, das betonte \textit{doch} zu erkennen. 

Betrachtet man \textit{auch doch}-Treffer in Korpora, hat man in den meisten Fällen den Eindruck, dass nicht beide Bestandteile als MPn vorkommen. Zwei Beispiele finden sich in (\ref{1178}) und (\ref{1179}). 

\begin{exe}
	\ex\label{1178} 
	\scriptsize
	{Als der Empfang vorüber war, hatte er allerdings einen Job, mit dem er selbst nicht gerechnet hatte. Trippe hatte mit seiner ungeheuren 					Überzeugungskraft Lindbergh dazu gebracht, \glqq techni\-scher Berater\grqq{} bei Pan Am zu werden. \textbf{Mit derselben Überredungskunst machte Trippe 		seine Betty schließlich \underline{auch doch} noch zur Ehefrau.}
	\hfill\hbox{(Z04/405.04570 Die Zeit (Online-Ausgabe), 27.05.2004)}}\\	
	$[$Konjunktionaladverb \textit{auch}, Adverb \textit{doch}$]$
\end{exe}

\begin{exe}
	\ex\label{1179} 
	\scriptsize
	{Frage: kann man nicht alle nicht auf den Artikel bezogenen Beiträge raus nehmen?--G.E.M.A. 23:31, 16. Jan. 2010 (CET)
	Nein. (\textbf{Das gäbe schlimmstenfalls dann \underline{auch doch} nur weitere Bewertungs\-scherereien nach dem Motto \glqq Was ist artikelbezogen			\grqq{}}). 
	\newline
	\hbox{}\hfill\hbox{(WDD11/L43.00760: Diskussion:Lectorium Rosicrucianum/Archiv/2010/1. Teilarchiv)}}\\	
	$[$MP \textit{auch}, Adverb \textit{doch}$]$
\end{exe}
An den Kalkulationen in Abschnitt~\ref{sec:präferenz} sieht man, dass auch bei der unmarkierten Abfolge eine große Anzahl von Treffern nicht der MP-Abfolge \textit{doch auch} entspricht. Von 500 Zufallstreffern bleiben 59 übrig. Ich habe oben schon angemerkt, dass die Untersuchung dieser Kombination schwierig ist, weil prinzipiell bei beiden Bestandteilen ein Non-MP-Pendant vorliegen kann. Die obigen Beispiele zeigen darüber hinaus, dass die auszuschließenden Formen subtiler vom MP-Gebrauch abweichen als in anderen Fällen und deshalb nicht leicht auszumachen und von diesem abzugrenzen sind. 	

\subsection{Die zwei \textit{auch doch}-Kontexte}
Trotz der Situation, dass man bei sehr vielen Treffern den Eindruck hat, dass es sich nicht um zwei MP-Vorkommensweisen handelt, finden sich in DECOW2014 40 Fundstellen für \textit{auch doch}, für die meiner Meinung nach gilt, dass man beide Partikeln der Kombination gut als MPn lesen kann. (\ref{1180}) bis (\ref{1188}) zeigt einige Beispiele.

\begin{exe}
	\ex\label{1180} 
	\scriptsize
	Ja, das auch. Aber auch hier setzt die Politik den Rahmen, ich erinnere mich an ein Schreiben des Ministeriums, in dem uns in der üblichen sanften Art 		nahegelegt wurde, die Schüler, denen Gymnasialeignung attestiert wurde, \textbf{\underline{auch doch} gefälligst bis zum (erfolgreich bestandenen) 			Abitur zu führen.}
	\hfill\hbox{(DECOW2014)}	
	\newline
	\hbox{}\hfill\hbox{(http://www.herr-rau.de/wordpress/2012/03/}
	\newline
	\hbox{}\hfill\hbox{wie-geht-kacken-und-das-achteinhalbjaehrige-gymnasium.htm)}
\end{exe}

\begin{exe}
	\ex\label{1181} 
	\scriptsize
	Und zum Thema Hechte in der Nacht:\\
	Der Hecht ist ein Augentier, welches Nachts die Beute erschnüffelt. Sprich nen Köderfisch richt der Kollege trotzdem. Nen Jerk wird er schwerlich 			erschnüffeln können. \textbf{Zumal Hechte denn \underline{auch doch} zu 90\% Tagaktiv bzw. Dämmerungsaktiv sind...}Ich hab in all den Jahren 4-5 Hechte 	nach Einbruch der Dämmerung gefangen aber unendlich viele mehr Tagsüber oder bei Dämmerung...
	\newline
	\hbox{}\hfill\hbox{(DECOW2014)}	
	\newline
	\hbox{}\hfill\hbox{(http://barsch-alarm.de/Forums/viewtopic/t=23812/start=0.html)}
\end{exe}						  
						
\begin{exe}
	\ex\label{1182} 
	\scriptsize
	@Titus : Wenn ich mich richtig entsinne hast Du das hier mal als zusätzliche Funktion gewünscht (siehe Kommentar Nr. 42 weiter oben ) Ich bin aber noch 	nicht dazu gekommen eine solche Funktion einzubauen, \textbf{da es aufgrund der Wahlmöglichkeit die man dazu haben sollte \underline{auch doch} kein 		kleiner Aufwand ist.} 
	\hfill\hbox{(DECOW2014)}	
	\newline
	\hbox{}\hfill\hbox{(http://www.crazytoast.de/plugin-wordpress-blogroll-widget-with-rss-feeds.html)}
\end{exe}	

\begin{exe}
	\ex\label{1183} 
	\scriptsize
	Ich finde ohne Sattel reiten prima, \textbf{weil man \underline{auch doch} viel genauer merkt, was unter einem los ist.} 
	\hfill\hbox{(DECOW2014)}	
	\newline
	\hbox{}\hfill\hbox{(http://www.wege-zum-pferd.de/forum/archive/index.php?t-5461.html)}
\end{exe}	

\begin{exe}
	\ex\label{1184} 
	\scriptsize
	um mich vllt klarer auszudrücken, ich wollte nicht sagen, dass veigar op ist, aber seine verwendete kombo war sehr stark. und ich weiss, es gibt für 		alles einen konter oder zumindest sollte es. aber in normalen spielen sehe ich nicht, was ich für gegner habe und kann mich erst im spiel 					dem\-entsprechend anpassen. dennoch bin ich der meinung, dass es op kombos und chars gibt. ist in einem solchen spiel auch nicht vermeidbar.\\	
	\noindent
	\textbf{Das ist dann \underline{ja auch doch} deine Meinung} ob sie stimmt und ob du es mit deinem spielvermögen beurteilen kannst ist was anderes, 		nicht böse gemeint.	 
	\hfill\hbox{(DECOW2014)}	
	\newline
	\hbox{}\hfill\hbox{(http://www.computerbase.de/forum/archive/index.php/t-697528-p-14.html)}
\end{exe}

\begin{exe}
	\ex\label{1185} 
	\scriptsize
	Ich fragte: \glqq Kann ich denn Schuh mal anprobieren, da ich ihn mir woanders hollen will\grqq{}\\

	Da sagte die Frau \glqq Neeee,so gehts nicht junger Mann\grqq{}\\
	
	Manuel\\
	21.06.2009, 21:11\\
	Ja ... naja \textbf{das is ja \underline{auch doch} ziemlich dreist.} xD			
	\hfill\hbox{(DECOW2014)}	
	\newline
	\hbox{}\hfill\hbox{(http://forum.torwart.de/de/archive/index.php/t-62037-p-4.html)}
\end{exe}

\begin{exe}
	\ex\label{1186} 
	\scriptsize
	Tut mir leid für Dich, dass es schon wieder Probleme gibt! Also Du hast mit Deinem echt die A.R.S.C.H.-Karte gezogen! Da ist meiner ja noch harmlos, 		obwohl ICH mich schon genug ärgere. Hast nicht schon überlegt, ob sie ihn Dir wandeln sollten? \textbf{Es gibt ja \underline{auch doch} einige im 			Board, bei denen alles funktioniert.}
	\hfill\hbox{(DECOW2014)}	
	\newline
	\hbox{}\hfill\hbox{(http://www.der206cc.de/forum/archive/index.php/t-2177.html)}
\end{exe}

\begin{exe}
	\ex\label{1187} 
	\scriptsize
	vielen dank für eure schnellen antworten. das BP – Bipolar, NP – Nonpolar ist hatte ich schon vermutet, aber offensichtlich das falsche daraus 				geschlossen. ;)\\
	\textbf{die beiden verschiedenen bezeichnungen sind aber \underline{auch doch} ein wenig irreführend.} (naja jetzt weis ich ja bescheid)\\
	gruß tim 		
	\hfill\hbox{(DECOW2014)}	
	\newline
	\hbox{}\hfill\hbox{(http://forum.musikding.de/vb/archive/index.php?t-11974.html)}
\end{exe}

\begin{exe}
	\ex\label{1188} 
	\scriptsize
	Find ich blöd dass du gehst, \textbf{weil du ja \underline{auch doch} sehr aktiv warst.}	
	\hfill\hbox{(DECOW2014)}	
	\newline
	\hbox{}\hfill\hbox{(http://www.websitepark.de/forum/archive/index.php/t-4754.html)}
\end{exe}						
Im Vergleich zum Auftreten der unmarkierten Abfolge \textit{doch auch} in diesem Kor\-pus sind dies natürlich sehr wenige Belege. Ich kann den genauen Wert nicht angeben. Es ist aber völlig klar und steht nicht zur Diskussion, dass die \textit{auch doch}-Treffer auch in diesen Daten deutlich unterrepräsentiert sind. Ich möchte betonen, dass meine Argumentation hinsichtlich der umgekehrten Abfolgen nicht derart verläuft, nachweisen zu wollen, dass die markierten Reihungen ähnlich gebräuchlich sind wie die unmarkierten. Es liegt zweifellos ein deutlicher Markiert\-heitsunterschied \is{Markiertheit} vor. Dennoch halte ich die in meiner Argumentation markierten Fälle nicht für ungrammatisch und non-existent. Ich gehe vielmehr davon aus, dass sie nicht gänzlich ausgeschlossen sind und sie mit einer gewissen Systema\-tik auftreten. Insbesondere aus dem letzten Grund möchte ich den Treffern den Status von Performanzfehlern absprechen. Ich bin der Meinung, dass in den erwähnten 40 Fällen, die durch (\ref{1180}) bis (\ref{1188}) illustriert werden, – anders als in den Beispielen in (\ref{1178}) und (\ref{1179}) – plausibel in beiden Fällen die MPn vorliegen können.

In DECOW findet man folglich derartige Daten mit einer gewissen Häufigkeit, die für meine Begriffe über die sporadischen Vorkommen in DGD2 und DeReKo hinausgehen. Der Frequenzunterschied zu der unmarkierten Abfolge bleibt natürlich sehr groß (weil auch in DECOW entsprechend viel mehr \textit{doch auch}-Treffer vorhanden sind). Der Vorteil dieser Daten ist allerdings, dass eine Menge vorliegt, die sich auf Muster untersuchen lässt.

Zwei Muster kristallisieren sich auch im Falle dieser Kombination heraus. Es gibt zwei Kontexte, die die Interpretation als MP für beide Elemente der Sequenz stützen. Hierbei handelt es sich zum einen um kausale Nebensätzen und zum anderen um die Dreierkombination \textit{ja auch doch}. Letztere wird auch im einzigen Hinweis auf die Abfolge \textit{auch doch}, den ich finden konnte (vgl. \citealt[254]{Hentschel1986}), erwähnt. 

Unter den 40 Treffern, denen ich die MP-Verwendung beider Elemente der Kombination zuschreibe, befinden sich 14 durch eine Konjunktion bzw. die Verb\-stellung (V1) ausgezeichnete kausale Nebensätze, 11 Kombinationen mit \textit{ja auch doch} sowie zwei Verbindungen aus diesen beiden Kontexten.

Es ist nicht so, dass das Auftreten dieser Kontexte stets mit der Interpretation von \textit{doch} und \textit{auch} als MP einhergeht (vgl. (\ref{1189}) und (\ref{1190})).

\begin{exe}
	\ex\label{1189} 
	\scriptsize
	\textbf{\textit{Da} mein Triebwagen jetzt \underline{auch doch} besser geworden ist}, als ich dachte, will ich mal auf der Börse nächstes WE schauen, 		ob ich vielleicht nen guten 4000 er günstig bekomme, ohne Achsen udn Kupplungen oder so.			
	\hfill\hbox{(DECOW2014)}	
	\newline
	\hbox{}\hfill\hbox{(http://alte-modellbahnen.xobor.de/t14781f2-Schrott-wird-flott-3.html)}
\end{exe}	

\begin{exe}
	\ex\label{1190} 
	\scriptsize
	Sollen sich doch die Leute die Köppe einrennen, wenn's ihnen Spass macht. Und wie's scheint, \textbf{steckt \textit{ja} \underline{auch doch} sehr viel 	mehr Strategie und Taktik dahinter}, als es auf den ersten Blick aussieht.			
	\newline
	\hbox{}\hfill\hbox{(DECOW2014)}	
	\newline
	\hbox{}\hfill\hbox{(http://www.comicforum.de/archive/index.php/t-89772.html)}
\end{exe}					 
Meiner Meinung nach sind dies aber Kontexte, die die Lesart beider Elemente als MPn stützen. Wenn die Partikel-Lesart vorliegt, scheint dies in den beiden Kontexten zu erfolgen.

Sucht man gezielt nach \textit{auch doch} in diesen beiden Kontexten, findet man mit mäßigem Aufwand weitere Treffer, die ich nicht für auffällig abweichend halte (vgl. (\ref{1191a}) bis (\ref{1194})).
	
\begin{exe}
	\ex\label{1191} 
	\scriptsize
	... früher war mein Blog hauptsächlich ein Naturfotografieblog. Heute ist er hauptsächlich ein People-fotografie-Blog, ich hab von den Fotomotiven her 		einmal ne 180 Grad Wendung gemacht :D Das liegt zunächst einmal daran, dass ich kaum noch Natur fotografiere, \textbf{\textit{weil} ich \textit{ja} 		\underline{auch doch} keine 36 Stunden Tage habe} (leider :D) und daher dann neben den Shootings für so etwas keine Zeit mehr bleibt.				
	\hfill\hbox{(Google-Suche 25.06.2015)}	
	\newline
	\hbox{}\hfill\hbox{(http://www.lichtreflexe-blog.de/2014\_10\_01\_archive.html)}
\end{exe}	
	
\begin{exe}
	\ex\label{1191} 
	\scriptsize
	Ich sehe es schon deutlich vor meinem dritten Auge: Alle einstigen Zonenklubs werden teilnehmen, das Gewinnerteam darf zwei Wochen Urlaub machen. In 		Nordkorea. Bikini, Sandburg, Folterkeller. Aus Chile wird eine Honeckermumie eingeführt, die den FDGB-Pokal überreicht. Die Spieler und Funktionäre 		können sich ein bisschen heldenhaft fühlen. Wie man hört, will die deutsche Wirtschaft Asiens letztem originalem Diktator Beine machen. 					\textbf{\textit{Weil} \underline{auch doch} dort unten alles besser werden soll.} 						
	\hfill\hbox{(Google-Suche 25.06.2015)}	
	\newline
	\hbox{}\hfill\hbox{(http://www.tagesspiegel.de/sport/willmanns-kolumne-dresdner-fans-wollen-de}
	\newline
	\hbox{}\hfill\hbox{n-fdgb-pokal-wieder-einfuehren/7601988-2.html)}	
\end{exe}								
						   
\begin{exe}
	\ex\label{1192} 
	\scriptsize
	Noch mehr aber ward ich allzeit dadurch beruhigt, wenn ich bedachte, wie Du am Kreuze den Vater in Dir für alle Deine Feinde um Vergebung batest; und 		da konnte ich denn den armen Judas trotz seines Selbstmordes nicht ausschließen. \textbf{Dazu war \textit{ja} \underline{auch doch} offenbar an dieser 		seiner letzten Tat nach der Schrift der in ihn fahrende Teufel schuld.} Daher also möchte ich wohl auch diesen Apostel, wenn schon nicht hier, so aber 		doch wenigstens irgendwo nicht im höchsten Grade unglücklich wissen.							
	\hfill\hbox{(Google-Suche 25.06.2015)}	
	\newline
	\hbox{}\hfill\hbox{(http://www.j-lorber.de/jl/gso2/gso2-007.htm)}
	\end{exe}						            
							           
\begin{exe}
	\ex\label{1193} 
	\scriptsize
	Ich denk mal eher, dass die 11 Punkte fuer Multiplayer apps sind ... Is ja auch logisch groesseres Display als iPhone. Mag sein dass es im Moment keine 		apps gibt die es nutzen aber wer weiß was die Zukunft bringt ;)\\
	— Coolix\\

	\textbf{Ich will neben meinen Finger aufm iPhone \textit{ja} \underline{auch doch} was erkennen} ... das ne Simple erklareung… \\
	Auch wenn das touchpad vom MacBook 11 Finger wie das Opas erkennt – es ist nur zum Mauszeiger steuern und nicht um was anzuschauen…\\
	— Gtc-michel89 								
	\hfill\hbox{(Google-Suche 25.06.2015)}	
	\newline
	\hbox{}\hfill\hbox{(http://www.iphone-ticker.de/multitouch-punkte-ipa}
	\newline
	\hbox{}\hfill\hbox{d-unterstutzt-11-iphone-nur-funf-10833/)}
\end{exe}								

\begin{exe}
	\ex\label{1194} 
	\scriptsize
	Inszeniert wurde sie von Kai Grehn, der sich durch eine Vielzahl verschiedener Kunstprojekte im Hörspielgenre einen Namen machen konnte. Auch \glqq Die 	Frau in den Dünen\grqq{} lässt er nicht ohne besondere Note erklingen. Das könnte einigen Hörern vielleicht etwas zu verkünstelt sein, aber es geht 		aufgrund des Inhaltes, \textbf{der \textit{ja} \underline{auch doch} eher auf einer psychologischen Ebene seinen \\ Schwerpunkt hat.}						
	\hfill\hbox{(Google-Suche 25.06.2015)}	
	\newline
	\hbox{}\hfill\hbox{(http://www.hoerspieltipps.net/archiv/diefrauindenduenen.html)}
\end{exe}					            
M.E. findet man die Abfolge in diesen beiden Kontexten zu einfach, um behaupten zu wollen, dass sie ungrammatisch ist und nicht existiert.

\subsection{Erklärung der markierten Abfolge}
Ich möchte weiter unten einen Vorschlag vorstellen, warum sich möglicherweise genau diese Kontexte eignen. Die Überlegungen erfolgen vor dem Hintergrund meiner Annahmen zu den umgekehrten Abfolgen in den Kapiteln~\ref{chapter:jud} und \ref{chapter:hue} und meinem Eindruck, dass MPn in Kombinationen abhängig vom Satztyp unterschiedliches Gewicht haben können. Es ist aber zweifellos so, dass in Verbindung mit der umgekehrten Reihung \textit{auch doch} noch einige empirische Aufgaben ausstehen, die ich zunächst adressieren möchte.

Insbesondere gilt es, festzustellen, ob es sich bei den beiden genannten Kontexten, die sich in den Daten herauskristallisieren, wirklich um genuine \textit{auch doch}-Umgebungen handelt. Es ist klar, dass die Verwendung von \textit{doch auch} prinzi\-piell weiter ist. Diese Reihung tritt auch gut in anderen Kontexten als kausalen Nebensätzen und größeren Kombinationen mit \textit{ja} auf. In Analogie zu meiner Untersuchung zu \textit{doch ja} müsste man aber festzustellen versuchen, ob Sprecher \textit{auch doch} tatsächlich innerhalb dieser Kontexte als besser und außerhalb dieser Umgebungen als schlechter erachten. Ob sich dies empirisch testen lässt und mit welcher Methode, muss (meine) weitere Forschung zeigen.

In Bezug auf die Korpusdaten wäre es wieder interessant, zu wissen, ob diese beiden Kontexte auch für \textit{doch auch} bzw. das Korpus an sich ein häufiger Auftre\-tenskontext sind. Es wäre zu überlegen, zum Zwecke dieser Erkenntnis die Treffer für die beiden Kontexte vor dem Hintergrund eines Erwartungswertes für die beiden Abfolgen gegenüberzustellen. Dieser Erwartungswert wäre allerdings äußerst aufwändig zu bestimmen. Selbst das Teilkorpus DECOW2014AX, mit dem ich zu anderen Fragestellungen dieser Arbeit exhaustive Suchen durchgeführt habe, gibt hier eine viel zu große Treffermenge aus, als dass man die Belege sortieren könnte. Wie ich schon verschiedentlich deutlich gemacht habe, ist dies aber insbesondere bei dieser Kombination absolut notwendig. Ein solches Vorgehen wäre folglich sehr mühsam, (davon abgesehen, dass das Korpus nur einzelne Sätze ausgibt und die Kontexte erst gesucht werden müssen). 

Trotz dieser Fragen, die es zu klären gilt, gehe ich (bis Gegenteiliges nachgewie\-sen ist) davon aus, dass diese beiden Kontexte eine Rolle spielen, wenn Sprecher die Abfolge \textit{auch doch} verwenden. Und ich möchte einige Überlegungen anstellen, warum die Umkehr der Abfolge genau in diesen Kontexten besser möglich zu sein scheint. Schon in Kapitel~\ref{chapter:jud} und \ref{chapter:hue} habe ich angenommen, dass MPn in Kombinationen nicht immer gleich gewichtet sind, in dem Sinne, dass der Satzkontext Einfluss auf das Gewicht einer Partikel nehmen kann. 

Die präferierte Abfolge \textit{doch auch} habe ich derart erklärt, dass diese Reihung das kommunikative Ziel, den cg zu erweitern und (dafür) die Themen auf dem Tisch zu entfernen, direkter abbildet, als die umgekehrte Abfolge. Die Partikel \textit{doch} adressiert das Thema auf dem Tisch, \textit{auch} führt eine Begründung und somit qualitative Bewertung eines anderen Sachverhalts an. Letzterer Diskursbeitrag darf vor dem Hintergrund des Diskursmodells aus \citet{Farkas2010} und dem dort angelegten Ziel/Zweck von Kommunikation (vgl. (\ref{1195})) als dem ersteren nachgeordnet angesehen werden.

\begin{exe}
	\ex\label{1195} Zwei fundamentale Antriebe für Gespräche 
		\begin{xlist}	
			\ex\label{1195a} Erweiterung des cg
			\ex\label{1195b} Herstellung eines stabilen Kontextzustands
			\newline
			\hbox{}\hfill\hbox {\citet[87]{Farkas2010}}
		\end{xlist}
\end{exe}
In Kapitel~\ref{chapter:jud}, Abschnitt~\ref{sec:unmarkiert} habe ich die präferierte Reihung \textit{ja doch} derart abgeleitet, dass diese Abfolge des Partikelbeitrags das Ziel einer Assertion im Diskurs direkter abbildet als die Abfolge \textit{doch ja}, weil \textit{ja} genau das bewirkt, was eine Assertion anstrebt: zum Teil des cg zu werden. Diesem Ziel kommt man mit der Abfolge \textit{ja doch} direkter nach, wenn man die Partikel, die dies zu leisten imstande ist, als erstes zur Applikation bringt und nicht erst darauf verweist, dass auch ein zur Debatte stehendes Thema angesprochen wird (\textit{doch}).  

Tritt nun zu \textit{auch doch} die Partikel \textit{ja} hinzu, könnte man argumentieren, dass die Adressierung des Themas entsprechend weniger relevant wird, da ohnehin sofort cg hergestellt wird. Die Folge ist, dass diese Partikel auch erst spät zur Anwendung gebracht werden kann und deshalb in dieser Dreierkombination am rechten Rand erscheint. Die Konstellation, aufgrund derer \textit{doch} in der Kombination \textit{doch auch} meiner Argumentation nach vorne steht, löst sich in diesem Kontext durch die Hinzunahme von \textit{ja} folglich auf und ermöglicht die späte Applikation von \textit{doch}.\footnote{Diese Erklärung setzt die Annahme voraus, dass sich die MPn in einer Dreierkombination (genauso wie ich es für eine Zweierkombination annehme) alle unter gleichem Skopus auf dieselbe Proposition beziehen.}

Ähnlich lässt sich auch für den kausalen Kontext argumentieren, dass der Aspekt der Adressierung des Themas hier in den Hintergrund rückt. 

Generell gehe ich davon aus, dass wenn Partikeln auftreten, ihr Diskursbeitrag im jeweiligen Satzkontext auch zur Anwendung kommt und beabsichtigt ist. Aus diesem Grund habe ich auch in meinen Ausführungen zu den V1- und \textit{Wo}-VL-Sätzen angenommen, dass ihre kausale Lesart keinen Einfluss auf die Abfolge der Partikeln nimmt. Die Adressierung des Themas liegt dort transparent vor und ist der qualitativen, kausalen Bewertung übergeordnet. Dennoch verwundert es nicht, wenn sich die Umkehr gerade im kausalen Kontext einstellen kann, d.h. dieser Satzkontext anfällig für die andere Reihung ist: Zweck eines Kausalsatzes ist es gerade, eine Begründung zu leisten. Tritt in dieser Umgebung eine Partikel auf, die diese kausale Lesart stützt, und eine andere, die die Adressierung des aktuell diskutierten Themas anzeigt, lässt sich folglich plausibel annehmen, dass gerade dieser Kontext die Hintergrundierung der Themaadressierung erlaubt. Die Konsequenz ist, dass die Partikel, die die Adressierung des aktuellen Diskursthemas kodiert (was für (assertive) Äußerungen prinzipiell hochrelevant ist), in genau diesem Kontext zurücktritt und somit später zur Applikation gebracht werden kann. Da jeder Kausalsatz ein assertiver Kontext ist, ist die Abfolge \textit{doch auch} natürlich möglich und auch geläufiger.

Vor dem Hintergrund meiner Überlegung, dass die umgekehrten Abfolgen in speziellen Kontexten, in denen die zweiten Bestandteile der Kombination weniger dominant sind, möglich werden, bietet sich auch eine Erklärung für den (soweit man dies sagen kann) positiven Einfluss des \textit{ja}-Kontextes und kausaler Nebensätze an.

Da man nicht ausschließen kann, dass sowohl der kausale Kontext als auch die Kombination mit \textit{ja} sowieso häufig auftreten, möchte ich in Betracht ziehen, dass auch allgemeinere (Verarbeitungs-)Prozesse einen Einfluss auf die Verwendung/Akzeptabilität der umgekehrten Abfolgen nehmen. In den DECOW-Daten, für die ich den MP-Gebrauch von \textit{auch} und \textit{doch} ansetze, findet sich auch jeweils ein Beleg mit \textit{wohl} und \textit{halt} (\textit{wohl auch doch}, \textit{halt auch doch}). Informell erfragte Sprecherurteile ergeben, dass Sprecher dazu tendieren, auch diese Kombinationen besser zu bewerten als das isolierte \textit{auch doch}. Da die Kombination \textit{ja auch} äußerst frequent ist, ist nicht auszuschließen, dass \glq bekannte Bestandteile\grq {} einen Einfluss auf die Bewertung nehmen. Verarbeitet man die Sequenz \textit{ja auch} als akzeptabel, stört in diesem Sinne auch das Hinzufügen von \textit{doch} weniger. Die Sequenzen \textit{wohl auch doch} und \textit{halt auch doch} treten in den Daten weniger auf als \textit{ja auch doch}, weil auch schon die Reihung \textit{ja auch} die Abfolgen \textit{wohl auch} und \textit{halt auch} deutlich übertrifft.

In Kapitel~\ref{chapter:jud}, Abschnitt~\ref{sec:ort} habe ich gegen die Kritik aus einem Gutachten argumentiert, dass die Reihung \textit{doch ja} durch eingefrorene Konstruktionen zustande komme, bei denen \textit{X}+\textit{doch} und \textit{ja}+\textit{X} ein festes Muster darstellen. In dem Sinne der dortigen Argumentation lehne ich diese Überlegung in Bezug auf genau diese Beispiele ab. Die aufzufindenden Dreierkombinationen, in denen \textit{auch doch} integriert vorkommt, könnten hier eher als ein solcher Fall eines bekannten Musters fungieren. Weist man nach, dass nicht nur die Dreierkombination mit \textit{ja} am linken Rand das Auftreten von \textit{auch doch} begünstigt, kann die Erklärung nicht mehr am konkreten Beitrag dieser Partikel hängen, sondern verliefe entlang allgemeinerer Verarbeitungsaspekte. Es bietet sich im Rahmen meines generellen Zugangs eine Erklärung an, die Bezug nimmt auf den Diskurseffekt von \textit{ja}. Die andere Erklärung möchte ich beim jetzigen Stand aber nicht ausschließen. In Bezug auf \textit{ja} widersprechen sich die beiden Erklärungen noch nicht einmal. Anderes gilt für \textit{wohl} und \textit{halt}. Beide Partikeln stellen nicht cg her, so dass die Hintergrundierung der Themaadressierung nicht angenommen werden kann. Dazu folgen sie \textit{doch} im unmarkierten Fall. 

Eine ähnliche Erklärung über die Verarbeitung bekannterTeilstrukturen lässt sich auch für das Auftreten von \textit{auch doch} in den kausalen Nebensätzen andenken: Stellte sich heraus, dass \textit{auch} in kausalen Nebensätzen frequent wäre (meine Verteilungsangaben zu \textit{da}-, \textit{denn}- und \textit{zumal}-Sätzen legen dies in Abschnitt~\ref{sec:korp} für \textit{zumal}-Sätze nahe), könnte auch das Vorkommen von \textit{auch} als Muster gelten, dessen Bekanntheit/Erwartetheit ausreicht, um ein hinzutretendes \textit{doch}, das an dieser Stelle eigentlich nicht gewünscht ist, zu \glq tolerieren\grq {}.\\

\noindent
Auch dieser Teil der Arbeit schließt folglich mit der Annahme, dass auch die umgekehrte Abfolge \textit{auch doch} nicht gänzlich ausgeschlossen werden sollte. Diese Reihung ist in den Daten sicherlich unterrepräsentiert und wird als weniger akzeptabel bewertet. Liegt eine gewisse Datenmenge vor (wie es DECOW ermöglicht), können m.E. aber Muster aufgedeckt werden, die zeigen, dass die Umkehr nicht völlig unsystematisch erfolgt und deshalb als Performanzfehler abgetan werden muss. Sicherlich sind aber weitere (vor allem empirische) Untersuchungen nötig, um diesen Aspekt weiter zu verfolgen. Die Daten, die ich anführe, geben für meine Begriffe aber zunächst ein Bild der realen Partikelverwendung.














 









