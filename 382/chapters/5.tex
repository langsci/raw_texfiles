\chapter{Modal particles in Italian: Introducing the case studies}\label{sec:5}
\hypertarget{Toc124860639}{}\section{Modal particles in Italian: Preliminary concepts}
\hypertarget{Toc124860640}{}
Before moving to the empirical part of the present work, this chapter introduces the last few theoretical notions necessary to properly frame the case studies and the subsequent discussion as well as the methodological guidelines behind the analysis. In this section, I will first present an overview of previous studies on pragmatic markers and modal particles in Italian. Subsequently, I will outline the sociolinguistic processes that led to the formation/development of regional varieties of Italian and discuss their relevance for the description of modal particles.

\largerpage[3]
\subsection{Modal particles in Italian: Previous studies}
\hypertarget{Toc124860641}{}
The standard references on discourse-pragmatic markers in Italian are the works of \citet{Bazzanella1995,Bazzanella2006}, the most comprehensive treatment to date of these topics. In these works, the labels \textit{segnali discorsivi} ‘discourse signals’ and \textit{discourse markers} are used, respectively. Under this heading, \citet[232--249]{Bazzanella1995} groups together items with a vast range of functions, including turn-taking devices (\textit{allora} ‘then’, \textit{dunque} ‘hence’, \textit{ecco} ‘that is’), fillers and reformulation markers (\textit{ehm} ‘uh’, \textit{diciamo} ‘let’s say’, \textit{cioè} ‘that is’), attention getters (\textit{guarda} ‘look’), and agreement devices (\textit{certo} ‘of course’, \textit{esatto} ‘exactly’). A functional taxonomy is also proposed, which encompasses three macro-functions: interactional, metatextual, and cognitive. These macro-functions are subdivided into more specific functions, providing a list which “is not meant to be exhaustive but is merely intended to outline the wide range of possibilities DMs can exploit in Italian and is proposed for comparison with other languages” (\citealt{Bazzanella2006}: 456–457). Therefore, a general categorization is provided which focuses on a fine-grained description of the functional spectrum of discourse markers: this taxonomy still represents a good introduction to these issues as regards Italian. The model proposed more recently by \citet{Ghezzi2014} also makes close reference to this taxonomy.\footnote{\citet{Ghezzi2014} distinguishes three macrofunctions (textual cohesion and coherence, social cohesion, and personal stance) that are substantially comparable to those identified by \citet{Bazzanella2006}: metatextual, interactional, and cognitive.}

In the general analyses of Italian discourse-pragmatic items mentioned above, little, if any, space, has been devoted to modal-particle-like elements. This is not surprising, since for Italian “the existence of such a group of words as German MPs has never been assumed. It has only been sporadically observed that some Italian lexemes (such as \textit{mai}, \textit{poi} and so on) present peculiar characteristics (phonetic, semantic, syntactic, etc.) distinguishing them from the traditional class of adverbs” \citep[92]{Coniglio2008}. However, it is worth citing those early sporadic observations – mainly by German scholars working on Romance languages – since they were the first to raise the question of whether modalization forms (that is, linguistic items expressing \textit{Abtönung}) can also be found in Italian: \citet{Stammerjohann1980}, \citet{Held1985,Held1988}, \citet{Burkhardt1985}, \citet{Radtke1985} and \citet{Masi1996}. These works mostly adopt a contrastive perspective, looking for Italian functional equivalents of German modal particles among Italian adverbs, connectives, syntactic constructions, and specific intonation contours. They do not refer to any particular theoretical frameworks and the results are thus hardly comparable with each other. Although they could appear outdated (and in many respects they are), they still constitute a source of interesting examples and early reflections on discourse-pragmatic functions in spoken Italian. Some of these works are also briefly cited by Andorno (\citeyear{Andorno2000}, \citeyear[180–181]{Andorno2003}), who first noticed the modal-particle-like uses of some Italian focus adverbs.\footnote{Since the topic was (almost) completely unexplored at that time, \citet[54]{Andorno2000} – who focuses on the acquisition of focus particles in Italian as a second language – does not elaborate on this point: “In mancanza di studi sistematici sulla questione, per non sommare le difficoltà di definizione della categoria a quelle relative all’analisi di una varietà non nativa, in questo lavoro non approfondiremo il tema del possibile uso modale degli avverbi focalizzanti, limitandoci a segnalare l’emergere – o il non emergere – di un tale uso” [In the absence of systematic studies on the issue, in this work we will not deal further with the topic of the possible modal use of focus particles, limiting ourselves to point out the emergence – or not – of such a use].}

\citet{Coniglio2008}, \citet{Cardinaletti2011}, and more recently \citet{CruschinaCognola2021} are among the few works which explicitly analyze specific uses of Italian adverbs (\textit{mai} ‘never’, \textit{poi} ‘then’, \textit{pure} ‘also’, \textit{ben} ‘well’, \textit{sì} ‘yes’) as modal particles. In particular, \citet[107]{Coniglio2008} describes these items as “semantically as well as syntactically very close to German MPs”. These papers lie in the wake of several works which have been dedicated to the German modal particles in the framework of generative grammar.  From that perspective, they are interesting elements for the study of the syntax/semantics interface, in terms of the relation between functional heads in the left-periphery of the clause and syntactic movement. These approaches mostly analyze modal particles as related to the left periphery of the clause, where the highest functional projections of the clause – including illocutionary force (ForceP) – are encoded \citep{Rizzi1997}. Moreover, Italian scholars working within the generative framework were the first to dedicate some attention to modal-particle-like elements which may be found in the dialects spoken across Italy and in regional varieties of Italian (see for instance \citealt{Cinque1991}; \citealt{Poletto2000}; \citealt{MunaroPoletto2005}).

To properly frame these issues, an overview of the sociolinguistic changes which affected Italian in the last century and are still ongoing is essential. In the following pages, I will briefly outline the composition of sociolinguistic repertoires in Italy, focusing on the relationship between Italo-Romance dialects, regional varieties of Italian, and standard Italian. Subsequently, I will introduce concepts such as \textit{restandardization}, \textit{demoticization,} and \textit{neo-standard}. I will explore how regionally marked and “low” features have started to penetrate the standard norm, how the traditional standard is progressively converging downward to spoken varieties, and what place discourse-pragmatic markers occupy in this process.

\subsection{Dialect/standard constellations in Italy}
\hypertarget{Toc124860642}{}
The sociolinguistic situation of Italy is remarkably varied. The national language, Italian, is spoken alongside more than fifteen Italo-Romance dialects, about fifteen historical linguistic minorities, and a considerable number of new linguistic minorities (the outcome of past and present movements of people and migrations).\footnote{On the Italo-Romance dialects see \citet{MaidenParry1997}, \citet{GrassiEtAl1997}, and \citet{NegroVietti2011}. On the so-called “historical linguistic minorities” see \citet{Aquila2011}, while on the so-called “new linguistic minorities” see \citet{Chini2011} and \citet{Goglia2018}. For a recent overview of sociolinguistic research in Italy, see \citet{Alfonzetti2017}.} As a whole, they compose a wide array of sociolinguistic repertoires, distributed along the peninsula and the islands. The primary work of reference on these topics – with particular attention to the Italian sociolinguistic continuum – is Berruto (\citeyear{Berruto2012}; see also \citealt{Berruto2018} for a recent and briefer recap).

\hspace*{-0.5pt}Standard Italian is a continuation of fourteenth-century Florentine, based upon the literary variety used by great authors of that period and codified by grammars at the beginning of the sixteenth century. Italo-Romance dialects (or simply \textit{dialetti} ‘dialects’) are not dialects of Italian, as they do not result from the geographic differentiation of Italian. In most areas, they are to be understood as linguistic systems separate from Italian: they derive from the Italo-Romance vernaculars spoken across the country ever since the Middle Ages – which were coeval with the Italo-Romance Florentine vernacular from which standard Italian developed – and they evolved in parallel with it. Italo-Romance dialects are hence \textit{primary dialects}, to use \citegen{Coseriu1980} terminology.\footnote{Using \citegen[2]{MaidenParry1997} terminology, Italo-Romance dialects are labelled \textit{dialects of Italy}, while the varieties which result from the geographical differentiation of Italian – such as the regional varieties of Italian – are labelled \textit{Italian dialects} (i.e. a local variety of standard Italian).}

Italo-Romance dialects were the only languages for daily use until the end of the nineteenth century – Italian being used at the time almost exclusively in writing and formal styles, and only by an educated minority of the population. Italo-Romance dialects, the low varieties of the repertoire, were hence in a \textit{diglossia} relationship with Italian, the high variety of the repertoire. As \citet{Berruto2018} points out, this situation dramatically changed in the course of the following century.

\begin{quote}
The most evident development in the sociolinguistic history of Italy since Unification (1861) is the shift from a situation in which the dialects were largely the most common (if not the only) vehicle of everyday spoken communication while Italian was used almost only in written domains, to a situation in which the dialects are normally used only in informal and in-group situations, mostly by lower socio-economic classes and by older people (indeed, with considerable differences between the regions). To the detriment of dialects, Italian has increasingly gained domains as well as “true” native speakers. \citep[498]{Berruto2018}
\end{quote}

\hspace*{-1.6pt}As a result, the relationship between Italian and Italo-Romance dialects evolved into a new one. During the twentieth century – as a consequence of various factors including generalized schooling and the diffusion of mass media, such as radio and television – the standard language spread across speakers and situations: the high variety of the repertoire also became the language for daily use, alongside the low varieties of the repertoire.

In fact, Italian is nowadays regularly used for formal spoken and written purposes, while Italo-Romance dialects, functionally subordinate to Italian, are restricted to the family domain and, more generally, to informal situations – standard Italian is the \textit{Dachsprache} ‘roof language’ of all Italo-Romance dialects, in the sense of \citet{Kloss1978}; see also \citet{Ammon1989}. At the same time, Italian is regularly used in informal situations as well, and both Italian and Italo-Romance dialects are employed for ordinary conversation. This type of linguistic repertoire has been termed \textit{dilalia} by \citet{Berruto1989}. In this kind of situation – where both Italian and Italo-Romance dialects can be employed for ordinary conversation – a massive number of primary dialect speakers shifted to Italian and contact between the varieties in the repertoire became increasingly intense.

Long-standing contact between Italo-Romance dialects and Italian eventually resulted in a range of intermediate varieties between the primary dialects and the standard variety of the national language. A modelling of different dialect/standard constellations in Europe – including dynamics of contact and processes of convergence – is offered by \citet{Auer2005}, while a discussion on the Italian situation (with a focus on the North-West) can be found in \citet{CerrutiRegis2014,CerrutiRegis2015}.\footnote{See also \citet{Berruto2005,Berruto2018}, who schematized four major (partially overlapping) classes of structural phenomena giving rise to new varieties in the contact area between the systems of Italian and dialect: (1) dialectization of Italian; (2) Italianization of dialect; (3) koineization; (4) hybridization.} In most areas, this range of intermediate varieties is to be considered as divided into two separate continua: the dialect continuum and the Italian continuum. The former consists of varieties resulting from the \textit{Italianization} of primary dialects, while the latter consists of varieties resulting from the \textit{dialectalization} of Italian. The first kind of continuum – for the sake of simplicity – is composed of rural dialects, dialects of small urban centers, and, if present, a more prestigious urban dialect (usually centered around a big urban center). The second kind of continuum, is composed of regional sub-standard varieties, regional standards, and the standard (national) language.\footnote{It is worth noting that some areas do not meet the characteristics of the most typical Italo-Romance scenario. In these areas, for historical reasons, primary dialects exhibit a lower degree of structural distance from Italian and a discrete boundary between two different linguistic systems cannot be identified; hence we are not dealing with two separate continua, i.e. the dialect continuum and the Italian continuum, but with a single dialect/standard continuum. Such is the case of Tuscan dialects, as well as with the dialects of Rome and other areas of Central Italy.}

For the purposes of this research, the second kind of continuum is particularly relevant. Regional sub-standard regional varieties are the most affected by the substrate influence of Italo-Romance dialects, and, as with regional varieties of Italian on the whole, they emerged among a primary dialect-speaking population after the spread of Italian as a common language for everyday purposes.\footnote{An example of regional sub-standard is the variety labelled \textit{italiano popolare} (‘popular Italian’) by Berruto (\citeyear{Berruto2012}), defined as a variety with heavy dialect interference, showing many substandard elements which deviate from standard Italian – often spoken by native dialect speakers who acquired Italian as a second language. } Conversely, regional standard varieties are the least affected by substrate influence and emerged in the wake of the establishment of a standard language ideology through literacy and schooling (see \citealt{Regis2017}: 148–150). Thus, regional varieties of Italian are varieties of the national language that are spoken in different geographical areas. They differ both from each other and from standard Italian at all levels of the language system, but especially with regard to phonetics, phonology, and prosody. They represent the Italian actually spoken in contemporary Italy: common Italian speakers regularly speak a regional variety of Italian, which are termed \textit{italiani regionali} ‘regional Italians’ (see \citealt{Cerruti2011}).

In fact, as Italian spread across speakers and situations, it turned into a multi-functional language, and provided itself with a bundle of co-occurring linguistic features which meet the requirement of \textit{immediacy} (\citealt{KochOesterreicher1985}) of spoken varieties. These linguistic features partly result from the well-known phonological and grammatical processes which arise naturally in many substandard spoken varieties across languages\footnote{On the so-called \textit{vernacular universals} see \citet{Chambers2004} and \citet{Trudgill2011}.} and are partly due to the transfer of linguistic features from Italo-Romance dialects to Italian (as regards the retention of substratum features). \citet{CerrutiEtAl2017} sum up the consequent processes as follows:

\begin{quote}
In any case, after a probable phase of idiosyncratic and/or inconsistent occurrence of features, the progressive stabilization of both nationwide shared and region-specific traits resulted in the emergence of more or less clearly demarcated varieties. More specifically, the relatively stable co-occurrence of certain substratum features, in various areas depending on the different substrata, gave rise to the emergence of different regional varieties of Italian [...]. In fact, regional varieties of Italian basically resulted from a process of “dialectalization of Italian”; that is, they essentially emerged as a consequence of the retention and subsequent stabilization of features coming from Italo-Romance dialects. Nowadays, common Italian speakers regularly speak a regional variety of Italian (alongside, in some cases, an Italo-Romance dialect). (\citealt{CerrutiEtAl2017}: 7)
\end{quote}

The development of regional varieties of Italian may be understood as one of the outcomes of \textit{demotization}, that is the process through which the standard language came to be used by the masses of the population, thus becoming “popular” (see for instance \citealt{Berruto2017}: 34–35; \citealt{CouplandKristiansen2011}; \citealt{AuerSpiekermann2011}).\footnote{The term is inspired by \citegen{Mattheier1997} \textit{Demotisierung} – based on the Greek word \textit{demos} ‘people’. Regional sub-standards and regional standards have also been claimed to represent two different phases of demotization (\citealt{Auer2017}: 367–368), the former resulting from the imperfect learning of the common language by primary dialect speakers and the latter issuing from the inclusion of regional features in standard usage.} This way, the massive spread of the standard language to very different communicative domains in the twentieth century has also had effects on the linguistic features of Italian – which until then was basically restricted writing and formal styles – and speakers at all levels of society began to have full access to the spoken standard. This expansion put pressure on the standard language which consequently developed an internal variability which is necessary to serve its manifold functions, leading to a large-scale structural transformation.

\subsection{Demotization and restandardization: What place for modal particles?}
\hypertarget{Toc124860643}{}
The process of demotization generally entails the influence of the spoken language on the standard variety: the latter, being no longer under the exclusive control of a small intellectual elite, ceases to be conformed only to the written language, and begins to be influenced by the spoken language. Hence, the standard variety has come to converge towards spoken informal varieties: many spoken informal features have come to be used and accepted even in formal and educated speech, as well as partly in formal and educated writing, thus leading to the progressive inclusion of formerly sub-standard features into standard usage (\citealt{Berruto2012}; \citealt{CerrutiEtAl2017}; \citealt{Cerruti2020}: 130).

However, this process did not affect the social prestige of the standard variety: as \citet[28]{CouplandKristiansen2011} point out, a fundamental characteristic of this type of sociolinguistic change is that “the ‘standard ideology’ as such stays intact while the valorization of ways of speaking changes”.\footnote{It should be clear that this process has not been accompanied by the weakening of the traditional, literary, standard variety of Italian, as the latter is still used and maintains official prestige (\citealt{Berruto2017}: 33–34). For this reason, it cannot be described as \textit{destandardization}. Unlike what typically happens with destandardization (as attested in Switzerland and Norway), there still is no evidence that the traditional standard is losing its official prestige or is replaced by competing varieties. Concerning destandardization and its relationship with demotization, see \citet{AuerSpiekermann2011} and \citet{CouplandKristiansen2011}.} In the Italian case, demotization has thus led the standard norm of Italian to increase in variability and to decrease in codification. Furthermore, a similar situation has been described for other European countries as well, for instance Germany and Denmark (\citealt{KristiansenCoupland2011}; \citealt{KristiansenGrondelaers2013}). Regional varieties, then, are not the only outcomes of demotization: this process also promotes the (on-going) establishment of a new standard norm, which encompasses the traditional literary standard, spoken language features and regional standard features. The outcome is a set of features which are \textit{standard by usage} (see \citealt{Ammon2003}: 2–5; see also \textit{usage-based standard ideology} in \citealt{AuerSpiekermann2011}).

In the Italian case, the clustering of these features has been described as a new standard variety, termed \textit{Italiano neo-standard} ‘neo-standard Italian’ by Berruto (\citeyear{Berruto2012}). This label indicates an accepted set of features that, in comparison to the traditional literary standard, “represent a lowering and a consolidation of a partially new norm, regionally slightly varied, closer to the spoken varieties and to the non-learned and non-bureaucratic styles” (\citealt{Berruto2012}: 27, \citeyear[33]{Berruto2017}). The process whereby the traditional standard is converging towards spoken, informal and regional varieties has also been named \textit{restandardization} (\citealt{Berruto2017}: 33–39) – a label that can also be used to entail “the coexistence between neo-standard Italian and the traditional standard” (\citealt{CerrutiEtAl2017}: 17; \citealt{Auer2017}: 366). Crucial for the distinction between neo-standard and traditional standard is the remark that the model speakers (and the usage domains) for the old and the new standard are different (\citealt{Berruto2017}: 36; \citealt{Auer2017}: 371): grammars and classical authors shaped the traditional standard (which is highly codified), while journalists and politicians play a major role for the new standard (which reflects less prescriptive values). For ordinary speakers, the neo-standard is not restricted to peripheral usage domains as the old standard was, but it is used throughout their everyday life and widespread in mass media. It is therefore “flexible enough to deal with manifold situations, differing in terms of co-participants, topics, speech activities” \citep[371]{Auer2017}.

Given this picture, the issue of the relationship between the neo-standard and regional (standard) varieties is a complex one: admittedly, it is not easy to tell them apart from each other, nor to determine the precise relationship between the two. On this point, opinions can (slightly) diverge. \citet[8]{CerrutiEtAl2017} argue that “neo-standard Italian is mainly characterized by regionally unmarked linguistic features, but it also contains region-specific features (viz. features of the regional standards), which are particularly abundant in spoken language”, while \citet[368]{Auer2017} – with reference to the German situation –  adopts a more radical point of view: “The neo-standard clearly is not a vehicle for the transportation of regional identities” thus identifying the neo-standard as a non-regionalized variety. Regional standard features are hence to be considered as “incorporated” into a large core of nationwide shared neo-standard features. This depends of course on the extent to which oral and informal features can be separated from regional features, but – at least for the Italian case – it is quite certain that neo-standard Italian indeed allows a certain amount of regional differentiation (see \citealt{Berruto2012}: 62–65).

\hspace*{-0.3pt}However, the most interesting point is another one. Although the neo-standard may contain regional features, \citet[368]{Auer2017} insists on the fact that these features are becoming “de-localized”. That is, regional features in the neo-standard do not necessarily correspond with the region the speaker comes from – or, in any case, they don’t have that specific indexical value anymore: on these issues there is already research evidence for phonetic and phonological phenomena (see \citealt{Crocco2017}; \citealt{PascaleEtAl2017}), but phenomena at other levels of analysis are likely also involved. For instance, it has been said that morpho-syntactic features play a role of primary importance in characterizing this partially renewed standard norm of Italian – including syntactic constructions such as right and left dislocations, hanging topic, topicalizations, and clefting (\citealt{CerrutiEtAl2017}: 9; \citealt{Auer2017}: 371). These constructions are textbook examples of phenomena at the syntax/pragmatics interface, being syntactic realizations of pragmatic relations pertaining to the coding of information structure. More generally, one might ask – besides the phonological, morpho-syntactic and lexical features – what is the place of pragmatic phenomena in neo-standard Italian?\footnote{Neo-standards have been described on the basis of some attitudinal components which qualify them as suited and convenient to multiple usage domains and communicative situations (see \citealt{Auer2017}: 371–373). From this perspective, attitudinal components such as orality, informality and subjectivity are strongly related to the functions that discourse-pragmatic markers express (reformulation, marking of discourse structure, expression of vagueness, expression of subjective attitudes). As a consequence, the question whether pragmatic phenomena and pragmatic markers can also be considered prominent features of neo-standards is well-founded.}

In fact, as already mentioned above, some research papers that have recently dealt with modal particles in Italian include sociolinguistic observations as well. Notable examples are works on \textit{già} ‘already’ as a backchecking form (\citealt{Squartini2013,Squartini2014}; \citealt{FedrianiMiola2014}; \citealt{Calaresu2015}) and works on \textit{mica} ‘(etym.) crumble’ as a non-canonical negation form (\citealt{PescariniPenello2012}; \citealt{Squartini2017}; \citealt{Ballarè2020}; \citealt{Cerruti2020}).\footnote{These works focused especially on northern regional varieties, as works on discourse-pragmatic markers in central and southern varieties are still underrepresented (see however \citealt{Scivoletto2022}; \citealt{BrucaleEtAl2022}). Moreover, another clarification is needed. Research on discourse-pragmatic markers in Italian which includes sociolinguistic observations does not entail the involvement of the debate about neo-standard. In fact, except for \citet{Cerruti2020}, the studies cited above do not mention this issue, focusing rather on the fact that some of the functions described are regionally marked.} These papers share the fact that some of the discourse-pragmatic functions under investigation only appear in regional varieties, while they are unknown to standard Italian: for this reason – whether or not in the foreground – sociolinguistic themes necessarily emerge.

This is for instance the case of \textit{mica}, examined by Squartini (\citeyear[213–222]{Squartini2017}; see also \citealt{Cerruti2020}: 132–136).

\ea%11
    \label{ex:key:11}
           Italian \citep{Squartini2017}\footnote{These examples are adapted from examples (9) and (10) in \citet[213--214]{Squartini2017}.}

\ea \label{ex:key:11a} “Ciao”, mi disse, “verrò a sposarti una notte di queste. […] Non sarai \textit{mica} già sposato?”, fece lei. “Purtroppo si”, dissi io.

\glt ‘“Hi”, she said, “I’ll come down and marry you one of these nights […] Are you not, \textit{by chance}, already married?”, she said. “Unfortunately, yes”, I said.’

\ex \label{ex:key:11b} Ma tu non eri \textit{mica} già sposato con Derganz?

\glt ‘But, weren’t you \textit{perhaps} already married to Derganz?!’
\z
\z

While the use of \textit{mica} in \REF{ex:key:11a} seems to be a standard use without special regional specialization, the occurrence of \textit{mica} in \REF{ex:key:11b} is not generally accepted by all speakers of Italian: “It is in fact a regional phenomenon restricted to varieties of Italian, whose geographical boundaries are not clear yet, but, in a preliminary delimitation, can be located in an area in the North-West of Italy, possibly centered in Lombardy and Northern Emilia” \citep[215]{Squartini2017}.\footnote{The difference between \textsc{mica}\textsubscript{1} \REF{ex:key:11a} and \textsc{mica}\textsubscript{2} \REF{ex:key:11b} can be explained in terms of polarity and mirativity: they signal a mismatch between what the speaker knows and what comes out from the current information flow. The two different uses express different orientations in the polarity contrast that characterize this mismatch. Specifically, \textsc{mica}\textsubscript{1} marks a negative expectation on the part of the speaker while the polarity value attributed in discourse is positive; \textsc{mica}\textsubscript{2} marks a positive expectation, while the polarity value attributed in discourse is negative (see \citealt{Squartini2017}: 217–222).}

Regionally marked elements reflect the phase of intense language contact between dialects and standard Italian, when regional varieties emerged retaining and adapting substrate dialectal features. In many cases, they indeed represent dialectal features transferred to regional varieties (see \citealt{Calaresu2015}; \citealt{FavaroGoria2019}; \citealt{Cerruti2020}), thus providing good case studies for the sociolinguistic processes sketched so far. In this respect, discourse-pragmatic markers are attractive because they can convey both the attitudinal components and the slight regional variability which characterize neo-standard Italian. In addition, they represent a remarkable observatory for the dynamic relationship between standard Italian, regional (standard) varieties, and the neo-standard.\footnote{Admittedly, as already mentioned above, it is not always easy to distinguish between regional (standard) varieties and neo-standard. In this respect, not being part of the neo-standard clearly does not entail a sub-standard status: for instance, the regional use of \textit{mica} exemplified above is not non-standard; it’s regional standard.} To wit, they possibly amount to those dialectal features transferred to the regional (standard) variety and eventually de-localized (in the sense that they are no longer indexical signs expressing regional identities) and included into the group of nationwide-shared neo-standard features.

\begin{quote}
Several features which were previously limited to the vernacular have indeed extended their reach to the standard. A number of features have moved “upwards” from secondary dialects of Italian, as the latter correspond to the vernaculars of those speakers who were socialized in Italian (e.g., the younger generations). Some were first transferred from primary dialects, which represent the vernaculars of those speakers who were socialised in an Italo-Romance dialect (as is typically the case of the older generations). Others, which were also transferred from primary dialects, have presumably always been used by both uneducated and educated speakers even in formal situations. \citep[131]{Cerruti2020}
\end{quote}

Whether some of these (regional) features have been promoted to the (neo-) standard variety is still in many cases an open research question. These sociolinguistic processes represent the background of the case studies presented in Chapters~\ref{sec:8} and~\ref{sec:9}.

\section{Data collection and methodology}
\hypertarget{Toc124860644}{}
The last section of this chapter introduces the data sources used in the case studies presented in the next chapters and the methodological choices adopted. Given the differences in objectives and methods between the first two studies and the other two, the discussion is split into two subsections. The case studies of Chapters 6–7 rely on corpus data and qualitative analysis: this is introduced in the first subsection. The case studies of Chapters~\ref{sec:8} and~\ref{sec:9} rely on data retrieved through sociolinguistics questionnaires and different data visualization techniques: this is introduced in the second subsection.

\subsection{Corpus-based analysis}
\hypertarget{Toc124860645}{}
The methodological approach used in Chapters~\ref{sec:6} and~\ref{sec:7} may be described as \textit{qualitative data analysis}. “Qualitative research in applied linguistics takes many forms and may best be defined as research that relies mainly on the reduction of data to words (codes, labels, categorization systems, narratives, etc.) and interpretative argument” \citep[1]{Benson2012}. Qualitative research is descriptive and interpretational: it encompasses attempts to determine the type of features occurring within a data sample and it relies on the grouping of data in different categories on the basis of bundles of common features. The identification and the fine-grained description of both categories and features – which reflect the interpretation of the data – constitute the core of this methodology.

The analysis of the data is made possible by the comparison between a starting hypothesis (often based on a reference theory) and empirical observations: this means collecting data and evaluating how the data relate to the theory. In particular – considering how data were collected for the present research – this takes the form of \textit{qualitative corpus-based analysis}. Corpus-based analysis employs grammatical categories recognized by a reference linguistic theory, but investigates their patterns of variation and their use empirically (see \citealt{Biber2010} among others). The use of corpora for the retrieval and the extraction of linguistic data is based on the conviction that linguistic investigations must be based on “real” data, that is, actual instances of oral or written communication as opposed to “made-up” data or data which is only built on introspection.\footnote{In principle, this also allows the replicability of results: the choice of corpora and analytical techniques is made transparent in order for the results to be verifiable and for follow-up studies to confirm or criticize the findings.} The corpora used for the data retrieval are briefly described below.

The main source used for the data collection is the KIParla corpus, a recent resource for the study of spoken Italian (see \citealt{MauriEtAl2019}; \citealt{BallarèEtAl2022}). Built with conversational data collected in Turin and Bologna, the corpus is constituted by two modules, KIP and ParlaTO.\footnote{The KIP subcorpus consists of approximately 70 hours of recorded speech collected in Turin and Bologna (35 hours per city approximately) and transcribed between 2016 and 2019. The ParlaTO subcorpus is a collection of spontaneous speech collected in Turin between 2018 and 2020 and it amounts to approximately 50 hours of speech (see \citealt{CerrutiBallarè2021}).} It is completely open-access and it is designed to be shared as a free resource through the NoSketch Engine interface. The parameters taken into account for the creation of the corpus stress “the relevance of extralinguistic factors (regarding both the socio-geographic profile/status of the speakers and the interactional contexts) in order to build a corpus suitable for investigating (socio)linguistic variation in contemporary Italian” (\citealt{MauriEtAl2019}: 1). The corpus interrogation allows access to the speakers’ metadata (age, place of origin and social group) and situational data about the conversational exchange, which are crucial factors for research in sociolinguistics and conversational analysis. So far, the KIParla includes various types of communicative situations (lessons, exams, interviews, and spontaneous conversations) linked to the academic setting. Future extensions will include other settings, but comparability will be ensured by the common classification of the factors defining the type of communicative situation. The different situation types were classified according to the following external factors: (i) the symmetrical vs. asymmetrical relationship between the participants; (ii) the presence vs. absence of previously established topics; (iii) the presence vs. absence of constraints on turn-taking. The authors believe “that using these three very general features is particularly helpful in the task of integrating new data recorded in other situations, without losing comparability with the other parts of the corpus” (\citealt{MauriEtAl2019}: 3).

In addition to the data extracted from the KIParla corpus, other examples come from the LIP corpus, one of the most important collections of spoken Italian data (see \citealt{MauroEtAl1993}). Consisting of approximately 58 hours of recordings – which amount to a total of approximately 490,000 words – the corpus was collected between 1990 and 1992 in four cities (Milan, Florence, Rome, and Naples). It samples five macro-types of communicative situations and different subtypes of discourse settings.\footnote{The five macro-types are: (Type A) bi-directional exchange, face to face, with free turn-taking (face-to-face conversations); (Type B) bi-directional exchange, not face to face, with free turn-taking (telephone conversations); (Type C) bi-directional exchange, face to face, with regulated turn-taking (including for instance assemblies, oral exams and interviews); (Type D) unidirectional exchange, with the addressee being present (including for instance lessons and sermons) and (Type E) distanced unidirectional exchange (television and radio programs). The same grid was used to (sub)classify situation types in the KIParla corpus.} These features (place of data collections, type of communicative situations) make it suitable for sociolinguistic research as well. The corpus is freely accessible through a digital version curated by the Karl-Franzens-Universität Graz (see \citealt{Schneider2002}).

Beside spoken data, the analysis also deals with examples of written language. These examples are extracted from the \textit{La Repubblica} corpus, a very large collection of newspaper texts \citep{BaroniEtAl2004}. This corpus contains all the articles published between 1985 and 2000 by the national daily \textit{La Repubblica}, amounting to about 320 million words. The corpus is tokenized, POS-tagged, lemmatized, and categorized in terms of genre (news-report and comment) and topic (labels such as “culture”, “economics”, “politics”, and so on). It is open-access and searchable through the NoSketch Engine interface.

The categorization and the analysis of corpus examples represent the bulk of Chapters~\ref{sec:6} and~\ref{sec:7}. In order to ensure layout consistency and a better readability, I adapted the examples extracted, leaving aside the transcription conventions specific to each corpus. The detailed reference to the source corpus has been reported under every example. In the examples, relevant adverbs are in italics. In the translations, different strategies have been employed: either a specific adverb/phrase (also in italics) or no translation (using instead the placeholder \textsc{ptc}).

\subsection{Sociolinguistic questionnaires}
\hypertarget{Toc124860646}{}
The case studies presented in Chapters~\ref{sec:8} and~\ref{sec:9} investigate Italian modal particles characterized by diatopic variation and/or diatopic markedness. The first one is dedicated to \textit{solo} ‘only’, with a focus on its use in the regional variety spoken in Piedmont (Northwestern Italy). The second one is intended to trace the distribution of a small set of modal particles in different regional varieties across Italy (with a focus on northern varieties). Crucially, both case studies aim at combining a pragmatic account of the items under analysis (grammatical status, illocutionary and common-ground-management functions) and a sociolinguistic account of their distribution (diatopic and, to a lesser extent, diaphasic variation). The details of the case studies will be discussed at the beginning of each chapter: I will now outline some common issues and the research methods applied.

Doing research on discourse-pragmatic markers characterized by diatopic variation raises a few methodological problems. The main one probably concerns the retrieval of data. Scattered instances of regionally marked elements can be found in the available corpora of spoken Italian, but it is rather a matter of chance. Especially in the case of modal-particle-like elements, the occurrences are extremely rare. Other examples can be found through targeted web extractions, but even this does not allow one to build a satisfactory dataset. Moreover, these data are not suitable for every research question. They can be used for a structural analysis of the contexts of use (type of speech acts, syntactic environment) but they give little to no information concerning specific semantic/pragmatic features of the constructions or sociolinguistic information. From a sociolinguistic viewpoint, only if metadata are available in spoken corpora (about the speakers and the communicative situations) is it possible to investigate the variation of these constructions, albeit still facing several difficulties. Data extracted from the web seldom allow us this possibility. In any case, depending on the specificities of the research question, corpora are often not the most-suited tool for sociolinguistic research.\textsuperscript{} \footnote{Among the corpora used cited above, the LIP corpus includes metadata about the communicative situation and the geographic context: speeches were collected in four Italian cities (Milan, Florence, Rome, Naples), thus allowing only limited research about diatopic variation. The KIParla corpus offers a rich apparatus of metadata and – even if the data were collected in only two cities (Turin and Bologna) – the hometown of speakers can always be retrieved, thus allowing more detailed sociolinguistic research.}

Having at the same time the objective and the necessity of collecting a significative amount of data about regionally marked modal-particle-like elements, I decided to turn to a different research methodology, namely \textit{sociolinguistic questionnaires}. They not only offer the possibility to be designed around a specific research question, but also to be tailored in order to combine more questions and different needs. Basic references on this methodology, which provided useful insights and suggestions are \citet{Schleef2013}, \citet{KrugSell2013} and \citet{Dollinger2015}. I designed two questionnaires, one for each case study, and they display some common aspects and some differences.

In broad terms, both questionnaires consist of three sections: a metadata introductory section, a section concerning the acceptability of the constructions under investigation, and a section with specific questions that depend on the focus of the case study. In this way, they combine data collection on the sociolinguistic markedness of the constructions and case-specific issues. In the metadata section, the respondents were asked to give general information which can be used to establish correlations between the given answers (linguistic data) and social factors. The following list sums up the social factors taken into consideration in the questionnaires.\footnote{In broad terms, they reflect the metadata model adopted by the KIParla corpus.}

\begin{itemize}
\item[-] gender
\item[-] educational level/degree
\item[-] year of birth
\item[-] occupation/profession
\item[-] native language(s)
\item[-] city/place of residence
\item[-] linguistic competence in one or more dialects
\item[-] city of high school attendance
\end{itemize}

Metadata are fundamental, since they give information about the respondents which can be subsequently put in relation with their answers – thus making it possible to find significative correlations between linguistic and social variables. For the present case studies, information about the “city of high school attendance” is particularly important. I decided to use this parameter to assess the regional variety spoken by the respondents, integrated when necessary by information about “native language(s)” and “linguistic competence in one or more dialects”.\footnote{Other social factors play no role in the present work, but the data collected may be used in future research highlighting different issues.}

High school years (age 14–18) are usually characterized by dense social and linguistic interactions – in varied communicative situations – favoring further development or enrichment of the linguistic and communicative competence. Close contact with peers coming from a relatively limited area (districts of a city or neighboring towns) promotes the acquisition of local linguistic features and sociolects, which are often not (yet) perceived as regionally marked or exclusive of a social group. Later years can witness more movements (to study, work or build relationships elsewhere) and more diverse social and linguistic contacts, all factors which can impact and modify one’s own idiolect – integrating features from other varieties. In this sense, the competence developed during youth and school socialization can be said to represent a good indicator of a regional language variety.\footnote{It should be remembered, however, that geographical varieties as such do not exist, rather, people are always speakers of a socio-geographical variety. Moreover, what environment or age is crucial for the acquisition and differentiation of pragmatic features is an open question. Useful hints on these (and related issues) can be found in \citet{Berruto2003}, who critically examines the concept of \textit{native speaker}.}

The design of the questionnaires has benefited from the models developed during previous research on these topics \citep{Favaro2019} – and from fruitful discussion and collaboration with colleagues (\citealt{FavaroGoria2019}). After a test phase, the questionnaires were spread in digital format through the web via mailing lists and group chats in order to collect answers from different Italian regions. No other specific sample characteristics were required. In the first questionnaire (April–September 2018), 570 answers were collected, and in the second one (October–December 2019), 180 answers were collected.\footnote{Further details will be discussed in the next two chapters. The original version of both questionnaires (in Italian) can be accessed online at \url{https://zenodo.org/records/10362289}.} The evaluation of the results – along with theoretical and empirical issues concerning the constructions under investigation – represents the bulk of Chapters~\ref{sec:8} and~\ref{sec:9}.

