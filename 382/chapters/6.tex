\chapter{From additivity to illocution: A case study on \textit{pure} ‘also’}\label{sec:6}
\hypertarget{Toc124860647}{}\section{\textit{pure}: Overview of the categories involved}
\hypertarget{Toc124860648}{}
This first case study concerns the adverb \textit{pure} ‘also’, which is widely used in Italian as a focus adverb. Since \citegen{König1991} seminal work, focus adverbs represent a thriving area of study as they are linked to some interesting problems that are debated at different levels of linguistic analysis.\footnote{Following \citet{Cesare2017}, I use the label \textit{focus adverbs} to refer to this class of elements. Other labels used in recent publications (see \citealt{Cesare2015}; \citealt{CesareAndorno2017}) include \textit{focus markers} and \textit{focusing modifiers}, as well as the “classic” label \textit{focus particles} \citep{König1991}.}

At the semantic level, the meaning of focus adverbs displays a complex interrelationship between semantic and pragmatic values: on the one hand, they have an impact on the propositional level, and on the other hand, they are responsible for the activation of several discourse inferences that can induce the emergence of new meanings. At the level of information structure, the issue of the exact relationship between the category of focus and the contribution of these adverbs to its identification is a delicate one: these items cannot properly induce focus by themselves, but their semantic contribution should be understood as sensitivity to the focus structure of a sentence (\citealt{König1993}: 978; \citealt{Cesare2010}). Regarding semantic change, focus adverbs show synchronic and diachronic overlap with other linguistic categories such as conjunctions, conjunctional adverbs, discourse markers, and modal particles (\citealt{König1991}: 16; 165).

In this section, an overview of the main functions of \textit{pure} will be provided. In doing this, I will highlight how its functional range covers different linguistic categories, spanning from information structure to illocutionary modification, modality, and concessivity.

\subsection{Information structure and focus adverbs}
\hypertarget{Toc124860649}{}
Speakers generate sentences appropriate for their communicative needs: in different communicative circumstances and according to the way speakers dy\-na\-mize information, the same propositional content can be expressed by different prosodic and morphosyntactic structures. Utterances that are equivalent in terms of their propositional content but not in terms of how information is integrated into the ongoing discourse, or with respect to how information is packaged according to the communicative situation, display a different \textit{information structure}.

Following \citet{Chafe1976,Chafe1987} and \citet{Lambrecht1994}, information structure will be defined here as a discursive dimension expressing the degree of activation that the propositional content of an utterance acquires in the informational flow that builds up discourse. A brief but useful introduction to this subject is offered by \citet{Krifka2008}. Taking up \citegen{Chafe1976} suggestion, \citet[243]{Krifka2008} defines information structure as a “phenomenon of information packaging that responds to the immediate communicative needs of interlocutors”. Information structure (henceforth IS) motivates the different ways in which linguistic information can be presented by interlocutors according to different communicative situations.

Two major IS categories have been recognized in literature: (i) those involving the mental representations of discourse referents – cognitive categories such as activation and identifiability; (ii) those indicating pragmatic relations between propositions and their elements – pragmatic categories such as topic and focus \citep[36]{Lambrecht1994}. More broadly, IS is linked to the notion of common ground (see the discussion in \chapref{sec:3}), the space that hosts the interplay between propositions whose knowledge is shared by the speaker and the addressee, assumptions about each other’s state of mind, and new information. This is relevant for IS since – as \citet[36]{Lambrecht1994} points out – information must be molded depending on whether it is presented by the speaker as already available to the addressee’s knowledge (presupposition) or as newly introduced by their utterance (assertion). The interplay between presupposition and assertion shapes the notion of \textit{focus}, which is directly involved in the semantic description of focus adverbs. There are two major ways of defining focus.

\citet[213]{Lambrecht1994} considers focus “the semantic component of a pragmatically structured proposition whereby the assertion differs from the presupposition”. If, in an utterance, a presuppositional structure is identified that divides the information into presupposed information (the set of propositions that the speaker thinks the interlocutor already knows or could take for granted) and asserted information (the proposition carrying new information that the addressee will share with the speaker once they have heard the utterance), it is possible to define focus as the pragmatic relation that associates an asserted component to an open variable in a presupposed proposition. The other main definition of focus as proposed by \citet[247]{Krifka2008} is that it “indicates the presence of alternatives that are relevant for the interpretation”.

Since \citet{König1991} a basic distinction has always been drawn between additive and exclusive focus adverbs, prototypically identified by items such as English \textit{also} (additive) and \textit{only} (exclusive): their Italian equivalents are \textit{anche} and \textit{pure} (additive) and \textit{solo} (exclusive).\footnote{The most important contributions on Italian focus adverbs are \citet{Ricca1999} and \citet{Andorno1999,Andorno2000}. In the Italian grammatical tradition, focus adverbs are usually referred to as \textit{focalizzatori} ‘focalizers’ \citep{Ricca1999} or \textit{avverbi focalizzanti} ‘focusing adverbs’ \citep{Andorno1999}.} One of the distinctive properties of focus adverbs is their interaction with the focus structure of an utterance: it is the information structure that determines the semantic contribution of focus adverbs to the utterance and variations in the information structure correspond to variations in their semantic contribution. Related to this, one of the most striking syntactic properties of such adverbs is their positional variability: they may occur in several positions in a sentence.

\ea%12
    \label{ex:key:12}

           Italian

\ea \label{ex:key:12a} Giorgio ha comprato \textit{pure} delle mele\textsuperscript{\textsc{focus}}

\glt ‘Giorgio \textit{also} bought some apples’

\ex\label{ex:key:12b} \textit{Pure} Giorgio\textsuperscript{\textsc{focus}} ha comprato delle mele

\glt ‘\textit{Also} Giorgio bought some apples’
\z
\z

In these examples, different positions of \textit{pure} correlate with different positions of the sentence stress (which signals the focus of the sentence, here shown by the words in small caps) and with different interpretations of the relevant sentence. Depending on their position, focus adverbs operate on different sentence constituents: in \REF{ex:key:12a}, the domain of association (on this term, see \citealt{Cesare2017}: 159–161) is \textit{delle mele} and the remaining part of the sentence is backgrounded; in \REF{ex:key:12b}, the domain of association is \textit{Giorgio} and the remaining part is backgrounded. The part of the sentence that the focus adverb operates on corresponds then to the part of the sentence in focus: changing the focus also changes the domain of association of the focus adverb.

Besides the functional value of operators on the focus of a sentence, focus adverbs also have a lexical meaning: they do not only signal a pragmatic relation, but they enrich it with specific semantic values. According to the description proposed by \citet[94--119]{König1991} for additive focus adverbs – the subclass to which \textit{pure} belongs – there are two features that play a crucial role in the semantic analysis of these expressions. The first one is the quantification effect, through which the value of the focused expression is related to a set of alternatives.
\newpage

\ea%13
    \label{ex:key:13}

          Italian

\ea\label{ex:key:13a}  Giorgio ha comprato \textit{pure} delle mele

\glt ‘Giorgio \textit{also} bought some apples’

\ex\label{ex:key:13b}  Giorgio ha comprato delle mele

\glt ‘Giorgio bought some apples’       [\textsc{assertion}]

\ex\label{ex:key:13c} Giorgio ha comprato qualcos’altro

\glt ‘Giorgio bought something else’     [\textsc{presupposition}]
\z
\z

A sentence like \REF{ex:key:13a} can be described as the sum of two propositions, represented here by sentences \REF{ex:key:13b} and \REF{ex:key:13c}. The sentence \textit{Giorgio only bought apples} contains the assertion that \textit{Giorgio bought apples} and builds on the presupposition that \textit{Giorgio bought something else} (which is outside of the scope of the negation, cf. \textit{It is not true that Giorgio also bought apples}, activating the same presupposition), thus suggesting that \textit{apples} are part of a larger set of elements (depending on the context) and that at least one of the possible alternatives satisfies the relevant open sentence. Focus adverbs contribute quantificational force to the meaning of a sentence: they quantify over the set of possible alternatives to the value of the focused expression. The meaning contribution of \textit{pure} is to include these alternatives as possible values for the open sentence in their scope, while at the same time asserting the validity of the sentence it has scope over.\footnote{There is a structural asymmetry in the meaning of the two sub-classes distinguished: additive focus adverbs trigger the presupposition that there is an alternative value under consideration that satisfies the open sentence in the scope of the adverb, while exclusive focus adverbs trigger a presupposition that corresponds to the relevant sentence in the scope of the adverb. In this way, additive focus adverbs do not contribute to the truth-conditional meaning of the sentence, while exclusive focus adverbs do (see \citealt{König1991}: 52–56).}

In addition to the selection of alternatives, some focus adverbs may induce a ranking into the set of possible alternatives which means that they induce scalar structures in the domain of quantification. In this case, the alternatives and the focus value are part of a set that is hierarchically arranged. Some adverbs can, by themselves, induce a scalar ordering (for example English \textit{even} and Italian \textit{persino}), others (like Italian \textit{pure}) are compatible with a scalar reading when this is suggested by the context:

\ea%14
    \label{ex:key:14}

          [\textit{La Repubblica} corpus – article.id: 2242, comment: education]

E l’accoglienza nelle scuole è stata considerata molto soddisfacente. \textit{Pure} gli studenti meno solerti, racconta chi ha assistito alla sperimentazione, hanno seguito con grande interesse

\newpage
\glt ‘And the reception in school has been considered beyond satisfactory. \textit{Even} less diligent students, said someone who witnessed the experiment, followed with great interest’
\z

In example \REF{ex:key:14}, the set of possible alternatives includes other groups of students ordered along a scale, from the less diligent ones to the very diligent ones. In contexts like these, scalar focus adverbs often activate an evaluation inference connected to the scalar ordering – that is, the value of the focus is characterized as ranking “high” or “low” on the scale.

In the case of \textit{pure}, the inference is rather connected with a scale of expectation in discourse, whereby the referent it has scope over can be said to be more or less expected in that context. As a part of its conventionalized meaning – when used in a scalar way – \textit{pure} activates the inference that the focus value ranks lower on the scale of expectation than the alternative values. I now give two more examples of \textit{pure} as a focus adverb:

\ea%15
    \label{ex:key:15}

          [\textit{La Repubblica} corpus – article.id: 2438, news-report: sport]

In palio i punti per l’attribuzione del titolo di combinata, che prevede \textit{pure} la disputa di uno slalom speciale. Ha vinto l’immancabile Svizzera

\glt ‘Points will be awarded/at stake for the attribution of the title of [alpine skiing] combined, which \textit{also} includes the execution of a special slalom. The invincible Swiss won’
    \z

\ea%16
    \label{ex:key:16}

          [\textit{La Repubblica} corpus - article.id: 2413, news-report: news]

A tale proposito i due enti hanno recentemente organizzato un’iniziativa promozionale comune a New York. Nel corso dei colloqui si è \textit{pure} parlato dello sviluppo dei trasporti marittimi ed aerei per migliorare i collegamenti tra Italia e Jugoslavia

\glt ‘In this respect the two institutions have recently organized a common sales initiative in New York. During the talks the issue of the development of maritime and air transport to improve the connections between Italy and Yugoslavia was \textit{also} addressed’
    \z

As noted above, the semantic contribution of focus adverbs crucially depends on their syntactic scope. Nevertheless, it is not always easy to clearly identify it. In example \REF{ex:key:15} \textit{pure} clearly has scope over the NP \textit{la disputa di uno slalom speciale}, as also suggested by its syntactic position. In example \REF{ex:key:16} the situation is different: even though it is quite unproblematic to say that the domain of association of \textit{pure} is the NP \textit{dello sviluppo}, the syntactic scope extends to include the VP \textit{si è parlato}. This is induced by the syntactic position of \textit{pure} – immediately after the finite verb form – from where it can operate on different sentence constituents.\footnote{This position is sometimes referred to as the position characterized by \textit{weitere Skopus} (\citealt{DimrothKlein1996}: 93) or \textit{portata ampia} \citep[51]{Andorno2000} – that is the position from where a focus adverb has “wide scope”.} This position is typically found in contexts where the additive focusing value of \textit{pure} is downsized and it is exploited for discourse dynamics rather than for highlighting a referent.

\ea%17
    \label{ex:key:17}

          [KIParla corpus - TOD2013]

c’hai ragione sì è vero è vero // c’ha \textit{pure} i suoi difetti Torino eh per carità però // naturalmente // cioè il negativo e il positivo c’è ovunque //

\glt ‘you’re right yes it’s true it’s true // it \textit{also} has its flaws Turin eh by all means but // of course // I mean you find negative and positive things everywhere’
    \z

\ea%18
    \label{ex:key:18}

          [\textit{La Repubblica} corpus - article.id: 1795, comment: sport]

Oh sì, gli stranieri sono trattati meglio, ma è \textit{pure} giusto

\glt ‘Oh yeah, foreigners are treated better, but it’s \textit{also} right’
    \z

In example \REF{ex:key:17} the narrow domain of association of \textit{pure} is the NP \textit{i suoi difetti}, but its scope extends on the whole utterance, which is marked as a concessive premise before introducing a contrast (connective \textit{però}). In example \REF{ex:key:18} it is not easy to identify a single element associated with the focus particle: \textit{pure} marks the whole utterance without evoking a real set of alternatives. These uses could be thought of as peripheral instances of \textit{pure} as a focus adverb. They share some of the features of \textit{pure} used as a modal particle (first of all, the syntactic position), which will be introduced in the next subsection.

\subsection{Illocutionary modification}
\hypertarget{Toc124860650}{}
Other uses of \textit{pure} are clearly not of the focus-adverb type. Example \REF{ex:key:19} below is an instance of a connective use of \textit{pure}, where the adverb – along with \textit{ma} ‘but’ – serves the function of connecting two sentences. The term \textit{conjunctional adverb} is sometimes used to refer to adverbs that operate as sentence connectives: “syntactically they belong to a given sentence, but functionally they do not actually modify it. Rather, they operate on the textual level, giving textual coherence to a sequence of sentences; thus they are functionally very close to conjunctions” (\citealt{RamatRicca1994}: 308).

\ea%19
    \label{ex:key:19}

          [\textit{La Repubblica} corpus - article.id:86, comment: economics]

Naturalmente, nessuno ha o può vantare ricette risolutive in proposito. Ma \textit{pure} qualcosa si poteva fare o, almeno, tentare

\glt ‘Of course, nobody has or can boast decisive actions in this respect. But \textit{yet} something could have been done or, at least, tried’
    \z

Example \REF{ex:key:20} is an instance of the use of \textit{pure} as a modal particle. In this case, the adverb operates on an imperative verb form with a mitigating function.

\ea%20
    \label{ex:key:20}

          [KIParla corpus - TOC1004]

// va bene si accomodi // comodatevi \textit{pure} //

\glt ‘// all right have a seat // have a seat \textit{please} //’
    \z

The fact that items operating as focus adverbs could also cover other functions has been noted since \citet[16]{König1991}, who cites the uses as conjunctional adverb and as modal particle: “Both ‘extensions’ in the use of focus particles can be observed in a wide variety of languages”. However, even though many single contributions have been devoted to single items, this is still an underexplored topic – at least from a typological perspective. A promising approach is represented by \citet{EckardtSpeyer2016}, who use the label \textit{focus cline} to refer to possible paths of development in the domain of focus and information structure:

\begin{quote}
We find the typical patterns of language change: emergence of new particles as well as bleaching and loss of constructions. The pathway of focus change starts where words develop into focus sensitive particles and associate with focus, it continues where they foster into conventionalized alternative-based constructions, and it ends where reference to alternatives or focus-background structure is lost. We will refer to the later stages as \textit{bleached focus}. (\citealt{EckardtSpeyer2016}: 503)
\end{quote}

The identification of a focus cline – starting with the emergence of focus sensitive particles and ending when their relation to the focus marking is lost – is one of the possible ways to describe the development of connective and modal uses of focus adverbs, and eventually their stabilization as autonomous senses of the same item (polysemy). One interesting issue then, revolves around the possible outcomes of the bleached-focus constructions. Both discourse markers and modal particles represent possible endpoints of the focus cline: the discourse structure and the coherence relations between sentences on the one hand, as in example \REF{ex:key:19} above; the relationship between a proposition and the speech act on the other, as in example \REF{ex:key:20}.

A further example of the modal use of additive focus adverbs is represented by the refutational use of \textit{too} that is found in some varieties of US English:

\ea%21
    \label{ex:key:21}

          English (\citealt{SchwenterWaltereit2010}: 88)

A:  You didn’t do your homework!

B:  I did \textit{too}!
    \z

The link between the focus-marking and the refutational use of \textit{too} has to be identified by the role played by additivity in discourse: “Just like the additive \textit{too}, this use makes reference in discourse to a contextually salient proposition, most often one derivable from a previous utterance, namely the syntactically negative proposition it denies. However, it is obvious that this way of referring back to a previous utterance is completely different from the way the additive \textit{too} operates” (\citealt{SchwenterWaltereit2010}: 88). The function of \textit{too} in this use is to deny the truth validity of the propositional content of a previous speaker’s utterance, reflecting a trajectory of change from a representational use of the adverb to an interactionally-bound interpersonal use.

Another example is the independent use of German \textit{auch} ‘too’ in questions \REF{ex:key:22a} or exclamations \REF{ex:key:22b} to mark pragmatically extreme states-of-affairs.

\ea%22
    \label{ex:key:22}

          German (\citealt{SchwenterWaltereit2010}: 98–99)

\ea\label{ex:key:22a}  Warum \textit{auch} hatte sie mein wichtigstes Wort nicht akzeptiert?

     ‘Why only didn’t she accept my most important word?’

\ex\label{ex:key:22b}     Was der Kerl \textit{auch} für Einfälle hat!

\glt ‘Hell, what sort of ideas does this guy have!’
    \z
    \z

A case of modal use of focus adverbs in Italian is represented by the use of \textit{pure} in \REF{ex:key:20} above. Even though this use of \textit{pure} could be said to represent the clearest case of the modal particle in Italian, it has not received a great deal of attention. Apart from sparse mentions in the works on Italian focus adverbs (see \citealt{Andorno2000}; \citealt{Ricca2017}), \citet{Held1983}, \citet[107–108]{Waltereit2006}, and \citet[115–118]{Coniglio2008} are among the few contributions which explicitly addressed the issue – but none of these works goes too much into details. Another example is \REF{ex:key:23}:

\ea%23
    \label{ex:key:23}

          [LIP corpus – Milan E1]

A:   dica \textit{pure} signorina cosa desidera

  B:   guardavo grazie

\glt  
A:  ‘\textit{please} miss tell me what you want’

  B:   ‘I was looking, thank you’
    \z

To give a preliminary description, in directive speech acts, \textit{pure} operates on the illocution carried by the verb and – depending on the case – it gives the directive the specific character of an invitation or permission to do something. As a pragmatic side-effect, the directive seems more polite/mitigated: this kind of directive may be included in the politeness/mitigation strategies available to speakers of Italian. The exchange in example \REF{ex:key:23} is cited also by \citet{Waltereit2006}, who observes:

\begin{quote}
\textit{Pure} is commonly used in offers; the speaker urges the hearer to something that lays in their own control area – for example, a seller can invite a client to express their own desire after they entered the shop. This is a modification of the speech act “directive”, because it is part of its preparatory conditions that the addressee would not do the relevant action on their own initiative \citep[66]{Searle1969}. In this way, the hearer expectations which are in the focus of directives are withdrawn. \citep[107]{Waltereit2006}\footnote{My translation of the original quote in German: “\textit{Pure} wird häufig in Angeboten verwendet; der Sprecher fordert den Hörer zu etwas auf, was in seinem, des Sprechers, eigenen Kontrollbereich liegt – z.B. kann ein Verkäufer damit die Kundin auffordern, ihren Wunsch zu äußern, nachdem sie schon das Geschäft betreten hat. Es handelt sich hierbei um eine Modifikation des Sprechaktes «Aufforderung», denn zu dessen Einleitungsbedingungen gehört es, dass der Angesprochene die jeweilige Handlung nicht von alleine tun würde \citep[66]{Searle1969}. Es wird hier also auf die bei Aufforderungen per se im Fokus stehende Hörerreaktion abgehoben”.}
\end{quote}

This description will be the starting point of the analysis in the following pages. In fact, none of the works cited above – despite containing many useful hints – has tested the theoretical claims on corpus data. As a consequence, the goal of the present case study is to expand this view through corpus examples: it will provide a bulk of examples of modal uses of \textit{pure} – including specific contexts of use not yet discussed, both in the spoken and in the written mode – and it will look at its distribution; it will provide a fine-grained analysis of the functions expressed and it will test the usefulness of the category of illocutionary modification to describe them.

As the examples of illocutive uses of \textit{too}, \textit{auch} and \textit{pure} have shown, the modal functions that (additive) focus adverbs can display are very different from each other and crucially depend on the diachronic trajectories of single items in specific languages – to wit, every case has its own specificities. From this perspective, the category of illocutionary modification tries to string together these items in view of the fact that – beyond the individual specificities – they share these features: (i) the scope on the speech act; (ii) a pragmatic effect on the illocution; (iii) a grammatical (rather than lexical) status. In the framework of Functional Discourse Grammar, these elements would be most likely classified as operators (that is, grammatical elements) at the layer of illocution, where items are found which account for “grammatical emphasis and mitigation of a specific Illocution” (\citealt{HengeveldMackenzie2008}: 83).

Modal instances of \textit{pure} can also be found in declarative sentences expressing assertions, along the lines of examples \REF{ex:key:17} and \REF{ex:key:18} above. Since they occur with indicatives, it is more difficult to clearly separate these uses from instances of \textit{pure} as a focus adverb – the illocutionary context being the same: in fact, it is better to think of them as a continuum of uses rather than a clear-cut divide. In this case, \textit{pure} contributes to emphasizing the illocutionary force:

\ea%24
    \label{ex:key:24}

          Italian

Deve \textit{pure} esserci una soluzione!

\glt ‘There must \textsc{ptc} be a solution!’
    \z

I will return to this topic commenting on more corpus examples in the next section. Before that, I will give an overview of uses where \textit{pure} cannot be described either as a focus adverb nor as a modal particle: they illustrate the semantic domains bordering the modal uses. This helps to place the expression of illocutionary modification in its broader grammatical environment.

\subsection{The surroundings: Modality, concessivity, discourse coherence}
\hypertarget{Toc124860651}{}
The last example showed how \textit{pure} can appear in declarative sentences with the main function of highlighting the force of an assertion. In particular, this is often the case of declarative sentences with modal verbs – as in \REF{ex:key:24} above, where an epistemic use of \textit{dovere} ‘must’ occurs. Moreover, \textit{pure} appears in several other contexts where categories related to the domain of modality play a major role. Recalling \citegen{Narrog2012} model of modal categories, as set forth in \chapref{sec:3}, these uses could be interpreted as the lower threshold of the functional spectrum of \textit{pure}, that is the area where illocutionary modification and proper modality meet. Nevertheless, it should be stressed that in these cases \textit{pure} does not autonomously mark modality, but it rather contributes to it by co-occurring with modal uses of verb forms.

The contribution of \textit{pure} in the marking of modality can be clearly noticed in the modal uses of the Italian future tense discussed by \citet{Squartini2012}. Examples \REF{ex:key:25} and \REF{ex:key:26} illustrate conjectural uses of the future tense, the second one also marked by \textit{pure}.

\ea%25
    \label{ex:key:25}

           Italian \citep[2118]{Squartini2012}

[Suonano alla porta] Sarà il postino

\glt ‘[The bell rings] It will be [be:\textsc{fut}] the postman’
    \z

\ea%26
    \label{ex:key:26}

           [LIP corpus – Rome B21]

B:   sabato è stata ’na giornata nera pe’ noi

A:   mazza certo ma guarda che se stai a pensa’ a quel

B:   il presidente sarà \textit{pure} contento

A:   a quel quel giocatore che faceva l’ultima partita lì e se n’annava

B:   ma il presidente sarà \textit{pure} contento che dici


\glt 
B:  ‘Saturday was a bad day for us’

A:  ‘yes, sure, but look, if you’re thinking about that’

B:  ‘the president will be \textsc{ptc} happy’

A:  ‘about that player who played his last game there and then left’

B:  ‘well, the president will be \textsc{ptc} happy, what do you think?’
    \z

A second set of modal uses of the future tense is represented by concessive uses, which can be interpreted as a further diachronic evolution of conjectural uses (\citealt{Squartini2012}: 2119; \citealt{BybeeEtAl1994}: 226–227). Example \REF{ex:key:27} and \REF{ex:key:28} illustrate the concessive use of the future tense, the second one also marked by \textit{pure}.

\ea%27
    \label{ex:key:27}

          Italian \citep[2121]{Squartini2012}

  Sarà alto quanto gli pare, ma lassù non ci arriva

\glt ‘He may be [be:\textsc{fut}] as tall as he likes, but he can’t reach there’
    \z

\ea%28
    \label{ex:key:28}

          [LIP corpus – Rome D14]

l’arroganza e l’impudenza di questo potere che sarà \textit{pure} in disfacimento ma continua ad autoriprodursi nel più totale disprezzo dell’opinione pubblica

\glt ‘the arrogance and the impudence of this power which \textit{might} be in decay but keeps self-reproducing in total disrespect of the public opinion’
    \z

As has already been said, this does not mean that \textit{pure} marks a concessive future, but it seems to be a preferential collocate of concessive futures. In both examples \REF{ex:key:27} and \REF{ex:key:28} it is not possible to analyze \textit{pure} as a normal instance of focus adverb: it retains its additive semantics, but there is no marked focus constituent, and no set of alternatives is evoked – thus resulting in a \textit{bleached focus construction} as in the definition given above.

\hspace*{-.5pt}Its (residual) additive semantics can be successfully exploited as soon as conces\-sivity is used in discourse. By using the concessive future with \textit{pure}, the speaker concedes an additional point which is exploited to mitigate a contrast in conversation: this combination represents a common argumentative move (\citealt{AnscombreDucrot1983}). The conceding move is thus introduced to acknowledge the validity of a first statement or point, before going on to claim the validity of a potentially contrasting second statement or point \citep[2123]{Squartini2012}. More generally, the adverb can occur in concessive contexts, most often (but not only) with verbs in the subjunctive mood:

\ea%29
    \label{ex:key:29}

          [\textit{La Repubblica} corpus - article.id: 1305, comment: weather report]

A Venezia invece, sia \textit{pure} per un attimo, ha fatto la sua ricomparsa il sole

\glt ‘In Venice instead, \textit{if only} for an instant, the sun showed up’
    \z

\ea%30
    \label{ex:key:30}

           Italian \citep[115]{Coniglio2008}

Ammesso \textit{pure} che riesca a vincere la gara.

\glt ‘Provided \textsc{ptc} that he manages to win the competition’
    \z

These examples show that \textit{pure} can contribute to mark concessive sentences and generally suggest that concessivity plays an important role in many contexts in which \textit{pure} occurs. This is not surprising, considering that additive focus adverbs frequently show up as components of concessive connectives and concessive conditionals (for instance Eng. \textit{even though}, \textit{even if} and Fr. \textit{quand même}; see \citealt{König1991}: 79–83; \citealt{HaspelmathKönig1998}: 584–589). The use of \textit{pure} as a connective (or as a component of a connective) represents the upper threshold of its functional spectrum, that is, the point where it does not operate inside a sentence (as the focus adverb does) or on the illocution (as the modal particle does), but where it rather serves the function of connecting sentences and expressing coherence relations between them.

\section{\textit{pure}: Corpus data}
\hypertarget{Toc124860652}{}
So far, I have given an overview of the main functions of \textit{pure} in contemporary Italian. To summarize, the central function is the use as an additive focus adverb at the layer of information structure. Beside this function, at least two modal uses of \textit{pure} have been identified, in different types of speech acts – assertives and directives – supposedly with different pragmatic effects. Among the peripheral uses, I mentioned some examples where \textit{pure} co-occurs with markers of modality, contributing to the expression of that grammatical category, and examples of \textit{pure} in concessive contexts. A separate set is represented by the connective uses of \textit{pure} which – apart from few mentions – are beyond the reach of this work. At least in present-day Italian, the notional domain of \textit{additivity} (see \citealt{CesareAndorno2017}) seems to represent the core semantic feature which may be found – to a greater or lesser extent – in the various contexts in which \textit{pure} appears. In the following pages, I will test this distribution against examples extracted from corpora – both from the spoken and the written language. In the final part of this section, I will take up the issue of additivity to address the role of contextual inferences in the emergence of the modal uses.

\subsection{Spoken language}
\hypertarget{Toc124860653}{}
A comprehensive discussion on the differences between spoken and written varieties of language is far beyond the scope of this work: it is well known that they differ in cognitive (production and processing of language chunks), social and structural terms (see for instance \citealt{ChafeTannen1987}; with reference to Italian, see \citealt{Berruto1985}; \citealt{Berretta1994}; \citealt{Pistolesi2016}). There are not many studies which explicitly address the issue of the differences regarding pragmatic markers in spoken and written varieties, but it is generally acknowledged that they are used differently in speech and writing (see \citealt{CribleCuenca2017}: 149–152 and references therein). This turns out to be particularly relevant for modal particles and other items operating on the illocution.

As already mentioned in \chapref{sec:3}, one way of analyzing the function of modal particles is as a tool of common ground management, whereby the reference to the common ground automatically implicates the presence of at least two viewpoints involved in the communication. As a tool of common ground management, modal particles often express a viewpoint that is not ascribed to the speaker alone, but crucially also involves the hearer. This does not exclude written texts, of course – the point being the kind of communicative situation rather than the medium of communication. However, data of spoken language can easily be thought of as the first place to look for linguistic tools of common ground management. This is especially the case with dialogues and conversations which represent communicative environments where the presence of both a speaker and a hearer is particularly manifest.

In order to extract a dataset to use as the basis for this analysis, I extracted all the occurrences of \textit{pure} from KIP and LIP, respectively 235 and 358. The first step of the qualitative analysis consisted in a preliminary annotation of the functions covered. \tabref{tab:key:6.1}  gives an overview of the functions of \textit{pure} in these two corpora (absolute and relative frequencies).

\begin{table}
\begin{tabularx}{\textwidth}{Xrrrr}
\lsptoprule
 & KIP [abs] & KIP [rel] & LIP [abs] & LIP [rel]\\
 \midrule
\textsc{focus} \textsc{adverb} & 195 & {0.82} & 302 & {0.84}\\
\textsc{mp} \textsc{assertive} & 8 & {0.05} & 3 & {< 0.01}\\
\textsc{mp} \textsc{directive} & 17 & {0.07} & 16 & {0.04}\\
\textsc{mp} \textsc{optative/hortative}  & 2 & {< 0.01} & 5 & {0.01}\\
\textsc{modal} \textsc{future} & 3 & {0.01} & 3 & {< 0.01}\\
\textsc{sia\_pure} & 3 & {0.01} & 12 & {0.03}\\
\textsc{che\_pure} & — & — & 8 & {0.02}\\
\textsc{connective} & 3 & {0.01} & 3 & {< 0.01}\\
\textsc{interjection} & 2 & {< 0.01} & 3 & {< 0.01}\\
\textsc{other} & 2 & {< 0.01} & 3 & {< 0.01}\\
\midrule
Total & 235 &  & 358 & \\
\lspbottomrule
\end{tabularx}
\caption{\label{tab:key:6.1}: Distribution of the functions of \textit{pure} in KIP and LIP}
\end{table}

A quick look at the figures shows that the proportions are quite similar, even though the two corpora have been collected in different cities and with almost thirty years between them.\footnote{A clarification is needed here. The extraction of data from KIP and LIP was aimed at building a general dataset for the subsequent analysis rather than at (quantitatively) comparing the data extracted from the first one with the data extracted from the second one. In this regard, \tabref{tab:key:6.1} must be primarily read as an overall presentation of the data and not as a comparison of the two corpora. Besides, despite some shared features, the two corpora were built according to criteria that are not fully comparable. Nevertheless, their dimensions are not so different: KIP consist of 70 hours of recordings/661175 tokens, while LIP consists of 58 hours of recordings/489178 tokens. In this respect, a quick quantitative comparison of the findings is not inappropriate.} The core of the qualitative analysis consists in the examination of the context of occurrence and the evaluation of the semantics and pragmatics of \textit{pure} within it – and their categorization, through an updated classificatory scheme.

I will now go through the labels used for the annotation. Starting from the lower end of \tabref{tab:key:6.1}, the three labels \textsc{interjection}, \textsc{connective,} and \textsc{other} refer respectively to the use of \textit{pure} as a focus adverb in isolation – that is, as a holophrastic element – to its use as a conjunctional adverb or as part of a conjunction (like \textit{pure se} ‘even if’) and to cases of dubious classification.\footnote{Including the occurrences of the omophonous adjective \textit{pure} (feminine plural of \textit{puro}), meaning ‘pure’.} The labels \textsc{sia\_pure} and \textsc{che\_pure} refer to collocations which – by virtue of their frequency and their relatively non-compositional meaning – may deserve a separate description: they will be dealt with in the next subsection, dedicated to examples of \textit{pure} in the written language. In the first line, the label focus adverb needs no further explanation, apart from the – largely expected – observation that this is the prototypical and more frequent function of \textit{pure} found in both corpora.

The labels \textsc{mp} \textsc{assertive}, \textsc{mp} \textsc{directive} and \textsc{mp} \textsc{optative/hortative} represent the modal uses, labeled according to the type of illocutionary force they modify. One delicate aspect of the classification was the (mis-)matching between the grammatical mood marked on the verb, the sentence type expressed, and the kind of illocution carried by it. In Italian, assertive speech acts (including asserting, claiming, stating, etc.) – usually corresponding to declarative sentences – are marked by the indicative mood: these cases are labeled as \textsc{mp} \textsc{assertive} in \tabref{tab:key:6.1}. On the other hand, directive speech acts (including orders, requests, suggestions, etc.) – when expressed by imperative sentences – can be marked both by imperatives and subjunctives. Imperative verb forms are used for the second person, singular and plural, while subjunctive verb forms are used for the third person, singular and plural, and also for the second person as a more polite/distanced form. These cases are labeled as \textsc{mp} \textsc{directive} in \tabref{tab:key:6.1}.

Declarative sentences with modal \textit{pure} are not so frequent and the examples are often ambiguous, since in many cases – as briefly mentioned in the preceding section – these uses of \textit{pure} could still be analyzed as focus adverbs. One convincing example of an illocutive use of \textit{pure} in a declarative sentence is the following:

\ea%31
    \label{ex:key:31}

          [KIParla corpus - TOD2002]

anche perché stavano facendo i lavori a casa mia \textit{dovevano pure} entrarci prima o poi

\glt ‘also because they were doing the [renovation] works they \textit{did have} \textsc{ptc} to get into my house \textit{anyway}, sooner or later’
    \z

\citet[115]{Coniglio2008} explains that “in declarative clauses, \textit{pur(e)} signals that the speaker has no evidence to prove that his assertion is true, but he still thinks it logical to suppose that it must be true”.\footnote{\citet[114]{Coniglio2008} claims that “In these contexts, the particle usually lacks its final -\textit{e}. […] Nevertheless, there are cases where both the full and the reduced form are possible”. Intuitively, I would say that the choice between the full and the reduced form mainly depends on personal and contextual choices. Still, it could be the case that in some highly conventionalized sentences, the reduced form is more widespread.} This analysis applies quite well to example \REF{ex:key:31}, but it may be too narrow for other cases. In particular, I don’t find an explanation, in terms of \textit{evidence} to support an opinion, very convincing. I would rather say that this use of \textit{pure} is related to the expression of \textit{assertivity} and to the strengthening of the illocutionary force expressed by the speech act. It contributes to encoding the speaker’s subjective expectation/evaluation on the communicated state-of-affairs.\footnote{This use of \textit{pure} usually appears in contexts that already express the speaker’s subjectivity. In particular, as example \REF{ex:key:31} shows, it is a common collocate of uses of the modal verb \textit{dovere} ‘must’ expressing subjective epistemic modality.}

At the same time, by marking the illocutionary force, it gives the assertion a prominent position in the conversational exchange. Specifically, the presence of the additive focus adverb projects the assertion against a set of possible alternative assertions. These assertions, however, do not represent alternative referents or states-of-affairs (as it happens when \textit{pure} works as a focus adverb) but rather contextual assumptions which are backgrounded by the modal use of \textit{pure}. In this manner, the assertion marked by the adverb does not represent an alternative among the others but the most relevant within the actual context. In some cases, the utterance marked by \textit{pure} slightly contrasts with a preceding statement or assumption: depending on contextual features, this can give to the modal use of \textit{pure} a counter-expectational flavor or the character of an assertion made despite lack of evidence.

Therefore, a more general way to analyze these examples is to describe the effect of \textit{pure} in terms of emphatic marking of the illocutionary force. In this respect, \citet{König2017} notes that, despite a very large range of different pragmatic effects, all illocutive uses of focus adverbs have something in common, namely that the markers seem to be associated with a verbal focus:

\begin{quote}
The alternatives in question are not denotations of other verbs, however, and so the more plausible analysis that we have instances of a verum focus, i.e. a focus on the assertion of truth, rather than on a specific overt constituent. Such an analysis can only be maintained, however, if we assume that there are several varieties of such focusing. \citep[37]{König2017}
\end{quote}

The seemingly “several varieties of such focusing” depend on the several kinds of contexts in which these constructions can appear, or more precisely on the several kinds of common ground assumptions at work in those contexts – to which the emphatic declarative sentence marked by \textit{pure} represents an alternative.\footnote{On verum focus, see \citet{Lohnstein2016}. I will not go any deeper into the relationship between modal particles and verum focus, which is however an already established research direction: see for instance \citet{Repp2013} and \citet{Abraham2017}.}

In the second context of use as a modal particle, \textit{pure} occurs in directive speech acts that have the form of imperative sentences:

\ea%32
    \label{ex:key:32}

          [LIP corpus – Florence C5]

bene vuole venire Lorenzo a proseguire un attimo non solo l’aspetto fisico passiamo oltre Roberto vai \textit{pure} al posto

\glt ‘all right Lorenzo wants to come to carry on a bit about not only the physical aspect let’s go beyond that Roberto go back to your seat \textit{please}.’
    \z

\ea%33
    \label{ex:key:33}

          [KIParla corpus - TOA3001]

non c’è problema se non volete venire all’esame mandatemi \textit{pure} un paper

\glt ‘there is no problem if you don’t want to come to the exam \textit{just} send me a paper’
    \z

In this kind of context, the directive sounds softened, resulting in most cases as an invitation to do something rather than an order in a narrow sense. This may be said to be the most evident effect of \textit{pure} on directives and the core of its illocutionary-modifying function. In terms of modification of the preparatory conditions, \citet[107–108]{Waltereit2006} explains that \textit{pure} marks a directive in which the speaker invites the hearer to do something that they would actually do by themselves – thus marking an inconsistency in the preparatory conditions of directives, according to which the addressee would not do the relevant action on their own initiative. More generally, directives with \textit{pure} express a granted permission/authorization to the addressee, thus specifying the kind of illocutionary act the speaker wants to perform, like in \REF{ex:key:32} above. This also gives \textit{pure} the character of a mitigating device. In other cases, the explicit marking of an authorization may sound redundant, since it is already clear that the addressee has the permission (or even the obligation) to do what the speaker asks for. In these cases, directives marked by \textit{pure} acquire the status of invitations/encouragements to do something, like in \REF{ex:key:33} above.

In \chapref{sec:3}, I gave a description of modal particles in terms of linguistic expressions that relate to the \textit{conditions} shaping a speech act and specify the \textit{intentions} in performing it. The natural habitat of directives with \textit{pure} are conversational contexts where it is self-evident that the addressee has the \textit{possibility} of performing some action which lays in the speaker’s control area (see \citealt{Waltereit2006}: 107) but still it is not sure that they will do that. On the speaker’s side, the ordinary interpretation of these contexts is that the hearer is waiting for an explicit signal to act. In this way, on directives \textit{pure} marks the speaker’s attention to the hearer’s expectations: by uttering a directive with \textit{pure} the speaker signals their active involvement in reading the hearer’s state of mind.\footnote{This helps to better understand the “modal” part in “modal particles”. A possible paraphrasis of these uses of \textit{pure} is with the modal verb \textit{potere} ‘can’ – whereby a sentence like \textit{fai pure} is (more or less) equivalent to \textit{puoi (anche) fare} ‘you can (also) do’. In this perspective, the modal use of \textit{pure} in directives seems to be related to the proper-modal domains expressed by \textit{potere} ‘can’ – ranging from ability (participant-internal possibility) to circumstantial possibility to permission (deontic possibility). In the case of \textit{pure}, circumstantial possibility and (especially) permission seem to be the relevant domains.} This shared attention for each other’s position in the conversational exchange and communicative expectations is reflected in the intentions of the speech act. Directives marked by \textit{pure} are no longer orders coming out of the blue, but rather invitations which seek to meet certain expectations. In this way, \textit{pure} specifies the aim with which the speech act is performed: the illocutionary point is adapted according to the common ground and the context of interaction.

A modal \textit{pure} also appears in related constructions, which are sometimes described as subcategories of imperative constructions:

\begin{quote}
Closely related to imperatives, i.e. constructions expressing directive speech acts such as commands, requests, advice, suggestions, invitations, etc., are formal markers frequently referred to as “hortatives”, “optatives”, “debitives”, “rogatives” and “monitories” […] Moreover, there is a difference in person associated with some of these labels: The label “imperative” is often restricted to second person directives, whereas “hortatives” is found for first and third person directives and “optatives” for directions addressed to third persons. (\citealt{KönigSiemund2007}: 313)
\end{quote}

In \tabref{tab:key:6.1}, the label \textsc{mp} \textsc{optative/hortative} refers to specific uses of the subjunctive mood, when it encodes optative and hortative illocutions. A semantically bleached \textit{pure} can also appear in these contexts, both with the third and the first person, as in examples \REF{ex:key:34} and \REF{ex:key:35} below. In optative contexts the speaker indicates to the addressee their wish that the positive situation evoked by the communicated content should come about. In hortative contexts the speaker encourages themselves or an addressee together with themselves to carry out the action evoked by the communicated content.

\ea%34
    \label{ex:key:34}

         [LIP corpus – Naples C4]

chi ci vuole eh giocare su questi cosi ci giocasse \textit{pure} io non voglio giocare non ho tempo da giocare penso

\glt ‘who wants uh to play on these things go ahead \textit{please} I don’t want to play I have no time to play I think’
    \z

\ea%35
    \label{ex:key:35}

          [KIParla corpus - TOD2014]\footnote{Example \REF{ex:key:35} is also very useful to observe the syntactic and functional differences between discourse markers (here occurring in a chain-like sequence: \textit{sì no vabbè eh}), modal particles (\textit{pure}) and adverbial adjuncts (\textit{senza vergogna}). Discourse markers operate on discourse chunks (in this case they introduce a new utterance) while modal particles (grammatically) operate on the illocution conveyed by the verb. In this case the adverbial adjunct also operates the layer of the illocution (it’s a speech act adverb of the type of \textit{frankly}) but as a lexical modifier.}

e ho trovato difficoltà a socializzare // sì no vabbè eh \textit{diciamolo pure} senza vergogna sono // più sociali i quartieri pieni d’extracomunitari che quelli pieni di torinesi

\glt ‘and I found it difficult to socialize // yeah no well uh \textit{let’s say it} \textsc{ptc} without shame // the neighborhoods full of immigrants are more social than those full of Turinese’
    \z

Lastly, some peripherical uses of \textit{pure} have been found, corresponding to what have been called in the last subsection the modal surrounding uses of the adverb. With modal futures (conjectural and concessive) – basically corresponding to declarative sentences – \textit{pure} contributes to express specific assertive speech acts such as conjectures/assumptions and to mark the conceding move (premise) of a concessive sentence.

\ea%36
    \label{ex:key:36}

          [KIParla corpus - TOD2014]

rispetto al paese // e chiaramente sta frazione \textit{c’avrà pure} un nome suppongo

\glt ‘compared to the town // and obviously this village \textit{must have} \textsc{ptc} a name I guess’
    \z

\ea%37
    \label{ex:key:37}

          [KIParla corpus – BOD1006]

perché // il professor paolino \textit{sarà pure} trasparente ma è anche adultero // è anche bugiardo

\glt ‘because // \textit{it may also be true that} professor paolino is honest but he is also an adulterer // he is also a liar’
    \z

\subsection{Written language}
\hypertarget{Toc124860654}{}
The distribution of the functions of \textit{pure} in written language appears to be quite different. I extracted the first 360 occurrences of \textit{pure} from the \textit{La Repubblica} corpus, to approximately equalize the number of occurrences extracted from the LIP corpus and thus enable a broad quantitative comparison.\footnote{Once more, the idea was to collect a general dataset for the subsequent analysis and not to (quantitatively) compare the written language data with the spoken language data. In this respect, I used the occurrences from the LIP only as a reference amount and I extracted the same number of occurrences as from the \textit{La Repubblica} corpus: a quick quantitative comparison is therefore possible, but it’s still a very unbalanced one.} I performed a manual annotation of the functions covered, using the same tag set applied for the spoken data. \tabref{tab:key:6.2} shows the distribution of the functions (absolute and relative frequencies).

\begin{table}
\begin{tabularx}{.8\textwidth}{Xrr}
\lsptoprule
 & REP [abs] & REP [rel]\\
 \midrule
\textsc{focus} \textsc{adverb} & 123 & {0.34}\\
\textsc{mp} \textsc{assertive} & — & —\\
\textsc{mp} \textsc{directive} & 3 & {<0.01}\\
\textsc{mp} \textsc{optative/hortative}  & 22 & {0.06}\\
\textsc{modal} \textsc{future} & 11 & {0.03}\\
\textsc{sia\_pure} & 88 & {0.24}\\
\textsc{che\_pure} & 91 & {0.25}\\
\textsc{connective} & 16 & {0.04}\\
\textsc{interjection} & — & —\\
\textsc{other} & 6 & {0.01}\\
\midrule
Total & 360 & \\
\lspbottomrule
\end{tabularx}
\caption{\label{tab:key:6.2} Distribution of the functions of \textit{pure}  in the \textit{La Repubblica}  corpus}
\end{table}

As regards the modal uses in directives and assertives, they are almost absent – confined to quotes and pieces of direct speech inserted in newspaper articles. This is not surprising, since these uses of \textit{pure} are the most bound to dialogical situations, which are not so common in the collection of journalistic prose on which the \textit{La} \textit{Repubblica} corpus has been built. Instead, occurrences of \textit{pure} in optatives and hortatives are more frequent, since they are less dependent on conversational structures. \REF{ex:key:38} and \REF{ex:key:39} are two examples:

\ea%38
    \label{ex:key:38}

          [\textit{La Repubblica} corpus - article.id: 1675, comment: politics]

E allora? Se il Pci, partendo dal suo 40 per cento, riesce ad aggregare una maggioranza, \textit{governi pure}. Non con noi però.

\glt ‘So what? If the PCI, starting from its 40 percent, manages to build a [political] majority, it \textit{may govern} \textsc{ptc}. Not with us, though.’
    \z

\ea%39
    \label{ex:key:39}

          [\textit{La Repubblica} corpus - article.id: 805, comment: culture]

Ciò posto, \textit{chiediamoci pure}: e come siamo messi quest’inverno che è così duro, più duro del solito?

\glt ‘That said, \textit{let’s ask ourselves} \textsc{ptc}: how are we doing in this winter that is so tough, tougher than usual?’
    \z

Examples with modal uses of the future tense together with \textit{pure} can also be found, both conjectural \REF{ex:key:40} and concessive \REF{ex:key:41}:

\ea%40
    \label{ex:key:40}

           [\textit{La Repubblica} corpus - article.id: 8, comment: politics]

Però \textit{ci sarà pure} una maniera per evitare l’equazione più industria = più inquinamento?

\glt ‘But \textit{there must be} \textsc{ptc} a way to avoid the equation more industry = more pollution?’
    \z

\ea%41
    \label{ex:key:41}

          [\textit{La Repubblica} corpus - article.id: 804, comment: politics]

In Occidente, se ne parla spesso con leggerezza come di una “gerontocrazia”. \textit{Sarà pure} così. Sta di fatto che vent’anni fa l’Urss era solo una grande potenza continentale; era, per così dire, confinata nel continente euro-asiatico; mentre, oggi, è una potenza planetaria, capace di intervenire in qualsiasi punto del globo.

\glt ‘The western world is often carelessly defined as a “gerontocracy”. It \textit{may be} \textsc{ptc} like this. But the fact is, USSR was twenty years ago only a big continental power; while today it’s a global power, able to intervene everywhere worldwide.’
    \z

The use as a focus adverb is the most frequent one, as in spoken data, but the (relative) frequency is lower. In fact, among the examples extracted, two other constructions appear with considerable frequency, labeled as \textsc{sia\_pure} and \textsc{che\_pure} in the annotation schema. The first one is a stable collocation, composed of the third person singular, present tense, of the verb \textit{essere} ‘to be’ in the subjunctive mood followed by \textit{pure}. It is part of the concessive contexts in which \textit{pure} can appear. The collocation \textit{sia} \textit{pure} works as a routinized concessive-conditional marker – similarly as the conjunction \textit{anche se} ‘even if’ – but it does not really introduce a subordinate sentence since it can only hold nominal constituents and not verb phrases.

\newpage
\ea%42
    \label{ex:key:42}

          [\textit{La Repubblica} corpus - article.id: 659, news-report: news]

Gran parte dei mercati rionali sono chiusi. Funzionano, \textit{sia pure} tra tante difficoltà, quelli all’ingrosso.

\glt ‘Most of the local markets are closed. Wholesale markets, \textit{despite} many difficulties, are still working.’
    \z

The label \textsc{che\_pure} refers to a collocation composed by a relative marker followed by \textit{pure}: in most cases it introduces an appositive relative clause.\footnote{The relative marker \textit{che} ‘which, that’ is the most frequent one, but other markers such as \textit{il quale}, \textit{del quale}, \textit{con il quale, dove, cui} can also be found.} In fact, the additive semantics of \textit{pure} lends itself well to appearing in an appositive relative clause, which has the function of adding information about the nominal constituents it refers to. However, in these contexts \textit{pure} can’t be analyzed as an additive focus adverb: it rather enriches the relative clause with a contrastive-concessive semantics.

\ea%43
    \label{ex:key:43}

           [\textit{La Repubblica} corpus - article.id: 1372, news-report: politics]

La protesta dei liberali, \textit{che pure} fanno parte del governo, non è una novità.

\glt ‘The protest of the liberals, \textit{even though} they are part of the government, is nothing new.’
    \z

Finally, the written data include some occurrences of \textit{pure} as a connective with contrastive meaning. In this case, \textit{pure} should be considered a literate variant of the more common \textit{eppure} ‘however, yet’ – the result of the univerbation with the connective \textit{e} ‘and’. Example \REF{ex:key:44} below illustrates this use (notice moreover that in this example the connective \textit{pure} is followed by the additive focus adverb \textit{anche} ‘also’).


\largerpage[2]%long distance
\ea%44
    \label{ex:key:44}

           [\textit{La Repubblica} corpus - article.id: 427, comment: culture]

Per quanto cercasse, sia negli anni preraffaelliti che dopo, di rappresentare con precisione e fedeltà ogni forma e ogni particolare di ciò che vedeva e che udiva, \textit{pure} anche a lui dovevano sembrare “dolci le armonie udite, ma più dolci quelle non udite”.

As much as he tried, both in the Pre-Raphaelite years and after them, to represent with precision and accuracy every form and every detail of what he saw and heard, \textit{nonetheless}, even he must have thought that “sweet were the heard harmonies, but sweeter those unheard”.’
    \z

Overall, the constructions found across the spoken and the written language are the same, even though they considerably differ regarding their frequency. In particular, the collocations \textit{sia pure} (with concessive-conditional meaning) and \textit{che pure} (with concessive-contrastive meaning) can be said to be a prominent feature of the written variety. The same holds for the use of \textit{pure} as a contrastive connective, albeit to a lesser extent. Modal uses of \textit{pure} in optatives and hortatives, as well as with modal futures, are also more common in the written variety – while modal \textit{pure} in directives and assertives are barely found in the written data.

\subsection{Additivity in interaction}
\hypertarget{Toc124860655}{}
To conclude this section, I will go back to the illocutive uses in order to discuss more examples, dwelling on the role of inferences and interactional dynamics in the emergence of modal functions. In particular, I would like to comment on some examples where the function of \textit{pure} is not so clearly defined – thus providing hints towards understanding the link between its additive semantics and the modal uses.\footnote{In the dataset these cases have mostly been labeled as \textsc{focus} \textsc{adverb} with a note specifying their vague status.}  As a starting point, I refer once more to \citet{Waltereit2006}, who discusses the issue in this way:

\begin{quote}
With the modalization form \textit{pure} the speaker acts as if the relevant proposition would be the second member of a pair of propositions connected by \textit{pure} ‘also’. In this way, the action to which the addressee is invited is not considered in isolation, but is portrayed as if it arose self-evidently from another state-of-affairs. It is thus only the second step in a co-oriented (co-occurrent) chain of actions […] As such, the modal particle \textit{pure} evokes the situation of saying ‘also’. (\citealt{Waltereit2006}: 107–108)\footnote{My translation of the original quote in German: “Mit der Abtönungsform \textit{pure} tut der Sprecher so, als ob die jeweilige Proposition das zweite Glied eines durch \textit{pure} ‘auch’ verbundenen Paares von Propositionen wäre. Die Handlung, zu der der Hörer aufgefordert wird, steht so nicht mehr allein, sondern sie wird so \textit{dargestellt}, dass sie sich gleichsam selbstverständlich aus einem anderen Sachverhalt ergibt. Sie ist so lediglich der zweite Schritt in einer ko-orientierten (gleichläufigen) Handlungskette […] Insofern evoziert die Abtönungspartikel \textit{pure} die Situation des ‘auch’-Sagens”.}
\end{quote}

The semantic profile of \textit{pure} as a focus adverb rests on the operation of additivity: as explained in the preceding section, the focus adverb activates the presupposition that something else is involved in the relevant proposition – another proposition or another referent, which can be recovered in the previous context or in the common ground. However, sometimes it is not evident what exactly this other proposition or referent evoked is, if any is evoked at all (see \citealt{SchwenterWaltereit2010}).

This situation may lead to different strategies to accommodate the presupposition: the hearer must make sense of the presence of \textit{pure} in that context and come up with an explanation of why the speaker used it. This process may induce the hearer to ascribe new shades of meaning to \textit{pure}, to associate it with contextual meanings and thus favor its reanalysis. The situation of \textit{saying ‘also’} can be described in this way: additive focus adverbs evoke the presence of another proposition or referent. If the hearer cannot recover these other contents, they have somehow to accommodate them. The elements at play are the shared knowledge in the common ground, further contextual features surrounding the speech act (conditions) and – importantly – what the hearer thinks the communicative goals of the speaker are (intentions). A prominent role is played by the inferential activity of the hearer: piecing together these elements, the presence of \textit{pure} can be justified and – in some cases – a new function can be configured. In the collected data several cases can be identified where the presupposition activated by \textit{pure} does not clearly refer to another proposition or another referent (it could be named a \textit{suspended} \textit{presupposition}). Many cases labeled as modal uses of \textit{pure} in assertive speech acts are actually of this kind – ambiguous instances where the content of the presupposition is not immediately recoverable:

\ea%45
    \label{ex:key:45}

          [KIParla corpus - TOD2012]

mh tra l’altro in quel periodo appunto spesso finivo lavora lavoravo da casa e finivo di lavorare anche alle due tre di notte // eh eh eh ed era un problema perché // magari chi stava nella camera accanto mi mi ascoltava sì disturbava e insomma eh lo posso \textit{pure} capire

\glt ‘uh by the way at that time in fact I often finished work I worked from home and I finished working even at two three in the night // hmm and this was a problem because // maybe the person who was staying in the room next to mine could listen to me yeah it was annoying hmm I can \textsc{ptc} understand it’
    \z

In example \REF{ex:key:45} there is no other recoverable proposition that the speaker can understand (\textit{lo posso pure capire} ‘I can also understand it’): it rather seems that \textit{pure} simply marks the addition of an utterance to the argumentation.\footnote{The speaker can understand that her late-night working activity can be a problem for the other flatmates, but it is not clear what \textit{else} she can understand. In this case, \textit{pure} serves a clear argumentative function (see \citealt{AnscombreDucrot1983}), marking the host sentence as the premise for a reversal of the argumentation: in the subsequent, the speaker explains how most of the problems in the flat are actually caused by other flatmates (and not from her).} Since they favor the hearer’s inferring activity, similar contexts are the better suited to trigger functional developments. This is more evident in the next couple of examples, where \textit{pure} appears with \textit{bisogna}, a marker of deontic modality.

\ea%46
    \label{ex:key:46}

           [\textit{La Repubblica} corpus – article.id: 1352, comment: culture]

Che è, tuttavia, al secolo, un materialista cibernetico, e scrive il suo libro nell’intento primordiale di porre a confronto il cervello e il computer, l’intelligenza neuronale e quella elettronica, per decidere se tra pensiero naturale e pensiero artificiale esista, o no, un confine invalicabile. Ma per decretare in merito, bisogna \textit{pure} capire che cosa è la mente: se no, come parlarne?

\glt ‘Who is, however, in his life, a cybernetic materialist, who writes his book with the aim of comparing brain and computer, neural intelligence and electronic intelligence, in order to determine whether an insuperable border exists or not, between natural and artificial thought. But to decide on this, we must \textsc{ptc} understand what the mind is: otherwise, how to talk about it?’
    \z

\ea%47
    \label{ex:key:47}

          [\textit{La Repubblica} corpus - article.id: 1627, comment: culture]

Non più, perché oggi si dà il caso che proprio là dove il socialismo è stato realizzato, proprio là dove gli ideali dei “rossi”, dei “sovversivi”, degli “extraparlamentari” si sono affermati, la psicoanalisi non c’è, o stenta ad esserci. Dunque: affermando “io sono rosso” (sovversivo, extraparlamentare), e mi compiaccio dell’avvento della psicoanalisi in Cina, ci si impiglia in una contraddizione – culturale – da cui bisogna \textit{pure} uscire.

\glt ‘Not anymore, because nowadays it turns out that precisely where socialism has been achieved, where the ideals of the “reds”, the “subversives”, the “extra-parliamentarians” have established themselves, psychoanalysis doesn’t exist or is struggling to exist. Therefore: by saying “I am red” (subversive, extra-parliamentarian), and I am pleased by the coming of psychoanalysis in China, one is caught in a – cultural – contradiction from which one has \textsc{ptc} to get out.’
    \z

In example \REF{ex:key:46}, again, it is not clear what one should understand in addition to what the mind is (\textit{bisogna pure capire cosa è la mente} “it is also necessary to understand what the mind is”): presumably a lot of things, but none of them is immediately recoverable in the preceding context. The same applies to example \REF{ex:key:47}, but the functional development seems to be already a step forward – with \textit{pure} clearly marking emphasis on the illocutionary force. In these cases, two factors are responsible for this. First, the lack of a clear referent for the presupposition to provide the conditions for the inferring of a new function. Second, the occurrence of \textit{pure} in crucial points of the argumentation (in both \REF{ex:key:46} and \REF{ex:key:47} it marks the endpoint of a long argument) comes to be linked with the emphatic assertivity of the host sentence.\footnote{The emphatic assertivity expressed by \textit{pure} can be also noticed in this example: 
\ea 
Passi \textit{pure} per le conversioni tardive dopo una vita passata a meditare, ma uno straccio di idea uno deve \textit{pure} averla. 
\glt‘Late conversions after a life spent meditating are \textsc{ptc}\textbf{ }fine, but one must \textsc{ptc} have a shred of idea.’
\z
In this case \textit{pure} emphasizes the illocutionary force of the main sentence (declarative sentence), which is presented as a contrast (it is introduced by \textit{ma} ‘but’) to a preceding sentence. Quite interestingly, the preceding sentence – which represents a concessive premise to the main sentence – is also marked by \textit{pure}. Thus, the example displays both a concessive \textit{pure} and an illocutive \textit{pure}.}

In these kinds of context, the pragmatic effect of \textit{pure –} by the absence of clearly-identified presupposition – ends up stressing the force of the speech act it has scope over. This is illustrated by the sequence below.

\ea%48
    \label{ex:key:48}

          this is \textit{also}\textsuperscript{add} the case >

this \textsc{is} \textit{also}\textsuperscript{add/ill} the case >

this \textsc{is} \textit{also}\textsuperscript{ill} the case
    \z

In order to further enrich this point, I will compare the behavior of \textit{pure} with \textit{anche} ‘also’, which is an additive focus adverb as well. In contemporary Italian, no conventionalized modal function of \textit{anche} is attested, but several contexts can be found, in which a reading as an additive focus adverb is – at least – problematic. I will examine a couple of these contexts to illustrate how the managing of contextual inferences in interaction can pave the way to the emerging of new (modal) functions.

The first example is the use of \textit{anche} in \REF{ex:key:49}, taken from a television advertisement and discussed by \citet[185–186]{Andorno2003}. To correctly contextualize the utterance, the following scenario must be imagined. It is nighttime, and a robber is communicating his terms to the police from inside a bank with a megaphone. During the negotiation, a man in his pajamas looks out of a window and warns the police to shut up and let the people sleep. Replying to the disapproving look of the police chief, an inspector says with a conciliatory tone:

\ea%49
\label{ex:key:49}
Italian \citep[186]{Andorno2003}\\
\gll Sono      \textit{anche}  le              tre.\\
be:3\textsc{pl} also \textsc{art} three\\
\glt ‘It’s \textsc{ptc} three in the morning.’
\z

In this example, \textit{anche} has a mitigating effect and normal focus-particle interpretation is excluded, since it is not possible to identify alternative values to a constituent in focus (as suggested by the unacceptable paraphrases like “It’s three a.m. and it’s also some other time” or “It’s three a.m. and it’s cold”). In this way, the additive semantics (that is, the presuppositional potential) of \textit{anche} is at odds with the impossibility of recovering a valid presupposition in the preceding context. Yet \textit{anche} keeps its value of additive particle, but the additive value serves the function of argumentative operator. Depending on the context, it can support the argument of the man in pajamas (“We have good reason to go on with our work, but we have to consider that it’s three in the morning”) or accept the possible reasons of the detective (“Actually it’s three in the morning, but this is only one of the facts we should consider in such a circumstance”). In this way, \textit{anche} doesn’t evoke alternative focus values, but alternative utterances and a rough paraphrase of the utterance could be: “Among the different things we can say, we have to say that it’s three in the morning”.\footnote{In order to explain this use of \textit{anche}, a process of syntactic reanalysis could be posited – from focus modifier to utterance modifier – and a pragmatic re-use of the adverb for argumentative purposes. On the other hand, it doesn’t appear strictly necessary to posit a second meaning for the adverb since its mitigating effect could be explained through the interaction of its additive semantics and the particular context of interaction, that is, taking into consideration the background of possible propositions that are at issue in the common ground and the inferences that the interlocutors can draw about the respective mental states. In this perspective, echoing \citet[31–33]{Smet2014}, example \REF{ex:key:49} could be better described as a \textit{hybrid} use of \textit{anche}, supported by some degree of structural indeterminacy – that is, it could be assigned both to the layer of information structure and to the layer of illocution.}

A second illocutionary context in which \textit{anche} shows a non-prototypical use is exemplified by the directive in \REF{ex:key:50}:

\ea%50
    \label{ex:key:50}

          Italian [from a chat group]

A:  Rob, passo a prenderti?

B:   Ok! Mi faccio trovare a pozzo per le 9 e venti circa

  A:   Fai \textit{anche} 25 che Mic tanto 5 min ritarda

\glt A:  ‘Rob, should I pick you up?’

B:  ‘Ok! I will be at Pozzo [metro station] around twenty past nine’

A:  ‘You can \textit{also} be there at twenty-five past nine. Mic is 5 minutes late anyway’
    \z

In this case too, a normal focus-adverb interpretation is excluded, since it is not possible to identify alternative values to a constituent in focus: in example \REF{ex:key:50} \textit{anche} modifies the directive without having scope over a sentence constituent. Typically, this happens in contexts where an inference of invitation or permission for the interlocutor to do something is at play and the semantic contribution of \textit{anche} spans from suggesting a generic set of actions that the interlocutor can do (e.g. wait a bit longer) to the mitigation of the directive.

In both cases, the behavior of \textit{anche} is comparable with the modal uses of \textit{pure} discussed so far and the two adverbs seem to be largely interchangeable in this kind of context. Nevertheless, it cannot be said that this modal use of \textit{anche} has the same conventional status of the modal use of \textit{pure}. In this respect, the different uses of \textit{pure} can be described in terms of polysemy, while for \textit{anche} an explanation in terms of (contextual) polyfunctionality is probably more appropriate. At first sight, it seems that the \textit{degree of conventionalization} in language use represents the major difference between these constructions, rather than functional distinctions in a narrow sense.\footnote{However, it should be noted that – at least in example \REF{ex:key:50} – \textit{anche} performs a mitigating function, while the prototypical functions of \textit{pure} in assertives is an emphatic one. Furthermore, it could also be the case that \textit{anche} and \textit{pure} in directives activate different inferences, whereby the first operates rather on the part of the speaker (\textit{invitation} to do something) and the second on the part of the addressee (\textit{permission} to do something). This issue is however not clear, and it would need more research.}

\section{\textit{pure}: Closing remarks}
\hypertarget{Toc124860656}{}
The examples discussed in this chapter confirm the remarkable polyfunctionality that the adverb \textit{pure} covers in contemporary Italian. Even considering only the occurrences as a modal particle, it appears in different types of speech acts and it performs different pragmatic effects: (i) directives can be specified as invitations and permissions; (ii) the illocutionary force of assertives can be reinforced and specified with counter-expectational flavor; (iii) specific illocutions like hortatives and optatives can also be marked by \textit{pure}.

The development of new functions is favored by the specific features of this adverb, which (like most focus adverbs) displays great syntactic variability, can appear in several positions in a sentence, and plays a crucial role in the activation of presuppositions and the management of further contextual inferences. Moreover, the semantic operation it activates – namely, additivity – can easily go beyond the domain of sentence semantics and be transferred to pragmatic and discourse/textual phenomena, which is crucial for the development of modal-particle-like functions: anchoring the performance of a speech act to the common ground, modal particles relate the respective illocution to contextual conditions. Overall, the data on \textit{pure} seem to fit well in the models of grammatical categories and semantic change discussed above (\citealt{Hengeveld2004,Hengeveld2017}; \citealt{Narrog2012,Narrog2017}; \citealt{TraugottDasher2002}). On the one hand, they confirm the contiguity of the grammatical domains of modality and illocutionary modification. On the other hand, they confirm that semantic change involves a progressive scope increase at the interpersonal level (context-level functions) – expanding from the communicated content (use of \textit{pure} as a focus adverb) to the illocution (use of \textit{pure} as a modal particle).

A more detailed study of development paths would need further theoretical discussion about the domain of modality (and its subcategories), its relationship with neighboring semantic domains and – importantly – cross-linguistic comparisons. This is however outside of the scope of the present research. In this respect, additivity has been recently approached by works that – from a typological perspective – aimed at drawing a semantic map of this functional domain (\citealt{Forker2016}; \citealt{Faller2020}). Even without discussing the details of this map (and its different versions), the contiguity between additivity, epistemic modality, concessivity and discourse coherence has been pointed out by both papers. From this perspective, the data on \textit{pure} discussed above seem to find a broader correspondence (and validation). Neither \citet{Forker2016} nor \citet{Faller2020} include illocutionary modification as a functional category bordering additivity and/or modality. Nonetheless, \citet[85]{Forker2016} mentions that in some languages additive particles convey emphasis or are used to intensify meaning. Future research should further develop the semantic map of additivity to include illocutionary modification as a node in the map (and possibly using examples of \textit{pure} to assess its validity).

