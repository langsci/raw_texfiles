\chapter{Boundedness, approximation, illocution: A case study on \textit{un po’} ‘a bit’}\label{sec:7}
\addtocontents{toc}{\protect\enlargethispage{\baselineskip}}
\hypertarget{Toc124860657}{}\section{\textit{un po’}: Overview of the categories involved}
\hypertarget{Toc124860658}{}
The case study presented in this chapter concerns the expression \textit{un po’} ‘a little, a bit’, which is widely used in Italian as a quantifier in pseudo-partitive constructions and as an adverbial degree modifier. Throughout the theoretical discussion and the data analysis, I will try to assess if some of its secondary uses can be traced back to the grammatical category of \textit{illocutionary modification} as defined above.

As far as I am aware, no work has been devoted specifically to constructions featuring \textit{un po’} so far. Moreover, compared to the previous chapter on \textit{pure}, there are other difficulties. In the case of focus adverbs, several works have been devoted to their context-level uses: thus, the link between focus adverbs, the marking of information structure and illocutionary modification represents a well-established theoretical point (\citealt{König1991}: 165–176). In the case of (pseudo-) partitive constructions this link has been studied less – or maybe it is less relevant for this grammatical category. However, some works have contributed to the study of the grammaticalization paths involving similar constructions (see for instance \citealt{Traugott2008}) and to tracing a map of the grammatical categories covered by the semantic domain of partitivity (\citealt{LuraghiHuumo2014}). These observations represent a useful starting point to investigate further pragmatic developments.

In this section an overview of the main functions of \textit{un po’} will be provided. In doing this, I will highlight how its functional range covers different domains and linguistic categories, from partitivity to verbal aspect, approximation and illocutionary modification.

\subsection{Pseudo-partitive constructions and degree adverbs}
\hypertarget{Toc124860659}{}
To approach the description of \textit{un po’}, I will start from the semantic domain of partitives – since this expression is often used in pseudo-partitive constructions and is functionally close to this semantic domain. Adopting the perspective of \citet{LuraghiHuumo2014}, the understanding of partitives relies on the notions of \textit{indefiniteness} and \textit{quantification}, including forms which have scope either over the nominal phrase or the verb phrase.

\hspace*{-1pt}Partitives represent a quite heterogenous category. Some languages have a ded\-icated partitive morphological case (for instance Finnish, Estonian and Basque), providing a good case for a formal definition of partitives. Other languages have a different array of formal means to express the same function: other case markers, adpositions, articles, and verbal morphologies. Among Romance languages, French and Italian feature so-called partitive articles, which are formed with the genitive preposition plus the definite article.\footnote{Diachronically, such articles can be shown to have originated within partitive constructions. For this reason, the label \textit{partitive article} is still used, even though these articles have little left to do with partitivity (see for instance \citealt{Stark2007}; \citealt{CarlierLamiroy2014}).} Functionally, the marking of indefiniteness (and non-specificity) is considered a defining feature of partitive case markers.

Partitive case markers can also be used to express part-whole relations, but this is not always the case – as they are often expressed by \textit{partitive constructions} such as English \textit{a piece of that cake} (see \citealt{Koptjevskaja-Tamm2006}; \citealt{Keizer2007}), indicating a part of a given whole. This generates some confusion in the literature: indeed, the term \textit{partitive} is most often used to refer to partitive constructions, that is, part-whole constructions. Partitive cases can also be found in such constructions, but not necessarily. The difference is exemplified by examples \REF{ex:key:51} and \REF{ex:key:52}:

\ea%51
    \label{ex:key:51}
           Dutch (\citealt{LuraghiHuumo2014}: 2)\\
\gll Fred   at     van   de     aardbeien.\\
Fred   eat:\textsc{pst.3sg}   of   \textsc{art.pl}   strawberry:\textsc{pl}\\

\glt ‘Fred ate of the (previously identified, belonging to a given set) strawberries.’
    \z % you might need an extra \z if this is the last of several subexamples

\ea%52
    \label{ex:key:52}
          Finnish (\citealt{LuraghiHuumo2014}: 2)\\
\gll Elmeri   löys-i     mansiko-i-ta.\\
Elmer   find-\textsc{3sg.pst}   strawberry\textsc{{}-pl-par}\\
\glt ‘Elmer found some (i.e. and indefinite quantity of not previously identified) strawberries.’
    \z

While the prepositional phrase \textit{van de aardbeien} ‘of the strawberries’ in \REF{ex:key:51} is a partitive construction and indicates a partition of a previously identified whole, the partitive NP \textit{mansikoita} ‘strawberries’ in \REF{ex:key:52} basically indicates indefiniteness, and does not refer to a part of a previously identified whole. This way, partitive constructions indicate a part of a given whole; partitive markers, instead, typically convey, at least in some contexts, the meaning of indefiniteness, which is not characteristic of part-whole relations.

Another distinction must be made between partitive case markers and pseudo-partitives. Partitives refer to a part/subset of a definite superset, while pseudo-partitives are generally taken to refer to an amount or quantity of some indefinite substance: they quantify over the kind of entity indicated by the nominal head of the phrase. In (proper) partitive constructions, the prepositional phrase embeds a (generally definite) nominal, as Italian \textit{due degli studenti} ‘two of the students’. Pseudo-partitive constructions feature the same preposition, this time taking a bare nominal complement, as Italian \textit{un bicchiere d’acqua} ‘a glass of water’. With reference to pseudo-partitives constructions, \citet{Keizer2007} observes:

\begin{quote}
Each of the differences observed can be accounted for by assuming that, unlike in partitives, the second noun in a pseudo-partitive construction does not form an embedded NP, i.e. that the second nominal element is not an independently referring expression. […] It seems therefore plausible to analyse pseudo-partitives as simple NPs headed by the second noun, with the first noun being part of a complex determiner (or quantifier). \citep[111]{Keizer2007}
\end{quote}

Getting to the focus of this chapter, the Italian element \textit{un po’} can be defined in some of its uses as quantifier in pseudo-partitive constructions. An example is represented by \REF{ex:key:53}:

\ea%53
    \label{ex:key:53}

          [KIParla Corpus - BOD2017]

ciao // ciao // ti chiami // francesco // okay francesco allora ehm io volevo farti \textit{un po’} di domande // eh innanzitutto volevo chiederti dove vivi attualmente // a bologna

\glt ‘hi // hi // your name is // Francesco // OK Francesco well uhm I wanted to ask you \textit{some} questions // uh first of all I wanted to ask you where you’re currently living // in Bologna’
    \z % you might need an extra \z if this is the last of several subexamples

Beside the use in pseudo-partitive constructions, \textit{un po’} displays also an adverbial use, acting as a degree modifier on adverbs \REF{ex:key:54a}, adjectives \REF{ex:key:54b} and verbal phrases \REF{ex:key:54c}:

\ea%54
    \label{ex:key:54}

          Italian [\url{http://www.treccani.it/vocabolario/poco}]

\ea  \label{ex:key:54a} un po’ più, un po’ meno; un po’ meglio, un po’ peggio;    l’ho fatto un po’ alla svelta

\glt ‘a little more, a little less; a little better, a little worse; I did it a bit quickly’

\ex \label{ex:key:54b} mi sento un po’ stanco; mi sembri un po’ pallido; è un ragazzo un po’ strano

\glt ‘I feel a little tired; you’re looking a little pale; he’s kind of a weird boy’

\ex \label{ex:key:54c} m’ha fatto un po’ ridere; mi ha fatto stare un po’ in ansia; fa un po’ caldo qui dentro

\glt ‘it made me laugh a bit; it made me feel a bit anxious; it’s a bit warm in here’
    \z
\z

This fits well into the grammaticalization path proposed by \citet{Traugott2008} for the diachronic development of these expressions:

\ea%55
    \label{ex:key:55}

           pre-partitive >

partitive >

quantifier >

degree modifier >

free adverb\footnotemark
\z
\footnotetext{The use as a free adverb corresponds to the holophrastic use of \textit{un po’}:
\ea

            A: sei stanco?   

B: un po’ 

\glt    A: ‘are you tired?’  
        
        B: ‘a bit’
\z
}

In its use as an adverbial modifier, \textit{un po’} doesn’t show a uniform semantics – especially when occurring in a post-verbal position. As it has already been noted for focus adverbs, adverbial modifiers in this position often have wide scope: although they may clearly have a syntactic association with the verb, the actual semantic scope may be over the object of the verb or over the entire predicate.

\citet[240–248]{Budd2014} offers a description of the functional domains of post-verbal partitives in Oceanic languages which may provide useful hints also for \textit{un po’}. When post-verbal partitives occur with an object NP there are in theory two possible readings, which are distinguishable to varying degrees depending on the semantics of the verb and of the object NP, as well as the discourse context (potentially, there is therefore a degree of ambiguity in some cases). In the first reading, only part of the NP’s referent is affected, while in the second reading, the object NP’s referent is partially affected. In Italian – although it is hard to find contexts showing a real ambiguity (the semantics of the verb or the presence of definite articles is usually enough to disambiguate) – \textit{un po’} can occur in both contexts. Examples \REF{ex:key:56} and \REF{ex:key:57} respectively illustrate these two readings.

\ea%56
    \label{ex:key:56}

          Italian \textsc{[affectedness} \textsc{of} \textsc{a} \textsc{part} \textsc{of} \textsc{an} \textsc{entity]}

\ea  Sposto \textit{un po’} di libri.

\glt ‘I move \textit{some} books around.’

\ex Racconto \textit{un po’} di storia.

\glt    ‘I tell \textit{some} of the story.’
\z
\z

\ea%57
    \label{ex:key:57}

          Italian \textsc{[partial} \textsc{affectedness} \textsc{of} \textsc{an} \textsc{entity/action]}

\ea Sposto \textit{un po’} i libri.

\glt ‘I move the books around \textit{a little}.’

\ex  Faccio \textit{un po’} di spesa.

\glt ‘I do \textit{some} shopping.’
    \z
    \z

In the first case, the complete accomplishment of one part of the whole is expressed, while in the second the incomplete accomplishment of a whole action is expressed. These two readings activate different inferences, possibly resulting in functional developments. The first path profiles the partiality of a referent and it can lead to the expression of non-specificity and indefiniteness. The second path profiles the incompleteness of an action (or an attempt to perform an actions) and it can lead to interpretations of aspectual nuance, that is, imperfectivity (see \citealt{LuraghiKittilä2014}: 56–58).

Quite surprisingly, \citet[545–547]{Budd2014} also reports examples of partitive markers apparently expressing the opposite aspectual meaning, that is perfectivity. In this third development path – arising from the reading exemplified by \REF{ex:key:57} – the meaning develops from “There is a certain amount of such an action” (partitive interpretation) to “Such an action has been done” (perfective interpretation). Finally, \citet[547–548]{Budd2014} reports examples of partitive markers giving a non-assertive tone to the utterance they appear in – such that an element of politeness is associated with their use: in requests and commands particularly, they have the effect of making the demand seem less impertinent or onerous. This raises the issue of a possible illocutive use of partitive markers/partitive-related constructions.

\subsection{Illocutionary modification}
\hypertarget{Toc124860660}{}
These last observations suggest that partitive markers can develop non-referen\-tial uses in the pragmatic domain. In the case of \textit{un po’} there is no previous literature available, but comparable constructions have been studied for other languages, especially English, and also Italian. \citet{Traugott2008}, as mentioned above, is a key reference work that studies the grammaticalization of \textit{\textsc{np} of \textsc{np}} patterns (for a Romance/Italian perspective see \citealt{Masini2016}; \citealt{Mihatsch2016}). \citet{Traugott2010} touches on the same issues from the perspective of (inter)subjectification – also providing examples of \textit{a bit}, which is close to Italian \textit{un po’}. \citet{NeelsHartmann2018} have studied the German constructions \textit{ein bisschen} (‘a bit\textsubscript{\textsc{dim}}’) and \textit{ein wenig} (‘a little’). Other constructions are comparable not on the basis of structural similarity, but on the basis of functional affinities, for example the Italian item \textit{un attimo} ‘an instant’ \citep{Voghera2017} and the German particle \textit{mal} ‘once’ (\citealt{KönigLi2018}). In this way – even without previous literature specifically dedicated to it – a comparative perspective can provide some hints for the functional analysis of \textit{un po’}. Nevertheless, illocutionary-flavored uses of this item are mentioned even in dictionaries:

\begin{quote}
A very peculiar use is showed by \textit{un po’} in imperative sentences, or sentences otherwise containing an order, an invitation, a request, where it sometimes has a mitigating value, sometimes a threatening tone […] in other cases it expresses resentment […] in still other cases it corresponds to an exclamation of encouragement. (\url{http://www.treccani.it/vocabolario/poco})\footnote{My translation of the original quote in Italian, which includes some examples: “Un uso particolarissimo ha \textit{un po’} in frasi imperative o comunque contenenti un ordine, un invito, una richiesta, nelle quali ha talora valore attenuativo, talora invece ha tono di minaccia: \textit{vedi un po’ tu se ci riesci}; \textit{mi dica un po’ cosa farebbe lei al mio posto}; \textit{vieni un po’ qua}; \textit{dimmi un po’: chi è che t’ha insegnato a rispondere così?}; \textit{dica un po’ lei}, \textit{sì}, \textit{lei!}; in altri casi esprime risentimento: \textit{senti un po’ che discorsi mi viene a fare!}; in altri ancora equivale a un’esclamazione d’incoraggiamento: \textit{indovina un po’ che cosa t’ho portato}; \textit{sentiamo un po’ ciò che vorresti}; \textit{riferiscimi un po’ quello che hai visto} (meno com. con quest’uso la forma \textit{poco}: \textit{guarda un poco qui}; \textit{dimmi un poco}).”}
\end{quote}

This quote acknowledges a modal use of \textit{un po’} in Italian. The pragmatic effect is described as attributable to two functional domains: the modification of speech acts (mitigating value, threatening tone) and the expression of intersubjective emotional attitudes (resentment, encouragement). Despite providing useful information, these descriptions don’t seem to grasp the central features of the modal uses of \textit{un po’} – for which a more detailed description is needed. Throughout the analysis I will consider two main points as decisive for a broad classification: the kind of speech act the adverb occurs in and the kind of pragmatic effect performed. The expression of intersubjective emotional attitudes – which has not played an important role in my analysis so far – will be (partially) included in the speech-act classification. The modal use of \textit{un po’} is particularly clear in directive speech acts, especially those coded on imperative sentences.

\ea%58
    \label{ex:key:58}

          [LIP corpus – Naples A1]

E:   Anna metti \textit{un po’} là per favore

B:   qua Vincenzo

E:   dove sta la borsa Franco mettiti \textit{un po’} più vicino a me va



\glt E:  ‘Anna put it \textsc{ptc} there please’

B:  ‘here Vincenzo’

E:  ‘where is the bag Franco move \textit{a bit} closer to me come on’
    \z % you might need an extra \z if this is the last of several subexamples

In the array of directive meanings, example \REF{ex:key:58} sounds like a request. In such a context, \textit{un po’} contributes to specifying the directive as a request, mildly downtoning the illocutionary force expressed by the imperative. Example \REF{ex:key:58} contains two occurrences of \textit{un po’}, expressing different functions. Here, the difference between its modal use (\textit{metti un po’ là} ‘put it a bit there’) and its adverbial use (\textit{un po’ più vicino} ‘a bit closer’) can be clearly noticed. In the first occurrence it operates on the verbal predicate – specifically, on the illocutionary layer, while in the second occurrence it modifies an adjective, functioning as a degree modifier construction. Besides mild requests, \textit{un po’} also appears in directives which rather express incitements and stressed requests:

\ea%59
    \label{ex:key:59}

          Italian
\ea \label{ex:key:59a} e levati \textit{un po’} questi occhiali

\glt ‘take off \textsc{ptc} these glasses’

\ex \label{ex:key:59b} e levati \textit{un po’}

\glt ‘get out \textsc{ptc} of here’
    \z
\z

Similar examples are rare in corpora of spoken language, but easily retrievable on the internet. The verb \textit{levarsi} has different meanings when used transitively \REF{ex:key:59a} or intransitively \REF{ex:key:59b}: in both cases \textit{un po’} seems to strengthen the force of the directive. In particular, in example \REF{ex:key:59a}, \textit{levati un po’ questi occhiali} should not be understood as ‘take off your glasses a bit (and let me glimpse your eyes)’, but rather as ‘come on, take off these glasses!’. Comparing \REF{ex:key:58} and \REF{ex:key:59}, it seems that the broader illocutionary context has a decisive impact in determining which kind of pragmatic effect \textit{un po’} plays in a directive, an issue I will come back to. Closely related to the directives (see the discussion in \chapref{sec:6}), \textit{un po’} can also appear in hortatives:

\ea%59
    \label{ex:key:60}
 [LIP corpus – Florence C5]

allora da qui dove andrà vediamo \textit{un po’} che ne so a Berlino forse

\glt ‘well from where will she go let’s see \textsc{ptc} I don’t know to Berlin maybe’
\z

Even if not acknowledged by the dictionary quote above, context-level uses of \textit{un po’} cover also another illocutive domain, namely assertions. In example \REF{ex:key:61} below \textit{un po’} operates on a non-gradable predicate, so that a value as degree modifier seems to be excluded. Significantly, \textit{un po’} occurs with a verb in the conditional mood – that is, an already modalized sentence. One possibility might be that, in such a context, \textit{un po’} further contributes to specify the assertion as a suggestion, giving to the utterance a non-assertive tone. In this perspective, it could be interpreted as a politeness element.

\ea%61
    \label{ex:key:61}

          [LIP corpus – Milan A11]

A:   subito scusa ma stando così le cose lui non deve chiedere scusa a nessuno

C:   questo qui è un atto di lei ha chiamato in causa il diritto quindi lui

A:   non deve chiedere scusa infatti

C:   cioè lei ha scelto per la risoluzione di diritto invece che

B:   secondo me lei ha scelto una la risoluzione di rappresaglia veramente

A:   esatto anch’ io la chiamerei \textit{un po’} rappresaglia

B:   cioè poteva anche semplicemente esprimergli esprimergli il suo disprezzo

\glt
A:  ‘right now sorry but if things are like this he must not apologize to anybody’

C:  ‘this is an action for her part she called into question the law so he’

A:  ‘he must not apologize indeed’

C:  ‘I mean she decided for a legal resolution instead of’

B:   ‘In my opinion she decided for a reprisal resolution actually’

A:  ‘That’s right I would call it \textsc{ptc} reprisal as well’

B:  ‘I mean she also could simply express express her contempt to him’
    \z % you might need an extra \z if this is the last of several subexamples

I will further analyze these constructions in the next section. Now, I will introduce a brief comparison that could help the subsequent analysis. In fact, even if there is no available research concerning \textit{un po’}, \citet{Voghera2017} dedicated a short paper to a very similar lexical item, namely \textit{un attimo} ‘an instant’. She sums up the semantic development of \textit{un attimo} in this way:

\begin{quote}
Starting from the original temporal function, \textit{un attimo} developed multiple functions, which derive from a double path of functional expansion. Firstly, we can recognize a semantic and pragmatic path, which brings to the use of \textit{un attimo} as vague quantifier and then as hedge. Secondly, there is a path towards textual uses, which exploits the possibility of using \textit{un attimo} as alerter in some imperative constructions and then as focuser […]. \citep[1]{Voghera2017}
\end{quote}

According to this explanation, \textit{un attimo} developed from lexical expression indicating a small portion of time to quantifier indicating a small quantity in general (that is, the function of \textit{un po’} in its content-level use). 

At this point it spreads to further contexts of use following two different paths. On the one hand, it gets to express information/relational/discourse vagueness as a hedge; on the other hand, it functions as an attention-getter in imperative sentences and as an interjection (\citealt{Voghera2017}: 392–394).\footnote{\citet{Voghera2017} does not conduct a proper diachronic study, she rather interprets synchronic data in a dynamic way. In order to explain how vagueness expressions apply to different linguistic levels (proposition, speech act, discourse), she builds on \citet[58]{Caffi2007}, who says: “In other words, speakers can use referential vagueness to reduce both their commitment to the precision of denotation, hence of their reference act, and their epistemic endorsement of the truth of the proposition”.} Among the examples cited in the paper, I report three of them where \textit{un attimo} could be easily replaced by \textit{un po’}:

\ea%62
    \label{ex:key:62}

          Italian (\citealt{Voghera2017}: 392–394)\footnote{These examples correspond respectively to examples (33), (46) and (49) in Voghera (2017: 392–394). Notice that, in the first example, the first verb is modified by \textit{un attimo} while the second one is modified by \textit{un pochino} – the diminutive form of \textit{un po’ –} with a similar pragmatic effect.}
\ea \label{ex:key:62a} insomma adesso ci pensa \textit{un attimo} e vede un pochino

\glt ‘well now he thinks \textit{a moment} about it and sees a bit’

\ex \label{ex:key:62b} non c’è male // senti \textit{un attimo} io ho chiamato papà in ufficio ma era già andato via

\glt ‘not bad // listen \textsc{ptc} I called dad in the office but he was already gone’

\ex \label{ex:key:62c} benissimo giriamo la carta e vediamo \textit{un attimo} quali itinerari proporreste

\glt ‘very good let’s turn the map and let’s see \textit{a moment} what routes you would suggest’
    \z
\z

As these examples show, the contexts of use are the same as identified for \textit{un po’}: assertions \REF{ex:key:62a}, directives expressed by imperative sentences \REF{ex:key:62b} and hortatives \REF{ex:key:62c}. However, the terminology and the categories used by \citet{Voghera2017} for her analysis of \textit{un attimo} are rather different from the ones adopted in this work. Concerning the overall classification, speech act theory is not referred to for the identification of the contexts of use. Concerning the analysis of the functions, terms like \textit{informational}/\textit{relational}/\textit{discourse vagueness}, \textit{hedge} and \textit{alerter} cannot be easily integrated in the present framework.\footnote{The term \textit{vagueness} has been used in many ways in linguistics and I don’t mean to sum them up: see for example \citet{JuckerEtAl2003} for vagueness in conversation. Useful hints on the term \textit{hedge} can be found in \citet{KaltenböckEtAl2010}.} Nevertheless, this brief comparison confirms also for Italian the existence of a development path which brings quantifiers to develop pragmatic functions in different kinds of speech acts.

The data analysis in the next section will try to further describe the pragmatic functions of \textit{un po’} and to assess to what extent the category of \textit{illocutionary modification} can be used to describe these uses. Before moving to the data, I will add something else on the semantic domain covered by (pseudo-)partitive constructions and grammatical categories relevant for their development paths, in order to place their context-level uses in a broader grammatical environment.

\subsection{The surroundings: Aspect, transitivity and verbal semantics}
\hypertarget{Toc124860661}{}
Following \citet{LuraghiHuumo2014}, I indicated the marking of indefiniteness and expression of part-whole relations as the core meaning of partitive markers and partitive constructions. Moreover, they can express secondary functions related to verbal semantics, such as non-assertive modality, imperfective aspect and low transitivity. To give an overview of the semantic domain covered by partitive markers and partitive constructions, I will refer mainly to \citet{LuraghiKittilä2014} who – besides a formal and functional typology of partitive markers – also outline their possible functional developments.

Among the functions of partitives related directly to verbal meanings, there is aspect marking (\citealt{LuraghiKittilä2014}: 38–40): in Estonian and Finnish, the partitive case contrasts with the accusative case; the partitive expresses imperfective aspect (and/or unbounded activities), while the accusative is associated with perfective aspect (and/or bounded activities). Linked to this, partitive coding is also associated with lower degrees of transitivity.\footnote{Imperfective aspect (on-going events, or events that were not completed successfully) is functionally directly related to low transitivity (see \citealt{HopperThompson1980}).} According to \citet[40–46]{LuraghiKittilä2014}, low transitivity manifests itself in three main ways: as partial affectedness  (which means that only a part of an entity is affected); as association with predicates ranking inherently lower for transitivity (for instance mental verbs, verbs of cognition and experience); and as a low degree of agency associated with the agent of the action.

Clearly, \textit{un po’} does not autonomously mark any of these functions in Italian, but some examples show a certain closeness with the semantic domains described so far:

\ea%63
    \label{ex:key:63}

           Italian
\ea\label{ex:key:63a}  vedo \textit{un po’} che fare   [\textsc{imperfective} \textsc{aspect]}

\glt ‘I’ll see \textsc{ptc} what to do’

\ex\label{ex:key:63b}  ha fatto \textit{un po’} un casino  [\textsc{low-transitivity}]

\glt ‘he did \textsc{ptc} a mess’
    \z
\z

In these examples, \textit{un po’} does not express a quantificational measure over the object of the predicate, but it contributes to mark specific shades of the verbal aspect or to refine the verbal semantics, for instance the degree of transitivity. The presence of \textit{un po’} in sentences like \REF{ex:key:63a} and \REF{ex:key:63b} can be related to the expression of an unbounded activity and to a low degree of agency, respectively. As I have already pointed out, these functions are particularly clear when partitive constructions appear after the finite verb form. \citet[56]{LuraghiKittilä2014} notice that “different inferences may arise from the occurrence of a partitive construction in the place of a direct object”. They trace two possible development paths, both starting from the meaning of partitive constructions (“A part of a referent undergoes the effects of an action/process”).

The first inference (“Only a part of a referent is involved”, expressing non-specificity) leads to indefiniteness: this path plays no role in the present analysis of \textit{un po’}. The second inference (“Action/process is partial”, expressing partiality) insists on the unboundedness of the event and leads to low transitivity and imperfectivity. This path seems to be relevant especially for the use of \textit{un po’} in assertions, as unboundedness is transferred from the semantic level to the speech-act level. Moreover, recalling the suggestion by \citet{Budd2014}, a third inferential path could be identified (“There is a certain amount of such an action”) which, insisting on the boundedness\footnote{On this category (cognitive before being linguistic), see among others \citet{Declerck1979}, \citet{Dahl1981}, \citet{Jackendoff1991} and \citet{Brinton1998a}.} of the event, leads instead to a perfective or punctual interpretation:

\ea%64
    \label{ex:key:64}

           Italian

guarda \textit{un po’} chi ha scritto queste parole

\glt ‘look \textit{a bit} who wrote these words’
\z

In imperative sentences like \REF{ex:key:64}, \textit{un po’} highlights the boundedness of the action expressed by the verb (perfective interpretation) or its punctual semantics, which is often an intrinsic feature of imperatives \citep[126]{Aikhenvald2010}. This could represent the onset of a development path – as the marking of boundedness is transferred from the propositional level to the speech act (or, in FDG terms, from the representational level to the interpersonal level). In particular, in sentences like \REF{ex:key:64} \textit{un po’} also contributes a mirative flavor (on mirativity, see \citealt{Delancey1997}; \citealt{HengeveldOlbertz2012}). This can be related to the perfective interpretation of imperative with \textit{un po’}, as the expression of surprise by definition concerns something that is accomplished, that is perceivable in its “perfectivity” (or at least that is perceived as such by the speaker). When \textit{un po’} conveys a mirative reading, it has always scope over a directive (\textit{ma pensa un po’!} ‘but guess what/imagine that!’, \textit{guarda un po’ chi arriva!} ‘look who’s coming!’), so the mirative value seems to develop as a sort of by-product of the context-level use of \textit{un po’} in directives – and in fact the conventionalized routines expressing surprise all feature an imperative with which the speaker asks the addressee to direct their attention to something (that, in particular, is surprising, and thus deserves to be noticed).

\section{\textit{un po’}: Corpus data}
\largerpage[2]
\hypertarget{Toc124860662}{}
In the last section, I gave an overview of the main functions of \textit{un po’} in contemporary Italian. Among the content-level uses, it can be used as a quantifier in pseudo-partitive constructions and as an adverbial degree modifier. In some contexts, this element also contributes to the expression of aspectual nuances of the predicate, related through inferential paths to the semantic domain of partitivity. Moreover, it displays uses linked to the illocutionary layer of the utterance. Two contexts of use have been identified, corresponding to different types of speech acts – assertives and directives – supposedly with different pragmatic effects. In both cases (assertives and directives) – regardless of the inferential path which is highlighted – the scope shift to the speech-act level may involve a reanalysis of the meaning of \textit{un po’}, which comes to mark new context-level functions. Now, through the analysis of data extracted from corpora – both from the spoken and the written language – I will provide a more detailed description of the contexts of use of \textit{un po’} and further discuss its modal functions.

\subsection{Spoken language}
\largerpage[2]
\hypertarget{Toc124860663}{}
I extracted 350 occurrences of \textit{un po’} both from the KIParla corpus and the LIP corpus, looking for tokens in post-verbal position. The next step was the manual annotation of the functions covered, following the categories identified in the last section. The extraction was mainly intended to gather a sample of corpus examples for the qualitative analysis – and not as a database for a quantitative analysis.\footnote{I repeat once more the disclaimer enunciated in \chapref{sec:6}. The extraction of data from KIP and LIP was aimed at building a general dataset for the subsequent analysis rather than at (quantitatively) comparing the data extracted from the first one with the data extracted from the second one. From this perspective, \tabref{tab:key:7.1} must be primarily read as an overall presentation of the data and not as a comparison of the two corpora. Moreover, the data displayed in \tabref{tab:key:7.1} cannot be compared fully with the data on \textit{pure} displayed by \tabref{tab:key:6.1} in \chapref{sec:6}. In particular, in the case of \textit{pure} I could extract all occurrences from both corpora, while in the case of \textit{un po’} – which is much more frequent – I couldn’t extract all occurrences and I limited myself to the arbitrary count of 350 occurrences for each corpus.} Nevertheless, the annotation of the functions allows us to make a (rough) count of the functional distribution in the samples. \tabref{tab:key:7.1} gives an overview of the functions covered by \textit{un po’} (absolute and relative frequencies).

\begin{table}[h]
\begin{tabularx}{0.8\textwidth}{Xrr@{\qquad}rr}
\lsptoprule
 & \multicolumn{2}{c}{KIP } & \multicolumn{2}{c}{LIP} \\
 \cmidrule(lr){2-3}\cmidrule(lr){4-5}
 & abs & rel & abs & rel\\
 \midrule
\textsc{pseudo-partitive} & 49 & {0.14} & 58 & {0.16}\\
\textsc{adverbial} \textsc{modifier} & 252 & {0.72} & 201 & {0.57}\\
\textsc{mp} \textsc{directive} & 5 & {0.01} & 46 & {0.13}\\
\textsc{mp} \textsc{optative/hortative} & 6 & {0.01} & 26 & {0.07}\\
\textsc{mp} \textsc{assertive} & 35 & {0.10} & 13 & {0.04}\\
\textsc{other} & 3 & {<0.01} & 6 & {0.01}\\
\midrule
Total & 350 &  & 350 & \\
\lspbottomrule
\end{tabularx}
\caption{\label{tab:key:7.1} Distribution of the functions of {un po’} in KIP and LIP}
\end{table}

\clearpage



The table shows some disparities in the distributions. While the frequency of \textit{un po’} in pseudo-partitive constructions and as an adverbial degree modifier are similar, the distribution of the illocutive uses is quite different. This depends possibly on the samples considered, which feature texts reflecting different communicative situations and sociolinguistic settings in a non-balanced way.\footnote{However, it is worth noting that – compared to the relatively stable distribution of the content-level uses – the context-level uses display more variation in frequency. This is in accordance with their behavior, which is more influenced by the contextual features of the communicative environment (participants, kind of interactions, topics of discussion). In other words, content-level uses of \textit{un po’} have a greater probability of appearing across different communicative situations and sociolinguistic settings, while context-level uses of \textit{un po’} heavily depend on them.}

I will now go through the labels used for the annotation. The first two labels refer to the content-level uses of \textit{un po’}, namely the use in pseudo-partitive constructions (\textsc{pseudo-partitive}) and the use as an adverbial degree modifier (\textsc{adverbial} \textsc{modifier}). As they have both been described in the previous section, I will not add much. The analysis of the sample reveals that the adverbial use of \textit{un po’} is by far the most frequent one. When expressing this function, \textit{un po’} modifies gradable words/concepts: adjectives \REF{ex:key:65}, adverbs \REF{ex:key:66} and verbal phrases \REF{ex:key:67}.

\ea%65
    \label{ex:key:65}

          [KIParla corpus – BOA3013]

mi sa che sei \textit{un po’} stanca // chissà perché poi // c’è freddo // il freddo stanca //

\glt ‘I think you’re \textit{a bit} tired // I wonder why // it’s cold // cold makes you tired //’
\z

\ea%66
    \label{ex:key:66}

          [KIParla corpus – TOD2011]

e quindi sono mh cose molto molto interessanti e molto profonde da da capire // impari a gestirti \textit{un po’ un po’} meglio // \textit{un po’} meglio sì //

\glt ‘and then they’re uh really interesting things and quite deep to to understand // you learn how to handle yourself \textit{a bit a bit} better // \textit{a bit} better yeah //’
\z


\ea%67
    \label{ex:key:67}

          [KIParla corpus – TOD2011]

// e mh e e e poi anche per eh uscire la sera magari per svagarsi \textit{un po’} dopo tutta la giornata passata in università //

\glt ‘// and uh and and and then also to uh go out in the evening maybe to have \textit{some} fun after spending the whole day at university //’
\z

Example \REF{ex:key:68} below is a case of \textit{un po’} in a pseudo-partitive construction, expressing a part-whole relation (referring to an indefinite referent).

\ea%68
    \label{ex:key:68}

          [KIParla corpus – TOD2012]

// però sì cioè c’erano \textit{un po’} di rumori appunto per lavori del comune quindi // in definitiva forse // quello quella dove sono stata meglio a livello di rumori // è stato è stato a san donato //

\glt ‘// but yeah I mean there were \textit{some} noises right for the works of the municipality so // in the end maybe // the one the one where I’ve felt better in terms of noises // was was in san donato //’
\z

Next to these, modal uses can be found. They have all been introduced in the last section: I will now give more examples and deepen the analysis. The overall classification is based on the kind of speech act in which \textit{un po’} occurs. Three of them have been identified: directives (\textsc{mp} \textsc{directive}), hortatives (\textsc{mp} \textsc{optative/hortative}) and assertions (\textsc{mp} \textsc{assertive}). In directive speech acts, \textit{un po’} operates on imperative verb forms (2nd person singular and plural). Looking at the examples extracted, it is perspicuous that a relatively small set of verbs appear with a certain frequency in these kinds of construction as seen in \tabref{tab:key:7.2}.

\begin{table}
\begin{tabularx}{0.5\textwidth}{Xrr}
\lsptoprule
 & {KIP} & {LIP}\\
 \midrule
\textit{sentire} ‘to hear’ & {–} & {5} \\
\textit{pensare} ‘to think’ & {1} & {5} \\
\textit{vedere} ‘to see’ & {–} & {9} \\
\textit{guardare} ‘to look’ & {1} & {12} \\
\textit{dire} ‘to say’ & {1} & {5} \\
other & {2} & {10} \\
\midrule
{Total} & {5} & {46}\\
\lspbottomrule
\end{tabularx}
\caption{\label{tab:key:7.2} Imperative verb forms occurring with \textit{un po’} in KIP and LIP}
\end{table}

In the sample from the LIP corpus, five verbs occur in the majority of examples (36 out of 46): three perception verbs (\textit{sentire} ‘to hear’, \textit{vedere} ‘to see’, \textit{guardare} ‘to look’), a psychological verb (\textit{pensare} ‘to think’), and the verb \textit{dire} ‘to say’. The other occurrences are represented by single (or double) occurrences of different verbs (for instance \textit{fare} ‘to do’, \textit{mettere} ‘to put’, \textit{chiudere} ‘to close’, \textit{chiamare} ‘to call’, \textit{provare} ‘to try’). The sample from the KIP corpus contains far less examples of this kind. A closer look at these examples and their conversational context suggests that constructions with imperative verb forms and \textit{un po’} could be grouped in three different sets. In a first set of examples, \textit{un po’} operates as a mitigating particle: it downtones the illocutionary force of the speech act, specifying directives as requests or invitations.

\ea%69
    \label{ex:key:69}

          [LIP corpus – Rome C9]

A:   volevo chiedere sempre a Manuela che cosa dunque eh dunque Calvino si è occupato quindi del problema della fiaba eccetera volevo sapere ha fatto eh ha prodotto un’opera interessante sulla fiaba e eh non un testo critico no una

B:  una raccolta

A:   una raccolta di fiabe intitolate

B:   Fiabe italiane

A:   Fiabe italiane ah che praticamente ecco raccontami \textit{un po’} di cosa cosa sono

\glt
A:  ‘I wanted to ask Manuela again what well uh well Calvino then worked on the issue of the folktale and so on I wanted to know he made uh he produced an interesting work on folktales and uh not a critical work no a’

B:  ‘a collection’

A:  ‘a collection of folktales called’

B:  ‘Italian folktales’

A:  ‘Italian folktales right that basically well tell me \textsc{ptc} what what they are’
    \z % you might need an extra \z if this is the last of several subexamples

In the context of an oral exam at university, \textit{un po’} downtones the force of the professor’s request (\textit{raccontami un po’} ‘tell me a bit’). Obviously, the professor is not asking the student to give a partial answer to question, rather they perform the speech act in a way which should not sound too overbearing: a possible paraphrase could be “I perform a directive speech act \textit{a bit}”.

Another example is \REF{ex:key:70} below. A man asks his partner to phone a friend (\textit{chiama un po’} ‘call a bit’). Again, \textit{un po’} does not operate on a referential level – there is no point in giving a partial phone call – but on the illocutionary level: it characterizes the directive as a mild request, as if to say “I am asking you this in a gentle way” (another contextual clue is represented by the expressions at the beginning of the utterance: \textit{amore ti prego} ‘sweetie please’).

\newpage
\ea%70
    \label{ex:key:70}

          [LIP corpus – Milan B34]

B:   amore ti prego chiama \textit{un po’} XYZ perché mi sa che te ne sei \textit{un po’} dimenticata

A:   sì me lo sono abbandonato no l’ho chiamato giovedì poi però io ero fuori

\glt
B:  ‘sweetie please call \textsc{ptc} XYZ because I think you forgot \textsc{ptc} about that ’

A:  ‘yes I forgot about it no I called him on Thursday but then I was out’
    \z % you might need an extra \z if this is the last of several subexamples

In a second set of examples, \textit{un po’} performs a rather different function: it operates on commands, requests, and demands, enriching them with a specific incitement or boosting flavor. In example \REF{ex:key:71} the speaker asks her mother – for the first time in twenty years – to cook beans. The presence of \textit{un po’} in the utterance (\textit{fammi un po’ i fagioli} ‘cook me \textsc{ptc} some beans’), is apparently not linked to the mitigation of the directive, it rather marks it as an unexpected proposal. Example \REF{ex:key:72} is even clearer. The speaker is complaining about the quality of a photo he received from a friend (it is crooked and the screen in the picture has dust on it): imagining an exchange of words with him, he utters a marked directive (\textit{togli un po’ quel dito di polvere} ‘take off \textsc{ptc} that inch of dust’). In this case, along with a specific flavor of casualness/unexpectedness, the utterance clearly sounds like a stressed request.

\ea%71
    \label{ex:key:71} [KIParla corpus – TOD2009]

eh e infatti ha fatto strano anche a mia mamma perché sono tornata e avevo voglia di fagioli // cosa che non ho mai chiesto in vent’anni di vita // mamma fammi \textit{un po’} i fagioli

\glt ‘uh indeed my mum was also surprised because I came back and I felt like having beans // a thing that I never asked for in twenty years of life // mum cook me \textsc{ptc} some beans’
\z

\ea%72
    \label{ex:key:72}

          [KIParla corpus – BOA3021]

però nel senso almeno fai la foto allo schermo da davanti // non metà // non in diagonale // non con il flash togli \textit{un po’} quel dito di polvere

\glt ‘but I mean at least take the picture of the screen from the front // not the half // not crooked // not with the flash take off \textsc{ptc} that inch of dust’
    \z % you might need an extra \z if this is the last of several subexamples

These examples are closely linked to a third set of examples. This last set includes almost the totality of examples featuring the five verbs listed in \tabref{tab:key:7.2} above. Expressions like \textit{senti un po’} ‘listen a bit’, \textit{guarda un po’} ‘look a bit’ or \textit{pensa un po’} ‘think a bit’ are high-frequency imperatives further marked by \textit{un po’}. Together they represent highly-routinized directives displaying a non-compositional meaning. Rather than expressing a directive speech act, they are used as attention-getters, to highlight specific parts in the conversational flow or to segment discourse chunks. In other words, they have reached a discourse-marker status (see \citealt{Waltereit2002} on \textit{guarda} ‘look’).\footnote{As discussed in the previous section, these constructions often convey a mirative flavor.}

\ea%73
    \label{ex:key:73}

          [KIParla corpus – TOD2003]

// mai io tra l’altro ho anche paura dei cani il mio fidanzato vorrebbe un canelupo cecoslovacco \textit{pensa un po’} // costa il lupo cecoslovacco //

\glt ‘// never by the way I’m even afraid of dogs my boyfriend would like to have a Czechoslovakian wolf dog \textit{can you imagine that} // a Czechoslovakian wolf is expensive //’
    \z % you might need an extra \z if this is the last of several subexamples

\ea%74
    \label{ex:key:74}

          [KIParla corpus – TOD1015]

è l’unico luogo della bibbia in cui sembra che insomma sta trinità c’è // solo che erasmo si rende conto che \textit{guarda un po’} proprio quel pezzetto nelle versioni greche originali non c’era

\glt ‘it’s the only passage of the bible where it seems that well there is this trinity // but Erasmus realizes that \textit{guess what} exactly that tiny piece was not included in the original Greek versions’
    \z % you might need an extra \z if this is the last of several subexamples

The difference between these two uses can be observed in example \REF{ex:key:75}, featuring both a routinized directive with discourse-marking function (\textit{senta un po’} ‘listen \textsc{ptc}’) and a marked imperative (\textit{faccia un po’ il conto} ‘you do the math \textsc{ptc}’). The first element is used to introduce a new utterance drawing the attention of the interlocutor. The second occurrence is an actual directive expressing a boosted request.

\ea%75
    \label{ex:key:75}

          [LIP corpus – Florence A9]

A:  sì appunto ora ecco è una cosa che devo verificare lei è in pensione non è che

B:  \textit{senta un po’} noi pensionati abbiamo diritto all’ esenzione

A:   voi pensionati avete diritto all’esenzione del ticket purché il reddito non superi i diciotto milioni l’anno

B:   un milione e due un milione e tre \textit{faccia un po’}\textbf{ }\textit{il conto}

\glt
A:  ‘yeah exactly now well it’s a thing that I have to verify you are retired isn’t it’

B:  ‘\textit{listen here} we pensioners have the right to the exemption’

A:  ‘you pensioners have the right to the exemption of the ticket so long as your income doesn’t exceed eighteen million per year’

B:  ‘one million and two one million and three \textit{you do the math} \textsc{ptc}’
    \z % you might need an extra \z if this is the last of several subexamples

A similar explanation holds also for hortatives. In this kind of speech act – closely related to directives – \textit{un po’} combines with first person plural subjunctives. The most typical case is \textit{vediamo un po’} ‘let’s see \textsc{ptc}’ (10 total occurrences in KIP, 46 total occurrences in LIP)\footnote{These numbers refer to the whole corpus, not to the smaller sample based on which I built \tabref{tab:key:7.1} above.} – also representing a fixed expression with a discourse-marker status.

\ea%76
    \label{ex:key:76}

          [KIParla corpus – BOA1015]

mh okay // che cosa possono dire adesso in italiano che prima non potevano dire // okay \textit{vediamo un po’}

\glt ‘hmm okay // what else can they now say in Italian that they couldn’t say before // okay \textit{let’s have a look}’
    \z % you might need an extra \z if this is the last of several subexamples

Leaving aside the constructions with \textit{un po’} that have reached a discourse-marker status, one could wonder if there is a common feature linking imperative constructions where \textit{un po’} gives a mitigating flavor to a directive (soft requests along the lines of examples \REF{ex:key:69} and \REF{ex:key:70} above) and imperative constructions where \textit{un po’} rather adds a boosting flavor (incitements and stressed requests along the lines of examples \REF{ex:key:71} and \REF{ex:key:72} above).

Overall, considering the illocutionary features of these constructions, it seems that \textit{un po’} contributes to specify imperative sentences as requests (of various kinds) – rather than commands or orders. In this respect, the core meaning contribution of \textit{un po’ –} directly derived from its adverbial semantics – is to mark the speech act as implying a minimal effort from the addressee to perform the actions. Besides that, contextual features that vary according to each speech event (urgency, unexpectedness, a certain casualness, and so on) contribute to further differentiating between soft and stressed requests, sudden proposals, and incitements. This way, the marking of a speech act/interactional frame where minimal effort is required on the part of the addressee, represents a conventional feature in constructions with \textit{un po’} as a modal particle. Since it explicitly marks the speaker’s attention to the addressee’s state of mind, this can be interpreted as a feature related to common ground management. That said, the mitigating/boosting flavor appears to be a contextual side-effect.

I now move on to the other important illocutionary context in which \textit{un po’} operates, namely assertions. Compared to the directives in the sample, the assertions appear less routinized: there are less fixed expressions and more variability. At the same time, these contexts of use show greater continuity with the use of \textit{un po’} as an adverbial degree modifier. However, in that case \textit{un po’} has scope over gradable expressions (adjectives, adverbs, verbal phrases), while – used as an operator on assertive speech acts – it has scope over the illocution. In this context, the scope extends over the finite verb form and, in some cases, over a nominal constituent bound to the verb. Thus, the particle does not modify the degree expressed by an adjective, adverb or verbal phrase, but it modifies the degree of assertivity of the utterance, that is, the illocutionary force expressed by the speech act.

Typically, it is used when speakers want to assert something, but are not completely sure or confident about what they are saying, when they want to limit the conversational impact of their utterance or when they want to give a flavor of uncertainty or vagueness to their assertions.\footnote{See \citet{Franken1997} and \citet{AllwoodEtAl2014} for a discussion of concepts such as \textit{uncertainty} and \textit{vagueness} in pragmatics.} Examples \REF{ex:key:77}, \REF{ex:key:78} and \REF{ex:key:79} illustrate this function:

\ea%77
    \label{ex:key:77}

          [KIParla corpus – BOA1001]

// periodo di inserimento eccetera // mh // eh mh // questa è la fase esplorativa in cui le cose che raccoglie sono \textit{un po’} vanno \textit{un po’} in più direzioni // mh //

\glt ‘// settling-in period and so on //hm // eh hm // this is the exploratory phase where the things she collects are \textsc{ptc} they go \textsc{ptc} in several directions // hm //’
    \z % you might need an extra \z if this is the last of several subexamples

\largerpage
\ea%78
    \label{ex:key:78}

          [KIParla corpus – TOD1014]

// che ne so esco in pigiama perché sono contro la società i valori borghesi // quindi io adesso andrò in giro vestita solo in pigiama e pantofole per esprimere // i punk facevano \textit{un po’} quello [...] quando arrivò la rivoluzione punk nelle strade di londra // e prima di new york in realtà // poi esplode \textit{un po’} a londra il fenomeno punk // fu una // uno schiaffo in faccia ai borghesi alla borghesia e alla // come dire al senso comune //

\glt ‘// I don’t know I go out wearing my pajamas because I’m against society the bourgeois values // so, I will go around wearing only my pajamas and my slippers to express // punks were doing \textsc{ptc} this [...] when the punk revolution arrived on the streets of London // and before that in New York actually // then it explodes \textsc{ptc} in London the punk phenomenon // it was // a slap in the face to the bourgeois the bourgeoisie and to the // how to say the common sense//’
    \z % you might need an extra \z if this is the last of several subexamples

\ea%79
    \label{ex:key:79}

          [KIParla corpus – BOD2015]

// no scherzo però secondo me quella è \textit{un po’} l’età in cui ti rendi conto // mh di cosa vuoi fare davvero // nel senso mh // come spiegare a abbandoni \textit{un po’} l’infanzia

\glt ‘// no I’m joking but in my view that’s \textsc{ptc} the age when you realize //hm what you want really do // I mean hm // how to explain that you leave \textsc{ptc} your childhood’
    \z % you might need an extra \z if this is the last of several subexamples

In example \REF{ex:key:79}, in particular, several contextual clues point to a scarce degree of speaker’s commitment with regard to the content being conveyed, namely two pause-fillers (\textit{mh}) and a reformulation marker (\textit{nel senso}); moreover, the process of reformulation itself is indexed explicitly by a question that the speaker perhaps asks himself in passing (\textit{come spiegare}). All these clues point to difficulties in online processing, and this explains the presence of \textit{un po’} marking the low level of assertivity characterizing the speech act. The conspiracy of all these contextual factors – which represent different discourse-pragmatic elements, such as interjections (\textit{mh}), discourse markers (\textit{nel senso}, \textit{come spiegare}), and modal particles (\textit{un po’}) – contribute to convey a broad sense of uncertainty in the interactional situation.

\hspace*{-.2pt}In \chapref{sec:3}, it has briefly been mentioned that illocutionary features of speech acts on the one hand, and conversation practices on the other – although closely intertwined – should be investigated through different analytical tools and categories. For this reason – even though taking into account the conversational effects of \textit{un po’} in assertions – I would analyze assertions featuring \textit{un po’} by means of speech act theory. Given that the core meaning conventionally associated with assertions can be described as “the speaker expresses the belief that the propositional content of the sentence is true” \citep{Searle1969}, it can be said that \textit{un po’} appears in utterances for which the speaker assumes little evidence for the propositional content conveyed.\textsuperscript{} This way, \textit{un po’} is used to mark assertions whose felicity conditions are not totally fulfilled. The conversational effects achieved by such utterances may be described through labels such as uncertainty or vagueness, but \textit{un po’} specifically operates on the illocutionary force of the assertion. In this perspective, an explanation in terms of \textit{low assertivity} allows one to hold these assertions together, regardless of the kind of conversational effect they produce (informational and discourse vagueness, face-saving utterances, mitigation, uncertainty).\footnote{Another possible label is \textit{approximated assertivity}. In particular, the label \textit{approximation} is used in Functional Discourse Grammar to refer to a class of operators (grammatical elements) operating at the layer of communicated content (see \citealt{Hengeveld2017}: 17; see also \citealt{HengeveldKeizer2011}). I will further discuss this issue in the next subsection.}

Through this explanation, it doesn’t seem inappropriate to relate the pragmatic effect of \textit{un po’} in these examples to the category of illocutionary modification as described above. On the one hand, \textit{un po’} modifies the intentions expressed through the performance of a speech act – and it can thus be considered as an operator that specifies the kind of speech act performed. On the other hand, it adjusts the speech act to the preexisting conditions, that is the combination of contextual factors forming the background on which the speech act is performed. Nevertheless, it is not always easy to assess to what extent this particle operates on the illocutionary features of the speech act rather than on the locutionary features of it, that is, on the propositional content evoked by the speech act. Since this is an important issue, I will address it in a separate subsection. Before that, I will briefly comment on the data of written language.

\subsection{Written language}
\hypertarget{Toc124860664}{}
The distribution of the functions of \textit{un po’} in the data of written language is similar to that of spoken language. I extracted 200 occurrences of post-verbal \textit{un po’} from the \textit{La Repubblica} corpus and I manually annotated the functions covered, using the same tag set applied to the spoken data. \tabref{tab:key:7.3} shows the distribution of the functions (absolute and relative frequencies).
\largerpage[2]

\begin{table}
\begin{tabularx}{0.8\textwidth}{Xr@{\qquad}r}
\lsptoprule
 & \multicolumn{2}{c}{REP}\\
 \cmidrule{2-3}
  & abs & rel\\
  \midrule
\textsc{pseudo-partitive} & 41 & {0.20}\\
\textsc{adverbial} \textsc{modifier} & 146 & {0.74}\\
\textsc{mp} \textsc{directive} & 4 & {0.02}\\
\textsc{mp} \textsc{optative/hortative} & – & –\\
\textsc{mp} \textsc{assertive} & 9 & {0.04}\\
\textsc{other} & – & –\\
\midrule
Total & 200 & \\
\lspbottomrule
\end{tabularx}
\caption{\label{tab:key:7.3} Distribution of the functions of \textit{un po’} in the \textit{La Repubblica} corpus}
\end{table}

The relative frequencies are similar to the ones of the spoken data (especially if compared with the sample of the KIP).\footnote{Also in this case, the comparison must be understood merely as a suggestion, since the samples were collected according to different criteria and they do not contain the same number of tokens.} The prototypical use of \textit{un po’} is the use as an adverbial degree modifier, followed by the use in pseudo-partitive constructions. The illocutionary uses are marginal, but still attested – showing that they are not exclusive to the spoken language. The qualitative analysis of the examples does not add significant detail to the overall picture either. Example \REF{ex:key:80} is an instance of a directive marked by \textit{un po’}. The explanation given for directives in the preceding subsection also holds for it.


\ea%80
    \label{ex:key:80}

          [\textit{La Repubblica} corpus – article.id: 558932, comment: sport]

Quest’ultimo, 37 anni, è un marinaio di foresta, come quasi tutti i grinder cioè quegli Obelix che smanovellano circa duemila giri a regata (fate \textit{un po’} i conti). Viene da Novara, giocava a pallavolo, lanciava il disco e faceva l’Isef alla Cattolica di Milano quando gli proposero di andare per mare.

\glt ‘The latter, 37 years old, is a forest seaman, like almost all grinders, that is, those Obelix who paddle about two thousand spins per race (do \textsc{ptc} the math). He comes from Novara, he used to play volleyball, he used to do discus throw, and was attending Isef at the Catholic University of Milan when he got the proposal to go to sea.’
    \z % you might need an extra \z if this is the last of several subexamples

Regarding assertions, the situation is quite different. The nine instances annotated as assertive – that is, modal uses of \textit{un po’} in assertive speech acts – could be actually better thought of as bridging contexts. In most of these cases (eight out of nine), \textit{un po’} occurs in predicative construction, between the verb \textit{essere} ‘to be’ and a nominal constituent. It is not clear whether it already operates on the assertivity of the illocution, or whether it can still be considered an operator on the communicated content (this issue will be further discussed in the next subsection). Example \REF{ex:key:81} below could be the only convincing illocutive use of \textit{un po’} in assertions found in this small sample: it does not modify the predicate (it’s not a partial interpretation), rather, it limits the speaker’s conviction expressed in the assertion. To wit, it operates on the illocutionary force of the assertion.

\ea%81
    \label{ex:key:81}

          [\textit{La Repubblica} corpus – Comment, Politics]

C’è una vignetta sul “Kurier” che mostra i capi dei due partiti di coalizione che cercano riparo da un temporale i cui fulmini tratteggiano due grandi simboli delle SS: nella sua asprezza interpreta \textit{un po’} il senso di delusione e di protesta con cui anche la stampa austriaca – escluso solo il diffuso “Kronenzeitung” – ha accolto la decisione del governo.

\glt ‘There is a cartoon on the “Kurier” that shows the leaders of the two coalition parties trying to find a shelter from a storm where lightning is sketched like two big symbols of the SS: in its bitterness, it portrays \textsc{ptc} the sense of delusion and protest with which also the Austrian press – excluding only the “Kronenzeitung” – accepted the government’s decision.
    \z % you might need an extra \z if this is the last of several subexamples

Moving on from here, I will further develop the discussion concerning the behavior of \textit{un po’} as a modal particle in assertions and the pragmatic functions expressed. As mentioned above, the problem revolves around the issue of the specific layer on which this particle operates, namely the illocutionary act in a narrow sense (force features of the speech act, including illocutionary point and preparatory conditions) or rather on certain aspects of the locutionary act (features of the propositional content evoked by the speech act). In the following, I will refer to FDG to show that it is not easy to draw a clear-cut divide between the illocution and the communicated content – and that the domain of action of \textit{un po’} in assertions possibly extends to both layers.

\subsection{Assertions with \textit{un po’}: Non-straightforward communication}
\hypertarget{Toc124860665}{}
The distinction between illocution and communicated content as separate grammatical layers is found in the model of Functional Discourse Grammar as a distinction pertaining to the interpersonal level. This level of analysis deals with the interaction between the speaker and the addressee (\citealt{HengeveldMackenzie2008}: 46–48). Rather than contributing to the semantic content of the expression in which it occurs, it is concerned with the attitude of the speaker towards the information they are transmitting: linguistic expressions operating at this level have no designating force.

As I have already sketched out in \chapref{sec:3}, the interpersonal level in FDG is organized in different pragmatic layers expressing scope relations. To briefly summarize, the \textit{move} consists of one or more (sequentially ordered) \textit{discourse acts}. Each act in turn consists of an \textit{illocution}, the \textit{speech participants} and a \textit{communicated content}. Finally, within the communicated content, one or more \textit{subacts of reference} and \textit{ascription} are executed by the speaker, by means of which he refers to entities and ascribes properties to these entities. For the present discussion, the distinction between illocution and communicated content is particularly relevant:

\begin{quote}
Whereas the Illocution indicates the conventionalized conversational use of a Discourse Act, and the Participants represent the essential Speaker-Addressee dyad, the Communicated Content contains the totality of what the Speaker wishes to evoke in his/her communication with the Addressee. In actional terms it corresponds to what \citet{Searle1969} calls the “representational act” and corresponds to the choices the Speaker makes in order to evoke a picture of the external world s/he wants to talk about. The Communicated Content is thus the unit within which the mapping to the Representational Level takes place. (\citealt{HengeveldMackenzie2008}: 87)
\end{quote}

The communicated content is made up by the execution of a set of subacts (reference and ascription), by which speakers can evoke referents and properties. Moreover, pragmatic relations determining the information structure of an utterance are encoded at this layer, such as focus (vs. background) and topic (vs. comment).\textsuperscript{}

\citet{HengeveldKeizer2011} further explore aspects related to the communicated content. Specifically, their paper is dedicated to linguistic elements expressing \textit{non-straightforwardness}, that is “grammatical and lexical strategies that are available to speakers to convey that the message they intend to communicate is not straightforwardly covered by the basic elements contained in their utterance; these include dummies such as \textit{whatshisname}, approximators such as \textit{like}, as well as exactness markers such as \textit{true}” (\citealt{HengeveldKeizer2011}: 1962). Accordingly, the degree of (non-)straightforwardness of a linguistic expression reflects the extent to which speakers are able or willing to provide the exact amount of information needed for successful or felicitous communication. While acknowledging an undeniable link between the linguistic coding of straightforwardness and such representational matters as predication and denotation, \citet[1964]{HengeveldKeizer2011} analyze straightforwardness as pertaining foremost to the interpersonal level, i.e. as modifying or specifying the actions performed by the speaker in their interaction with an addressee.

These observations can help to better describe the behavior of \textit{un po’} in assertive speech acts. In the analysis provided above, a distinction has been made between content-level uses of \textit{un po’} (pseudo-partitive constructions, adverbial degree modifier uses) and context-level uses (illocutionary operator). In FDG terms, this distinction is reflected in the separation between a representational and an interpersonal level. The issue now is to further explain the status of context-level uses of \textit{un po’.} The advantage of referring to FDG is that this framework assumes a distinction – within the interpersonal level – between illocution and communicated content. In this respect, the analysis in the previous section – although favoring an analysis of modal \textit{un po’} in terms of low assertivity – led to the conclusion that labels such as vagueness or approximation may be used to describe the conversational effect brought by assertions with \textit{un po’}. Although very tricky to define in a systematic way, these labels seem to refer to the communicated content – in the sense of the information evoked by a speech act – rather than to the illocution in a narrow sense. Bearing these observations in mind, I will explore some examples in order to assess whether \textit{un po’} in assertions can be described – at least in some cases – as an operator at the layer of the communicated content rather than on the illocution.

\citet[1965–1975]{HengeveldKeizer2011} and \citet{Hengeveld2017} report several examples of lexical and grammatical expressions which code different values of non-straightforwardness, applying to each of the units involved at this layer: the two types of subact, ascriptive and referential, and the communicated content as a whole. I give here three examples of \textit{sort of}.

\ea%82
    \label{ex:key:82}

          English (\citealt{Hengeveld2017}: 21–22)\footnote{Examples (32a–c) correspond respectively to examples (12), (13) and (14) in Hengeveld (2017: 21–22).}

\ea\label{ex:key:82a}  We’re looking for a \textit{sort-of} manager to book us shows.

\ex\label{ex:key:82b}  I think I can more or less understand in general terms what happens up until \textit{sort of} the impressionist time, maybe just post-impressionist.

\ex\label{ex:key:82c}  McCain backtracks on gay adoption, \textit{sort of}.
    \z
\z

In \REF{ex:key:82a} \textit{sort of} directly modifies a lexical element and has the function of indicating that this lexical element only approximately designates what the speaker has in mind. In this case, \textit{sort of} operates at the layer of the ascriptive subact, as it is the appropriateness of the ascription of a property that is at stake here. In \REF{ex:key:82b} \textit{sort of} has scope over the entire noun phrase \textit{the impressionist time}, which in this case serves as a measure that is roughly indicative of the end point of the period about which the speaker has some understanding. In such cases, \textit{sort of} is said to operate on the referential subact, as the unit being modified is referential in nature. Finally, in \REF{ex:key:82c} \textit{sort of} modifies the entire preceding utterance: it qualifies this utterance as expressing approximately what the speaker has in mind. In this example, \textit{sort of} modifies the entire communicated content, the message transmitted by the speaker. \citet{HengeveldKeizer2011} define \textit{sort of} as an expression of \textit{approximation.} Let’s compare this distribution with Italian examples of \textit{un po’}.

\ea%83
    \label{ex:key:83}

          [LIP corpus – Milan E8]

la prima è che il mondo occidentale è abbastanza abituato a vedere \textit{un po’} dei profili in ogni momento storico e invece ci sono delle situazioni storiche.

\glt ‘the first thing is that the western world is quite used to seeing \textit{sort of} profiles in every historical moment whereas there are only historical situations.’
    \z % you might need an extra \z if this is the last of several subexamples

\ea%84
    \label{ex:key:84}

          [LIP corpus – Milan E11]

vi sono però dei difficili rapporti con gli azionisti tedeschi con la cordata che fa \textit{un po’} da cordone sanitario.

\glt ‘there are however difficult relationships with the German shareholders with the group acting \textsc{ptc} as a cordon sanitaire.
    \z % you might need an extra \z if this is the last of several subexamples

\ea%85
    \label{ex:key:85}

          [LIP corpus – Milan C9]

d’altro canto questo è un po’ il problema di tutti i movimenti spontanei eh su questi su questi temi io credo che adesso si apra una fase diversa una fase nuova nella quale eh sostanzialmente gli obiettivi possono essere due primo è \textit{un po’} quello a cui noi cerchiamo di contribuire con questo concerto e cioè quello di tenere alta l’attenzione e la solidarietà

\glt ‘on the other hand this is kind-of the problem of all spontaneous movements uh on these on these topics I think that now a different phase is beginning a new phase where uh basically the goals can be of two kinds first it is \textsc{ptc} the one we’re trying to support with this concert that is keeping up the awareness and solidarity’
    \z % you might need an extra \z if this is the last of several subexamples

These examples from my data sample can be very well analyzed along the lines of the explanation given for \REF{ex:key:82a}, \REF{ex:key:82b} and \REF{ex:key:82c}. In \REF{ex:key:83} \textit{un po’} operates at the layer of the ascriptive subact, while in \REF{ex:key:84} it operates at the layer of the referential subact. In \REF{ex:key:85} it operates on the whole communicated content. In all three cases, \textit{approximation} seems a good label to describe its pragmatic effect. These examples suggest that the context-level uses of \textit{un po’} – or at least some of them – can be better described as operators at the layer of the communicated content rather than on the illocution. However, doubts remain for examples like \REF{ex:key:85}, where it is not clear whether \textit{un po’} expresses approximation on the communicated content or low commitment on the assertivity. This is not surprising after all, as \citet{HengeveldKeizer2011} themselves highlight:

\begin{quote}
It would be worthwhile to see to what extent these distinctions are relevant at other layers of the Interpersonal Level (the Illocution or the Discourse Act) as well. However, since the same expressions are often used to mark non-straightforwardness at the different layers, it is not always easy to determine to which level these expressions apply. For the same reason, it turns out to be difficult to distinguish between approximation/exactness on the one hand and mitigation/reinforcement on the other. (\citealt{HengeveldKeizer2011}: 1975)
\end{quote}

In the absence of clear formal features that distinguish the use of the same element to express different functions at different layers, these kinds of observation leave some space for scope vagueness. They moreover recall the (partially problematic) issue of the exact contribution of \textit{un po’}, which – apart from approximation – has been described as marking reinforcement in some cases.

More precisely, \textit{un po’} seems to operate as a focus marker in some examples, contributing to separate a focused part of the utterance (often new information) from a backgrounded one (\citealt{HengeveldKeizer2011}: 1975). Again, this is not surprising since pragmatic functions such as focus and topic are coded – from an FDG perspective – exactly at the layer of the communicated content and overlapping of functions are to be expected. Consider example \REF{ex:key:86} below. In this case \textit{un po’} could actually be analyzed as contributing to mark the focus (in a pseudo-cleft syntactic structure), but it is clear that the gap between this kind of construction and the reanalysis of \textit{un po’} as a marker that modifies the illocutionary features of the whole speech act represents a short step:

\ea%86
    \label{ex:key:86}

          [KIParla corpus - BOD2001]

// eh la cosa è \textit{un po’} quella che vivi in in una bolla però poi effettivamente sì certo conosci un po’ il luogo però forse non lo conosci proprio in tutti i pro e i contro //

\glt ‘// well, the thing is \textsc{ptc} that that you live in in a bubble but then actually yeah sure you know the place a bit but maybe you don’t know it exactly with all the pros and cons //’
    \z % you might need an extra \z if this is the last of several subexamples

\hspace*{-0.6pt}These last observations discussed the relationship between communicated content and illocution on the one hand, and between approximation, mitigation and reinforcement on the other – showing that the context-level uses of \textit{un po’} display a complex distribution which cover different pragmatic domains.\footnote{It is perhaps important to point out that – despite this underlying complexity – \textit{context-level uses} remains an informative and inclusive enough label, beyond the micro-layers which are detectable at the pragmatic level.}

Nevertheless, even if many examples of \textit{un po’} can be interpreted as more relevant for the communicated content layer, the fact remains that many others are clearly involved in the modification of the illocutionary force. Consider example \REF{ex:key:87} below. In this case, there is no doubt that \textit{un po’} marks low commitment on the assertivity. There are different factors that point to a scarce degree of speaker’s commitment with regard to the content being conveyed: the reported information (the fact that Leo is dating someone) could not have been verified, or the speaker does not feel entirely entitled to share it. This being the situation, the speaker uses \textit{un po’} to reduce the level of assertivity of their speech act. In this case, therefore, \textit{un po’} refers to the network of conversational expectations which constitutes the common ground of the conversation, thus qualifying itself as a full-fledged illocutionary operator.

\ea%87
    \label{ex:key:87}

          [KIParla corpus - BOD2014]

// leo lo vedo molto bene in ’sto periodo // ho visto \textit{un po’} che ha una tipa che fanno parecchie cose // beh come sempre // solito solito leo // un uomo di successo //

\glt ‘// leo I see him in good shape at the moment // I saw \textsc{ptc} that he has a girl they do lot of stuff // well as usual // same same leo // a man of success //’
    \z % you might need an extra \z if this is the last of several subexamples

\section{\textit{un po’}: Closing remarks}
\hypertarget{Toc124860666}{}
The analysis conducted in this chapter has allowed us to describe the polyfunctionality of \textit{un po’} – from the content-level uses in pseudo-partitive constructions and as an adverbial degree modifier up to its context-level use as a modal particle. As in the case of \textit{pure}, it can be noticed that the adverb appears in different types of speech acts and it performs different pragmatic effects: directives can be marked as involving a minimal effort on the side of the addressee and consequently specified as mild requests (with a mitigation flavor) as well as rather sudden incitements (with a reinforcing overtone); the illocutionary force of assertives can be downtoned (low commitment on the assertivity); specific illocutions like hortatives can also be marked by \textit{un po’}.

Moreover, going through the data, I also tried to highlight the connections with neighboring functions: in directives, \textit{un po’} can contribute to specifying aspectual features (perfectivity, low-transitivity), in assertions it operates in different ways on the presentation of information (approximation of the communicated content, focus-marking). Going back to the problematic issue of whether illocutionary modification may be used as a category to define the illocutive uses of \textit{un po’} – that is, can \textit{un po’} be considered a modal-particle-like element in its illocutive uses? – it has been suggested that \textit{un po’} can be described as an illocutionary operator, though a non-prototypical one: even though in some cases it has little effect on the common ground management and exhibits, rather, a strong link with the communicated content, in other cases it definitely operates on the illocutionary force of the speech act.

Highlighting the connections among these different functions did not lead to the creation of a semantic map – more typological research would be needed for this – but it can nevertheless give an idea of how different functional domains are close to each other and can be linked by contextual inferences. In its content-level uses (pseudo-partitive constructions, adverbial degree modifier uses), \textit{un po’} expresses part-whole relations and indefiniteness/unspecificity: traces of these meanings can be found also in other uses.\footnote{In this respect, approximation could be considered a pragmatic counterpart of non-specificity.} As suggested by \citet{LuraghiKittilä2014} and \citet{Budd2014}, different inferential paths may lead to the expression of new functions, based on what meaning shade is profiled. In the case of \textit{un po’,} both boundedness and unboundedness can play a role, depending on which perspective is assumed on the partiality of an action, event or referent (that is, depending on whether what has been done, or what has been left undone, is profiled). These different profiles can lead – among other things – to the expression of aspectual nuances of verb forms, perfectivity on the one hand (boundedness) and imperfectivity and low-transitivity on the other (unboundedness). Finally, these meaning features are sensitive to the illocutive types in which they are inserted: in the context of specific speech acts, the meanings conveyed by \textit{un po’} can be reanalyzed as functions at the pragmatic level. Thereby, \textit{un po’} comes to be used as a marker of specific functions at different pragmatic layers (low commitment on the speaker’s side, low effort on the hearer’s side, vagueness and approximation, force regulation).

To conclude, a closing note on the relations between my analysis of \textit{un po’} and the Functional Discourse Grammar theoretical framework. As it has been said in the preceding chapters, the grammatical categories assumed by FDG can be useful to describe systematic functions at the interpersonal level – which are often neglected in the description of pragmatic markers – and, at the same time, to highlight the connections with other grammatical categories pertaining to other levels of analysis. This holds also in the case of \textit{un po’}. Specifically, the distinction introduced in FDG between the layer of illocution and the layer of communicated content has allowed a fine-grained description of the context-level uses of \textit{un po’}. Conversely – now with specific reference to the dynamic model elaborated by \citet{Narrog2012} – it seems that modality plays no role in the description of the uses of \textit{un po’}, or in its functional development. I recall that modality is considered by \citet{Narrog2012} to be the main semantic domain leading to illocutionary modification: forms marking modality can widen their scope relations and move up to express illocutionary functions. My analysis of \textit{un po’} – exactly as the analysis of \textit{pure –} suggests that modality is not the only domain that linguistic expressions can cross to reach the layer of illocutionary modification. Other domains, such as aspect, event quantification, and information structure can also play a role at the semantics/pragmatics interface. Future research will further develop this issue.

