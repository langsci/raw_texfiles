\chapter{Introduction}\label{sec:1}
\hypertarget{Toc124860606}{}\section{Research background}
\hypertarget{Toc124860607}{}
Linguistic interactions flourish on a myriad of grounds and through a multitude of performers. Numerous elements compose the countless situations in which interactions take place and are formed. Along with the language materials (sounds, words, syntactic structures and grammatical rules), linguistic interactions are molded by social, cultural and cognitive factors~– that is, extralinguistic factors. Together, these factors compose the \textit{context} that hosts and shapes them.

Context is a basic notion of pragmatics, which~– according to mainstream definitions~– studies language use in context as well as how context contributes to the meaning of linguistic expressions. Pragmatics explains how the transmission of meaning depends not only on the linguistic knowledge (grammar and lexicon) of speakers and addressees, but also on what surrounds an utterance, namely, context. Even though, from time to time, socio-cultural factors can also be considered, pragmatics often makes use of a narrower notion of context. From this perspective, context indicates a specifically \textit{mental landscape}, where different mental entities meet and coexist. Theories of mind, theories of oneself and others, memories of past and future interactions, the communicative intent of the speaker, the expectations of the hearer, any pre-existing knowledge about those involved, and any pre-existing knowledge about the world.

Fundamentally, languages display linguistic expressions that explicitly refer to this complex tangle of mental entities. The research presented in this volume is rooted in a pragmatic perspective on the negotiation of meaning. It studies linguistic expressions which explicitly code a connection between utterances and mental landscapes~– to wit, between speech acts and the contexts of their performance. Drawing on Italian data~– both from the standard and regional varieties~– I will examine the properties and the behavior of a set of adverbs variously modifying the speech acts in which they appear. Among other functions, they specify the way a speech act should be interpreted in the context of interaction, modify its illocutionary force, and allow the speaker (and the hearer) to refer to presupposed/inferred meanings active in the common ground and shared during interactions.

\section{Research directions}
\hypertarget{Toc124860608}{}
Dealing with linguistic expressions that code interactional functions, the present work fits into the research field on pragmatic markers, which has been developing and growing for the last forty years and is now represented by many different theoretical and empirical approaches. The research presented in this volume does not fully line up with any specific theoretical framework: rather, it touches upon different aspects of linguistic theory that might be useful to describe the elements under investigation and their functions. Specifically, it relies upon three main theoretical sources to formulate hypotheses on the properties and the behavior of the elements under investigation: firstly, Hansen’s (\citeyear*{Hansen2008,Hansen2012}) works on the semantics/pragmatics interface; secondly, Waltereit’s (\citeyear*{Waltereit2001,Waltereit2006}) works on the functions of modal particles (which in turn is rooted in speech act theory); and finally, the theoretical framework developed by Functional Discourse Grammar (\citealt{HengeveldMackenzie2008}) for what concerns illocutionary modification and the layered structure of grammatical categories. Reference to these sources will allow me to spell out the research directions of the present work: 
(i) studying the different uses of a set of Italian adverbs and distinguishing content-level uses from context-level uses; 
(ii) examining in detail specific uses of adverbs at the speech-act level and the features of illocutionary modification as a grammatical category; 
(iii) investigating the relationship between different uses of the same item and the distribution of functions in the grammatical system. Further reference to issues concerning semantic change and sociolinguistic variation complete the framework adopted here. The combination of these different aspects also suggests new directions in the theoretical treatment of modal particles.

Nonetheless, the objectives of the present research are empirical rather than theoretical. The main goal of this study is to provide a description of modal uses of a set of Italian adverbs: among others, I will discuss specific uses of \textit{pure} ‘also’, \textit{anche} ‘also’, \textit{solo} ‘only’, and \textit{un po’} ‘a bit’. A terminological note should be made here. I will use different labels to refer to the elements under investigation, depending on what specific aspect I want to highlight. The labels \textit{modal particles} and \textit{modal-particle-like elements} refer to the adverbs under investigation as part of a specific semantic/pragmatic class of elements. The first label is the most common in linguistic research and therefore the one that creates less confusion (because of this, it also appears in the title of this book). The second is sometimes used throughout the volume to highlight that some languages (including Italian) don’t display clear paradigms of modal particles but rather adverbs (and other elements) that in specific contexts of use display functions similar to those of modal particles. The labels \textit{modal uses}/\textit{functions} and \textit{illocutive uses}/\textit{functions} refer to the semantic and pragmatic properties of the adverbs under investigation in specific contexts of use. The first clearly redirects to the label \textit{modal particles} but it can be misleading since one could interpret it as referring to the grammatical domain of \textit{modality} (which includes the expression of epistemic modality and related categories, which is only marginally relevant for modal particle research). For this reason, I sometimes use the labels \textit{illocutive uses}/\textit{functions of adverbs}. These are also somewhat problematic, since every speech act has an illocutionary force, which is better understood as the global property of an utterance and not specific to some of its components. Strictly speaking, there are no \textit{illocutive uses of adverbs}. In this sense, these labels should be interpreted as referring to those contexts of use where the specific adverbs express functions related to the modification of a speech act and its illocutionary force.

With reference to Italian, the issues mentioned so far have not yet been (thoroughly) touched upon by previous research. In this respect, the research directions revolve around the following questions: What are modal uses of adverbs in Italian? How can their properties and functions be described? What are their contexts of use? Dealing with such questions, a further research direction has proven to be inevitable, namely the issue of sociolinguistic variation~– since some modal uses can only be found in certain varieties of Italian.

To summarize, the goals of the present research are:

\begin{itemize}
\item describing the modal uses of a set of Italian adverbs in terms of semantic features and pragmatic functions
\item describing the modal uses of a set of Italian adverbs in terms of contexts of use (the kind of speech acts and conversational routines in which they appear)
\item connecting this description to the issue of polyfunctionality and meaning description at the semantics/pragmatics interface
\item connecting this description to issues of semantic change
\item connecting this description to issues of language variation, with a focus on regional varieties of Italian
\end{itemize}
\section{Structure of this book}
\hypertarget{Toc124860609}{}
Overall, this research aims at being the first full-length study on modal uses of adverbs in Italian. It consists of two parts: the first part offers a topic-based literature review and sets out the theoretical framework. The second part presents two case studies on modal uses of adverbs in standard Italian and two case studies on modal uses of adverbs in regional varieties of Italian. In this way, the research features four case studies: although they share a common theoretical framework and similar research methods, they are somewhat independent of each other. Nevertheless, cross-referencing between them ensures that the work is coherent.

Following this introduction, \chapref{sec:2} briefly discusses previous work on pragmatic markers and the place of modal particles in this field of research. \chapref{sec:3} examines modal particles more closely: after introducing three key concepts of pragmatics (speech acts, implicatures and presuppositions), it sets out the functions of modal particles and discusses the grammatical category of illocutionary modification. \chapref{sec:4} examines the issue of how to deal with meaning at the semantics/pragmatics interface and links it to issues of language change (semantic change and reanalysis) and language variation (conventionalization of new functions and sociolinguistic perspectives on pragmatic phenomena).

Opening the second part, \chapref{sec:5} paves the way for the empirical case studies, discussing previous work on modal particles in Italian as well as the sociolinguistic background necessary for describing these elements, and presenting the research methods employed (corpus analysis and sociolinguistic questionnaires). \chapref{sec:6} deals with modal uses of additive focus adverbs: \textit{pure} and \textit{anche}, both meaning ‘also’. \chapref{sec:7} deals with modal uses of the quantifier/degree adverb \textit{un po’} ‘a bit’. \chapref{sec:8} deals with modal uses of the exclusive focus adverb \textit{solo} ‘only’, with a focus on the regional variety of Italian spoken in Piedmont (a region in the north-west of Italy). \chapref{sec:9} deals with the broader distribution of modal uses of adverbs in regional varieties of Italian, with a focus on northern varieties. A concluding chapter closes the work, summing up the main findings, highlighting strong and weak points, and suggesting future research directions.

