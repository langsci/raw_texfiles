\chapter{Modal particles and language variation: A case study on \textit{solo} ‘only’}\label{sec:8}
\hypertarget{Toc124860667}{}\section{\textit{solo}: Overview of the categories involved}
\hypertarget{Toc124860668}{}
This chapter deals with the different uses of the adverb \textit{solo} ‘only’. In addition to its prototypical use as an exclusive focus adverb, it has developed some secondary uses – among which we find illocutive uses as an illocutionary operator tied to specific speech-act types.

This case study aims at giving an in-depth description of the illocutive uses of \textit{solo} and – on a methodological level – it aims at showing how an integrated approach that combines pragmatics and sociolinguistics is necessary in order to address the different theoretical and empirical issues raised by similar phenomena. In particular – compared to the approach taken for the two previous case studies of the two previous chapters – this case study includes more explicitly the issues concerning language change and variation discussed in \chapref{sec:4}. This way, several topics discussed in the previous chapters will be combined to get a comprehensive analysis of the illocutive uses of \textit{solo} and their development: pragmatic categories and their position in the grammar, sociolinguistic variation, semantic change and reanalysis. However, although closely intertwined, each of the three sections specifically examines one of these aspects.

This section – which also relies on previous research (\citealt{Favaro2019,Favaro2020}) – sets out the background of the case study, discussing the main semantic and pragmatic features of the modal uses of \textit{solo} and describing the questionnaire used to investigate them.

\subsection{From focus marking to illocutionary modification}
\hypertarget{Toc124860669}{}
In its prototypical use, \textit{solo} belongs to the category of focus adverbs, which count among the linguistic items marking the information structure of an utterance – specifically, the pragmatic relation of focus. Basic notions of information structure – as well as general characteristics of focus adverbs – have been discussed in \chapref{sec:6}, so I will not repeat them here. Let us however briefly examine the semantic features of \textit{solo} as a focus adverb again. Adapting example \REF{ex:key:13} of \chapref{sec:6}, the lexical meaning of \textit{solo} can be described as follows.

\ea%88
    \label{ex:key:88}

           Italian
\ea \label{ex:key:88a} Giorgio ha comprato \textit{solo} delle mele.

\glt ‘Giorgio \textit{only} bought apples’

\ex \label{ex:key:88b} Giorgio ha comprato delle mele.

\glt ‘Giorgio bought apples’      [\textsc{presupposition}]

\ex\label{ex:key:88c} Giorgio non ha comprato nient’altro.

\glt ‘Giorgio didn’t buy anything else’   [\textsc{assertion}]
    \z
\z

A sentence like \REF{ex:key:88a} can be described as the sum of two propositions, represented here by sentences \REF{ex:key:88b} and \REF{ex:key:88c}. The sentence \textit{Giorgio only bought apples} builds on the presupposition that \textit{Giorgio bought apples} (which is outside of the scope of the negation, cf. \textit{It is not true that Giorgio bought only apples}, activating the same presupposition) and contains the assertion that \textit{Giorgio didn’t buy anything else}, thus suggesting that \textit{apples} are part of a larger set of elements (depending on the context) and that none of the possible alternatives satisfies the relevant open sentence (\textit{Giorgio bought x}). Recalling the description proposed by \citet[94--119]{König1991}, this semantic property of focus adverbs can be described as a quantification effect, which relates the value of the focused expression to a set of alternatives. The meaning contribution of \textit{solo} is to exclude these alternatives as possible values for the open sentence in its scope.

As I pointed out in \chapref{sec:6}, some focus adverbs may induce a ranking into the set of possible alternatives, inducing scalar structures in the domain of quantification: the alternatives and the focus value appear as part of a hierarchically arranged set. Like \textit{pure}, \textit{solo} does not induce a scalar ordering by itself, but it is compatible with it when this is suggested by the context:

\ea%89
    \label{ex:key:89}

           Italian

È \textit{solo} un bambino!

\glt ‘He is \textit{only} a child!’

(+> not a boy / not an adult / not an old man)
    \z % you might need an extra \z if this is the last of several subexamples

\ea%90
    \label{ex:key:90}

          English \citep[96]{König1991}

Is \textit{only} a B grade required?

(+> not higher grades)
    \z % you might need an extra \z if this is the last of several subexamples

In \REF{ex:key:89} and \REF{ex:key:90}, the sets of possible alternatives to the focus value (respectively, age groups and academic grades) are \textit{per se} ordered sets. In contexts like these, scalar focus adverbs often activate an evaluation inference connected to the sca\-lar ordering – that is, the value of the focus is characterized as ranking “high” or “low” on the scale. As a part of its conventionalized meaning – when used in a scalar way – \textit{solo} activates the inference that the excluded alternative values rank higher on the scale than the value in focus: in most cases, when \textit{solo} is associated with an order, the value of its focus is evaluated as minimal.

Given the features of the prototypical use, I turn now to uses of \textit{solo} which cannot be described as focus adverbs. Concerning their scope, they do not operate on sentence constituents (like NPs or VPs) but on other units. Concerning their meaning, they do not display the effect of quantification: no set of alternative referents is opposed to a focused one. As \citet[503]{EckardtSpeyer2016} put it, “reference to alternatives of focus-background structure is lost”. Thus, these uses of \textit{solo} represent \textit{bleached focus constructions}, functional developments which occupy a more forward position on the focus cline (see the discussion in \chapref{sec:6}).

A first set of functional developments of \textit{solo} consists of functional expansions towards the domains of discourse coherence and conversational structure: these are the uses of the adverb as a \textit{conjunctional adverb} and as a \textit{discourse marker} which I will not develop here (see \citealt{Favaro2020}: 117–120).\footnote{Similar uses of exclusive focus adverbs have been described for other languages, too: König (\citeyear[106--107]{König1991}, \citeyear[34–38]{König2017}) gives a brief description of English \textit{only} and German \textit{nur} as conjunctional adverbs, \citet{Brinton1998b} considers the diachronic evolution of \textit{only} as a conjunctional adverb. With regard to German, \citet[78--81]{ModicomDuplâtre2018} investigate the use of \textit{nur} ‘only’ as a connective and \citet[337–338]{AuerGünthner2005} account for the use of \textit{bloß} ‘only’ as a discourse marker.} A second set of functional developments of \textit{solo} is represented by its illocutive uses – functional expansions towards the domain of speech act specification and common ground management. The involvement of these functional domains – which I discussed together in \chapref{sec:3} under the label of \textit{illocutionary modification –} leads us to the consideration that \textit{solo} in its illocutive uses employs a modal-particle-like element.\footnote{The illocutive uses of \textit{solo} partially bring to mind some functions of English \textit{just} (\citealt{König1991}: 116–119). See also \citet{Lee1987,Lee1991}, \citet[153–174]{Aijmer2002}, \citet{MolinaRomano2012} and Beeching (\citeyear[76–96]{Beeching2016}, \citeyear{Beeching2017}).}

Quite importantly, previous research suggested that these uses are unevenly distributed in Italian \citep{Favaro2019}. This fact is probably due to some kind of sociolinguistic markedness (regional and diaphasic variation, for instance) and they are almost absent in digital corpora of spoken Italian. A small questionnaire survey permitted the collection of some real-life examples for analysis and to test them with acceptability judgments and possible paraphrases. Two main contexts of occurrence – directive speech acts and assertive speech acts – were identified for these uses.

\subsection{Contexts of occurrence: Directive and assertive speech acts}
\hypertarget{Toc124860670}{}
In the first kind of context, \textit{solo} occurs in directive speech acts such as orders, exhortations and instructions. In \REF{ex:key:91} an example of \textit{solo} in a directive is provided.\footnote{Examples \REF{ex:key:91}--\REF{ex:key:96} – which I have already discussed in \citet{Favaro2020} – come from different sources. Some of them were featured in the first questionnaire I developed on these issues (see \citealt{Favaro2019}), where I relied on personal introspection, spontaneous speech and every-day conversations heard in Turin to collect the stimuli. Other examples were extracted from the internet (social media and forum discussions) and are featured in the new questionnaire discussed later in this chapter. The text in square brackets represents additional contextual information which was provided in the questionnaires to better interpret the stimulus.}

\ea%91
    \label{ex:key:91}

          Italian \citep[121]{Favaro2020}

[Silvia’s brother has broken his sister’s bike which he had borrowed without asking and keeps apologizing to her profusely. Silvia says to her brother:]

  Guarda, sparisci \textit{solo}!

\glt ‘Look, \textit{just} beat it!’
    \z % you might need an extra \z if this is the last of several subexamples

From a syntactic point of view, \textit{solo} is positioned immediately after the finite verb form, and it has scope over the whole utterance: these two features are common to all illocutive uses of \textit{solo}. Concerning the scope, it is problematic to identify an overt sentence constituent in the scope of \textit{solo}: the adverb seems to be associated with a verbal focus, but the alternatives in question are not denotations of other verbs. Related to this, from a semantic point of view, the quantificational effect with exclusive meaning is expressed in a different way: the adverb does not evoke alternative referents as opposed to a focused one, but other propositions activated in the common ground (CG). In this way, the presence of \textit{solo} in the directive seems to require different CG structures compared to the same utterance without it (\textit{Guarda}, \textit{sparisci!}).

In example \REF{ex:key:91}, the presence of \textit{solo} explicitly points to a set of propositions present in the CG (for instance, the interlocutor’ opinion on the appropriateness of continued apologies), which – in the speaker’s perspective – are not valid in this specific context. In the case of the directive without \textit{solo}, this connection with the common ground is not explicitly established and the speech act is not projected against a background of other propositions. For these reasons, it is unsatisfactory to define \textit{solo} as an exclusive focus adverb in this kind of construction: at first sight, it rather emphasizes the speech act which acquires a salient position in the conversational exchange.\footnote{Intonation is often a crucial variable to identify the contexts of use where \textit{solo} functions as a modal particle. Nevertheless, although this kind of utterance may be characterized by distinctive intonation patterns, suprasegmental aspects count among the features that contribute to identify modal uses of \textit{solo}, but they are not enough to distinguish them from the prototypical use as a focus adverb.} In these uses, \textit{solo} can be considered an \textit{illocutionary operator}, a grammatical item operating at the layer of illocution.

Another example is the brief exchange in \REF{ex:key:92}, where someone is talking about the volleyball team of Bra, a town in northern Italy. In the answer, \textit{solo} operates on the imperative verb form \textit{stai (zitto)} ‘stay (quiet)’:

\ea%92
    \label{ex:key:92}

           Italian \citep[122]{Favaro2020}

A:  Io dico BRA campione d’Italia U16 venendo da due anni di dominio U14. Altre previsioni?

B:  Stai \textit{solo} zitto ke porti ancora sfiga!!!

\glt
A:  ‘I say BRA Italian champion U16 after two years of domination in the U14. Other predictions?’

B:  ‘\textit{Just} shut up, you’re gonna jinx it!!!’
    \z % you might need an extra \z if this is the last of several subexamples

The meaning of \textit{solo} in this kind of directive constructions may be said to be threefold. First, it contributes to the CG management, marking a contrast between the speech act and a belief attributed to the addressee. In example \REF{ex:key:92}, the speaker attributes a proposition to the addressee’s mind (it could be \textit{You can say your predictions as well}): in this sense, the illocutive use of \textit{solo} in directives is \textit{polyphonous} (\citealt{Ducrot1984}; \citealt{DetgesGévaudan2018}: 307), since it targets not only the speaker’s viewpoint but also that of the addressee. The speaker contrasts this proposition with the directive, presenting it as the obvious action the addressee should undertake. Second, by highlighting this contrast, \textit{solo} operates on the illocutionary force, giving emphasis to the directive. Finally, as a conversational side effect, it has a closing effect on the conversation: the interlocutor is “discouraged” from continuing the discussion on that topic. Another case in point is example \REF{ex:key:93}:

\ea%93
    \label{ex:key:93}

          Italian \citep[123]{Favaro2020}

[Roberta asks Anna about her schedule for the following day]

  R:  Hai tanto da fare domani?

  A:  Lascia \textit{solo} stare, sono piena tutto il giorno!

\glt
R:  ‘Are you very busy tomorrow?’

A:  ‘Don’t \textit{even} ask, I’m busy all day!’
    \z % you might need an extra \z if this is the last of several subexamples

In this example, again, the speaker attributes a belief to the addressee’s mind (\textit{We could arrange something together}) and contrasts it with an emphatic directive. As a CG management tool, \textit{solo} points to the information ascribed to the addressee and specifies the role of the speech act towards it. Integrating speech acts in the common ground is one of the typical functions of modal particles. As discussed in \chapref{sec:3}, \citet{Waltereit2001,Waltereit2006} analyzes these forms as linguistic items operating at the speech act level: they signal a speech situation where the preparatory conditions of a speech act are not (completely) fulfilled, specifying how the “defective” speech act should be correctly interpreted in that speech situation.

Following these suggestions, the meaning of \textit{solo} in this context of use can be analyzed in terms of specification of the preparatory conditions of the speech act, in order to integrate the new directive into the assumed CG. According to the preparatory conditions of directives, it is not obvious to both the speaker and the addressee that – in the normal course of events – the addressee should do what is expressed by the directive (\citealt{Searle1969}: 66; \citealt{Waltereit2001}: 1403). By contrast, \textit{solo} overtly marks an obvious directive, pointing to (and excluding) a set of propositions activated in the CG, and highlighting the only one – the directive – that the speaker considers to be valid in the speech situation. From the speaker’s perspective, in examples like \REF{ex:key:92} and \REF{ex:key:93}, \textit{solo} signals this friction marking emphasis on a taken-for-granted directive.

In the second kind of illocutionary context, \textit{solo} occurs in assertions conveying evaluations. A first example is \REF{ex:key:94}, taken from a blog discussion about the football transfer market: someone is talking about the possibility that Belotti, a player of the Turin Football Club, might be sold by the club. Here \textit{solo} gives a declarative sentence the character of an exclamation.

\ea%94
    \label{ex:key:94}

          Italian \citep[124]{Favaro2020}

Per me se parte Belotti a certe cifre va \textit{solo} bene: coi suoi soldi si rifarebbe la squadra, modulo offensivo ma con difensori di livello!

\glt ‘In my view if Belotti leaves for good money it’s \textit{just} fine: with his money they could remake the team, an attack formation but with high-level defenders!’
    \z % you might need an extra \z if this is the last of several subexamples

In a similar fashion to its use with directives, \textit{solo} fulfils three functions at once in this kind of assertion. It contributes to the CG management, by signaling a discrepancy between the presented information and some general knowledge that is assumed to be present in the CG (and thus also entertained by the addressee). Second, it strengthens the illocutionary force, marking emphasis on the assertion. Moreover, on the conversational side, the emphatic assertion has a closing potential on the conversation, as if it could express the last word on the current discussion. In the context of \REF{ex:key:94}, the speaker considers a proposition like \textit{Someone thinks that selling Belotti is (not)}\textbf{ }\textit{a good idea} as active in the CG. It is important to note that in this case – as in the next one – both the affirmative and the negative proposition could be at issue, depending on the context. In fact, what the speaker wants to contrast is the possibility that their assertion could be challenged or questioned, and not necessarily one of the two versions (that largely depend on the communicative situation). Another case in point is \REF{ex:key:95}:

\ea%95
    \label{ex:key:95}

          Italian \citep[124]{Favaro2020}

[Giorgio, annoyed by a long discussion with friends]

In effetti, prima di parlare informati, ha \textit{solo} ragione Ceci a dire che ti inventi certe cose!

\glt ‘Actually, before you talk inform yourself, Ceci is \textit{absolutely} right saying that you make up things!’
    \z % you might need an extra \z if this is the last of several subexamples

In the context of \REF{ex:key:95}, a proposition like \textit{Someone thinks that Ceci is (not)}\textbf{ }\textit{right} is active in the CG. As in the previous example, the speaker corrects this proposition with his emphatic assertion, presenting it as the obvious proposition one should take into account. In terms of speech act specification, presenting an assertion as it should be obvious to the addressee is contrary to the preparatory conditions of assertions \citep[66]{Searle1969}. This is the result of excluding the alternative propositions in the CG as non-valid: this way, the \textit{challengeability} (see \citealt{Kroon1995}) of the proposition conveyed by the assertion is cancelled – according to which the speaker recognizes that some opinion can’t be negotiated with the addressee – and no room is left for possible disagreement. The emphatic assertion marked by \textit{solo} is then the only one that is valid and, in this sense, it should be obvious to the addressee: in the speaker’s perspective, in examples like \REF{ex:key:94} and \REF{ex:key:95}, \textit{solo} marks emphasis on a non-challengeable assertion.

In many cases, the proposition contrasted by the speaker corresponds to a common belief so that the emphatic assertion involves some degree of counter-expectation. These features provide a clue as to the explanation of how and in what kind of conversational contexts this use of \textit{solo} can emerge. Following the above line of reasoning, example \REF{ex:key:96} illustrates a possible bridging context from the focus adverb use \textit{solo} to its use as an illocutionary operator:

\ea%96
    \label{ex:key:96}

          Italian \citep[125]{Favaro2020}

  A:  Non ci credo, questo freddo a maggio!

B:  Mah, i tedeschi sono \textit{solo} contenti se anche a maggio ci sono sei gradi, così  possono usare ancora un po’ le loro giacche colorate e i thermos all’università.

\glt
  A:  ‘I can’t believe it, such cold weather in May!’

B:  ‘Well, \textit{actually} Germans are \textit{just} happy if even in May it is six degrees, so that they can still use their colorful jackets and their thermos’ at university.’
    \z % you might need an extra \z if this is the last of several subexamples

In \REF{ex:key:96} it is difficult to ascertain whether \textit{solo} is an exclusive focus adverb. The crucial parameter is again the scope extension, since it is not clear if its scope extends over the predicate or on the whole utterance. The problems of defining the semantics of \textit{solo} in these contexts and the problem of its scope relate to each other: the vagueness of the scope extension (predicate or utterance) corresponds to a vague semantics: its value as a focus adverb and its values as an illocutionary operator are not clearly distinguishable. This meaning vagueness is the consequence of several factors that combine in similar sentences – the starting point being an assertive speech act where an evaluation expressed by the speaker triggers the scalar reading of \textit{solo}. In this evaluative context, there is a mismatch between the conventional meaning associated with the scalar use of \textit{solo} (which – excluding values higher on some scale – is usually associated with low values) and the kind of predicate, that ranks high on a possible scale.\footnote{This fact is reminiscent of the phenomenon of scale inversion, which may be displayed by a scalar exclusive focus adverb when the context expresses a sufficient condition (see \citealt{König1991}: 101; \citealt{ModicomDuplâtre2018}: 81–84). However, neither the examples given for \textit{solo} in directives nor the examples for assertive speech acts are cases of contexts expressing sufficient conditions, and the emergence of the illocutive meaning of \textit{solo} must be explained otherwise. In the case of example \REF{ex:key:96}, the ironic context could also play a role.} This fact contrasts with the normal interpretation of \textit{solo} as an exclusive focus adverb.

Furthermore, \REF{ex:key:96} constitutes a counter-expectation context since it is assumed that people are not happy for the temperature to be six degrees in May. In utterances like \REF{ex:key:96} there is a clash between two contextual factors (the evaluative context and counter-expectation context) and two semantic factors (the conventional meaning of \textit{solo} and a high-ranking predicate). This sum of factors constitutes a fixed argumentative move (\citealt{DetgesWaltereit2016}; see the discussion in \chapref{sec:4}) through which speakers takes advantage of a slightly deviating use of \textit{solo} to index CG information and correct it with their emphatic assertion. The progressive routinization of this construction is accompanied by the contextual syntactic reanalysis of \textit{solo} from focus adverb at the sentence level to emphatic operator targeting the illocution of the utterance, and pragmatically used for argumentative purposes.

\subsection{The questionnaire}
\hypertarget{Toc124860671}{}
So far, I have described the semantic/pragmatic features of the modal uses of \textit{solo}, pointing out their contexts of occurrence and their function within the speech act. Nevertheless, there are some questions which this kind of analysis – that is, the description of examples in context, largely based on personal interpretation – cannot answer.

First, the issue about the sociolinguistic markedness of the modal uses, that is, how standard and regional uses can be distinguished and how they are placed along other dimensions of variation. Second, the issue of how to describe the relationship between focusing and modal uses of \textit{solo} – both from a synchronic point of view and with reference to the emergence of the new functions. This touches on topics such as the management of contextual inferences, conventionalization, and reanalysis. In this respect, also the question of whether the uses in directive and assertive speech acts belong to the same evolutionary path or to different ones can be raised.

The research instrument used to investigate these issues is a sociolinguistic questionnaire, aimed at collecting speakers’ judgments on the modal uses of \textit{solo}. It consists of 12 stimuli: 6 utterances where the adverb appears in a directive construction and 6 utterances where it appears in assertions. Two stimuli of each category were inserted as controls, as standard uses of \textit{solo} as a focus adverb. Most of the stimuli have been proposed in the form of adapted cartoons, so that there was enough context to clarify what reading we wanted to suggest. The questionnaire collected 570 answers ({April 2018} – {September 2018}).\footnote{See \chapref{sec:5} for general information about the questionnaire design. In the next pages, the structure of the questionnaire and the stimuli are translated into English: the original Italian version can be accessed online at \url{https://zenodo.org/records/10362289}. The questionnaire has been developed in collaboration with Eugenio Goria, a colleague at the University of Turin. I worked with him both for the design of the questionnaire and for a first round of analysis of the results (more focused on sociolinguistic issues). Part of that work has been presented in \citet{FavaroGoria2019}.}

The questionnaire is divided into two parts. In the first one, for each stimulus, the respondents are invited to comment on the use of specific constructions. Concerning this task, I use the label \textit{reported language use} to refer to the (reported) familiarity with these uses – expressed through a personal evaluation of them. This parameter includes an evaluation concerning the “passive familiarity” with a construction (“Have you ever heard such a sentence?”) and an evaluation concerning the “active familiarity” (“Do you use such a sentence?”). For four stimuli – two directives and two assertions – the type of sociolinguistic markedness perceived was also asked for (“In what kind of context are you likely to hear similar sentences?”).

In the second part, the kind of meaning attached to the utterance was investigated. Two more questions were asked for eight of the stimuli (three directives, three assertions plus the two controls). The first one is an open question (“Would it make a difference if the sentence were without \textit{solo}?”) where the respondents can provide a free reading of the proposed stimulus, thus suggesting what kind of inferences and secondary meanings they link to it. This is useful to test if the speakers’ insights match our own hypotheses. The second one is a multiple-choice question with three possible answers: a paraphrase expressing emphasis on the illocutionary force, a paraphrase expressing management of the common ground, and finally the possibility to choose both meanings for the proposed stimulus or something else.\footnote{The possible paraphrases necessarily reflect personal choices. Reducing a meaning category to a paraphrase represents a weak point of this methodology. However, it is interesting since it allows us to understand how speakers construe the meaning of the utterance and hence what kind of discourse inferences are activated in that context.} The general structure of the questionnaire is summed up in \tabref{tab:key:8.1}.

\begin{table}
\begin{tabularx}{\textwidth}{lXX}
\lsptoprule
 & Question & Answer\\
 \midrule
Reported language use & Have you ever heard such a sentence? & {}--- often
\newline {}--- sometimes
\newline {}--- never\\
& Do you use such a sentence? & {}--- often
\newline {}--- sometimes
\newline {}--- never\\
\tablevspace\\
Sociolinguistic
status & In what kind of context are you likely to hear similar sentences? & open response\\
\tablevspace\\
Meaning evaluation & Would it make a difference if the sentence were without \textit{solo}? & open response\\
%\hhline%%replace by cmidrule{-~~}
\tablevspace\\
& With this sentence the speaker means… & 3 possible alternatives
+ open response\\
\lspbottomrule
\end{tabularx}
\caption{\label{tab:key:8.1} Structure of the questionnaire on the illocutive uses of \textit{solo}}
\end{table}

As discussed in \chapref{sec:5}, sociolinguistic issues are often fundamental to giving an in-depth description of Italian modal-particles-like elements. The present case study on \textit{solo} will mainly refer to diatopic variation, but future research on this or similar constructions could take into consideration other social variables – possibly developing questionnaires which focus on specific issues. This questionnaire aimed at collecting data (and give a description) both on the sociolinguistic distribution of these constructions and the development of their functions. In this respect, questionnaires combining acceptability judgments and meaning evaluations represent a promising methodology – and are consistent with the idea that variation and change of linguistic structures are deeply interwoven.

\newpage
\section{Reported language use and regional variation}
\largerpage[-4]
\hypertarget{Toc124860672}{}
This section deals with the sociolinguistic features of the modal uses of \textit{solo}. This was investigated in the first part of the questionnaire, which contained evaluations about the acceptability of the constructions and open questions about their sociolinguistic status. Through this, I wanted to collect information about the diatopic variation of these constructions. As mentioned in the last section, the questionnaire consists of 12 stimuli: six directives and six assertions (for each category, one stimulus represents a prototypical use of \textit{solo} as a focus adverb). The distinction between these two illocutive contexts constituted one of the assumptions that shaped the design of the questionnaires (and that was further confirmed by the answers collected). The discussion in the next pages follows this basic distinction. In this section, I will first show the results concerning the overall reported language use of the modal uses of \textit{solo}. Subsequently, I will deal with the issue of diatopic variation, starting from the hypothesis that the modal uses of \textit{solo} are more common in the regional variety of Italian spoken in Piedmont than in other regional varieties.

\subsection{Overall reported language use}
\hypertarget{Toc124860673}{}
In the first part of the questionnaire, the acceptability\footnote{For reasons of convenience, I will sometimes use the more common label \textit{acceptability} instead of \textit{reported language use}, which is however not fully suited to describe the kind of judgement that the respondents were requested to give. In this respect, the label \textit{acceptability} must be understood as “familiarity with a construction”.} of the illocutive uses of \textit{solo} was investigated. For each stimulus, the respondents were invited to evaluate both the passive familiarity with the stimulus featuring the adverb (“Have you ever heard such a sentence?”) and the active familiarity with it (“Do you use such a sentence?”). There are three possible answers: \textit{spesso} ‘often’, \textit{qualche volta} ‘sometimes’ and \textit{mai} ‘never’. I will focus here on the results concerning passive acceptability. In fact, the tendency in both sets of answers is substantially the same, with the difference that the results concerning the passive acceptability of the constructions are always slightly higher than those concerning its active use. As can be expected in a sociolinguistic questionnaire, respondents are more likely to admit that they recognize a construction rather than using it actively. The boxplots in \figref{fig:key:8.1} and \figref{fig:key:8.2} show the overall results of the (passive) acceptability of the two illocutionary contexts.\footnote{The proposed stimuli are utterances along the lines of examples (91–96) discussed in the previous section, which have been modelled after constructed examples evaluated in previous research, and real examples extracted from the web or heard in every-day conversations. In the presentation of the results – in this and in the following cases – the relevant utterances are shown beside the boxplots. For the whole stimuli, see the original version of the questionnaire (online at \url{https://zenodo.org/records/10362289}).}

\begin{figure}
\includegraphics[height=.42\textheight]{figures/FavaroLSIfinal-img001.png}


{\small\raggedright
D1 = \textit{devi solo avere pazienza!} ‘you only need to be patient’

D2 = \textit{sparisci solo!} ‘just beat it, just get out of here’

D3 = \textit{levati solo!} ‘just geat out of the way’

D4 = \textit{stai solo zitto!} ‘just shut up’

D5 = \textit{lascia solo stare!} ‘just give it up, just don’t bother’

D6 = \textit{lasciami solo in pace!} ‘just leave me alone’\par
}
\caption{\label{fig:key:8.1} Directives with \textit{solo}: “Have you ever heard such a sentence?”}
\end{figure}

\begin{figure}
\includegraphics[height=.45\textheight]{figures/FavaroLSIfinal-img002.png}


{\small\raggedright
A1 = \textit{lo spero solo} ‘I just hope so’

A2 = \textit{è solo bello} ‘it’s just nice’

A3 = \textit{sono solo contenti} ‘they are just happy’

A4 = \textit{ha solo ragione} ‘she is absolutely right’

A5 = \textit{mi farebbe solo piacere} ‘I am just glad’

A6 = \textit{va solo bene} ‘it’s just fine’\par}

\caption{\label{fig:key:8.2} Assertions with \textit{solo}: “Have you ever heard such a sentence?”}
\end{figure}

In the graphs, the labels D1–D6 and A1–A6 on the horizontal axis (labeled \textit{(sub)corpora}) correspond to the proposed stimuli and the numerical values on the vertical axis (labeled \textit{linguistic variable}) to the possible answers: 1.0 corresponds to “never”, 2.0 to “sometimes” and 3.0 to “often”. The collected answers are graphically represented by the boxplots, obtained through the Lancaster Stats Tool Online \citep{Brezina2018}. The box graphically represents the area where most answers are concentrated, and the more or less dense blue ovals represent the number of answers for each value. The bold black lines (which allow a quick comparison of the results) correspond to the median value of the answers of each stimulus, while the red line represents the mean value.\footnote{The use of numerical values to represent the responses might give the impression that an ordinal variable (the ordered rank “never”, “sometimes”, “often”) has been treated here as an interval scale (1, 2, 3). This is a methodological simplification (equal intervals on the number scale do not represent equal differences between the responses). In this case, numeric values have been associated to each of the possible answers for the purposes of data visualization rather than for an actual quantitative evaluation of the results. In fact, the representation of the mean value – which is quite uncommon for boxplots – represents an in-built feature of the boxplots obtained through the Lancaster Stats Tool online (and for this, the software requires numerical values to work). Moreover, no statistical significance testing has been used for these distributions. In this regard, even if occasional reference to the mean values will be made, the boxplots in the next pages must be interpreted primarily using median values (as well as the position of the boxes).}

From\largerpage{} the comparison of the two graphs a few important remarks can be made. The first box in each graph corresponds to the control stimulus (standard uses of \textit{solo} as a focus adverb): they clearly result as being more acceptable than the others, as the median value of the answers is 3.0, which corresponds to the answer “often”. However, otherwise, the results are quite different across the two series: the use of \textit{solo} in directives is overall less acceptable than its use in assertions. If we focus on the results of \figref{fig:key:8.1}, we notice that the proposed stimuli attained quite mixed values (D2, D3 and D5 have 1.5, D4 is over 2.0 and D6 just below) and they are overall much lower than the control stimulus.

The picture looks different for the answers in \figref{fig:key:8.2}: in fact, the proposed stimuli – with the exception of A2 (\textit{è solo bello} ‘it’s just nice’) – attain values not too far from the value of the controls (mean values are between 2.0 and 2.5, that is, over the threshold of “sometimes”). This observation allows us to draw a first general conclusion: the non-focusing uses of \textit{solo} were evaluated as less acceptable in utterances expressing directives and more acceptable in utterances expressing assertions. The directives then result in more marked constructions than the assertions, further away from the prototypical use of \textit{solo}, and are probably more prone to show facts of linguistic variation.

As regards the differences within each set, a few observations can be made. In \figref{fig:key:8.1}, examples D2 and D3 are directive speech acts where \textit{solo} appears in simple imperative constructions: \textit{sparisci solo} ‘just beat it’ and \textit{levati solo} ‘just get out of the way’. On the other hand, D4, D5, and D6 correspond to partially conventionalized expressions: \textit{stai solo zitto} ‘just shut up’, \textit{lascia solo stare} ‘just give it up, just don’t bother’, \textit{lasciami solo in pace} ‘just leave me alone’. However, the acceptability values don’t allow to clearly separate these two subgroups. In fact, D2, D3, and D5 show a similar value and they rank lower than D4 and D6: this suggests that the acceptability depends primarily on the features of single stimuli.

Moving to \figref{fig:key:8.2} – in a similar fashion – A2 and A3 are assertive speech acts where \textit{solo} appears in a predicative construction of the type \textit{to be} + \textsc{adj}: \textit{è solo bello} ‘it’s just nice’, \textit{sono solo contenti} ‘they are just happy’. On the other hand, A4, A5, and A6 represent multi-word expressions – partially conventionalized verbal phrases like \textit{va solo bene} ‘it’s just fine’, \textit{ha solo ragione} ‘she is absolutely right’, and \textit{mi fa solo piacere} ‘I am just glad’. In this case, these expressions turn out to be slightly more acceptable than the non-conventionalized ones: this can probably be explained with higher frequency of occurrence in the common language. Despite this, while A3, A4, A5 and A6 attain similar values, A2 ranks lowest on the scale and thus represents the exception in this group: again, it seems that the acceptability depends primarily on the specific features of each stimulus.

\subsection{Reported language use across regions}
\hypertarget{Toc124860674}{}
As part of the analysis of the reported language use, I wanted to investigate aspects of the sociolinguistic markedness of these constructions. In particular, an important point is to establish whether the acceptability of these constructions displays regional variation. As pointed out in the preceding section, previous research advanced the hypothesis that the illocutive uses of \textit{solo} are mostly found in the regional variety of Italian spoken in Piedmont \citep{Favaro2019}. The data collected through the questionnaire can be used for this purpose. 

Since there were 120 answers from Piedmont, I organized the data in order to get comparable groups. To achieve this, I excluded the regions with less than 30 answers, and I aggregated the remaining answers in two groups based on the region of origin. This way, I got three groups with similar numbers: a set of answers from Piedmont (120), a set of answers from other northern Italian regions (130, from Lombardy, Veneto and Emilia-Romagna), and a set of answers from central-southern Italy (125, from Lazio, Apulia and Sicily). The results are graphically represented by boxplots, one for each context. Figures \ref{fig:key:8.3}–\ref{fig:key:8.6} show the boxplots of four stimuli containing directives (D2, D3, D4 and D5).

The boxplots show that the median value (bold black line) of the answers from Piedmont ranks higher than the median values of the two other groups for each of the contexts under examination.{\interfootnotelinepenalty=10000\footnote{Note that the other boxes (non-Piedmont) contain different regions each. This grouping (and the method chosen) runs the risk of equalizing the differences between these regions. However, previous research (\citealt{FavaroGoria2019}) showed that these differences are not significant. This issue is also addressed by the case study in \chapref{sec:9}, which separately considers four regions of northern Italy.}} This way, they confirm the assumption that the modal use of \textit{solo} in directives is more acceptable in Piedmont than in the other two groups. This is particularly evident for the first two contexts (D2 and D3) – corresponding to simple imperatives – and slightly less pronounced in the case of semi-conventionalized multi-word imperatives (D4 and D5). On the other hand, the values extrapolated from the answers of the northern group and those of the central-southern group are very similar across the different stimuli. This provides strong evidence in favor of assigning this specific use of \textit{solo} to the regional variety of Italian spoken in Piedmont. 

The picture is rather different when looking at Figures \ref{fig:key:8.7}–\ref{fig:key:8.10}, which show the boxplots of four stimuli containing assertive speech acts (A2, A3, A4 and A6).

The boxplots show similar distributions for each group in each of the contexts under consideration. In addition to being more acceptable overall, the modal uses of \textit{solo} in assertive speech acts also show almost no regional variation. Even its use in predicative constructions (A2 and A3) is not considered acceptable in Piedmont. At most, a difference could be identified between northern and central-southern regions: in Figures \ref{fig:key:8.8}, \ref{fig:key:8.9} and \ref{fig:key:8.10} the median value corresponds to “often” for Piedmont and the northern group, and to “sometimes” for the central-southern group. Nevertheless, on the basis of these data, the modal uses of \textit{solo} in assertive speech acts cannot be assigned to any specific regional variety of Italian – thus resulting in a feature which can be found in the standard variety and/or in different regional varieties spoken across the peninsula.\largerpage[2.25]

These observations contribute to further mark the difference between the two illocutionary contexts where the modal uses of \textit{solo} can be found: both the results concerning overall acceptability and the regional markedness trace a clear division between the two sets. This issue will arise again in the next section, which deals with the meaning of these constructions. Before that, I will briefly conclude this section with some more observations about the sociolinguistic status characterizing the illocutive uses of \textit{solo}.

\begin{figure}[p]
\includegraphics[height=.45\textheight]{figures/FavaroLSIfinal-img003.png}

\caption{\label{fig:key:8.3} D2 (\textit{sparisci solo}): regional variation}
\end{figure}

\begin{figure}[p]
\includegraphics[height=.45\textheight]{figures/FavaroLSIfinal-img004.png}

\caption{\label{fig:key:8.4} D3 (\textit{levati solo}): regional variation}
\end{figure}

\begin{figure}[p]
\includegraphics[height=.45\textheight]{figures/FavaroLSIfinal-img005.png}

\caption{\label{fig:key:8.5} D4 (\textit{stai solo zitto}): regional variation}
\end{figure}

\begin{figure}[p]
\includegraphics[height=.45\textheight]{figures/FavaroLSIfinal-img006.png}

\caption{\label{fig:key:8.6} D5 (\textit{lascia solo stare}): regional variation}
\end{figure}

\begin{figure}[p]
\includegraphics[height=.45\textheight]{figures/FavaroLSIfinal-img007.png}

\caption{\label{fig:key:8.7}: A2 (\textit{è solo bello}): regional variation}
\end{figure}

\begin{figure}[p]
\includegraphics[height=.45\textheight]{figures/FavaroLSIfinal-img008.png}

\caption{\label{fig:key:8.8} A3 (\textit{sono solo contenti}): regional variation}
\end{figure}

\begin{figure}[p]
\includegraphics[height=.45\textheight]{figures/FavaroLSIfinal-img009.png}

\caption{\label{fig:key:8.9}: A4 (\textit{ha solo ragione}): regional variation}
\end{figure}

\begin{figure}[p]
\includegraphics[height=.45\textheight]{figures/FavaroLSIfinal-img010.png}

\caption{\label{fig:key:8.10}: A6 (\textit{va solo bene}): regional variation}
\end{figure}


\subsection{What kind of sociolinguistic status?}
\hypertarget{Toc124860675}{}
For four stimuli (D2, D4, A2 and A6) an open-ended question was also proposed, namely “In what kind of context are you likely to hear similar sentences?”. The aim of having this question was to collect personal evaluations and direct opinions – influenced by the researcher’s perspective as little as possible – on the kind of sociolinguistic markedness attributed by speakers to the illocutive uses of \textit{solo}. Obviously, this was not meant to be a systematic account, but rather a way to get some clues as to various point of views. No prevailing opinion emerged from the answers, but they are interesting in order to get a sense of the respondents’ concept of sociolinguistic variation. In fact, the answers – which are very disparate (and often rather playful) – revolve around three main directions, which correspond to actual dimensions of variation described by sociolinguistic theory: diatopic variation, diaphasic variation and variation across age groups.\footnote{I could not find clear mentions to diastratic variation, unless we consider certain social groups as the expression of specific social classes. In fact, a couple of respondents answered that you can hear similar expressions among \textit{zarri} ‘thugs, rough guys’.}

The only answer which reaches a large consensus concerns the diaphasic variation. Many respondents assign these (and similar) expressions to conversations in informal settings. Some respondents further specify that they are found in conversations among peers. With respect to other dimensions of variation, no answer reaches such a large consensus. Although some respondents assign these expressions to the regional variety of Italian spoken in Piedmont (some of them even describe them as calques of the Piedmontese dialect), a few others assign them to southern varieties. In fact, this seems to correspond to the results of the regional acceptability: although more acceptable in Piedmont, these expressions are not totally ruled out in other varieties. Lastly, some respondents consider these expressions as a feature of the informal varieties used by adolescents and young people. However, other respondents include adults and older people in their answers. To conclude, apart from the right (but not surprising) assignment of the illocutive uses of \textit{solo} to informal conversations – especially among peers – the open-ended question has not provided consistent answers about the sociolinguistic status characterizing these constructions.\footnote{If nothing else, the answers show that the respondents have a clear idea of the dimensions of sociolinguistic variation to which these and similar expressions can possibly be subjected to.}\clearpage

\section{A look into the emergent functions of \textit{solo}}
\hypertarget{Toc124860676}{}
The sociolinguistic analysis of the previous section – mainly based on acceptability judgments – gave two main results. First, it revealed a clear difference between the two illocutive contexts – the illocutive use of \textit{solo} in directives resulting less acceptable than its use in assertions. Second, the illocutive use of \textit{solo} in directives – although found across different regions – is more acceptable for speakers from Piedmont and it can be considered an especially widespread feature in Piedmontese Regional Italian. Such a difference was not found for the assertions, which present a more stable distribution across different regions. Keeping this in mind, I now turn to a semantic/pragmatic description of the modal uses of \textit{solo}, by analyzing the answers of the second part of the questionnaire (namely, the part about meaning evaluation). This part of the analysis builds upon a limited dataset. As has already been said, the second part of the questionnaire covers only eight stimuli (D1, D2, D4, D5 for the directives, A1, A3, A4, A6 for the assertions). Moreover, I will only consider the 120 answers of respondents from Piedmont, since they seemed to have more familiarity with these constructions.

The analysis builds upon the concepts discussed in \chapref{sec:4}: inferences in interaction, reanalysis and conventionalization. In this regard, synchronic studies have the advantage of enabling the investigation of the actual mechanisms through which inferences are dealt with in the actions of the participants. In fact, compared to diachronic data, they can give a better understanding of the contextual meanings and of the inferences that can be drawn in conversation, allowing one to capture subtle meaning variations and their position in the conventionalization route. In light of this, I use the term \textit{emergent functions} in the sense of “appearing/emerging in specific contexts”, that is, a function which is still context-bound and not yet fully conventionalized.

\subsection{Open questions: Detecting inferences}
\hypertarget{Toc124860677}{}
I focus now on the results concerning the meaning of the constructions, addressed by the second part of the questionnaire. As a first step I analyzed the answers to the open questions (“Would it make a difference if the sentence were without \textit{solo}?”). The goal of this part was to provide a space where respondents could give a free reading of the proposed stimulus, using their own categories and expressing their own insights. Analyzing the answers, the main aim was to identify what kind of interpretation the respondents give to these utterances and their contexts, thus throwing light on what kind of inferences and secondary meanings they link to it. Many suggestions have arisen, and it is not possible to give an overview of all of them, but the great majority of answers match the working hypothesis of two “clouds” of emerging meanings, the first one related to the emphatic marking of the illocutionary force, the second related to common ground management. \tabref{tab:key:8.2} shows some relevant examples.

\begin{table}
\small
\begin{tabularx}{\textwidth}{lQQQ}
\lsptoprule
Stimulus & Exclusive reading & Illocutionary-force reading & Common-ground reading\\
\midrule
D1 & Without \textit{solo} it would mean that patience is not the only thing you need to do that activity. & Here \textit{solo} reinforces the concept. & —\\
\tablevspace
D2 & — & It has reinforcing value. & With \textit{solo} we understand that Hobbes has said what Calvin thought.\\
\tablevspace
D4 & — & It would be less emphatic. & In this case \textit{solo} contributes to make sense of the second part of the sentence. Without it, there would be no connection.\\
\tablevspace
A1 & — & It would be less emphatic. & —\\
\tablevspace
A3 & — & The sentence would be less strong. & Here \textit{solo} implies an unexpected contrast between the two opinions.\\
\tablevspace
A4 & — & Here \textit{solo} reinforces her stance. & Without \textit{solo} there would be no direct comparison between what is happening in that moment and what Cecilia says usually happens.\\
\lspbottomrule
\end{tabularx}
\caption{\label{tab:key:8.2} Answer to the open question: “Would it make a difference if the sentence were without \textit{solo}?”}
\end{table}

As expected, for the context D1 (\textit{devi solo aver pazienza} ‘you only need to be patient’) – that is, the control stimulus – a prototypical exclusive reading is found (“Without \textit{solo} it would mean that patience is not the only thing you need to do that activity”) and an emphatic reading (“Here \textit{solo} reinforces the concept”), suggesting that this inference is the first one to come into play. In the other contexts no exclusive reading is mentioned, and most answers suggest an emphatic reading, related to the marking of the illocutionary force: for example, “It has reinforcing value” for D2 (\textit{sparisci solo!} ‘just beat it, just get out of here’) or “It would be less emphatic” for D4 (\textit{stai solo zitto!} ‘just shut up’). At the same time, however, some respondents suggest a different kind of reading, which seems to be related to common ground management, that is, a reference to some proposition activated in the context of exchange or attributed to the interlocutor’s mind: for example, “With \textit{solo} we understand that Hobbes has said what Calvin thought” (referring to the two characters in the cartoon) for D2 or “In this case \textit{solo} helps us making sense of the second part of the sentence. Without it there would be no connection [with the first one]” for D4.

This picture also applies to the answers regarding the assertions with some minor differences. In the context A1 (\textit{lo spero solo!} ‘I just hope so’) – the control context – the emphatic reading clearly prevails over the exclusive one. In the other contexts, the emphatic reading is always present, but many respondents give answers explicitly attributable to a common ground reading, like “Here \textit{solo} implies an unexpected contrast between the two opinions” for A3 (\textit{sono solo contenti} ‘they are just happy’) and “Without \textit{solo} there would be no direct comparison between what is happening in that moment and what Cecilia says usually happens” for A4 (\textit{ha solo ragione} ‘she is absolutely right’).

The analysis of the answers to the open question allows us to reach some conclusions. First of all, it confirms the starting hypothesis that the emergent functions of \textit{solo} are linked to two different domains, the marking of illocutionary force on the one hand, and common ground management on the other hand. Overall, the first domain clearly prevails in the answers, but it is remarkable that some respondents explicitly mention the common-ground-related functions.\footnote{This was not necessarily an expected result. Functions related to common ground management are quite elusive and their identification requires some attention. By explicitly mentioning it in the answers, the respondents demonstrate the importance of this feature (as well as a high degree of linguistic self-awareness).} I keep on calling them \textit{emergent functions} because it is almost impossible to identify contexts in which a reading based on the notion of exclusiveness (the main semantic feature of the prototypical use of \textit{solo} as a focus adverb) is totally ruled out. Per contra – with the self-explaining exception of the control contexts – this kind of reading is never overtly mentioned by the respondents. For this reason, it is better to explain these emergent functions in terms of contextual meanings, still linked to inferences activated in the context of interaction but already on the conventionalization path.

Now, some more issues need to be considered. Do the inferences equally appear in directives and assertions? Can the two inferences be combined or are they mutually exclusive? What do they reveal about the conventionalization paths of these constructions? In the next section I will try to give some possible answers to these questions.

\subsection{Multiple-choice questions: Sorting inferences}
\hypertarget{Toc124860678}{}
Moving to a quantitative view of the multiple-choice questions, no major differences in the distribution of the functions across directives and assertives can be found.


\begin{figure}
\includegraphics[height=.3\textheight]{figures/FavaroLSIfinal-img011.png}
\caption{\label{fig:key:8.11} Bar plot of the functions of \textit{solo} in directives and assertions}
\end{figure}

For these counts, I cross-referenced the answers regarding the meaning with the answers about acceptability. I only considered the respondents who stated that they heard these constructions “sometimes” or “often”. This slightly reduces the number of answers, but has the advantage of excluding potentially “sloppy” answers from respondents who don’t recognize the constructions under analysis. As for the overall frequency of the answers – now excluding the control stimuli D1 and A1 – both in the directives and in the assertions, the emphatic reading (in \figref{fig:key:8.11} labeled as \textsc{ill}, which stands for illocutionary force) prevails over the common ground reading (in \figref{fig:key:8.11} labeled as \textsc{cg}), which is still well represented.\footnote{These are the exact values: considering the directives, 141 answers for ILL, 103 for CG and 53 for both; considering the assertions, 171 answers for ILL, 126 for CG and 43 for both.}

However, a closer look at the single contexts complicates the picture. The mosaic plots in \figref{fig:key:8.12} and \figref{fig:key:8.13} show the distribution of the three possible meaning options for each stimulus.\footnote{The mosaic plots use the χ\textsuperscript{2}{}-statistic and have been created through the software of statistical analysis R (\citealt{R_core_team2020}). See also \citet[199–222]{Levshina2015}.} Overall, the results point to an irregular distribution of the three possible meaning options across the contexts. In the case of directives, for example, the common ground reading dominates in context D4 (\textit{stai solo zitto}), whereas in context D5 (\textit{lascia solo stare}) the emphatic reading clearly prevails. Context D2 (\textit{sparisci solo}) shows a more balanced situation.

\vfill
\begin{figure}[H]
 \includegraphics[height=.3\textheight]{figures/FavaroLSIfinal-img012.png}


\begin{tabularx}{.6\textwidth}{XrYY}
\lsptoprule
 & D2 & D4 & D5\\
 \midrule
 ILL & 37 & 40 & 64\\
 CG & 32 & 56 & 12\\
 both & 24 & 20 & 9\\
 TOT & 93 & 116 & 85\\
\lspbottomrule
\end{tabularx}
\caption{ Mosaic plot of the functions of \textit{solo} in directives}
\label{fig:key:8.12}
\end{figure}
\vfill
\pagebreak

The assertions also show an irregular distribution. In this case, the common ground reading dominates in context A3 (\textit{sono solo contenti}), whereas in context A4 (\textit{ha solo ragione}) the empathic reading prevails. Context A6 (\textit{va solo bene}) shows a more balanced situation.



\begin{figure}
\includegraphics[height=.3\textheight]{figures/FavaroLSIfinal-img013.png}


\begin{tabularx}{0.6\textwidth}{XrYY}
\lsptoprule
 & A3 & A4 & A6\\
\midrule
 ILL & 32 & 84 & 55\\
 CG & 62 & 22 & 42\\
 both & 19 & 10 & 14\\
 TOT & 113 & 116 & 111\\
\lspbottomrule
\end{tabularx}
\caption{Mosaic plot of the functions of \textit{solo} in assertions}
\label{fig:key:8.13}
\end{figure}

In these graphs the color of the shading corresponds to the sign of the residuals, that is, the differences between the observed and expected frequency divided by the square root of the expected value. Positive residuals (frequency is greater than what can be expected by chance) are indicated by blue rectangles, negative residuals (frequency is smaller than what can be expected by chance) by pink rectangles. The analysis reveals significant differences in the functions assigned to \textit{solo} by respondents in different contexts.

\hspace*{-.1pt}Nevertheless, also considering the irregular distribution of the functions across the single stimuli and the two broader illocutionary contexts, it is hard to identify an explanatory variable for this distribution other than the specificities of each context of occurrence: some contexts favor an illocutionary force reading (D5, A4), other contexts favor a common ground reading (D4, A3). For this reason, it is not possible to hypothesize a single path of development from the exclusive meaning to the emphatic reading and then to the common ground reading (or the other way around). It is probably better to conceive two parallel paths~– corresponding to different inferences that can both arise from the use of \textit{solo} as a focus particle in specific conversational contexts – leading to different readings. However, they can co-exist in the same construction as different shades of meaning, which can be foregrounded or backgrounded according to the context of interaction.

\subsection{Closing remarks: Conventionalization in the modal uses of \textit{solo}}
\hypertarget{Toc124860679}{}
The empirical research conducted through the questionnaire gave several results about the distribution and the meanings of the modal uses of \textit{solo}. They appear in two illocutionary contexts – directive speech acts and assertive speech acts – which show both similarities and differences.

The main difference corresponds to their geographic distribution: although both were found across different regional varieties of Italian, the use in directive speech acts is mainly found in the regional variety spoken in Piedmont, while the use in assertive speech acts does not clearly show diatopic markedness. Regarding their meaning, the two illocutionary contexts are rather similar. I have described the properties of two different emerging functions: an emphatic reading – when the adverb mainly strengthens the illocutionary force – and a common ground reading – when the adverb contributes to signal a contrast between its host utterance and some proposition activated in the common ground.

The analysis of the answers of the third part of the questionnaire (open questions and multiple-choice questions about the meanings of these constructions) showed that the emphatic reading is the most frequent one. However, the common ground reading is also well represented and, in some cases (D4 and A3 specifically), it is even more frequent in the answers. The absence of correlation between the kind of speech act (directives or assertions) and a specific reading, supports the hypothesis that the emergent functions differ according to the specific contexts of the stimulus rather than with respect to the speech act they occur in: different inferences arise in different conversational contexts. The emergence of new functions can then be described as the sum of minor semantic changes mediated by the gradual conventionalization of discourse inferences – which correspond to different facets of meaning in the emergent uses: those more linked to the expression of the illocutionary force and those more linked to the management of the common ground.

These observations are linked with the broader discussion on the meaning of focus adverbs, which turned out to be an exemplar case study to investigate structural indeterminacy. These adverbs prototypically have scope over sentence constituents and act as a modifier of focus, but they can also extend their scope over the illocution – projecting the proposition over a background of other propositions activated in the common ground. Only in few cases is there structural evidence demonstrating that these particles operate at different grammatical layers, but in the clearest ones (for instance in the case of \textit{pure}) they show different meanings according to the layer they operate on. Moreover, the issue of structural evidence is not essential – at least at this point – and it is more informative to focus on how speakers face language usage than on the kind of schematic abstractions they derive from it.

As pointed out by \citet[43]{Smet2014}: “Especially where the evidence is dubious […] the syntactic structure language users assign may simply leave the problematic aspects of structure unspecified”. In this respect, the ambiguous uses of \textit{solo} discussed above – characterized by variable syntactic scope – show that underspecified syntactic patterns are an ideal locus for language change. Underspecification also plays a role at the semantic level: the analysis of the answers revealed an overlap of functions, whereby different facets of meaning often co-exist with the same stimulus. Illocutionary operators contribute, on the one hand, to the modification of the illocutionary force, and on the other hand, to managing the relationship between the utterance they appear in and the context of interaction. I see this fact as the natural consequence of meaning negotiation in interaction, the locus where the selection of contextual meanings takes place. The simultaneous presence of different readings in context is due to the activation of different inferences, driven by the hearer’s attempt to correctly interpret the speaker’s utterance.

As shown by the results, the type of conversational context plays a crucial role in defining which meaning prevails. With increase in frequency, some contexts of use can turn into semi-fixed argumentative routines that speakers can use to index common ground information or to modify the illocutionary force of a speech act: this way, the bond between a construction and a specific (interactional) function gets stronger. In conclusion, the new functions of \textit{solo} are interpreted as inferences organized along a cline of conventionalization: arising from the use of focus adverbs in discourse, they are progressively incorporated in its conventionalized meaning.

