\chapter{At the semantics/pragmatics interface: Meaning, change, variation}\label{sec:4}
\addtocontents{toc}{\protect\enlargethispage{2\baselineskip}}
\hypertarget{Toc124860626}{}\section{The boundary between semantics and pragmatics}
\hypertarget{Toc124860627}{}
In the last chapter, the main features of modal particles have been defined and illocutionary modification has been identified as the grammatical category that subsumes their functions. On the one hand, the occurrence of modal particles in different kinds of speech acts provides a key to classify their prototypical uses; while on the other, the indexical relationship they establish with the conditions underlying the performance of speech acts provides a key to investigate their effect on illocutionary force.

Nevertheless, reference to speech act theory and to a layered model of grammar that features a category called illocutionary modification is not sufficient to fully grasp the issue of the \textit{meaning} of modal particles. The functional features of these elements – as many other discourse-pragmatic elements – call into question the interplay between coded meanings and different kinds of inferred meanings (both from a synchronic and a diachronic perspective).

While adverbial forms that operate in the discourse-pragmatic domain show a strong conventional meaning that constrains their use, the typical polyfunctionality of these elements reflects different kinds of implicatures which arise from their use in interaction. Thus, exploring the role of contextual inferences is crucial to describe not-fully-conventionalized uses of adverbs in the discourse-pragmatic domain and to hypothesize patterns of semantic change.

As a consequence, the question arises of how to describe the different contribution of conventional and non-conventional levels of meaning to the behavior of discourse-pragmatic elements. These observations lead to the more general issue of defining the boundaries between semantics and pragmatics. For the purposes of the present research, the heart of the matter can be summarized like this: depending on which criteria are chosen to draw the distinction between semantics and pragmatics, the functions of discourse-pragmatic elements can be assigned to one domain or the other.

In the next pages, drawing on works like \citet{Hansen1998a,Hansen2008,Hansen2012} and \citet{Ariel2008,Ariel2010}, I will explore the issue of the border between pragmatics and semantics and how this influences our understanding of the meaning of discourse-pragmatic elements.

\subsection{Drawing the semantics/pragmatics boundary}
\hypertarget{Toc124860628}{}
Following the French tradition initiated by \citet{DucrotEtAl1980}, \citet{Hansen2008} doesn’t primarily conceive of the distinction between semantics and pragmatics as a distinction between elements that contribute to the truth-conditional meaning of a sentence and elements that do not:

\begin{quote}
Rather, semantic meaning is that which is – or appears to be – coded in linguistic expressions, while pragmatic meaning is the interpretative “surplus” that remains when we subtract semantic (or coded) meaning from that which is taken to be the object of a given speaker’s communicative intentions in a given context. Pragmatic meaning arises as a result of the interaction between coded meanings and the linguistic co-text and situational context in which they appear. Hearers may be assumed to arrive at an interpretation of the pragmatic meaning of a given utterance by attempting to unify the coded meanings of the words and constructions that make up that utterance with what they know (or have reason to believe is the case) about its co-text and context. (\citealt{Hansen2008}: 12–13)
\end{quote}

In this perspective, the distinction between semantics and pragmatics is redesigned as a distinction between conventional meanings, intrinsically bound to linguistic expressions, and spontaneous meanings, bound instead to the communicative intentions of the speaker and to the interpretations of the hearer in a particular communicative situation, and highly determined by the context of utterance (see also \citealt{Ariel2010}: 93–119 for an identical position).

In the latter case, they are bound to what is inferentially meant (and not explicitly said) by the speaker. The decisive feature of pragmatic meanings is their inferential and defeasible nature: they arise as a consequence of the interaction between an utterance and the specific context where it is produced but there is nothing really fixed and binding in their status: they can be easily cancelled by some subsequent information added to the discourse. In other words, they represent an inference corresponding to a possible interpretation of what is said by the speaker in a specific context, but not a compulsory one. Following this line of reasoning, among the different types of (commonly classified) pragmatic meanings illustrated in the previous section (conversational implicatures, conventional implicatures, and presuppositions), only conversational implicatures turn out to be actually pragmatic meanings, due to their inferential nature. Conventional implicatures and presuppositions, due to their coded nature (in the speaker’s knowledge of language and in some specific lexical triggers), belong to semantics (\citealt{Hansen2008}: 26–33).

A further consequence of this view is that certain types of meaning which, because of their non-truth-conditionality, have traditionally been regarded as pragmatic in nature, are redefined here as being semantic meanings due their non-inferential, coded nature. Concerning pragmatic markers, the major consequence of this view on the semantics/pragmatics boundary is that these elements turn out to have a much less “pragmatic” meaning. In this sense, taking for instance the case of Italian, the whole spectrum of uses of linguistic elements defined as “pragmatic” because of their non-truth-conditionality – from the rebuttal-mitigating use of the adverb \textit{veramente} ‘really’ (\citealt{RiccaVisconti2014}) to the more interjection-like use of \textit{guarda} ‘look’ as a discourse marker \citep{Waltereit2002} – are actually characterized by a strong conventional (“semantic”) meaning: it is coded in the linguistic expressions and as such it must be learned, even if the context of occurrence plays an important role in how the elements are interpreted.

Let’s take another example of an element with both a truth-conditional and a non-truth conditional reading:

\ea%10
    \label{ex:key:10}
          Italian

\ea \label{ex:key:10a}  Lo riassumo \textit{brevemente} qui, domani ne parliamo meglio.

\glt ‘I sum it up \textit{briefly} here, we’ll talk about it better tomorrow.’

\ex \label{ex:key:10b} \textit{Brevemente}, quello che è successo ieri è che Giorgio ha frainteso la mia posizione.

\glt ‘\textit{Briefly}, what happened yesterday is that Giorgio misinterpreted my position.’
\z
\z


Drawing the boundary between semantics and pragmatics on the basis of truth-conditionality, a sharp distinction should be drawn between the “semantic” use of \textit{brevemente} (10a) and a “pragmatic” one \REF{ex:key:10b}.\footnote{On this issue, see also \citet{Dijk1979} and \citet[76–86]{Sweetser1990}.} Such a distinction, however, clouds the evidence that both meanings are \textit{coded} in the linguistic expression. What really counts here is that \textit{brevemente} displays different functions at different levels of the grammar, namely, it acts as a predicate modifier in \REF{ex:key:10a} and as a speech-act modifier in \REF{ex:key:10b}. A sharp distinction between the two uses – or even more the definition of \textit{brevemente} in \REF{ex:key:10a} as an adverb and in \REF{ex:key:10b} as a pragmatic marker – does not underline the importance of polyfunctional behavior of this and similar items.

\subsection{Content-level and context-level expressions}
\hypertarget{Toc124860629}{}
In many cases, in fact, it is not a given linguistic item which is either propositional (“semantic”) or non-propositional (“pragmatic”) in nature, but rather different uses of it. For this reason, Hansen (\citeyear[14–17]{Hansen2008}, \citeyear[592–594]{Hansen2012}) proposes a different terminology:

\begin{itemize}
   
\item \textsc{content-level} \textsc{use}: Any use of a linguistic item in which the meaning of that item bears saliently either on a state-of-affairs in some real or imagined world referred to in its host clause or on the relation between that state-of-affairs and other (real or imagined) states-of-affairs.

\item \textsc{context-level} \textsc{use}: Any use of a linguistic item in which that item primarily expresses the speaker’s comment on the relation between a described state-of-affairs and the discourse itself (including, but not limited to, the way it is represented linguistically) or on the wider speech situation (including, but not limited to, the subjective attitudes to the state-of-affairs in question that may be entertained by the speaker, the hearer, or some relevant third party).

\end{itemize}

Redefining the boundary between semantics and pragmatics in this way means that linguistic items and constructions functioning at the context level do have a semantics (and not only a pragmatics). This division of labor in terms of meaning conventionality – especially in a work that highlights the role of polyfunctionality/polysemy in the description of linguistic meaning – has the advantage of not determining either an a priori distinction between “semantic” and “pragmatic” uses of an item nor absolute evaluations about which level of meaning it should be assigned to (content- or context-level).\footnote{As noted by \citet[16]{Hansen2008} herself, this separation is in principle compatible with the hierarchically layered representation of adverbs and adverbial expressions proposed by Functional (Discourse) Grammar (see \citealt{Hengeveld1989}; \citealt{DikEtAl1990}; \citealt{RamatRicca1998}; \citealt{HengeveldMackenzie2008}). In this respect, the distinction between content-level uses and context-level uses roughly corresponds to the distinction between linguistic expressions that operate at the Representational Level and those that operate at the Interpersonal Level – as outlined in the previous chapter.}

As a consequence, the question arises of how to describe their semantics, since context-level meanings are not normally referential in nature. Many of the scholars that advocate a uniform handling of encoded meanings as semantic would, however, draw a distinction between \textit{conceptual} \textit{meanings} vs \textit{procedural meanings} \citep{Blakemore1987}. Conceptual meanings provide contents through which the addressee can construe a representation of reality corresponding to the expressions uttered by the speaker: they match concepts in our mental encyclopedias. Procedural meanings, on the other hand, do not themselves enter the semantic representation of the utterance: they provide instructions to hearers on how the conceptual meanings expressed in an utterance should be combined and processed.\footnote{It is important to underline that the conceptual vs. procedural distinction does not coincide in any way with the distinction between truth-conditional and non-truth-conditional meaning; on the contrary, it supports the enriched view of semantics that has been illustrated so far (see \citealt{Blakemore2006}: 230).}

The conceptual vs. procedural distinction, as a basic distinction between meaning that contributes to the contentful structure of the utterance (coding entities, activities, qualities and so on) and meanings that do not, has also been adopted in other research frameworks.\footnote{\citet[594]{Hansen2012} too refers to this distinction, though again, following the French tradition deriving from \citet{DucrotEtAl1980} prefers to use the term instructional (over procedural): “context-level items are thus seen as providing processing instructions to the hearer, indicating how the contents of the host clause as a whole should be contextualized so as to be integrated into a coherent mental representation of the discourse”.}  Among others, \citet{TraugottDasher2002} include the distinction between conceptual (\textit{contentful}, in their terminology) and procedural meanings in their model of semantic change:

\begin{quote}
Meanings expressed by nouns, verbs, adjectives, prepositions, and adverbs in some of their uses are usually of the contentful type. By contrast, procedural meanings are primarily indexical of SP/W’s [speaker/writer’s] attitudes to the discourse and the participants in it; they index metatextual relations between propositions or between propositions and the non-linguistic context. They include discourse markers (\textit{well}, \textit{in fact}, \textit{so} in some of their meanings), various connectives (\textit{and}, \textit{but}), and express SP/W’s view of the way these propositions should be understood to be connected. (\citealt{TraugottDasher2002}: 10)
\end{quote}

Nevertheless, drawing the pragmatics/semantics divide in this way, and the concept of procedural meaning, are not enough to fully describe the meaning of elements that display uses as pragmatic markers. Rather, many elements show a coexistence of both content-level and context-level uses, and their contribution to the meaning of an utterance can quite often be explained only through an overlap of coded meanings and contextual inferences, showing a steady crossing of the extralinguistic/linguistic divide. The next subsection will deal with the \textit{polyfunctionality} of discourse-pragmatic elements, while the entirety of the next section will deal with the relationship between contextual inferences and coded meanings.\textsuperscript{} \footnote{Undeniably, the issue of how to distinguish between code and context is far more complex than outlined in this chapter – and would require a much longer discussion, which is however outside of the scope of this study. In addition to the references cited so far, see for instance \citet{BellighWillems2021}.}

\subsection{Polyfunctionality at the semantics/pragmatics interface}
\hypertarget{Toc124860630}{}
Most commonly, lexemes and constructions displaying context-level uses also display content-level uses or have homophonous counterparts that operate at the representational level. The uses of \textit{brevemente} ‘briefly’ in example \REF{ex:key:10} above are a case in point. The same applies for the best-studied English pragmatic markers (\textit{well}, \textit{you know}, \textit{like}): they all have homophonous counterparts in other word classes.\footnote{A classical reference on these three items is \citet{Schourup1985}. Studies on single elements include \citet{Jucker1993} and \citet{Schourup2001} on \textit{well}, \citet{Östman1981} on \textit{you know}, \citet{Arcy2017} on \textit{like}.} In fact, in the case of elements operating at the semantics/pragmatics interface – regardless of the fact that their homophonous counterparts are adverbs, verbal constructions or other sources – this appears to be the rule rather than the exception and this fact should be systematically accounted for in their description. In my view, the best way to do this is to investigate how the different functions relate to each other, both synchronically and diachronically.

Several studies on pragmatic markers adopted this perspective, suggesting that the relation between the different functions of an item is better caught when tracing the emergence of new functions over time. In this sense, the synchronic coexistence of several functions for a single item may reflect a diachronic process of change through which the new functions gradually developed. This fact – well known in grammaticalization studies – is referred to by \citet[22--23]{Hopper1991} as \textit{layering}: “Within a functional domain, new layers are continually emerging. As this happens, the old layers are not necessarily discarded, but may remain to coexist with and interact with the new layers”. The result of this process is the polyfunctionality of items operating at the semantic/pragmatics interface, the coexistence of coded and inferred meanings, and – for some items – the coexistence of content-level and context-level functions. This coexistence leads to (at least) two intertwined problems: firstly, how to deal theoretically with polyfunctionality in the description of an item, and secondly, where to place the dividing line between different uses.

The existence of multiple senses or uses of a linguistic unit is a recurrent problem in linguistic analysis affecting all meaningful elements of language alike such as content words, function words (such as prepositions and auxiliaries), and affixal categories (such as affixes marking tense and case), which are sometimes referred to together as \textit{grams} \citep{BybeeEtAl1994}. The linguistic analysis of polyfunctionality – the association of a single form with several different interpretations – has long been a highly debated issue and three main positions can be distinguished (see \citealt{TraugottDasher2002}: 11–16; \citealt{Haspelmath2003}: 212–213; \citealt{Hansen2008}: 34–40):

\begin{itemize}
    
\item \textsc{monosemy}: According to this view, lexemes and grams have just a vague abstract meaning, and all the various functions that can be distinguished are not linguistically independent but result from the interaction with linguistic or non-linguistic context in order to yield a specific interpretation.

\item \textsc{polysemy}: According to this view, there are different senses or meanings attached to each lexeme and gram, but these meanings are related to each other in some fashion that needs to be specified, so that it is by no means an accident that the different senses have the same formal expression.

\item \textsc{homonymy}: According to this view, separate meanings (in the sense of underlying representations) are recognized for each of the functions and consequently different homophonous grams or lexemes apply for each different meaning.\footnote{As pointed out by Haspelmath (\citeyear[212]{Haspelmath2003} footnote 2), “from a semantic point of view, polysemy and homonymy are similar in that both involve different senses. The fundamental semantic problem has often been seen as that of distinguishing between \textit{vagueness} (= monosemy) and \textit{ambiguity} (= polysemy or homonymy)”. See also \citet{Tuggy1993} and \citet[37--39]{Hansen2008}, who discuss a variety of heuristic tools available for this purpose.}

\end{itemize}

From a diachronic perspective, polysemy seems to be the most attractive option. The synchronic coexistence of different functions is typically the result of diachronic sense extensions and – since these synchronic relations are expected to reflect the possible paths of change – it is assumed that these functions are related to one another in ways that can be motivated \citep[598]{Hansen2012}. Therefore, a polysemic approach is preferable since it explicitly aims at highlighting the step-by-step process by which meaning extension takes place over time.

However, such analyses of meaning extensions are often a matter of debate and it can be difficult to handle the terms \textit{use}, \textit{sense} and \textit{polysemy} in a consistent way – especially when the issue arises of how to describe the polysemy of an item, distinguishing the different conventional senses and the contextual uses. Moreover, it is important to stress that not every meaning extension of a given form necessarily end up with semanticization (that is, with a coded polysemy). In fact, it is common to find short-term innovations within the context of single speech events, which are better described as specific uses of a form rather than separate senses. Therefore, terminological caution can be useful: “I mostly refer to different functions of an expression, rather than “senses” (= conventional meanings) or “uses” (= contextual meanings), because often it is not easy to tell whether we are dealing with different senses or just different uses. The term “function” is meant to be neutral between these two interpretations” \citep[212]{Haspelmath2003}.

In this work, I also tend to employ the terms \textit{functions} and \textit{polyfunctionality} in this neutral interpretation. By doing so, I do not commit myself to a specific claim about which functions are part of the conventionalized linguistic knowledge and therefore constitute different senses, and which functions only arise in different utterances depending on the pragmatic context – a problem that continuously shows up in the description of pragmatic markers. However, I regularly employ the term \textit{use} in a broad sense as well, to indicate the different readings (\textit{uses}) of linguistic expression and the different contexts (of \textit{use}) in which it can appear.

\section{Context-level expressions in the light of language change}
\hypertarget{Toc124860631}{}
The effects of the overlap between coded and inferred meanings can be observed both from a synchronic perspective – relevant to the meaning description of pragmatic markers in interaction – as well as from a diachronic perspective – relevant to the emergence of new functions. After having examined the issue of how to describe the meaning contribution of items operating at the semantics/pragmatics interface, I will introduce in the following pages a diachronic perspective on them, with a discussion about semantic change and models concerned with “the possibility that grammar is often pragmatics turned code” \citep[111]{Ariel2008}. This perspective aims at highlighting the fact that meanings that appear in the pragmatic domain first – as inferences arising out of and exploited in the flow of speech – can progressively conventionalize as coded meanings of linguistic items. In other words: different types of meaning (inferred and coded, content-level and context-level) and more broadly the divide between semantics and pragmatics, are sensitive to language change, and items and meanings can cross the divide moving from one domain to the other.

\subsection{Dynamizing the semantics/pragmatics interface}
\hypertarget{Toc124860632}{}
As mentioned above, the synchronic coexistence of several functions for the same item may reflect the diachronic process of change through which the new functions gradually developed: synchronic polyfunctionality often reflects diachronic change. In the case of elements operating at the semantics/pragmatics interface, works additionally adopting a diachronic perspective have shown that in most cases the various discourse-pragmatic functions of a linguistic item have developed gradually over long periods, often starting from a relatively content-level source meaning, and progressively developing context-level functions (\citealt{Waltereit2006}; \citealt{Hansen2008}).

These findings suggest the existence of strong cross-linguistic tendencies of development, linking diachronic work on pragmatic markers with the broader research field of semantic change. Works like \citet{Traugott1989,Traugott1995,Traugott2010} and \citet{TraugottDasher2002} identified regular tendencies of change (\tabref{tab:key:4.1}).

\begin{table}
\begin{tabular}{lccccc}
\lsptoprule
i. & non-subjective & > & subjective & > & {intersubjective}\\
%\midrule
ii. & content & > & content/procedural & > & procedural\\
%\midrule
iii. & s-w-proposition & > & s-o-proposition & > & s-o-discourse\\
%\midrule
iv. & truth-conditional &  \multicolumn{3}{c}{>} & non-truth-conditional\\
\lspbottomrule
\end{tabular}
\caption{\label{tab:key:4.1} Tendencies of semantic change (see \citealt[40]{TraugottDasher2002})}
\end{table}

Through different case studies (concerning modal verbs, discourse markers, performatives and honorifics), this strand of research demonstrated that meanings (i) tend to become increasingly subjective (i.e. increasingly grounded in the speaker’s subjective perspective), and possibly even intersubjective (i.e. explicitly grounded in the relationship between speaker and hearer). Moreover, (ii) meanings that were conceptual at the outset tend to become increasingly procedural in nature, and (iii) constructions that originally had scope within the host proposition tend to progressively extend their scope to the level of the proposition and then up to the level of discourse. Finally, (iv) meanings that were truth-conditional at the outset tend to become non-truth-conditional. These tendencies clearly match changes from content-level to context-level meanings \citep[599]{Hansen2012}.

In particular, it is fundamental to the idea of “semantic change (change in code) as arising out of the pragmatic uses to which speakers or writers and addressees or readers put language, and most especially out of the preferred strategies that speakers/writers use in communicating with addressees” (\citealt{TraugottDasher2002}: xi). To properly frame this idea, it is useful to build on the proposal of \citet{Levinson1995,Levinson2000} – taken up by \citet[16–17]{TraugottDasher2002} – and distinguish three levels of meaning relevant to a lexeme:

\begin{itemize}
    
\item \textsc{coded} \textsc{meanings}:  These are semantic meanings, that is, conventions of a language at a given time. They represent a non-cancelable conventional link between the form of a lexeme and its meaning.

\item \textsc{utterance-type} \textsc{meanings}: These are pragmatic preferred meanings, that is, regular conventions of use in language-specific communities. They represent typical associations between a lexeme and a commonly used implicature, but are nevertheless cancelable.

\item \textsc{utterance-token} \textsc{meanings}: These are pragmatic nonce meanings, that is, inferences that have not been crystallized into commonly used implicatures. They arise in context at the time, based on encyclopedic or specific situational knowledge.\footnote{Referring to utterance-token and utterance-type meaning, \citet{Levinson2000} uses, respectively, the terms \textit{particularized conversational implicature} (PCI) and \textit{generalized conversational implicature} (GCI). \citet{TraugottDasher2002} speak of (G)IIN, that is \textit{(generalized) invited inferences.} The concept of \textit{invited inference} is substantially the same as that of \textit{implicature}, but \citet{TraugottDasher2002} use it to emphasize the role of the interactive negotiation of meanings between the interlocutors and the active role of speakers in rhetorical strategizing. Hence the name of their model: \textit{Invited Inferencing Theory of Semantic Change} (IITSC).}

\end{itemize}

\citet[24–25]{TraugottDasher2002} treat these different kinds of meaning as relevant both on a cognitive level – related to the processing of the information flow in interaction – and on a communicative/rhetorical level – related to the interactional negotiation of meaning between the interlocutors. They consider the meaning of utterances and constructions as deeply rooted in the context of interaction and dependent on the strategic use of language operated by the speakers, these facts constituting the precondition of the origin of language change in discourse strategies. In this view, “the chief driving force in processes of regular semantic change is pragmatic” (\citealt{TraugottDasher2002}: 24) and the main goal of their theory of semantic change is to explain the conventionalization of pragmatic meanings (linked to contextual inferences) and their reanalysis as semantic meanings (coded). This closely reminds us of the claim by \citet[111]{Ariel2008}: “The argument is that pragmatics, together with other extragrammatical triggers, provides the raw materials and impetus for grammar”.

In this model, the starting point of semantic change is the innovative use of a lexeme or a construction in an utterance: a speaker may begin to strategically exploit a conversational implicature (utterance-token meaning or invited inference/IIN) associated with a lexeme or a construction and may innovatively extend this use in a new linguistic environment. If the new uses acquire social value and therefore become salient in a community, they are likely to gradually spread to other speakers and to other linguistic contexts where they start to appear regularly (utterance-type meaning or generalized invited inference/GIIN). The last stage of this change is described as follows:

\begin{quote}
They are considered GIINs so long as the original coded meaning is dominant or at least equally accessible, but when that original meaning becomes merely a trace in certain contexts, or disappears, then the GIIN can be considered to have become semanticized as a new polysemy or coded meaning. (\citealt{TraugottDasher2002}: 35)
\end{quote}

It is therefore expected that the overall diachronic process follows a path from coded meanings to utterance-token meaning (IINs) to utterance-type meanings (GIINs) – which are pragmatically motivated polysemies – to finally reach the status of new semantic polysemies (coded meanings). This model – intuitively very clear (see however the criticism by \citealt{HansenWaltereit2006}) – has the fundamental virtue of recognizing the role played by pragmatic meanings arising in interaction in the emergence of innovative uses of lexemes and constructions. This way, the dynamics of change are placed in the “natural environment” of language use.

The descriptive and analytical approach taken by the present research shares many assumptions of usage-based models of grammar (\citealt{Hopper1987}; \citealt{Bybee2007,Bybee2010}; \citealt{Harder2012}): “In the usage-based approach, grammar is seen as an emergent system consisting of fluid categories and dynamic constraints that are in principle always changing under the influence of general cognitive and communicative pressures of language use” \citep[830]{Diessel2011}. Concerning language change, these models argue that the emergence of new functions must be explained through language usage rather than through reference to a pre-existing language faculty or underlying language structures. From this perspective, non-conventional meaning – that is, meaning associated with \textit{actual usage events} – constitutes the input to a process that has linguistic units at the output end. The following quote from \citet{DetgesWaltereit2016} introduces all relevant concepts for the subsequent discussion:

\begin{quote}
We have argued that the triggers for change are recurrent communicative functions. High frequency, in turn, leads to routinization of these items. Routinization, we would argue, is an aspect inherent to language use that affects all modules of grammar. First of all, at the semantics/discourse interface, the original inference wrapped up in the respective argumentative move turns into the new procedural function of the linguistic item. Secondly, at the syntax/discourse interface, the item undergoes reanalysis (\citealt{DetgesWaltereit2002}). It loses its original syntactic compositionality. (\citealt{DetgesWaltereit2016}: 654)
\end{quote}

\hspace*{-2pt}The concept of \textit{high frequency} is a core principle of usage-based models of grammar: the more a linguistic sign acquires communicative and cognitive sa\-lience, the more it is used, becoming routinized/entrenched in speakers’ cognitive systems and communicative habits (see \citealt{Ariel2008}: 149–211 for a comprehensive discussion about salient discourse patterns). One of the key notions of the quote is \textit{routinization}, and it will be discussed along with \textit{conventionalization} in the coming section. Before that, two additional “interface issues” mentioned in the quote must be addressed. Concerning the semantics/pragmatics interface (semantics/discourse interface in the quote), the role of \textit{discourse inferences} in the emergence of new procedural functions will be discussed. Concerning the syntax/discourse interface, the concept of \textit{reanalysis} will be called into play.

\subsection{Inferences in interaction}
\hypertarget{Toc124860633}{}
When talking about the motivations for language change, many works invoke factors such as the creative use of language or the urge to communicative successfully (see for instance \citealt{Haspelmath1999}; \citealt{HopperTraugott2003} [1993]: 24). \citet{Waltereit2011,Waltereit2012} highlights the strong link between processes of language change and the discourse domain, identifying in discourse strategies the motivation for and the starting point of grammatical changes: “What we see, then, is that an important subset of functional change is governed by the patterns of communication the relevant items are being used for creatively by speakers, rather than by the lexical properties of these items themselves” \citep[65]{Waltereit2012}.

Traditionally, the research approaches specifically interested in discourse and interactional dynamics are frameworks like \textit{interactional linguistics} and \textit{conversation analysis} (\citealt{Levinson1983}: 284–370; \citealt{Clift2016}; \citealt{Couper-KuhlenSelting2017}). These models are aimed at studying how utterances implement actions in discourse, privileging an empirical analysis of how language acts in communicative exchanges. They investigate the procedural infrastructure of interaction, examining the sequential placement of utterances in conversation, adjacency pairs, the organization of turns, and the dynamics of turn-taking.

However, these models have rarely been interested in processes of language change (but see \citealt{Couper-Kuhlen2011}). \citet{EhmerRosemeyer2018} explicitly highlight the connection between research on interaction and research on language change – finding in \textit{pragmatic inferencing} an important contact point:

\begin{quote}
From a usage-based perspective on language, it is pragmatic inference that is particularly important to the study of interaction and language change. Whereas entailments are unlikely to be discussed in discourse (e.g., upon hearing \textit{All of my friends are reading} I am unlikely to react by asking \textit{Are some of} \textit{your friends reading?}), pragmatic inferences are frequently dealt with in interaction and may, for example, become the topic of conversation (e.g., I might react to \textit{ALL of my friends are reading} by saying \textit{So does this} \textit{mean I am not your friend?}). In addition, it is a commonplace in historical linguistics that meaning change is often derived from pragmatic inferences. (\citealt{EhmerRosemeyer2018}: 536)
\end{quote}

Before continuing, a short remark is needed about the notion of \textit{inference}. In \chapref{sec:3}, the notion of \textit{implicature} was discussed in detail. The notion of \textit{inference} could be seen as its counterpart on the hearer side. As \citet{Huang2017} puts it:

\begin{quote}
By way of summary, a speaker conversationally implicates, the addressee infers, but a conversational implicature itself is not an inference. The addressee may or may not succeed in figuring out the speaker’s m-intended conversational implicature as an inference. Nevertheless, it is the speaker’s expectations about the appropriate inferences the addressee can reasonably be expected to draw that make the production and comprehension of a conversational implicature a rational, shared-goal activity. \citep[157]{Huang2017}
\end{quote}

Implicatures are a type of speaker meaning that goes beyond what is (literally) said. Inferences refer to the cognitive processes by which participants figure out meaning beyond what is said. Inferences arise in context – given the utterance and certain contextual conditions – and are responsible for the difference between literal meanings (“what is said”) and communicative meanings (“what is meant”, which corresponds to what is said plus what is implicated by the speaker~– that is, to be inferred by the hearer). Cued by indirectness, they can represent a grey area in communication: it is up to active discursive negotiation to uncover inferred meanings that have not been meant – or otherwise to accept them as acceptable or even relevant in the context.

This notion of inference lies at the core of an approach to the study of language change which tries to take proper account of conversational dynamics, aimed also at observing in real time (as far as possible) the interactional conditions which favor the emergence of new meanings. Thus – from the perspective of language change research – giving due consideration to interactional dynamics is necessary for detecting the mutual construction of meaning between speaker and hearer and the local management of discourse inferences. Moreover, since pragmatic inferences are frequently dealt with in interaction, interaction turns out to be the locus and in some cases the trigger of language change. As it has been discussed above, many models of language change assign a central role to inferences, conceiving the emergence of new meanings as the gradual conventionalization of inferences arising in discourse. Continuing along this research direction – but somehow reversing the perspective – several studies have argued that meaning change may not only occur as the result of the conventionalization of speaker-based conversational implicatures, but also as hearer-based reanalysis (\citealt{DetgesWaltereit2002}; \citealt{Eckardt2009}; \citealt{SchwenterWaltereit2010}; \citealt{RosemeyerGrossman2018}).

Taking stock of these studies, \citet{EhmerRosemeyer2018} aim at examining the relations between inferencing, interaction and language change, showing how work with diachronic data has assigned the notion of inference a central place in explanations of meaning change. With this approach, the focus on interaction is not the same as in conversation-analytical approaches – aimed at mapping the relationship between conversation structures and the actions performed by the interlocutors – but is, rather, intended to “demonstrate the importance of employing a contextualized model of the roles of speaker and hearer in the synchronic and diachronic emergence of meaning” (\citealt{EhmerRosemeyer2018}: 547). In fact, the explicit reference to the respective roles of speaker and hearer in interaction helps to better understand how certain meanings are intentionally suggested on one side of the conversation and how additional meanings are inferred by the other side, shedding light on the continuous process of meaning negotiation between the interlocutors.

The crucial point is when – irrespective of the speaker’s original intent – inferred meanings are accepted by the hearer as the most salient ones in a particular context and bound to specific linguistic constructions. The pairing of inferred meanings with novel contexts and specific linguistic forms is the environment in which change can start to take place. This perspective on language change highlights the role of the hearer:

\begin{quote}
Thus, meaning change commonly appears to arise in situations in which the hearer draws an inference on the basis of the use of a linguistic construction in a context in which its use is unexpected […]. Crucially, the resulting historical change is unmotivated from the perspective of the speaker. Although the speakers exploit the semantic potential of using a linguistic construction in novel contexts and anticipate the inference by the hearers that a divergent reading is intended, they do not necessarily expect the conventionalization of this inference (\citealt{EhmerRosemeyer2018}: 547)
\end{quote}

In this perspective, what needs to be done is to study closely what happens in the continuous exchange of explicit and inferred meanings between speaker and hearer and moreover – trying to include a structural point of view – understand what happens at the syntax/discourse interface. Most of these issues revolve around the concept of \textit{reanalysis}, which I will now explore.

\subsection{Reanalysis}
\hypertarget{Toc124860634}{}
In the last subsection, pragmatic inferencing was identified as the contact point between interaction and language change, representing an important descriptive and analytical notion for both research directions. Now, pragmatic inferencing provides the link to introduce the concept of \textit{reanalysis}, as they both pertain to the hearer’s sphere of activity: inferencing relates to comprehension rather than production and – in a similar way – reanalysis is normally seen as a hearer-driven change (\citealt{DetgesWaltereit2002}). As \citet[57]{Waltereit2018} puts it: “It is therefore natural to ask what is the relationship between the two”.

According to traditional definitions (for instance \citealt{Langacker1977}: 58), \textit{reanalysis} is that type of language change that assigns a new underlying structure to a surface sequence without overtly modifying that sequence. In their comparison between grammaticalization and reanalysis, \citet{DetgesWaltereit2002} essentially follow this definition. Reanalysis has mainly been used to explain morpho-syntactic changes and it is more generally linked to the structuralist-generative view of language change. It involves two abstract syntactic representations and a syntactic ambiguity arising in a particular surface sequence: when reanalysis occurs, there is an abrupt shift between the two representations.

This concept of reanalysis has been intensively debated by recent works (\citealt{Smet2009,Smet2014}; \citealt{Whitman2012}) and its suitability and usefulness – both on a theoretical and on an empirical level – has been questioned (see \citealt{Waltereit2018} for a summary of this debate). Questioning both abruptness of change and the role of ambiguity, De Smet (\citeyear[1748–1751]{Smet2009}, \citeyear[28–37]{Smet2014}) argued that the traditional concept of reanalysis can be broken down into underlying mechanisms of change that better fit into current usage-based models. This perspective argues for gradience of change and structural indeterminacy (uses of linguistic items that cannot be assigned to one single abstract representation). However, the question arises of how long underlying mechanisms can be identified – which are more basic, more specific, and better defined.

Alternatively, reanalysis has also been used in the literature to refer not to a specific \textit{type} of language change, but to the fact that something \textit{has} changed. From this perspective, reanalysis is seen as the formal signal of an innovation, relevant to essentially any kind of language change. Commenting on both these views, \citet[60--61]{Waltereit2018} suggests that – at this point of the discussion – “reanalysis is not a phenomenon in the empirical domain, but an analytical category on the theoretical plane”. He also points out that the reason for the overlap of these different readings of \textit{reanalysis} may be that both imply a hearer inference that is not specifically prompted by the speaker.

Whatever interpretation of the concept one can maintain, the local management of inferences in interaction seems to be the common feature that holds together both interpretations of reanalysis – and possibly also other types of language change, from semantic change to grammaticalization. From this perspective, the rise and the management of inferences – as a basic feature of human linguistic behavior – turns out to be the main empirical phenomenon to be closely investigated: for these reasons, it seems reasonable to start from this kind of analysis, no matter what broader type of language change it represents.

There have been many attempts to model the process whereby a context-specific inferential meaning is reanalyzed as a new encoded function. \citet{Heine2002}, discussing the role of context-induced reinterpretation in grammaticalization processes, proposes this kind of scenario.\footnote{Among other proposals, see for instance \citet{Diewald2002} and \citet{Ramat2012}.}


\begin{table}
\begin{tabularx}{\textwidth}{llQQ}
\lsptoprule
Stage & Context & Resulting meaning\\
\midrule
I & Initial stage & Unconstrained & Source meaning\\
\hspace{1mm}\\
II &  Bridging context & There is a specific context giving rise to an inference in favor of a new meaning & Target meaning foregrounded\\
\hspace{1mm}\\
III & Switch context & There is a new context which is incompatible with the source meaning & Source meaning backgrounded\\
\hspace{1mm}\\
IV & Conventionalization & The target meaning no longer needs to be supported by the context that gave rise to it; it may be used in new contexts & Target meaning only\\
\lspbottomrule
\end{tabularx}
\caption{A scenario of how linguistic expressions acquire new grammatical meanings \citep[86]{Heine2002}}
\label{tab:key:4}
\end{table}

In short, the scenario is as follows. The starting point is the normal use of a construction – referred as the \textit{source meaning} – in an array of different contexts (Stage I). If the use of this construction to imply a certain, non-literal meaning is found to be particularly successful in specific types of contexts, this meaning can, over time, become firmly associated with the construction. In those contexts, there is another meaning – referred as the \textit{target meaning –} which represents a more plausible interpretation of the utterance concerned (Stage II). At this point, as an effect of frequency, the new form-meaning pairing may acquire the conventional character that is the defining feature of grammatical constructions: these contexts no longer allow for an interpretation of the construction in terms of the source meaning (Stage III). If this stage is reached, the new form-meaning pairing is freed from the contextual constraints that gave rise to it. It will start occurring in new contexts and eventually generalize to a whole variety of contexts (Stage IV).

What this model highlights is the metonymical character of the changes involved, that is, the progressive shift from one form-meaning association to another which takes place in context (see also \citealt{TraugottDasher2002}: 27–34, 78–81). \citet[201]{Koch2001} defines metonymy as “a linguistic effect upon the content of a given form, based on a figure/ground effect along the contiguity relations within a given frame and generated by pragmatic processes”. In the scenario described above, this corresponds to the figure/ground shift between meanings: the source meaning shifts from the foreground to the background while the target meaning (prompted by a context-induced inference) shifts from the background to the foreground. In a way, metonymy is the conceptual counterpart of pragmatic inferencing, so that – following the process of inferring – lexical expressions or grammatical constructions gradually shift from one conceptual meaning to another, while at the same time they expand from a specific context of use to a greater variety of contexts. Even though this model could represent a good approximation of the dynamics through which a discourse inference can be reanalyzed as coded meaning, it also shows some problematic points to be noticed here.

First, the notion of \textit{bridging context} is problematic: although it is useful at a theoretical level, it is hard to assess its validity at a descriptive level, since bridging contexts can be recognized only retrospectively once a change has already happened. Especially when working with diachronic data, it is the presence of different meanings for the same construction in synchrony that allows us to reconstruct their diachronic relationship and thus identify contexts where both meanings can (hypothetically) be intended. Second, these models do not convincingly deal with the protagonists of interactions: \citet{Heine2002} doesn’t say almost anything about the respective roles of the speaker and the hearer, while \citet{TraugottDasher2002} focus almost exclusively on the “creative” role of the speaker – capable of inviting inferences and deliberately insert constructions in unexpected contexts – and overlook the role of the hearer.

While the appearance of constructions in unexpected contexts is definitely one of the triggers of reanalysis, it seems more realistic that this happens largely by chance, rather than from an explicit choice of the speaker – and consequently, that hearers compensate this unexpectedness with their inferring activity. Besides the speaker’s creative activity, closer reference to the hearer’s inferring activity and an accurate inspection of interactional patterns – aimed at identifying which inferred meanings are possibly activated in a specific context – is essential to develop a plausible model of meaning change.

\section{Context-level expressions in the light of language variation}
\hypertarget{Toc124860635}{}
Referring back to the quote by \citet{DetgesWaltereit2016} mentioned above, a last aspect of reanalysis and language change processes shall be discussed now: conventionalization and the spread of an innovation in the social community of speakers. This section will address this issue and, going in this direction, it will introduce a sociolinguistic perspective to the present research. Discussing the interdependence of variation and change processes in linguistic systems, the gradual character of reanalysis is reevaluated through a variational perspective~– getting to a sociolinguistic-informed notion of conventionalization. This paves the way to the topic of sociolinguistic variation of pragmatic markers.

\subsection{Argumentative routines and conventionalization}
\hypertarget{Toc124860636}{}
There are different ways of conceptualizing the process through which a new meaning or function gradually comes to be firmly associated with a linguistic expression. \citet{Heine2002} defines it \textit{conventionalization}, including it as the fourth step in the scenario:

\begin{quote}
Most context-induced inferences remain where they are: they are confined to bridging contexts, they are what has variously been described as “contextual meanings” or “pragmatic meaning”. But some of them, i.e. those acquiring switch contexts, may develop some frequency of use, they no longer need to be supported by context and they turn into “normal” or “inherent” or “usual” or “semantic” meanings. \citep[85]{Heine2002}
\end{quote}

\citet[35]{TraugottDasher2002} hold a similar view: “The prime objective of IITSC [Invited Inferencing Theory of Semantic Change; see the discussion in the previous section] is to account for the conventionalizing of pragmatic meanings and their reanalysis as semantic meanings”. Following these examples, the term \textit{conventionalization} will be used in the present research to indicate the progressive inclusion of emergent contextual functions in the coded meaning of a linguistic expression. Nevertheless, other terms have also been used in the literature to denote this process or to highlight specific aspects of it.

Opting for a different terminological choice, \citet{DetgesWaltereit2016} talk about \textit{routinization} to explain processes of change. According to them, the driving force linking different kinds of change is to be found in \textit{argumentative routines}, that is, preferred ways of saying something on the part of the speakers. Argumentative routines allow the speaker to reach their communicative goals in a simple but effective way: they represent “familiar paths” in interaction, on which speakers can count to get the desired results. The source of the differences are the speakers’ communicative strategies – determined by individual preferences and contextual choices – and the different domains of grammar addressed by them (propositional content, speech act, discourse structure). The common aspect among different kinds of change is represented by the fact that they are all driven by patterns of language use: “Routinization is not a feature of language itself – it is rooted in language use” (\citealt{DetgesWaltereit2016}: 637). The gradual rooting of frequent argumentative moves in usage give rise to communicative routines which – in the course of time – are \textit{reanalyzed} as parts of the grammar.

Routinization has both a cognitive and a communicative facet. On the cognitive side, it represents the progressive fixation of language sequences, grammatical patterns, and argumentative moves in the speakers’ language knowledge (entrenchment). On the communicative side, it results in the high frequency in discourse of a linguistic item – and in the progressive ruling out of alternatives to perform the same action or express the same meaning. What emerges from this process is the only possible choice in a specific context. In this perspective, routinization is a usage-based-grammar-flavored term: it revolves around the idea that increasing frequency of use of linguistic forms and the simultaneous entrenchment in speakers’ competence are the crucial points in language change. Routinization and conventionalization are clearly not equivalent terms, but neither are they entirely dissimilar to each other. They instead represent complementary aspects in processes of language change: on the one hand, the gradual fixation of an expression/function in the speakers’ language knowledge – on the other, its progressive acceptance as a coded part of the communicative habits of a community.

Finally, with reference to reanalysis, \citet{Smet2012,Smet2014} has discussed in detail the concept of \textit{actualization}, defined as “the process following syntactic reanalysis whereby an item’s syntactic status manifests itself in new syntactic behavior” (\citealt{Smet2012}: 601) – which can be partially compared to the process of conventionalization, whereby an item spreads to new contexts of use. \citet{Smet2014} aims at integrating the concept of reanalysis in usage-based models of grammar, showing how it could be less abrupt than usually assumed: he builds a model of gradual – in a way, barely noticeable – change where a new function evolves through a cascade of small steps of reanalysis in slightly changed contexts. Moreover, he highlights how the spread to new syntactic contexts will first affect those contexts that most resemble the original usage contexts of the construction, showing that actualization proceeds from one environment to another on the basis of similarity relations between environments. The logical consequence of this view is a conflation of the two notions: “If reanalysis can be gradual in this way, the temporal primacy of reanalysis over actualization is no longer logically necessary, and the process of reanalysis can be reconceived as simply part of actualization (which then becomes something of a misnomer)” (\citealt{Smet2012}: 629).

Echoing this line of reasoning, \citet{EhmerRosemeyer2018} link it to the discussion about the role of inferencing in interaction. They argue that the gradual affirmation in specific contexts of use – and the expansion to new ones – can also be explained in terms of the degree of expectedness of hearer-based inferences:

\begin{quote}
We could thus expect scenarios such as the ones described above in which the original reanalysis is highly unexpected (and consequently, salient) in discourse. However, once reanalysis has taken place, the same inference becomes much less unexpected in those usage contexts that most resemble the original reanalysis context. These contexts are favored in the actualization process because of cognitive ease; the hearers can use an already established reanalysis pattern based on a more or less conventionalized inference to deal with this new utterance type. (\citealt{EhmerRosemeyer2018}: 548)
\end{quote}

The consequence of these observations is the downsizing of the distinction between reanalysis and actualization/conventionalization – and the necessity, instead, of paying greater attention to the degree of expectation of inferences in a given discourse situation and how they are dealt with in intermediate phases of change.

\subsection{Degrees of conventionalization}
\hypertarget{Toc124860637}{}
Accordingly, reanalysis and conventionalization go hand in hand rather than being two separate phases of a process: a form is reanalyzed while it spreads through the social community of speakers. The more it spreads, the more the new form/function pair becomes fixed as a new convention. Thus, reanalysis is strictly bound to the process of selection and diffusion of innovative usage patterns through the community. Yet how should the diffusion of morpho-syntactic and semantic variants in a speech community be modeled?

Possible suggestions come from the analogy that has been established by some researchers between current models of sound change – and specifically the work of \citet{Ohala1981,Ohala1993} – and models of language change in other domains. \citet{Croft2000,Croft2010} has discussed these arguments in depth, which have also been taken up by \citet{Waltereit2012}, \citet{GrossmanNoveck2015} and \citet{EhmerRosemeyer2018}. \citet{Ohala1981,Ohala1993} proposed that sound change is a result of the way hearers perceive the speech signal. In actual speech two instances of the same phoneme are never entirely identical. This is due to a number of phonetic bias factors (mainly related to the mechanical and physiological aspects of sound production and perception) which, in speech production, result in a pool of synchronic variation. This represents the basis for sound change:

\begin{quote}
In the perception mechanism, hearers typically filter contextual variation out from the speech signal. However, they sometimes fail to do so, analyze a part of the contextual variation as the articulatory goal and even filter out a part of the signal that was part of the original articulatory goal. Thus, errors in speech perception can in the long run lead to sound change. (\citealt{EhmerRosemeyer2018}: 543)
\end{quote}

Consequently, sound change represents the result of how listeners perceive and represent the speech signal and – at least in some cases – its origin is to be found in this articulatory variation: “In other words, sound change is the result of the selection of a variant out of the range of variation inherent in normal speech, rather than requiring any specific departure from the conventions that are underlying those representations” \citep[54]{Waltereit2012}.

Departing from these ideas, the works cited above have tried to extend this line of reasoning to other areas of language change. The assumption is that – with regard to the semantic level of linguistic constructions – the mechanism of interpreting a contextual feature as a coded one might look very similar to the cases of phonetic variation. For instance, \citet{Croft2010} suggests that morphosyntactic change comes about in the same way, whereby it is triggered by natural variation of lexical choice in discourse. With reference to the above-mentioned model of sound change, the interpretation given to hearer-based reanalysis would arise as a result of failed attempts, or misunderstandings, in the retrieval of the intended meaning – which in turn have the potential to give rise to subtle variation of meaning. This idea is truly fascinating, although it has also received some criticism (see the discussion in \citealt{Waltereit2012}: 55). However, the reference to inherent variation is important – in my perspective – to better describe the relationship between reanalysis and conventionalization.

In this scenario, the role of a well-identifiable structural reanalysis is downsized, in favor of a dynamic view of synchronic variation – as linguistic constructions gain more and more frequency and contexts of use which are characterized by an inherent meaning variation. In my view, this point has both an empirical motivation and theoretical implications: in the synchronic landscape, variation in usage is something observable and to some extent – however difficult and questionable – measurable. This is not the case with reanalysis – especially if based on the identification of bridging contexts – which, as discussed above, can be interpreted as such only retrospectively.

In this regard, \citet{EhmerRosemeyer2018} introduce the useful concept of \textit{degrees of conventionalization}:

\begin{quote}
While the use of a construction in a novel context leads to an ad-hoc inference by the hearer (corresponding to a particularized implicature on the speaker side), repeated exposure to the same novel usage will lead to the conventionalization of this inference. The degree of conventionalization of an inference has an important influence on the perception and management of inferences, as conventionalized inferences are arguably drawn on a less conscious level and are more robust. This may impact the usage contexts of the constructions that the inferences are associated with. […] This means that assuming degrees of the conventionalization of an inference and observing the reflexes of this process in interaction can be useful in determining at which point an inference has become part of the encoded meaning of a construction. (\citealt{EhmerRosemeyer2018}: 548)
\end{quote}

On this view, reanalysis should be theoretically defined not as a phenomenon clearly distinct from the diffusion of the newly reanalyzed construction, but rather as an integral part of it. Pushing this reasoning further, the degrees of conventionalization should be interpreted (and operationalized) as degrees of variation in the acceptability of constructions which are more or less present in the competence (and in the performance) of speakers as a consequence of the inherent variation of use.

\subsection{Language variation in the pragmatic domain}
\hypertarget{Toc124860638}{}
The theoretical reasoning about degrees of conventionalization will now be suspended: I will go back to it in the empirical part, supported by the analysis of data. The remainder of this section will be devoted to bringing the above reasoning back to the bigger picture. This is fostered by the concept of \textit{language variation}, which appeared several times in the discussion about different interpretations of reanalysis and conventionalization.

Living language – the language which crowds the multiple spaces of our social interactions – is no static entity. Without wishing to summarize here the main assumptions and findings of variationist sociolinguistics (in this regard, see \citealt{Bayley2013}), it is a well-established point that language is ineradicably subject to variation (\citealt{Labov1963,Labov1972}) and characterized by a structured heterogeneity (\citealt{WeinreichEtAl1968}: 99–100). Moreover, variation is sometimes stable, but sometimes it leads to change in a linguistic system. Thus, variationist sociolinguistics is ultimately concerned with the circumstances – internal and external – which determine the evolution of linguistic systems.

The relationship between language variation and language change has always been a crucial theme for variationist sociolinguistics (again, see the seminal paper by \citet{WeinreichEtAl1968}, but it has progressively expanded to other theoretical approaches to language change, above all grammaticalization theory (see \citealt{NevalainenPalander-Collin2011}; \citealt{Poplack2011}). In this respect, a fundamental idea is that change is characterized (and preceded) by variation, proceeding gradually across time and linguistic contexts, with a period of oscillation between conservative and innovative forms: “A does not become B; rather, A and B alternate for a period and the frequency of one (or more) com\-peting variants increases, spreading in both linguistic and social space” (\citealt{Arcy2013}: 485). From this perspective, while variation can be quite stable and its presence does not automatically entail that a change is occurring, the process of change, by contrast, always entails variation.

\hspace*{-1.2pt}Variationist sociolinguistics was originally developed for the analysis of phonological variation and has been successfully applied to the analysis of morpho-syntactic variation. Discourse-pragmatic features, however, do not easily satisfy the defining criteria of the variationist concept of linguistic variable.

\begin{quote}
Because discourse-pragmatic features have unique semiotic and distributional properties, it is not easy to apply the parameters outlined above to their conceptualisation as variables or to their quantitative analysis. Firstly, discourse-pragmatic features are typically semantically bleached and therefore cannot be defined in terms of semantic equivalence between variants. Secondly, they are typically both referentially and syntactically optional, and thus eschew straightforward reporting as non-occurrences (see, however, \citealt{Arcy2005}). Consequently, it is not immediately obvious on what basis to identify co-variants of a discourse-pragmatic variable and how to produce accountable results. \citep[28]{Pichler2013}
\end{quote}

Nevertheless, some studies (\citealt{Terkourafi2011}; \citealt{Pichler2013,Pichler2016}; \citealt{Arcy2005,Arcy2017}) have recently argued in favor of extending the variationist analysis to discourse-pragmatic features (quotatives, tags, discourse-pragmatic markers), a task that they consider important and feasible despite its complexity – trying to group different elements under a single functional category which can be treated on a par with the classic variables.

Finally, another approach called \textit{variational pragmatics} has recently emerged (see \citealt{BarronSchneider2009}; see also \citealt{Placencia2011}), aimed at investigating the relationship between language variation and pragmatic phenomena. Despite the name, this framework does not necessarily take into con\-sideration linguistic variables in the above-described sense: “rather, variational pragmatics investigates how par\-ticular speech acts, routines, or even broader notions such as politeness, are realized across varieties of the same language” (\citealt{CameronSchwenter2013}: 466). This approach does not preclude variationist methodology, but it doesn’t focus on the variant forms and their internal linguistic conditioning, but rather on the macro-social processes and cultural values associated with speaker strategies for carrying out prag\-matic routines in natural discourse. Primarily, it focuses on patterns of macro-social pragmatic variation across dialects and sub-varieties of a given language, thus trying to assess the impact of sociolinguistic and geographical variables on pragmatic aspects.

A discussion related to variation across varieties of Italian (and their impact on pragmatic phenomena) represent the core of the next chapter, which concludes the theoretical part of the present work and introduces the case studies.

