\chapter{Conclusions}\label{sec:10}
\addtocontents{toc}{\protect\enlargethispage{2\baselineskip}}
\hypertarget{Toc124860693}{}\section{A look back at the theoretical framework}
\hypertarget{Toc124860694}{}
This research used a lot of concepts in the attempt of elaborating a theoretical framework for the analysis of modal particles in Italian. The discussion conducted in the first part of this work contributed towards the definition of a flexible framework – not strictly based on a pre-existing theoretical model, but rather composed by a set of connected concepts (coming from different sources) necessary to give an insightful description of modal particles. Let me revise them briefly.

\chapref{sec:2} and \chapref{sec:3} were dedicated to a re-evaluation of the notion of \textit{modal particle}. After briefly introducing the functional category of pragmatic markers (and its subdivisions), I identified the main features of the linguistic expressions I wanted to investigate – namely the “speech-act tuners” variously defined as modal particles, modal-particle-like elements or illocutionary operators throughout this work. In this respect, the fundamental concepts discussed are those of \textit{common ground} and \textit{illocutionary modification}. They served to elaborate a working definition of modal particles as linguistic elements which operate on the \textit{conditions} under which the speech act is performed (integrating the speech act in the common ground and contributing to manage the information flow with respect to the shared knowledge) – and which specify the \textit{intentions} with which speech acts are performed (contributing to refine the illocutionary point of the speech act in an interpersonal perspective). By using modal particles, speakers point both to the underlying conditions that allow the performance of a speech act and to the communicative intention that defines the orientation of that speech act in the interactional space. Thus, modal particles simultaneously operate on (and show the intertwining of) the different dimensions of a speech act: its felicity conditions, its illocutionary force and the proposition carried by it.

\chapref{sec:4} took a step back to consider modal particles and linguistic elements operating at the semantics/pragmatics interface from a broader perspective. In particular, the issue of how to describe their meaning was addressed. Following works like \citet{Hansen2008,Hansen2012} and \citet{Ariel2008,Ariel2010}, the distinctions between pragmatics and semantics was defined as a divide between non-coded vs. coded meanings, that is inference vs. convention. This perspective on the semantics/pragmatics interface led me to draw a distinction between content-level vs. context-level uses of linguistic expressions, that is a distinction between uses having a bearing on a state-of-affairs/proposition (or on the relation between two states-of-affairs/propositions) and uses having a bearing on the relation between a state-of-affairs/proposition and contextual entities (the discourse itself, the mental states of the interlocutors). In this respect, the diachronic tendencies which cross this divide were also considered, discussing how the same linguistic element can display both content-level and context-level uses and how inferential meanings can become coded meanings over time.

The discussion revolved around the concepts of \textit{reanalysis} and \textit{conventionalization}. Following \citet{Smet2009,Smet2012,Smet2014}, the relationship between them was redefined as a matter of degree rather than an abrupt step: new constructions/functions are reanalyzed as they spread to new contexts of use – and alongside their diffusion across a speech community. In this respect, the concept of \textit{degrees of conventionalization} (\citealt{EhmerRosemeyer2018}) has been employed as a useful concept to describe the gradual diffusion of constructions/functions across a speech community. Specifically, it represents a descriptive tool that helps to empirically address the issue of how to define what is coded and what is non-coded, overcoming the setting of a predefined divide between them and handling it in terms of the acceptability of the constructions/functions across a speech community. The fact that constructions/functions display varying degrees of conventionalization is often reflected in their different sociolinguistic status. Following this line of reasoning, the issue of language variation in the pragmatic domain has been raised and I pointed out that in many cases a sociolinguistic perspective is necessary to give a better description of modal particles and other discourse-pragmatic elements, including their diffusion in different language varieties and their diverse functions across them.

Overall, the concepts that I combined for this research compose a “data-driven” theoretical framework – that is, a framework mainly oriented to the discussion of the data, which calls into play different concepts whenever they can be useful to deal with specific aspects of the data. Such an approach has both advantages and drawbacks. On the one hand, it is flexible and versatile – and it allowed a discussion of modal particles from multiple perspectives, from their grammatical functions to issues of language change and variation. On the other hand, it may lack a certain systematical nature and it somewhat blends the difference between description and analysis of data – which I often addressed together. In this respect, the reference to Functional Discourse Grammar (\citealt{HengeveldMackenzie2008}) represented an effort to compare my data with an established theoretical framework of linguistic analysis. Albeit not systematical, the reference to the layered model of grammatical categories posited by FDG has allowed me to place the modal/illocutive functions under investigation in a broader picture. This way, the relationship between content-level and context-level uses of the same adverb/element could be reassessed as a relationship between neighboring functions in a layered model of grammatical categories, which are connected to each other by specific scope relations and predictable patterns of development.

\section{Three outcomes of this research}
\hypertarget{Toc124860695}{}
An overview of the main findings of the single case studies has already been presented at the end of each relevant chapter. I will not repeat them here. Moreover, the conclusions related to the sociolinguistic issues discussed in this work (regional markedness of modal particles, variation in the acceptability, involvement of modal particles in the sociolinguistic changes affecting contemporary Italian) have already been presented in the closing section of \chapref{sec:9}. I will not repeat them here, either. What remains to be done is to highlight general conclusions cross-cutting all case studies and revise the research questions presented in the introduction: What are the modal uses of adverbs in Italian? How can their properties and functions be described? What are their contexts of use? The discussion will be divided in three parts.

\subsection{Modal particles in Italian}
\hypertarget{Toc124860696}{}
The question of whether modal particles can be found in Italian has been addressed by few scholars and hardly ever in a systematic way (\citealt{Coniglio2008} and \citealt{Squartini2017} count among the rare exceptions). This does not point only to a lack of sufficient scientific consideration: as a matter of fact, Italian does not display a well-identifiable set of modal particles. The same holds for the Romance language family as a whole: despite some well-studied examples (see for instance \citealt{Hansen1998b} on Fr. \textit{bien}; \citealt{Waltereit2004,Waltereit2020} on Fr. \textit{quand même}), few “real” modal particles have been identified in Romance languages and – perhaps more importantly – no Romance language displays a coherent paradigm of modal particles comparable to what can be found in German (by far the best-studied case of a language with modal particles). Overall, the results of this research do not question any of these facts.

However, considering things in more detail lead us to partially reassess this situation. This meant – in the context of the present research – to consider the functions of modal particles in a broader perspective (including cross-linguistic comparison), to adopt a functional perspective on this category, and to include less-studied language varieties (dialects, regional varieties) in the description. I often used labels such as \textit{modal-particle-like elements} and \textit{modal}/\textit{illocutive uses of adverbs} to refer to the elements under analysis, suggesting a prototype-based approach to this category. The description of data from regional varieties of Italian further increased the inventory of elements interpretable as such. In this respect, it is certain that Italian and regional varieties of Italian do display modal-particle-like elements – and the present research offered a new description of a set of them.

The elements that I have been analyzing are adverbs from different subclasses: focus adverbs (\textit{pure}, \textit{anche}, \textit{solo}), degree adverbs (\textit{un po’}), temporal adverbs (\textit{poi}), phasal adverbs (\textit{già}). All of them display illocutive uses. When used as such, they have scope over the speech act and they contribute to expressing how it must be interpreted in the relevant communicative exchange: they can modify its illocutionary force, they can point out specific aspects of the underlying interactional context, and they can do both. This functional status is common to all these adverbs. Regarding other aspects, more differences can be noticed: these adverbs differ regarding the frequency found in corpora, the degree of conventionalization of their illocutive functions, and in their sociolinguistic distribution. Nevertheless, they all represent clear examples of modal-particle-like elements in Italian. Among them, at least one – \textit{pure –} can even be considered a prototypical example of modal particle: it is highly frequent in spoken data, it shows a firm pan-Italian distribution, and it displays clear conventionalized illocutive functions.

\subsection{Illocutionary modification as a grammatical category}
\hypertarget{Toc124860697}{}
The central point of my analysis of modal-particle-like elements insists on the acknowledgment of illocutionary modification as a grammatical category. Building upon the cross-linguistic considerations formulated by \citet{Waltereit2001,Waltereit2006}, \citet{Hengeveld2004}, \citet{HengeveldMackenzie2008} and \citet{Narrog2012}, I defined illocutionary modification as a category capable of bringing together both (the functions of) modal particles in a narrow, language-specific, sense and similar elements found across different languages. In turn, illocution is recognized as a core grammatical domain, the one that maps communicative intentions onto conventionalized linguistic expressions.

\begin{quote}
The Illocution of a Discourse Act captures the lexical and formal properties of that Discourse Act that can be attributed to its conventionalized interpersonal use in achieving a communicative intention. Communicative intentions include such Discourse Act types as calling for attention, asserting, ordering, questioning, warning, requesting, etc., which may map onto Illocutions such as Vocative, Declarative, Imperative, etc. There is no one-to-one relation between a specific communicative intention and an Illocution, as languages may differ significantly in the extent to which they make use of linguistic means to differentiate between communicative intentions. (\citealt{HengeveldMackenzie2008}: 68–69)
\end{quote}

With reference to the quote, modal particles are among the linguistic means that contribute to the differentiation between communicative intentions. As the case studies showed, \textit{pure} can specify imperative illocutions as invitations and permissions, \textit{un po’} can specify imperative illocutions as requests and \textit{solo} can specify imperative illocutions as peremptory orders.

Communicative intentions, however, do not exist as such – out of the blue – but are expressed against a background of previous discourse acts, previous assumptions and future steps in discourse. In this respect, modal particles also refer to the underlying conditions of speech acts, specifying how they should be interpreted in a specific (conversational) context. As the case studies showed, \textit{pure} marks directives that redundantly meet certain expectations on the part of the hearer, \textit{un po’} marks an interactional frame where minimal effort on the part of the addressee is required, and \textit{solo} marks directives that contrast with some assumption active in the common ground that the speaker considers not valid in the relevant conversational context. In summary, these elements enrich a basic illocution with subtler communicative intentions and explicit reference to common-ground conditions. In my view, these represent the core functions of the grammatical domain of illocutionary modification.

Illocutionary modification as a category represents an attempt to include pragmatic facts in a model of grammar. Despite being a profitable approach, it must be reminded that pragmatic facts – by referring to the use of language in real contexts, with countless nuances – are not describable as consistently as semantic facts. Describing pragmatic facts often requires reference to real life situations, unspoken things, inferences. Categories such as communicative intentions and common-ground conditions are (theoretically and empirically) different from present tense and imperfective aspect. Nevertheless, they are all formally coded by human languages and – in this perspective – illocutionary modification is worth exploring.

Future research will further investigate this category, both from a theoretical and an empirical point of view. The compatibility between illocutionary modification and speech act theory represents a possible research direction. With reference to other grammatical domains, the relationship between illocutionary modification and modality is a key point \citep{Narrog2012}. The relationship between illocutionary modification and information structure also represents an underexplored key point: functional developments of focus adverbs such as \textit{pure}, \textit{anche}, and \textit{solo} suggest a strong link between them. Finally, more typological research and cross-linguistic comparison is needed in order to get a better inventory of the linguistic means expressing it across different languages.

\subsection{Routines in interaction, inferences and illocutive functions}
\hypertarget{Toc124860698}{}
An essential step in the analysis of modal particles is represented by the description of their contexts of usage. I first focused on the types of speech acts they are featured in – that I used as a decisive factor for the classification – and then on the most salient interactional patterns in which modal particles appear. Both factors contribute to shaping the pragmatic functions expressed. In a sense, illocutive functions emerge precisely from inserting speech acts in interactional routines, that is from the interplay between illocutionary force and interactional context.

This has been shown in the discussion about additivity in interaction in \chapref{sec:6} and in the description of the emergent functions of \textit{solo} in \chapref{sec:8}. Especially in the latter case – which could benefit from the analysis of questionnaire data – the interactional context (rather than the illocutive context) has proven to be the most decisive factor influencing the emerging meanings, orienting the choice between the “emphatic reading” and the “common ground reading” of the illocutive use of \textit{solo}. In addition to this, the semantic features of the source constructions represent a third factor since they can constrain the range of contexts in which an element can be used and its spread to new contexts. However, semantic features of source constructions should probably not be interpreted as the decisive factor in shaping the emergence of illocutive functions.

This has been shown by the data of the second questionnaire – and especially by the answers of the last two stimuli (backchecking markers and emphatic markers). What is striking in both cases is that a very diverse set of elements can get to express the same pragmatic function. Backchecking markers can emerge from cleft constructions, phasal and temporal adverbs (\textit{già}, \textit{poi}), focus adverbs (\textit{pure}), and comparative adverbs (\textit{più}). Emphatic markers can emerge from degree adverbs (\textit{un po’}), focus adverbs (\textit{solo}), and temporal adverbs (\textit{poi}, \textit{mo’}). It follows from this that a diverse array of semantic features, when inserted in the relevant illocutive and interactional context (and in the appropriate syntactic slot), can develop the same pragmatic function – because they are put against the same background of contextual inferences and used in similar conversational routines.

In my opinion, this aspect is the decisive factor in orienting the reanalysis process and defining the development path of an adverb. Some semantic features of the source constructions are progressively bleached under the “pressure” of external factors (such as the communicative intention), which in the long term will define the characteristics of the emerging constructions. Other semantic features shall remain unchanged – like the fact that focus adverbs evoke alternatives, or the fact that \textit{già} evokes a transition between phases – but they are reanalyzed in interaction and transferred to the discourse-pragmatic level. This way, modal uses of focus adverbs evoke alternative propositions in the common ground (and not alternative referents/states-of-affairs) and the backchecking \textit{già} evokes a transition between conversational phases (and not between states-of-affairs).

In this perspective, with reference to the model of hearer-based reanalysis described in \chapref{sec:4}, the attempt to infer the communicative intent associated with a speech act – and the way it is encoded on grammatical constructions – is the main factor triggering the process of reanalysis and, at the same time, the one that shapes the emerging functions. This holds for many cases of semantic change – in different domains of lexicon and grammar – but appears to be central in the case of illocutionary operators, since they are precisely markers of communicative intent: in a sense, they come to express the routinized communicative process that shaped their functions. As a result, prototypical uses of modal particles can be in the best way described as stereotypes of conversational exchanges.

\section{A forward look at future research}
\hypertarget{Toc124860699}{}
This work has identified a set of Italian modal particles, described their functions and contexts of use. Moreover, it has hopefully shown the feasibility of an approach which equally considers the functional domain of these elements (illocutionary modification), aspects related to their development (reanalysis of contextual inferences, routines in interaction), and variation (occurrences in regional varieties, variation in their acceptability). If this basic framework will be considered satisfactory, future research will explore these specific issues more deeply and further develop the main categories used. It will revise the case studies presented and conduct new ones, and it will improve the employed methodologies. Future work will refine the theoretical framework in terms of consistency and compactness – further developing specific concepts such as illocutionary modification, argumentative routines, degrees of conventionalization, and (pragmatic) salience.

On the “grammatical” side – the one this research mostly focused on – more theoretical and empirical work is needed to get to a comprehensive definition of modal particles, modal-particle-like elements and illocutionary modification. On the “conversational” side – occasionally touched upon in this research – more work is needed to further develop a model of hearer-based reanalysis and to improve the understanding of how inferences are processed and managed in interaction. A close analysis of the diversity of conversational environments is perhaps the best way to understand the behavior of modal-particle-like elements: including a conversation-analytical perspective can tell a lot about their use and development. Finally, behind everything else, the question remains of how to define \textit{conventions} in language: how they are negotiated in conversation, routinized in usage, and coded in grammar.

