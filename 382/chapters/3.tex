\chapter{Modal particles and illocutionary modification}\label{sec:3}
\hypertarget{Toc124860614}{}\section{Three central topics of pragmatics}
\hypertarget{Toc124860615}{}
To better frame the discussion on modal particles, it is necessary to introduce three core concepts of pragmatic theory which are essential to understand their functions and, generally, the behavior of discourse-pragmatic elements in interaction: speech acts, implicatures and presuppositions. As \citet[21]{Levinson1983} puts it: “Given a linguistic form uttered in a context, a pragmatic theory must account for the inference of presuppositions, implicatures, illocutionary force”. When investigating language use in context, speech acts occupy a prominent position, since delivering speech acts can be argued to be the central function of language \citep{Searle1969}: “the rest of the linguistic apparatus, with all of its complex syntax and propositional structure, is there to serve this purpose. For speech acts are the coin of conversation, and conversation the core niche for language use and acquisition” (\citealt{Levinson2017}: 215–216).

In my perspective, the description of modal particles is closely linked with the speech acts they occur in, and a satisfactory analysis of their functions cannot ignore their role at the level of illocution. On the other hand, implicatures are responsible for the rich overlap of different meaning levels that arise in conversation. The interplay of different meaning levels plays a crucial role in the semantic description of modal particles and their evolution over time: a satisfactory analysis of the emerging of their functions cannot ignore how the contextual reanalysis of implicatures/inferences shapes them. Finally, presuppositions come into play both in the semantic description of modal particles and their source constructions (some of which are presupposition-triggering elements) – and in the description of the conversational context in which they operate as well.

\subsection{Speech acts}
\hypertarget{Toc124860616}{}
The concept of \textit{speech act} refers to the fact that utterances, in addition to conveying meaning (whatever it may be), perform specific actions (\textit{do things}) and change reality through statements, orders, promises, warnings and so on. This concept was first developed by \citet{Austin1961,Austin1962}, who pointed out that utterances are not mere meaning-bearers, but they also have specific \textit{forces} that provide the clues to understanding in what ways, in uttering a sentence, one might be said to be performing actions.

Dealing with these subjects, the theory of speech acts describes how utterances have action-like properties, what exactly (and how many) these properties are, and how they are reflected in linguistic forms. When saying something, three kinds of acts are simultaneously performed (see \citealt{Levinson1983}: 236):

\begin{itemize}
    
\item \textsc{locutionary} \textsc{act}:    The utterance of a sentence with determinate sense and reference.

\item \textsc{illocutionary} \textsc{act}: The making of a statement, offer, promise, etc. in uttering a sentence, by virtue of the conventional \textit{force} associated with it (or with its explicit performative paraphrase).

\item \textsc{perlocutionary} \textsc{act}: The bringing about of effects on the audience by means of uttering the sentence, such effects being special to the circumstances of utterance.

\end{itemize}

The illocutionary act – the actual act normally referred to when talking about speech acts – corresponds then to the force associated with an utterance (that is, the force of ordering, warning, promising etc.). This force displays a certain degree of conventionality, depending on the issuance of a certain kind of utterance in an appropriate context and in accord with an established social/communicative procedure. In contrast, a perlocutionary act completely depends on context and on the specific circumstances of issuance, “and is therefore not conventionally achieved just by uttering that particular utterance, and includes all those effects, intended or unintended, often indeterminate, that some particular utterance in a particular situation may cause” \citep[237]{Levinson1983}.

Like all other actions, speech acts can succeed or fail in reaching their goals for a range of reasons. In the case of speech acts, these reasons are named \textit{felicity conditions}: both appropriate subjective states of the speaker as well as appropriate external circumstances are needed for a speech act to be valid. \citet{Searle1969} influentially systematized Austin’s work and proposed a typology of speech acts based on the felicity conditions they are bound to: \textit{representatives} (statements and the like), \textit{directives} (questions, requests, orders), \textit{commissives} (threats, promises, offers), \textit{expressives} (thanking, apologizing, congratulating), and \textit{declarations} (christening, declaring war, firing, which rely on elaborate institutional backgrounds).\footnote{Yet another approach working on speech act classification is represented by conversation analysis (see \citealt{SacksEtAl1974}; \citealt{Schegloff2017}), which studies all sorts of fundamental organizations for interactive language use: turn-taking, repair and sequence organization. In conversation analysis different types of speech acts can be identified through the sequential position that they hold in conversation: pre-closings (e.g. the exchange of \textit{well} before goodbyes in phone calls), assessments (evaluations of shared events or things), repair initiators (like \textit{Excuse me?}), pre-invitations (\textit{What are you doing on Friday night?}), and so forth (see \citealt{Levinson2017}: 202).}

Closer to linguistic theory is the problem of the \textit{grammar} of speech acts, i.e. how their conventionalized use in achieving a communicative intention matches formal linguistic properties. As \citet[202--203]{Levinson2017} underlines, “one of the central puzzles is that speech acts are not for the most part simply or directly coded in the linguistic form: for example, \textit{Where are you going?} could be an idle question, or a challenge, or a reprimand, or a prelude (a pre-) to a request for a ride or to an offer to give you a ride, and the relevant response depends on the correct attribution”. Illocutionary forces are often formally coded in major sentence types (imperatives, interrogatives, declaratives) and by explicit performative verbs, but this is not always the case. In addition, they can also be expressed through idiomatic means (interjections and related expressions). The absence of one-to-one mapping between form and function is further confirmed by indirect speech acts: for instance, prototypically directives are coded by imperative sentences, but interrogative sentences can code them as well.\footnote{The standard reference on these issues is \citet{SadockZwicky1985}. In the German linguistic tradition, the relationship between illocutionary force, sentence type (\textit{Satzmodus}) and their formal manifestations (for instance, \textit{mood} as a verb category) has been debated in several works (\citealt{Meibauer1987}; \citealt{Rosengren1992}; \citealt{Altmann1993}; \citealt{MeibauerEtAl2013}). Modal particles can be counted among the formal features contributing to mark a specific sentence type \citep{Thurmair1989}.}

However, as \citet[214]{Levinson2017} points out, many surface elements can help to narrow down an illocutionary force. For example, adverbs like \textit{please} mark unambiguous requests or pleadings, adverbs like \textit{obviously} or \textit{frankly} mark statements, and interjections like \textit{wow} or \textit{my gosh} mark exclamations. In addition, as noted by \citet{SadockZwicky1985} in their typological overview on syntactic means to distinguish speech acts, there are also minor sentence types that are indeed specialized for signaling a specific illocutionary force.\footnote{Citing from their examples, a case in point in English are particular syntactic constructions to express suggestions (\textit{How about a walk?}, \textit{Why not stay here?}, \textit{Let’s tour the island!}), exclamatives (\textit{What a beautiful day!}, \textit{Of all the stupid things to do!}) and optatives (\textit{If only I’d done it!}, \textit{May the best man win!}).} These observations are fundamental also when researching pragmatic markers and modal particles. Many of them could in fact count as elements autonomously displaying a specific illocutionary force \citep[75]{Austin1962} or elements that help to specify/modify the illocutionary force of the speech act in which they appear – as they occur only in particular illocutionary types or display functions related to their modification.

\subsection{Implicatures}
\hypertarget{Toc124860617}{}
A second fundamental notion in pragmatics is that of \textit{implicature}, as first developed in the work of \citet{Grice1975}. He introduced the term \textit{implicature} to separate what speakers explicitly \textit{say} when they utter a sentence and the constellation of further meanings that are \textit{implied} (suggested, meant) by that sentence in a particular context.

The notion of implicature has to be split in two: \textit{conversational implicature} and \textit{conventional implicature}. \citet[156]{Huang2017} defines the first category as follows: “Conversational implicature is definable as any meaning or proposition expressed implicitly by a speaker in his or her utterance of a sentence which is meant without being part of what is said in the strict sense”. A conversational implicature is part of what a speaker means, though not part of what a sentence explicitly says: it constitutes, therefore, a component of speaker meaning.

Conversational implicatures are bound to what Grice calls the \textit{cooperative principle}, a basic principle that emphasizes the rational nature of human communication as a shared-goal human activity, where both the speaker and the addressee act as rational and cooperative agents. The cooperative principle comes with four attendant maxims (quantity, quality, relation, manner) which represent general guidelines tacitly recognized by both the speaker and the addressee in order to achieve successful communication (see \citealt{Levinson1983}: 101–102).

Conversational implicatures arise as a consequence of the interplay (and often of some incompatibility) between the cooperative principle and the maxims, on the one hand, and a specific utterance in its context of occurrence on the other:

\ea%4
    \label{ex:key:4}

          Alice:  Do you know if George is at the library?

Beth:  I haven’t seen any red bike in the courtyard.

(+> George isn’t in the library)

  Alice:  He is probably late today.
    \z % you might need an extra \z if this is the last of several subexamples

In example \REF{ex:key:4}, Beth implies (+>) that George has not yet arrived at the library – since she can’t see his bike in the courtyard (and she assumes that he would cycle to the library, as usual). The utterance spoken by Beth could be said to contradict the maxim of quantity: the information she provides is more informative than required. However, relying on the cooperative principle, Alice accepts Beth’s utterance as a meaningful contribution to the talk exchange and consequently, she can correctly infer what Beth wants to suggest.

The second category of implicature put forward by Grice is \textit{conventional implicature}: “By conventional implicature is meant a non-truth-conditional meaning which is not derivable in any general considerations of cooperation and rationality from the saying of what is said, but arises solely because of the conventional features attached to particular lexical items and/or linguistic constructions” \citep[176]{Huang2017}. Contrary to conversational implicatures – though not contributing to the truth conditions of the sentence – conventional implicatures are not based on the cooperative principle, but on speakers’ knowledge of the language. A conventional implicature is therefore independent from what the speaker wants to express or implicate, but is an integral part of the meaning conventionally attached to words and phrases.

Taking an example from the class of focus particles, in \REF{ex:key:5} \textit{even} (a scalar focus particle) conventionally implicates some sort of unexpectedness, surprise, or unlikeness.

\ea%5
    \label{ex:key:5}
    
          \textit{Even} Philip now cycles to work: since last summer, he has been very worried about global warming.
    \z % you might need an extra \z if this is the last of several subexamples

This implicature – though part of what the speaker subjectively wants to express – is coded as a fixed component of the conventional meaning of \textit{even}.

\hspace*{-2.3pt}Among the linguistic expressions that activate conventional implicatures, there are different kinds of connectives, sentence adverbs, quantifiers, honorifics and many others (see \citealt{Levinson1983}: 127–130; \citealt{Huang2017}: 175–180). I will return to conversational and conventional implicatures later on to assess their place across the semantics/pragmatics divide and to discuss the relationship between them from a diachronic perspective.

\subsection{Presuppositions}
\hypertarget{Toc124860618}{}
A third key-notion of pragmatics is that of \textit{presupposition}. Presuppositions represent pieces of \textit{background information} (or are presented as such by the speaker): they are background assumptions against which the main contribution of an utterance is to be assessed. A useful analogy here is the notion of figure and ground in Gestalt psychology: in a picture, a figure stands out only relative to a background, and there are well-known visual illusions or “ambiguities” where figure and background are reversible, demonstrating that the perception of each is relative to the perception of the other. The analogy is that the figure of an utterance is what is asserted or what is the main point of what is said, while the ground is the set of presuppositions against which the figure is assessed \citep[180]{Levinson1983}.

In contrast to conversational implicatures – and similarly to conventional implicatures – presuppositions are closely bound to the actual linguistic structure of sentences: they are triggered by certain words and constructions and, in this sense, built into linguistic expressions. There are many of such presupposition-triggering expressions, that is, linguistic elements that function as sources of presuppositions (see \citealt{Levinson1983}: 181–184; \citealt{Geurts2017}: 180). An incomplete list includes definite articles, quantifiers, factive verbs (such as \textit{regret} and \textit{realize}), change-of-state verbs (such as \textit{stop} and \textit{continue}), temporal subordinators (such as \textit{while} and \textit{since}), cleft sentences, and focus particles (such as \textit{only} and \textit{too}). Let’s see an example:

\ea%6
    \label{ex:key:6}

          Alice has stopped writing poetry.

(→ Alice has been writing poetry)
    \z % you might need an extra \z if this is the last of several subexamples

In example \REF{ex:key:6}, the fact that Alice has stopped writing poetry presupposes that she has been writing poetry. The fact that presuppositions really form a set of background assumptions can be demonstrated by changing the polarity of the proposition (\textit{Alice has not stopped writing poetry}) or by converting it into a question (\textit{Has Alice stopped writing poetry?}). Both sentences share the presupposition (→) that \textit{Alice has been writing poetry}: in both cases, the background assumption remains the same. “Thus the main point of an utterance may be to assert or to deny or to question some proposition, and yet the presuppositions can remain constant, or – to employ our analogy – the figure can vary within limits, and the ground remain the same” (\citealt{Levinson1983}: 180–181).\footnote{This is not the case for ordinary entailments (see \citealt{Levinson1983}: 191–198; \citealt{Geurts2017}: 180–181).}  Put otherwise: “Generally speaking, presuppositions tend to escape from any embedded position in the sense that, whenever a sentence $\varphi $ contains an expression triggering the presupposition that χ, an utterance of $\varphi $ will imply that χ is true” \citep[181]{Geurts2017}.

Widening the perspective, it may be said that presuppositions are part of the context in which a speech act is performed. How to better define \textit{context}, however, is not straightforward.\footnote{For an overview of issues related to this concept, see \citet{Fetzer2017}. See also \citet{FetzerFischer2007}.} As \citet[182]{Geurts2017} puts it: “The standard answer is that presuppositions are part of the common ground between speaker and hearer: by using an expression which triggers the presupposition that χ, the speaker signals (or acknowledges) that χ is already part of the common ground (\citealt{Stalnaker1973,Stalnaker1974}). At any given moment in the discourse, the common ground consists of the information all participants accept as true at that point”.\footnote{See also \citet[128]{Sæbø2016} who discusses the relationship between presuppositions and information structure: “Classically, presuppositions have been considered as conditions that the point of evaluation must meet for the sentences that carry them to be true or false. For the phenomena at issue in this article, however, they are more appropriately, in accordance with the dominant picture today, regarded as conditions that the context, or the Common Ground, must meet in order to be updated with the sentence”.} Thus, the common ground represents a mental and interactional space which accounts for the coexistence of different kinds of inferences.

\section{Modal particles}
\hypertarget{Toc124860619}{}
Broadly speaking – in the languages which display such a class – modal particles (MPs) are usually seen as “conveying certain ‘pragmatic presuppositions’ about the context of utterance, including in particular the relationship between speaker and hearer” (\citealt{Hansen1998a}: 42; \citealt{Diewald2013}: 33). Modal particles establish a link between the speech act they occur in (including the propositional content conveyed) and the interlocutors’ expectations based on the previous discourse and the extralinguistic situation.

From this perspective, they count as linguistic means which are used to explicitly manage some implicit content of the communicative exchange: they accommodate what is said (propositional content) and what is performed (a speech act) into the context of conversation – which is formed not only by extralinguistic references (real-world entities), but also by the set of knowledge shared by the interlocutors (the common ground).

\hspace*{-.9pt}This section will give an overview of the semantic/pragmatic domains involved in the description of the functions of modal particles. I will first comment on the idea of the speech-act theoretic approach to the functions of MPs elaborated by \citet{Waltereit2001,Waltereit2006}, linking it subsequently to the concepts of \textit{common ground} and \textit{illocutionary force}.

\subsection{The functions of modal particles}
\hypertarget{Toc124860620}{}
The major influence for the approach adopted here is represented by the work of \citet{Waltereit2001,Waltereit2006}. A fundamental characteristic of this work is the combination of a semantic and pragmatic analysis of the functions of (German) modal particles with a contrastive interest, aimed at detecting the formal manifestations of the same functions in languages that do not have a specific class of particles to express them, specifically Romance languages.

As \citet[1391]{Waltereit2001} puts it: “The intuition underlying this article is that if, e.g., the Romance languages have fewer modal particles than German, they should have other means of expressing the same function. It is claimed that this function essentially consists in accommodating the speech act at minimal linguistic expense to the speech situation”. Accordingly, this specific function (which is sometimes referred to with the German term \textit{Abtönung} ‘shading, modalization’) turns out to be a universal category and languages which do not have modal particles as a separate word class utilize a diverse array of linguistic means to express it: specific uses of adverbs, intonation, dislocation, tag questions, specific uses of tense, and diminutive morphology.

Thanks to these characteristics, such an approach seems to be particularly suitable to study modal-particle-like elements in languages that do not have a clearly defined paradigm of modal particles. \citet[1393--1397]{Waltereit2001} develops a speech-act-theoretic approach based on two hypotheses:

\begin{itemize}
\item \textsc{hypothesis} \oldstylenums{1}: Modalization is essentially a speech-act-level operation insofar as the preparatory conditions of the speech act are modified at minimal linguistic expense.

\item \textsc{hypothesis} \oldstylenums{2}: Modalization forms modify the preparatory conditions, as they evoke a speech situation in which the desired preparatory conditions are fulfilled. Thereby they enable the speaker to justify her speech act.

\end{itemize}

\hspace*{-4.2pt}This approach makes crucial reference to the preparatory conditions of a speech act, which “describe the way the speech act fits into the social relation of speaker and addressee, and they describe how their respective interests are concerned by the act” \citep[1397]{Waltereit2001}.\footnote{Preparatory conditions count among the felicity conditions of speech acts (see \citealt[239]{Levinson1983}).} In this perspective, modalization forms appear in speech situations where the preparatory conditions of a speech act are not (entirely) fulfilled. In the case that speakers wish to perform a speech act that is not sufficiently licensed by the speech situation or that might have undesired side-effects, modal particles and equivalent constructions explicitly signal it, providing clues as to the “justification” of the speech act in the relevant speech situation.

These kinds of circumstances are exemplified through the analysis of the function of the German modal particle \textit{ja} (\citealt{Waltereit2001}: 1398–1399). This particle occurs (mainly) in the speech-act class of assertions, which – according to \citet{Searle1969} – display as part of their preparatory conditions that it is not obvious to both speaker and addressee what the propositional content conveyed by the speech act is.

\ea%7
    \label{ex:key:7}

          German \citep[1398]{Waltereit2001}

Die Malerei war \textit{ja} schon immer sein Hobby.

\glt ‘\textit{(As you know)}, painting has always been his hobby.’
    \z % you might need an extra \z if this is the last of several subexamples

The modal particle \textit{ja} occurs exactly in speech situations in which this condition is not fulfilled, namely the propositional content of the speech act is known to both interlocutors, either because it is part of their previous knowledge or because it is inferable from the external context. By inserting \textit{ja} in the assertion, the speaker signals an inconsistency in the preparatory conditions of the speech act (Hypothesis 1) and – at the same time – “that the assertion counts as a relevant contribution to conversation even if its propositional content is obvious to the addressee” \citep[1398]{Waltereit2001}. By doing so, the performance of the speech act is justified.

Moreover, he relates the modal use of \textit{ja} to the function of \textit{ja} as an answering particle (comparable to English \textit{yes}). The argument he provides is that – used as an answering or confirming particle – \textit{ja} creates a speech situation where speaker and addressee agree on the content of a situationally relevant proposition (i.e., the utterance that it is used to respond to). As \citet[1399]{Waltereit2001} puts it, “the \textit{result} of saying \textit{ja} as a (non-modal) response token (i.e., speaker and addressee agree about the propositional content of the first utterance) corresponds to a \textit{presupposition} or \textit{implicature} of its use as a modal particle (i.e., speaker and addressee agree about the propositional content of the assertion containing the particle, by virtue of there being strong evidence for it)”. In other words, the link between the modal particle employed and the non-modal counterpart consists in the fact that the modal particle evokes a speech situation in which the desired preparatory conditions for the utterance containing the modal particle are fulfilled (Hypothesis 2).\footnote{The reference to the external speech situation also plays a role in the development of modal particles. \citet[1410--1414]{Waltereit2001} conceives of the diachronic rise of modalization forms as an instance of (metonymic) semantic change involving the speech act domain. In this perspective, some contextual features of typical speech situations belonging to the speech act domain, but not directly bearing on the potential illocutionary force of the relevant items, can also become part of the meaning of these items. For work on metonymic change in the speech act domain (for example the recruitment of speech act verbs) see \citet{Koch1999,Koch2001}.}

This approach has been developed in later works (\citealt{Waltereit2006}; \citealt{WaltereitDetges2007}; \citealt{DetgesWaltereit2009}; \citealt{DetgesGévaudan2018}), applying it to the analysis of Romance forms and constructions which display functions similar to those described for German modal particles. In particular, \citet{DetgesGévaudan2018} highlight the link between \textit{Abtönung} and the theory of \textit{linguistic polyphony} as proposed by \citet{Ducrot1984}, according to which “the speaker can evoke different ‘viewpoints’ (i.e. illocutionary attitudes, propositional viewpoints etc.) and/or different ‘voices’ (i.e. reference to other speakers’ discourse in various forms such as irony, imitation, direct reported speech etc.)” (\citealt{DetgesGévaudan2018}: 307).

In this perspective, the function performed by modal particles is redefined as a form of polyphony, aimed at accommodating (integrating) a speech act to the viewpoints active in a specific speech situation (common ground). Integrating speech acts in the common ground, modal particles evoke a viewpoint that is not simply ascribed to the speaker alone, but crucially involves also the hearer. Thus, “they provide clues as to how the speaker’s assertion ties in with the addressee’s world- and discourse knowledge as hypothesized by the speaker” (\citealt{DetgesGévaudan2018}: 308). This is shown in the next example for different modal particles:

\ea%8
    \label{ex:key:8}

          German (\citealt{DetgesGévaudan2018}: 307–308)

Das ist \textit{ja}/\textit{doch}/\textit{vielleicht}/\textit{aber} großartig!

\glt ‘That’s \textit{MP} great!’

\begin{itemize}
    
\item \textit{ja}: speaker’s assertion \textit{is} obvious to addressee

\item \textit{doch}: speaker’s assertion \textit{should} be obvious to addressee

\item \textit{vielleicht}: asserted proposition \textit{unexpected} for speaker and \textit{unknown} to addressee

\item \textit{aber}: asserted proposition \textit{unexpected} for speaker but \textit{known} to addressee
\end{itemize}
    \z % you might need an extra \z if this is the last of several subexamples


\subsection{The many facets of common ground}
\hypertarget{Toc124860621}{}
The last observations have called into question a set of concepts which need further clarification, such as \textit{common ground}, \textit{speaker-hearer link}, and \textit{viewpoints}. They all refer to the capacity of modal particles of indexing mental representations: “the presence of MPs signals the presence of an active speaker, who wants to stress her own mental representation of a certain fact and to attribute a certain attitude and state of knowledge to the hearer” \citep[283]{Coniglio2012}.

In fact, many works have stressed this aspect of the semantics of modal particles rather than their connection to the preparatory conditions of a speech act. This means that the conceptual space around modal particles (and modal-particle-like elements) can be expanded to include other concepts (see the papers collected in \citealt{AbrahamLeiss2012}). From this perspective, a fundamental role is played by the notion of \textit{common ground}, cited for example by \citet{Grosz2016}: “As a working definition, we can define discourse particles [his term for modal particles] as a closed class of functional elements that contribute to \textit{common ground management} in the spirit of \citet{Krifka2008}; that is, they encode specific instructions on how the Common Ground should or should not be modified in the subsequent discourse”.

Common ground is defined by \citet[245]{Krifka2008} as “information that is mutually known to be shared in communication and continuously modified in communication”. The concept of common ground (henceforth CG) refers to a universe of discourse where the speaker and the addressee share the knowledge of some propositions and formulate assumptions about each other’s states of mind. Through this model, it is possible to distinguish presuppositions as requirements for the input CG, and assertions as the proposed change in the output CG. Furthermore, \citet[246]{Krifka2008} separates CG content – that is, the truth-conditional information (propositional content) in the CG – from the CG management – that is, information about the manifest communicative interests and goals of the participants. The identification of the dimension of CG management – understood as the conversational push given by one of the interlocutors so that CG content develops in an intended or desired direction – is fundamental for the analysis of the functions of modal particles.

Citing the modal particle \textit{ja} as a prime example, \citet[337]{Grosz2016} explains that “\textit{ja(p)} triggers a presupposition that the contextually given speaker believes that the modified proposition \textit{p} is true; it furthermore presupposes a belief concerning the contextually given addressee, namely that she either knows that \textit{p} is true, or that the truth of \textit{p} is evident in the utterance context”. In terms of common ground management, \textit{ja} conveys that uttering the proposition it scopes over does not have the main goal of updating the common ground, since it is already present in the common ground or evident in the utterance of context. According to this explanation, \textit{ja} acts as a presupposition trigger, that is to say, as a linguistic item signaling that (part of) the conveyed information must be considered as given or taken for granted.

However, in my view, the crucial point is the fact that the particle does not operate as a trigger for presuppositions related to the propositional content \textit{per se}, but it acts as a trigger for presuppositions that influence the way the speech act must be interpreted in the context of interaction, namely, they are also relevant for the common ground management. From this perspective, this approach is not distant from that of \citet{Waltereit2001} discussed in detail in the previous subsection: preparatory conditions are part of the communicative conditions necessary to successfully perform a speech act (and to convey its propositional content) according to a specific context of interaction. Therefore, modification of the preparatory conditions and activation of presuppositions are both part of the common ground as a dynamic conversational dimension and they both pertain to the functions of modal particles as tools of common ground management.

Moreover, recalling the arguments of \citet{DetgesGévaudan2018}, modal particles should be analyzed as a means to manage the speaker-hearer link. In this respect, it must not be forgotten that modal particles operate in the sphere of a speech act and they thereby relate a specific \textit{illocution} to the common ground. Another strand of research (see for instance \citealt{Jacobs1991}; \citealt{Coniglio2012}) has focused specifically on the kind of relation which exists between modal particles and the illocutionary force: “They interact with the illocution and must be anchored to a speaker as the ‘author’ of a certain speech act and as the possessor of a certain mental representation” \citep[255]{Coniglio2012}.

In this approach, modal particles play a crucial role in the way speakers as performers of a speech act want to achieve their communicative point: particles take up a certain illocutionary type and modify it by restricting and specifying it. For example, as \citet[262--263]{Coniglio2012} shows, a default imperative clause can be further specified as a cogent order (this is the case of the stressed particle \textit{ja}) or a less peremptory order, a request, a suggestion or a piece of advice (this is the case of the particle \textit{mal}). Modal particles can thus be analyzed as operators on the \textit{illocutionary point} of a speech act – that is, that component of the illocutionary force which represents the basic purpose of a~speaker~in making an utterance. This fact is further proven by the observation that many modal particles tend to be associated with specific illocutionary types: as a rule, each modal particle can only occur in a subset of all illocutionary types available (\citealt{Jacobs1991}; \citealt{DetgesWaltereit2009}).

\subsection{Conditions and intentions}
\hypertarget{Toc124860622}{}
As is clear from the preceding discussion, the status of modal particles is still highly debated and the term covers a wide range of concepts – from common ground to illocutionary force, from presuppositions to preparatory conditions – which cannot all be revised here in detail. Moreover, a large part of this debate has developed through the analysis of German modal particles and thus builds on categories that cannot always be easily transferred to other languages (especially the syntactic ones): typological work on modal-particle-like elements remains a desideratum.

In this sense, some (first) suggestions can be found in \citet{Hengeveld2004} and \citet{HengeveldMackenzie2008}, where the label \textit{illocutionary operator} is used to broadly refer to grammatical items that emphasize or mitigate the force of a specific illocutionary act. The functions of illocutionary operators and their syntactic scope admittedly bring to mind those of modal particles, which could be seen as a language-specific manifestation of this category. Indeed, among the examples they cite, the use of the Dutch \textit{maar} ‘only’ with a mitigating function in declarative and imperative illocutions appears to be a typical example of modal particle:

\ea%9
    \label{ex:key:9}

          Dutch (\citealt{HengeveldMackenzie2008}: 83)
\ea  Je moet \textit{maar} gaan fietsen.

\glt ‘You should go for a bike ride, you know.’

\ex  Ga \textit{maar} fietsen.

     ‘Why not go for a bike ride?’
    \z
\z

Considering the objectives of this research and the fact that I focus on Italian – a language where no definite class of modal particles is found – I will not aim at giving a comprehensive theoretical analysis of modal particles and their functions. Nevertheless, in the previous pages some features of modal particles on which there is a substantial degree of agreement have been highlighted and I would like to close this section with three statements that I will use as guidelines for my analysis.

First, the sphere of action – or “natural habitat” – of modal particles is the speech act: from a syntactic perspective, these elements – having scope over the whole host utterance – operate at the layer of the illocution. Secondly, from a discourse-pragmatic perspective, they operate on the \textit{conditions} under which the speech act is performed: as markers of common ground management, they fine-tune the speech act by repairing problems that arise from the violation of a preparatory condition. Moreover, they contribute to the managing of the information flow with respect to what has been explicitly mentioned in the discourse while also considering what can be indirectly inferred from previous discourse elements, shared knowledge, and world knowledge. Thus, they play a fundamental role in the speaker-hearer link. Finally, they specify the \textit{intentions} with which speech acts are performed: by matching the speaker’s communicative tension with their viewpoint on the hearer’s expectations, they refine the illocutionary point of the speech act, adapting it to the context of interaction and facilitating its interpretation in an interpersonal perspective.

In this way, modal particles simultaneously operate on (and show the intertwining of) the different dimensions of a speech act: its felicity conditions, its illocutionary force, and the proposition carried by it. These different aspects define the grammatical domain in which modal particles operate: following \citet[1192]{Hengeveld2004}, \citet[22--25]{Waltereit2006} and \citet[13]{Narrog2012} I will name it \textit{illocutionary modification}.

\section{Illocutionary modification}
\hypertarget{Toc124860623}{}
At the beginning of this chapter notions such as \textit{speech act} and \textit{illocutionary force} were introduced from a broad theoretical perspective. In order to analyze the functions of modal particles and modal-particle-like elements, it is necessary to consider how these notions fit into models of (core) grammatical functions.

On the one hand, this will allow me to better frame the behavior of modal particles as grammatical means that serve as indicators of illocutionary force. On the other hand, it will highlight the boundaries (and the overlapping areas) between illocutionary modification and other grammatical categories.

In the next few pages – combining notions coming from modal particle research (\citealt{Waltereit2001,Waltereit2006}; \citealt{Coniglio2012}) with notions from the research tradition in functional grammar (\citealt{Hengeveld2004}; \citealt{HengeveldMackenzie2008}; \citealt{Narrog2012}) – I will spell out the concept of \textit{illocutionary modification} in more detail, since it represents the bulk of the analytical framework adopted in the present work.

\subsection{Illocution and illocutionary modification}
\hypertarget{Toc124860624}{}
Introducing the speech-act categories above, I highlighted the absence of a one-to-one correspondence between speech acts and linguistic forms. Rather, a conventional association between sentence types (which are marked by various grammatical means such as verb forms and intonation) and illocutionary forces can be observed: declaratives, in their prototypical function, express assertions, interrogatives express requests for information, imperatives express orders and exclamatives express the speaker’s feelings towards a fact.

This can lead us to argue for a basic distinction, mainly for descriptive purposes, between the force component expressed in the form of the sentence (abstract sentential force), which is independent of the context of use as part of the literal meaning of the sentence, and the force which characterizes the actual illocutionary act (speech act).\footnote{Many partially overlapping concepts can be found in the literature referring to these distinctions, such as \textit{sentence type}, \textit{sentence mood}, \textit{illocution} and \textit{illocutionary type} (on this, see the short summary presented by \citealt{AlmEtAl2018}: 3–5).} For instance, \citet{Hengeveld2004} distinguishes between the basic illocution of a sentence and further modifications to it:

\begin{quote}
The basic illocution of a sentence can be defined as the conversational use conventionally associated with the formal properties of that sentence (cf. \citealt{SadockZwicky1985}: 155), which together constitute a sentence type. Apart from word order and intonation, these formal properties may include specific mood morphemes, which may in these cases be interpreted as the morphological markers of basic illocutions. \citep[1191]{Hengeveld2004}
\end{quote}

\begin{quote}
Basic illocutions may be further modified by markers of what I here call illocutionary modification. Like basic illocution, illocutionary modification should be interpreted in terms of the conversational use of sentences. But unlike basic illocution, markers of illocutionary modification do not identify sentences as speech acts of certain types, but rather mark much more general communicative strategies on the part of the speaker: they reinforce or mitigate the force of the speech act \citep[1192]{Hengeveld2004}
\end{quote}

The concept of \textit{basic illocution} refers to categories such as declarative, interrogative, imperative, prohibitive, optative and so on. These are abstract illocutionary primitives identified by the grammatical distinctions (morphosyntactic and phonological) represented across languages. Conversely, the label \textit{illocutionary modification} refers to the various grammatical means that modify the illocutionary force of a speech act and further differentiate between communicative intentions (specific illocutionary forces such as making statements and requests, giving orders, warnings and permissions).

Languages display a wide range of constructions to mark illocutionary modification: prosody, word order, syntactic constructions, specific uses of adverbs, and morphology (see \citealt{Waltereit2001}; \citealt{HengeveldMackenzie2008}: 81–84). Illocutionary modification is not yet an established category in linguistic analysis nor has it been consistently applied to the analysis of modal particles. Nevertheless, it could be a useful category to include modal particles and modal-particle-like elements in a broader cross-linguistic perspective. Moreover, the detailed work already existing on the semantics and pragmatics of modal particles could serve to refine the notion of illocutionary modification and make it a well-established category in typological research.

So far, as already mentioned above, this label has been used mainly within the framework of Functional Discourse Grammar (FDG), a typologically-based theory of language structure (\citealt{HengeveldMackenzie2008}; \citealt{Keizer2015}). FDG relies on the idea that grammatical categories are organized in layers, connected to each other by scope relations: “in Functional Discourse Grammar scope relations are defined in terms of different pragmatic and semantic layers. Pragmatic layers together constitute the interpersonal level in this model, while semantic layers together constitute the representational level” \citep[15]{Hengeveld2017}. The hierarchical organization of layers and levels is represented in \tabref{tab:key:3.1}, adapted from \citet[16]{Hengeveld2017}.

\begin{table}
\footnotesize
\begin{tabular}{lclclclcl}
\lsptoprule
\multicolumn{9}{c}{Interpersonal}\\
\multicolumn{9}{c}{Level}\\
\midrule
Discourse & > & Illocution & > & Communicated & > & Referential & > & Ascriptive\\
Act & & & & Content & & Subact & & Subact \\
\multicolumn{9}{c}{˅}\\
\multicolumn{9}{c}{Representational}\\
\multicolumn{9}{c}{Level}\\
\midrule
Proposition & > & Episode & > & State-of-Affairs & > & Configu-& > & Property\\
& & & & & & rational & & \\
& & & & & & Property & & \\
\lspbottomrule
\end{tabular}
\caption{\label{tab:key:3.1} Scope relations in FDG}
\end{table}

The interpersonal level deals with all the formal aspects of a linguistic unit that reflect its role in the interaction between speaker and addressee. At the interpersonal level scope relations are defined in terms of different pragmatic layers, which are the most relevant for the present research. Moving inside out, there are the \textit{ascriptive subact} and the \textit{referential subact}, which are the building blocks of the communicated content; the \textit{communicated content} itself, which represents the message transmitted in an utterance; the \textit{illocution}, which specifies the communicative intention of the speaker; and the \textit{discourse act}, which is the basic unit of communication. In a similar way, the representational level deals with the relation that obtains between language and the non-linguistic world it describes. At the representational level, scope relations are defined in terms of different semantic layers, which range from \textit{property} to \textit{proposition}.

Moreover, the basic content of each layer may be further specified by operators and modifiers: operators capture specification by grammatical means, while modifiers capture specification by lexical means. At the representational level, categories such as aspect, tense and modality are coded. At the interpersonal level, categories such as mirativity, approximation, reportativity and, in fact, illocutionary modification are coded.\footnote{A detailed list of the modifiers and operators used in FDG can be found in \citet[492]{Hattnher2015} and \citet[17]{Hengeveld2017}.} From this perspective, illocutionary modification corresponds to specification by grammatical means at the layer of illocution.

\subsection{Illocutionary modification and its grammatical surroundings}
\hypertarget{Toc124860625}{}
Further reference to FDG notions will take advantage of the layered structure of grammatical categories posited by this framework to include discourse-pragmat\-ic functions in a broader picture of grammatical functions. More specifically, this means that discourse-pragmatic functions will be described not as isolated or marginal points of the grammatical system, but as an integral part of it – neighboring and (partially) overlapping with other grammatical categories.

Discourse-pragmatic functions pertain to the interpersonal level – the “natural” environment of pragmatic markers – where they act as operators on communicated content, illocutions, and discourse acts. However, their largely observed polyfunctionality makes things more complex: many items cross the divide between the interpersonal and representational level and end up in the domain of other grammatical categories. Among other things, the analysis presented in the case studies will seek to highlight how some markers do not cover only discourse-pragmatic functions at the interpersonal level, but can also act as markers of tense, aspect, event quantification, and modality.

In this sense, a further reference is represented by the works of \citet{Narrog2012,Narrog2017}, which focus on semantic change in the domain of modality. For the purposes of the present research, the crucial feature of this model is the explicit inclusion of illocutionary modification as a grammatical category bordering the domain of modality: “A further step beyond modality and mood are illocutionary force and illocutionary (force) modification (IM), i.e. the expression of the communicative purpose of an utterance, such as making a statement, a promise, or a prediction, and its modification” \citep[13]{Narrog2012}.

This category is not strictly modal (illocutionary modification usually does not change the factuality of a sentence) and it is further up the scale of speech act orientation than modality: it mainly concerns the interaction of speaker and hearer in discourse. Illocutionary modification is thus firmly envisioned as part of a grammatical system and involved in the change processes affecting it Moreover, it is involved in predictable paths of change which can be empirically tested through the description of the behavior and the development of single items. Importantly, \citet{Narrog2012} also mentions what kind of items instantiate this category and are concerned in its change dynamics:

\begin{quote}
Among the categories related to the domain of modality, hearer orientation and discourse orientation are most obviously at play in the category of illocutionary force. Many discourse markers in English, sentence-final particles in Japanese, and \textit{Modalpartikeln} in German very directly code the speaker’s attention to the addressee as a participant in the speech event, or to the discourse context. (\citealt{Narrog2012}: 50–51).
\end{quote}

This quote calls into question exactly the items cited in the previous chapters as members of the overarching category of pragmatic markers, thus including in all respects discourse-pragmatic functions in a broader model of grammatical categories and their development.

Based on a combination of notions used in the research on modal particles and notions used in the research tradition of functional grammar, this concept of illocutionary modification constitutes the main analytical category of the present work. The case studies presented in \chapref{sec:6}--\chapref{sec:9} aim precisely at identifying in which contexts of use certain Italian adverbs operate as markers of illocutionary modification. Moreover, they represent a test bench for the hypothesis that illocutionary modification subsumes three main functions: (i) modification of the illocutionary force of a speech act (reinforcement and mitigation); (ii) marking of the communicative intention expressed (specification of the illocutionary point of the speech act); (iii) marking of the conditions under which the speech act is performed (integration of a speech act in the relevant common ground and management of contextual inferences). In addition to this, by exploring the network of functions covered by single items, the case studies will introduce new evidence concerning the position of illocutionary modification in a layered model of grammar and the surrounding categories.
