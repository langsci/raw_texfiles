\chapter{Modal particles in regional varieties of Italian: Expanding the view}\label{sec:9}
\hypertarget{Toc124860680}{}\section{Multiple ways to illocutionary modification}
\hypertarget{Toc124860681}{}
This chapter widens the picture given so far, taking into account other Italian modal-particle-like elements and drawing attention to other functional domains connected to the speech-act level (backchecking, emphasis, mitigation). This chapter is not devoted to the in-depth analysis of a single element, but rather deals with a larger set of adverbs: some of them have already been introduced in the previous chapters (\textit{pure}, \textit{anche}, \textit{un po’}), others will be introduced here for the first time (\textit{già}, \textit{poi}). By doing so, it aims to discuss different functional domains, in order to point out research directions that cross the fields of pragmatics and sociolinguistics.\footnote{While the case study on \textit{solo} presented in the previous chapter adopted a semasiological approach (form-to-function) – picking up a linguistic element to describe its meanings and functions – the present chapter adopts instead an onomasiological approach (function-to-form): given a functional domain, the linguistic elements expressing it are investigated.}

The empirical analysis revolves mostly around a second sociolinguistic questionnaire, which I used to collect data about the acceptability of specific constructions. Being the last chapter of this study, the final section offers a first round of conclusions concerning the sociolinguistic issues discussed throughout the research.

\subsection{Backchecking markers}
\hypertarget{Toc124860682}{}
Building upon previous research (\citealt{BazzanellaEtAl2005}; \citealt{Hansen2008}; \citealt{HansenStrudsholm2008}; \citealt{Välikangas2004}), \citet{Squartini2013,Squartini2014} studies the discursive uses of Romance phasal adverbs, with a focus on the French adverb \textit{dèjà} and the Italian cognate form \textit{già}, both meaning ‘already’. Both adverbs show – among other context-level functions\footnote{French \textit{déjà} can be used as an emphatic marker in directive speech act as in example \REF{ex:key:100} below, while Italian \textit{già} can be used as an interjection and as a discourse marker (\citealt{Squartini2013}: 172–181).} – a backchecking use in interrogative sentences. Backchecking particles are used in questions to signal that the requested piece of information belongs to common knowledge and, as such, used to be well known also to the speaker, who, however, has forgotten it. In contrast with the phasal use – referring to the temporal constituency of a single situation denoted in the propositional content – in this case “by using \textit{déjà}/\textit{già} the speaker is instead discursively qualifying the speech act (the question itself), in these cases signaling that the question might be considered as redundant and only due to a contingent extralinguistic fact (an accidental tip of the tongue)” (\citealt{Squartini2013}: 167–168).

\ea%97
    \label{ex:key:97}

          French \citep[195]{Squartini2014}

Quel est \textit{déjà} le nom de cet acteur qui se prénomme Robert et joue au côté de Marilyn Monroe dans Rivière sans retour?

\glt ‘What is \textit{already} the name of that actor whose first name is Robert, who acts with Marilyn Monroe in River of No Return?’
\z % you might need an extra \z if this is the last of several subexamples

\ea%98
    \label{ex:key:98}

          Italian \citep[200]{Squartini2014}

  com’è \textit{già} che si fa a calcolare la media?

\glt ‘how do you calculate (lit. ‘how is it already that you calculate’) the average mark?’
    \z % you might need an extra \z if this is the last of several subexamples

In these examples, the original TAM marker ‘already’ does not indicate anteriority with respect to a given state-of-affairs denoted in the proposition, it rather refers to the whole informational content of the utterance as already possessed by the speaker and momentarily forgotten due to a contingent extralinguistic fact: it is the speaker’s knowledge of the whole propositional content of the question to be marked as information “already” given and shared by the speaker and the addressee \citep[1999]{Squartini2014}.

In order to give a full account of Italian \textit{già}, considering regional varieties is particularly relevant. In fact, as regards contemporary Italian, the occurrence of this interactional \textit{già} is considered a regional feature. Speakers from the north-west of Italy, especially those from Piedmont, seem to behave like their French neighbors in admitting \textit{già} as an interrogative marker, but the same does not hold for other regional varieties of Italian. However, there is some controversy on the actual geographical extent of this phenomenon. \citet[113–114]{Cerruti2009} explicitly lists it among the diatopic-marked features typically characterizing regional varieties of the Northwest of Italy. Conversely, \citet[55]{BazzanellaEtAl2005} suggest that the interrogative use of \textit{già} should not be considered as a regional phenomenon, since it possibly extends to the standard variety of Italian.

\citet{FedrianiMiola2014} document the areality of this phenomenon by extending the analysis to other European languages and pragmatic markers with the same function occurring in other Italian areas neighboring Piedmont. According to their description, the same backchecking function expressed by \textit{già} in the regional variety spoken in Piedmont/Northwest Italy is fulfilled by \textit{più} ‘more’ in Ligurian Regional Italian and by \textit{pure} ‘also’ in Emilian Regional Italian. Moreover, they suggest that there is also evidence for a wider area of employment of \textit{già} as a backchecking particle, ranging from northern and western Lombardy to Romagna (\citealt{FedrianiMiola2014}: 181). One of the stimuli of the questionnaire is intended to collect empirical data on this point.

In Italian, another element that shows a similar behavior in some contexts is \textit{poi} ‘then’, as mentioned by Coniglio (\citeyear[111--114]{Coniglio2008}; see also \citealt{Bazzanella1995}: 226–227; \citealt{CruschinaCognola2021}). According to his explanation, \textit{poi} can be used in questions to “express the speaker’s concern or interest with respect to the information being asked for” \citep[111]{Coniglio2008}. More precisely, I would consider \textit{poi} in questions as a backchecking-like element: in fact, it refers to previously given information or to a previously mentioned topic of conversation. The same adverb occurs with a similar function also in (mostly negative) assertions: in this case, apart from marking the whole propositional content as “given”, the adverb gives a counter-expectational flavor to the utterance. \citet[112]{Coniglio2008} argues that “by using this particle, the speaker aims at mitigating the too strong assertion that is present in the preceding linguistic or extralinguistic context”.

\ea%99
    \label{ex:key:99}

          Italian \citep[112]{Coniglio2008}

  \ea\label{ex:key:99a} Ha \textit{poi} cantato alla festa?

\glt ‘Did \textsc{ptc} she sing at the party?’

   \ex \label{ex:key:99b} Non è \textit{poi} così male!

\glt ‘It’s not \textsc{ptc} that bad!’
    \z % you might need an extra \z if this is the last of several subexamples
    \z

Through the question marked by \textit{poi} in \REF{ex:key:99a}, the speaker recalls a state-of-affairs known to both them and the addressee, or a previous conversation about the same topic (in this case, the option of singing at the party). This way, \textit{poi} signals that the question is not out of the blue, but rather refers to a piece of knowledge shared by the interlocutors. Similarly, the assertion in \REF{ex:key:99b} marks a contrast with the previous assumption, corresponding to shared knowledge or information present in the preceding context (in this case, the assumption that something is bad).

\subsection{Markers of emphasis and mitigation}
\hypertarget{Toc124860683}{}
Regarding French \textit{déjà}, it must be noticed that the modal uses of the adverb are not limited to the tip-of-the-tongue situation, rather, it shows a wider distribution than (Piedmontese Regional) Italian \textit{già}. \citet[497–498]{HansenStrudsholm2008} and \citet[213–216]{Hansen2008} point out that \textit{déjà} shows further interactional uses, both in imperative (this use being extremely rare in contemporary French) and interrogative sentences.

\ea%100
    \label{ex:key:100}

          French (\citealt{Hansen2008}: 213–214)

 \ea \label{ex:key:100a} Montre-moi \textit{déjà} ce que tu sais faire!

\glt ‘\textit{Just} show me what you can do!’

\ex \label{ex:key:100b} A.   mhm moi j’ai bien aimé ce film-là

C.   mhm

A.   parce qu’il y a un cadre historique qui est très bien...

B.   rendu

A.   euh oui

C.   c’était quelle guerre \textit{déjà}? // la guerre de cent ans là

\glt

A.   ‘mhm I liked that movie’

C.   ‘mhm’

A.   ‘because the historical setting is very well...’

B.   ‘portrayed’

A.   ‘er yes’

C.   ‘what war was that, \textit{now}? // the one-hundred year war’
    \z
\z

In the case of imperatives \REF{ex:key:100a}, the adverb “signals that the action requested is seen by the speaker as the first in a potential series of related actions to be carried out by either the hearer or the speaker. As already noted above, it strongly implicates, moreover, that the action marked is a prerequisite to some other action. As such, \textit{déjà} in imperatives may be said to have a slightly boosting effect” \citep[215]{Hansen2008}. In interrogative sentences \REF{ex:key:100b}, on the contrary, \textit{déjà} marks the information requested as previously known but not retrievable at the moment of conversation: “Instead of having the status of brand-new information, which would imply an unequal distribution of knowledge among speaker and hearers, the requested information is transformed, in advance of its production, into a simple reminder, and the interactional equilibrium is thereby preserved” \citep[213--214]{Hansen2008}.

This discussion is summed up by \citet[192–197]{Squartini2014}, who also explores the functions of \textit{già}/\textit{déjà} across French, standard Italian and northwestern regional varieties of Italian. The different distribution of the adverb in these three varieties makes it possible to identify two separate subdomains, both in general terms belonging to pragmatics but referring to different functional areas: information state and illocutionary modification. The use of \textit{già}/\textit{déjà} in questions referring to given information (backchecking) represents the first subdomain, while the use of \textit{déjà} in contexts like those of examples \REF{ex:key:100a} and \REF{ex:key:100b} represents the second subdomain. In the transition from one to the other the connection to the propositional content of the utterance progressively vanishes, and the adverb comes to operate as a pure illocutionary modifier of the speech act in itself.

\begin{quote}
Being connected to information state, backchecking can be conceived as still linked to the propositional content of the utterance (the degree of novelty of the propositional content), and appears therefore reasonable as an intermediate stage between content-level uses and fully-fledged context-level uses \citep{Hansen2008}, the latter being totally anchored to the illocutionary domain of the speech act and more extensively compatible with questions as illocutionary types in general. In this respect, evolving from backchecking to interrogative implies that the connection with information state tends to be loosened as pragmaticalization proceeds, and \textit{déjà} becomes increasingly connected to the speech act in itself, instead of exclusively marking the degree of novelty of the requested information. (\citealt{Squartini2014}: 207–208)
\end{quote}

With reference to the terminology used in Functional Discourse Grammar, information state/backchecking can be interpreted as a transition area, representing a bridge between the representational level (TAM markers) and the interpersonal level (illocutionary modification).\footnote{This closely recalls the issue about the use of \textit{un po’} as a marker of approximation in assertive speech acts, as discussed in \chapref{sec:7}. Thus, functions coded at the layer of communicated content such as approximation and backchecking can be kept separated from illocutionary modification, for the former refers to information managing, while the latter directly interfaces to the pragmatics of speech acts.}  Nonetheless, it must be noticed that in other languages phasal adverbs equivalent to \textit{già}/\textit{déjà} can develop illocutionary function without covering the backchecking functional domain. This is the case of US English \textit{already} \REF{ex:key:101} and Spanish \textit{ya} ‘already’ \REF{ex:key:102} in imperative sentences.

\ea%101
    \label{ex:key:101}

          English (\citealt{HansenStrudsholm2008}: 497)

\glt ‘Open your eyes, \textit{already}!’
    \z % you might need an extra \z if this is the last of several subexamples

\ea%102
    \label{ex:key:102}

          Spanish (\citealt{HansenStrudsholm2008}: 498)

  ¡Cállate \textit{ya}!

\glt ‘Shut up, now!’
    \z % you might need an extra \z if this is the last of several subexamples

In standard Italian, neither the use as a downtoner in interrogatives nor as a booster in imperatives is attested for \textit{già}. However – considering regional varieties – an illocutive use of \textit{già} is found in Sardinian Regional Italian, where it can be used as an assertive modal operator.

\ea%103
    \label{ex:key:103}

         Sardinian Regional Italian \citep[120]{Calaresu2015}

A.   Sto ancora aspettando il libro

\glt ‘I’m still waiting for the book’

B:   \textit{Già} te lo porto io domani

\glt ‘\textsc{ptc} I’ll bring it to you tomorrow’
    \z % you might need an extra \z if this is the last of several subexamples

As discussed by \citet{Calaresu2015}, in assertive speech acts (always in preverbal position) \textit{già} can be used by speakers to highlight their commitment to the utterance, that is, to endorse their own assertion.\footnote{In this sense, \citet{Calaresu2015} suggests that this use of \textit{già} underlies performativity and that a possible paraphrase of the use of \textit{già} in example \REF{ex:key:103} would be: “I assure you already here and now that I’ll bring you the book tomorrow”. In this case, the phasal value of \textit{già} is transferred from the semantic level of the proposition to the pragmatic/performative level of the speech act.}

Moving to other elements, the situation is similar. I mentioned in the last subsection a backchecking-like use of the temporal adverb \textit{poi} ‘then’ in direct questions and assertions. In other contexts, the same adverb seems to have developed a behavior more directly related to the modification of the illocutionary force of the speech act.

\ea%104
    \label{ex:key:104}

          Italian (questionnaire data)

[Elena and her sister Lucia must go back to work after lunch, but Lucia seems rather willing to take a nap on the sofa. Elena says]:

   Dai Luci, stai \textit{poi} su che dobbiamo fare i lavori in giardino!

\glt ‘Come on Lucia, get \textsc{ptc} up, we have to work in the garden!’
    \z % you might need an extra \z if this is the last of several subexamples

\ea%105
    \label{ex:key:105}

           Italian (questionnaire data)

[Simone to Vittorio, who always chooses original dishes when they go out for dinner]

Vitto, certo che la pizza all’ananas fa \textit{poi} schifo!

\glt ‘Vitto, you know, pineapple pizza is \textsc{ptc} disgusting!’
    \z % you might need an extra \z if this is the last of several subexamples

In both cases – a directive speech act in \REF{ex:key:104} and an assertive speech act in \REF{ex:key:105} – \textit{poi} has a boosting effect on the illocutionary force. Moreover, while the backchecking-like uses of \textit{poi} in examples \REF{ex:key:99a} and \REF{ex:key:99b} above are featured in standard Italian, the uses mentioned exemplified by \REF{ex:key:104} and \REF{ex:key:105} have probably a restricted distribution – possibly corresponding to areas in Romagna (a region in the northeast of Italy).

Since this chapter focuses on sociolinguistics issues, a fine-grained analysis of the illocutive functions of these elements (phasal or temporal adverbs) is out of its scope (see \citealt{Hansen2008} on phasal adverbs in French). Nevertheless, it is worth highlighting that they show the great variety of development paths that lead to the functional domain of illocutionary modification and – at the same time – the great variety of outcomes which can derive from the same lexical source. They should be added to the inventory of modal-particle-like elements in Italian and Romance varieties, and they show once more “how peripheral and low prestige diatopic varieties may synchronically exhibit a range of not attested, or only fragmentarily attested uses, in the textual history of more standard varieties, thus helping to reconstruct plausible paths of grammaticalization possibly valid as well for other adverbs and textual varieties” \citep[113]{Calaresu2015}.

\subsection{The questionnaire}
\hypertarget{Toc124860684}{}
The second questionnaire I designed for this research is aimed at collecting sociolinguistic data on Italian modal particles. Although there have been several references in the literature, no empirical data concerning their distribution is available. The phasal adverb \textit{già} discussed in the preceding subsections is a good case in point. It has been widely studied (\citealt{BazzanellaEtAl2005}; \citealt{Squartini2013,Squartini2014}), also from a sociolinguistic perspective which considers dialectal data and regional varieties (\citealt{FedrianiMiola2014}; \citealt{Calaresu2015}), but no large-scale data are available that can confirm (or reject) the hypothesis of diatopic markedness and – if the hypothesis is the confirmed – associate this markedness with a specific geographic area. The questionnaire is intended to be a first step in this direction. Moreover, it is intended to be a sociolinguistic counterpart of the corpus analysis of \chapref{sec:6} and \chapref{sec:7}: the elements that have been described there (\textit{anche}, \textit{pure}, \textit{un po’}) also appear in the stimuli of the second questionnaire.

I designed the questionnaire with three goals in mind. First, I wanted to collect data on the acceptability/reported language use\footnote{For more on these two terms, see the discussion in \chapref{sec:8}.} of modal uses of Italian adverbs. Second, I wanted to collect suggestions about other possible modal-particle-like elements not yet described in the literature. Third, I wanted to understand if some of these elements can be considered sociolinguistic variants – that is, if they express the same pragmatic function in different language varieties. The general design of the questionnaire is inspired by the previous one about the modal uses of \textit{solo}. No direct comparison between the two questionnaires was planned, but a similar design has the advantage of giving more coherence to the analysis. In order to do this, the acceptability judgments use the same scale as the first questionnaire (with some minor differences).

The second questionnaire consists of 16 stimuli and it is divided in two parts: the first one has 14 stimuli, the second one only two. In the first part, the respondents are invited to comment on the use of specific constructions through two different questions (“Have you ever heard such a sentence?” and “Do you use such a sentence?”). Moreover, the respondents are invited to suggest possible alternatives (“Is there another word you would use instead of x in the same context?”) or leave an open comment (“Do you have any other comment on this sentence?”).

In the second part – given a specific context – the respondents are invited to choose the sentence they would use out of different possibilities, or to suggest another one. The general structure of the questionnaire is summed up in \tabref{tab:key:9.1}.\footnote{See \chapref{sec:5} for general information about the questionnaire design. Here, the structure of the questionnaire and the stimuli are translated into English: the original version in Italian can be accessed online at \url{https://zenodo.org/records/10362289}.}

\begin{table}
\begin{tabularx}{\textwidth}{XXQ}
\lsptoprule
 & Question & Answer\\
\midrule
 Reported
language use & Have you ever heard such a sentence? & {}--- yes, sometimes
\newline {}--- no\\
 & Do you use such a sentence? & {}--- yes, often
\newline {}--- yes, sometimes
\newline {}--- no\\
\tablevspace
Suggestions & Is there another word you would use instead of x in the same context? & open response\\
& Do you have any other comment on this sentence? & open response\\
\tablevspace
Pragmatic variants & [given a specific context]
\newline Which sentence would you use in this context? & Multiple answer options + open response\\
\lspbottomrule
\end{tabularx}
\caption{\label{tab:key:9.1} Structure of the questionnaire on modal particles in Italian}
\end{table}

\section{Questionnaire data}
\hypertarget{Toc124860685}{}
Having mentioned a few more Italian elements showing illocutionary modification functions, I will now look into the data collected through the questionnaire. In a similar way to the evaluation of the first questionnaire, I will focus on the results concerning the reported language use and the geographical variation they display.

Even though they are based on a limited sample, the results are interesting, and they can be converted into data charts on Italian modal-particle-like elements. In the presentation of the findings, I consider a general picture – answers have been collected from almost every Italian region – which is however unbalanced towards northern Italy, since most answers come from regions in the north (Piedmont, Lombardy, Veneto, Emilia-Romagna).

\newpage
I will start discussing the first part of the questionnaire (stimuli 1–14) – to compare the usage rate of different illocutionary constructions, represented by specific uses of the adverbs \textit{anche}, \textit{pure}, \textit{un po’} and \textit{poi}. Afterwards, I will discuss stimulus 15 and stimulus 16, respectively dedicated to backchecking markers and to emphasis markers.

\subsection{Overall reported language use}\largerpage[-2]
\hypertarget{Toc124860686}{}
The results concerning the overall reported language use of a stimulus are represented by the answers to the question “Do you use such a sentence?”. In this sense, these results depict the active usage of the constructions by the respondents. The possible answers refer to the same scale used in the first questionnaire: \textit{sì, abitualmente} ‘yes, often’, \textit{sì, qualche volta} ‘yes, sometimes’ and \textit{no} ‘no’. They have been converted to numeric values (1.0 counts as “no”, 2.0 as “sometimes” and 3.0 as “often”) for the purpose of data visualization. The collected answers are graphically represented by boxplots, obtained through the Lancaster Stats Tool Online \citep{Brezina2018}. While elaborating the boxplots, I grouped the answers into four different sets based on the featured marker.\footnote{In the presentation of the results – in this and in the next cases – the relevant utterances are shown beside the boxplots. For the whole stimuli, see the original version of the questionnaire (online at \url{https://zenodo.org/records/10362289}).}

The first group includes the stimuli featuring the modal uses of \textit{anche} ‘also’. These uses of \textit{anche} have been thoroughly discussed in \chapref{sec:6}: with reference to the labels used in \figref{fig:key:9.1}, “anche\_1” and “anche\_3” are directive speech acts with imperative verb forms, while “anche\_2” is an assertive speech act with an indicative verb form. Broadly speaking, in these constructions \textit{anche} might be said to represent a mitigation marker. Looking at the results, it can be noticed that the illocutive uses of \textit{anche} all attain a median value of 2.0 (bold black line), which corresponds to “sometimes”. For each context, some respondents answered “no” (especially in the third context: 49 out of 180 respondents, namely almost one third), but it can doubtless be concluded that these modal uses of \textit{anche} are commonly used.\footnote{Concerning the third example (\textit{vedi anche tu}), many respondents further commented that they would rather use the same utterance without \textit{anche}. Possible alternatives to it (also suggested in the comment section) are \textit{pure}, \textit{un po}’ and \textit{poi} – namely all other elements investigated by the first part of the questionnaire. I will further discuss this fact – which is interesting for an overall evaluation of Italian modal-particle-like elements – in the conclusion.}


\begin{figure}[t]
\includegraphics[height=.45\textheight]{figures/FavaroLSIfinal-img014.png}

{\raggedright\small
anche\_1 = \textit{fai anche le 6} ‘you can \textsc{ptc} be there at 6 p.m.’

anche\_2 = \textit{sono anche le 3} ‘it is \textsc{ptc} 3 a.m.’

anche\_3 = \textit{vedi anche tu} ‘you can \textsc{ptc} think about that’\par}

\caption{\label{fig:key:9.1} Reported language use: modal uses of \textit{anche} }
\end{figure}

\begin{figure}[t]
\includegraphics[height=.45\textheight]{figures/FavaroLSIfinal-img015.png}

{\raggedright\small
pure\_1 = \textit{prendi pure} ‘please take it’

pure\_2 = \textit{stai pur certo} ‘just be sure’

pure\_3 = \textit{deve pure esserci} ‘there must \textsc{ptc} be’

pure\_4 = \textit{sarà pure bravo} ‘he may \textsc{ptc} be good’\par}

\caption{\label{fig:key:9.2} Reported language use: modal uses of \textit{pure} }
\end{figure}

The second group includes the illocutive uses of \textit{pure} ‘also’, discussed in \chapref{sec:6} as well. With reference to the labels used in \figref{fig:key:9.2}: “pure\_1” represents a directive speech act where \textit{pure} has a mitigating function, “pure\_2” is a directive speech act where \textit{pure} marks emphasis\footnote{These specific uses of \textit{pure} – where the adverb gives the directive the character of a warning or an intimidation \textit{–} have not been discussed in the \chapref{sec:6}, since they sound outdated (or very literate) in contemporary Italian. They can be quite surely traced back to the use of \textit{pur(e)} as an exclusive focus adverb in Old Italian (see \citealt{Ricca2017}; \citealt{Favaro2021}: 117–129). In these contexts, the truncated form \textit{pur} is the only acceptable variant. In some respects, the data concerning \textit{solo} discussed in \chapref{sec:8} represent a similar development path.}, “pure\_3” is an assertive speech act with an epistemic use of \textit{dovere} ‘must’ and “pure\_4” features a concessive future. The presence of all these uses in the corpus data – although in low numbers – suggested that they are a stable presence in contemporary Italian: the results of the questionnaire confirm this observation. In fact, the four stimuli attain high values. In particular, the median value of “pure\_1” and “pure\_3” is 3.0, which corresponds to “often”. Looking at these results, it can be concluded that the illocutive uses of \textit{pure} represent the most common instances of modal particles in Italian. In the case of “pure\_1” – that is, \textit{pure} in a directive speech act as a mitigation device – 139 out of 180 respondents answered “often” to the question about the active usage of such an utterance.\footnote{On the contrary, the use of \textit{pure} in directive speech acts as an emphatic marker (“pure\_2”) attains lower values: as has already been pointed out, it sounds antiquated and it is mainly found in fixed expressions, precisely like \textit{stai (pur) certo} ‘be sure’.}




The results concerning \textit{un po’} ‘a bit’ (\figref{fig:key:9.3}) are different. The stimuli proposed in the questionnaire include both its use in directives (labeled “unpo\_1”) and assertions (labeled “unpo\_2”). The utterance represented by “unpo\_3” is a case of a directive in a partially fixed expression (\textit{vedi di calmarti} ‘calm yourself’, literally ‘look at calming yourself’). This last example attains the highest value (mean value is between 2.0 and 2.5), while the other two attain lower values (mean values are between 1.5 and 2.0). However, such uses are not that rare in corpus data and these results slightly contrast with the corpus findings. It could be the case that – unlike the illocutive uses of \textit{pure} – these uses are less conventionalized or perceived as such by the respondents. Indeed, looking at the answers to the open question “Do you have any other comment on this sentence?” several respondents – who however admit using similar utterances – point out that this use \textit{un po’} is \textit{improprio} ‘inappropriate’, \textit{non corretto} ‘not correct’ or \textit{grossolano} ‘gross’. Other respondents answered they have just realized the existence of such a use. No similar comments are found for the illocutive uses of \textit{pure}.\footnote{Going through the answers to the question “Is there another word you would use instead of \textsc{x} in the same context?”, many respondents \REF{ex:key:36} answer \textit{un attimo} ‘a moment’– precisely as has been pointed out in \chapref{sec:7}.}


\begin{figure}[t]
\includegraphics[height=.45\textheight]{figures/FavaroLSIfinal-img016.png}

{\raggedright\small
unpo\_1 = \textit{provala un po’} ‘give it a try \textsc{ptc}’

unpo\_2 = \textit{è un po’ quello il fatto} ‘that is \textsc{ptc} the fact’

unpo\_3 = \textit{vedi un po’ di calmarti} ‘calm \textsc{ptc} yourself’\par}

\caption{\label{fig:key:9.3} Reported language use: modal uses of \textit{un po’}}
\end{figure}

The last group includes illocutive uses of \textit{poi} ‘then’. This adverb has not been described in detail in this study, but a short outline of its context-level uses has been given in the preceding section. I mentioned two uses related to the pragmatic domain of backchecking, in interrogatives and assertions: they are represented here by the utterances labeled as “poi\_1” and “poi\_4”, comparable to examples (99a–b) above. Moreover, I tentatively included two uses – most probably geographically marked – related to the emphatic marking of speech acts, in assertives (“poi\_2”) and directives (“poi\_3”), corresponding to examples \REF{ex:key:104} and \REF{ex:key:105} above. The results are shown in \figref{fig:key:9.4}. A clear difference emerges between the modal uses of \textit{poi} related to backchecking and those related to emphasis on the illocutionary force. While the first pair attains a median value of 3.0, the second pair attains a median value of 1.0: although some respondents answer “sometimes” or even “often” (7 respondents for “poi\_2” and 5 respondents for “poi\_3”), most of them don’t recognize this use. These results seem to confirm what has been suggested in the last section: the backchecking uses of \textit{poi} are features of standard Italian, while the emphatic uses are probably found only in regional varieties.


\begin{figure}[t]
\includegraphics[height=.45\textheight]{figures/FavaroLSIfinal-img017.png}


{\raggedright\small
poi\_1 = \textit{sei poi andata?} ‘did you go \textsc{ptc}?’

poi\_2 = \textit{fa poi schifo} ‘it is \textsc{ptc} disgusting’

poi\_3 = \textit{stai poi su} ‘get \textsc{ptc} up’

poi\_4 = \textit{non sono poi così lontane} ‘they are not \textsc{ptc} so far’\par}

\caption{\label{fig:key:9.4} Reported language use: modal uses of \textit{poi}}
\end{figure}

\largerpage[2]
In fact, looking at the answers to the question “Do you have any other comment on this sentence?”, a few respondents answered – both with regard to “poi\_2” and “poi\_3” – that these uses are typically found in varieties spoken in Emilia-Romagna.{\interfootnotelinepenalty=10000\footnote{This fact is in accordance with the starting assumption. In fact, while developing the stimuli and designing the questionnaire, both “poi\_2” and “poi\_3” have come to my attention thanks to people from Faenza, a city in Romagna, situated southwest from Ravenna and southeast from Bologna.}} However, the data from Emilia Romagna and from Piedmont are not so dissimilar, even though they are rated slightly higher in Piedmont.\footnote{Coincidentally, both regions obtained 36 answers thus allowing a comparison based on an identical sample.} Only “poi\_3” shows some more evidence of regional markedness: Emilia-Romagna is represented by three respondents answering “often” and seven respondents answering “sometimes”, while for Piedmont no respondent answered “often” and three respondents answered “sometimes”. Overall, Emilia-Romagna has thus ten respondents who assert to actively using this construction, Piedmont only three. To conclude, some evidence of the regional markedness of “poi\_2” and “poi\_3” has been found, but more research is needed to confirm the results.

\subsection{Backchecking markers in interrogatives}
\hypertarget{Toc124860687}{}
Stimulus 15 of the questionnaire deals with backchecking strategies, which have been introduced in the preceding section. It adopts an onomasiological perspective (function-to-form): having identified backchecking as a pragmatic functional domain, the different formal strategies which can code it are investigated. In practical terms – that is, in the context of a questionnaire – this means presenting a conversational context in which backchecking strategies can be used and asking the respondents which specific strategy they would choose. The context provided in the questionnaire is the most typical backchecking context referred in the literature, namely a tip-of-the-tongue situation where someone has forgotten the name of a person and asks the interlocutor to provide this information once more.

\ea%106
    \label{ex:key:106-1}

          Italian (questionnaire data)

  [Anna does not remember the name of Irene’s cousin]

Ire, com’\textit{è che} si chiamava tua cugina?   [cleft sentence]

 Ire, come si chiamava \textit{già} tua cugina?    [\textit{già} ‘already’]

 Ire, come si chiamava \textit{più} tua cugina?   [\textit{più} ‘more’]

 Ire, come si chiamava \textit{pure} tua cugina?   [\textit{pure} ‘also’]

\glt ‘Ire, what was your cousin’s name again?’
    \z % you might need an extra \z if this is the last of several subexamples

The respondents could choose one or more of the answers proposed, or also suggest other possibilities. The first option is a cleft sentence, while the other three options display different backchecking particles (see the previous section for a brief description of their characteristics).

As a working hypothesis, the cleft sentence was suggested as a feature found in the standard variety of Italian, while the backchecking particles as features of different regional varieties. This has been substantially confirmed by the collected data. Moreover, many respondents answered that both options are perfectly acceptable. Correspondingly, it must be concluded that cleft structures and specific backchecking particles are both strategies available to speakers, who can contextually choose whether to use one or the other. Nevertheless, they never (or very rarely) appear simultaneously in the same sentence.

The selection of different strategies by respondents results in seven possible groups of answers. Two more are represented by answers of respondents who find acceptable more than one particle (labeled as “cleft+mix”) and by answers of respondents who don’t find any of the proposed options (labeled as “other”) acceptable. In this last category, most of the instances are represented by answers rejecting the use of the backchecking imperfect (see \citealt{Waltereit2001}: 1405–1407 on this), featured in all the options provided in the questionnaire. As an alternative, many respondents provide an utterance with a cleft syntactic structure and a present tense (\textit{Com’è che si chiama tua cugina?}). In \figref{fig:key:9.5}, the answers of respondents from four regions of northern Italy are graphically displayed: Piedmont (36 answers), Lombardy (47 answers), Veneto (18 answers) and Emilia-Romagna (36 answers).\footnote{As has been said above, the analysis of the second questionnaire mostly focuses on data from northern Italy: with the partial exceptions represented by Apulia (10 answers) and Sicily (14 answers), very few data have been collected from regions in central-southern Italy.}


\begin{figure}
\begin{tikzpicture}
	\pie
	[color = {red!80,
			black!50,
			yellow!80},
	rotate = 200,
	sum = auto,
	radius = 1.5,
	pos = {0,5}
	]{18/,                  %cleft+\textit{già},
		7/,                 %cleft,
		11/}               %\textit{già}}
	\node at (0,3) {Piedmont};

	\pie
	[color = {red!80,
			black!50,
			gray!20},
	rotate=160,
	sum = auto,
	radius = 1.5,
	pos = {5,5}
        ]{2/,                %cleft+\textit{già},
         36/,                %cleft,
		  9/}                %other
		\node at (5,3) {\large Lombardy};

	\pie
	[color={gray!20,
			black!50},
	rotate=90,
	sum = auto,
	radius = 1.5,
	pos = {0,0}
	]{3/,                   %other
		15/}                %cleft
		\node at (0,-2) {\large Veneto};

	\pie
	[color = {gray!20,
			violet!70,
			blue!50,
			cyan!80,
			green!80,
			yellow!80,
			red!80,
			black!50},
	rotate=90,
	sum = auto,
	radius = 1.5,
	pos = {5,0},
	text = legend
	]{2/other,
	1/cleft+mix,
		1/pure,
		3/cleft+pure,
		1/cleft+\textit{più},
		2/\textit{già},
		8/cleft+\textit{già},
		18/cleft
	}
        \node at (5,-2) {\large Emilia-Romagna};
\end{tikzpicture}
\caption{\label{fig:key:9.5} Backchecking strategies in four regions of northern Italy}
\end{figure}

Piedmont offers an interesting and consistent picture: half of the respondents select both the cleft structure and the particle \textit{già} as acceptable backchecking strategies, the other half is divided between respondents choosing either the cleft structure or the particle \textit{già} (almost a third). Overall, the great majority of respondents use the particle \textit{già} as a backchecking strategy, either in alternation with the cleft structure or as a main strategy. These data confirm therefore that \textit{già} is a typical (but not exclusive) feature of Piedmontese Regional Italian.\footnote{Half of the respondents from Piedmont use both a standard feature (the cleft structure) and a regional feature (the particle \textit{già}). This can be interpreted in the light of the formation of regional standard varieties (“dialectization of Italian” in \chapref{sec:5}): both nationwide and region-specific traits are featured in regional (standard) varieties.}

Also  taking into account the different number of respondents, Lombardy and Veneto show a similar situation. The cleft structure is by far the most common backchecking strategy, while some respondents propose alternative answers (labeled “other”): cleft structures without backchecking imperfect, the imperfect without cleft structure, and even the plain question with the present tense and no cleft structure (therefore with no overt backchecking marking, relying only on contextual interpretation). At the same time, two respondents from Lombardy select both the cleft structure and the particle \textit{già} as acceptable backchecking strategies, showing that this particle is used not only in Piedmont.

Lastly, Emilia-Romagna offers the richest and most complex situation: almost each of the nine possible group of answers is featured in its graph.\footnote{The only possibility which does not appear in the answers from Emilia-Romagna is the particle \textit{più} ‘more’ (not in alternation with the cleft structure). According to \citet{FedrianiMiola2014} this feature is found in the regional variety spoken in Liguria. In fact – although the questionnaire includes only one respondent from Liguria – their answer selects the particle \textit{più} alone as a backchecking strategy.} Half of the respondents choose the cleft structure, while another third selects both the cleft structure and other particles as acceptable backchecking strategies (indeed, all the particles suggested by the questionnaire are featured in the answers). Few respondents choose the options with particles only (\textit{già} or \textit{pure}) or suggest different options (no cleft structure, no backchecking imperfect: the label for this group of answers is “other”). Two respondents affirm to use both \textit{pure} and \textit{già} to express backchecking and also add \textit{poi} to the list (the cleft structure is included among the possibilities: the label for this group is “cleft+mix”). Thus, the emerging picture results as being very rich and varied.\footnote{Emilia-Romagna has a different dialectal history compared to Piedmont and Veneto. In contrast to these latter regions, where cities like Turin and Venezia represented unifying centers for the dialect use \citep{Regis2011}, no dialectal koiné developed in Emilia-Romagna – possibly leaving space for the coexistence of more variants.} In particular, the particle \textit{pure} (alone or in alternation with the cleft structure) – which, according to \citet{FedrianiMiola2014}, is typical of this regional variety when used as a backchecking form – has been selected by four respondents.

Overall, it is possible to distinguish three types. In the case of Lombardy and Veneto – apart from a few exceptions – the pan-Italian strategy (cleft structure) is the only option to mark backchecking. In the case of Piedmont, alongside a pan-Italian strategy (cleft structure), a regionally marked strategy is found (the particle \textit{già}), which is recognized by almost all respondents: it represents therefore a regional standard feature. In the case of Emilia-Romagna instead, a pan-Italian strategy is attested (cleft-structure) alongside other strategies linked to a strong inter-individual/inter-group variation. From this pool of different strategies – quite interestingly – ten respondents from Emilia-Romagna select \textit{già}, either as an alternative to a cleft structure or by itself. Along with the two answers from Lombardy, this fact confirms that \textit{già} as a backchecking marker actually shows a supra-regional distribution.\footnote{Recalling the discussion about demotization in \chapref{sec:5}, these data allow us to consider the backchecking \textit{già} as involved in the process by which regional features (among them regional standard features) are de-localized and spread across different regions. If included into a larger core of nationwide shared features, backchecking \textit{già} could be considered a candidate feature for neo-standard Italian (which shows different regional standard features in different geographical areas).}

\subsection{Markers of emphasis in imperatives}
\hypertarget{Toc124860688}{}
In the preceding section, I briefly described the use of \textit{già} as a modal operator on assertions (this use is found in the regional variety of Italian spoken in Sardinia) and two uses of \textit{poi} in assertions and directives (possibly limited to the regional variety of Italian spoken in Emilia-Romagna). Here, I focus on directives, considering elements that express boosting of the illocutionary force: Stimulus 16 of the questionnaire deals with elements marking \textit{emphasis} on directive speech acts.

A specification is needed here: as pointed out by \citet[1026]{Schwenter2003}, categories such as \textit{emphasis} and \textit{mitigation} are somehow “intuitive and pre-theoreti\-cal labels” and a more fine-grained pragmatic analysis should rather avoid them in the description of the function of pragmatic markers, using instead more precise categories. Although I am aware of this, I decided to use it in this section – for two main reasons. First, a fine-grained analysis of pragmatic functions is not the goal of this section, which rather focuses on the usage variation of certain markers. Second, to make possible the comparison, I needed a category broad enough to include different markers which have their own specificities but at the same time also show commonalities (the data of the questionnaire are a strong proof in this sense). In this sense, the intuitive character of the label \textit{emphasis} represents a common thread of the functions of several markers and – even more importantly~– its pre-theoretical character makes it easily recognizable to the respondents.\footnote{Moreover, I am not aware of any better label used in the literature. Functional Discourse Grammar also uses labels such as \textit{emphasis} (or \textit{reinforcement}) and \textit{mitigation} to indicate illocutionary force modification at the speech act level (see \citealt{Hengeveld2004}: 1192; \citealt{HengeveldMackenzie2008}: 83).}

Exactly like the preceding questionnaire example, this stimulus also adopts an onomasiological perspective (function-to-form): having identified emphasis as a pragmatic functional domain, the different formal strategies that code it are investigated. In the questionnaire, a conversational context was given in which an emphatic particle can be used and the respondents were asked which specific strategy they would choose. The context provided in the questionnaire is a typical one where several different elements can appear to mark emphasis, namely a directive expressed by a conventionalized multi-word expression (\textit{stai zitto} ‘shut up’).

\ea%106
    \label{ex:key:106-2}

          Italian (questionnaire data)

[Giacomo, sick of Mario during a discussion unnecessarily proceeding for a whole hour]

Senti, stai \textit{solo} zitto, che hai torto marcio! [\textit{solo} ‘only’]

  Senti, stai \textit{un po’} zitto, che hai torto marcio! [\textit{un po’} ‘a bit’]

Senti, stai \textit{mo’} zitto, che hai torto marcio! [\textit{mo’} ‘now’]

  Senti, stai \textit{poi} zitto, che hai torto marcio!   [\textit{poi} ‘then’]

\glt ‘Look, shut \textsc{ptc} up, you’re dead wrong!’
    \z % you might need an extra \z if this is the last of several subexamples

As in the preceding case, the respondents could choose one or more of the answers proposed, or also suggest other possibilities. The first two particles have been described in previous chapters of this study. As shown in \chapref{sec:8}, the use of \textit{solo} ‘only’ in directives is found acceptable by speakers across Italy, but it is used more in the regional variety spoken in Piedmont. The use of \textit{un po’} ‘a bit’ in directives has been described in \chapref{sec:7} as a feature found in in the standard variety of Italian: the questionnaire data seem to validate this assumption. The particle \textit{mo’} ‘now’ shows several uses in spoken varieties and/or dialects, but no specific research is available about it.\footnote{Most probably, its etymology goes back to Latin \textit{modo} ‘only, just, now’, which was already used as a modal particle in Latin (see \citealt{Kroon2011}: 177).} The particle \textit{poi} ‘then’ has been introduced in the previous section and it appeared in four stimuli in the first part of the questionnaire: its use in directives is likely to be traced back to the regional variety spoken in Emilia-Romagna.

Going through the answers, it became clear that the use of \textit{un po’} in directives is actually a supra-regionally non-marked option: this option was selected by respondents from every region. I outlined nine possible groups of answers. The first four are represented by answers selecting a single particle. Three more groups are represented by answers selecting both \textit{un po’} and one of the other particles as emphatic markers in directives. Two more are represented by answers of respondents who find three or more particles acceptable (this group is labeled as “mix”) and by answers of respondents who don’t find acceptable any of the proposed options (this group is labeled as “other”). It should be also highlighted that none of these particles can be used in combination with others. In \figref{fig:key:9.6}, the answers of respondents from four regions of northern Italy are displayed.

\begin{figure}
\begin{tikzpicture}
	\pie
	[color = {red!80,
			black!50,
			gray!20,
			blue!50,
			yellow!80},
	rotate = 200,
	radius=1.5,
	pos={0,5},
	sum = auto
	]{ 22/,                                    %\textit{un po'/solo},
		5/,                                    %\textit{un po'},
		1/,                                    %other,
		1/,                                    %\textit{mo'},
		7/}                                    %\textit{solo}}
        \node at (0,3) {\large Piedmont};

	\pie
	[color = {red!80,
			black!50,
			gray!20,
			purple!80,
			cyan!80,
			yellow!80},
	rotate=160,
	pos={5,5},
	sum = auto,
	radius=1.5
        ]{5/,                        %\textit{un po'/solo},
         24/,                        %\textit{un po'},
          8/,                        %other,
          1/,                        %mix,
          1/,                        %\textit{un po/mo'},
          6/}                        %\textit{solo}}
		\node at (5,3) {\large Lombardy};

	\pie
	[color={gray!20,
			red!80,
			black!50},
	rotate = 90,
	sum = auto,
	radius=1.5,
	pos = {0,0}
      ]{5/,                  %other
		4/,                  %\textit{un po'/solo},
		9/}                  %\textit{un po'}}
		\node at (0,-2) {\large Veneto};

	\pie
	[color = {gray!20,
			purple!80,
			blue!50,
			cyan!80,
			teal!80,
			green!80,
			yellow!80,
			red!80,
			black!50},
	rotate=90,
	pos={5,0},
	sum = auto,
	radius=1.5,
	text = legend
      ]{4/other,
        4/mix,
		3/\textit{mo'},
		4/\textit{un po'/mo'},
		1/\textit{poi},
		2/\textit{un po'/poi},
		2/\textit{solo},
		4/\textit{un po'/solo},
		12/\textit{un po'}}
        \node at (5,-2) {\large Emilia-Romagna};
\end{tikzpicture}
\caption{\label{fig:key:9.6} Emphatic particles in directives in four regions of northern Italy}
\end{figure}

Let’s start from the results concerning Veneto (18 answers), which displays the simplest picture. Half of the respondents select \textit{un po’} as preferred emphatic marker, while the other half is divided between respondents who select both \textit{un po’} and \textit{solo}, and respondents who don’t choose any of the options suggested. In this last case, respondents propose a simple directive without particles as an alternative (labeled as “other”).\footnote{Some respondents suggested other strategies as well: prosody, additional discourse markers (\textit{vai}, \textit{va’} ‘go’), different verbal phrases (\textit{statti zitto}, \textit{vedi di starti zitto} ‘shut up’) or other particles (\textit{pur} ‘also’).}

Lombardy (47 answers) shows a similar but richer picture. More than half of the respondents select \textit{un po’} as their preferred emphatic marker, corroborating the idea that this adverb is the unmarked option. The absence of an overt marking of emphasis is also a common option. Moreover, \textit{solo} is selected as an option by eleven respondents, either as an alternative to \textit{un po’} (five answers) or by itself (six answers). Two respondents also select other particles, namely \textit{mo’} and \textit{poi}.

Piedmont (36 answers) shows instead a quite different picture. In fact, the most selected option is not \textit{un po’} (five answers), but both \textit{un po’} and \textit{solo} as equivalent emphatic markers (22 answers). Consistent with the findings of the other questionnaire, the option of selecting only \textit{solo} is also common (seven answers), more than in other regions. Few answers select \textit{mo’} or no particle. Thus, the Piedmontese graph of emphatic markers closely recalls the graph concerning the backchecking strategy, where \textit{un po’} plays the role of the cleft structure and \textit{solo} plays the role of \textit{già}, and most answers select both strategies.

Similar to the previous question, Emilia-Romagna (36 answers) offers the richest and most complex situation: in this case, each one of the nine possible emphatic strategies are featured in the answers from this region. The option \textit{un po’} was selected by one third of the respondents, while four respondents selected the option without particles. Both \textit{mo’} and \textit{poi} were selected by respondents (either alone or as an alternative to \textit{un po’}), and this seems to confirm their regional markedness since – apart from isolated cases in Piedmont and Lombardy – they are chosen with some frequency only in the answers from this region. Nevertheless, \textit{solo} is also selected by some respondents, as a further confirmation of its supra-regional distribution. Lastly, four respondents indicate three or more particles – namely all the options suggested in the questionnaire, in different combinations (labeled “mix” in the graph). These results seem to confirm for Emilia-Romagna what has already emerged in the previous question, namely, the coexistence of different features in a varied linguistic space.

Although they cannot lead to conclusive statements, these results have provided empirical material to different assumptions made in the preceding sections. Speakers of Italian have access to a nationwide-spread emphatic marker, namely \textit{un po’}, which can be used to boost the illocutionary force of directives.\footnote{Data from other regions, albeit limited, confirm the use of \textit{un po’} as an emphatic marker.} This is by no means a compulsory option – in many cases a specific prosodic contour is enough – but a structural possibility that respondents from different regions recognize and actively use. Other emphatic markers also exist, even though they are less common and regionally flavored. Among them, at least \textit{solo} seems to have achieved a supra-regional distribution. Other markers such as \textit{poi} and \textit{mo’} show a more limited distribution, but still contribute to prove the existence of illocutionary operators in different (northern) regional varieties of Italian. Specific contributions which investigate the semantic and pragmatic characteristics of these particles in-depth remain a call for future research. For now, however, the use of the category of \textit{emphasis} – although not completely satisfactory for the reasons listed above – has proved useful for a broad comparison of different markers operating on the illocutionary force of directives.

\section{Sociolinguistic issues: Closing remarks}
\hypertarget{Toc124860689}{}
This section closes the last chapter of the present research and offers a first round of final comments. Before moving to the general conclusions of the research, I want to review the sociolinguistic issues discussed in the last two chapters. In fact, although strictly linked to the broader picture, these issues deserve an individual discussion. I will sum up the main findings of the two questionnaires and I will discuss what these data show about the relationship between modal particles and (contemporary) sociolinguistic changes in Italian. Taken together, these issues show the many challenges and opportunities offered by a sociolinguistic-oriented study of pragmatic phenomena.

\subsection{Findings from the questionnaires}
\hypertarget{Toc124860690}{}
The second questionnaire had three main goals: collecting data about the usage of a set of adverbs with illocutionary functions, collecting data about their sociolinguistic status (with a focus on geographical variation), and testing the feasibility of adopting an onomasiological (function-to-form) approach to the variation of modal particles. In the design of the questionnaire, these three goals were not dealt with separately, but were rather addressed jointly. The stimuli of the first part integrate the corpus data discussed in \chapref{sec:6} and \chapref{sec:7}, investigating the usage rate of modal constructions with \textit{anche}, \textit{pure}, and \textit{un po’} previously described, also considering their regional variation. Moreover, data about \textit{poi –} which a few references (\citealt{Bazzanella1995}; \citealt{Coniglio2008}) have cited as a modal-particle-like element in Italian – were also collected.

The high values attained by the acceptability judgments and the relative uniformity across regions (excluding two uses of \textit{poi}) allow for the conclusion that these uses are a stable presence in the standard variety of Italian or show at least a pan-national distribution. The issue of geographical variation was addressed more specifically with the last two stimuli, which also test a function-to-form approach to variation of pragmatic markers.

Having identified backchecking interrogatives and emphatic imperatives as pragmatic domains subject to being expressed by diverse pragmatic markers, these two stimuli investigated their distribution across regions. The results, main\-ly based on the data from four regions in northern Italy, confirm that variation of pragmatic markers does exist, and it is reflected by the respondents’ answers. Regionally marked elements have been found for two regions: a backchecking \textit{già} and an emphatic \textit{solo} in Piedmont, a backchecking \textit{pure} and emphatic \textit{poi} and \textit{mo’} in Emilia-Romagna. Moreover, the results show a supra-regional diffusion of these uses since they also appear in other regions. At the same time, it became clear that speakers have a standard (not-regionally marked) variant at their disposal for both domains, namely the cleft construction for backchecking interrogatives and \textit{un po’} for emphatic directives.

From this perspective, the second questionnaire can be seen as encompassing the first one (or the first one can be seen as an extension of the second one). The first questionnaire focused on the modal uses of \textit{solo} in directive and assertive speech acts, aiming at collecting data about their distribution (reported language use, geographical variation) and their meaning. It therefore represents an in-depth case study on a single particle, which could be used as a model for future research on other elements. The data collected allowed a better understanding of the behavior of \textit{solo} in the two different illocutive contexts: modal uses of \textit{solo} in assertions and directives which differ both in terms of acceptability and geographical distribution.

\hspace*{-1.5pt}The results show that the use of \textit{solo} in assertive speech acts is more widespread and it can probably be assigned to the standard variety of Italian, while the use in directive speech acts – generally less widespread – results more acceptable for respondents from Piedmont. The meaning analysis showed that different semantic features of these constructions (emphasis on the illocutionary force and common ground management) are recognized by the respondents and can coexist in the same context. By crossing the acceptability data with the meaning data, hypotheses on the development paths of these constructions have been formulated. This led to the conclusion that the two features do not develop one after the other, but they rather represent different paths of semantic change, which can however intersect if the conversational context allows it. Thus, the conversational context appears to be the main factor in determining both the acceptability values and the specific function of these constructions.

At a general level, the results discussed in \chapref{sec:8} and \chapref{sec:9} demonstrate that questionnaires can be fruitfully used to investigate discourse-prag\-mat\-ic phenomena. In particular, this methodology proved decisive when collecting data about linguistic expressions and constructions which are difficult to trace or even absent in corpora. It moreover allows the possibility of linking them to a set of metadata, thus enabling a sociolinguistic study of pragmatic markers – selecting each time the dimension of variation to be explored. In the questionnaires both multiple-choice questions and open questions were used. Multiple-choice questions make it possible to collect significant amounts of data which can also be analyzed from a quantitative perspective. In addition, open questions make it possible to broaden the research assumptions by directly asking the respondents about specific issues, possibly including their personal and pre-theoretical categories in the analysis framework – and thus collecting further hints which would be otherwise difficult to get.

In the presentation of the data, I took advantage of various data visualization possibilities (box plots, bar plots, mosaic plots, and pie charts) which are very effective to render different kinds of data. To conclude, questionnaires have proved to be an adaptable and powerful tool to conduct research on pragmatic markers, also from a sociolinguistic perspective: once the research question is defined, specific designs can be developed in order to investigate a wide array of different aspects.

\subsection{Modal particles and sociolinguistic changes in Italian}
\hypertarget{Toc124860691}{}
Looking at the results of the two questionnaires, the question arises what they say about the on-going sociolinguistic changes affecting contemporary Italian, that is to say, whether the results reflect the sociolinguistic changes described in \chapref{sec:5} and how. Evidently, as previously stated, it is not possible to draw definitive conclusions from these results: although the sample taken into consideration is not small, more data – especially from central and southern Italy – are needed to further corroborate the findings. Nevertheless, I would like to highlight three interesting points.

The first point concerns the overall reported language use of the constructions described above. Given a certain variation in the acceptability degree, most of the uses have been widely recognized by the respondents who also affirm to actively use them and largely agree on their functions. This fact supports the idea that an average use of these particles does exist. This involves elements which have been present in Italian for a long time (\textit{pure}), elements which are present in the regional standards, most probably transferred from the base dialects (\textit{poi} and \textit{solo} in directives, backchecking \textit{già}), and elements which arguably represent more recent innovations (\textit{un po’}, \textit{solo} in assertions). In this respect, it should be mentioned that no grammar or schoolbook provides indications on these elements which – although also found in written texts – are basically features of spoken varieties and in particular dialogic speech. In this light, they are evidence of processes leading to the spontaneous fixation of a set of uses (“standard by usage”), some of them at a regional level, others at a pan-Italian level.

The second point concerns precisely the geographical markedness of specific elements. The results of the questionnaires show that some uses are specific to certain regions, while others are found across regions. In the latter case, uses are found which can be in all likelihood assigned to the national standard (\textit{pure}, \textit{un po’}), while in the former the situation could be more complex – and it’s not totally clear if all regionally marked elements have the same sociolinguistic status.

While some represent features of the regional standard variety (\textit{già} and \textit{solo} in Piedmont), others could also be features in regional sub-standard varieties (\textit{poi} and \textit{mo’} in directives in Emilia-Romagna): more research is needed on this point. Anyway, the most interesting finding is that the results show evidence for supra-regional circulation of regionally marked uses. The clearest example is perhaps the backchecking use of \textit{già} which, besides being widely used in Piedmont, is also found in Lombardy and Emilia-Romagna – despite being in competition with both a standard alternative (cleft sentence) and other regionally-marked particles (\textit{pure}, \textit{poi} and \textit{più}). Whether this and other features will keep on spreading is an open issue: it cannot be excluded that some of them will further change their sociolinguistic status, coming to be included among the pan-national uses of Italian.

Finally, modal-particle-like elements generally reflect the ongoing process of convergence among the regional varieties of Italian. In fact, the supra-regional diffusion of regionally marked features is consistent with the current tendency represented by a decrease of regional markedness of certain features – which are increasingly used by speakers with different origins. “Due to both internal migrations and increasing exchanges and mobility, regional varieties of Italian are including linguistic features that come from other regional varieties, especially among the younger generation. The regional markedness of spoken Italian is thus noticeably decreasing nowadays. The present younger generation in particular speaks a sort of ‘composite’ RI [regional Italian], at least in terms of phonetics and phonology” \citep[23]{Cerruti2011}. The modal particles investigated by the questionnaire should also be included in this perspective: pragmatic phenomena have proved to be an interesting viewpoint on variation phenomena and on the on-going sociolinguistic changes affecting contemporary Italian, both with regard to the emergence of a new standard and to the reduced regional markedness of certain features.

\subsection{Salience and variation: A path for future research}
\hypertarget{Toc124860692}{}
It is clear that, in order to investigate the historical trajectories of single items, fine-grained work should be devoted to the analysis of the language-contact dynamics which involve standard Italian, dialects and regional varieties. This issue has not been addressed in detail by the present research, which focused on the functional positions occupied by discourse-pragmatic elements in the grammatical system rather than on the processes of language contact they reflect. Referring precisely to works on the diffusion of contact-induced changes, Cerruti (\citeyear{Cerruti2009}: 268--269, \citeyear{Cerruti2020}: 131) points out that the filling of structural gaps seems to facilitate the establishment of some substrate features as part of the standard norm. In this regard, certain constructions have been transferred from substrate dialects to regional varieties of Italian because they represented “useful” and “strategic” features, which provided regional Italian lexical and grammatical constructions absent in the standard.\footnote{See for instance \citet[89]{CerrutiRegis2014} on the focus particle \textit{solo più}, literally ‘only more’, which is a loan translation of Piedmontese \textit{mac pi} (with the same meaning). This construction is a regional standard feature of the Italian spoken in Piedmont and expresses a meaning for which there are no grammaticalized constructions in standard Italian. However, it is now also found in speech productions of speakers coming from other regions.} This seems to be the case for at least some of the elements discussed in the last two chapters (specific uses of \textit{solo}, \textit{già} and \textit{poi}).

Future work will further develop this line of research, investigating on the one hand the development paths of single items, and on the other hand the broader sociolinguistic dynamics that constrain them. In this respect, a relevant theoretical contribution to the discussion can be found in the notion of \textit{salience}. \citet{KerswillWilliams2002} discuss salience as an explanatory concept in language change resulting from dialect contact. They define salience as “a property of a linguistic item or feature that makes it in some way perceptually and cognitively prominent” (\citealt{KerswillWilliams2002}: 81): in the analysis of the dynamics of language contact or internal variation, salience can explain why specific features are perceived as more prominent by speakers and are thus more likely to be transferred between varieties. Describing the interplay of linguistic internal and external factors which constitute this notion, they highlight the role of extra-linguistic factors (cognitive, pragmatic, interactional, and sociodemographic factors) as central in defining salience, because they “directly motivate speakers to behave in a certain way” (\citealt{KerswillWilliams2002}: 106).

Although salience may be described foremost in cognitive terms (\citealt{Rácz2013}: 23–43; \citealt{Schmid2007}; \citealt{TomlinMyachykov2015}), this notion also leaves space for an interpretation in terms of sociolinguistic indexation. For instance, \citet{Cheshire1996,Cheshire1997,Cheshire2009} works with a pragmatically based notion of salience, focusing on the variation of syntactic patterns whose position and interactional roles makes them perceptually salient in the utterance environment. From this perspective, grammatical features are perceived as salient if they are recognized as fulfilling specific pragmatic and interactional functions.\footnote{See also \citet[168–211]{Ariel2008} for a discourse-oriented discussion of this concept – focusing specifically on the role of salient discourse patterns in shaping grammar. Other important references on these topics are \citet{Du_bois1987,Du_bois2003}.} The emphasis on the pragmatic character of salience is particularly well-suited for the properties of the markers described in the present work. Combining interactional properties (inferencing, common ground management) and pragmatic functions (specification of the illocutionary force), these items derive their salient character from their prominent role in providing an utterance with specific features that are strategically used in the construction of discourse. Moreover, they often occur in marked syntactic environments (interrogative sentences, imperative sentences) which further makes their occurrence salient to speakers. Thus, discourse-prag\-mat\-ic markers and analogous constructions meet several criteria which favor their transfer from one variety to another.\footnote{For an example on Italian data, \citet{Cerruti2020} uses the notion of salience to explain the diffusion of different constructions where \textit{mica} operates as a non-canonical negative marker across (sub-standard and standard) regional varieties of Italian and into the standard language. }

In the analysis carried out about \textit{solo}, backchecking markers and markers of emphasis can make a profitable use of this notion. Specifically, I see salience as the contact point between two aspects involved in the description given above. On the one hand, pragmatic markers represent key points in interactions, since they code specific functions on utterances: the conversational exchange between interlocutors also depends on the interpretation of their meaning. Thus, their interactional roles make them perceptually salient in the utterance and in the conversational dynamics. On the other hand, precisely the fact that specific pragmatic functions are perceived as salient by the speakers could represent one of the reasons why (linguistic elements expressing) these functions are charged with indexicality – namely the fact that they \textit{point to} social identities and social meanings \citep{Silverstein2003}. Thus, salience represents a feature leading key points in interaction to develop (social) indexicalities.\footnote{In some streams of contemporary sociolinguistics (see \citealt{Eckert2008,Eckert2012}), concepts such as \textit{indexicality} and \textit{indexical fields} are used to explain the fluid nature of linguistic variables, interpreted as constellations of general and flexible meanings that become more specific in the context of stylistic practice and performance. For a comparison between this approach to sociolinguistic variables with the variationist (Labovian) approach – as well as their implications for sociolinguistic theory – see \citet{GuyHinskens2016}.}

\ea%107
    \label{ex:key:107}

          key points in interaction > salience > (social) indexicality
    \z % you might need an extra \z if this is the last of several subexamples

Salient functions represent linguistic domains which favor the coexistence and turnover of forms: they attract different linguistic elements and so they represent an ideal locus for variation.  In some cases, a specific syntactic position can relate to interactional salience. Most of the modal particles described throughout this work appear after the finite verb form. This is consistent with the fact that they have scope over the illocutionary force, which is expressed by the verb.\footnote{However, this is not a universal fact. For instance, German modal particles are also mostly found after the finite verb – possibly due to the rigid and regular syntactic structure found in German \citep{Abraham1991} – but Japanese modal particles mostly appear in the sentence-final position (\citealt{IzutsuIzutsu2013}).} Providing a space for immediate verbal modification and for the coding of pragmatic functions, the postverbal position is perceived as salient: several interactional functions – mitigation, emphasis and common ground management – are coded (by expressions appearing) here. Thus, this position can be filled with different markers, attracting variation phenomena and leading to the development of different social indexicalites.

