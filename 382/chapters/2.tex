\chapter{Pragmatic markers}\label{sec:2}
\hypertarget{Toc124860610}{}\section{What are pragmatic markers?}
\hypertarget{Toc124860611}{}
In his textbook on pragmatics, \citet{Levinson1983} – discussing discourse deixis – mentions that

\begin{quote}
there are many words and phrases in English, and no doubt most languages, that indicate the relationship between an utterance and the prior discourse. Examples are utterance-initial usages of \textit{but, therefore, in} \textit{conclusion, to the contrary, still, however, anyway, well, besides, actually, all in all, so, after all,} and so on. It is generally conceded that such words have at least a component of meaning that resists truth-conditional treatment. What they seem to do is indicate, often in very complex ways, just how the utterance that contains them is a response to, or a continuation of, some portion of the prior discourse. We still await proper studies of these terms […] (\citealt[87--88]{Levinson1983})
\end{quote}

Levinson’s choice of introducing these items when discussing discourse deixis says something about the absence at that time of an autonomous research field dedicated to these items.\footnote{He also mentions pragmatic markers in the chapters about \textit{conversational implicature} (\citealt[162--163]{Levinson1983}) and \textit{conversational structure} (\citealt[365]{Levinson1983}), confirming the pervasiveness of these items across the pragmatic domain.} Fifteen years after Levinson’s claim that “we still await proper studies of these terms”, the research on what have been called in the meantime \textit{pragmatic markers} (PMs) or \textit{discourse markers} (DMs)\footnote{These are only two of the (almost) countless possibilities to name these items: among them, \textit{pragmatic particles} and \textit{discourse particles}, \textit{pragmatic connectives} and \textit{discourse connectives} can also be found.} was defined “a growth industry in linguistics” by \citet[932]{Fraser1999}. Some years later – in the introduction of an edited volume that represents a key publication in this field – \citet{Fischer2006} takes up the metaphor and goes even further:

\begin{quote}
There are very many studies of discourse particles on the market, and by now it is almost impossible to find one’s way through this jungle of publications. For a newcomer to the field, it is furthermore often very difficult to find the bits and pieces that constitute an original model of the meanings and functions of discourse particles. \citep[1]{Fischer2006}
\end{quote}

Similar claims always come back in this research field. In recent times, \citet{FedrianiSansò2017} begin the introduction of one of the latest volumes published on these topics in this way:

\begin{quote}
In the last decades, research on pragmatic markers (henceforth PMs), discourse markers (DMs) and modal particles (MPs) has produced a generous amount of literature, and a hardly quantifiable number of new approaches and novel insights into their nature and function. A look at this literature is likely to discourage any attempt to edit yet another book on these three elusive entities, and to engage one more time in the often non-rewarding attempt to delimit and define them (both in a standalone fashion and in comparison with one another). (\citealt[1]{FedrianiSansò2017})
\end{quote}

It can be provocatively stated that the growth of research contributions on pragmatic markers is inversely proportional to our understanding of these items or – more gently – that it goes hand in hand with the difficulty of elaborating a shared model for their analysis.\footnote{See also \citet[75]{Waltereit2015}: “More broadly, though, it is difficult to avoid the impression that despite the great number of publications that have been dedicated to discourse markers (including in a historical perspective) in recent years, progress has been a little slow”.}

In fact, the number of approaches and terminologies used in this research field are too numerous and often too vague to sum them up, so that even recommending overview works becomes problematic, because they often reflect the same fragmentation and difficulties. This has substantial consequences for the definition and the classification of the phenomena being investigated and – above all~– on the comparability of research results, with the consequence that sometimes one has the unpleasant impression of not even knowing what exactly is being investigated. \citet{Schourup1999} sums up the problem this way:

\begin{quote}
While it is widely agreed that such expressions play a variety of important roles in utterance interpretation, there is disagreement in regard to such fundamental issues as how the discourse marker class should be delimited, whether the items in question comprise a unified grammatical category, what type of meaning they express, and the sense in which such expressions may be said to relate elements of discourse. \citep[227]{Schourup1999}
\end{quote}

This position has been echoed several times. Some years later, according to \citet[419--420]{Lewis2011}, “there is little consensus on whether they are a syntactic or a pragmatic category, on which types of expressions the category includes, on the relationship of discourse markers to other posited categories”. Similar claims – as a kind of haunting refrain – persist to the present date, where “there is little to no consensus as to what entities constitute the inventory of PMs and DMs in a single language and cross-linguistically” and as to “the subdivisions within the functional spectrum covered by PMs and DMs” (\citealt[4]{FedrianiSansò2017}).\footnote{See also, for instance, \citet[99]{Crible2017}: “Discourse marker (DM) research today, after several decades of flourishing productivity, still faces many terminological, theoretical and methodological issues which restrain large-scale progress in the field, despite the multiplicity of theoretical frameworks and approaches taken by many valuable works”.}

Other commonplace statements in this research field concern the terminology: \citet[5]{DegandEtAl2013} note that “it has become standard in any overview article or chapter on DMs to state that reaching agreement on what makes a DM is as good as impossible, be it alone on terminological matters”. In sum, it seems that certainty and consensus are largely overcome by doubt and disagreement in this research field. And yet – perhaps at an intuitive level – the prototypical pragmatic markers are easily identified in various languages:

\ea%1
    \label{ex:key:1}

          English \citep[1027]{Schourup2001}

\textit{Well}, isn’t it beautiful outside!
    \z  

\ea%2
    \label{ex:key:2} 

          Italian (\citealt[455]{Bazzanella2006})

\textit{Ecco}, \textit{cioè}, \textit{voglio dire}, non sono del tutto d’accordo.

\glt ‘Well, that’s to say, I mean, I don’t completely agree.’
    \z  

\ea%3
    \label{ex:key:3} 

          Greek (\citealt[662]{NikiforidouEtAl2014})

\textit{Ela} min arxisis tis grinies pali mu lei ekinos.

\glt ‘Come on, don’t start grumbling again, he says.’
    \z  

Even without trying to understand what the words in italics have in common – which admittedly isn’t an easy task and would immediately lead to classification problems – it might be said that they somehow make the examples sound natural: they do not refer to anything in external reality but they are involved in facilitating spontaneous speech production and in letting the interaction go smoothly.\footnote{In fact, some of the most creative definitions of pragmatic markers – not really part of the scientific description of these items – are indeed the most convincing, from \textit{discourse lubricant} and \textit{conversational greaser} to \textit{discourse glue} (see \citealt[1]{Brinton1996} and references therein).}

Leaving aside terminologies and classifications for now, I use the term \textit{pragmatic marker} as a “cover term for a range of seemingly heterogeneous forms” (\citealt{HansenRossari2005}: 178) that operate at the communicative level rather than at the propositional one.\footnote{The contrasting terms, \textit{communicative} vs. \textit{propositional} are used in a pre-theoretical way for now. Similar contrasts are represented by pairs such as \textit{context-level} vs. \textit{content-level}, \textit{interpersonal} vs. \textit{representational}, \textit{procedural} vs. \textit{conceptual}, \textit{use-conditional meaning} vs. \textit{truth-conditional meaning}. Such pairs are used by different theoretical frameworks and none of them is fully equivalent to the others. Some of them will be discussed in more detail below.}  Works like \citet[29--40]{Brinton1996} and \citet{AijmerSimon-Vandenbergen2011} – which also use \textit{pragmatic marker} as an umbrella term – provide an insightful overview of this topic (research tradition, delimitation of the research field, comparison of terminologies and classifications, and list of functions), as well as discussions of different approaches and methodologies.

A clarification is needed at this point. The present work does not deal with pragmatic markers as a general category, but rather with a quite limited class of items operating on the illocutionary act (that is modal particles, which could be considered at most as a subset of pragmatic markers). For this reason, the choice of beginning with a broad discussion of pragmatic markers may appear unjustified, a fact of which I am well aware, yet on the other hand – as ought to be clear from what has been said thus far – starting from this point is unavoidable, since the research tradition has often lumped together different issues in one package leading to the establishment of a sort of customary practice which is difficult to overcome. This is not necessarily inappropriate – and indeed there are some good reasons for doing it, first of all the multifunctionality of several items\footnote{It is quite common that the same linguistic item shows both a use as a discourse marker and a use as a modal particle: many of the case studies collected in \citet{DegandEtAl2013} insist on this point. Moreover, as discussed in \chapref{sec:4}, several issues concerning the meaning and the development of discourse-pragmatic elements are transversal to the whole category.} – but the impression cannot be avoided that perhaps some issues are grouped and discussed together as a matter of routine, more than for the real necessity to have a common container (and a cover term) for a highly heterogenous set of items:

\begin{quote}
It might seem practical to group elements that are complex to distinguish, but the cognitive soundness and methodological efficiency of such an approach remain to be demonstrated. In fact, to my knowledge, no corpus study has ever identified and analyzed such a large range of items in authentic data. It would seem that the merit of the PM category is therefore mainly theoretical and metalinguistic, and does not correspond to an empirically-founded category of similar expressions in language use. (\citealt{Crible2017}: 102–103)
\end{quote}

Bearing these warnings in mind, let’s have a look at the whole set and extract what is needed.

\section{Approaches to pragmatic markers}
\hypertarget{Toc124860612}{}
Even without giving a precise theoretical value to the notion of \textit{pragmatic marker,} there are some features that can certainly be identified as characterizing all linguistic items with discourse-pragmatic functions. The following three points are highlighted by \citet[3--4]{FedrianiSansò2017}. First, they have a non-truth-conditional value: they can be detached from the utterance in which they appear without affecting its propositional meaning. Second – on a broad behavioral level – their functions may be intended as \textit{procedural}: pragmatic markers are items that place constraints on the interpretation by providing instructions to the hearer as to how the proposition/utterance is to be processed, in order for both participants to co-build a coherent mental representation of discourse. Third – in the normal case – there is typically more than one pragmatic marker that can potentially serve a given function in a given language and – conversely – each marker has more than one function depending on different (socio-)linguistic variables.

Formal features (the position in the utterance, morphological features, intonation contours) should be – at least for now – left aside, since they would immediately force me to deal with the subcategorization of markers and functions or language-specific issues. This means that the category of pragmatic markers – and the same goes for most of the possible subcategories – holds as a consistent group of elements only insofar as they share a global discourse-pragmatic function, and should thus be understood as a functional category, rather than a formal one. As stated by \citet[27]{Hansen2006}: “I do not conceive of discourse markers as constituting a part of speech, for it seems that very few linguistic items are exclusively devoted to this function. Rather, a great many, often formally quite different, linguistic items may have one or more discourse-marking uses alongside one or more non-discourse-marking uses”. Before shifting the focus to modal particles, I will shortly mention two relevant works on discourse-pragmatic elements, in order to introduce issues related to the description of their functions and their subcategorization.

A first influential approach to discourse markers is represented by \citet{Blakemore1987}. This work fits into the larger framework of Relevance Theory (RT), which has been developed since \citegen{Wilson1986} book (see \citealt{Wilson2017} for a recent overview). Relevance Theory is interested in understanding and describing the linguistic means of encoding information about the inferential processes that make communication possible. RT is grounded in a Principle of Relevance which consists of a cognitive facet (“Human cognition tends to be geared to the maximization of relevance”) and a communicative one (“Every utterance communicates a presumption of its own optimal relevance”) (see \citealt{Wilson2017}: 83--85).

Within this framework, the \textit{conceptual} vs. \textit{procedural} distinction was introduced – a cognitive distinction between “two ways in which linguistic meaning can contribute to the inferential processes involved in utterance interpretation: either it may encode constituents of the conceptual representations that undergo these processes, or it may encode procedural information or constraints on those processes” \citep[229]{Blakemore2006}. In this perspective, discourse-pragmatic elements represent an ideal case study. Relevance Theory has focused mainly on markers with discourse-structuring and textual functions – elements like \textit{but}, \textit{so}, \textit{after all}, \textit{nevertheless} – that is, non-truth-conditional expressions that connect sentences together (\citealt{Blakemore1987,Blakemore2002}; see also \citealt{JuckerZiv1998}). In these studies, discourse markers are analyzed as expressions that restrict the inferential processes in communication – facilitating the speaker’s aim of achieving relevance for a minimum cost in processing and guiding the hearer in the correct interpretation of the utterance and its context \citep[230]{Blakemore2006}. Discourse markers in a relevance-theoretic perspective are thus good examples of linguistic expressions that encode procedural meanings: they don’t refer to conceptual representations but to inferential processes, encoding information about which of these inferential processes yields the intended interpretation.\footnote{Relevance Theory also points out that the distinction between conceptual and procedural meaning is not equivalent to the distinction between truth-conditional and non-truth-conditional meaning \citep[230]{Blakemore2006}, paving the way towards a discussion about how to draw the distinction between pragmatics (traditionally associated with non-truth-conditional meaning) and semantics (traditionally associated with truth-conditional meaning) – and their respective role in the description of conceptual and procedural meaning. I will further discuss this issue in \chapref{sec:3}.}

A second seminal work on discourse markers dates to the same year, but builds on a completely different framework. \citet{Schiffrin1987} fits in the larger framework of discourse analysis, relying on the assumption that language is context-sensitive and designed for communication. \citet[21--29]{Schiffrin1987} builds her analysis on a model of discourse that identifies five \textit{discourse planes}, to be understood as the different discourse components on which discourse markers work: participation framework, information state, ideational structure, action structure, and exchange structure. This work has contributed to identify the set of English expressions – structurally defined as “sequentially dependent elements that bracket units of talk” \citep[31]{Schiffrin1987} – which have been at the center of research on discourse markers in the following years: interjections (\textit{oh}), adverbs (\textit{well}, \textit{now} and \textit{then}), connectives (\textit{and}, \textit{but}, \textit{or}, \textit{so} and \textit{because}) and lexicalized phrases (\textit{y’know} and \textit{I mean}).

Discourse markers are seen to work as \textit{contextual coordinates of talk} with indexical functions: “markers index the location of an utterance within its emerging local contexts. It is the indexical function of markers which is the key to understand why they are used: markers propose the contextual coordinates within which an utterance is produced and designed to be interpreted” \citep[315]{Schiffrin1987}. In this way, the five planes of discourse and the role of discourse markers build a model of discourse coherence, where the markers index their host utterances to one or more of these five planes and thus integrate them, contributing to the production of coherent discourse. In this model, the \textit{multifunctionality} of discourse markers is highlighted, since they can operate simultaneously on different planes of discourse. By doing so, they integrate the many different processes underlying the construction of discourse, and thus help to create coherence.

These works have set the framework for much of the subsequent research. Notions such as \textit{procedural meaning}, \textit{discourse planes} and \textit{multifunctionality} have played an important role in the research on discourse markers to date, and they are regularly employed in very recent accounts of discourse-pragmatic elements such as \citet{Ghezzi2014} and \citet{Crible2017,Crible2018}. In particular, the approach and the terminological choices adopted by \citet{Crible2017,Crible2018} are largely compatible with the approach taken in the present work. Crible builds on \citet[28]{Hansen2006}, using the label \textit{pragmatic marker} as an overall “cover term for all those non-propositional functions which linguistic items may fulfil in discourse”. Thus, discourse markers are conceived as a subclass of this overarching category:

\begin{quote}
DMs are a grammatically heterogeneous, syntactically optional, polyfunctional type of pragmatic marker. Their specificity is to function on a metadiscursive level as procedural cues to constrain the interpretation of the host unit in a co-built representation of on-going discourse. They do so by either signaling a discourse relation between the host unit and its context, making the structural sequencing of discourse segments explicit, expressing the speaker’s meta-comment on their phrasing, or contributing to the speaker-hearer relationship. \citep[35]{Crible2018}
\end{quote}

Other subclasses include interjections, modal particles, response signals, politeness expressions and tag questions.\footnote{For the syntactic and functional criteria used to delimit the subclass of discourse markers from the other subclasses see the detailed discussion in Crible (\citeyear[105--109]{Crible2017}, \citeyear[34--37]{Crible2018}).} One of them, namely \textit{modal particles}, will be the focus of the coming pages.

\section{Identifying a subclass: Modal particles}
\hypertarget{Toc124860613}{}
While discourse markers (DMs) are an arguably universal category, modal particles (MPs) – also known as \textit{Abtönungspartikeln} in the German linguistic tradition (lit. ‘shading particles’) – are commonly viewed as specific to certain languages. The presence of modal particles is widely recognized (and exceptionally well-researched) for German (\citealt{Weydt1969,Weydt1979}; \citealt{Thurmair1989}; \citealt{Abraham1991}; \citealt{Meibauer1994}; \citealt{König1997}; \citealt{Waltereit2001,Waltereit2006}; \citealt{Zimmermann2011}; \citealt{BayerStruckmeier2017}) but contested for English and Romance languages.\footnote{The references on this issue are impossible to sum up: for English, see for instance \citet{Haselow2011} and \citet{FischerHeide2018}. References to modal particle research in Romance languages will be frequent throughout the whole work: a good starting point is \citet{Waltereit2001}. More broadly, the cross-linguistic distribution of modal particles is still under-researched and no typological work on this topic has appeared to date. To give a few references, studies on modal particles have appeared for Dutch (\citealt{Foolen1993}; \citealt{Vismans1994}), Danish (\citealt{Davidsen-Nielsen1996}; \citealt{Hansen1998a}: 41–46), Swedish \citep{Aijmer1996}, Slavic languages (\citealt{DedaićMišković-Luković2010}), Japanese (\citealt{IzutsuIzutsu2013}), Chinese (\citealt{Li2006}; \citealt{Fang2021}), Hindi \citep{Montaut2016} and many other languages. The issue of cross-linguistic distribution is further complicated by the question of whether (and how) it should be distinguished between languages which feature modal particles as a (consistent) word class and languages which feature modal-particle-like elements and/or elements that can be described as modal particles in some of their uses. Among the recent publications that adopt a cross-linguistic-oriented approach see \citet{ModicomDuplâtre2020}, \citet{GergelEtAl2022} and \citet{ArtiagoitiaEtAl2022}.} For this reason – adding to the difficulty of clearly describing their functions – modal particles occupy an ambiguous position in the research on discourse-pragmatic elements: sometimes they are seen as “special guests”, sometimes as “unwelcome guests”, and sometimes even as “gatecrashers”.

In recent times, several works have dealt with the controversial relationship between modal particles and other discourse-pragmatic elements, first of all discourse markers (among others, see \citealt{Hansen1998a} and the papers collected in \citealt{DegandEtAl2013}). With reference to this debate, \citet{Detges2015} gives this clear and concise definition of modal particles:\largerpage

\begin{quote}
While DMs are defined by purely functional criteria, the definition of MPs includes function as well as form. In German, MPs are syntactically integrated, (mainly) unstressed particles; they appear in the middle-field, next to the inflected form of the predicate. Syntactically, MPs have scope over sentences. Functionally, they fine-tune speech acts, by “repairing” problems arising from the violation of some felicity condition (\citealt{Waltereit2001,Waltereit2006}). Unlike DMs which indicate two-place relations between sequentially ordered chunks of discourse, MPs mark a relationship between a speech act and some element in the common ground, usually a belief (“a proposition”) attributed to the addressee. \citep[132]{Detges2015}
\end{quote}

The articles collected in \citet{DegandEtAl2013} give an interesting overview of the different ways of treating the relationship between DMs and MPs, from those who sharply separate them to those who give them an equal footing. Even though there are many formal\footnote{These are foremost syntactic criteria. In particular, German modal particles occur exclusively in the so-called \textit{middle field} (the domain between the initial and final verbal elements of the Germanic clause). In Japanese they mostly occur in the sentence-final position instead. Overall modal particles are sensitive to syntactic constraints and tend to occur in fixed sentence positions according to language-specific syntactic structures.}  and functional reasons to see them as separate pragmatic categories (as in Detges’ definition cited above), the problem of their relationship was not pulled out of thin air, since many linguistic items can cover both functions, giving the impression of a close link between the two categories. Nevertheless, considering the different processes of change through which they emerge (\citealt{WaltereitDetges2007}; \citealt{DetgesWaltereit2009}), they should be better dealt with separately.

Modal-particle-like expressions (or, citing again \citealt{Detges2015}: 136, expressions with a “certain MP-like flavour” in some of their uses) will be at the center of the present research. The discussion about their functions and their place in the grammar of a language, their diachronic development and their synchronic variation, their relationship with other discourse-pragmatic elements will come up throughout the whole first part of the work. Different views on these categories crucially also depend on how their functions are conceived of and defined, a topic that I will explore further in the next chapter.

