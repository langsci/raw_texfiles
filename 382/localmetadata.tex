\author{Marco Favaro} %use this field for editors as well
\title{Modal particles in Italian}
\subtitle{Adverbs of illocutionary modification and sociolinguistic variation}
\BackBody{This study investigates the properties of a set of Italian adverbs (among others: \textit{pure} ‘also’, \textit{solo} ‘only’, \textit{un po’} ‘a bit’) that, in specific contexts of use, modify the speech acts in which they appear. On the one hand, these elements specify the way in which a speech act should be interpreted with reference to the specific interactional context, modifying its illocutionary force. On the other hand, they index presupposed/inferred meanings active in the common ground of the interaction, integrating the speech act in the common ground. These functions closely resemble those of the elements that, especially in the German linguistic tradition, are called modal particles. Drawing on original data from Italian – both from the standard language and regional varieties – the goal of the study is to describe the synchronic features of these elements and to explain the emergence of the modal uses. For this purpose, it jointly employs theoretical notions of pragmatics (speech act theory, inferences in interaction), models of language change (reanalysis and conventionalization) and the descriptive tools of sociolinguistic approaches. Through the presentation of four case studies, integrating corpus and questionnaire data, the present work gives a thorough analysis of the modal functions and the contexts of use of the adverbs under investigation: it explores their role at the semantics/pragmatics interface, it discusses their place in a layered model of grammar and it examines their distribution across different language varieties.}
\ISBNdigital{978-3-96110-428-4}
\ISBNhardcover{978-3-98554-086-0}
\BookDOI{10.5281/zenodo.10259474}
\typesetter{Sebastian Nordhoff, Hannah Schleupner}
\proofreader{Carla Bombi}

\renewcommand{\lsSeries}{orl}
\renewcommand{\lsSeriesNumber}{6}
\renewcommand{\lsID}{382}
