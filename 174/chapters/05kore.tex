\section{Koreanic}\label{sec:5.7}
\subsection{Classification of Koreanic}\label{sec:5.7.1}

\ili{Korean} has a North \ili{Korean} (Pyongyang) and a South \ili{Korean} (Seoul) standard. Here primarily the latter will be considered. In addition, \ili{Korean} is officially recognized as a minority language in \isi{China}, where it has developed its own standardized version of \ili{Korean} based on the language spoken in Yanbian, Jilin province (L. \citealt{BrownYeon2015}: 466). But apart from the standard languages, \ili{Korean} also contains a considerable amount of dialectal variation. Usually, six different dialect areas are recognized (L. \citealt{BrownYeon2015}: 461), but it has become increasingly clear that \ili{Yukcin} has to be considered a seventh dialect (e.g., \citealt{King2006a}: 130).

\ea\upshape%1
    \label{ex:kore:1}
\begin{forest}  for tree={grow'=east,delay={where content={}{shape=coordinate}{}}},   forked edges  
[
    [Northwestern (Pyongan Province)]
    [Northeastern (Hamgyong Province)]   
    [\ili{Yukcin} (Hamgyong Province)]
    [Central (Gyeonggi{,} Hwanghae{,} Gangwon{,} and Chungcheong Provinces)]
    [Southwestern (Jeolla Province)]
    [Southeastern (Gyeongsang Province)]
    [\ili{Jeju} (\ili{Jeju} island)]
]
\end{forest}   
    \z

\citet[58]{Sohn1999} also differentiates between seven dialect zones, but instead of \ili{Yukcin} he regards Chungcheong, included in the Central Dialect above, as a separate entity. \ili{Jeju} clearly is the most aberrant member of the \ili{Korean} dialects (e.g., \citealt{Kiaer2014}). \citet{Vovin2013a} even goes so far as to consider \ili{Jeju} a \ili{Koreanic} language in its own right. He claims that the primary division is between \ili{Jeju} on the one side and the varieties spoken on the \isi{Korean Peninsula} on the other. In his view, \ili{Yukcin}, part of the Northeastern dialect area, is also sufficiently different from the rest of the dialects to consider it a separate language. But \citet[8]{Lee2015} recently came to the rather convincing conclusion “that the early historical relationships among \ili{Koreanic} variants are considerably non-treelike”. In general, it may thus be better to conceptualize \ili{Koreanic} as a dialect \isi{continuum} with strong mutual contacts that make a classification into different languages problematic.

Within the Northeast Asian area, apart from the \isi{Korean Peninsula} and adjacent regions in \isi{China}, significant numbers of \ili{Korean} speakers can also be found on \isi{Sakhalin}, in \isi{Japan}, and in Central \isi{Asia}. The language in Central \isi{Asia}, mostly in Uzbekistan and Kazakhstan, has its origin in Northeastern and \ili{Yukcin} dialects, while the language spoken on \isi{Sakhalin} is ultimately derived from the Southeast of \isi{Korea} \citep[128]{King2006a}. It is primarily the language spoken in Central Asia---also known as \ili{Kolyemal} (Koryo language)---that will be included in this chapter. The \ili{Korean} dialects in \isi{China} are not very well described, but one can roughly state that “Yanbian \ili{Korean} has its roots in Hamgyong dialect, whereas the variety of \ili{Korean} spoken in Liaoning is of the Pyongan variety and that of Heilongj[i]ang is based on Gyeongsang” (L. \citealt{BrownYeon2015}: 466, corrected). Given the scarcity of resources, only the variety spoken in Yanbian, Jilin province, will be included in this study (\citealt{ZhaoXi1982}; \citealt{XuanDewu1985}). In \isi{Japan}, apart from mainland \ili{Korean} dialects, we also find speakers of \ili{Jeju}, especially in \\isi{\=Osaka} \citep{Saltzman2014}.

\subsection{Question marking in Koreanic}\label{sec:5.7.2}

When it comes to \isi{question marking}, \ili{Korean} has a complicated split system that depends on the speech level. The \isi{interrogative} forms in \ili{Korean} qualify as \isi{interrogative} mood markers because they are in complementary distribution with declarative markers. In other words, the \isi{interrogative} suffixes replace the declarative ones and are not merely attached to them. This is a major difference compared to most languages in \isi{Northeast Asia}.

\ea%2
    \label{ex:kore:2}
    \ili{Korean} (Jilin)\\
    \ea
    \gll narssi-ka  tʃoh-\textbf{{ta}}.\\
    weather-\textsc{nom}  good-\textsc{decl.plain}\\
    \glt ‘The weather is good.’
    
    \ex
    \gll narssi-ka  tʃoh-\textbf{{ni}}?\\
    weather-\textsc{nom}  good-\textsc{q.plain}\\
    \glt ‘Is the weather good?’ (\citealt{XuanDewu1985}: 57)
    \z
    \z

Descriptions disagree in the number of forms and speech levels in \ili{Korean}. \tabref{tab:kore:1} shows these according to the \isi{analysis} by \citet{Song2005}, who distinguishes six different levels. There are declarative, \isi{interrogative}, imperative, and propositive endings. The suffixes are usually called “sentence enders”, because they always take the last position in a sentence and are not restricted to verbs as such, but can also attach to verbal adjectives. Consider the following examples from Jilin \ili{Korean}.

\ea%3
    \label{ex:kore:3}
    \ili{Korean} (Jilin)\\
    \ea
    \gll ka-\textbf{{nɯnka}}?\\
    go-\textsc{q}.\textsc{fam}\\
    \glt ‘Are (you) going?’
    
    \ex
    \gll k‘ɯ-\textbf{{nka}}?\\
     big-\textsc{q}.\textsc{fam}\\
    \glt ‘Is (it) big?’ (\citealt{XuanDewu1985}: 31)
    \z
    \z

\begin{table}
\caption{Korean sentence enders \citep[125]{Song2005}}
\label{tab:kore:1}

\begin{tabularx}{\textwidth}{XXXXl}
\lsptoprule
& \textbf{Statements} & \textbf{Questions} & \textbf{Commands} & \textbf{Proposals}\\
\midrule
Plain & -(n)ta & -ni/-(nu)nya & -ela/-ala & -ca\\
Intimate & -e/-a & -e/-a & -e/-a & -e/-a\\
Familiar & -ney & -na/-nunka & -key & -sey\\
Semi-formal & -o & -o & -(u)o & -(u)psita\\
Polite & -eyo/-ayo & -eyo/-ayo & -eyo/-ayo & -eyo/-ayo\\
Deferential & -(su)pnita & -(su)pnikka & -(u)sipsio & -(u)sipsita\\
\lspbottomrule
\end{tabularx}
\end{table}

Some of the sentence enders can be further analyzed. The first element in \textit{-n-unya}, \textit{-n-un-ka}, and maybe in \textit{-n-i}, as well as the medial element in \textit{-sup-ni-kka} may be an indicative marker. The suffix \textit{-sup} is an addressee honorific while the suffix \textit{-un} has been called a “pre-nominal-modifier” suffix. The polite forms are identical with the intimate forms except for an additional suffix \textit{-yo} (\citealt{Sohn1994}: 337ff.). In the Chungcheong dialect, often included into the Central dialect, but treated as a separate dialect by \citet[58]{Sohn1999}, this takes the characteristic form \textit{-yu} (L. \citealt{BrownYeon2015}: 462).

Some of the forms in \tabref{tab:kore:1} are not restricted to one function. In fact, of the \isi{interrogative} forms mentioned, only the plain, familiar, and deferential forms are not also found in statements, commands, or proposals. \citet[449]{Sohn2015} lists additional variants for plain statements (\textit{-la} instead of \textit{-ta}) and semi-formal (\textit{-(s)o/-(s)wu} instead of only \textit{-o}) \isi{questions}.

The sentence endings in the officially recognized variety of \ili{Korean} spoken in \isi{China} are very similar to standard \ili{Korean} (\tabref{tab:kore:2}). The authors mention additional forms not shown here such as \textit{-tʃi} or \textit{-tʃio}, which are found in all sentence types. These probably correspond to the committal \textit{-ci} and its \isi{combination} with the polite marker \textit{-ci-yo} > \textit{-cyo} in Standard \ili{Korean} (see below). There are, furthermore, the endings \textit{-(nɯn)tʃi}, \textit{-(nɯn)ja}, and \textit{-najo} that are restricted to the \isi{interrogative} sentence type. Their exact difference in meaning remains unclear. But these are clearly combinations of other elements already encountered. The element \textit{-nɯn} is known from the complex familiar \isi{interrogative} ending \textit{-nɯn-ka} and \textit{-najo} is the familiar \isi{interrogative} ending \textit{-na} in \isi{combination} with the suffix \textit{-jo} known from the polite speech level. The last elements in \textit{-nɯn-tʃi} and \textit{-nɯn-ja} are probably the marker \textit{-tʃi} seen before and the intimate marker \textit{-(j)ə/-a}, respectively, both of which are \isi{speech act} neutral.

\begin{table}
\caption{Sentence enders in Korean as spoken in China (\citealt{XuanDewu1985}: 62f.; \citealt{ZhaoXi1982}: 75) listed analogous to \tabref{tab:kore:1}}
\label{tab:kore:2}

\begin{tabularx}{\textwidth}{XXXXl}
\lsptoprule
& \textbf{Statements} & \textbf{Questions} & \textbf{Commands} & \textbf{Proposals}\\
\midrule
Plain & (-nɯ)-(n)ta & -ni & -(j)əra/-ara & -tʃa\\
Intimate & -(j)ə/-a & -(j)ə/-a & -(j)ə/-a & -(j)ə/-a\\
Familiar & -ne & -na/-(nɯn)ka & -ke & -se\\
Semi-formal & -(s)o & -(s)o & -(s)o & -(ɯ)psita\\
Polite & -(j)əjo/-ajo & -(j)əjo/-ajo & -(j)əjo/-ajo & -(j)əjo/-ajo\\
Deferential & -(sɯ)pnita & -(sɯ)pnikka & -(ɯ)sipsiɣo & -(ɯ)psita\\
\lspbottomrule
\end{tabularx}
\end{table}

As opposed to standard \ili{Korean} \textit{-(u)si-psita}, \ili{Chinese} \ili{Korean} \textit{-(ɯ)psita} lacks the element \textit{-si} that is present in \textit{-(u)si-psio}/\textit{-(ɯ)si-psiɣo} and has been characterized as a “subject honorific suffix” \citep[344]{Sohn1994}. For Standard \ili{Korean} \citet[151]{Kim-Renaud2012} mentions an additional set of so-called “superdeferentials”, the \isi{interrogative} form of which is \textit{-((u)si)naikka}. According to her, the familiar \isi{interrogative} forms (called “deferential equal”) are \textit{-(n)(u)nka(yo)} and \textit{-((u)si)na(yo)}. In the latter form, both the honorific suffix \textit{-si} and the polite marker \textit{-yo} are optional, and the same is true for \textit{-(u)si-psita/-(ɯ)psita} and other sentence enders. Variants with either the vowel \textit{e} or \textit{a} depend on the vowel in the preceding syllable. The variant with \textit{a} follows syllables that contain an \textit{a} or an \textit{o}, otherwise the variant with \textit{e} is employed. This is a special kind of restricted \isi{vowel harmony} still present in \ili{Korean}. \tabref{tab:kore:3} shows all attested standard \ili{Korean} variants with the help of two verbs and two adjectives.

\begin{table}
\caption{Interrogative paradigms of two verbs and two adjectives in Korean (\citealt{Sohn1994}: 15-16)}
\label{tab:kore:3}

\begin{tabularx}{\textwidth}{XXXXl}
\lsptoprule
& \textbf{mek- ‘to eat’} & \textbf{po- ‘to see’} & \textbf{coh ‘good’} & \textbf{si ‘sour’}\\
\midrule
Plain & mek-ni

mek-nunya & po-ni

po-nunya & coh-(u)ni

coh-unya & si-ni

si-nya\\
Intimate & mek-e & po-a & coh-a & si-e\\
Familiar & mek-na

mek-nunka & po-na

po-nunka & (coh-na)

coh-unka & (si-ne)

si-nka\\
Semi-formal & mek-so

mek-\textbf{uo} & po-o & coh-so & si-o\\
Polite & mek-eyo & po-ayo & coh-ayo & si-eyo\\
Deferential & mek-supnikka & po-pnikka & coh-supnikka & si-pnikka\\
\lspbottomrule
\end{tabularx}
\end{table}

The use of the different speech levels is highly complex and has been very well summarized by \cite[126f.]{Song2005}, whose concise description is worth quoting in an abbreviated form. See \citet{Brown2011} for details.

\largerpage
\begin{quote}
The \textbf{plain} speech style is used between friends or siblings whose age difference is not substantial (perhaps a one or two year age gap; in \ili{Korean} culture, a three or more year age difference is regarded as substantial), or by old speakers (e.g. parents or teachers) to young children. [...]

The \textbf{intimate} speech level is referred to as \textit{panmal} ‘half talk’ in \ili{Korean}. This level is similar to the plain level in that it is used between close friends and siblings (both before middle age), by young school children to adult family members (especially their (grand)mother but probably not their (grand)father) or by a man to his (younger) wife. [...]

The \textbf{familiar} speech level is used to someone who has a lower social status than the speaker. When this level is chosen, however, the speaker is signal[l]ing a reasonable amount of courtesy to the hearer. [...] it is typically used by male adults to younger male adults who are probably under the former’s influence (e.g. protégés or former students), or to their sons-in-law. [...]

The \textbf{semi-formal} speech-level [...] has almost completely fallen into disuse and may indeed sound old-fashioned to young people’s ears. It is definitely a speech level associated with the older generation. If used, however, it is to someone with lower social status than the speaker and it is regarded as a slightly more courteous speech level than the familiar speech level. [...]

The \textbf{polite} speech level, together with the intimate speech level, is the most commonly used speech level, but, unlike the intimate speech level -- which is emblematic of intimacy, familiarity or friendliness -- it is used when politeness or courtesy is called for, regardless of the social status of the hearer, as long as they are old enough (university students and older). [...]

Finally, the \textbf{deferential} speech level is the highest form of deference to the hearer. This speech level is thus used to people with unquestionable seniority. It is never used to someone with equal or inferior social status. [...] (my boldface)
\end{quote}

However complicated the internal division of \isi{question marking} may be, it does not depend on the \isi{question type}. The following \isi{content question}s display the same question markers as did the \isi{polar question}s above. Interrogatives remain \textit{in situ} \citep[265]{Sohn1999} but nevertheless are often in sentence initial position.

\ea%4
    \label{ex:kore:4}
    \ili{Korean} (Jilin)\\
    \gll \textbf{{muɣəs}}-ɯr  ha-\textbf{{nɯnka}}?\\
    what-\textsc{acc}  do-\textsc{q.fam}\\
    \glt ‘What are (you) doing?’ (\citealt{XuanDewu1985}: 42)
    \z

\ea%5
    \label{ex:kore:5}
    \ili{Korean}\\
    \gll \textbf{{mues}}-ul  ha-\textbf{{ni}}?\\
    what-\textsc{acc}  do-\textsc{q.plain}\\
    \glt ‘What are (you) doing?’ \citep[146]{Song2005}\footnote{In casual speech this sentence is said to be pronounced \textit{mwel hani}.}
    \z

\noindent Notice the slight dialectal differences such as the presence of an intervocalic consonant in Jilin \ili{Korean} \textit{mu}\textbf{\textit{ɣ}}\textit{əs} as opposed to standard \ili{Korean} \textit{mues} (also cf. \textit{-(ɯ)si-psi}\textbf{\textit{ɣ}}\textit{o} versus \textit{-(u)si-psio}), as well as the difference in speech level. Alternative questions do not exhibit an obligatory \isi{disjunction}. Instead, each alternative takes one of the \isi{interrogative} sentence enders listed above. Naturally, the two markers have to be identical, i.e. are from the same speech level.

\ea%6
    \label{ex:kore:6}
    \ili{Korean} (Jilin)\\
    \gll kitʃ‘a-ka  məntʃə  o-r-\textbf{{ka}},      tʃatoŋtʃ‘a-ka  o-r-\textbf{{ka}}?\\
    train-\textsc{nom}  first  arrive-\textsc{prs}-\textsc{q.fam}  car-\textsc{nom}  arrive-\textsc{prs}-\textsc{q.fam}\\
    \glt ‘Does the train or does the car arrive first?’ (\citealt{XuanDewu1985}: 94)
    \z

\ea%7
    \label{ex:kore:7}
    \ili{Korean}\\
    \gll wuli-ka    ka-l-\textbf{{kka.yo}} salam-ul  ponay-l-\textbf{{kka.yo}}?\\
    we-\textsc{nom}  go-\textsc{prs}-\textsc{q.fam}    person-\textsc{acc}  send-\textsc{prs}-\textsc{q.fam}\\
    \glt ‘Shall we go or shall (we) send someone?’ \citep[122]{Sohn1994}
    \z

\noindent In the latter example the same politeness marker \textit{-yo} that we have already encountered in the polite level endings \textit{-a.yo} {\textasciitilde} \textit{-e.yo} is found in the Standard \ili{Korean} example.

There is an optional \isi{disjunction} \textit{an-i-myen} ‘\textsc{neg}-\textsc{cop}-\textsc{cond}’ that literally means ‘and if not’ \citep[20]{Sohn1994} and is thus a parallel to Mongolian \textit{eswel} (\sectref{sec:5.8.2}).

\ea%8
    \label{ex:kore:8}
    \ili{Korean}\\
    \gll yongho-ka  te  khu-\textbf{{ni}}, \textbf{{animyen}} nami-ka  te khu-\textbf{{ni}}?\\
    \textsc{pn}-\textsc{nom}  more  big-\textsc{q.plain}  or    \textsc{pn}-\textsc{nom}  more big-\textsc{q.plain}\\
    \glt ‘Is Yongho taller or Nami?’ \citep[20]{Sohn1994}
    \z

Negative \isi{alternative question}s may make use of a negative verb such as in the idiomatic expression in \REF{ex:kore:9}.

\ea%9
    \label{ex:kore:9}
    \ili{Korean}\\
    \gll ka-l-\textbf{{kka}} \textbf{{ma}}-l-\textbf{{kka}}?\\
    go-\textsc{prs}-\textsc{q}  \textsc{neg}-\textsc{prs}-\textsc{q}\\
    \glt ‘whether to go or not’ \citep[392]{Sohn1999}
    \z

\noindent When the first alternative is a copula, the second alternative has to be the negative counterpart of a copula.

\ea%10
    \label{ex:kore:10}
    \ili{Korean}\\
    \gll canton    iss-\textbf{{ni}} \textbf{{eps}}-\textbf{{ni}}?\\
    change    \textsc{cop}-\textsc{q.plain}  \textsc{neg}-\textsc{q.plain}\\
    \glt ‘Do you have change or not?’ (\citealt{Kim-Renaud2012}: 150)
    \z

\noindent These are constructions very similar to those of surrounding languages such as \ili{Japonic}, \ili{Mongolic}, or \ili{Tungusic} (see Chapter 6).

\citet[2783]{Yoon2010} investigated the relative frequency of question types. In this study there were 70\% \isi{polar question}s (including \isi{tag question}s), 29\% \isi{content question}s and only 3\% \isi{alternative question}s. However, 15\% of all polar \isi{questions} were actually tag \isi{questions}. The situation is thus very similar to \ili{Japanese} (\sectref{sec:5.6.2}). Tag question markers usually have the form \textit{ku-ci-yo} ‘do.so-\textsc{comm}-\textsc{pol}’ (> \textit{kucyo}) and are attached to a \isi{declarative sentence}.

\begin{quote}
Instead of completing a statement with a sentence ending and then adding a \isi{tag question} such as \textit{ku-ci-yo}, it is possible to put \textit{-ci} or the contracted form of its negative form \textit{-canh} [< \textit{-ci-anh}] into the sentence ending of the main statement without using it in a separate \isi{tag question}. Such a \isi{tag question} marked in the sentence ending is called a “pseudo-\isi{tag question}” by some researchers (\citealt[2788]{Yoon2010}, my brackets)
\end{quote}

\noindent The author has recorded the two examples in (\ref{ex:kore:11}):

\ea%11
    \label{ex:kore:11}
    \ili{Korean}\\
    \ea
    \gll hwuchwu  mac-\textbf{{cyo}}?\\
    black.pepper  correct-\textsc{comm.pol}\\
    \glt ‘It is black pepper, right?’
    
    \ex
    \gll wenlay    khu-\textbf{{canh}}-\textbf{{a.yo}}?\\
    originally  tall-\textsc{comm.neg}-\textsc{pol}\\
    \glt ‘She has been tall since birth, right?’ \citep[2788]{Yoon2010}
    \z
    \z

I was unable to find information on \isi{focus question}s in the literature available to me. The following examples were elicited from a native speaker in \isi{South Korea} via internet in April 2016. Focus was expressed in this \isi{case} with word initial position of the focused element. The \isi{analysis} roughly follows \citet{Song2005}.

\ea%12
    \label{ex:kore:12}
    \ili{Korean}\\
    \ea
    \gll nayil hakkyo-ey  ka-\textbf{{p}}\textbf{{nikka}}?\\
    tomorrow  school-\textsc{dir}  go-\textsc{q.def}\\
    \glt ‘Do (you) go to school tomorrow?’
    
    \ex
    \gll \uline{hakkyo-ey}   nayil    ka-\textbf{{pnikka}}?\\
    school-\textsc{dir}  tomorrow  go-\textsc{q.def}\\
    \glt ‘Do (you) go \textit{to school} tomorrow?’
    
    \ex
    \gll \uline{tangsin-un}  nayil    hakkyo    ka-\textbf{{pnikka}}?\\
    2\textsc{sg}-\textsc{top}  tomorrow  school    go-\textsc{q.def}\\
    \glt ‘Do \textit{you} go to school tomorrow?’ (elicited, slightly adjusted)
    \z
    \z 

\noindent Similar to \ili{Japanese}, the \isi{question marker} does not change its form and remains in sen\-tence-final position. A topic marker \textit{-(n)un} attaches to the focused pronoun in the last example that takes sentence initial position. The other sentences do not have an overt pronoun, as “Koreans tend to avoid second-person pronouns altogether” \citep[75]{Song2005}. The second sentence differs from the first in the sentence initial position of \textit{hakkyo-ey} (cf. \citealt{Song2005}: 107).

Available descriptions of \isi{questions} in \textbf{Jeju} are not very specific or detailed. \citegen[13f.]{Kiaer2014} otherwise good description only gives an unanalyzed list of 19 different \isi{interrogative} endings: \textit{-ka(ko)}, \textit{-n’ga(go)}, \textit{-nya}, \textit{-ne}, \textit{-nda}, \textit{-tia(ti)}, \textit{-lle}, \textit{-chi}, \textit{-k’o}/\textit{-llogo}, \textit{-ra}, \textit{-men}, \textit{-sǒ}, \textit{-an}, \textit{-sun}/\textit{-mnekka}, \textit{-ptega}, \textit{-ptegang}, \textit{-sugang}, \textit{-sukkwa(gwa)}, and \textit{-suga(kka)}. Unfortunately, there is no information on the semantic or pragmatic differences between all these suffixes and it is doubtful that they all simply mark \isi{questions}. One may only speculate that they fall within different registers that are based on politeness. The  interesting examples given by \citet{Kiaer2014} lack a morpheme \isi{analysis} and a \isi{glossing}, which makes their \isi{analysis} rather unclear. For instance, the three sentences in (\ref{ex:kore:13}) were all translated as ‘Where are you going?’.

\ea%13
    \label{ex:kore:13}
    \ili{Jeju}\\
    \ea
      ǒdi  kamdi?\\
    
    \ex
      ǒdŭi  kamini?\\
    
    \ex
      ǒdi  kamsini?\\
    \glt ‘Where are you going?’ (\citealt{Kiaer2014}: 14, 16, 17)
    \z
    \z 

\noindent The \isi{interrogative} \textit{ǒdi} corresponds, of course, to \ili{Korean} \textit{eti} ‘where (to)’ and \textit{ka-} in both languages means ‘to go’. The suffix \textit{‘-m} is considered to be a marker for the present tense but is better understood as an indicative marker (e.g., \citealt{Saltzman2014}: passim). The \isi{analysis} of \textit{‘-amsi} as a marker for progressive aspect is equally problematic. The final \textit{-ni} might be comparable to the plain question ending in \ili{Korean}. But neither \textit{-ni} nor \textit{-di} are listed as an \isi{interrogative} ending by Kiaer, who also leaves open the difference between \textit{ǒdi} and \textit{ǒdŭi} (maybe a typographic error). \citet{Sohn1999} provides a more complete \isi{analysis} of \ili{Jeju} \isi{interrogative} sentence enders, which is given in \tabref{tab:kore:4} below. Among these we find the two plain level question markers \textit{-(e)m-ti(ya)} and \textit{-(e)m-sini}, which correspond to \textit{-m-di} and \textit{-m-sini} in (\ref{ex:kore:13}a, \ref{ex:kore:13}c), but no correspondence to \textit{-mini} (\ref{ex:kore:13}b) was found. Possibly, \textit{-mi-ni} is the same ending as \textit{-(e)m-si-ni}, but without the suffix \textit{-si}. It may be noted that the expression ‘Where are you going?’ is a common greeting in \ili{Korean} that exists on different speech levels. In this expression the marker \textit{-si} is optional on all speech levels, which corroborates the \isi{analysis} of the \ili{Jeju} ending as \textit{-mi-ni}.

\ea%14
    \label{ex:kore:14}
    \ili{Korean}\\
    \ea
    \gll eti  ka-\textbf{{si.pnikka}}?\\
    where  go-\textsc{q}.\textsc{def}\\

    \ex
    \gll eti  ka-\textbf{{pnikka}}?\\
    where  go-\textsc{q}.\textsc{def}\\
    
    \ex
    \gll eti  ka-\textbf{{s(i).eyo}}?\\
    where  go-\textsc{q}.\textsc{pol}\\
    
    \ex
    \gll eti  ka-\textbf{{yo}}?\\
    where  go-\textsc{q}.\textsc{pol}\\
    
    \ex
    \gll eti  ka-\textbf{{si.o}}?\\
    where  go-\textsc{q}.\textsc{semf}\\
    
    \ex
    \gll eti  ka-\textbf{{o}}?\\
    where  go-\textsc{q}.\textsc{semf}\\
    
    \ex
    \gll eti  ka-\textbf{{si.na}}?\\
    where  go-\textsc{q}.\textsc{fam}\\
    
    \ex
    \gll eti  ka-\textbf{{na}}?\\
    where  go-\textsc{q}.\textsc{fam}\\
    
    \ex
    \gll eti  ka-\textbf{{si.e}}?\\
    where  go-\textsc{q}.\textsc{int}\\
    
    \ex
    \gll eti  ka-(\textbf{{a}})?\\
    where  go-(\textsc{q}.\textsc{int)}\\
    
    \ex
    \gll ?eti  ka-\textbf{{si.ni}}?\\
    where  go-\textsc{q}.\textsc{plain}\\
    
    \ex
    \gll eti  ka-\textbf{{ni}}?\\
    where  go-\textsc{q}.\textsc{plain}\\
    \glt ‘Where are you going?’ (L. \citealt{Brown2011}: 47; \citealt{Lee2000}: 264f.;   \citealt{Song2005}: 158; \citealt{YeonBrown2011}: 8)
    \z
    \z

\citet{Sohn1999} includes the following \ili{Jeju} example that corresponds functionally to the deferential speech level in the standard \ili{Korean} example (\ref{ex:kore:14}a) above.

\ea%15
    \label{ex:kore:15}
    \ili{Jeju}\\
    \gll \textbf{{etu}} ley  ka-m-swu-\textbf{{kkwa}}?\\
    where  to  go-\textsc{ind}-\textsc{ah}-\textsc{q}.\textsc{def}\\
    \glt ‘Where are you going?’ (\citealt{Sohn1999}: 75, from Lee I.S.)
    \z

\noindent \citet[49]{Saltzman2014} reanalyzed the sentence and calls \textit{ley} (\textit{-re} according to her) an ablative and \textit{-swu} (\textit{-su} in her rendering) a formal present tense marker, both of which are problematic. If \textit{ley} indeed functions as an ablative, the sentence should rather have been translated as something like ‘Where do you come from?’ In fact, according to \citet[75]{Sohn1999}, Standard \ili{Korean} may add the marker \textit{lo} instead of \textit{ley}. Clearly, this is the instrumental or directional \isi{case} marker \textit{(u)lo} and not an ablative \citep[115]{Song2005}. A comparable sentence from \ili{Jeju} in the past tense given by \citet{Kiaer2014} is the following:

\ea%16
    \label{ex:kore:16}
    \ili{Jeju}\\
    \gll ǒdi  ka-ng    wa-m-su-\textbf{{gwa}}?\\
    where  go-\textsc{pst}    ?\textsc{aux}-\textsc{ind}-\textsc{ah}-\textsc{q}.\textsc{def}\\
    \glt ‘Where did you go?’ (\citealt{Kiaer2014}: 10, my tentative \isi{analysis})
    \z

Here the marker \textit{-m-su-gwa} is the same as \textit{-m-su-kkwa} in \REF{ex:kore:15} above and corresponds to the standard \ili{Korean} deferential \isi{interrogative} \textit{-(su)p-ni-kka}. Note that \textit{-sup} (\textit{-p} when following a vowel) is an addressee honorific suffix, \textit{-ni} is an indicative marker and only \textit{-kka} is the actual \isi{question marker} \citep[341]{Sohn1994}. Thus, phonological differences apart, \ili{Jeju} \textit{-m-swu-kkwa} (\textit{-m-su-gwa}) and standard \ili{Korean} \textit{-sup-ni-kka} contain the same functional elements but apparently use the addressee honorific suffix and the indicative marker in reversed order.

Apart from \ili{Jeju}, other dialects have special sentence enders as well. \tabref{tab:kore:4} summarizes those dialectal \isi{interrogative} sentence enders that deviate from the standard language. Question marking in the Chungcheong dialect is very similar to Standard \ili{Korean}, but \textit{-o}, \textit{-e-yo}, and \textit{-sup-ni-kka} have the forms \textit{-wu}, \textit{-e-yu}, and \textit{-sup-ni-kkya} instead, which exhibit slight phonological differences. Other endings such as \textit{-nya} are identical:

\ea%17
    \label{ex:kore:17}
    \ili{Korean} (Chungcheong)\\
    \gll ni  pap  mek-ess-\textbf{{nya}}?\\
    2\textsc{sg}  meal  eat-\textsc{pst}-\textsc{q}.\textsc{plain}\\
    \glt ‘Did you have your meal?’ \citep[71]{Sohn1999}
    \z

\begin{table}
\caption{Selected interrogative sentence enders in Korean dialects based on \citet[66-76]{Sohn1999}; some dialectal forms identical to standard forms were excluded; see also \citet{Yeon2012}}
\label{tab:kore:4}

\begin{tabularx}{\textwidth}{lQQQ}
\lsptoprule
& \textbf{\textsc{plain}} & \textbf{\textsc{int}}\textbf{, \textsc{fam}}\textbf{, \textsc{semf}} & \textbf{\textsc{pol}}\textbf{, \textsc{def}}\\
\midrule
\ilit{Jeju} & -(e)m-ti(ya), -(e)m-sini, -esinya & -m-kka, -m-kko, [-em-se], -(e)m-singa & -(e)m-swu-kkwa, (-wu)-kkwa\\
Hamgyong & -wa, -m & -wu, [-m-mey], [-cipi], -nungka, -m-twu(ng) & -sswu-ta, -m-mengi, -p-syo, -p-m-mi-kka\\
Pyongan & -(u)m-ma, [-(u)wa]\footnotemark & [-(u)m-mey], -wu, -kan & -(u)op-ni-kka, -(su)p-ney-kka, -(su)p-mey-kka\\
Jeolla & -eya, -nya & [-elawu] & -(su)p-ni-kkye\\
Gyeongsang & PQ -na, \textsc{cop} -ka

CQ -no, \textsc{cop} -ko & -neng-kyo & (-(si)p)-ni-kk(y)e\\
Standard & -ni, -(n)u-nya & [-e], -na, [-(u)o] & [-e.yo], -(su)p-ni-kka\\
\lspbottomrule
\end{tabularx}
\end{table}

\footnotetext{ The declarative form is \textit{-(u)wa-yo}.}

Square brackets in \tabref{tab:kore:4} indicate forms that are not restricted to \isi{questions}. Some examples from the dialects follow.

\ea%18
    \label{ex:kore:18}
    \ili{Korean} (Hamgyong)\\
    \gll ka-\textbf{{wu}}?\\
    go-\textsc{q}.\textsc{fam}\\
    \glt ‘Does (she) go?’ \citep[67]{Sohn1999}
    \z

\newpage 
\ea%19
    \label{ex:kore:19}
    \ili{Korean} (Pyongan)\\
    \gll \textbf{{etu.m}} ey  ka-si-p-ney-\textbf{{kka}}?\\
    where  to  go-\textsc{sh}-\textsc{ah}-\textsc{ind}-\textsc{q}\\
    \glt ‘Where are you going?’ \citep[69]{Sohn1999}
    \z

\ea%20
    \label{ex:kore:20}
    \ili{Korean} (Jeolla)\\
    \gll ni  pap  muk-ess-\textbf{{nya}}?\\
    2\textsc{sg}  meal  eat-\textsc{pst}-\textsc{q}.\textsc{plain}\\
    \glt ‘Did you have your meal?’ \citep[74]{Sohn1999}
    \z

\noindent Several examples from the 19th century, mostly based on the Pyongan dialect \citep[238]{King1987}, can be found in the \textit{Corean Primer} by \citet{Ross1877}. For example, the ender \textit{-um-mê} in \REF{ex:kore:21} corresponds to \textit{-(u)m-mey} in modern Hamgyong and Pyongan dialects.

\ea%21
    \label{ex:kore:21}
    \ili{Korean} (Pyongan)\\
    \gll \textbf{{moosoon}} băpi  iss-\textbf{{um.mê}}?\\
    what    meal  \textsc{cop}-\textsc{q}.\textsc{fam}\\
    \glt ‘What food is there?’ (\citealt{Ross1877}: 13)
    \z

\noindent For other dialects equally old materials are not available to me.

There are differences in \isi{intonation} as well. In Jeolla and Chungcheong both falling and rising \isi{intonation} are possible, whereas the standard \ili{Korean} equivalent necessarily has rising \isi{intonation}. Polar questions in Gyeongsang generally have a falling \isi{intonation}. See \citet[66-76]{Sohn1999} and \citet{Jeon2015} for additional information.

\tabref{tab:kore:4} does not list forms encountered in \textbf{Yukcin} or \textbf{Kolyemal}. But some information on these dialects has been collected by Ross J. King. Instead of the standard \ili{Korean} \textit{-(su)p-ni-kka}, \ili{Yukcin} has \textit{-mdung} \citep[238]{King1987}, which appears to have a cognate in \ili{Kolyemal} \textit{-(ɨ)mdo} {\textasciitilde} \textit{-mdu} \citep[262]{King1987}. To my knowledge, no other \ili{Korean} dialect mentioned thus far has a comparable form (\tabref{tab:kore:4}). \ili{Kolyemal} furthermore has \textit{-na}, \textit{-o}, and \textit{-ja}, which correspond to Standard \ili{Korean} \textit{-na}, \textit{-o}, and \textit{-nya}, respectively. There are two polite markers, \textit{-ga} and \textit{-ge} that exhibit the same vowel difference as \textit{-a} {\textasciitilde} \textit{-e} in Standard \ili{Korean}. But their exact etymology and function remain unclear to me.

\ea%22
    \label{ex:kore:22}
    \ili{Korean} (\ili{Kolyemal})\\
    \ea
    \gll \textbf{{misi}}-ř    ha-\textbf{{ja}}?\\
    what-\textsc{acc}  do-\textsc{q}.\textsc{plain}\\
    \glt ‘What are you doing?’
    
    \ex
    \gll ka-\textbf{{mdo}}?\\
    go-\textsc{q}.\textsc{def}\\
    \glt ‘Are you going?’
    
    \ex
    \gll \textbf{{ɔdi}}-ř    ka-\textbf{{n.ga}}?\\
    where-\textsc{acc}  go-\textsc{q}.\textsc{pol}\\
    \glt ‘Where are you going?’ \citep[243, 262]{King1987}
    \z
    \z 

There is insufficient information on tag, \isi{focus}, and \isi{alternative question}s from the dialects. But like Standard \ili{Korean}, almost all dialects have the same marking in polar and content \isi{questions}. \textit{Gyeongsang} is exceptional among modern dialects in making a distinction between polar \textit{-no} and \isi{content question}s \textit{-na}. After copulas these markers take the forms \textit{-ko} and \textit{-ka}, but preserve the distinction between polar and content \isi{questions}. This distinction cannot be found in more honorific speech levels.

\ea%23
    \label{ex:kore:23}
    \ili{Korean} (Gyeongsang)\\
    \ea
    \gll ni \textbf{{etey}} ka-ss-\textbf{{no}}?\\
    2\textsc{sg}  where  go-\textsc{pst}-\textsc{q}.\textsc{plain}\\
    \glt ‘Where did you go?’
    
    \ex
    \gll i  ke \textbf{{nwu}} chayk  i-\textbf{{ko}}?\\
    this  ?\textsc{nom}  who  book  \textsc{cop}-\textsc{q}.\textsc{plain}\\
    \glt ‘Whose book is this?’
    
    \ex
    \gll pap  mun-\textbf{{na}}?\\
    meal  eat-\textsc{q}.\textsc{plain}\\
    \glt ‘Did you eat?’
    
    \ex
    \gll kuk  i  ni  chayk  i-\textbf{{ka}}?\\
    that  ?\textsc{nom}  2\textsc{sg}  book  \textsc{cop}-\textsc{q}.\textsc{plain}\\
    \glt ‘Is that your book?’ \citep[72]{Sohn1999}
    \z
    \z

\noindent This pattern is a relic from Middle \ili{Korean} that was lost in the other dialects during the \textbf{Pre-Modern Korean} period (\tabref{tab:kore:5}). More exactly, the Middle \ili{Korean} marker \textit{-ko} was replaced by \textit{-ka}, which from then on marked both polar and content \isi{questions} \citep[456]{Sohn2015}.

\begin{table}
\caption{Selected Pre-Modern Korean verb endings in the 19\textsuperscript{th} century \citep[456]{Sohn2015}}
\label{tab:kore:5}
\fittable{
\begin{tabular}{lllll}
\lsptoprule
& \textbf{Statements} & \textbf{Questions} & \textbf{Commands} & \textbf{Proposals}\\
\midrule
Plain & -ta/-la, -eta, -ma & -nja/-njo, -lja & -la, -ala/ela & -ca\\
Intimate & -ci &  & -a & \\
Familiar & -ney, -lsjej, -msjej & -nka/-nko & -kej & -sej\\
Semi-formal & -(s)o & -o, -lka & -o, -kwulje & \\
Polite & -(j)o &  &  & \\
Deferential & -ita, -olsita, -oyta & -iska, -pnójka/-pnika & -sjosje, -psio & -saita, -psita\\
\lspbottomrule
\end{tabular}
}
\end{table}

Several of these sentence enders still encountered in 19th century \ili{Korean} are no longer in use in modern Standard \ili{Korean}, e.g. the semi-formal \isi{interrogative} ending \textit{-lka}.

The difference between polar and \isi{content question}s was still present in \ili{Middle Korean}, which also had a further \isi{question marker} \textit{-ta} that was later lost. A good description of Middle \ili{Korean} \isi{question marking} and its relation to Contemporary \ili{Korean} (CK) was recently given by \citet[448]{Sohn2015}.

\begin{quote}
The \isi{interrogative} endings were (a) \textit{-(k)o/-sko}, (b) \textit{-(k)a/-ska}, and (c) \textit{-ta}. \textit{-(s)ko} occur[r]ed in question-word question sentences, and \textit{-(s)ka} in yes-no \isi{questions}. Both \textit{-(k)o} and \textit{-(k)a} also attached directly to a copula complement, as in \textit{i-nón sang-ka pel-a?} (CK \textit{i-nun sang i-nka pel-inka?}) ‘Is this a prize or a punishment?’ After the mood suffixes \textit{-ni} [indicative] and \textit{-li} [prospective], the endings \textit{-ko} and \textit{-ka} lost the consonant \textit{-k}, and became new question endings \textit{-nio/-njo} and \textit{-nia/-nja/-nje} on the one hand and \textit{-lio/-ljo} and \textit{-lia/-lja/-lje} on the other (CK \textit{-n}[\textit{j}]\textit{a/-ni}; \textit{-lya}). The question ender \textit{-ta}, which is obsolete in CK, was frequently used in a sentence whose subject is a second person, as in \textit{kutuj-nón enu cek-uj tolao-l-ta?} (CK \textit{kutay-nun encey tolao-keyss-eyo?}) ‘When will you return?’ The three-way (a, b, c) distinction has been lost in CK, except that the Gyeonsang dialect retains the \textit{-ko}/\textit{-ka} distinction. (slightly corrected)
\end{quote}

\noindent Consider the following examples that illustrate the markers \textit{-ka}, \textit{-ko}, and \textit{-ta}, respectively.

\ea%24
    \label{ex:kore:24}
    \ili{Middle Korean}\\
    \ea
    \gll i  twu  salóm  i  cinsillo    nej    hangkes-\textbf{{ka}}?\\
    this  two  person  \textsc{nom}  truly    2\textsc{sg}.\textsc{gen}  master-\textsc{q}\\
    \glt ‘Are these two persons truly your masters?’
    
    \ex
    \gll hjenljang-ón    sto \textbf{{mjes}} salóm-\textbf{{ko}}?\\
    wise.person-\textsc{top}  also  how.many  person-\textsc{q}\\
    \glt ‘Also, how many wise people were there?’
    
    \ex
    \gll kutuj-nón \textbf{{enu}} cek-uj    tolao-l-\textbf{{ta}}?\\
    2\textsc{sg-top}  which  time-\textsc{loc}  return-?\textsc{pros}-\textsc{q}\\
    \glt ‘When will you return?’ (\citealt{Sohn2012}: 102, 103)
    \z
    \z 

As can be seen from the example given in the above quotation, \isi{alternative question}s take two \isi{polar question} markers. For a better understanding, the example is analyzed in more detail in \REF{ex:kore:25}.

\ea%25
    \label{ex:kore:25}
    Middle \ili{Korean}\\
    \gll i-nón    sang-\textbf{{ka}} pel-\textbf{{a}}?\\
    this-\textsc{top}  prize-\textsc{q}    punishment-\textsc{q}\\
    \glt ‘Is this a prize or a punishment?’ \citep[102]{Sohn2012}
    \z

The complete set of \ili{Middle Korean} sentence enders is given in \tabref{tab:kore:6}. As in modern \ili{Korean}, there are four different sentence types, but only four speech levels.

\textbf{Old Korean} had two \isi{interrogative} sentence enders \textit{-ku} \zh{古}, \zh{遣}, \zh{故} and \textit{-ka} \zh{去}, too, but both marked \isi{polar question}s (\citealt{Nam2012}: 58f.). The distinction between polar (\textit{-kə} \zh{去}) and content (\textit{-ko} \zh{古}, \textit{-s.ko.a} \zh{叱濄}) \isi{question marker}s was only introduced in Late Old \ili{Korean} \citep[66]{Nam2012}. The \ili{Old Korean} question markers display a form similar to \ili{Tungusic} and \ili{Mongolic} on the one side (Old \ili{Korean} \textit{-ku}) and to \ili{Japonic} on the other (Old \ili{Korean} \textit{-ka}) (§§\ref{sec:5.10.2}, \ref{sec:5.8.2}, \ref{sec:5.6.2}). Question marking in the \ili{Jurchenic} branch of \ili{Tungusic} strongly differs from the other branches. There are more and different question markers and all have forms similar to \ili{Koreanic} (\tabref{tab:kore:7}).

\begin{table}
\caption{Middle Korean verb endings \citep[449]{Sohn2015}}
\label{tab:kore:6}

\begin{tabularx}{\textwidth}{Xllll}
\lsptoprule
& \textbf{Statements} & \textbf{Questions} & \textbf{Commands} & \textbf{Proposals}\\
\midrule
Plain & \textit{-ta/-la} & \textit{-ko/-ka, -nje/-njo, -ta} & \textit{-(ke)la} & \textit{-cje (-cela)}\\
Neutral & \textit{-ni/-noj} & \textit{-ni} & \textit{-kola/kolje} & ?\\
Moderate respect & \textit{-ng-ta} & \textit{-nó-ni-ska/sko} & \textit{-esje/-asje} & ?\\
Deferential & \textit{-ngi-ta} & \textit{-nó/ni-ngi-ska/sko} & \textit{-sjosje} & \textit{-sa-ngi-ta}\\
\lspbottomrule
\end{tabularx}
\end{table}

\begin{table}
\caption{Similar question markers in Middle Korean and Jurchenic (\sectref{sec:5.10.3})}
\label{tab:kore:7}

\begin{tabularx}{\textwidth}{Xl}
\lsptoprule

\textbf{Middle Korean} & \textbf{Manchu}\\
\midrule
-ni & =ni\\
-nia/-nio & =nio, ?Bala =ŋɔ\\
-(k)a/-(k)o & =o, \ili{Alchuka} =(k)ɔ\\
?-nja; ?\ili{Korean} -na; ?Gyeongsang -nA & =nA\\
\lspbottomrule
\end{tabularx}
\end{table}

The exact source and time of \isi{borrowing} remain unclear. But since Classical \ili{Manchu} already had all markers, they were borrowed before 1600. A major difference is that question markers replace declarative endings in \ili{Koreanic} but usually attach to them in \ili{Jurchenic} (note, however, forms such as \ili{Bala} \textit{ənə=ŋɔ} ‘go=\textsc{q}’). \ili{Manchu} \textit{=o} usually seems to follow copulas (free or bound), which also speaks in favor of a connection with \ili{Korean}. Remember that the Gyeongsang dialect has the form \textit{-ka} {\textasciitilde} \textit{-ko} following copulas, and \ili{Alchuka} \textit{=kɔ} preserves a velar plosive in this form as well. For instance, \ili{Manchu} \textit{-mbi=o} ‘-\textsc{ipfv}=\textsc{q}’ (containing the copula \textit{bi}) exactly corresponds to \ili{Alchuka} \textit{-mei=k}\textit{ɔ}. Similar to \ili{Korean} sentence enders, \ili{Jurchenic} markers may also attach to non-verbal elements but remain in sentence-final position, e.g. \ili{Bala} \textit{amin=ŋɔ} ‘father=\textsc{q}’. \ili{Korean} \textit{-o} is not restricted to \isi{questions} but may also mark imperatives, for instance, and \ili{Manchu} also has a polite imperative marker \textit{-rAo} that may contain the same element, possibly attached to the imperfective participle \textit{-rA} that also appears in the prohibitive \textit{ume} V\textit{-rA}. But more research with the help of large scale corpora is necessary to determine the exact meaning and use of those markers in \ili{Manchu}.

\subsection{Interrogatives in Koreanic}\label{sec:5.7.3}

\ili{Korean} interrogatives exhibit two dominant resonances, \textit{e{\textasciitilde}} and \textit{m{\textasciitilde}}. The first has previously been compared with \ili{Old Japanese} (\sectref{sec:5.6.3}). Similar to several other surrounding languages, the \isi{interrogative} ‘who’ does not belong to any of these groups but rather starts with \textit{n{\textasciitilde}}. In the Chungcheong dialect the \isi{resonance} \textit{e{\textasciitilde}} has the form \textit{we{\textasciitilde}} instead \citep[267]{King2006b}. \tabref{tab:kore:8} summarizes those interrogatives found in the literature available to me for Standard \ili{Korean}, \ili{Korean} as spoken in Jilin as well as \ili{Jeju}.

\citet[69]{Sohn1999} mentions a Pyongan \isi{interrogative verb} \textit{ekha} ‘to do how’ that he renders as the periphrastic sequence \textit{etheh-key ha-} in standard \ili{Korean} (\textit{ha-} ‘to do’). \ili{Jeju} has a periphrastic sequence \textit{ʌt̤}\textit{ʌŋ-ha}, too \citep[65]{Saltzman2014}. The \isi{interrogative} meaning ‘who’ is an amalgamation of the original \isi{interrogative} with the content \isi{question marker} \citep[456]{Sohn2015}. Note that \textit{nwuku} still has the nominative form \textit{nwu-ka} in Standard \ili{Korean}. The \isi{combination} \textit{enu-cey} ‘which time’ is the source of the contracted form \textit{encey} ‘when’ \citep[262]{Sohn1999}.

\begin{table}
\caption{Interrogatives from Korean (\citealt{Sohn1999}: 208ff., 256, 273, 396, 403; \citealt{Yoon2010}: 2784, in square brackets), Korean spoken in China (\citealt{XuanDewu1985}: 29, 161), and Jeju (\citealt{Kiaer2014}; \citealt{ChengHarrison2014}, in square brackets; \citealt{Saltzman2014}, in parentheses)}
\label{tab:kore:8}

\begin{tabularx}{\textwidth}{Xlll}
\lsptoprule
& \textbf{South Korea} & \textbf{Jilin} & \textbf{Jeju}\\
\midrule
who & nwu(ku), nwu-ka \textsc{nom} & nuku & [nuge]\\
what & mues & muɣəs & musin’gŏ {\textasciitilde} musigŏ, [musinggeo]\\
what + noun & musun &  & \\
what + noun (time) & myech & mjətʃ‘ & \\
which & enu & ənɯ & ŏnŭgŏ, \{ʌnɨ\}\\
what kind of & etten &  & \\
how & ecci, ettehkey &  & \{ʌ{t̤}ʌŋ\}\\
where & eti & əti & ŏdi, \{ʌtɨ\}\\
when & encey & əntʃe & \\
how much & elma & ərma & \\
how long & [elmana] &  & \\
why & way &  & \\
what sort of & weyn &  & \\
\lspbottomrule
\end{tabularx}
\end{table}

There are also several forms meaning ‘who’ that are a \isi{combination} of \textit{enu-} with one of the three bound nouns \textit{ay} ‘child’, \textit{salam} ‘person’, and \textit{pun} ‘respected person’ (\citealt{Sohn1999}: 207f., passim; \citealt{Song2005}: 73, passim). The \isi{interrogative} \textit{musun} is derived from \textit{mues-i-n} ‘what-\textsc{cop}-\textsc{rel}’ and \textit{etten} from \textit{e-tte-ha-n} ‘which-kind-\textsc{cop}-\textsc{rel}’ \citep[256]{Sohn1999}. \ili{Korean} \textit{weyn} similarly derives from \textit{way-i-n}. A form without the relative marker \textit{-n} but with the adverbializer \textit{-}\textit{key} ‘so that, to’ is probably the source of \textit{etteh-}\textit{key} ‘how’ (cf. \citealt{Sohn1999}: 376). In these forms \textit{e-} seems to be the actual \isi{interrogative} marker that must also be the ultimate source of \textit{elma}, \textit{ecci}, \textit{enu}, and \textit{eti}. In \ili{Jeju} the interrogatives \textit{musi(n’)-gŏ} ‘what’ (\ili{Korean} \textit{mues}) and \textit{ŏnŭ-gŏ} ‘which’ (\ili{Korean} \textit{enu}) contain a suffix \textit{-gŏ} that could correspond to \ili{Korean} \textit{kes} ‘thing’ or \textit{-kes-i} ‘-thing-\textsc{nom}’, which is regularly pronounced \textit{-key} in informal speech \citep[155]{Song2005}. This assumption is corroborated with data from \textit{Kolyemal}, among which we find \textit{misi-ge} ‘what thing’. That in \ili{Kolyemal} \textit{ɔndʒe-ge} ‘when’ (\ili{Korean} \textit{encey}) the same suffix is present is unlikely from a functional perspective. \tabref{tab:kore:9} summarizes \ili{Kolyemal} interrogatives and their direct \ili{Korean} cognates.

\begin{table}
\caption{Kolyemal interrogatives in comparison with Korean (\citealt{King1987}: 263; \citealt{Sohn1999})}
\label{tab:kore:9}

\begin{tabularx}{\textwidth}{XXl}
\lsptoprule
& \textbf{Kolyemal} & \textbf{Korean}\\
\midrule
who & nugi & nwu(ku)\\
who (\textsc{acc/dir}) & nugi-ř & nwukwu-l(ul)\\
how many & me(t) & myech\\
what (thing) & misi-\textbf{ge} & mues\\
what (\textsc{acc/dir}) & misi-ř & mues-ul (> mwe-l)\\
what kind of & musun & musun\\
where & ɔdɨ-\textbf{mæ} & eti\\
whither (\textsc{acc/dir}) & ɔdi-ř & eti-lo\\
whence (\textsc{abl}) & ɔdi-sɔ & ?eti-eyse\\
when (thing?) & ɔndʒe-\textbf{ge} & encey\\
why & ɔtʃtʃæ & ecci\\
how & ɔttɔk\textsuperscript{h}æ & ettehkey\\
\lspbottomrule
\end{tabularx}
\end{table}

The \isi{resonance} \textit{e{\textasciitilde}} in \ili{Korean} has the form \textit{ɔ{\textasciitilde}} in \ili{Kolyemal}. The \isi{case} suffix \textit{-ř} combines the function of a directive with that of an accusative, as can be seen in \textit{nugi-ř} ‘what-\textsc{acc}’ but \textit{ɔdi-ř} ‘where-\textsc{dir}’. In \ili{Korean} both the accusative \textit{-(l)ul} and the instrumental \textit{-(u)lo} also have the function of a directive, but the first is likely the source of \ili{Kolyemal} \textit{-ř} (\citealt{Song2005}: 112, 115). \ili{Kolyemal} \textit{ɔdɨ-mæ}, like Pyongan \textit{etu-m} in example (19) above, derives from \ili{Middle Korean} \textit{etu-mej} (see below).

Similar to \ili{Japanese} (\sectref{sec:5.6.3}), \ili{Korean} displays parallel paradigms in \textit{demonstratives} and one \isi{interrogative} stem. Like \ili{Japanese} (\textit{ko-}, \textit{so-}, and \textit{a-}, older \textit{ka-}), \ili{Korean} has a three way distinction of \isi{demonstratives} (\textit{i}, \textit{ku}, and \textit{ce}). But while \ili{Japanese} has exactly the same paradigms for the \isi{interrogative} stem \textit{do-}, the paradigm of \ili{Korean} \textit{e-} exhibits several irregularities (\tabref{tab:kore:10}).

\begin{table}
\caption{Full paradigms of Korean demonstratives and the selective interrogative \citep[296]{Sohn1994}}
\label{tab:kore:10}

\begin{tabularx}{\textwidth}{XXXXl}
\lsptoprule
& \textbf{\textsc{prox}} & \textbf{\textsc{prox.h}} & \textbf{\textsc{dist}} & \textbf{which}\\
\midrule
stem & i & ku & ce & e-\textbf{nu}\\
\textit{eki} ‘place’ & \textbf{yeki} & \textbf{keki} & \textbf{ceki} & \textbf{eti}\\
\textit{kes} ‘thing’ & i kes & ku kes & ce kes & enu kes\\
\textit{il} ‘thing, fact’ & i il & ku il & ce il & enu il\\
\textit{kos} ‘place’ & i kos & ku kos & ce kos & enu kos\\
direction & i ccok & ku ccok & ce ccok & enu ccok\\
way & i-li & ku-li & ce-li & \textbf{eti-lo}\\
kind of & i-le-n & ku-le-n & ce-le-n & \textbf{e-tte}-n\\
to be ... way & i-leh-ta & ku-leh-ta & ce-leh-ta & \textbf{e-tteh}-ta\\
to do ... way & i-le-n-ta & ku-le-n-ta & ce-le-n-ta & -\\
\lspbottomrule
\end{tabularx}
\end{table}

The paradigms not only contain \isi{case} endings but also certain bound nouns that have typological and probably areal parallels in \ili{Manchu} (\sectref{sec:5.10.3}). Unlike the adverbs \textit{yeki} ‘here’, \textit{keki} ‘there’, and \textit{yeki} ‘over there’, which are based on the demonstrative stems in \isi{combination} with \textit{eki} ‘place’, the \isi{interrogative} \textit{eti} ‘where’ has a \isi{case} marker \textit{-ti}. In \ili{Jeju}, this suffix can also be found in the \isi{demonstratives} (\tabref{tab:kore:11}).

\begin{table}
\caption{Jeju demonstratives and the selective interrogative in neutral and locative form \citep[21]{Saltzman2014}}
\label{tab:kore:11}

\begin{tabularx}{\textwidth}{XXXXl}
\lsptoprule
& \textbf{\textsc{prox}} & \textbf{\textsc{prox.h}} & \textbf{\textsc{dist}} & \textbf{which}\\
\midrule
selective & i {\textasciitilde} jo/jʌ & kɨ & tɕʌ & ʌ-\textbf{nɨ}\\
locative & \textbf{jo}-ti & kɨ-ti & tɕʌ-ti & ʌ-ti\\
\lspbottomrule
\end{tabularx}
\end{table}

As regards the irregular \ili{Jeju} stem \textit{jo-}, note that \ili{Korean} also has the diminutive demonstrative stems \textit{yo}, \textit{ko}, and \textit{co} \citep[114]{Sohn1994}.

The \isi{interrogative} \textit{enu} ‘which’ is likely analyzable and based on the stem \textit{e-}. The ending \textit{-nu} might, according to \citet[322]{Vovin2005}, have a connection to a \ili{Japanese} attributive ending (\ili{Old Japanese} \textit{-nö}). While the \ili{Jeju} \isi{interrogative} \textit{ʌti} ‘where’ can, at least synchronically, be analyzed as \textit{ʌ-ti} ‘which-\textsc{loc}’, this is probably not true for \ili{Korean} \textit{eti}. Diachronically, however, both \ili{Jeju} \textit{ʌti} and \ili{Korean} \textit{eti} go back to \ili{Middle Korean} \textit{e-tuj}, the second part of which is a bound noun meaning ‘place’. \citet[322]{Vovin2005} assumes that the form can be reconstructed as Proto-\ili{Korean}(ic) *\textit{èntúy}, thus allowing an \isi{analysis} of the first part as the forerunner of \ili{Korean} \textit{enu} ‘which’ and a connection with \ili{Proto-Japonic} *\textit{entu} ‘where’. His reasoning is based on the fact that the \textit{t} should have regularly changed to \textit{l} in this position without the \textit{n} present. \ili{Middle Korean} furthermore has an extended form \textit{etu-mej} ‘where’ that might be comparable to \ili{Old Ryūkyūan} \textit{idu-ma} (\sectref{sec:5.6.3}).

In general, the set of \ili{Middle Korean} interrogatives is very similar to modern \ili{Korean}, only one form (\textit{hjen} ‘how many’) having been entirely lost (\tabref{tab:kore:12}). The exact differences between the forms meaning ‘what’ remain unclear to me.

\citet[319]{Vovin2005} mentions an additional \ili{Middle Korean} form \textit{e:styé} {\textasciitilde} \textit{e:sté} {\textasciitilde} \textit{e:styéy} ‘how’ that, according to him, goes back to *\textit{e-is-ti} ‘how-exist-\textsc{adv}’.

\newpage %solid chapter border

\begin{table}[t]
\caption{Middle Korean interrogatives \citep[98]{Sohn2015} in comparison with modern Korean}
\label{tab:kore:12}
\begin{tabularx}{\textwidth}{Xll}
\lsptoprule
& \textbf{Middle Korean} & \textbf{Korean}\\
\midrule
who & nwu(-ko {\textasciitilde} -kwu) & nwuku {\textasciitilde} nwu-\\
how many & mjes & myech\\
what & musus, muzus, musu, musuk, musum & mues, musun\\
where & etuj & eti\\
where & etu-mej & etu-m (Pyongan)\\
whither & etu-le & eti-lo\\
which & enu, enwu, enó & enu\\
how much & enma & elma\\
when & enu-cjej & encey\\
how many & \textbf{hjen} & -\\
\lspbottomrule
\end{tabularx}
\end{table}
