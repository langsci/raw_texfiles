\section{Mongolic}\label{sec:5.8}
\subsection{Classification of Mongolic}\label{sec:5.8.1}

\ili{Mongolic} languages form a \isi{language family} with about a dozen modern members. According to \citet[232]{Janhunen2006} they may be classified as in \REF{exfig:mong:1}. \cite[388-389]{Rybatzki2003b} assumes a slightly different classification with six groups. Of these, the so-called Northern (\ili{Khamnigan Mongol}, \ili{Buryat}) and South-Central groups (\ili{Shira Yughur}) are part of Central \ili{Mongolic} and \ili{Shirongolic}, respectively, in \citegen{Janhunen2006} classification.

% \ea\upshape%1
\begin{figure}
\caption{Classification of Mongolic}
    \label{exfig:mong:1}
\begin{forest}  for tree={grow'=east,delay={where content={}{shape=coordinate}{}}},   forked edges  
[
    [\ili{Dagur}(ic)
        [\ili{Dagur}]
    ]
    [Central \ili{Mongolic}
        [\ili{Khamnigan Mongol}]
        [\ili{Buryat}]
        [Mongol proper]
        [\ili{Ordos}]
        [\ili{Oirat}]
    ]
    [\ili{Shirongolic}
    	[\ili{Shira Yughur}]
        [Huzhu \ili{Mongghul}]
        [Minhe \ili{Mangghuer}]
        [\ili{Bonan}]
        [\ili{Kangjia}]
        [\ili{Santa}]
    ]
    [\ili{Moghol}(ic)
    	[\ili{Moghol}]
    ]
]
\end{forest}   
%     \z
\end{figure}

The two classifications agree, however, in the number of languages as well as in some details such as the isolated positions of \ili{Dagur} and \ili{Moghol}. A recently discovered language that was added to \REF{exfig:mong:1} is called \ili{Kangjia} and belongs to the \ili{Shirongolic} branch. According to \citet[347]{Kim2003}, “the \ili{Kangjia} ‘language’ would appear to be intermediate between \ili{Bonan} and \ili{Santa}.” However, it may actually be more closely related to \ili{Bonan} \citep[66]{Siqinchaoketu2002}. Central \ili{Mongolic} has also been called Common \ili{Mongolic} by \cite[3f.]{Janhunen2012c}, and is said to contain also the \ili{Khorchin} group of dialects that was not listed as a separate entry. \ili{Khorchin} is spoken in western parts of \isi{Manchuria} (the modern provinces of Heilongjiang, Jilin, and Liaoning), but mostly in the adjacent parts of eastern Inner \isi{Mongolia} \citep[4]{Janhunen2012c}. \ili{Chakhar \ili{Mongolian}}, not listed above, belongs to the same branch as \ili{Khalkha}. It is spoken in Inner \isi{Mongolia} and is said to be the language spoken by the descendants of the last emperor of the \ili{Mongolian} Yuan dynasty and his followers who fled from Peking in 1368 \citep[1]{Sechenbaatar2003}. Kalmyk, also not mentioned, can be considered an aberrant dialect of \ili{Oirat} and is the only \ili{Mongolic} language located in \isi{Europe}. \ili{Moghol}, located in Afghanistan, is probably extinct today and will for the most part be excluded here.

\newpage 
The \ili{Mongolic} language \ili{Shira Yughur} or Eastern Yughur (\textit{d\=o}\textit{ngbù yùgù yǔ} \zh{东部裕固语} in \ili{Chinese}) should not be confused with the \ili{Turkic} language Yellow \ili{Uyghur} that is also called Sarig or Western Yughur (\textit{x\={\i}bù yùgù yǔ} \zh{西部裕固语} in \ili{Chinese}, see \sectref{sec:5.11}). There are also different \ili{Chinese} designations for \ili{Bonan} (\textit{bǎo\=an yǔ} \zh{保安语}), \ili{Santa} (\textit{d\=o}\textit{ngxi\=ang yǔ} \zh{东乡语}), and a collective name for Huzhu \ili{Mongghul} and Minhe \ili{Mangghuer} (\textit{tǔzú yǔ} \zh{土族语}), which are also known as Monguor in the West. Of the languages mentioned in \REF{exfig:mong:1} only \ili{Moghol} is located outside \isi{Northeast Asia}. All \ili{Mongolic} languages except for \ili{Buryat} and Kalmyk, which are for the most part spoken in \isi{Russia} as well as \ili{Moghol} in Afghanistan, are located within \isi{Mongolia} and \isi{China}.

\subsection{Question marking in Mongolic}\label{sec:5.8.2}

Question marking in \ili{Mongolic} is not very complex. \citet[27]{Janhunen2003a} gives a good summary of the marking of \isi{questions} in \ili{Proto-Mongolic}.

\begin{quote}
When no \isi{interrogative} pronoun or pronominal verb was present in the sentence, \textit{interrogation} in Proto-\ili{Mongolic} was expressed by a sentence-final \isi{interrogative} particle, which may be reconstructed as either *\textit{gü} (> *=\textit{gU}), as in \ili{Buryat} and \ili{Khamnigan Mongol}, or *\textit{xU} (> *=\textit{UU}), as in most other \ili{Mongolic} languages. In \isi{questions} containing an \isi{interrogative} word, no particle was originally needed, but in Common \ili{Mongolic} the copular form *\textit{bü-}(\textit{y})\textit{i} > *\textit{büi} ‘being, present’ was grammaticalized in such sentences into what may be termed a \textit{corrogative} particle.
\end{quote}

Consider the following examples from Written \ili{Mongolian} in which the forms are still relatively well preserved.

\ea%2
    \label{ex:mong:2}
     Written \ili{Mongolian}\\
    \ea
    \gll {ta} {sayin=}\textbf{{uu}}?\\
    2\textsc{pl}  good=\textsc{q}\\
    \glt ‘Are you well?’
    
    \ex
    \gll {ta} \textbf{{ken}} \textbf{{bui}}?\\
    2\textsc{pl}  who  \textsc{cop}>\textsc{q}\\
    \glt ‘Who are you?’ (\citealt[53]{Janhunen2003b}, transcription changed)
    \z
    \z

The marking of \isi{polar question}s with a sentence-final clitic is, of course, an areal trait. The development of a marker in \isi{content question}s, on the other hand, sets \ili{Mongolic} apart from most languages of the area. But similarly there exists a special content \isi{question marker} in some \ili{Turkic} languages that shares a functional background in a copula (\sectref{sec:5.11.2} and \sectref{sec:6}).

In modern \ili{Mongolic} languages the question markers have gone through phonetic erosion. As we will see further below the polar \isi{question marker} fused with certain verb endings and copulas, especially in \ili{Shirongolic} languages. Some individual \ili{Mongolic} languages have additionally adopted question markers from other languages. Consider the following examples from a variety of \ili{Buryat} spoken in \isi{China}.

\ea%3
    \label{ex:mong:3}
    \ili{Buryat} (Shineken)\\
    \ea
    \gll {xugsʲ}in-tei    oolz-aa=\textbf{{g}}=sʲ{a?}\\
    old.woman-\textsc{com}  meet-\textsc{p.ipfv}=\textsc{q}=2\textsc{sg}\\
    \glt ‘Did you meet the old woman?’
    
    \ex
    \gll \textbf{{xen}}-tei    oolz-aa=\textbf{{b}}{=sʲ}{a?}\\
    who-\textsc{com}  meet-\textsc{p.ipfv}=\textsc{q}=2\textsc{sg}\\
    \glt ‘Whom did you meet?’
    
    \ex
    \gll {sʲ}ii  nam-tai  ɔsʲ-nɔ=\textbf{{ba}}?\\
    2\textsc{sg}  1\textsc{sg}.\textsc{obl}-\textsc{com}  reach-\textsc{prs}=\textsc{q}\\
    \glt ‘You go with me, don’t you?’ (\citealt{Yamakoshi2011a}: 170-171)
    \z
    \z 

\noindent Sentence (\ref{ex:mong:3}a) illustrates the polar \isi{question marker} \textit{=go {\textasciitilde} =gu} {\textasciitilde} =\textit{g}, sentence (\ref{ex:mong:3}b) the optional “corrogative” particle \textit{=be} {\textasciitilde} \textit{=b}, and sentence (\ref{ex:mong:3}c) the marker \textit{=ba}, which is a recent \isi{borrowing} from \ili{Chinese} \textit{ba} \zh{吧} that can be found in several languages of \isi{China} (\sectref{sec:6}, and \sectref{sec:5.9.2.1}). In Shineken \ili{Buryat} \textit{=ba} is mutually exclusive with the agreement marker. Alternative questions display \isi{double marking} with \textit{=go {\textasciitilde} =gu} {\textasciitilde} =\textit{g}.

\ea%4
    \label{ex:mong:4}
    \ili{Buryat} (Shineken)\\
    \gll {bii} {enee-g-uur-ee} {jab-xa=}\textbf{{g}}=bi, teree-g-uur-ee jab-xa=\textbf{{g}}{=bi?}\\
    1\textsc{sg.nom}  this-\textsc{e}-\textsc{inst}-\textsc{refl}  go-\textsc{p.fut}=\textsc{q}=1\textsc{sg} that-\textsc{e}-\textsc{inst}-\textsc{refl}  go-\textsc{p.fut}=\textsc{q}=1\textsc{sg}\\
    \glt ‘Should I go in this or in that direction?’ (\citealt{Yamakoshi2006}: 153)
    \z

In non-verbal sentences the content \isi{question marker} can also attach to word classes such as adjectives and interrogatives.

\ea%5
    \label{ex:mong:5}
    \ili{Buryat} (Shineken)\\
    \ea
    \gll {sʲ}inii xubuun=sʲe \textbf{{alin}}=\textbf{{be}}?\\
    2\textsc{sg.gen} boy.\textsc{nom}=2\textsc{sg}.\textsc{poss}  which=\textsc{q}\\
    \glt ‘Where is your boy?’ (\citealt{Yamakoshi2011b}: 116, shortened)
    
    \ex
    \gll \textbf{{al}}\textbf{{ʲ}}\textbf{{an}}=in    hain=\textbf{{be}}?\\
    which=3\textsc{sg}.\textsc{poss}  good=\textsc{q}\\
    \glt ‘Which one is good?’ \citep[5]{Yamakoshi2007b}
    \z
    \z

\citet{Janhunen2003a} appears to believe that the \isi{question marker} in \ili{Buryat} and \ili{Khamnigan Mongol} has a different origin than the one found in other \ili{Mongolic} languages. Interestingly, both \ili{Buryat} and \ili{Khamnigan Mongol} had intense \isi{contact} with dialects of the \ili{Tungusic} language \ili{Evenki}. In both Khamnigan \ili{Evenki} and \ili{Khamnigan Mongol} the enclitic has the form \textit{=gv}. \citet[95]{Janhunen1991} speculated that it may have been borrowed from one language to the other, but left the direction of \isi{borrowing} open. Given that many \ili{Tungusic} languages preserve a cognate of the enclitic in Khamnigan \ili{Evenki} (see \sectref{sec:5.10.2}), it seems likely that it was borrowed from \ili{Evenki} into \ili{Khamnigan Mongol}. But Khamnigan \ili{Evenki} may reflect influence from \ili{Khamnigan Mongol}, and in turn has lost the property of consonant alternation that is still present in \ili{Evenki} proper (\textit{=gu} {\textasciitilde} \textit{=ku} {\textasciitilde} \textit{=ŋu} {\textasciitilde} \textit{=vu}). The enclitic \textit{=gi(i)} in the \ili{Tungusic} language \ili{Solon}, on the other hand, is probably a secondary loan from a \ili{Mongolic} source (possibly \ili{Buryat} \textit{=gü}). Apart from the not unlikely scenario that individual \ili{Mongolic} languages have borrowed the \ili{Tungusic} \isi{question marker}, the other \ili{Mongolic} \isi{question marker} reconstructed by Janhunen as *\textit{xU}, could potentially also have a very old connection to \ili{Proto-Tungusic} *\textit{Ku} because it already existed at the proto-level of both language families. As is often the case, the etymology of the markers is not \isi{transparent} in either \ili{Mongolic} or \ili{Tungusic}. Also note a similar marker \textit{-ku} (written as \zh{古, 遣, 故}) in Old \ili{Korean} (\sectref{sec:5.7.2}).

The form of the \isi{question marker} in \ili{Middle Mongol} was probably \textit{=UU}, that is \textit{=üü} {\textasciitilde} \textit{=uu}. In Written Mongol (Uyghur script), the enclitic has the form \textit{-(ju)gu} {\textasciitilde} \textit{-(ju)qu} {\textasciitilde} \textit{-(ju)qhu} when following vowels and \textit{-ugu} {\textasciitilde} \textit{-uqu} {\textasciitilde} \textit{-uqhu} otherwise \citep[79]{Rybatzki2003a}. According to \cite[45]{Street2008}, the plosives were not present in the spoken language but rather indicated a hiatus, which can be seen from other scripts used to write \ili{Middle Mongol}. The \isi{vowel harmony} may represent a problem for the comparison with \ili{Tungusic}, but the older records of \ili{Middle Mongol} show a strong functional \isi{similarity} to \ili{Tungusic}. While the enclitic has a strict sentence-final position in modern \ili{Mongolian}, it was mobile at earlier stages and could attach to a focused element. In other words, the functional scope included not only polar but also \isi{focus question}s.

\begin{quote}
The \isi{interrogative} particle in early Middle \ili{Mongolian} was what may be termed a \textbf{floating particle}: for purposes of \isi{emphasis} it could float from one point to another on the surface structure of a sentence, though at a deeper level remaining in construction with the remainder of the sentence as a whole [i.e., marking the whole sentence as question]. (\citealt{Street2008}: 76, my square brackets)
\end{quote}

\noindent A typological parallel for a change from a mobile to a sentence-final question particle can be observed in the transition from Old to Modern \ili{Japanese} (\sectref{sec:5.6.2}). In Middle \ili{Mongolian} \isi{alternative question}s were also marked with the same enclitic that attached once on each alternative. Consider the following examples from \ili{Middle Mongol}.

\ea%6
    \label{ex:mong:6}
    \ili{Middle Mongol} (Arabic script; Secret History)\\
    \ea
    \gll {burut-b=}\textbf{{uu}}?\\
    escape-\textsc{term}=\textsc{q}\\
    \glt ‘Did (you) escape?’
    
    \ex
    \gll {caq=}\textbf{{u’u}} {gür-be?}\\
    time=\textsc{q}    arrive-\textsc{term}\\
    \glt ‘Has \textit{the time} arrived?’
    
    \ex
    \gll {jöb=}\textbf{{ü’ü}} {tab=}\textbf{{u’u}} {ügüle-rün?}\\
    appropriate=\textsc{q}  ?convenient=\textsc{q}    say-\textsc{cvb}\\
    \glt ‘Saying: is it appropriate, is it convenient?’ \citep[79]{Rybatzki2003a}
    \z
    \z 

The same functional scope can be reconstructed for \ili{Proto-Tungusic} (\sectref{sec:5.10.2}). Furthermore, the two \isi{proto-languages} combine this with a similar phonological shape, which is unlikely to be a coincidence.

As indicated by Janhunen in the above quotation, the etymology of the marker *\textit{büi} is \isi{transparent} and has its origin in a participle form of the copula *\textit{bü-}, most likely the so-called deductive *\textit{-(y)i} ‘\textsc{prs}.\textsc{ipfv}’ \citep[24]{Janhunen2003a}. The term “corrogative” is frequently employed by Janhunen but has never been explained adequately from a functional perspective or in terms of \isi{grammaticalization}. According to the \isi{analysis} followed in this book, it may simply be called a content \isi{question marker}. While in \ili{Mongolian} it has an eroded form similar to \ili{Buryat}, it may also appear in a form that is still identical to the copula.

\ea%7
    \label{ex:mong:7}
    \ili{Mongolian}\\
    \gll {cii} \textbf{{xedzee}} {yab-sen} \textbf{{bwai}}?\\
    2\textsc{sg}  when  depart-\textsc{pst}  \textsc{cop}>\textsc{q}\\
    \glt ‘When did you go?’ \citep[255]{Janhunen2012c}
    \z

\noindent As noted above, a similar content \isi{question marker} exists in some surrounding \ili{Turkic} languages that has its origin in a copula that in turn goes back to a demonstrative (\sectref{sec:5.11.2}).

\textbf{Dagur} differs from other \ili{Mongolic} languages in that there is a different polar \isi{question marker}. There is no, or at least no obligatory, content \isi{question marker}.

\ea%8
    \label{ex:mong:8}
    \ili{Dagur}\\
    \ea
    \gll en  bitig=\textbf{{yee}}?\\
    this  book=\textsc{q}\\
    \glt ‘Is this a book?’
    
    \ex
    \gll {ʃii} \textbf{{ani}}{-ʃi?}\\
    2\textsc{sg}  who-2\textsc{sg}\\
    \glt ‘Who are you?’ (\citealt{Tsumagari2003}: 150; \citealt{Chaolu1994c}: 11)
    \z
    \z

The data by \citet{ZhongSuchun1982}, collected in 1963 in Morin Daba, show a similar situation but make it clear that the polar \isi{question marker} can be used optionally in \isi{content question}s, too. This indicates that it is not only formally, but also functionally different from other \ili{Mongolic} languages. Apparently, \ili{Dagur} also has borrowed \ili{Chinese} \textit{ba} \zh{吧}.

\ea%9
    \label{ex:mong:9}
    \ili{Dagur} (Morin Daba)\\
    \ea
    \gll ənə  warkəl-ii  waa-səŋ=\textbf{{jəə}}?\\
    this  clothes-\textsc{acc}  wash-\textsc{pst}=\textsc{q}\\
    \glt ‘Have (you) washed these clothes?’
    
    \ex
    \gll \textbf{{xaanə}} {itʃ-bəi-ʃii=}\textbf{{jəə}}?\\
    whither  go-\textsc{fut}-2\textsc{sg}=\textsc{q}\\
    \glt ‘Where are you going?’
    
    \ex
    \gll daŋg    oo-dʒ      ul  bol-səŋ \textbf{{baa}}?\\
    tobacco  drink-\textsc{cvb.ipfv}  \textsc{neg}  can-\textsc{pst}  \textsc{q}\\
    \glt ‘Smoking is prohibited here, right?’ (\citealt{ZhongSuchun1982}: 76)
    \z
    \z 

There is one example of an \isi{alternative question} that exhibits the marker \textit{jumoo} once on each alternative. This is probably a recent loan from an Inner \ili{Mongolian} dialect, in which the latter part is the \isi{question marker} \textit{=UU}, that will be further explained below.

\newpage 
\ea%10
    \label{ex:mong:10}
    \ili{Dagur}\\
    \gll bii  əidəə {jaw-oo}{s-minʲ} dʒuɣi-ɣu \textbf{{jum}}\textbf{.}\textbf{{o}}\textbf{{o}}? {tiidaa} {jaw-oo}{s-minʲ} {dʒuɣi-ɣu} \textbf{{jum}}\textbf{.}\textbf{{o}}\textbf{{o}}?\\
    1\textsc{sg}  this.way  go-\textsc{cvb.cond}-1\textsc{sg}  right-\textsc{ipfv}  \textsc{q} that.way  go-\textsc{cvb.cond}-1\textsc{sg}    right-\textsc{ipfv}  \textsc{q}\\
    \glt ‘Will I go this way or that way?’ (\citealt{Chaolu1994c}: 18)
    \z

In the \ili{Dagur} dialect spoken in Tarbagatai (\textit{tǎchéng} \zh{塔城}) in \isi{Xinjiang}, the usual \isi{polar question} marker has the vowel-harmonic forms \textit{-ja} {\textasciitilde} \textit{-jə} {\textasciitilde} \textit{-jo} and also marks alternative as well as \isi{content question}s. It remains unclear whether it can also be found in \isi{focus question}s. The marker was given as a suffix but is reanalyzed as an enclitic here.

\ea%11
    \label{ex:mong:11}
    \ili{Dagur} (Tacheng)\\
    \ea
    \gll ərgun-šin  xaǰir-səŋ=\textbf{{ja}}?\\
    husband-2\textsc{sg}  come.back-\textsc{pst}=\textsc{q}\\
    \glt ‘Did your husband come back?’
    
    \ex
    \gll {səəs-šin} \textbf{{xərəə}},    bəraan=\textbf{{ja}} {očog=}\textbf{{jo}}?\\
    urine-2\textsc{sg}  how.much  much=\textsc{q}  little=\textsc{q}\\
    \glt ‘How much urine do you have? Is it a lot or a little?’
    
    \ex
    \gll {šii} \textbf{{xəǰəə}} ənd  ir-səŋ-ši=\textbf{{jə}}?\\
    2\textsc{sg}  when  here  come-\textsc{pst}-2\textsc{sg}=\textsc{q}\\
    \glt ‘When did you come here?’ (\citealt{Yu2008}: 85, 86)
    \z
    \z 

\noindent The functional scope of the \isi{question marker} in \ili{Dagur} suggests an areal connection to several surrounding languages (\sectref{sec:6}).

In some \isi{content question}s there is a copula that could be the “corrogative” form found in other \ili{Mongolic} languages. As in Shineken \ili{Buryat} the agreement marker follows the copula, but in \ili{Dagur} the sentence additionally takes the usual \isi{question marker}, which makes it unlikely that the copula fulfills the role of a \isi{question marker}.

\ea%12
    \label{ex:mong:12}
    \ili{Dagur} (Tacheng)\\
    \gll {šii} \textbf{{xaan}}-aar  ir-səŋ-\textbf{{b}}{-ši=}\textbf{{jə}}?\\
    2\textsc{sg}  where-\textsc{abl}  come-\textsc{pst}-\textsc{cop}-2\textsc{sg=q}\\
    \glt ‘Where did you come from?’ (\citealt{Yu2008}: 86)
    \z

\noindent According to \citet[79]{Yu2008}, the marker \textit{-jə} sometimes fuses with the preceding suffix and the verb in example \REF{ex:mong:12} is realized as /\textit{irzbɨšə}/. If the element =\textit{jə} that is sometimes found in \isi{questions} in the \ili{Tungusic} languages \ili{Sibe} and Aihui \ili{Manchu} is indeed a \isi{question marker}, then its most likely source is \ili{Dagur}. Clearly, \ili{Dagur} was also the origin of the \isi{question marker} \textit{=jee} in \ili{Oroqen} (see  \sectref{sec:5.10.2}).

To my knowledge there are no explicit descriptions of \isi{questions} in \textbf{Moghol}, but \citet{Weiers1972} mentions several examples of polar and content \isi{questions} that appear to be generally unmarked morphosyntactically. Presumably, there was a different \isi{intonation} contour that cannot be reconstructed for now. Given its peripheral position outside of \isi{Northeast Asia}, \ili{Moghol} will not be further addressed here.

The Northern subgroup of \ili{Mongolic} as identified by \citet{Rybatzki2003b}, i.e. \ili{Khamnigan Mongol} and \ili{Buryat}, basically share the \isi{question marking} of Shineken \ili{Buryat} seen above. Both the polar \isi{question marker} as well as the “corrogative” particle are still present in both languages. Khamnigan also has adopted the \ili{Mandarin} marker \textit{ba} \zh{吧}. The \ili{Khamnigan Mongol} “corrogative” particle \textit{bei} has been borrowed into Khamnigan \ili{Evenki} (\sectref{sec:5.10.2}).

\ea%13
    \label{ex:mong:13}
    \ili{Khamnigan Mongol}\\
    \ea
    \gll {hain=}\textbf{{gu}}?\\
    good=\textsc{q}\\
    \glt ‘Is it good?’
    
    \ex
    \gll {tere} \textbf{{ken}} \textbf{{bei}}?\\
    3\textsc{sg}  who  \textsc{q}\\
    \glt ‘Who is he?’ \citep[97]{Janhunen2003c}
    
    \ex
    \gll ənə  kobcaxon=cini  tɔrg-ɔɔr  ɔjɔ-gd-ɔɔ=\textbf{{gu}}, {bisi=}\textbf{{gu}}?\\
    this  clothes=2\textsc{sg.poss}  silk-\textsc{inst}  sew-\textsc{pass}-\textsc{p.ipfv}=\textsc{q} \textsc{neg}=\textsc{q}\\
    \glt ‘Are these clothes made of silk or not?’\glt
    
    \ex
    \gll bii    kara-ku-du=min    tabi-tee-ta \textbf{{ba}}?\\
    1\textsc{sg.nom}  see-\textsc{p.fut}-\textsc{dat}=1\textsc{sg.poss}  fifty-\textsc{prop}-2\textsc{pl}  \textsc{q}\\
    \glt ‘I guess you are about fifty years old, right?’ (\citealt{Yamakoshi2007a}: 132, 127)\z\z

As in Shineken \ili{Buryat}, an agreement suffix may follow the question markers in Standard \ili{Buryat}. The marker \textit{=gü} marks polar, alternative, and maybe \isi{focus question}s.

\ea%14
    \label{ex:mong:14}
    \ili{Buryat}\\
    \ea
    \gll yeshe  münge-tei  hen=\textbf{{gü}}?\\
    \textsc{pn}  money-\textsc{com}  \textsc{cop.pst}=\textsc{q}\\
    \glt ‘Did Yeshe have money?’
    
    \ex
    \gll {shi} \textbf{{xen}}{-tei}{=}\textbf{{b}}{=shi}?\\
    2\textsc{sg}  who-\textsc{com}=\textsc{q}=2\textsc{sg}\\
    \glt ‘Who are you with?’
    
    \ex
    \gll {yaba-xa-m=}\textbf{{gü}},  bai-xa-m=\textbf{{gü}}?\\
    go-\textsc{p}.\textsc{fut}-1\textsc{sg}=\textsc{q}  stay-\textsc{p}.\textsc{fut}-1\textsc{sg}=\textsc{q}\\
    \glt ‘Shall I go or stay?’ (\citealt{Skribnik2003}: 120, 119)
    \z
    \z 

\noindent This is probably also true for \ili{Khamnigan Mongol}, but no example for a plain \isi{alternative question} has been found in the relevant literature.

In order to compensate for the lack of information in most grammatical descriptions, the following examples of \textbf{Cyrillic \ili{Khalkha} \ili{Mongolian}} were elicited in October 2015 from a \ili{Mongolian} informant of Outer \isi{Mongolia} living in Germany. The \isi{analysis} and transcription partly follows \citet{Janhunen2012c}. As noted before, \isi{polar question}s are usually marked with the enclitic \textit{=UU}.

\ea%15
    \label{ex:mong:15}
    Cyrillic \ili{Khalkha} \ili{Mongolian}\\
    \gll ci  surguul-ruu-g.aa  yaw-j      bai-g.aa  youm=\textbf{{uu}}?\\
    2\textsc{sg}  school-\textsc{dir}-\textsc{poss}.\textsc{refl}  depart-\textsc{cvb.ipfv}  \textsc{cop}-\textsc{p}.\textsc{ipfv}  \textsc{cop}=\textsc{q}\\
    \glt ‘Are you going to school?’
    \z

\noindent As in this example \REF{ex:mong:15}, the enclitic sometimes combines with a copula, derived from a word meaning ‘thing’ (\citealt{Janhunen2012c}: 221, 228). This form also appears in \ili{Dagur} as \textit{=jumo}\textit{o} and some \ili{Tungusic} languages (see \sectref{sec:5.10.2}), all of which were probably borrowed from central \ili{Mongolian} dialects spoken in Inner \isi{Mongolia}. It also seems likely that the polar \isi{question marker} found its way from \ili{Mongolian} (\textit{=uu} {\textasciitilde} \textit{=oo}) into \ili{Oroqen} (\textit{=oo}), where it has an additional meaning of fear or doubt. Focus questions are identical to polar \isi{questions} in form but exhibit an additional intonational peak on the focused element (indicated by underlining in \ili{Mongolian} and with italics in the translation). Unlike \ili{Middle Mongol} and some \ili{Tungusic} languages, the \isi{question marker} does not express \isi{focus} itself and cannot take any other position in the sentence.

\ea%16
    \label{ex:mong:16}
    Cyrillic \ili{Khalkha} \ili{Mongolian}\\
    \gll ci  \uline{surguul-ruu-g.aa} yaw-j      bai-g.aa  youm=\textbf{{uu}}?\\
    2\textsc{sg}  school-\textsc{dir}-\textsc{poss}.\textsc{refl}  depart-\textsc{cvb.ipfv}  \textsc{cop}-\textsc{p}.\textsc{ipfv}  \textsc{cop}=\textsc{q}\\
    \glt ‘Are you going \textit{to school}?’
    \z

Both plain and \isi{negative alternative question}s require two question markers as well as a disjunctive. The disjunctive \textit{eswel} literally meaning ‘(and) if not’ could be analyzed as \textit{es-wel} ‘\textsc{neg-cvb}.\textsc{cond}’ and can also be employed as a standard disjunctive \citep[221]{Janhunen2012c}. This has a typological parallel in \ili{Korean} \textit{an-i-myen} ‘\textsc{neg}-\textsc{cop}-\textsc{cond}’ (\sectref{sec:5.7.2}).

\ea%17
    \label{ex:mong:17}
    Cyrillic \ili{Khalkha} \ili{Mongolian}\\
    \ea
    \gll ci  tzai  uu-x=\textbf{{uu}}, \textbf{{eswel}} airag  uu-x=\textbf{{uu}}?\\
    2\textsc{sg}  tea  drink-\textsc{p}.\textsc{fut=q}  or  kumis  drink-\textsc{p}.\textsc{fut}=\textsc{q}\\
    \glt ‘Do you drink tea or kumis?’
    
    \ex
    \gll ci  surguul-ruu-g.aa  yaw-a.x=\textbf{uu}    \textbf{eswel} {yaw-a.x-}\textbf{{güi}}{=}\textbf{{y.uu}}?\\
    2\textsc{sg}  school-\textsc{dir}-\textsc{poss}.\textsc{refl}  depart-\textsc{p}.\textsc{fut}=\textsc{q} or  depart-\textsc{p}.\textsc{fut}-\textsc{neg}=\textsc{q}\\
    \glt ‘Do you go to school or not?’
    \z
    \z

Alternative questions may also take the extended \isi{question marker} \textit{youm=uu} (Benjamin Brosig p.c. 2016).

\ea%18
    \label{ex:mong:18}
    Cyrillic \ili{Khalkha} \ili{Mongolian}\\
    \gll ci  tzai  uu-x    youm=\textbf{{uu}}, \textbf{{eswel}} airag  uu-x {youm=}\textbf{{uu}}?\\
    2\textsc{sg}  tea  drink-\textsc{p}.\textsc{fut}  \textsc{cop}=\textsc{q}    or  kumis  drink-\textsc{p}.\textsc{fut} \textsc{cop}=\textsc{q}\\
    \glt ‘Do you drink tea or kumis?’
    \z

Apparently, the \isi{question marker} \textit{=UU} has expanded its scope and sometimes also appears in content \isi{questions}.

\ea%19
    \label{ex:mong:19}
    Cyrillic \ili{Khalkha} \ili{Mongolian}\\
    \gll {ci} \textbf{{xedzee}} surguul-ruu-g.aa  yaw-a.x=\textbf{{uu}}?\\
    2\textsc{sg}  when    school-\textsc{dir}-\textsc{poss}.\textsc{refl}  depart-\textsc{p}.\textsc{fut}=\textsc{q}\\
    \glt ‘When are you going to school?’
    \z

\noindent But according to other sources, \ili{Khalkha} also has the expected “corrogative” particle.

\ea%20
    \label{ex:mong:20}
    Cyrillic \ili{Khalkha} \ili{Mongolian}\\
    \gll \textbf{{xen}} tsai  uu-san \textbf{{be}}?\\
    who  tea  drink-\textsc{p}.\textsc{pfv}  \textsc{q}\\
    \glt ‘Who drank tea?’ \citep[171]{Svantesson2003}
    \z

Some verbal endings in \ili{Mongolian} have a slightly different but predictable form in the \isi{interrogative} than those in the declarative. These are summarized in \tabref{tab:mong:1}.

\begin{table}
\caption{Special interrogative endings in \ili{Mongolian} according to \cite[183f., 255, 298]{Janhunen2012c}); differences are marked with boldface}
\label{tab:mong:1}

\begin{tabularx}{\textwidth}{XXXl}
\lsptoprule
& \textbf{Meaning} & \textbf{Declarative} & \textbf{Interrogative}\\
\midrule
Mood & \textsc{vol} & yab-\textbf{ii.y} & yab-\textbf{y}=oo\\
Finite & \textsc{dur} & yab-\textbf{e.n’} & yab-\textbf{n}=oo\\
& \textsc{conf} & yab-l(=aa) \textsc{emph} & yab-l=oo/yab-laa=y.oo\\
& \textsc{term} & yab-\textbf{eb} & yab-\textbf{b}=oo\\
& \textsc{res} & yab-j & yab-j=oo\\
Participle & \textsc{fut} & yab-\textbf{ex} & yab-\textbf{x}=oo\\
& \textsc{hab} & yab-\textbf{deg} & yab-\textbf{dg}=oo\\
& \textsc{prf} & yab-\textbf{sen} & yab-\textbf{sn}=oo\\
& \textsc{imprf} & yab-aa & yab-aa=y.oo\\
\lspbottomrule
\end{tabularx}
\end{table}

Similarly to other languages of the region, descriptions of \ili{Mongolic} languages usually do not mention \isi{tag question}s and it remains open whether they are absent or were simply ignored. The following elicited example is marked with a marker \textit{tee} that appears to be ultimately derived from the distal demonstrative \textit{te-} and can roughly be translated as ‘is it like this?’.

\ea%21
    \label{ex:mong:21}
    Cyrillic \ili{Khalkha} \ili{Mongolian}\\
    \gll ci  surguul  ruu-g.aa  yab-na \textbf{{tee}}?\\
    2\textsc{sg}  school    \textsc{dir}-\textsc{poss}.\textsc{refl}  depart-\textsc{dur}  so\\
    \glt ‘You are going to school, right?’
    \z

Another \isi{tag question} type encountered in \ili{Mongolian} consists of a negative copula followed by a polar \isi{question marker}.

\ea%22
    \label{ex:mong:22}
    \ili{Darkhat} \ili{Mongolian}\\
    \gll {ir-sen} \textbf{{biš}}=\textbf{{oo}}?\\
    come-\textsc{p.pfv}  \textsc{neg}=\textsc{q}\\
    \glt ‘(S)he arrived, didn’t (s)he?’ \citep[188]{Ragagnin2011}
    \z

Descriptions of \ili{Mongolic} languages usually also do not mention \isi{intonation} contours. But \citet[192]{Karlsson2003} made the following interesting observations for \ili{Khalkha} \ili{Mongolian}.

\begin{quote}
Focus in \isi{questions} is signaled by a rising gesture, the LH [low high]. However, depending on the segmental conditions, the gesture can be realized just as a tonal peak, synchronized with the second mora, making it similar to the focal H in declaratives. Interrogatives have a terminal low boundary tone, which is characteristic for most informants, while the high final rise is optional. All this makes the \isi{intonation} of interrogatives similar to that of declaratives. The reason for this seems to be the strong formal signaling of interrogatives by using question particles. Thus, \isi{intonation} has a redundant role in forming the \isi{interrogative} mode in \ili{Mongolian}. (my square brackets)
\end{quote}

\noindent We may thus conclude the following: \isi{polar question}s are obligatorily marked with the enclitic \textit{=UU} and have an optional rising \isi{intonation}. In \isi{focus question}s there is an additional peak on the focused element. This makes the structure of \isi{focus} \isi{questions} quite different from \ili{Middle Mongol}, where, as seen above, the enclitic attaches to the element in \isi{focus}. In addition, interrogatives in \isi{content question}s obligatorily receive “the same tonal gesture” as focused elements in \isi{focus} \isi{questions} (\citealt{Svantesson2005}: 93).

In \textbf{Chakhar \ili{Mongolian}} the polar \isi{question marker} is \textit{=UU} (\textit{=ůů}, \textit{=uu}) or \textit{=y.UU} when following a vowel. According to \citet[182]{Sechenbaatar2003} “the material shape of the \isi{interrogative} particle links Chakhar with \ili{Khalkha}, marking a distinction with regard to several other Inner \ili{Mongolian} dialects, such as, for instance, Baarin, in which the \isi{interrogative} particle appears in the invariable shape \textit{=ii}.” The optional “corrogative” particle has the form \textit{=w} {\textasciitilde} \textit{=b} or \textit{=wéé} {\textasciitilde} \textit{=béé}. The forms with a plosive are found following the \isi{nasals} \textit{m} or \textit{ŋ}.

\ea%23
    \label{ex:mong:23}
    \ili{Chakhar \ili{Mongolian}}\\
    \ea
    \gll ax=cin    õnõõdõr  yaw-n=\textbf{{uu}}?\\
    brother=2\textsc{sg}  today    depart-\textsc{dur}=\textsc{q}\\
    \glt ‘Is your brother leaving today?’
    
    \ex
    \gll {e.n} \textbf{{yamar}} {yum=}\textbf{{béé}}?\\
    this  what  thing=\textsc{q}\\
    \glt ‘What kind of thing is this?’ (\citealt{Sechenbaatar2003}: 182, 107)
    \z
    \z

\textbf{\ili{Khorchin} \ili{Mongolian}} likewise has the enclitic \textit{=(j)UU} that marks polar, alternative, and possibly \isi{focus question}s.

\ea%24
    \label{ex:mong:24}
    \ili{Khorchin} \ili{Mongolian}\\
    \ea
    \gll {ənə} {bɔl} {nɔm=}\textbf{{ʊʊ}}?\\
    this  \textsc{top}  book=\textsc{q}\\
    \glt ‘Is this a book?’
    
    \ex
    \gll {bii} {ənuur} {jab-aal} {taar-n=}\textbf{{ʊʊ}}, {tiiš-əə} {jab-aal} {taar-n=}\textbf{{ʊʊ}}?\\
    1\textsc{sg}  here  go-\textsc{cvb.cond}  fit-prs=\textsc{q} there-\textsc{refl}  go-\textsc{cvb.cond}  fit-\textsc{prs}=\textsc{q}\\
    \glt ‘Do I have to go in this or that direction?’ (\citealt{Yamakoshi2015}: 282, 292)
    \z
    \z

There is also an enclitic \textit{=(j)ii} that marks \isi{polar question}s as well as, optionally, \isi{content question}s. This might indicate an areal connection to \ili{Ainu}, \ili{Dagur}, \ili{Korean}, \ili{Japanese}, \ili{Manchu}, \ili{Ōgami}, and \ili{Ulcha} (\sectref{sec:6}). Perhaps, the expansion of \ili{Khalkha} \textit{=(y)UU} can also be explained as an areal trait connected to this.

\ea%25
    \label{ex:mong:25}
    \ili{Khorchin} \ili{Mongolian}\\
    \ea
    \gll tɛr  xun  jɛb-lɛɛ=\textbf{{j.ii}}?\\
    that  person  depart-\textsc{conf}=\textsc{q}\\
    \glt ‘Did that man go?’ \citep[71]{Chaganhada1991}
    
    \ex
    \gll {čii} \textbf{{xən}}{=}\textbf{{ji}}?\\
    2\textsc{pl}  who=\textsc{q}\\
    \glt ‘Who are you?’
    
    \ex
    \gll {činii} {aab=čin’} \textbf{{xaa}} {bii}?\\
    2\textsc{sg.gen}  father=2\textsc{sg.poss}  where  \textsc{cop}\\
    \glt ‘Where is your father?’ (\citealt{Yamakoshi2015}: 282, 284)
    \z
    \z 

However, \ili{Khorchin} might also exhibit the “corrogative” marker. Compare the following two examples from \ili{Khorchin} and \ili{Khalkha}, respectively (Benjamin Brosig p.c. 2018).

\ea%26
    \label{ex:mong:26}
    \ili{Khorchin} \ili{Mongolian}\\
    \gll ən  tɛxaa \textbf{{xən}}{-ɛɛ} \textbf{{jimɛɛ}}?\\
    this  chicken  who-\textsc{gen}  \textsc{cop.q}\\
    \glt ‘Whose chicken is this?’ \citep[71]{Chaganhada1991}
    \z

\ea%27
    \label{ex:mong:27}
    Cyrillic \ili{Khalkha} \ili{Mongolian}\\
    \gll en  taxyaa \textbf{{xen}}-ii(-x)    youm \textbf{{bwai}}?\\
    this  chicken  who-\textsc{gen}(-\textsc{nom})  \textsc{cop}  \textsc{q}\\
    \glt ‘Whose chicken is this?’
    \z

\noindent Without the nominalizer, \textit{youm} is perhaps better understood as ‘thing’.

According to \citet[15]{Brosig2014}, \ili{Khorchin} has two further \isi{question marker}s \textit{=me} and \textit{=mu}. Their exact scope and etymological relation remain unclear. However, \textit{=me} apparently can mark polar and \isi{content question}s while \textit{=mu} appears at least in polar and \isi{alternative question}s, e.g. \textit{nogon=}\textbf{\textit{mu}}\textit{, xar=}\textbf{\textit{mu}}? ‘Is (it) green or black?’ \citep[15]{Brosig2014}.

\ea%28
    \label{ex:mong:28}
    \ili{Khorchin} \ili{Mongolian}\\
    \gll {zaqi-d} \textbf{{yuu}} {xii=}\textbf{{me}}?\\
    \textsc{pn-dat}    what  do\textsc{=q}\\
    \glt ‘What are you going to do in Jarud?’\footnote{In \ili{Chinese} this place is called \textit{z\=aqí} \zh{扎旗}.} \citep[15]{Brosig2014}
    \z

An imperfective marker \textit{-n} is said to have been assimilated to the following \isi{question marker} in this example. \ili{Khorchin} also has borrowed the \ili{Mandarin} marker \textit{ba} \zh{吧}. According to \citet[72]{Chaganhada1991} it has a long vowel (\textit{baa}), just like the adjacent languages \ili{Solon}, \ili{Oroqen}, and \ili{Dagur}.

\ea%29
    \label{ex:mong:29}
    \ili{Khorchin} \ili{Mongolian}\\
    \gll {činii} {ax=čin'} {bas} {duč} {bɔl-ɔɔdue=}\textbf{{ba}}?\\
    2\textsc{sg.gen}  e.brother=2\textsc{sg.poss}  also  forty  become-\textsc{neg.ipfv}=\textsc{q}\\
    \glt ‘Your elder brother is not yet forty, right?’ \citep[287]{Yamakoshi2015}
    \z

An authochthonous equivalent of \ili{Mandarin} \textit{ba} \zh{吧} used mostly by older speakers is the \isi{combination} \textit{=i=}\textbf{\textit{dee}} \citep[16]{Brosig2014}. In \isi{tag question}s either \textit{ba} or \textit{=(y)UU} may be employed, which has parallels in the \ili{Tungusic} language \ili{Sibe} (\sectref{sec:5.10.2}) and in \ili{Mandarin} (\sectref{sec:5.9.2}).

\ea%30
    \label{ex:mong:30}
    \ili{Khorchin} \ili{Mongolian}\\
    \ea
    \gll {tərə} {xun} {baxš} bišə, \textbf{{tiim=ba}}?\\
    that  person  teacher    \textsc{neg}  like.that=\textsc{q}\\
    \glt ‘That person is not a teacher, right?’
    
    \ex
    \gll \textbf{{xədən}} {čag} {bɔl-ǰ=}\textbf{{ji}}?    xɔjɔr {čag=}\textbf{{ʊʊ}}? \textbf{{tiim=uu}}?\\
    when  hour  become-\textsc{cvb.}\textsc{ipfv}=\textsc{q}  two  hour=\textsc{q}    like.that=\textsc{q}\\
    \glt ‘What time is it? Two o’clock? Right?’ (\citealt{Yamakoshi2015}: 283, 286)
    \z
    \z

Benjamin Brosig (p.c. 2016) mentions a couple of additional particles such as \textit{qi} (identical to \textit{ʃii} below) of not absolutely clear origin. It is perhaps best classified as a \isi{tag question} marker. \ili{Mongolian} has a negative copula, \textit{bish} in the spoken and \textit{bous} in the literary language, that might somehow be connected to a word meaning ‘other’ (\citealt{Janhunen2003a}: 27; \citealt{Janhunen2012c}: 251). There is a parallel \isi{grammaticalization} of adjectives meaning ‘different’ to a negative copula in \ili{Tungusic} (\citealt{Hölzl2015a}: 146). According to \citet{Chaganhada1991} it has the form \textit{biʃii} in \ili{Khorchin} and has developed into a \isi{question marker}. Under my \isi{analysis}, however, the final \textit{=ii} might be nothing but the \isi{question marker}. From this perspective, \textit{biʃ=ii} is probably a \isi{tag question} marker almost identical to \ili{Darkhat} \textit{biš-oo} and \textit{bish=uu} in \ili{Mongolian} according to \citet[251]{Janhunen2012c}. This construction has exact typological parallels in several \ili{Tungusic} (\sectref{sec:5.10.2}) and \ili{Turkic} languages (\sectref{sec:5.11.2}). In what appears to be another type of \isi{tag question}, \ili{Khorchin} \textit{biʃ} may also be followed by an emphatic enclitic (\textit{biʃ=j.əə}) that has the form \textit{=(y)AA} in \ili{Mongolian} according to \citet[93]{Janhunen2012c}. Perhaps the form \textit{ʃii} is a contracted form of \textit{biʃ=ii}.

\ea%31
    \label{ex:mong:31}
    \ili{Khorchin} \ili{Mongolian}\\
    \gll ən  udur  ʃin  tabən  sar-iin    nəgən \textbf{{ʃii}}?\\
    this  day  new  five  month-\textsc{gen}  one  \textsc{q}\\
    \glt ‘Is today not the first day of May?’ \citep[71]{Chaganhada1991}
    \z

\noindent The same marker \textit{ʃii} can also be found in \isi{tag question}s following the element \textit{tii.n}, which is probably derived from the distal demonstrative (cf. \citealt{Janhunen2012c}: 130), similar to \REF{ex:mong:21} from Khalka. \citet[72]{Chaganhada1991} translates \textit{tiin} \textit{ʃii}, which may be attached to a \isi{declarative sentence}, as a \isi{tag question}. In \ili{Khalkha} there is also a \isi{question tag} \textit{tiim bish=üü} (Benjamin Brosig p.c. 2018).

There are few clear descriptions for \isi{questions} in \textbf{Ordos}. But there is evidence that it preserves the original \isi{question marker} as \textit{=(j)uu} and lacks the “corrogative” particle (Stefan Georg p.c. 2015).

\ea%32
    \label{ex:mong:32}
    \ili{Ordos}\\
    \ea
    \gll {yabu-b=}\textbf{{uu}}?\\
    go-\textsc{term}=\textsc{q}\\
    \glt ‘Did he go?’ \citep[208]{Georg2003a}
    
    \ex
    \gll {t‘e.re} \textbf{{j}}\textbf{{ɯɯ}} {ge-}\textsc{d}{ž\=e-n?}\\
    3\textsc{sg}  what  say-\textsc{res-3}\\
    \glt ‘What does he say?’ (\citealt{Mostaert1937}: lix)
    \z
    \z

\noindent Most other dialects will be ignored here for lack of data and reasons of space.

\largerpage[2]
There are also few good materials for \isi{questions} in \ili{Oirat}, which is why there will first be a descriptions of \isi{questions} in the closely related language (or aberrant dialect) \ili{Kalmyk}. In \textit{Kalmyk} the \isi{interrogative} particle \textit{=u} marks polar \isi{questions} and similar to \ili{Mongolian} (\tabref{tab:mong:1}) fuses with some suffixes, e.g. \textit{-na} ‘\textsc{dur}’ vs. \textit{-nu} ‘\textsc{dur.q}’ and \textit{-la} ‘\textsc{conf}’ vs. \textit{-lu} ‘\textsc{conf.q}’ \citep[42]{Benzing1985}. The “corrogative” particle is preserved as \textit{=b} {\textasciitilde} \textit{=w}, e.g., \textbf{\textit{kem}}\textit{=}\textbf{\textit{b}}? ‘who is it?’. In addition, there is another \isi{question marker} \textit{=iy} {\textasciitilde} \textit{=i} that seems to be employed in \isi{alternative question}s as well as \isi{polar question}s, e.g. \textit{xol=}\textbf{\textit{iy}}? ‘Is it far?’.

\ea%33
    \label{ex:mong:33}
    \ili{Kalmyk}\\
    \ea
    \gll ter  ir-v=\textbf{{u}}?\\
    3\textsc{sg}  come-\textsc{pfv}=\textsc{q}\\
    \glt ‘Did he come?’
    
    \ex
    \gll {endr} \textbf{{yamaran}} ödr?  sän  ödr=\textbf{{iy}} mu  ödr=\textbf{{iy}}?\\
    today  which    day  good  day=\textsc{q}  bad  day=\textsc{q}\\
    \glt ‘What kind of day is it today, (is it) a good day or a bad day?’ (\citealt{Benzing1985}: 42f.)
    \z
    \z

\noindent Note that in this example a \isi{content question} is followed by an \isi{alternative question} (see \sectref{sec:4.4}). Bläsing (2003) does not mention the marker, but quite clearly, this is the same element we have already seen above, e.g. \ili{Khorchin} \textit{=(j)ii}. The following Kalmyk sentences were elicited from a native speaker living in Germany in January 2016 via internet. The transliteration and \isi{analysis} are mine but roughly follow \cite{Bläsing2003}.

\ea%34
    \label{ex:mong:34}
    \ili{Kalmyk}\\
    \ea
    \gll \textbf{{al’daran}} {yow-jana-c?}\\
    whither  go-\textsc{prog}-2\textsc{sg}\\
    \glt ‘Where are you going?’
    
    \ex
    \gll ci  manhdur  shkol-d    yow-jana-c?\\
    2\textsc{sg}  tomorrow  school-\textsc{all}  go-\textsc{prog}-2\textsc{sg}\\
    \glt ‘Are you going to school tomorrow?’
    
    \ex
    \gll \uline{ci}  manhdur  shkol-d    yow-jana-c?\\
    2\textsc{sg}  tomorrow  school-\textsc{all}  go-\textsc{prog}-2\textsc{sg}\\
    \glt ‘Are \textit{you} going to school tomorrow?’
    \z
    \z 

\noindent No \isi{question marker} appears in \isi{content question}s. In \isi{focus question}s the \isi{focus} is apparently expressed with the help of an additional peak on the focused element.

The situation in Kalmyk is indeed very similar to \textbf{Oirat} proper, for which \citet{Birtalan2003} mentions the question markers \textit{=UU} {\textasciitilde} =(\textit{y)}\textit{UU}, \textit{=ii}, as well as \textit{=w} {\textasciitilde} \textit{=b}.

\ea%35
    \label{ex:mong:35}
    \ili{Oirat}\\
    \gll sään  bään=\textbf{{uu}}?\\
    good  \textsc{cop}=\textsc{q}\\
    \glt ‘Are you well?’ \citep[227]{Birtalan2003}
    \z

As opposed to Benzing, she treats the form \textit{=ii}, which we have already encountered in Baarin, \ili{Khorchin}, and Kalmyk, as a variant of the polar \isi{question marker}. In fact, this is the most likely \isi{analysis} as its form \textit{=ii} {\textasciitilde} \textit{=y.ii} is completely parallel to the standard marker \textit{=UU {\textasciitilde} =y.UU} \citep[183]{Janhunen2012c}. In some \ili{Tungusic} languages there are question markers that were probably borrowed from \ili{Mongolian} \textit{=(y)ii}, notably Ongkor \ili{Solon} \textit{-ii} as well as, less likely due to geographical distance, \ili{Even} \textit{-ii}, \ili{Negidal} \textit{-i}, and maybe \ili{Uilta} \textit{-}\textit{(y)i} (\sectref{sec:5.10.2}).

\textbf{\ili{Shirongolic}} languages also preserve the original polar \isi{question marker}, but display a more complicated picture than Central \ili{Mongolic}. In \textit{Shira Yughur}---classified by \citet{Rybatzki2003b} as the only South-Central language instead of as \ili{Shirongolic}---the polar \isi{question marker} \textit{=uu} {\textasciitilde} \textit{=j.uu} sometimes fused with the preceding verb ending, but there is no clear information as to when and how often this happened. The durative marker \textit{-nAi} (and variants) always has the form \textit{-nam} before the question particle. The “corrogative” particle appears to be optional. Alternative as well as negative alternative \isi{questions} take two question markers.

\newpage 
\ea%36
    \label{ex:mong:36}
    \ili{Shira Yughur}\\
    \ea
    \gll {tʃə} {qutad} largə  ʃda-daɢ=\textbf{{u}}\textbf{{u}}?\\
    2\textsc{sg}  \textsc{pn}  speak  able-\textsc{hab}=\textsc{q}\\
    \glt ‘Can you speak \ili{Chinese}?’
    
    \ex
    \gll tʃəmiin    dʒa \textbf{{jimar}} dʉge-də  ɔɔl-sɔn \textbf{{b}}\textbf{{ə}}?\\
    2\textsc{sg}.\textsc{gen} be  what  time-\textsc{dat}  get-\textsc{pst}   \textsc{q}\\
    \glt ‘Which year were you born in?’ (\citealt{Chaolu1994a}: 8, 9f.)
    
    \ex
    \gll tʃə.ma-də  qudaʁa    bii=\textbf{{j.u}}, \textbf{{uɣui}}{=}\textbf{{j.u}}?\\
    2\textsc{sg}.\textsc{obl}-\textsc{loc}  knife    \textsc{ex}=\textsc{q}    \textsc{neg}=\textsc{q}\\
    \glt ‘Do you have a knife?’ \citep[58]{Zhaonasitu1981b}
    \z
    \z 

\noindent The last example (\ref{ex:mong:36}c) is a negative \isi{alternative question} that shows a negative existential because of the existential in the first alternative. Similar to the situation in \ili{Khalkha} before, there is also one example of the polar \isi{question marker} in what appears to be a \isi{content question} (cf. \ili{Mongolian} \textit{-x=oo} in \tabref{tab:mong:1} above).

\ea%37
    \label{ex:mong:37}
    \ili{Shira Yughur}\\
    \gll {cimiin} \textbf{{keen}}-di ög-\textbf{{k’uu}}?\\
    2\textsc{sg}.\textsc{acc}  who-\textsc{dat}  give-\textsc{fut}.\textsc{q}\\
    \glt ‘To whom shall I give you?’ \citep[280]{Nugteren2003}
    \z

In sum, \ili{Shira Yughur} \isi{interrogative} constructions pattern with Central \ili{Mongolic} and have to be differentiated from the more complex system found in \ili{Shirongolic} languages.

\textit{Bonan}, like \ili{Ordos}, lacks the “corrogative” particle in \isi{content question}s, which are morphsyntactically unmarked.

\ea%38
    \label{ex:mong:38}
    \ili{Bonan}\\
    \gll {dʐoma} \textbf{{χala}} {o-to}?\\
    \textsc{pn}  where  go-\textsc{pfv}\\
    \glt ‘Where did Droma go?’ \citep[261]{Fried2010}
    \z

For polar \isi{questions} \ili{Bonan} preserves the \ili{Mongolic} \isi{interrogative} marker that has the form \textit{-u}. But its use is more complicated than in these \ili{Mongolic} languages we have encountered before: “When \textit{-u} is suffixed to imperfective lexical verbs, it replaces the imperfective suffix (\textit{-tɕi}/\textit{-tɕo}). Similarly, when it is suffixed to perfective verbs, it replaces the perfective suffixes \textit{-to} and \textit{-tɕə}.” \citep[258]{Fried2010} The suffix thus attaches directly to the verb stem.

\ea%39
    \label{ex:mong:39}
    \ili{Bonan}\\
    \gll {tɕʰə} {nudə} {natʰə-}\textbf{{u}}?\\
    2\textsc{sg}   today  dance-\textsc{q}\\
    \glt ‘Did you dance today?’ \citep[259]{Fried2010}
    \z

The \isi{interrogative} marker \textit{-u} fused with several verb endings and copulas, see \tabref{tab:mong:3} and \tabref{tab:mong:4}.

\begin{table}
\caption{Special interrogative forms in Bonan (\citealt{WuHugjiltu2003}: 339, 343)}
\label{tab:mong:3}

\begin{tabularx}{\textwidth}{XXXl}
\lsptoprule
& \textbf{Meaning} & \textbf{Declarative} & \textbf{Interrogative}\\
\midrule
\textsc{narr} & \textsc{prs, fut} & -m & -mu\\
\textsc{dur} & \textsc{prs, fut} & -na & -nu\\
\textsc{term} & \textsc{pst} & -wa > -o & -wu > -u\\
\lspbottomrule
\end{tabularx}
\end{table}

\begin{table}
\caption{Special interrogative copula forms in Bonan (\citealt{WuHugjiltu2003}: 340, 343)}
\label{tab:mong:4}

\begin{tabularx}{\textwidth}{XXXl}
\lsptoprule
& \textbf{Subjective} & \textbf{Objective} & \textbf{Interrogative}\\
\midrule
\textsc{exist} & wi & wa & wu\\
\textsc{cop}.\textsc{emph} & mbi & mba & mbu\\
\lspbottomrule
\end{tabularx}
\end{table}

\ea%40
    \label{ex:mong:40}
    \ili{Bonan}\\
    \ea
    \gll {ode-}\textbf{{nu}}?\\
    go-\textsc{dur}.\textsc{q}\\
    \glt ‘Will he go?’ (\citealt{WuHugjiltu2003}: 343)
    
    \ex
    \gll tɕin-da    tɕa {dawʑi} \textbf{{w}}\textbf{{ɵ}}\textbf{{u}}?\\
    2\textsc{sg}-\textsc{dat} tea  leaf  \textsc{exist}.\textsc{q}\\
    \glt ‘Do you have tea?’
    
    \ex
    \gll tə.rə  aagɵ  da  ɬɵma \textbf{{mbɵ}}\textbf{{u}}?\\
    that  girl  also  student  \textsc{cop}.\textsc{emph}.\textsc{q}\\
    \glt ‘Is that woman also a student?’ (\citealt{Chaolu1994e}: 8, 6)
    \z
    \z 

The copula forms are given as \textit{wɵ}\textit{u} and \textit{mbɵu} by \citet{Chaolu1994e}. According to \citet[260]{Fried2010}, the forms are declarative \textit{wi} \textsc{subj} vs. \textit{wa} \textsc{obj}, \isi{interrogative} \textit{wu(u)} \textsc{subj} vs. \textit{wa-u} \textsc{obj} and declarative \textit{bi} \textsc{subj} vs. \textit{ba} \textsc{obj}, \isi{interrogative} \textit{bu} \textsc{subj} vs. \textit{ba-u} \textsc{obj}. The copula starting with \textit{b-} is used in nominal copula clauses, the copula starting with \textit{w-} in all other clause types \citep[260]{Fried2010}. In addition, the \isi{Gansu} variety of \ili{Bonan} has borrowed the \ili{Chinese} polar \isi{question marker} \textit{ma} \zh{吗} (\citealt{WuHugjiltu2003}: 343) and has a special marker \textit{-si}, allegedly for “\isi{rhetorical questions}”, that can also mark \isi{alternative question}s.

\ea%41
    \label{ex:mong:41}
    \ili{Bonan}\\
    \ea
    \gll {χeɕaŋ} {jaŋgətɕə} {natʰə} {kʰər-}\textbf{{si}}?\\
    \textsc{pn}  how    dance  be.required-\textsc{q}\\
    \glt ‘How should (one) dance Leru?’
    
    \ex
    \gll {pə} [{hkutə} {or}{ə-}\textbf{{si}} \textbf{{ə}}\textbf{{s}}\textbf{{ə}} {or}{ə-}\textbf{{si}}] {ə}{sə} {med-o}.\\
    1\textsc{sg}  yesterday  rain-\textsc{q}  \textsc{neg}  rain-\textsc{q}  \textsc{neg}  know-\textsc{term}\\
    \glt ‘I don’t know [whether it rained or not yesterday].’ (\citealt{Fried2010}: 99, 227)
    \z
    \z

An example of a plain \isi{alternative question} was given by Chaolu Wu \REF{ex:mong:42}. Interestingly, only the first alternative has an overt \isi{question marker}. A similar situation can be seen in \ili{Santa}, \ili{Kangjia}, and \ili{Mangghuer} and is an areal feature.

\ea%42
    \label{ex:mong:42}
    \ili{Bonan}\\
    \gll {bə} en-sa    ɵ.d {kə-}{saŋ=}\textbf{{wu}},    taŋ-sa {ɵ.d} {kar-saŋ?}\\
    1\textsc{sg}  this-\textsc{abl}  go  be.required-\textsc{p.pfv}=\textsc{q}  that-\textsc{abl} go  be.required-\textsc{p.pfv}\\
    \glt ‘Will I go this way or that way ?’ (\citealt{Chaolu1994e}: 15)
    \z

In \ili{Bonan} there is also the \ili{Chinese} \isi{question marker} \textit{ba} \zh{吧}.

\ea%43
    \label{ex:mong:43}
    \ili{Bonan}\\
    \gll dedə    ‘gudə    sə  edəro \textbf{{ba}}?\\
    old.man.\textsc{voc}  yesterday  \textsc{neg}  ?tired  \textsc{q}\\
    \glt ‘Grandpa, you weren’t too tired yesterday, right?’ (\citealt{Buhe1982}: 59)
    \z

In \textbf{\ili{Kangjia}} the \isi{question marker} has the form \textit{-ʉ} and has the two variants \textit{-vʉ} and \textit{-bʉ}. It fused with more suffixes than in \ili{Bonan} resulting in the forms listed in \tabref{tab:mong:5}. In addition there are also two markers \textit{ba}, \textit{le}, and \textit{sa} that are most likely of \ili{Mandarin} \ili{Chinese} origin (e.g., \ili{Mandarin} \textit{ba} \zh{吧}, Xining \textit{lɛ}\textsuperscript{53} \zh{呢}, Hezhou \textit{ʐa}\textsuperscript{3}, see \sectref{sec:5.9.2.1}) As in \ili{Bonan}, content \isi{questions} are usually unmarked.

\begin{table}
\caption{Special question forms in Kangjia (\citealt{Siqinchaoketu1999}: passim; 2002: passim)}
\label{tab:mong:5}

\begin{tabularx}{\textwidth}{XXl}
\lsptoprule

\textbf{Meaning} & \textbf{Declarative} & \textbf{Interrogative}\\
\midrule
\textsc{nfut} & -m, & -mʉ\\
\textsc{nfut} & -na, ... & -nʉ\\
\textsc{pst} & -va, -pa, -ba & -vʉ\\
\textsc{progr} & -si, -sina & -sʉ, -sʉnʉ\\
\textsc{progr} & -dʒi(na) & -dʒinʉ\\
\textsc{progr (}-dʒi igʉ >) & -dʒigʉ & -dʒigʉ ʉ\\
\textsc{pfv} & -sʉn & -sʉn ʉ/bʉ/vʉ\\
\textsc{ipfv} & -gʉ(n) & -gʉ(n) ʉ/bʉ\\
\textsc{hab (}-si-gʉ >) & -s(ɯ)gʉ & -s(ɯ)gʉ bʉ\\
\textsc{fut (}-gʉi > -gi) & -gʉ(n)-i/a & -gʉ(n) bʉ\\
\lspbottomrule
\end{tabularx}
\end{table}

\newpage 
\ea%44
    \label{ex:mong:44}
    \ili{Kangjia}
    \ea
    \gll tʃi  meda-na \textbf{{ba}}?\\
    2\textsc{sg}  know-\textsc{nfut}  \textsc{q}\\
    \glt ‘You know, right?’
    
    \ex
    \gll ajɔ!  en.de \textbf{ma}-ge-dʒi    re-va \textbf{{sa}}?\\
    \textsc{excl}  here  how-\textsc{v}-\textsc{cvb.ipfv}  come-\textsc{pst}  \textsc{q}\\
    \glt ‘Oh! Why have you come here?’
    
    \ex
    \gll tʃi-ni    kʉn \textbf{{le}}?\\
    2\textsc{sg}-\textsc{gen}  person  \textsc{q}\\
    \glt ‘What about your person?’ (\citealt{Siqinchaoketu1999}: 215, 222, 217)
    \z
    \z 

Alternative questions have only one marker attached to the first alternative.

\ea%45
    \label{ex:mong:45}
    \ili{Kangjia}\\
    \ea
    \gll tʃi  mede-\textbf{{nʉ}}?\\
    2\textsc{sg}  know-\textsc{nfut}.\textsc{q}\\
    \glt ‘Do you know?’
    
    \ex
    \gll te  tʃi-ni-gʉ \textbf{{bʉ}}?\\
    that  2\textsc{sg}-\textsc{gen}-?\textsc{n}  \textsc{q}\\
    \glt ‘Is that yours?’
    
    \ex
    \gll re-\textbf{{vʉ}}, \textbf{{se}} {re-va?}\\
    come-\textsc{pst}.\textsc{q}  \textsc{neg}  come-\textsc{pst}\\
    \glt ‘Has (she) come or not?’ (\citealt{Siqinchaoketu2002}: 71, 169, 217)
    \z
    \z 

Tag \isi{questions} in \ili{Kangjia} take the sentence-final marker \textit{ere} {\textasciitilde} \textit{are}. Note a formally similar \isi{tag question} marker \textit{ale} in the \ili{Turkic} language \ili{Tuvan} (\sectref{sec:5.11.2}).

\ea%46
    \label{ex:mong:46}
    \ili{Kangjia}\\
    \gll te  kʉn  lausɯ    mari, \textbf{{ere}}?\\
    that  person  teacher    \textsc{neg}  \textsc{q}\\
    \glt ‘That person isn’t a teacher, right?’ \citep[197]{Siqinchaoketu1999}
    \z

\ili{Kangjia} has a copula system similar to \ili{Bonan} (\tabref{tab:mong:6}).

\begin{table}
\caption{Special interrogative copula forms in Kangjia in analogy to Bonan (Kangjia/Bonan) (\citealt{Siqinchaoketu1999}: 196f., 216, passim; 2002: passim).}
\label{tab:mong:6}

\begin{tabularx}{\textwidth}{XXXl}
\lsptoprule
& \textbf{Subjective} & \textbf{Objective} & \textbf{Interrogative}\\
\midrule
\textsc{exist} & i/wi & va/wa & vʉ/wu\\
\textsc{cop}.\textsc{emph} & mbi/mbi & mba/mba & mbʉ/mbu\\
\lspbottomrule
\end{tabularx}
\end{table}

Similar to \ili{Bonan} and \ili{Kangjia}, \textbf{Santa} preserves the \ili{Mongolic} \isi{interrogative} marker as \textit{-}\textit{u}, which also fused with the preceding verbal ending or copula (\tabref{tab:mong:7}). There likewise does not appear to be a “corrogative” particle.

\begin{table}
\caption{Finite tense aspect markers in Santa (\citealt{Kim2003}: 358; \citealt{Napoli2014}: 39); in \citet{Chaolu1994b} the interrogative forms are given as \textit{-nu} and \textit{-wo-u}}
\label{tab:mong:7}

\begin{tabularx}{\textwidth}{XXXl}
\lsptoprule
& \textbf{Meaning} & \textbf{Declarative} & \textbf{Interrogative}\\
\midrule
\textsc{dur} & \textsc{-ipfv} & -ne & -nu\\
\textsc{term} & \textsc{-pfv} & -wo & -wu\\
\textsc{progr} & -\textsc{cvb}.\textsc{ipfv cop} & -zhi wo > -zho & -zhi wu > -zhu\\
\lspbottomrule
\end{tabularx}
\end{table}

\ea%47
    \label{ex:mong:47}
    \ili{Santa}\\
    \ea
    \gll {ire-}\textbf{{nu}}?\\
    come-\textsc{dur.q}\\
    \glt ‘Will you come?’
    
    \ex
    \gll {kijie-zhi} \textbf{{wu}}?\\
    sleep-\textsc{cvb}.\textsc{ipfv}  \textsc{cop}.\textsc{q}\\
    \glt ‘Is (s)he sleeping?’
    
    \ex
    \gll {chi} \textbf{{khala}} {echi-ne?}\\
    2\textsc{sg}  whither  go-\textsc{dur}\\
    \glt ‘Where are you going?’ (\citealt{Kim2003}: 358)
    
    \ex
    \gll ghoni=chini    ire-\textbf{{wu}}?\\
    sheep=2\textsc{sg}.\textsc{gen}  come-\textsc{term}.\textsc{q}\\
    \glt ‘Have your sheep come back?’ (\citealt{Todaeva1959}: 284, modified transcription based on \citealt{Kim2003})\z\z

The form \textit{-mu} found in the following \isi{alternative question} \REF{ex:mong:48} was not mentioned by Kim but is probably comparable to an identical form in \ili{Bonan}, the so-called narrative \isi{interrogative}. In \ili{Santa} as well, only the first alternative receives a \isi{question marker}.

\ea%48
    \label{ex:mong:48}
    \ili{Santa}\\
    \gll bi  ənə {man-sa} {jawu-}\textbf{{mu}} {ha} {man-sa} {jawu-nə?}\\
    1\textsc{sg}  this  direction-\textsc{abl}  go-\textsc{q}    that  direction-\textsc{abl}  go-\textsc{dur}\\
    \glt ‘Will I go this way or that way?’ (\citealt{Chaolu1994b}: 14)
    \z

\citet[295]{Todaeva1959}, who did fieldwork among the \ili{Santa} in the middle of the 50s, mentions two additional \isi{interrogative} particles \textit{la} and \textit{ba}. The latter is clearly of \ili{Chinese} origin (\textit{ba} \zh{吧}, \citealt{LiuZhaoxiong1981}: 83).

\newpage 
\ea%49
    \label{ex:mong:49}
    \ili{Santa}\\
    \ea
    \gll hhe  shi  noghi  we-\textbf{{la}}?\\
    that  \textsc{cop}  dog  \textsc{cop}-?\textsc{q}\\
    \glt ‘Is that a dog?’
    
    \ex
    \gll e.ne  kun  e.de  ire-ne \textbf{{ba}}?\\
    this  man  here  come-\textsc{dur}  \textsc{q}\\
    \glt ‘This man is coming, right?’ (\citealt{Todaeva1959}: 295, modified transcription based on \citealt{Kim2003})
    \z
    \z

Perhaps \textit{la} is a loan from Hezhou \ili{Mandarin} \textit{la}\textsuperscript{3} \zh{啦} (\sectref{sec:5.9.2.1}). \citet[360]{Field1997} claims that \ili{Santa} has tag \isi{questions} that have the form of a regular \isi{polar question} followed by the irrealis negator \textit{uliə}, which is a very unexpected construction for a \isi{tag question}. In fact, an \isi{analysis} as a negative \isi{alternative question} in which only the first alternative is overtly marked is more likely. Such a situation can also be found in Karlong \ili{Mongghul} (see \citealt{Faehndrich2007}: 221) and Minhe \ili{Mangghuer} (see below).

\ea%50
    \label{ex:mong:50}
    \ili{Santa}\\
    \gll imani  mədʑiə=\textbf{{nu}} \textbf{{uliə}}?\\
    faith  know=\textsc{q}  \textsc{neg}\\
    \glt ‘Do you know the faith or not?’ \citep[360]{Field1997}
    \z

\noindent The same construction is also mentioned by \citet[79]{LiuZhaoxiong1981}. A slightly different \isi{analysis} of the use of negators for \isi{question marking} has recently been given by \citet[41]{Napoli2014}. According to him, there are two negators that can fulfill this function.

\begin{quote}
Events marked with the non-perfective marker -\textit{ne} can only receive the negator \textit{(u)lie}. Since this negator can only be used with events marked with this marker, the finite marker can be dropped. This is not the case with \textit{wuye}, since it can negate events marked with -\textit{wo} and -\textit{zho}. Therefore, in order to specify the tense-aspect of the sentence, the marker is obligatory. 
\end{quote}

\ea%51
    \label{ex:mong:51}
    \ili{Santa}\\
    \gll chi  baza-de  echi-wo \textbf{{wuye}}?\\
    2\textsc{sg}  \textsc{pn}-\textsc{loc}    go-\textsc{term}  \textsc{neg}\\
    \glt ‘Did you go to Linxia or not?’ \citep[41]{Napoli2014}
    \z

\noindent However, other sources do not mention the negator \textit{wuye} at all. \ili{Santa} has a partially productive negative verb \textit{ui-} (\citealt{LiuZhaoxiong1981}: 73; cf. S. \citealt{Kim2003}: 362) that could be the basis for \textit{wuye}. For instance, consider the following example.

\ea%52
    \label{ex:mong:52}
    \ili{Santa}\\
    \gll tȿɯ-ni    ada  uai-\textbf{{nu}}, \textbf{{u}}{-wo?}\\
    2\textsc{sg}-\textsc{gen}  father  \textsc{cop}-\textsc{q}    \textsc{neg}-\textsc{term}\\
    \glt ‘Is your father alive or not?’ (\citealt{LiuZhaoxiong1981}: 105, simplified)
    \z

\textbf{Mongghul} has no “corrogative” particle in content \isi{questions} (Stefan Georg p.c. 2015), but preserves the polar \isi{question marker}. \citet{Faehndrich2007} has collected several descriptions of \isi{question marking} in \ili{Mongghul} that exhibit certain differences but usually agree in the presence of three question markers such as neutral \textit{uu}, \textit{nuu} after objective \textit{-a}, and \textit{juu} after subjective \textit{-ii} in Karlong. The descriptions disagree about the \isi{analysis} of the question markers as particles or suffixes. Here, the original variant has been analyzed as enclitic \textit{=(y)uu} and all other forms as suffixes. In Karlong \ili{Mongghul} the forms are

\begin{quote}
\textit{nu:}, after words ending in the objective suffix -\textit{a}, \textit{ju:}, after words ending in the subjective suffix \textit{-i:}, and \textit{u:}, which is used after words ending in other vowels, including /a/ which is not the objective suffix. Short high vowels are deleted before the \isi{interrogative} particle \textit{u:}. \citep[221]{Faehndrich2007}
\end{quote}

\noindent In example (\ref{ex:mong:53}b) it appears in a \isi{focus question} and does not stand sentence-finally. It does not, however, attach to the apparent \isi{focus} in the sentence, which is \textit{ɕge pɨsee} ‘big belt’. The situation is thus unlike \ili{Middle Mongol}. Whether the \isi{focus} position is sentence initial or preverbal remains unclear, but might be responsible for the sentence-final position of the personal pronoun. But see also \sectref{sec:5.11.2} on \ili{Turkic} languages for second person markers following the \isi{question marker}.

\ea%53
    \label{ex:mong:53}
    \ili{Mongghul} (Karlong)\\
    \ea
    \gll tɕɨ  dʑiehun-la=\textbf{{uu}}?\\
    2\textsc{sg}  marry-\textsc{v=q}\\
    \glt ‘Are you married?’
    
    \ex
    \gll ɕge  pɨsee  sge-dʑ-a-\textbf{{nuu}} {tɕɨ?}\\
    big  belt  see-\textsc{perf}-\textsc{obj=q}  2\textsc{sg}\\
    \glt ‘Have you seen a big belt?’ (\citealt{Faehndrich2007}: 221, 220)
    \z
    \z

\ea%54
    \label{ex:mong:54}
    \ili{Mongghul}\\
    \gll {qi} \textbf{{anji}} {xji}-{gu-i?}\\
    2\textsc{sg}  where  go-\textsc{p}.\textsc{fut}-\textsc{subj}\\
    \glt ‘Where are you going?’ (\citealt{Georg2003b}: 303, shortened)
    \z

\citet{Åkerman2012} gives a much more elaborate description of the \isi{interrogative verb} forms in \ili{Mongghul}. Similar to other \ili{Shirongolic} languages, the \isi{question marker} fused with several verb suffixes and copulas (\tabref{tab:mong:8}). There is complex \isi{interaction} of \isi{question marking} with the domains of tense, aspect, clause type, and perspective. Similar to \citegen{Faehndrich2007} claim, the \isi{question marker} \textit{-nu} only appears after the objective forms. However, the \isi{interaction} of question markers with the other suffixes is much more complicated. In some cases the \isi{question marker} simply fused with the suffix, e.g. \textit{-wuu} < \textit{-wa} + \textit{-u}. In other cases, those in which \citet{Faehndrich2007} postulated the \isi{question marker} \textit{-juu}, the \isi{analysis} is somewhat unclear.

\begin{table}
\caption{Special interrogative suffixes and copulas in Mongghul (\citealt{Åkerman2012}: 13ff.)}
\label{tab:mong:8}

\begin{tabularx}{\textwidth}{XXl}
\lsptoprule

\textbf{Meaning} & \textbf{Declarative} & \textbf{Interrogative}\\
\midrule
\textsc{pfv} \textsc{subj} & -wa & -wuu\\
\textsc{ipfv} \textsc{subj} & -nii & -niu\\
\textsc{state} \textsc{subj} & -jii & -jiu\\
\textsc{fut} \textsc{subj} & -gu.nii, -gui & -gu.niu, -guu\\
\textsc{ipfv} \textsc{neut} & -ni/-nu & -nu\\
\textsc{ipfv} \textsc{neut} & -m/-n & -muu\\
\textsc{fut} \textsc{neut} & -m/-n & -muu\\
\textsc{cop} \textsc{neut} & wei & wei-u\\
\textsc{pfv} \textsc{obj} & -jia & -jia-nu\\
\textsc{ipfv} \textsc{obj} & -na & -na-nu\\
\textsc{fut} \textsc{obj} & -gu.na & -gu.na-nu\\
\textsc{cop} \textsc{obj} & wa {\textasciitilde} ma {\textasciitilde} na {\textasciitilde} la {\textasciitilde} ra & wa-nu {\textasciitilde} ma-nu {\textasciitilde} na-nu {\textasciitilde} la-u {\textasciitilde} ra-nu\\
\lspbottomrule
\end{tabularx}
\end{table}

The question suffix \textit{-niu}, for example, might be analyzable as \textit{-ni-u}, but following Faehndrich an \isi{analysis} as \textit{-n-iu} is more likely. Only in some cases is there a \isi{question marker} with a long vowel, e.g. \textit{-m-uu}.

\ea%55
    \label{ex:mong:55}
    \ili{Mongghul}\\
    \ea
    \gll {qi} {Mongghul} {uguo} {mudie-}\textbf{{nu}}?\\
    2\textsc{sg}  \textsc{pn}    language  know-\textsc{ipfv}.\textsc{q}\\
    \glt ‘Do you know (how to speak) \ili{Mongghul}?’
    
    \ex
    \gll {qi} {wazar} {Minyuen-de} {sou-}\textbf{{niu}}?\\
    2\textsc{sg}   city  \textsc{pn}-\textsc{loc}    sit-\textsc{subj}.\textsc{ipfv}.\textsc{q}\\
    \glt ‘Do you live at Minyuan in the city?’
    
    \ex
    \gll {qi} {laosi} {wei}-\textbf{{u}}?\\
    2\textsc{sg}  teacher  \textsc{subj}.\textsc{cop}-\textsc{q}\\
    \glt ‘Are you a teacher?’
    
    \ex
    \gll {nie-nu} {moxi} shda-gu.na-\textbf{{nu}}?\\
    this-\textsc{acc}  read  can-\textsc{obj}.\textsc{fut}-\textsc{q}\\
    \glt ‘Can you read this?’
    
    \ex
    \gll {ngan} {Zhonggui} {kun} {na-}\textbf{{nu}}?\\
    3\textsc{sg}  \textsc{pn}    person  \textsc{obj}.\textsc{cop}-\textsc{q}\\
    \glt ‘Is he \ili{Chinese}?’ (\citealt{Åkerman2012}: 14)
    \z
    \z 

Alternative \isi{questions} take two question markers that have to be identical in form.

\ea%56
    \label{ex:mong:56}
    \ili{Mongghul}\\
    \gll {qi} {niudur} {xi-}\textbf{{gu}}\textbf{.}\textbf{{niu}}, {malang} {xi-}\textbf{{gu}}\textbf{.}\textbf{{niu}}?\\
    2\textsc{sg}  today    go-\textsc{fut.subj.q}  tomorrow  go-\textsc{fut.subj.q}\\
    \glt ‘Do you go today or tomorrow?’ (\citealt{Åkerman2012}: 14)
    \z

Probably the most aberrant \ili{Mongolic} language with respect to the marking of \isi{questions} is \textit{Mangghuer}. Instead of a simple particle there is a rather elaborate paradigm of forms which, as in \ili{Mongghul}, includes the dimension of perspective (\tabref{tab:mong:9}), typical for adjacent \ili{Tibetic} languages (\sectref{sec:5.9.2}). Nevertheless, the suffixes marking \isi{polar question}s clearly contain the original \isi{interrogative} particle.

\begin{table}
\caption{Paradigm of question marking in Mangghuer (\citealt{Slater2003b}: 316; \citealt{Dixon2012}: 386f.)}
\label{tab:mong:9}

\begin{tabularx}{\textwidth}{XXXXl}
\lsptoprule
&  & \textbf{\textsc{pfv}} & \textbf{\textsc{ipfv}} & \textbf{\textsc{fut}}\\
\midrule
\textsc{decl} or CQ & \textsc{subj} & -ba & -la bi & -ni\\
& \textsc{obj} & -jiang & -lang & -kunang\\
PQ & \textsc{subj} & -bu & -la bi-u & -nu\\
& \textsc{obj} & -jinu & -leinu & -kuninu\\
\lspbottomrule
\end{tabularx}
\end{table}

\ea%57
    \label{ex:mong:57}
    Minhe \ili{Mangghuer}\\
    \ea
    \gll {bi} {ri-}\textbf{{jinu}}?\\
    1\textsc{sg}  come-\textsc{pfv}.\textsc{obj}.\textsc{q}\\
    \glt ‘Did I come?’
    
    \ex
    \gll {qi} {ri-}\textbf{{bu}}?\\
    2\textsc{sg}  come-\textsc{pfv}.\textsc{subj}.\textsc{q}\\
    \glt ‘Did you come?’
    
    \ex
    \gll gan  ri-\textbf{{jinu}}?\\
    3\textsc{sg}  come-\textsc{pfv}.\textsc{obj}.\textsc{q}\\
    \glt ‘Did (s)he come?’ \citep[198]{Slater2003a}\z\z

There is one example of a negative \isi{alternative question} in which only the first alternative receives \isi{question marking} while the second takes a negative marker. In the original, \textit{nu} was written detached from the preceding word.

\ea%58
    \label{ex:mong:58}
    Minhe \ili{Mangghuer}\\
    \gll {ta} {ghula} {qijige} {kerli}{=}\textbf{{nu}} \textbf{{lai}}{-kerli}?\\
    2\textsc{pl}  two  flower  want=\textsc{q}  \textsc{neg}-want\\
    \glt ‘Do you want two flowers or not?’ (\citealt{ZhuYongzhong1997}: 437)
    \z

As expected, there are also special \isi{interrogative} forms of copulas, as shown in \tabref{tab:mong:10}. Again, the original \isi{question marker} can clearly be recognized, but is not completely analyzable.

\begin{table}
\caption{Special interrogative copulas in Mangghuer \citep[318]{Slater2003b}; \citet[199]{Slater2003a} in addition has the variant \textit{meinu}, which is identical to \textit{beinu} in meaning; negative copulas in addition have special attributive forms, \textsc{subj} \textit{(u)gui} and \textsc{obj} \textit{(u)guang}, while there are no such special forms for declarative and interrogative copulas}
\label{tab:mong:10}

\begin{tabularx}{\textwidth}{XXXl}
\lsptoprule
& \textbf{Declarative} & \textbf{Question} & \textbf{Negative}\\
\midrule
\textsc{subj} & bi & biu & puzhi\\
\textsc{obj} & bang & beinu & puzhang\\
\lspbottomrule
\end{tabularx}
\end{table}

\newpage 
\ea%59
    \label{ex:mong:59}
    Minhe \ili{Mangghuer}\\
    \ea
    \gll bi  laoshi \textbf{{meinu}}?\\
    1\textsc{sg}  teacher    \textsc{cop}.\textsc{obj}.\textsc{q}\\
    \glt ‘Am I a teacher?’
    
    \ex
    \gll qi  laoshi \textbf{{biu}}?\\
    2\textsc{sg}  teacher    \textsc{cop}.\textsc{subj}.\textsc{q}\\
    \glt ‘Are you a teacher?’
    
    \ex
    \gll gan  laoshi \textbf{{meinu}}?\\
    3\textsc{sg}  teacher    \textsc{cop}.\textsc{obj}.\textsc{q}\\
    \glt ‘Is (s)he a teacher?’ \citep[199]{Slater2003a}\z\z

Content questions do not take the interrogative forms of copulas.

\ea%60
    \label{ex:mong:60}
    Minhe \ili{Mangghuer}\\
    \gll {tasi} \textbf{{ang}}=ji-ku-ni    bi?\\
    2\textsc{pl}  where=\textsc{dir}-\textsc{ipfv}-\textsc{n}  \textsc{cop}.\textsc{subj}\\
    \glt ‘Where are (all of) you from?’ (\citealt{ChenZhaojun2005}: 16)
    \z

For the Halchighul dialect of \ili{Mangghuer} \citet[61]{Zhaonasitu1981a} mentions the markers \textit{ba} (\ili{Chinese} \textit{ba} \zh{吧}) and \textit{ȿa} (perhaps Hezhou \textit{ʐa}\textsuperscript{3}) with similar meanings.

At a first glance the situation in \ili{Mongghul} is very different from the other languages mentioned thus far, but this is partly due to the difference in description. In fact, \ili{Mongghul} has a strikingly similar system that is given again in \tabref{tab:mong:11}, following the \isi{analysis} by \citet[316]{Slater2003b} and \citet[386f.]{Dixon2012} for \ili{Mangghuer}. In fact, this new \isi{analysis} allows us to analyze some of the forms further than would be possible otherwise. The so-called “why-question” marker \textit{-ji} in \ili{Mangghuer} not shown in \tabref{tab:mong:11} might correspond to \textit{-jii} ‘\textsc{state.subj}’ in \ili{Mongghul}.

\begin{table}
\caption{Paradigm of question marking in Mongghul (\citealt{Åkerman2012}: 13ff.) in comparison with Mangghuer (Mongghul/Mangghuer)}
\label{tab:mong:11}
\begin{tabularx}{\textwidth}{XXlll}
\lsptoprule
&  & \textbf{\textsc{pfv}} & \textbf{\textsc{ipfv}} & \textbf{\textsc{fut}}\\
\midrule
DECL or CQ & \textsc{subj} & -wa/-ba & -nii/\textbf{-la bi} & \textbf{-gu}-nii/-ni\\
& \textsc{obj} & -jia/-jiang & -na/-\textbf{l}ang & -gu-na/-kunang\\
PQ & \textsc{subj} & -wuu/-bu & -niu/\textbf{-}\textbf{la bi-}\textbf{u} & \textbf{-gu}-niu/-nu\\
& \textsc{obj} & -jia-nu/-jinu & -na-nu/-\textbf{l}einu & -gu-na-nu/-kuninu\\
\lspbottomrule
\end{tabularx}
\end{table}

The perspective neutral forms were left aside to make the system more comparable with \ili{Mangghuer}. The two paradigms show both striking similarities and differences. Altogether, the \isi{interaction} between the domains of perspective, aspect, and tense is almost identical. In general, however, the \ili{Mongghul} forms are more readily analyzable. There are slight phonological changes as can be seen from correspondences such as \ili{Mangghuer} \textit{-ba} and Monggul \textit{-wa}. \ili{Mangghuer} has apparently innovative imperfective forms that are a \isi{combination} of the copula \textit{bi}, \isi{interrogative} \textit{bi-u} (\ili{Mongghul} neutral copula \textit{wei}, \textit{wei-u}), and a so-called “imperfective auxiliary linker” \textit{-la} \citep[143]{Slater2003a} that might correspond to the verbal purposive suffix \textit{-la} in \ili{Mongghul} often found before auxiliaries (\citealt{Åkerman2012}: passim). The unexpected objective imperfective forms \textit{-lang} and \textit{-leinu}, corresponding to \ili{Mongghul} \textit{-na} and \textit{-na-nu}, have been contaminated by \textit{-la} but are preserved in the future. In \ili{Mongghul} the future forms are still identical to the imperfective forms, except for the future participle marker \textit{-gu} \citep[300]{Georg2003b}. In \ili{Mangghuer} \textit{-ku} is restricted to the objective forms. This parallel allows an at least historically valid \isi{analysis} of the two \ili{Mangghuer} forms into \textit{-ku-nang} (\ili{Mongghul} \textit{-gu-na}) and \textit{-ku-ni-nu} (\ili{Mongghul} \textit{-gu-na-nu}).



\begin{table} 
\caption{Overview of polar and content question markers in Mongolic languages; intonation patterns are excluded}
\label{tab:mong:12}

\begin{tabularx}{\textwidth}{llQ}
\lsptoprule

\textbf{Language} & \textbf{PQ} & \textbf{CQ}\\
\midrule
\ilit{Dagur} & =yee\# & =yee\#\\
Tacheng \ilit{Dagur} & -jA\# & -jA\#\\
\ilit{Moghol} & - & -\\
\ilit{Khamnigan Mongol} & =gu\# & bei\#\\
\ilit{Buryat} & =gü\# & be {\textasciitilde} =b\#\\
Shineken \ilit{Buryat} & =go\#/=gu\#/=g & be {\textasciitilde} =b\#\\
\ili{Mongolian} & =(y)UU\# (\tabref{tab:mong:1}), =(y)ii\# & bwii\# {\textasciitilde} bwai\#, V-b {\textasciitilde} V-e.b, ?=(y)UU\#\\
\ilit{Khorchin} & =(j)UU\#, =(j)ii\#, =mu\#, ?=me\# & =(j)ii\#, be\#, =me\#\\
\ilit{Oirat} & =(y)UU\#, =ii\# & =b\# {\textasciitilde} =w\#\\
\ilit{Kalmyk} & =u\#, =iy\# {\textasciitilde} =i\# & =b\# {\textasciitilde} =w\#\\
\ilit{Ordos} & =UU\# & -\\
\ilit{Shira Yughur} & =(j)uu\# & bə\#, ?=(j)uu\#\\
\ilit{Mongghul} & =uu, (Tables \ref{tab:mong:8}, \ref{tab:mong:9}) & -\\
\ilit{Mangghuer} & =u, (Tables \ref{tab:mong:10}, \ref{tab:mong:11}) & -\\
\ilit{Bonan} & V-u, (\tabref{tab:mong:3}), V-si, ma\# & -\\
\ilit{Kangjia} & V-ʉ, (\tabref{tab:mong:5}), sa\# & -, le\#\\
\ilit{Santa} & V-u, (\tabref{tab:mong:7}), la\# & ?\\
\lspbottomrule
\end{tabularx}
\end{table}




In sum, for most \ili{Mongolic} languages the information given for \isi{question marking} in grammatical descriptions is not sufficient for a full typology. \tabref{tab:mong:12} summarizes the different \isi{interrogative} marking strategies in \ili{Mongolic} languages for polar and \isi{content question}s, exclusively. From \tabref{tab:mong:12} it becomes apparent that the internal diversity of \isi{interrogative} particles within \ili{Mongolic} is less pronounced than, for example, \ili{Tungusic}. In fact, \ili{Mongolic} languages may be classified into four groups according to their \isi{polar question} markers. Most languages preserve the original polar \isi{question marker} \textit{=UU}. \ili{Dagur} has the form \textit{=yee} instead, \ili{Moghol} apparently lacks any morphosyntactic \isi{question marker}, and \ili{Buryat}, together with \ili{Khamnigan Mongol}, probably borrowed the marker from an \ili{Ewenic} (\ili{Tungusic}) language. \ili{Shirongolic} also forms a subgroup for itself because in all the languages the \isi{question marker} fused with other elements, which results in a much more complicated situation. The \isi{interrogative} marker \textit{-mu} has, according to \citet[384]{Sandman2012}, been borrowed from \ili{Bonan} into the \ili{Sinitic} language \ili{Wutun}, but a \ili{Turkic} origin is more likely (see §§ \ref{sec:5.9.2.1}, \ref{sec:5.11.2}, \ref{sec:6.1}).

\ili{Shirongolic} languages also have special \isi{interrogative} forms for copulas that are given in \tabref{tab:mong:13}.

\begin{table}
\caption{Special interrogative copulas in \ili{Shirongolic} languages}
\label{tab:mong:13}

\fittable{
\begin{tabular}{lll}
\lsptoprule
& \textbf{Declarative} & \textbf{Question}\\
\midrule
\ilit{Bonan} (Wu) & wi \textsc{subj}, wa \textsc{obj,}

mbi \textsc{subj}, mba \textsc{obj} & wu,

mbu\\
\ilit{Kangjia} & i \textsc{subj}, va \textsc{obj,}

mbi \textsc{subj}, mba \textsc{obj} & vʉ,

mbʉ\\
\ilit{Bonan} (Fried) & wi \textsc{subj}, wa \textsc{obj,}

bi \textsc{subj}, ba \textsc{obj} & wu(u) \textsc{subj}, wa-u \textsc{obj},

bu \textsc{subj}, ba-u \textsc{obj}\\
\ilit{Santa} & wo & wu\\
\ilit{Mongghul} & wei \textsc{neut},

wa, ... \textsc{obj} & wei-u \textsc{neut},

wa-nu, ... \textsc{obj}\\
\ilit{Mangghuer} & bi \textsc{subj},

bang \textsc{obj} & bi-u \textsc{subj},

beinu \textsc{obj}\\
\lspbottomrule
\end{tabular}
}
\end{table}

Some of the forms are still analyzable into a copula and a \isi{question marker}, e.g. \ili{Mongghul} \textit{wei-u}. In other languages such a situation may have existed before phonetic erosion and contraction set in, e.g. \ili{Bonan} \textit{wi} + \textit{-u} = \textit{wu}. Monggul and \ili{Mangghuer} have different copula forms depending on the category of perspective, while in some variants of \ili{Bonan}, \ili{Kangjia}, and \ili{Santa} this difference is leveled in the \isi{interrogative}. Special \isi{interrogative} forms of copulas are also known from \ili{Japonic} (\ili{Shuri}, \sectref{sec:5.6.2}) and \ili{Ainuic} (\sectref{sec:5.1.2}).

\subsection{Interrogatives in Mongolic}\label{sec:5.8.3}

There are few good descriptions of interrogatives in \ili{Mongolic}. Most treatments such as those in \citet{Janhunen2003} mention only a handful of forms and leave them mostly unanalyzed. Most grammatical descriptions for \ili{Mongolic} languages also do not mention the syntactic behavior of interrogatives. But they seem to generally remain \textit{in situ} (e.g., \citealt{Fried2010}: 134; \citealt{Napoli2014}: 40). Nevertheless, there are quite reliable reconstructions for Proto-\ili{Mongolic} by \citet[20]{Janhunen2003a} that can serve as a basis for further \isi{analysis} (\tabref{tab:mong:14}, see also \citealt{Poppe1955}: 229f.).

\begin{table}
\caption{Proto-Mongolic reconstructions by \citet[20]{Janhunen2003a} and their modern \ili{Mongolian} correspondences (\citealt{Janhunen2012c}: 130ff.; Benjamin Brosig p.c. 2018)}
\label{tab:mong:14}

\begin{tabularx}{\textwidth}{llQl}
\lsptoprule

\textbf{Meaning} & \textbf{Analysis} & \textbf{Proto-Mongolic} & \textbf{\ili{Mongolian}}\\
\midrule
who &  & *ke/n & xen\\
who \textsc{pl} & \textsc{pl} -d & *ke-d & xed\\
when & & *ke.li & WM keli\\
how &  & *ke.r & xer\\
how many & \textsc{gen} of *ke-d [?] & *ke.d.ü.n & xedn\\
how much & \textsc{acc} of *ke-d [?] & *ke.d.ü.(y)i & xedii\\
when & \textsc{loc} \textsc{*-A} & *ke.d.i.x-e > *ke.j.i.x-e [?] & xedzee\\
where & \textsc{loc} \textsc{*-A} & *ka.mix-a > *kaxa/n-a [?] & xaan’\\
what & & *yan & WM yan\\
what &  & *ya.xu/n & yuun\\
what kind of &  & *ya.m.bar > *yamar & yamer\\
what thing & & *ya.xu.ma > *yexüme & youm\\
to do what & *ki- ‘to do’ & *ya.xa+ki- > *yaxa- [?] & yaa-\\
which &  & *ali/n & alyn\\
\lspbottomrule
\end{tabularx}
\end{table}

Some developments assumed by Janhunen are marked with a question mark as they are not very plausible. There is evidence of a form *\textit{kamixa} ‘where’ in some languages such as Written \ili{Oirat} \textit{xamigha(a)}. In \ili{Middle Mongol}, a form \textit{xamiya} is attested twice (Benjamin Brosig p.c. 2018). Locative interrogatives are not derived from the selective \isi{interrogative} as in \ili{Tungusic} but nevertheless display parallels with the \isi{demonstratives} (\tabref{tab:mong:15}). The \ili{Turkic} language \ili{Dolgan} has a form \textit{kanna} ‘where, whither’, and a related form \textit{xanna} is found in \ili{Yakut}. There are surprisingly similar forms in \ili{Mongolic}, e.g. \ili{Khamnigan Mongol} \textit{kaana} or \ili{Buryat} \textit{xaana} ‘where’. But the \ili{Yakut} and \ili{Dolgan} forms are contractions of an \isi{interrogative} that is still analyzable in other \ili{Turkic} languages including \ili{Sarig Yughur} \textit{qan-ta} ‘which-\textsc{loc}’ (\sectref{sec:5.11.3}).

\begin{table}
\caption{Spatial deictics in \ili{Mongolian} according to \citet[131]{Janhunen2012c}, slightly reduced}
\label{tab:mong:15}

\begin{tabularx}{\textwidth}{XXXl}
\lsptoprule
& \textbf{\textsc{prox} (hearer)} & \textbf{\textsc{dist}} & \textbf{\textsc{int}}\\
\midrule
\textsc{loc} & naa-n’ & tzaa-n’ & xaa-(n’)\\
\textsc{loc abl} & naa-n-aas & tzaa-n-aas & xaa-n-aas\\
\textsc{lat} & naa-sh & tzaa-sh & xaa-sh\\
\textsc{prol} & naa-g.oor & tzaa-g.oor & xaa-g.oor\\
\lspbottomrule
\end{tabularx}
\end{table}

\tabref{tab:mong:16} shows five of the interrogatives that can be found in most modern \ili{Mongolic} languages.

\begin{table}
\caption{Five Proto-Mongolic interrogatives and their modern representatives}
\label{tab:mong:16}
\small
\begin{tabularx}{\textwidth}{lQQQQQ}
\lsptoprule
& \textbf{*ken}

\textbf{‘who’} & \textbf{*yaxun}

\textbf{‘what’} & \textbf{*alin}

\textbf{‘which’} & \textbf{*kejixe}

\textbf{‘when’} & \textbf{*kaxana}

\textbf{‘where’}\\
\midrule
\ili{Dagur} & xeng & yoon & aly & xejee & xaan\\
\ili{Mongolian} & xen & yuun & alyn & xedzee & xaan’\\
\ili{Buryat} & xen & yüün & ali & xezee & xaana\\
\ili{Khamnigan Mongol} & ken & yeen & ali & kejie & kaana\\
\ili{Ordos} & ken & yüün & ali & kejee & kaa\\
Written \ili{Oirat} & ken & you/n & ali & kezee & \textbf{xamigha(}\textbf{a)}\\
\ili{Oirat} & ken & yuu/n & äl {\textasciitilde} äl-k & keze & \textbf{xamaa}\\
\ili{Kalmyk} & ken & yuun & aly(-k) & kezä & \textbf{xama}, \textbf{aly-}\textbf{d}\\
\ili{Shira Yughur} & ken & \textbf{yima} & aali & kejee & xana\\
\ili{Santa} & kien & yang & ali & giezhe & khala\\
\ili{Bonan} & kang & yang & \textbf{ane} & kece(-) & hala\\
\ili{Kangjia} & kɔ & jɔ {\textasciitilde} jaŋ & \textbf{ani(ɣe)} & gədʒe & χana\\
Huzhu \ili{Mongghul} & ken & ya(a)n & ali & kijee & \textbf{an-ji(i)}\\
Minhe \ili{Mangghuer} & kan & ya, yang & \textbf{a(yi)ge} & kejie & \textbf{ang(-ji)}\\
\ili{Moghol} & ken {\textasciitilde} kiyan & \textbf{emah} {\textasciitilde} \textbf{imas} etc. & ? & keja & ?\\
\lspbottomrule
\end{tabularx}
\end{table}

According to \citet[20]{Janhunen2003a} the stem *\textit{ke-} originally had the meaning ‘who’ as well as ‘what’, which is an unlikely scenario from a cross-lingusitic point of view. As has been shown by \citet{Cysouw2005}, the only place worldwide where this pattern is not extremely rare or altogether absent is South America.

Proto-\ili{Mongolic} had two resonances (submorphemes), one in *\textit{k{\textasciitilde}} that is still present in most \ili{Mongolic} languages but changed to \textit{x{\textasciitilde}} in \ili{Dagur}, \ili{Buryat} and \ili{Mongolian}, and one in *\textit{y{\textasciitilde}} that has survived up to today. Similar changes from *\textit{k{\textasciitilde}} to > *\textit{x{\textasciitilde}} can be seen in \ili{Turkic} (e.g., \ili{Khakas}, \sectref{sec:5.11.3}) or \ili{Tungusic} languages (e.g., \ili{Nanai}, \sectref{sec:5.10.3}). Only the \isi{interrogative} *\textit{ali/n} ‘which’ does not fit into either type. All \ili{Mongolic} languages thus have what has been called K-interrogatives in this study. Furthermore, \ili{Mongolic} also possesses the \isi{KIN-interrogative}. \ili{Amuric} (\sectref{sec:5.2.3}) and especially \ili{Tungusic} languages (\sectref{sec:5.10.3}) exhibit several interrogatives that may have been borrowed from \ili{Mongolic}.

In the following, I will address interrogatives in individual \ili{Mongolic} languages in turn. \tabref{tab:mong:17} summarizes the interrogatives found in four descriptions of \textbf{Dagur}. The etymology of most of these forms has already been given above.

\begin{table}
\caption{Interrogatives in Dagur (\citealt{Martin1961}: 30f., passim; \citealt{ZhongSuchun1982}: 52; \citealt{Chaolu1996}: 22; \citealt{Tsumagari2003}: 141f.; \citealt{Yu2008}: 63, passim)}
\label{tab:mong:17}

\begin{tabularx}{\textwidth}{lQllQQQ}
\lsptoprule
& \textbf{Martin} & \textbf{Zhong} & \textbf{Chaolu} & \textbf{Tsumagari} & \textbf{Yu et al.}\\
\midrule
which & ali, alin (attr.) & alj & alʲ & aly {\textasciitilde} alin- & ali {\textasciitilde} aalj\\
who & hen & xən, anii & xən, aniin & xeng, aniing & anija {\textasciitilde} anja\\
what & joo & joo & jɔɔ & yoo {\textasciitilde} yoon- & jo, jooke, joon\\
how many/much &  & jookee & jɔɔkəə & yookie {\textasciitilde} yiekie & jooked\\
why & iuuu & juguu &  & yoondaa, yuguu & jugoo\\
what (kind of) & iamare {\textasciitilde} iamere & jamər & jamər & yamer & jamər\\
how many & hede, heden (attr.) & xədəŋ & xəd & xed & xəd, xədii, xədən (attr.)\\
when & hejee & xədʒee & xədʒəə & xejee & xəǰəə, xəǰəər\\
how & here & xər &  & xer, yoondaa & xərəə\\
where & haane & xaanə & xaanə & xaan & xaan\\
whither &  &  & xaidaa & xaidaa & xaan, xandiin\\
whence &  &  &  &  & xaanaar\\
\lspbottomrule
\end{tabularx}
\end{table}

\citet{Chaolu1996} also mentions an \isi{interrogative verb} \textit{jee-} ‘to do what’. The form \textit{iuuu} {\textasciitilde} \textit{juguu} etc. ‘how, why’ may be cognate with \ili{Buryat}, Chakhar, \ili{Khalkha}, and Khamnigan \textit{yaa/g-aad} ‘how, why’, which is a perfective \isi{converb} form of an \isi{interrogative verb} that has the form \textit{-g/-AA(r)} > *\textit{-g/-AAd} in \ili{Dagur} (\citealt{Tsumagari2003}: 145f.). The medial \textit{-g-} may either be part of the \isi{converb} or, less likely, of the verbalizer that has the form \textit{-ge} in \ili{Buryat} or Khamnigan. The \isi{resonance} *\textit{k{\textasciitilde}} changed to *\textit{x{\textasciitilde}} in \ili{Dagur}, but not in all dialects. The change did not take place in the Qiqihar dialect, which has forms such as \textit{kuu} ‘person’ or \textit{kər} ‘how’ as opposed to \textit{xuu} and \textit{xər} in other dialects (\citealt{Ding1995}: 191). \ili{Dagur} preserved the original \isi{interrogative} \textit{xən} ‘who’ but also has an innovative form \textit{anii} that ultimately might be somehow related to *\textit{ali(n)} ‘which’. Similarly, the two \ili{Tungusic} \isi{contact} languages of \ili{Dagur}, \ili{Oroqen} and \ili{Solon} have a form \textit{a(a)wu}, which originally meant ‘which one’ but has extended its meaning and has partly replaced the form \textit{ni(i)} ‘who’ that goes back to \ili{Proto-Tungusic} (\sectref{sec:5.10.3}). If true, this could be an instance of a shared \isi{grammaticalization}. But the exact etymology of the \ili{Dagur} \isi{interrogative} is not entirely clear. The suffix in \textit{yoon-daa} is likely a dative ending followed by the reflexive marker, i.e \textit{-d-AA} \citep[143]{Tsumagari2003}. This is also the \isi{analysis} of the form \textit{joon-d-ee} found among the \ili{Dagur} \isi{interrogative} paradigms collected by \citet[30]{Martin1961}. Martin also lists a plain dative form \textit{joon-de}, which is likely the source of Nanmu \ili{Oroqen} \textit{joonde} and could also somehow be connected with \ili{Solon} \textit{yoodon}.

The interrogatives in \textit{Khamnigan Mongol} and \textit{Buryat} (\tabref{tab:mong:18}) are very similar. Khamnigan has a more conservative phonology and preserves the initial *\textit{k{\textasciitilde}}, which changed to \textit{x{\textasciitilde}} in \ili{Buryat}. Interrogatives with the meaning ‘why’ and ‘how’ are derived with the help of the same \isi{case} and converbial markers in both languages. \citet{Yamakoshi2007a} mentions a Khamnigan form \textit{kədui cag-} ‘what time’, which is probably a loan translation of \ili{Mandarin} \textit{j\u{\i} diăn} (\sectref{sec:5.9.3.1}). Castrén (1857a) collected several paradigms of interrogatives and \isi{demonstratives} that are given in a simplified and analyzed version in \tabref{tab:mong:19}. The paradigms are clearly identical in their \isi{case} forms, but the \isi{demonstratives} take an additional stem augmentation \textit{/n}.

\begin{table}
\caption{Interrogatives in Buryat (\citealt{Yamakoshi2011a}: 170; \citealt{Skribnik2003}: 111) and Khamnigan Mongol (\citealt{Janhunen2003c}: 92; \citealt{Yamakoshi2007a}: passim); Buryat also has \textit{xedii-dexi} ‘how manieth’ and \textit{xedii-lüülen} ‘in a group of how many’; some variants were excluded}
\label{tab:mong:18}

\begin{tabularx}{\textwidth}{lQQQQ}
\lsptoprule
& \textbf{Shineken Buryat} & \textbf{Standard Buryat} & \textbf{Khamnigan}
\textbf{(Janhunen)} & \textbf{Khamnigan (Yamakoshi)}\\
\midrule
who & xe/n & xe/n & ken & kən\\
how many & xedii & xedii.n- &  & kədui\\
when & xezee & xezee & kejie & kəzie\\
how &  & xer & ker & \\
where & xaa-(na) & xaa-(na) & kaa-na & kaa-na\\
whither & xaa-sʲ & xai-sha & kaa-si & kaa-s\\
whence &  & xaana-haa &  & \\
through where & xaa-g-oor & xaa/g-uur &  & \\
what kind of & jamar & yamar & yamar & jamar\\
what & jun, juu & yüü/n & \mbox{yee/n {\textasciitilde} yuu/n} & jun, joon\\
why (what-\textsc{dat}) & juun-de & yüün-de & yeen-du & joon-do\\
to do what (what-\textsc{v}-) & jaa- & yaa-(ge-) & yaa-(g-) & joo ki-\\
how, why (\textsc{ipfv}.\textsc{cvb}) & jaa-zʲa & yaa-zha &  & \\
how, why (\textsc{pfv}.\textsc{cvb}) &  & yaa-g-aad & yaa-g-aad & \\
which & alʲ & ali &  & ali\\
\lspbottomrule
\end{tabularx}
\end{table}

\begin{table}
\caption{Simplified paradigms of interrogatives and demonstratives in Buryat according to \cite[31ff.]{Castrén1857a}; only singular forms and not all variants are shown}
\label{tab:mong:19}

\begin{tabularx}{\textwidth}{XXXXXl}
\lsptoprule
& \textbf{this} & \textbf{that} & \textbf{who} & \textbf{what} & \textbf{which}\\
\midrule
plain & e.nê & te.rê & ke/n & jụ/n & ałi/n\\
\textsc{acc} & enê/n-i & terê/n-i & ken-i & jụn-i & ałin-i\\
\textsc{dat} & enê/n-de & terê/n-de & ken-de & jụn-de & ałin-da\\
\textsc{abl} & enê/n-ehe & terê/n-ehe & ken-ehe & jụn-ehe & ałin-aha\\
\textsc{inst} & enê/n-er & terê/n-er & ken-er & jụn-er & ałin-ar\\
\textsc{com} & enê/n-tei & terê/n-tei & ken-tei & jụn-tei & ałin-tai\\
\lspbottomrule
\end{tabularx}
\end{table}

\begin{table}
\caption{Paradigms of interrogatives and demonstratives in Dagur according to \cite[28ff.]{Martin1961} in analogy to \tabref{tab:mong:19}.}
\label{tab:mong:20}

\begin{tabularx}{\textwidth}{XXlXXl}
\lsptoprule
& \textbf{this} & \textbf{that} & \textbf{who} & \textbf{what} & \textbf{which}\\
\midrule
plain & e.ne & te.re & he/n & joo & ali\\
\textsc{acc} & e.n-ii & te.r-ii & hen-ii & joon-ii & ali-i\\
\textsc{dat} & e.n-de & te(.re)/n-de & hen-de & joon-de & alin-de\\
\textsc{abl} & e.n-eese & te.r-eese & hen-eese & joon-oose & ali-eese\\
\textsc{inst} & e.n-eere & te.r-eere & hen-eere & joon-oore & ali-eere\\
\textsc{com} & e.n-tei & te.re-tei & hen-tei & joon-tei & ali-tei\\
\lspbottomrule
\end{tabularx}
\end{table}

As in \ili{Dagur} and other languages below the dative \isi{case} form of the \isi{interrogative} meaning ‘what’ has acquired the meaning ‘why’. For comparison, \tabref{tab:mong:20} lists \ili{Dagur} demonstrative and \isi{interrogative} paradigms, but excludes reflexive \isi{case} markers. There are some differences in phonology and \isi{morphology} such as the lack of the \textit{/n} in several \ili{Dagur} forms. However, given the overall \isi{similarity} of paradigms, these will not be listed for all languages below.

In \textit{\ili{Mongolian}} the same change from *\textit{k{\textasciitilde}} to \textit{x{\textasciitilde}} as in \ili{Buryat} occured (\tabref{tab:mong:22}). According to Mostaert’s account of \ili{Ordos} (\tabref{tab:mong:23}), the initial \textit{k{\textasciitilde}} is preserved except for \textit{χaa} ‘where’.

\begin{table}
\caption{Interrogatives in \ili{Mongolian} (\citealt{Janhunen2012c}: 130ff., 255f.) and in Chakhar \citep{Sechenbaatar2003}, Darkhat (\citealt{Gáspár2006}: 46), and Khorchin dialects (\citealt{Yamakoshi2015}: passim); not all forms and variants are listed}
\label{tab:mong:21}

\begin{tabularx}{\textwidth}{QlQlQ}
\lsptoprule
& \textbf{\ili{Mongolian}} & \textbf{Chakhar} & \textbf{Darkhat} & \textbf{Khorchin}\\
\midrule
who & xe/n & xeng {\textasciitilde} xen- & xen & xən\\
how many & xed//n & \mbox{xed, xede-ng (attr.)} &  & xədən\\
how much & xedii & xedii &  & xədii, xədəə\\
when & xedzee & xejee, xediis & xejee & xəǰəə\\
how & xer &  &  & \\
where & xaa(-n’) & xaa(-na=n) & xaa(-nă) & xaa\\
what & yuu/n & yuu, yuu/n & yuu & jʊʊ\\
why (\textsc{dat}) & yuun-d &  & yuun-d & \\
how, by what means (\textsc{inst}) & yuu-geer &  & yuu-gaar & \\
what kind of, how & yamer & yamar & yamăr & jamar\\
to do what & yaa-, yuu xii- &  & yaa- & jaa(-x)-, jʊʊ xii-\\
how (\textsc{cvb.ipfv}) & yaa-j &  & yaa-ǰ & jaa-ǰ\\
why (\textsc{cvb.pfv}) & yaa-gaad &  & yaa-gaad & \\
how much/many &  & gecneeng &  & \\
which & aly//n & aly {\textasciitilde} äly/n {\textasciitilde} älya/n & äl\textsuperscript{i} & æl'\\
\lspbottomrule
\end{tabularx}
\end{table}

There are additional forms such as \ili{Khorchin} \textit{jʊʊ gə-ǰ} ‘what say-\textsc{cvb.ipfv} > why’, which is completely parallel to \ili{Manchu} \textit{ai se-me}, and a \ili{Mongolian} \isi{interrogative verb} \textit{xaa-c-} (< \textit{xaa-oc-}) ‘to go where’. The \isi{semantic scope} of \textit{yamer} ‘what kind of, how’ suggests a connection with \ili{Turkic} languages (\sectref{sec:5.11.3}). Chakhar \textit{gecneeng} has a cognate in \ili{Ordos} \textit{ɢe‘tś‘ineen} and Cyrillic \ili{Khalkha} \textit{xecneen} (Benjamin Brosig p.c. 2016). Instead of \textit{xedii} \ili{Khalkha} usually has the complex form \textit{xer olon} ‘how much’ (Benjamin Brosig p.c. 2016), which might be a calque of a common European formation transmitted via \ili{Russian} \textit{kak mnogo}/\textcyrillic{как много}. \citet[202]{Georg2003a} only mentions a few forms for \ili{Ordos} (\textit{ken} ‘who’, \textit{gecineen} ‘how much’, \textit{kejee} ‘when’, \textit{kaa} ‘where’, \textit{yüü/n} ‘what’, and \textit{yamar} ‘what kind of’). But in his list \textit{kaa} still preserves the initial \textit{k-}.

\newpage 
Similar to \ili{Ordos}, in both \textbf{Oirat} and \textbf{Kalmyk} the *\textit{k{\textasciitilde}} remained stable in the stem *\textit{ke-} but changed to \textit{x} in the stem *\textit{kaa-} (\tabref{tab:mong:22}). The same is true for \ili{Shira Yughur} and maybe for \ili{Santa} as well (see below). The form \textit{aly-}\textit{d} ‘where’ in Kalmyk clearly is a locative (dative) form of the \isi{interrogative} \textit{aly} ‘which’. Spoken \ili{Oirat} \textit{äl-k} and Kalmyk \textit{aly-k} in all likelihood have the same origin as \ili{Mangghuer} \textit{ali-ge}. Instead of \ili{Kalmyk} \textit{xamaran} ‘whither’ my informant employed the form \textit{al’daran}, based on \textit{aly} ‘which’.

\begin{table}
\caption{Interrogatives in Oirat and Kalmyk (\citealt{Birtalan2003}: 220; \citealt{Bläsing2003}: 239)}
\label{tab:mong:22}

\begin{tabularx}{\textwidth}{Xlll}
\lsptoprule
& \textbf{Written Oirat} & \textbf{Spoken Oirat} & \textbf{Kalmyk}\\
\midrule
who & ken & ken & ken\\
when & kezee & keze & kezä\\
how much/many & kedüi {\textasciitilde} kedüü & kedn {\textasciitilde} kedü & kedü {\textasciitilde} kedü/n-\\
what & you/n & yuu/n & yuun\\
to do what &  &  & yagh-\\
why (-ad \textsc{cvb.pfv}) &  &  & yagh-ad\\
what kind of & yamaaru & yamr/n & yamr {\textasciitilde} yamaran\\
where (-d \textsc{loc}) & xamigha {\textasciitilde} xamighaa & xamaa & xama, aly-d\\
whither &  &  & xamaran\\
which & ali & äl {\textasciitilde} äl-k & aly {\textasciitilde} aly-k\\
\lspbottomrule
\end{tabularx}
\end{table}

In \textbf{Shira Yughur}, however, roughly half of the interrogatives show a \isi{resonance} in \textit{y{\textasciitilde}}. Most interrogatives are either inherited from \ili{Proto-Mongolic} or have a straightforward explanation such as a contraction with a following verb or the presence of a \isi{case} marker. Only the form \textit{yima} ‘what’ clearly differs from \ili{Mongolian}. Its explanation is probably related to the change of meaning of the \isi{interrogative} \textit{yaan} from ‘what’ to ‘how’.

The \textbf{Bonan} \isi{interrogative} \textit{χala} ‘where’ similar to \ili{Santa} \textit{khala} has a liquid \textit{l} instead of a nasal \textit{n} (cf. \ili{Mongolian} \textit{xaan’}). Whether \textit{anə} ‘which’ is cognate with \ili{Mongolian} \textit{alyn} ‘which’ or rather \ili{Dagur} \textit{anii} ‘who’ remains unclear to me. The forms \textit{jant\textsuperscript{h}}\textit{oχ} and \textit{yamtig} from the two different descriptions probably represent dialectal variants of one and the same \isi{interrogative}. The \isi{interrogative} \textit{yamten’ge} ‘how much’ is said to result from a fusion with the numeral \textit{nege} ‘one’. This development of the numeral ‘one’ appears to have been influenced by \ili{Tibetic} (§§\ref{sec:3.5}, \ref{sec:5.9.3.2}). Both ‘how’ and ‘why’ are clearly based on the \isi{interrogative verb} \textit{jaŋ-gə-}, which in turn is transparently derived from \textit{jaŋ-} ‘what’.

\begin{table}
\caption{Interrogatives in Shira Yughur (\citealt{Nugteren2003}: 273; \citealt{Zhaonasitu1981b}: 27, passim), and Ordos (\citealt{Mostaert1937}: passim)}
\label{tab:mong:23}

\begin{tabularx}{\textwidth}{Qlll}
\lsptoprule
& \textbf{Shira Yughur} &  & \textbf{Ordos}\\
\midrule
& \textbf{Nugteren} & \textbf{Zhaonasitu} & \\
\midrule
who & ken & ken & ke‘n\\
what &  &  & jɯɯ, juun\\
how much (niɣe ‘one’) & keedi & keedə niɣe & ke‘\textsc{d}\textsuperscript{u}ii, ɢe‘tś‘ineen\\
how many & keden & keedə & ke‘dɯ\\
when & kejee & kedʒee & ke‘\textsc{d}žee\\
where & xana & χana & χaa(-na)\\
whither & xagshi & χaɢʃə & \\
whence & xanasa &  & \\
to go where (*kaana od- >) & xanad- &  & \\
how (\textsc{inst}) & yaan &  & j{u͔}{u͔}-gaar\\
why (\textsc{dat}, \textsc{cvb.pfv}) & yaan-di & jaan-də & juun-\textsc{d}{u͔}, juundaan, jaaχ\textsuperscript{k}χ{u͔}{u͔}n\\
what & yima & ima & \\
what kind of, how & yimar & imar & jamar\\
to do how & yaa-gi- & ima-ɣə- & \\
to do what & yima-gi- &  & jaa-(χ\textsuperscript{k}χi-)\\
to happen how (*yama bol-) & yimal- &  & \\
which & aali & aalə & ali\\
\lspbottomrule
\end{tabularx}
\end{table}

\textbf{Kangjia} has the same derivation \textit{jaŋ-gi-} ‘to do what’ but the stem \textit{jaŋ} ‘what’, possibly in analogy to \textit{kɔ} ‘who’, also has the alternative form \textit{jɔ} ‘what’. As opposed to \ili{Bonan}, but similar to \ili{Santa}, there is an \isi{interrogative} stem \textit{ma-}. As opposed to \ili{Bonan} \textit{yamten’ge} ‘how much’, \ili{Kangjia} has \textit{ma-tu niɣe} that is derived from \textit{matu} but is likewise based on the numeral ‘one’.

\largerpage[-1]
The origin of the \textbf{Santa} \isi{interrogative} \textit{dʑidʑiən-də} is \ili{Chinese} \textit{j\u{\i}-diăn} ‘what time’. But it contains an autochthonous dative (locative) marker \textit{-d}\textit{ə} that is also present in the complex expression \textit{ali orŋ}\textit{-d}\textit{ə}, which literally means ‘at which place’ and has parallels in several languages such as \ili{Mandarin} \ili{Chinese} \textit{zài shénme dìfang} ‘(\textsc{cop}.)\textsc{loc} what place’ but also \ili{Kangjia} \textit{ani satʃa}. Another loan from \ili{Mandarin} is the second part of \textit{yan shihou} ‘what time’ (\ili{Mandarin} \textit{shíhòu} ‘time’) that is also present in \ili{Kangjia} \textit{ani-ɣe sɯ$\chi əʉ$-dʉ}.

\begin{table}
\caption{Interrogatives in Bonan (\citealt{Fried2010}: 144, 261; \citealt{WuHugjiltu2003}: 337, 342) and Kangjia (\citealt{Siqinchaoketu1999}: 185ff., passim; 2002: 73); see also \citet[187]{Todaeva1963}}
\label{tab:mong:24}

\begin{tabularx}{\textwidth}{lQQl}
\lsptoprule
& \textbf{Bonan} &  & \textbf{Kangjia}\\
\midrule
& \textbf{Fried} & \textbf{Hugjiltu} & \\
\midrule
who & k\textsuperscript{h}aŋ & kang & kɔ\\
whose (-gaŋ \textsc{poss}) & k\textsuperscript{h}aŋ-gaŋ & kang-g(h)ang & \\
how many & k\textsuperscript{h}ət\textsuperscript{h}oŋ & kudung {\textasciitilde} kutung & gʉdɔ {\textasciitilde} gədo\\
which (niɣe ‘one’) & anə & ane & ani, aniɣe\\
when (-dʉ \textsc{dat}) &  &  & aniɣe sɯχəʉ-dʉ\\
where (satʃa ‘place’) &  &  & ani satʃa\\
what thing &  &  & jama(-sʉ(n))\\
where & χala & hala & χana\\
whence &  &  & χana-sa(la)\\
when & k\textsuperscript{h}ətɕə {\textasciitilde} k\textsuperscript{h}ətɕə-χaŋ & kece {\textasciitilde} kece-xangnang & gədʒe\\
what (=gə \textsc{sg}.\textsc{ind}) & jaŋ {\textasciitilde} jaŋ-gə & yang & jɔ {\textasciitilde} jaŋ\\
to do what (-gə/-ge/-gi \textsc{v}) & jaŋ-gə- & yang-ge- & jaŋ-gi-\\
how (-tɕə/-je \textsc{cvb.ipfv}) & jaŋ-gə-tɕə & yang-ge-je & \\
why (-saŋ \textsc{p.pfv}, ?-da \textsc{dat}) & jaŋ-gə-saŋ & yang-ge-da & \\
what kind of & jant\textsuperscript{h}oχ & yamtig & \\
how much (nege/niɣe ‘one’) &  & yamten’ge & ma-tu niɣe\\
how (-tu \textsc{adj}) &  &  & ma-tu\\
to do how (-gi \textsc{v}) &  &  & ma-tu-gi- {\textasciitilde} ma-gi-\\
\lspbottomrule
\end{tabularx}
\end{table}

\begin{table}
\caption{Interrogatives in Santa (\citealt{Chaolu1994b}: passim; \citealt{Kim2003}: 356; \citealt{Napoli2014}: 40); see also \citet[360]{Field1997}; not all variants are listed}
\label{tab:mong:25}

\begin{tabularx}{\textwidth}{llXX}
\lsptoprule
& \textbf{Chaolu Wu} & \textbf{Kim} & \textbf{Napoli}\\
\midrule
who & kiən & kien & kien\\
how much/many & giəduŋ, giədu-ʁaŋ & giedun & giedun\\
when &  & giezhe & giezhe\\
where & qala & khala & khala\\
whence &  &  & khala-se\\
what & jaŋ, ja & yang & yan\\
why (gie-zhi ‘do/say-\textsc{cvb}.\textsc{ipfv}’) &  &  & yan gie-zhi\\
which & ali & ali & ali\\
where (-də \textsc{dat}) & ali orŋ-də &  & \\
how (gie- ‘to do/say’) & matu gie-, matu-kaŋ &  & matu gie-\\
when, what time & dʑidʑiən-də &  & yan shihou\\
\lspbottomrule
\end{tabularx}
\end{table}

The form \textit{matu} ‘how’ looks very untypical for \ili{Mongolic}. According to \citet[194]{Siqinchaoketu1999}, \ili{Kangjia} \textit{ma-} is an abbreviated form of \textit{jama}, which seems possible but is in need of further explanation. A more plausible alternative would be a \ili{Sinitic} origin, e.g. \ili{Mandarin} \textit{mà} (\sectref{sec:5.9.3.1}). The second part of \textit{ma-tu} could be a derivational suffix that attaches to nouns to form adjectives (\citealt{Kim2003}: 352). It may be noted that in \ili{Kangjia} the suffix \textit{-tu} is optional in the \isi{interrogative verb} \textit{ma(-tu)-gi-} ‘to do how’. Interestingly, in \ili{Santa} \textit{matu} is usually followed by the verb \textit{gie-} ‘to say, to make, to think’ unless it has the form \textit{ma-tu-kaŋ}. In fact, \ili{Santa} \textit{ma-tu gie-} looks suspiciously similar to \ili{Kangjia} \textit{ma-tu-gi-}. The suffix \textit{-kaŋ} possibly derives nouns from adjectives (\textit{-ghang} in \citealt{Kim2003}: 352), which would explain why it is followed by a copula in the following example.

\newpage
\ea%61
    \label{ex:mong:61}
    \ili{Santa}\\
    \ea
    \gll {tɕi} \textbf{{ma.tu}} \textbf{{gie}}{-wo?}\\
    2\textsc{sg}  how  do-\textsc{term}\\
    \glt ‘How are you doing?’
    
    \ex
    \gll ən.udu.ku  tɕientɕi \textbf{{ma.tu.kaŋ}} wo?\\
    today    weather  how    \textsc{cop}\\
    \glt ‘How is the weather today?’ (\citealt{Chaolu1994b}: 17)
    \z
    \z

\noindent In \citet[288]{Todaeva1959} we find an additonal form \textit{ma.tu.n-ni} ‘what kind of’ with a third person possessive ending. The complex \isi{interrogative} \textit{yan gie-zhi} ‘why, what for’ can be analyzed as ‘what do/say-\textsc{cvb}.\textsc{ipfv}’ and is a parallel to \ili{Bonan} \textit{yang-}\textit{ge-je}. Not mentioned in \tabref{tab:mong:25} are \isi{plural} forms such as \textit{jan-la} (\citealt{MaGuoliang1986}: 174), which carries the special \ili{Santa} \isi{plural} marker of unclear origin. The initial *\textit{k} has three or four different reflexes (\textit{k}, \textit{g}, and \textit{q} {\textasciitilde} \textit{kh}). It may be noted that today in both \ili{Santa} and \ili{Kangjia} the \isi{interrogative} ‘who’ is the only \isi{interrogative} starting with a \textit{k-}, for which there may well be functional rather than phonological reasons (\sectref{sec:6}).

The majority of interrogatives in \textbf{Mongghul} start with \textit{a{\textasciitilde}}, several with \textit{k{\textasciitilde}} and only one or two with \textit{y{\textasciitilde}} (\tabref{tab:mong:26}). \citet[127]{Faehndrich2007} mentions one form \textit{tiɢaan} ‘how many’ that seems to have been borrowed from an unknown source. For the Halchighul dialect of \ili{Mongghul}, \cite[151]{Schröder1964} lists the interrogatives \textit{kän} ‘who’, \textit{yan} ‘what’, \textit{ali} ‘which’, and \textit{kidi} ‘how much’.

\begin{table}
\caption{Interrogatives in Huzhu Mongghul (\citealt{Chaolu1994c}: passim; \citealt{Faehndrich2007}: 127; \citealt{Dpal-ldan-bkra-shis1996}: passim)}
\label{tab:mong:26}

\begin{tabularx}{\textwidth}{lllQ}
\lsptoprule
& \textbf{Chaolu Wu} & \textbf{Faehndrich} & \textbf{Dpal-ldan-bkra-shis}\\
\midrule
who & ken & kani \textsc{subj}, kana \textsc{obj} & ken\\
how much/many &  & kɨdɨ, tiɢaan & kidi-hangi\\
when (-dɨ \textsc{dat}) & kɵdʑee & kɕee, ali-sxuu-dɨ & kijee, ali sghuu\\
which & alɵ &  & ali\\
which one &  &  & alinga, alingi\\
what kind of &  &  & amahgi sanba\\
how, why & amaɢa, amarr- & amagɨdʑa & amaga, amakiji\\
where & andʑii & andʑii \textsc{subj}, andʑa \textsc{obj} & anji(i)\\
what & jaan & jaanii \textsc{subj},  jaana \textsc{obj} & yan, yanna\\
\lspbottomrule
\end{tabularx}
\end{table}

\clearpage %solid chapter boundary
The second part of \textit{amahgi sanba} ‘what kind of’ mentioned by \cite[232, 241]{Dpal-ldan-bkra-shis1996} means ‘kind, type, pattern’. The form \textit{ali sghuu}/\textit{ali-sxuu-dɨ} literally means ‘at which time’. A speciality of \ili{Mongghul} is the presence of a perspective distinction in several interrogatives, which was only mentioned by \citet{Faehndrich2007}.

\begin{table}
\caption{Interrogatives in Minhe Mangghuer (\citealt{Slater2003a}: 55, 86; \citealt{Dpal-ldan-bkra-shis1996}: passim)}
\label{tab:mong:27}

\begin{tabularx}{\textwidth}{XXl}
\lsptoprule
& \textbf{Slater} & \textbf{Dpal-ldan-bkra-shis}\\
\midrule
who & kan & kan\\
how much/many &  & kedu\\
when &  & kejie\\
why, how (=la \textsc{inst}, =ji \textsc{dir}) & ya=la, ya=ji, ya & ya-la, ya-ge, ya-ji\\
what & yang, ya & yang\\
which (one) & ayi-ge & ali, a-ge, ali-ge\\
what kind of & amer-da & yamer(-da)\\
where (from yang?) & ang & ang-ji\\
\lspbottomrule
\end{tabularx}
\end{table}

Neither \cite{Slater2003a,Slater2003b}, nor \cite{Dpal-ldan-bkra-shis1996} give a clear \isi{analysis} of the \textbf{Mangghuer} interrogatives (\tabref{tab:mong:27}). But \textit{kan}, \textit{kedu}, \textit{kejie}, \textit{yang}, \textit{ali}, and \textit{yamer} are clearly of \ili{Proto-Mongolic} origin. The form \textit{angji} ‘where’, also present in \ili{Mongghul} as \textit{anjii}, probably contains a \isi{case} ending that was given as a directive \textit{=ji} by \citet[312]{Slater2003b} and is specific to \ili{Mangghuer}. Problematically, it expresses only direction but not location, for which there is the dative/locative \textit{=du}. The form \textit{ya=ji} ‘why’ thus literally means ‘where to’ (cf. \ili{English} \textit{to what end}). The comparison of the two forms \textit{amerda} and \textit{yamerda} (both with an unclear suffix \textit{-da}) with and without initial approximant suggest that the form \textit{ang} might be a variant of the interroagtive \textit{yang} ‘what’. The \isi{interrogative} \textit{ayige} or \textit{age} probably contains the indefinite \isi{singular} marker \textit{=ge}, which is either derived from the \ili{Chinese} classifier \textit{ge} \zh{个} (via the loan \textit{yige} ‘one’, from \textit{yí-gè} ‘one-\textsc{clf}’), or from the autochthonous numeral \textit{nige} ‘one’ (cf. \citealt{Slater2003a}: 100). This \isi{analysis} is corroborated by the form \textit{ali-ge} ‘which one’. But the first part \textit{ayi-} or \textit{a-} remains unclear from a language-internal perspective. Most likely it has been borrowed from a \ili{Sinitic} language (\sectref{sec:5.9.3.1}). The corresponding form in \ili{Mandarin} is \textit{nă(-yi)-ge} with or without the numeral ‘one’. This may explain the difference between \textit{ayi-ge} and \textit{a-ge}. In \ili{Sinitic} languages of the area the \isi{interrogative} lost the initial nasal, e.g. Hezhou/Linxia \textit{a-ʒi}\textsuperscript{24}\textit{-gə {\textasciitilde} a-ji}\textsuperscript{24}\textit{-gə}.
\clearpage %solid chapter boundary