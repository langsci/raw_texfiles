\chapter{An overview of language families in Northeast Asia}

The validity of all fourteen language families of \isi{NEA} has been proven by means of the classical comparative method. \citet{Hammarström2016} list about 430 different language families worldwide. Of these, \ili{Niger-Congo} (called “Atlantic-Congo”, 1430 languages) and \ili{Austronesian} (1274 languages) are, in terms of individual languages, the two largest ones. \ili{Indo-European} (583 languages) and \ili{Trans-Himalayan} (475 languages) follow in places three and four. All other families found in \isi{NEA} are considerably smaller, with several dozen languages at most. As regards the size of the individual languages, i.e. the number of speakers, there are similarly pronounced differences. By counting native speakers only, \ili{Mandarin} is the largest language worldwide with about one billion speakers. \ili{English} has less than half the number of native speakers, but including second language learners, it must clearly be considered the largest language in the world, with perhaps up to twice as many speakers as \ili{Mandarin}. \ili{Russian} (ca. 150 million, \citealt{Cubberley2002}), \ili{Japanese} (ca. 130 million, \citealt{Hasegawa2015}), \ili{Korean} (ca. 75 million, \citealt{Song2005}), \ili{Ukrainian} (ca. 36 million, \citealt{Young2006}), \ili{Uzbek} (ca. 20 million, \citealt{Johanson2006b}), \ili{Kazakh} (ca. 10 Mio, \citealt{Muhamedowa2016}), \ili{Uyghur} (ca. 10 million, \citealt{TuohutiLitifu2012}), \ili{Mongolian} (ca. 5 million, \citealt{Janhunen2003}), and \ili{Amdo Tibetan} (ca. 1.3 million, \citealt{Ebihara2011}: 42), have more than one million speakers. Of the rest, only \ili{Shuri}, \ili{Yakut}, \ili{Oirat}, \ili{Tuvan}, and \ili{Buryat}, and perhaps \ili{Santa}, have between 200,000 and one million speakers. Most of the remaining languages have well below fifty thousand speakers. But note that several languages, including \ili{Mandarin}, \ili{English}, \ili{Russian}, \ili{Ukrainian}, \ili{Uzbek}, and \ili{Kazakh}, are represented in \isi{NEA} only by a fraction of the total number of speakers.

The names \textit{Paleo-Siberian} or \textit{\ili{Paleo-Asiatic}} (\textit{paleoaziatiskije jazyki} in \ili{Russian}) are sometimes still used as labels for several language families (e.g., \citealt{TsumagariKurebitoEndo2007}), especially \ili{Amuric}, \ili{Chukotko-Kamchatkan}, \ili{Yeniseic}, and \ili{Yukaghiric}, sometimes expanded to include \ili{Ainuic}. But this label should be avoided whenever possible, as it does not refer to any valid genetic, areal, or typological grouping.

\ili{Ainu}, \ili{Korean}, \ili{Nivkh}, and sometimes even \ili{Japanese}, are considered to be linguistic \textit{isolates} that are not related to any other known language. However, the difference between a \isi{language isolate} and a \isi{language family} is a \isi{matter of degree} rather than kind. Historically, an isolate \textit{necessarily} is part of a larger stock that has already disappeared, or the relationship to other languages is too remote to be detectable. A case in point is the language \ili{Ket}. It is known to be part of the \ili{Yeniseic} \isi{language family}, but is its sole survivor. Recent years have seen the rise of the so-called \textit{\ili{Dene-Yeniseian} hypothesis}, which claims a genetic connection between \ili{Yeniseic} and the \ili{Na-Dene} languages in North America. Without the historical attestation of now extinct varieties of \ili{Yeniseic}, neither the \ili{Yeniseic} \isi{language family} nor its connection to \ili{Na-Dene} would be known today, and \ili{Ket} would simply count as a linguistic isolate. \ili{Japanese} is certainly not an isolate, but together with the \ili{Ryūkyūan} languages forms the \ili{Japonic} or \ili{Japanese}-Ryūkyūan \isi{language family}. In addition, \ili{Ainu}, \ili{Korean}, \ili{Nivkh}, and \ili{Japanese} all have a certain amount of internal diversity that is usually described as dialectal variation. Given the absence of any clear definition of what characterizes a language as opposed to a dialect, a clear distinction between an isolate and a \isi{language family} cannot be drawn. In order to make the description analogous to the other language families, the designation of the language families of \ili{Ainu}, \ili{Korean}, and \ili{Nivkh} will be \ili{Ainuic}, \ili{Koreanic}, and \ili{Amuric} \citep{Janhunen1996}, respectively.

A special group of Northeast Asian languages is formed by several \textit{pidgins}, \textit{creoles}, and \textit{mixed languages}. Their classification is open to debate and depends on the theory of genetic relatedness one adopts (\citealt{Operstein2015}: 1–3). The \isi{pidgins}, both of which are extinct by now, were called \ili{Govorka} (\ili{Taimyr Pidgin Russian}, \ili{Russian} x \ili{Nganasan}), and \ili{Chinese Pidgin Russian} x \ili{Chinese}). Both are strongly based on \ili{Russian}, which is why they will be treated together with the other \ili{Indo-European} languages (§§\ref{sec:2.5}, \ref{sec:5.5}). Mixed languages include Copper Island \ili{Aleut} (\ili{Aleut} x \ili{Russian}) and \ili{Eynu} (\ili{Uyghur} x \ili{Persian}). For practical purposes these will be treated together with \ili{Eskaleut} (§§\ref{sec:2.4}, \ref{sec:5.4}) and \ili{Turkic} (§§\ref{sec:2.11}, \ref{sec:5.11}), respectively. An \ili{Ainu-Itelmen hybrid} will not be included as it is extinct and has not been recorded to a sufficient degree \citep[81]{Fortescue2003}. \ili{Yilan Creole}, the only language of \isi{Taiwan} included in this study, is basically \ili{Japanese} (§§\ref{sec:2.6}, \ref{sec:5.6}), but has been strongly influenced by Austronesian languages. The status of several varieties in the \isi{Amdo Sprachbund}, especially \ili{Gangou}, Hezhou, \ili{Tangwang}, and \ili{Wutun} (all \ili{Sinitic} x \ili{Turkic} x \ili{Mongolic} x \ili{Tibetic}), remains somewhat unclear. But there are some indications that they are creolized varieties of \ili{Sinitic} and thus will all be treated together with \ili{Trans-Himalayan} (Sino-\ili{Tibetan}, §§\ref{sec:2.9}, \ref{sec:5.9}). Several languages, including \ili{Alchuka}, \ili{Bala}, \ili{Kili}, \ili{Kilen}, and Ussuri \ili{Nanai}, are to different degrees a mixture of several \ili{Tungusic} languages and therefore treated in \sectref{sec:2.10} and \sectref{sec:5.10} on \ili{Tungusic}.

The \ili{Indo-European} languages \ili{Latin}, \ili{Sanskrit}, and \ili{Prakrit} as well as the Semitic languages \ili{Arabic}, \ili{Aramaic}, and \ili{Hebrew}, all of which were at some point used as literary languages in parts of \isi{NEA}, will be excluded. The two \ili{Indo-European} languages \ili{Dutch} and \ili{Portuguese} had only a short-lived and, at least for the purposes of this study, unimportant presence in the maritime southeast of \isi{NEA}. Today, globalization brings many different languages from all around the world into \isi{NEA}, especially the larger cities in the south. But apart from \ili{English}, these languages will be neglected, too. \isi{NEA} may have been home to languages and whole language families that have disappeared without leaving any records. Some of them may be accessible through the study of loanwords. A case in point is the hypothetical language of the \ili{Rouran} empire (\zh{柔然}, 330-555 CE) around \isi{Mongolia}, for which \citet{Vovin2004} has collected a small amount of material. He concludes that it is probably not related to any surrounding language known to us today. Unfortunately, almost nothing is known about its grammatical structure, let alone its \isi{grammar of questions}. Another language or family of languages that apparently has disappeared without trace \citep{Fortescue2013} was presumably spoken by the recently discovered \textit{\isi{Paleo-Eskimos}}.

\begin{quote}
Paleo-Eskimos likely represent a single \isi{migration} pulse into the Americas from \isi{Siberia}, separate from the ones giving rise to the Inuit and other Native Americans, including \ili{Athabaskan} speakers. Paleo-Eskimos, despite showing cultural differences across time and space, constituted a single population displaying genetic continuity for more than 4000 years. On the contrary, the \isi{Thule} people, ancestors of contemporary Inuit, represent a population replacement of the \isi{Paleo-Eskimos} that occurred less than 700 years ago. (\citealt{RaghavanDeGiorgio2014}: 1020)
\end{quote}

\largerpage
\noindent This is by no means the only prehistoric population that is attested in \isi{NEA}, but the recency of their spread would in principle make them accessible with the standard tools of historical linguistics. Recently, genetic studies came to the conclusion that not only populations in \isi{Chukotka}, but also Kets, Nganasans, Selkups, Yukaghirs \citep{FlegontovChangmai2016}, and speakers of \ili{Eskaleut} and \ili{Na-Dene} languages \citep[175, 183]{Reich2018} are genetically related to the \isi{Paleo-Eskimos}. It would be tempting to connect this evidence with the \ili{Dene-Yeniseian} hypothesis \citep{Vajda2010}, but thus far we cannot bring together the linguistic and genetic data as there are too many possible variables. It has by now been demonstrated that not only the Paleo-Eskimos, but in fact all native American populations can be traced back to \isi{Asia}. In other words, all extant and innumerable extinct indigenous American languages necessarily have their origin in \isi{NEA} in prehistoric times. The so-called \textit{Beringian Standstill Model} assumes that a population had lived relatively isolated in \isi{Beringia}, now mostly covered by water, before entering the Americas when the glaciers were on their retreat and the sea levels started to rise (e.g., \citealt{Moreno-MayarPotter2018}). \citet[1]{LlamasFehren-Schmitz2016}, based on genetic evidence, recently argued “that a small population entered the Americas via a coastal route around 16.0 kya, following previous isolation in eastern \isi{Beringia} for {\textasciitilde}2.4 to 9 thousand years after separation from eastern Siberian populations.” (corrected) In other words, the predecessors of most native American languages---possibly excluding speakers of \ili{Na-Dene}, hypothetical \ili{Paleo-Eskimo}, and \ili{Eskaleut}, all of which spread over North America much later---were still around in \isi{Beringia}, arguably a part of \isi{NEA} back then, as recently as 16,000 years ago. It is plausible to assume that this Beringian area harbored a certain amount of linguistic and genetic diversity. For example, there is evidence for a population that today only left some genetic traces in Amazonia and is more closely related to \isi{Australasians} (see \citealt[176-181]{Reich2018} and references therein). This time depth of up to 24,000 years of separation of Siberian and these early native American populations lies well beyond the perhaps 10,000 or so years that are, given ideal circumstances, accessible by means of the comparative method. This means that, from a purely linguistic point of view, generally only a fraction of \isi{prehistory}, namely the \isi{Holocene} (from ca. 9,500 BCE, \citealt{Bellwood2013}: 5f.), is actually accessible. Even so, the age of most language families in \isi{NEA} is considerably lower and does not even approach that age. The data in \tabref{tab:2:1} are only approximations and different authors give different estimates. The data quoted were chosen because their point of view seems to be by and large the most accurate according to my current understanding.
\newpage

\begin{table}[t!]
\caption{Approximate rounded age and  \isi{homeland} of the 14 language families; arrows indicate the possible location of the pre-proto languages}
\label{tab:2:1}

\begin{tabularx}{\textwidth}{llQQ}
\lsptoprule

\textbf{Family} & \textbf{Estim. age} & \textbf{Location} & \textbf{Source}\\
\midrule
\ilit{Trans-Himalayan} & ?9000-8000 & ?eastern \isit{Himalayas} & \citealt{BlenchPost2014}\\
\tablevspace
\ilit{Indo-European} & 6500 & north of Black Sea & \citealt{AnthonyRinge2015}\\
\tablevspace
\ilit{Eskaleut} & 5000 & \isit{Chukotka} → Southern \isit{Alaska} & \citealt{Fortescue2013}\\
\tablevspace
\ilit{Uralic} & 5000 & western \isit{Siberia} & \citealt{Janhunen2009}\\
\tablevspace
\ilit{Chukotko-Kamchatkan} & 4000 & \isit{Kamchatkan isthmus} & \citealt{Fortescue2005}\\
\tablevspace
\ilit{Japonic} & 2200 & southwest \isit{Korea} → Honshū & \citealt{Janhunen2010}\emph{\textup{;}} \citealt{LeeHasegawa2011}\\
\tablevspace
\ilit{Turkic} & 2200 & south \isit{Siberia}, \isit{Mongolia} & \citealt{Yunusbayev2015}\\
\tablevspace
\ilit{Tungusic} & 2000 & middle \isit{Amur} & \citealt{Pevnov2012}\\
\tablevspace
\ilit{Yukaghiric} & ?2000 & \isit{Baikal} → middle \isit{Lena}, \isit{Indigirka} & \citealt{Häkkinen2012}\emph{\textup{;}} \citealt{Maslova2003a}\\
\tablevspace
\ilit{Yeniseic} & 1500-2200 & northern \isit{China} → south \isit{Siberia} & \citealt{Vajda2004}\emph{\textup{;}} \citealt{VovinVajdadelaVaissière2016}\\
\tablevspace
\ilit{Ainuic} & 1300 & \isit{Honshū} → \isit{Hokkaid\=o} & \citealt{LeeHasegawa2013}\\
\tablevspace
\ilit{Mongolic} & 800 & northeastern \isit{Mongolia} & \citealt{Janhunen2003a}\\
\tablevspace
\ilit{Amuric} & 500-1500 & Upper \isit{Amur} & \citealt{Fortescue2011}\emph{\textup{;}} \citealt{Janhunen2010}\\
\tablevspace
\ilit{Koreanic} & 500-1000 & southeast \isit{Korea} & \citealt{Janhunen2010}\\
\lspbottomrule
\end{tabularx}
\end{table}

\sectref{sec:2.1} to \sectref{sec:2.14} will briefly introduce all 14 language families of \isi{NEA} in alphabetical order. Details of the internal classification of the language families, as well as their grammars of questions, will be described in Chapter 5.

\clearpage 
\section{Ainuic}\label{sec:2.1}

\citet[463]{Bugaeva2012} estimates that there are about 100,000 ethnic \ili{Ainu}, of whom only a handful still speaks the language. Historically, there are three major groups of dialects, the \isi{Sakhalin} dialects, the \isi{Kuril} Islands dialects, and the \isi{Hokkaid\=o}dialects (e.g., \citealt{Bugaeva2012}: 461). Proto-\ili{Ainuic} has roughly been dated “to the last centuries of the first millenium A.D.” \citep[155]{Vovin1993}. The spread of the three branches probably started in northern Hokkaid\=o (\citealt{LeeHasegawa2013}) and covered a vast area reaching \isi{Sakhalin} in the Northwest and the \isi{Kuril} Islands and maybe even the tip of southern \isi{Kamchatka} in the Northeast. Today, most \ili{Ainu} have shifted to \ili{Japanese} and the last speakers are only found on Hokkaid\=o. Most of the \isi{Sakhalin} \ili{Ainu} moved to \isi{Japan} after the Second World War and the \isi{Kuril} Island \ili{Ainu} were relocated as early as 1884. Both groups of dialects are extinct today. Genetic research has revealed that the \ili{Ainu} are the result of an \isi{admixture} from the continental \isi{Okhotsk people} (perhaps connected to the \ili{Nivkh}) into the Satsumon population, which itself goes back to the \isi{J\=omon} population (\citealt{Sato2007}). It is known through the study of place names in the T\=ohoku region of \isi{Honshū} that speakers of \ili{Ainu} or a language closely related to \ili{Ainu} once must have lived there as well. According to \citet[33]{Bentley2008b}, \ili{Chinese} recordings of Yamatai \isi{toponyms}, presumably located in southern \isi{Japan}, are predominantly \ili{Japanese}, but may also contain several \ili{Ainuic} elements. The most likely scenario that also takes recent genetic studies into consideration (\citealt{Jinam2012}), is that the \ili{Ainu}, because of the arrival of the \ili{Japonic}-speaking Yayoi people in Honshū, migrated from \isi{Honshū} to \isi{Hokkaid\=o}, where they mixed with people from the \ili{Amuric} speaking Okhotsk population, but preserved their language and subsequently spread to the surrounding regions (\citealt{LeeHasegawa2013}: 5). Up to this point in time, no genetic connections of \ili{Ainuic} with other languages or language families have been proven. The best but still not absolutely convincing attempt to clarify the \isi{prehistory} of the \ili{Ainu} language has perhaps been made by \citet[175]{Vovin1993}, who could “definitely say that Proto-\ili{Ainu} is unrelated to any of the neighbouring languages.” He proposed a possible connection with \ili{Austroasiatic} but this is not generally accepted. \citet[219]{MatsumuraOxenham2013} summarized research on the origin of the \isi{J\=omon} population and concluded “that it ultimately derived from the modern human colonizers of Late Pleistocene \isi{Southeast Asia} and Australia, who subsequently mixed with later migrants from the northern part of \isi{East Asia} during the early J\=omon period (c. 12-7 kya) or before”. This would be in accordance with Vovin’s claim of a southern origin, but given the great time depth of the \isi{J\=omon} culture of 12 ky and the extremely shallow time depth of \ili{Ainuic}, no further hypothesis can be drawn on possible linguistic connections. For the time being, \ili{Ainuic} has to be recognized as a stock on its own, but with possible connections to \isi{Mainland Southeast Asia} and beyond.

The \isi{contact} languages of \ili{Ainuic} were \ili{Japonic} in the South, and \ili{Amuric} in the North (e.g., \citealt{Vovin2016}). There is also strong \isi{contact} to \ili{Russian} as well as the \ili{Tungusic} language \ili{Uilta} on \isi{Sakhalin} and, on the southern tip of \isi{Kamchatka}, to \ili{Itelmen}. \ili{Ainu} used to be a \isi{lingua franca} in southern \isi{Sakhalin} during the 19th century, and was even used by the \ili{Japanese} (\citealt{Yamada2010}: 65).

\section{Amuric (Nivkh)}\label{sec:2.2}

The designation \ili{Amuric} has been introduced by \citet{Janhunen1996} to refer to the \isi{language family} to which \ili{Nivkh}, previously called \ili{Gilyak}, belongs. The internal diversity appears to be similar to that of \ili{Ainuic}, with some dialects being mutually unintelligible \citep[7]{Gruzdeva1998}. No relation with other languages has been proven, although \citet{Fortescue2011} recently argued for the possibility of a remote relationship with \ili{Chukotko-Kamchatkan} languages, which has yet to be verified. There are at most a few hundred speakers left out of a population of a few thousand. \ili{Amuric} has often been linked with the \isi{Okhotsk culture} (5th to 13th century AD), which reached as far as Hokkaid\=o and the \isi{Kuril} Islands \citep{Fortescue2011} and had a strong impact on the \ili{Ainu} (see also \citealt{Vovin2016}). Based on evidence from the cultural lexicon, \citet[294]{Janhunen2010} assumes an origin of \ili{Amuric} further to the south in central \isi{Manchuria}. However, this contradicts both the assumption that \ili{Tungusic} was spoken along the middle \isi{Amur} (\sectref{sec:2.10}) and the hypothesis that the \isi{Okhotsk culture} was \ili{Amuric}-speaking. Today, \ili{Nivkh} is spoken along the mouth of the \isi{Amur} and in some villages on \isi{Sakhalin} and perhaps by a few speakers who were resettled in Hokkaid\=o after the last world war (\citealt{Fortescue2016}: 1ff.).

\ili{Nivkh} had intense contacts with several \ili{Tungusic} languages (e.g., \citealt{Gusev2015b}) both at the lower \isi{Amur} (e.g., \ili{Negidal}, \ili{Ulcha}), and on \isi{Sakhalin} (\ili{Uilta}, \isi{Sakhalin} \ili{Evenki}), where there was also \isi{contact} with \ili{Ainuic} and, for a short period, with \ili{Japanese} (see also \citealt{Yamada2010}). In addition, there is some evidence for old contacts between \ili{Amuric} and \ili{Ainuic} (see \citealt{Vovin2016}). The most important \isi{contact} language today is \ili{Russian}, and most \ili{Nivkh} have switched to speaking \ili{Russian}.

\section{Chukotko-Kamchatkan}\label{sec:2.3}
\largerpage
The status of \ili{Chukotko-Kamchatkan} (or Luoravetlan) as a \isi{language family} is not recognized by some authors, notably \citet{GeorgVolodin1999}. But \citet{Fortescue2003,Fortescue2005,Fortescue2011} has quite convincingly shown that it has a firm basis. The \isi{language family} falls into two major branches, \ili{Itelmen} (Kamchadal) on the one hand and a more diverse branch including Chuckchi, \ili{Alutor}, \ili{Koryak} (Nymylan), and \ili{Kerek}, on the other hand. All scholars agree that Chuckhi, \ili{Alutor}, \ili{Koryak}, and \ili{Kerek} are related, and the controversy surrounds the question of whether \ili{Itelmen} belongs to the same \isi{language family} or not. Concerning the origin of \ili{Chukotko-Kamchatkan} (CK), \citet[3]{Fortescue2005} assumes the following scenario.

\begin{quote}
The linguistic “centre of gravity”---suggesting the original CK “\isi{homeland}”---lies around the Kamchatkan isthmus [...], an area presumably reached from the west along the coast of the Okhotsk Sea long before the introduction of the \isi{reindeer}-herding from further west within the last thousand years or so [...]. The time at which proto-CK may have been spoken in this general area by hunters of wild caribou has been estimated as somewhere around four thousand years [...]; this coincides with the beginnings of the Neolothic cultures of Tarya on \isi{Kamchatka} and (a little later) Ust-Belaya on \isi{Chukotka}.
\end{quote}

In agreement with an original location further to the west and perhaps to the south, \citet{Fortescue2011} has recently argued for an old genetic relation of \ili{Chukotko-Kamchatkan} with \ili{Amuric}, which seems possible but remains to be verified. A recent genomic study has shown that the \ili{Chukchi} derive about 40\% of their genome from a back-\isi{migration} of a native American population to \isi{Asia} \citep[184]{Reich2018}. If the same is true for all \ili{Chukotko-Kamchatkan}-speaking populations, this opens up the possibility that Pre-Proto-Chukot\-ko-Kamchatkan, or a \isi{contact} language thereof, can be traced to North America.

Two historically attested dialects of \ili{Itelmen} as well as \ili{Kerek} have already disappeared, and all the remaining languages except for \ili{Chukchi}, which has about 10,000 speakers, are highly endangered. Concerning the lifestyle of the speakers of \ili{Chukotko-Kamchatkan}, \citet[416]{Anderson2006c} mentions an interesting split.

\begin{quote}
Along the coasts, \ili{Chukchi} people live as sea mammal hunters, like the local Yup’ik populations, but they live as reindeer herders in the interior. Approximately three-quarters of the \ili{Chukchi} live as reindeer herders. Northern Kamchatkan groups mainly practice reindeer-oriented economies and fishing and sea mammal hunting along the coasts. The \ili{Itelmen} live primarily as \isi{subsistence} fishers.
\end{quote}

\noindent The herding of \isi{reindeer} must be a relatively recent innovation brought to the Northeast of \isi{NEA} by other people from the west, but may have been the driving factor in a secondary expansion of \ili{Chukchi}.

\ili{Chukotko-Kamchatkan} languages had \isi{contact} mostly with \ili{Even}, parts of \ili{Yupik}, \ili{Yukaghiric}, \ili{Russian} and, less importantly, \ili{English}. \ili{Itelmen} seems to have had \isi{contact} with \ili{Ainuic} as well.

\section{\ili{Eskaleut} (Eskimo-Aleut)}\label{sec:2.4}

\ili{Eskaleut} languages are for the most part not spoken in \isi{NEA}, but in \isi{Alaska}, Canada, and Greenland (e.g., \citealt{Berge2006}). The primary split is between Eskimo and \ili{Aleut}, the former having an additional division between \ili{Yupik}, Inuit, and perhaps \ili{Sirenikski} (e.g., \citealt{FortescueJacobsonKaplan2010}: x). In this study only those \ili{Eskaleut} languages spoken in or in the vicinity of \isi{NEA} will be included. These are Sirenik(ski), which is extinct, and \ili{Naukan}(ski) \ili{Yupik} on the mainland, Central Siberian \ili{Yupik} on St. Lawrence Island, and \ili{Aleut} as well as Mednyj \ili{Aleut} on the \ili{Aleut} Islands. The languages have all reached their present location from \isi{Alaska}, where the \isi{homeland} of \ili{Eskaleut} was probably located. Very early, at least several thousand years ago, the \ili{Aleut} started migrating along the \ili{Aleut} islands towards \isi{Asia}.

\begin{quote}
It can only be surmised that the movement that separated \ili{Aleut} from Eskimo occurred soon after the first arrival of the Eskimo-\ili{Aleut} family in \isi{Alaska} over \isi{Bering Strait}, at least four thousand years ago and some two thousand years before the Inuit-\ili{Yupik} split. The linguistic evidence suggests at least two major phases here---an ongoing spread westwards as far as the outermost Near Islands (reached some 2,500 years ago), overlaid in more recent times (only a few hundred years ago) by a wave bearing specifically Eastern \ili{Aleut} influence from the Alaskan peninsula. (\citealt{Fortescue2013}: 344f.)
\end{quote}

The best known and most important expansion of Eskimo was about a thousand years ago to northern Canada and Greenland. But there were migrations on the Asian side as well, which are more important for the present study \citep{Fortescue2004}.

\begin{quote}
On the Asian side of \isi{Bering Strait}, at approximately the same time as the \isi{Thule} \isi{migration} eastward from North \isi{Alaska}, a westward expansion of \isi{Punuk} culture whaling people probably speaking Central Siberian \ili{Yupik} was initiated. This eventually reached as far as the Kamchatkan isthmus in the 15th century, as linguistic evidence suggests, although the Eskimo presence must have been short-lived or absorbed by maritime Koryaks and---especially---Kereks \citep[344]{Fortescue2013}
\end{quote}

It is, of course, generally accepted that Pre-\ili{Proto-Eskaleut} had been located on the Asian side before crossing over to \isi{Alaska}, but according to \citet[558]{Berge2010} and \citet{Fortescue2013} this must have been at least 4000 years ago. The possible existence of a few Eskimo loanwords in \ili{Tungusic} languages cannot change that basic fact (cf. \citealt{Vovin2015}).

\citet[344]{Fortescue2013} hypothesizes that \ili{Sirenikski} may be “a pocket of archaic Eskimo much influenced by \ili{Chukchi}.” \ili{Aleut} probably had \isi{contact} with unknown languages in \isi{Alaska} and perhaps the \ili{Aleut} Islands. Both \ili{Aleut} and \ili{Yupik} as spoken in \isi{Asia} had strong \isi{contact} with \ili{Russian} and, less importantly, with \ili{English}.

\section{Indo-European}\label{sec:2.5}

\ili{Indo-European} is the most widespread and the largest \isi{language family} worldwide in terms of speakers. About one third of the global population speaks an \ili{Indo-European} language. \ili{Proto-Indo-European} was presumably located on the \isi{Pontic-Caspian steppe}, perhaps about 4500 BCE (\citealt{AnthonyRinge2015}), although there are competing but in my eyes much less likely hypotheses, for example of a location in Anatolia south of the Black Sea (e.g., \citealt{Heggarty2013}). There is \isi{convergent evidence} from the \isi{human genome}, archaeology, and linguistics for the location on the \isi{Pontic-Caspian steppe} (e.g., \citealt{Anthony2007}; \citealt{AllentoftSikora2015}; \citealt{AnthonyRinge2015}; \citealt{HaakLazaridis2015}; \citealt{JonesGonzales-Fortes2015}). According to one prominent view, the subsequent spread and the divergence of \ili{Indo-European} branches can be summarized as follows:

\begin{quote}
\textbf{Archaic Proto-Indo-European} (partly preserved in Anatolian) probably was spoken before 4000 BCE; \textbf{early Proto-Indo-European} (partly preserved in Tocharian) was spoken between 4000 and 3500 BCE; and \textbf{late Proto-Indo-European} (the source of Italic and Celtic with the \isi{wagon}/\isi{wheel} vocabulary) was spoken about 3500-3000 BCE. Pre-\ili{Germanic} split away from the western edge of late \ili{Proto-Indo-European} dialects about 3300 BCE, and Pre-\ili{Greek} split away about 2500 BCE, probably from a different set of dialects. Pre-\ili{Baltic} split away from Pre-\ili{Slavic} and other northwestern dialects about 2500 BCE. Pre-\ili{Indo-Iranian} developed from a northeastern set of dialects between 2500 and 2200 BCE. (\citealt{Anthony2007}: 82, my boldface)
\end{quote}

\ili{Indo-European} has a dozen major branches, four of which have, or formerly had, representatives in \isi{Northeast Asia} as defined here: Tocharian, \ili{Iranian} (part of \ili{Indo-Iranian}), (East) \ili{Slavic}, and (West) \ili{Germanic}. Historically speaking, \ili{Indo-European} languages entered \isi{Northeast Asia} at roughly three different times.

Pre-Tocharian, which may have branched off from \ili{Indo-European} about 5300 years ago (before all other branches except Anatolian), probably reached the \isi{Altai} mountains shortly afterwards and is associated with the Afanasievo culture (ca. 3300-2500 BCE) (\citealt{Mallory2010}: 51; \citealt{AnthonyRinge2015}: 208). The Afanasievo culture showed a southward expansion, which would explain why Tocharian is only attested further south in the \isi{Tarim} basin in two different forms known as \ili{Tocharian A} (East) and B (West) (e.g., \citealt{Winter1998}). There are indications of the existence of a third language (\ili{Tocharian C}), which is attested exclusively in loanwords (\citealt{Mallory2010}: 48f.). Tocharian has been extinct for at least a thousand years.

\begin{quote}
\ili{Tocharian A}, found in documents near \isi{Turfan} and \isi{Qarashähär}, and \ili{Tocharian B}, found mainly around \isi{Kucha} in the west but also in the same territory as \ili{Tocharian A}. The documents, dating from the 6th to the 8th centuries CE, suggest that \ili{Tocharian A} was by that time probably a dead liturgical language, while \ili{Tocharian B} was still very much in use. In addition to Tocharian, administrative texts have been discovered in \ili{Prakrit}, an Indian language from the territory of \isi{Krorän} [lóulán \zh{楼兰}]; these documents contain many proper names and items of vocabulary that would appear to be borrowed from a form of Tocharian (sometimes known as \ili{Tocharian C}) spoken by the native population. The Kroränian documents date to ca. 300 CE and provide our earliest evidence for the use of Tocharian. For our purposes here, it is also very important to note that the earliest evidence for the mummified remains of “westerners” in the \isi{Tarim} Basin is found in cemeteries at \isi{Xiaohe} [\zh{小河}] (Small River) and \isi{Qäwrighul} [gǔmùg\=ou \zh{古墓沟}], both of which are located in the same region as \ili{Tocharian C}. (\citealt{Mallory2010}: 48f., my square brackets)
\end{quote}

There are alternative names for \ili{Tocharian A}, such as \textit{Agnean} after the \ili{Sanskrit} name Agni (\textit{y\=anqí} \zh{焉耆}) for the city of Karashahr, and for \ili{Tocharian B}, such as \textit{Kuchean} after the city of \isi{Kucha} (\textit{qiū}\textit{z\={\i}} \zh{龟兹} and variants) (e.g., \citealt{Fortson2010}: 400; \citealt{GengShimin2012}).

\ili{Tocharian} was in \isi{contact} with several \ili{Iranian} languages that entered the Northeast Asian scene after Tocharian, but were probably present in the \isi{Tarim} basin as early as 1300 BCE \citep[50]{Mallory2010}. \ili{Iranian} together with Indic and maybe \ili{Nuristani} as an independent subbranch, forms the \ili{Indo-Iranian} branch of \ili{Indo-European} (\citealt{Fortson2010}: 202f.). \ili{Iranian} language history is usually divided into an Old \ili{Iranian} (until the 4th or 3rd century BCE), a Middle \ili{Iranian} (until the 8th or 9th century CE), and a Modern \ili{Iranian} period (e.g., \citealt{Schmitt2000}: 3). \ili{Iranian} languages only had a wide distribution in \isi{NEA} during the Middle \ili{Iranian} period. The two languages Khotanese (\textit{hétián sàiyǔ} \zh{和田塞语}, in the South of the \isi{Tarim} basin, ca. 5th to 10th century CE, \citealt{Emmerick2009}: 377ff.) and Tumshuquese (\textit{túmùshūkè sàiyǔ} \zh{图木舒克塞语}, in the North, 7th to 8th century CE), closely related and usually collectively called \ili{Saka} (\textit{sàiyǔ} \zh{塞}), were more restricted in their distribution than \ili{Sogdian} (\citealt{Emmerick2009}; \citealt{GengShimin2011}). \ili{Sogdian} (\textit{sùtèyǔ} \zh{粟特语}, ca. 4th to 11th centuries CE, \citealt{Yoshida2009}: 279ff.) was originally spoken in present-day Uzbekistan and Tajikistan, but “the Sogdians played an active role as international traders along the Silk Road between \isi{China} and the West, with the result that the \ili{Sogdian} language became a kind of \isi{lingua franca} in the region between Sogdiana and \isi{China}” \citep[279]{Yoshida2009}. Regarding modern \ili{Iranian}, only the \isi{Pamir} languages \ili{Sarikoli} (\textit{sà}\textit{l\u{\i}kù’er} \zh{萨里库尔}) and \ili{Wakhi} (\textit{wǎ}\textit{hǎn} \zh{瓦军}), treated as dialects of one language called \textit{tǎjíkèyǔ} \zh{塔吉克语} (\citealt{GaoErqiang1985}: 101) but not to be confused with the \ili{Tajik} language, as well as the mixed \ili{Persian}-\ili{Uyghur} language \ili{Eynu}, are spoken in \isi{NEA}. However, the discussion will also briefly mention \ili{Yaghnobi}, which is located in Tajikistan but represents the only modern language that is closely related with \ili{Sogdian}.

The last period of \ili{Indo-European} influx brought Eastern \ili{Slavic} as well as \ili{Germanic} languages into \isi{NEA}. Together with the Baltic languages, \ili{Slavic} forms the Balto-\ili{Slavic} branch of \ili{Indo-European}. Only the East \ili{Slavic} languages \ili{Russian} and \ili{Ukrainian} expanded into \isi{NEA}. \ili{Russian} is not only the dominant language of the Russian Federation, but has also had some influence on several languages outside of \isi{Russia}, such as \ili{Mongolian} or \ili{Uyghur}. Many speakers of languages in the Russian Federation are bilingual in \ili{Russian} or are even shifting to \ili{Russian} as their primary language. \ili{Ukrainian} only plays a marginal role, but nevertheless can be found scattered across the \ili{Russian}-speaking area. \ili{Slavic} originates in Eastern \isi{Europe}, perhaps northwest of the Black Sea (\citealt{Fortson2010}: 420f.) and the \ili{Russian} expansion beyond the Urals only started in the 16th century. By 1625 the Russians reached the \isi{Yenisei}, and by the end of the 17th century they had conquered most of \isi{Siberia}, excluding only Outer \isi{Manchuria}, \isi{Chukotka}, and southern \isi{Kamchatka} \citep[102]{Forsyth1992}. This means that \ili{Russian} played no role in \isi{NEA} until about 400 years ago. There is a mixed \ili{Russian}-\ili{Ukrainian} language called \ili{Surzhyk}, of which some speakers are most likely also found in \isi{NEA}, but which must be neglected for lack of sufficient information \citep{Bilaniuk2004}.

\largerpage[2]
Only West \ili{Germanic} languages are marginally represented in \isi{NEA} by scattered minorities of \ili{German} (especially \ili{Altai Low German}) speakers living in southern \isi{Siberia} as well as a certain amount of influence from American \ili{English} as spoken in \isi{Alaska} and the \ili{Aleut} Islands. \ili{Yiddish} is included here mostly because of the existence of a Jewish Autonomous Oblast in \isi{Russia} close to Khabarovsk, where a handful of \ili{Yiddish} speakers can be found and where it has an official status. \ili{Yiddish} is a descendant of primarily southeastern \ili{Middle High German} that was extensively influenced by \ili{Slavic}, \ili{Hebrew}, and \ili{Aramaic} (\citealt{JacobsPrincevanderAuwera1994}). \ili{Altai Low German} (or \ili{Plautdiitsch}) “is the descendant of the Low \ili{German} (Low Prussian and Pommeranian) dialects once spoken in the Danzig area.” \citep[13]{Nieuweboer1999} There is only limited information on \isi{questions} in \ili{Altai Low German}, but Standard \ili{German}, a liturgical language for Siberian speakers of \ili{German} dialects, can give some rough indications about how the blanks may be filled in. There was an \ili{English} jargon introduced with \ili{English}-speaking whale hunting crews especially in \isi{Chukotka} (\citealt{deReuse1996}). \ili{English} is perhaps the major foreign language in large parts of \isi{NEA} and there are many native speakers, often soldiers, in \isi{Japan} and \isi{South Korea}. Furthermore, it often serves as a \textit{lingua franca} in international communication.

\section{Japonic (Japanese-Ryūkyūan)}\label{sec:2.6}

The \ili{Japonic} \isi{language family} most likely had its origin on the \isi{Korean Peninsula} and only later expanded into the \ili{Japanese} archipelago. This expansion is connected with the Yayoi people, originally perhaps farmers along the Yangtze, who after 850 BCE via \isi{Korea} spread to \isi{Japan} where they arrived by about 400 BCE (e.g., \citealt{Janhunen2003e}; \citealt{LeeHasegawa2011}; \citealt{MatsumuraOxenham2013}: 219; \citealt{Siska2017}: 2f.). The Yayoi people mixed with and replaced the original \isi{J\=omon} population, their \isi{hunter-gatherer} lifestyle as well as their languages. Peripheral areas such as Hokkaid\=o and the \isi{Ryūkyūan Islands} preserve stronger traces of the J\=omon genome. But while \ili{Ainuic} languages in Hokkaid\=o may represent the last remnants of the J\=omon languages, Ryūkyūan languages are clearly related to \ili{Japanese}. According to \citet[202]{Vovin2013a}, the southward \isi{migration} of \ili{Ryūkyūan} only started in the 9th century.

According to one classification, \ili{Japanese} can be divided into Old (592-794), Late Old (794-1192), Middle (1192-1603), and Early Modern \ili{Japanese} (1603-1867) (\citealt{Hasegawa2015}: 5ff.). \ili{Old Japanese} can be further divided into Eastern, Central, and Western \ili{Old Japanese}. Eastern \ili{Old Japanese} was spoken in what today is the Kant\=o area in the 8th century CE, while Western \ili{Old Japanese} is the language from Nara \citep{Kupchik2011}. \ili{Hachij\=o} is the only modern descendant of Eastern \ili{Old Japanese} \citep[9]{Kupchik2011}. Central \ili{Old Japanese}, thought to be the predecessor of Modern \ili{Japanese}, is almost unknown (but see \citealt{Kupchik2011}: 7f., 852). \ili{Old Japanese} has to be distinguished from Classical \ili{Japanese}, which was based on Late \ili{Old Japanese} as defined above and served as a literary language \citep{Tranter2012b}. There is evidence for the former presence of \ili{Para-Japonic} or \ili{Japonic} languages on the \isi{Korean Peninsula} as well as on Jeju Island \citep{Vovin2013b}, but no information relevant for this study can be obtained from these long-gone varieties (see also \citealt{Beckwith2007} and especially \citealt{Pellard2005} for some discussion).

\ili{Japonic} had \isi{contact} with \ili{Ainuic}, \ili{Koreanic}, \ili{Sinitic}, \ili{Amuric}, \ili{Uilta} etc. Modern \ili{Japanese}, furthermore, has been influenced by several European languages and especially \ili{English}. Contact with Austronesian on \isi{Taiwan} led to the emergence of \ili{Yilan Creole}. The dialects of \ili{Japanese} as well as Ryūkyūan languages are both increasingly being replaced by Standard \ili{Japanese}, which itself is based on the T\=oky\=o dialect in the Eastern dialect area (\citealt{Sanada2007}). \ili{Yilan Creole} is under \ili{Chinese} influence.

\section{Koreanic}\label{sec:2.7}
 
The internal dialectal differences of \ili{Korean} should not be underestimated, and some of these dialects, notably \ili{Jeju} on \ili{Jeju} island and \ili{Yukcin} in the Northeast, have been said to exhibit language-like differences with regard to other varieties of \ili{Korean}. It is therefore possible to speak of the \textit{\ili{Koreanic}} \isi{language family} instead of a \textit{Korean} isolate. Regarding the origin of \ili{Koreanic}, \citet[201]{Vovin2013a} has recently argued for a location in the north:

\begin{quote}
It appears that the \isi{migration} of the \ili{Korean}[ic] speakers to their present location was quite straightforward, from southern \isi{Manchuria} in the north to the \isi{Korean Peninsula} in the south. The linguistic process of Koreanization took several centuries, and it appears that proto-\ili{Korean}[ic] or pre-Old \ili{Korean} gradually replaced [Para-]\ili{Japonic} languages between the 3rd and 8th centuries \textsc{ce}. The central and southern parts of the \ili{Korean} Peninsula were originally [Para-]\ili{Japonic} speaking. (my square brackets)
\end{quote}

Today, \ili{Koreanic} is distributed across the entire \ili{Korean} Peninsula as well as adjacent parts of \isi{China}, parts of \isi{Sakhalin}, and even Central \isi{Asia}. Theoretically, Central Asian \ili{Korean} (\ili{Kolyemal}) as spoken in eastern Uzbekistan, for example, is located outside of \isi{Northeast Asia}. However, given its location very close to \isi{Xinjiang} and the fact that it preserves several conservative features that were lost in \isi{Korea}, it will also be included.

\ili{Korean} is historically attested in several stages that may be called Old \ili{Korean}, Middle \ili{Korean}, and Modern \ili{Korean}, but recent descriptions disagree on how exactly the historical stages of \ili{Korean} should be classified. \citet{Whitman2015} considers Old \ili{Korean} to be the language of Unified \isi{Silla} (668-935 CE), while \citet[41]{Nam2012} argues that the Old \ili{Korean} period already began in the 5\textsuperscript{th} century CE.

\begin{quote}
We divide Old \ili{Korean} (OK) into Early, Mid and Late Old \ili{Korean} (EOK, MOK, LOK). EOK was the \ili{Korean} of the Three Kingdoms period, roughly from the start of the fifth century until Silla unified the Three Kingdoms in the 660s. MOK was the \ili{Korean} of the Unified Silla [Sinla] period, from the 660s until the 930s when Koryŏ [Kolye] re-unified the country. LOK was the language of the earlier part of the Koryŏ dynasty from the 930s till the mid-thirteenth century.
\end{quote}

\noindent The languages that were spoken before or during Unified Silla are only poorly attested. Very likely these languages included \ili{Para-Koreanic} and \ili{Para-Japonic}, but no relevant material is available for the purposes of this study, which is why they have been excluded here altogether. Old \ili{Korean} was followed by Middle \ili{Korean}, more exactly Early Middle \ili{Korean} (10\textsuperscript{th} to 14\textsuperscript{th} centuries) and Late Middle \ili{Korean} (15\textsuperscript{th} and 16\textsuperscript{th} centuries), roughly divided by the invention of the Hangul script in 1446 \citep{Sohn2012}.

\ili{Koreanic} had \isi{contact} with Southern \ili{Tungusic}, \ili{Japonic}, and \ili{Sinitic}, which forms a very strong ad- and superstrate. Both \ili{Japonic} and \ili{Koreanic} derive a large amount of vocabulary from \ili{Sinitic}. Today, \ili{English} is an important \isi{contact} language as well.

\section{(Khitano-)Mongolic}\label{sec:2.8}

There are a dozen \ili{Mongolic} languages and all are spoken in \isi{Northeast Asia} except for \ili{Kalmyk} (an aberrant dialect of \ili{Oirat}) and \ili{Moghol} in Afghanistan (Janhunen \citeyear{Janhunen2003}, \citeyear{Janhunen2006}). Apart from the \ili{Mongolic} languages proper, there is what has been termed \ili{Para-Mongolic} (\citealt{Janhunen2003d,Janhunen2012d}), i.e. sister languages of the \ili{Proto-Mongolic} lineage (e.g., \ili{Khitan}). All known Para-\ili{Mongolic} languages are extinct and given the scarce material, Para-\ili{Mongolic} languages will be excluded from the discussion. The age of the \ili{Mongolic} \isi{language family}, i.e. the time of the break-up of the Proto-\ili{Mongolic} unity, is thought to be only about 800 years (e.g., \citealt{Janhunen2012c}: 3). If one includes Para-\ili{Mongolic}, the family must be much older, but \citegen[8]{Janhunen2012b} estimate of an age of about 1500 to 2500 years before present shows that the details are far from clear. In addition to the modern \ili{Mongolic} languages there are historical records of older stages, notably so-called \ili{Middle Mongol}, which “is the technical term for the \ili{Mongolic} languages recorded in documents during, or immediately after, the time of the Mongol empire(s), in the thirteenth to the early fifteenth centuries.” \citep[57]{Rybatzki2003a} In addition, there is written Mongol, a literary language written with the \ili{Uyghur} alphabet that has a history of about 800 years and exhibits several archaic features \citep{Janhunen2003b}. The recently partly deciphered Hüis Tolgoi inscription from \isi{Mongolia} seems to represent a form of early \ili{Mongolic} and is considerably older than \ili{Middle Mongol} (e.g., \citealt{Vovin2017}). The “\isi{homeland}” problem is notoriously difficult for many language families. However, for \ili{Mongolic} it quite clearly was located somewhere in present-day northeastern \isi{Mongolia}, the place where the \ili{Mongolic} expansion had its starting point (\citealt{Janhunen2003}: xxxiv). But \ili{Proto-Mongolic} itself formed a larger family with \ili{Para-Mongolic}, and the question about the original location of this proto-language of Proto- and \ili{Para-Mongolic} (\citealt{Janhunen2012d}: 114 proposes the name \textit{\ili{Khitano-Mongolic}}, also adopted here, and \citealt{Shimunek2014,Shimunek2017} \textit{\ili{Serbi-Mongolic}}), is less easy to answer. \citet[10]{Janhunen2012b} assumes that it was located further to the south in present-day \isi{Liaoning} or eastern Inner \isi{Mongolia}:

\begin{quote}
There is a particularly clear parallelism in the expansion of the \ili{Mongolic} [including \ili{Para-Mongolic}] and \ili{Tungusic} language families. Once they had occupied their protohistorical positions on both sides of the \isi{Liao} basin, they both assumed a general northward trend of expansion. In the light of the available information on the history and protohistory of the region, the \ili{Mongolic} \isi{homeland} has to be placed in southwestern \isi{Manchuria} (Liaoxi), while the \ili{Tungusic} Homeland can hardly have been located anywhere else but in southeastern \isi{Manchuria} (Liaodong), though quite possibly also extending to the northern part of the \isi{Korean Peninsula}. (my square brackets)
\end{quote}

\noindent On \ili{Tungusic}, see \sectref{sec:2.10}. Janhunen’s assumption of a Pre-Proto-\ili{Mongolic} \isi{homeland} situated roughly in eastern \isi{Manchuria} is corroborated by some historical facts, such as the \ili{Khitan} \isi{Liao-dynasty} (\zh{辽}, 916-1125 CE) that roughly derived from this region.

\ili{Mongolic} in general shows strong influence from \ili{Turkic} languages and \textit{vice versa} \citep{Schönig2003}. Individual \ili{Mongolic} languages participated in different linguistic areas that sometimes overlap and display a different strength of convergence. Shirongolic is an integral part of the so-called \isi{Amdo Sprachbund}. \ili{Dagur}, together with the two \ili{Tungusic} languages \ili{Solon} and \ili{Oroqen}, formed a small \isi{linguistic area} for itself, but during the Qing-dynasty (1636-1911) were also under the strong influence of yet another \ili{Tungusic} language, \ili{Manchu}. Similar to \ili{Tungusic}, \ili{Mongolic} languages today can be classified as to whether they are under the influence of the national language of \isi{Russia} (\ili{Kalmyk}, \ili{Buryat}) or \isi{China} (\ili{Dagur}, \ili{Shirongolic} etc.). But unlike \ili{Tungusic}, this only partly applies to the \ili{Mongolic} languages spoken in "Outer \isi{Mongolia}", where \ili{Russian} influence appears to be receding, and does not apply at all to \ili{Moghol} in Afghanistan. A national language itself, \ili{Mongolian} of course influences all \ili{Mongolic} languages spoken in \isi{Mongolia}.

\section{Trans-Himalayan (Sino-Tibetan)}\label{sec:2.9}

It has been pointed out that the name \textit{Sino-Tibetan} is somewhat misleading and it will not be used in this book. The traditional view, as advocated by \citet{LaPolla2013}, for example, claims that \ili{Sino-Tibetan} has two main branches, \ili{Sinitic} and \ili{Tibeto-Burman}. According to this view, the origin of Sino-\ili{Tibetan} (and not only of \ili{Sinitic}) is usually said to have been around the \isi{Yellow River}. Some of the justified criticism to previous approaches to the family has been aptly summarized by \citet[93]{BlenchPost2014}:

\begin{quote}
“Reconstructions” have been proposed which have failed to take many languages of high phyletic significance into account; these forms have been repeatedly quoted without remark in the literature, in the process gaining a lustre they hardly deserve. Sino-\ili{Tibetan} has no agreed internal structure, and yet its advocates have been happy to propose dates for its origin, expansion and \isi{homeland} in stark contradiction to the known archaeological evidence. A focus on “high cultures” (\ili{Chinese}, \ili{Tibetan}, \ili{Burmese}) has led to an emphasis on these languages and their written records, something wholly inappropriate for a phylum where an overwhelming proportion of its members speak unwritten languages.
\end{quote}

\noindent Therefore, the more adequate and neutral name \textit{\ili{Trans-Himalayan}} (\citealt{vanDriem2014}) will be employed here instead, which does not imply a split into only two main branches and suggests an origin and center of diversity further to the southwest. In fact, most \ili{Trans-Himalayan} languages are located in South or \isi{Southeast Asia}. According to \citet{vanDriem2014} and \citet{BlenchPost2014}, the geographical distribution of the different branches suggests an origin of the whole \isi{language family} in the eastern \isi{Himalayas}. Under this assumption, \ili{Sinitic} would be the northernmost of many different branches of the family. Needless to say, this innovative view is not yet accepted by all researchers and deserves further investigation (see \citealt{LaPolla2016} for a discussion).

This study only includes languages from three of a total of perhaps 42 different subbranches of \ili{Trans-Himalayan} (\citealt{vanDriem2014}), namely \ili{Sinitic}, \ili{Tibetic} (a subbranch of Bodish), and \ili{Qiangic}. The age of \textit{\ili{Sinitic}} depends on the definition. Traditionally, old stages of \ili{Chinese} are divided into \textit{Old Chinese} and \textit{Middle Chinese}. However, a new approach developed by \citet{Norman2014}, which focuses on evidence from the spoken languages, makes a distinction into \textit{\ili{Common Dialectal Chinese}} (CDC, the proto-language of all modern \ili{Chinese} languages except Min) and \textit{Early Chinese} (EC, the proto-language of Min, CDC etc.). Roughly speaking, CDC can be compared with the Romance languages and Early \ili{Chinese} with Italic. If \textit{Sinitic} refers to CDC and its descendants, then the age is perhaps about 2000 years. If, however, \textit{Sinitic} refers to the whole branch of \ili{Trans-Himalayan} (i.e., (pre-)EC), then \ili{Sinitic} is perhaps some 1500 years older. The latter view will be adopted here. However, Norman was reluctant to estimate the ages of the two \isi{proto-languages}. While Norman’s is perhaps the best approach to the history of \ili{Chinese} yet, this study necessarily takes a pragmatic stance. Compared with \ili{Indo-European}, the \isi{reconstruction} of \ili{Chinese} is still in its infancy and goes beyond the possibilities of this study, which will mostly be focusing on modern \ili{Chinese} languages. In order to capture some of the history of \ili{Chinese}, I will refer to the recent study by \citet{BaxterSagart2014a,BaxterSagart2014b}, who employ the term \textit{Old Chinese} as a more or less useful cover term for the earlier period of \ili{Sinitic}:

\begin{quote}
\emph{\textup{We use the term “\ili{Old Chinese}” in a broad}} sense to refer to varieties of \ili{Chinese} used before the unification of \isi{China} under the Qín \zh{秦} dynasty in 221 \textsc{bce}. The earliest written records in \ili{Chinese} are oracular inscriptions on bones and shells from about 1250 \textsc{bce} (in the late Sh\=ang \zh{商} dynasty, which was overthrown by the Zh\=ou \zh{周} in 1045 \textsc{bce}), so this is an interval of about 1,000 years. Obviously there must have been many varieties of \ili{Chinese} during this period, widely distributed in time and space. (\citealt{BaxterSagart2014a}: 1)
\end{quote}

Throughout its history, \ili{Sinitic} had intense language contacts with many surrounding languages (see \citealt{Matthews2010}). Especially intense was the influence on \ili{Korean} and \ili{Japanese}, which derive a large amount of their vocabulary from \ili{Sinitic}. \ili{Mandarin} today is the dominant language of \isi{China}, and has already started to replace several minority languages throughout the country. Just like \ili{Russian} dominates the northern half of \isi{NEA}, \ili{Mandarin} has a leading position in the southern half.

Following \citet{Tournadre2014}, it is perhaps best not to speak of the \ili{Tibetan}, but of the \ili{Tibetic}, branch, which goes back to Old \ili{Tibetan} (ca. 7th to 9th century CE) as its proto-language, which is closely related to the \ili{Classical Tibetan} language:

\begin{quote}
‘\ili{Classical Tibetan}’ is an idealization, referring both to over a millennium of written history and to a tradition of prescriptive grammar which many of the authors of the texts, in some cases down to the present, made greater or lesser efforts to conform to. [...] The term ‘\ili{Old Tibetan}’ is used to refer to written material from before about 1000 CE, primarily inscriptions and documents found in the Dun-huang caves (\citealt{DeLancey2003}: 255f.)
\end{quote}

Today \ili{Tibetic} encompasses about 200 different varieties distributed over an extremely large area, which can, according to \citet{Tournadre2005,Tournadre2005}, be classified into eight “sections”. Only some varieties from the eastern (\ili{Baima}, \ili{Cone}, \ili{Zhongu}) and northeastern sections (\ili{Amdo Tibetan}, \ili{gSerpa}) will be included here. \ili{Amdo Tibetan} is of special importance for this study because of its dominant position in the \isi{Amdo Sprachbund} (\citealt{SandmanSimon2016}, \sectref{sec:3.5}).

Whether \ili{Qiangic} is a valid subgroup of \ili{Trans-Himalayan}, and which languages it should cover, is an ongoing debate. \citet{Chirkova2012} argues that it should be reconceptualized as an areal rather than a genetic group of \ili{Trans-Himalayan} languages. Without a final solution to the problem at hand, this study retains the common designation as \ili{Qiangic}, which is first and foremost a pragmatic decision. In \isi{NEA} only one language is usually classified as \ili{Qiangic}:

\begin{quote}
\ili{Tangut} (also known as the \isi{Xixia} language) is an extinct Tibeto-Burman language that was spoken in the Xixia empire that existed from 1038 to 1227 in northwestern \isi{China}. The language was buried in oblivion till 1908 when the \ili{Russian} geographer P.K. Kozlov discovered the ruins of a \ili{Tangut} city at Khara Khoto. \citep[602]{Gong2003}
\end{quote}

\ili{Baima}, tentatively classified as \ili{Tibetic} here, is sometimes also treated as a \ili{Qiangic} language \citep[139]{Chirkova2012}.

There have been many attempts to connect \ili{Trans-Himalayan} with other language families, none of which is widely accepted. A \textit{\ili{Sino-Tibetan-Austronesian}} hypothesis that also includes \ili{Tai-Kadai} as a branch of Austronesian is currently being debated (see \citealt{Sagart2016}), but does not seem to be gaining acceptance.

\section{Tungusic}\label{sec:2.10}

\ili{Tungusic} is the name of a \isi{language family} that includes about a dozen to twenty different languages distributed over a vast area in \isi{Siberia} and Northern \isi{China}. Experts do not agree on the exact number of languages, primarily because of the complex network of dialects and mutual influence. Instead of \textit{Tungusic}, some researchers prefer the name \textit{\ili{Manchu}-Tungusic} (e.g., \citealt{Pevnov2012}), but I will continue to use the name \textit{Tungusic} as a convenient label for the whole \isi{language family}. The name \textit{Tungusic} historically referred to the \ili{Evenki} or the \ili{Even} and their languages, but today does not designate any specific variety. In addition, if understood in the old sense, the name \textit{\ili{Manchu-Tungusic}} actually refers to only two or three of many more languages. In addition, the term suggests a primary split of the \isi{language family} into \ili{Manchu} and \ili{Tungusic}, which is not necessarily accurate (e.g., \citealt{Ikegami1974}; \citealt{Georg2004}; \citealt{Janhunen2012b}; \citealt{Hölzl2015a,Hölzl2017b}). What is more, the name \textit{Tungusic} belongs to a long tradition of referring to the whole \isi{language family} (e.g., \citealt{Benzing1956}).

\ili{Tungusic} today is usually classified into four different groups (\citealt{Ikegami1974}; \citealt{Georg2004}), which can be called \ili{Jurchenic}, \ili{Nanaic}, \ili{Udegheic}, and \ili{Ewenic} \citep{Janhunen2012b}. According to one hypothesis that will be followed here, the first two form the southern \ili{Tungusic} branch, and the latter two the northern branch. \cite{Janhunen1996,Janhunen2005,Janhunen2012b} assumes that \ili{Proto-Tungusic} was spoken in southern \isi{Manchuria}, east of the \isi{Liao} river and partly in the north of \isi{Korea}:

\begin{quote}
The linguistic facts suggest that the \ili{Tungusic} family represents a classic case of \isi{language spread} from a relatively compact \isi{homeland}. Against the overall ethnohistorical picture of \isi{Northeast Asia}, it appears likely that the \ili{Tungusic} \isi{homeland} was located in the region comprising Southern \isi{Manchuria} and Northern \isi{Korea}, the historical habitat of the Jurchen-\ili{Manchu}. From here \ili{Tungusic} expansion took \ili{Tungusic} to the Armur basin, where \ili{Nanai}, Udeghe, and Ewenki branches subsequently emerged. These initial expansions of \ili{Tungusic} may have taken place between 2000 and 1000 years ago \citep[216-233]{Janhunen1996}. \citep[39]{Janhunen2005}
\end{quote}

\noindent But a more plausible location appears to have been further north, as has also been claimed by \citet{Pevnov2012} and \citet{Vovin2013a}. An educated guess for an original location of \ili{Tungusic} should probably pinpoint the confluence of the \isi{Amur}, the \isi{Sunggari}, and the \isi{Ussuri}. From this region \ili{Jurchenic} expanded southwards along the \isi{Sunggari} and the \isi{Ussuri}, \ili{Nanaic} followed the lower \isi{Amur} northwards, \ili{Udegheic} spread along the eastern tributaries of the lower \isi{Amur} and the \isi{Ussuri}, and \ili{Ewenic} speakers migrated along the \isi{Amur} river towards the northeast and to some extent followed the left tributaries such as the Bureya and the Zeya. Parts of \ili{Ewenic} (mostly \ili{Evenki} and \ili{Even}) then rapidly covered almost all of \isi{Siberia}. This expansion of \ili{Ewenic} has also been recognized by \citet[39]{Janhunen2005}:

\begin{quote}
The modern distribution of \ili{Tungusic} is largely the result of the secondary expansion of the Ewenki branch, which very probably began from the Middle Armur region no more than 1000 years ago. This expansion spread \ili{Tungusic} over the whole of \isi{Siberia}, from the Okhotsk Sea in the east to the \isi{Yenisei} basin in the west, and from Lake \isi{Baikal} in the south to the \isi{Arctic Ocean} in the north. The expansion has continued until recent times, especially in Northeast \isi{Siberia}. Territories reached only in the 19th century include \isi{Kamchatka} (Ewen) and \isi{Sakhalin} (Siberian Ewenki).
\end{quote}

\noindent \citet{Janhunen2005} is right in \isi{pointing} out the internal homogeneity of both \ili{Evenki} and \ili{Even}, which indicates a very recent spread. Even today the number of \ili{Ewenic} languages is highest in \isi{Manchuria}.

A recent study found evidence that the direct ancestors of some \ili{Tungusic}-speaking peoples have been living in \isi{Manchuria} for at least 8000 years:

\begin{quote}
We report genome-wide data from two hunter-gatherers from Devil’s Gate, an early Neolithic cave site (dated to {\textasciitilde}7.7 thousand years ago) located in \isi{East Asia}, on the border between \isi{Russia} and \isi{Korea}. Both of these individuals are genetically most similar to geographically close modern populations from the \isi{Amur} Basin, all speaking \ili{Tungusic} languages, and, in particular, to the Ulchi. (\citealt{Siska2017}: 1)
\end{quote}

\noindent This is no proof, of course, that the ancestors of the \ili{Tungusic} \isi{language family} were spoken in the area as well. However, the genetic continuity might suggest that there may not have been a \isi{language shift} from some unknown languages to the \ili{Tungusic} languages family (or its predecessor), which would be expected to leave clearer traces of genetic \isi{admixture}. Another recent genetic study, for example, found that the \ili{Udihe} appear to be “the result of \isi{admixture} between local Amur-\isi{Ussuri} populations and \ili{Tungusic} populations from the north.” (\citealt[1]{Duggan2013}) Unfortunately, it is still too early to draw any substantial linguistic conclusions based on these results.

There are too many instances of \isi{language contact} of \ili{Tungusic} languages all over \isi{NEA} to be summarized here in detail. \ili{Manchu} used to be an important superstrate language for all languages in \isi{Manchuria} and also had a certain impact on \ili{Mandarin} (e.g., \citealt{Tsumagari1997}). \ili{Manchu} itself has a pronounced \ili{Mongolic}, \ili{Para-Mongolic}, and perhaps \ili{Koreanic} adstrate. \ili{Sibe} had \isi{contact} with \ili{Mongolic} languages such as \ili{Khorchin} and, the group of speakers who were relocated to \isi{Xinjiang} in 1764, with several \ili{Turkic} languages such as \ili{Uyghur}. Several \ili{Tungusic} languages had \isi{contact} with \ili{Amuric} languages along the lower \isi{Amur}. \ili{Evenki} had \isi{contact} with several \ili{Mongolic} languages such as \ili{Buryat}, with \ili{Nivkh} on \isi{Sakhalin}, as well as with \ili{Yakut}, \ili{Yeniseic}, \ili{Yukaghiric}, and \ili{Samoyedic}. \ili{Even} had \isi{contact} with \ili{Chukotko-Kamchatkan} as well as \ili{Yakut}, and partly replaced \ili{Yukaghiric}. \ili{Oroqen} and especially \ili{Solon} had an almost symbiotic relation to the \ili{Mongolic} language \ili{Dagur} (e.g., \citealt{Janhunen1997}). The same is true for the two Khamnigan \ili{Evenki} dialects with \ili{Khamnigan Mongol} (e.g., \citealt{Janhunen1991}; \citealt{Janhunen2003c}). Before the advent of \ili{Russian} and \ili{Mandarin} influence, \ili{Khitano-Mongolic} exerted the most important influence over all of \ili{Tungusic} \citep{Doerfer1985}.

\section{Turkic}\label{sec:2.11}

\ili{Turkic} languages are widespread today, from the Arctic Sea in the north to \isi{Qinghai} in the south and from \isi{Manchuria} in the east to Turkey in the west (excluding recent migrations to Germany, for instance). The spread of \ili{Turkic} all over Eurasia had its beginnings in southern \isi{Siberia} and northern \isi{Mongolia}, where the oldest \ili{Turkic} records, the Orkhon inscriptions, were found \citep{Golden1998}. \ili{Turkic} has perhaps six main branches, \ili{Oghur}, \ili{Khalaj}, Siberian, \ili{Uyghur}-Karlak, \ili{Kipchak}, and \ili{Oghuz} (\citealt{Johanson1998}: 81f.; \citealt{Johanson2006a}: 161f.). First \ili{Oghur}, today only represented by \ili{Chuvash} in European \isi{Russia}, and then perhaps \ili{Khalaj} (in Iran) split away from the rest. Most languages covered here are from the Siberian branch, but languages from all branches except \ili{Oghur} and \ili{Khalaj} are today located in \isi{NEA}. This study excludes the by now perhaps extinct archaic \ili{Turkic} language \ili{Khotong} from \isi{Mongolia}, for which no data are available to me \citep[148]{Shimunek2015}.

The classification above only includes modern \ili{Turkic} languages, but there are historically attested varieties of \ili{Turkic} that will be briefly mentioned as well, notably Old \ili{Turkic} and \ili{Chagatay}.

\begin{quote}
Old \ili{Turkic} is taken to be the language underlying three corpora. The first one consists of official or private inscriptions in the runiform script, dating from the seventh to tenth centuries, in the territory of the second Türk empire and the \ili{Uyghur} steppe empire - preset-day \isi{Mongolia} - and the Yenisey basin. The second and most extensive corpus consists of ninth to thirteenth century Old \ili{Uyghur} manuscripts from northwest \isi{China} in \ili{Uyghur}, runiform and other scripts. [...] The third corpus consists of eleventh-century texts from the Karakhanid state, mostly in \ili{Arabic} script [...]. \citep[138]{Erdal1998}
\end{quote}

\begin{quote}
Chaghatay can be defined as a succession of stages of written \ili{Turkic} in Central \isi{Asia}. In many respects it is also a continuation of earlier stages, notably of Karakha\-nid \ili{Turkic}, with Kharezmian \ili{Turkic} as a transitional stage. It cannot be defined as a fixed entity in time and space. Chaghaty sources are a hybrid collection of different varieties of \ili{Turkic}, who from the late fifteenth century onwards more or less tried to focus on a specific model known as Classical Chaghatay. (\citealt{BoeschotenVandamme1998}: 166)
\end{quote}

\noindent Chagathay influenced several written languages, including the \ili{Kipchak} languages \ili{Tatar} and \ili{Kazakh}, the \ili{Oghuz} language \ili{Turkmen}, and the \ili{Uygur-Karluk} languages \ili{Uzbek} and \ili{Uyghur} (\citealt{BoeschotenVandamme1998}: 168). In fact, the \ili{Uygur-Karluk} branch is sometimes also called the \ili{Chagatay} branch of \ili{Turkic}. 

The extensive \isi{contact} between \ili{Turkic} and other languages has been summarized by \citet{Schönig2003} and \citet{Johanson2010}. \ili{Turkic} languages in general had strong \isi{contact} with \ili{Mongolic}. But individual languages underwent a plethora of \isi{contact} situations that cannot all be summarized here. \ili{Yakut} had \isi{contact} with \ili{Buryat}, and later with \ili{Evenki}, \ili{Yukaghir}, and \ili{Nganasan}, which led to the emergence of \ili{Dolgan}. In the southwest there is \isi{contact} with \ili{Iranian} and in the southeast with \ili{Sinitic}. In the \isi{Amdo} region there is a strong \isi{interaction} with \ili{Mongolic} and \ili{Sinitic} varieties as well as with \ili{Amdo Tibetan}.

\section{Uralic}\label{sec:2.12}

\ili{Uralic} (e.g., \citealt{Sinor1988}; \citealt{Abondolo1998}) is a \isi{language family} with a very long history comparable to that of \ili{Indo-European}. The primary split of the \isi{language family} separates the \ili{Samoyedic} languages from \ili{Finno-Ugric}. Despite the rather small comparative corpus between \ili{Finno-Ugric} and \ili{Samoyedic}, their genetic relation is usually recognized. \ili{Proto-Samoyedic} perhaps split about 2000 years ago, while \ili{Finno-Ugric} and \ili{Samoyedic} had a common origin in \ili{Proto-Uralic} about 5000 years ago \citep[68]{Janhunen2009}. The location of the \ili{Uralic} \isi{homeland} is disputed, but \citet[71]{Janhunen2009} argues for “the borderline between the Ob and \isi{Yenisei} drainage areas in \isi{Siberia}” and thus for a region at the edge of \isi{NEA}. Given the connection of \ili{Uralic} with \ili{Yukaghiric} (\sectref{sec:2.14}), Pre-\ili{Proto-Uralic} could even have been spoken in \isi{NEA}. However, only the \ili{Samoyedic} branch is clearly represented in \isi{NEA} (e.g., \citealt{Janhunen1998}).

\begin{quote}
Listed roughly from north to south, these are (older designations given in parentheses): \ili{Nganasan} (Tavgy), \ili{Enets} (Yenisei-Samoyed), \ili{Nenets} (Yurak), \ili{Selkup} (Ostyak-Samoyed), \ili{Kamass}(ian), and \ili{Mator} (Motor). The southernmost languages \ili{Kamass} and \ili{Mator}, are now no longer spoken: \ili{Mator} was replaced by \ili{Turkic} idioms during the first half of the nineteenth century, and the fact that it is known at all today is because of intensive philological work done with word lists; the last \ili{Kamass} speaker died in 1989. [...] only \ili{Nenets} is spoken by a relatively large number of people (some 27,000); \ili{Selkup}, which has sharp dialectal divisions, has fewer than 2,000 speakers; \ili{Nganasan}, some 600; and \ili{Enets}, perhaps 100. \citep[2]{Abondolo1998}
\end{quote}

Elena Skribnik (p.c. 2017) informed me that in \isi{NEA} there are also a few speakers of, for example, \ili{Estonian}. However, such isolated groups will mostly be neglected in this study (but see \citealt{Miestamo2011} and \sectref{sec:5.12.2}). Together with \ili{Yeniseic}, \ili{Samoyedic} forms the western border of the \isi{NEA} area. \ili{Samoyedic} may have been spoken by the Tagar culture (ca. 1000-200 BCE) in the Minusinsk basin (\citealt{Janhunen2009}: 72; \citealt{Parpola2012}: 294), but this remains somewhat speculative. Just like \ili{Yeniseic}, \ili{Samoyedic} spread along the \isi{Yenisei} northwards, while those varieties left behind were slowly replaced by languages from other families.

\ili{Samoyedic} had \isi{contact} with several \ili{Finno-Ugric}, \ili{Yeniseic}, and \ili{Turkic} languages as well as \ili{Evenki}, \ili{Russian}, and perhaps some early form of Tocharian. \ili{Selkup} had an especially strong \isi{interaction} with the \ili{Yeniseic} language \ili{Ket}.

\section{Yeniseic (Yeniseian)}\label{sec:2.13}

Typologically, \ili{Yeniseic} is the most atypical Siberian \isi{language family} (\sectref{sec:3.5}). Today it is represented by only one language, namely \ili{Ket}. But there used to be several other \ili{Yeniseic} languages (\ili{Arin}, \ili{Assan}, \ili{Kott}, \ili{Pumpokol}, \ili{Yugh}) that have since disappeared. \ili{Yeniseic} substrate \isi{toponyms}, largely river names that have endings such as \textit{{}-ul}, \textit{{}-ses}, or \textit{{}-det}, cover a large area from the Irtysh in the west to northern \isi{Mongolia} in the east and indicate a more widespread distribution in the past \citep[474]{Vajda2009b}. The \isi{homeland} of \ili{Yeniseic} may have been the \isi{Altai} region, especially the Karasuk culture (1200-700 BCE) (\citealt[1f.]{FlegontovChangmai2016}). According to \citet[33]{Vajda2010}, less than 100 \ili{Ket} are still able to speak the language.

There is some evidence to suggest that a \ili{Yeniseic} language was one of the language of the historic \isi{Xiongnu} (\zh{匈奴}) in northern \isi{China}, the main rivals of the Han dynasty (206 BCE to 220 CE) (cf. \citealt{VovinVajdadelaVaissière2016} and references therein). In addition, \citet{Vajda2010} has made a strong argument for a genetic connection between \ili{Yeniseic} and \ili{Na-Dene} languages, called the \textit{\ili{Dene-Yeniseian} hypothesis}. Apart from \ili{Eskaleut}, this would be the first \isi{language family} discovered that connects languages in \isi{Asia} and the Americas. The theory is currently gaining acceptance as new pieces are added to the puzzle (e.g., \citealt{Vajda2013}), and, at least for the moment, it seems that there are fewer critics than proponents. Nevertheless, more research over the following years will show whether the hypothesis can stand the test of time. If Dene-Yeniseian turns out to be a valid genetic unit, there are several different possible explanations for its modern distribution. One possibility would be to assume a location of the proto-language somewhere in (south)eastern \isi{NEA}. From there, \ili{Yeniseic} moved westwards, whereas \ili{Na-Dene} moved northwards to finally cross \isi{Beringia}. But \citet{SicoliHolton2014} have recently argued for an alternative that assumes an original location in the Beringian area. \ili{Yeniseic}, according to them, is the result of a back-\isi{migration} into \isi{Asia}. However, this goes against the general rule of thumb that migrations in \isi{NEA} usually follow a south-to-north direction. In any case, the \isi{migration} of \ili{Yeniseic} down the \isi{Yenisei} and of Na-Dene from \isi{Alaska} southwards are widely accepted and must be common ground for any additional hypothesis. The question of the time depth of the hypothetical \ili{Proto-Dene-Yeniseian} language remains unsettled for now, but must necessarily be many thousand years older than \ili{Proto-Yeniseic} (see \sectref{sec:5.13.4}).

\section{Yukaghiric}\label{sec:2.14}

The term \textit{Yukaghiric} is employed here to refer to the \isi{language family} usually called \textit{Yukaghir}. However, there are two rather different extant \ili{Yukaghiric} languages, which is why a specialized designation for the \isi{language family} seems in order to avoid confusion. These two languages are called Kolyma \ili{Yukaghir} (Odul) and Tundra \ili{Yukaghir} (Wadul). \citet{TsumagariKurebitoEndo2007} classify Tundra \ili{Yukaghir} as “seriously endangered” and \isi{Kolyma} \ili{Yukaghir} as “moribund” as there are only several dozen elderly speakers left for both languages (\citealt[130]{Matić2014}). \ili{Yukaghiric} languages were mostly replaced by \ili{Even} (\ili{Tungusic}), \ili{Yakut} (\ili{Turkic}), \ili{Chukchi}, and \ili{Kerek} (\ili{Chukotko-Kamchatkan}), as well as \ili{Russian} (\ili{Slavic}). War with and exploitation by the Russians, together with \isi{smallpox} epidemics, decimated their number drastically. Where there were an estimated 4500-5000 \ili{Yukaghir} in the 17\textsuperscript{th} century, only 150-200 remained at the end of the 19\textsuperscript{th} century, but their number has been growing again ever since (\citealt[3]{Rédei1999}; \citealt{Forsyth1992}: 74-80).

\ili{Yukaghiric} languages must have been extremely widespread in northeastern \isi{Siberia} until the 17\textsuperscript{th} century. According to \citet{VolodkoStarikovskaya2008}, the \ili{Yukaghir} were even involved in the formation of the \ili{Samoyedic}-speaking \ili{Nganasan} much further to the west. Even so, they seem to have reached the northern parts of \isi{NEA} from a location further south. \cite[93]{Häkkinen2012} argues that

\begin{quote}
\ili{Yukaghir}[ic] can be derived from the west, as it was spoken earlier near the \isi{Lena}. We may assume that \ili{Yukaghir}[ic] at some point in the past migrated down the \isi{Lena}, just as \ili{Yakut} did later, and that Early Proto-\ili{Yukaghir}[ic] was spoken somewhere near the Upper \isi{Lena} and the region of Lake \isi{Baikal}, the watershed area between the \isi{Lena} and \isi{Yenisei} river systems. (my square brackets)
\end{quote}

\noindent If Häkkinen’s assumption is correct, this brings \ili{Yukaghiric} geographically much closer to other language families such as \ili{Tungusic}, \ili{Samoyedic}, Khitano-\ili{Mongolic}, \ili{Turkic}, and \ili{Yeniseic}. A southern origin of the \ili{Yukaghir} is also corroborated by evidence from mitochondrial DNA analyses (\citealt{VolodkoStarikovskaya2008}). Häkkinen’s conclusions are built on an assumption of a direct \isi{contact} of \ili{Yukaghir} with \ili{Uralic} languages. \citet[61]{Janhunen2009} explicitly denies a connection between \ili{Uralic} and \ili{Yukaghiric}. But most researchers do not exclude the possibility of a genetic connection (e.g., \citealt{Pispane2013}) or at least \isi{contact} (e.g., \citealt{Rédei1999}; \citealt{Aikio2014}). The separation of the two \ili{Yukaghiric} languages has been estimated to date back to about 2,000 years ago \citep[28]{Maslova2003a}, which remains rather speculative and might be an overestimation. For instance, personal pronouns in the two extant \ili{Yukaghiric} varieties are basically identical, which would not be expected after such a long period of separation. The location of Pre-\ili{Proto-Yukaghiric} in the south of \isi{NEA}, on the other hand, must be much older and has been tentatively dated to the early-middle \isi{Holocene} (\citealt[1097]{VolodkoStarikovskaya2008}) and thus might be much earlier than “Early Proto-\ili{Yukaghir}” as was assumed by \cite[93]{Häkkinen2012}.