\section{Japonic}\label{sec:5.6}
\subsection{Classification of Japonic}\label{sec:5.6.1}

As is by now well established, \ili{Japanese} is not an isolated language, as was, for instance, claimed by \citet[89]{Shibatani1990}. Instead, \ili{Japanese} is merely the major, but by no means the only, representative of a \isi{language family} called \ili{Japanese}-\ili{Ryūkyūan}  or simply \ili{Japonic} (e.g., \citealt{Tranter2012a}: 3f.). A simplified classification of \ili{Japonic} languages may tentatively be represented as in \figref{exfig:japa:1} (based on \citealt{Pellard2009}: 264; \citealt{Chien2010}; \citealt{Shimoji2010}; \citealt{Hasegawa2015}: 21ff.), excluding most historically attested and possible \ili{Para-Japonic} languages. Only those \ili{Ryūkyūan}  languages or dialects mentioned during this section are listed.


% \ea\upshape%1
\begin{figure}
\caption{Classification of Japonic}
    \label{exfig:japa:1}
\small
\begin{forest}  for tree={grow'=east,delay={where content={}{shape=coordinate}{}}},   forked edges  
[
    [\ili{Japanese}
        [\ili{Old Japanese}
        	[East
            	[\ili{Hachij\=o} dialect]
            ]
            [Central
            	[\ili{Japanese} dialects]
                [\ili{Yilan Creole}]
            ]
            [\textsuperscript{†}West
            ]
        ]
    ]
    [Ryūkyūan
        [Northern
        	[Amami
            	[\ili{Ura}]
                [\ili{Yuwan}]
                [\ili{Okinoerabu}]
                [\ili{Shodon}]
            ]
            [Okinawan
            	[\ili{Shuri}]
                [\ili{Tsuken}]
            ]
        ]
        [Southern (Sakishima)
        	[Miyako
            	[\ili{Tarama}]
                [\ili{Ikema}]
                [\ili{Ōgami}]
                [\ili{Irabu}]
                [\ili{Hirara}]
                [\ili{Nakachi}]
            ]
            [Macro-Yaeyama
            	[Yaeyama
                	[\ili{Hateruma}]
                	[\ili{Hatoma}]
                	[\ili{Miyara}]
                ]
                [Yonaguni
                	[\ili{Dunan}]
                    [\ili{Sonai}]
        		]
            ]
        ]
    ]
]
\end{forest}   
%     \z
% \normalsize 
\end{figure}

The primary split in \ili{Japonic} is between \ili{Japanese} and Ryūkyūan. Mainland \ili{Japanese} constitutes a dialect \isi{continuum} that can roughly be classified into four larger areas called Eastern, Central, Western, and Kyūshū (\citealt{Hasegawa2015}: 21f.). The \ili{Hachij\=o} dialects and the Okinawan dialect influenced by Ryūkyūan form separate groups in themselves. For reasons of space and lack of sufficient information, a focus will be on Modern Standard \ili{Japanese} in this study.\footnote{A \textit{Handbook of \ili{Japanese} Dialects} has been announced by De Gruyter for 2019.} A special case is \ili{Yilan Creole} spoken on \isi{Taiwan} and has thus also been listed separately. \ili{Even} though the lexicon is mostly based on \ili{Japanese}, Yilan is a creole language that also exhibits certain influences from \ili{Austronesian} languages, especially \ili{Atayal} (\citealt{Chien2010}). Ryūkyūan languages spoken in the Ryūkyū archipelago may be classified into two main branches, Northern and Southern Ryūkyūan, each of which splits into two branches \citep{Shimoji2010}. Southern Ryūkyūan has also been called Sakishima \citep{Bentley2008a}. Yonaguni is treated as a separate branch of Ryūkyūan by \citet{Izuyama2012} and as a separate subbranch of Southern Ryūkyūan by \citet[242]{Bentley2008a}, but is often included within the Macro-Yaeyama subbranch of Southern Ryūkyūan, the other branch of which is Miyako. Northern \ili{Ryūkyūan}  can be divided into Amami and Okinawan. There is a large amount of variation among Ryūkyūan. According to \citet[380]{Lawrence2012}, there are 35 “dialects” within Miyako and 20 within Yaeyama alone. Of course, a classification into languages and dialects is difficult and even somewhat spurious. But clearly the \isi{Ryūkyūan Islands} can be considered a treasure trove of \isi{linguistic diversity}, of which only some parts can be included in this chapter. As is common practice today in the study of Ryūkyūan, the place name will also indicate the language spoken at that place, i.e. \ili{Irabu} on \ili{Irabu} island etc.

\subsection{Question marking in Japonic}\label{sec:5.6.2}

\citet[295]{TranterKizu2012} give a good summary of \isi{question marking} strategies in modern \textbf{Standard Japanese}.

\begin{quote}
Questions of all types including \textit{wh}-\isi{questions} and \textit{yes/no} \isi{questions} are expressed by a change in \isi{intonation} and the addition of a particle at the end of the sentence: familiar-style \textit{-no}, \textit{-Ø}, or polite-style \textit{-ka}. Soliloquy-type \isi{questions} that do not necessarily require a \isi{response} from a hearer use \textit{-kana(a)} or \textit{-kashira} (female). There is no change in \isi{word order}, and no \isi{fronting} in \textit{wh}-word \isi{questions}. Questions that present alternatives, including those that ask a question in an affirmative form with a negative alternative of the same situation, have the structure of two separate \isi{questions}.
\end{quote}

\noindent The speech level differences are not as strongly developed as they are in \ili{Koreanic} \isi{questions} (\sectref{sec:5.7.2}). The default and polite \isi{question marker} in \ili{Japanese} is the sentence-final and possibly enclitic particle \textit{ka} \jp{か}.
The marker can be found in polar, alternative, and \isi{content question}s.

\ea%2
    \label{ex:japa:2}
    \ili{Japanese}\\
    \ea
    \gll dare.ka    ga  kimashita \textbf{{ka}}?\\
    someone  \textsc{nom}  come.\textsc{pst.pol}  \textsc{q}\\
    \glt ‘Did someone come here?’ \citep[104]{Hasegawa2015}
    
    \ex
    \gll sono  hon wa omoshiroi  desu \textbf{{ka}} tsumaranai  desu \textbf{{ka}}?\\
    that  book  \textsc{top}  interesting  \textsc{cop}  \textsc{q}  boring    \textsc{cop} \textsc{q}\\
    \glt ‘Is that book interesting or boring?’ (\citealt{Hinds1984}: 159f.)
    
    \ex
    \gll o-namae  wa \textbf{{nan}} desu \textbf{{ka}}?\\
    \textsc{hon}-name  \textsc{top}  what  \textsc{cop}  \textsc{q}\\
    \glt ‘What is (your) name?’ (own knowledge)
    \z
    \z 

\noindent The same marker also appears at the end of what seem to be \isi{focus question}s. The following two examples were elicited from a native speaker living in Germany in November 2015. The \isi{glossing} follows \citet{Hasegawa2015}.

\ea%3
    \label{ex:japa:3}
    \ili{Japanese}\\
    \ea
    \gll ashita \textbf{{wa}} gakk\=o  ni  iki-masu \textbf{{ka}}?\\
    tomorrow  \textsc{top}  school  to  go-\textsc{npst.pol}  \textsc{q}\\
    \glt ‘Is it tomorrow that you are going to school?’
    
    \ex
    \gll ashita,    gakk\=o  ni \textbf{{wa}} iki-masu \textbf{{ka}}?\\
    tomorrow  school  to  \textsc{top}  go-\textsc{npst.pol}  \textsc{q}\\
    \glt ‘Is it to school that you are going to tomorrow?’
    \z
    \z

\noindent Similar to \ili{Korean} (\sectref{sec:5.7.2}) and \ili{Wutun} (\sectref{sec:5.9.2.1}), it apparently is the topic marker \textit{wa} that follows the focused or perhaps rather topicalized element while the sentence otherwise is identical to a plain \isi{polar question}. \ili{Japanese} has a special way of forming \isi{topic question}s that contain the same topic marker \textit{wa} but have a truncated form.

\ea%4
    \label{ex:japa:4}
    \ili{Japanese}\\
    \gll an\=o,  ky\=odai \textbf{{wa}}?\\
    uh  siblings  \textsc{top}\\
    \glt ‘So, do you have any brothers or sisters?’ \citep[166]{Hinds1984}
    \z

In \textbf{Old Japanese}, the particle \textit{ka} already existed but differed from the modern \ili{Japanese} one in its syntactic behavior. According to \citet[1220]{Vovin2009}, it was present in both Eastern and Western \ili{Old Japanese} and has the same scope as in modern \ili{Japanese}. But, in contrast to the strict sentence-final position today, the particle could appear in other positions as well. Apparently, the particle also marked \isi{focus question}s and attached to the focused element in both \isi{focus} and \isi{content question}s.

\ea%5
    \label{ex:japa:5}
    \ili{Old Japanese}\\
    \ea
    \jp{嚢伽多佐例}\\
    \gll \textbf{{ta}}=\textbf{{ka}} ta-sar-e?\\
    who=\textsc{q}    \textsc{emp}?-go.away-\textsc{ev}\\
    \glt ‘Who goes away?’
    
    \ex
    \jp{今夜可君之我許来益武}\\
    \gll KÖ  YÖPI=\textbf{{ka}} KYIMYI-NKA    WA-Nkari K-YI-[i]mas-am-u?\\
    this  night=\textsc{q}  lord-\textsc{poss}    1\textsc{sg}-\textsc{dir} come-\textsc{inf}-\textsc{hon}-\textsc{tent}-\textsc{attr}\\
    \glt ‘Is it \textit{tonight} that (my) lord will come to me?’ (Western; \citealt{Vovin2009}: 1220, 1225)
    \z
    \z

\noindent Typologically, this is a change similar to the one observed from \ili{Middle Mongol} to modern \ili{Mongolian} (see \sectref{sec:5.8.2}).

In Eastern \ili{Old Japanese} \textit{=ka} is attested as a marker for polar, \isi{focus}, and \isi{content question}s and triggers \textit{\isi{kakari musubi}} ‘\isi{focus} concord’ (see further below): It “forces the main verb to take an attributive suffix, regardless of whether it follows or precedes the verb” \citep[834]{Kupchik2011}.

\ea%6
    \label{ex:japa:6}
    \ili{Old Japanese} (Eastern)\\
    \ea
    \jp{安杼加安我世牟}\\
    \gll \textbf{{aNtö}}{=}\textbf{{ka}} a-Nka se-m-u?\\
    what=\textsc{q}  1\textsc{sg}-\textsc{poss}  do-\textsc{tent}-\textsc{attr}\\
    \glt ‘What should I do?’
    
    \ex
    \jp{於不世他麻保加}\\
    \gll opuse-tamap-o=\textbf{{k}}\textbf{{a}}?\\
    assign.\textsc{inf}-\textsc{hon}-\textsc{attr=}  \textsc{q}\\
    \glt ‘Has (the emperor) given (me) the order?’
    
    \ex
    \jp{夜麻尓可祢牟}/\jp{毛夜杼里波奈之尓}\\
    \gll yama-ni=\textbf{{ka}} ne-m-u mwo  /  yaNtör-i pa na-si-ni?\\
    mountain-\textsc{loc=q}  sleep-\textsc{tent}-\textsc{attr}  \textsc{foc}  /  lodge-\textsc{n} \textsc{top}  not.exist-\textsc{fin}-\textsc{loc}\\
    \glt ‘Shall (I) sleep \textit{in the mountains} since there is no lodging (here)?’ (\citealt{Kupchik2011}: 834, 835)
    \z
    \z 

\noindent According to \citet[1229]{Vovin2009}, the particle has a cognate in \ili{Ryūkyūan} languages and can be traced back to \ili{Proto-Japonic}.

\begin{quote}
The \isi{interrogative} particle \textit{ka} {\textasciitilde} \textit{ga} (< *\textit{-N ka}) is well attested in both \ili{Old Ryūkyūan} and modern dialects. However, as far as I can tell, Ryukyuan \textit{ka} {\textasciitilde} \textit{ga} appears exclusively in \textit{wh-}\isi{questions} [CQ]. Thus, in all probability, WOJ [Western \ili{Old Japanese}] \textit{ka} in general \isi{questions} [PQ] represents a \ili{Japanese} innovation, and we should reconstruct PJ [\ili{Proto-Japonic}] *ka, \isi{interrogative} particle in \textit{wh-}\isi{questions}.
\end{quote}

\noindent An example from \ili{Old Ryūkyūan} is the following:

\ea%7
    \label{ex:japa:7}
    Old Ryūkyūan\\
    \jp{けおわのかしよらしよ}\\
    \gll keo  wa \textbf{{no=ka}} s-i-yor-asiyo?\\
    today  \textsc{top}  what=\textsc{q}  do-\textsc{inf}-exist-\textsc{sup}\\
    \glt ‘What would (they) do today?’ \citep[1229]{Vovin2009}
    \z

Old \ili{Korean} had a similar marker \textit{-ka} \zh{去} that might be somehow related to the \ili{Japonic} form (\sectref{sec:5.7.2}). But as we will see later in some \ili{Ryūkyūan} languages, there is also the possibility that the marker is the result of a language internal development from a \isi{focus} marker.

\ili{Japanese} exhibits an instance of the \isi{grammaticalization} from \isi{nominalization} to \isi{question marker} through ellipsis of the following copula and original \isi{question marker}.

\ea%8
    \label{ex:japa:8}
   \ili{Japanese}\\
    \gll doko  e  iku \textbf{{no}} (desu \textbf{{ka}})?\\
    where  \textsc{all}  go  \textsc{n}>\textsc{q}  \textsc{cop}  \textsc{q}\\
    \glt ‘Where are you going?’ \citep[163]{Hinds1984}
    \z

The suffix \textit{no} originally may have been the genitive \isi{case} marker \citep[258]{Shibatani1990}. See \sectref{sec:5.1.2} on \ili{Ainuic} and below on \ili{Ryūkyūan} for similar developments from nominalizer to \isi{question marker} that may be due to \isi{contact} with \ili{Japanese}. According to \citet[297]{Hasegawa2015} \textit{no} adds “various nuances, typically softening the locution when addressing an interlocutor. It is, therefore, considered mildly feminine even though male speakers also use this particle.” The most aberrant \ili{Japanese} dialect, \textit{\ili{Hachij\=o}}, has a marker \textit{kai} that was written attached to a preceding word or with a hyphen and translated with \ili{Japanese} \textit{ka} \jp{か}. Presumably, it is either a particle or an enclitic. Content \isi{questions} seem to remain unmarked (\citealt{KKK1950}: 130, 208). The Tsuruoka dialect in northern Honshū marks polar \isi{questions} with \textit{ga} and \isi{content question}s with \textit{na} (\citealt{Matsumori2012}: 323, 325). In the Ei dialect in southern Kyūshū, both polar and content \isi{questions} take the marker \textit{ka} or its formal variant \textit{kana} (\citealt{Matsumori2012}: 342).

\textbf{Yilan Creole} has the two optional sentence-final markers \textit{ga} and \textit{no}, corresponding to \ili{Japanese} \textit{ka} and \textit{no}, respectively. As opposed to \ili{Japanese} \textit{ka}, \textit{ga} apparently does not appear in \isi{content question}s, which remain unmarked. This may be due to influence from \ili{Atayal} or \ili{Chinese}. Polar questions generally have a rising \isi{intonation}.

\ea%9
    \label{ex:japa:9}
    \ili{Yilan Creole}\\
    \ea
    \gll kore ga \textbf{{nani}}?\\
    this  \textsc{top} what\\
    \glt ‘What (is) this?’
    
    \ex
    \gll anta kyolai aru \textbf{{ga}}?\\
    2\textsc{sg}  sibling have  \textsc{q}\\
    \glt ‘(Do) you have siblings?’
    
    \ex
    \gll anta no hoyin ga qalux \textbf{{no}}?\\
    2\textsc{sg}  \textsc{gen}  dog  \textsc{top}  black  \textsc{q}\\
    \glt ‘(Is) your dog black?’\footnote{The words \textit{hoyin} ‘dog’ and \textit{qalux} ‘black’ have been borrowed from \ili{Atayal}.}
    
    \ex
    \gll anta teykan ’suw?\\
    2\textsc{sg}  chair    heavy\\
    \glt ‘(Is) your chair heavy?’ (\citealt{Peng2015}: 52, 54, 55)\footnote{The words \textit{teykan} ‘chair’ and \textit{’suw} ‘heavy’ derive from \ili{Atayal}.}\z\z

\noindent \ili{Yilan Creole} \isi{questions} thus behave very similarly to those in \ili{Japanese} but have a slightly different form and \isi{semantic scope}.

The last \isi{question marker} mentioned by \citet[295]{TranterKizu2012} as quoted above is \textit{-ka.na(a)} or \textit{-ka.shira}, formerly used in women’s speech, employed for \isi{questions} to oneself. According to \citet[294]{Hasegawa2015} \textit{ka.shira} “expresses \isi{uncertainty} and \isi{curiosity}” and has been translated as ‘I wonder’. As we will see below, \ili{Ryūkyūan} languages have similar markers containing an element \textit{-ka-} {\textasciitilde} \textit{-ga-} that was translated in the same way.

Both Eastern and Western \textbf{Old Japanese} had another \isi{question marker} \textit{ya} found in polar and \isi{focus question}s. Its behavior in these two dialect groups is rather similar, but there are minor differences. For Eastern \ili{Old Japanese} we have the following description by \citet[832]{Kupchik2011}:

\begin{quote}
When in the sentence-final position, it follows the copula \textit{tö,} the defective verb \textit{tö} ‘think,’ or the evidential form of the verb. The examples with the evidential are used to make ironic \isi{questions} [...]. When this particle is fronted to a pre-verbal position, the verb form must take the attributive suffix [...]. Unlike WOJ, where \textit{ya} is amply attested directly after the final form of a verb or the final exclamative -\textit{umo} \citep[1211]{Vovin2009}, such usages are unattested in EOJ.
\end{quote}

In Western \ili{Old Japanese},  the non-final position -- presumably found in \isi{focus question}s -- also accompanies the attributive form of the verb. In case it is sentence-final -- in \isi{polar question}s -- it may follow final, evidential, and exclamative forms, but not attributive ones \citep[1211]{Vovin2009}.

\ea%10
    \label{ex:japa:10}
    \ili{Old Japanese} (Eastern)\\
    \ea
    \jp{宇恵古奈宜}/\jp{賀久古非牟等夜}\\
    \gll uwe kwo-na-N-kyi / Nka-ku kwopiy-m-u tö=\textbf{{ya}}?\\
    sow.\textsc{inf}  dim-water-\textsc{loc}-leeks  /  be.thus-\textsc{inf} long.for-tent-\textsc{fin}  think/say=\textsc{q}\\
    \glt ‘Do (you) think (I) love the sowed water leeks so much?’
    
    \ex
    \jp{阿須由利也}/\jp{加曳我伊牟多祢牟}\\
    \gll asu-yuri=\textbf{{ya}} / kaye-Nka muta ne-m-\textbf{{u}}?\\
    tomorrow-\textsc{abl=q}  /  reed-\textsc{poss}  together.with sleep-\textsc{tent}-\textsc{attr}\\
    \glt ‘From tomorrow shall (I) sleep together with the reeds?’ \citep[832]{Kupchik2011}
    \z
    \z

\ea%11
    \label{ex:japa:11}
    \ili{Old Japanese} (Western)\\
    \ea
    \jp{儺波企箇輸揶}\\
    \gll na  pa  kyik-as-u=\textbf{{ya}}?\\
    2\textsc{sg}  \textsc{top}  ask-\textsc{hon}-\textsc{fin=q}\\
    \glt ‘Shall (I) ask you?’
    
    \ex
    \jp{枳彌波夜那祇}\\
    \gll kyimyi    pa=\textbf{{ya}} na-\textbf{{kyi}}?\\
    lord    \textsc{top=q}    \textsc{neg}-\textsc{attr}\\
    \glt ‘Don’t (you) have a lord?’ (\citealt{Vovin2009}: 1211, 1215)
    \z
    \z

\noindent Similarly to the particle \textit{ka}, \citet[1219]{Vovin2009} assumes that \textit{ya} has cognates in \ili{Ryūkyūan} and that it can be traced back to \ili{Proto-Japonic}.

\begin{quote}
The cognates \textit{ya} {\textasciitilde} \textit{yaa} of the Western \ili{Old Japanese} \isi{interrogative} particle \textit{ya} are well attested in modern Ryukyuan dialects, although in most dialects \textit{ya} {\textasciitilde} \textit{yaa} have the function of a confirmation seeker, like MdJ \textit{ne}, and not an \isi{interrogative} particle. As far as I can tell, \textit{ya} {\textasciitilde} \textit{yaa} occurs only in sentence-final position.
\end{quote}

But according to \citet[305]{Shinzato2015}, the \ili{Old Japanese} marker rather corresponds to the \ili{Ryūkyūan} \isi{question marker} \textit{(y)i}, on which see below. Whether \ili{Ainu} \textit{ya} may be compared remains an open question, but it may well have been borrowed from older stages of \ili{Japanese} (\sectref{sec:5.1.2}). A sentence-final particle \textit{ya} in Standard \ili{Japanese} is usually accompanied by falling \isi{intonation} and does not express \isi{questions} \citep[298]{Hasegawa2015}.

In \textbf{Standard Japanese} there is another sentence-final particle \textit{kk}e, the function of which \citet[2687]{HayashiM2010} explains as follows:

\begin{quote}
Thus, unlike \textit{ka} and \textit{no}, \textit{kke} makes implicit reference to knowledge or information previously held by the speaker and shared with the addressee, but which the speaker has somehow forgotten or is unsure about. The particle then serves to enlist collaborative participation of the addressee in the process of regaining that knowledge/information.
\end{quote}

\ea%12
    \label{ex:japa:12}
    \ili{Japanese}\\
    \gll are  ichi-nen  deshita \textbf{{kke:}}?\\
    \textsc{excl}  one-year  \textsc{cop.pst.pol}  \textsc{q}\\
    \glt ‘Wait, is (your visa valid) for one year?’ (\citealt{HayashiM2010}: 2687, simplified)
    \z

There is also a special marker \textit{tte} for \isi{echo questions}, which is a variant of the quotative marker \textit{to} used in casual speech. But \textit{to} cannot function as a sentence-final particle (\citealt{Hasegawa2015}: 310f.).

\ea%13
    \label{ex:japa:13}
    \ili{Japanese}\\
    \gll dare  deshita \textbf{{tte}}?\\
    who  \textsc{cop}.\textsc{pst.pol}  \textsc{quot}>\textsc{q}\\
    \glt ‘Who did you say it was?’ \citep[165]{Hinds1984}
    \z

\noindent According to \citet[165]{Hinds1984}, the marker has its origin in an ellipsis of the subsequent \isi{speech act} verb followed by the \isi{question marker} \textit{ka}.

In a comparative study of \isi{question-response sequences} in ten different languages, \ili{Japanese} had the highest ratio of \isi{polar question}s (85\%), as opposed to \isi{content question}s (15\%). There was only one \isi{alternative question}. But 39\% percent of the polar \isi{questions} had a declarative form and 30\% were actually \isi{tag question}s (\citealt{HayashiM2010}: 2686). There were three different \isi{tag question} markers, \textit{janai}, \textit{desh\=o}, and \textit{yo ne}. The first is a negative copula \textit{ja-nai} ‘\textsc{cop}-\textsc{neg}’ and can roughly be translated as ‘isn’t it?’. It has the shorter version \textit{jan} and a more polite variant \textit{janai desu ka}. The tag marker \textit{desh\=o} and its less polite variant \textit{dar\=o} are actually so-called conjectural copula forms meaning ‘probably be’ \citep[80]{Hasegawa2015} and “ask for the addressee’s confirmation to the speaker’s conjecture” (\citealt{HayashiM2010}: 2689). They roughly correspond to \ili{English} tag \isi{questions} such as ‘is it?’. The last form \textit{yo ne} is a \isi{combination} of two different markers the function of which goes well beyond the marking of \isi{questions} (see \citealt{Hasegawa2015}: 299ff.). According to \citet[2690]{HayashiM2010}, “these particles are used sentence-finally to make an assertion while seeking confirmation/agreement to it from the addressee.” In \isi{combination}, \textit{yo ne} was translated as ‘don’t you think?’ But \textit{ne} can also mark \isi{questions} on its own. It has a variant \textit{na} that is usually used by men.

\ea%14
    \label{ex:japa:14}
    \ili{Japanese}\\
    \gll ii  tenki    da \textbf{{na}}?\\
    good  weather  \textsc{cop}  \textsc{q}\\
    \glt ‘It is a fine day, isn’t it?’ \citep[296]{Hasegawa2015}
    \z

\noindent Whether this might be a cognate of a \isi{question marker} found in several \ili{Ryūkyūan} languages remains unclear to me.

The marking of \isi{questions} in \textbf{\ili{Ryūkyūan} languages} is less well described than for \ili{Japanese}. In general, there are similar patterns with sentence-final particles, but in some languages there are question suffixes and the pattern of \isi{question marking} may be quite complex. In \textbf{Ura} (spoken on Amami \=Oshima) polar \isi{questions}, for instance, there is either a rising \isi{intonation} or a simple sentence-final clitic \textit{=na} {\textasciitilde} \textit{=nja}.

\ea%15
    \label{ex:japa:15}
    \ili{Ura}\\
    \gll kuri=ja  hon=\textbf{{na}}?\\
    this=\textsc{top}  book=\textsc{q}?\\
    \glt ‘Is this a book?’ \citep[27]{Shigeno2010}
    \z

There is an additional marker for “self-\isi{questions}”, the \isi{semantic scope} of which was not given. It might belong to other forms meaning ‘I wonder’, e.g. \ili{Japanese} \textit{-ka.na}.

\ea%16
    \label{ex:japa:16}
    \ili{Ura}\\
    \gll an  ʔcju=ja \textbf{{taru}}{=}\textbf{{kai}}?\\
    that  person=\textsc{top}  who=\textsc{q}\\
    \glt ‘Who is that person?’ \citep[27]{Shigeno2010}
    \z

\citet{Shigeno2010} does not further specify whether content \isi{questions} receive a special marking or not, but among his examples there are the markers \textit{=joo} (in CQ), \textit{=kana} (in CQ), and \textit{=ja(a)} (in PQ), that were glossed as question markers but not further explained.

\ea%17
    \label{ex:japa:17}
    \ili{Ura}\\
    \ea
    \gll wan=ga kak-ju-Ø-n=\textbf{{ja}}?\\
    1\textsc{sg}=\textsc{nom}  write-\textsc{ipfv}-(\textsc{npst})-\textsc{adn}=\textsc{q}\\
    \glt ‘Should I write?’
    
    \ex
    \gll an ʔcju=nkja=ja \textbf{{icu}}{=raga} kuma=nan ur-i=\textbf{{joo}}?\\
    that  person=\textsc{appr}=\textsc{top}  when=\textsc{abl}  here=\textsc{loc}1  exist-\textsc{npst}=\textsc{q}\\
    \glt ‘What time did those people get here?’
    
    \ex
    \gll \textbf{{nan}}=cjuu=no=\textbf{{kana}}?\\
    what=\textsc{quo}\textsc{t}=\textsc{gen}=\textsc{q}\\
    \glt ‘How should (I) express this?’ (\citealt{Shigeno2010}: 20, 23, 30)
    \z
    \z 

\noindent The enclitic \textit{=ja(a)} is formally identical to the topic and persuasion markers.

A much better description can be found for the closely related language \textbf{Yuwan} (also spoken on Amami \=Oshima). In this variety, the marking of \isi{questions} is much more complicated and displays a typologically very interesting pattern. Similar to \ili{Ura}, polar \isi{questions} are either expressed with rising \isi{intonation} or an enclitic \textit{=na}.

\ea%18
    \label{ex:japa:18}
    \ili{Yuwan}\\
    \gll uro=o    koow-an=\textbf{{na}}?\\
    2\textsc{sg}=\textsc{top}  buy-\textsc{neg}=\textsc{q}\\
    \glt ‘Don’t you buy it?’ \citep[337]{Niinaga2015}
    \z

But \isi{questions} in \ili{Yuwan} may also be expressed by means of affixes. The information is insufficient to decide about the distribution of these three different marking strategies.

\ea%19
    \label{ex:japa:19}
    \ili{Yuwan}\\
    \gll uro=o    koo-ju-\textbf{{mɨ}}?\\
    2\textsc{sg}=\textsc{top}  buy-\textsc{ipfv}=\textsc{q}\\
    \glt ‘Do you buy it?’ \citep[337]{Niinaga2015}
    \z

Altogether, there are the three suffixes, \textit{-mɨ} for \isi{polar question}s, \textit{-u} for content \isi{questions}, and \textit{-ui} for \isi{focus question}s. If that is not enough, the latter two suffixes are not used in isolation but obligatorily combine with \isi{focus} markers that are specific to the question types, i.e. \textit{=ga} in content and \textit{=du} in \isi{focus} \isi{questions}.

\begin{quote}
The clitic \textit{=du} cannot appear with \textit{-u}, while \textit{=ga} cannot appear with \textit{-ui}. This kind of phenomenon, where the presence of a \isi{focus} clitic correlates with the type of verbal \isi{inflection}, is known as \textit{\isi{kakari musubi}} in \ili{Japanese} linguistics \citep[75]{Niinaga2010}
\end{quote}

\noindent The phenomenon called \textit{\isi{kakari musubi}} will be discussed in more detail below. As seen above, neutral \isi{polar question}s take no \isi{focus} marking.

\ea%20
    \label{ex:japa:20}
    \ili{Yuwan}\\
    \ea
    \gll kurɨ=ba tu-ju-\textbf{{mɨ}}?\\
    this=\textsc{acc} take-\textsc{ipfv}-\textsc{q}\\
    \glt ‘Will (you) take this?’
    
    \ex
    \gll nuu=ba=\textbf{{ga}} tu-jur-\textbf{{u}}?\\
    what=\textsc{acc}=\textsc{foc}   take-\textsc{ipfv}-\textsc{q}\\
    \glt ‘What will (you) take?’
    
    \ex
    \gll kurɨ=ba=\textbf{{du}} tu-jur-\textbf{{ui}}?\\
    this=\textsc{acc}=\textsc{foc} take-\textsc{ipfv}-\textsc{q}\\
    \glt ‘Will (you) take \textit{this}?’ (\citealt{Niinaga2010}: 76f.)
    \z
    \z 

From a \isi{diachronic} perspective the content and \isi{focus question} markers perhaps contain the same element \textit{-u}. The element \textit{-i} possibly has a cognate in \ili{Shuri} \textit{-i(i)} (or perhaps \textit{=ji}). Clearly, \ili{Yuwan} -\textit{mɨ} is cognate with \ili{Shuri} and \ili{Tsuken} \textit{-mi}. It has been proposed that these also contain an actual \isi{question marker} -\textit{ɨ} {\textasciitilde} \textit{-i}. The \isi{focus} marker \textit{=du} may also appear in declarative sentences while \textit{=ga} is restricted to \isi{content question}s \citep[75]{Niinaga2010}. The three question markers exhibit an interesting \isi{interaction} with polarity and tense (\tabref{tab:japa:1}). In the past tense the question markers attach to the “declarative” (past) marker \textit{-tar}, the loss of the \textit{r} before consonants is regular. In non-past tense, on the other hand, the question markers replace the declarative \textit{-i} (perhaps cognate with \ili{Shuri} \textit{-i} ‘\textsc{prs.ptcp}’).

\begin{table}
\caption{Declarative and question markers in Yuwan \citep[64]{Niinaga2010}}
\label{tab:japa:1}

\begin{tabularx}{\textwidth}{XXXl}
\lsptoprule
& \textbf{Category} & \textbf{Assertion} & \textbf{Negation}\\
\midrule
\textsc{npst} & \textsc{decl} & -i & -an\\
& PQ & \textbf{-mɨ} & \textbf{-amɨ}\\
& FQ & \textbf{-ui} & -\\
& CQ & \textbf{-u} & -\\
\textsc{pst} & \textsc{decl} & -tar & -an-tar\\
& PQ & -ta-mɨ & -an-ta-mɨ\\
& FQ & -tar-ui & -an-tar-ui\\
& CQ & -tar-u & -an-tar-u\\
\lspbottomrule
\end{tabularx}
\end{table}

In the non-past the polar \isi{question marker} has a special negative form \textit{-amɨ} as opposed to the plain negative \textit{-an}. Negative forms of \textit{-ui} and \textit{-u} apparently only exist in the past tense.

Another \isi{question marker} \textit{=ga(i)} is always used in \isi{combination} with the suppositional enclitic \textit{=daroo}. The following sentence was translated as a \isi{tag question} by Niinaga.

\ea%21
    \label{ex:japa:21}
    \ili{Yuwan}\\
    \gll an ʔcjoo sjensjee=ja ar-an=\textbf{{daroo}}=\textbf{{ga(i)}}?\\
    that  person.\textsc{top}  teacher=\textsc{top}  \textsc{cop}-\textsc{neg}.\textsc{npst}=\textsc{supp}=\textsc{q}\\
    \glt ‘(I) suppose that that person is not a teacher, is that right?’ \citep[73]{Niinaga2010}
    \z

\citet[72]{Niinaga2010} also used the gloss ‘confirmative question’ for \textit{=ga(i)}. Another enclitic \textit{=jəə} “is used only with intentional \isi{inflection} to confirm the hearer acknowledges the intention of the speaker” \citep[72]{Niinaga2010}.

\ea%22
    \label{ex:japa:22}
    \ili{Yuwan}\\
    \gll waŋ=ga  ik-joo=\textbf{{jəə}}?\\
    1\textsc{sg}=\textsc{nom}  go-\textsc{int}=\textsc{q}\\
    \glt ‘I will go, right?’ \citep[329]{Niinaga2015}
    \z

\Citet{vanderLubbeTokunaga2015} give an overview of two dialects spoken on \textbf{Okinoerabu} among the Amami islands, Masana in the west and Kunigami in the east. But most examples for \isi{questions} are from Masana. \textbf{Masana} has the same enclitic \textit{=na} {\textasciitilde} \textit{=nja} for \isi{polar question}s as several languages mentioned above, but in \isi{content question}s the same form \textit{=joo} as in \ili{Ura} is found.

\ea%23
    \label{ex:japa:23}
    \ili{Okinoerabu} (Masana)\\
    \ea
    \gll ʔatia-ŋ=\textbf{{nja}}?\\
    \textit{know-}\textsc{ind}\textit{=}\textsc{q}\\
    \glt ‘Do you know?’
    
    \ex
    \gll \textbf{{ʔuda}}{=gatʃi=}\textbf{{joo}}?\\
    where=\textsc{dir}=\textsc{q}\\
    \glt ‘Where are you going?’ (\citealt{vanderLubbeTokunaga2015}: 353)
    \z
    \z

\noindent The dubitative suffix \textit{-ra} usually combines with the \isi{focus} marker \textit{=ga} and was translated as ‘I wonder if’ but is not a \isi{question marker} in the strict sense. Another dubitative marker \textit{-ro} on the other hand is “used to ask \isi{questions} in a less direct way” (\citealt{vanderLubbeTokunaga2015}: 357).

\ea%24
    \label{ex:japa:24}
    \ili{Okinoerabu} (Masana)\\
    \gll kiba-ti      mee-\textbf{{ro}}?\\
    work.hard-\textsc{med}   exist.\textsc{hon}-\textsc{dub}\\
    \glt ‘Are you working hard?’ (a greeting) (\citealt{vanderLubbeTokunaga2015}: 357)
    \z

Exactly the same description was given for three other markers. The enclitic \textit{=kaja} could be related to \ili{Shuri} \textit{=gajaa}. Both can be found in content \isi{questions}, e.g. \textbf{\textit{taru}}\textit{=}\textbf{\textit{kaja}}? ‘Who would that be?’ (\citealt{vanderLubbeTokunaga2015}: 362). The origin of the other two (PQ \textit{=sa}, CQ \textit{=do}) remains unclear for now. According to \citet[361]{vanderLubbeTokunaga2015} “in the past tense, the medial \isi{converb} is used rather than the past tense suffix \textit{-ta-}.” In \textbf{Kunigami} and other varieties in the eastern part, the verbal suffix \textit{-jee} is employed instead of the enclitic \textit{=na} {\textasciitilde} \textit{=nja}. This might be a cognate of \ili{Yuwan} \textit{=jəə} and \ili{Ōgami} \textit{-ɛɛ} that we will soon encounter, e.g. \textit{kuruma ʔa-}\textbf{\textit{jee}}? ‘car \textsc{cop}-\textsc{q}’ ‘Is there a car?’ (\citealt{vanderLubbeTokunaga2015}: 362).

\textbf{Shuri} (or Okinawan) as spoken on Okinawa has several question markers and displays strong similarities to other languages mentioned thus far. There is a particle \textit{naa} that has a short vowel in \ili{Ura}, \ili{Yuwan}, \ili{Tsuken}, \ili{Tarama}, and \ili{Ikema} and in these languages has sometimes been analyzed as enclitic, sometimes as freestanding particle. It has been translated as a \isi{tag question} by \citet{Miyara2015}, but may also be a plain polar \isi{question marker}.

\ea%25
    \label{ex:japa:25}
    \ili{Shuri}\\
    \gll kamadee=ga  maŋgo  tʃuku-ta-n=\textbf{{naa}}?\\
    \textsc{pn}=\textsc{nom}  mango  grow-\textsc{pst}-\textsc{ind}=\textsc{q}\\
    \glt ‘Kmadee grew mangoes, didn’t he?’ \citep[394]{Miyara2015}
    \z

\noindent But \ili{Shuri} also has an \isi{interrogative verb} \isi{morphology}. In some cases it is not entirely certain that we are not dealing with enclitics instead, but for purposes of comparison all forms have been given as suffixes. Similar to \ili{Yuwan}, there is a \isi{polar question} suffix \textit{-mi}, but content \isi{questions} take the suffix \textit{-ga}. According to \citet[95]{Uemura2003}, as well as \citep[181f.]{Arakaki2003}, however, the actual \isi{question marker} for polar \isi{questions} is \textit{-i} and the \textit{-m} originally was an affirmative, declarative, or indicative marker that has the form \textit{-\textsc{n}} in other contexts. According to \citet{Arakaki2010}, the suffix \textit{-\textsc{n}} is an evidential marker for “direct evidence”. As opposed to \ili{Yuwan}, which uses the plain negative \textit{-an} and the \isi{interrogative} negative \textit{-amɨ}, \ili{Shuri} retains its original form in the negative, i.e. \textit{-(r)a\textsc{n}-i}. While in \ili{Yuwan} the new polar \isi{question marker} simply attaches to the past tense form (\textit{-ta-mɨ}), \ili{Shuri} has an amalgamated form \textit{-ti(i)} that in all likelihood goes back to a \isi{combination} of the past tense marker \textit{-ta} and the \isi{interrogative} \textit{-i}. However, \citet[145]{Uemura2003} and \citet[67]{Arakaki2015} seem to suggest a \isi{combination} of the past participle and the \isi{question marker} instead. In \isi{content question}s, \textit{-ga} takes the last position, is fully analyzable, and always replaces the indicative ending \textsc{-n}. \tabref{tab:japa:2} gives an overview of \ili{Shuri} verb forms with a focus on \isi{interrogative verb} \isi{morphology}.

\begin{table}
\caption{Shuri verb forms illustrated with the verbs \textit{‘uki-} ‘to wake up’ and \textit{kac-} ‘to write’ according to \citet[180f., passim]{Arakaki2003}; partly reanalyzed (cf. \citealt{Uemura2003})}
\label{tab:japa:2}

\begin{tabularx}{\textwidth}{XXl}
\lsptoprule
& \textbf{Vowel stem} & \textbf{Consonant stem}\\
\midrule
\textsc{prs.ptcp} & ‘uki-i & kac-i\\
\textsc{npst-ind} & ‘uki-ju-\textsc{n} & kac-u-\textsc{n}\\
\textsc{neg (npst)} & ‘uki-ra\textsc{n} & kac-a\textsc{n}\\
\textsc{npst.neg-q (PQ)} & ‘uki-ran-\textbf{i} & kac-an-\textbf{i}\\
\textsc{npst-ind.q (PQ)} & ‘uki-ju-\textbf{mi} & kac-u-\textbf{mi}\\
\textsc{npst-q (CQ)} & ‘uki-ju-\textbf{ga} & kac-u-\textbf{ga}\\
\textsc{npst.neg-q (CQ)} & ‘uki-ra\textsc{n}-\textbf{ga} & kac-a\textsc{n}-\textbf{ga}\\
\textsc{pst.ptcp} & ‘uki-ti & kac-i\\
\textsc{pst-ind} & ‘uki-ta-\textsc{n} & kac-a-\textsc{n}\\
\textsc{neg-pst-ind} & ‘uki-ra\textsc{n-}ta-\textsc{n} & kac-a\textsc{n-}ta-\textsc{n}\\
\textsc{neg-pst.q (PQ)} & ‘uki-ra\textsc{n-}\textbf{ti} & kac-a\textsc{n-}\textbf{ti}\\
\textsc{pst.q (PQ)} & ‘uki-\textbf{ti(i)} & kac-\textbf{i(i)}\\
\textsc{pst-q (CQ)} & ‘uki-ta-\textbf{ga} & kac-a-\textbf{ga}\\
\textsc{neg-pst-q (PQ)} & ‘uki-ra\textsc{n-}ta-\textbf{ga} & kac-a\textsc{n-}ta-\textbf{ga}\\
\lspbottomrule
\end{tabularx}
\end{table}

\citet[95]{Uemura2003} furthermore mentions the partly suppletive copula forms \textit{’ja-\textsc{n}} (affirmative), \textit{’ja-mi} (\isi{polar question}), and \textit{ʔa-ran-i} (\isi{negative polar question}). Consider some examples with \isi{interrogative verb} \isi{morphology}.

\newpage 
\ea%26
    \label{ex:japa:26}
    \ili{Shuri}\\
    \ea
    \gll kamadee=ga  maŋgoo  tʃuku-ju-\textbf{{mi}}?\\
    \textsc{pn}=\textsc{nom}  mango    grow-\textsc{prs}-\textsc{q}\\
    \glt ‘Will Kamadee grow mangoes?’
    
    \ex
    \gll \textbf{{taa}}=ga    maŋgoo  tʃuku-ta-\textbf{{ga}}?\\
    who=\textsc{nom}  mango    grow-\textsc{pst}-\textsc{q}\\
    \glt ‘Who grew mangoes?’ (\citealt{Miyara2015}: 393, 394)
    \z
    \z 

\noindent Whether the suffix \textit{-i(i)} seen above has to be differentiated from the particle \textit{=ji} found in \isi{focus question}s, remains unclear.

\ea%27
    \label{ex:japa:27}
    \ili{Shuri}\\
    \gll kamadee=ga=\textbf{{du}} maŋgo  tʃuku-ju-\textbf{{ru}}=\textbf{{ji}}?\\
    \textsc{pn}=\textsc{nom}=\textsc{foc}    mango  grow-\textsc{prs}-\textsc{nind}=\textsc{q}\\
    \glt ‘Is it Kamadee who grows mangoes?’ \citep[394]{Miyara2015}
    \z

\noindent According to the description by \cite[181f.]{Arakaki2003}, the \isi{question marker} \textit{-i(i)} attaches directly to the verb stem and replaces the usual past tense ending \textit{-a-\textsc{n}}.

\ea%28
    \label{ex:japa:28}
    \ili{Shuri}\\
    \ea
    \gll wa\textsc{n}=nee  tigami    kac-a-\textsc{n}?\\
    1\textsc{sg}=\textsc{top}  letter    write-\textsc{pst}-\textsc{ind}\\
    \glt ‘I wrote a letter.’
    
    \ex
    \gll ‘jaa=ja  tigami    kac-\textbf{{ii}}?\\
    2\textsc{sg}=\textsc{top}  letter    write-\textsc{q}\\
    \glt ‘Did you write a letter?’ \citep[181]{Arakaki2003}
    \z
    \z

\noindent However, if \textit{-ti(i)} indeed stems from \textit{-ta} + \textit{-i} (or \textit{-ti} + \textit{-i}), perhaps \textit{-i(i)} can be analyzed as \textit{-a} + \textit{-i} (or \textit{-i} + \textit{-i}). The occasional long vowel (\textit{-tii}, \textit{-ii}) in \citegen{Arakaki2003} and \citegen{Uemura2003} data might be a reflex of this. In (\ref{ex:japa:27}) above, the \isi{focus} marker \textit{=du} requires the non-indicative ending \textit{-ru} on the verb. The \ili{Yuwan} verbal ending \textit{-ui}---combined with \textit{=du} as well---possibly contains a cognate of \ili{Shuri} \textit{-ji} (or perhaps \textit{-i(i)}). Content \isi{questions} in \ili{Yuwan} only have the ending \textit{-u}. In \ili{Shuri}, if the \isi{focus} marker \textit{=ga} is present, again identical to the \isi{question marker} in \isi{content question}s, the verb takes the question or dubitative marker \textit{-ra}. This pattern can be found in both content and \isi{focus question}s. See below on \textit{kakari musubi} for further information on this phenomenon.

\ea%29
    \label{ex:japa:29}
    \ili{Shuri}\\
    \ea
    \gll \textbf{{nuu}} tʃi-yu-\textbf{{ga}}?\\
    what  wear-\textsc{prs}-\textsc{q}\\
    \glt ‘What do you wear?’
    
    \ex
    \gll \textbf{{nuu}}=\textbf{{ga}} tʃi-yu-\textbf{{ra}}?\\
    what=\textsc{foc}  wear-\textsc{prs}-\textsc{q}\\
    \glt ‘What do you wear?’ (\citealt{Nagano-Madsen2015}: 204)
    
    \ex
    \gll kamadee=ga=\textbf{{ga}} maŋgoo    tʃuku-ju-\textbf{{ra}} \textbf{{jaa?}}\\
    \textsc{pn}=\textsc{nom}=\textsc{foc}    mango    grow-\textsc{prs}-\textsc{q}  \textsc{q}\\
    \glt ‘Is it Kamadee who will grow mangoes?’ \citep[394]{Miyara2015}
    \z
    \z 

\noindent Apart from all the different forms mentioned, the last example has yet another particle \textit{jaa}, originally glossed as ‘I wonder’, that can also appear as a part of the complex form \textit{ga-jaa}. As noted above, it may be related to the form \textit{ya} in \ili{Old Japanese}. The first element is unlikely to be the content \isi{question marker} \textit{-ga} because \textit{gajaa} can also appear in \isi{polar question}s. The description is insufficient to give a good summary here but \textit{(ga)jaa} appears in \isi{focus} and \isi{content question}s.

\ea%30
    \label{ex:japa:30}
    \ili{Shuri}\\
    \gll kamadee=ja \textbf{{nuu}} tʃuku-ju-\textbf{{gajaa}}?\\
    \textsc{pn}=\textsc{nom}  what  grow-\textsc{prs}-\textsc{q}\\
    \glt ‘Kamadee is going to grow what?’ \citep[395]{Miyara2015}
    \z

As opposed to other \ili{Ryūkyūan} languages the marker \textit{-ka} does not mark neutral \isi{questions} but rather suggestions.

\ea%31
    \label{ex:japa:31}
    \ili{Shuri}\\
    \gll ʔari=ga  ʔi-i-ʃe=e    tʃik-an-\textbf{{ka}}?\\
    3\textsc{sg}=\textsc{nom}  say-\textsc{prs}-\textsc{n}-\textsc{top}?  listen-\textsc{neg}-\textsc{sgs}\\
    \glt ‘Shall we not listen to him?’ \citep[395]{Miyara2015}
    \z

Intonation in \ili{Shuri} is exceptionally well described and too complex to go into every detail here (see \citealt{Nagano-Madsen2015}). Several important points have been summarized as follows:

\begin{quote}
In \ili{Japanese}, both yes-no and wh-\isi{questions} are accompanied by final rising pitch. In Okinawan, neither yes-no \isi{questions} nor wh-\isi{questions} are accompanied by final rising pitch. Like a yes-no question, Okinawan wh-question has \isi{intonation} composed of its lexical accent type unless the verb is immediately preceded by a wh-word. When a verb is immediately preceded by a wh-word, the lexical accent of the verb is usually strongly compressed or rather deleted. [...]

Although the most usual form of forming interrogatives in Okinawan is with a mood suffix, it is not impossible to make an utterance that has (declarative) indicative mood suffix +N, which is produced with a final rising pitch. Furthermore, it is quite common to form an \isi{interrogative} with the sentence-final question particle \textit{na}, which is also produced with a final rising pitch. (\citealt{Nagano-Madsen2015}: 209)
\end{quote}

\textbf{Tsuken} (spoken on \ili{Tsuken} island close to Okinawa) has a polar \isi{question marker} \textit{-mi} that probably is related to the marker \textit{-mi} in \ili{Shuri} or \textit{-mɨ} in \ili{Yuwan}. At first glance, the \isi{question marker} replaces the declarative ending in the following examples in a non-past tense. But in fact, \textit{-mi} must goes back to *\textit{-n-i} as in \ili{Shuri} and \ili{Yuwan}.

\ea%32
    \label{ex:japa:32}
    \ili{Tsuken}\\
    \ea
    \gll ʔjaa=ga  kak-u-\textbf{{n}}.\\
    2\textsc{sg}=\textsc{nom}  write-\textsc{npst-decl}\\
    \glt ‘You will write.’
    
    \ex
    \gll ʔjaa=ga  kak-u-\textbf{{mi}}?\\
    2\textsc{sg}=\textsc{nom}  write-\textsc{npst}-\textsc{q}\\
    \glt ‘Will you write?’ \citep[92]{Matayoshi2010}
    \z
    \z

\noindent But there is also a cognate of the marker \textit{=na} {\textasciitilde} \textit{=nja} in \ili{Yuwan} and other languages that enclitically attaches to the sentence. It does not replace the declarative marker but rather attaches to it.

\ea%33
    \label{ex:japa:33}
    \ili{Tsuken}\\
    \gll kuruma=kara si-sa-n=\textbf{{na}}?\\
    car=\textsc{abl}    come-\textsc{pst}-\textsc{decl}=\textsc{q}\\
    \glt ‘Did you come by car?’ \citep[102]{Matayoshi2010}
    \z

\noindent The distribution between the two markers also remains unclear in \ili{Tsuken} but probably is connected to the verb ending. Content \isi{questions} have a marker \textit{=ga} that, as in \ili{Shuri}, looks suspiciously similar to the \isi{focus} marker \textit{=ga} \citep[102]{Matayoshi2010}. A connection with the nominative/genitive \textit{=ga} seems unlikely.

\ea%34
    \label{ex:japa:34}
    \ili{Tsuken}\\
    \gll \textbf{{taa}}=ga sa=\textbf{{ga}}?\\
    who=\textsc{nom}  do=\textsc{q}\\
    \glt ‘Who does?’ \citep[94]{Matayoshi2010}
    \z

There is no example in which the plain \isi{focus} marker \textit{=ru} is found in a question, which does not mean, however, that this is impossible. The same is true for the \isi{focus} marker \textit{=du} in the language \ili{Tarama}.

\textbf{Tarama} (spoken on \ili{Tarama} and Minna among the Miyako islands) otherwise has a straightforward pattern with \textit{=na} found in polar \isi{questions} and \textit{=ga} in \isi{content question}s. Again, the optional \isi{focus} marker in \isi{content question}s is identical in form with the \isi{question marker}.

\ea%35
    \label{ex:japa:35}
    \ili{Tarama}\\
    \ea
    \gll kure=e    kam=nu  sïma=\textbf{{na}}?\\
    this=\textsc{top}  god=\textsc{gen}  island=\textsc{q}\\
    \glt ‘Is this an island of god?’
    
    \ex
    \gll naa=ju=ba \textbf{{nuu}}=ti=\textbf{{ga}} ïï=\textbf{{ga}}?\\
    name=\textsc{acc}=\textsc{top}  what=\textsc{quot}=\textsc{foc}  say=\textsc{q}\\
    \glt ‘\isi{What is your name?}’ \citep[417]{Aoi2015}
    \z
    \z

There are also examples where there is only one marker with the form \textit{=ga}. Aoi glosses the form as question, but it might well be the \isi{focus} marker.

\ea%36
    \label{ex:japa:36}
    \ili{Tarama}\\
    \gll \textbf{{nuu}}=\textbf{{ga}} sï-tar?\\
    what=?\textsc{q}  do-\textsc{pst}\\
    \glt ‘What happened (with you)?’ \citep[419]{Aoi2015}
    \z

\textbf{Ikema} (spoken on \ili{Ikema}, \ili{Irabu}, and Miyako among the Miyako islands) also has the two question markers \textit{=na} (PQ, FQ) and \textit{=ga} (CQ). But, as opposed to \ili{Yuwan}, for instance, the \isi{focus} marker \textit{=du} appears not only in \isi{focus} but also in \isi{content question}s.

\ea%37
    \label{ex:japa:37}
    \ili{Ikema}\\
    \ea
    \gll husɨ=nu=\textbf{{du}} mii-rai    ui=\textbf{{na}}?\\
    star=\textsc{nom}=\textsc{foc}    look-\textsc{pot}  \textsc{cont.npst}=\textsc{q}\\
    \glt ‘Can you see \textit{the stars}?’
    
    \ex
    \gll \textbf{{nau}}=nu=\textbf{{du}} mii-rai    ui=\textbf{{ga}}?\\
    what=\textsc{nom}=\textsc{foc}  look-\textsc{pot}  \textsc{cont.npst=q}\\
    \glt ‘What can you see?’ (\citealt{HayashiY2010}: 173)
    \z
    \z

The \ili{Hirara} dialect of Miyako has yet another distributional pattern. According to \citet[620]{Koloskova2008}, there is a distinction between three \isi{focus} markers, namely \textit{=ga} in content \isi{questions}, \textit{=nu} in polar \isi{questions}, and \textit{=du} in declaratives. In \ili{Ikema} a special \isi{question marker} for \isi{topic question}s is \textit{=da}, which is always combined with the topic marker. In Masana (\ili{Okinoerabu}) the \isi{question marker} \textit{=do} can also be combined with the topic marker \textit{=wa} (\citealt{vanderLubbeTokunaga2015}: 362).

\ea%38
    \label{ex:japa:38}
    \ili{Ikema}\\
    \gll vva=a=\textbf{{da}}?\\
    2\textsc{sg}=\textsc{top}=\textsc{q}\\
    \glt ‘How about you?’ (\citealt{HayashiY2010}: 173, fn. 16)
    \z

Questions in \textbf{Ōgami} (spoken on \ili{Ōgami} next to Miyako and in one village on Miyako itself) have a pitch that “is high and level and falls sharply on the last syllable” \citep[146]{Pellard2010}. Similar patterns may exist for other \ili{Ryūkyūan} languages but usually were not stated as clearly. There are two optional question markers, a by now familiar particle \textit{=ka} and a suffix \textit{-ɛɛ} that “is limited to past tense forms, the copula and stative verbs” \citep[151]{Pellard2010}. It may be worth noting that it is identical to a suffix that derives agent nouns \citep[118]{Pellard2009} and we might be dealing with a development parallel to \ili{Japanese} \textit{no}.\footnote{For the following examples only \citet{Pellard2009} in \ili{French} was quoted, but they can usually also be found in \citet{Pellard2010} in \ili{English}.}

\ea%39
    \label{ex:japa:39}
    \ili{Ōgami}\\
    \gll \textbf{{nauɾipa}}{=}\textbf{{tu}} kuu-tataɾ-\textbf{{ɛɛ}}?\\
    why=\textsc{foc}  come-\textsc{pst.neg}-\textsc{q}\\
    \glt ‘Why didn’t you come?’ \citep[211]{Pellard2009}
    \z

I was unable to find a good example for the sentence-final particle =\textit{ka} in \cite{Pellard2009,Pellard2010}. The only example is an embedded \isi{content question}.

\ea%40
    \label{ex:japa:40}
    \ili{Ōgami}\\
    \gll [\textbf{{nau}}=iu  as-sipa=\textbf{{tu}} tau-kaɯ=\textbf{{ka}}] ss-ai-n?\\
    what=\textsc{acc}  do-\textsc{circ}=\textsc{foc}    good-\textsc{v}=\textsc{q}  know-\textsc{pot}-\textsc{neg}\\
    \glt ‘I don’t know [what I should do].’ \citep[225]{Pellard2009}
    \z

There is a special marker \textit{mukaɾa} for embedded \isi{polar question}s comparable to \ili{English} \textit{if}/\textit{whether} or \ili{German} \textit{ob} \citep[221]{Pellard2009}. The \isi{focus} marker \textit{=tu} is sometimes found attached to a verb as well and we might be dealing with a development of a \isi{question marker} as in \ili{Irabu}, but \citet[192]{Pellard2009} is not very clear about this.

\ea%41
    \label{ex:japa:41}
    \ili{Ōgami}\\
    \gll vva=a    pssnii=pa    asi=\textbf{{tu}}?\\
    2\textsc{sg}=\textsc{top}  siesta=\textsc{top.obj}    do=?\textsc{q}\\
    \glt ‘Have you taken a siesta?’ \citep[221]{Pellard2009}
    \z

Questions in the language \textbf{Irabu} (spoken on \ili{Irabu} among the Miyako islands) exhibit an interesting \isi{interaction} with \isi{focus} marking. According to \citet[118]{Shimoji2011a},

\begin{quote}
when a \isi{focus} marker is present, a \isi{question marker} is optional, and its form is identical to that of the \isi{focus} clitic in the same clause. I treat these two (i.e., the \isi{focus} marker and \isi{question marker}) as different morphemes owing to the fact that they show different allomorphic patterns, even though the \isi{focus} marker may be the historical source of the \isi{question marker}.
\end{quote}

\noindent If only a \isi{question marker} is present, it attaches sentence-finally to the verb. This is a plain \isi{polar question}.

\ea%42
    \label{ex:japa:42}
    \ili{Irabu}\\
    \gll vva=a      uri=u    až-tar=\textbf{{ru}}?\\
    2\textsc{sg}=\textsc{top}    that=\textsc{acc}  say-\textsc{pst}=\textsc{q}\\
    \glt ‘Did you say that?’ \citep[119]{Shimoji2011a}
    \z

\noindent In the following two examples both \isi{focus} and question markers appear. The first example is a \isi{focus question}, the second a \isi{content question}.

\newpage 
\ea%43
    \label{ex:japa:43}
    \ili{Irabu}\\
    \ea
    \gll vva=ga=\textbf{{ru}} uri=u    až-tar=\textbf{{ru}}?\\
    2\textsc{sg}=\textsc{nom}=\textsc{foc}  that=\textsc{acc}  say-\textsc{pst}=\textsc{q}\\
    \glt ‘Did you say that?’
    
    \ex
    \gll vva=a \textbf{{nau}}=ju=\textbf{{ga}} až-tar=\textbf{{ga}}?\\
    2\textsc{sg}=\textsc{top}  what=\textsc{acc}=\textsc{foc}  say-\textsc{pst}=\textsc{q}\\
    \glt ‘What did you say?’ \citep[118]{Shimoji2011a}
    \z
    \z

The two markers \textit{=ru} and \textit{=ga} are probably cognate with \ili{Yuwan} \textit{=du} and \textit{=ga}, where they express only \isi{focus}. The fact that the question markers are optional if the \isi{focus} marker is present, might be a hint of the historical development. Presumably, the \isi{focus} marker \textit{=ru} was reanalyzed as a \isi{question marker} in \isi{focus} \isi{questions} and subsequently also marked \isi{polar question}s. From there it may have spread back to \isi{focus question}s in its new position attached to verbs. But in the absence of any historical data, this scenario must remain speculative. \citet{Shimoji2011a} has one example of an embedded \isi{alternative question} that shows \isi{double marking} and no \isi{disjunction}.

\ea%44
    \label{ex:japa:44}
    \ili{Irabu}\\
    \gll [ssibara=\textbf{{ru}} a-tar=\textbf{{ru}} maibara=\textbf{{ru}} a-tar=\textbf{{ru}}] mmja  s-sa-n-Ø{=suga}\\
    back=\textsc{foc}  \textsc{cop}-\textsc{pst}=\textsc{q}  front=\textsc{foc}  \textsc{cop}-\textsc{pst}=\textsc{q} \textsc{intj}  know-\textsc{thm}-\textsc{neg}-\textsc{npst}=but\\
    \glt ‘But I’m not sure [whether (the house) was behind or in front].’ (\citealt{Shimoji2011a}: 132f.)
    \z

\noindent The presence of the \isi{focus} marker in \ili{Irabu} excludes realis marking on the verb (see below on \textit{\isi{kakari musubi}}). \citet[396]{Lawrence2012} briefly mentions \isi{question marking} in the \textbf{\ili{Nakachi}} variety of Miyako, which shows a somehow reminiscent pattern. In \isi{content question}s there is only the marker \textit{-ga} on the \isi{interrogative} itself while \isi{polar question}s have the \isi{focus} marker \textit{-ru}, exclusively. In polar \isi{questions} the slightly different \textit{-ro} is found sentence-finally.

\textbf{Hateruma} is the name of one of the Yaeyama islands but as usual is also used to refer to the language spoken there Aso (\citeyear*{Aso2010,Aso2015}). Due, however, to relatively recent population movements, the language is also spoken on another Yaeyama island, namely \ili{Ishigaki}. \ili{Hateruma} has four inferential suffixes \textit{=kaja}, \textit{=sa}, \textit{=dore}, and \textit{=pacï}, the first three of which may correspond to the forms found in \ili{Okinoerabu} above, i.e. \textit{=kaja}, \textit{=sa}, \textit{=do} \citep[208]{Aso2010}. But polar \isi{questions} are also expressed with the enclitic \textit{=naa} while content \isi{questions} remain unmarked. This is a rather untypical pattern for a \ili{Japonic} language but is the norm in most other languages in \isi{Northeast Asia} (see Chapter 6).

\ea%45
    \label{ex:japa:45}
    \ili{Hateruma}\\
    \ea
    \gll da=Ø    sinsin=\textbf{{naa}}?\\
    2\textsc{sg}=(\textsc{core})  teacher=\textsc{q}\\
    \glt ‘Are you a teacher?’
    
    \ex
    \gll kuri=Ø=ja \textbf{{nu}} ja-Ø?\\
    this=(\textsc{core})=\textsc{top}  what  \textsc{cop}-\textsc{npst}\\
    \glt ‘What is this?’ \citep[210]{Aso2010}
    \z
    \z

\noindent Often an enclitic such as \textit{=ba} is found in content \isi{questions}, but this has instead an emphatic or \isi{focus} function.

\textbf{Hatoma} is another Yaeyama variety. While \ili{Hateruma} is a small island south of the main island Iriomote, \ili{Hatoma} is an even smaller island on the north of it \citep[189]{Aso2010}. \ili{Hatoma} exhibits an interesting split between past and non-past content \isi{questions} \citep[396]{Lawrence2012}, the former, like \isi{polar question}s, being marked by (probably rising) \isi{intonation} alone and the latter showing a second split. Non-past \isi{content question}s usually have an attributive form of a verb followed by the marker \textit{-wa}. But if an \isi{interrogative} phrase stands sentence-finally, it takes the marker \textit{-ja}, instead. Apparently, the difference lies in the clause type with either a verbal or a non-verbal predicate. Content \isi{questions} thus have three different markings.

\ea%46
    \label{ex:japa:46}
    \ili{Hatoma}\\
    \ea
    \gll \textbf{{nu}}\textbf{\textsc{n}}\textbf{{ti}} kana\textsc{n=}\textbf{{wa}}?\\
    why  write.\textsc{neg=q}\\
    \glt ‘Why won’t wou write?’
    
    \ex
    \gll kjuu=ja \textbf{{nuu}}-nu  pii=\textbf{{ja}}?\\
    today=\textsc{top}  what-\textsc{gen}  day=\textsc{q}\\
    \glt ‘What day is it today?’
    
    \ex
    \gll \textbf{{nuusi}} nat-taa?\\
    how  become-\textsc{pst}\\
    \glt ‘What happened?’
    
    \ex
    \gll waa  aca-\textsc{n} k-ii    ffir-u\textsc{n}?\\
    2\textsc{sg}  tomorrow-also    come-\textsc{inf}  give.me-\textsc{aff}\\
    \glt ‘Will you come tomorrow, too?’ \citep[396]{Lawrence2012}\z\z

Descriptions of \ili{Ryūkyūan} languages almost never give information on other question types such as \isi{alternative question}s, \citet[397]{Lawrence2012} being an exception. \ili{Hatoma} alternative \isi{questions} either display simple \isi{juxtaposition} or \isi{double marking} with the form \textit{=kajaa}.

\ea%47
    \label{ex:japa:47}
    \ili{Hatoma}\\
    \ea
    \gll kuree    turu  kaburee?\\
    this.\textsc{top}  bird  bat\\
    \glt ‘Is this a bird or a bat?’
    
    \ex
    \gll kuree    turu=\textbf{{kajaa}} kaburee=\textbf{{kajaa}}?\\
    this.\textsc{top}  bird=\textsc{q}    bat=\textsc{q}\\
    \glt ‘(I wonder) is this a bird or a bat?’ \citep[397]{Lawrence2012}
    \z
    \z

\noindent Cognates of the marker \textit{=kajaa} were already encountered in \ili{Okinoerabu} and \ili{Shuri}. In \ili{Hatoma} it can also be found in (less direct) \isi{content question}s such as \textbf{\textit{taa}}\textit{=}\textbf{\textit{kajaa}}? ‘(I wonder) who (is it)?’ \citep[396]{Lawrence2012}.

The last Yaeyama variety to be considered here is called \textbf{Miyara} or Miyaran, spoken on \ili{Ishigaki} island (\citealt{Izuyama2003}; \citealt{Davis2015}). In \ili{Miyara} both polar and \isi{focus question}s may be expressed with the help of rising \isi{intonation} alone. In \isi{focus} \isi{questions} an additional \isi{focus} marker \textit{=du} appears and triggers the loss of the indicative ending on the verb.

\ea%48
    \label{ex:japa:48}
    \ili{Miyara}\\
    \ea
    \gll naoja=ja sinbun jum-u-\textbf{{n}}?\\
    \textsc{pn}=\textsc{top}  newspaper    read-\textsc{prs}-\textsc{ind}\\
    \glt ‘Will/does Naoya read the newspaper?’
    
    \ex
    \gll naoja=ja sinbun=\textbf{{du}} jum-{u?}\\
    \textsc{pn}=\textsc{top}  newspaper=\textsc{foc}  read-\textsc{prs}\\
    \glt ‘Will/does Naoya read the \textit{newspaper}?’
    
    \ex
    \gll naoja=ja \textbf{{noo}}=\textbf{{du}} jum-u?\\
    \textsc{pn}=\textsc{top}  what=\textsc{foc}    read-\textsc{prs}\\
    \glt ‘What does/will Naoya read?’ (\citealt{Davis2015}: 260)
    \z
    \z 

Content \isi{questions} have the same (optional) \isi{focus} marker but exhibit falling \isi{intonation}. Notice the absence of the final \textit{-}\textsc{n} from \isi{content question}s even if the \isi{focus} marker \textit{=du} is not present.

\ea%49
    \label{ex:japa:49}
    \ili{Miyara}\\
    \gll \textbf{{zïma}}=ge har-u?\\
    where=\textsc{dir}  go-\textsc{prs}\\
    \glt ‘Where are you going?’ (\citealt{Davis2015}: 261)
    \z

\ili{Miyara} also has the dubitative particle \textit{kajaa} as well as a particle \textit{i} that “indicates a \isi{request} for agreement” (\citealt{Izuyama2003}: 28f.). Details remain unclear, but the latter might be comparable with \textit{=ji} in \ili{Shuri}.

Yonaguni is the westernmost island of the Yaeyama islands, only about 100 km off the coast of \isi{Taiwan}. Here only two Yonaguni dialects will be addressed, \ili{Dunan} and \ili{Sonai}. In \textit{Dunan} polar and \isi{focus question}s are marked with a sentence-final clitic \textit{=na}. Content \isi{questions} have their own sentence-final marker \textit{=nga}. There is an additional \isi{focus} marker in \isi{focus} (\textit{=du}) and content \isi{questions} (\textit{=ba}). A non verbal \isi{content question} has the \isi{question marker} \textit{=ja} instead of \textit{=nga}.

\ea%50
    \label{ex:japa:50}
    \ili{Dunan}\\
    \ea
    \gll khuruma  mut-i    bu=\textbf{{na}}?\\
    car    hold-\textsc{med}  \textsc{ipfv}=\textsc{q}\\
    \glt ‘Do (you) have a car?’
    
    \ex
    \gll suuti  khat-i=\textbf{{du}} bu-ru=\textbf{{na}}?\\
    book  write-\textsc{med}=\textsc{foc}  \textsc{ipfv}-\textsc{ptcp}=\textsc{q}\\
    \glt ‘(Are you) writing a book?’
    
    \ex
    \gll nda=ja    tharu=nki \textbf{{nu}}(=\textbf{{ba}})  thura=\textbf{{nga}}?\\
    2\textsc{sg}=\textsc{top}  \textsc{pn}=\textsc{dir}    what(=\textsc{foc})  give=\textsc{q}\\
    \glt ‘What will you give to Taro?’
    
    \ex
    \gll \textbf{{nma}}=ŋa(=\textbf{{ba}})    dunan-ccima=\textbf{{ja}}?\\
    where=\textsc{nom}(=\textsc{foc})  \textsc{pn}-island=\textsc{q}\\
    \glt ‘Where exactly is Yonaguni island?’ (\citealt{YamadaPellardShimoji2015}: 468, 466, 469)\z\z

\noindent Whether \textit{=ja} might be connected to \ili{Okinoerabu} and \ili{Ura} \textit{=joo} remains unclear to me.

In \textbf{Sonai} the situation is very similar to \ili{Dunan} (\citealt{Izuyama2012}: 442ff.). The polar \isi{question marker} has the form \textit{=na(i)} and \isi{content question}s have two different markers with the same distribution, \textit{=ga} in verbal and \textit{=ja(a)} in non-verbal clauses. In addition, there is a dubitative form \textit{=kaja(a)} roughly meaning ‘I wonder’ as in \ili{Hatoma} and other varieties. The two elements \textit{-du} and \textit{-ba} obviously correspond to \ili{Dunan} \textit{=du} and \textit{=ba}. \citet[443]{Izuyama2012} calls them \isi{focus} and selective particles but writes them attached to the preceding word with the help of a hyphen. The question markers on the other hand were written detached from the preceding word. I reanalyze all of them as enclitics.

\ea%51
    \label{ex:japa:51}
    \ili{Sonai}\\
    \ea
    \gll num-i=\textbf{{na}}?\\
    drink-\textsc{ind}=\textsc{q}\\
    \glt ‘Have you drunk it?’
    
    \ex
    \gll \textbf{\textsc{n}}\textbf{{ma}}-\textsc{n}ki  h-ju\textsc{n}=\textbf{{ga}}?\\
    where-\textsc{all}  go-\textsc{pfv}=\textsc{q}\\
    \glt ‘Where has (she) gone?’
    
    \ex
    \gll u=ja \textbf{{nu}}=\textbf{{ja}}?\\
    this=\textsc{top}  what=\textsc{q}\\
    \glt ‘What is this?’
    
    \ex
    \gll \textbf{{nu}}=\textbf{{ba}} \textsc{n}da    munu=\textbf{{ja}}?\\
    what-\textsc{sel}  2\textsc{sg}.\textsc{gen}  thing=\textsc{q}\\
    \glt ‘Which is yours?’
    
    \ex
    \gll \textsc{n}da \textbf{{ici}}=\textbf{{ba}} s-u\textsc{n}=\textbf{{ga}}?\\
    2\textsc{sg}  when=\textsc{sel}  come-\textsc{pfv}=\textsc{q}\\
    \glt ‘When did you come?’
    
    \ex
    \gll \textbf{{ta}}=\textbf{{ba}}=\textbf{{du}} tata-\textsc{n}=\textbf{{ga}}?\\
    who=\textsc{sel}=\textsc{foc}    make.stand-\textsc{conc}=\textsc{q}\\
    \glt ‘Whom do you make stand?’
    
    \ex
    \gll \textbf{{nu}}=\textbf{{ba}}=\textbf{{du}} ut-iru=\textbf{{kaja}}?\\
    what=\textsc{sel}=\textsc{foc}  fall-\textsc{conc}=\textsc{q}\\
    \glt ‘I wonder which one will fall down?’ (\citealt{Izuyama2012}: 439, 419, 425, 444, 421)\z\z

\noindent As in \ili{Ikema}, the \isi{focus} marker \textit{=du} is also found in \isi{content question}s and is not restricted to \isi{focus question}s as in \ili{Yuwan}.

\tabref{tab:japa:3} summarizes the marking of \isi{questions} in \ili{Japonic} languages. Given the lack of information on alternative \isi{questions}, these have been excluded from the summary. In general, it appears that alternative \isi{questions} show the double marked type and lack a \isi{disjunction}. Forms with an additional semantic component such as those translated with ‘I wonder’ are excluded from the list as well.

Most languages have different markers for polar and content \isi{questions}. \ili{Ōgami} and \ili{Japanese} are exceptional in allowing the same marker. Apart from \ili{Hateruma} and \ili{Yilan Creole} all languages have \isi{content question} markers. Little information is available on \isi{focus question}s. In some languages such as \ili{Dunan}, \ili{Ikema}, and \ili{Japanese} they have the same marking as \isi{polar question}s, plus an additional \isi{focus} marker. In \ili{Yuwan} and \ili{Shuri} there are special question markers, but \ili{Shuri} also allows the question and \isi{focus} markers from content \isi{questions} to enter \isi{focus} \isi{questions}. The only languages without at least an optional polar \isi{question marker} are \ili{Hatoma} and \ili{Miyara}.

\begin{table}[p]
\caption{The marking of polar, focus, and content questions in \ilit{Japonic}; whether or not a focus marker is optional is not indicated}
\label{tab:japa:3}
\small
\begin{tabularx}{\textwidth}{lQQl}
\lsptoprule
& \textbf{PQ} & \textbf{CQ} & \textbf{FQ}\\
\midrule
East OJ & ya, ka & ka & \textsc{attr} + ya, ?ka\\
West OJ & ya, ka & ka & \textsc{attr} + ya, ?ka\\
\ilit{Hachij\=o} & ?kai & ?- & ?\\
\ilit{Japanese} & ka\#, no\# & ka\# & ka\# + wa \textsc{top}\\
\ilit{Yilan Creole} & ga\#, no\# & - & ?\\
\ilit{Ura} & =na\# {\textasciitilde} =nja\# & =joo\# & ?\\
\ilit{Yuwan} & V-mɨ, =na\#

(\tabref{tab:japa:1}) & V-u + =ga \textsc{foc} & V-ui + =du \textsc{foc}\\
\ilit{Okinoerabu} & =na\#, \textsc{ind} -ŋ, -Ø + =nja\#/-jee\# \textsc{+ pst} -ti instead of -ta & =joo\# & ?\\
\ilit{Shuri} & V-mi, V-i, V-ti, =naa\# (\tabref{tab:japa:2}) & V-ga, V-ra + =ga \textsc{foc} & =ji\# + =du \textsc{foc},

V-ra + =ga \textsc{foc}\\
\ilit{Tsuken} & V-mi, =na\# & =ga\# & ?\\
\ilit{Tarama} & =na\# & =ga\# + =ga \textsc{foc} & ?\\
\ilit{Ikema} & =na\# & =ga\# + =du \textsc{foc} & =na\# + =du \textsc{foc}\\
\ilit{Ōgami} & =ka\#, ?=tu, \textsc{pst}, \textsc{cop}, \textsc{stat}.\textsc{v} -ɛɛ & =ka\#, \textsc{pst}, \textsc{cop}, \textsc{stat}.\textsc{v} -ɛɛ

+ =du \textsc{foc} & ?\\
\ilit{Irabu} & =ru\# + =ru \textsc{foc} & =ga\# + =ga \textsc{foc} & ?\\
\ilit{Hateruma} & =naa\# & - & ?\\
\ilit{Hatoma} & - & \textsc{pst -, attr} + =wa, =ja\# (non-verbal) & ?\\
\ilit{Miyara} & - & lack of \textsc{ind} -\textsc{n} + =du \textsc{foc} & lack of -\textsc{n} + =du \textsc{foc}\\
\ilit{Dunan} & =na\# & =nga\#, =ja\# (non-verbal)

+ =ba \textsc{foc} & =na\# + =du \textsc{foc}\\
\ilit{Sonai} & =na(i)\# & =ga\#, =ja(a)\# (non-verbal)

+ =ba \textsc{sel,} + =du \textsc{foc} & ?\\
\lspbottomrule
\end{tabularx}
\end{table}

A typologically rare phenomenon of \ili{Japonic} languages that is relevant for \isi{interrogative} constructions is a kind of \textit{\isi{focus} concord}, usually called \textit{kakari musubi} (KM) ‘governing (and) concordance’ (cf. \citealt{Shimoji2010}: 11; \citealt{Shinzato2013}). We have already encountered a special type in \ili{Yuwan} above that is limited to \isi{interrogative} constructions. Specifically, the \isi{focus} markers \textit{=du} in \isi{focus} \isi{questions} and \textit{=ga} in \isi{content question}s necessarily are followed by the verb endings \textit{-ui} and \textit{-u}, respectively. Usually, however, the phenomenon is not restricted to \isi{questions} but can also be found in declarative sentences. More generally, \textit{\isi{kakari musubi}} can be characterized as “a syntactic agreement construction in which specific particles called \textit{kakari joshi} (\textit{kakari} particles, KP henceforth) correlate with particular predicate conjugational endings other than regular finite forms to end a sentence.” \citep[299]{Shinzato2015}

\begin{table}[p]
\caption{KPs in \ili{Proto-Japonic}, \ili{Old Japanese}, and \ili{Old Ryūkyūan} according to \cite[306ff.]{Shinzato2015}}
\label{tab:japa:4}
\small
\begin{tabularx}{\textwidth}{XXXl}
\lsptoprule
& \textbf{Proto-Japonic} & \textbf{Old Japanese} & \textbf{Old Okinawan}\\
\midrule
Group I & *kö(swo) & koso & su\\
& *työ & so & do\\
& *ka & ka & ga\\
Group II & ? & ya & ?yi\\
& ? & namu & -\\
\lspbottomrule
\end{tabularx}
\end{table}

KM is attested in some \ili{Ryūkyūan} languages as well as \ili{Old Japanese}, but not in modern \ili{Japanese}. Altogether, \ili{Japonic} has five different \textit{kakari} particles, of which we have already encountered \textit{ka} and \textit{ya}. In Old Okinawan only three of them have clear cognates (\tabref{tab:japa:4}). The first three of the markers may go back to \isi{demonstratives} (cf. \ili{pre-modern Japanese} \isi{demonstratives} \textit{ko-}, \textit{so-}, \textit{ka-}, see \sectref{sec:5.6.3}). According to \citet{Shinzato2015}, \textit{\isi{kakari musubi}} is similar to an it-cleft construction, i.e. a way of marking \isi{focus}. This may be the reason why the \textit{kakari} particles are also found in \isi{focus} as well as \isi{content question}s. The verbal ending triggered by the KP is usually an adnominal form. Modern \ili{Ryūkyūan} languages nevertheless show several deviations from this rule. In \ili{Miyara} and \ili{Irabu}, for instance, there is no adnominal form of the verb (\citealt{Davis2015}: 257). In \ili{Miyara} the presence of the \isi{focus} marker leads to the loss of the indicative ending (see also example \ref{ex:japa:48} above).

\newpage 
\ea%52
    \label{ex:japa:52}
    \ili{Miyara}\\
    \ea
    \gll naoja=ja sinbun jum-u-\textbf{{n}}?\\
    \textsc{pn}=\textsc{top}  newspaper    read-\textsc{prs}-\textsc{ind}\\
    \glt ‘Naoya will read/reads the newspaper.’
    
    \ex
    \gll naoja=ja sinbun=\textbf{{du}} jum-{u?}\\
    \textsc{pn}=\textsc{top}  newspaper=\textsc{foc}  read-\textsc{prs}\\
    \glt ‘Naoya will read/reads the \textit{newspaper}.’ (\citealt{Davis2015}: 260)
    \z
    \z

The phenomenon found in \ili{Irabu} has been called \textit{quasi-kakari musubi} \citep{Shimoji2011b}. Instead of the obligatory presence of a certain verb ending (usually adnominal), \ili{Irabu} not only excludes the presence of realis marking but allows other types of endings (including irrealis, mood-neutral etc.).

\ea%53
    \label{ex:japa:53}
    \ili{Irabu}\\
    \ea
    \gll ba=a    kuruma=u=\textbf{{du}} vv-\textbf{{tar}}.\\
    1\textsc{sg}=\textsc{top}  car-\textsc{acc}=\textsc{foc}    sell-\textsc{pst}\\
    \glt ‘I sold a \textit{car}.’
    
    \ex
    \gll *ba=a    kuruma=u=\textbf{{du}} vv-\textbf{{tam}}.\\
    1\textsc{sg}=\textsc{top}  car-\textsc{acc}=\textsc{foc}    sell-\textsc{pst}\\
    \glt ‘I sold a \textit{car}.’ \citep[120]{Shimoji2011b}
    \z
    \z

\citet[121]{Shimoji2011b} calls these two different types positive and negative concordance. For a phenomenon similar to \textit{\isi{kakari musubi}} in \isi{NEA} see \sectref{sec:5.14.2} on \isi{question marking} in \ili{Yukaghiric}.

\subsection{Interrogatives in Japonic}\label{sec:5.6.3}

Interrogatives in \ili{Japonic} languages are not very well described. Most descriptions available to me simply mention one or two forms but do not dwell on their \isi{analysis}, etymology, or usage. The major exception in the Western literature is \citet[297--336]{Vovin2005}. Some interrogatives such as ‘who’, ‘what’, and ‘when’ are probably of \ili{Proto-Japonic} origin (\tabref{tab:japa:5}).

These forms represent three major groups of \isi{interrogative} present in \ili{Japonic} languages that start with *\textit{t-}, *\textit{n-}, and *\textit{e-}, respectively. \ili{Japonic} has neither KIN- nor K-in\-ter\-rogatives. The \ili{Proto-\ili{Japonic}} \isi{interrogative} \textbf{*}\textit{ta-} ‘who’ is basically present in all \ili{Japonic} languages. Written \ili{pre-modern Japanese} still had \textit{ta-re} instead of modern day \textit{da-re} \citep[63]{Aston1904}. \ili{Yilan Creole} has an initial liquid instead (\textit{la-re}). In some languages the base stem is used as \isi{interrogative} while other languages exhibit different suffixes. The suffix \textit{-re} in \ili{Japanese} and its equivalents in some of the other languages is probably related to the suffix found in \textit{do-re} ‘which’ as well as the \isi{demonstratives} (see below). Its meaning is somewhat unclear but it may be treated as a stem extension.

\begin{table}[t]
\caption{{\ili{Japonic}} interrogatives for ‘who’, ‘what’, and ‘when’; many Southern {\ili{Ryūkyūan}} forms stem from \cite[298-299]{Bentley2008a}; EOJ = Eastern \ili{Old Japanese}, WOJ = Western Old Japanese, PMJ = \ili{pre-modern Japanese}, OR = \ili{Old Ryūkyūan} (\citealt{Vovin2005}; \citealt{Kupchik2011}); the \textit{N} stands for prenasalization; transcription of \ili{Shodon} glottal stop modified}
\label{tab:japa:5}
\small
\begin{tabularx}{\textwidth}{lQll}
\lsptoprule
& \textbf{who} & \textbf{what} & \textbf{when}\\
\midrule
EOJ & ta-re \jp{多例}, \jp{多礼}, ...

ta-(N)ka \jp{多賀}, \jp{他加}, ... & ?aN- \jp{安}, ... < *\textbf{ani}

(nani \jp{奈尓},...)\footnotemark & itu \jp{,}\\
\ilit{Hachij\=o}\footnotemark & \textbf{d}a-re \jp{だれ},

\textbf{d}a-ga \jp{たが} & \textbf{ani} \jp{あに} & ?\\
WOJ & ta-(re) \jp{多禮}, \jp{多礼}, ...

ta-Nka \jp{多賀}, \jp{他賀}, ... & nani \jp{奈爾}, \jp{那爾}, ... & itu \jp{伊都}\\
PMJ & ta-re, (\textbf{d}a-re), ta-ga & nani & itsu\\
\ilit{Japanese} & \textbf{d}a-re \jp{誰}, \jp{だれ} & nani ({\textasciitilde} naɴ) \jp{何}, \jp{なに} & itsu \jp{何時}, \jp{いつ}\\
\ilit{Yilan Creole} & \textbf{l}a-re & nani & ?\\
\ilit{Ura} & ta-ru & na\textbf{n} & icu\\
\ilit{Yuwan} & ta-rɨ {\textasciitilde} ta-ru & nuu & ɨcɨ\\
\ilit{Okinoerabu} & ta-ru {\textasciitilde} ta-ŋ & nuu & ʔitʃi\\
\ilit{Shodon} & tha-r(u-), thaa-ga & nu(u) {\textasciitilde} nu(u) & ʔyit(i)\\
OR & ta(a) \jp{たあ}, ta-ru \jp{たる},

ta-ga \jp{たが} & nau \jp{なお} & itu ?[itsï] \jp{いつ}\\
\ilit{Shuri} & taa & nuu & ʔitʃi\\
\ilit{Tsuken} & taa & ? & ?\\
\ilit{Hirara} & ta-ru {\textasciitilde} too & noo & itsï\\
\ilit{Tarama} & taa-ga, tau & nuu & itsï\\
\ilit{Ikema} & ta-ru & nau & ?\\
\ilit{Nagahama} & ta-ru & nau & itsï\\
\ilit{Ōgami} & ta-ɾu {\textasciitilde} tau & nau & iks\\
\ilit{Irabu} & ta-ru & nau & ic\\
\ilit{Ishigaki} & ta-ru {\textasciitilde} taa & noo & itsï\\
\ilit{Kohama} & ta-ru & nuu & itsu\\
\ilit{Kuroshima} & ta-rï {\textasciitilde} taa & nuu & itʃiya\\
\ilit{Hateruma} & ta-ru {\textasciitilde} ta(a) & nu(u) & icï\\
\ilit{Hatoma} & taa & nuu & itsi\\
\ilit{Miyara} & ta-ru & noo & itsï\\
\ilit{Dunan} & thá & nû & ?\\
\ilit{Sonai} & ta(a), ta-ŋa, takka & nu(u) & ici\\
\lspbottomrule
\end{tabularx}
\end{table}

\addtocounter{footnote}{-2}
\stepcounter{footnote}\footnotetext{ This form is rare and probably originates in the Western dialect.}
\stepcounter{footnote}\footnotetext{ Hachij\=o data taken from http://www008.upp.so-net.ne.jp/ohwaki/hougen.htm. (Accessed 2016-01-19.)}

The suffixes \textit{-Nka} in \ili{Old Japanese} and \textit{-ga} in Old \ili{Ryūkyūan} are said to have a possessive function (\citealt{Vovin2005}: 298ff.). \ili{Hachij\=o} \textit{-ga}, \ili{Shodon} \textit{-ga}, \ili{Tarama} \textit{-ga}, \ili{Okinoerabu} \textit{-ŋ}, and \ili{Sonai} \textit{-ŋa} are likely of the same origin. It may be worth noting, however, that in these languages the suffix combines the function of both the genitive as well as the nominative (\citealt{Izuyama2012}: 417; \citealt{vanderLubbeTokunaga2015}: 352; \citealt{Aoi2015}: 415).

\newpage 
The \ili{Proto-Ryūkyūan} \isi{interrogative} meaning ‘what’ probably had the form \textbf{*}\textit{nau}. Forms such as \ili{Miyara} \textit{noo} have gone through regular sound changes, in this case *\textit{au} > \textit{oo} (cf. \citealt{Davis2015}: 258). But the connection with \ili{Japanese} \textit{nani} or \ili{Hachij\=o} \textit{ani} is not completely straightforward. At least one \ili{Ryūkyūan}  language has a form closer to \ili{Japanese} (\ili{Ura} \textit{nan}), but this may be due to \isi{contact} with \ili{Japanese}. To my knowledge, the best, albeit problematic, explanation has been put forward by \cite[305–313]{Vovin2005}. He reconstructs a \ili{Proto-\ili{Japonic}} form *\textit{nanu}, in which the *\textit{n-} is said to be a prefix with unclear meaning. Eastern \ili{Old Japanese}, Vovin claims, has a form without the prefix, as can be seen from a comparison of WOJ \textit{naNtö, naNsö} and EOJ \textit{aNtö, aNse}. According to Vovin, the final \textit{-i} might derive from a suffix \textit{-(C)i}, the meaning of which was not given. He assumes an irregular sound change in \ili{Ryūkyūan} , namely the loss of the intervocalic \textit{n}, resulting in *\textit{nau}. \citet[313]{Vovin2005} also notices a \isi{similarity} of his \isi{reconstruction} with \ili{Austronesian} *\textit{n-anu} with an unclear prefix. \citet[310]{Blust2013} reconstructs the \ili{Proto-Austronesian} form as *\textit{anu} ‘what’, and we will encounter the \ili{Atayal} form \textit{nanu\textsuperscript{ʔ}} ‘what’ at the end of this section. The \isi{similarity} is indeed striking, but depends on whether \ili{Proto-\ili{Japonic}} *\textit{nanu} is a correct \isi{reconstruction} or not.

However, Vovin’s explanation does not seem very plausible. For example, instead of postulating an otherwise unknown prefix \textit{n-}, it is much more likely that Eastern \ili{Old Japanese} simply lost the initial nasal that is present in \ili{Ryūkyūan}  as well. Let us first consider the \ili{Japanese} forms \textit{naze} and \textit{nado} meaning ‘why’. According to \citet[333]{Vovin2005} they have the form \textit{naNsö} and \textit{naNtö} in (Western) \ili{Old Japanese} and are combinations of \textit{nani} with the two defective verbs \textit{tö} ‘to say’ and \textit{sö} {\textasciitilde} \textit{se} ‘to do’ (or a particle \textit{sö}). Given the strong connection of the categories of \textsc{reason} and \textsc{action}, this seems plausible. \citet[336]{Vovin2005} claims that \ili{Ryūkyūan}  has no cognates of the two forms, and indeed, of the references used in \sectref{sec:5.6.2} only \citet[106]{Shimoji2011a} mentions the two forms \textit{nausi} ‘how’ and \textit{nautti} ‘why’ for \ili{Irabu}. \cite[268, 298f.]{Bentley2008a} gives some additional forms (e.g., \ili{Hirara} \textit{nooʃii} ‘how’, \textit{nooti} ‘why’) and reconstructs Southern \ili{Ryūkyūan}  (Sakishima) *\textit{naWo-se} ‘how’ and *\textit{naWo-nVte-} ‘why’. The \textit{W} stands for a somewhat unclear semi-vowel *\textit{j} or *\textit{w} (\citealt{Bentley2008a}: 218f.). Apart from certain innovations and additional suffixes found in some languages, there certainly are cognates of the \ili{Japanese} interrogatives. \ili{Ryūkyūan}  forms such as \ili{Irabu} \textit{nau-si} suggest a derivation that is directly based on \textit{nau} ‘what’ and the same may be true for the \ili{Old Japanese} equivalents, i.e. they might be derived from *\textit{nanu} instead of \textit{nani}. The nasal found in some forms such as Yonaguni \textit{nundi}, according to Bentley, was part of the suffix instead of the stem (also cf. \ili{Shuri} \textit{nuuntʃ}\textit{i} ‘why’, \citealt{Miyara2015}: 387).

The \isi{interrogative} ‘when’ can be reconstructed as \textbf{*}\textit{etu} (\citealt{Vovin2005}: 330, see \citealt{Pellard2008}: 143, passim for details on vowels). The \isi{interrogative} can be found in all \ili{Japonic} languages for which sufficient material is available. The \isi{analysis} of PJ *\textit{etu} is an open question but it can be classified with several other interrogatives with the \isi{resonance} *\textit{e{\textasciitilde}} > \textit{i{\textasciitilde}} (\tabref{tab:japa:6}). WOJ in addition has the forms \textit{iNtu-ti} ‘where’ as well as \textit{iku-Nta} ‘how many/much’ and EOJ \textit{iNtu-si} ‘which’.

\begin{table}
\caption{Interrogatives in \ili{Proto-Japonic}, Western \ili{Old Japanese} (WOJ), Eastern Old Japanese (EOJ), \ili{pre-modern Japanese} (PMJ), \ili{Japanese} (J), and \ili{Proto-Ryūkyūan} (PR) starting with \textit{i{\textasciitilde}} < *\textit{e{\textasciitilde}} according to \cite[63ff.]{Aston1904}, \cite[297ff.]{Vovin2005}, and \cite[589ff.]{Kupchik2011}; partly modified transcription; the \textit{N} stands for prenasalization}
\label{tab:japa:6}

\begin{tabularx}{\textwidth}{XXXXXXX}
\lsptoprule
& \textbf{PJ} & \textbf{WOJ} & \textbf{EOJ} & \textbf{PMJ} & \textbf{J} & \textbf{PR}\\
\midrule
which & *entu-re & iNtu-re & itu-re & idzu-re & do-re & *edu-re\\
when & *etu & itu & itu & itsu & itsu & *etu\\
how & *eka & ika & ika & ika & ika & *eka\\
how many & *eku & iku- & ? & iku- & iku & *eku\\
\lspbottomrule
\end{tabularx}
\end{table}

\largerpage
Several scholars have compared the interrogatives in *\textit{e{\textasciitilde}} with \ili{Koreanic} *\textit{e-} (e.g., \citealt{FrellesvigWhitman2004}: 289; \citealt{Vovin2005}: 319, 322; \sectref{sec:5.7.3}). However, a comparison based on one vowel must be treated with caution.

The \ili{Old Japanese} \isi{interrogative} \textit{ika} ‘how’ is not very common, is usually limited to Western \ili{Old Japanese} and is followed by one of the defective copulas \textit{n-} and \textit{tö-} or the still more productive \textit{nar-}, which is a contraction of \textit{n-i ar-} ‘\textsc{cop}-\textsc{inf} exist-’ (\citealt{Vovin2005}: 313--319; \citealt{Kupchik2011}: 593f.). Among the cognates in \ili{Ryūkyūan}  languages we find \ili{Old Ryūkyūan} forms such as \textit{ika} \jp{いか}, \textit{ikya} \jp{いきや} {\textasciitilde} \textit{ka} \jp{か}, \textit{kya} \jp{きや} etc. and \ili{Shuri} \textit{‘icaa} {\textasciitilde} \textit{caa} (see \citealt{Vovin2005}: 318 for a more exhaustive list). In both cases there are forms with and without the initial vowel that is responsible for the palatalization of the following velar consonant. Vovin’s (2005: 317) problematic and somehow unclear conclusion is that the \isi{interrogative} has to be analyzed as *\textit{e-ka}. But this is no explanation for why the initial element---which must be considered the \isi{interrogative} as such---can simply be omitted. It is more reasonable to assume that \textit{ika} was considered an inseparable \isi{interrogative} by the speakers, which is why the, maybe irregular, loss of the vowel did not affect its \isi{interrogative} status as such. The same criticism also applies to his explanation of the other interrogatives that will be addressed in the following. \ili{Japanese} \textit{ikaga} ‘how’ derives from the \ili{Old Japanese} fixed expression \textit{ika n-i ka} ‘how \textsc{cop}-\textsc{inf} \textsc{q}’ (\citealt{Vovin2005}: 314, fn. 120). \citet[319]{Vovin2005} compares the hypothetical element \textit{-ka} with \ili{Korean} but leaves open any further detail.

Apart from the locative endings, the \ili{Old Japanese} \isi{interrogative} \textit{iNtu-ku} ‘where’ has a direct cognate in \ili{Old Ryūkyūan} \textit{idu-ma} > \textit{zuma} \jp{すま} as well as in modern \ili{Ryūkyūan}  languages such as \ili{Miyara} \textit{zïma} (\citealt{Vovin2005}: 321; \citealt{Davis2015}: 261). The second part \textit{-ma} is claimed to be a noun meaning ‘place’, but in this case the \isi{interrogative} \textit{idu-} would be expected to have the meaning ‘what’ or ‘which’ rather than ‘where’. In fact, from a typological perspective PJ \textit{*entu} (together with the extended form \textit{*entu-re}) likely was a selective \isi{interrogative} ‘which’ at first and only later developed into a locative \isi{interrogative} ‘where’, as it was combined with a locative marker or a noun meaning ‘place’ (\textit{-ma}). Several languages of the region have parallel developments and this scenario is corroborated by data from some \ili{Ryūkyūan}  languages such as \ili{Irabu} \textit{nzi} ‘which’ versus \textit{nza} ‘where’ that may go back to the plain and derived forms, respectively. \ili{Ōgami} still has the non-palatalized forms \textit{nti} ({\textasciitilde} \textit{iti}) versus \textit{nta} ({\textasciitilde} \textit{ita}) that make this development seem more plausible. However, \ili{Ryūkyūan}  languages show much stronger variation in forms meaning ‘where’ than in those interrogatives previously encountered. Some of \citegen[321, especially fn. 123]{Vovin2005} otherwise good explanations for those deviations are somewhat speculative and cannot be taken at face value. Among the dialects mentioned in \sectref{sec:5.6.2}, for example, we find the forms listed in \tabref{tab:japa:7}. A possible explanation for \ili{Shuri} \textit{maa} is the loss of the first part of \textit{idu-ma}. All other forms can, following Vovin, be derived directly from \textit{idu-ma} or rather its predecessor PR *\textit{eNtuma} (\citealt{Vovin2005}: 321, fn 123). But this is certainly not true for \ili{Yuwan} \textit{daa}, in which the first part was deleted as well (cf. \ili{Okinoerabu} \textbf{\textit{ʔu}}\textit{da}).

\begin{table}
\caption{Interrogative forms meaning ‘where’ in \ili{Japonic}}
\label{tab:japa:7}

\begin{tabularx}{\textwidth}{Xl}
\lsptoprule

\textbf{Language} & \textbf{Form}\\
\midrule
Eastern \ilit{Old Japanese} & iNtu- \jp{伊豆}\\
Western \ilit{Old Japanese} & iNtu-ku \jp{伊豆久}\\
(written) \ilit{pre-modern Japanese} & idzu-ko\\
\ilit{Hachij\=o} & do-ko\\
\ilit{Japanese} & do-ko \jp{何処}\\
\ilit{Yilan Creole} & do-ko\\
\ilit{Ura} & ʔuda\\
\ilit{Yuwan} & daa\\
\ilit{Okinoerabu} & ʔuda\\
OR & idu-ma > zuma \jp{すま}\\
\ilit{Shuri} & maa\\
\ilit{Tsuken} & maa\\
\ilit{Ōgami} & nta {\textasciitilde} ita\\
\ilit{Irabu} & nza\\
\ilit{Hateruma} & za\\
\ilit{Miyara} & zïma\\
\ilit{Dunan} & nmâ\\
\ilit{Sonai} & ɴma\\
\lspbottomrule
\end{tabularx}
\end{table}

Vovin mentions a Northern \ili{Ryūkyūan}  form \textit{raa}, not encountered thus far, that is probably a variant of \textit{daa}. The distinction between \textsc{location}, \textsc{direction}, and \textsc{source} has not been given for the majority of languages. Most likely, the difference in most languages is indicated with \isi{case} markers as in (Eastern) \ili{Old Japanese} (\textit{iNtu-yu} ‘where from’), \ili{Japanese} (\textit{doko ni} ‘where (to)’, \textit{doko e} ‘where to’, \textit{doko kara} ‘where from’, my knowledge), or \ili{Ura} (\textit{ʔuda=ne} ‘where’, \textit{ʔuda-gatʃi} ‘where to’, \citealt{vanderLubbeTokunaga2015}: 361).

In modern \ili{Japanese} only a few forms in \textit{i{\textasciitilde}} survive (e.g., \textit{itsu}, \textit{ikura}), which is due to a replacement with forms built on the stem \textit{do-}. The fact that all forms are analyzable shows that this is a relatively new system. In fact, the \isi{interrogative} stem \textit{do-} in \ili{Japanese} is completely in line with the \isi{demonstratives} (\tabref{tab:japa:8}). These paradigms were clearly at least partly present in \ili{Old Japanese} (\tabref{tab:japa:9}). But in standard \ili{Japanese} the distal demonstrative \textit{ka-} has been replaced with \textit{a-} and \ili{Old Japanese} still lacked the stem \textit{do-}. Interestingly, written \ili{pre-modern Japanese} still had forms based on the stems \textit{ka-} and \textit{idzu-} (\tabref{tab:japa:10}). In \ili{Japanese} the word \textit{kare} started out as a demonstrative, changed its meaning to a male third person pronoun and also means ‘boyfriend’ today.

\begin{table}
\caption{Parallels in demonstratives and interrogatives in Japanese (based on \citealt{Dixon2012}: 407; \citealt{Hasegawa2015}: 332); the \ili{Kansai} dialect has a regular form \textit{a-ko} instead of the irregular \textit{a.so-ko}; some endings were omitted}
\label{tab:japa:8}

\begin{tabularx}{\textwidth}{lXXXl}
\lsptoprule
& \textbf{proximal} & \textbf{medial} & \textbf{distal} & \textbf{interrogative}\\
\midrule
pronominal & ko-re & so-re & a-re & do-re\\
adnominal & ko-no & so-no & a-no & do-no\\
place & ko-ko & so-ko & a.\textbf{so}-ko & do-ko\\
thing/person (vulgar) & ko-itsu & so-itsu & a-itsu & do-itsu\\
direction/person (polite) & ko-chira & so-chira & a-chira & do-chira\\
type/kind & ko-nna & so-nna & a-nna & do-nna\\
adverb & ko-o & so-o & a-a & do-o\\
\lspbottomrule
\end{tabularx}
\end{table}

\begin{table}
\caption{Old Japanese demonstrative and interrogative paradigms (\citealt{Vovin2005}: 272; \citealt{Kupchik2011}: 583, partly modified); there are additional forms such as \textit{wote} ‘that (over there)’ not shown here}
\label{tab:japa:9}
\begin{tabularx}{\textwidth}{XXXXl}
\lsptoprule
& \textbf{proximal} & \textbf{medial} & \textbf{distal} & \textbf{interrogative}\\
\midrule
pronominal & kö-(re) & sö-(re) & ka-(re) & WOJ iNtu-re\\
adnominal & kö-nö & sö-nö & ka-nö & ?\\
place & kö-kö & sö-kö & - & WOJ iNtu-ku\\
\lspbottomrule
\end{tabularx}
\end{table}

\begin{table}
\caption{Paradigms of written \ili{pre-modern Japanese} demonstrative and interrogative paradigms (\citealt{Aston1904}: 60ff.)}
\label{tab:japa:10}

\begin{tabularx}{\textwidth}{XXXXl}
\lsptoprule
& \textbf{proximal} & \textbf{medial} & \textbf{distal} & \textbf{interrogative}\\
\midrule
pronominal & ko-(re) & so-(re) & ka-(re) & idzu-(re)\\
adnominal & ko-no & so-no & ka-no & idzu\textbf{-re}-no\\
place & ko-ko & ?so-ko & ? & idzu-ko\\
\lspbottomrule
\end{tabularx}
\end{table}

This paradigmatic parallel between pre-modern \textit{idzu-} and modern \textit{do-} might suggest that it is in fact the same etymological entity in a different phonological shape. In some \ili{Ryūkyūan}  languages there is a form without the initial vowel as well. For example, \ili{Okinoerabu} \textbf{\textit{ʔu}}\textit{duru} ‘which’ and \textbf{\textit{ʔu}}\textit{da} ‘where’ (\citealt{vanderLubbeTokunaga2015}: 350) must directly correspond to \textit{dɨru} and \textit{daa} in \ili{Yuwan}. The paradigms in \ili{Hachij\=o} are very similar to modern \ili{Japanese}, but there is a different distal stem \textit{u-} that looks similar to the medial stem in Ryūkyūan (\tabref{tab:japa:11}). In general, the Northern Ryūkyūan languages, especially Amami Ryūkyūan languages, have a pattern very similar to \ili{Japanese}. Except for \ili{Miyara}, the Southern Ryūkyūan languages do not exhibit the same similarities in demonstrative and \isi{interrogative} paradigms. \tabref{tab:japa:12} to \tabref{tab:japa:16} show paradigms for those languages that were described in sufficient detail. Also, northern Ryūkyūan shares the distal stem \textit{a-} with modern \ili{Japanese}, while southern Ryūkyūan still has \textit{ka-}, as does \ili{Old Japanese}. What is more, the extension of the demonstrative and the \isi{interrogative} are only found in northern \ili{Ryūkyūan}  and are not necessarily identical in form. In \ili{Yuwan} and \ili{Shuri}, for example, the \isi{demonstratives} have the extension \textit{-rɨ} {\textasciitilde} \textit{-ri}, but the demonstrative has \textit{-ru}. In \ili{Dunan}, the extension can only be found in the distal demonstrative.

\begin{table}[t]
\caption{Paradigms of Hachij\=o demonstrative and interrogative paradigms (\citealt{KKK1950}: 204f.); cf. \textit{dare} ‘who’; several dialectal forms were omitted}
\label{tab:japa:11}

\begin{tabularx}{\textwidth}{XXXXl}
\lsptoprule
& \textbf{proximal} & \textbf{medial} & \textbf{distal} & \textbf{interrogative}\\
\midrule
pronominal & ko-re & so-re & u-re & do-re\\
adnominal & ko-no & so-no & u-no & do-no\\
place & ko-ko & so-ko & u-ko & do-ko\\
\lspbottomrule
\end{tabularx}
\end{table}

\begin{table}[b]
\caption{Paradigms of Yuwan (Amami) demonstrative and interrogative paradigms (\citealt{Niinaga2010}: 50f.); cf. \textit{ta-rɨ/ru} ‘who’; see also \cite[123-124]{Martin1970}}
\label{tab:japa:12}

\begin{tabularx}{\textwidth}{XXXXl}
\lsptoprule
& \textbf{proximal} & \textbf{medial} & \textbf{distal} & \textbf{interrogative}\\
\midrule
pronominal & ku-rɨ & u-rɨ & a-rɨ & dɨ-\textbf{ru}\\
adnominal & ku-n & u-n & a-n & dɨ-n\\
place & ku-ma & u-ma & a-ma & \textbf{daa}\\
\lspbottomrule
\end{tabularx}
\end{table}

\begin{table}
\caption{Paradigms of Shuri (Okinawan) demonstrative and interrogative paradigms \citep[387]{Miyara2015}; form in square brackets from \cite{OCLS1999–2003}}
\label{tab:japa:13}

\begin{tabularx}{\textwidth}{XXXXl}
\lsptoprule
& \textbf{proximal} & \textbf{medial} & \textbf{distal} & \textbf{interrogative}\\
\midrule
pronominal & ku-ri & ʔu-ri & ʔa-ri & dʒi-\textbf{ru}\\
adnominal & ku-nu & ʔu-nu & ʔa-nu & [dʒi-nu]\\
place & ku-ma & ʔu-ma & ʔa-ma & \textbf{maa}\\
\lspbottomrule
\end{tabularx}
\end{table}

\begin{table}
\caption{Paradigms of Ōgami (Miyako) demonstrative and interrogative paradigms (\citealt{Pellard2009}: 123; 2010: 129), cf. \textit{ta-ɾu} ‘who’; no forms with the Ōgami adnominal (genitive) \textit{-nu} are available; gaps are filled with forms from Miyako proper in square brackets (\citealt{OCLS1999–2003})}
\label{tab:japa:14}

\begin{tabularx}{\textwidth}{XXXXl}
\lsptoprule
& \textbf{proximal} & \textbf{medial} & \textbf{distal} & \textbf{interrogative}\\
\midrule
pronominal & ku-ɾi & u-ɾi & ka-ɾi & \textbf{nti} {\textasciitilde} \textbf{iti}\\
adnominal & [ku-nu] & ? & [ka-nu] & [nza-nu]\\
place & ?ku-ma & u-ma & ka-ma & \textbf{nta} {\textasciitilde} \textbf{ita}\\
\lspbottomrule
\end{tabularx}
\end{table}

\begin{table}
\caption{Paradigms of Miyara (Yaeyama) demonstrative and interrogative paradigms \citep[24]{Izuyama2003}, cf. \textit{ta-ru} ‘who’; there are also the forms \textsc{n}\textit{ge {\textasciitilde}} \textsc{n}\textit{ga} ‘there (medial)’ and \textit{zɪ}\textsc{n}\textit{ge {\textasciitilde} zɪ}\textsc{n}\textit{ga} ‘where’ (+ -\textit{ge {\textasciitilde} -ga})}
\label{tab:japa:15}

\begin{tabularx}{\textwidth}{XXXXl}
\lsptoprule
& \textbf{proximal} & \textbf{medial} & \textbf{distal} & \textbf{interrogative}\\
\midrule
pronominal & ku-ri & u-ri & ka-ri & zɪ-ri\\
adnominal & ku-nu & u-nu & ka-nu & ?\\
place & ku-ma & u-ma & ka-ma & zɪ-ma\\
\lspbottomrule
\end{tabularx}
\end{table}

\begin{table}
\caption{Paradigms of Dunan (Yonaguni) demonstrative and interrogative paradigms (\citealt{YamadaPellardShimoji2015}: 454, 456f.)}
\label{tab:japa:16}

\begin{tabularx}{\textwidth}{XXXXl}
\lsptoprule
& \textbf{proximal} & \textbf{medial} & \textbf{distal} & \textbf{interrogative}\\
\midrule
pronominal & khú & ú & kha-ri & ?\\
adnominal & khu-nu & u-nu & kha-nu & ?\\
place & khû-ma & û-ma & khá-ma & \textbf{nmâ}\\
\lspbottomrule
\end{tabularx}
\end{table}

Apparently, instead of the selective \isi{interrogative}, Yonaguni uses an objective \isi{interrogative}, e.g. \ili{Sonai} \textit{nu} ‘what’ \textit{nu-nu} ‘what-\textsc{adj}’ \citep[431]{Izuyama2012}.

Less complicated than the locative forms are the quantitative interrogatives ‘how much’ and ‘how many’ that are based on PJ *\textit{eku}. Two suffixes, \textit{-Nta} (maybe a collective) and \textit{-ra} (maybe a \isi{plural}) can sometimes be found attached to the stem (\citealt{Vovin2005}: 330, fn. 129). Whether *\textit{eku} was analyzable or not remains an open question. Middle \ili{Japanese} had another variant \textit{iku-tu} ‘how many’ that is not attested in \ili{Old Japanese}. \ili{Ryūkyūan}  languages have cognates of \ili{Old Japanese} *\textit{eku} and *\textit{ekura} as well as of Middle \ili{Japanese} \textit{ikutu}. Similar to *\textit{eka} the initial vowel was sometimes lost and in some cases led to the palatalization of the following velar, e.g. Benoki \textit{kassaa} \citep[332]{Vovin2005}, but \ili{Yuwan} \textit{ikjassa} \citep[51]{Niinaga2010} < \textit{iku-ra} ‘how much’. In some languages the \isi{interrogative} *\textit{eku} is preserved and is usually combined with a classifier, e.g. \ili{Okinoerabu} \textit{ʔiku-tʃi} ‘how many things’, \textit{ʔiku-tai} ‘how many people’ (\citealt{vanderLubbeTokunaga2015}: 351). In \ili{Japanese} this pattern has been taken over by \textit{naɴ-} followed by a classifier, e.g. \textit{nan-mei} \jp{何名} ‘how many people’ (which is the source of Yilan \textit{name}, \citealt{Peng2015}: 53).

\tabref{tab:japa:17} shows those interrogatives found in written and spoken pre-modern {Jap\-a\-nese} interrogatives. Except for those forms based on \textit{idzu}, the interrogatives are still present in modern \ili{Japanese}. There are the resonances \textit{i{\textasciitilde}} and \textit{n{\textasciitilde}}. Today there is also a \isi{resonance} in \textit{d{\textasciitilde}}, but in written pre-modern \ili{Japanese}, the \isi{interrogative} \textit{tare} ‘who’ was unique in that it did not exhibit any of the resonances. \ili{Japanese} \textit{dare} with an initial \textit{d} might be an innovation based on \textit{dore}.


\begin{table}
\caption{Pre-modern Japanese interrogatives (\citealt{Aston1904}: 63ff.); forms marked with an asterisk * are limited to the written language; not all derivations are shown}
\label{tab:japa:17}

\begin{tabularx}{\textwidth}{Xl}
\lsptoprule
\textbf{Meaning} & \textbf{Form}\\
\midrule
who & ta-re*, da-re\\
what & nani\\
what-\textsc{pl} & nani-ra\\
adnominal & nani-no (> nanno)\\
who (\ili{Japanese} \textit{hito} ‘man’) & nani-bito\\
why & naze\\
why & nado\\
which & idzu-re*, do-re\\
which (adnominal) & idzu-re-no*, do-no\\
where & idzu-ko*, do-ko\\
whither & idzu-chi*, do-chi\\
whither & idzu-kata*\\
how, what manner & ika\\
% how, what manner & ika-ni\\
how many & iku-tsu\\
how much & iku-ra\\
when & itsu\\
\lspbottomrule
\end{tabularx}
\end{table}

Few descriptions of \ili{Ryūkyūan}  languages available to me give such an exhaustive list of interrogatives. Some \isi{questions} are thus hard to \isi{answer}. But the limited data allow the observation that, from a typological point of view, the \isi{interrogative systems} are very different from one another. In \ili{Hateruma}, for instance, all attested interrogatives except \textit{icï} ‘when’ are only two phonemes long and none is readily analyzable synchronically (\textit{nu} ‘what’, \textit{za} ‘where’, \textit{ne} ‘why, how’, \textit{ta} ‘who’, \citealt[199]{Aso2010}; \citeyear[429]{Aso2015}).

\clearpage %solid chapter boundary
\begin{table}[t]
\caption{Ōgami interrogatives (\citealt{Pellard2009}: 132; \citeyear{Pellard2010}: 129); my tentative analysis based on \cite{Pellard2009,Pellard2010}}
\label{tab:japa:18}

\begin{tabularx}{\textwidth}{Xll}
\lsptoprule

\textbf{Meaning} & \textbf{Form} & \textbf{Analysis}\\
\midrule
who & ta-ɾu {\textasciitilde} tau & < PJ, loss of ɾ, analogy to nau?\\
when & iks & < PJ\\
how many & if- + \textsc{clf} & e.g., if-taɯ ‘how many people’\\
what & nau & < PR\\
which & nti {\textasciitilde} iti & \\
where & nta {\textasciitilde} ita & analogy to nti {\textasciitilde} iti\\
why & nau-ɾipa & circumstantial \isi{converb} -ɾipa\\
how & nau-pasi & \\
how much & nau-nu-pusa, nti-ka-pusa & \\
\lspbottomrule
\end{tabularx}
\end{table}

In \ili{Ōgami}, on the other hand, the interrogatives are up to nine phonemes long and some are at least partly analyzable (\tabref{tab:japa:18}). \ili{Ōgami} has two main resonances \textit{i{\textasciitilde}} and \textit{n{\textasciitilde}} as well as one form \textit{taɾu} ‘who’ that does not partake in any of them. The two \ili{Ōgami} forms inquiring about quantity apparently are based on \textit{nau} ‘what’ and \textit{nti} ‘which’, respectively, and can be analyzed as \textit{nau-nu-pusa {\textasciitilde} nti-ka-pusa}. The exact meaning of the suffixes remains unclear, however. A connection to the desiderative form \textit{-pus} is unlikely on semantic grounds. The second part of \textit{nau-pasi} also remains unclear. There is a circumstantial \isi{converb} form \textit{-}\textit{ɾipa} \citep[146]{Pellard2009} that might have been attached to a hypothetical \isi{interrogative verb} \textit{nau-} ‘to do what’, yielding \textit{nau-ɾipa} ‘why’.

In \ili{Yuwan} \textit{nuusjattu} probably has a similar background and may be an amalgamated form containing the elements \textit{nuu} ‘what’, the verbalizer \textit{-s(j)ar}, and the past causal \isi{converb} \textit{-tattu} (\citealt{Niinaga2010}: 66, 71). \ili{Japanese} \textit{d\=o} \textit{yatte} literally means ‘doing how’ and can be analyzed into \textit{d\=o} ‘how’ and the so-called \textit{te}-form (roughly gerund) of the verb \textit{yaru} ‘to do, to give, to put’. These few cases suffice to show a strong connection between the two categories of \textsc{activity} and \textsc{reason} (\sectref{sec:4.3}).


The Amami languages \textbf{Yuwan} and \textbf{Shodon} as well as the Okinawan language \textbf{Shuri} (\tabref{tab:japa:19}) exhibit a pattern very similar to \ili{Japanese} and have the three resonances \textit{n{\textasciitilde}}, \textit{i{\textasciitilde}} and \textit{d{\textasciitilde}} (> \textit{dʒ} in \ili{Shuri}). But the languages preserve an initial unvoiced aspirated plosive \textit{t} in the \isi{interrogative} meaning ‘who’.

\begin{table}
\caption{Shodon (\citealt{Martin1970}: 123f.), Yuwan \citep[51]{Niinaga2010}, and Shuri \citep[387]{Miyara2015} interrogatives; accents removed, modified transcription for glottal stops; form in square brackets from \cite{OCLS1999–2003}}
\label{tab:japa:19}

\begin{tabularx}{\textwidth}{XlXl}
\lsptoprule

\textbf{Meaning} & \textbf{Shodon} & \textbf{Yuwan} & \textbf{Shuri}\\
\midrule
who & tha-r(u-), thaa-ga & ta-rɨ {\textasciitilde} ta-ru & taa\\
what & nu(u) & nuu & nuu\\
why &  & nuusjattu & nuuntʃi\\
which one & dir & dɨru & dʒiru\\
which (adn.) & din & dɨn & [dʒi-nu]\\
where & da(a) & daa & maa\\
what kind of & ʔyikhyassyun &  & \\
when & ʔyit(i) & ɨcɨ & ʔitʃi\\
how many & ʔyitkhut(i) {\textasciitilde} ʔyitkhu(u)t(i) &  & \\
how much & ʔyikhyassa & ikjassa & tʃassa\\
how &  &  & tʃaʃʃi\\
\lspbottomrule
\end{tabularx}
\end{table}

The only polysemy that has been described can be found in \ili{Hateruma} \textit{ne}, which covers both \textsc{manner} and \textsc{reason}.

\largerpage
For the most part, interrogatives in \textbf{Yilan Creole} are identical or almost identical to \ili{Japanese} (e.g., \textbf{\textit{l}}\textit{are} ‘who’, \textit{nani} ‘what’, \textit{doko} ‘where’, \textit{ikura} ‘how much’, \textit{name} ‘how many (people)’, \citealt{Peng2015}: 52ff.). One interesting phenomenon as opposed to Standard \ili{Japanese} (\ref{ex:japa:2}c) is the use of an \isi{interrogative} basically meaning ‘who’ instead of ‘what’ in \isi{questions} about names, see also (\ref{ex:japa:35}b) from \ili{Tarama} (see \citealt{Idiatov2007}; \citealt{Hölzl2014b} for a general discussion). This may be due to influence from \ili{Austronesian} languages, maybe via \ili{Mandarin} \ili{Chinese} as spoken on \isi{Taiwan}.

\ea%54
    \label{ex:japa:54}
    \ili{Yilan Creole}\\
    \gll anta no namae ga \textbf{{lare}}?\\
    2\textsc{sg}  \textsc{gen}  name  \textsc{top}  who\\
    \glt ‘\isi{What is your name?}’ \citep[54]{Peng2015}
    \z

\ea%55
    \label{ex:japa:55}
    \ili{Japanese}\\
    \gll anata  no  namae  wa \textbf{{nan}} desu  ka?\\
    2\textsc{sg}  \textsc{gen}  name  \textsc{top}  what  \textsc{cop}  \textsc{q}\\
    \glt ‘\isi{What is your name?}’ (constructed in analogy to \ili{Yilan Creole})
    \z

\ea%56
    \label{ex:japa:56}
    \ili{Mandarin} \ili{Chinese} (\isi{Taiwan})\\
    \gll nǐ  de  míngzi  shì \textbf{{shéi}}?\\
    2\textsc{sg}  \textsc{gen}  name  \textsc{cop}  who\\
    \glt ‘\isi{What is your name?}’\footnote{This sentence was given to me by a native speaker from \isi{Taiwan} during my talk at \textit{The 8th International Conference on Construction Grammar} (\citealt{Hölzl2014b}). \ili{Chinese} also has further constructions.}
    \z

\ea%57
    \label{ex:japa:57}
    Maryinax \ili{Atayal}\\
    \gll \textbf{{ima}}\textbf{{\textsuperscript{ʔ}}} {a\textsuperscript{ʔ}} {ralu\textsuperscript{ʔ}}{=su\textsuperscript{ʔ}}?\\
    who  \textsc{nom}.\textsc{nrf}  name=2\textsc{sg}.\textsc{bg}\\
    \glt ‘\isi{What is your name?}’ \citep[271]{Huang1996}
    \z

\noindent This might be an areal trait that has its origin in Austronesian languages where it is a rather typical phenomenon (\citealt{Blust2013}: 509f.). Standard \ili{Chinese} as spoken in the People’s Republic of \isi{China} usually employ the \isi{interrogative} \textit{shénme} ‘what’. Other varieties of \ili{Atayal} such as Wulai in turn employ \textit{nanu\textsuperscript{ʔ}} ‘what’ instead of \textit{ima\textsuperscript{ʔ}} ‘who’ in official contexts, which may have its origin in \ili{Chinese} \citep[293]{Huang1996}. While in \ili{Yilan Creole} the use of \textit{lare} ‘who’ may have an origin in \ili{Austronesian}, the whole construction rather resembles \ili{Chinese} and especially \ili{Japanese}, except for the lack of the copula.