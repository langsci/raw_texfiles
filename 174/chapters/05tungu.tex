\section{Tungusic}\label{sec:5.10}
\subsection{Classification of Tungusic}\label{sec:5.10.1}

Recently, \citet[16]{Janhunen2012b} suggested the following classification of \ili{Tungusic} that is based mostly on previous work done by other researchers (\citealt{Ikegami1974,Lie1978,Doerfer1978a,Georg2004}). The designations of languages has been slightly adapted. The primary split in \ili{Tungusic} is between northern and southern \ili{Tungusic}, both of which subsequently went through a secondary split. Thus, there are four main branches that, following Janhunen, may be called \ili{Ewenic}, \ili{Udegheic}, \ili{Nanaic}, and \ili{Jurchenic}. There are, however, several minor problems with Janhunen’s classification. For example, it does not show the strong dialectal division of some of the languages. \ili{Even} and \ili{Evenki}, for instance, are said to have about 12 and 50 dialects, respectively (e.g., \citealt{Malchukov1995,Atknine1997}). The classification of \ili{Solon} into three different languages is too detailed, whereas the dialects of \ili{Oroqen} are not even mentioned (e.g., \citealt{WhaleyLi2000}, see \figref{exfig:tungu:1}).

% \ea\upshape%1
\begin{figure}
\caption{Classification of Tungusic}
    \label{exfig:tungu:1}
    \small 
    \fittable{
\begin{forest}  for tree={grow'=east,delay={where content={}{shape=coordinate}{}}},   forked edges
[
    [Northern \ili{Tungusic}
        [\ili{Ewenic}
        	[Siberian \ili{Ewenic}
            	[\ili{Even} proper]
                [\ili{Arman}]
                [\ili{Evenki} proper]
                [\ili{Negidal}]
                [\ili{Oroqen}]
                [Urulga Khamnigan \ili{Evenki}]
            ]
            [Manchurian \ili{Ewenic}
            	[Mankovo Khamnigan \ili{Evenki}]
                [Nonni \ili{Solon}]
                [Hailar \ili{Solon}]
                [Ongkor \ili{Solon}]
            ]
        ]
        [\ili{Udegheic}
        	[\ili{Udihe} proper]
            [\ili{Oroch}]
        ]
    ]
    [Southern \ili{Tungusic}
        [\ili{Nanaic}
        	[\ili{Nanai} proper]
            [\ili{Kili}]
            [\ili{Kilen}]
            [Ulchaic
            	[\ili{Ulcha} proper]
                [\ili{Uilta} (Orok)]
            ]
        ]
        [\ili{Jurchenic}
			[\isi{Jurchen} proper]
            [\ili{Manchu}]
            [\ili{Sibe}]
        ]
    ]
]
\end{forest}   
}
%     \z
\end{figure}

While the dialectal divison of \ili{Nanaic}, \ili{Udegheic}, and \ili{Ewenic} is rather well understood, there is almost no attempt at a classification of \ili{Jurchenic}. The \ili{Jurchenic} branch has been named after \isi{Jurchen}, the oldest attested \ili{Tungusic} language. Several scholars have tried to give an adequate account of the relation of \isi{Jurchen} and \ili{Manchu}, the second oldest attested \ili{Tungusic} language. \citet[6]{Janhunen2012b} claims that, despite “slight variation in the dialectal basis”, the three \ili{Jurchenic} languages \isi{Jurchen}, \ili{Manchu}, and \ili{Sibe} “may be classified as a \isi{diachronic} sequence of a single language”. However, even if we consider \isi{Jurchen} an archaic form of \ili{Manchu} as does Janhunen, apparently following \citet[12]{Doerfer1978a}, this is imprecise and somewhat misleading. Doerfer’s classification, of course, was written before the bulk of information necessary became available during the 1980s, when mainly \ili{Chinese} linguists started to produce grammatical descriptions of \ili{Jurchenic} varieties. Except for \ili{Sibe}, these have mostly been neglected in western descriptions. I tentatively propose a new classification of \ili{Jurchenic} (e.g., \citealt{Hölzl2017c,Hölzl2018b}, \figref{exfig:tungu:2}).

% \ea%2
\begin{figure}
 \caption{Proposed new classification of Jurchenic}

    \label{exfig:tungu:2}
\begin{forest}  for tree={grow'=east,delay={where content={}{shape=coordinate}{}}},   forked edges  
[
    [\ili{Manchuic}
        [Written \ili{Manchu} (since 1600)
        [Written \ili{Sibe}]
        ]
        [\ili{Manchu} dialects
        	[˚\ili{Jurchen B} (ca. 1500)]
        		[\textsuperscript{(†)}Manchurian dialects]
            		[Dzungarian dialect(s) (Spoken \ili{Sibe})]
        ]
    ]
    [\ili{Balaic}
        [˚Written \isi{Jurchen} (12\textsuperscript{th}-15\textsuperscript{th} c.)]
        [˚\ili{Jurchen  A} (ca. 15\textsuperscript{th} c.)]
        [\textsuperscript{†}\ili{Bala} dialects]
    ]
    [Alchukaic
    	[\textsuperscript{†}\ili{Alchuka} (sociolects?)]
    ]
]
\end{forest}   

˚ = only historically attested
\textsuperscript{†} = no speaker left
\textsuperscript{(}\textsuperscript{†}\textsuperscript{)} = moribund, almost extinct\\
%     \z
\end{figure}


The exact branching structure, especially the precise relation of the three hypothetical branches, has yet to be investigated. Until recently, \ili{Alchuka} and \ili{Bala} were almost unknown in the West (e.g., \citealt{MuYejun1985,MuYejun1986,MuYejun1987,Ikegami1999} [1993]; \citealt{Hölzl2014a}: 212; \citealt{Hölzl2015a}: 136 fn. 27; \citealt{Hölzl2017c,Hölzl2018b,Hölzl2018b}). \ili{Bala} is basically \ili{Jurchenic} but exhibits some influence from several other \ili{Tungusic} languages (\citealt{MuYejun1985,MuYejun1986}). \ili{Alchuka} preserves some archaic features (e.g., an initial \textit{k-}, and a verbal suffix \textit{-ʐï} < *\textit{-si}, \citealt{Hölzl2017c,Hölzl2018b}), but also has unique innovations (such as the loss of several word internal consonants) and displays some interference from \ili{Manchuic}. The existence of two distinct \isi{Jurchen} languages has also been recognized by \citet{Kiyose2000}. They have been called \ili{Jurchen  A} (Bureau of Translators, \citealt{Kiyose1977}) and \ili{Jurchen  B} (Bureau of Interpreters, \citealt{Kane1989}) in analogy with similar cases, such as \ili{Tocharian A} and B. Given that \ili{Sibe}, located in Dzungaria since 1764, has been relatively isolated for over two hundred years and was strongly influenced by \ili{Khorchin} \ili{Mongolian} before that, it has to be kept apart from those dialects still located in \isi{Manchuria} (e.g., Aihui, Lalin/Jing, Sanjiazi, Yibuqi). All modern varieties, except \ili{Bala} and \ili{Alchuka}, may be classified as \ili{Manchu} dialects that, together with Written \ili{Manchu} and \ili{Jurchen  B}, form the \ili{Manchuic} branch of \ili{Jurchenic}. \ili{Bala}, together with \ili{Jurchen  A} (\citealt{MuYejun1987} also saw this connection), form a branch on their own called \ili{Balaic}. The distinction between \ili{Jurchen  A} and Written \isi{Jurchen} is mostly heuristic in nature. Technically speaking, if the above classification is correct, the forerunner of \ili{Alchuka} might be called “\isi{Jurchen} C” but does not seem to be attested. Only the somewhat mysterious language of the \ili{Kyakala} in \isi{China} (\textit{kiyakara} in \ili{Manchu}) had to be excluded for lack of data, but it seems to be a mixture of different \ili{Jurchenic} varieties as well as, perhaps, some other \ili{Tungusic} languages (see \citealt{Hölzl2018a} for details).


\subsection{Question marking in Tungusic}\label{sec:5.10.2}

In \textbf{Evenki}, there are two ways of expressing \isi{polar question}s. The first relies on a change of \isi{intonation}: “The \isi{focus}, as a rule, attracts the intonational nucleus on to itself, the intonational contour being higher and more prolonged than that of the corresponding positive sentence” (\citealt{Nedjalkov1997}: 4f.). The following example can mean both ‘They killed the elk.’ (\textit{ty} being the intonational nucleus) and ‘Did they kill the elk?’ (with the tone peak on \textit{tyva vaa}).

\ea%3
    \label{ex:tungu:3}
    \ili{Evenki}\\
    \gll nuŋartyn  moty-va vaa-re-Ø.\\
    3\textsc{pl}    elk-\textsc{acc}  kill-\textsc{non.fut}-(3\textsc{pl})\\
    \glt ‘They killed the elk./Did they kill the elk?’ (\citealt{Nedjalkov1997}: 4f.)
    \z

The second way of expressing a \isi{polar question} makes use of an enclitic \textit{=Ku} that can have several variants depending on the preceding sounds, \textit{=gu}, \textit{=ku}, \textit{=ŋ}\textit{u}, and \textit{=vu}. The enclitic attaches to the verb in polar \isi{questions}, to the focused element in \isi{focus question}s, and occurs twice in \isi{alternative question}s.

\ea%4
    \label{ex:tungu:4}
    \ili{Evenki}\\
    \ea
    \gll si  ulle-ve      d’ep-che-s\textbf{{=ku}}?\\
    2\textsc{sg}    meat-\textsc{acc}    eat-\textsc{pst}-2\textsc{sg}=\textsc{q}\\
    \glt ‘Have you \textit{eaten} the meat?’
    
    \ex
    \gll si  ulle-ve\textbf{{=gu}} d’ep-che-s?\\
    2\textsc{sg}    meat-\textsc{acc}=\textsc{q}    eat-\textsc{pst}-2\textsc{sg}\\
    \glt ‘Did you eat \textit{meat/the meat}?’
    
    \ex
    \gll tar  asatkan  soŋo-d’oro-n=\textbf{{ŋu}},  in’ekte-d’ere-n=\textbf{{ŋu}}?\\
    that    girl    cry-\textsc{prs}-3\textsc{sg}=\textsc{q}  laugh-\textsc{prs}-3\textsc{sg}=\textsc{q}\\
    \glt ‘Is that girl crying or laughing?’ (\citealt{Nedjalkov1997}: 136, 7)
    \z
    \z

\noindent This \isi{question marker} can be traced back to \ili{Proto-Tungusic} \citep[147]{Benzing1956} and exhibits formal and functional similartities to the \ili{Mongolic} \isi{question marker} (\sectref{sec:5.8.2}), which might indicate an old loan relationship of unclear direction. \ili{Buryat} \textit{=gü} and \ili{Khamnigan Mongol} \textit{=gv} might be relatively recent loans from \ili{Evenki}.

Since \ili{Tungusic} has different negators depending on the clause type and other factors (\citealt{Hölzl2015a}), negative \isi{alternative question}s show different patterns as well. For standard \isi{negation}, many \ili{Tungusic} languages employ a negative verb. The \isi{question marker} that is found once in polar but twice in alternative \isi{questions} attaches to the first alternative and to the conjugated negative verb while the rest of the second alternative, including the lexical verb, is deleted. Below we will encounter \isi{negative alternative question}s with other negators.

\ea%5
    \label{ex:tungu:5}
    \ili{Evenki}\\
    \gll eme-d{’e-n}{=}\textbf{{ŋu}} e{.le} ta{.r} asi, \textbf{{e}}{-te-n}{=}\textbf{{ŋu}}?\\
    come-\textsc{fut}-3\textsc{sg}-\textsc{q}  here  that  woman    \textsc{neg}-\textsc{fut}-3\textsc{sg}-\textsc{q}\\
    \glt ‘I wonder if that woman will come here or not.’ \citep[7]{Nedjalkov1997}
    \z

Content \isi{questions} do not show the enclitic and remain unmarked.

\ea%6
    \label{ex:tungu:6}
    \ili{Evenki}\\
    \gll \textbf{{e}}\textbf{{kun}}{-duk} eme-che-s?\\
    what-\textsc{abl}  come-\textsc{pst}-2\textsc{sg}\\
    \glt ‘Where did you come from?’ \citep[3]{Nedjalkov1997}
    \z

Focus questions may also remain unmarked morphosytactically, in which case the focused element seems to take second position (cf. \citealt{Nedjalkov1997}: 135).

\newpage 
\ea%7
    \label{ex:tungu:7}
    \ili{Evenki}\\
    \ea
    \gll si [er   dukuvun-ma]  ga-cha-s?\\
    2\textsc{sg}  this  book-\textsc{acc} buy-\textsc{pst}-2\textsc{sg}\\
    \glt ‘Did you buy this book?’
    
    \ex
    \gll [er   dukuvun-ma] si ga-cha-s?\\
    this  book-\textsc{acc}  2\textsc{sg}  buy-\textsc{pst}-2\textsc{sg}\\
    \glt ‘Did \textit{you} buy this book?’ \citep[5]{Nedjalkov1997}
    \z
    \z

Question marking in \ili{Evenki} dialects does not appear to differ much from Standard \ili{Evenki}. The following examples were drawn from the \textbf{Sakhalin} dialect that belongs to the eastern group of dialects \citep{Atknine1997}. In this dialect the enclitic has a long vowel but displays the same \isi{semantic scope} and morphosyntactic behavior.

\ea%8
    \label{ex:tungu:8}
    \ili{Evenki} (\isi{Sakhalin})\\
    \ea
    \gll kilivliil,  saa-ra-s    ee-ra-s=\textbf{{kuu}}?\\
    girl    know-\textsc{aor}-2\textsc{pl}  \textsc{neg}-\textsc{aor}-2\textsc{pl=q}\\
    \glt ‘Girls, have you noticed?’
   
    \ex
    \gll ile-ve=\textbf{{guu}} e-chee-s    xenu-re?\\
    man-\textsc{acc}=\textsc{q}  \textsc{neg}-\textsc{pst}-2\textsc{sg}    notice-\textsc{cn}\\
    \glt ‘Haven’t you noticed the man?’
    
    \ex
    \gll ŋene-ŋeet-y-vun=\textbf{{ŋuu}} \textbf{{e}}{-ŋeet-y-vun=}\textbf{{ŋuu}}?\\
    go-\textsc{deb}-\textsc{e}-1\textsc{pl}.\textsc{ex}=\textsc{q}    \textsc{neg}-\textsc{deb}-\textsc{e}-1\textsc{pl}.\textsc{ex}=\textsc{q}\\
    \glt ‘Should we go or not?’
    
    \ex
    \gll \textbf{{eedaa}} ekeendeek-lee-vun      eme-nni?\\
    why  dancing.place-\textsc{loc}-1\textsc{pl}.\textsc{ex}.\textsc{poss}  come-2\textsc{sg}\\
    \glt ‘Why have you come to our dancing place?’ (\citealt{BulatovaCotrozzi2004}: 64, 72, 19, 12)
    \z
    \z

There is also an enclitic \textit{=too} of unclear origin that marks \isi{polar question}s and does not seem to exist in Standard \ili{Evenki} \citep{Nedjalkov1997}.

\ea%9
    \label{ex:tungu:9}
    \ili{Evenki} (\isi{Sakhalin})\\
    \gll ta.r-gachiin  beje-ŋelii  kuxii-ŋeet-y-c=\textbf{{too}}?\\
    that-\textsc{eval}  man-\textsc{com}  fight-\textsc{deb}-\textsc{e}-2\textsc{sg}=\textsc{q}\\
    \glt ‘You are supposed to fight with such a man?’ (\citealt{BulatovaCotrozzi2004}: 19)
    \z

\textbf{Khamnigan Evenki} preserves the original enclitic as \textit{=gv} but differs from \ili{Evenki} in having borrowed both \ili{Russian} \textit{=li} (\sectref{sec:5.5.2.2}), as well as the corrogative marker \textit{bei} from \ili{Khamnigan Mongol} (\sectref{sec:5.8.2}). In \ili{Mongolic} the marker is derived from the copula and even in Khamnigan \ili{Evenki} seems to be mutually exclusive with the autochthonous copula \textit{bi-}. The fact that Khamnigan \ili{Evenki} \textit{=gv} does not show a variation, as in \ili{Evenki}, may indicate influence from \ili{Khamnigan Mongol}.

\newpage 
\ea%10
    \label{ex:tungu:10}
    \ili{Evenki} (Khamnigan)\\
    \ea
    \gll aya  bi-si-ndi=\textbf{{gv}}?\\
    good  be-\textsc{prs}-2\textsc{sg}=\textsc{q}\\
    \glt ‘Are you well?’
    
    \ex
    \gll e.r-si \textbf{{nii}} \textbf{{bei}}?\\
    this-2\textsc{sg}.\textsc{poss}  who  \textsc{q}\\
    \glt ‘Who is this (one of yours)?’
    
    \ex
    \gll sii \textbf{{nii}} bi-si-ndi?\\
    2\textsc{sg}  who  be-\textsc{prs}-2\textsc{sg}\\
    \glt ‘Who are you?’
    
    \ex
    \gll murin=\textbf{{li}},  vkvr=\textbf{{li}}?\\
    horse=\textsc{q}  cow=\textsc{q}\\
    \glt ‘(Is it) a horse or a cow?’ (\citealt{Janhunen1991}: 95f.)\z\z

In \textbf{Even} the enclitic has the variants \textit{=gu}, \textit{=ku}, and \textit{=ŋu} \citep[19]{Malchukov1995} and as in \ili{Evenki} marks polar, \isi{focus}, and \isi{alternative question}s. Content questions remain unmarked.

\ea%11
    \label{ex:tungu:11}
    \ili{Even}\\
    \ea
    \gll {i-d’i-m=}\textbf{{gu}}?\\
    enter-\textsc{fut}-1\textsc{sg}=\textsc{q}\\
    \glt ‘Shall I come in?’ \citep[165]{Malchukov2001}
    
    \ex
    \gll tiniv=\textbf{{gu}} {em-}{ri-}{n?}\\
    yesterday=\textsc{q}  come-\textsc{pst}-3\textsc{sg}\\
    \glt ‘Did he come \textit{yesterday}?’ (Andrej Malchukov p.c. 2013)
    
    \ex
    \gll uliki-w=\textbf{{gu}} bu-ri-s,    hulica-m=\textbf{{gu}}?\\
    squirrel-\textsc{acc}=\textsc{q}  give-\textsc{pst}-2\textsc{sg}  fox-\textsc{acc}=\textsc{q}\\
    \glt ‘Did you give (him/her) a squirrel or a fox?’ \citep[111]{Benzing1955}\z\z

\ea%12
    \label{ex:tungu:12}
    \ili{Even}\\
    \gll etiken \textbf{{i}}-le    hör-re-n?\\
    old.man  which-\textsc{all}  go-\textsc{non}.\textsc{fut}-3\textsc{sg}\\
    \glt ‘Where has the old man gone?’ \citep[19]{Malchukov1995}
    \z

\ili{Even} has a further \isi{question marker} \textit{=i} \citep[138]{Malchukov2008} with possible parallels in \ili{Negidal}, \ili{Solon}, and, less likely, \ili{Uilta}. Dialects of \ili{Even} show basically the same \isi{question marking} patterns. Consider the following examples from the western dialect area.

\ea%13
    \label{ex:tungu:13}
   \ili{Even} (Western)\\
    \ea
    \gll ta.wa.r \textbf{{\textsuperscript{i}}}\textbf{{ɛ.k}} bi-d’i-n?\\
    that    what  \textsc{cop}-\textsc{fut}-3\textsc{sg}\\
    \glt ‘What is it?’ (said in riddles)
    
    \ex
    \gll oldo-ɯ    oldo-mi-s=\textbf{{gɯ}}?\\
    fish-\textsc{acc}  fish-\textsc{v}-2\textsc{sg}=\textsc{q}\\
    \glt ‘Do you catch fish?’ (\citealt{Sotavalta1978}: 30, 28, simplified)
    \z
    \z

For the eastern dialect area (from the river Anadyr), \citet[217]{Schiefner1874} has an example of an \isi{alternative question} without a \isi{question marker} but with what appears to be a \isi{disjunction} \textit{tömi}. Given that disjunctions are very rare in the northern part of \isi{NEA} (\sectref{sec:6.4}), but also exist in Kolyma \ili{Yukaghir} (\sectref{sec:5.14.2}), an areal connection seems possible. \citet[179]{Malchukov2001} argues that imperative sentences in \ili{Even} “may be used in interrogative sentences to ask for permission”.

\ea%14
    \label{ex:tungu:14}
    \ili{Even}\\
    \gll kosci-\textbf{{de}}{-ku?}\\
    fetch-\textsc{fut}.\textsc{imp}-1\textsc{sg}\\
    \glt ‘Shall I fetch (the reindeer)?’ \citep[165]{Malchukov2001}
    \z

\noindent This might indicate a certain connection to the \ili{Chukotko-Kamchatkan} languages in which there is a general affinity of imperatives to \isi{question marking} (\sectref{sec:5.3.2}).

\cite{Matić2016} claims that \ili{Even} has a special category of \isi{tag question}s that developed out of the negative verb.

\ea%15
    \label{ex:tungu:15}
    \ili{Even} (Tompo)\\
    \gll adʒịt=ta,  ta-la    ịh-ha-p \textbf{{e}}{-he-p?}\\
    truth=?and  that-\textsc{all}  arrive-\textsc{nfut}-1\textsc{pl}  \textsc{neg}-\textsc{nfut}-1\textsc{pl}\\
    \glt ‘And indeed, we have arrived there, haven’t we?’ (\citealt{Matić2016}: 171)
    \z

\noindent It may be noted that the construction actually has the form of an elliptical \isi{negative alternative question} (\textit{or not?}), but with \isi{juxtaposition} instead of \isi{double marking}. This certainly explains the fact that, as in many other examples from \ili{Tungusic} languages, the negative verb takes the same suffixes as the lexical verb. An interesting phenomenon is the optional presence of a contrastive or adversative enclitic C\textit{=kA} {\textasciitilde} V\textit{=kkA} that precedes the negative verb, but is not restricted to \isi{questions} \citep[112]{Benzing1955}.

\ea%16
    \label{ex:tungu:16}
    \ili{Even} (Tompo)\\
    \gll hi{i=}\textbf{{kke}} \textbf{{e}}{-he-ndi} e{.re}{.k} kụŋa{a-w} čọrda-ndị?\\
    2\textsc{sg}=\textsc{contr}  \textsc{neg}-\textsc{nfut}-2\textsc{sg}  this  child-\textsc{acc}  beat-2\textsc{sg}\\
    \glt ‘You beat up this child, didn’t you?’ (\citealt{Matić2016}: 172)
    \z

\noindent The \isi{word order} in this last example \REF{ex:tungu:16} is indeed problematic for the \isi{analysis} as \isi{alternative question} and strongly speaks in favor of \citegen{Matić2016} \isi{analysis}, although there are other examples with relatively free \isi{word order} above (e.g., \ref{ex:tungu:11}).

For \textbf{Arman}, \cite{DoerferKnüppel2013}---the only source readily available---do not have examples for any \isi{question type}. Given its very close relation to \ili{Even}, we may speculate that the marking of \isi{questions} was similar. However, several interrogatives are attested and will be presented in \sectref{sec:5.10.3}.

Questions in \textbf{Oroqen} are usually marked with an enclitic that has the variants \textit{=ŋee} {\textasciitilde} \textit{=ŋ}\textsc{ee} after \isi{nasals} and \textit{=jee} {\textasciitilde} \textit{=j}\textsc{ee} in all other positions (\citealt{HuZengyi2001}: 157). It cannot be cognate with \ili{Evenki} \textit{=Ku} which exists in \ili{Oroqen} as well. Most likely it has a connection to \textit{=yee} in the \ili{Mongolic} language \ili{Dagur} (but see \citealt{Whaley2005}). The enclitic marks polar and alternative \isi{questions}. No instance has been found where it marks \isi{focus question}s.

\ea%17
    \label{ex:tungu:17}
    \ili{Oroqen} (Nanmu)\\
    \ea
    \gll tari naan  nammu-ŋi  bəi=\textbf{{ŋee}}?\\
    3\textsc{sg}  also  \textsc{pn}-\textsc{gen}    man=\textsc{q}\\
    \glt ‘Is (s)he also from Nanmu?’
    
    \ex
    \gll miin    əgdi  nin.ut=\textbf{{ŋee}} unaadʒi=\textbf{{ŋee}}?\\
    \textsc{superl} old  boy=\textsc{q}    girl=\textsc{q}\\
    \glt ‘Is your oldest (child) a boy or a girl?’ (\citealt{Chaoke2007}: 141, 152)
    \z
    \z

In the Shengli dialect, the enclitic has a variant \textit{=\textsc{n}i} after \isi{nasals}. This throws some doubt on the connection with \ili{Dagur} but opens up the possibility of a comparison with \ili{Solon} \textit{=gi(i)}.

\ea%18
    \label{ex:tungu:18}
    \ili{Oroqen} (Shengli)\\
    \gll ɔlɔ-jɔ    {pi-xi-n=\textbf{\textsc{n}}\textbf{i}}?\\
    fish-\textsc{part}  \textsc{cop}-\textsc{prs}-3\textsc{sg}=\textsc{q}\\
    \glt ‘Is there any fish?’ (\citealt{HanMeng1993}: 307)
    \z

Content \isi{questions} do not have the enclitic and are unmarked morphosyntactically as in \ili{Evenki}, and \ili{Even}. This appears to be a difference to \ili{Dagur}, but as we will see for the Xunke dialect of \ili{Oroqen} below, the enclitic optionally also marks \isi{content question}s.

\ea%19
    \label{ex:tungu:19}
    \ili{Oroqen} (Chaoyang)\\
    \gll ʃii \textbf{{ɪkʊn}} dʒaalɪn    ə.ləə  əmə-tʃə-j?\\
    2\textsc{sg}  what  reason    here  come-\textsc{pst}-2\textsc{sg}\\
    \glt ‘Why did you come here?’ (\citealt{HuZengyi2001}: 148)
    \z

Another enclitic has the form \textit{=oo} and expresses a certain fear that something has happened (\citealt{HuZengyi2001}: 157).

\ea%20
    \label{ex:tungu:20}
    \ili{Oroqen} (Chaoyang)\\
    \gll tari  jabʊ-tʃ{aa=}\textbf{{oo}}?\\
    3\textsc{sg}    go-\textsc{pst}=\textsc{q}\\
    \glt ‘Is (s)he going?’ (\citealt{HuZengyi2001}: 157)
    \z

\noindent Quite clearly, this is a loan from \ili{Mongolian} \textit{=UU} that may have acquired a special semantics in \ili{Oroqen}. Alternative \isi{questions} may be marked with the enclitic \textit{=jɔɔmaa} {\textasciitilde} \textit{=jooməə} that is of \ili{Mongolic} origin and may combine with a cognate of \ili{Evenki} \textit{=Ku}.

\newpage
\ea%21
    \label{ex:tungu:21}
    \ili{Oroqen} (Chaoyang)\\
    \ea
    \gll əri  bajta  tədʒəə \textbf{{jooməə}},  olook \textbf{{jooməə}}?\\
    this  matter  real  \textsc{q} false  \textsc{q}\\
    \glt ‘Is this matter true or false?’
    
    \ex
    \gll əri  bəjə  ʊnaadʒɪ-tʃi \textbf{{jɔɔmaa.gʊʊ}},  utə-tʃi \textbf{{jooməə}}.\textbf{{guu?}}\\
    this  person  girl-\textsc{poss}  \textsc{q}    boy-\textsc{poss}  \textsc{q}\\
    \glt ‘Does he (this man) have a girl or a boy?’ (\citealt{HuZengyi2001}: 158)
    \z
    \z

Apparently, \ili{Oroqen} also has borrowed the \ili{Chinese} marker \textit{ba} \zh{吧}, but sometimes has two vowel harmonic variants \textit{baa} and \textit{bəə}. It has a long vowel as in some varieties of \ili{Khorchin}, \ili{Dagur} (\sectref{sec:5.8.2}), and \ili{Solon} (see below).

\ea%22
    \label{ex:tungu:22}
    \ili{Oroqen} (Chaoyang)\\
    \gll əri  mʊrin  aja  mʊrin \textbf{{baa}}?\\
    this  horse  good  horse  \textsc{q}\\
    \glt ‘This horse is a good one, right?’ (\citealt{HuZengyi2001}: 157)
    \z

\ili{Oroqen} has also borrowed the \ili{Chinese} interrogative \isi{disjunction} \textit{háishì} \zh{还是} ‘or.\textsc{q}’ for \isi{alternative question}s. As in \ili{Chinese}, no additional \isi{question marker} is present in this example.

\ea%23
    \label{ex:tungu:23}
    \ili{Oroqen}\\
    \gll yabu-ʃa \textbf{{haʃi}} \textbf{{yə}}{-ʃa?}\\
    walk-\textsc{pst}  or  what-\textsc{pst}\\
    \glt ‘Did you go or what?’ (\citealt{LiFengxiang2005}: 56)
    \z

\largerpage[2]
A slightly different picture can be drawn for the Xunke dialect of \ili{Oroqen}, which has a large amount of \isi{question marker}s (\citealt{ZhangLiZhang1989}: passim). The enclitic \textit{=j\textsc{e}} marks polar and, optionally, content \isi{questions}, which makes a connection to \ili{Dagur} clear. One of their examples given is an \isi{alternative question} that contains the two markers \textit{=jɔ} and \textit{=jə}. These must be vowel harmonic variants of \textit{=j\textsc{e}}. Thus, the enclitic is even more similar to some subdialects of \ili{Dagur} that also exhibit \isi{vowel harmony} in this form. Xunke \ili{Oroqen} has likewise borrowed the markers \textit{=ɔɔ} (expressing doubt) from \ili{Mongolian} \textit{=UU}, and perhaps \textit{baa} {\textasciitilde} \textit{bəə} from \ili{Chinese} \textit{ba} \zh{吧}. Alternative \isi{questions} may either be marked twice with one of the two markers \textit{ɔɔmal} and \textit{jɔɔma} or may take a \isi{disjunction} \textit{aaki} that may either stand alone or may be combined with other question markers. The origin of \textit{ɔɔmal} is unclear but possibly may be treated as a variant of \textit{jɔɔma}. Furthermore, there is a \isi{tag question} marker \textit{unti}, which looks somewhat similar to the negative copula in \ili{Solon} and \ili{Oroqen} that developed out of an adjective meaning ‘different’ (\citealt{Hölzl2015a}). However, in Xunke \ili{Oroqen}, the forms are \textit{oŋto} ‘\textsc{neg}’ and \textit{wʊntʊ} ‘different’ (\citealt{ZhangLiZhang1989}: 183).

\ea%24
    \label{ex:tungu:24}
    \ili{Oroqen} (Xunke)\\
    \ea
    \gll nɔɔnin  tɔrɔki=\textbf{{jɔ}} waa-tɕa \textbf{{aaki}} gujtɕən=\textbf{{jə}}?\\
    3\textsc{sg}  boar=\textsc{q}    kill-\textsc{pst}  or  roe.deer=\textsc{q}\\
    \glt ‘Did (s)he kill a boar or a roe deer?’
\newpage    
    \ex
    \gll ɕii \textbf{{i.rə}} ŋənə-ni=\textbf{{j}}\textbf{\textsc{e}}?\\
    2\textsc{sg}  where  go-2\textsc{sg}=\textsc{q}\\
    \glt ‘Where are you going?’
    
    \ex
    \gll bii    umun.nə  ŋənə-tɕə-w, \textbf{{unti}}?\\
    1\textsc{sg}.\textsc{nom}  once    go-\textsc{pst}-1\textsc{sg}  right\\
    \glt ‘I have been there once, right?’ (\citealt{ZhangLiZhang1989}: 123, 126, 131)\z\z

In \textbf{Huihe Solon}, there is an enclitic \textit{=gi(i)}, which is accompanied by an additional rising \isi{intonation}. It marks polar and \isi{alternative question}s.

\ea%25
    \label{ex:tungu:25}
    \ili{Solon} (Huihe)\\
    \ea
    \gll eri  üxür  aya=\textbf{{gii}}?\\
    this  ox  good=\textsc{q}\\
    \glt ‘Is this ox good?’ \citep[7]{Tsumagari2009a}
    
    \ex
    \gll ʃi.n-i    bəj-ʃi aja=\textbf{{gi}}, ərʉ=\textbf{{gi}}?\\
    2\textsc{sg}.\textsc{obl}-\textsc{gen}  body-2\textsc{sg.poss}  good=\textsc{q}  bad=\textsc{q}\\
    \glt ‘Are you well (or sick)?’ \citep[316]{Chaoke2009}
    \z
    \z

Despite functional, formal, and distributional similarities, \ili{Solon} \textit{=gi(i)} and \ili{Evenki} \textit{=Ku} are probably not direct cognates of each other because there is no sound law that would justify the different vowel qualities (e.g., \citealt{Benzing1956,Doerfer1978b}). Maybe it is a loan from a \ili{Mongolic} language, e.g. \ili{Buryat} \textit{=gü} (\sectref{sec:5.8.2}). Content \isi{questions} usually do not show any marking.

\ea%26
    \label{ex:tungu:26}
    \ili{Solon} (Huihe)\\
    \gll sii \textbf{{ii}}\textbf{.}\textbf{{lee}} tegeji-ndi?\\
    2\textsc{sg}  where  live-\textsc{prs}.2\textsc{sg}\\
    \glt ‘Where do you live?’ \citep[15]{Tsumagari2009a}
    \z

Like \ili{Oroqen}, \ili{Solon} also has a marker \textit{baa} with a long vowel that must derive from \ili{Chinese} \textit{ba} \zh{吧} and a form \textit{yeeme} that, similar to \ili{Khorchin} \ili{Mongolian} \textit{jimɛɛ}, can also appear in \isi{content question}s (\sectref{sec:5.8.2}).

\ea%27
    \label{ex:tungu:27}
    \ili{Solon} (Huihe)\\
    \ea
    \gll ta.ri  üli-see \textbf{{baa}}?\\
    3\textsc{sg}  go-\textsc{pst}    \textsc{q}\\
    \glt ‘He went, didn’t he?’
    
    \ex
    \gll eri si \textbf{{oxon}} \textbf{{yeeme}}?\\
    this  \textsc{cop}  what  \textsc{q}\\
    \glt ‘What is this?’ \citep[15]{Tsumagari2009a}
    \z
    \z

\clearpage    
As in \ili{Oroqen}, \isi{alternative question}s appear to preserve a cognate of \ili{Evenki} \textit{=Ku}. Consider the following \isi{negative alternative question}.

\ea%28
    \label{ex:tungu:28}
    \ili{Solon} (?Huihe)\\
    \gll ʃii  mʊrın-ʃı    bəjə=\textbf{{guu}}, \textbf{{aaʃın}}{=}\textbf{{gʊʊ}}?\\
    2\textsc{sg}  horse-2\textsc{sg.poss}  man=\textsc{q}    \textsc{neg}=\textsc{q}\\
    \glt ‘Do you have horses or not?’ (\citealt{HuZengyiChaoke1986})
    \z

There is limited information on other dialects of \ili{Solon}, especially the \textbf{Ongkor} dialect formerly spoken in \isi{Xinjiang}. However, there apparently were morphosyntactically unmarked \isi{questions} that probably had a special intonational contour, e.g. \textit{śi mandii?} ‘Are you strong?’ \citep[11]{Aalto1979} In addition, there are two forms \textit{=ii} and \textit{=uu}, both of which are probably loans from \ili{Mongolian} \textit{=(y)}\textit{ii {\textasciitilde}} \textit{=(y)}\textit{UU} (\sectref{sec:5.8.2}). In \ili{Even}, \ili{Negidal}, and \ili{Uilta} there are markers similar to \textit{=ii} (see below). Content \isi{questions} remain unmarked.

\ea%29
    \label{ex:tungu:29}
    \ea
    \ili{Solon} (Ongkor)\\  
    \gll ə̬r  uktu  ulu-r    uktu=\textbf{{ii}}?\\
    this  road  walk-\textsc{ptcp}  road=\textsc{q}\\
    \glt ‘Is this road the road (usually) travelled?’
    
    \ex
    \gll baxuu-dže=\textbf{{uu}} \textbf{{e}}{-dže=}\textbf{{uu}}?\\
    find-\textsc{pst}=\textsc{q}    \textsc{neg}-\textsc{pst}=\textsc{q}\\
    \glt ‘Was it found or not?’
    
    \ex
    \gll \textbf{{jam}} iśi-ndii?\\
    which.one  see-\textsc{prs}.2\textsc{sg}\\
    \glt ‘What do you see?’ (\citealt{Aalto1979}: 8, 9, modified transcription)
    \z\z

\noindent The \isi{interrogative} \textit{jam} in (\ref{ex:tungu:29}c) is probably a loan from \ili{Jurchenic} that can also be seen in Nonni \ili{Solon} as \textit{jemu} (\ref{ex:tungu:30}b, see \sectref{sec:5.10.3}). There is even less information on the \textit{Nonni} dialect of \ili{Solon}. Nevertheless, at least some examples have been collected by \cite{Ivanovskij1982}). One dubious example of a negative \isi{alternative question} apparently relies on \isi{juxtaposition}. Several \isi{content question}s remained unmarked as well, and an optional polar \isi{question marker} has the form \textit{=gi}.

\ea%30
    \label{ex:tungu:30}
    \ili{Solon} (Nonni)\\
    \ea
    \gll ši.n-i    enin  šamine    žu-de    bi-si-ɳ=\textbf{{gi}}?\\
    2\textsc{sg}.\textsc{obl}-\textsc{gen}  mother  ?father    house-\textsc{loc}  \textsc{cop}-\textsc{prs}-3=\textsc{q}\\
    \glt ‘Are your parents at home (still alive)?’
    
    \ex
    \gll \textbf{{jemu}} gerbi-či?\\
    which  name-2\textsc{sg}.\textsc{poss}\\
    \glt ‘\isi{What is your name?}’ (\citealt{Ivanovskij1982}: 1)
    \z
    \z

Except for \ili{Oroqen}, \textbf{Negidal} is probably the most aberrant \ili{Ewenic} language with respect to \isi{question marking}. At first glance, the situation is similar to \ili{Evenki} as there is a marker that is cognate with \textit{=Ku}. Note the absence of the consonant in the form \textit{=ʊʊ}, which, like Ongkor \ili{Solon} \textit{=uu}, is quite similar to \ili{Mongolian}. There are also unmarked \isi{polar question}s that probably have an \isi{intonation} similar to \ili{Evenki}.

\ea%31
    \label{ex:tungu:31}
    \ili{Negidal}\\
    \ea
    \gll noŋan    naa.bəjə-ni=\textbf{{ŋuu}},  naa-nɪ=\textbf{{ʊʊ}}?\\
    3\textsc{sg}    \textsc{pn}-3\textsc{sg.poss}=\textsc{q}  \textsc{pn}-3\textsc{sg.poss}=\textsc{q}\\
    \glt ‘Is he a \ili{Negidal} or a \ili{Nanai}?’
    
    \ex
    \gll sii  saa-s?\\
    2\textsc{sg}  know-2\textsc{sg}\\
    \glt ‘Do you know?’ (\citealt{Kazama2002a}: 80, 65)
    \z
    \z

But \citet[10]{KhasanovaPevnov2003} mention a morphological marking of \isi{questions} in \ili{Negidal} as in the following example. Incidentally, the example also contains a further \isi{question marker} \textit{=i} (cf. example \ref{ex:tungu:31}a from \ili{Even} above).

\ea%32
    \label{ex:tungu:32}
    \ili{Negidal}\\
    \gll ii-\textbf{{ǰə}}{-}\textbf{{m}}{=}\textbf{{i}}?\\
    enter-\textsc{fut}.\textsc{q}-1\textsc{sg}.\textsc{q}=\textsc{q}\\
    \glt ‘Shall I come in?’ \citep[127]{Kazama2002a}
    \z

According to them, the \isi{interrogative} future differs from the general future in two points. First, the \isi{interrogative} future has a short vowel as opposed to the plain future. Second, a different personal ending is employed (e.g., 1\textsc{sg} \textit{-m} instead of \textit{-v}). Compare the following pair of sentences:

\ea%33
    \label{ex:tungu:33}
    \ili{Negidal}\\
    \ea
    \gll \textbf{{e}}\textbf{{eva}} iche-\textbf{{ǯa}}{-}\textbf{{m}}?\\
    what    see-\textsc{fut}.\textsc{q}-1\textsc{sg.q}\\
    \glt ‘What will I see?’
    
    \ex
    \gll oǯ{a-va    iche-}\textbf{{ǯ}}\textbf{{ee}}{-}\textbf{{v}}.\\
    track-\textsc{acc}  see-\textsc{fut}-1\textsc{sg}\\
    \glt ‘I will see the tracks.’ (\citealt{KhasanovaPevnov2003}: 10)
    \z
    \z

The morphological \isi{interrogative} marking is found in polar, content, as well as in \isi{alternative question}s and can combine with \isi{interrogative} enclitics. Consider the following \isi{open alternative question} with both morphological and enclitic markers.

\ea%34
    \label{ex:tungu:34}
    \ili{Negidal}\\
    \gll mozhet  bolotkı    bi-\textbf{{ǰə}}{-}\textbf{{m}}{=}\textbf{{ŋu}} \textbf{{ee}}{-}\textbf{{ǰa}}{-}\textbf{{m}}{=}\textbf{{ŋu}}?\\
    maybe    autumn    be-\textsc{fut}.\textsc{q}-1\textsc{sg}=\textsc{q}  what-\textsc{fut}.\textsc{q}-1\textsc{sg}=\textsc{q}\\
    \glt ‘Is it perhaps already autumn or what?’ \citep[114]{Kazama2002a}
    \z

Previous descriptions of \ili{Negidal} apparently did not mention this interesting feature (see \citealt{Kazama2002a}: 107, 114, 115). According to \citet{Ikegami1985}, in \ili{Tungusic} languages there are generally two different sets of personal endings (\tabref{tab:tungu:1}). In \ili{Negidal}, Set 1 is used after past forms in \textit{-čaa} as well as future forms in \textit{-ǰa(-ŋaa)} and also has a possessive function with nouns. Set 2, on the other hand, can be found after present stems in \textit{-ja} or underived stems. \citet[91]{Ikegami1985} also notes that, according to Kolesnikova \& Konstantinova, the future ending \textit{-ǰa} takes the first person inclusive marker \textit{-p} instead of \textit{-t}. This might indicate a confusion resulting from the \isi{interrogative} marking and may show that \citegen{KhasanovaPevnov2003} assumptions are correct.

\begin{table}
\caption{Personal endings in Negidal according to \cite[88f.]{Ikegami1985}, from Cincius, adjusted}
\label{tab:tungu:1}

\begin{tabularx}{\textwidth}{XXl}
\lsptoprule
& \textbf{Set 1} & \textbf{Set 2}\\
\midrule
1\textsc{sg} & \textbf{\textit{-v}} & \textbf{\textit{-m}}\\
2\textsc{sg} & \textit{-s} & \textit{-s}\\
3\textsc{sg} & \textit{-n} & \textit{-n}\\
1\textsc{pl.incl} & \textbf{\textit{-t}} & \textbf{\textit{-p}}\\
1\textsc{pl.excl} & \textit{-vun} & \textit{-vun}\\
2\textsc{pl} & \textit{-sun} & \textit{-sun}\\
3\textsc{pl} & \textbf{\textit{-tin}} & \textit{-}\\
\lspbottomrule
\end{tabularx}
\end{table}

Accordingly, Set 2 would additionally be used in interrogatives, while Set 1 is found in declarative sentences. There does not appear to be any further description of this phenomenon for \ili{Negidal}, or for any other \ili{Tungusic} language for that matter. However, a possible areal connection can be found in \ili{Yukaghiric} (\sectref{sec:5.14.2}). As in \ili{Negidal}, the \ili{Yukaghiric} interrogative suffixes are restricted to the first person (\isi{singular} \textit{-m}, \isi{plural} -\textstyleStrong{{\textit{uok {\textasciitilde} -ook}}}). But the connection to \ili{Yukaghiric} is not without its problems. First of all, \ili{Yukaghir} languages are spoken several thousand kilometers north of \ili{Negidal} and in \ili{Yukaghiric} the special interrogative suffixes are only found in \isi{content question}s. Furthermore, \ili{Yukaghiric} lacks any special interrogative tense markers. But as specified in \sectref{sec:2.14} we may assume that \ili{Yukaghiric} was once spoken in a much larger territory and that its speaker probably migrated northward along the \isi{Lena} river from an earlier location close to Lake \isi{Baikal}, which reduces the distance to the \ili{Negidal}. But even if the areal connection turns out to be wrong, we are dealing with an interesting typological parallel in which \isi{interrogative} agreement marking is mostly restricted to the first person and the third person \isi{plural} remains unmarked.

In \textbf{Udihe} polar \isi{questions} can be marked by \isi{intonation} only, which is said to be higher and somewhat longer than that of declarative sentences (\citealt{NikolaevaTolskaya2001}: 807). An element may be moved to a \isi{focus} position, typically in front of the verb, which is different from \ili{Evenki} as seen above (\citealt{NikolaevaTolskaya2001}: 841).

\newpage 
\ea%35
    \label{ex:tungu:35}
   \ili{Udihe}\\
    \gll \ulp{uti}{25} \ule{nii}   ŋənəə-ni  bi  bagdəə-mi  bua-la?\\
    that  man  go.\textsc{pst}-3\textsc{sg}  1\textsc{sg}.\textsc{nom}  born.\textsc{pst}-1\textsc{sg}  place-\textsc{loc}\\
    \glt ‘Has \textit{he} (the man) gone to my birthplace.’ \citep[41]{Girfanova2002}
    \z

An alternative is the use of an enclitic =\textit{nu} {\textasciitilde} \textit{=gu}, cognate of \ili{Evenki} \textit{=Ku}, that attaches to the verb in \isi{polar question}s and to the the element in \isi{focus} in \isi{focus question}s. As in \ili{Evenki}, the scope of the marker also encompasses \isi{alternative question}s.

\ea%36
    \label{ex:tungu:36}
    \ili{Udihe}\\
    \ea
    \gll Iwana  zugdi-du  bi-s’e=\textbf{{nu}}?\\
    Ivan  house-\textsc{loc}  \textsc{cop}-\textsc{pfv}=\textsc{q}\\
    \glt ‘Is Ivan at home?’
    
    \ex
    \gll si  bagä{ä-za=}\textbf{{nu}} bi-s’e-i?\\
    2\textsc{sg}  other.side-\textsc{n}-\textsc{q}  \textsc{cop}-\textsc{pfv}-2\textsc{sg}\\
    \glt ‘Were you on the other side of the river?’
    
    \ex
    \gll xeleba    bie=\textbf{{nu}} \textbf{{anči}}{=}\textbf{{nu}}?\\
    bread    be.\textsc{prs}.\textsc{hab}=\textsc{q}  \textsc{neg}=\textsc{q}\\
    \glt ‘Is there bread or not?’ (\citealt{NikolaevaTolskaya2001}: 809, 812)\z\z

Content \isi{questions} do not normally take any morphosyntactic marking.

\ea%37
    \label{ex:tungu:37}
    \ili{Udihe}\\
    \gll \textbf{{j’e}}-le    ñansule-i?\\
    which-\textsc{loc}  study-2\textsc{sg}\\
    \glt ‘Where do you study?’ (\citealt{NikolaevaTolskaya2001}: 801)
    \z

The \isi{semantic scope} of \textit{=nu} is thus identical to \ili{Evenki}, but \ili{Udihe} has a further enclitic \textit{=nA} that has a contrastive function. It remains dubious whether this form has any connection with \ili{Manchu} \textit{=nA} (see below).

\ea%38
    \label{ex:tungu:38}
    \ili{Udihe}\\
    \gll xuda=\textbf{{na}}?\\
    fur=\textsc{q}\\
    \glt ‘And what about the \textit{fur}?’ (\citealt{NikolaevaTolskaya2001}: 808)
    \z

The enclitic is also often used together with an \isi{interrogative} word. Within the following question the contrastive \isi{focus} lies not on the \isi{interrogative} but on the river.

\ea%39
    \label{ex:tungu:39}
    \ili{Udihe}\\
    \gll ei=\textbf{{ne}} \textbf{{j’e}}\textbf{.}\textbf{{u}} bäsa-ni?\\
    this=\textsc{q}    what  river-3\textsc{sg}\\
    \glt ‘And what is \textit{this} river called?’ (\citealt{NikolaevaTolskaya2001}: 808)
    \z

\newpage 
Alternative \isi{questions} may also be formed with \textit{-(e)s(i)} of unknown origin. In example (\ref{ex:tungu:40}a) a \isi{content question} is followed by an \isi{alternative question} (\sectref{sec:4.4}).

\ea%40
    \label{ex:tungu:40}
    \ili{Udihe}\\
    \ea
    \gll \textbf{{j’e}}-we    xokto-ni,    käŋaa-\textbf{{s}} ogböö-\textbf{{s}}?\\
    what-\textsc{acc}  footstep-3\textsc{sg}.\textsc{poss}  deer-\textsc{q}    elk-\textsc{q}\\
    \glt ‘Whose footsteps are these, a deer’s or an elk’s?’
    
    \ex
    \gll iwana  sin-du    kusige-we  bu-ge, \textbf{{e}}{-si-ni} bu-oo-\textbf{{s(i)}}?\\
    \textsc{pn}  2\textsc{sg}.\textsc{obl}-\textsc{dat}  knife-\textsc{acc}  give-\textsc{pfv}  neg-\textsc{pst}-3\textsc{sg} give-\textsc{pst}-\textsc{q}\\
    \glt ‘Has Ivan given you the knife or not?’ (\citealt{NikolaevaTolskaya2001}: 811, 255)
    \z
    \z

This latter construction is probably not a \isi{tag question} construction but an \isi{alternative question} with a \isi{question marker} on the second alternative only, which is also attested for \ili{Kilen} and \ili{Manchu}.

According to \citet[351]{NikolaevaTolskaya2001}, there are tag \isi{questions} that are formed with the help of the \isi{interrogative} \textit{j’e.u} ‘what’, which may be attributed to \ili{Russian} influence (cf. \sectref{sec:5.5.2.2}).

\ea%41
    \label{ex:tungu:41}
    \ili{Udihe}\\
    \gll em’e-i, \textbf{{j’e.u}}?\\
    come.\textsc{pfv}-2\textsc{sg}  what\\
    \glt ‘You came, didn’t you?’ (\citealt{NikolaevaTolskaya2001}: 351)
    \z

\ili{Udihe} and \ili{Oroch} have a very interesting open \isi{alternative question} construction in which the second alternative is an inflected \isi{interrogative verb}. This pattern has been adopted by \ili{Kilen} from \ili{Udihe}. We have already observed a similar construction in \ili{Oroqen} \REF{ex:tungu:23}, but with the \ili{Chinese} \isi{disjunction} instead of \isi{double marking}.

\ea%42
    \label{ex:tungu:42}
    \ili{Udihe}\\
    \gll su  xulisee-u=\textbf{{nu}} \textbf{{jaa}}{-u=}\textbf{{nu}}?\\
    2\textsc{pl}  go.\textsc{pst}-2\textsc{pl}=\textsc{q}    what.\textsc-2\textsc{pl}=\textsc{q}\\
    \glt ‘Did you go, or what?’ (\citealt{NikolaevaTolskaya2001}: 811)
    \z

\ea%43
    \label{ex:tungu:43}
   \ili{Kilen}\\
    \gll su  ənə-xəi=\textbf{{nə}} \textbf{{ja}}{-xəi=}\textbf{{nə}}?\\
    2\textsc{pl}  go-\textsc{perf}=\textsc{q}    what-\textsc{perf}=\textsc{q}\\
    \glt ‘Did you go or what?’ (\citealt{ZhangPaiyu2013}: 158)
    \z

\ea%44
    \label{ex:tungu:44}
    \ili{Oroch}\\
    \gll agduči-za-i=\textbf{{nu}} \textbf{{jaa}}{-za-i=}\textbf{{nu}}?\\
    tell-\textsc{fut}-1\textsc{sg}=\textsc{q}    what-\textsc{fut}-1\textsc{sg}=\textsc{q}\\
    \glt ‘Should I tell or what?’ (\citealt{TolskayaTolskaya2008}: 98, from Avrorin)
    \z

Example \REF{ex:tungu:44} from \textbf{Oroch} shows the \isi{interrogative} enclitic which, similar to \ili{Nanai} and \ili{Udihe}, has the form \textit{=nu} and is optional in \isi{polar question}s. Content \isi{questions} remain unmarked.

\ea%45
    \label{ex:tungu:45}
    \ili{Oroch}\\
    \ea
    \gll sii  čihala-i-si?\\
    2\textsc{sg}  agree-\textsc{prs}-2\textsc{sg}\\
    \glt ‘Do you agree?’
    
    \ex
    \gll sii \textbf{{jaa}}-va    xuana-i-si?\\
    2\textsc{sg}  what-acc  argue-\textsc{prs}-2\textsc{sg}\\
    \glt ‘What do you argue about?’
    
    \ex
    \gll sii  amba    bi-si=\textbf{{nu}}?\\
    2\textsc{sg}  evil.spirit  \textsc{cop}-\textsc{prs}=\textsc{q}\\
    \glt ‘Are you an evil spirit (the devil)?’ (\citealt{AvrorinBoldyrev2001}: 184)\z\z

In 1958, a team of unknown scientists from \isi{China} gave a handful of comparative word lists for five \ili{Tungusic} languages in \isi{China}. Their list also contains two sentences that can be translated as ‘when do you come back?’ (\zh{多怎回来}) and ‘where do you go?’ (\citealt{NDSSLD1958}: 82). Unfortunately, they transcribed all languages with the help of \ili{Chinese} characters, which makes the \isi{analysis} less easy. Additionally, some characters were written incorrectly. The following gives the corrected sentences in \ili{Chinese} transcription and its rendering in official Pinyin spelling followed by a rough approximation of the original languages. The transcription, \isi{analysis} and \isi{glossing} is mine. Interestingly enough, the set of languages is not completely identical to the five officially recognized languages today. There are no sentences from \ili{Sibe}, but from \textbf{Hezhen} (\textit{hèzh\=en} \zh{赫真}) which refers to the dialect of \ili{Nanai} spoken in \isi{China}. \ili{Hezhen} is not very well known (cf. \citealt{AnJun1986}: 79–86) and probably extinct by now, while \textit{Kilen} (\textit{qíléng} \zh{奇楞}) has been described in several grammatical sketches. The \ili{Hezhen} data are thus potentially very important. Both \ili{Hezhen} and \ili{Kilen} are classified together as the Hezhe (\textit{hèzhé} \zh{赫哲}) language and are treated as dialects by the authors of \citet{NDSSLD1958}. Of the five languages only \ili{Hezhen} and \ili{Kilen} are included here for the sake of brevity.

\ea%46
    \label{ex:tungu:46}
    \ili{Hezhen}\\
    \ea
    \zh{哈利吉朱衣西} [ha li ji zhu yi xi]\\
    \gll \textbf{{xali}} dʑi-dʑu-i-ɕi?\\
    when  come-\textsc{regr-prs-2sg}\\
    \glt ‘When do you come back?’
    
    \ex
    \zh{好西額奴衣西} [hao xi e nu yi xi]\\
    \gll \textbf{{xaoɕi}} enə-i-ɕi?\\
    where.to  go-\textsc{prs-2sg}\\
    \glt ‘Where do you go?’ (\citealt{NDSSLD1958}: 82)
    \z
    \z

\ea%47
    \label{ex:tungu:47}
    \ili{Kilen}\\
    \ea
    \zh{阿黑額莫土衣西} [a hei\footnote{The character \textit{hei} \zh{黑} should instead read \textit{li} \zh{里}.} e mo tu yi xi]\\
    \gll \textbf{{ali}} emə-tu-i-ɕi?\\
    when  come-\textsc{regr-prs-2sg}\\
    \glt ‘When do you come back?’
    
    \ex
    \zh{鴨勒額訥衣西} [ya le e ne yi xi]\\
    \gll \textbf{{ya}}-le    ene-i-ɕi?\\
    which-\textsc{all}  go-\textsc{prs-2sg}\\
    \glt ‘Where do you go?’ (\citealt{NDSSLD1958}: 82)
    \z
    \z

In both languages \isi{content question}s do not take any morphosyntactic marking. As will be further explained in \sectref{sec:5.10.3}, \ili{Kilen} interrogatives exhibit affinities with \ili{Udegheic}, which explains the absence of the initial consonant in \textit{ali}, as opposed to \ili{Hezhen} \textit{hali} ‘when’, and the \isi{interrogative} \textit{yale}, instead of \ili{Hezhen} \textit{xaosi} ‘whither’ (\ili{Udihe} \textit{ali}, \textit{j’ele}, \ili{Nanai} \textit{xaali}, \textit{xaosi}).\footnote{Both \ili{Hezhen} and \ili{Kilen} show characteristics that suggest a basic connection to \ili{Nanai}, e.g. the absence of an initial consonant in \textit{ene-} ‘to go’ (\ili{Nanai} \textit{ənə-}, \ili{Udihe} \textit{ŋene-}, \ili{Manchu} \textit{gene-}). The verb \textit{emə-} ‘to come’ in \ili{Kilen} was most likely borrowed from \ili{Udihe} (\ili{Nanai} \textit{\.{ɟ}i-}, \ili{Udihe} \textit{eme-}, \ili{Manchu} \textit{ji-}).} \citet[241]{Schmidt1928b} mentions the \ili{Samar} sentence \textit{xajadži džidžisi?} ‘Where did you come from?’ \ili{Samar} is not very well-known, but is clearly very similar to \ili{Nanai} as well (e.g., \ili{Nanai} \textit{xajaǰi} ‘whence’).

There are several descriptions of \ili{Kilen} that differ more or less strongly from each other. According to \citet[157f.]{ZhangPaiyu2013}, \textbf{Kilen} expresses \isi{polar question}s with rising \isi{intonation} on the last word of the sentence, e.g. \textit{ɕi} \textit{sa?} ‘Do you know?’. However, \ili{Kilen} was heavily influenced by \ili{Chinese}, in fact, \ili{Chinese} may by now have replaced \ili{Kilen} completely, leaving \ili{Kilen} extinct. Following to \citet[158]{ZhangPaiyu2013}, \ili{Kilen} borrowed the three \isi{interrogative} particles \textit{ba} \zh{吧}, \textit{ma} \zh{吗}, and \textit{(y)a} \zh{啊}/\zh{呀}, all of which are possible in the sentence above, e.g. \textit{ɕi} \textit{sa=}\textbf{\textit{a}}? ‘Do you really know?’. Most likely, \textit{=a} is not of \ili{Chinese} origin, however. Several examples of polar \isi{questions} in \citet{ZhangZhangDai1989} were either unmarked (showing rising \isi{intonation}) or marked with the final \isi{question marker} \textit{=a}. Note that it never followed anything but the second person \isi{singular} agreement form \textit{-ɕi} and was always written attached to it. Nevertheless, it is better analyzed as enclitic =\textit{a} that may appear in both polar and \isi{content question}s, which might speak instead in favor of a connection with \ili{Manchu} \textit{=o}.

\ea%48
    \label{ex:tungu:48}
    \ili{Kilen}\\
    \ea
    \gll ɕi  arki-wə  ɔmi-mi    bi-ɕi=\textbf{{a}}?\\
    2\textsc{sg}  alcohol-\textsc{acc}  drink-\textsc{cvb}  \textsc{cop}-2\textsc{sg=q}\\
    \glt ‘Are you drinking alcohol?’
    
    \ex
    \gll \textbf{{ia}}-lə    ənə-ji-ɕi=\textbf{{a}}?\\
    which-\textsc{all}  go-\textsc{prs}-2\textsc{sg}=\textsc{q}\\
    \glt ‘Where are you going?’ (\citealt{ZhangZhangDai1989}: 87f.)
    \z
    \z

Note the absence of the marker in the otherwise identical sentence (47b) above. \citet{AnJun1986} already mentions examples with the \ili{Chinese} enclitic \textit{ba} \zh{吧}. In his data, \textit{a} is not written attached to the preceding word and can also follow forms other than the second person \isi{singular}.

\ea%49
    \label{ex:tungu:49}
    \ili{Kilen}\\
    \gll ɕi.n-i    agə-ɕi      biχan    fuli-m    ən-$\chi ə$-ni=\textbf{{a}}?\\
    2\textsc{sg}.\textsc{obl}-\textsc{gen}  e.brother-2\textsc{sg}.\textsc{poss}  wilderness  hunt-\textsc{cvb}  go-\textsc{pst}-3\textsc{sg}=\textsc{q}\\
    \glt ‘Did your elder brother go to hunt?’ (\citealt{AnJun1986}: 36)
    \z

In \isi{alternative question}s the \ili{Chinese} \isi{interrogative} disjunctive \textit{háishì} \zh{还是} ‘or.\textsc{q}’ may be employed but is combined with the marker \textit{=a}.

\ea%50
    \label{ex:tungu:50}
    \ili{Kilen}\\
    \gll ɕi  əi-wə \textbf{{xəɕi}} ti-wə    gələ-ji-ɕi=\textbf{{a}}?\\
    2\textsc{sg}  this-\textsc{acc}  or.\textsc{q}  that-\textsc{acc}  ?want-\textsc{prs}-2\textsc{sg}=\textsc{q}\\
    \glt ‘Do you want this or that?’ (\citealt{ZhangZhangDai1989}: 45, simplified)
    \z

Unlike other \isi{alternative question} constructions among \ili{Tungusic} languages, the \isi{question marker} appears only once and does not attach to the elements in \isi{focus}.

A further enclitic called a “contrastive particle” by \citet[159]{ZhangPaiyu2013} seems to have been borrowed from \ili{Udihe} and marks polar and \isi{focus question}s.

\ea%51
    \label{ex:tungu:51}
    \ili{Kilen}\\
    \ea
    \gll ɕi adɔqɔli\textbf{{=nə}}? \\
    2\textsc{sg}    cold=\textsc{q}\\
    \glt ‘Are you cold (or not)?’
    
    \ex
    \gll suɾsaɾə\textbf{{=nə}} talaxa?\\
    tasty=\textsc{q}  grilled.fish\\
    \glt ‘Is the grilled fish tasty (or not)?’ (\citealt{ZhangPaiyu2013}: 159)
    \z
    \z

As seen in example \REF{ex:tungu:69} above, it also marks \isi{alternative question}s. Most likely it has been borrowed from \ili{Udihe} \textit{=nu}, but it exhibits certain similarities to \ili{Udihe} \textit{=nA} as well.

\textbf{Nanai} is the best described language from the \ili{Nanaic} branch and there is even a good description of question \isi{intonation} by \citet[294]{Baitchura1979} (underlining removed).

\begin{quote}
In general \isi{questions}, the tone movement in the vowel of the final syllable has a clearly and strongly manifested rising character, whereas the mean and the maximal tone heights surpass those of the preceding vowels in cases in which no \isi{interrogative} particle is present in the sentence. If there is such a particle (e.g., nu), the rise of the tone at the end of the sentence is not so high, its pitch being a little lower in comparison to the tone heights of vowels at the beginning of the sentence.
\end{quote}

There are sentences with and without the enclitic. In 1858, Venjukov recorded a \isi{polar question} without enclitic among the \isi{Ussuri} \ili{Nanai}.

\ea%52
    \label{ex:tungu:52}
    \ili{Nanai} (\isi{Ussuri})\\
    \gll anda, duman bira  goró?\\
    friend  \textsc{pn}    river  far\\
    \glt ‘Friend, is the Duman river far away?’ (\citealt{Alonso2011}: 14, from Venjukov)
    \z

Similar to \ili{Evenki} or \ili{Udihe}, the enclitic \textit{=nu} in \ili{Nanai} does not appear in \isi{content question}s (which remain unmarked), but marks more than one \isi{question type}, including polar, alternative, and possibly \isi{focus question}s.

\ea%53
    \label{ex:tungu:53}
    \ili{Nanai} (Najkhin)\\
    \ea
    \gll swə aja-so=\textbf{{nu}}?\\
    2\textsc{pl}  good-2\textsc{pl}.\textsc{poss}=\textsc{q}\\
    \glt ‘Are you all well?’
    
    \ex
    \gll swə sogdata-wa  wanda-mari  maŋbo-ci=\textbf{{nu}} namo-ci=\textbf{{nu}} ənə-i-su?\\
    2\textsc{pl} fish-\textsc{acc} take-\textsc{cvb.pl}  river-\textsc{all}=\textsc{q}  sea-\textsc{all}=\textsc{q} go-\textsc{prs}-2\textsc{pl}\\
    \glt ‘Do you go to the river or sea to catch fish?’
    
    \ex
    \gll si \textbf{{xali}} əusi  \.{ɟ}i-ci-si?\\
    2\textsc{sg}  when  here  come-\textsc{pst}-2\textsc{sg}\\
    \glt ‘When did you come here?’ (\citealt{KoYurn2011}: 155, 68, 52)\z\z

\noindent Regarding the last sentence compare example \REF{ex:tungu:74} from \ili{Hezhen} above.

In \textbf{Ulcha} the enclitic marks \isi{focus}, alternative, and (optionally) \isi{content question}s. Such an extension of scope can also be observed in \ili{Mongolian} (\sectref{sec:5.8.2}).

\ea%54
    \label{ex:tungu:54}
    \ili{Ulcha}\\
    \ea
    \gll lʊča  gasa=\textbf{{nʊʊ}}?\\
    \textsc{pn}  village=\textsc{q}\\
    \glt ‘(Is it) a \ili{Russian} village?’
    
    \ex
    \gll \textbf{{xasu}} aňan=\textbf{{nʊʊ}},  nadan  aňan=\textbf{{nʊʊ}} ňuŋgun    aňan=\textbf{{nʊʊ}} bi-či-ni?\\
    how.many  year=\textsc{q} seven  year=\textsc{q} six    year=\textsc{q}    \textsc{cop}-\textsc{pst}-?3\textsc{sg}\\
    \glt ‘How many years has it been, seven years (or) six years?’ (\citealt{Kazama2002b}: 79, 86)
    
    \ex
    \gll saaŋxai, \textbf{{xai.mi}} soŋg-i?\\
    \textsc{pn}    why    cry-?\textsc{prs}\\
    \glt ‘Sanghai, why are you crying?’ (\citealt{Schmidt1923b}: 235f.)\z\z

Within \ili{Nanaic}, \textbf{Uilta} has the most interesting marking of \isi{questions}. Polar \isi{questions} in \ili{Uilta} have both rising \isi{intonation} and an interrogative clitic \textit{=(y)i} that might be related to the one found in \ili{Even}, \ili{Negidal}, and Ongkor \ili{Solon}, although these were perhaps borrowed from \ili{Mongolic}. It always follows the verb (Patryk Czerwinski p.c. 2018). In addition, there is a specialized marker \textit{=ga} {\textasciitilde} \textit{=ka} for \isi{content question}s that cannot be found in any other \ili{Tungusic} language.

\ea%55
    \label{ex:tungu:55}
    \ili{Uilta} (Southern)\\
    \ea
    \gll eri aya ulaa=\textbf{{yi}}?\\
    this  good  reindeer=\textsc{q}\\
    \glt ‘Is this a good reindeer?’
    
    \ex
    \gll \textbf{{ŋui}}  sinda-xa-ni\textbf{{=ga}}?\\
    who  come-\textsc{pst}-3\textsc{sg}=\textsc{q}\\
    \glt ‘Who came?’ \citep[15]{Tsumagari2009b}
    \z
    \z

Apparently, the marker in \isi{content question}s is not obligatory as there are also several examples without it.

\ea%56
    \label{ex:tungu:56}
    \ili{Uilta} (Northern)\\
    \gll \textbf{{khoni}} bi-si    sii?\\
    how  \textsc{cop}-\textsc{prs}  2\textsc{sg}\\
    \glt ‘How are you?’\footnote{Regarding the use of the \isi{interrogative}, cf. \ili{Russian} \textit{kak dela?}/\textcyrillic{как дела?}} \citep[150]{Funk2000}
    \z

The \ili{Uilta} polar and \isi{content question} markers can almost certainly be attributed to influence from \ili{Amuric} (see Sections 3.1 and 5.2.2). Within \citegen{Ikegami1997} dictionary there are not only examples with \textit{=ga}, but also with a marker \textit{=gəə}.

\ea%57
    \label{ex:tungu:57}
    \ili{Uilta}\\
    \gll tari  nari \textbf{{ŋui}}{=}\textbf{{gəə}}?\\
    that  person  who=\textsc{q}\\
    \glt ‘Who is that person?’ \citep[145]{Ikegami1997}
    \z

However, riddles recorded by Ikegami contain yet another variant with a final \textit{-k}.

\ea%58
    \label{ex:tungu:58}
    \ili{Uilta} (Southern)\\
    \gll \textbf{{xai}}{=}\textbf{{gəək}}?\\
    what=\textsc{q}\\
    \glt ‘What is (this)?’ \citep[93]{Ikegami1958}
    \z

The origin of the final \textit{-k}, which can also appear in children's games, remains partly unclear and other examples with similar constructions exhibit the marker \textit{=ga}, instead.

\ea%59
    \label{ex:tungu:59}
    \ili{Uilta} (Southern)\\
    \gll eri \textbf{{xai}}{=}\textbf{{ga}}?\\
    this  what=\textsc{q}\\
    \glt ‘What is this?’ \citep[15]{Tsumagari2009b}
    \z

Patryk Czerwinski (p.c. 2018) was so kind to check with some of the last speakers of the northern dialect. According to his fieldwork, the variant with \textit{=gəək} is still used as an ‘‘embellishment’’ of \isi{questions} and is marked with respect to the other variants. \textit{xai tari?}, \textit{xai=ga tari?}, and \textit{xai=gəək tari?} are said to have more or less the same meaning ‘What is that?’.

\cite[21, 50]{Nakanome1928} already mentioned two different forms, the unproblematic form <ga> and yet another variant written as <ṅö> that was probably pronounced with a \isi{velar nasal} [ŋ] and a vowel quality comparable to the form \textit{=gəə} recorded by Ikegami, i.e. [ŋə] (see \sectref{sec:3.1}).

\ea%60
    \label{ex:tungu:60}
    \ili{Uilta}\\
    \gll \textbf{{hai}}-wö    gade-si=\textbf{{ṅö}}?\\
    what-\textsc{acc}  buy-2\textsc{sg=q}\\
    \glt ‘What do you (want to) buy?’ \citep[52]{Nakanome1928}
    \z

In addition, there are variants with a fricative in intervocalic position, e.g. [ŋui=ɣə], [ŋui=ɣə(ə)k] 'who-\textsc{q}' (Patryk Czerwinski p.c. 2018). Most likely, we are dealing with one enclitic that undergoes both vowel harmonic and consonant alternations depending on the preceding syllable (i.e. \textit{=KA(A)}). In my eyes, \ili{Nivkh} \textit{=ŋa} is the most likely source of this enclitic in \ili{Uilta} (see \sectref{sec:3.1}).

It is an open question whether \textit{=gəək} is an independent form or a variant of \textit{=KA(A)}. A \textit{-k} can also appear in answers to riddles and might be a suffix. However, the form \textit{=gəək} apparently exhibits no \isi{vowel harmony} and only appears in special contexts, which might suggest that it is in fact a different form (Patryk Czerwinski p.c. 2018).

In the northern dialect, the \isi{question marker} seem to be more strongly fused with the preceding elements (\textit{-čee} \textbf{<} \textit{-či} ‘3\textsc{pl}’ + \textit{=KA} ‘\textsc{q}’).

\ea%61
    \label{ex:tungu:61}
    \ili{Uilta} (Northern)\\
    \gll \textbf{{xooni}} to-li-\textbf{{čee}}?\\
    how  do-\textsc{p.fut}-3\textsc{pl.q}\\
    \glt ‘How are they doing?’\footnote{\jp{どうすればよいでしょう}? in \ili{Japanese}. For the use of the \isi{interrogative}, cf. \ili{Russian} \textit{kak dela?}/\textcyrillic{как дела?}} (\citealt{Yamada2016}: 192)
    \z

No examples for alternative, \isi{focus}, and \isi{tag question}s have been found in the relevant literature (e.g., \citealt{Ikegami2002}). According to Patryk Czerwinski (p.c. 2018), \isi{focus question}s do not show any difference with respect to polar \isi{questions}. He elicited the following two alternative \isi{questions} for me. The \isi{analysis} roughly follows \cite{Tsumagari2009b}.

\ea%62
    \label{ex:tungu:62}
    \ili{Uilta} (Northern)\\
    \ea
    \gll sii \textbf{xoo}-tai ŋəɲɲee-si, oskoola-tai \textbf{yyuu}, duku-takki \textbf{yyuu}?\\
    2\textsc{sg} which-\textsc{dir} go.\textsc{prs}-2\textsc{sg} school-\textsc{dir} \textsc{q} house-\textsc{dir.refl.poss} \textsc{q}\\
    \glt ‘Where are you going, are you going to school or to your house?‘
    
    \ex
    \gll sii oskoola-tai ŋəɲɲee-si=\textbf{yi}, duku-takki \textbf{yyuu}?\\
    2\textsc{sg} school-\textsc{dir} go.\textsc{prs}-2\textsc{sg}=\textsc{q} house-\textsc{dir.refl.poss} \textsc{q}\\
    \glt ‘Are you going to school? Or to your house?‘
    
    \ex
    \gll \textbf{xai}-wa dəptu-li-\textbf{ssee}, sundattaa, ulissəə \textbf{yyuu}?\\
    what-\textsc{acc} eat-\textsc{fut}-2\textsc{sg.q} fish.\textsc{acc} meat.\textsc{acc} \textsc{q}\\
    \glt ‘What will you eat, fish or meat?‘\z\z

In the second example, there is the polar \isi{question marker} \textit{=(y)i} at the verb in the first alternative. The second alternative takes what appears to be a \isi{question marker} \textit{yyuu}. In the first example, because of the ellipsis of the verb, the marker \textit{yyuu} is found on each alternative. In the third example, the marker \textit{yyuu} only appears on the second and last alternative. The preceding \isi{content question}s exhibits a fused \isi{question marker} similar to the one seen before (perhaps \textit{-si} + \textit{=KA} > \textit{-see}). The only possible \isi{question tag} in \ili{Uilta} is \textit{ii} ‘yes‘ (similar to \ili{Russian}), although this is difficult to identify, given the formal resemblance with the polar \isi{question marker} \textit{=yi(i)} (Patryk Czerwinski p.c. 2018).

Question marking similarly aberrant to that in \ili{Uilta} can be observed in the entire \ili{Jurchenic} branch, but especially in Written \ili{Manchu}. For \textbf{Manchu}, book three of the \textit{Qingwen Qimeng} (\citealt{Wuge1730}, translated by \citealt{Wylie1855}) lists a number of \isi{interrogative} forms: \textit{na}, \textit{ne}, \textit{no}, \textit{nu}, \textit{ya}, all of which are probably enclitics. The first three must be vowel-harmonic variants of one form \textit{=nA}, which is similar to \ili{Udihe}, although a connection remains doubtful. The enclitic \textit{=nu} may be cognate of \ili{Evenki} \textit{=Ku}, \ili{Udihe} \textit{=nu}, and \ili{Nanai} \textit{=nu}, but is not often encountered.

\ea%63
    \label{ex:tungu:63}
    \ili{Manchu}\\
    \ea
    \gll waka-ra-rak\=u=\textbf{{na}}?\\
    mistake-\textsc{v}-\textsc{p}.\textsc{ipfv}.\textsc{neg}\textbf{=}\textsc{q}\\
    \glt ‘Will (you) not blame (me) then?’
    
    \ex
    \gll gene-rak\=u=\textbf{{ne}}?\\
    go-\textsc{p}.\textsc{ipfv}.\textsc{neg}=\textsc{q}\\
    \glt ‘Will (you) not go?’
    
    \ex
    \gll o.jo-rak\=u=\textbf{{no}}?\\
    become-\textsc{p}.\textsc{ipfv}.\textsc{neg}=\textsc{q}\\
    \glt ‘Will it not do?’
    
    \ex
    \gll gisu-re-rak\=u=\textbf{{nu}}?\\
    word-\textsc{v}-\textsc{p}.\textsc{ipfv}.\textsc{neg}=\textsc{q}\\
    \glt ‘Will (you) not speak?’ (\citealt{Wuge1730,Wylie1855}: 171)\z\z

The marker \textit{=ya} is also not very frequent and, might have been borrowed from \ili{Chinese} \textit{=(y)a} \zh{啊}/\zh{呀}, e.g. \textit{inu=}\textbf{\textit{ya}}? ‘Is it so?’ It appears that \textit{=ni} and \textit{=o} are not only the most neutral but also the most frequent \isi{question marker}s. There is not much information about the two enclitics, but both appear in polar, alternative, and \isi{content question}s, which makes \ili{Manchu} different from most other \ili{Tungusic} languages, but not \ili{Oroqen}. The enclitic \textit{=ni} has the reduced form \textit{n} after the existential negator \textit{ak\=u}.

\ea%64
    \label{ex:tungu:64}
    \ili{Manchu}\\
    \ea
    \gll ere gese  baita  geli  bi=\textbf{{ni}}?\\
    this  like  matter  also  \textsc{cop}=\textsc{q}\\
    \glt ‘Was there ever anything like this?’
    
    \ex
    \gll ere \textbf{{ai}} turgun=\textbf{{ni}}?\\
    this  what  reason=\textsc{q}\\
    \glt ‘What is the reason for this?’
    
    \ex
    \gll si  ere  bithe=be  hola-ki    se-mbi=\textbf{{o}}?\\
    2\textsc{sg}  this  book=\textsc{acc}  read-\textsc{opt}  say-\textsc{ipfv}=\textsc{q}\\
    \glt ‘Do you want to read this book?’
    
    \ex
    \gll geren  niyalma-i  dorgi  fala-ci    aca-ra-ngge \textbf{{we}}{-be=}\textbf{{o}}?\\
    all  person-\textsc{gen}  inside  punish-\textsc{cond}  meet-\textsc{ipfv}-\textsc{n} who-\textsc{acc}=\textsc{q}\\
    \glt ‘Who among all men is to be punished?’ (\citealt{Wuge1730,Wylie1855}: 131, 133, 140)
    
    \ex
    \gll si  min-de    bu-mbi=\textbf{{o}} bu-\textbf{{rak}}\textbf{{\=u.}}\textbf{{n}}?\\
    2\textsc{sg}  1\textsc{sg}.\textsc{obl}-\textsc{dat}  give-\textsc{ipfv}=\textsc{q}  give-\textsc{ipfv}.\textsc{neg.q}\\
    \glt ‘Are you going to give it to me, or not?’ (\citealt{Hauer2007}: 67, from the \textit{Jinpingmei})
    \z
    \z 

Possibly influenced by \textit{ak\=u}\textit{n}, the words \textit{sain} ‘good’, \textit{tašan} ‘false‘, and \textit{yargiyan} ‘true’ have the special \isi{interrogative} forms \textit{saiy\=un}, \textit{tašun}, and \textit{yargiy\=un}. The last example (\ref{ex:tungu:64}e) consists of a negative \isi{alternative question} in which, unlike any other \ili{Tungusic} language except \ili{Uilta}, two different question markers may be employed. In \ili{Manchu}, there is a wealth of such verb doubling constructions for \isi{questions} in which the second verb is always negated (\tabref{tab:tungu:2}). Only one of these patterns marks both verbs with a question marker and two do not have any marker at all, which may be due to \ili{Chinese} influence. In one case, the second alternative takes two markers. In most cases, there is one marker found at the second negated verb (cf. \ili{Kilen} and \ili{Udihe} above).

\begin{table}
\caption{Negative \isi{alternative question} patterns in Manchu (\citealt{Gorelova2002}: 325f.); \textit{-mbi} ‘\textsc{ipfv}’, \textit{-rA} ‘\textsc{p.ipfv}’, \textit{-hA} ‘\textsc{p.pfv}’, \textit{-hAk}\textit{\=u}/\textit{-rak}\textit{\=u} ‘\textsc{neg}’, \textit{se-} ‘to say’, \textit{bi-} ‘\textsc{cop}’}
\label{tab:tungu:2}

\begin{tabularx}{\textwidth}{XXl}
\lsptoprule

\textbf{V1} & \textbf{V2} & \textbf{AUX}\\
\midrule
V-ra & V-rak\=u & \\
V-ra & V-rak\=u=\textbf{n} & \\
V-ra & V-rak\=u & bi=\textbf{o}\\
V-ra & V-rak\=u & se-me=\textbf{o}\\
V-ra & V-rak\=u=\textbf{nio} & \\
V-ha & V-hak\=u & \\
V-ha & V-hak\=u=\textbf{n} & \\
V-ha & V-hak\=u=\textbf{n} & bi-he=\textbf{o}\\
V-mbi & V-rak\=u & \\
V-mbi & V-rak\=u=\textbf{n} & \\
V-mbi=\textbf{o} & V-rak\=u=\textbf{n} & \\
V-mbi-he & V-mbi-hek\=u & \\
V-mbi-he & V-mbi-hek\=u=\textbf{n} & \\
\lspbottomrule
\end{tabularx}
\end{table}

Given its \isi{semantic scope} and the possibility that \textit{=o} alone may mark a \isi{negative alternative question}, a connection to \ili{Kilen} \textit{=a} seems possible. There are further constructions not mentioned by Gorelova, which have no \isi{reduplication} of the verb.

\ea%65
    \label{ex:tungu:65}
    \ili{Manchu}\\
    \ea
    \gll gebu  ali.bu-ha=\textbf{{o}} \textbf{{ak\=un}}?\\
    name  submit-\textsc{p.pvf}=\textsc{q}  \textsc{neg.q}\\
    \glt ‘Have (you) enrolled (for the exam) or not?’ (\citealt{vonMöllendorff1892}: 28)
    
    \ex
    \gll ere kemuni tolhin \textbf{{waka}}\textbf{{=}}\textbf{{o}} se{-}{me}\textbf{{=}}\textbf{{o}}?\\
    this  still    dream  \textsc{neg}=\textsc{q}    say-\textsc{cvb.ipfv}=\textsc{q}\\
    \glt ‘This isn’t a dream or is it?’ (\citealt{DiCosmo2006}: 87, 104, 131)
    \z
    \z

As opposed to the previous constructions, this last example (\ref{ex:tungu:65}b) has the same \isi{question marker} used twice, which may be due to the presence of the negative copula \textit{waka}, after which apparently only \textit{=o} can be found. \citet[72]{AixinjueluoYingsheng1987b} argued that a \ili{Mandarin} \isi{interrogative} construction with sentence-final \textit{yŏu ma?} \zh{有吗} ‘\textsc{ex} \textsc{q}’, apparently found in the Peking dialect, is a calque of \ili{Manchu} \textit{bi=o?} ‘\textsc{ex}=\textsc{q}’.

It is often claimed that the two markers \textit{=ni} and \textit{=o} may also be attached behind one another to form the complex marker \textit{=nio}.

\ea%66
    \label{ex:tungu:66}
    \ili{Manchu}\\
    \gll ere sain ak\=u=\textbf{{nio}}?\\
    this  good  \textsc{nex}=\textsc{q}\\
    \glt ‘Isn’t this good?’ (\citealt{Wuge1730,Wylie1855}: 134)
    \z

This would be a very unusual pattern among \ili{Tungusic} languages. But apart from this \isi{analysis} into two question markers, which is a rather unexpected, there is a more plausible explanation that treats \textit{=nio} as one marker that was borrowed from \ili{Korean} (see below). Also remember that, following \textit{ak\=u}, the marker \textit{=ni} usually takes the form \textit{-n}.

In \textbf{Sibe}, polar \isi{questions} are regularly expressed with the enclitic \textit{=na} that seems to correspond to the \ili{Manchu} form \textit{=nA} above but does no exhibit \isi{vowel harmony}. It marks polar and \isi{alternative question}s. In both polar and \isi{content question}s, there is sometimes an element \textit{=jə} that might correspond to \ili{Dagur} \textit{=yee}. But its status as a \isi{question marker} remains rather dubious. Like many languages in \isi{China}, \ili{Sibe} has adopted the \ili{Mandarin} \isi{question marker} \textit{ba} \zh{吧}.

\ea%67
    \label{ex:tungu:67}
    \ili{Sibe}\\
    \ea
    \gll ʂi  guldʐa-ji  nan=\textbf{{na}}?\\
    2\textsc{sg}  \textsc{pn}-\textsc{gen}    person=\textsc{q}\\
    \glt ‘Are you from Yining?’
    
    \ex
    \gll ʂi \textbf{{evʂi}}{=}\textbf{{jə}}?\\
    2\textsc{sg}  how=?\textsc{q}\\
    \glt ‘How are you?’
    
    \ex
    \gll tʂəhsə avqa-ji    arvun \textbf{{evʂi}}?\\
    yesterday  sky-\textsc{gen}  form  how\\
    \glt ‘How was the weather yesterday?’ (\citealt{Chaoke2006}: 207, 280, 277)
    
    \ex
    \gll ə.r  mi.{n} dʐaq=\textbf{{na}},  ʂi.{n} dʐaq=\textbf{{na}}?\\
    this  1\textsc{sg.gen}  thing=\textsc{q}  2\textsc{sg}.\textsc{gen}  thing=\textsc{q}\\
    \glt ‘Is this mine or yours?’
    
    \ex
    \gll so.{n} tə.{va} {mori\textsuperscript{n}} gum ambu \textbf{{ba}}?\\
    2\textsc{pl}.\textsc{gen}  there  horse  all  big  \textsc{q}\\
    \glt ‘In your place all horses are big, aren’t they?’ (\citealt{Zikmundová2013}: 49, 95)
    \z
    \z 

In general, there are very few descriptions of possible tag \isi{questions} in \ili{Tungusic} languages. \ili{Sibe} is somehow exceptional because at least two different \isi{tag question} patterns were recorded.

\ea%68
    \label{ex:tungu:68}
    \ili{Sibe}\\
    \ea
    \gll {tə-}{s} ɢanbi ʂi.{n-}{i-}{ŋə} vaq, \textbf{{məndʐaŋ}}\textbf{{=}}\textbf{{ba}}?\\
    that-\textsc{pl}  pencil    2\textsc{sg}.\textsc{obl}-\textsc{gen}-\textsc{n}  \textsc{neg} right=\textsc{q}\\
    \glt ‘Those aren’t your pencils, right?’
    
    \ex
    \gll tə.{r} əm nan siv, \textbf{{məndʐaŋ}}\textbf{{=}}\textbf{{na}}?\\
    that  one  man  teacher    right=\textsc{q}\\
    \glt ‘That one is a teacher, right?’
    
    \ex
    \gll ʂi mi.{ni-}{d} {utʂi-}{v} {li-}{maq} ʂinda, \textbf{{o}}\textbf{{-}}\textbf{{m}}{=}\textbf{{na}}?\\
    2\textsc{sg}  1\textsc{sg}.\textsc{obl}-\textsc{dat}  door-\textsc{acc} open-\textsc{cvb}.\textsc{pfv}  stay[\textsc{imp]} be-\textsc{ipfv}=\textsc{q}\\
    \glt ‘Hold the door open for me, will you?’ (\citealt{Chaoke2006}: 91, 102, 343)\z\z

The first type could be a calque and partial loan from \ili{Mandarin} \textit{duì ma/ba} \zh{对马}/\zh{吧}. The latter type with the verb \textit{o-} ‘to become, to be, to be permissible’ \citep{Norman2013} possibly is a calque of \ili{Mandarin} \textit{kĕy\u{\i} ma} \zh{可以吗} or \textit{xíng ma} \zh{行吗} (\sectref{sec:5.9.2.1}). There are also parallels in \ili{Khorchin} \ili{Mongolian} (\sectref{sec:5.8.2}).

Records of \ili{Sibe} from the beginning of the 20th century that were strongly influenced by Written \ili{Manchu} have been recorded by Muromski. They contain several \isi{question marker}s, \textit{=na(a)}, \textit{=ńu(u)} {\textasciitilde} \textit{=ńü}, \textit{=ü} {\textasciitilde} \textit{=’u}, and \textit{=o} (\citealt{Kałużyński1977}: 53). The marker \textit{=U} might be of \ili{Mongolic} origin and is the only one that is unknown from \ili{Manchu}. It appears to have fused with the imperfective or dictionary form \textit{-mbi} of Written \ili{Manchu}.

\ea%69
    \label{ex:tungu:69}
    \ili{Sibe}\\
    \gll mi.n-i    gała-ci    tuči-\textbf{{mbü}}?\\
    1\textsc{sg}.\textsc{obl}-\textsc{gen}  hand-\textsc{abl}  come.out-\textsc{ipfv}.\textsc{q}\\
    \glt ‘Can you escape my hands?’ (\citealt{Kałużyński1977}: 53)
    \z

\textbf{Sanjiazi Manchu} also shows some of this variation. The following examples were collected in 1961 and contain the markers \textit{=nɔ}, \textit{=nu}, and \textit{=ni}. The last one seems to be restricted to \isi{content question}s that are optionally unmarked, while the other two (\textit{=nU}) appear in \isi{polar question}s. Enhebatu treats them as variants of the same form.

\ea%70
    \label{ex:tungu:70}
    \ili{Manchu} (Sanjiazi)\\
    \ea
    \gll ɯr  baiti-bɯ  dɔndʑi-$\gamma $=\textbf{{nɔ}}?\\
    this  matter-\textsc{acc}  hear-\textsc{p.pfv}=\textsc{q}\\
    \glt ‘Did you hear about this matter?’
    
    \ex
    \gll ɯ.l{ɯ} \textbf{{ai}} {dʐa\textsuperscript{q}}{xa=}\textbf{{ni}}?\\
    this  what  thing=\textsc{q}\\
    \glt ‘What (kind of thing) is this?’
    
    \ex
    \gll \textbf{{ai}} gɯvvɯ?\\
    what  name\\
    \glt ‘What (is your) name?’ (\citealt{Enhebatu1995}: 300, 296, 55)\z\z

\citet[45]{KimJuwon2008}, who did fieldwork in Sanjiazi in 2005 and 2006, recorded the markers \textit{=no} and \textit{=nə}. They claim that the latter is a loan from \ili{Mandarin} \textit{ne} \zh{呢}. Sanjiazi has also borrowed \ili{Mandarin} \textit{ba} \zh{吧}. For \isi{alternative question}s the \ili{Chinese} \isi{disjunction} \textit{háishì} \zh{还是} ‘or.\textsc{q}’ has been adopted.

\ea%71
    \label{ex:tungu:71}
    \ili{Manchu} (Sanjiazi)\\
    \ea
    \gll ɕi  qaŋqxɣ-ʁɣ \textbf{{ba}}?\\
    2\textsc{sg}  be.thirsty-\textsc{p.pfv}  \textsc{q}\\
    \glt ‘You must be thirsty, right?’ \citep[55]{Enhebatu1995}
    
    \ex
    \gll najnaj  čičikal=dili  ǰi-xə-niŋŋə \textbf{{haishi}} xaəlbin=dili  ǰi-xə-niŋŋə?\\
    grandmother  \textsc{pn}=\textsc{abl} come-\textsc{p.pfv}-\textsc{n}  or.\textsc{q}  \textsc{pn}=\textsc{abl}  come-\textsc{p.pfv}-\textsc{n}\\
    \glt ‘Is your grandmother from Qiqihar or from Harbin?’ (\citealt{KimJuwon2008}: 214)
    \z
    \z

\citet[46]{KimJuwon2008} mention an enclitic \textit{=ja} {\textasciitilde} \textit{=jə} that they call an “‘intimacy’ particle”. It may appear in \isi{questions} but is not restricted to them. A connection to the \ili{Sibe} and \ili{Dagur} enclitic seems more likely than with \ili{Mandarin} \textit{ya} \zh{呀}.

The \textbf{Yibuqi} dialect of \ili{Manchu} presents a situation very similar to Sanjiazi \ili{Manchu}. The usual \isi{question marker} has the form \textit{=no}, \isi{content question}s remain unmarked, and the \ili{Mandarin} \isi{disjunction} may be employed in plain and \isi{negative alternative question}s.

\newpage
\ea%72
    \label{ex:tungu:72}
    \ili{Manchu} (Yibuqi)\\
    \ea
    \gll pi    put\textsc{a} tʂə-mi=\textbf{{no}}?\\
    1\textsc{sg.nom}  food  eat-\textsc{ipfv}=\textsc{q}\\
    \glt ‘(Can) I eat the food?’
    
    \ex
    \gll ɕiɕk‘ə \textbf{{və}} tɕi-ɣə?\\
    yesterday  who  come-\textsc{p.pfv}\\
    \glt ‘Who came yesterday?’
    
    \ex
    \gll ɕi  tɕi-mi \textbf{{xɛʂʅ}} pi    kənə-mi?\\
    2\textsc{sg}  come-\textsc{ipfv}  or  1\textsc{sg}  go-\textsc{ipfv}\\
    \glt ‘Do you come or do I go?’\footnote{Here, the \isi{disjunction} may also have the form \textit{ʂʅ}.} (\citealt{ZhaoJie1989}: 164, 184, 188)\z\z

The Yibuqi dialect additionally borrowed the \ili{Mandarin} polar \isi{question marker} \textit{ma} \zh{吗}.

\ea%73
    \label{ex:tungu:73}
    \ili{Manchu} (Yibuqi)\\
    \gll so  kəm  tɕi-ɣə \textbf{{m}}\textbf{\textsc{a}}?\\
    2\textsc{pl}  all  come-\textsc{p.pfv}  \textsc{q}\\
    \glt ‘Have you all come?’(\citealt{ZhaoJie1989}: 154)
    \z

\textbf{Aihui Manchu} has the standard polar \isi{question marker} \textit{=no}. A form \textit{=je} similar to \ili{Sibe} is attested, but its meaning is not perfectly clear. Content questions usually remain unmarked. Alternative questions take the \ili{Mandarin} \isi{disjunction} \textit{háishì} \zh{还是} ‘or.\textsc{q}’.

\ea%74
    \label{ex:tungu:74}
    \ili{Manchu} (Aihui)\\
    \ea
    \gll ɕi  mandʐo  gisun    baʁa.na-m=\textbf{{no}}?\\
    2\textsc{sg}  \ili{Manchu}  language  be.able-\textsc{ipfv}=\textsc{q}\\
    \glt ‘Can you speak \ili{Manchu}?’
    
    \ex
    \gll \textbf{{ɛ}}-bəri    jov-ʁa?\\
    what-\textsc{dir}  go-\textsc{p.pfv}\\
    \glt ‘Where did (she) go?’
    
    \ex
    \gll \textbf{{an}}\textbf{{ə}} gəl  mɛri-m      dʑi-ɣə{=}\textbf{{je}}?\\
    why  again  return-\textsc{cvb}.\textsc{ipfv}  come-\textsc{p.pfv}=?\textsc{q}\\
    \glt ‘Why have you come back again?’
    
    \ex
    \gll ə.rə dʐaqa  ʂʅ  fə-{niŋŋə} \textbf{{xɛʂʅ}} {itɕi-ŋŋə?}\\
    this  thing  \textsc{cop}  old-\textsc{n}    or.\textsc{q}  new-\textsc{n}\\
    \glt ‘Is this thing old or new?’ (\citealt{WangQingfeng2005}: 210, 229, 228, 243)
    \z
    \z 

For Aihui \ili{Manchu} a \isi{tag question} different from \ili{Sibe} has been recorded. It may have been partly calqued from \ili{Mandarin} \textit{duì bu duì} \zh{对不对} or \textit{duì ba} \zh{对吧}.

\newpage 
\ea%75
    \label{ex:tungu:75}
    \ili{Manchu} (Aihui)\\
    \gll bi    agə-dərə  gəm  adʑigən, \textbf{{ino}} \textbf{{vaqa}}{=}\textbf{{ba}}?\\
    1\textsc{sg}.\textsc{nom}  e.brother-\textsc{abl}  all  little    correct  wrong=\textsc{q}\\
    \glt ‘I’m smaller than all (my elder) brothers, isn’t that right?’ (\citealt{WangQingfeng2005}: 236)
    \z

It may be noted that both \textit{inu} and \textit{waka} also function as positive and negative one word answers, respectively, in Written \ili{Manchu}.

The two languages \textbf{Bala} and \textbf{Alchuka} add important pieces to the puzzle. Both preserves a cognates of \ili{Manchu} \textit{=o}. Compare the following two sentences.

\ea%76
    \label{ex:tungu:76}
    ?\ili{Bala}\\
    \gll ɕi.n    nianli \textbf{{ai}}-və-t‘      bi=\textbf{{ɔ}}?\\
    2\textsc{sg}.\textsc{gen}  washing.hammer  what-place-\textsc{loc}  \textsc{cop}=\textsc{q}\\
    \glt
    \z

\ea%77
    \label{ex:tungu:77}
    Lalin/Jing \ili{Manchu}\\
    \gll ɕi.n-i    nijandʒ‘a.k‘u \textbf{{ai}}-ba-de    bi-x=\textbf{{ɔ}}?\\
    2\textsc{sg}.\textsc{obl}-\textsc{gen}  washing.hammer  what-place-\textsc{loc}  \textsc{cop}-\textsc{pfv}=\textsc{q}\\
    \glt ‘Where is your washing hammer?’\footnote{\zh{洗衣棒锤} in \ili{Chinese}. \citet{Norman2013} translates Written \ili{Manchu} \textit{niyanca-kû} as ‘a wooden stick for beating starched clothes while washing’.} (\citealt{MuYejun1987}: 25)
    \z

\ili{Alchuka}, in addition to \textit{=ɔ}, has a variant \textit{=kɔ} with an unaspirated [k]. This form is related to \ili{Manchu} \textit{=o} as well, as can be observed from a comparison of \ili{Alchuka} \textit{əl}\textit{ə-mei=}\textbf{\textit{k}}\textbf{\textit{ɔ}} ‘fear-\textsc{ipfv}=\textsc{q}’ (\citealt{MuYejun1986}: 16) with \ili{Manchu} \textit{gele-mbi=}\textbf{\textit{o}} (\citealt{AixinjueluoYingsheng1987a}: 15) that were attested in the same sentence. \ili{Bala} also has a form \textit{=ŋɔ} that is most likely cognate with Sanjiazi \textit{=nɔ}, Aihui \ili{Manchu} \textit{=no}, Yibuqi \ili{Manchu} \textit{=no}, and \ili{Manchu} \textit{=nio}.

\ea%77
    \label{ex:tungu:78}
    \ili{Bala}\\
    \ea
    \gll ɕi ənə=\textbf{{ŋɔ}}?\\
    2\textsc{sg}    go=\textsc{q}\\
    \glt ‘Are you going?’
    
    \ex
    \gll ɕi.{n} amin=\textbf{{ŋɔ}}?\\
    2\textsc{sg}.\textsc{gen}  father=\textsc{q}\\
    \glt ‘Is it your father?’ (\citealt{MuYejun1987}: 31)
    \z
    \z

\tabref{tab:tungu:3} summarizes \isi{interrogative} markers in \ili{Tungusic} languages. \ili{Kyakala}, \ili{Jurchen  A}, \ili{Jurchen  B}, \ili{Kili}, and \ili{Arman} have been excluded for lack of information. To the best of my knowledge, the origin of the \ili{Jurchenic} question markers have never been described satisfactorily. But given their presence in \ili{Jurchenic}, exclusively, and the lack of a good internal etymology, a \isi{borrowing} from a neighboring language seems plausible. I argue that most of them (\ili{Manchu} \textit{=o}, \textit{=n(i)}, \textit{=nio}, \textit{=nA}) were perhaps borrowed from \ili{Koreanic}, which had longstanding contacts with \ili{Jurchenic}. The details are presented in \sectref{sec:5.7.2}. \ili{Manchu} \textit{=nu} might be inherited from \ili{Proto-Tungusic}. Aihui \ili{Manchu} \textit{-je}, Sanjiazi \ili{Manchu} \textit{-jA} as well as \ili{Sibe} \textit{-jə} may have been borrowed from \ili{Dagur}. The \isi{disjunction} in \ili{Kilen}, \ili{Oroqen}, and \ili{Manchu} dialects was borrowed from \ili{Mandarin}.

Regarding the syntactic behavior of interrogatives in content \isi{questions}, \cite[343f.]{MalchukovNedjalkov2010} offer the following summary.

\begin{quote}
Question formation need not involve WH-movement in \ili{Tungusic} languages. For some languages, WH-\isi{fronting} seems to be a preferred option, as for example in \ili{Evenki} (\citealt{Nedjalkov1997}: 7f.). For \ili{Even}, on the other hand, WH-\isi{fronting} is associated with emphatic/rhetorical \isi{questions}; in regular constituent \isi{questions} the \isi{interrogative} pronoun remains \isi{in situ} \citep{Malchukov2008}. In Written \ili{Manchu}, question words also remain \isi{in situ} \citep[222]{Gorelova2002}. In \ili{Udihe} (\citealt{NikolaevaTolskaya2001}: 799), the position of focused elements including question words is strictly before the verb.
\end{quote}

However, note that, according to the description by \citet[42]{Girfanova2002}, \ili{Udihe} behaves like \ili{Evenki} in putting the question word in sentence initial position. \ili{Even} \citet[799, 805]{NikolaevaTolskaya2001} agree that the \isi{interrogative} \textit{ii-}\textit{mi/j’e-}\textit{mi} ‘why’ that is of converbal origin may optionally stand in clause initial position as well.

\begin{table}
\caption{Question markers in Tungusic languages}
\label{tab:tungu:3}

\begin{tabularx}{\textwidth}{lllQ}
\lsptoprule

\textbf{Language} & \textbf{PQ} & \textbf{CQ} & \textbf{AQ}\\
\midrule
\ilit{Even} & =Ku\# & - & 2x =Ku\#\\
\ilit{Evenki} & =Ku\# & - & 2x =Ku\#\\
\ilit{Evenki} (Sakh.) & =Kuu\#, =too\# & - & 2x =Ku\#\\
\ilit{Evenki} (Kh.) & =Kv\# & bei\# & 2x =li\#\\
\ilit{Negidal} & \tabref{tab:tungu:1}, =Kʊʊ\#, =i\# & -, \tabref{tab:tungu:1} & \tabref{tab:tungu:1}, 2x =Kʊʊ\#\\
Chaoyang \ilit{Oroqen} & =YEE\#, =oo\#, bAA\# & - & 2x =YEE\#, 2x jOOmAA\# (+2x =gUU)\\
Xunke \ilit{Oroqen} & =jA\#, =ɔɔ\#, bAA\# & =jA\# & (2x =jA +) aaki ‘or’, 2x jɔɔma\#, 2x ɔɔmal\#\\
Huihe \ilit{Solon} & =gi(i)\# & (jeeme) & 2x =gi(i)\#\\
Ongkor \ilit{Solon} & =uu\#, =ii\# & - & 2x =uu\#\\
\ilit{Udihe} & =Ku\#, =nA\# & - & 2x =Ku\#, 2x =nA\#, 1-2x -(e)s(i)\#\\
\ilit{Oroch} & =nu\# & - & 2x =nu\#\\
\ilit{Nanai} & =nu\# & - & 2x =nu\#\\
\ilit{Kilen} & =nə\#, =a\#, =ma\#, =ba\# & - & 2x =nə\#, xəɕi ‘or’ + =a\#\\
\ilit{Uilta} & =(y)i\# & -KA\# & =(y)i\#, yyuu\#\\
\ilit{Ulcha} & =nʊʊ\# & ?=nʊʊ\# & 2x =nʊʊ\#\\
\ilit{Manchu} & =o\#, =n(i)\#, =nio\#, =nA\#, =nu\# & =o\#, =n(i)\# & \tabref{tab:tungu:2}\\
Aihui \ilit{Manchu} & =no\# & -, ?-je\# & xɛʂʅ ‘or’\\
Yibuqi \ilit{Manchu} & =no\#, m\textsc{a}\# & - & xɛʂʅ ‘or’\\
Sanjiazi \ilit{Manchu} & =nɔ\#, =nu\#, ba\# & =ni\#, ?-jA\#, =nə & haishi ‘or’\\
\ilit{Sibe} & =na(a)\#, ba\#, ?-jə\# & - ?-jə\# & 2x =na(a)\#\\
\ilit{Alchuka} & =(k)ɔ\# & =n(i)\#, ? & ?\\
\ilit{Bala} & =ɔ\#, =ŋɔ\# & =ɔ\#, ? & ?\\
\lspbottomrule
\end{tabularx}
\end{table}

\subsection{Interrogatives in Tungusic}\label{sec:5.10.3}%\footnotemark

\ili{Tungusic} interrogatives have been treated in some detail before.
% \footnotetext{Some isolated aspects of this subsection have been presented at the conference of the \textit{Polish \isi{Cognitive Linguistics} Association} (\citealt{Hölzl2015c}).} 
The classical but partly outdated \isi{reconstruction} can be found in \cite[114f.]{Benzing1956}. The most exhaustive lists of cognates that nevertheless lack many important data can be found in \cite[264ff.]{Cincius1949} and \cite{Cincius1975/77}. \citet{Kazama2003} elaborates on \cite{Cincius1975/77} and also includes data from \ili{Kilen} and \ili{Sibe} but still is not exhaustive. Not to be underestimated are the data collected in \cite{Schmidt1923a,Schmidt1923b,Schmidt1928a,Schmidt1928b} for \ili{Samagir}, \ili{Samar}, \ili{Ulcha}, \ili{Nanai}, \ili{Oroch}, \ili{Udihe}, \ili{Negidal}, and \ili{Evenki}. Of these, the first two varieties are almost unknown otherwise. Schmidt mentions \ili{Samagir} \textit{ekon} ‘what’ and \ili{Samar} \textit{xai} ‘what’, which is sufficient to classify the two as \ili{Ewenic} (e.g., \ili{Evenki} \textit{ekun}) and \ili{Nanaic} (e.g., \ili{Nanai} \textit{xaɪ}), respectively (see also \citealt{Doerfer1978a}). \tabref{tab:tungu:5} gives an extended list of cognates for those five interrogatives that have the widest distribution among \ili{Tungusic} languages. For the references, see the more detailed descriptions below. The use of \ili{Tungusic} interrogatives or \isi{demonstratives} as correlatives has recently been investigated in detail by \citet[185-226]{Baek2016}.

All languages except for some subdialects of \ili{Solon} and \ili{Oroqen} preserve the \isi{interrogative} ‘who’. The form has been reconstructed as *\textit{ŋüi} \citep[115]{Benzing1956} or *\textit{ŋui} {\textasciitilde} *\textit{ŋɵi} \citep[68]{Kazama2003} for \ili{Proto-Tungusic} and as *\textit{ŋii} for Proto-\ili{Ewenic} (\citealt{Janhunen1991}: 70f.). Only Kazama’s \isi{reconstruction} based on Ikegami is erroneous. The original *\textit{ü} regularly changed to \textit{i} in Northern but to \textit{u} in Southern \ili{Tungusic}. In some \ili{Ewenic} languages such as \ili{Solon} or \ili{Oroqen} as well as \ili{Udegheic}, the \isi{velar nasal} changed to an \textit{n} while it apparently was lost in all of \ili{Jurchenic} and \ili{Nanaic}, except for \ili{Uilta} and \ili{Ulcha}. These are not regular developments but have certain parallels, e.g. \ili{Evenki} \textbf{\textit{ŋ}}\textit{ina.kin}, \ili{Solon} \textbf{\textit{n}}\textit{ini.xin}, \ili{Uilta} \textbf{\textit{ŋ}}\textit{inda}, \ili{Nanai} \textit{enda}, \ili{Manchu} \textit{inda.h\=u}\textit{n}, but \ili{Udihe} \textit{in’e.i} ‘dog’ (cf. \citealt{Benzing1956}: 68). The short vowel in some northern \ili{Tungusic} languages must be a secondary innovation that is partly shared by the \isi{interrogative} \textit{i(i)-}. \ili{Kilen} \textit{ni} was borrowed from \ili{Udegheic}, and \ili{Kili} \textit{ŋii} from \ili{Ewenic}. A form \textit{p‘ə} ‘who’ mentioned by \citet[14]{MuYejun1986} for \ili{Alchuka} is most unexpected and cannot be explained with the reconstructed form *\textit{ŋüi}. Problematically, a [\textit{p\textsuperscript{h}}] in \ili{Alchuka} usually corresponds to an \textit{f} in \ili{Manchu} (e.g., \ili{Alchuka} \textit{p‘i}, \ili{Manchu} \textit{fi} ‘brush’) and \ili{Manchu} \textit{we} clearly corresponds to \ili{Nanai} \textit{ui} (e.g., \ili{Manchu} \textbf{\textit{we}}\textit{si-hun} and \ili{Nanai} \textbf{\textit{ui}}\textit{si} ‘up’). It is not very plausible to assume that \ili{Nanai} \textit{ui} or \ili{Manchu} \textit{we} are not related to \ili{Uilta} \textit{ŋui} or \ili{Ulcha} \textit{(ŋ)ui}. Assuming that the \ili{Alchuka} form is not a mistake, it is most likely related to \ili{Manchu} \textit{we}, but details remain obscure for the moment.

\begin{table}
\small
\caption{List of cognates of five Tungusic interrogatives}
\label{tab:tungu:5}
\small
\fittable{
\begin{tabular}{llllll}
\lsptoprule

\textbf{Language} & \textbf{who} & \textbf{what, which} & \textbf{which,} \textbf{what} & \textbf{how many} & \textbf{how}\\
\midrule
\ilit{Proto-Tungusic} & *ŋüi & *ja- & *Kai & *Kadu & *Kooni\\
\ilit{Even} & ŋi & ja-k & i-rəə-\textbf{k} & adi & on\\
\ilit{Arman} & ŋii & jaa-ḳ & \textbf{iää}-ra-\textbf{k} & aadii & oon\\
\ilit{Evenki} & ŋi & e-kun & i-r & ady & oon\\
\ilit{Negidal} & nii, n’ii, ŋii & ee-xun, ee-kun & ii- & adii & oon(i)\\
\ilit{Solon} (Huihe) & nii, \textbf{aawu} & o-xon & ii & adi & \textbf{iittü}\\
\ilit{Solon} (Ongkor) & \textbf{a($\gamma $)uu} & jo-xon & i(i) & adi & \\
\ilit{Oroqen} (Nanmu) & \textbf{awu} & i-hun & i-r(i) & adi & ooni\\
\ilit{Oroqen} (Chaoyang) & nii & ɪ-kʊn & i-ri & ? & ɔɔn\\
Kha. \ilit{Evenki} B. & nii & i-kun & ii-r & adii & oon\\
Kha. \ilit{Evenki} U. & nii & ie-kun & ii-r & adii & oon\\
\ilit{Udihe} & ni(i) & j’e-u & ii- & adi & ono\\
\ilit{Oroch} & n’ii & jaa-u & i- & ady & oni\\
\ilit{Kili} & ŋii & ?ii- {\textasciitilde} e- & i- & ad\textbf{ii} & \\
\ilit{Kilen} (ZhY) & ni & ja & - & ad\textbf{i} & ɔni\\
\ilit{Kilen} (ZhP) & ni & ja & - & at\textbf{i} & ɔmə\textbf{ɕi}\\
\ilit{Kilen} (AJ) & ni & ja & \textbf{χ}ai & \textbf{χ}ad\textbf{i}, \textbf{χ}adu & on(n)i\\
\ilit{Kilen} (Ling) & ui & - & \textbf{h}ai & \textbf{h}adu & \textbf{h}ɔni-\textbf{biʃi}\\
\ilit{Nanai} & ui & - & \textbf{x}aɪ & \textbf{x}ado & \textbf{x}ooni\\
\ilit{Uilta} & ŋui & - & \textbf{x}ai & \textbf{xasu} & \textbf{x}ooni\\
\ilit{Ulcha} & (ŋ)ui & - & \textbf{x}ai & \textbf{x}adu & \textbf{x}oon(i)\\
\ilit{Alchuka} & ?\textbf{p‘}ə & ? & \textbf{k}ai- & \textbf{k}utu & \textbf{katiram}\\
\ilit{Bala} &  &  & a(i)- &  & \\
\ilit{Manchu} & we & ya & ai & udu & \textbf{absi}, \textbf{adarame}\\
Aihui \ilit{Manchu} & və & ja & ɛ & \textbf{ɛdik} & \textbf{avɕe}\\
Sanjiazi \ilit{Manchu} & wə & ja & aj & udə & \textbf{adəlmən}\\
Yibuqi \ilit{Manchu} & və & jA & ɛi & utu & \textbf{atər(ə)mə}\\
\ilit{Sibe} & və & ya & ai & ut & \textbf{afś(}\textbf{e)}\\
\lspbottomrule
\end{tabular}
}
\end{table}

Within the \isi{interrogative} system of \ili{Tungusic} *\textit{ŋüi} has a special position as it is unrelated to the other interrogatives. The same is also true for *\textit{ja-} ‘what, which’. \citegen{Benzing1956} \ili{Proto-Tungusic} \isi{reconstruction} *\textit{jaa-} and \citegen[70f.]{Janhunen1991} Proto-\ili{Ewenic} \isi{reconstruction} *\textit{ie-} seem to show the wrong vowel quantity and quality, respectively. As for the development of the vowels, note a parallel development in \ili{Tungusic} *\textit{jaa-sa} ‘eyes’ > \ili{Evenki} \textit{ee.sa}, Borzya \textit{ii.sa} etc. (\citealt{Benzing1956}: 25; \citealt{Janhunen1991}: 34). Only \ili{Nanaic} languages have no reflex of *\textit{ja-}, \ili{Kilen} being a special case as the \isi{interrogative} \textit{ja} has been adopted from \ili{Udegheic} or, less likely, from \ili{Jurchenic}. \citet[199]{LiLinjing2011} mentions a \ili{Kilen} form \textit{ya.o}, for which only \ili{Udihe} \textit{j’e.u} or perhaps \ili{Oroch} \textit{jaa.u} can be the source. The extension seen in this form exists only in northern \ili{Tungusic}. \ili{Kili} seems to have variation between \textit{ii- {\textasciitilde} e-}, derived from \ili{Ewenic}.

The \isi{interrogative} *\textit{Kai} (\citealt{Benzing1956} reconstructed *\textit{xai}) is preserved in all branches but is absent in some parts of \ili{Kilen} and exists only in relics in \ili{Udegheic}. Benzing assumed the presence of a suffix attached to a stem *\textit{xa-}, but no direct evidence for this has been found. The initial consonant has been regularly lost in most of northern \ili{Tungusic} and \ili{Jurchenic} and in most cases changed to a x-like sound in \ili{Nanaic}. \citet[56, 75]{Kazama2003} did not recognize the connection between \textit{i(i)-} and \ili{Nanai} \textit{xaɪ} etc. Admittedly, the stem extension (e.g., \ili{Evenki} \textit{i-r} ‘which’) can only be found in northern \ili{Tungusic}. However, this must be a secondary innovation of some \ili{Ewenic} languages that spread from the \isi{demonstratives} (e.g., \ili{Evenki} \textit{e-r} ‘this’, \textit{ta-r} ‘that’).

There is a certain amount of confusion surrounding the relation of the two stems *\textit{ja-} and *\textit{Kai-}. For instance, \citet[27]{Doerfer1985} tried to show that they go back to one form, but his explanation is extremely speculative and does not appear to be actually based on any hard evidence. Nevertheless, the two forms are problematic as they have several properties in common, and are partly interchangeable. First, while northern \ili{Tungusic} languages have an \isi{interrogative verb} based on *\textit{ja-}, the \isi{interrogative} *\textit{Kai-} has both nominal and verbal properties in \ili{Nanaic}. This interesting difference can be shown with the help of \ili{Nanai} and \ili{Kilen}, which has been strongly influenced by \ili{Udihe} in this regard (\tabref{tab:tungu:6}).

\begin{table}
\caption{Ambiguous interrogative stems in Nanai \citep{Kazama2007} and Kilen (\citealt{ZhangPaiyu2013}: 162)}
\label{tab:tungu:6}

\begin{tabularx}{\textwidth}{XXXl}
\lsptoprule
& \textbf{Nanai} & \textbf{Kilen} & \textbf{Meaning}\\
\midrule
verbal & xai-xa-ni & ja-xəi-ni & what-\textsc{pst}-3\textsc{sg}\\
nominal & xai-wa & ja-wə & what-\textsc{acc}\\
\lspbottomrule
\end{tabularx}
\end{table}

\ili{Ewenic} and \ili{Udegheic} roughly pattern with \ili{Kilen} while \ili{Jurchenic} is close to \ili{Nanai} but apparently is unique in showing an obligatory verbalizer (e.g., \ili{Manchu} \textit{ai-na-}, \ili{Bala} \textit{a-na-}, \ili{Alchuka} \textit{kai-na-}). However, \ili{Udihe} also has an optional derived form \textit{j’e-ne-}. This split has not only been overlooked by \citet{Benzing1956} but also by several other scholars such as \citet[99]{TolskayaTolskaya2008}. Interestingly, some forms have the same derivation but are based on different stems. For example, interrogatives meaning ‘why’ usually have a verbal basis and are really \isi{converb} forms of the \isi{interrogative verb}, e.g. \ili{Even} \textit{ja-mi}, \ili{Udihe} \textit{j’e-mi}, but \ili{Nanai} \textit{xaɪ-mi} and \ili{Manchu} \textit{ai-na-me}. Apart from Oroch, \ili{Udihe} is exceptional as it also has the form \textit{ii-mi} that can be compared with \ili{Nanai}. Second, the two \isi{interrogative} stems are partly interchangeably in \ili{Jurchenic}. In \ili{Manchu} dialects, for example, there is \isi{synchronic} variation between alternative forms such as \textit{ai-erin} {\textasciitilde} \textit{ya-erin} ‘when’ or \textit{ai-ba-} {\textasciitilde} \textit{ya-ba-} ‘where’ without apparent differences in meaning (see below for more examples).

However, the fact that languages as distantly related and located as \ili{Even} and \ili{Manchu} have traces of both stems is clear evidence for their existence at a very early stage in the development of \ili{Tungusic}. Furthermore, *\textit{Kai} is part of a larger group of interrogatives that share a \isi{resonance} in *\textit{K{\textasciitilde}} that most likely is etymologically connected. But even in \ili{Proto-Tungusic} their exact derivation must have already been obscure. For example, Benzing, based on the assumption of \isi{analyzability} of *\textit{xa-i}, reconstructed the \isi{interrogative} meaning ‘how’ as *\textit{xaoni}, which has to be rejected, as there is no indication of an original diphthong. Many modern languages preserve a long vowel, which is why I reconstruct the form as *\textit{Kooni} instead. \cite[70f.]{Janhunen1991} assumed a stem *\textit{xoo-}, but there is no clear evidence that \textit{-ni} might have been a suffix. This \isi{interrogative} is preserved everywhere except for \ili{Solon} and \ili{Jurchenic}. In a similar vein, \citegen{Benzing1956} \isi{reconstruction} *\textit{xaduu} ‘how much’ with a long vowel has no real basis as most languages simply have a short vowel. Based on the distribution of northern \ili{Tungusic} \textit{i} and southern \ili{Tungusic} \textit{u}, but \ili{Nanai} \textit{o}, as well as a comparison with \ili{Mongolic} (on which see below), the form may probably be reconstructed as *\textit{Kadu} instead. In the latter two interrogatives there are some irregular developments such as a progressive vowel assimilation in \ili{Udihe} \textit{on}\textbf{\textit{o}} ‘how’ (cf. \ili{Oroch} \textit{on}\textbf{\textit{i}}) and a retrogressive assimilation in \ili{Jurchenic}, e.g. \ili{Manchu} \textbf{\textit{u}}\textit{du} ‘how many’ (cf. \ili{Ulcha} \textit{x}\textbf{\textit{a}}\textit{du}).

Benzing assumed a \ili{Proto-Tungusic} \isi{resonance} in *\textit{x{\textasciitilde}}. But in my opinion, new evidence (e.g., \citealt{MuYejun1986,Hölzl2017c}) points to a possible \isi{reconstruction} as plosive (see also \citealt{Rozycki1993}). This assumption is based on data from \ili{Alchuka} that exhibit what could be a conservative feature lost in all other \ili{Tungusic} languages. However, given its unclear phonetic status, for now I use a label \textit{K-} in the reconstructions instead. The limited data from \ili{Alchuka} contain four interrogatives with an initial unaspirated velar plosive \textit{k} (or perhaps \textit{g}) that is not present in \ili{Manchu} (\tabref{tab:tungu:7}). It has been suggested to me by András Róna-Tas (p.c. 2015) that the consonant might be a secondary innovation in \ili{Alchuka}. The initial consonant is attested in about two dozen instances, and it may well be a secondary innovation in some of them. However, the fact that it systematically appears in many attested interrogatives and has a correspondence in \ili{Nanaic} \textit{x-} suggests that at least in this position it should be of \ili{Proto-Tungusic} origin.\footnote{I am currently preparing a more detailed investigation of the problem.}

\begin{table}
\caption{Selected Alchuka interrogatives (\citealt{MuYejun1985,MuYejun1986,MuYejun1987,MuYejun1988a,MuYejun1988b}) with Manchu cognates \citep{Norman2013}; inner-Tungusic loanwords are in parentheses}
\label{tab:tungu:7}

 \begin{tabularx}{\textwidth}{XXl} 
  \lsptoprule
   & \ilit{Alchuka} & \ilit{Manchu}\\
  \midrule
   for what reason & (\textbf{\_}ei) t`uku & \textbf{\_}ai turgun\\
   how & \textbf{k}atiram & \textbf{\_}adarame\\
   how many & \textbf{k}utu & \textbf{\_}udu\\
   to do what & \textbf{k}ai-na-mei & \textbf{\_}ai-na-mbi\\
   what & (\textbf{\_}ei) & \textbf{\_}ai\\
   what has happened & \textbf{g}ai-na-hanbie & \textbf{\_}ai-na-habi\\
   what (is it) & \textbf{k}ent`aka & \textbf{\_}antaka\\
   when & (\textbf{\_}ant`aŋgi) & \textbf{\_}atanggi\\
   why & (\textbf{\_}einu) & \textbf{\_}ainu\\
\lspbottomrule
\end{tabularx}
\end{table}

From the typological criterion adopted in this study, interrogatives in \ili{Alchuka} qualify as \isi{K-interrogatives}. Regardless of the exact \isi{reconstruction} that I intend to clarify in future studies, \ili{Proto-Tungusic} clearly has to be classified in the same way.

\begin{table}
\caption{Cognates of *\textit{Ka-bV-sɨ-kii}}
\label{tab:tungu:8}

\begin{tabularx}{\textwidth}{XXl}
\lsptoprule

\textbf{Language} & \textbf{Form} & \textbf{Meaning}\\
\midrule
\ilit{Even} & awaskii {\textasciitilde} awuskii & whither\\
\ilit{Negidal} & awaskii & whither\\
\ilit{Evenki} & awaskii & whither\\
\ilit{Oroch} (Schmidt) & avasee & whither\\
\ilit{Kilen} (An Jun) & \textbf{χ}aoɕi & whither\\
\ilit{Ulcha} (Schmidt) & \textbf{x}avasi & whither\\
\ilit{Nanai} & \textbf{x}aosi & whither\\
\ilit{Manchu} & a\textbf{b}si & whither, how\\
\ilit{Sibe} (Zikmundová) & a\textbf{f}ś(e) & whither, how\\
\lspbottomrule
\end{tabularx}
\end{table}

\cite[114f.]{Benzing1956} has three more reconstructions (*\textit{xalii} ‘when’, *\textit{xason} ‘how much’, and *\textit{xaba-sıkii} ‘whither’), all of which exhibit several deficits. Only the last one is attested in \ili{Jurchenic} languages. The first two may perhaps be corrected to *\textit{Kaali} and *\textit{Kasu(n)} (see the description of individual languages below). The last form *\textit{xaba-sıkii} poses several problems that cannot be solved easily, but I propose the slightly different \isi{reconstruction} *\textit{Ka-bV-sɨ-ki(i)} instead (\tabref{tab:tungu:8}). Apparently we are dealing with a \isi{case} form, more exactly a directive, of an otherwise unknown \isi{interrogative} starting with \textit{K{\textasciitilde}} that has parallels in the \isi{demonstratives}, e.g. \ili{Even} \textit{ə-wə-ski(i)} ‘in this direction, hither’, \textit{ta-wa-ski(i)} {\textasciitilde} \textit{ta-wu-ski(i)} ‘in that direction, thither’, and \textit{a-wa-ski(i)} {\textasciitilde} \textit{a-wu-ski(i)} ‘in what direction, whither’ (\citealt{Benzing1955}: 77f.; \citealt{Benzing1955}: 86, 113f.). \ili{Manchu} preserves the forms \textit{ebsi} ‘hither’ (\ili{Alchuka} \textit{kə’uʐï}), \textit{yabsi} ‘how very’, and \textit{absi} ‘whither’ which has acquired the meaning ‘how, why’.

\ili{Sibe} \textit{afś(e)} is a regular continuation of \ili{Manchu} \textit{absi}. It seems possible that the final element was only present in \ili{Ewenic} but not in the other \ili{Tungusic} languages. Note that there are several \isi{case} forms that may either stand alone or may be combined with a comparable suffix, e.g. \ili{Solon} dative \textit{-dU}, ablative \textit{-dU-xi}. Apart from the \isi{case} suffix, there is another element *\textit{-bV}, possibly of nominal origin, that might also be present in \ili{Proto-Tungusic} *\textit{Ka-bV-gu(u)} ‘which’, an \isi{interrogative} that had not been reconstructed by \citet{Benzing1956} (\tabref{tab:tungu:9}). \ili{Even} \textit{aw-gic} ‘whither’ could go back to the same source *\textit{Ka-bV-}.

\begin{table}
\caption{Cognates of *\textit{Ka-bV-gu(u)} ‘which one > who’ (cf. \citealt{Hölzl2014b}); Schmidt = \cite{Schmidt1923b}, Grube = \cite{Grube1900}, Castrén = \cite{Castrén1856}}
\label{tab:tungu:9}

\begin{tabularx}{\textwidth}{XXl}
\lsptoprule

\textbf{Language} & \textbf{Form} & \textbf{Meaning}\\
\midrule
\ilit{Uilta} & \textbf{x}aawu & which (one)\\
\ilit{Ulcha} (Schmidt) & \textbf{x}avu & which (one)\\
?\ilit{Nanai} (Grube) & \textbf{x}awui & which (one)\\
\ilit{Evenki} (Castrén) & awguu {\textasciitilde} a{b̴}guu & which (one)\\
\ilit{Evenki} & avgu & which (one)\\
\ilit{Evenki} (Khamnigan) & abguu & which (one)\\
\ilit{Even} & awug {\textasciitilde} awag & which (one)\\
\ilit{Negidal} & L avvu, avgu, au, U avgavu & which (one)\\
\ilit{Solon} (Hailar) & a$\gamma $uu & who\\
\ilit{Solon} (Huihe) & awu & who\\
\ilit{Oroqen} (Nanmu) & awu & who\\
\lspbottomrule
\end{tabularx}
\end{table}

\newpage 
In \ili{Uilta}, the initial consonant has been preserved as \textit{x-}, but intervocalic *V\textit{b}V and *V\textit{g}V have both been regularly lost (\citealt{Benzing1956}: 30, 34). The final *\textit{uu} must have changed to \textit{wu} following the newly formed long vowel \textit{aa}. \ili{Uilta} in addition has a special accusative form \textit{xaakkoo} (\citealt{Tsumagari2009b}: 4, 7f.) which might indicate the presence of an earlier consonant other than \textit{w} since only stems ending in -CV show this type of assimilation of the accusative marker \textit{-BA} and the geminate \textit{kk} indicates a plosive. This consonant may have been a relic of the original *\textit{g}. My \isi{reconstruction} is almost identical to \citegen[68]{Kazama2003} *\textit{xabagu}. But the vowel in the second syllable is not entirely certain as it has been lost in several languages and shows variation between \textit{a} {\textasciitilde} \textit{u} in \ili{Even}. The intervocalic *V\textit{b}V changed to \textit{w} in northern \ili{Tungusic} languages. The \ili{Even} variant \textit{awu.n} indicates that the final *\textit{-gu(u)} is a suffix that replaces the unstable nasal. In \ili{Solon}, the *\textit{b} > \textit{w} was lost and the *\textit{g} changed to \textit{$\gamma $}. After the \textit{$\gamma $} had been lost in some \ili{Solon} dialects, the final long vowel must have changed to \textit{wu} as in \ili{Uilta}. The second possibility that \ili{Solon} \textit{a$\gamma $uu} {\textasciitilde} \textit{awu} goes back to a form without the suffix *\textit{-gu(u)} in which the *\textit{b} changed to \textit{$\gamma $} is less likely due to the presence of a long vowel that can only be traced back to the suffix. \ili{Oroqen} \textit{awu} is a \ili{Solon} loanword. Some points remain unclear, however. For example, does Khamnigan have a \textit{b} instead of the expected \textit{w}, because of the following \textit{g} and how does the Upper Amgun \ili{Negidal} form \textit{avgavu} fit into the picture? Possibly there was a variance between different suffixes, such as in \ili{Evenki} \textit{idy-}\textit{vu}, \textit{idy-gu} ‘which one’. \cite[4f.]{Cincius1975/77} includes \ili{Manchu} \textit{absi} ‘how’ in the list of cognates, which is clearly a mistake.

There is one rather problematic \isi{interrogative} that has several functions and can have both verbal and nominal properties. In \isi{interrogative} sentences the meaning is extremely broad as it may be translated as ‘who’, ‘what’, ‘which’, ‘where’ or ‘how many’ (\citealt{BulatovaGrenoble1999}: 24). Given its unclear semantics it has been glossed as \textsc{int} (\isi{interrogative}). Consider the following example from \ili{Evenki}.

\newpage 
\ea%78
    \label{ex:tungu:79}
    \ili{Evenki}\\
    \gll \textbf{{aŋ}}\textbf{{ii}} \textbf{{aŋ}}\textbf{{ii}}{-$\beta $a} \textbf{{aŋ}}\textbf{{ii}}{-ǯa-ra-n?}\\
    \textsc{int}  \textsc{int}-\textsc{acc}  \textsc{int}-\textsc{ipfv}-\textsc{prs}-3\textsc{sg}\\
    \glt ‘Who is doing what?’ (\citealt{BulatovaGrenoble1999}: 26)
    \z

Problematically, the word may also be used in declarative sentences where it may “replace nearly any verb” (\citealt{BulatovaGrenoble1999}: 26) or may also function as a demonstrative. Given that cognates from the \ili{Nanaic} branch do not show an initial consonant, this word is clearly of a different origin than the other interrogatives. The best treatment of this unusual word has been given by \cite[301ff.]{Idiatov2007}. Elaborating on \cite{Cincius1975/77}, he gives the following account. The word started out as a noun meaning something like ‘thing’, which in \ili{Evenki} may have been combined with the genitive or the alienable \isi{possession} marker. The second step was the development of a “placeholder or filler”, such as \ili{English} \textit{whatchamacallit} \citep[302]{Idiatov2007}. This function is attested in several other \ili{Tungusic} languages. The last step was from a placeholder to an \isi{interrogative}. Since the last function is restricted to \ili{Evenki}, the forms from other languages will not be treated here any further.

\ili{Tungusic} interrogatives exhibit several striking similarities to \ili{Mongolic} that cannot be explained by chance (\tabref{tab:tungu:10}). These comparisons do not stand on their own but join well-known similarities in the personal pronouns and \isi{demonstratives}.

\begin{table}
\caption{Similar interrogatives in Mongolic and Tungusic}
\label{tab:tungu:10}

\begin{tabularx}{\textwidth}{XXl}
\lsptoprule
& \textbf{Mongolic} & \textbf{Tungusic}\\
\midrule
what & *ya-xu/n & *ja- (northern + *-ku/n)\\
to do what & *ya-xa- & *ja- (only northern and \ili{Kilen})\\
how many/much & *kedü- & *Kadu\\
when & *keli & *Kaali\\
\lspbottomrule
\end{tabularx}
\end{table}

This is not the place to present a discussion of a possible genetic connection between \ili{Mongolic} and \ili{Tungusic}, but it should be pointed out that \isi{language contact} could also account for these similarities (see \sectref{sec:5.8.3}). Most likely, the forms have been borrowed by \ili{Tungusic} because the \isi{morphology} involved is also known from other elements in \ili{Mongolic}, such as \isi{demonstratives}, e.g. *\textit{e.li}, *\textit{te.li} or *\textit{e.dü-}, *\textit{te.dü-} \citep[20]{Janhunen2003a}. The first two from the list also appear to have been borrowed by \ili{Nivkh} (\sectref{sec:5.2.3})

The following will address interrogatives in the individual branches of \ili{Tungusic} in turn. \tabref{tab:tungu:11} gives some interrogatives from \ili{Arman} and \textbf{Even}. \ili{Even} has some unique developments in \isi{interrogative} paradigms (\tabref{tab:tungu:12}). While the stem extension of the \isi{interrogative} \textit{i-}, an extension from the \isi{demonstratives}, is shared by most \ili{Ewenic} languages, \ili{Even} \textit{i-rəə-k} exhibits a further innovation. The final \textit{-k} stems from the \isi{interrogative} \textit{ja-k} and even found its way into the \isi{demonstratives} and may tentatively be analyzed as a newly formed nominative marker that is restricted to these four stems.

\begin{table}
\caption{Interrogatives in different dialects of Arman and Even (\citealt{DoerferKnüppel2013}, modified; \citealt{Benzing1955,Sotavalta1978}: 12, passim, modified; \citealt{Schiefner1874}); DK = Doerfer \& Knüppel, B = Benzing, S = Sotavalta, Sch = Schiefner; case forms and several alternatives are not shown}
\label{tab:tungu:11}

\begin{tabularx}{\textwidth}{lllll}
\lsptoprule
& \textbf{\ili{Arman} (DK)} & \textbf{\ili{Even} (B)} & \textbf{Western \ili{Even} (S)} & \textbf{Anadyr (Sch)}\\
\midrule
who & ṇii, ŋii & ŋi & ŋii & ŋi\\
what & jaa-\textbf{ḳ} & ja-\textbf{k} & jɛ-\textbf{k} & ja-\textbf{k}\\
which & \textbf{iää}-ra-\textbf{k} & i-rəə-\textbf{k} & i-rä-\textbf{k} & i-\\
to do what & jaa- & ja- & iɛ- & (i)jä-\\
why (\textsc{cvb}) & jaa-m{i̮}, jaa-ŋgai & ja-mi & iɛ-gaji & (i)jä-m(i)\\
which one &  & awug &  & \\
where & i-lee, \textbf{iää}-laa & i-ləə & i-lä, i-la, i-DDɯ & i-la\\
how & oon, uun & on & \textsuperscript{i}ɛtat, \textsuperscript{i}ɛkat & on\\
when & ooḳ & ok &  & ok\\
how much/many & aadii, adal & adi & adi & \\
how much/big & as{̇u}n & asun & ahun {\textasciitilde} ahɯn & \\
\lspbottomrule
\end{tabularx}
\end{table}

\begin{table}
\caption{Nominative and accusative case forms of interrogatives and demonstratives in Even (\citealt{Benzing1955}: 77, 79)}
\label{tab:tungu:12}

\begin{tabularx}{\textwidth}{XXXXXl}
\lsptoprule
& \textbf{who} & \textbf{what} & \textbf{which} & \textbf{this} & \textbf{that}\\
\midrule
\textsc{nom} & ŋi & ja-\textbf{k} & i-\textbf{rəə}-\textbf{k} & ə-\textbf{rəə}-\textbf{k} & ta-\textbf{ra}-\textbf{k}\\
\textsc{acc} & ŋi-w & ja-w & i-\textbf{rə}-w & ə-\textbf{rə}-w & ta-\textbf{ra}-w\\
\lspbottomrule
\end{tabularx}
\end{table}

The \textbf{Evenki} and \ili{Negidal} \isi{interrogative systems} are extremely similar to one another. Generally, the forms tend to be a bit longer than those in \ili{Even}. In Khamnigan \ili{Evenki}, while the Urulyungui dialect preserves a small difference between \textit{ie-} and \textit{ii-}, the two interrogatives *\textit{ja-} and *\textit{Kai-} completely coalesced into \textit{i(i)-} in the Borzya dialect. In general, Khamnigan \ili{Evenki} interrogatives appear to be more closely related to \ili{Oroqen} than to \ili{Evenki}. Apart from this partly shared sound change, both groups have also changed the initial \isi{velar nasal} to an alveolar nasal in \textit{nii} ‘who’ and have a form \textit{aali} ‘when’ instead of \textit{ookin} in \ili{Evenki}. But apart from \textit{awu} ‘who’, which is borrowed from \ili{Solon}, \ili{Oroqen} does not have a cogante of \textit{abguu} ‘which one’. The Khamnigan form \textit{iir-giiji} {\textasciitilde} \textit{iir-giid} has a cognate in \ili{Oroqen} \textit{iri-}\textit{gidə} and \ili{Evenki} \textit{ir-git}. \ili{Evenki} dialects, such as the one from \isi{Sakhalin}, exhibit a very similar \isi{interrogative} system but shows some regular phonological differences (e.g., \textit{a\textbf{x}un} ‘how many’, \citealt{BulatovaCotrozzi2004,Atknine1997}).

\begin{table}
\caption{Evenki (\citealt{Nedjalkov1997}: 3-18, 135-136, 214-216, 318ff.), Negidal (\citealt{Cincius1982}: 34, passim), Khamnigan Evenki (\citealt{Janhunen1991}: 70f.), and Aoluguya Evenki interrogatives \citep[171, 238]{Hasibateer2016}; U = Upper Amgun, L = Lower Amgun dialect of Negidal. B = Borzya, U = Urulyungui dialect of Khamnigan Evenki}
\label{tab:tungu:13}

\begin{tabularx}{\textwidth}{QlQQl}
\lsptoprule

\textbf{Meaning} & \textbf{Evenki} & \textbf{Negidal} & \textbf{Khamnigan} & \textbf{Aoluguya}\\
\midrule
who & ŋi & nii, n’ii, ŋii & nii & niː, nɪː\\
what & e-kun & L ee-xun,\newline U ee-kun & B i-kun,\newline U ie-kun & æː-qɷn, æː-ʁɷn\\
what kind of\newline (\textit{-dy} ‘\textsc{adj’}) & e(e)-ku-dy &  & &\\
what kind of (\textsc{pl})\newline (\textit{-ma} ‘\textsc{adj’}) & e(e)-ku-ma &  & &\\
why\newline (\textit{-da} ‘-\textsc{cvb}’) & ee-da & ee-daa(j) & &\\
how much & ady & adii & adii & adɪ, addi\\
how many times & adyra & adiijaa & &\\
by what number\newline (\textit{-}\textit{tA}\textit{l} ‘\textsc{quant}’) & ady-tal & adiital & &\\
which & anty &  & &\\
what, which & aŋi & aŋ, aŋi, aŋe ‘yes, alright, okay’ & &\\
how many & asun & asun & &\\
which, what kind & avady &  & & irəgeɕin, irgəːtʃin\\
whither &  & awaskii & &\\
what, which & avgu & L avvu, avgu, au, U avgavu & abguu &\\
which & i-r & ii & & ɪː-ra, iːʂ\\
where & i-du & ii-duu, ii-laa & ii-dvv, B i-lee, U ii-lee & iː-du, ɪː-la, ɪː-ra\\
which one & idyvu, idygu &  & &\\
how much/many & oki &  & ooki & iːrba, ɪrɢaː\\
when & ookin & L ooxin,\newline U ookin & aali & ɔːqin, ɔʁːin\\
how & oon & oon(i) & oon & æːχa\\
\lspbottomrule
\end{tabularx}
\end{table}

There are descriptions for several \ili{Oroqen} dialects, the interrogatives of which are given in \tabref{tab:tungu:17}. Only a selection of \isi{case} forms is included. The complex forms \textit{ixun-tʃaalin} or \textit{ɪkʊn dʒaalɪn} ‘why’ and \textit{adi erin-du} ‘when’ contain the \ili{Manchu} loanwords \textit{jalin} ‘reason’ and \textit{erin} ‘time’. The second part in \textit{iri-}\textit{gətʃin} ‘what kind of’ is probably not \ili{Manchu} \textit{hacin} ‘kind, sort, class, item’ (from \ili{Korean}) because there is a similar suffix in other \ili{Ewenic} languages \citep[100]{Benzing1956}, e.g. \ili{Evenki} \textit{-gAchin} ‘similar to, just as, like’ \citep[56]{Nedjalkov1997}, e.g. Aoluguya \ili{Evenki} \textit{irəgeɕin} {\textasciitilde} \textit{irgəːtʃin}. The suffix \textit{-du} is a locative and dative \isi{case} marker that can also be found in \textit{oki-du} ‘when’ (based on \textit{oki} ‘how much’, influenced by \ili{Solon}), \textit{ixu-tu} ‘when’ (based on \textit{i-xun} ‘what’ with stem extension) and \textit{(i)i-tu} ‘where’ (based on \textit{i-(xun)} ‘what’ without stem extension). The etymology of \textit{idʒirgee} ‘which one’ remains unclear for now. The Nanmu forms \textit{awu} ‘who’, \textit{oonde} ‘what kind of’, and \textit{joonde} ‘why’ have been adopted from \ili{Solon}. The same is probably true for \textit{iktu} ‘how’ as mentioned by Chaoke. The origin of \textit{j\textsc{ee-}ma} ‘which one’ remains unclear, but a connection to \ili{Mongolic} seems plausible. One can observe a slow phonological convergence of the two different stems \textit{i-hun} ‘what’ and \textit{i-r(i)} ‘which’.

\begin{table}
\caption{Interrogatives in different dialects of Oroqen (\citealt{HuZengyi2001}: 101, 148, 261; 1986: 94; \citealt{Chaoke2007}: 47, 257f.; \citealt{ZhangLiZhang1989}: 56, 141, 184; \citealt{HanMeng1993}: 43, 264-265; CK = \citealt{Chaoke2014a}: 164-165, passim)}
\label{tab:tungu:17}
\small
\begin{tabularx}{\textwidth}{lQQQlQQ}
\lsptoprule
& \textbf{Chaoyang} & \textbf{Wulubutie} & \textbf{Nanmu} & \textbf{Xunke} & \textbf{Shengli} & \textbf{CK}\\
\midrule
who & nii & nii & \textbf{awu} & nii & nii & ni\\
what & ɪkʊn & ɪkʊn & ihun & ikʊn & ixun & ikun\\
when & aalɪ(ɪ) & aalɪ & aala & aali & aali, \textbf{ixu-tu} & oki-du, \textbf{adi erin-du}\\
how many &  &  & adi &  & ati & adi\\
what kind of & iri-gətʃin, iŋŋətʃin, innuwəən & iri-gətʃin, iŋŋətʃin, innuwəən & \textbf{oondi} & iri-gətɕin & iri-kətʃ’in, iŋNetʃ’in & \\
which & i-ri & i-ri & i-r(i) &  & (i)i-ri & i-ri\\
which one &  &  &  & \textbf{j\textsc{ee}}\textbf{ma} & \textbf{jeman} & \\
which one (\textsc{pl?}) & idʒirgee & idʒirgee &  & idʑirgə & itʃərkee & \\
where & i-ləə, i-du &  & i-lə & i-rə & i-ləə, iitu & i-le(-ni/ŋi)\\
whither & iri-gidə & iri-gidə & iri-gidə & iri-gidə & iri-kita, iir-tə & \\
why & ɪda, \textbf{ɪkʊn dʒaalɪn} &  & iida, iima, \textbf{joonde} &  & \textbf{ixun-tʃaalin} & \\
how & ɔɔn & ɔɔn & ooni & ɔɔn & (w)ɔɔn & oon, \textbf{iktu}\\
how much & ɔɔkɪ &  & oohi & ɔɔki & ɔɔxi & (o)oki\\
\lspbottomrule
\end{tabularx}
\end{table}

\textbf{Solon} interrogatives are probably the most aberrant among \ili{Ewenic} languages. The \isi{interrogative} \textit{ni(i)} ‘who’ has been almost completely replaced (see \tabref{tab:tungu:20}). The unexpected vowel quality in \textit{(j)o-xon} ‘what’ can possibly be attributed to influence from \ili{Dagur} (\textit{yoo(n)} ‘what’). \textit{ohi-}\textit{du} ‘when’ similar to \ili{Evenki} is based on \textit{ohi} ‘how much’, but contains an additional locative marker. This form has been adopted by one \ili{Oroqen} dialect while Ongkor \ili{Solon} \textit{aali} ‘when’ in turn can perhaps be traced back to influence from \ili{Oroqen}. \ili{Dagur} \textit{yoondaa} ‘how, why’ is perhaps the source of \textit{joodaa} ‘why’.

\begin{table}[t]
\caption{Interrogatives in different dialects of Solon (\citealt{Chaoke2009}: 35f., 250ff., 351f., 355; \citealt{Tsumagari2009a,Poppe1931}: 110); CK = Chaoke, T = Tsumagari, P = Poppe, R = Ramstedt (\citealt{Aalto1976}; 1977, modified), K = Kamimaki (Lie \citealt{Lie1978}: 175, 177, modified); case forms are not listed}
\label{tab:tungu:20}

\begin{tabularx}{\textwidth}{lQQQQQ}
\lsptoprule
& \textbf{Huihe (CK)} & \textbf{Huihe (T)} & \textbf{Hailar (P)} & \textbf{Ongkor (R)} & \textbf{Ongkor (K)}\\
\midrule
who & awu, \textbf{ni} & aawu, \textbf{nii} & a$\gamma $uu & auu {\textasciitilde} a$\gamma $uu & ahu\\
what & o-hoŋ & ii, o-xon & ii & i(i),\newline \textbf{j}o-xon & o-hon, u-hun\\
when & ohi-du & ooxi-du & ooxii-du & \textbf{aali} & oke-du\\
what kind of & ondi & oondii & oondii & oondin & onde\\
which & iggʉ & \textbf{ii-}\textbf{r}, iggü & iixɯɯ &  & \\
where & i-lə & ii-lee & ii-ləə & ii-lə & i-le, i-lo\\
how & ittʉ & iittü & iittɯɯ &  & \\
why & ida, joodoŋ & yoodon & iidaa, joodaa & ida, joxon-du & yoda\\
how many & adi & adi & adii & adi & aade\\
how much & ohi & ooxi & ooxii, \textbf{oondii} & ookɪ & ooke,\newline \textbf{one}\\
\lspbottomrule
\end{tabularx}
\end{table}

The origin of \textit{i}\textbf{\textit{gg}}\textit{ʉ} ‘which’ and \textit{i}\textbf{\textit{tt}}\textit{ʉ} ‘how’ is unclear, but in \ili{Solon} a geminate suggests the earlier presence of a consonant cluster as can be seen in many examples, e.g. \ili{Evenki} \textit{i}\textbf{\textit{rg}}\textit{i}, \ili{Solon} \textit{i}\textbf{\textit{gg}}\textit{i} ‘tail’. Possibly, the forms are based on the stem \textit{i(i)-r(i)}, followed by a \isi{case} ending. At least synchronically the form with the suffix \textit{-r(i)} has no wide distribution among \ili{Solon} dialects, which usually employ the bare stem \textit{i(i)}. But further evidence for this view can be gleaned from the \isi{demonstratives} \textit{e-ri} ‘this’ and \textit{ta-ri} ‘that’ that still have the extensions, and the derived forms \textit{e}\textbf{\textit{tt}}\textit{ü} ‘in this way’ and \textit{ta}\textbf{\textit{tt}}\textit{ü} ‘in that way’ (\citealt{Tsumagari2009a}: 3, 6). Problematically, from a \isi{synchronic} perspective no \isi{case} marker has the expected form *\textit{-gü} or *\textit{-tü}. At least the latter may have a connection with the dative \textit{-du} {\textasciitilde} \textit{-dü} that in \ili{Evenki} also has a variant \textit{-tu} with an unvoiced consonant. Ramstedt’s Ongkor \ili{Solon} materials have been recorded in Tacheng (\citealt{Lie1978}: 128). From the city Alimtu, Muromskij collected several unproblematic forms including \textit{au} ‘who’, \textit{ad’} {\textasciitilde} \textit{ad\~\i} ‘how many’, \textit{ile} ‘whither’, \textit{ida} ‘why’, \textit{on(i)} ‘how’ (\citealt{Lie1978}: passim, \citealt{Kałużyński1971}: passim). \citet[63]{ChaokeKajia2014} mention a form \textit{antie} ‘how’ that seems to correspond to \ili{Evenki} \textit{anty} ‘which’.

\textbf{Kili} is a mixture of \ili{Nanaic} with \ili{Ewenic} elements \citep{Doerfer1978a}, but judging from the interrogatives alone, \ili{Kili} appears to have stronger affinities to \ili{Ewenic} than to \ili{Nanaic} (e.g., \textit{ŋii} ‘who’, \textit{e-ma} ‘what’, \textit{adii} ‘how many’, \textit{ali} ‘when’, \textit{i-du} ‘where’, \textit{ii-daj} ‘why’, \textit{osi} ‘which one’, \citealt{Kazama2003}: passim; see \citealt{Sunik1958} for details). Note the characteristic form \textit{ŋii} as well as the absence of the initial consonant \textit{x-}. \ili{Kili} \textit{osi} remains obscure but has a cognate in \ili{Kilen} \textit{ɔɕi}. The two interrogatives \textit{iidaj} and \textit{ema} are also characteristic of \ili{Ewenic}, but might stem from two different sources as indicated by the different length and quality of the initial vowel.

As expected, interrogatives in \textbf{Udegheic} languages show affinities with \ili{Ewenic}. \tabref{tab:tungu:23} gives an overview of some forms attested in \ili{Udihe} and \ili{Oroch}.

\begin{table}
\caption{Udihe (\citealt{NikolaevaTolskaya2001}: 348ff.; \citealt{TolskayaTolskaya2008}: 100) and Oroch interrogatives (\citealt{AvrorinBoldyrev1978,Lopatin1957}, collected in 1924, modified); not all variants are listed}
\label{tab:tungu:23}

\begin{tabularx}{\textwidth}{XXll}
\lsptoprule
& \textbf{Udihe} & \textbf{Oroch} & \\
\midrule
& NT  & AB & Lopatin\\
\midrule
who & ni(i) & n’i & n\textsuperscript{(y)}i\\
what & j’e-u & jā(-ʊ) & ya(-u)\\
what kind of & je-me & &\\
why & je(-ne)-mi, ii-mi & jæ-mi, jə̄-mi, jā-na-mi, jī-mi & \\
what, where, why & j’efe & jāvʊ & yava\\
when & ali & āli & ale\\
how many & adi & adi & ady\\
how & ono & ōn’i & oni\\
which & onobui & ōn’i bi & \\
\lspbottomrule
\end{tabularx}
\end{table}

\citet[348]{NikolaevaTolskaya2001} claim that \textit{j’e-fe} ‘what, where, on which place’ is an accusative form. It seems, however, that this form rather corresponds to \ili{Manchu} \textit{ya-ba} and \ili{Sibe} \textit{ya-va} ‘where, which place’. The strange looking form \textit{onobui} is probably a contraction of \textit{ono} ‘how’ with an inflected form of the copula \textit{bi-}, which has a parallel in \ili{Kilen}. Apart from this, several more \ili{Udihe} interrogatives have been adopted by \ili{Kilen} as well (see below).

The \isi{interrogative} \textit{ni(i)} ‘who’ is declined as the word \textit{nii} (or \textit{ninta}) ‘man’ but does not have an etymological connection to it, as claimed by \citet{Schulze2007}. Instead, the forms correspond to \ili{Nanai} \textit{ui} and \textit{nai}, respectively, and are similar only by chance. But one cannot exclude the possibility of a folk etymological connection. The \isi{interrogative} \textit{j’eu} exhibits some irregularities. Apart from the nominative forms, the paradigms are parallel to the \isi{demonstratives} (\tabref{tab:tungu:24}). The ending \textit{-u} in \textit{j’e-u} or \textit{jaa-u} ‘what’ is identical in origin with \ili{Evenki} \textit{-kun} in \textit{e-kun}. In \ili{Oroch}, but not in \ili{Udihe}, this extension is also found in most \isi{case} forms. This is a secondary leveling that has a parallel in \ili{Evenki} and \ili{Oroqen}. \ili{Udihe} \textit{j’euxi} ‘whither’ probably corresponds to \ili{Nanai} \textit{xaosi} but is based on a different stem. As in \ili{Nanai} the same suffix \textit{-uxi} can otherwise only be found in the \isi{demonstratives}.

\begin{table}
\caption{Interrogative and demonstrative paradigms in Udihe (\citealt{NikolaevaTolskaya2001}: 100, 343f., 348) and Oroch (\citealt{AvrorinBoldyrev2001}: 193, 197)}
\label{tab:tungu:24}
\small
\begin{tabularx}{\textwidth}{llllQllll}
\lsptoprule
& \textbf{Udihe} &  &  &  & \textbf{Oroch} &  &  & \\
\midrule
& \textbf{who} & \textbf{what} & \textbf{this} & \textbf{that} & \textbf{who} & \textbf{what} & \textbf{this} & \textbf{that}\\
\midrule 
\textsc{nom} & ni & j’e-\textbf{u} & \textbf{ee}, \textbf{ei} & \textbf{ute}, \textbf{uti}, \textbf{tee}, \textbf{tii} & n’ii & jaa-\textbf{u} & \textbf{ei} & \textbf{tei}, \textbf{ti}\\
\textsc{acc} & ni-we & j’e-we & a-wa & (u)ta-wa & n’ii-ve & jaa-va & ee-ve & taa-va\\
\textsc{dat} & ni-du & j’e-du & o-du & (u)ta-du & n’ii-du & jaa-\textbf{u}-du & ee-du & taa-du\\
\textsc{lat} & ni-tigi & j’e-\textbf{uxi} & a-\textbf{uxi}, i-tigi & \mbox{(u)ta-\textbf{uxi}}, \mbox{(u)ta-tigi} & n’ii-ǰiǰi & jaa-\textbf{u}-ǰiǰi & ee-ǰiǰi & taa-ǰiǰi\\
\textsc{loc} & ni-le & j’e-le & o-lo & (u)ta-la & n’ii-le & jaa-\textbf{u}-la & ee-le & taa-la\\
\textsc{prol} & ni-li & j’e-li & o-li & (u)ta-li & n’ii-li & jaa-\textbf{u}-li & ee-li & taa-li\\
\textsc{abl} & ni-digi & je-digi & o-digi & (u)ta-zi & n’ii-dui & jaa-\textbf{u}-dui & ee-dui & taa-dui\\
\textsc{inst} & ni-zi & j’e-zi & o-zi & \mbox{(u)ta-digi} & n’ii-ǰi & jaa-\textbf{u}-ǰi & ee-ǰi & taa-ǰi\\
\lspbottomrule
\end{tabularx}
\end{table}

In \ili{Udihe}, forms such as the dative \textit{j’e-du} or the locative \textit{j’e-le} have the variants \textit{ii-du} and \textit{ii-le}. According to \citet[349]{NikolaevaTolskaya2001}, this only represents a difference in pronunciation. However, \textit{ii-} really is the relic of a different stem of which no nominative or citation form is left in \ili{Udihe} (\tabref{tab:tungu:26}). \ili{Oroch} likewise has these alternative forms, e.g. \textit{i-du} \citep{Schmidt1928a}.

\begin{table}[t]
\caption{Selected case forms of two different interrogatives in Udihe, Even, and Manchu}
\label{tab:tungu:26}

\begin{tabularx}{\textwidth}{XlXXXXl}
\lsptoprule
& \multicolumn{2}{X}{\textbf{Udihe}} & \multicolumn{2}{X}{\textbf{Even}} & \multicolumn{2}{X}{\textbf{Manchu}}\\
\midrule
\textsc{nom} & j’e-\textbf{u} & - & ja-\textbf{k} & i-\textbf{rəə}-\textbf{k} & ya & ai\\
\textsc{dat} & j’e-du & ii-du & ja-du & i-du & ya-de & ai-de\\
\textsc{loc} & j’e-le & ii-le & ja-la & i-ləə & - & -\\
\textsc{cvb} & j’e(-ne)-mi & ii-mi & ja-mi & - & - & ai-na-me\\
\lspbottomrule
\end{tabularx}
\end{table}

In \ili{Manchu} the locative *\textit{-lA} is only preserved in relics (e.g. \textit{ama-la} ‘behind’) and the stem extension is restricted to the \isi{demonstratives} \textit{e-re} ‘this’ and \textit{te-re} ‘that’. Strangely, \ili{Udihe} also shows this variation between two stems in the \isi{interrogative} \textit{ii-}\textit{mi} {\textasciitilde} \textit{j’e-}\textit{mi} ‘why’. Given that these are \isi{converb} forms, \ili{Udegheic} is the only branch in which both stems can function as \isi{interrogative} verbs. \ili{Udihe} \textit{ii-mi} directly compares with \ili{Nanai} \textit{xaɪ}\textit{-mi} and \textit{j’e-}\textit{mi} with \ili{Even} \textit{ja-mi}. \ili{Udihe} furthermore has the variant \textit{j’e-ne-}\textit{mi}, which is similar to \ili{Manchu} \textit{ai-na-me}, but is based on the other stem.

Unlike all northern \ili{Tungusic} languages and most of \ili{Jurchenic}, \textbf{Nanaic} interrogatives form a coherent system in which all forms share the \isi{resonance} \textit{x{\textasciitilde}}. The only exceptions is the \isi{interrogative} meaning ‘who’ that was already different in \ili{Proto-Tungusic}, as well as some isolated \ili{Uilta} forms that have perhaps been borrowed from \ili{Nivkh} (\textit{sado}, \textit{saa}, \textit{nuulu}) (see \sectref{sec:5.2.3}). Patryk Czerwinski (p.c. 2018) elicited the forms \textit{sadu} and emphatic \textit{sadoo} from a northern \ili{Uilta} speaker.

\begin{table}
\caption{Interrogatives in Najkhin Nanai \citep{Kazama2007}, Ussuri Nanai (\citealt{Sem1976}), Ulcha \citep{Majewicz2011}, and Uilta (\cite{Ikegami1997}; \cite{Tsumagari2009b}; \cite{Majewicz2011}); accents removed}
\label{tab:tungu:27}

\begin{tabularx}{\textwidth}{llQlQl}
\lsptoprule
& \textbf{N. Nanai} & \textbf{U. Nanai} & \textbf{Ulcha} & \textbf{\ili{Uilta} (M)} & \textbf{\ili{Uilta} (TI)}\\
\midrule
who & ui & ui & nui & ŋuj & ŋui\\
what & xai & χay & xaj & xaj & xai\\
to do what & xai- & χay(-ra)- & xaj- &  & xai-\\
why & xai-mi & \textbf{χajo} &  &  & xai-mi\\
when & xaali & χ\=al’y & xali & xali & xaali\\
which one & xamačaa & χamatsa & xamata &  & \textbf{xaawu}\\
what for & xaigoji & χaoɴk’y, χaʊnk’y,

χaoɴk’yd’y & xajd’u & xajbu & \\
where & xai-do & χay-dʊ &  & xaj-du, \textbf{sado} & xai-du\\
whither & xaosi & χaos’(y), χaʊs’y, χay-χaydu & xajban & xavasaj, xoty, \textbf{nuulu} & xawasai\\
whence & xajaǰi &  & xajdani & xavedu & xamaččuu\\
how & xooni & χ\=on’i & xon(i) & xooni & xooni\\
how many & xado & χado, χadʊ & xasu, xadum & xasu & xasu\\
\lspbottomrule
\end{tabularx}
\end{table}

\newpage 
With respect to the other \ili{Nanaic} languages, \textbf{Kilen} exhibits a very different set of interrogatives (\tabref{tab:tungu:31}). Not only is there a rather confusing variation in the origin of the individual forms, but different accounts show striking differences as well. The forms that most closely resemble \ili{Nanai} have been collected by \citet{Ling1934} and they might actually represent \ili{Hezhen} instead of \ili{Kilen}. All other descriptions show variations between some forms of \ili{Nanaic} and some of \ili{Udegheic} or \ili{Jurchenic} origin that have been borrowed. \textit{χadi}, if this is not a typo, is an especially interesting form as it combines features typical of southern and northern \ili{Tungusic}. It shares the initial consonant typical for \ili{Nanaic}, but has a final \textit{-i} that can only be of northern \ili{Tungusic} origin. \ili{Udihe} loans include \textit{ni}, \textit{adi}, \textit{oni}, and maybe several more such as \textit{ja}, \textit{uki}, and \textit{ɔnibiɕi}, although the latter has also been recorded with an initial consonant atypical of \ili{Udihe}. \ili{Manchu} elements include the nouns \textit{jaka} ‘thing’ (in \textit{ia}\textit{-}\textit{mə-dʑaka}) and perhaps \textit{erin} ‘time’ (in \textit{ia-}\textit{ma-ərin}, \textit{adi}/\textit{ya erin-du}, and \textit{iaɾin}). The \isi{interrogative} \textit{ja} is certainly of \ili{Udihe} origin, because \citet[199]{LiLinjing2011} mentions a form \textit{ya-o} ‘what’ that can only stem from \ili{Udihe} \textit{j’e-u} but not \ili{Manchu} \textit{ya}. The \isi{interrogative} \textit{iətin} ‘when’ appears to be a \isi{combination} of \textit{ja} and perhaps an otherwise unknown noun meaning ‘time’ or suffix that can also be found in \ili{Manchu} \textit{atanggi} ‘when’, see below. The forms \textit{onnomi} and \textit{onaqami} are obscure but may contain the \isi{converb} marker \textit{-mi}. As seen before, \cite[82]{NDSSLD1958} mentions the two \ili{Kilen} forms \textit{ya-le} \zh{鴨勒} and \textit{ali} \zh{阿里}.\footnote{The character \zh{里} was incorrectly written as \zh{黑}.} \ili{Kilen} \textit{ɔɕimkən} is a contraction of \textit{ɔɕi} and the numeral \textit{əmkən} ‘one’, which is most likely of \ili{Jurchenic} origin (\ili{Manchu} \textit{emken}). The \isi{interrogative} \textit{ɔɕi} has a cognate in \ili{Kili} \textit{osi} and \textit{ikti} ‘how’ perhaps in \ili{Oroqen} \textit{iktu} ‘how’. Both remain unclear etymologically.

\begin{table}[t]
\caption{Interrogatives in different descriptions of Kilen (\citealt{Ling1934}: 243, 245; \citealt{AnJun1986}: 38, 63; \citealt{ZhangLiZhang1989}: 40, 44f., 70, 74f., 88, 144; \citealt{ZhangPaiyu2013}: 95, 162f.; \citealt{Chaoke2014b}: 164 et passim); the table also contains all available case forms. L = Ling Chunsheng, AJ = An Jun, ZZD = Zhang et al., Z = Zhang, CK = Chaoke; forms from \citet{AnJun1984} in square brackets}
\label{tab:tungu:31}

\begin{tabularx}{\textwidth}{llQQQQ}
\lsptoprule
& \textbf{L} & \textbf{AJ} & \textbf{ZZD} & \textbf{Z} & \textbf{CK}\\
\midrule
who & \textbf{ui} & ni & ni & ni & ni\\
what, which &  & ja & ia & ia/ja & ya\\
what & \textbf{h}ai & \textbf{χ}ai, [\textbf{χ}ajə] &  &  & \textbf{h}ay\\
what (thing) &  &  & ia-mə-dʑaka &  & ya-ma\\
which &  & [ɔɕə] & ɔɕimkən & ɔɕi & iri\\
where &  &  & ja/ia-du & ia-tu & i-lə, i-du\\
where from &  &  &  & ia.tu-tiki & \\
where to &  & \textbf{χaoɕi} & ia-lə & ia-tu-lə & \\
when, what time &  &  & ia-ma-ərin & \textbf{iaɾin} & \textbf{iətin}, ya erin-du, adi erin-(du/ni)\\
how & \textbf{h}ɔnibiʃi & ombiɕə, on(n)i, onnomi, [onbiɕə] & ɔni, ɔnibiɕi ‘what’ & ɔməɕi & oni, ikti\\
how many & \textbf{h}ad\textbf{u} & \textbf{χ}ad\textbf{i} {\textasciitilde} \textbf{χ}ad\textbf{u}, [adi] & adi & ati & adi\\
how much &  &  &  & uki & uki\\
why &  & onaqami & unakəmi & ɔŋnəmi, ɔməmi & \\
\lspbottomrule
\end{tabularx}
\end{table}

Interrogatives in \textbf{Jurchenic} show marked differences from the other \ili{Tungusic} languages. Almost no information is available for \ili{Bala} and the few interrogatives available for \ili{Alchuka} have already been given in \tabref{tab:tungu:7}. \ili{Sibe} and Written \ili{Manchu}, on the other hand, are exceptionally well described. \tabref{tab:tungu:32} gives an overview of interrogatives in \ili{Manchuic} languages. Aihui \ili{Manchu} \textit{ɛdzəxə {\textasciitilde}} \textit{ɛdzə$\gamma $,} \textit{ɛdik}, according to \citet[149]{Enhebatu1995}, has a cognate in Sanjiazi \ili{Manchu} \textit{aizɯg}, \textit{aizɿg}, \textit{aizɤɯ} ‘how much, many’, but remains obscure. \ili{Sibe} \textit{yask(}\textit{ə)} might be comparable as well, but seems to be based on \textit{ya} instead of \textit{ai}. The \ili{Manchu} form \textit{adarame} (\ili{Alchuka} \textit{katiram}) has never been analyzed in a clear manner. There are several possibilities, but the most likely scenario is a derivation from the \isi{interrogative} \textit{ai} that subsequently lost the \textit{i} as in other forms. If the final \textit{-me} is the imperfective \isi{converb} form as a comparison with \ili{Nanai} \textit{xaɪ}\textit{-mi} ‘why’ might suggest, then at least one of the other elements present might have been a verbalization. Both \textit{-dA} and \textit{-rA} are attested in this usage, but their \isi{combination} would be most unusual. Problematically, the verbal \isi{interrogative} in \ili{Manchu} has the regular form \textit{ai-na-} (\ili{Bala} \textit{a-na-}, \ili{Alchuka} \textit{kai-na-}). In fact, \ili{Manchu} also has the expected form \textit{ai-na-me} ‘how’. Perhaps the form has to be analyzed as *\textit{a(i)-da-ra-me} with an unclear derivation of the \isi{interrogative} stem. The forms \textit{ainu} and \textit{antaka} (\ili{Alchuka} \textit{kent’aka}) are even more obscure but probably derive from *\textit{Kai-}, too.

\begin{table}[t]
\caption{Interrogatives in Manchuic (\citealt{Norman2013,Zikmundová2013,WangQingfeng2005,KimJuwon2008,ZhaoJie1989}); most case forms and some variants are not listed}
\label{tab:tungu:32}

\begin{tabularx}{\textwidth}{lQQQQl}
\lsptoprule
& \textbf{Written} & \textbf{Sibe} & \textbf{Aihui} & \textbf{Sanjiazi} & \textbf{Yibuqi}\\
\midrule
who & we & və & və & wə & və\\
what & ai & ai & ɛ & aj & ɛi\\
to do what & aina- & ainə- &  &  & \\
which & ya & ya &  & ja & j\textsc{a}\\
when & atanggi,

ai erinde & aitiŋ, ya erin-t & ɛtiŋ(ŋe),

ɛ(j)irin & ajtiŋ, aj əlin & ɛi t‘iŋkə\\
what kind of thing & ai-jaka, ya jaka & ai jaq(ə) & atȿaʁa, \textbf{ɛd}\textbf{ɛre} & aj ǰakə & ɛi tʂ\textsc{a}χa\\
what (is it) & antaka & antq &  &  & \\
which one & ya emken, yaka & yam, yamkə\textsuperscript{n} & ja(m)kən, jama &  & j\textsc{a} əmk‘ə\\
where & aiba-, aibi-, yaba-,

ya-, ai- & ɛ.vi-,

yava-,

ye-, ai- & ɛba-,

java & jawu- {\textasciitilde}

jawə(-) & j\textsc{a}p\textsc{a-}\\
how & absi, adarame, ainame & afś(e) & əvɕe {\textasciitilde} avɕe & adəlmən & atər(ə)mə\\
why & ainu, ai turgun & a\textsuperscript{n} & an {\textasciitilde} aŋ, ana {\textasciitilde} anə & ai tulxun & an\textsc{a}\\
how many & udu & ut, \textbf{yask(}\textbf{ə)} & \textbf{udɛ}\textbf{le, ɛ}\textbf{dzəxə {\textasciitilde}} \textbf{ɛdzə$\gamma $,}

\textbf{ɛdik} & udə, \textbf{udali} & utu\\
\lspbottomrule
\end{tabularx}
\end{table}

Unfortunately, there is almost no record of \textbf{\isi{Jurchen} A} interrogatives. However, there is a form that has been reconstructed as *\textit{wanon} \zh{晚灣} ‘how’. \citet[137]{Kiyose1977} hypothesized that it might be connected to \ili{Manchu} \textit{antaka}, but this is not very convincing. At first glance, no similar \isi{looking} form is attested in any other \ili{Jurchenic} language. The only tentative solution that I can think of, apart from an altogether unknown \isi{interrogative} or a mistake, is to compare *\textit{wanon} with \ili{Manchu} \textit{ai-na-me}, which has the same meaning and shows some remote formal resemblance. The initial *\textit{w-} is extremely problematic but could perhaps be a reflex of an initial consonant, cf. \ili{Alchuka} \textbf{\textit{k}}\textit{ai-na-}. The \isi{interrogative} \textit{ai} lost the \textit{i} in some other instances as well, cf. \ili{Bala} \textit{a-na-}. What has been reconstructed as *\textit{-n} might thus be a \isi{converb} form that does not, however, match \ili{Manchu} \textit{-me} or \ili{Bala} \textit{-mi} (\citealt{MuYejun1987}: 30). The \isi{converb} has been recorded in the form \textit{-n} in the modern Aihui dialect, but such a comparison would be anachronistic. Nevertheless, a \isi{converb} form would be expected because the following word was a verb. \citet[140]{Kiyose1977} assumes that *\textit{ain} \zh{爱因} may be the same \isi{interrogative} as \ili{Manchu} \textit{ainu}, which seems accurate. \citet[144]{Kiyose1977} also mentions a form \textit{adi} \zh{阿的} that has been translated as ‘etc.’ This might correspond to \ili{Manchu} \textit{udu} ‘how many/much’, but, if true, is closer to northern \ili{Tungusic}. Notice that \ili{Manchu} \textit{udu} may also mean ‘several’, which is a bit closer semantically.

\begin{table}[t]
\caption{Analytical interrogatives in Manchu (\citealt{Hauer2007,Norman2013,Hölzl2015c})}
\label{tab:tungu:33}

\begin{tabularx}{\textwidth}{llXl}
\lsptoprule

\textbf{Form} & \textbf{Meaning} & \textbf{Basis} & \textbf{Meaning}\\
\midrule
ai-ba-, ya-ba- & where & ba & place\\
ai-erin- & when & erin & time\\
ya-emu, ya-emken & which one & emu {\textasciitilde} emken & one\\
ai-hacin, ya-hacin & what kind of & hacin & kind, sort, class\\
ai-haran & why & haran & reason\\
ya-ici & whither & ici & direction\\
ai-jaka, ya-jaka & what sort of & jaka & thing\\
ai-niyalma, ya-niyalma & what person, who & niyalma & person\\
ai-se- & to say what & se- & say-\\
ai-se-me & why & se-me & say-\textsc{cvb.ipfv} > \textsc{quot}\\
ai-turgun & why & turgun & reason\\
atanggi & when & ? & ?time\\
\lspbottomrule
\end{tabularx}
\end{table}

It is well-known that \ili{Manchu} has a very special and aberrant position among \ili{Tungusic} languages. In fact, the differences are so strong that I have previously put forward the possibility that, in \citegen{Operstein2015} terminology, \ili{Manchu} is a \isi{contact} variety that shows a certain amount of \isi{simplification} (\citealt{Hölzl2012}). An additional argument in favor of this hypothesis is the existence of many analyzable interrogatives that consist of either \textit{ai-} or \textit{ya-} together with a noun or, in some cases, another element (\tabref{tab:tungu:33}). Most forms are not normally written in one word in \ili{Manchu}, but may be analyzed as compounds. Most dialectal forms that go back to these compounds have not been listed above. Most of these formations are very \isi{transparent}. In some cases there is an unexpected semantics, such as in \textit{ai-se-me} ‘why’. \citet{Hauer2007} additionally mentions the form \textit{ainam.baha-} ‘to get how’, which is a \isi{combination} \textit{ainame} ‘how’ and \textit{baha-} ‘to get’. It is highly doubtful that \textit{bi-} in \textit{ai-bi-} is the copula \textit{bi} as claimed by \citet[219]{Gorelova2002} because it would be impossible to attach a \isi{case} marker to it, e.g. \textit{ai-bi-de} ‘where’. Most likely it really is a variant of \textit{ba} ‘place’, e.g. \textit{ai-ba-de}. \ili{Manchu} \textit{ai-ba-} has interesting correspondences in the \isi{demonstratives}. While the \isi{demonstratives} have the usual form \textit{e-(re)} ‘this’ and \textit{te-(re)} ‘that’, the correspondence of \textit{ai-ba-} is \textit{u-ba-} ‘here’ and \textit{tu-ba-} ‘there’ A special case is represented by \textit{atanggi}, which seems to be based on \textit{ai} ‘what’. The form may be amalgamated, but, based on the meaning ‘when’, one may suspect a noun meaning ‘time’ to underlie the second part. In fact, \textit{ai-erin-} is such a compound. But there is no word meaning ‘time’ with an adequate form in \ili{Manchu} and \textit{atanggi} is not synchronically analyzable. \tabref{tab:tungu:34} shows all dialectal cognates available.

\begin{table}[t]
\caption{Cognates of the Jurchenic temporal interrogative; not all variants are shown}
\label{tab:tungu:34}

\begin{tabularx}{.8\textwidth}{XX}
\lsptoprule

\textbf{Language} & \textbf{Form}\\
\midrule
\ilit{Manchu} & atanggi\\
Aihui \ilit{Manchu} & ɛtiŋ(ŋe), ɛtiŋŋi\\
Sanjiazi \ilit{Manchu} & aitiŋ, aitiŋga, aitiŋŋe\\
\ilit{Sibe} & aitiŋ\\
Yibuqi \ilit{Manchu} & ɛi t’iŋkə\\
\ilit{Bala} & a(n)t’aŋniə, at’ani\\
\ilit{Alchuka} & ant’aŋgi\\
\ilit{Kilen} & ?iətin\\
\lspbottomrule
\end{tabularx}
\end{table}

The \ili{Alchuka} and \ili{Bala} forms might indicate a connection with \ili{Manchu} \textit{antaka}. However, according to \cite{MuYejun1988a,MuYejun1988b}, the \textit{-n-} in one form of \ili{Bala} and in \ili{Alchuka} is an innovation. The \ili{Kilen} form \textit{iətin} has most likely been borrowed but is probably built on \ili{Manchu} \textit{ya} instead of \textit{ai}. Given the absence of a fitting word meaning ‘time’ within \ili{Jurchenic}, it may have been borrowed from another language, not unlike \ili{Manchu} \textit{hacin} ‘sort’, which according to \cite[100]{Benzing1956}, stems from \ili{Korean}. Middle \ili{Korean} \textit{enu-cjej} ‘when’ matches the \ili{Jurchenic} form typologically (‘what-time’), but not formally (\sectref{sec:5.7.3}). A connection to (Eastern) \ili{Old Japanese} \textit{tökyi} \zh{等伎} ‘time’ (\citealt{Kupchik2011}: 60, 106) seems extremely implausible, but this is the only form I was able to find in surrounding languages that at least looks remotely similar. The interesting proposal by \cite{Alonso2017} that \textit{atanggi} might be related to other words ending in \textit{+nggi} such as \textit{senggi} ‘blood’ is not plausible on semantic grounds and would leave the first part \textit{ata-} unanalyzed. For now, the origin of the interrogative remains obscure.

\clearpage %solid chapter boundary