\section{Chukotko-Kamchatkan}\label{sec:5.3}
\subsection{Classification of Chukotko-Kamchatkan}\label{sec:5.3.1}

\ili{Chukotko-Kamchatkan} (or Luoravetlan) is a small family that includes five languages in two different branches (\citealt{Fortescue2003}: 51f.; \citealt{Anderson2006c}).

\clearpage %solid chapter boundary
\ea%1
    \label{ex:chuk:1}
\begin{forest}  for tree={grow'=east,delay={where content={}{shape=coordinate}{}}},   forked edges  
[
    [\ili{Chukotian}
        [\ili{Alutor}]
        [\ili{Chukchi}]
        [\textsuperscript{†}\ili{Kerek}]
        [\ili{Koryak} (Nymylan)]
    ]
    [\ili{Itelmen} (Kamchadal)
        [\textsuperscript{†}Eastern]
        [\textsuperscript{†}Southern]
        [Western]
    ]
]
\end{forest}   
    \z

\ili{Itelmen} formerly consisted of three different languages or dialect groups, of which all but the Western group have already become extinct. \ili{Kerek} disappeared during the 1990s. Recently, it has been proposed that \ili{Amuric} (\sectref{sec:5.2}) may be distantly related to \ili{Chukotko-Kamchatkan} \citep{Fortescue2011}, but this hypothesis remains unproven.

\subsection{Question marking in Chukotko-Kamchatkan}\label{sec:5.3.2}

Given the lack of data on other question types, the following will focus primarily on polar and \isi{content question}s. \textbf{Alutor} marks \isi{polar question}s by means of probably rising \isi{intonation} and an optional question particle. Unlike most other languages treated in this study, the particle does not stand sentence-finally but initially, which, except for some \ili{Indo-European} languages (\sectref{sec:5.9.2}), represents a stark contrast with \isi{NEA} (Chapter 6). Content \isi{questions} have no question particle.

\ea%2
    \label{ex:chuk:2}
    \ili{Alutor}\\
    \ea
    \gll \textbf{{matka}} ta-lɣu-ŋi?\\
    \textsc{q}    \textsc{pot}-kill.wild.reindeer-\textsc{pot}.\textsc{pfv}[3\textsc{sg}.S]\\
    \glt ‘Will he kill a wild reindeer?’

    \ex
    \gll \textbf{{miɣɣa}} iv-i?\\
    who.\textsc{abs}.\textsc{sg}  say-3\textsc{sg.}S[\textsc{pfv]}\\
    \glt ‘Who said (that)?’ (\citealt{Nagayama2011}: 293, 294)
    \z
    \z

\noindent In all \ili{Chukotko-Kamchatkan} languages, interrogatives seem to take sentence initial position, which likewise differentiates them from the rest of \isi{NEA}. Interestingly, the initial question particle itself looks similar to \ili{Chukotko-Kamchatkan} interrogatives starting with \textit{m{\textasciitilde}} (see \sectref{sec:5.3.3}). \citet[416]{Fortescue2005} translates \textit{matka} as ‘or’ and lists it with forms such as \ili{Chukchi} \textit{mec-} ‘somewhat’. While the exact derivation remains unexplained, there is also a \ili{Koryak} form \textit{met(’)ke} ‘or’ that appears to be a direct cognate of \ili{Alutor} \textit{matka}. Content questions are likewise unmarked in \ili{Koryak}.

\ea%3
    \label{ex:chuk:3}
    \ea \ili{Koryak}\\
    \gll \textbf{{met’ke}} jenny  e-jem-ke?\\
    \textsc{q}    maybe  \textsc{neg}-come-\textsc{circ}\\
    \glt ‘Perhaps (she) does not come?’

    \ex
    \gll\textbf{{meki}} ib-i?\\
    who.\textsc{abs}.\textsc{sg}  say-3\textsc{sg.}S[\textsc{pfv]}\\
    \glt ‘Who said (that)?’ \citep[51]{Zhukova1997}
    \z
    \z

It seems that in \ili{Chukchi} and \ili{Kerek} both polar and \isi{content question}s are generally unmarked.

\ea%4
    \label{ex:chuk:4}
    \ili{Kerek}\\
    \ea
    \gll milˀej  jallaju-ŋi-n?\\
    gun  take-\textsc{prf}-3\textsc{sg}.O\\
    \glt ‘Will you take this gun?’

    \ex
    \gll\textbf{{maki}} jatč-i?\\
    who.\textsc{abs}.\textsc{sg}  come-\textsc{prf}\\
    \glt ‘Who came?’ (\citealt{Volodin2001}: 156, 157)
    \z
    \z

\ea%5
    \label{ex:chuk:5}
    \ili{Chukchi}\\
    \ea
    \gll koolo  enmec    ɣe-ɣjew-iɣǝt?\\
    \textsc{intj}  already    \textsc{pf}-awaken-2\textsc{sg}\\
    \glt ‘My goodness, you’re up already?’

    \ex
    \gll \textbf{{tite}} ŋan  ŋotqen    n-ǝ-qit-ǝ-qin?\\
    when  \textsc{deict}  \textsc{dem}.3\textsc{sg}.\textsc{abs}  \textsc{hab}-\textsc{e}-freeze-\textsc{e}-3\textsc{sg}\\
    \glt ‘When does it freeze there?’ (\citealt{Dunn1999}: 86, 72)
    \z
    \z

\ili{Chukchi} furthermore has an element \textit{ǝtlon}, glossed as a \isi{question marker}, that appears in both polar and \isi{content question}s and was translated as ‘on earth’, i.e. it adds a certain \isi{emphasis}. It may also fuse with the \isi{interrogative} \textit{ˀǝmi} ‘where’ to form the more complex emphatic \isi{interrogative} \textit{ˀǝmitlon} ‘where on earth’ (\citealt{Dunn1999}: 289f.). Its syntactic position is not absolutely clear, however, but seems to be relatively free.

\ea%6
    \label{ex:chuk:6}
    \ili{Chukchi}\\
    \gll anə  kəkel, \textbf{{ətlon}} \textbf{{iˀam}}, \textbf{req}-ə-lˀet-ə-rko:n?\\
    so  \textsc{intj}  \textsc{q}  why  what-\textsc{e}-\textsc{dur}-\textsc{e}-\textsc{prog.voc}\\
    \glt ‘Oh my! Why, what on earth are you doing?’ \citep[55]{Dunn1999}
    \z

\noindent It does not seem to be a true \isi{question marker}, but nevertheless appears in \isi{interrogative} contexts. Functional equivalents can be found in \ili{Yiddish} (\sectref{sec:5.5.2.2}) and Tundra \ili{Nenets} (\sectref{sec:5.12.2}).

Polar questions in \ili{Itelmen} have final rising \isi{intonation} but otherwise are identical to equally unmarked \isi{content question}s.

\ea%7
    \label{ex:chuk:7}
    \ili{Itelmen}\\
    \gll kni-n    qitkineŋ  çi-ze-n?\\
    \textsc{pp}.2\textsc{sg}-\textsc{poss}  brother    be.available-\textsc{prs}-3\textsc{sg}\\
    \glt ‘Do you have a brother?’ (\citealt{GeorgVolodin1999}: 214)
    \z

Interrogatives in \isi{content question}s optionally take a suffix \textit{-s}, which is said to be a \isi{question marker} that expresses additional \isi{emphasis}.

\ea%8
    \label{ex:chuk:8}
    \ili{Itelmen}\\
    \ea
    \gll \textbf{{k’e}} ç‘eˀ\c{l}-en    k’o\c{l}-ki?\\
    who    \textsc{com.v}-2\textsc{sg}.3\textsc{sg}  come-\textsc{inf}.\textsc{v}\\
    \glt ‘With whom have you come?’

    \ex
    \gll \textbf{{manke}}-\textbf{{s}} kǝmma-n  mni\c{l}  k-meç-knen?\\
    where-\textsc{q}  \textsc{pp}.1\textsc{sg}-\textsc{poss}  all  \textsc{inf}.III-disappear-\textsc{circ}\\
    \glt ‘Where have all my (people?) gone?’ (\citealt{GeorgVolodin1999}: 134, 214)
    \z
    \z

\ili{Itelmen} is the only \ili{Chukotko-Kamchatkan} language for which descriptions of \isi{focus question}s are available to me. They follow an intriguing pattern that has a variable personal marker on the verb.

\ea%9
    \label{ex:chuk:9}
    \ili{Itelmen}\\
    \ea
    \gll \textbf{{isx}}-enk    n-zǝl-aɬ-\textbf{{in}} kza  kǝma-nk?\\
    father-\textsc{loc}  \textsc{imprs}-give-\textsc{fut}-2\textsc{sg}.\textsc{obj.O}  you  me-\textsc{dat}\\
    \glt ‘Will father give \textit{you} to me?’

    \ex
    \gll isx-enk    n-zǝl-aɬ-\textbf{{um}} kza \textbf{{kǝma}}-nk?\\
    father-\textsc{loc}  \textsc{imprs}-give-\textsc{fut}-1\textsc{sg}.\textsc{obj.IO}  you  me-\textsc{dat}\\
    \glt ‘Will father give you \textit{to me}?’ (\citealt{Bobaljik2002}: 3)
    \z
    \z

\noindent In this example, either the direct or the indirect “object” are represented with an agreement marker on the verb. The presence of the marker expresses the focusing of the respective constituent.

\ili{Chukotko-Kamchatkan} languages have a strong \isi{interaction} of \textbf{imperatives} and \isi{question marking}, which is yet another untypical feature for \isi{NEA}. For example, \citet[325]{Nedjalkov1994} mentions the interesting fact that \isi{imperative} verb forms in \ili{Chukchi} may appear in \isi{content question}s where their meaning changes to marking future tense.

\ea%10
    \label{ex:chuk:10}
    \ea
    \ili{Chukchi}\\
    \gll \textbf{{myn}}-le-rkyn?\\
    \textsc{imp}.1\textsc{pl}-go-\textsc{ipfv}\\
    \glt ‘Let us fly!’

    \ex
    \gll\textbf{{minky.ty}} \textbf{{myn}}-le-rkyn?\\
    over.where  \textsc{imp}.1\textsc{pl}-go-\textsc{ipfv}\\
    \glt ‘Over what place shall we fly?’ (\citealt{Nedjalkov1994}: 325)
    \z
    \z

\citet[171]{GeorgVolodin1999} claim that imperatives in \ili{Itelmen} may also have a future and prospective meaning, but this does not appear to be restricted to \isi{questions}. The phenomenon in \ili{Chukchi} has a more straightforward parallel in the more closely related language \ili{Kerek}, for which \citet[158]{Volodin2001} noted the following phenomenon:

\begin{quote}
Interrogative sentences in \ili{Kerek} are often viewed as a special type of imperative utterances that presuppose a speech \isi{response}. Any \isi{interrogative} sentence can be interpreted as a reduced imperative sentence of the type “Tell (\isi{answer}) me, if...”. This view may be confirmed by the strong formal ties existing between imperative and \isi{interrogative} meanings demonstrated by \ili{Chukchi}-\ili{Koryak} (and \ili{Chukchi}-Kamchatkan) languages.
\end{quote}

\noindent In both \ili{Kerek} and \ili{Chukchi} the \isi{imperative} markers in \isi{questions} exhibit an additional modal overtone such as ‘can’ or ‘must’ \citep[157]{Volodin2001}.

\ea%11
    \label{ex:chuk:11}
    \ili{Kerek}\\
    \gll \textbf{{manka}} \textbf{{nə}}-xaxau-n?\\
    why    \textsc{imp}.3\textsc{sg}-go-3\textsc{sg}.S\\
    \glt ‘Why does he have to go?’ \citep[156]{Volodin2001}
    \z

\noindent The \isi{imperative} marker is not obligatory, however, and as in \ili{Chukchi} all examples provided by Volodin are content \isi{questions}. Whether this feature is shared by \ili{Alutor} and \ili{Koryak} remains unclear for now. Interestingly, \isi{interrogative} \isi{morphology} in the adjacent \ili{Yukaghiric} languages (see \sectref{sec:5.14.2}) as well as in Central Alaskan \ili{Yupik} (\sectref{sec:5.4.2}) is also restricted to \isi{content question}s. See also \sectref{sec:5.10.2} on \ili{Even}, a \ili{Tungusic} language that had \isi{contact} with \ili{Chukotko-Kamchatkan} and exhibits the use of imperative forms in \isi{questions} as well.

The marking of \isi{questions} in \ili{Chukotko-Kamchatkan} summarized in \tabref{tab:chuk:1} exhibits no similarities to \ili{Amuric} or to most of \isi{NEA}, for that matter.

\begin{table}
\caption{Summary of question marking in Chukotko-Kamchatkan}
\label{tab:chuk:1}

\begin{tabularx}{\textwidth}{XXl}
\lsptoprule

\textbf{Language} & \textbf{PQ} & \textbf{CQ}\\
\midrule 
\ilit{Chukchi} & - & - (\textsc{imp-V})\\
\ilit{Alutor} & \#matka & -\\
\ilit{Kerek} & - & - (\textsc{imp-V})\\
\ilit{Koryak} & \#met’ke & -\\
\ilit{Itelmen} & - & (\#\textsc{int}-s)\\
\lspbottomrule
\end{tabularx}
\end{table}


\subsection{Interrogatives in Chukotko-Kamchatkan}\label{sec:5.3.3}

Several Proto-\ili{Chukotko-Kamchatkan} (PCK) interrogatives have been reconstructed by \citet{Fortescue2005}. \tabref{tab:chuk:2} lists them with cognates from all five languages, but not all variants and only \isi{singular} forms are shown. Each language has some additional forms, e.g. \textit{la\textsuperscript{ʔ}}\textit{lsxe\textsuperscript{ʔ}}\textit{n} ‘how much/many’, \textit{manke} ‘whence, how’, \textit{manx\textsuperscript{ʔ}}\textit{al} ‘whither’, \textit{əŋqa} ‘what’, and \textit{əŋqan-kit} ‘what-\textsc{caus} > why’ in \ili{Itelmen} (\citealt{GeorgVolodin1999}: 136, passim), \textit{maŋ-ki}, \textit{maja} ‘where’, \textit{maŋ-kət(iŋ)} ‘whence’, \textit{maŋ-kepəŋ} ‘whence, along where’, \textit{maŋ-in\textsuperscript{j}}\textit{as} ‘how many, how long’, and \textit{taʕər} ‘how much’ in \ili{Alutor} (\citealt{Nagayama2011}: 293f.), and \textit{ˀemi} ‘where’, \textit{iˀam} ‘why’, \textit{mik-ə-ne} ‘whither’, \textit{tˀer} ‘how much/many’ etc. in \ili{Chukchi} (\citealt{Dunn1999}: 66, passim). The most important \isi{resonance} of \ili{Chukotko-Kamchatkan} languages is \textit{m{\textasciitilde}}.

\begin{table}
\caption{Proto-Chukotko-Kamchatkan (PCK) interrogatives and their cognates in individual languages according to \cite[56, 173, 175ff., 287]{Fortescue2005} and \cite{Dunn1999,Dunn2000}}
\label{tab:chuk:2}

\begin{tabularx}{\textwidth}{QllQQQl}
\lsptoprule
\small
\textbf{Meaning} & \textbf{PCK} & \textbf{Chukchi} & \textbf{Kerek} & \textbf{Koryak} & \textbf{Alutor} & \textbf{Itelmen}\\
\midrule
what~kind & *mæŋin & meŋin & maŋin & meŋin & maŋin & min\\
who (\textsc{abs.sg}) & *mikæ & mik(ə)- & maki & meki, maki (Kamen) & miɣɣa & k’e\\
where & *miŋ(kə) & miŋkə & miŋkiil (\textsc{all}) & miŋkə & miŋkə & maʔ, mank\\
how & *miŋkəði & miŋkəri & miŋkii & miŋkəje & ?maŋkət & ?mank\\
what & *ðæq- & req-/ceq- & jaq- & jeq- & taq-, teq- (Palana) & saq\\
when & *titæ & tite & sita & tite & tita & it’e\\
\lspbottomrule
\end{tabularx}
\end{table}

\cite[263, 282]{Fortescue2005} reconstructs, furthermore, Proto-\ili{Chukotian} (PC) stems that lack a cognate in \ili{Itelmen}, i.e. PC *\textit{ʀæmi} ‘where’ (\ili{Chukchi} \textit{ʔemi}, \ili{Kerek} \textit{Xam}, and \ili{Koryak} \textit{hemmi}, \ili{Alutor} \textit{-}) and PC *\textit{tæʀər} ‘how much’ (\ili{Chukchi} \textit{tˀer}, \ili{Kerek} \textit{tˀaj}, \ili{Koryak} \textit{teʀ}\textit{i}, and \ili{Alutor} \textit{taʀər}). \ili{Itelmen} likewise exhibits interrogatives without clear equivalents in \ili{Chukotian} such as one meaning ‘what’ (Eastern \textit{nkc}, Southern \textit{nakxej}, and Western \textit{ăŋqa}, \citealt{Fortescue2005}: 399). \citet[1372]{Fortescue2011} compares PCK *\textit{ðæq-} ‘what’ with \ili{Nivkh} \textit{t\textsuperscript{h}}\textit{a-}/\textit{řa-} (\sectref{sec:5.2.3}) and tentatively reconstructs PCKA *\textit{tʌ(q)-}. However, this \isi{reconstruction} is still too speculative, given that the genetic connection between the two families has not been proven beyond doubt. This stem in \ili{Chukotko-Kamchatkan} cannot only have nominal but also verbal properties.

\ea%12
    \label{ex:chuk:12}
    \ili{Alutor}\\
    \gll ɣəttə \textbf{{taq}}-ətkən?\\
    2\textsc{sg}.\textsc{abs}.\textsc{sg}  what-\textsc{ipfv[2sg.S]}\\
    \glt ‘What are you doing?’ \citep[294]{Nagayama2011}
    \z

\newpage     
\ea%13
    \label{ex:chuk:13}
    \ili{Koryak} (Kamenskoye)\\
    \gll nɪ-\textbf{{ya’q}}{-iɣi?}\\
    \textsc{hab}-what-2\textsc{sg}\\
    \glt ‘What are you doing?’ \citep[730]{Bogoras1922}
    \z

\ea%14
    \label{ex:chuk:14}
    \ili{Koryak}\\
    \gll n-\textbf{{re’q}}-iɣɪt?\\
    \textsc{hab}-what\textsc{-}2\textsc{sg}\\
    \glt ‘What are you doing?’ \citep[730]{Bogoras1922}
    \z

\ea%15
    \label{ex:chuk:15}
    \ili{Chukchi}\\
    \gll nə-\textbf{{req}}-iɣət?\\
    \textsc{hab}-what-\textsc{2sg}\\
    \glt ‘What are you doing?’ \citep[368]{Dunn1999}
    \z

\ili{Chukchi} earlier made a characteristic difference between \textit{req-} as used by men and \textit{ceq}\textit{-} as used by women (\citealt{KämpfeVolodin1995}: 8). But this is just the effect of a more general pattern in which women pronounced \textit{r} as \textit{c} that seems to have been lost by now \citep{Dunn2000}. Another language in \isi{Northeast Asia} that makes some distinctions between the \isi{grammar of questions} of women and men is \ili{Japanese} (\sectref{sec:5.6.2}). Similar to \ili{Ket} (\sectref{sec:5.13.3}), interrogatives can be incorporated into the verb. When incorporated the meaning of \textit{req-}/\textit{raq-} {\textasciitilde} \textit{rˀe-}/\textit{rˀa-} changes from ‘what’ to ‘why’.

\ea%16
    \label{ex:chuk:16}
    \ili{Chukchi}\\
    \ea
    \gll \textbf{{raq}}-etə    nə-wetgawe-gˀət?\\
    what-\textsc{dat}  \textsc{prs}-speak-2\textsc{sg}.S\\
    \glt ‘Why do you speak?’

    \ex
    \gll nə-\textbf{{raq}}-ə=wetgawe-gˀət?\\
    \textsc{prs}-what-\textsc{e}=speak-2\textsc{sg}.S\\
    \glt ‘Why do you speak?’

    \ex
    \gll \textbf{{rˀa}}-etə    ŋəta-gˀət?\\
    what-\textsc{dat}  come-2\textsc{sg}.S\\
    \glt ‘Why did you come?’

    \ex
    \glt \textbf{\textit{rˀa}}\textit{=ŋəta-gˀət?}\\
    what=come-2\textsc{sg}.S\\
    \glt ‘Why did you come?’ (\citealt{Spencer1995}: 457, from Skorik)
    \z
    \z

\noindent As examples (\ref{ex:chuk:16}a) and (\ref{ex:chuk:16}c) illustrate, the meaning ‘why’ is otherwise expressed with the dative form of the \isi{interrogative}. See \sectref{sec:5.8.3} and \sectref{sec:5.10.3} for a somewhat similar development in \ili{Khorchin} and \ili{Manchu}.

Interrogatives in \ili{Chukotko-Kamchatkan} languages have elaborated paradigms (see \citealt{Nagayama2011}: 293f. on \ili{Alutor}; \citealt{Bogoras1922}: 726ff. on \ili{Koryak}; \citealt{GeorgVolodin1999}: 134-136 on \ili{Itelmen}). In \ili{Chukchi} the paradigms correspond to the second [+\textsc{hum}] and first declension [+/-\textsc{hum}] of nouns, respectively (\tabref{tab:chuk:3}). In order to make clear the distinction found in the second declension into collective suffixes on the one hand and number/\isi{case} suffixes on the other, the sign Ø indicates which of the markers is absent. The layering of suffixes follows the order \textsc{v-coll-num/case}. The first declension has no collective suffixes. Locative interrogatives and \isi{demonstratives} have parallel paradigms (\citealt{Dunn1999}: 286f.), e.g. \textit{ŋut-ku} ‘\textsc{dem.prox}-\textsc{loc}’, \textit{ŋen-ku} ‘\textsc{dem.dist}-\textsc{loc}’, and \textit{miŋ-ke} ‘where-\textsc{loc}’. The ablative (\textit{m}\textbf{\textit{e}}\textit{ŋ-qo(rə)}) and allative (\textit{miŋ-kəri}) have the same forms throughout.

\begin{table}
\caption{Chukchi interrogative paradigms according to \citet[87]{KämpfeVolodin1995}}
\label{tab:chuk:3}

\begin{tabularx}{\textwidth}{Xll}
\lsptoprule

\textbf{Glossing} & \textbf{who (2nd decl.)} & \textbf{what (1st decl.)}\\
\midrule
stem & mik- & req-\\
\textsc{abs.sg} & meŋi-Ø-n & r”etnyt-Ø\\
\textsc{abs.pl} & miky-Ø-nti & r”etnyt-et\\
\textsc{loc-erg (-coll)} & miky-ne-Ø & req-e (\textsc{inst-erg})\\
\textsc{loc-erg (+coll)} & miky-ryk-Ø & req-yk (\textsc{loc})\\
\textsc{abl (-coll)} & mek-Ø-gypy, (meky-na-jpy) & r”a-/raq-gypy\\
\textsc{abl (+coll)} & meky-r-gypy & -\\
\textsc{all (-coll)} & meky-na-Ø (/-gty) & raq-ety\\
\textsc{all (+coll)} & meky-ryk-y & -\\
\textsc{orient (-coll)} & miky-Ø-gjit & reqy-gjit\\
\textsc{orient (+coll)} & miky-ry-gjit & -\\
\textsc{desig (+/-coll)} & miky-Ø-ny & req-y\\
\lspbottomrule
\end{tabularx}
\end{table}

In \ili{Alutor}, participle forms of the \isi{interrogative verb} may take \isi{case} markers as well.

\ea%17
    \label{ex:chuk:17}
    \ili{Alutor}\\
    \gll ənŋin \textbf{{taq}}-ə-lʔ-u    qa  paninalʔ-u?\\
    well  what-\textsc{e}-\textsc{ptcp}-\textsc{abs.pl}  \textsc{emph}  ancestor-\textsc{abs.pl}\\
    \glt ‘Well, what did (our) ancestors do?’ \citep[133]{Nagayama2016}
    \z

\noindent Predicatively used interrogatives can also take person and number markers.

\ea%18
    \label{ex:chuk:18}
    \ili{Alutor}\\
    \gll \textbf{{mik}}{-ine-}\textbf{{ɣət}} ɣəttə    un\textsuperscript{j}un\textsuperscript{j}u-jɣət?\\
    who-\textsc{poss}-2\textsc{sg.pred}  2\textsc{sg.abs}  child-2\textsc{sg.pred}\\
    \glt 'Whose child are you?' \citep[121]{Nagayama2016}
    \z

\noindent Unlike \ili{Chukchi} or \ili{Itelmen}, but similar to \ili{Aleut} (\sectref{sec:5.4.3}), \ili{Alutor} and \ili{Koryak} not only have \isi{plural} but also \isi{dual} forms.

In sum, \ili{Chukotko-Kamchatkan} interrogatives deviate strongly from other \isi{NEA} languages. No K-interrogatives are present and only \ili{Itelmen} \textit{k’e} has been tentatively classified as a \isi{KIN-interrogative}, although it likely derives from what has been reconstructed as PCK *\textit{mikæ}. Complex paradigms with sandhi effects, \isi{ergative} marking, \isi{dual} number (e.g., \ili{Koryak} \textit{ma’ki} ‘\textsc{abs.sg}’, \textit{ma’kinti} ‘\textsc{abs.du}’, \textit{maku’wɣi} ‘\textsc{abs.pl}’, \citealt{Bogoras1922}), and incorporation set \ili{Chukotko-Kamchatkan} apart from most other languages in \isi{NEA}. However, ambivalent \isi{interrogative} stems meaning ‘(to do) what’ are shared with \ili{Tungusic}, \ili{Eskaleut}, and \ili{Samoyedic}. Especially \ili{Itelmen} exhibits an \isi{opaque} \isi{interrogative} system that resists any \isi{synchronic} attempt for \isi{analysis}. An exhaustive \isi{diachronic} \isi{analysis} can only be accomplished by experts on the language.