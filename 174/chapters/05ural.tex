\section{Uralic}\label{sec:5.12}
\subsection{Classification of Uralic}\label{sec:5.12.1}

Leaving aside the possible existence of so-called Para-\ili{Uralic} for which no direct evidence is available, \ili{Uralic} may be classified as follows \citep[65]{Janhunen2009}.
\ea\upshape%1
    \label{ex:ural:1} 
\noindent    
\ili{Uralic} → \ili{Samoyedic}\\
~~~~~~\ili{Finno-Ugric} → \ili{Mansic}\\
~~~~~~~~~~~~Finno-\ili{Khantic} → \ili{Khantic}\\
~~~~~~~~~~~~~~~~~~\ili{Finno-Permic} → \ili{Permic}\\
~~~~~~~~~~~~~~~~~~~~~~~~\ili{Finno-Volgaic} → \ili{Mariic}\\
~~~~~~~~~~~~~~~~~~~~~~~~~~~~~~\ili{Finno-Saamic} → \ili{Saamic}\\
~~~~~~~~~~~~~~~~~~~~~~~~~~~~~~~~~~~~\ili{Finno-Mordvinic} → \ili{Mordvinic}\\
~~~~~~~~~~~~~~~~~~~~~~~~~~~~~~~~~~~~~~~~~~\ili{Finnic} \& Para-\ili{Finnic} \\
    \z

\ili{Uralic} is usually divided into two main branches, \ili{Samoyedic} and \ili{Finno-Ugric}, the latter of which shows strong internal diversity and can be classified into about seven subbranches. However, only the \ili{Samoyedic} branch (e.g., \citealt{Janhunen1977}; \citeyear{Janhunen1998}; \citealt{Hajdú1988}) will be treated here. \citet[459]{Janhunen1998} mentions two possible classifications of \ili{Samoyedic} languages which he calls the conventional and the alternative classification. Both classifications share the assumption that \ili{Enets} and \ili{Nenets} as well as \ili{Selkup} and \ili{Kamass} are relatively closely related, but differ in whether \ili{Nganasan} and \ili{Mator} should be granted a separate status or not. As for \ili{Enets} and \ili{Nenets}, the focus here will mostly lie on Tundra \ili{Nenets} and Forest \ili{Enets}, mostly excluding other dialects. A language called \ili{Yurats} probably was a transitional vernacular between \ili{Enents} and \ili{Nenets} and will be excluded for lack of data \citep[457]{Janhunen1998}.

\subsection{Question marking in Uralic}\label{sec:5.12.2}

Marking strategies for polar \isi{questions} have been surveyed by \citet{Miestamo2011} for all of \ili{Uralic} (\tabref{tab:ural:1}). In general, \ili{Uralic} languages form a relatively clear western border of \isi{Northeast Asia}. Marking with initial, second position or preverbal particles, question affixes, and question \isi{word order} are all features that set \ili{Uralic} apart from other languages in \isi{Northeast Asia}. Some of these features such as \isi{word order} for marking \isi{polar question}s rather have affinity with European languages, especially \ili{Germanic} (§§\ref{sec:4.2.1}, \ref{sec:5.5.2.1}).

\begin{table}
\caption{Polar question marking strategies in Uralic (adapted from \citealt{Miestamo2011}: 8); Int. = Intonation, IP = initial particle, PP = preverbal particle, FP = final particle, 2ndC = second position clitic, WO = word order, AnA = A-not-A}
\label{tab:ural:1}

\fittable{
\begin{tabular}{llllllllll}
\lsptoprule
& \textbf{Int.} & \textbf{IP} & \textbf{PP} & \textbf{FP} & \textbf{Clitic} & \textbf{2ndC} & \textbf{WO} & \textbf{Affix} & \textbf{AnA}\\
\midrule
\ili{Estonian} & + & + &  & + &  &  & + &  & \\
\ili{Finnish} &  &  &  & + &  & + & + &  & \\
\ili{Veps} & + &  & + &  & + &  &  &  & \\
\ili{Votic} &  &  &  &  &  & + &  &  & \\
Central-Southern \ili{Saami} & + & + &  &  &  &  &  &  & \\
Northern \ili{Saami} &  & + &  &  &  & + &  &  & \\
Skolt \ili{Saami} & + &  &  &  &  & + &  &  & +\\
\ili{Mari} &  &  &  & + &  &  &  &  & \\
Erzya \ili{Mordvin} & + &  &  &  &  &  &  &  & \\
Moksha \ili{Mordvin} & + &  &  &  &  &  &  &  & \\
\ili{Komi-Zyrian} & + &  &  &  &  & + & + &  & +\\
\ili{Udmurt} & + &  &  &  & + &  &  &  & \\
\ili{Hungarian} & + &  &  & + & + &  &  &  & \\
\ili{Khanty} & + &  &  &  &  &  &  &  & \\
\ili{Mansi} &  & + &  &  & + &  &  &  & \\
\ili{Enets} & + &  &  &  &  &  &  & + & \\
\ili{Kamass} &  &  &  &  & + &  &  &  & \\
\ili{Nenets} & + &  &  & + &  &  &  & + & \\
\ili{Nganasan} & + &  &  &  &  &  &  & + & \\
\ili{Selkup} &  &  & + &  &  &  &  &  & \\
\lspbottomrule
\end{tabular}
}
\end{table}

As we will see in this section, not only the marking of \isi{polar question}s but also the \isi{semantic scope} of the \isi{question marker}s differentiates \ili{Samoyedic}, especially northern \ili{Samoyedic}, from most other languages in this study.

\largerpage
The most complex system of asking \isi{questions} can be found in \textbf{\ili{Nganasan}}, which has recently been described by \citet[17]{Miestamo2011}. It is worth quoting his good summary in full length.

\newpage 
\begin{quote}
PIs are expressed by the \isi{interrogative} mood or by \isi{intonation} alone. The \isi{interrogative} mood suffixes are different in different tense-aspect categories (they follow all other verb \isi{morphology} but the person suffixes). In the \textbf{present} (aorist), the suffix is \textit{-ŋu/-ŋa}, and this suffix replaces the imperfective and perfective aspect suffixes used in the indicative present. However, the aspect suffixes mark aspect only redundantly (and only in the indicative present): the aspect distinction is a lexical one and imperfective and perfective verbs differ in their stems as well (except for a small number of biaspectual stems) — the semantic distinction is thus not lost in the \isi{interrogative}. In the \textbf{preterite}, the \isi{interrogative} suffix is \textit{-hu/-ha}, and it replaces the preterite suffix used in the indicative. In the \textbf{future} expressed with \textit{-sutə}, the final vowel of the verb (the \textit{ə} of the future marker or the vowel of the person suffix) is lengthened if the verb is in final position in the \isi{interrogative}. The \isi{interrogative} \textbf{iterative} marker is \textit{-kəə}, which differs from the indicative iterative \textit{-kə} by the lengthening of the vowel. The \isi{interrogative} \textbf{future} may also be expressed by -\textit{ntəŋu/-ntəŋa}, which is a \isi{combination} of the progressive aspect suffix \textit{-ntə} and the present \isi{interrogative} suffix \textit{-ŋu/-ŋa}; according to Larisa Leisiö (p.c.), the aorist and future would differ in the progressive \isi{interrogative} in that the future would contain two instances of the progressive marker, but in actual usage, this repetition often does not happen and the distinction is then not made formally. The \isi{interrogative} \textbf{renarrative} suffix is \textit{-ha} instead of the indicative renarrative -\textit{hamhu}, i.e. the second syllable of the marker is dropped in the \isi{interrogative}. Other moods do not take \isi{interrogative} suffixes, although some of them may be used in polar interrogatives. The remote past and the future-in-the-past are used without \isi{interrogative} marking in \isi{questions}. The \isi{interrogative} mood can also be used in content \isi{questions}. (my boldface)
\end{quote}

\noindent The same markings are present not only in polar and \isi{content question}s, but also in \isi{alternative question}s. The first two sentences are negative \isi{questions}, present and iterative, showing that the \isi{question marker}s under \isi{negation} attach to the so-called negative verb—a feature that \ili{Uralic} shares with \ili{Tungusic} (e.g., \citealt{Hölzl2015a})—rather than the lexical verb itself. Example \REF{ex:ural:4} is an open \isi{alternative question} in which the second of the two markers attaches to the \isi{interrogative verb}.

\ea%2
    \label{ex:ural:2}
    \ili{Nganasan}\\
    \ea
    \gll \textbf{{maa$\delta ə$}} təti  ńa-mtə      ńi-\textbf{{ŋ}}\textbf{{i̮}}-ŋ      heli̮si̮-ˀ?\\
    why  this  partner-\textsc{acc}.2\textsc{sg}  \textsc{neg}-\textsc{prs}.\textsc{q}-2\textsc{sg}  help-\textsc{con}\\
    \glt ‘Why don’t you help this friend of yours?’
    
    \ex
    \gll ti̮ŋ  ŋonə-ntuˀ  ńi-\textbf{{kəə}}{-r}i̮ˀ  hourə-ˀ?\\
    2\textsc{pl}  self-2\textsc{pl}  \textsc{neg}-\textsc{it}.\textsc{q}-2\textsc{pl}  fetch.wood-\textsc{con}\\
    \glt ‘Don’t you go to bring your firewood yourselves?’
    
    \ex
    \gll ma-kal’i  i-\textbf{{ŋu}}{-ruˀ} \textbf{{maa}}-\textbf{{ŋu}}{-ruˀ?}\\
    home-\textsc{car}  be-\textsc{prs}.\textsc{q}-2\textsc{pl}  what-\textsc{prs}.\textsc{q}-2\textsc{pl}\\
    \glt ‘Are you homeless or what?’ (\citealt{Gusev2015a}: 109, 121)\z\z

\clearpage %solid chapter boundary    
\noindent Question marking in \ili{Nganasan} is markedly different from all other \ili{Uralic} languages as well as from most other languages included in this study. Even from a global perspective, it qualifies as one of the more complex \isi{interrogative systems}. Because of morphophonological alternations the exact form of the \isi{question marker}s is too complicated to be given here in full detail (see \citealt{Helimski1998}: 489). As one can ascertain from the second part of the open \isi{alternative question}, \isi{content question}s also display the same \isi{question marking}. When the \isi{interrogative} has no verbal characteristics, the marking attaches to the verb of the clause.

\ea%3
    \label{ex:ural:3}
    \ili{Nganasan}\\
    \gll ti̮{n,} \textbf{{maa}} ənti̮-d’i    əmə  tuj-\textbf{{ŋu}}{-ruˀ?}\\
    2\textsc{pl}  what  so-inf    here  come-\textsc{prs}.\textsc{q}-2\textsc{pl}\\
    \glt ‘What have you come here for?’ \citep[121]{Gusev2015a}
    \z

Polar questions in \textbf{Forest Enets} have final rising \isi{intonation}, while in content \isi{questions} there is a peak on the \isi{interrogative} \citep[353]{Siegl2013}. Similar to \ili{Nganasan}, there is a special past tense \isi{question marker} that appears in both polar and \isi{content question}s and combines with \isi{polar question} \isi{intonation}. Except for the past tense, \isi{questions} remain unmarked morphosyntactically. No example for an \isi{alternative question} was found, although the comparison with other \ili{Samoyedic} languages suggests that they probably exhibit the \isi{double marking} type. Forest \ili{Enets} lacks indirect speech and thus has no \isi{indirect questions} \citep[198]{Siegl2013}.

\ea%4
    \label{ex:ural:4}
    \ili{Enets} (Forest)\\
    \ea
    \gll uu  mosra-\textbf{{sa}}{-d}\\
    2\textsc{sg}  work-\textsc{pst}.\textsc{q}-2\textsc{sg}\\
    \glt ‘Did you work?’ \citep[402]{Siegl2012}
    
    \ex
    \gll \textbf{{kokođ}} to-\textbf{{sa}}{-d}?\\
    whence  come-\textsc{pst}.\textsc{q}-2\textsc{sg}\\
    \glt ‘Where did you come from?’ \citep[355]{Siegl2013}
    \z
    \z

\noindent The past tense \isi{interrogative} suffix takes the forms \textit{-sa}, \textit{-d’a}, \textit{-t’a}, or \textit{-č’a}, depending on the preceding word (Künnap 1999b: 27). Interestingly, while the \isi{answer} to a past tense question of course must also be in the past tense, both the tense suffix as well as the position of the agreement marker differ from the question construction.

\ea%5
    \label{ex:ural:5}
    \ili{Enets} (Forest)\\
    \gll karaul-xuđ to-đ-\textbf{{ud}}{’.}\\
    \textsc{pn}-\textsc{abl}.\textsc{sg}  come-1\textsc{sg}-\textsc{pst}\\
    \glt ‘I came from Karaul.’ \citep[404]{Siegl2012}
    \z

\noindent According to \citet[403]{Siegl2012}, this unusual situation of a tense suffix following an agreement marker is connected with the development of the question suffix.

\begin{quote}
In the \ili{Enets} and \ili{Nenets} languages, a new secondary past tense construction based on the finite verb and a free-standing auxiliary emerged. Later, the free-standing auxiliary merged with the finite verb, resulting in the unusual ordering where tense follows personal endings. Although the reasons for this unusual instance of change, as well as for the prior tense/aspect system preceding this change, await a more thorough investigation and \isi{reconstruction}, the triggered change resulted in the emergence of a new mood which is only used in \isi{questions} with general past tense reference.
\end{quote}

\noindent It may be worth noting that, typologically, the situation is similar to \ili{Nganasan}. In both languages there is an integration of \isi{question marking} and tense (or aspect). But there is only one marker in \ili{Enets}, while there are several in \ili{Nganasan}, and there is no formal identity of the respective markers.

In \textbf{Tundra \ili{Nenets}} there is a very similar situation to that in Forest \ili{Enets}. Polar questions display “pitch raising on the penultimate and ultimate syllables, which may make the sentence-final vowel longer.” \citep[267]{Nikolaeva2014} Polar, content, and \isi{alternative question}s exhibit the same past tense \isi{question marker} \textit{-sa} that has a palatalized dialectal variant \textit{-s’a} and changes to \textit{-se} before agreement markers \citep[97]{Nikolaeva2014}. An \textit{s} (\textit{s’}) regularly changes into \textit{c} (\textit{c’}) following consonants \citep[20]{Nikolaeva2014}.

\ea%6
    \label{ex:ural:6}
    \ili{Nenets} (Tundra, Taymyr)\\
    \gll ŋawor-ma-nʔ    xarwa-daʔ?\\
    eat-\textsc{n}-\textsc{dat}    want-2\textsc{pl}\\
    \glt ‘Do you want to eat?’ (\citealt{Mus2015b}: 90, from Nenyang)
    \z

Interrogatives are either \textit{in situ} or fronted \citep[266]{Nikolaeva2014}.

\ea%7
    \label{ex:ural:7}
    \ili{Nenets} (Tundra)\\
    \ea
    \gll \textbf{{xəqman}}{-}\textbf{{ca}}{-n°?}\\
    say.what-\textsc{pst}.\textsc{q}-2\textsc{sg}\\
    \glt ‘What did you say?’
    
    \ex
    \gll pidər° ti-m xada-\textbf{{sa}}{-r°}?\\
    2\textsc{sg}  reindeer-\textsc{acc}  kill-\textsc{pst}.\textsc{q}-2\textsc{sg}>\textsc{sg}.\textsc{obj}\\
    \glt ‘Did you kill the reindeer?’
    
    \ex
    \gll wera to-\textbf{{sa}}, \textbf{{n}}\textbf{{’}}\textbf{{i}}{-}\textbf{{sa}}?\\
    \textsc{pn}  come-\textsc{pst}.\textsc{q}  \textsc{neg}-\textsc{pst}.\textsc{q}\\
    \glt ‘Did Wera come or not?’ (\citealt{Nikolaeva2014}: 95, 265, 267)\z\z

Forest \ili{Nenets} has the same \isi{question marker} found in Tundra \ili{Nenets} and Forest \ili{Enets} and presumably exhibits the same \isi{semantic scope}. Consider an example of a \isi{content question} in the past tense.

\newpage 
\ea%8
    \label{ex:ural:8}
    \ili{Nenets} (Forest)\\
    \gll \textbf{{kuńana}} me-\textbf{{sa}}{-n?}\\
    where    \textsc{cop}-\textsc{pst.q}-2\textsc{sg}\\
    \glt ‘Where were you?’ \citep[115]{Mikola2004}
    \z

Another way of forming a \isi{polar question} usually addressed to oneself is the use of a dubitative enclitic. The enclitic can also be found in \isi{content question}s and marks \isi{alternative question}s if used twice, and thus has the same \isi{semantic scope} as the suffix.

\ea%9
    \label{ex:ural:9}
    \ili{Nenets} (Tundra)\\
    \ea
    \gll n’{abako-m’}{i} tuu{t°-bə-ta=}\textbf{{w°h}}?\\
    elder.sister-1\textsc{sg}  come.\textsc{fut}-\textsc{cond}-\textsc{dub}\\
    \glt ‘(I wonder) will my sister come or not?’
    
    \ex
    \gll \textbf{{s’axah}} t{uut°ə}{=}\textbf{{w°h}}?\\
    when    come.fut=\textsc{dub}\\
    \glt ‘When on earth will he come?’
    
    \ex
    \gll t’{uku} yəxa yor’{a=}\textbf{{w°h}} \textbf{{n}}\textbf{{’ii}}{=}\textbf{{w°h}}?\\
    this  river  deep-\textsc{dub}  \textsc{neg}-\textsc{dub}\\
    \glt ‘Is this river deep or not?’ (\citealt{Nikolaeva2014}: 267, 268)\z\z

\noindent The dubitative enclitic usually has the form \textit{=m}\textit{°h} but changes to \textit{=w}\textit{°h} after vowels and to \textit{=(}\textit{°)}\textit{h} after \textit{m}. Examples \REF{ex:ural:4} and (\ref{ex:ural:9}c) of \isi{negative alternative question}s exhibiting negative auxiliaries as second alternatives follow a construction very similar to several other languages in \isi{NEA}.

An interesting \isi{alternative question} with a \isi{focus} that is not on the verb is the following in which the verb takes the question suffix. The first alternative precedes and the second follows the verb. The second alternative has rising \isi{intonation} towards the end.

\ea%10
    \label{ex:ural:10}
    \ili{Nenets} (Tundra)\\
    \gll noxa-m xada-\textbf{{sa}}{-n°,} t{’on}{’a-m?}\\
    polar.fox-\textsc{acc}  kill-\textsc{pst}.\textsc{q}-2\textsc{sg}  fox-\textsc{acc}\\
    \glt ‘Did you kill a polar fox or a red fox?’
    \z

\noindent As can be seen, there is only one \isi{question marker}. Probably, this is the result of ellipsis of the originally reduplicated verb \textit{xada-sa-n°}. The following example, which I reanalyze as an open \isi{alternative question}, also has this structure.

\ea%11
    \label{ex:ural:11}
    \ili{Nenets} (Tundra)\\
    \gll ti-m    xada-\textbf{{sa}}{-n}{\textsuperscript{o}},  ŋan’i \textbf{{ŋ}}\textbf{{ə}}\textbf{{mke}}{-m?}\\
    reindeer-\textsc{acc}  kill-\textsc{pst}.\textsc{q}-2\textsc{sg}  other  what-\textsc{acc}\\
    \glt ‘Did you kill a reindeer or what (did you kill) instead?’ \citep[268]{Nikolaeva2014}
    \z

Tundra \ili{Nenets} has yet another clitic \textit{=t’iq} {\textasciitilde} \textit{=d’iq} absent in eastern dialects that may be found in \isi{questions} but is not a \isi{question marker} as such.

\begin{quote}
The \isi{interrogative} clitic is used in \isi{questions}, most typically, in \isi{rhetorical questions}, but sometimes also information \isi{questions}. Its function consists in strengthening the \isi{interrogative} force, roughly in the same way as the ‘on earth’ expression in \ili{English} \citep[123]{Nikolaeva2014}
\end{quote}

\ea%12
    \label{ex:ural:12}
    \ili{Nenets} (Tundra)\\
    \gll \textbf{{xən}}\textbf{{’}}\textbf{{ah}} m{’iŋa-dəm=}\textbf{{t}}\textbf{{’}}\textbf{{iq}}?\\
    whither  go-1\textsc{sg}=\textsc{emph}.\textsc{q}\\
    \glt ‘Where on earth am I going?’ \citep[123]{Nikolaeva2014}
    \z

\noindent There are typologically comparable emphatic elements in \ili{Chukchi} and \ili{Yiddish} \isi{questions}.

For the extinct language \textbf{Mator}, only two content \isi{questions} were recorded. \citet[164]{Helimski1997} claims that both exhibit a suffix \textit{-s} possibly related to the past tense \isi{question marker} in \ili{Enets} and \ili{Nenets}. Given the fact, however, that both sentences were translated into the present tense, this seems rather unlikely. \ili{Mator} has been extinct for over 150 years, which is why more information cannot be obtained.

Unlike all other languages in \isi{Northeast Asia}, \textbf{Selkup} has a preverbal \isi{polar question} particle derived from the \isi{interrogative} \textit{qaj} ‘what’ \citep[18]{Miestamo2011}.

\ea%13
    \label{ex:ural:13}
    \ili{Selkup} (Taz)\\
    \gll tat \textbf{{qaj}} qən-na-ntɨ?\\
    2\textsc{sg}  \textsc{q}  go-\textsc{co}-2\textsc{sg}\\
    \glt ‘Are you leaving?’(\citealt{Wagner-Nagy2015}: 149, from Kuznecova)
    \z

\noindent The \isi{interrogative} \textit{qaj} possibly has a \ili{Turkic} origin (see \sectref{sec:5.11.3}). Content \isi{questions} remain unmarked. \citet[142]{Wagner-Nagy2015} is not clear whether final rising \isi{intonation} affects only polar or also content \isi{questions}.

\ea%14
    \label{ex:ural:14}
    \ili{Selkup} (Taz)\\
    \gll tat  qum-ɨt-ɨp \textbf{{qajɨtqo}} ašša  apstɨ-s-al?\\
    2\textsc{sg}  man-\textsc{pl}-\textsc{acc}  why    \textsc{neg}  feed-\textsc{pst}-2\textsc{sg}.O\\
    \glt ‘Why did you not give the people any food?’ (\citealt{Wagner-Nagy2015}: 142, from Kuznecova)
    \z

\noindent According to \cite[111]{Castrén1855}, \isi{alternative question}s display the marker \textit{kai} in front of each alternative (a feature shared with \ili{Ket}, \sectref{sec:5.13.2}), and in \isi{negative alternative question}s the second alternative has the form \textit{kai aṡa?} ‘or not?’ (i.e., \textit{qaj} ‘what > \textsc{q}’, \textit{ašša} ‘\textsc{neg}’, \citealt{Wagner-Nagy2015}).

\citet[15]{Miestamo2011} analyzes \textit{Kamass}, extinct since 1989, as having an enclitic polar \isi{question marker} \textit{=a}. The marker attaches to the verb and does not appear in \isi{content question}s. Alternative questions are marked twice with the marker \textit{=bV}, like \textit{=a} given with a hyphen but called particle by \cite[35f.]{Künnap1999a}. In line with \citegen{Miestamo2011} \isi{analysis}, it is treated as an enclitic here. In addition, the example contains a disjunctive \textit{aali} ‘or’, which comes from \ili{Russian} \citep[189]{Joki1944}.

\ea%15
    \label{ex:ural:15}
    \ili{Kamass}\\
    \ea
    \gll tən  or\=on    tərl\=eē̮le    mə{lal=}\textbf{{a}}?\\
    2\textsc{sg}  hole.\textsc{lat} skip.\textsc{ger}  can.\textsc{fut}=\textsc{q}\\
    \glt ‘Can you skip the hole?’
    
    \ex
    \gll \textbf{{šinde}} dəγin  amna?\\
    who  there  sit.\textsc{prs}.3\textsc{sg}\\
    \glt ‘Who is sitting there?’
    
    \ex
    \gll man    amn\=oˀla=\textbf{{bo}} \textbf{{aali}} maˀan    j\={\i}la xallolaˀ=\textbf{{bo}}?\\
    1\textsc{sg}.\textsc{loc}  live.\textsc{fut}.2\textsc{pl}=\textsc{q}  or  home.\textsc{gen}  people.\textsc{lat} go.\textsc{fut}.2\textsc{pl}=\textsc{q}\\
    \glt ‘Are you going to live here at my place or are you going home to your own people?’ (Künnap 1999a: 35, 36)\z\z

\noindent The \isi{question marker} \textit{=bV} could have a \ili{Turkic} origin (\sectref{sec:5.11.2}), but note that the extinct \ili{Kott} language, according to \cite{Castrén1858}, has two question markers \textit{â} and \textit{bo}, both of which seem to have parallels in \ili{Kamass} (\sectref{sec:5.13.2}).

\begin{table}
\caption{Summary of question marking in Uralic}
\label{tab:ural:2}

\begin{tabularx}{\textwidth}{lQlQ}
\lsptoprule
& \textbf{PQ} & \textbf{CQ} & \textbf{AQ}\\
\midrule
Forest \ili{Enets} & V-sa ‘\textsc{pst}’ & id. & ?2x id.\\
\ili{Kamass} & V=a & - & 2x =bV + aali ‘or’\\
\ili{Mator} & ? & ?-s & ?\\
\ili{Nganasan} & V-ŋu/-ŋa ‘\textsc{prs}’, V-hu/-ha ‘\textsc{pst}’, V-sutə ‘\textsc{fut}’, V-kəə ‘\textsc{it}’, V-ha ‘\textsc{renarr}’ & id. & 2x id.\\
\ili{Selkup} & qaj V ‘what>\textsc{q}’ & - & 2x qaj V\\
Tundra \ili{Nenets} & V-sa ‘\textsc{pst}’, =w°h ‘\textsc{dub}’ & V-sa ‘\textsc{pst}’ & 2x V-sa ‘\textsc{pst}’, 2x =w°h ‘\textsc{dub}’\\
\lspbottomrule
\end{tabularx}
\end{table}

\tabref{tab:ural:2} summarizes marking of polar, content, and alternative \isi{questions}. Little information on tag or \isi{focus question}s is available to me, but possibly there is a \isi{tag question} marker in \ili{Nenets} that has the form \textit{-xava} ‘is it not so?’ (\citealt{Miestamo2011}: 16f.).

In general, northern and southern \ili{Samoyedic} languages have quite distinct \isi{question marking} strategies. The form and \isi{semantic scope} of the northern \ili{Samoyedic} markers set the languages apart from most other languages in \isi{Northeast Asia}. \tabref{tab:ural:3} gives an overview of two of the question suffixes in northern \ili{Samoyedic} and their cognates in southern \ili{Samoyedic}. The \ili{Mator} suffix \textit{-s-} has tentatively been added, but its exact meaning remains unclear \citep[164]{Helimski1997}.

\begin{table}
\caption{Samoyedic tense markers based on \cite[115f.]{Mikola2004}}
\label{tab:ural:3}

\begin{tabularx}{\textwidth}{lXXXXll}
\lsptoprule

\textbf{PS} & \textbf{Nganasan} & \textbf{Nenets} & \textbf{Enets} & \textbf{Mator} & \textbf{Kamass} & \textbf{Selkup}\\
\midrule
\textsc{pret} *-så- & -sua- etc. & -sa- \textsc{q} & -sa- \textsc{q} & -s- ?\textsc{q} & ? & -s-, -h-\\
\textsc{aor} *-ŋ(å)- & -ŋu- \textsc{q} & -ŋ(a)- & -ŋ(a)- & -ga & -γV-, -gV- & -ŋ-, -n-, Ø\\
\lspbottomrule
\end{tabularx}
\end{table}

As shown in \sectref{sec:5.11.2}, the Southern \ili{Siberian Turkic} language \ili{Khakas} has a similar development from a past tense into a \isi{question marker} (\textit{-ǯAŋ}) that seems to have been influenced by \ili{Samoyedic}. Because of the large geographical distance, the \ili{Negidal} future \isi{question marker} presumably has no areal connection to \ili{Samoyedic} (\sectref{sec:5.10.2}).

\subsection{Interrogatives in Uralic}\label{sec:5.12.3}

For reasons of space only limited aspects of interrogatives in \ili{Samoyedic} can be presented here. The interested reader is referred to \cite{Mus2009,Mus2013,Mus2015b} and references therein, who has given a very detailed description of \ili{Samoyedic} interrogatives, especially those from the northern languages, and in particular those from Tundra \ili{Nenets}. Unfortunately, her description lacks a clear historical or morphological \isi{analysis}.

All \ili{Samoyedic} languages have a \isi{resonance} in \textit{k{\textasciitilde}} (> \textit{x{\textasciitilde}} in Tundra \ili{Nenets}), and thus have K-interrogatives. Only some languages have what is called a \isi{KIN-interrogative} (e.g., \ili{Mator} \textit{kim}, Forest \ili{Nenets} \textit{kim’a}). Both features are inherited from \ili{Proto-Samoyedic}. \cite[15, 62f., 69, 75, 91]{Janhunen1977} reconstructs the following \ili{Proto-Samoyedic} interrogatives *\textit{ki.m(ɜ)} {\textasciitilde} \textit{*ki̮.}\textit{mä} ‘who’, *\textit{ku-} ‘what, which’, *\textit{ku.nå} ‘where’, *\textit{kä-} ‘what, how’, *\textit{kä.nə} ‘how much’ , *\textit{m{e̮}} ‘what’, and *\textit{ə.m-} ‘what’. Derivations in individual languages, the meaning of the stems, and whether the reconstructed forms are as clearly analyzable as indicated by the hyphens, remain extremely unclear, however. The first three reconstructions share a \isi{resonance} in *\textit{k{\textasciitilde}} and thus are probably related historically. In several languages the initial consonant changed to a fricative in some forms such as Forest \ili{Enets} \textit{sän}, Tundra \ili{Nenets} \textit{s’an°} that are cognates of \ili{Nganasan} \textit{kanə} and thus derive from *\textit{kä.n\^ə} ‘how much’. Generally, most interrogatives seen below can be grouped with one of these reconstructions. The initial \textit{ŋ-} in Tundra \ili{Nenets} \textit{ŋəmke} and Forest \ili{Nenets} \textit{ŋami} is prothetic \citep[466]{Janhunen1998} and the forms are thus derived from *\textit{ə.m-} ‘what’ \citep[15]{Janhunen1977}. Note that the \textit{ŋ-} only appears in the Central (\textit{ŋamge}) but not the Western (\textit{amge}) and Eastern dialects (\textit{amge}) of Tundra \ili{Nenets} \citep[93]{Mus2015a}.

Let us now briefly consider the \isi{interrogative systems} in individual \ili{Samoyedic} languages, starting with \textit{Nganasan} (\tabref{tab:ural:4}). There is only one \isi{resonance} in \textit{k{\textasciitilde}} and only the categories of \textsc{person} (*\textit{k-}) and \textsc{thing} have special forms without this \isi{resonance}. The \isi{interrogative} \textit{maa-d\textsuperscript{j}}\textit{aa} ‘why’ is derived from \textit{maa} ‘what’ with the help of what appears to be an allative. A form \textit{syly}/\textcyrillic{сулу} ‘who’, borrowed from \ili{Nganasan}, is attested for \ili{Taimyr Pidgin Russian} (\sectref{sec:5.5.3.3}).

\begin{table}
\caption{Nganasan interrogatives (\citealt{Helimski1998}: 500f.; \citealt{Kortt1985}: passim; \citealt{Castrén1855}: 47, 49, 50, 65, 74); not all variants listed, accents removed}
\label{tab:ural:4}

\begin{tabularx}{\textwidth}{lQQQ}
\lsptoprule

\textbf{Meaning} & \textbf{Helimski} & \textbf{Kortt \& Simčenko} & \textbf{Castrén}\\
\midrule
who & sïlï(-ŋuna) & syly & sele\\
what & maa(-ŋuna) & ma’ & maa\\
why (?-\textsc{all}) & maa-d\textsuperscript{j}aa & ma-d‘a & maajaaŋ\\
when & kaŋge & kanga & kaŋaŋ\\
how many & kanə & kano’ & kana’, kanaŋ\\
“the how manieth” & kanə-mtu(ə) & kangkoj & kanamtua, kanagüi’\\
how many times & kan-üʔ &  & kani’\\
which, what kind of & kuə, kun\textsuperscript{j}iə, kured\textsuperscript{j}i & kunie, kurodi, kuredi & kunie, kurajee\\
where & kunu, kun\textsuperscript{j}ini & kuninu & kuninu\\
whence & kun\textsuperscript{j}i$\delta ə$ & kunida & kunida\\
along where &  & kunimenu & kunimanu\\
whither & kun\textsuperscript{j}i, kundə & kuni’ & kuni’aaŋ, kunijaaŋ\\
how & kun\textsuperscript{j}i-ʔi\textsuperscript{i}a & id. & id.\\
\lspbottomrule
\end{tabularx}
\end{table}

In Forest \textbf{Enets} there is also a \isi{resonance} in \textit{k{\textasciitilde}}. The interrogatives \textit{obu} ‘what’, \textit{še} ‘who’, and \textit{sän} ‘how much’ do not exhibit this \isi{submorpheme}, although the latter two historically had an initial *\textit{k} as well.

\begin{table}
\caption{Forest Enets (\citealt{Siegl2013}: 195ff.; \citealt{Künnap1999b}: 5, 22, 27, 30, 40) and Enets interrogatives (\citealt{Castrén1855}: 76, 81, 82, 90, 91); not all variants listed}
\label{tab:ural:5}

\begin{tabularx}{\textwidth}{lQQQ}
\lsptoprule
& \textbf{Siegl} & \textbf{Künnap} & \textbf{Castrén}\\
\midrule
who & še & seea & sio, sie\\
how much & sän &  & senno\\
what & obu & obu, abbua & awuo\\
why (\textsc{trsl}) & obu-š &  & \\
whence & ko-ko-đ & kuho$\delta $ & kuro, kudo, koohoro\\
where & ku-ni-n & kunn{e̮} & kokohone\\
whither & ku-ʔ & kuoˀ & kuu\\
along where &  &  & kuuno’one\\
how & kuń & kuńˀ, kud’, kuˀon & kuuno’ kurahaane\\
when & kuna &  & kun(n)e\\
which, what kind of & kursi & kurs{e̮} & hooke\\
\lspbottomrule
\end{tabularx}
\end{table}

\largerpage
The interrogatives meaning ‘where’, ‘whither’, and ‘whence’ have separate forms, although the first two share a stem \textit{ku-}, while the last is based on \textit{ko-}. Instead of \textit{obu} ‘what’, Tundra \ili{Enets} has the \isi{interrogative} \textit{miˀ} (\citealt{Künnap1999b}: 5) or \textit{mii’} (\citealt{Castrén1855}: 97). Forest \ili{Enets} exhibits an interesting \isi{interrogative} with the meaning ‘which of two’ that has its own paradigms shown in \tabref{tab:ural:6} (e.g., \textit{koki-juʔ} ‘who of us two’).

\begin{table}
\caption{Paradigms of the Forest Enets interrogative ‘which of two’ \citep[198]{Siegl2013}}
\label{tab:ural:6}

\begin{tabularx}{\textwidth}{XXl}
\lsptoprule
& \textbf{\textsc{du}} & \textbf{\textsc{pl}}\\
\midrule
\textsc{1} & koki-juʔ & =1\textsc{du}\\
\textsc{2} & koki-riʔ & koki-raʔ\\
\textsc{3} & koki-điʔ & koki-đuʔ\\
\lspbottomrule
\end{tabularx}
\end{table}

Apart from other \ili{Samoyedic} languages (e.g., Tundra \ili{Nenets} \textit{xujumʔ} ‘which of two’, \citealt{Mus2015b}: 79), this \isi{interrogative} has no functional parallel in \isi{Northeast Asia}, but in \ili{Proto-Indo-European} *\textit{k\textsuperscript{w}}\textit{oteros} (\sectref{sec:5.5.3}).

\textbf{Nenets} interrogatives exhibit two resonances, one in \textit{k{\textasciitilde}} or \textit{x{\textasciitilde}}, and another in \textit{s’{\textasciitilde}} or \textit{š{\textasciitilde}}. Initial *\textit{k-} regularly changed to \textit{x-} in Tundra \ili{Nenets}, but remained stable in Forest \ili{Nenets} (\citealt{Hajdú1988}: 4). The initial \textit{s’-} or \textit{š-} likewise goes back to *\textit{k-} (cf. \citealt{Janhunen1977}: 62f.). As mentioned before, the initial \textit{ŋ-} is prothetic \citep[466]{Janhunen1998}.

\begin{table}
\caption{Tundra Nenets (\citealt{Nikolaeva2014}: 50, 265, passim), Forest Nenets (\citealt{Mus2013}: passim), and {Nenets} interrogatives according to \cite[3, 10, 32, 327]{Castrén1855}; the Tundra Nenets forms in square brackets are from \cite{Mus2013,Mus2015b}; not all variants listed}
\label{tab:ural:7}

\begin{tabularx}{\textwidth}{Xlll}
\lsptoprule
& \textbf{Tundra Nenets} & \textbf{Forest Nenets} & \textbf{Castrén}\\
\midrule
who & xiib’a & kim’a & hübea, hibea etc.\\
which & xə-n’a-ŋi° & ku-ńa-ŋi & hu-naa-ŋy\\
where & xə-n’a-na & ku(-ńa)-na & hu-naa-na\\
whither & xə-n’a-h & ku(-ńa)-ŋ & hu-naa\\
whence & xə-n’a-d° & ku(-ńa)-t & hu-naa-d\\
along where & [xə-n’a-mna] &  & hu-na-mna\\
how & xə(n)c’er°q & kušeʔ, kušeł & hunder, hunzier etc.\\
to say what & xəqman- &  & ?ha-maan\\
what (kind of) & xurka &  & hurk(k)a\\
how many & s’an° & šan & saŋooka, sambir\\
when & s’ax°h & šaxaŋ, šajna, šana & ?saha’\\
what size & [s’aŋkar] & šam’an & ?saŋum, saŋuna ‘how long’\\
what (kind of) & ŋəmke & ŋami & (ŋ)amge(e)\\
why & ŋəmke & ŋameʔ, ŋamiŋaš & (ŋ)amge(e)jemn̴e\\
\lspbottomrule
\end{tabularx}
\end{table}

\largerpage
The form meaning ‘when’ is derived from ‘how many’. Tundra \ili{Nenets} has only one form, whereas Forest \ili{Nenets} makes a distinction into different forms for ‘what (kind of)’ and ‘why’. The \isi{interrogative} \textit{ŋəmke} has the irregular accusative \isi{plural} \textit{ŋəwo} \citep[25]{Nikolaeva2014} and exhibits a function similar to \textit{xurka}. The two forms are sometimes interchangeable.

\ea%16
    \label{ex:ural:16}
    \ili{Nenets} (Tundra)\\
    \gll \textbf{{xurka}}{/}\textbf{{ŋəmke}} l'ekarə-ŋe° tara-\textbf{{sa}}?\\
    which    doctor-\textsc{ess}  needed-\textsc{pst.q}\\
    \glt ‘What kind of doctor did he work as?’ \citep[261]{Nikolaeva2014}
    \z

Locative \isi{demonstratives} and interrogatives in Forest \ili{Enets} show partly parallel par\-a\-digms with special morphological markers \textit{-n} ‘\textsc{loc}’, \textit{-ʔ} ‘\textsc{all}’, and \textit{-}\textit{đ} ‘\textsc{abl}’ that are otherwise only known from postpositions (\tabref{tab:ural:8}).

\begin{table}
\caption{Demonstrative and interrogative paradigms in Forest Enets (\citealt{Siegl2013}: 197, 204); modified analysis}
\label{tab:ural:8}

\begin{tabularx}{\textwidth}{XXXl}
\lsptoprule
& \textbf{\textsc{prox}} & \textbf{\textsc{dist}} & \textbf{\textsc{int}}\\
\midrule
\textsc{lat} & \textbf{äu}(-ʔ) & to-ni-ʔ & ku-ʔ\\
\textsc{loc} & äku(-\textbf{xu})-n & to-ni-n & ku-ni-n\\
\textsc{abl} & äku(-\textbf{xu})-đ & to-ni-đ & \textbf{ko}-\textbf{ko}-đ\\
\lspbottomrule
\end{tabularx}
\end{table}

While all three stems share the same \isi{case} markers, there are differences in the formation of the stems that are only insufficiently understood. \citet[204]{Siegl2013} admits that the “spatial deixis system of Forest \ili{Enets} is far from being clear”. However, a comparison with Tundra \ili{Nenets} sheds some light on the situation.

In Tundra \ili{Nenets} the suffix \textit{-ŋi°} ({\textasciitilde} \textit{-(x)°} {\textasciitilde} \textit{-y°}) in the selective \isi{interrogative} \textit{xə-n’a-ŋi°} is an attributive form \citep[52]{Nikolaeva2014}. The locative usually has the form \textit{-xən(’)a}, the dative has the 2nd and 3rd person possessive form \textit{-xəh-}, and the ablative has the form \textit{-xəd°} (\citealt{Nikolaeva2014}: 62ff.). Apparently, these forms contain an element \textit{-xə} that is missing in the locative interrogatives that simply add the \isi{case} markers \textit{-na} ‘\textsc{loc}’, \textit{-h} ‘\textsc{dat}’, and \textit{-d°} ‘\textsc{abl}’, but attach to an element \textit{-n’a} instead that has been translated as ‘at, by’ \citep[50]{Nikolaeva2014}. The prolative, found in \textit{xə-n’a-mna}, usually has the slightly different form \textit{-mən(’)a(h)}. Apart from the locative forms listed in \tabref{tab:ural:7}, \citet[50]{Nikolaeva2014} mentions the shorter forms \textit{xu-na}, \textit{xu-h}, \textit{xu-d°}, and \textit{xu-mna}. This variation can also be seen in Forest \ili{Nenets}, e.g. \textit{ku(-ńa)-na} ‘where’. Forest \ili{Enets} shows a less clear picture, but it can be noted that both the \isi{case} markers (\textit{-n}, \textit{-ʔ}, \textit{-}\textit{đ}) and stem formations (\textit{-ni}, \textit{-xu}, \textit{-ko}) have parallels in Tundra \ili{Nenets} (\textit{-na}, \textit{-h}, \textit{-d°}, and \textit{-n’a}, \textit{-xə}, \textit{-ko}). The last of the suffixes can perhaps be found in Tundra \ili{Nenets} \isi{demonstratives} such as e.g., \textit{t’u}\textbf{\textit{ko}}\textit{-xə-na} ‘there’, which seems to correspond to Forest \ili{Enets} \textit{to-ni-n} but has different derivations. The comparison with \ili{Nganasan} in \tabref{tab:ural:9} illustrates basically the same pattern.

\begin{table}
\caption{Paradigms of the locative interrogative in Nganasan and Tundra Nenets}
\label{tab:ural:9}

\begin{tabularx}{\textwidth}{XXXX}
\lsptoprule
& \textbf{Nganasan} & \textbf{Tundra Nenets} & \\
\midrule
where & ku-ni-nu & xə-n’a-na & xu-na\\
whither & ku-ni-’ & xə-n’a-h & xu-h\\
whence & ku-ni-da & xə-n’a-d° & xu-d°\\
along where & ku-ni-menu & xə-n’a-mna & xu-mna\\
\lspbottomrule
\end{tabularx}
\end{table}

\noindent \citet{Helimski1998} recorded \isi{synchronic} variation in \ili{Nganasan} with (\textit{ku-n\textsuperscript{j}}\textit{i-ni}) and without the stem extension (\textit{ku-nu}) as well.

The Tundra \ili{Nenets} \isi{interrogative} stem \textit{xə-}, mistakenly called an “\isi{interrogative} prefix” by \cite[3204f.]{Wagner-Nagy2016}, fused with the negative verb \textit{n’i-}, resulting in the complex form \textit{xə}\textit{n’a-} ‘how not’ \citep[281]{Nikolaeva2014}. The interrogatives \textit{xii}\textit{b’}\textit{a} ‘(to be) who’ and \textit{ŋəmke} ‘(to be) what’ may be either verbal or nominal without requiring any derivation (see \sectref{sec:5.4.3} on \ili{Yupik} and \sectref{sec:5.10.3} on \ili{Tungusic}).

\ea%17
    \label{ex:ural:17}
    \ili{Nenets} (Tundra)\\
    \ea
    \gll {(pidər\textsuperscript{o}}) \textbf{{x}}\textbf{{ii}}\textbf{{b}}\textbf{{’}}\textbf{{a}}{-n}{\textsuperscript{o}}{-}\textbf{{s’}}\textbf{{\textsuperscript{o}}}?\\
    2\textsc{sg}    who-2\textsc{sg}-\textsc{pst}.\textsc{q}\\
    \glt ‘Who were you?’
    
    \ex
    \gll \textbf{{x}}\textbf{{ii}}\textbf{{b}}\textbf{{’}}\textbf{{a}}-h  teda    səwa?\\
    who-\textsc{gen}  reindeer.3\textsc{sg}  good\\
    \glt ‘Whose reindeer is good?’ (\citealt{Nikolaeva2014}: 257, 251)
    \z
    \z

\noindent Full paradigms are not attested but see \cite{Mus2009,Mus2015b} for a partial list of forms.

In Tundra \ili{Nenets} there is an \isi{interrogative} \textit{xəq}\textit{man-} with the meaning ‘to say what’, with the verb \textit{man-} ‘to say’ as a second element (see \ref{ex:ural:7} above). Given the special meaning, one cannot exclude an areal connection to Kolyma \ili{Yukaghir} \textstyleStrong{{\textit{monoʁod-}}}\textstyleStrong{{ with the same meaning that exhibits the verb} }\textstyleStrong{{\textit{mon-}}}\textstyleStrong{{ ‘to say’ as a first part}} (\sectref{sec:5.14.3}). The verb for ‘to say’ was already similar in the respective \isi{proto-languages} \citep[274]{Nikolaeva2006}, but the mere existence of an \isi{interrogative} with this specific meaning in \isi{NEA} is extremely rare and might indicate a \isi{contact} phenomenon.

The extinct language \textbf{Mator} had a \isi{resonance} in \textit{k{\textasciitilde}} (e.g., \textit{ki̮}\textit{m} ‘who’, \textit{kumna} ‘how many’, \textit{kulgu} ‘which’, \textit{kagan} ‘when’) and at least one form, \textit{amgan} ‘why’ (\citealt{Helimski1997}: passim), without it that might be connected with Tundra \ili{Nenets} \textit{(ŋ)amge} ‘what’. As in \ili{Nganasan}, \ili{Enets}, and \ili{Nenets}, the locative forms seem all to be built on a stem \textit{ku-}, but no stem extension can be found, e.g. \textit{ku-na} ‘where’, \textit{kuŋa} ‘whither’, \textit{kuj} ‘whence’. \ili{Mator} \textit{kulgu} ‘which’ could correspond to Tundra \ili{Nenets} \textit{xurka}.

\begin{table}
\caption{Selkup interrogatives from different dialects (\citealt{Wagner-Nagy2015}: 152, passim; \citealt{Castrén1855}: 111, 113, 126); not all variants listed}
\label{tab:ural:10}

\begin{tabularx}{\textwidth}{XXXl}
\lsptoprule
& \textbf{Taz dialect} & \textbf{Ob dialect} & \textbf{Castrén}\\
\midrule
what & qaj & qaj & kai\\
why & qajqo &  & kaitko, kaiŋo\\
who & kutɨ & qod & kud, kod\\
when & kuššan {\textasciitilde} kuššat & quʒan & kussai, kuṡal, kunzei\\
where & kun {\textasciitilde} kut & kun, kut’t’an & kun, kaigan\\
whither & kuččä & qu, kučet & ku, kaind\\
whence & kuunɨ & kut’aut, quute & kun, kaigan\\
how &  &  & kaindek\\
how much &  &  & kaana\\
\lspbottomrule
\end{tabularx}
\end{table}

The \textbf{Selkup} \isi{interrogative} system (\tabref{tab:ural:10}) exhibits two resonances in \textit{k{\textasciitilde}} and \textit{q{\textasciitilde}}. The form \textit{kutɨ} {\textasciitilde} \textit{qod} seems to have replaced the original form meaning ‘who’. The interrogatives \textit{qaj}, and \textit{kaindek} (and less likely \textit{kuššan} {\textasciitilde} \textit{quʒan}) seem to derive from a \ili{Turkic} source (\sectref{sec:5.11.3}). According to \cite[111]{Castrén1855}, \ili{Selkup} also has an \isi{interrogative} \textit{kak} {\textasciitilde} \textit{kaŋ} ‘how’ that was borrowed from \ili{Russian} \textit{kak}/\textcyrillic{как}.

\ili{Kamass} (\tabref{tab:ural:11}) has two resonances in \textit{g{\textasciitilde}} and \textit{k{\textasciitilde}}, both of which derive from *\textit{k-}. The initial \textit{š-} in the \isi{interrogative} meaning ‘who’ goes back to *\textit{k-} as well \citep[69]{Janhunen1977}. The individual forms remain largely obscure synchronically.

\begin{table}[t]
\caption{Kamass interrogatives (\citealt{Künnap1999a}: 19, 26, 28; \citealt{Castrén1855}: 179, 180, 181, 183, 184; cf. \citealt{Joki1944}: 145)}
\label{tab:ural:11}

\begin{tabularx}{\textwidth}{llQ}
\lsptoprule
& \textbf{Künnap} & \textbf{Castrén}\\
\midrule
what & (ə)mbi(i) & ümbi\\
why, what for & əmbiile, mooˀ, mo & ümbi ila’, ümbi naaman\\
who & ši̮ndi, šəndi, šində, šəmdə & ṡimdi\\
what (else) & šombi & \\
which, who (of the two) & giˀiˀ & gi{d̴}i, ki{d̴}i\\
which, what kind & gigəˀ & \\
where (to) & gijen & gi{d̴}igän\\
where to & giibər & gi{d̴}ibir, gi{d̴}re\\
where from & giˀiˀ & gi{d̴}igä’\\
how & kadəˀ, kədəˀ & kada’\\
which, what kind & kajet, kəjet & ka{d̴}et ‘how’\\
when & kaamən & kaaman\\
how many/much & kümen, gilʒi & khümän\\
which & kümeeŋgit & khümäŋit ‘the how manieth’\\
\lspbottomrule
\end{tabularx}
\end{table}

\largerpage[2]
In sum, the \isi{interrogative systems} in \ili{Samoyedic} display a bewildering diversity of forms that in this study is only overcome by \ili{Indo-European} and \ili{Trans-Himalayan}. No \isi{interrogative} has been fully preserved in all \ili{Samoyedic} languages, many exhibit idiosynchratic derivations, and only a few forms have a relatively wide distribution (e.g., *\textit{ku.nå} ‘where’), which either indicates strong language contacts or, what is more likely, perhaps a longer time of separation than the usually accepted 2000 years (e.g., \citealt{Janhunen2009}). In comparison, \ili{Tungusic}, which is estimated to be of more or less the same age (e.g., \citealt{Janhunen2005}), presents a much more coherent picture with many forms found throughout the entire family (\sectref{sec:5.10.3}). For this reason, the above discussion was not able to give an adequate overview of historical developments, which only an expert in these languages can provide.

\clearpage %solid chapter boundary