\section{Amuric}\label{sec:5.2}
\subsection{Classification of Amuric}\label{sec:5.2.1}

\ili{Nivkh} is usually considered a linguistic isolate (e.g., \citealt{Anderson2006d}), but there may be some reason to assume a connection to \ili{Chukotko-Kamchatkan} languages \citep{Fortescue2011} (\sectref{sec:5.3}). Apart from that, there is perhaps enough internal variation to consider it a small \isi{language family} that will be called \ili{Amuric} \citep{Janhunen1996}. However, these varieties are traditionally called \textit{dialects} instead of \textit{languages} \citep[7]{Gruzdeva1998}. The relation of these so-called dialects has been characterized by \citet[7]{Gruzdeva1998} as follows:

\begin{quote}
\textsc{ad} and \textsc{esd} are rather different: their speakers affirm that they do not understand each other. \textit{N}\textsc{sd} (or the Shmidt dialect) occupies an in[t]ermediate position between these two. As for \textsc{ssd} (or the Poronaisk dialect), it has essential differences in phonology, grammar, and vocabulary from the other three dialects, especially from \textsc{ad}.
\end{quote}

\noindent The \isi{Amur} dialect has also been spoken on northwestern \isi{Sakhalin}. \citet{Shiraishi2006} has additionally argued for the existence of a West \isi{Sakhalin} dialect (WSD) that is different from, but closely related to the \isi{Amur} dialect (see also \citealt{ShiraishiTangiku2013}). This has not been recognized by \citet{Fortescue2016}. In sum, there are the following varieties.

\ea%1
    \label{ex:amur:1}
\begin{forest}  for tree={grow'=east,delay={where content={}{shape=coordinate}{}}},   forked edges  
[
    [Amur-West-\isi{Sakhalin}
        [\isi{Amur} dialect (AD)]
        [West \isi{Sakhalin} dialect (WSD)]
    ]
    [North \isi{Sakhalin} dialect (NSD)
    ]
    [East \isi{Sakhalin} dialect (ESD)
    ]
    [South \isi{Sakhalin} dialect (SSD)
    ]
]
\end{forest}
    \z

Most examples will be drawn from AD and ESD. The somewhat obscure transcription of some publications has been changed and roughly follows \citet[203]{ShiraishiTangiku2013}.

\subsection{Question marking in Amuric}\label{sec:5.2.2}

According to \citet[45]{Gruzdeva1998}, \ili{Nivkh} makes a distinction between two types of \isi{polar question} markers. The first type is a suffix that directly attaches to the verb stem and has the form \textit{-l(o)} in both \isi{Amur} and East \isi{Sakhalin} dialects. The form \textit{-lo} is more polite and ceremonious than \textit{-l}, which seems to have a more colloquial flavor (\citealt{NedjalkovOtaina2013}: 116).

\ea%2
    \label{ex:amur:2}
    \ili{Nivkh}\\
    \gll tʃʰi     ra-\textbf{{l(o)}}?\\
    2\textsc{sg}    drink-\textsc{fin}.\textsc{q}\\
    \glt ‘Did you drink?’ \citep[45]{Gruzdeva1998}
    \z

\ea%3
    \label{ex:amur:3}
    \ili{Nivkh} (\isi{Amur})\\
    \gll if    pʰrɨ-\textbf{{l(o)}}?\\
    3\textsc{sg}    come-\textsc{q}\\
    \glt ‘Did (s)he come?’ (\citealt{NedjalkovOtaina2013}: 116)
    \z

The second type attaches to a finite verbal form or other elements in \isi{focus}. It has the form \textit{=l(a)} {\textasciitilde} \textit{=lo} in the \isi{Amur} dialect and the form \textit{=l(a)} {\textasciitilde} \textit{=lu} in the East \isi{Sakhalin} dialect. It was also written with a hyphen but is reanalyzed as enclitic here. The semantic difference between the two markers, which are perhaps etymologically connected, remains unclear.

\clearpage %solid chapter boundary
\ea%4
    \label{ex:amur:4}
    \ili{Nivkh} (\isi{Amur})\\
    \gll tʃʰi     ra-d=\textbf{{la}}?\\
    2\textsc{sg}    drink-\textsc{fin=q}\\
    \glt ‘Did you drink?’
    \z

\ea%5
    \label{ex:amur:5}
    \ili{Nivkh} (East \isi{Sakhalin})\\
    \gll tʃʰi     ra-d=\textbf{{lu}}?\\
    2\textsc{sg}    drink-\textsc{fin=q}\\
    \glt ‘Did you drink?’ \citep[45]{Gruzdeva1998}
    \z

Polar and \isi{focus question}s have the same marker that attaches to the verb in the former and to the element under \isi{focus} in the latter.

\ea%6
    \label{ex:amur:6}
    \ea
    \ili{Nivkh} (\isi{Amur})\\
    \gll ɨtɨk pʰrɨ-dʒ=\textbf{{la}}?\\
    father    come-\textsc{ind=q}\\
    \glt ‘Has father come?’

    \ex
    \gll ɨtɨk=\textbf{{la}} pʰrɨ-dʒ?\\
    father=\textsc{q}  come-\textsc{ind}\\
    \glt ‘Is it \textit{father} who has come?’ (\citealt{NedjalkovOtaina2013}: 124)
    \z
    \z

Content questions may be unmarked if they have a special \isi{intonation} that was left unspecified by \citet[46]{Gruzdeva1998}. Otherwise they have a \isi{question marker} different from that for \isi{polar questions} (\isi{Amur} \textit{=ŋa, =at(a)}, East \isi{Sakhalin} \textit{=ŋa, =ŋu, =ara}). The markers may either attach to the verb or the \isi{interrogative} (phrase). They have been reanalyzed as enclitic here. Interrogatives remain \textit{in situ}.

\ea%7
    \label{ex:amur:7}
    \ili{Nivkh} (East \isi{Sakhalin})\\
    \ea
    \gll {tʃʰin} \textbf{{tʰamdʒi}} {pʰ-vo-ux} \textbf{tʰamdʒi} ŋa turʰ-pɨrʰk  ɲi-tʰa-d-ɣun?\\
    2\textsc{pl}  what.kind  \textsc{refl}-village-\textsc{loc} what.kind  animal meat-?only  eat-\textsc{hab}-\textsc{ind}-\textsc{pl}\\
    \glt ‘What kinds of animal meat do you eat in your village?’ (\citealt{ChaeHeekyung2013}: 132)

    \ex
    \gll {tʃʰi} \textbf{{tʰa}}.k-to$\chi $  vi-d=\textbf{{ŋa}}?\\
    2\textsc{sg}  where-\textsc{dat}  go-\textsc{ind}=\textsc{q}\\
    \glt ‘Where are you going (roughly)?’ \citep[182]{Gruzdeva2008}
    \z
    \z

\ea%8
    \label{ex:amur:8}
    \ili{Nivkh} (\isi{Amur})\\
    \ea
    \gll \textbf{{aŋ}} {pʰrɨ-dʒ=}\textbf{{at}}?\\
    who  come-\textsc{ind=q}\\
    \glt ‘Who came?’

    \ex
    \gll {tʃʰi} \textbf{{sidʒ}}{=}\textbf{{ŋa}} j-ɨsru-dʒ?\\
    2\textsc{sg}  what=\textsc{q}  \textsc{obj}-pursue-\textsc{ind}\\
    \glt ‘Whom do you pursue?’ \citep[46]{Gruzdeva1998}
    \z
    \z

\ea%9
    \label{ex:amur:9}
    \ili{Nivkh} (\isi{Amur})\\
    \gll \textbf{{sidʒ}} ɲivɣ=\textbf{ŋa} {jiv-dʒ?}\\
    what  person=\textsc{q}  have-\textsc{ind}\\
    \glt ‘What (kind of) man is (here)?’ (\citealt{NedjalkovOtaina2013}: 124)
    \z

\noindent The existence of separate and overtly marked polar and \isi{content question} markers seems to have been adopted by the \ili{Tungusic} language \ili{Uilta} (\sectref{sec:5.10.2}).

No clear examples for \isi{tag question}s and only one example for a negative \isi{alternative question} have been found. The \isi{analysis} of this example from von Glehn \citep[31]{Grube1892} remains partly obscure for me but is sufficiently clear to show that there is no \isi{disjunction} and that each alternative takes a marker \textit{lo}. In the \isi{Amur} dialect this may either correspond to the enclitic \textit{=l(a)} {\textasciitilde} \textit{=lo} or to the suffix \textit{-l(o)}. However, \cite[125, 209]{NedjalkovOtaina2013} mention a suffix \textit{-lu} found in the \isi{Amur} dialect, misleadingly called “particle” despite being given with a hyphen, that seems to have dubitative meaning and marks indirect \isi{alternative question}s. Given that it may also have the form \textit{-lo}, it seems possible that this is the form recorded by von Glehn.

\ea%10
    \label{ex:amur:10}
    \ili{Nivkh} (\isi{Amur})\\
    \gll [{tu-nɨ-dʒ-}\textbf{lu} qa-nɨ-dʒ-\textbf{{lu}}] pʰanpʰara-r      hum-dʒ.\\
    go.upstream-\textsc{fut}-\textsc{n}-\textsc{dub}  go.downstream-\textsc{fut}-\textsc{n}-\textsc{dub} not.know-\textsc{cvb.nar}.3\textsc{sg}  be-\textsc{ind}\\
    \glt ‘He does not know [whether to go upstream or downstream].’ (\citealt{NedjalkovOtaina2013}: 209)
    \z

\noindent An etymological connection to the other two question markers seems likely but to my knowledge there has not been an investigation of this topic. The same marker also appears in indirect \isi{polar questions} (\citealt{NedjalkovOtaina2013}: 220) and \isi{content question}s (see \ref{ex:amur:12}a,b below). This quite clearly shows that it should be kept apart from the actual question markers. On the contrary, it may be a marker for indirect \isi{questions}, exclusively. \isi{Rhetorical questions} in \ili{Nivkh} are marked with \textit{-rla} {\textasciitilde} \textit{-tla}.

\ea%11
    \label{ex:amur:11}
    \ili{Nivkh} (\isi{Amur})\\
    \gll if  pʰrɨ-\textbf{{rla}}?\\
    3\textsc{sg}  come-\textsc{q}\\
    \glt ‘Did (s)he really come?’ (\citealt{NedjalkovOtaina2013}: 116)
    \z

A special marker that is said to expect a positive \isi{answer} and thus perhaps comes close to a \isi{question tag} is (probably sentence-final) <\textit{y}> as recorded by von Schrenck \citep{Grube1892}. \citet[262]{Austerlitz1956} mentions a marker \textit{=ii}, reanalyzed as enclitic here, that he translates as ‘isn’t it?’ and it might be the same as <\textit{y}>, e.g. \textit{ŋav=ii?} ‘a sparrow’s nest, isn’t it?’. The \ili{Tungusic} language \ili{Uilta} (\sectref{sec:5.10.2}) not only has a content \isi{question marker} \textit{=ga} {\textasciitilde} \textit{=ka} that most likely derives from \ili{Nivkh} \textit{=ŋa} (\sectref{sec:3.1}), but also has a polar \isi{question marker} \textit{=(y)i} that could to stem from this enclitic in \ili{Nivkh}.

\begin{table}
\caption{Summary of question marking in Amuric.}
\label{tab:amur:1}

\begin{tabularx}{\textwidth}{XXXXX}
\lsptoprule
& \textbf{PQ} & \textbf{CQ} & \textbf{AQ} & \textbf{FQ}\\
\midrule
AD & V=l(a)/=lo & =ŋa, =at(a) & 2x =l(a)/=lo, ?2x =lu/=lo & FOC=l(a)/=lo\\
ESD & V=l(a)/=lu & =ŋa, =ŋu, =ara & ? & FOC=l(a)/=lu\\
\lspbottomrule
\end{tabularx}
\end{table}

Slightly adjusting \citegen[79, 172]{Fortescue2016} reconstructions, Proto-\ili{Amuric} must have had the \isi{question marker}s *\textit{=la} {\textasciitilde} \textit{=lo}, *\textit{-rla} {\textasciitilde} \textit{-rlo}, *\textit{=ŋa, =ata}, and *\textit{=i} with somewhat unclear distribution.

\subsection{Interrogatives in Amuric}\label{sec:5.2.3}

Descriptions of interrogatives in \ili{Nivkh} are usually insufficient, especially for the South and North \isi{Sakhalin} dialects. \tabref{tab:amur:2} shows those forms collected by \citet{Mattissen2003} and \citet{Fortescue2016} to which WSD data has been added (\citealt{ShiraishiTangiku2013}). The \isi{Amur} and West \isi{Sakhalin} dialects have a \isi{resonance} in \textit{ř{\textasciitilde}} and the East \isi{Sakhalin} dialect in \textit{tʰ}\textit{{\textasciitilde}} that go back to the same origin. Interrogatives meaning ‘what’ and ‘when’, and, except for ESD, also the \isi{interrogative} meaning ‘who’ do not participate in this \isi{resonance}. The \isi{resonance} has been recorded as \textit{š{\textasciitilde}} by von Schrenck and as \textit{s{\textasciitilde}} by von Glehn \citep{Grube1892}. For example, von Schrenck had a form \textit{ša-} ‘which, what kind of’ (AD \textit{řa-}) as well as its regular locative form \textit{ša-in} ‘where’ (AD \textit{řa-in}, \citealt{Fortescue2011}: 144).

\citet[111]{Fortescue2016} speculates that AD \textit{aŋ} derives from \textit{nar-ŋa} ‘who-\textsc{q}’. If correct, a typological parallel can be found in \ili{Korean} (\sectref{sec:5.8.3}).  ESD \textit{tʰ}\textit{au-nt/-d} ‘who’ is perhaps a secondary innovation based on the selective \isi{interrogative}. Interestingly, almost all listed interrogatives are monosyllabic. But there are some longer forms as well, as the following two examples from the \isi{Amur} dialect illustrate.

\ea%12
    \label{ex:amur:12}
    \ili{Nivkh} (\isi{Amur})\\
    \ea
    \gll [\textbf{{jagut}} imŋ  pʰrɨ-dʒ-\textbf{{lu}}] if  pʰanpʰara-dʒ.\\
    how.3\textsc{pl}  3\textsc{pl}  come-\textsc{ind}-\textsc{dub}    3\textsc{sg}  not.know-\textsc{ind}\\
    \glt ‘He does not understand [how they came there].’

    \ex
    \gll [\textbf{{jagur}} pʰrɨ-dʒ-\textbf{{lu}}] pʰanpʰara-dʒ.\\
    how.3\textsc{sg}  come-\textsc{ind}-\textsc{dub}    not.know-\textsc{ind}\\
    \glt ‘(He) does not understand [how (he) came (there)].’ (\citealt{NedjalkovOtaina2013}: 220)
    \z
    \z

\begin{table}
\caption{Nivkh interrogatives according to \citet[14]{Mattissen2003} and \citet[passim]{Fortescue2016}, WSD according to \citet[206]{ShiraishiTangiku2013}; not all variants listed}
\label{tab:amur:2}
\small
\begin{tabularx}{\textwidth}{>{\raggedright}p{1.7cm}llQQQQ}
\lsptoprule
\textbf{Proto-Amuric} & \textbf{AD} & \textbf{WSD} & \textbf{ESD} & \textbf{NSD} & \textbf{SSD}\\
\midrule
*nar\newline ‘who’ & aŋ {\textasciitilde} aɣ & aŋ & nar {\textasciitilde} nař, \textbf{tʰ}\textbf{au-nt/-d} & nar {\textasciitilde} nař & nat\\
\tablevspace 
*tu-nt\newline ‘what’ & si-dʒ & si-tʃ {\textasciitilde} si-d\textsuperscript{j} & ru-(n)t/-d & ru-t, řu-t, su-t & ru-nt, lu-nt\\
\tablevspace 
*ta-nt\newline ‘which’ & \mbox{řa-dʒ {\textasciitilde} tʰa-dʒ} & ?\footnotemark & tʰa-d & ? & ?\\
\tablevspace 
\mbox{*taŋz {\textasciitilde} *taŋr}\newline ‘how much/many’ & řa-ŋs & řa-ŋs & \mbox{tʰa-ŋs {\textasciitilde} tʰa-gs} & řa-ŋspaklu ‘some’ & \mbox{tʰa-ŋk {\textasciitilde} řa-nkř} ‘some’\\
\tablevspace 
*ta-\newline ‘where’ & řa-r & řa-ŋ {\textasciitilde} řa-g & tʰa-s & ? & řa-k, tʰa-k\\
\tablevspace 
*aɣr\newline ‘when’ & ɨɣr & ɨɣr & aɣř {\textasciitilde} ɨɣř & \textbf{ɨrŋa} & axř\\
\tablevspace 
*ja-(nɨ-ŋ)\newline ‘how, why’ & ja-ŋu-t/-r & jaŋ-gu-nɨ-tʃ & \mbox{ja-ɲř {\textasciitilde} ja-nř} & ja-na-gu-t & ja-nɨ-ŋ, jan-ř, ja-nɨ-g\\
\lspbottomrule
\end{tabularx}
\end{table}

\footnotetext{ Given the parallel in the AD and ESD, one may assume that the WSD has the form \textit{řa-tʃ {\textasciitilde} řa-d}\textit{\textsuperscript{j}} ‘which’.}

Also observe the dubitative suffix \textit{-lu} used for \isi{indirect questions} presented in \sectref{sec:5.2.2}. In \textit{jagur} {\textasciitilde} \textit{jagut} the element \textit{-r} (2\textsc{sg}, 3\textsc{sg}) {\textasciitilde} \textit{-t} (1\textsc{sg}, 1\textsc{pl, 2pl, 3pl}) is the narrative \isi{converb} marker that is also part of the \isi{rhetorical question} marker \textit{-r-la} {\textasciitilde} \textit{-t-la} previously noted (\citealt{NedjalkovOtaina2013}: 40). The forms also contain an old causative marker \textit{-ku} {\textasciitilde} \textit{-$\gamma $u} {\textasciitilde} \textit{-gu} {\textasciitilde} \textit{-xu} that apparently has mostly lost its function (\citealt{NedjalkovOtaina2013}: 42). Apparently, \citet{Gruzdeva1998} and \citet{Mattissen2003} do not mention any of these forms, but they have been listed as \textit{ja-ge-r} (von Schrenck), \textit{ja}\textbf{\textit{ŋ}}\textit{(-o-r)} (von Glehn), \textit{ja-g-r} (Seeland), and \textit{ja}\textbf{\textit{n}}\textit{-g-r} (Lebedew) by \citet{Grube1892}. As in two of these examples, AD and WSD sometimes contain a consonant \textit{-ŋ} which---\citet[87]{NedjalkovOtaina2013} speculate---might be a dialectal difference. According to \citet[81]{Fortescue2016}, the \textit{-ŋ} could be a participle form. \tabref{tab:amur:3} shows the paradigm of these forms as can be reconstructed with the help of different descriptions.

\begin{table}
\caption{Simple AD and WSD interrogative paradigms of the form meaning ‘how’ (\citealt{NedjalkovOtaina2013}: 40, 220, \citealt{Shiraishi2006}: 65, \citealt{ShiraishiTangiku2013}: 206)}
\label{tab:amur:3}

\begin{tabularx}{\textwidth}{XXl}
\lsptoprule
& \textbf{\textsc{sg}} & \textbf{\textsc{pl}}\\
\midrule
1 & ja(ŋ)-gu-t & ja(ŋ)-gu-t\\
2 & ja(ŋ)-gu-r & ja(ŋ)-gu-t\\
3 & ja(ŋ)-gu-r & ja(ŋ)-gu-t\\
\lspbottomrule
\end{tabularx}
\end{table}

But according to \citet[206]{ShiraishiTangiku2013} there are also some longer forms such as WSD \textit{jaŋ-gu-nɨ-tʃ} ‘how’. The WSD suffix \textit{-tʃ} is the same as AD \textit{-dʒ} ‘\textsc{ind}’ that attaches to what appears to be the future marker \textit{-nɨ} (\citealt{NedjalkovOtaina2013}: 209) or perhaps the verb \textit{-nɨ} {\textasciitilde} \textit{-nu} ‘to do’ as in SSD \textit{ja-nɨ-ŋ} \citep[81]{Fortescue2016}. \citet[369]{NedjalkovOtaina2013} mention in addition an AD form \textit{jaar} ‘why’ that must be related to these forms but has a long vowel and lacks the causative suffix (see \citealt{Fortescue2016}: 81 for additional variants). According to \citet[238]{Mattissen2003} the stem \textit{ja-} (optionally with a derivation \textit{ja-$\gamma $a-} not encountered thus far) actually means ‘to do what’. The forms \textit{ja-} as well as \textit{ja-ʁo} may also be employed as an attribute, e.g. AD \textit{ja-ɲivx} ‘what person’, \textit{ja-ʁo-dəf} ‘what kind of house’. These patterns are extremely similar to \ili{Mongolic} (*\textit{ya-}\textit{xu/n} ‘what’, *\textit{ya-}\textit{xa-} ‘to do what’, \sectref{sec:5.8.3}) and \ili{Tungusic} (*\textit{ja-(kun)} ‘what’, *\textit{ja-} ‘to do what’, \sectref{sec:5.10.3}).\footnote{According to \citet[209]{NedjalkovOtaina2013} and \citet[81]{Fortescue2016}, the initial \textit{j-} is a third person \isi{singular} marker---a hypothesis first proposed by Jakobson---while \textit{a-} is the actual \isi{interrogative verb} meaning ‘to do what’. But the connection with \ili{Tungusic} and \ili{Mongolic} makes this very unlikely.} Possibly, the \ili{Nivkh} forms are \ili{Tungusic} loans that in turn derive from \ili{Mongolic}. The converbal origin of forms meaning ‘how’ or ‘why’ might also suggest a connection with \ili{Mongolic} or \ili{Tungusic}. Within \ili{Nivkh} there are completely parallel forms in the \isi{demonstratives}, e.g. AD \textit{ho-(ʁo)-} ‘be like that, do thus’, \textit{ho(ŋ)-gu-r/t} ‘thus, in that way’ etc. (\citealt{NedjalkovOtaina2013}: 87f.).

Suffixes in the locative (AD \textit{řa-r}, ESD \textit{tʰ}\textit{a-s} ‘where’) and the quantitative interrogatives (AD \textit{řa-ŋs}, ESD \textit{tʰ}\textit{a-ŋs {\textasciitilde} t}\textit{ʰ}\textit{a-gs} ‘how much/many’) have parallels in spatial expressions and \isi{demonstratives}, cf. AD \textit{tu-r} ‘here’, \textit{hu-r} ‘there’, \textit{tu-ŋs} ‘this much’, \textit{hu-ŋs} ‘that much’ (\citealt{Gruzdeva1998}: 26f., 36), ESD \textit{tu-s}, \textit{hu-s}, and \textit{tu-nks}, \textit{hu-nks} with a slightly different form \citep[170]{Gruzdeva2008}. \citet[14]{Mattissen2003} furthermore mentions AD \textit{řa-kr {\textasciitilde} t}\textit{ʰ}\textit{a-kr} ‘where’ that has a suffix also known from spatial expressions and \isi{demonstratives}, e.g. ESD \textit{tu-kř} ‘here’, \textit{hu-kř} ‘there’ \citep[181]{Gruzdeva2008}. The difference between \textit{-s} and \textit{-kř} is that the former designates a precise and the latter a non-precise location \citep[178]{Gruzdeva2008}. Another suffix \textit{-nx} roughly patterns with the latter in meaning, e.g. ESD \textit{tʰ}\textit{a-nx} ‘where’ \citep[184]{Gruzdeva2008}. It is possible to attach a \isi{case} marker such as the dative to the locative forms, e.g. ESD \textit{tʰ}\textit{a-s-to$\chi $}, \textit{tʰ}\textit{a-k-to$\chi $} ‘where to’ (\citealt{Gruzdeva2008}: 179, 182). Thus, similar to \ili{Tungusic} the forms meaning ‘where’ are derived from the selective \isi{interrogative} (AD \textit{řa-dʒ {\textasciitilde} t}\textit{ʰ}\textit{a-dʒ}, ESD \textit{tʰ}\textit{a-d}).

The forms meaning ‘what’ may be analyzed as a stem and the nominalizer (indicative) *\textit{-nt} > AD \textit{-dʒ}, ESD \textit{-nt} {\textasciitilde} \textit{-(n)d} etc. \citep[1366]{Fortescue2011}. The same element is present in the selective \isi{interrogative} and ESD \textit{tʰ}\textit{au-nt, tʰ}\textit{au-d} ‘who’, as well as some \isi{demonstratives} (\tabref{tab:amur:4}). Notice that von Schrenck recorded the \isi{Amur} dialect form meaning ‘what’ as \textit{si-č} {\textasciitilde} \textit{si-}\textbf{\textit{n}}\textit{č} \citep{Grube1892}, which preserves a nasal that is also present in ESD \textit{ru-d {\textasciitilde} ru-}\textbf{\textit{n}}\textit{t} ‘what’.

\begin{table}
\caption{Amuric demonstratives and interrogatives “indicating a person or an object” (\citealt{Gruzdeva1998}: 26ff.)}
\label{tab:amur:4}

\begin{tabularx}{\textwidth}{XXl}
\lsptoprule
& \textbf{AD} & \textbf{ESD}\\
\midrule
\textsc{dem.prox} & tɨ-dʒ & tu-d {\textasciitilde} tu-nt {\textasciitilde} tɨ-nt\\
\textsc{dem.dist} 1 & hɨ-dʒ & hu-d {\textasciitilde} hu-nt {\textasciitilde} hɨ-nt\\
\textsc{dem.dist} 2 & a-dʒ & ahu-d {\textasciitilde} ehu-d {\textasciitilde} ehɨ-nt\\
\textsc{dem.dist} 3 & aehɨ-dʒ & aix-nt\\
\textsc{dem.}invisible & ku-dʒ & ku-d {\textasciitilde} ku-nt\\
what & si-dʒ & ru-d {\textasciitilde} ru-nt\\
which & řa-dʒ {\textasciitilde} tʰa-dʒ & tʰa-d\\
who & - & tʰau-d {\textasciitilde} tʰau-nt\\
\lspbottomrule
\end{tabularx}
\end{table}

Demonstratives with the suffix may take number and \isi{case} markers (e.g., AD \textit{tɨ-dʒ-Ø-ɣir} ‘this-\textsc{ind}(-\textsc{sg})-\textsc{inst}’), without, they may function as attributive forms (e.g., AD \textit{tɨ urk} ‘this night’). Perhaps a similar situation can be observed for the interrogatives \textit{tʰ}\textit{amdʒi} ‘what kind of’ (\citealt{ChaeHeekyung2013}: 135) versus \textit{tʰ}\textit{amdʒi-d} ‘how’ \citep[1372]{Fortescue2011} in the ESD (similar to \textit{ja-dʒ} {\textasciitilde} \textit{ja-} in AD, \citealt{Mattissen2003}: 238).

\citet[1371]{Fortescue2011} assumes that \ili{Nivkh} \textit{tʰ}\textit{a-}/\textit{řa-} is related to *\textit{ðæq} in Proto-\ili{Chukotko-Kamchatkan} (e.g., \ili{Chukchi} \textit{räq}, \ili{Alutor} \textit{taq}). He reconstructs a common proto-form for both as *\textit{tʌ(q)-} (\sectref{sec:5.3.3}). But as long as the hypothetical \isi{language family} is not accepted by a majority of scholars, this must be treated with caution. Two interrogatives from \ili{Nivkh} may have found their way into the \ili{Tungusic} language \ili{Uilta} (\sectref{sec:5.10.3}). The \ili{Uilta} materials collected by Bronisław Piłsudski contain the two forms \textit{nuulú} ‘whither’ and \textit{sádo} ‘where’ (\citealt{Majewicz2011}: 388, 430). The second \isi{interrogative} also has the form \textit{saa} ‘where’ with a long vowel and is most likely a loan from West \isi{Sakhalin} \ili{Nivkh} \textit{řa}\textit{-}\textit{g} ‘where’ (cf. \citealt{Ikegami1997}\emph{\textup{;}} \citealt{Pevnov2009}: 122). Note that von Glehn recorded several forms starting with \textit{s{\textasciitilde}} \citep{Grube1892}. Allegedly, the ESD also has the forms \textit{nu-nt {\textasciitilde} nu-d} ‘what’. \citet[1372]{Fortescue2011} speculates that these forms are actually indefinites and may contain a contracted form of the noun \textit{nə-} ‘thing’. But if \ili{Uilta} \textit{nuulu} is indeed from \ili{Nivkh}, it must be connected somehow to this form in the East \isi{Sakhalin} dialect.