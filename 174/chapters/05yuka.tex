\section{Yukaghiric}\label{sec:5.14}
\subsection{Classification of Yukaghiric}\label{sec:5.14.1}

Today, there are two surviving but endangered, or rather moribund, \ili{Yukaghiric} languages called Tundra \ili{Yukaghir} (Wadul) and Kolyma \ili{Yukaghir} (Odul). Two varieties called \ili{Chuvan} and \ili{Omok} are usually included in the list of \ili{Yukaghiric} languages. Both have already disappeared and are not well recorded (e.g., \citealt{Anderson2006e}).
\ea\upshape%1
    \label{ex:yuka:1}
\begin{forest}  for tree={grow'=east,delay={where content={}{shape=coordinate}{}}},   forked edges   
[
    [Tundra (Wadul)
    ]
    [Kolyma (Odul)
    ]
    [†\ili{Chuvan}
    ]
    [†\ili{Omok}
    ]
]
\end{forest}    
    \z

\noindent According to \citet{Nikolaeva2008}, however, “the linguistic status of \ili{Chuvan} and \ili{Omok} did not much differ from the status of other varieties of Old \ili{Yukaghiric}, and therefore referring to them as separate languages within the larger family to the exclusion of other known \ili{Yukaghir} idoms is unnecessary”. Old \ili{Yukaghiric} is a cover term used by \citet{Nikolaeva2008} for those varieties recorded during the 18\textsuperscript{th} and 19\textsuperscript{th} centuries. Given the limited information on languages other than Tundra and Kolyma \ili{Yukaghir} this chapter will be concerned primarily with these two modern languages.


\subsection{Question marking in Yukaghiric}\label{sec:5.14.2}

\textbf{Kolyma \ili{Yukaghir}} makes a difference between polar and \isi{content question}s in that only the latter take morphological marking. Polar questions are either expressed with rising \isi{intonation} or take an enclitic\textit{=duu} that appears twice in \isi{alternative question}s as well as in negative alternative \isi{questions} (\citealt{Nagasaki2011}: 245; \citealt{Maslova2003a}: 475-478). The semantic difference between polar \isi{questions} marked with \isi{intonation} or the enclitic is unknown to me.


\ea%2
    \label{ex:yuka:2}
    \ili{Yukaghir} (Kolyma)\\
    \ea
    \gll omo-s’    šoromo-k?\\
    good-\textsc{attr}  person-\textsc{pred}\\
    \glt ‘Is (he) a good person?’
    
    \ex
    \gll me-n’oho-j={duu}?\\
    \textsc{pred.foc}-fall-\textsc{intr}.3\textsc{sg}=\textsc{q}\\
    \glt ‘Did he fall down?’
    \ex
    \gll igeje  čičegej-gen!  ad-i={duu} šašaqa-daj-m={duu}?\\
    rope  strech-\textsc{imp.3sg}  strong-\textsc{intr}.3\textsc{sg}=\textsc{q}  tear-\textsc{caus}-\textsc{tr.S}=\textsc{q}\\
    \glt ‘Let the rope stretch! Is it strong (enough), or will he tear it up?’
    
    \ex
    \gll kudede{=}\textbf{{d}}\textbf{{uu}} \textbf{{oj}}{{}-l}{’e}{=}\textbf{{d}}\textbf{{uu}}?\\
    kill=\textsc{q} \textsc{neg}-be=\textsc{q}\\
    \glt ‘Have I killed it or not?’(\citealt{Maslova2003a}: 475-477)\z\z

\noindent The enclitic also exists in the \ili{Turkic} languages \ili{Yakut} and \ili{Dolgan}, where it has the form \textit{=duo} {\textasciitilde} \textit{=duu} (\sectref{sec:5.11.2}). The enclitic does not exist in Tundra \ili{Yukaghir} \citep[150]{Nikolaeva2006}, which also suggests a \ili{Turkic} origin.

\citet{Schiefner1871} published some material of a variety spoken along the Anadyr that is closely related to Kolyma \ili{Yukaghir} but possibly has affiliations with \ili{Chuvan} \citep[28]{Nikolaeva2006}. This variety does not appear to exhibit the enclitic. Instead, \isi{polar question}s remain unmarked and (negative) \isi{alternative question}s have a \isi{disjunction} of \ili{Russian} origin \citep[101]{Nikolaeva2006}. The tentative \isi{analysis} roughly follows \citet{Maslova2003a}.

\ea%3
    \label{ex:yuka:3}
    \ili{Yukaghir} (Kolyma, Anadyr)\\
    \gll mot  adó  kêt’ alí el{=kêt’?}\\
    1\textsc{sg}  son  come.\textsc{intr}.3\textsc{sg}  or \textsc{neg}=come.\textsc{intr}.3\textsc{sg}\\
    \glt ‘Did my son come or not?’ \citep[92]{Schiefner1871}
    \z

\noindent This absence of the enclitic in a variety of \ili{Yukaghir} spoken further away from \ili{Yakut} and \ili{Dolgan} is a further indication that it can be traced back to \ili{Turkic}. \citet[254]{Nagasaki2011} recorded a \isi{tag question} that was formed with the help of \ili{Russian} \textit{da}/\textcyrillic{да} ‘yes’ attached to a \isi{declarative sentence}.

According to \cite[66f.]{Maslova2003b} \isi{polar question}s in \textbf{Tundra Yukaghir} are formed with the help of the apparently sentence initial particle \textit{eld’e}. But the particle also appears in \isi{content question}s.


\ea%4
    \label{ex:yuka:4}
    \ili{Yukaghir} (Tundra)\\
    \ea
    \gll eld’e,  tide-ŋ    mit  t’ald’ed’uo  el=men’-me-k?\\
    well \textsc{dem.inv-foc}  1\textsc{pl}  ring \textsc{neg}=take-\textsc{tr}-2\textsc{sg}\\
    \glt ‘Well, haven’t you taken that ring of ours?’
    
    \ex
    \gll eld’e, neme-le  men’-me-k?\\
    well  what-\textsc{foc}  take-\textsc{tr}-\textsc{fc}\\
    \glt ‘Well, what have you bought?’ (\citealt{Maslova2001}: 48, 42)
    \z
    \z

\noindent A sentence initial \isi{question marker} indicates a connection with \ili{Chukotko-Kamchatkan} (\sectref{sec:5.3.2}). However, as the translation indicates, the word \textit{eld’e} is probably not a real question particle. Neither the exact meaning, nor its origin are discussed by Maslova. \cite[154f.]{Nikolaeva2006} assumes an underlying stem *\textit{el-} that could mean something like ‘good’, apparently unrelated to the negator \textit{el=} as seen in (\ref{ex:yuka:4}a).

\cite[66f.]{Maslova2003b} mentions two further particles, the dubitative \textit{quolem} (formally similar to interrogatives starting with \textit{quo{\textasciitilde}}) and hesitative \textit{ejk}. Furthermore, she claims, “if these particles are absent, the verb takes the Negative marker”. However, on the same page she gives an example of what appears to be a polar or \isi{focus question} that neither shows the particles, nor \isi{negation}.

\ea%5
    \label{ex:yuka:5}
    \ili{Yukaghir} (Tundra)\\
    \gll tet-{ek} Id’ilwej?\\
    2\textsc{sg}-\textsc{foc} \textsc{pn}\\
    \glt ‘Are you Idilway?’ \citep[67]{Maslova2003b}
    \z

Data given by \citet{Schmalz2012} confirms the hypothesis that unmarked \isi{polar question}s do not have any of the above mentioned particles. Consider the following examples with \isi{focus} marking on a constituent and on the verb, respectively.

\ea%6
    \label{ex:yuka:6}
    \ili{Yukaghir} (Tundra)\\
    \ea
    \gll tet-{ek} werwe-l?\\
    2\textsc{sg}-\textsc{foc}  be.strong-\textsc{n.}S\textsc{foc}\\
    \glt ‘Are you strong?’
    
    \ex
    \gll (mörde(ŋ)) me=möri-mk?\\
    news \textsc{pred.foc}=hear-\textsc{tr}.2\textsc{pl}\\
    \glt ‘Have you heard (the news)?’ (\citealt{Schmalz2012}: 69, 71)
    \z
    \z

Presumably, \isi{polar question}s can be indicated with \isi{intonation} only, as is also possible in Kolyma \ili{Yukaghir}. This suggests a connection with some \ili{Chukotko-Kamchatkan} languages (\sectref{sec:5.3.2}).

The proclitic \textit{me=} seen in (\ref{ex:yuka:6}b) can also be found in \isi{questions} with a denominal verb “to ask for mere confirmation of already known information” \citep[88]{Schmalz2012}. This indicates a functional \isi{similarity} to \isi{tag question}s in \ili{English}.

\ea%7
    \label{ex:yuka:7}
    \ili{Yukaghir} (Tundra)\\
    \gll me{=brigadir-ŋo-d’ek?}\\
    \textsc{pred.foc}=team.leader-be-\textsc{intr}.2\textsc{sg}\\
    \glt ‘(You) are the team leader, (aren’t you)?’ \citep[88]{Schmalz2012}
    \z

Alternative questions also differ from Kolyma \ili{Yukaghir} in that Tundra \ili{Yukaghir} uses a disjunctive connective \textit{ejk}, identical to the alleged hesitative \textit{ejk}.

\ea%8
    \label{ex:yuka:8}
    \ili{Yukaghir} (Tundra)\\
    \gll uo  purie-le ejk samnaldaŋn’e-le  aptaa-nu-m?\\
     child  berry-\textsc{acc} or  mushroom-\textsc{acc}  gather.\textsc{inch}-\textsc{dur}-\textsc{tr}.3\textsc{sg}\\
    \glt ‘Is the child picking berries or mushrooms?’ \citep[83]{Schmalz2012}
    \z

\noindent It is difficult to decide from the limited data whether \textit{ejk} has to be analyzed as a \isi{disjunction} or as a single \isi{question marker}, which is a possible marking pattern in some languages. Schmalz presents one instance of yet another possible \isi{disjunction}, \textit{uuri}, of unknown origin \citep[445]{Nikolaeva2006}.

\ea%9
    \label{ex:yuka:9}
    \ili{Yukaghir} (Tundra)\\
    \gll tet  ile me=čaal’-uon’ \textbf{uuri} me{=n’aawe-j.}\\
    2\textsc{sg}  reindeer \textsc{pred.foc}=be.bay-\textsc{intr.3sg}  or \textsc{pred.foc}=be.white-\textsc{intr.3sg}\\
    \glt ‘Is your reindeer bay or white?’ \citep[88]{Schmalz2012}
    \z

\ili{Yukaghiric} content \isi{questions} are more complicated than other question types and involve morphological marking on the verb. In \textbf{Kolyma Yukaghir} there is a split between three different paradigms. Special \isi{interrogative} marking is the default choice, except for so-called intransitive subjects (better called S) and direct objects (better called O), in which case \isi{focus} marking is employed \citep[245]{Nagasaki2011}. This distribution has certain \isi{ergative} characteristics, but within \isi{focus} marking two paradigms exist for transitive (so-called \textit{me}-participle, Tables \ref{tab:yuka:4} and \ref{tab:yuka:5} below) and intransitive verbs (\textit{l}-participle, Tables \ref{tab:yuka:2} and \ref{tab:yuka:3} below) \citep[240]{Nagasaki2011}. A questioned A (transitive subject) requires no \isi{focus} or agreement marking.

\ea%10
    \label{ex:yuka:10}
    \ili{Yukaghir} (Kolyma)\\
    \ea
    \gll tet qanin kelu-{k}?\\
    2\textsc{sg}  when  come-2\textsc{sg.q}\\
    \glt ‘When do you come?’ (questioned peripheral argument)
    
    \ex
    \gll kin{-tek} kelu-{l}?\\
    who-\textsc{pred.foc}    come-\textsc{ptcp}\\
    \glt ‘Who came?’ (questioned S)
    
    \ex
    \gll tet lem{-}{dik} ooʒe-t-{mo}?\\
    2\textsc{sg}  what-\textsc{pred.foc}  drink-\textsc{fut}-\textsc{ptcp}.2\textsc{sg}\\
    \glt ‘What will you drink?’ (questioned O)
    
    \ex
    \gll kin kudede?\\
    who  kill\\
    \glt ‘Who killed (it)?’ (questioned A) (\citealt{Nagasaki2011}: 245, 240)\z\z
 
In Kolyma \ili{Yukaghir}, interrogatives either stand sentence-initially or remain \textit{in situ} \citep[481]{Maslova2003a}. Sentence-initial position of interrogatives in \isi{NEA} is rare, but can also be found in \ili{Evenki} (\sectref{sec:5.10.3}). Note the additional predicative \isi{focus} marker \textit{-(le)k} (which has a special form on these two interrogatives) that is included in the \isi{case} par\-a\-digm by Maslova (\citeyear{Maslova1997}: 459f, \citeyear[88]{Maslova2003a}). It appears on nominal predicates as well as on intransitive subjects (S) and direct objects (O) and is thus not restricted to \isi{questions} \citep[227]{Nagasaki2011}. According to \citet[459]{Maslova1997} it is zero marked on “third person pronouns, proper nouns, and possessive NPs”.

Basically the same pattern of content \isi{question marking} was already in place in the 19th century, as can be seen from the following sentences given by \citet{Schiefner1871} for the variety already encountered above. Again, the tentative \isi{analysis} tries to follow \citet{Maslova2003a}.

\ea%11
    \label{ex:yuka:11}
    \ili{Yukaghir} (Kolyma, Anadyr)\\
    \ea
    \gll kanin kawe-i-ta-je-{k}?\\
    when    go-?\textsc{pfv}-\textsc{fut}-\textsc{intr}-2\textsc{sg.q}\\
    \glt ‘When will you leave?’ (questioned peripheral argument)
    
    \ex
    \gll kịn{-ak} kallu-{l} ta?\\
    who-\textsc{pred.foc}    come-\textsc{ptcp}  there\\
    \glt ‘Who came over there?’ (questioned S)
    
    \ex
    \gll mịt lom{dak} aa-ta-{m}?\\
    \textsc{1pl}  what-\textsc{pred.foc}  do-\textsc{fut}-\textsc{ptcp}.\textsc{1pl}\\
    \glt ‘What will we do?’ (questioned O)\footnote{Note that Kolyma \ili{Yukaghir} has ‘1\textsc{sg’} \textit{-me} but ‘1\textsc{pl’} \textit{-l} (\tabref{tab:yuka:5}).}
    
    \ex
    \gll kin ólo?\\
    who  steal\\
    \glt ‘Who stole (it)?’ (questioned A) (\citealt{Schiefner1871}: 101, 92)\z\z

\noindent The focused \isi{interrogative} \textit{kịn-ak} seems be closer to Tundra \textit{kin-ek} than to Kolyma \textit{kin-tek}. However, this could also be an artifact of the recording.

Content \isi{questions} in \textit{Tundra \ili{Yukaghir}} are better understood than \isi{polar question}s and exhibit a close affinity to those in Kolyma \ili{Yukaghir}. There are verbal suffixes that “are only used in specific [i.e. content] \isi{questions} to peripheral constituents” \citep[20]{Maslova2003b}. \citegen{Matić2014} summary of how content \isi{questions} are marked can be seen in \tabref{tab:yuka:1}. Marking of \isi{content question}s is thus basically identical to Kolyma \ili{Yukaghir}.

\begin{table}
\caption{Content questions in Tundra Yukaghir (\citealt{Matić2014}: 132, modified)}
\label{tab:yuka:1}
\begin{tabularx}{\textwidth}{lXXX}
\lsptoprule
& S/O & A & Oblique\\
\midrule
agreement & S/O \isit{focus} & - & \isit{interrogative}\\
marking on \isit{interrogative} & \isit{focus} \isit{case} & - & -\\
position of \isit{interrogative} & \multicolumn{3}{c}{preverbal or sentence initial position}\\
\lspbottomrule
\end{tabularx}
\end{table}

\ea%12
    \label{ex:yuka:12}
    \ili{Yukaghir} (Tundra)\\
    \ea
    \gll neme-lek  joorǝ-t-{ook}?\\
    what-\textsc{inst}  play-\textsc{fut}-1\textsc{pl.q}\\
    \glt ‘What are we going to play with?’ (questioned peripheral argument)
    
    \ex
    \gll kin{-}{ek} ewrǝ-{l}?\\
    who-\textsc{foc}  come-\textsc{ptcp}\\
    \glt ‘Who has arrived?’ (questioned S)
    
    \ex
    \gll neme{-}{lǝ} ńeed’i-t-{mǝŋ}?\\
    what-\textsc{foc}  tell-\textsc{foc}-\textsc{ptcp}.2\textsc{sg}\\
    \glt ‘What will you tell (us)?’ (questioned O)
    
    \ex
    \gll tet-qanǝ kin tite  gitńǝr  wee?\\
    2\textsc{sg}-\textsc{acc}  who  so  up.to  do\\
    \glt ‘Who did that to you?’ (questioned A) (\citealt{Matić2014}: 131f.)\z\z

\noindent The \isi{focus} marker has the form \textit{-lǝ(ŋ)} {\textasciitilde} \textit{-(ǝ)k} (\citealt{Matić2014}: 131). \citet{Maslova2003b}: 8, 52) gives the form as \textit{-le(ŋ)} {\textasciitilde} \textit{-(e)k} and again includes it in the \isi{case} paradigm. According to \citet[55]{Schmalz2012}, \textit{-le(ŋ)} usually attaches to nouns and \textit{-(e)k} to pronouns. Interestingly, \textit{kin(-ek)} ‘who’ thus behaves like pronouns and \textit{neme(-le)} ‘what’ like nouns, which is a common cross-linguistic pattern (\sectref{sec:4.3}). The variant \textit{-leŋ} tends to be a \isi{focus} marker and \textit{-le} an accusative \citep[54]{Maslova2003b}. The obligatory \isi{combination} of \isi{focus} markers with certain verb forms has a typological parallel in \ili{Japonic}, where a similar phenomenon is called \textit{\isi{kakari musubi}} (\sectref{sec:5.6.1}). Tables \ref{tab:yuka:4}  and \ref{tab:yuka:5} exclude paradigms for marking of A (transitive subject) as they have almost no special marking; see (\ref{ex:yuka:10}d), (\ref{ex:yuka:11}d), and (\ref{ex:yuka:12}d). In Tundra \ili{Yukaghir} the third person pronouns take the forms \textit{tud} and \textit{titt}. The verb furthermore remains unmarked except for third person \isi{plural} -\textit{ŋu} \citep[56]{Schmalz2012}.

\begin{table}
\caption{Focus marking in intransitive clauses in Tundra Yukaghir \citep[56]{Schmalz2012}; \textit{uu(l)-} ‘to go’}
\label{tab:yuka:2}
\begin{tabularx}{\textwidth}{X ll@{\hspace*{3cm}}ll}
\lsptoprule
& \multicolumn{2}{l}{Verb Focus} & \multicolumn{2}{l}{Subject Focus (S)}\\
\midrule
1\textsc{sg} & met & mer=uu-je-\textbf{ŋ} & met-ek & uu-l\\
2\textsc{sg} & tet & mer=uu-je-k & tet-ek & uu-l\\
3\textsc{sg} & tude.l & mer=uu-j & tude.l & uu-l\\
1\textsc{pl} & mit & mer=uu-je-li & mit-ek & uu-l\\
2\textsc{pl} & tit & mer=uu-je-mut & tit-ek & uu-l\\
3\textsc{pl} & titte.l & mer=uu-\textbf{ŋi} & titte.l & uu-\textbf{ŋu}-l\\
\lspbottomrule
\end{tabularx}
\end{table}

\begin{table}
\caption{Focus marking in intransitive clauses in Kolyma Yukaghir (\citealt{Maslova2003a}: 140, 144, 234; \citealt{Nagasaki2011}: 230); \textit{šohie} ‘get lost, disappear’, \textit{amde-} ‘to die’; constructed in analogy to \tabref{tab:yuka:2}}
\label{tab:yuka:3}
\begin{tabularx}{\textwidth}{X ll@{\hspace*{3cm}}ll}
\lsptoprule
& \multicolumn{2}{l}{Verb Focus} & \multicolumn{2}{c}{Subject Focus (S)}\\
\midrule
1\textsc{sg} & met & m=amde-je-\textbf{Ø} & met-ek & šohie-l\\
2\textsc{sg} & tet & m=amde-je-k & tet-ek & šohie-l\\
3\textsc{sg} & tude.l & m=amde-j & tude.l & šohie-l\\
1\textsc{pl} & mit & m=amde-j-l’i & mit-ek & šohie-l\\
2\textsc{pl} & tit & m=amde-j-met & tit-ek & šohie-l\\
3\textsc{pl} & titte.l & m=amde-\textbf{ŋ}i & titte.l & šohie-\textbf{ŋ}i-l\\
\lspbottomrule
\end{tabularx}
\end{table}

\begin{table}
\caption{Focus marking in transitive clauses in Tundra Yukaghir \citep[56]{Schmalz2012}; \textit{ai-} ‘to shoot’}
\label{tab:yuka:4}
\begin{tabularx}{\textwidth}{X ll@{\hspace*{3cm}}ll}
\lsptoprule
& \multicolumn{2}{l}{Verb Focus} & \multicolumn{2}{l}{Object Focus (O)}\\
\midrule
1\textsc{sg} & met & mer=ai-\textbf{ŋ} & met & ai-meŋ\\
2\textsc{sg} & tet & mer=ai-mek & tet & ai-meŋ\\
3\textsc{sg} & tude.l & mer=ai-m & tude.l & ai-mele\\
1\textsc{pl} & mit & mer=ai-j & mit & ai-l\\
2\textsc{pl} & tit & mer=ai-mk & tit & ai-mk\\
3\textsc{pl} & titte.l & mer=ai-\textbf{ŋa} & titte.l & ai-\textbf{ŋu}-mle\\
\lspbottomrule
\end{tabularx}
\end{table}

\begin{table}
\caption{Focus marking in transitive clauses in Kolyma Yukaghir (\citealt{Maslova2003a}: 140, 144; \citealt{Nagasaki2011}: 221, 230); \textit{juø-} ‘to see, to look at’, \textit{aa-} ‘to make’; constructed in analogy to \tabref{tab:yuka:4}}
\label{tab:yuka:5}
\begin{tabularx}{\textwidth}{X ll@{\hspace*{3cm}}ll}
\lsptoprule
& \multicolumn{2}{l}{Verb Focus (FUT)} & \multicolumn{2}{l}{Object Focus (O)}\\
\midrule
1\textsc{sg} & met & aa-t-\textbf{Ø} & met & juø-me\\
2\textsc{sg} & tet & aa-te-mek & tet & juø-me\\
3\textsc{sg} & tude.l & aa-te-m & tude.l & juø-mele\footnotemark{}\\
1\textsc{pl} & mit & aa-te-j & mit & juø-l\\
2\textsc{pl} & tit & aa-te-met & tit & juø-met\\
3\textsc{pl} & titte.l & aa-\textbf{ŋ}i-te-m & titte.l & juø-\textbf{ŋ}i-le\\
\lspbottomrule
\end{tabularx}
\end{table}

\footnotetext{ This suffix takes the form \textit{-mle} if following the future marker \textit{-te}.}

The special \isi{interrogative verb} endings from both languages are collected in Tables \ref{tab:yuka:6} and \ref{tab:yuka:7}, comparing them with the declarative endings. The suffixes \textit{-m(e)} and \textit{{}-je} that sometimes appear in front of the agreement markers express transitivity and intransitivity, respectively \citep[141]{Maslova2003a}. For the most part, the paradigms in Tundra and Kolyma \ili{Yukaghir} are extremely similar or even identical. One difference is the presence of a first person \isi{singular} agreement marker \textit{-ŋ} in Tundra \ili{Yukaghir} that is absent in Kolyma \ili{Yukaghir}. The same difference can be observed in the transitive verb \isi{focus} paradigms (Tables \ref{tab:yuka:4}, \ref{tab:yuka:5}). Furthermore, Tundra \ili{Yukaghir} has a special second \isi{plural} ending \textit{-mk} in the transitive paradigm instead of the expected \textit{-mut} (also compare Tables \ref{tab:yuka:4}, \ref{tab:yuka:5}).

\begin{table}
\caption{Tundra Yukaghir non-future endings \citep[18]{Maslova2003b}; for the interrogative only future endings are available, showing the additional future suffix \textit{-t(e)}}
\label{tab:yuka:6}
\begin{tabularx}{\textwidth}{XXXl}
\lsptoprule
& \textbf{\textsc{tr nonfut}} & \textbf{\textsc{itr nonfut}} & \textsc{q} \textbf{\textsc{fut}}\\
\midrule
1\textsc{sg} & -ŋ & -je-ŋ & -te-m\\
2\textsc{sg} & -me-k & -je-k & -te-k\\
3\textsc{sg} & -m-Ø & -j-Ø & -t-Ø\\
1\textsc{pl} & -j & -je-l’i & -t-uok\\
2\textsc{pl} & -mk & -je-mut & -te-mut\\
3\textsc{pl} & -ŋa (\textsc{fut} -ŋu-te-m) & -ŋi-Ø (\textsc{fut} -ŋu-te-j) & -ŋu-t-Ø\\
\lspbottomrule
\end{tabularx}
\end{table}

\begin{table}
\caption{Kolyma Yukaghir non-future endings according to \citet[140]{Maslova2003a}; alternations of \textit{j} not shown here include \textit{d’} and \textit{č} \citep[43]{Maslova2003a}; alternative forms in square brackets according to \cite[228f.]{Nagasaki2011}}
\label{tab:yuka:7}
\begin{tabularx}{\textwidth}{Xlll}
\lsptoprule
& \textbf{\textsc{tr nonfut}} & \textbf{\textsc{itr nonfut}} & \textsc{q} \textbf{\textsc{nonfut}}\\
\midrule
1\textsc{sg} & -Ø & -je-\textbf{Ø} & -m\\
2\textsc{sg} & -me-k/[-mi-k] & -je-k & -k\\
3\textsc{sg} & -m-Ø & -j-Ø/[-Ø] & -Ø\\
1\textsc{pl} & -j & -je-l’i [-j(ii)-li] & -l-ook/[-uɵk]\\
2\textsc{pl} & -met & -je-met [-j(e)-met] & -met\\
3\textsc{pl} & -ŋaa/[-ŋam] (\textsc{fut} -ŋi-te-m) & -ŋi-Ø (\textsc{fut} -ŋi-te-j) & -ŋi-Ø\\
\lspbottomrule
\end{tabularx}
\end{table}

There is the possibility that \isi{interrogative} agreement forms in \ili{Negidal}---most unusual for a \ili{Tungusic} language---may be traced back to \ili{Yukaghiric} influence (\sectref{sec:5.10.2}). Similar to both Kolyma and Tundra \ili{Yukaghir}, \ili{Negidal} has special agreement forms for the first person \isi{singular} \textit{-m} as well as the \isi{plural} (inclusive) \textit{-p}, and the third person \isi{plural} remains unmarked. The formal \isi{similarity} in the \isi{singular} is accidental, but the typological parallel is unlikely to be due to chance. However, \ili{Negidal} has the same marking throughout all question types and combines this with other \isi{question marker}s.


\subsection{Interrogatives in Yukaghiric}\label{sec:5.14.3}

\cite{Nikolaeva2006} reconstructed several \ili{Proto-Yukaghiric} interrogatives. The form *\textit{kin} ‘who’ is very similar to forms with the same meaning in several surrounding languages (the so-called KIN-interrogatives, Chapters 3 and 6). The interrogatives *\textit{qa-} ‘which’ and *\textit{qo-} (> \textit{quo-} in Tundra \ili{Yukaghir}) ‘where’ must be related, historically. They suggest a connection between the two categories of \textsc{selection} and \textsc{place}, the latter usually being derived from the former. However, as is often the case, a \isi{reconstruction} of clear-cut \isi{interrogative} stems is rather questionable. More generally, \ili{Yukaghiric} exhibits the common \textit{K{\textasciitilde}} \isi{resonance} present in many languages of the area (Chapters \ref{sec:3} and \ref{sec:6}). Proto-\ili{Yukaghir} ?*\textit{leme} ‘what’ may have started with an *\textit{n} instead of an *\textit{l} (Kolyma \ili{Yukaghir} \textit{leme} {\textasciitilde} \textit{neme}, Tundra \ili{Yukaghir} \textit{neme}) as did *\textit{noŋoon} ‘what for’. \tabref{tab:yuka:9} gives a more exhaustive list of forms from the two extant \ili{Yukaghiric} languages. Most forms start with a \textit{q{\textasciitilde}}, only a few with \textit{n{\textasciitilde}} ({\textasciitilde} \textit{l{\textasciitilde}}) and \textit{kin} ‘who’ has a special position in both languages. Interestingly, the functional distribution of the resonances \textit{k{\textasciitilde}}, \textit{n{\textasciitilde}}, \textit{q\textasciitilde} is almost identical to \ili{Turkic} languages (\sectref{sec:5.11.3}). In contrast to what \citegen{Nikolaeva2006} reconstructions suggest, the two \ili{Yukaghiric} languages share several very specific interrogatives that can be traced back directly to the proto-language.

\begin{table}
\caption{Interrogatives in Kolyma (\citealt{Nagasaki2011}: 245; \citealt{Maslova2003a}: 238, 250) and Tundra Yukaghir \citep[41]{Maslova2003b}}
\label{tab:yuka:9}
\begin{tabularx}{\textwidth}{lXXX}
\lsptoprule

\textbf{Meaning} & \textbf{Kolyma} &  & \textbf{Tundra}\\
\midrule
& \textbf{Nagasaki} & \textbf{Maslova} & \textbf{Maslova}\\
\midrule
who & kin & kin & kin\\
what & leme & leme {\textasciitilde} neme & neme\\
what for & nooŋon & noŋon & \\
by what & numun &  & \\
to say what & monoʁod- &  & \\
which & qadi &  & qadi\\
what (verb modifier) &  &  & qadinol\\
how many/much & qamun & qamun & qabu-n {\textasciitilde} qabu-d\\
where & qon & qo-n & qadaa\\
to be where &  & qol-l’e- & \\
whence & qot & qo-t & qadaa-t\\
whither & qaŋide & qa-ŋide & \\
along what route &  &  & qadaa-n\\
how & qodo {\textasciitilde} qode & qodo & quode\\
to be how &  & qodo-l’e- & quode-ban-\\
at what place & qadun & qadoon- & \\
when & qanin & qanin & qan’in\\
why & qodit & qodi-t & quodii\\
how often & qamlid’e &  & qaml’id’e\\
to be how & qodimie- & qodimie & \\
to be how many/much & qamloo- &  & qamlal\\
\lspbottomrule
\end{tabularx}
\end{table}

A difference can be found in the locative interrogatives, i.e. Kolyma \textit{qon} versus Tundra \textit{qadaa} ‘where’. Additionally, while in Tundra \ili{Yukaghir} \isi{case} markers attach directly to the locative \isi{interrogative}, the \isi{case} marker replaces the final \textit{-n} in Kolyma \ili{Yukaghir}. \cite[186, 208]{Schmalz2013}, in his otherwise excellent grammar of Tundra \ili{Yukaghir}, analyzes the initial \textit{q-} as the analyzable \isi{interrogative} stem for the interrogatives in \ili{Yukaghir}, which might be too far-fetched. The \isi{resonance} in \textit{q\textasciitilde}, of course, could indicate an original etymological connection, but similarities with \isi{demonstratives} are perhaps better analyzed as the result of an additional \isi{resonance} phenomenon or paradigmatic analogy to the \isi{demonstratives} (e.g., \citealt{Diessel2003}, \citealt{BickelNichols2007}). \cite[186, 208]{Schmalz2013} also mentions some additional interrogatives for Tundra \ili{Yukaghir} that have not been listed above, such as \textit{quodeband'e} ‘what kind of’ that he analyzes as \textit{quode} ‘how’, \textit{pan-} ‘to be’, and the participle \textit{-je} etc.