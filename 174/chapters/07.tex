\chapter{Conclusion}

According to \citet[429]{EvansLevinson2009}, “we are the only species with a communication system that is fundamentally variable at all levels.” The investigation of \isi{linguistic diversity} thus should be a major concern of linguistics in general and of typology in particular. The main research question of this study was, following \citet[248]{Bickel2007}, “what’s where why?”

\begin{quote}
Asking “what’s where?” targets \isi{universal} preferences as much as geographical or genealogical skewings, and results in probabilistic theories stated over properly sampled distributions. Asking “why?” is based on the premises that (i) typological distributions are historically grown and (ii) that they are interrelated with other distributions. \citep[239]{Bickel2007}
\end{quote}

\noindent Therefore, the present study is not a classical \isi{synchronic} typological investigation, but focused on the distribution and preliminary explanation of \isi{linguistic diversity} found in the limited geographical area of \isi{Northeast Asia} (\isi{NEA}), tentatively defined as the area north of the \isi{Yellow River} and east of the \isi{Yenisei} (e.g., \citealt{Chard1974}). Another question formulated in the Introduction was whether the concept of \isi{Northeast Asia} makes sense from the point of view of \isi{areal linguistics}. The answer is certainly yes, but with limitations. The definition of \isi{Northeast Asia} as a concept strongly depends on its opposition with \isi{Mainland Southeast Asia} (MSEA, \citealt{EnfieldComrie2015}). Regarding the number of languages (\isi{language diversity}), \isi{NEA} with with perhaps up to 150 languages ranks much lower than MSEA, which is the home of up to 600 different languages. In terms of different linguistic stocks, however, \isi{NEA} has 14 instead of only 5 found in MSEA (\isi{phylogenetic diversity}). In comparison, the region of New Guinea is home to approximately 1200 languages from about 35 language families on an area of only 850,000 km\textsuperscript{2} \citep{Foley2000}. In \isi{NEA}, \isi{Mongolia} alone is larger than that area. There are similarly pronounced regional differences in \isi{linguistic diversity} within \isi{Northeast Asia}. The highest concentration of languages can be found in peripheral regions such as the \isi{Amdo} region, the \isi{Ryūkyūan Islands}, in the \isi{Amur} river shed, and around the \isi{Altai} extending northwards along the \isi{Yenisei} as well as southward along adjacent mountainous regions. Following \citet{Nichols1992,Nichols1997} these can be characterized as \textit{residual} or \textit{accretion zones}. Language diversity is at its lowest in central parts around \isi{Mongolia}, northern \isi{China}, central \isi{Siberia}, \isi{Korea}, and central parts of \isi{Japan}, which qualifies as a large coherent \textit{spread zone}. Regarding \isi{phylogenetic diversity}, there is quite a different distribution that peaks around the eastern part of \isi{NEA} along the \isi{Pacific Rim} (Pacific \isi{NEA}), where representatives of 12 of the 14 language families of \isi{NEA} can be found (e.g., \citealt{Anderson2010}). Historically, however, both \ili{Yeniseic} and \ili{Samoyedic}, which are the only exceptions, have been spoken further towards the southeast as well. No doubt there is a multitude of reasons for these strong differences in \isi{linguistic diversity}, including climatic (e.g., \isi{temperature}, \isi{precipitation}), geographical (e.g., \isi{landscape roughness}, \isi{river density}), and cultural factors (e.g., \isi{subsistence} patterns, \isi{agriculture}, \isi{pastoralism}, hunting and gathering) (e.g., \citealt{Nichols1992}; \citealt{Nettle1999}; \citealt{AxelsenManrubia2014}). Not only is there a complex mixture of different causes located on different \isi{time scales}, but the importance of individual factors varies from region to region. These factors clearly also influence the size of languages, which is greatest in the southeast (\ili{Mandarin}, \ili{Japanese}, \ili{Korean}) and decreases towards the west and especially towards the north and seems to be directly correlated with the distribution of \isi{population density} and environmental factors such as \isi{climate} and vegetation, and, consequently, the existence of \isi{agriculture}. Understandably, the exact causes of \isi{phylogenetic} and \isi{language diversity} could not be investigated within this study, which focused on \isi{structural diversity}, more precisely the diversity found in the \textit{\isi{grammar of questions}}, i.e. those aspects of any given language that are specialized for asking \isi{questions}. The primary distinction made in the \isi{grammar of questions} of a given language is in \isi{question marking}, \isi{interrogative}s, and optional additional \isi{functional domain}s such as \isi{coordination}, \isi{focus}, and \isi{negation}. A comparison of the \isi{structural diversity} of the grammar of \isi{questions} found in MSEA and \isi{NEA} was not feasible as there are simply too many languages to investigate in MSEA. The obvious next step should thus be to expand the typology proposed in this study to \isi{Mainland Southeast Asia} (see \citealt{Clark1985}; \citealt{Huang1996}; \citealt{Huang1999}; \citealt{Enfield2010}; \citealt{Rajasingh2014}, etc.) and to other regions from around the globe. Nevertheless, there is evidence that \isi{NEA} and MSEA, despite manifold differences (Chapter 3), together form one large area with a preponderance of sentence-final \isi{polar question} markers \citep{Dryer2013k}. For reasons of space this study necessarily also excluded responses and answers, which is yet another avenue for further research (e.g., \citealt{EnfieldStiversLevinson2010}). Future studies should also pay more attention to \isi{intonation} in \isi{questions}, which was for the most part neglected here for mere lack of information (e.g., \citealt{Sicoli2014} and references therein). However, this study identified many important aspects of the \isi{grammar of questions} in \isi{NEA} and beyond, ranging from general principles (Chapter 4) to specific aspects of the languages of \isi{NEA} (Chapters 5, 6). Given the focus on one area, the typology of \isi{questions} presented in this study was necessarily limited. I intend to elaborate on it in future studies with a global coverage. For example, the exact distribution and explanation of KIN- and K-interrogatives can only be settled with the help of a global sample of languages. The total discussion mentions over 450 languages and dialects from \isi{NEA} and beyond (see the Language Index). Altogether about 900 glossed examples were given. The aim was to achieve both a cross-linguistically plausible typology and a resolution of the \isi{linguistic diversity} of \isi{Northeast Asia} as much as possible (\citealt{VoegelinVoegelin1964}: 2).

Chapters 3 and 6 identified several important areal features such as KIN- and K-interrogatives that have a strong basis in \isi{Northeast Asia} as well as more localized instances of diffusion and convergence such as in the so-called \isi{Amdo Sprachbund}. Concerning the \isi{grammar of questions}, the \ili{Tungusic} languages play a less important role for \isi{NEA} than was assumed in \sectref{sec:3.4}. However, there is no point in arguing whether \isi{Northeast Asia} qualifies as a clear-cut \textit{linguistic area}, given the problematic status of the concept itself (e.g., \citealt{Campbell2006}). In terms of \isi{structural diversity}, \isi{Northeast Asia} admittedly has a relatively clear boundary towards the southeast, i.e. \isi{Mainland Southeast Asia} (e.g., \citealt{EnfieldComrie2015}), but not towards the west (e.g., \citealt{HeggartyRenfrew2014b}). While there are certain features such as the existence of \isi{front rounded vowels} that are relatively widespread in \isi{NEA}, these can often also be found in the adjacent regions towards the west, such as \isi{Europe}. The reason for this seems to be in the fact that \isi{NEA} over millennia was the starting point for a multitude of population movements and linguistic spreads over all of northern Eurasia towards the west (e.g., \citealt{Nichols1997}: 376f.). Another major direction of spread was from southern \isi{NEA} towards the north, often following the rivers \isi{Yenisei} and \isi{Lena} (e.g., \citealt{Skribnik2004}: 151). Not only do all three large language families of \isi{Europe}, \ili{Indo-European}, \ili{Uralic}, and \ili{Turkic}, derive from a location further to the east or even from \isi{NEA}, but the ancestors of \textit{all} native Americans and their languages necessarily had their origin within \isi{NEA} as well (e.g., \citealt{LlamasFehren-Schmitz2016} and references therein). \isi{Northeast Asia} thus holds a key position for regions as far apart as western \isi{Europe} and the Americas. One of the best examples for the importance of especially southern \isi{NEA} for the dispersal of peoples and languages is the recent discovery of the so-called Mal’ta specimen found near lake \isi{Baikal} that is about 24,000 years old (\citealt{RaghavanSkoglund2014}). It represents a population called the \textit{\isi{Ancient North Eurasians}} that lack a closer relation to modern East Asians. Instead, \isi{Ancient North Eurasians} are one of four major founding lineages thus far identified for modern Europeans in the west (\citealt{JonesGonzales-Fortes2015}), and also significantly contributed to the genome of the Kets along the middle \isi{Yenisei} in the north (\citealt{FlegontovChangmai2016}) as well as of native Americans that initially spread towards \isi{Beringia} in the northeast (\citealt{RaghavanSkoglund2014}). Even though the \isi{time scales} involved are too large to be accessible through historical linguistics, such population movements certainly were also connected with the spread of languages. Take the \isi{Yamnaya} culture in the \isi{Pontic-Caspian steppe}, for example, which is thought to have brought both the ANE genome as well as the \ili{Indo-European} languages into \isi{Europe} (\citealt{Anthony2007}; \citealt{AnthonyRinge2015}; \citealt{AllentoftSikora2015}; \citealt{HaakLazaridis2015}; \citealt{JonesGonzales-Fortes2015}). While \isi{NEA} played a crucial role in the spread of populations to other parts of the world, it was itself reached by populations and thus most likely by languages from as far south as southern \isi{China} (\citealt{HongQi2013}) and \isi{Southeast Asia} or perhaps \isi{Australiasia} (\citealt{RaghavanSteinrücken2015}; \citealt{SkoglundMallick2015}; \citealt{Reich2018}: 176-181), which again left traces as far apart as northern and eastern \isi{Europe} and South America, respectively. These results have potential implications for the search of long-term relations between languages that cannot be restricted to \isi{NEA} alone.

The title of this study promised \textit{an ecological perspective} and the Introduction tentatively identified the approach as a so-called \textit{\isi{ecological typology}}. This approach shares its appreciation of human and \isi{linguistic diversity} with several other approaches (e.g., \citealt{EvansLevinson2009}; \citealt{Levinson2012a}), but in addition follows the so-called \textit{\isi{ecological commitment}} (\citealt{Hölzl2015e}: 186) that the description of language “should be reconcilable with what is known from ecological research”, which was formulated in analogy to the well-known \textit{cognitive commitment} that continues to define \isi{Cognitive Linguistics} (e.g., \citealt{Evans2012}). While the cognitive approach sees “language as an integral part of \isi{cognition}” \citep[539]{Langacker2008}, the ecological approach---and what was tentatively called \textit{ecological typology} is only a part of it---in my interpretation sees language as an integral part of \isi{ecology}, i.e. the \textit{\isi{organism-environment system}} (e.g., \citealt{Järvilehto1998}; \citealt{Odling-SmeeLaland2009}). A \isi{similarity} of the two approaches is the attempt to find explanations in general principles (\citealt{Hölzl2015d}: 185), cf. the \textit{generalization commitment} in \isi{Cognitive Linguistics} (e.g., \citealt{Evans2012}). In my eyes, \textit{\isi{ecology}} is a valuable cover term for an emerging field of investigations that, for the explanation of \isi{linguistic diversity} and language structure, acknowledges a multitude of different \textit{reasons} (e.g., \citealt{SteffensenFill2014}; \citealt{Bickel2015}; \citealt{DeBusser2015}) that take effect on different \textit{\isi{time scales}} or \textit{\isi{causal frames}} (e.g., \citealt{Enfield2014}). This conceptual shift promises deep implications of which not even the surface could be scratched by this study. Linguistic diversity cannot be considered independently of a multitude of factors, ranging from the invention of the \isi{wheel}, over the domestication of the \isi{reindeer} or the biochemistry of the \isi{brain}, up to the amount of \isi{precipitation}.

In one sense that was emphasized throughout this book, \isi{ecology} “represents a shift of emphasis from a single language in isolation to many languages in \isi{contact}.” (\citealt{VoegelinVoegelin1964}: 2) Following \citet{SteffensenFill2014}, this was called \textit{\isi{symbolic ecology}}. The subheading \textit{An ecological perspective} thus mainly refers to the aspect of \isi{language contact} within the entire linguistic landscape of \isi{Northeast Asia}. The influence of other ecologies such as those mentioned in the Introduction (e.g., cognitive, natural, sociocultural) are only beginning to be understood and consequently had a subordinate position (e.g., \citealt{DeBusser2015}). Nevertheless, there are indications that these influences should not be underestimated and deserve further research (e.g., \citealt{AxelsenManrubia2014}; \citealt{Everett2016}). An investigation of the impact of \isi{climate}, for instance, is necessarily based on a global sample of languages which could not be achieved within this regional study. However, a comparison of the results for \isi{NEA} in this study and a global sample by \citet{Dryer2013m} suggests a possible climatic influence on \isi{question marking} and especially \isi{intonation}: The lack of languages in \isi{NEA} that mark \isi{polar question}s with \isi{intonation} alone and do not have further \isi{question marking} strategies (but see \sectref{sec:5.3.2}) could be attributed to the fact that, for some reason, such languages are usually located in the \isi{tropics}. In fact, \citet[1322]{Everett2015} recently found more convincing evidence for a possible climatic influence on language structure:

\begin{quote}
The sound systems of human languages are not generally thought to be ecologically adaptive. We offer the most extensive evidence to date that such systems are in fact adaptive and can be influenced, at least in some respects, by climatic factors. Based on a survey of laryngology data demonstrating the deleterious effects of aridity on vocal cord movement, we predict that complex tone patterns should be relatively unlikely to evolve in arid [and cold] climates.
\end{quote}

\largerpage
\noindent In many cases such as this there may be several reasons for a certain phenomenon. Concerning the occurrence of tones in MSEA but not in \isi{NEA} there are further possible explanations, including \isi{language contact} or even the occurrence of certain genes \citep{Dediu2011}. Of course, a language can only mark \isi{questions} with the help of \isi{tones} if the language possesses tones in the first place (\citealt[593]{HymanLeben2000}).

Additionally, \sectref{sec:4.4} has tentatively proposed an \isi{ecological theory of questions}, which describes them as a form of \textit{\isi{exploratory behavior}} (e.g., \citealt{Gibson1988}) in the \textit{\isi{dialogical array}} (\citealt{Gibson1979}; \citealt{Hodges2009}). This \isi{exploration} can be explained with \textit{specific epistemic \isi{curiosity}} (\citealt{Berlyne1954,Loewenstein1994}), which itself is evoked by so-called “\isi{collative}” (i.e. novel, changing, complex, conflicting, surprising, or uncertain, \citealt{Berlyne1978}) properties of the \isi{organism-environment system} (\citealt{Järvilehto1998,Turvey2009}). Humans seek comprehension and clarity, and there are several ways of achieving this, including mental problem solving, physical exploration, or asking \isi{questions}. However, like other types of \isi{exploratory behavior}, \isi{questions} are a proactive process (e.g., \citealt{Gibson1988}: 5f.). Questions are not merely a \isi{request} for information, but crucially involve \isi{predictions} by the speaker and thus depend on our previous experience.