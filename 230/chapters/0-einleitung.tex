\chapter{Einleitung}

Wie in vielen anderen Sprachen hat sich der deutsche Definitartikel aus einem adnominal gebrauchten \isi{Demonstrativum} entwickelt  \parencite{Oubouzar1992,Szczepaniak2011a}. Die entscheidende Phase des Wandels spielt sich im Althochdeutschen (Ahd.) ab, also in der Zeit von 750--1050 n. Chr. Das ursprüngliche \isi{Demonstrativum}  \object{dër}\footnote{Im Folgenden wird die normalisierte Form \object{dër} verwendet, um auf das althochdeutsche (ahd.) System Bezug zu nehmen. Alle nachfolgenden Beispiele werden -- wenn nicht anders angegeben -- in der überlieferten Form dargestellt.} verliert in dieser Zeit seine demonstrative, d.h. verweisende Funktion und erschließt Gebrauchskontexte, in denen die eindeutige Identifizierbarkeit des Referenten auch unabhängig von der Gesprächssituation gewährleistet ist. Obwohl diese Entwicklung durch die ahd. Überlieferungslage gut dokumentiert ist, fehlen bislang systematische Korpusuntersuchungen. \is{Korpuslinguistik} Mit \textcite{Hodler1954} und \textcite{Oubouzar1989,Oubouzar1992,Oubouzar1997a} liegen zwar durchaus größere empirische Arbeiten vor, allerdings fehlen darin klare semantische Analysekriterien. Auch Brückenkontexte \is{Brückenkontext} \parencite{Heine2002a}, die \textcite{Himmelmann1997} für außereuropäische Sprachen ausgearbeitet hat, sind bisher ununtersucht geblieben. Die Studien von \textcite{Abraham1997}, \textcite{Philippi1997} und \textcite{Leiss2000}, die mögliche Antworten auf die Frage liefern, warum sich der \isi{Definitartikel} überhaupt entwickelt hat, stützen sich auf Analysen von Einzelbeispielen. Das Gleiche gilt für die Untersuchungen von \textcite{Demske2001}, \textcite{Kraiss2012,Kraiss2014} und \textcite{Schlachter2015}, welche den funktionalen Wandel in den Fokus rücken.
In der vorliegenden Arbeit wird die Entwicklung des Definitartikels \is{Definitartikel} erstmals an einer größeren Datenmenge computergestützt und mit korpuslinguistischen Methoden \is{Korpuslinguistik} untersucht. Ermöglicht wird dieses Vorgehen durch die zunehmende Digitalisierung und \isi{Annotation} historischer Daten der letzten Jahre. Die Textgrundlage für die Korpusuntersuchung \is{Korpuslinguistik} bilden die fünf größten ahd. Textdenkmäler aus dem  \object{Referenzkorpus Altdeutsch} \parencite{Donhauser2014}.\footnote{Die Daten entsprechen einer Vorversion zur Version 0.1 des Referenzkorpus.}\is{Korpus}

\section{Zielsetzung} 

Mit der Untersuchung soll der funktionale Wandel des Definitartikels \is{Definitartikel} empirisch erschlossen werden. Das Ziel ist, über systematische Analysen von Gebrauchskontexten, in denen \object{dër} erscheint, die ersten Stufen des Grammatikalisierungspfades \is{Grammatikalisierungspfad} \parencite{Lehmann2015} für das Deutsche zu rekonstruieren und auf den Prüfstand zu stellen. Es wird davon ausgegangen, dass zu den ursprünglich prag"-ma"-tisch-definiten (situationsabhängigen) Gebrauchskontexten \is{Pragmatische Definita}
 se"-man"-tisch-definite (situationsunabhängige) \is{Semantische Definita} hinzukommen \parencite{Lobner1985,Himmelmann1997}. 
Die Entwicklung des Definitartikels spielt sich aber nicht nur auf  Morphem-, \is{Morphem} sondern auch auf Phrasenebene \is{Phrase} ab, da sich erst in Verbindung mit einem \isi{Substantiv} aus dem ursprünglichen Demonstrativum \is{Demonstrativartikel} ein \isi{Definitartikel} entwickeln konnte. In anderen Kontexten grammatikalisieren Demonstrativa \is{Demonstrativum} bspw. zu Konjunktionen \is{Konjunktion} oder Personalpronomen \is{Personalpronomen} \parencite{Diessel1999}. Aus der bisherigen Forschung \parencite[u.a.][]{Oubouzar1989,Oubouzar1992}
lässt sich ableiten, dass nicht alle Substantivtypen \is{Substantiv} gleichermaßen mit \object{dër} kombiniert werden. Aufbauend auf \textcite{Szczepaniak2011a} und \textcite{Enger2011} wird die Hypothese aufgestellt, dass der kognitiv-linguistische Faktor \isi{Belebtheit} (in Interaktion mit \isi{Individualität}, \isi{Relevanz} und \isi{Agentivität}) bestimmt, welche Substantive \is{Substantiv} determiniert werden. Es wird davon ausgegangen, dass die \isi{Expansion} bei menschlichen und kommunikativ-relevanten Referenten beginnt und weiter entlang der Belebtheitsskala \is{Belebtheitshierarchie} in Richtung Abstrakta \is{Abstraktum} und \isi{Massennomen} verläuft. Der Zusammenhang von \isi{Belebtheit} und \object{dër}-Gebrauch soll mit einer Korpusuntersuchung \is{Korpuslinguistik} empirisch nachgewiesen werden.  

Die Untersuchung ist im Rahmen der historischen \isi{Konstruktionsgrammatik} verankert \parencite[s. u.a.][]{Traugott2003,Bergs2008,Traugott2013}. Eine zentrale Annahme der \isi{Konstruktionsgrammatik} ist, dass Sprache aus einem stukturierten und über die Zeit veränderbaren Netzwerk \is{Konstruktikon} von konventi\-onalisierten Form-Funktionspaaren, den Konstruktionen, \is{Konstruktion} besteht und sich gebrauchsbasiert wandelt \parencite{Bybee2010,Bybee2013}. Die Entwicklung des Definitartikels \is{Definitartikel} wird dabei als \isi{Konstruktionalisierung} aufgefasst. Vor diesem theoretischen Hintergrund wird auch der Frage nachgegangen, inwiefern form- oder funktionsseitig ähnliche Nominalphrasen \is{Nominalphrase (NP)} den Wandel von [\object{dër}\,+\,N] analogisch \is{Analogie} beeinflussen. Zentral ist dabei die Idee, dass sich \object{dër} als \object{Default}-Marker für \isi{Definitheit} innerhalb eines Determiniererschemas \is{Determiniererschema}  etabliert, welches Sprecherinnen und Sprecher auf Basis ähnlicher adnominaler Definitheitsmarker, etwa  [Possessivartikel\,+\,N] \is{Possessivum}  oder [Genitivattribut\,+\,N] \is{Genitivattribut} abstrahieren \parencite[vgl. für das Englische][]{Sommerer2015}.

\section{Aufbau} 

Der Hauptteil der Arbeit besteht aus vier Theoriekapiteln  (Kapitel~\ref{chapter:theorie}--\ref{chapter:belebtheit}), einem Methodenteil (Kapitel~\ref{methode}) und einem Ergebnisteil, der aus der Ergebnispräsentation (Kapitel~\ref{ergebnisse}) und der theoretischen Diskussion der Ergebnisse (Kapitel~\ref{bicpic}) besteht. Der letzte Teil (Kapitel~\ref{kapitel:zusammenfassung}) fasst die zentralen Ergebnisse der Arbeit zusammen und weist auf offene Fragen hin sowie mögliche Anknüpfungspunkte, die sich aus den Ergebnissen ableiten lassen. 

Im zweiten Kapitel erfolgt eine Auseinandersetzung mit den Grammatikalisierungspfaden \is{Grammatikalisierungspfad} von \textcite{Greenberg1978} und \textcite{Lehmann2015}, die die wichtigsten funktionalen und formalen Schritte des Wandels vom ursprünglichen Demonstrativ- \is{Demonstrativartikel} zum \isi{Definitartikel} abbilden. Anschließend wird die Quelle dieser \isi{Grammatikalisierung}
 genauer betrachtet, da im Gegensatz zu typischen Grammatikalisierungsprozessen \is{Grammatikalisierung} kein lexikalisches, sondern ein grammatisches Element am Anfang der Entwicklung steht. Nachdem im Anschluss die wichtigsten Parameter der Grammatikalisierung zusammengetragen werden, geht die zweite Hälfte des Kapitels auf die konstruktionsgrammatische Perspektive \is{Konstruktionsgrammatik} ein. Es wird dafür argumentiert, dass es sich bei der Entwicklung des Definitartikels \is{Definitartikel} um einen Fall von \isi{Konstruktionalisierung}
 handelt, genauer um die Herausbildung des Schemas \is{Schema} [\isi{Definitartikel}\,+\,N]. Angetrieben wird dieser Wandel durch Type- und Token-Entrenchment \is{Type-Entrenchment}\is{Token-Entrenchment}sowie Analogie- und Reanalysemechanismen.\is{Analogie}\is{Reanalyse} 

Das dritte Kapitel fasst die wichtigsten Erkenntnisse aus der Forschung zur Entwicklung des Definitartikels \is{Definitartikel} im Althochdeutschen zusammen. Nach einem Überblick über die Strategien, die das Althochdeutsche kennt, um \isi{Definitheit} auszudrücken, steht die Frage nach dem Warum im Mittelpunkt. Zunächst werden die klassischen Gründe, die die Forschung für die Herausbildung des Artikels anführt, kritisch beleuchtet. Anschließend wird ein weiteres Entstehungsszenario vorgestellt, das den Faktor Expressivität mit ins Spiel bringt: Es wird davon ausgegangen, dass der Demonstrativartikel \is{Demonstrativartikel} ursprünglich dazu diente, diskurswichtige Referenten zu exponieren. Der inflationäre Einsatz dieser Strategie hat den funktionalen Wandel und die Konventionalisierung von [\object{dër}\,+\,N] angetrieben. Das Kapitel schließt mit einer Übersicht der Faktoren, die den Wandel determinieren, sowie einer Diskussion möglicher Brückenkontexte. \is{Brückenkontext}

Im vierten Kapitel werden Kriterien zusammengetragen, mit denen sich De"-mon"-stra"-tiv- von Definitartikeln funktionsseitig voneinander abgrenzen lassen. Zu Beginn werden die wichtigsten Theorien aus der Definitheitsforschung \is{Definitheit} skizziert. Damit die funktionale Reichweite von Demonstrativartikeln \is{Demonstrativartikel} sichtbar wird, werden im nächsten Teil Gebrauchskontexte erläutert, in denen Demonstrativ\-artikel typischerweise vorkommen. Danach stehen Kontexte im Fokus, die  Definitartikeln vorbehalten sind. 
Im Rahmen der Löbnerschen Definitheitstheorie \parencite{Lobner1985}  werden die für Definitartikel genannten Gebrauchskontexte als sogenannte semantische \is{Semantische Definita} Definitheitskontexte \is{Definitheitskontext} beschrieben -- in Abgrenzung zu \is{Pragmatische Definita} pragmatischen Definitheitskontexten, die zur Domäne der Demon"-stra"-tiv"-artikel \is{Demonstrativartikel} gehören. Aus der bisherigen Forschung lässt sich ableiten, dass das ahd. \object{dër} in beiden Kontexten auftreten kann. Mit der Korpusuntersuchung \is{Korpuslinguistik} soll das Gebrauchsspektrum erstmals systematisch an einer größeren Datenmenge untersucht werden. 

Im fünften Kapitel wird dafür argumentiert, dass die Entwicklung des Definitartikels \is{Definitartikel} eine Form von belebtheitsgesteuertem Wandel darstellt. Zu Beginn wird die kognitiv-linguistische Kategorie Belebtheit, welche sich über unter-\linebreak schiedliche Belebtheitshierarchien abbilden lässt, definiert. Anschließend wird mithilfe bisheriger Erkenntnisse aus der historischen Sprachwissenschaft, der Sprachtypologie sowie der kognitiven Linguistik gezeigt, warum es sinnvoll ist, \isi{Belebtheit} als Einflussfaktor hinzuzuziehen. Das hier theoretisch modellierte Belebtheitsmodell umfasst auch Abstrakta \is{Abstraktum}
und \isi{Massennomen} und bezieht somit den Faktor \isi{Individualität} mit ein. Außerdem werden Korrelationen von \isi{Belebtheit} und \isi{Relevanz} sowie semantischer Rolle \is{Semantische Rolle} aufgezeigt.

Das sechste Kapitel widmet sich der Untersuchungsmethode. Am Anfang wird diskutiert, inwiefern sich das Althochdeutsche mit korpuslinguistischen Methoden \is{Korpuslinguistik} erschließen lässt. Danach wird die Textauswahl begründet. Um eine vollständige Transparenz zu gewährleisten, werden die Schritte der Datenaufbereitung genau erläutert. Eine besondere methodische Herausforderung ist die Anno\-ta\-tion \is{Annotation} von Belebtheitskategorien: \is{Belebtheit} Auf Basis der Übersetzungen aus dem \object{Referenzkorpus Altdeutsch}\is{Korpus} wurden Konzept-Types generiert, welche mit Hilfe von \isi{Annotationsrichtlinien} doppelt annotiert \is{Annotation} und über \object{Inter Annotator Agreements}  evaluiert wurden. Außerdem wurde eine Stichprobe an Nominalphrasen \is{Nominalphrase (NP)} nach Definitheitskontexten, semantischer Rolle \is{Semantische Rolle} sowie morphosyntaktischen Merkmalen \is{Annotation} annotiert. Auch Differenzbelege \is{Differenzbeleg} (Abweichungen von der lat. Vorlage, die bei einigen Texten relevant sind) wurden gekennzeichnet. Der letzte Abschnitt des Kapitels zeigt, wie qualitative und quantitative Analysemethoden bei der Untersuchung kombiniert werden. 
  
Im siebten Kapitel erfolgt die Präsentation der Ergebnisse. Zu Beginn wird ein Überblick gegeben, wie häufig [\object{dër}\,+\,N] in den einzelnen Texten vorkommt. Anschließend wird das funktionale Spektrum der \isi{Konstruktion} beleuchtet. Hierzu wird die Distribution von pragmatischen \is{Pragmatische Definita} und semantischen Definitheitskontexten, \is{Semantische Definita} in denen \object{dër} erscheint, präsentiert. Anschließend werden die Häufigkeiten von Superlativkonstruktionen \is{Superlativ} und ausgewählten Unika \is{Unikum} mit und ohne Determinierung gegenüberstellt; beide Kontexte repräsentieren semantische Definitheit. \is{Semantische Definita} Der dritte Abschnitt des Kapitels enthält die Ergebnisse zu den kognitiven Faktoren \isi{Belebtheit}, \isi{Individualität}, \isi{Relevanz} und semantische Rollen.\is{Semantische Rolle} Im Anschluss wird die NP-Perspektive \is{Nominalphrase (NP)} eingenommen und gezeigt, wie salient \object{dër} als Phraseneinleiter \is{Phrase} in den einzelnen Texten ist. Auch das \is{Wortstellung} Stellungsverhalten \is{Nominalsyntax} einzelner \isi{Determinierer} und attributiver Adjektive \is{Adjektiv} sowie Interaktionen von \object{dër} mit schwach flektierten Adjektiven \is{Flexion} wird hier offengelegt. 

Das achte Kapitel stellt die Ergebnisse in einen größeren theoretischen Zusammenhang: Wann vollzieht \object{dër} den funktionalen Wandel zum Definitartikel und welche Brückenkontexte \is{Brückenkontext} dienen als Sprungbrett? Die empirischen Erkenntnisse werden genutzt, um den Entwicklungspfad des Definitartikels \is{Definitartikel} neu zu modellieren. Zudem wird ein Expansionspfad \is{Expansion} entlang der \isi{Belebtheitshierarchie} vorgeschlagen. Die Ergebnisse zur Struktur der Nominalphrase \is{Nominalphrase (NP)}  stützen die Annahme, dass die \isi{Konstruktionalisierung}  von [\object{dër} + N] als Teil einer übergeordneten \isi{Schematisierung}, nämlich im Rahmen der Herausbildung eines Determiniererschemas,\is{Determiniererschema} abgelaufen ist. Ferner wird auf Basis der Ergebnisse gezeigt, wie sowohl Type- als \is{Type-Entrenchment} auch Token-Entrenchment \is{Token-Entrenchment} den grammatischen Wandel beeinflussen.

Das neunte Kapitel fasst die zentralen Ergebnisse zusammen und formuliert Anknüpfungspunkte für zukünftige Studien. 
