\documentclass[output=paper,
modfonts,nonflat
]{langsci/langscibook} 
\author{Gustavo Godoy\affiliation{PPGAS-Museu Nacional, UFRJ}%
\lastand Wyrapitã Ka'apor%
}%
\title{Ka'apor}
\lehead{G.\ Godoy and Wyrapitã Ka'apor}
\ourchaptersubtitle{A’i ymanhar ke kurumĩ rehe pynu ixo.}
\ourchaptersubtitletrans{‘A long time ago, an old woman farted on a boy’}  
% \abstract{noabstract}
\ChapterDOI{10.5281/zenodo.1008797}

\maketitle

\begin{document}

\section{Introduction} 


	This narrative concerns an old woman who farted on a boy, making him ill. The title, a synthesis of the narrative, was proposed by Karairan (Raimundo Tembé) on the 4th of August, 2016:
    
\ea a'i ymanihar ke kurumĩ pynu ixo \\[.3em]
\gll aʔi		ɪman-haɾ-kɛ	kuɾumĩ	ɾɛhɛ	∅-pɪnu	i-ʃɔ \\
old.woman	formerly\textsc{-nmlz}\textsc{-afc}	boy	at	3-fart	3-be \\
\glt ‘A long time ago, a woman farted on a boy.’ \\
\z
\ea aja me'ẽ ke kurumĩ u'y sepetu ixapekwar rupi kutuk \\[.3em]
\gll aja-ɛ̃-kɛ	kuɾumĩ		uʔɪ-sɛpɛtu	i-ʃapɛ	  kʷaɾ	ɾupi	∅-kutuk \\
\textsc{ana}\textsc{-nmlz}\textsc{-afc}	boy		arrow-spit	3-vagina	  hole	by	3-pierce \\
\glt ‘Afterwards, the boy pierced her through the anus with a wood-tipped arrow.’ \\
\z

	The story was narrated by Wyrapitã Ka’apor, who is also known as Jamói, and recorded on the 15th of June 2014 in the village of \emph{Xie pihun rena} (Centro Novo do Maranhão County, in the state of Maranhão). The transcription and analysis of the text were carried out by the storyteller and Gustavo Godoy, while he was conducting field research for his Master’s degree in Anthropology \citep{Godoy2015}.



\section{The Ka’apor and their languages}
	The Ka’apor are an eastern Amazonian people who live in the western part of the state of Maranhão. In the 19th century, they lived in the state of Pará and, before 1800, further west in the Tocantins River basin \citep[30--32]{Balée1994}. In 1911, the Brazilian government began the process of pacification of the Ka’apor, who were known for attacking local colonists \citep{Ribeiro1962}; the Ka’apor population in 1928, when so-called pacification was concluded, has been estimated at 5,000 people. The current population in the Upper Turiaçu Indigenous Land is 2,300. Some Ka’apor also live among the Tembé (\textsc{tupian}), in the Alto Guamá Indigenous Land.
    
\begin{figure}[h!]
  \caption{The Upper Turiaçu Indigenous Land and villages mentioned}
  \centering
\includegraphics[width=\textwidth]{figures/turiacu.pdf}
\end{figure}
    
	The Ka’apor language is classified as belonging to the eastern branch of the Tupian family; it is part of the Tupi-Guarani set of the Maweti-Guarani sub-branch. The Ka’apor lexicon \citep{Corrêa-da-Silva1997,Balée2006} shows colonial influences through the Língua Geral Amazônica (\textsc{tupian}). Its argument marking is similar to Nheengatu \citep{Corrêa-da-Silva2002}, which descends from Língua Geral.
    
    \newpage 
	Spoken Ka’apor is a vital language, actively transmitted between generations. Nonetheless, there has been a decline in the transmission of traditional genres, such as formulaic greetings and some types of songs. The Ka’apor have additionally developed a local sign language for communicating with the deaf, which has been in use since at least the 1940s \citep{KakumasuKakumasu1988,Ferreirand,Ferreira1984,Ferreira2010}.


\section{Notes on the narrative}

	The theme of the narrative presented here is also found among other indigenous peoples. The Bororo (\textsc{macro-jê}) version tells of the origin of disease and begins with a grandmother farting on her grandson’s face after he refused to submit to male initiation. The Kĩsêdjê (\textsc{jê}) version deploys a flatulent mother-in-law \citep[1--2]{Nonato2016}. The Kuikuro (\textsc{upper xinguan cariban}) version is very similar to that of the Kĩsêdjê (Franchetto, recordings and field notes from 1981).
    
	While the Ka’apor and Bororo versions should be considered “serious" narratives, for the Kĩsêdjê and the Kuikuro, the theme of the flatulent mother-in-law is a feature of short and funny (“ugly") narratives. It is interesting to note the structural correspondences of the relations between the protagonists: a grandmother and grandson for the Bororo; an   older woman and young boy for the Ka’apor; a mother-in-law and son-in-law in the Upper Xingu.
    
	In the Ka’apor narrative, the eschatological event is intimately tied to the description of the behaviour of ancient killers, who were submitted to post-ho\-mi\-cide seclusion. When the seclusion was over, the killers had to come out during the beer drinking ceremony (a drink made from the fermentation of cashews or manioc flat bread).
    
	The plot involves four characters: a young boy (the protagonist), an elder woman (the antagonist), the youth’s brother (co-protagonist) and the war chief (auxiliary). The boy grows increasingly pale (\emph{tawa} ‘yellow’), as the days go by, because of the intestinal gasses (\emph {pɪnu}) that an old woman discharged on him. His brother grows suspicious and makes a small wooden-tipped arrow (\emph {uʔɪ sepetu}) for the sick boy to kill the old woman. Feigning sleep, the boy stabs the arrow into the old woman’s anus as she tries to fart on him once again. The malevolent woman soon falls ill and dies. The young boy confesses the homicide to the war chief (\emph {tuʃa}), who advises him to go into post-homicide seclusion, lying down in his room (\emph {kapɪ}). Finally, the young killer comes out during the beer drinking ceremony.

\newpage 	
	The narrative is, in fact, a reminder of this now defunct part of the beer ritual: the coming out of the killer. Furthermore, one of the four characters, the war chief, is a figure of the past.
    
	The beer-drinking ceremony brings to an end the restrictions that fall on those who find themselves in states of susceptibility, such as the killer; its central moment is the naming ritual. In the narrative, the war chief tells the boy that he should come out after the “lifting of the children” (line 34) when the sponsors present the name of a baby. The beverage ritual is here called \emph{akaju ɾɪkwɛɾ ŋã ʔu}, “the cashew beer drinking moment” (line 37).
\begin{figure}[h!]
  \caption{The storyteller Jamói, at the moment when the men ask about the name of his goddaughter (\emph{tajɪɾ-aŋa}). Her name, \emph{Nanã akɪɾ} ‘unripe pineapple', is then announced.}
  \centering
\includegraphics[width=\textwidth]{figures/2017-03-27__1_.png}
\end{figure}
    
	The killer would appear in the beer drinking ceremony carrying a pack of arrows. He remained at the place of the ceremony, looking east, immobile and impassive. An old man would go towards him with words of reprobation and revenge: “Hu-Hu-Hu! You attacked your comrade! Now it is you who will be (attacked)! Now I will draw your blood!”
    
\largerpage[2]    
	The old man carried a tooth (from a trahira - a fish with a big mouth and sharp teeth - or from a Brazilian squirrel) which he used to scar the killer's legs (line 42), so as to expel the morbid blood in his body (line 43). This blood contained part of the dead person. Failure to extract it would cause the killer to go mad, affected by the murdered enemy. After scarification, the killer’s vulnerability would come to an end, and he could then leave his room and wander the outside world without any danger (line 45) \emph{pɛ sɔɾɔka ɾɛhɛ wata atu i-ʃɔ tĩ}).


\section{Notes on transcription and annotation} 
	The first line of the transcription is orthographic. The Ka’apor system of alphabetic writing is based on the phonemic analysis carried out by \citet{Kakumasu1964,KakumasuKakumasu1988}. Ka’apor has 15 phonemic consonants /p, t, k, kʷ, m, n, ŋ, ŋʷ, s, ʃ, j, ɾ, w, h, ʔ/; six phonemic oral vowels /i, ɪ, ɛ, a, ɔ, u/ and five nasal vowels /ĩ, ɛ̃, ã, ɔ̃, ũ/. In the orthography, the graphemes <ng, ’, x, y, e, o> represent IPA /ŋ, ʔ, ʃ, ɪ, ɛ, ɔ/. The <ái, úi> sequence represent /aj, uj/. 
    
   
\setcounter{equation}{0}
\section{A’i ymanhar ke kurumĩ rehe pynu ixo.}
\translatedtitle{‘A long time ago, an old woman farted on a boy’}\\
\translatedtitle{‘Muito tempo atrás, uma velha pedou em um menino’}\footnote{Recordings of this story are available from \url{https://zenodo.org/record/997433}}

\ea ta'yn uker ou 'y pytun rahã pame ame'ẽ a'i ai pynu oho ehe je \\[.3em]
\gll taʔɪn u-kʷɛɾ ɔ\footnotemark{}-u ʔɪ pɪtun ɾahã pamɛ amɛʔɛ̃ aʔi ai pɪnu\footnotemark{} ɔ-hɔ ɛhɛ jɛ\footnotemark{} \\
child 3-sleep 3-lay.down \textsc{pfv} night \textsc{sr} each \textsc{dei} old.woman roguish fart 3-go 3.at \textsc{hsy} \\
\glt ‘Night after night, while a boy was lying (down), it is said that an old roguish woman came to fart on him.’ \\
‘Dizem que, noite após noite, quando um menino estava deitado, uma velha escrota peidava em cima dele.’ \\
\fnminus
\fnminus
\footnotetext{There is no number distinction in the third person prefix, glossed simply as ‘3’. The allomorph of this prefix in active verbs is \emph{u-}, when the verbal root is monosyllabic and does not contain the vowel /ɔ/. If the monosyllabic root has the vowel /ɔ/, the allomorph \emph{ɔ-} is used. The third person person prefix \emph{ɔ-}  with the auxiliary verb \emph{-u} ‘to lay down' is a exception to this rule.}
\fnplus
\footnotetext{When the verbal root has more than one syllable, no third person prefix is attached to it. In other words, the person mark is a zero allomorph (∅).}
\fnplus
\footnotetext{Ka'apor does not have a system with several morphemes indexing different sources of the information, as in the grammaticalized evidentials used in Kotiria realis statements (cf.  chapter 5 in this volume) or  the more complex evindentiality  systems of other Tupian languages, such as Kamauirá, Tapirapé, and Karo. The reported evidential \emph{jɛ} ‘\textsc{hsy}' (hearsay or reportative modality) is the only evidential morpheme in Ka'apor. \emph{jɛ} does not imply disbelief on the part of the speaker in relation to the content of the mythical narrative, and it occurs in almost all sentences. This contrasts with the sparse occurrence of the quotative reported evidential \emph{-yu'ka} and its pragmatic use in the Kotiria narrative (cf. \fnref{fn:kotiria:9} of the Kotiria narrative, Chapter 5). Indeed, in Ka'apor mythological narratives, the morpheme \emph{je} codes a diffuse source of information, more like the ‘diffuse' evidential \emph{-yu'ti} in Kotiria (cf. \citealt{Stenzel2008}).}
\z

\newpage 
\ea pe pytun je tĩ pe pynu oho je tĩ \\[.3em]
\gll pɛ pɪtun jɛ tĩ pɛ pɪnu ɔ-hɔ jɛ tĩ \\
then night \textsc{hsy} again then fart 3-ho \textsc{hsy} again \\
\glt ‘Each night that came, she farted on him again.’ \\
‘Outra noite chegava e ela peidava de novo.’ \\
\z

\ea ta'yn ukwer ta je tĩ, ko a'i ihái ke piɾok hũ je pe i'aɾ pe pynu hũ \\[.3em]
\gll taʔɪn u-kʷɛɾ-ta jɛ tĩ kɔ aʔi i-haj-kɛ piɾɔk-hũ jɛ pɛ i-ʔaɾ-pɛ pɪnu -hũ \\
child 3-sleep\textsc{-fut} \textsc{hsy} again \textsc{dei} old.woman 3-skirt\textsc{-afc} strip\textsc{-intens} \textsc{hsy} then 3-above\textsc{-loc} fart \textsc{-intens} \\
\glt ‘When the boy went to sleep again, the old woman took off her skirt, it is said; then (she) farted on him a lot.’ \\
‘O menino ia dormir de novo e a tal velha tirava a sua saia; então, peidava um monte nele.’ \\
\z

\ea pynupynu ate ehe je \\[.3em]
\gll pɪnu$\sim$pɪnu-atɛ ɛhɛ jɛ \\
fart$\sim$\textsc{red}-\textsc{intens} 3.at \textsc{hsy} \\
\glt ‘She was really farting a lot on him.’ \\
‘Ficava peidando muito mesmo nele.’ \\
\z

\ea pe oho je tĩ \\[.3em]
\gll pɛ ɔ-hɔ jɛ tĩ \\
then 3-go \textsc{hsy} again \\
\glt ‘She went again (towards the boy).’ \\
‘E ela foi de novo (até o menino).’ \\
\z

\ea ta'yn ke itawa imu parahy ahy ipe je \\[.3em]
\gll taʔɪn-kɛ i-tawa i-mu paɾahɪ-ahɪ i-pɛ jɛ \\
child\textsc{-afc} 3-yellow 3-bɾother angry-\textsc{intens} \textsc{3-dat} \textsc{hsy} \\
\glt ‘The boy was yellowish (sick), his brother was very angry with him.’ \\
‘O menino estava amarelado, seu irmão ficou bravo.’ \\
\z

\largerpage
\ea “ne tawa te ne ke ã ne jyty'ym te amõ 'y” \\[.3em]
\gll nɛ-tawa-tɛ nɛ-kɛã nɛ-jɪtɪʔɪm-tɛ amɔ̃ ʔɪ \\
\textsc{2sg}-yellow-\textsc{intens} \textsc{2sg-afc} \textsc{2sg}-lazy\textsc{-intens} another \textsc{pfv} \\
\glt ‘“You are yellowish and very lazy too.”’ \\
‘“Você está amarelado e está preguiçoso também.”’ \\
\z

\ea pandu 'ym anu ta'yn je tĩ  \\[.3em]
\gll pandu-ʔɪm$\sim$anu taʔɪn jɛ tĩ \\
tell-\textsc{neg$\sim$red} child \textsc{hsy} again \\
\glt ‘The boy didn’t say anything.’ \\
‘O menino não contava.’ \\
\z

\ea pe pytun je tĩ, pe a'i [...] pytun pyter pe je \\[.3em]
\gll pɛ pɪtun jɛ tĩ pɛ aʔi [...] pɪtun pɪtɛɾ-pɛ jɛ \\
so night \textsc{hsy} again so old.woman [...] night middle-\textsc{loc} \textsc{hsy} \\
\glt ‘Another night, the old ... [hesitation] in the middle of the night.’ \\
‘Então, de noite, novamente, então, a velha ... [hesitação] era no meio da noite.’ \\
\z

 
\ea pe a'i ihon ixo je 'y \\[.3em]
\gll pɛ aʔi i-hɔn i-ʃɔ jɛ ʔɪ \\
then old.woman 3-go \textsc{3-aux} \textsc{hsy} \textsc{pfv} \\
\glt ‘Then the old woman went (towards the boy).’ \\
‘Então a velha foi (até o menino).’ \\
\z

\ea pe ihái ke musyryk je, pe ta'yn ukwer atu je Pũũũ! japũi rehe pynu je \\[.3em]
\gll pɛ i-haj-kɛ mu-sɪɾɪk jɛ pɛ taʔɪn u-kʷɛɾ-atu jɛ pũũũ i-apũi ɾɛhɛ pɪnu jɛ \\
then 3-skirt\textsc{-afc} \textsc{caus-}strip \textsc{hsy} then child 3-sleep-\textsc{intens} \textsc{hsy} \textsc{ideo} 3-nose \textsc{loc} fart \textsc{hsy} \\
\glt ‘So, (she) she raised her own skirt, the child was sleeping deeply: \textit{Puum}! She farted in his nose.’ \\
‘Ela levantou a saia, o menino estava dormindo bem: \textit{Puum}! Ela peidou no nariz dele.’ \\
\z

\ea pe ... pe wera uwyr tĩ, pe imu panu ipe “myja ne xoha tĩ” \\[.3em]
\gll pɛ pɛ wɛɾa uwɪɾ tĩ pɛ i-mu panu i-pɛ mɪja nɛ-ʃɔ-ha tĩ \\
then then light come again then 3-brother tell 3-\textsc{dat} \textsc{q} \textsc{2sg}-be\textsc{-nmlz} again \\
\glt ‘So ... so the light (of the morning) came again, and his brother said: “What’s wrong with you?”’ \\
‘Amanheceu de novo e o irmão perguntou: “O que está errado com você?”’ \\
\z

\ea “epandu ihẽ pe rahã", pe imu pandu, “xe amõ a'i ihẽ rehe pynu ixo tĩ” “pytun rahã pame ihẽ rehe pynu” je; “a'erehe ihẽ ke atawa tái” \\[.3em]
\gll ɛ-pandu ihɛ̃-pɛ ɾahã pɛ i-mu panu ʃɛ amɔ̃ aʔi ihɛ̃-ɾɛhɛ pɪnu i-ʃɔ tĩ pɪtun ɾahã pamɛ ihɛ̃-ɾɛhɛ pɪnu jɛ aʔɛ-ɾɛhɛ ihɛ̃-kɛ a-tawa taj \\
\textsc{2sg.imp}-tell \textsc{1sg-dat} \textsc{hort} so 3-brother tell \textsc{dei} another old.woman \textsc{1sg-loc} fart 3-be again night \textsc{sr} each \textsc{1sg-loc} fart \textsc{hsy} 3-about \textsc{1sg-afc} \textsc{1sg}-yellow \textsc{intens} \\
\glt ‘“Talk to me!” and his brother said: “The old woman is farting on me; every night she farts on me! That's why I'm turning yellow.”’ \\
‘“Conte para mim!” e o irmão respondeu: “A velha está peidando em mim; todas as noites! Por isso estou amarelando.”’ \\
\z

\ea pe imu ... u'y ra'yr mujã ipe je, u'y sepetu, yrapar ra'yr \\[.3em]
\gll pɛ i-mu uʔɪ ɾaʔɪr mujã i-pɛ jɛ uʔɪ sɛpɛtu ɪɾapaɾ ɾaʔɪɾ \\
so 3-brother arrow small make 3-\textsc{dat} \textsc{hsy} arrow spit bow small \\
\glt ‘Then, his brother ... made a small arrow with a tip of wood and a small bow  for him.’ \\
‘Então, o irmão ... fez uma flechinha para ele com ponta de madeira e um pequeno arco.’ \\
\z

\ea pe ko[me'ẽ] ... pe “ejingo rahã kỹ" aja ipe je \\[.3em]
\gll pɛ kɔ[mɛʔɛ̃] pɛ ɛ-jingɔ ɾahã kɪ̃ aja i-pɛ jɛ \\
so th(is) so \textsc{2sg.imp}-shoot \textsc{hort} \textsc{kɪ̃}\footnotemark{} \textsc{ana} 3-\textsc{dat} \textsc{hsy} \\
\glt ‘So this ... so “Shoot!” he said to him.’ \\
‘Então ... “Flecha!” assim disse (o irmão) para ele.’ \\ 
\footnotetext{\citet{KakumasuKakumasu1988} translate the morpheme \emph{kɪ̃} as ‘definitive intention’.} 
\z

\ea pe ta'yn ukwer uwyr je tĩ, pytun rahã  \\[.3em]
\gll pɛ taʔɪn u-kʷɛɾ uwɪɾ jɛ tĩ pɪtun ɾahã \\
so child 3-sleep come \textsc{hsy} again noite \textsc{sr} \\
\glt ‘The boy went to sleep again, at night.’ \\
‘O menino foi dormir de novo, de noite.’ \\
\z

\largerpage[2]
\ea pe a'i ai tur je tĩ \\[.3em]
\gll pɛ aʔi ai tuɾ jɛ tĩ \\
so old.woman rogue come \textsc{hsy} again \\
\glt ‘The old woman came again.’ \\
‘A velha chegou novamente.’ \\
\z
\newpage 

\ea pe sa'e a'i ihái, xirur ke pirok hũ je ihai ai ke  \\[.3em]
\gll pɛ saʔɛ aʔi i-haj ʃiɾuɾ-kɛ piɾɔk-hũ jɛ i-haj ai jɛ  \\
so guy old.woman 3-skirt short-\textsc{afc} strip-\textsc{intens} \textsc{hsy} 3-skirt bad \textsc{hsy} \\
\glt ‘So he ... the old woman raised her own old skirt, her pants.’ \\
‘Aí ele ... a velha levantou sua saia, sua calça, levantou alto, a sua saia surrada.’ \\
\z

\ea xape ai jumupirar te’e xoty je \\[.3em]
\gll i-ʃapɛ ai ju-mu-piɾaɾ tɛʔɛ i-ʃɔtɪ jɛ \\
3-anus bad \textsc{refl}-\textsc{caus}-open free 3-towards \textsc{hsy} \\
\glt ‘Her disgusting asshole opened towards the boy.’ \\
‘Seu cu nojento abriu muito na direção do menino.’ \\
\z

\ea pe sa'e u'y ke hykýi je Sõõ xape kwar rupi ate jingo mondo je a'i ai ahem ate oho je  \\[.3em]
\gll pɛ saʔɛ uʔɪ-kɛ hɪkɪj jɛ Sõõ ʃapɛ kʷaɾ ɾupi atɛ jingɔ mɔnɔ jɛ aʔi ai ahɛm atɛ ɔ-hɔ jɛ  \\
so guy arrow-\textsc{afc} pull \textsc{hsy} \textsc{ideo} anus hole in \textsc{intens} shoot throw \textsc{hsy} old.woman roguish scream \textsc{intens} 3-go \textsc{hsy} \\
\glt ‘So, he took the arrow and \textit{Sõõ!} he shot it into her asshole, the arrow was stuck in her asshole. The nasty old woman screamed a lot.’ \\
‘Então, ele armou a flecha \textit{Sõõ!} flechou bem no buraco de seu cu e a (velha) ficou com a flecha encravada. A velha foi gritando muito.’ \\
\z

\ea pe a'i ai pynu ‘ym je ‘y amõ ku'em rahã sa’e … amõ wera uwyr je ‘y \\
\gll pɛ aʔɪ ai pɪnu ʔɪm jɛ ʔɪ amɔ̃ kuʔɛm ɾahã saʔɛ amɔ̃ wɛɾa uwɪɾ jɛ ʔɪ \\
so old.woman roguish fart \textsc{neg} \textsc{hsy} \textsc{pfv} another morning \textsc{sr} guy another light come \textsc{hsy} \textsc{pfv} \\
\glt ‘Then, the old woman didn’t fart anymore; the next morning ... Another day came.’ \\
‘Então a velha não peidou mais, na outra manhã ... Chegou outro dia.’ \\
\z

\ea pe pytun oho tĩ; a'i ju...ju... juwyr ‘ym oho pytun rahã \\[.3em]
\gll pɛ pɪtun ɔhɔ tĩ aʔi ju...ju... juwɪɾ ʔɪm ɔhɔ pɪtun ɾahã \\
so night 3-go again old.woman [hesitation] return \textsc{neg} 3-go night \textsc{sr} \\
\glt ‘So, night came again; [hesitation] the old woman didn’t return.’ \\
‘Veio a noite novamente; [hesitação] a velha não voltou, quando foi a noite novamente.’ \\
\z

\ea pe ... pe atu u'y ipi'a kwar [rupi]\footnotemark{} upen u'am 'y \\[.3em]
\gll pɛ pɛ atu uʔɪ i-piʔa kʷaɾ [ɾupi] upɛn uʔam ʔɪ \\
so so \textsc{intens} arrow 3-vagina hole [by] broke \textsc{aux}.vertical \textsc{pfv} \\
\glt ‘So ... So, the arrow broke in her vagina.’ \\
‘Então ... Então, a flecha quebrou no buraco da vagina dela.’ \\
\footnotetext{Not performed in speech, but indicated in the analysis.}
\z

\ea amõ ukwer rahã a'i ame'ẽ a'i ke ma'e ahy je 'y \\[.3em]
\gll amõ u-kʷɛɾ ɾahã aʔi amɛʔɛ̃ aʔi-kɛ maʔɛ-ahɪ jɛ ʔɪ \\
another 3-sleep \textsc{sr} old.woman \textsc{ana} old.woman-\textsc{afc} sickness \textsc{hsy} \textsc{pfv} \\
\glt ‘The next day, that old woman got sick.’ \\
‘No outro dia, a velha, aquela velha, adoeceu.’ \\
\z

\ea pe sawa'e, ame'ẽ ta'ynuhu ukwa je  \\[.3em]
\gll pɛ sawaʔɛ amɛʔɛ̃ taʔɪn-uhu u-kʷa je \\ so man \textsc{ana} child-\textsc{intens} 3-know \textsc{hsy} \\
\glt ‘That young boy already knew.’ \\
‘Aquele homem, aquele menino sabia.’ \\
\z

\ea “pe'ẽ a'i ihẽ rehe pynu ixo riki ã, ame'ẽ pytun rahã”  \\[.3em]
\gll pɛʔɛ̃ aʔi ihɛ̃ ɾɛhɛ pɪnu i-ʃɔ ɾikiã amɛʔɛ̃ pɪtun ɾahã \\
\textsc{dei} old.woman \textsc{1sg} in fart 3-be emphasis \textsc{ana} night \textsc{sr} \\
\glt ‘“That old woman was farting on me that night.”’ \\
‘“Aquela velha estava peidando em mim aquela noite.”’ \\
\z

\ea sa'e ke pe túiha ke \\[.3em]
\gll saʔɛ-kɛ pɛ tuj-ha-kɛ \\
guy-\textsc{afc} \textsc{dei} stay-\textsc{nmlz-afc} \\
\glt ‘There was the boy.’ \\
‘Ali estava o menino.’ \\
\z

\newpage 
\ea yman te pe a'i ke manõ je ‘y \\[.3em]
\gll ɪman-tɛ pɛ aʔi-kɛ manɔ̃ jɛ \\
lately-\textsc{intens} so old.woman-\textsc{afc} die \textsc{hsy} \\
\glt ‘Later, the old woman died.’ \\
‘Passou um tempo, a velha morreu.’ \\
\z

\ea pe ame'ẽ ta'yn tuxa pe pandu je \\[.3em]
\gll pɛ amɛʔɛ̃ taʔɪn tuʃa-pɛ panu jɛ \\
so \textsc{ana} child chief-\textsc{dat} tell \textsc{hsy} \\
\glt ‘Then, the child talked to the chief:’ \\
‘Então aquele menino falou para o tuxaua:’ \\
\z

\ea “ihẽ ame'ẽ a'i ke ajukwa” \\[.3em]
\gll ihɛ̃ amɛʔɛ̃ aʔi-kɛ a-jukʷa \\
\textsc{1sg} \textsc{ana} old.woman-\textsc{afc} \textsc{1sg}-kill \\
\glt ‘“I killed that old woman.”’ \\
‘“Eu matei aquela velha.”’ \\
\z

\ea pe tuxa aja ipe je \\[.3em]
\gll pɛ tuʃa aja i-pɛ jɛ \\
so chief \textsc{dei} 3-\textsc{dat} \textsc{hsy} \\
\glt ‘So, the warrior chief said to him:’ \\
‘E o tuxaua assim falou para ele:’ \\
\z

\ea “kawĩ, ta'ynta jahupir rahã, epandu kỹ hetaha pe” \\[.3em]
\gll kawĩ taʔɪn-ta ja-hupiɾ ɾahã ɛ-panu kɪ̃ hɛta-ha-pɛ \\
beer child-\textsc{pl} \textsc{1pl}-lift \textsc{sr} \textsc{2sg.imp}-tell \textsc{kɪ̃} many-\textsc{nmlz}-\textsc{dat} \\
\glt ‘“Beer, when we will lift the children, talk to the people.”’ \\
‘“Cauim, quando levantamos as crianças, fale para o grupo.”’ \\
\z

\ea “apo kapy pe te'e eju rĩ” \\[.3em]
\gll apɔ kapɪ-pɛ tɛʔɛ ɛ-ju ɾĩ \\
now room-\textsc{loc} \textsc{tɛʔɛ} \textsc{2sg.imp-}lay.down \textsc{ipfv} \\
\glt ‘“Now, go off to your room.’” \\
‘“Agora, vá deitar no teu quarto.’” \\
\z

\newpage 
\ea pe sa'e kapy pe túi je \\[.3em]
\gll pɛ saʔɛ kapɪ-pɛ tuj jɛ \\
so guy room-\textsc{loc} stay \textsc{hsy} \\
\glt ‘He stayed in the room (in seclusion).’ \\
‘Ele ficou no quarto (em reclusão).’ \\
\z

\ea pe ... akaju rykwer ngã'u je 'y, pe sa'e uhem je 'y \\[.3em]
\gll pɛ akaju ɾɪkʷɛɾ ŋã\footnotemark{}-ʔu jɛ ʔɪ pɛ saʔɛ u-hɛm jɛ ʔɪ \\
so cashew juice  \textsc{3pl}-ingest \textsc{hsy} \textsc{pfv} so guy 3-exit  \textsc{hsy} \textsc{pfv} \\
\glt ‘Then ... Later, after they	drank cashew beer, he came out.’ \\
‘Então ... Depois deles beberem o cauim do caju, ele saiu.’ \\
\footnotetext{Although there is no distinction between plural and singular inthe third person prefix, Ka'apor has the free pronoun \emph{ŋã} to index third person plural.}
\z

\ea kapy ngi uhem \\[.3em]
\gll kapɪ ĩ u-hɛm \\
room \textsc{abl} 3-exit \\
\glt ‘He left the seclusion room.’ \\
‘Ele saiu do quarto de reclusão.’ \\
\z

\ea kujã ta ke upa mupinim \\[.3em]
\gll kujã-ta-kɛ upa mu-pinim \\
mulher-\textsc{pl}-\textsc{afc} all \textsc{caus}-painted \\
\glt ‘All the women were painted.’ \\
‘Todas as mulheres foram pintadas.’ \\
\z

\ea aja rahã sa'e a'e uhem ta'ynuhu, ta'ynuhu je \\[.3em]
\gll aja ɾahã saʔɛ aʔɛ u-hɛm taʔɪn-uhu taʔɪnuhu jɛ \\
\textsc{ana} \textsc{sr} guy 3 3-exit child-\textsc{intens} child-\textsc{intens} \textsc{hsy} \\
\glt ‘At this moment, the young man left the room.’ \\
‘Foi neste o momento que o jovem saiu da reclusão.’ \\
\z

\ea ma'e… huwy ke \\[.3em]
\gll maʔɛ huwɪ-kɛ \\
hesitation blood-\textsc{afc} \\
\glt ‘And ... The blood ... ’ \\
‘Ee ... Sangue ...’\\
\z

\ea upa tymã mu’i huwy ke upa muhãi je \\[.3em]
\gll upa tɪmã muʔi huwɪ-kɛ upa muhãj jɛ \\
all leg scarify blood-\textsc{afc} all disperse \textsc{hsy} \\
\glt ‘The lower part of his legs was scarified; the (bad) blood was totally dispersed.’ \\
‘A parte de baixo de suas pernas foram escarificada; todo o sangue (ruim) foi retirado.’ \\
\z

\ea huwy ahyha ke, a’i ruwy ke ame'ẽja saka je \\[.3em]
\gll huwɪ ahɪ–ha-kɛ aʔi ɾuwɪ-kɛ amɛʔɛ̃-ja saka jɛ \\
blood pain-\textsc{nmlz}-\textsc{afc} old.woman blood-\textsc{afc} \textsc{dei-ana} like \textsc{hsy} \\
\glt ‘The evil and toxic blood, it was as if it were the blood of the old woman.’ \\
‘O sangue insalubre e mórbido, era como se fosse o sangue da velha.’ \\
\z

\ea pe ... aja rahã katu je tĩ \\[.3em]
\gll pɛ aja ɾahã katu jɛ tĩ \\
so \textsc{ana} \textsc{sr} good \textsc{hsy} again \\
\glt ‘So ... In this way he became well again.’ \\
‘Então ... deste jeito o jovem ficou bom.’ \\
\z

\ea pe soroka rehe wata atu ixo tĩ \\[.3em]
\gll pɛ sɔɾɔka ɾɛhɛ wata atu i-ʃɔ tĩ \\
so outside by walk good 3-be again \\
\glt ‘And he was able to walk outside again.’ \\
‘E ele pode andar de novo tranquilamente do lado de fora.’ \\
\z

\section*{Acknowledgments}
I thank Jamói for his teachings; he is one of my main  consultants among the Ka'apor. I am also grateful to Bruna Franchetto and Kristine Stenzel for their revisions and suggestions,  Luiz Costa for a first English translation of the Introduction, and Manuella Godoy, who helped me with the final details of the map. My work on Ka’apor as an MA student was supported by a grant to the Graduate Program in Social Anthropology  (UFRJ) from the of Program for Continuing Academic Development (CAPES, Brazilian Ministry of Education).

\section*{Non-standard abbreviations}

\begin{tabularx}{.42\textwidth}{lQ}
\textsc{afc} & affected \\
\textsc{ana} & anaphora or cataphora \\
\textsc{dei} & deixis \\
\textsc{hort} & hortative \\
\end{tabularx}
\begin{tabularx}{.55\textwidth}{lQ}
\textsc{hsy} & hearsay, reported evidential \\
\textsc{ideo} & ideophone \\
\textsc{intens} & intensifier \\
\textsc{sr} & subordinator \\
\end{tabularx}
 
{\sloppy
\largerpage
\printbibliography[heading=subbibliography,notkeyword=this]
}
\end{document}
