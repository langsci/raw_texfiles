\documentclass[output=paper,
modfonts,nonflat
]{langsci/langscibook} 
\author{Antonio Guerreiro\affiliation{University of Campinas, Brazil}%
\and Ageu Kalapalo%
\and Jeika Kalapalo%
\lastand Ugise Kalapalo%
}%
\title{Kalapalo}
\lehead{A.\ Guerreiro, Ageu Kalapalo, Jeika Kalapalo \& Ugise Kalapalo}
\ourchaptersubtitle{Kamagisa etsutühügü}
\ourchaptersubtitletrans{‘Kamagisa sang for the first time’}  
% \abstract{noabstract}
\ChapterDOI{10.5281/zenodo.1008777}

\maketitle

\begin{document}

\section{Introduction} 

Kalapalo is a dialectal variaton of the Upper Xingu Carib Language. The narrative presented in this chapter is around 12 minutes long and was recorded in 2010, with seventy-year-old \textit{Ageu Kalapalo}. He tells us how a man named \textit{Kamagisa}\footnote{\textit{Ageu} calls him \textit{Kamagisa}, but most people insist the character's correct name would be \textit{Kumagisa}. We have decided to keep \textit{Ageu}'s original pronunciation.} married a Snake Woman and learned from his father-in-law a suite of songs of the Xinguan mortuary ritual (\textit{egitsü}, broadly known as “Quarup"). When \textit{Kamagisa} decided to move permanently to his wife’s village, he performed a ritual for himself and taught the songs to another human singer. The events take place in \textit{Hagagikugu}, an important historical site for the Kalapalo and Nahukua peoples. \textit{Ageu} also explains how these same events are reflected in verses sung in Kamayurá, a Tupi-Guarani language (Tupian), an example of the inter-relatedness of history, narratives and music in the Xinguan multiethnic and multilingual network.

	The Kalapalo are a Carib-speaking people who live in the southern region of the Xingu Indigenous Land, in northern Mato Grosso, Brazil. They are a population of over 700 people living in ten villages, but most of them are concentrated in \textit{Aiha} (their oldest and biggest village, with more than 270 people) and \textit{Tankgugu}. Alongside the pressures they've been suffering from farming, illegal fishing and logging, as well as from the Brazilian government, the Kalapalo have been able to maintain their lifestyle, with their narratives (\textit{akinha}) playing a very important part. As some say, \textit{akinha} are neither “myths" nor “stories", but actual \textit{history}: they tell about events that made the world the way it is today.
    
\begin{figure} 
%   \includegraphics[width=1.0\textwidth]{figures/map_eng.jpg}
\includegraphics [width=\textwidth] {figures/xinguland.pdf}
  \caption{Upper Xingu villages, southern area of the Xingu Indigenous Land}
\end{figure}
    
    The telling of \textit{akinha} is a complex verbal art, and also a highly valued knowledge. Only a few people can be considered true “masters (or owners) of stories" (\textit{akinha oto}). To be so, one must learn them from good storytellers, and tell them “beautifully" to others. A good story-telling involves several abilities, such as keeping a regular rhythm, making good use of parallelism to call the audience's attention to events or characters \citep{Franchetto2003}, and giving details that make the listener actually “see" what is being told. The combination of these stylistic features may induce the listeners (especially children and younger people) into an almost dreamlike state, from which they must be “awakened" after an \textit{akinha} ends so they do not get sleepy, lazy, or — what would be even worse — they do not keep thinking about the spirit-beings of which they might have heard, since it could cause them to be attacked by these dangerous beings.

\newpage     
    \textit{Kamagisa's} narrative brings together several issues of Xinguan thought: the problem of mortality; the possibility of metamorphosis of humans into spirit-beings, and vice-versa; the details demanded by ritual communication and action; and the multilingual character intrinsic to Xinguan life. Perhaps all these issues could be thought of as versions of an all-encompassing one: the problem of communication (and, thus, of \textit{translation}) that imposes itself on the relations between different kinds of people, such as consanguines and affines, humans and non-humans, the living and the dead, fellow villagers and foreigners, etc.
          
    The narrative was transcribed, translated, and analyzed using ELAN, with the help of \textit{Jeika} and \textit{Ugise Kalapalo}. The narrative is phonetically transcribed in the first line, and morphologically segmented in the second one. The third line presents the glosses, and the last two contain  free translations in English and Portuguese. The transcription, morphological segmentation, and most glosses follow the works of Bruna \citet{Franchetto1986,Franchetto2002,Franchetto2003}, Ellen \citet{Basso2012}, Mara \citet{Santos2007}, Mutua \citet{Mehinaku2010}, and Aline Varela \citet{Rabello2013}. We thank Bruna Franchetto for her continuous help with the Kalapalo language.

    Let's now follow \textit{Kamagisa} on his unexpected journey to the world of the spirits.

\section{\textit{Kamagisa etsutühügü}}
\translatedtitle{‘\textit{Kamagisa} sang for the first time’}\\
\translatedtitle{‘\textit{Kamagisa} cantou pela primeira vez’}\footnote{Recordings of this story are available from \url{https://zenodo.org/record/997435}}

\ea  ihaü̃ heke onta\\[.3em]
\gll i-haü̃ heke on-ta\\
     3-cousin \textsc{erg} repudiate-\textsc{dur}\\
\glt ‘His cousin was repudiating him.’
\glt ‘A prima dele o estava rejeitando.’
\z 

\ea  indzahatohoi indzahatohoi itsa\\[.3em]
\gll i-ndzaha-toho-i i-ndzaha-toho-i i-tsa\\
     3-fiancée-\textsc{ins}-\textsc{cop} 3-fiancée-\textsc{ins}-\textsc{cop} to.be-\textsc{dur}\\
\glt ‘His fiancée, she was his fiancée.’
\glt ‘Sua noiva, ela era sua noiva.’
\z 

\ea  ah indisüi ijogu indisü\\[.3em]
\gll ah 0-indi-sü-i i-jo-gu 0-indi-sü\\
     \textsc{expl} 3-daughter-\textsc{poss}-\textsc{cop} 3-mother's.brother-\textsc{poss} 3-daughter-\textsc{poss}\\
\glt ‘Ah, she was his daughter, his mother's brother's daughter.’
\glt ‘Ah, ela era filha dele, filha do irmão de sua mãe.’
\z 

\ea  Kamagisa haü̃ha ihaü̃\\[.3em]
\gll Kamagisa haü̃-ha i-haü̃\\
     Kamagisa cousin-\textsc{ha}{\footnotemark} 3-cousin\\
\glt ‘\textit{Kamagisa's} own cousin, his cousin.’
\glt ‘A própria prima de \textit{Kamagisa}, sua prima.’
\z 
\footnotetext{Please see Franchetto's introduction to chapter 2 for discussion of the \textsc{ha} particle, found in both Kuikuro and Kalapalo.}

\ea  ülepe hale egei\\[.3em]
\gll üle-pe hale ege-i\\
     \textsc{log}-\textsc{ntm} \textsc{cntr} \textsc{dist}-\textsc{cop}\\
\glt ‘Because of that ...’
\glt ‘Por causa disso ...’
\z 

\ea  onta leha iheke tsuẽ ekugu onta iheke\\[.3em]
\gll on-ta leha i-heke tsuẽ ekugu on-ta i-heke\\
     repudiate-\textsc{dur} \textsc{compl} 3-\textsc{erg} a.lot true repudiate-\textsc{dur} 3-\textsc{erg}\\
\glt ‘She was repudiating him, she was truely repudiating him a lot.’
\glt ‘Ela o estava rejeitando, ela o estava rejeitando demais.’
\z 

\ea  ihitsü heke\\[.3em]
\gll i-hitsü heke\\
     3-wife \textsc{erg}\\
\glt ‘His wife did it.’
\glt ‘Sua esposa o fez.’
\z 

\newpage 
\ea  ihaü̃ hekeha\\[.3em]
\gll i-haü̃ heke-ha\\
     3-cousin \textsc{erg}-\textsc{ha}\\
\glt ‘His cousin did so.’
\glt ‘Sua prima fez assim.’
\z

\ea  inhalü etengalü embege etengalü inhalü\\[.3em]
\gll inhalü e-te-nga-lü embege e-te-nga-lü inhalü\\
     \textsc{neg} 3-go-\textsc{hab}-\textsc{pnct} tentatively 3-go-\textsc{hab}-\textsc{pnct} \textsc{neg}\\
\glt ‘Nothing, in vain he used to go after her; he used to go after her, and nothing.’
\glt ‘Nada, ele tentava ir atrás dela em vão; ele tentava ir atrás dela, e nada.’
\z

\ea  onta leha iheke leha\\[.3em]
\gll on-ta leha i-heke leha\\
     repudiate-\textsc{dur} \textsc{compl} 3-\textsc{erg} \textsc{compl}\\
\glt ‘She would readily repudiate him.’
\glt ‘Ela prontamente o rejeitava.’
\z

\ea  üle hinhe leha etijakilü leha\\[.3em]
\gll üle hinhe leha et-ija-ki-lü leha\\
     \textsc{log} \textsc{purp} \textsc{compl} \textsc{3.dtr}-hammock's.rope-\textsc{vblz}-\textsc{pnct} \textsc{compl}\\
\glt ‘Because of that, he untied the ropes of his hammock.’
\glt ‘Por causa disso, ele desamarrou as cordas de sua rede.’
\z

\ea  apokinenügü iheke leha tühitsü apokinenügü\\[.3em]
\gll apoki-ne-nügü i-heke leha tü-hitsü apoki-ne-nügü\\
     drop-\textsc{vblz}-\textsc{pnct} 3-\textsc{erg} \textsc{compl} \textsc{refl}-wife drop-\textsc{vblz}-\textsc{pnct}\\
\glt ‘He left her, he left his own wife.’
\glt ‘Ele a deixou, ele deixou sua própria esposa.’
\z

\ea  apokinenügü leha iheke\\[.3em]
\gll apoki-ne-nügü leha i-heke\\
     drop-\textsc{vblz}-\textsc{pnct} \textsc{compl} 3-\textsc{erg}\\
\glt ‘He soon left her.’
\glt ‘Ele logo a deixou.’
\z

\ea  üle ineke hale egei\\[.3em]
\gll üle ineke hale ege-i\\
     \textsc{log} \textsc{purp} \textsc{cntr} \textsc{dist}-\textsc{cop}\\
\glt ‘Because that made him sad ...’
\glt ‘Porque aquilo o deixou triste ...’
\z

\ea  üle ineke hale egei tingunkgingu ineke Kakakugu heke sijatitselü eke heke beha\\[.3em]
\gll üle ineke hale ege-i t-ingunkgingu ineke Kakakugu heke s-ijati-tse-lü eke heke be-ha\\
     \textsc{log} \textsc{purp} \textsc{cntr} \textsc{dist}-\textsc{cop} \textsc{refl}-thought \textsc{purp} Kakakugu \textsc{erg} 3-offer-\textsc{vblz}-\textsc{pnct} snake \textsc{erg} \textsc{aug}-\textsc{ha}\\
\glt ‘... because that made him sad, because his thoughts made him sad, \textit{Kakakugu} made him an offer, a big snake did so.’
\glt ‘... porque aquilo o deixou triste, porque seus pensamentos o deixaram triste, \textit{Kakakugu} fez uma oferta a ele, uma grande cobra o fez.’
\z

\ea  eke helei Kakakugui\\[.3em]
\gll eke h-ele-i Kakakugu-i\\
     snake \textsc{ha}-3.\textsc{dist}-\textsc{cop} Kakakugu-\textsc{cop}\\
\glt ‘He is a snake, \textit{Kakakugu}.’
\glt ‘Ele é uma cobra, \textit{Kakakugu}.’
\z

\ea  ahütü kuge hüngü\\[.3em]
\gll ahütü kuge hüngü\\
     \textsc{neg} human \textsc{neg}\\
\glt ‘He is not human.’
\glt ‘Ele não é humano.’
\z

\ea  itseke beja\\[.3em]
\gll itseke beja\\
     spirit \textsc{ep}\\
\glt ‘A spirit, indeed.’
\glt ‘Um espírito, de fato.’
\z

\newpage 
\ea  itseke\\[.3em]
\gll itseke\\
     spirit\\
\glt ‘A spirit.’
\glt ‘Um espírito.’
\z

\ea  itseke\\[.3em]
\gll itseke\\
     spirit\\
\glt ‘A spirit.’
\glt ‘Um espírito.’
\z

\ea  jatsi jatsi jasu nügü iheke\\[.3em]
\gll jatsi jatsi jasu nügü i-heke\\
     poor poor pitiful \textsc{pnct} 3-\textsc{erg}\\
\glt ‘“Poor me, poor me, pitiful me," he [\textit{Kamagisa}] said.’
\glt ‘“Pobre de mim, pobre de mim, que pena de mim," ele [\textit{Kamagisa}] disse.’
\z

\ea  uonlü nika iheke nügü iheke\\[.3em]
\gll u-on-lü nika i-heke nügü i-heke\\
     \textsc{1}-repudiate-\textsc{pnct} \textsc{ep} 3-\textsc{erg} \textsc{pnct} 3-\textsc{erg}\\
\glt ‘“Is it true that she rejected me?" he said.’
\glt ‘“Será verdade que ela me rejeitou?" ele disse.’
\z

\ea  üle hinhe hale egei inhaha sinügü\\[.3em]
\gll üle hinhe hale ege-i 0-inha-ha s-i-nügü\\
     \textsc{log} \textsc{purp} \textsc{cntr} \textsc{dist}-\textsc{cop} 3-to-\textsc{ha} 3-come-\textsc{pnct}\\
\glt ‘Because of that he came to him.’
\glt ‘Por causa disso ele veio até ele.’
\z

\ea  Kakakugu suü̃ enügüha\\[.3em]
\gll Kakakugu s-uü̃ e-nügü-ha\\
     Kakakugu 3-father come-\textsc{pnct}-\textsc{ha}\\
\glt ‘\textit{Kakakugu}, her [\textit{Kamagisa's} future wife] father, came.’
\glt ‘\textit{Kakakugu}, o pai dela [da futura esposa de \textit{Kamagisa}], veio.’
\z

\ea  suü̃ha tetinhü inha\\[.3em]
\gll s-uü̃-ha t-e-ti-nhü 0-inha\\
     3-father-\textsc{ha} \textsc{refl}-come-\textsc{ptcp}-\textsc{nanmlz} 3-to\\
\glt ‘Her father is the one who came to him.’
\glt ‘O pai dela é quem veio até ele.’
\z

\ea  suü̃ enügü\\[.3em]
\gll s-uü̃ e-nügü\\
     3-father come-\textsc{pnct}\\
\glt ‘Her father came.’
\glt ‘O pai dela veio.’
\z

\ea  ülepe\\[.3em]
\gll üle-pe\\
     \textsc{log}-\textsc{ntm}\\
\glt ‘Then ...’
\glt ‘Então ...’
\z

\ea  aiha\\[.3em]
\gll aiha\\
     done\\
\glt ‘... done.’
\glt ‘... pronto.’
\z

\ea  ingilabe lahale ingila itseke hisuü̃gü gele ukenübata egei\\[.3em]
\gll ingila-be lahale ingila itseke hisuü̃-gü gele uk-enü-bata ege-i\\
     before-\textsc{aug} \textsc{cntr} before spirit kin-\textsc{poss} \textsc{adv} \textsc{du}-stay-\textsc{temp} \textsc{dist}-\textsc{cop}\\
\glt ‘This was a long time ago, when we were still kin to the spirits.’
\glt ‘Isso foi há muito tempo atrás, quando nós ainda éramos parentes dos espíritos.’
\z

\ea  ülepe\\[.3em]
\gll üle-pe\\
     \textsc{log}-\textsc{ntm}\\
\glt ‘Then ...’
\glt ‘Então ...’
\z

\ea  sinügü\\[.3em]
\gll s-i-nügü\\
     3-come-\textsc{pnct}\\
\glt ‘... he came.’
\glt ‘... ele veio.’
\z

\ea  Kamagisa inha etelü\\[.3em]
\gll Kamagisa inha e-te-lü\\
     Kamagisa to 3-go-\textsc{pnct}\\
\glt ‘He went to \textit{Kamagisa}.’
\glt ‘Ele foi até \textit{Kamagisa}.’
\z

\ea  ihumita hegei tindisü inha\\[.3em]
\gll i-humi-ta h-ege-i t-indi-sü inha\\
     3-send-\textsc{dur} \textsc{ha}-\textsc{dist}-\textsc{cop} \textsc{refl}-daughter-\textsc{poss} to\\
\glt ‘He was sending him to his daughter.’
\glt ‘Ele o estava enviando para sua filha.’
\z

\ea  ẽ tindisü inha\\[.3em]
\gll ẽ t-indi-sü inha\\
     \textsc{aff} \textsc{refl}-daughter-\textsc{poss} to\\
\glt ‘Yes, to his daughter.’
\glt ‘Sim, para sua filha.’
\z

\ea  ülepe\\[.3em]
\gll üle-pe\\
     \textsc{log}-\textsc{ntm}\\
\glt ‘Then ...’
\glt ‘Então ...’
\z

\ea  aiha\\[.3em]
\gll aiha\\
     done\\
\glt ‘... done.’
\glt ‘... pronto.’
\z

\ea  igelü leha iheke leha\\[.3em]
\gll ige-lü leha i-heke leha\\
     take-\textsc{pnct} \textsc{compl} 3-\textsc{erg} \textsc{compl}\\
\glt ‘He soon took him.’
\glt ‘Ele logo o levou.’
\z

\ea  Kamagisa hogijü iheke leha\\[.3em]
\gll Kamagisa hogi-jü i-heke leha\\
     Kamagisa find-\textsc{pnct} 3-\textsc{erg} \textsc{compl}\\
\glt ‘He had already found \textit{Kamagisa}.’
\glt ‘Ele já havia encontrado \textit{Kamagisa}.’
\z

\ea  kuge bejetsa atühügü leha\\[.3em]
\gll kuge be-jetsa atühügü leha\\
     human \textsc{aug}-\textsc{ev} become \textsc{compl}\\
\glt ‘He had already become just like a person.’
\glt ‘Ele já havia se tornado exatamente como uma pessoa.’
\z

\ea  eke atühügü kugei leha\\[.3em]
\gll eke atühügü kuge-i leha\\
     snake become human-\textsc{cop} \textsc{compl}\\
\glt ‘The snake had already become a person.’
\glt ‘A cobra já havia se tornado uma pessoa.’
\z

\ea  hm eingadzu inha etete nügü iheke\\[.3em]
\gll hm e-ingadzu inha e-te-te nügü i-heke\\
     \textsc{expl} \textsc{2sg}-sister to \textsc{2sg}-go-\textsc{imp} \textsc{pnct} 3-\textsc{erg}\\
\glt ‘Hm, “Go to your sister," he said.’
\glt ‘Hm, “Vá para sua irmã," ele disse.’
\z

\ea  eingadzu inha kete nügü iheke\\[.3em]
\gll e-ingadzu inha k-e-te nügü i-heke\\
     \textsc{2sg}-sister to \textsc{du}-go-\textsc{imp} \textsc{pnct} 3-\textsc{erg}\\
\glt ‘“Let's go to your sister," he said.’
\glt ‘“Vamos para sua irmã," ele disse.’
\z

\ea  üngele kilüha\\[.3em]
\gll üngele ki-lü-ha\\
     \textsc{3.log} say-\textsc{pnct}-\textsc{ha}\\
\glt ‘That one said so.’
\glt ‘Aquele é quem disse isso.’
\z

\ea  Kakakugu kilüha\\[.3em]
\gll Kakakugu ki-lü-ha\\
     Kakakugu say-\textsc{pnct}-\textsc{ha}\\
\glt ‘\textit{Kakakugu} said so.’
\glt ‘\textit{Kakakugu} disse isso.’
\z

\ea  suü̃ kilüha\\[.3em]
\gll s-uü̃ ki-lü-ha\\
     3-father say-\textsc{pnct}-\textsc{ha}\\
\glt ‘Her father said so.’
\glt ‘O pai dela disse isso.’
\z

\ea  ẽhẽ nügü iheke\\[.3em]
\gll ẽhẽ nügü i-heke\\
     \textsc{aff} \textsc{pnct} 3-\textsc{erg}\\
\glt ‘“Yes," he answered.’
\glt ‘“Sim," ele respondeu.’
\z

\ea  ẽhẽ nügü iheke\\[.3em]
\gll ẽhẽ nügü iheke\\
     \textsc{aff} \textsc{pnct} 3-\textsc{erg}\\
\glt ‘“Yes," he answered.’
\glt ‘“Sim," ele respondeu.’
\z

\largerpage
\ea  ngukuagi tsügüha inhüngü tadüponhokokinhü igia\\[.3em]
\gll ngukuagi tsügüha iN-üngü t-adüponhoko-ki-nhü igia\\
     cerrado.palm \textsc{ep} 3-home \textsc{refl}-small.mound-\textsc{ins}-\textsc{nanmlz} this.way\\
\glt ‘It seems that his house is a small \textit{cerrado} palm called \textit{ngukuagi}, that stands on a small mound like this.’
\newpage
\glt ‘Parece que a casa dele é uma pequena palmeira do cerrado chamada \textit{ngukuagi}, que fica desse jeito em cima de um morrinho.’
\z

\ea  uge uü̃pe kita ũãke\\[.3em]
\gll uge uü̃-pe ki-ta ũãke\\
     \textsc{1} father-\textsc{ntm} say-\textsc{dur} \textsc{ev.pst}\\
\glt ‘My deceased father used to say:’
\glt ‘Meu finado pai costumava dizer:’
\z

\ea  Kakakugu üngü hegei\\[.3em]
\gll Kakakugu üngü h-ege-i\\
     Kakakugu home \textsc{ha}-\textsc{dist}-\textsc{cop}\\
\glt ‘“That's \textit{Kakakugu's} house."’
\glt ‘“Aquela é a casa de \textit{Kakakugu}."’
\z

\ea  inhüngü hegei\\[.3em]
\gll iN-üngü h-egei-i\\
     3-house \textsc{ha}-\textsc{dist}-\textsc{cop}\\
\glt ‘That's his house.’
\glt ‘Aquela é a casa dele.’
\z

\ea  inhüngü\\[.3em]
\gll iN-üngü\\
     3-house\\
\glt ‘His house.’
\glt ‘A casa dele.’
\z

\ea  üle hujati\\[.3em]
\gll üle huja-ti\\
     \textsc{log} midst-\textsc{all}\\
\glt ‘Into the middle of that ...’
\glt ‘Para o meio daquilo ...’
\z

\newpage 
\ea  ah etelü leha\\[.3em]
\gll ah e-te-lü leha\\
     \textsc{expl} 3-go-\textsc{pnct} \textsc{compl}\\
\glt ‘... ah, he readily went!’
\glt ‘... ah, ele foi prontamente!’
\z

\ea  ngukuagi\\[.3em]
\gll ngukuagi\\
     cerrado.palm\\
\glt ‘Into the small \textit{cerrado} palm.’
\glt ‘Para dentro da pequena palmeira do cerrado.’
\z

\ea  ẽ inhüngü hegei Kakakugu\\[.3em]
\gll ẽ iN-üngü h-egei-i Kakakugu\\
     \textsc{aff} 3-home \textsc{ha}-\textsc{dist}-\textsc{cop} Kakakugu\\
\glt ‘Yes, that's his house, \textit{Kakakugu's} house.’
\glt ‘Sim, aquela é a casa dele, a casa de \textit{Kakakugu}.’
\z

\ea  ülepe\\[.3em]
\gll üle-pe\\
     \textsc{log}-\textsc{ntm}\\
\glt ‘Then ...’
\glt ‘Então ...’
\z

\ea  aiha\\[.3em]
\gll aiha\\
     done\\
\glt ‘... done’
\glt ‘... pronto’
\z

\ea  ah tindisü tuponga leha ijatelü leha iheke\\[.3em]
\gll ah t-indi-sü t-upo-nga leha ija-te-lü leha i-heke\\
     \textsc{expl} \textsc{refl}-daughter-\textsc{poss} \textsc{refl}-above-\textsc{all} \textsc{compl} hammock's.rope-\textsc{vblz}-\textsc{pnct} \textsc{compl} 3-\textsc{erg}\\
\glt ‘Ah, he then tied his hammock just over his daughter's.’
\glt ‘Ah, então ele amarrou sua rede logo acima da de sua filha.’
\z

\ea  ijatelü leha\\[.3em]
\gll ija-te-lü leha\\
     hammock's.rope-\textsc{vblz}-\textsc{pnct} \textsc{compl}\\
\glt ‘Tied his hammock.’
\glt ‘Amarrou sua rede.’
\z

\ea  sangagübe teh ah sangatepügü\\[.3em]
\gll s-anga-gü-be teh ah s-anga-te-pügü\\
     3-genipapo-\textsc{poss}-\textsc{aug} \textsc{itj} \textsc{expl} 3-genipapo-\textsc{vblz}-\textsc{pfv}\\
\glt ‘She had a great \textit{genipapo} painting - wow! - she was painted with \textit{genipapo}.’\footnote{\textit{Genipapo} is the fruit of the \textit{Genipa americana} tree. The Kalapalo extract from it a clear liquid used for skin painting, as well as for decorating ceramics and wooden benches. After the liquid oxidizes, it turns black, and may stay on the skin for several days. The liquid may be mixed with charcoal soot to make the painting even darker.}
\glt ‘Ela tinha uma linda pintura de jenipapo - uau! - ela estava pintada com jenipapo.’
\z

\ea  indisüha\\[.3em]
\gll 0-indi-sü-ha\\
     3-daughter-\textsc{poss}-\textsc{ha}\\
\glt ‘His daughter was.’
\glt ‘A filha dele estava.’
\z

\ea  indisü angatepügü\\[.3em]
\gll 0-indi-sü anga-te-pügü\\
     3-daughter-\textsc{poss} genipapo-\textsc{vblz}-\textsc{pfv}\\
\glt ‘His daughter was painted with \textit{genipapo}.’
\glt ‘A filha dele estava pintada com jenipapo.’
\z

\ea  ülepe leha\\[.3em]
\gll üle-pe leha\\
     \textsc{log}-\textsc{ntm} \textsc{compl}\\
\glt ‘Just after that ...’
\glt ‘Logo depois disso ...’
\z

\newpage 
\ea  ah ihitsüi leha itsa leha\\[.3em]
\gll ah i-hitsü leha i-tsa leha\\
     \textsc{expl} 3-wife \textsc{compl} to.be-\textsc{dur} \textsc{compl}\\
\glt ‘... ah, she was already his wife.’
\glt ‘... ah, ela já era sua esposa.’
\z

\ea  ihitsüi leha\\[.3em]
\gll i-hitsü-i leha\\
     3-wife-\textsc{cop} \textsc{compl}\\
\glt ‘Already his wife.’
\glt ‘Já era sua esposa.’
\z

\ea  ihitsüi leha itsa leha\\[.3em]
\gll i-hitsü-i leha i-tsa leha\\
     3-wife-\textsc{cop} \textsc{compl} to.be-\textsc{dur} \textsc{compl}\\
\glt ‘She was already his wife.’
\glt ‘Ela já era sua esposa.’
\z

\ea  aiha\\[.3em]
\gll aiha\\
     done\\
\glt ‘Done.’
\glt ‘Pronto.’
\z

\ea  ingati beja inhalü ikungalü leha iheke leha\\[.3em]
\gll ingati beja i-nha-lü iku-nga-lü leha i-heke leha\\
     lie.together \textsc{ep} to.be-\textsc{hab}-\textsc{pnct} to.have.sex-\textsc{hab}-\textsc{pnct} \textsc{compl} 3-\textsc{erg} \textsc{compl}\\
\glt ‘He always lay with her in his hammock, and he always had sex with her.’
\glt ‘Ele sempre se deitava com ela em sua rede, e ele sempre fazia sexo com ela.’
\z

\newpage
\ea  ikungalü beja iheke leha\\[.3em]
\gll iku-nga-lü beja i-heke leha\\
     to.have.sex-\textsc{hab}-\textsc{pnct} \textsc{ep} 3-\textsc{erg} \textsc{compl}\\
\glt ‘He always really had sex with her.’
\glt ‘Ele realmente sempre fazia sexo com ela.’
\z

\ea  ah ngikona tüilü iheke\\[.3em]
\gll ah ngiko-na t-üi-lü i-heke\\
     \textsc{expl} thing-\textsc{ep} \textsc{refl}-make-\textsc{pnct} 3-\textsc{erg}\\
\glt ‘Ah, who knows how he did it ...’
\glt ‘Ah, quem sabe como ele fazia isso ...’
\z

\ea  aiha\\[.3em]
\gll aiha\\
     done\\
\glt ‘Done.’
\glt ‘Pronto.’
\z

\ea  sakihata iheke tita gisüki\\[.3em]
\gll s-aki-ha-ta i-heke tita gi-sü-ki\\
     3-word-\textsc{vblz}-\textsc{dur} 3-\textsc{erg} mortuary.effigy song-\textsc{poss}-\textsc{ins}\\
\glt ‘He was teaching him ... with songs of mortuary effigies.’
\glt ‘Ele o estava ensinando ... com cantos de efígies mortuárias.’
\z

\ea  tita gisüki\\[.3em]
\gll tita gi-sü-ki\\
     mortuary.effigy song-\textsc{poss}-\textsc{ins}\\
\glt ‘Songs of mortuary effigies.’
\glt ‘Cantos de efígies mortuárias.’
\z

\ea  tita gisüki ah sakihata iheke\\[.3em]
\gll tita gi-sü-ki ah s-aki-ha-ta i-heke\\
     mortuary.effigy song-\textsc{poss}-\textsc{ins} \textsc{expl} 3-word-\textsc{vblz}-\textsc{dur} 3-\textsc{erg}\\
\glt ‘With songs of mortuary effigies... ah, he was teaching him.’
\glt ‘Com cantos de efígies mortuárias... ah, ele o estava ensinando.’
\z

\ea  Kakakugu heke\\[.3em]
\gll Kakakugu heke\\
     Kakakugu \textsc{erg}\\
\glt ‘\textit{Kakakugu} was.’
\glt ‘\textit{Kakakugu} estava ensinando.’
\z

\ea  ah tindisü ngiso akihata iheke\\[.3em]
\gll ah t-indi-sü ngiso aki-ha-ta i-heke\\
     \textsc{expl} \textsc{refl}-daughter-\textsc{poss} husband word-\textsc{vblz}-\textsc{dur} 3-\textsc{erg}\\
\glt ‘Ah, he was teaching his daughter's husband.’
\glt ‘Ah, ele estava ensinando o marido de sua filha.’
\z

\ea  akihata iheke\\[.3em]
\gll aki-ha-ta i-heke\\
     word-\textsc{vblz}-\textsc{dur} 3-\textsc{erg}\\
\glt ‘He was teaching.’
\glt ‘Ele estava ensinando.’
\z

\ea  akihata iheke\\[.3em]
\gll aki-ha-ta i-heke\\
     word-\textsc{vblz}-\textsc{dur} 3-\textsc{erg}\\
\glt ‘He was teaching.’
\glt ‘Ele estava ensinando.’
\z

\ea  etelü hõhõ tüti inha\\[.3em]
\gll e-te-lü hõhõ tü-ti inha\\
     3-go-\textsc{pnct} \textsc{emph} \textsc{refl}-mother \textsc{dat}\\
\glt ‘He went to visit his mother for a while.’
\glt ‘Ele foi visitar sua mãe por um tempo.’
\z

\ea  tüti inha hõhõ sinügü Kamagisa enügü\\[.3em]
\gll tü-ti inha hõhõ s-i-nügü Kamagisa e-nügü\\
     \textsc{refl}-mother \textsc{dat} \textsc{emph} 3-come-\textsc{pnct} Kamagisa come-\textsc{pnct}\\
\glt ‘\textit{Kamagisa} came to visit his mother for a while.’
\glt ‘\textit{Kamagisa} veio visitar sua mãe por um tempo.’
\z

\ea  tüti inha\\[.3em]
\gll tü-ti inha\\
     \textsc{refl}-mother \textsc{dat}\\
\glt ‘To his mother.’
\glt ‘Até sua mãe.’
\z

\ea  ihitsü ike leha\\[.3em]
\gll i-hitsü ike leha\\
     3-wife with \textsc{compl}\\
\glt ‘With his wife.’
\glt ‘Com sua esposa.’
\z

\ea  ihitsü\\[.3em]
\gll i-hitsü\\
     3-wife\\
\glt ‘His wife.’
\glt ‘Sua esposa.’
\z

\ea  tünho akuã leha teta leha\\[.3em]
\gll tü-nho akuã leha te-ta leha\\
     \textsc{refl}-husband shadow \textsc{compl} go-\textsc{dur} \textsc{compl}\\
\glt ‘She was walking right behind her husband, like his shadow.’
\glt ‘Ela ia andando logo atrás de seu marido, como sua sombra.’
\z

\ea  inde giti atani etimbelüko\\[.3em]
\gll inde giti atani et-imbe-lü-ko\\
     here sun \textsc{temp} \textsc{3.dtr}-arrive-\textsc{pnct}-\textsc{pl}\\
\glt ‘When the sun was here, they arrived.’
\glt ‘Quando o sol estava nessa posição, eles chegaram.’
\z

\ea  etimbelüko\\[.3em]
\gll et-imbe-lü-ko\\
     \textsc{3.dtr}-arrive-\textsc{pnct}-\textsc{pl}\\
\glt ‘They arrived.’
\glt ‘Eles chegaram.’
\z

\ea  totomonaha\\[.3em]
\gll t-oto-mo-na-ha\\
     \textsc{refl}-kin-\textsc{pl}-\textsc{all}-\textsc{ha}\\
\glt ‘To his kin.’
\glt ‘Nos seus parentes.’
\z

\ea  totomo\\[.3em]
\gll t-oto-mo\\
     \textsc{refl}-kin-\textsc{pl}\\
\glt ‘His kin.’
\glt ‘Seus parentes.’
\z

\ea  tetuna beja\\[.3em]
\gll t-etu-na beja\\
     \textsc{refl}-village-\textsc{all} \textsc{ep}\\
\glt ‘To his own village, indeed.’
\glt ‘Para sua própria aldeia, de fato.’
\z

\ea  Kamagisa etu leha\\[.3em]
\gll Kamagisa etu leha\\
     Kamagisa village \textsc{compl}\\
\glt ‘They were already at \textit{Kamagisa's} village.’
\glt ‘Eles já estavam na aldeia de \textit{Kamagisa}.’
\z

\ea  Kamagisa etuna etimbelüko\\[.3em]
\gll Kamagisa etu-na et-imbe-lü-ko\\
     Kamagisa village-\textsc{all} \textsc{3.dtr}-come-\textsc{pnct}-\textsc{pl}\\
\glt ‘They arrived at \textit{Kamagisa's} village.’
\glt ‘Eles chegaram na aldeia de \textit{Kamagisa}.’
\z

\ea  atibe Kamagisa enta nügü iheke\\[.3em]
\gll atibe Kamagisa e-nta nügü i-heke\\
     \textsc{itj} Kamagisa come-\textsc{dur} \textsc{pnct} 3-\textsc{erg}\\
\glt ‘“There he comes, \textit{Kamagisa}," they said.’
\glt ‘“Lá vem ele, \textit{Kamagisa}," disseram.’
\z

\ea  inhalü ihitsü ingilüi ihametijaõ heke\\[.3em]
\gll inhalü i-hitsü ingi-lü-i i-hameti-jaõ heke\\
     \textsc{neg} 3-wife see-\textsc{pnct}-\textsc{cop} 3-sister.in.law-\textsc{pl} \textsc{erg}\\
\glt ‘His wife could not be seen by her sisters-in-law.’
\glt ‘Sua esposa não podia ser vista pelas cunhadas dela.’
\z

\ea  inhalü ingilü ihekeni\\[.3em]
\gll inhalü ingi-lü i-heke-ni\\
     \textsc{neg} see-\textsc{pnct} 3-\textsc{erg}-\textsc{pl}\\
\glt ‘They didn't see her.’
\glt ‘Elas não a viam.’
\z

\ea  uendeha ike eteta\\[.3em]
\gll uende-ha ike e-te-ta\\
     there-\textsc{ha} 3.with 3-go-\textsc{dur}\\
\glt ‘But she was there, she was going with him.’
\glt ‘Mas ela estava lá, ela estava andando com ele.’
\z

\ea  ike eteta\\[.3em]
\gll ike e-te-ta\\
     3.with 3-go-\textsc{dur}\\
\glt ‘She was going with him.’
\glt ‘Ela estava andando com ele.’
\z

\ea  ihitsü tetaha\\[.3em]
\gll i-hitsü te-ta-ha\\
     3-wife go-\textsc{pnct}-\textsc{ha}\\
\glt ‘His wife was going.’
\glt ‘Sua esposa estava andando.’
\z

\ea  ülepe\\[.3em]
\gll üle-pe\\
     \textsc{log}-\textsc{ntm}\\
\glt ‘Then ...’
\glt ‘Então ...’
\z

\ea  etimbelüko\\[.3em]
\gll et-imbe-lü-ko\\
     \textsc{3.dtr}-arrive-\textsc{pnct}-\textsc{pl}\\
\glt ‘... they arrived.’
\glt ‘... eles chegaram.’
\z

\ea  ahametigüko akongo aketsugei nügü iheke tingajomo heke\\[.3em]
\gll a-hameti-gü-ko ako-ngo akets=uge-i nügü i-heke t-ingajomo heke\\
     2-sister.in.law-\textsc{poss}-\textsc{pl} with-\textsc{nmlz} \textsc{ev}=\textsc{1}-\textsc{cop} \textsc{pnct} 3-\textsc{erg} \textsc{refl}-sisters \textsc{erg}\\
\glt ‘“I'm in the company of your sister-in-law," he said to his sisters.’
\glt ‘“Eu estou acompanhado de sua cunhada," ele disse para suas irmãs.’
\z

\ea  ahametigüko akongo aketsugei\\[.3em]
\gll a-hameti-gü-ko ako-ngo akets=uge-i\\
     2-sister.in.law-\textsc{poss}-\textsc{pl} with-\textsc{nmlz} \textsc{ev}=\textsc{1}-\textsc{cop}\\
\glt ‘“I'm in the company of your sister-in-law."’
\glt ‘“Eu estou acompanhado de sua cunhada."’
\z

\ea  itaginhitüe tsüha ahametigüko nügü iheke\\[.3em]
\gll itaginhi-tüe tsüha a-hameti-gü-ko nügü i-heke\\
     greet-\textsc{imp}.\textsc{pl} \textsc{ep} 2-sister.in.law-\textsc{poss}-\textsc{pl} \textsc{pnct} 3-\textsc{erg}\\
\glt ‘“You may greet your sister-in-law," he said.’
\glt ‘“Vocês podem cumprimentar sua cunhada," ele disse.’
\z

\ea  ẽ uhitseke geleha\\[.3em]
\gll ẽ uhitseke gele-ha\\
     \textsc{aff} in.vain \textsc{adv}-\textsc{ha}\\
\glt ‘“Ok," they said in vain.’
\glt ‘“Tudo bem," elas disseram à toa.’
\z

\ea  amago nika nügü ngapa iheke\\[.3em]
\gll amago nika nügü ngapa i-heke\\
     2\textsc{pl} \textsc{ep} \textsc{pnct} \textsc{ep} 3-\textsc{erg}\\
\glt ‘“Are you really here?" they may have said.’
\glt ‘“Você está mesmo aí?" talvez elas tenham dito.’
\z

\ea  shhh ekei bele itüinjü iheke\\[.3em]
\gll shhh eke-i bele itüin-jü i-heke\\
     \textsc{ideo} snake-\textsc{cop} \textsc{cu} answer-\textsc{pnct} 3-\textsc{erg}\\
\glt ‘“\textit{Shhh}" - she answered in the snake's language.’
\glt ‘“\textit{Shhh}" - ela respondeu na língua das cobras.’
\z

\ea  üngele hekeha\\[.3em]
\gll üngele heke-ha\\
     \textsc{3.log} \textsc{erg}-\textsc{ha}\\
\glt ‘That one did so.’
\glt ‘Ela fez isso.’
\z

\ea  ihametigüko heke\\[.3em]
\gll i-hameti-gü-ko heke\\
     3-sister.in.law-\textsc{poss}-\textsc{pl} \textsc{erg}\\
\glt ‘Their sister-in-law did.’
\glt ‘A cunhada delas fez.’
\z

\ea  Kakakugu indisü hekeha\\[.3em]
\gll Kakakugu indi-sü heke-ha\\
     Kakakugu daughter-\textsc{poss} \textsc{erg}-\textsc{ha}\\
\glt ‘\textit{Kakakugu's} daughter did so.’
\glt ‘A filha de \textit{Kakakugu} fez isso.’
\z

\ea  shhh ah nügü iheke\\[.3em]
\gll shhh ah nügü i-heke\\
     \textsc{ideo} \textsc{expl} \textsc{pnct} 3-\textsc{erg}\\
\glt ‘“\textit{Shhh}," ah, she said!’
\glt ‘“\textit{Shhh}," ah, ela disse!’
\z

\ea  aiha\\[.3em]
\gll aiha\\
     done\\
\glt ‘Done.’
\glt ‘Pronto.’
\z

\ea  ah tihü hakilü hale iheke\\[.3em]
\gll ah t-ihü haki-lü hale i-heke\\
     \textsc{expl} \textsc{refl}-body reveal-\textsc{pnct} \textsc{cntr} 3-\textsc{erg}\\
\glt ‘Ah, and then she revealed her body.’
\glt ‘Ah, e então ela revelou seu corpo.’
\z

\ea  atsakilü lahale atütüi\\[.3em]
\gll a-tsaki-lü lahale atütü-i\\
     3-appear-\textsc{pnct} \textsc{cntr} beautiful-\textsc{cop}\\
\glt ‘She appeared beautiful.’
\glt ‘Ela apareceu muito bonita.’
\z

\ea  hm atsakilü leha\\[.3em]
\gll hm a-tsaki-lü leha\\
     \textsc{expl} 3-appear-\textsc{pnct} \textsc{compl}\\
\glt ‘Hm, she appeared.’
\glt ‘Hm, ela apareceu.’
\z

\ea  atsakilü leha\\[.3em]
\gll a-tsaki-lü leha\\
     3-appear-\textsc{pnct} \textsc{compl}\\
\glt ‘She appeared.’
\glt ‘Ela apareceu.’
\z

\ea  kogetsi\\[.3em]
\gll kogetsi\\
     tomorrow\\
\glt ‘The next day ...’
\glt ‘No dia seguinte ...’
\z

\ea  etinga inhalüko unditü ukugahipügü\\[.3em]
\gll etinga i-nha-lü-ko unditü ukugahi-pügü\\
     lie.on.hammock be-\textsc{hab}-\textsc{pnct}-\textsc{pl} long.hair hang.dowards-\textsc{pfv}\\
\glt ‘When they lay down together, her hair was hanging downwards.’
\glt ‘Quando eles se deitaram juntos, o cabelo dela estava pendurado em direção ao chão.’
\z

\ea  ande leha ihametijaõ heke ingingalü leha\\[.3em]
\gll ande leha i-hameti-jaõ heke ingi-nga-lü leha\\
     now \textsc{compl} 3-sister.in.law-\textsc{pl} \textsc{erg} see-\textsc{hab}-\textsc{pnct} \textsc{compl}\\
\glt ‘Now her sisters-in-law could already see her.’
\glt ‘Agora suas cunhadas já podiam vê-la.’
\z

\ea  ihametijaõ\\[.3em]
\gll i-hameti-jaõ\\
     3-sister.in.law-\textsc{pl}\\
\glt ‘Her sisters-in-law.’
\glt ‘Suas cunhadas.’
\z

\ea  aiha\\[.3em]
\gll aiha\\
     done\\
\glt ‘Done.’
\glt ‘Pronto.’
\z

\ea  ülepe sinünkgo leha tetuna beha\\[.3em]
\gll üle-pe s-i-nü-nkgo leha t-etu-na beha\\
     \textsc{log}-\textsc{ntm} 3-come-\textsc{pnct}-\textsc{pl} \textsc{compl} \textsc{refl}-village-\textsc{all} \textsc{ev}\\
\glt ‘Then, they came back to their village.’
\glt ‘Então, eles voltaram para sua aldeia.’
\z

\ea  tseta leha ihütisoho heke sakihata\\[.3em]
\gll tseta leha i-hüti-soho heke s-aki-ha-ta\\
     there \textsc{compl} 3-shame-\textsc{ins} \textsc{erg} 3-word-\textsc{vblz}-\textsc{dur}\\
\glt ‘There, his father-in-law was teaching him.’
\glt ‘Lá, seu sogro o estava ensinando.’
\z

\ea  igeki beha\\[.3em]
\gll ige-ki beha\\
     \textsc{prox}-\textsc{ins} \textsc{ev}\\
\glt ‘About this.’
\glt ‘Sobre isso.’
\z

\ea  tita gisüki\\[.3em]
\gll tita gi-sü-ki\\
     mortuary.effigy song-\textsc{poss}-\textsc{ins}\\
\glt ‘About the songs of mortuary effigies.’
\glt ‘Sobre os cantos de efígies mortuárias.’
\z

\ea  tita gisüki\\[.3em]
\gll tita gisü-ki\\
     mortuary.effigy song-\textsc{poss}-\textsc{ins}\\
\glt ‘About the songs of mortuary effigies.’
\glt ‘Sobre os cantos de efígies mortuárias.’
\z

\ea  aiha etsuhukilü leha inha leha\\[.3em]
\gll aiha etsuhuki-lü leha 0-inha leha\\
     done finish-\textsc{pnct} \textsc{compl} 3-\textsc{dat} \textsc{compl}\\
\glt ‘Done, it was finished for him.’
\glt ‘Pronto, estava tudo concluído para ele.’
\z

\ea  hm untsi nügü iheke\\[.3em]
\gll hm untsi nügü i-heke\\
     \textsc{expl} uterine.nephew \textsc{pnct} 3-\textsc{erg}\\
\glt ‘“Hm, nephew," he said.’
\glt ‘“Hm, sobrinho," ele disse.’
\z

\ea  etsuke hetsange hõhõ ihakitomi eheke nügü iheke\\[.3em]
\gll etsu-ke hetsange hõhõ i-haki-tomi e-heke nügü i-heke\\
     debut-\textsc{imp} \textsc{hort} \textsc{emph} 3-reveal-\textsc{purp} 2-\textsc{erg} \textsc{pnct} 3-\textsc{erg}\\
\glt ‘“You may sing for the first time, to reveal your songs," he said.’
\glt ‘“Você pode estrear, para revelar seus cantos," ele disse.’
\z

\ea  nügü iheke\\[.3em]
\gll nügü i-heke\\
     \textsc{pnct} 3-\textsc{erg}\\
\glt ‘He said.’
\glt ‘Ele disse.’
\z

\ea  ihütisoho kilü\\[.3em]
\gll i-hüti-soho ki-lü\\
     3-shame-\textsc{ins} say-\textsc{pnct}\\
\glt ‘His father-in-law said.’
\glt ‘Seu sogro disse.’
\z

\ea  Kakakugu kilü beha\\[.3em]
\gll Kakakugu ki-lü beha\\
     Kakakugu say-\textsc{pnct} \textsc{ev}\\
\glt ‘\textit{Kakakugu} said that.’
\glt ‘\textit{Kakakugu} disse isso.’
\z

\ea  etsuke hetsange hõhõ ah ihakitomi eheke nügü iheke\\[.3em]
\gll etsu-ke hetsange hõhõ ah i-haki-tomi e-heke nügü i-heke\\
     debut-\textsc{imp} \textsc{hort} \textsc{emph} \textsc{expl} 3-reveal-\textsc{purp} 2-\textsc{erg} \textsc{pnct} 3-\textsc{erg}\\
\glt ‘“You may now sing for the first time, ah, to reveal them," he said.’
\glt ‘“Você agora pode estrear, ah, para revelá-los," ele disse.’
\z

\ea  ihakitomi nügü iheke\\[.3em]
\gll i-haki-tomi nügü i-heke\\
     3-reveal-\textsc{purp} \textsc{pnct} 3-\textsc{erg}\\
\glt ‘“To reveal them," he said.’
\glt ‘“Para revelá-los," ele disse.’
\z

\ea  ah nügü iheke\\[.3em]
\gll ah nügü i-heke\\
     \textsc{expl} \textsc{pnct} 3-\textsc{erg}\\
\glt ‘“Ah," he said.’
\glt ‘“Ah," ele disse.’
\z

\ea  nügü iheke Kamagisa heke\\[.3em]
\gll nügü i-heke Kamagisa heke\\
     \textsc{pnct} 3-\textsc{erg} Kamagisa \textsc{erg}\\
\glt ‘He said to \textit{Kamagisa}.’
\glt ‘Ele disse para \textit{Kamagisa}.’
\z

\ea  atütüi beja itsalü leha iheke leha\\[.3em]
\gll atütü-i beja i-tsa-lü leha i-heke leha\\
     beautiful-\textsc{cop} \textsc{ep} 3-hear-\textsc{pnct} \textsc{compl} 3-\textsc{erg} \textsc{compl}\\
\glt ‘He had already listened to it really well.’
\glt ‘Ele já os havia escutado muito bem.’
\z

\ea  itsalü leha atütüui ekugu leha\\[.3em]
\gll i-tsa-lü leha atütü-i ekugu leha\\
     3-hear-\textsc{pnct} \textsc{compl} beautiful-\textsc{cop} true \textsc{compl}\\
\glt ‘He listened to it really well.’
\glt ‘Ele os havia escutado muito bem.’
\z

\ea  ülepe etimbelü tetuna\\[.3em]
\gll üle-pe et-imbe-lü t-etu-na\\
     \textsc{log}-\textsc{ntm} \textsc{2.dtr}-come-\textsc{pnct} \textsc{refl}-village-\textsc{all}\\
\glt ‘Then he arrived in his village.’
\glt ‘Então ele chegou em sua aldeia.’
\z

\ea  tikongoingo akihalü hõhõ iheke\\[.3em]
\gll t-iko-ngo-ingo aki-ha-lü hõhõ i-heke\\
     \textsc{refl}-with-\textsc{nmlz}-\textsc{fut} word-\textsc{vblz}-\textsc{pnct} \textsc{emph} 3-\textsc{erg}\\
\glt ‘First he taught the one who was going to be his singing companion.’
\glt ‘Primeiro ele ensinou aquele que seria seu companheiro de canto.’
\z

\ea  ah tikongoingo akihalü engü beja otohongoingo beja iginhundote\\[.3em]
\gll ah t-iko-ngo-ingo aki-ha-lü engü beja otohongo-ingo beja igi-nhun-dote\\
     \textsc{expl} \textsc{refl}-with-\textsc{nmlz}-\textsc{fut} word-\textsc{vblz}-\textsc{pnct} \textsc{con} \textsc{ep} other.similar-\textsc{fut} \textsc{ep} song-\textsc{vblz}-\textsc{adv}\\
\glt ‘Ah, he taught the one who was going to be his companion, his other, when he was to sing.’
\glt ‘Ah, ele ensinou aquele que seria seu companheiro, seu outro, quando ele fosse cantar.’
\z


\newpage 
\ea  otohongoingo tsüha\\[.3em]
\gll otohongo-ingo tsüha\\
     other.similar-\textsc{fut} \textsc{ev.uncr}\\
\glt ‘That one who would be his companion.’
\glt ‘Aquele que seria seu companheiro.’
\z

\ea  üngele akihalü hõhõ iheke\\[.3em]
\gll üngele aki-ha-lü hõhõ i-heke\\
     \textsc{3.log} word-\textsc{vblz}-\textsc{pnct} \textsc{emph} \textsc{3sg}-\textsc{erg}\\
\glt ‘First he taught him.’
\glt ‘Primeiro ele o ensinou.’
\z

\ea  üngele akihalü.\\[.3em]
\gll üngele aki-ha-lü\\
     \textsc{3.log} word-\textsc{vblz}-\textsc{pnct}\\
\glt ‘Taught him.’
\glt ‘O ensinou.’
\z

\ea  aiha etükilü\\[.3em]
\gll aiha etüki-lü\\
     done complete-\textsc{pnct}\\
\glt ‘Done, it was complete.’
\glt ‘Pronto, estava completo.’
\z

\ea  etükilü leha inha\\[.3em]
\gll etüki-lü leha 0-inha\\
     complete-\textsc{pnct} \textsc{compl} 3-\textsc{dat}\\
\glt ‘It was complete for him.’
\glt ‘Estava completo para ele.’
\z

\ea  otohongo inha\\[.3em]
\gll otohongo inha\\
     other.similar \textsc{dat}\\
\glt ‘For his companion.’
\glt ‘Para seu companheiro.’
\z

\ea  osiha\\[.3em]
\gll osi-ha\\
     \textsc{hort}-\textsc{ha}\\
\glt ‘“Let's go."’
\glt ‘“Vamos lá."’
\z

\ea  osiha ai hale tüti heke nügü iheke\\[.3em]
\gll osi-ha ai hale tü-ti heke nügü i-heke\\
     \textsc{hort}-\textsc{ha} \textsc{purp} \textsc{cntr} \textsc{refl}-mother \textsc{erg} \textsc{pnct} 3-\textsc{erg}\\
\glt ‘“Let's go," and then he said to his mother:’
\glt ‘“Vamos lá," e então ele disse para sua mãe:’
\z

\ea  ama\\[.3em]
\gll ama\\
     mother\\
\glt ‘“Mother."’
\glt ‘“Mãe."’
\z

\ea  ah kupuke hõhõ nügü iheke\\[.3em]
\gll ah k-upu-ke hõhõ nügü i-heke\\
     \textsc{expl} \textsc{du}-make.a.visual.imitation-\textsc{imp} \textsc{emph} \textsc{pnct} 3-\textsc{erg}\\
\glt ‘“Ah, make our image," he said.’
\glt ‘“Ah, faça nossa imagem," ele disse.’
\z

\ea  ah kupuke hõhõ nügü iheke\\[.3em]
\gll ah k-upu-ke hõhõ nügü i-heke\\
     \textsc{expl} \textsc{du}-make.a.visual.imitation-\textsc{imp} \textsc{emph} \textsc{pnct} 3-\textsc{erg}\\
\glt ‘“Ah, make our image," he said.’
\glt ‘“Ah, faça nossa imagem," ele disse.’
\z

\ea  kupuke ah nügü baha iheke\\[.3em]
\gll k-upu-ke ah nügü baha i-heke\\
     \textsc{du}-make.a.visual.imitation-\textsc{imp} \textsc{expl} \textsc{pnct} \textsc{ev} 3-\textsc{erg}\\
\glt ‘“Make our image," ah, that's what he said.’
\glt ‘“Faça nossa imagem," ah, isso é o que ele disse.’
\z

\ea  ẽhẽ nügü iheke\\[.3em]
\gll ẽhẽ nügü i-heke\\
     \textsc{aff} \textsc{pnct} 3-\textsc{erg}\\
\glt ‘“Yes," she said.’
\glt ‘“Sim," ela disse.’
\z

\ea  isi heke tsüle togokige ihenügü togokibe bahale\\[.3em]
\gll isi heke tsüle togokige ihe-nügü togoki-be bahale\\
     mother \textsc{erg} \textsc{ep} cotton spin-\textsc{pnct} cotton-\textsc{aug} \textsc{adv}\\
\glt ‘Then his mother spun cotton, a lot of cotton.’
\glt ‘Então sua mãe fiou algodão, muito algodão.’
\z

\ea  igia kugube sueletu sagagebe otohongo\\[.3em]
\gll igia kugu-be s-uele-tu s-agage-be otohongo\\
     this.way true-\textsc{aug} 3-girth-\textsc{poss} 3-alike-\textsc{aug} other.similar\\
\glt ‘A roll was this big, and another one also had the same size.’
\glt ‘Um rolo era grande desse jeito, e um outro tinha o mesmo tamanho.’
\z

\ea  tita etikoguingo hegei\\[.3em]
\gll tita etiko-gu-ingo h-ege-i\\
     mortuary.effigy belt-\textsc{poss}-\textsc{fut} \textsc{ha}-\textsc{dist}-\textsc{cop}\\
\glt ‘This is what would become the effigy's belt.’
\glt ‘Isso é o que se tornaria o cinto da efígie.’
\z

\ea  ege hungu jetsa inke tsapa akago heke tüita\\[.3em]
\gll ege hungu jetsa in-ke tsapa akago heke t-üi-ta\\
     \textsc{dist} similar \textsc{ev} see-\textsc{imp} \textsc{ep} 3\textsc{pl} \textsc{erg} \textsc{refl}-make-\textsc{dur}\\
\glt ‘Like those, look those they are making.’\footnote{\textit{Ageu} refers to the cotton belts some young men were making at a neighboring house.}
\glt ‘Como aqueles, veja aqueles que eles estão fazendo.’
\z

\ea  üle hunguingo hegei\\[.3em]
\gll üle hungu-ingo h-ege-i\\
     \textsc{log} similar-\textsc{fut} \textsc{ha}-\textsc{dist}-\textsc{cop}\\
\glt ‘It was meant to be like those.’
\glt ‘Era pra ser como aqueles.’
\z

\ea  üle hunguingo hegei\\[.3em]
\gll üle hungu-ingo h-ege-i\\
     \textsc{log} similar-\textsc{fut} \textsc{ha}-\textsc{dist}-\textsc{cop}\\
\glt ‘It was meant to be like those.’
\glt ‘Era pra ser como aqueles.’
\z

\ea  isi ngihetanümi\\[.3em]
\gll isi ng-ihe-ta-nümi\\
     mother \textsc{log}-spin-\textsc{dur}-\textsc{pnct.cop}\\
\glt ‘What his mother was spinning.’
\glt ‘O que sua mãe estava fiando.’
\z

\ea  isi ngihetanümi togokigeha\\[.3em]
\gll isi ng-ihe-ta-nümi togokige-ha\\
     mother \textsc{log}-spin-\textsc{dur}-\textsc{pnct.cop} cotton-\textsc{ha}\\
\glt ‘The cotton his mother was spinning.’
\glt ‘O algodão que sua mãe estava fiando.’
\z

\ea  aiha\\[.3em]
\gll aiha\\
     done\\
\glt ‘Done.’
\glt ‘Pronto.’
\z

\ea  togokige etükilü\\[.3em]
\gll togokige etüki-lü\\
     cotton complete-\textsc{pnct}\\
\glt ‘The cotton was ready.’
\glt ‘O algodão estava pronto.’
\z

\ea  osiha\\[.3em]
\gll osi-ha\\
     \textsc{hort}-\textsc{ha}\\
\glt ‘“Let's go."’
\glt ‘“Vamos lá."’
\z

\ea  aibeha\\[.3em]
\gll aibeha\\
     done\\
\glt ‘Done.’
\glt ‘Pronto.’
\z

\ea  tita ikenügü ihekeni\\[.3em]
\gll tita ike-nügü i-heke-ni\\
     mortuary.effigy to.cut-\textsc{pnct} 3-\textsc{erg}-\textsc{pl}\\
\glt ‘They cut down a log for an effigy.’
\glt ‘Eles derrubaram uma tora para fazer uma efígie.’
\z

\ea  tita ikenügü leha ihekeni\\[.3em]
\gll tita ike-nügü leha i-heke-ni\\
     mortuary.effigy to.cut-\textsc{pnct} \textsc{compl} 3-\textsc{erg}-\textsc{pl}\\
\glt ‘They had already cut down a log for an effigy.’
\glt ‘Eles já tinham derrubado uma tora para fazer uma efígie.’
\z

\ea  ületoho\\[.3em]
\gll üle-toho\\
     \textsc{log}-\textsc{ins}\\
\glt ‘To do that.’
\glt ‘Para fazer isso.’
\z

\ea  tühutoho hegei\\[.3em]
\gll tü-hu-toho h-ege-i\\
     \textsc{refl}-imitation-\textsc{ins} \textsc{ha}-\textsc{dist}-\textsc{cop}\\
\glt ‘That was his image.’
\glt ‘Aquela era sua imagem.’
\z

\ea  agetsiha agetsi tita\\[.3em]
\gll agetsi-ha agetsi tita\\
     one-\textsc{ha} one mortuary.effigy\\
\glt ‘There was only one, one effigy.’
\glt ‘Havia apenas uma, uma única efígie.’
\z

\ea  ülepea higei ah titabe ige tüingalü higei\\[.3em]
\gll üle-pe-a h-ige-i ah tita-be ige t-üi-nga-lü h-ege-i\\
     \textsc{log}-\textsc{ntm}-\textsc{caus} \textsc{ha}-\textsc{prox}-\textsc{cop} \textsc{expl} mortuary.effigy-\textsc{aug} \textsc{prox} \textsc{refl}-make-\textsc{hab}-\textsc{pnct} \textsc{ha}-\textsc{dist}-\textsc{cop}\\
\glt ‘It's since this that we have been making effigies.’
\glt ‘É desde então que nós temos feito efígies.’
\z

\ea  ülepeaha\\[.3em]
\gll üle-pe-a=ha\\
     \textsc{log}-\textsc{ntm}-as=\textsc{ha}\\
\glt ‘Since this.’
\glt ‘Desde então.’
\z

\ea  aiha\\[.3em]
\gll aiha\\
     done\\
\glt ‘Done.’
\glt ‘Pronto.’
\z

\ea  ülepebe\\[.3em]
\gll üle-pe-be\\
     \textsc{log}-\textsc{ntm}-\textsc{aug}\\
\glt ‘Then.’
\glt ‘Então.’
\z

\ea  ah inhegikaginenügü bele iheke\\[.3em]
\gll ah inh-egikagi-ne-nügü bele i-heke\\
     \textsc{expl} 3-sing.closely-\textsc{vblz}-\textsc{pnct} \textsc{cu} 3-\textsc{erg}\\
\glt ‘Ah, he sang behind the effigy.’
\glt ‘Ah, ele cantou atrás da efígie.’
\z

\ea  angi taka kangaki etelüko inhalü hungube\\[.3em]
\gll angi taka kanga-ki e-te-lü-ko inhalü hungu-be\\
     \textsc{int} \textsc{adv} fish-\textsc{ins} 3-go-\textsc{pnct}-\textsc{pl} \textsc{neg} similar-\textsc{aug}\\
\glt ‘Did they go fishing? It doesn't seem so ...’
\glt ‘Será que eles foram pescar? Não parece que foram ...’
\z

\ea  ẽ kangaki muke etelüko\\[.3em]
\gll ẽ kanga-ki muke e-te-lü-ko\\
     \textsc{aff} fish-\textsc{ins} \textsc{adv} 3-go-\textsc{pnct}-\textsc{pl}\\
\glt ‘Yes, they must have gone fishing.’
\glt ‘Sim, eles devem ter ido pescar.’
\z

\ea  kangaki hõhõ etelüko\\[.3em]
\gll kanga-ki hõhõ e-te-lü-ko\\
     fish-\textsc{ins} \textsc{emph} 2-go-\textsc{pnct}-\textsc{pl}\\
\glt ‘First they went fishing.’
\glt ‘Primeiro eles foram pescar.’
\z

\ea  ah ületohokibe\\[.3em]
\gll ah üle-toho-ki-be\\
     \textsc{expl} \textsc{log}-\textsc{ins}-\textsc{ins}-\textsc{aug}\\
\glt ‘Ah, to do so.’
\glt ‘Ah, para fazer isso.’
\z

\ea  inhalü ihagitoguiha inhalü\\[.3em]
\gll inhalü i-hagito-gu-i-ha inhalü\\
     \textsc{neg} 3-guest-\textsc{poss}-\textsc{cop}-\textsc{ha} \textsc{neg}\\
\glt ‘He didn't have guests, no.’
\glt ‘Ele não tinha convidados, não.’
\z

\ea  etsuta hale egea hale egei\\[.3em]
\gll etsu-ta hale egea hale ege-i\\
     debut-\textsc{dur} \textsc{cntr} like.that \textsc{cntr} \textsc{dist}-\textsc{cop}\\
\glt ‘He was just singing for the first time.’
\glt ‘Ele estava apenas estreando.’
\z

\ea  hm etsuta\\[.3em]
\gll hm etsu-ta\\
     \textsc{expl} debut-\textsc{dur}\\
\glt ‘Hm, he was singing for the first time.’
\glt ‘Hm, ele estava estreando.’
\z

\ea  aiha\\[.3em]
\gll aiha\\
     done\\
\glt ‘Done.’
\glt ‘Pronto.’
\z

\ea  ah iginhun leha\\[.3em]
\gll ah igi-nhun leha\\
     \textsc{expl} song-\textsc{vblz} \textsc{compl}\\
\glt ‘Ah, he started to sing!’
\glt ‘Ah, ele começou a cantar!’
\z

\ea  nhagati bele ekü telü tita telü leha egea\\[.3em]
\gll nhaga-ti bele ekü te-lü tita te-lü leha egea\\
     hole-\textsc{all} \textsc{ev} \textsc{con} go-\textsc{pnct} mortuary.effigy go-\textsc{pnct} \textsc{compl} like.that\\
\glt ‘The effigy was put standing straight in a hole.’
\glt ‘A efígie foi colocada de pé em um buraco.’
\z

\ea  üle egikagineta bele ihekeni\\[.3em]
\gll üle egikagi-ne-ta bele i-heke-ni\\
     \textsc{log} sing.closely-\textsc{vblz}-\textsc{dur} \textsc{ev} 3-\textsc{erg}-\textsc{pl}\\
\glt ‘This is what they were siging about behind it.’
\glt ‘Era sobre isso que eles estavam cantando atrás dela.’
\z

\ea  ah totohongo ake leha\\[.3em]
\gll ah t-otohongo ake leha\\
     \textsc{expl} \textsc{refl}-other.similar \textsc{com} \textsc{compl}\\
\glt ‘Ah, together with his companion.’
\glt ‘Ah, junto com seu companheiro.’
\z

\ea  tüngakihapügü ake tsüha\\[.3em]
\gll tüng-aki-ha-pügü ake tsüha\\
     \textsc{refl}-word-\textsc{vblz}-\textsc{pfv} \textsc{com} \textsc{ev.uncr}\\
\glt ‘With the one he taught.’
\glt ‘Com aquele que ele ensinou.’
\z

\ea  tüngakihapügü ake\\[.3em]
\gll tüng-aki-ha-pügü ake\\
     \textsc{refl}-word-\textsc{vblz}-\textsc{pfv} \textsc{com}\\
\glt ‘With the one he taught.’
\glt ‘Com aquele que ele ensinou.’
\z

\ea  ah iginhundako leha\\[.3em]
\gll ah igi-nhu-nda-ko leha\\
     \textsc{expl} song-\textsc{vblz}-\textsc{dur}-\textsc{pl} \textsc{compl}\\
\glt ‘Ah, they were singing.’
\glt ‘Ah, eles estavam cantando.’
\z

\ea  aiha akinügü leha\\[.3em]
\gll aiha aki-nügü leha\\
     done finish-\textsc{pnct} \textsc{compl}\\
\glt ‘Done, it was finished.’
\glt ‘Pronto, estava terminado.’
\z

\ea  akinügü\\[.3em]
\gll aki-nügü\\
     finish-\textsc{pnct}\\
\glt ‘It was finished.’
\glt ‘Estava terminado.’
\z

\ea  ülepe\\[.3em]
\gll üle-pe\\
     \textsc{log}-\textsc{ntm}\\
\glt ‘Then ...’
\glt ‘Então ...’
\z

\ea  kohotsi inhügü iginhuko leha\\[.3em]
\gll kohotsi inhügü igi-nhu-ko leha\\
     at.dusk become song-\textsc{nmlz}-\textsc{pl} \textsc{compl}\\
\glt ‘When dusk came, they were singing again.’
\glt ‘Quando chegou o entardecer, eles estavam cantando novamente.’
\z

\ea  ah iginhundako leha\\[.3em]
\gll ah igi-nhuN-da-ko leha\\
     \textsc{expl} song-\textsc{vblz}-\textsc{dur}-\textsc{pl} \textsc{compl}\\
\glt ‘Ah, they were singing again.’
\glt ‘Ah, eles estavam cantando novamente.’
\z

\ea  kohotsi\\[.3em]
\gll kohotsi\\
     at.dusk\\
\glt ‘At dusk.’
\glt ‘Ao entardecer.’
\z

\ea  ülepe mitote\\[.3em]
\gll üle-pe mitote\\
     \textsc{log}-\textsc{ntm} at.dawn\\
\glt ‘And then at dawn ...’
\glt ‘E depois ao amanhecer ...’
\z

\ea  aibeha inhügü gehale\\[.3em]
\gll ai-be-ha inhügü gehale\\
     \textsc{hort}-\textsc{aug}-\textsc{ha} become \textsc{adv}\\
\glt ‘... ready, one more time.’
\glt ‘... pronto, mais uma vez.’
\z

\ea  inhügü gehale\\[.3em]
\gll inhügü gehale\\
     become \textsc{adv}\\
\glt ‘One more time.’
\glt ‘Mais uma vez.’
\z

\ea  ah ta beja iheke\\[.3em]
\gll ah ta beja i-heke\\
     \textsc{expl} \textsc{dur} \textsc{ep} 3-\textsc{erg}\\
\glt ‘Ah, his words!’
\glt ‘Ah, suas palavras!’
\z

\ea  Kamagisa kita leha\\[.3em]
\gll Kamagisa ki-ta leha\\
     Kamagisa say-\textsc{dur} \textsc{compl}\\
\glt ‘What \textit{Kamagisa} was saying.’
\glt ‘O que \textit{Kamagisa} estava dizendo.’
\z

\ea  iginhundako leha\\[.3em]
\gll igi-nhun-da-ko leha\\
     song-\textsc{vblz}-\textsc{dur}-\textsc{pl} \textsc{compl}\\
\glt ‘They were singing.’
\glt ‘Eles estavam cantando.’
\z

\ea  aiha\\[.3em]
\gll aiha\\
     done\\
\glt ‘Done.’
\glt ‘Pronto.’
\z

\ea  aiha mitote\\[.3em]
\gll aiha mitote\\
     done at.dawn\\
\glt ‘Done, at dawn ...’
\glt ‘Pronto, ao amanhecer ...’
\z

\ea  ah sitogupe onginügübe\\[.3em]
\gll ah s-ito-gu-pe ongi-nügü-be\\
     \textsc{expl} 3-fire-\textsc{poss}-\textsc{ntm} bury-\textsc{pnct}-\textsc{aug}\\
\glt ‘... Ah, his fire was buried.’\footnote{During the last night of a mortuary ritual, a fire is kept in front of the deceased's effigy. While it is kept burnig, the soul of the deceased is present among the living. At dawn, the fire is buried while a chief makes a speech exhorting the deceased to leave permanently to the village of the dead.}
\glt ‘... Ah, seu fogo foi enterrado.’
\z

\newpage 
\ea  engü bejaha sitogupeha onginügü leha mitote\\[.3em]
\gll engü beja-ha s-ito-gu-pe-ha ongi-nügü leha mitote\\
     \textsc{con} \textsc{ep}-\textsc{ha} 3-fire-\textsc{poss}-\textsc{ntm}-\textsc{ha} bury-\textsc{pnct} \textsc{compl} at.dawn\\
\glt ‘Yes, his fire was buried at dawn.’
\glt ‘Sim, seu fogo foi enterrado ao amanhecer.’
\z

\ea  apungu baha egei leha\\[.3em]
\gll apungu baha ege-i leha\\
     end \textsc{ev} \textsc{dist}-\textsc{cop} \textsc{compl}\\
\glt ‘That was the end.’
\glt ‘Aquilo foi o fim.’
\z

\ea  apungu leha egei inhalü hale ikinduko\\[.3em]
\gll apungu leha ege-i inhalü hale ikindu-ko\\
     end \textsc{compl} \textsc{dist}-\textsc{cop} \textsc{neg} \textsc{cntr} wrestling-\textsc{pl}\\
\glt ‘That was the end, they didn't wrestle \textit{ikindene}.’\footnote{\textit{Ikindene} is a combat sport practiced by all the peoples of the Upper Xingu, and it is considered one of the most important features of their regional society. Wrestling is the \textit{climax} of the Quarup mortuary ritual, but, since \textit{Kamagisa} didn't have any guests, his ritual also didn't have \textit{ikindene}.}
\glt ‘Aquilo foi o fim, eles não lutaram \textit{ikindene}.’
\z

\ea  inhalü hale ikinduko egea gele hegei gele\\[.3em]
\gll inhalü hale ikindu-ko egea gele h-ege-i gele\\
     \textsc{neg} \textsc{cntr} wrestling-\textsc{pl} like.that \textsc{adv} \textsc{ha}-\textsc{dist}-\textsc{cop} \textsc{adv}\\
\glt ‘They didn't wrestle \textit{ikindene}, it was just like that.’
\glt ‘Eles não lutaram \textit{ikindene}, foi só daquele jeito.’
\z

\ea  tuhunügü gele\\[.3em]
\gll tu-hu-nügü gele\\
     \textsc{refl}-make.an.image-\textsc{pnct} \textsc{adv}\\
\glt ‘Just his image was made.’
\glt ‘Só sua imagem foi feita.’
\z

\newpage 
\ea  etsunügü gele Kamagisa etsunügü\\[.3em]
\gll etsu-nügü gele Kamagisa etsu-nügü\\
     debut-\textsc{pnct} \textsc{adv} Kamagisa debut-\textsc{pnct}\\
\glt ‘Just \textit{Kamagisa}'s debut and self-image-making.’
\glt ‘Só a estreia e feitura da própria imagem de \textit{Kamagisa}.’
\z

\ea  ama\\[.3em]
\gll ama\\
     mother\\
\glt ‘“Mother."’
\glt ‘“Mãe."’
\z

\ea  jeugagi je jeuga jeuga jeuga\\[.3em]
\glt [\textit{Kamagisa} was singing in Kamayurá, a Tupi-Guarani language]\footnote{Multilinguism in ritual language is discussed in the final section.}
\glt [\textit{Kamagisa} estava cantanto em Kamayurá, uma língua Tupi-Guarani]
\z

\ea  ama jeuga ipugu inkgete umbüngaitsü tühügü hegei iheke\\[.3em]
\gll ama jeuga ipu-gu i-nkgete u-mbüngai-tsü tühügü h-egei i-heke\\
     mother macaw feather-\textsc{poss} bring-\textsc{imp} 1-armlet-\textsc{poss} \textsc{pfv} \textsc{ha}-\textsc{dist}-\textsc{cop} 3-\textsc{erg}\\
\glt ‘“Mother, bring my macaw feathers armlets," that's what he said.’
\glt ‘“Mãe, traga meus braceletes de pena de arara," é o que ele disse.’
\z

\ea  hm Kamagisa kilü\\[.3em]
\gll hm Kamagisa ki-lü\\
     \textsc{expl} Kamagisa say-\textsc{pnct}\\
\glt ‘Hm, \textit{Kamagisa} said.’
\glt ‘Hm, \textit{Kamagisa} disse.’
\z

\ea  ületse ingugiha isi heke indzüngaitsü inginügü\\[.3em]
\gll üle-tse ingugi-ha isi heke i-ndzüngai-tsü ingi-nügü\\
     \textsc{log}-\textsc{dim} decision.solution-\textsc{ha} mother \textsc{erg} 3-armlet-\textsc{poss} bring-\textsc{pnct}\\
\glt ‘It was soon solved, and his mother brought his armlets.’
\glt ‘Isso logo se resolveu, e sua mãe trouxe seus braceletes.’
\z

\ea  jahu ehe jahu e\\[.3em]
\glt [\textit{Kamagisa} was still singing in Kamayurá]
\glt [\textit{Kamagisa} ainda estava cantando em Kamayurá]
\z

\ea  ama jahu puguha ulekugu inkgete\\[.3em]
\gll ama jahu pugu-ha u-leku-gu i-nkgete\\
     mother oropendola feather-\textsc{ha} 1-headdress-\textsc{poss} bring-\textsc{imp}\\
\glt ‘“Mother, bring my oropendola [\textit{Psarocolius sp.}] feathers headdress."’\footnote{The oropendolas — \textit{xexéu}, in Portuguese, and \textit{kui}, in Kalapalo — are birds of the \textit{Psarocolius} genus, whose tail feathers are highly esteemed for their vivid yellow tones.}
\glt ‘“Mãe, traga minha plumária de penas de xexéu [\textit{Psarocolius sp.}]."’
\z

\ea  leku heke ẽ ẽ hügeku heke\\[.3em]
\gll leku heke ẽ ẽ hügeku heke\\
     headdress \textsc{erg} \textsc{aff} \textsc{aff} headdress \textsc{erg}\\
\glt ‘He was talking about a headdress called \textit{leku}, yes, yes, about a headdress also called \textit{hügeku}.’
\glt ‘Ele estava falando de uma plumária chamada \textit{leku}, sim, sim, sobre uma plumária também chamada de \textit{hügeku}.’
\z

\ea  Kamajula hegei\\[.3em]
\gll Kamajula h-ege-i\\
     kamayurá \textsc{ha}-\textsc{dist}-\textsc{cop}\\
\glt ‘That is in the Kamayurá language.’
\glt ‘Aquilo está na língua Kamayurá.’
\z

\ea  egea ta iheke kamajulai\\[.3em]
\gll egea ta i-heke kamajula-i\\
     like.that \textsc{dur} 3-\textsc{erg} kamayurá-\textsc{cop}\\
\glt ‘He was saying that in Kamayurá.’
\glt ‘Ele estava dizendo aquilo em Kamayurá.’
\z

\ea  kamajula\\[.3em]
\gll kamajula\\
     kamayurá\\
\glt ‘In Kamayurá.’
\glt ‘Em Kamayurá.’
\z

\ea  Kamajula heke küngüa iganügü jahu nügü iheke\\[.3em]
\gll Kamajula heke küngüa iga-nügü jahu nügü i-heke\\
     kamayurá \textsc{erg} oropendola name-\textsc{pnct} oropendola \textsc{pnct} 3-\textsc{erg}\\
\glt ‘The Kamayurá call the oropendola “\textit{jahu}," that's what they say.’
\glt ‘Os Kamayurá chamam o xexéu de “\textit{jahu}," isso é o que eles dizem.’
\z

\ea  aiha\\[.3em]
\gll aiha\\
     done\\
\glt ‘Done.’
\glt ‘Pronto.’
\z

\ea  apungu leha\\[.3em]
\gll apungu leha\\
     end \textsc{compl}\\
\glt ‘That was the end.’
\glt ‘Aquele foi o fim.’
\z

\ea  etsuhukilü leha\\[.3em]
\gll etsuhu-ki-lü leha\\
     end-\textsc{vblz}-\textsc{pnct} \textsc{compl}\\
\glt ‘It was over.’
\glt ‘Tinha acabado.’
\z

\ea  etsuhukilü\\[.3em]
\gll etsuhu-ki-lü\\
     end-\textsc{vblz}-\textsc{pnct}\\
\glt ‘Over.’
\glt ‘Acabado.’
\z

\ea  ama nügü iheke\\[.3em]
\gll ama nügü i-heke\\
     mother \textsc{pnct} 3-\textsc{erg}\\
\glt ‘“Mother," he said.’
\glt ‘“Mãe," ele disse.’
\z

\ea  ama\\[.3em]
\gll ama\\
     mother\\
\glt ‘“Mother."’
\glt ‘“Mãe."’
\z

\ea  uentomila aketsange uteta leha igei leha\\[.3em]
\gll u-en-tomi-la aketsange u-te-ta leha ige-i leha\\
     \textsc{1}-wait-\textsc{purp}-\textsc{neg} \textsc{int} \textsc{1}-go-\textsc{dur} \textsc{compl} \textsc{prox}-\textsc{cop} \textsc{compl}\\
\glt ‘“This is so that I'm never coming back, I am leaving."’
\glt ‘“Isso é para que eu nunca mais volte, eu estou partindo."’
\z

\ea  üle igakaho hegei etsuta\\[.3em]
\gll üle igakaho h-ege-i etsu-ta\\
     \textsc{log} ahead \textsc{ha}-\textsc{dist}-\textsc{cop} debut-\textsc{dur}\\
\glt ‘Before that he sang for the first time.’
\glt ‘Antes disso ele estreou.’
\z

\ea  etsuta hegei\\[.3em]
\gll etsu-ta h-ege-i\\
     debut-\textsc{dur} \textsc{ha}-\textsc{dist}-\textsc{cop}\\
\glt ‘That was his first time singing.’
\glt ‘Aquela foi a primeira vez dele cantando.’
\z

\ea  ama\\[.3em]
\gll ama\\
     mother\\
\glt ‘“Mother."’
\glt ‘“Mãe."’
\z

\ea  uenügüti hestange keiti\\[.3em]
\gll u-e-nügü-ti hetsange k-e-iti\\
     \textsc{1}-come-\textsc{pnct}-wish \textsc{hort} \textsc{imp}.\textsc{proh}-\textsc{2sg}-wish\\
\glt ‘“You shall not want me to come back."’
\glt ‘“Você não deve querer que eu volte."’
\z

\ea  ah ahati heke seku ũãke utihunhetatühügü ũãke\\[.3em]
\gll ah a-hati heke seku ũãke u-ti-hu-nhe-ta-tühügü ũãke\\
     \textsc{expl} \textsc{2sg}-niece \textsc{erg} \textsc{ep} \textsc{ev.pst} \textsc{1}-throat-swell-\textsc{vblz}-\textsc{dur}-\textsc{pfv} \textsc{ev.pst}\\
\glt ‘“Ah, your niece made my throat swell with sadness in the past."’
\glt ‘“Ah, sua sobrinha fez minha garganta ficar inchada de tristeza no passado."’
\z

\ea  ah nügü leha iheke tüti heke\\[.3em]
\gll ah nügü leha i-heke tü-ti heke\\
     \textsc{expl} \textsc{pnct} \textsc{compl} 3-SG \textsc{refl}-mother \textsc{erg}\\
\glt ‘Ah, he said so to his mother.’
\glt ‘Ah, ele disse para sua mãe.’
\z

\ea  üi üi üi isi honunda leha\\[.3em]
\gll üi üi üi isi honu-nda leha\\
     \textsc{ideo} \textsc{ideo} \textsc{ideo} mother cry-\textsc{dur} \textsc{compl}\\
\glt ‘“\textit{Üi, üi, üi}" - his mother was crying.’
\glt ‘“\textit{Üi, üi, üi}" - sua mãe estava chorando.’
\z

\ea  kogetsi kogetsi leha egei sinügü etsutühügüngine\\[.3em]
\gll kogetsi kogetsi leha ege-i s-i-nügü etsu-tühügü-ngine\\
     tomorrow tomorrow \textsc{compl} \textsc{dist}-\textsc{cop} 3-come-\textsc{pnct} debut-\textsc{pfv}-\textsc{all}\\
\glt ‘On the next day, the day after he came, after he had sung for the first time.’
\glt ‘No dia seguinte, no dia após sua vinda, depois que ele havia cantado pela primeira vez.’
\z

\ea  etsutühügü\\[.3em]
\gll etsu-tühügü\\
     debut-\textsc{pfv}\\
\glt ‘He sang for the first time.’
\glt ‘Ele cantou pela primeira vez.’
\z

\newpage 
\ea  etsutühügüngine leha sinügü\\[.3em]
\gll etsu-tühügü-ngine leha s-i-nügü\\
     debut-\textsc{pfv}-\textsc{all} \textsc{compl} 3-come-\textsc{pnct}\\
\glt ‘After he had sung for the first time, he came.’
\glt ‘Depois que ele havia cantado pela primeira vez, ele veio.’
\z

\ea  apungu ekugu leha inhalü leha totomona tünga tetuna etelüi leha\\[.3em]
\gll apungu ekugu leha inhalü leha t-oto-mo-na t-ünga t-etu-na e-te-lü-i leha\\
     end true \textsc{compl} \textsc{neg} \textsc{compl} \textsc{refl}-kin-\textsc{pl}-\textsc{all} \textsc{refl}-house \textsc{refl}-village-\textsc{all} 3-go-\textsc{pnct}-\textsc{cop} \textsc{compl}\\
\glt ‘He went away for good, he didn't ever come back to his kin, to his house, to his village.’
\glt ‘Ele foi embora de vez, ele nunca mais voltou para seus parentes, para sua casa, para sua aldeia.’
\z

\ea  inhalü leha\\[.3em]
\gll inhalü leha\\
     \textsc{neg} \textsc{compl}\\
\glt ‘Never again.’
\glt ‘Nunca mais.’
\z

\ea  apungui leha etelü\\[.3em]
\gll apungu-i leha e-te-lü\\
     end-\textsc{cop} \textsc{compl} 3-go-\textsc{pnct}\\
\glt ‘He went away for good.’
\glt ‘Ele foi embora de vez.’
\z

\ea  tsakeha\\[.3em]
\gll tsa-ke-ha\\
     listen-\textsc{imp}-\textsc{ha}\\
\glt ‘Listen.’
\glt ‘Ouça.’
\z

\newpage
\ea  uitsajingugu kitseha\\[.3em]
\gll uitsajingugu ki-tse-ha\\
     ? say-\textsc{imp}-\textsc{ha}\\
\glt ‘Say “\textit{uitsajingugu}."’\footnote{We could never find a proper translation of this word. A nahukua man once said it would mean `my little shin!', and explained that one should say so to avoid getting lazy after listening to a narrative.}
\glt ‘Diga “\textit{uitsajingugu}.''{}’
\z

\ea  upügü hegei\\[.3em]
\gll upügü h-ege-i\\
     last \textsc{ha}-\textsc{dist}-\textsc{cop}\\
\glt ‘This is the end.’
\glt ‘Este é o fim.’
\z

\section{Comments: alterity and translation}

This \textit{akinha}, ‘story, narrative', is formally similar to all others the Kalapalo recount, and they share a narrative style with the Kuikuro, Matipu, and Nahukua, their Carib-speaking neighbors \citep{Basso1985,Franchetto1986}. An \textit{akinha} stands apart from ordinary talk by means of stylistic resources that mark its “frontiers" to the listeners, usually beginning and ending with the word \textit{tsakeha}, ‘listen' (followed, in the end, by the expression \textit{upügü hegei}, ‘this is the end', when the narrator declares there's nothing left to be told). \textit{Ageu} usually marks these “frontiers" when telling stories, but this one is an exception, since it begins without his calling any special attention to it. \textit{Kamagisa's} story, like other narratives, is also internally divided into thematic blocks that may be identified by opening and closing lines, such as \textit{ülepe}, ‘then' and \textit{aiha}, ‘done'.
    
    The first block (lines 1-28) tells how a hurtful event led to \textit{Kamagisa's} separation from his kin, followed by his first contact with \textit{Kakakugu}, an \textit{itseke}, a powerful spirit-being. In the second block (lines 29-58), we are told about \textit{Kamagisa's} displacement to the spiritual world, resulting (block 3, lines 59-68) in his marriage with \textit{Kakakugu's} daughter (a Snake Woman). When, in the fourth block (lines 69-111), \textit{Kamagisa} returns to his village accompanied by his new wife, she is still invisible to her affines, and speaks a language incomprehensible to humans. As they lie together and have frequent sexual relations, the Snake Woman's body becomes visible to her affines (block 5, lines 112-120), and, when \textit{Kamagisa} returns to his father-in-law's village, he finishes his learning of a special knowledge. He learns songs that will lead him to ask his mother to ‘make an image' for him (block 6, lines 121-162), a request followed by the description of important steps in the preparation of the \textit{egitsü} mortuary ritual: the cutting of a special tree from which the effigy is made (block 7, lines 163-174), the temporal sequencing of the song performances (block 8, lines 175-203), a brief explanation of the musical language (block 9, lines 204-224), and, finally, a sad farewell to the human world (block 10, lines 225-246). A similar structure can be found in other narratives, in which a deception or fight with someone's kin may lead a character away from the human world, provoking his contact with Others (spiritual beings, enemies, or non-Indians) that will become the source of some special knowledge that he or she will transmit to humans.\footnote{It is noteworthy that this narrative inverts several aspects of the myth of \textit{Arakuni} as told by the Arawak-speaking Wauja. \textit{Arakuni} is \textit{loved} (not repudiated) by his \textit{sister} (a forbidden woman, the opposite of a cross-cousin as a preferred spouse); instead of leaving of his own will, his mother is the one who \textit{sends him away}; and finally, instead of marrying a spirit-being, \textit{he becomes one himself} (a great snake). When leaving the human world, Arakuni sings until he is fully transformed into a spirit-being, and his chants are now part of the Quarup repertoire. I would like to thank Aristóteles Barcelos Neto for calling my attention to the relations between both narratives.}
    
    This narrative calls our attention to a trope, contained both in its title and its events. I've decided to translate \textit{Kamagisa} \textit{etsutühügü} as ‘\textit{Kamagisa} Sang for the First Time' because this is the sense in which this expression is usually understood. More specifically, the verb \textit{etsunügü} can be translated as ‘to debut', as I've done in the glosses. However, \textit{etsunügü} has also two other meanings closely related to the final scenes of the narrative: it can also mean ‘to make an image of oneself' (such as a self-portrait, a \textit{selfie} picture with a cell phone or, in this case, a mortuary effigy), and ‘to set a date for leaving'. While debuting as a singer, Kamagisa also performed the other two actions. First, he made an image of himself, a \textit{tita}, a mortuary effigy which is also called \textit{kuge hutoho} (‘made in order to imitate a person', or ‘the image of a person'). By doing so, he revealed to his mother his intention to leave his kin and his village once and for all, since his feelings were deeply hurt by his former fiancée (his mother's brother's daughter, MBD). During the \textit{egitsü}, or Quarup, the production and display of mortuary effigies is done in order for the dead to depart and leave their kin behind \citep{Guerreiro2011,Guerreiro2015}. Kamagisa, in this sense, was acting like a dead person, performing his own mortuary ritual. \textit{Etsunügü}, then, combines different actions performed by \textit{Kamagisa} in a single word — who, by making an effigy of himself, created both the context for revealing songs learned from the spirits and for his final departure.
    
\begin{figure}[h!]
  \includegraphics[width=1.0\textwidth]{figures/Effigies.jpg}
  \caption{Kalapalo boy sitting nearby two mortuary effigies at Aiha. Photo: Marina Pereira Novo}
\end{figure}
    
     \textit{Kamagisa's} story also introduces us to the multiethnic and multilingual composition of Xinguano rituals, and the means of translating myths into songs, songs into action, and action into creative or transformative social relations. The songs \textit{Kamagisa} learned form a musical suite, or \textit{gepa}, named after \textit{Kamagisa's} village \textit{Hagagikugu}. This village resulted from the fission of the Akuku, an ancient Carib-speaking people linguistically and sociopolitically closely related to the Kalapalo. Some say they were actual Kalapalo ancestors, as we can also see in the literature \citep{Basso2001}. Others, however, insist the Akuku were a different people, more closely related to the Nahukua. In any case, \textit{Kamagisa's} story tells about the origins of a suite of songs considered to be special Kalapalo knowledge, and that's why the Kalapalo are seen as their true ‘owners' or ‘masters' (\textit{otomo}). However, most Kalapalo can't understand but a few words of it, since the songs are almost entirely in Kamayurá, a language of the Tupi-Guarani family. We're facing here a fairly common (and fascinating) situation in the Upper Xingu: we're talking about the origin myth of a suite of Kalapalo songs, sung mostly in a Tupi-Guarani language, and which plays a central role in a ritual with a probable Arawakan origin.

	When interacting with different forms of alterity, the problem of communication comes to the fore, and \textit{Kamagisa's} narrative shows how translations can be produced by several media: what one sings, even though it's not completely understood, may be translated into actions, that, in turn, can be translated into social relations. As Rafael \citet{Bastos1983,Bastos1983} already argued some time ago, this suggests that, if there is anything like a \textit{lingua franca} in the Upper Xingu, it is their rituals and the communication system they compose from myths, musical and choreographic performances, and bodily decoration.

\section*{Acknowledgments}
Writing of this chapter was made possible by financial support from the São Paulo Research Foundation (FAPESP) for the Young Researcher Project “Transforming Amerindian regional systems: the Upper Xingu case" (process number 2013/26676-0). 

\section*{Non-standard abbreviations}

\begin{tabularx}{.45\textwidth}{lX}
\textsc{aff } & affirmative \\
\textsc{aug } & augmentative \\
\textsc{cntr } & contrastive \\
\textsc{com } & comitative \\
\textsc{con } & connective \\
\textsc{cu } & cumulative effect \\
\textsc{dtr } & detransitivizer \\
\textsc{ep } & epistemic \\
\textsc{emph } & emphatic \\
\textsc{ev } & evidential \\
\textsc{expl } & expletive \\
\textsc{hab } & habitual aspect \\
\textsc{ha } & ha particle \\
\end{tabularx}
\begin{tabularx}{.5\textwidth}{lQ}
\textsc{hort } & hortative \\
\textsc{ideo } & ideophone \\
\textsc{int } & intensifier \\
\textsc{itj } & interjection \\
\textsc{log } & logophoric \\
\textsc{nanmlz } & non-agent nominalizer\\
\textsc{nmlz } & nominalizer \\
\textsc{ntm } & nominal tense marker \\
\textsc{pnct } & punctual aspect \\
\textsc{purp } & purposive \\
\textsc{temp } & temporal marker \\
\textsc{uncr } & uncertainty \\
\textsc{vblz } & verbalizer \\
\end{tabularx}
 
{\sloppy
\printbibliography[heading=subbibliography,notkeyword=this]
}
\end{document}
