\documentclass[output=paper,
modfonts,nonflat
]{langsci/langscibook} 
\author{Bruna Franchetto\affiliation{Museu Nacional, Federal University of Rio de Janeiro, Brazil}%
\and Carlos Fausto\affiliation{Museu Nacional, Federal University of Rio de Janeiro, Brazil}%
\and Ájahi Kuikuro%
\lastand Jamalui Kuikuro Mehinaku%
}%
\title{Kuikuro}
\lehead{B.\ Franchetto, C.\ Fausto, Ájahi Kuikuro \& Jamalui Kuikuro Mehinaku}
\ourchaptersubtitle{Anha ituna tütenhüpe itaõ}
\ourchaptersubtitletrans{‘The woman who went to the village of the dead’}
% \abstract{noabstract}
\ChapterDOI{10.5281/zenodo.1008774}

\maketitle

\begin{document}

\section{Introduction}
\emph{Anha ituna tütenhüpe itaõ}, ‘The woman who went to the village of the dead’, is a narrative lasting roughly twenty minutes. It was registered by Bruna Franchetto and Carlos Fausto on the 23rd of November, 2004, in both audio and video formats, in \emph{Ipatse}, the main Kuikuro village (Southern Amazonia, State of Mato Grosso, Brazil). Bruna Franchetto began her linguistic and anthropological research on the Upper Xingu Carib languages, particularly on the Kuikuro dialect, in 1977. Carlos Fausto began his anthropological research among the Kuikuro in 1998.

The storyteller was \emph{Ájahi}, a woman who was around 65 years old at that time, a renown ritual specialist and expert singer of the female rituals of \emph{Jamugikumalu} and \emph{Tolo}.

\begin{figure}[h!]
  \caption{Bruna Franchetto and \emph{Ájahi} Kuikuro (Photo: \emph{Takumã} Kuikuro, 2001)}
  \centering
\includegraphics[width=\textwidth]{figures/ajahi.jpg}
\end{figure}

Basic annotation of the recording – orthographic transcription and translation - was done by \emph{Jamalui}, a Kuikuro researcher, using ELAN, with the assistance of Carlos Fausto. \emph{Asusu} Kuikuro helped with the penultimate revision, carried out in June 2016. The annotated text has been revised more than once by Franchetto, who added the interlinear analysis.

The transcription line is orthographic. Kuikuro (alphabetic) writing was developed by 
indigenous teachers, in collaboration with Bruna Franchetto, in the 1990s. The correspondences between “letters" or groups of letters (digraphs and trigraphs) and symbols from the International Phonetic Alphabet (IPA), when different, are as follows: <ü> /ɨ/, <j> /ʝ/, <g> /ʀ̆/ (uvular flap), <ng> /ŋ/, <nh> /ɲ/, <nkg> /ŋɡ/; N represents a subspecified fluctuating nasal.

\section{Kuikuro: people and language} 
Kuikuro is the name by which one of the dialects of the Upper Xingu Carib Language, in the Xinguan Southern Branch of the Carib family, is known \citep{MeiraFranchetto2005,Meira2006}. It is spoken by approximately 600 people, distributed in six villages in the region known as “Upper Xingu", in the headwaters of the Xingu river, Southern Amazonia, Brazil. They inhabit the southeastern region of the Xingu Indigenous Park, between the Culuene and the Buriti rivers, where they have lived since at least the second half of the 16th century. Archaeological, linguistic, and ethnological research all point to the upper Rio Buriti region as the homeland of the Kuikuros' ancestors. This region was occupied by Carib groups who had travelled from the west of the Rio Culuene, possibly in the 17th century. The denomination “Kuikuro" derives from the toponym for the place where, at the beginning of the 19th century, the first Kuikuro village (\emph{Kuhi ikugu}, ‘Needle Fish Creek’) was erected as the residence of a recognized autonomous member of the Upper Xingu system. This toponym has been frozen as a permanent ethnonym since the first written ethnographic record by Karl von den Steinen at the end of the 19th century \citep{Steinen1894}.
 

\begin{figure}[t]
  \caption{Villages of the Upper Xingu people, Xingu Indigenous Land.}
% \includegraphics [width=\textwidth] {figures/Figure-3_Map_Xingu_land.jpg}
% \includegraphics [width=\textwidth] {figures/Figure-3_Map_Xingu_land.pdf}
\includegraphics [width=\textwidth] {figures/xinguland.pdf}
\end{figure}

Upper Xingu Carib dialects are distinguished mainly by different prosodic structures \citep{SilvaFranchetto2011}. The speakers of these varieties are part of the Upper Xingu Carib sub-system, which in turn is tied to the multilingual and multiethnic regional system known as the Upper Xingu. This comprises the drainage basin of the headwaters of the Xingu River, itself one of the largest southern tributaries of the Amazon. Thanks to the collaborative work of archaeologists, linguists, and anthropologists \citep{FranchettoHeckenberger2001,FaustoEtAl2008,Franchetto2011}, we are beginning to understand the historical origins of this regional system. We can now confidently claim that this system was formed over the last four hundred years, incorporating people from different origins in a continuous and dynamic process. Speakers of languages belonging to the three major linguistic groupings in South America (Arawak, Carib, and Tupi) and one linguistic isolate (Trumai) created a unique social system that remains functional today. 

Kuikuro should be considered a stable, albeit vulnerable, language/variety. Its stability derives from the protection of the Kuikuro territory over the last fifty years, the gradual and late start of formal schooling in the last twenty years, and a linguistic and cultural heritage that is highly valued, both internally and externally, since the Kuikuro are part of the Upper Xingu region, which has been constructed as a Brazilian national icon of “Indianness". Its vulnerability is due to a variety of factors including conflict between the indigenous language and the dominant language (Portuguese), schooling, the growing presence of written media and television, the increasing mobility of individuals and families between villages and towns, and prolonged stays in town.  Another extremely relevant and ambiguous factor related to the preservation or weakening of the indigenous language is contact with missionaries, which has become increasingly intense. In contrast, initiatives that seek to strengthen the indigenous language have been put into effect by researchers (linguists and anthropologists) in participative documentation projects that including the production of videos via the training of indigenous filmmakers, publications and the supervision of indigenous researchers. This process needs to continue in order to effectively safeguard the indigenous language.

The morphosyntactic characteristics of Kuikuro can be summed up in the following generalizations:

\begin{itemize}
 \item  It is a highly agglutinative, head final, and ergative language. 
 \item  Any head constitutes a prosodic unit with its internal argument. 
 \item A unique set of prefixal person markers indexes internal (absolutive) argument on verbs, nominals, and postpositions. Kuikuro, similar to many other Amerindian languages, makes a morphological  distinction between first person plural inclusive and first person plural exclusive. The abbreviations ‘1.2’ and ‘1.3’ were chosen as glosses for first person plural inclusive and first person plural exclusive, respectively. The prefixed morpheme \emph{ku-} (\emph{kuk-} with nominal and verbal stems beginning with a vowel) is the phonological exponent of the abstract person features [ego\&tu] or [actor\&participant].The prefixed morpheme \emph{ti-} (\emph{tis-} with vowel-initial stems) is the phonological exponent of  the abstract person features [ego\&alter] or [actor\&non-participant].
 \item There are no auxiliaries and there is no explicit agreement between verbs and their arguments. 
 \item Argument nominals are bare, underdetermined for number and definiteness.
 \item The aspectual inflection of the Kuikuro verb includes three main suffixes: punctual (\textsc{pnct}, an event or action seen as instantaneous, without inherent temporal duration), durative (\textsc{dur}) and perfect (\textsc{prf}). Tense is outside the verbal word.
\end{itemize}
 

The ubiquity of the clitic \emph{ha} is noteworthy, a grammatical morpheme whose function and meaning are still under investigation. We have not yet defined a specific meta-linguistic gloss for it. It certainly marks an important component of the syntactic packaging of information, acting at the interface with discourse. In fast speech, \emph{ha} cliticizes to the following word, while in slow speech it cliticizes to the preceding word. \citet{Kalin2014} offers an interesting formal explanation of the same particle in Hixkaryana, a Northern Carib language.

The grammatical morpheme, possible cliticizable, \emph{leha}, interpreted as a completive aspect, also occurs in most Kuikuro sentences. It “closes" main as well as secondary predications and is possibly responsible for the finiteness of the verbal word. Like the clitic \emph{ha}, it can occur more than once in the same utterance. It is interesting to see the complementary positioning of \emph{ha} and \emph{leha} in  lines 4 and 5, giving a semantically differentiating nuance whose subtlety we are still unable to grasp completely.

For more information on Kuikuro grammar, see, among other publications: \citet{Franchetto1986,Franchetto2010,Franchetto2015,Santos2007,FranchettoSantos2009,FranchettoSantos2010,LimaEtAl2013}.

\section{The narrative}
\emph{Anha ituna tütenhüpe itaõ} ‘The woman who went to the village of the dead’, offers a distinctively feminine version of the pan-xinguan and better-known narrative ‘\emph{Agahütanga}, the trip of a living man to the village of the dead'. In the more – let us call it – “masculine" version, \emph{Agahütanga} cries for a dead friend, who descends from the celestial world of the dead to take him on a journey of knowledge. During the journey, \emph{Agahütanga} witnesses the fatal obstacles that can dissolve the dead into the nothingness of smoke. He manages to reach the celestial village, whose owner is a two-headed vulture, devourer of the dead, which is also where the stars that mark the seasonal calendar reside. \emph{Agahütanga} returns to earth but was killed by a jealous and greedy enemy (and sorcerer).

Kuikuro and Kalapalo masculine versions of this narrative were published in \citet{Carneiro1977} and \citet[91--140]{Basso1985}, respectively. Another Kuikuro version was recorded in 2003 by Carlos Fausto, and deposited (transcribed and translated), alongside the present version by \emph{Ájahi}, in the Kuikuro digital archives hosted by the DoBeS (Documentation of Endangered Languages, Max Pank Institute of Psycholinguistics and VolkswagenStiftung) Program and by the ProDoclin (Program for the Documentation of Indigenous Languages, Museu do Índio, FUNAI, Rio de Janeiro). Two short Kamayurá (masculine) versions, both in Portuguese, were published in \citet[122--130]{VillasBoas1970} and \citet[200--201]{Agostinho1974}.

	In \emph{Ájahi}’s version, a woman is taken, by her dead mother-in-law and by her longing (a dangerous and virtually fatal feeling) for her dead husband, through the path of the dead (\emph{anha}) to their celestial village. “Dead" is a possible and viable translation for the word \emph{anha}. \emph{Akunga} or \emph{akuã} designates an animic principle that animates living beings, a shadow of a thing or person, or a double. The \emph{akunga} of a dead person is \emph{akungape}, an ex-\emph{akunga},  the word receiving the suffix \emph{-pe} that marks the nominal past tense or, better, the nominal terminative aspect. “The \emph{akunga} remains inside the body of the deceased until the grave begins to be filled in; but then, feeling the weight of fresh earth being heaped upon it, it slips out of the body and abandons the grave. At about this point, the soul ceases to be called \emph{akunga} and is referred to as añá [\emph{anha}], the name it will bear from here on” \citep[3]{Carneiro1977}.

 \emph{Ájahi} describes  the astonishing village of the dead, pointing to the three essential elements of the structure of a true Xinguan village (lines 145-148). Beside the \emph{hugogo}, the \emph{kuakutu} (an Arawakan word) known, in Portuguese, as “men’s house", is a small traditional house built in the middle of the plaza and is the place where only men gather daily for pleasure, shamanistic sessions, political discussions, body painting, sharing food, and ritual concentration. \emph{Tajühe} is the third defining element of village architecture: the prototypical house of a big chief, much larger than the common houses, decorated internally and externally, including two “ears" (large triangular mats at the ends of the horizontal rod supporting the ceiling, twisted with moriche palm leaves (\emph{Mauritia flexuosa}) fibers. Every house is a body, with a butt, belly, back, and earrings.

\begin{figure}[h!]
  \caption{The Kuikuro village of \emph{Ipatse} drawn by \emph{Sepé} Kuikuro, 2006} 
  \centering
\includegraphics[width=\textwidth]{figures/ipatse.jpg}
\end{figure}

From the village of \emph{anha}, the living visitor has to be upside down in order to see the world of the living. Hiding behind her mother-in-law, the woman sees her dead-husband's-soul and his dead-brother's-soul returning from a fishing trip. She sees him with another woman. In lines 81 and 171, \emph{Ájahi} introduces this personnage: \emph{Itsangitsegu}, the wife of the dead husband in the afterworld, the celestial village of the dead. \emph{Itsangitsegu} is an \emph{itseke} (a hyper-being or supernatural being), not a dead person. She is the \emph{itseke} that gets the old dead who are weak, takes care of them, puts them in seclusion, and feeds them until they are raised up again. She is an \emph{itseke} with only one breast and one buttock, but she is very generous and all women are jealous of her. \emph{Itsangitsegu} is the mother of the twins Sun and Moon, the ancestors who created the human species. A character of a long and founding mythical saga, \emph{Itsangitsegu} was killed by her mother-in-law Jaguar, and, after her death, was honored by Sun and Moon in the first \emph{egitsü}, known as “Kwaryp", the great Xinguan intertribal ritual that marks the end of the mourning period after the deaths of chiefs and outstanding ritual specialists and singers.

In the afterworld, other words are used, referring to an inside-out world. The woman returns from this exceptional journey to the world of the living, but her fate on this side will be determined by the consequences of having dared to transpose an impassable frontier. It is impossible not to note the similarities between this narrative and the Greek myth of Orpheus and Eurydice: Orpheus crosses through the door to the world of the dead because of his devastating desire to have Eurydice, his dead wife, back. Hades, the god of the underworld, grants him the possibility of bringing her back to the world of the living, but if he turns to see his beloved, who is following him out of the underworld, he will lose her forever. Orpheus disobeys, and Eurydice is transformed into a statue of salt.
	
    These are the most relevant linguistic and structural characteristics of the narrative.	The text recurrently makes use of the suffix \emph{–pe}, glossed as \textsc{ntm} (“nominal tense marker"), as in \emph{kakisükope} (\emph{k-aki-sü-ko-pe}, 1.2-word-\textsc{poss-pl-ntm}), which can be loosely translated as ‘our former words’ or ‘those which were our words’, referring to the words of the living that the dead seek out and transform, in their language of the dead, into other words. In lines 128-141, \emph{Ájahi} insists on the contrast/complementarity between the language of the dead and the language of the living. The suffix \emph{-pe} here means that the dead are trying to recover their language (that they used when they were alive), but in this effort they only find synonyms in the language of the dead.  \emph{–pe} could also be analysed as a “nominal terminative aspect" and is attested for common nouns and proper names, as well as in possessed and absolute noun phrases, conveying the death/destruction/end of the referent(s), a change of form/identity of the referent(s), or a loss of functionality of the referent(s) \citep{FranchettoSantos2009,FranchettoThomas2016}.

	Another relevant aspect is the use of specific forms and expressions during verbal interaction between affinal relatives, particularly between mother-in-law and daughter-in-law. In these interactions, the use of the second person plural instead of the second person singular is obligatory, as in, for example, \emph{emuguko} (\emph{e-mugu-ko}, 2-son-\textsc{pl}), `your son’ (the woman speaking to her mother-in-law about her own husband, the son of the mother-in-law). The Kuikuro kinship term for one's parents-in-law is \emph{hüsoho}, a nominalization (\emph{-soho}) of the verbal stem \emph{hü},  meaning `to feel shy/respect'. So, \emph{hüsoho} means, literally, `made to feel shy/respect'.

	Concerning the noteworthy traits of the narrative structure (see also \citealt{Basso1985,Franchetto2003}) we draw attention to the following:

\begin{itemize}
\item More than half of the text is a direct citation of dialogue between the characters, with a predominance of verbal forms inflected by performative modes (imperative, hortative, imminent future), interjections, as well as epistemic markers (including evidentials) that modulate the attitudes and communicative intentions of the interacting speakers. 
In Kuikuro there are several \emph{verba dicendi}, such as the roots \emph{ki} (say), \emph{itagi} (talk), \emph{aki} (language/word). Particularly interesting is the very frequent use of just the aspectual morphemes \emph{tagü} (durative) and \emph{nügü} (punctual) immediately at the end of the directly reported speech.
\item The storyteller regularly marks the progressive development of the narrative by logophoric connectives (\emph{lepe, ülepe}) and by movement verbs (for instance, \emph{te}, ‘to go’). This is how the textual units that we may equate to paragraphs are marked, grouping phrases/enunciations, the minimal textual units represented by the lines in the transcription. 
\item We have sought to keep most of the repetitions, which constitute the parallelistic characteristics of the narrative style, configuring its rhythm and, at times, changes of perspective before events that we would otherwise suppose to be understood as a single unit.
\end{itemize}
 

The narrative presented here is an example, in female voice, of the Kuikuro art of telling. Its apprehension is synergistically verbal and visual: the scenes succeed one another by movement through space-time, characters become animated one before another in their voice-bodies. Listeners are captured as if in a dream.

\section{Anha ituna tütenhüpe itaõ}
\translatedtitle{‘The woman who went to the village of the dead’}\\
\translatedtitle{‘A mulher que foi para a aldeia dos mortos’}\footnote{Recordings of this story are available from \url{https://zenodo.org/record/997443}}

\ea tüma akinha ititüi \\[.3em]
\gll
tü-ma		akinha ititü-i \\
\textsc{q-dub} 	story 	name-\textsc{cop} \\\
\glt ‘Which is the name of the story?’ 	(question by Bruna Franchetto) \\
‘Como é o nome da estória?’		(pergunta de Bruna Franchetto)\\
\z

\ea anha ituna tütenhüpe itaõ \\[.3em]
\gll
anha	itu-na		tü-te-nhü-pe				itaõ\\
dead	place-\textsc{all} 	\textsc{ptcp}-go.\textsc{ptco-nanmlz}-\textsc{ntm} 	woman\\
\glt ‘The woman who went to the village of the dead, the woman’ \\
‘A mulher que foi para a aldeia dos mortos, a mulher’ \\
\z

\ea anha ituna etelü\\[.3em]
\gll
anha	itu-na		e-te-lü \\
dead 	place-\textsc{all}	3-go-\textsc{pnct} \\
\glt ‘She went to the village of the dead’\\
‘Ela foi para a aldeia dos mortos’
\z

\ea inhope apünguha\\[.3em]
\gll
i-nho-pe		apünguN=ha\\
3-husband-\textsc{ntm}	die.\textsc{pnct=ha}\\
\glt ‘Her husband had died’\\
‘O esposo dela tinha falecido’\\
\z

\ea inhope apüngu leha\\[.3em]
\gll
i-nho-pe	apünguN	leha \\
3-husband-\textsc{ntm} 	die.\textsc{pnct} 	\textsc{compl}\\
\glt ‘Her husband died’ \\
‘O esposo dela faleceu’\\
\z

\ea  lepe inhaka leha itsagü \\[.3em]
\gll üle-pe		i-ngaka{\footnotemark}{} 	leha 		i-tsagü\\
\textsc{log}-\textsc{ntm}	3-instead.of	\textsc{compl} 	3.be-\textsc{dur}\\
\glt ‘Then, she stayed in his place (she went into morning for him)’ \\
‘Depois, ela foi ficando no lugar dele (foi ficando de luto por ele)’\\
\footnotetext{The velar nasal palatalizes after the high front vowel, at the morphemic boundary (see \citealt{Franchetto1995}).}
\z

 \ea hombei leha hombei leha itsagü\\[.3em]
\gll hombe-i	leha	hombe-i	leha	i-tsagü\\
widow-\textsc{cop} 	\textsc{compl}  widow-\textsc{cop} 	\textsc{compl}  3.be-\textsc{dur} \\
\glt ‘A widow, she was widowed’\\
‘Enviuvou, ficou viúva’\\
\z
 
\ea ülepei leha itsagü leha hombe tamitsi\\[.3em]
\gll üle-pe-i	leha	i-tsagü		leha	hombe	tamitsi\\
\textsc{log}-\textsc{ntm}-\textsc{cop} 	\textsc{compl}  3.be-\textsc{cont} 	\textsc{compl}  widow 	longtime\\
\glt ‘After this, she remained widowed for a long time’\\
‘Depois disso, ela ficou viúva por muito tempo’\\
\z

\ea ülepe leha aiha\\[.3em]
\gll üle-pe leha aiha{\footnotemark}{} \\
\textsc{log}-\textsc{ntm} 	\textsc{compl}  	done \\
\glt ‘After this, done’ \\
‘Então, acabou’ (o luto)\\
\footnotetext{The meaning of the particle \emph{aiha} is here roughly translated as ‘done'. \emph{aiha} has a clear discursive function when it is used to close a block or scene of a narrative.}
\z

\ea engü isangatelü leha\\[.3em]
\gll engü	is-anga-te-lü		leha\\
then 	3-jenipa-\textsc{vblz}-\textsc{pnct} 	\textsc{compl} \\
\glt ‘Then she was painted with genipapo’\\
‘Então, ela foi pintada com jenipapo’\\
\z

 \ea isangatelü leha ihombundão{\footnotemark}{} heke isangatelü leha\\[.3em]
\gll is-anga-te-lü		leha	i-hombundaõ 	heke	is-anga-te-lü		leha\\
3-jenipa-\textsc{vblz-pnct} 	\textsc{compl}	3-widow.\textsc{col} 	\textsc{erg} 	3-jenipa-\textsc{vblz}\textsc{-pnct}	\textsc{compl} \\
\glt ‘The brothers of her dead husband painted her’ \\
‘Os irmãos do falecido esposo pintaram-na’\\
\footnotetext{This word could be the result of: \emph{hombe(N)+(C)aõ} (widow+collective). The brothers-in-law of a woman are her potential sexual partners and potential spouses. Here, they are referred to as a group of associated “widowers" and they are responsible for important duties and functions towards the widow.}
\z


\ea lepene leha itsagü\\[.3em]
\gll lepene	leha 	i-tsagü\\
then 	\textsc{compl}  3.be-\textsc{cont}\\
\glt ‘Then, she remained’\\
‘Depois, ela ficou’\\
\z

\ea anhü tülimo heke ijimo\footnote{At morphemic boundaries , the consonant /l/ palatalizes to [ɟ] after the high front vowel /i/.} ijimo ijimo\\[.3em]
\gll anhü	tü-limo	heke	i-limo		i-limo	i-limo\\
son \textsc{refl}-children 	\textsc{erg} 	3-children 3-children 3-children\\
\glt ‘“My dears!" (she said) to her children, children of this size, and this size, and this size'\\ ‘“Meus queridos!”, (disse) para os seus filhos, filhos desse tamanho, desse tamanho e desse tamanho’\\
\z

\ea anhü apa tuhipe kunhigake ouünko tuhipe\\[.3em]
\gll anhü apa	tuhi-pe		ku-ng-ingi-gake{\footnotemark}{}	o-uüN-ko	tuhi-pe\\
son father 	garden-\textsc{ntm} 	1.2-\textsc{obj}-ver-\textsc{imp.ctf}		2-father-\textsc{pl} 	garden-\textsc{ntm}\\
\glt ‘“My dears! Go to see the father’s old garden, your father’s old garden!”’\\
‘“Meus queridos! Vamos ver a roça que era do pai, a roça que era do pai de vocês!”’\\
\footnotetext{The inflectional morphemes of the imperative mood are sensitive to the directional egocentered kind of movement involved: centrifugal imperative (\textsc{imp.ctf}, go to … ), centripetal imperative (\textsc{imp.ctp}, come to … ), and imperative (\textsc{imp}, no movement).}
\z

\ea ouünko tuhipe kunhigake ika kigeke \\[.3em]
\gll o-uüN-ko	tuhi-pe		ku-ng{\footnotemark}-ingi-gake	ika	kigeke	\\
2-father-\textsc{pl} 	garden-\textsc{ntm} 	1.2-\textsc{obj}-see-\textsc{imp.ctf} 		wood 	let’s.go\\
\glt ‘“Let’s go see your father’s old garden, let’s go to cut wood!”’\\
‘“Vamos lá ver a roça que era do pai de vocês, vamos catar lenha!”’\\
\footnotetext{The object marker \emph{ng-}, prefixed to the verbal stem, is the spelled-out trace of the object (patient). Observe that the agent appears in absolutive case. See \citet{Franchetto2010} and \citet{FranchettoSantos2010} for an analysis of this type of construction, which these authors called “de-ergativized", due to a special kind of downgraded transitivity. This construction characterizes relative or focus sentences, where the relativized or the focused argument is the object, as well as some sentences with the verb inflected for imminent future, imperative mood, or hortative mood.}
\z

\ea ehe ijimo telü leha\\[.3em]
\gll ehe	i-limo		te-lü		leha\\
\textsc{itj} 	3-children 	go-\textsc{pnct}	\textsc{compl} \\
\glt ‘“Yes!” her children went away’\\
‘“Sim!”, seus filhos foram’\\
\z

\ea etelüko leha etelüko leha\\[.3em]
\gll e-te-lü-ko	leha	e-te-lü-ko	leha\\
3-go-\textsc{pnct-pl} 	\textsc{compl} 3-go-\textsc{pnct-pl} 	\textsc{compl} \\
\glt ‘They went away, they went away’\\
‘Eles foram, eles foram’\\
\z

\newpage 
\ea eh tigati leha kuigi andati leha \\[.3em]
\gll eh tigati	leha	kuigi	anda-ti		leha\\ 
\textsc{itj} there 	\textsc{compl} garden 	\textsc{loc-ill} 	\textsc{compl}  \\
\glt ‘Yes, right to the garden’\\
‘Sim, direto para a roça’\\
\z

\ea inhünkgo leha  itsuhipüati leha tünho tuhipüati \\[.3em]
\gll i-nhüN-ko	leha		i-tuhi{\footnotemark}-püa{\footnotemark}-ti		 leha	tü-nho		tuhi-püa-ti\\
3.be-\textsc{pnct-pl}	\textsc{compl}  	3-garden-ex.place-\textsc{ill} 	\textsc{compl} \textsc{refl}-husband 	garden-ex.place-\textsc{ill} \\
\glt ‘They reached the place of the father’s old garden’\\
‘Chegaram no lugar que tinha sido a roça dele’\\
\addtocounter{footnote}{-1}
\footnotetext{The consonant /t/ palatalizes to [ts] after the high front vowel /i/ at morphemic boundaries (see \citealt{Franchetto1995})}
\stepcounter{footnote}
\footnotetext{The suffix \emph{-püa} is used to characterize a place where something was previously located (a village, a garden).}
\z

\ea jatsitsü	jatsitsü\\[.3em]
jatsitsü	 jatsitsü \\
poor.man poor.man\\
\glt ‘“Poor man, poor man!” (the widow said)’\\
‘“Coitado, coitado!”, (a viúva disse)’\\
\z

\ea ige inhambalüila tinika ulimo uün etsujenügü uãke nügü iheke\\[.3em]
\gll ige	inhamba-lü-i-la 	tinika	u-limo		uüN etsuje-nügü	uãke 
nügü		i-heke\\
\textsc{prox} 	eat-\textsc{pnct-cop-priv} \textsc{adv} 	1-children 	father die-\textsc{pnct} 		\textsc{pst}
 say.\textsc{pnct}	3-\textsc{erg} \\
\glt ‘“My children’s father died without eating this (the manioc from his garden),” she said’\\
‘“O pai dos meus filhos morreu sem se alimentar disto (da roça dele)”, ela disse’\\
\z

\newpage 
\ea ige inhambangatüingi hõhõ ataiti uãke ulimo uün heke\\[.3em]
\gll ige		inhamba-nga-tüingi	hõhõ	ataiti 	uãke	u-limo		uüN heke\\
\textsc{dem.prox} 	eat-\textsc{avd} 		\textsc{emp} 	\textsc{irr}	\textsc{pst} 	1-children 	father \textsc{erg}\\
\glt ‘“My children’s father could have eaten this”’\\
“O pai dos meus filhos poderia ter se alimentado disto”’\\
\z

\ea iheke uãke tünhope heke{\footnotemark} \\[.3em]
\gll i-heke	uãke	tü-nho-pe		heke \\
ele-\emph{erg}	\emph{pst} 	\textsc{refl}-husband-\textsc{ntm} 	\textsc{erg} \\
\glt ‘“It was him,” (talking) about her dead husband’ \\
‘“Ele, faz tempo”, (falando) do seu falecido esposo’ \\
\footnotetext{See \citet{Franchetto2010} for a description and analysis of the coexisting functions and meanings of the postposition \emph{heke}, as a case (ergative) marker of the external argument of a “transitive verb", and as a perspective locative.}
\z

\ea tita leha inilundagü inilundagü leha\\[.3em]
\gll tita	leha	ini-luN-tagü		ini-luN-tagü		leha\\
there 	\textsc{compl}  cry-\textsc{vblz-dur}		cry-\textsc{vblz}-\textsc{cont} 	\textsc{compl} \\
\glt ‘There she was crying, crying’\\
‘Ficou lá chorando, chorando’\\
\z

\ea inhope tuhi heke isotünkgitsagü\\[.3em]
\gll i-nho-pe	tuhi	heke	is-otüN-ki-tsagü\\
3-husband-\textsc{ntm}	garden 	\textsc{erg} 	3-sorrow-\textsc{caus}-\textsc{dur} \\
\glt ‘The garden of her dead husband was making her deeply sorrowful’ \\
‘A roça do falecido esposo fazia com que ela sentisse muita pena’\\
\z

 \ea ülepe leha etelü indeha eitsüe\\[.3em]
\gll üle-pe		leha		e-te-lü		inde=ha	e-i-tsüe\\
\textsc{log}-\textsc{ntm} \textsc{compl}		3-go-\textsc{pnct} 	here=\textsc{ha} 	2-be-\textsc{imp.pl} \\
\glt ‘Then she went away, “Stay here!” (she said to her own children)’\\
‘Depois ela foi, “Fiquem aqui!”,	(ela falou para seus filhos)’\\
\z

\ea tülimo ngondilü leha iheke \\[.3em]
\gll tü-limo	ngondi-lü	leha	i-heke\\
\textsc{refl}-children 	leave-\textsc{pnct} 	\textsc{compl}	3-\textsc{erg} \\
\glt ‘She left her children there’\\
‘Ela deixou seus filhos lá’\\
\z

\ea ilaha utetai\\[.3em]
\gll ila=ha		u-te-tai\\
there=\textsc{ha} 	1-go-\textsc{fut.im} \\
\glt ‘“I am going that way”’\\
‘“Eu vou para lá”’\\
\z

\ea ilaha nhingadzetai ige nhigüintsai\\[.3em]
\gll ila=ha		ng-ingaNtse-tai	ige	ng-igüiN-tsai\\
there=\textsc{ha} 	\textsc{obj}-look-\textsc{fut.im} 		\textsc{prox} 	\textsc{obj}-surround-\textsc{fut.im} \\
\glt ‘“I’ll have a look and take a walk (around the garden)”’\\
‘“Vou dar uma olhada nisso e vou dar uma volta nisso (na roça)”’\\
\z

\ea lepe leha etelü leha igüinjüi leha\\[.3em]
\gll üle-pe		leha	e-te-lü		leha	iguiN-jü-i		leha\\
\textsc{log}-\textsc{ntm} 	\textsc{compl}	3-go-\textsc{pnct} 	\textsc{compl}	surround-\textsc{pnct-cop} 	\textsc{compl} \\
\glt ‘Then she went away and took a walk around (the garden)’\\
‘Depois ela foi-se e deu uma volta (na roça)’\\
\z

\ea üle hata ah nügü iheke ukugesube ukugesube \\[.3em]
\gll üle	hata	ah	nügü	i-heke	ukuge=sube	ukuge=sube{\footnotemark} \\
\textsc{log} 	when 	\textsc{itj} 	\textsc{pnct}	3-\textsc{erg}	people=\textsc{ep} 	people=\textsc{ep}\\
\glt ‘Meanwhile, “Ah!" she said, “Is it people? Is it people?”’\\
‘Enquanto isso, “Ah!", ela disse, “Será que é gente? Será que é gente?”’\\
\footnotetext{ \emph{sube} and, in the following line, \emph{beki} are clitic particles expressing a feeling of surprise, fear and dramatic curiosity.}
\z 

\ea tübeki ekisei nügü iheke\\[.3em]
\gll tü=beki	ekise-i		nügü	i-heke\\
\textsc{q=ep} 		3.\textsc{dist-cop} 	\textsc{pnct} 	3-\textsc{erg} \\
\glt ‘“Who can that person be?” she said’\\
‘“Quem será aquela pessoa?”, ela disse’\\
\z

\ea lepe leha isinügü leha etuhupüngenügü\\[.3em]
\gll üle-pe		leha	is-i-nügü	leha	et-uhupünge-nügü\\
\textsc{log}-\textsc{ntm}	\textsc{compl}	3-come-\textsc{pnct} 	\textsc{compl}	3\textsc{dtr}-disguise-\textsc{pnct} \\
\glt ‘Then she came back and disguised herself’\\
‘Depois ela voltou e disfarçou’\\
\z

\ea ijopenümi leha ihüsoho einhügü leha\\[.3em]
\gll i-lope-nümi leha	i-hüsoho	ei-nhügü	leha\\
3-come.toward-\textsc{pnct.cop}	\textsc{compl}	3-mother.in.law 	come-\textsc{pnct}	\textsc{compl} \\
\glt ‘Her mother-in-law came toward her’\\
‘A sogra vinha em sua direção’\\
\z

\ea ijopenümi leha kagahuku akata leha\\[.3em]
\gll i-lope-nümi	leha		kagahuku	akata		leha\\
3-come.toward-\textsc{pnct.cop}	\textsc{compl}	fence 	along.inside 	\textsc{compl} \\
\glt ‘Toward her along on the inside of the fense’\\
‘Na direção dela acompanhando a cerca (da roça) por dentro’\\
\z

\ea aka nügü iheke uãki eitsako nügü iheke\\[.3em]
\gll aka	nügü	i-heke	uã-ki	e-i-tsa-ko	nügü	i-heke\\
\textsc{itj}	\textsc{pnct}	3-\textsc{erg}	\textsc{q-ins}	2-be-\textsc{dur-pl} 	\textsc{pnct}	3-\textsc{erg} \\
\glt ‘“Wow!” she (the mother-in-law) said to her: “What are you doing here?” she said to her’\\
‘“Nossa!”, ela disse: “O que vocês estão fazendo aqui?”, ela disse’\\
\z

\ea eh nügü iheke inde muke utetagü emuguko tuhipe ingiale\\[.3em]
\gll eh	nügü	i-heke	inde	muke	u-te-tagü	e-mugu-ko	tuhi-pe	ingi-ale\\
\textsc{itj}	\textsc{pnct}	3-\textsc{erg}   here	\textsc{ep}	1-go-\textsc{dur} 	2-son-\textsc{pl} 	garden-\textsc{ntm} 	see-while\\
\glt ‘“Yeah!” she (the woman) said to her, “I'm walking here looking at your son’s old garden”’\\
‘“Sim!” ela disse “estou indo por aqui olhando a roça que foi do teu filho”’\\
\z

\ea uinilale nügü iheke uinilale utetagü inde\\[.3em]
\gll u-inilu-ale	nügü	i-heke	u-inilu-ale	u-te-tagü	inde\\
1-cry-while 	\textsc{pnct}	3-\textsc{erg}	1-cry-while 	1-go-\textsc{dur}	here\\
\glt ‘“Crying,” she said to her, “I’m going here crying”’\\
‘“Chorando”, ela lhe disse, “estou indo por aqui chorando”’\\
\z

\largerpage
\ea ehẽ nügü iheke einilundako kahegei\\[.3em]
\gll ehẽ	nügü	i-heke	e-iniluN-ta-ko=kaha	ege-i\\
\textsc{itj} 	\textsc{pnct}	3-\textsc{erg}	2-cry-\textsc{dur}-\textsc{pl}=\textsc{ep}	\textsc{dist-cop} \\
\glt ‘“Yes,” she (the mother-in-law) said, “you are really crying”’\\
‘“Sim”, ela (sogra) disse, “você está chorando mesmo”’\\
\z

\ea elimo uünkoi ailene inatagü{\footnotemark} \\[.3em]
\gll e-limo		uüN-ko-i	ailene	inata-gü \\
2-children	father-\textsc{pl-cop}	feast	nose-\textsc{poss} \\
\glt ‘“Your children’s father used to be the first of the feast"' \\
‘“O pai dos seus filhos era sempre o primeiro da festa”’\\
\footnotetext{To be the first one is expressed as ‘to be the nose (of something)’: here, the nose of the feast.}
\z

\ea kogetsi epetsakilü kogetsi epetsakilü\\[.3em]
\gll kogetsi		epetsaki-lü	kogetsi		epetsaki-lü\\
tomorrow	adorn-\textsc{pnct}	tomorrow 	adorn-\textsc{pnct}\\
\glt ‘“One day he adorned himself and the other day he adorned himself (also)”’\\
‘“Um dia se enfeita, outro dia se enfeita (também)”’\\
\z

\ea ilango gitse elimo uünkoi\\[.3em]
\gll ila-ngo		gitse	e-limo		uüN-ko-i\\
there-\textsc{nmlz}	\textsc{ep}	2-children	father-\textsc{pl}-\textsc{cop} \\
\glt “Your children’s father was so”\\
‘“O pai dos seus filhos era assim”’\\
\z

\ea tingakugui gitse etengatohokoi inhaka nügü iheke\\[.3em]
\gll tingakugu-i	gitse	e-te-nga-toho-ko-i		i-ngaka	nügü	i-heke \\
weeping-\textsc{cop}	\textsc{ep}	2-go-\textsc{hab-insnmlz}-\textsc{pl-cop}	3-instead.of	\textsc{pnct}	3-\textsc{erg} \\
\glt ‘“You will always be weeping for him,” she (the mother-in-law) said’\\
‘“Você ficará sempre lamentando por ele”, ela (a sogra) disse’\\
\z

\ea ehẽ nügü iheke\\[.3em]
\gll ehẽ	nügü	i-heke\\
\textsc{aff}	\textsc{pnct}	3-\textsc{erg}\\
\glt ‘“Yes,” she (the woman) said’\\
‘“Sim”, ela (a mulher) disse’\\
\z

\largerpage[2]
\ea lepe inho ügühütuki leha isakihata leha iheke\\[.3em]
\gll üle-pe		i-nho		ügühütu-ki	leha	is-aki-ha-ta		leha	i-heke\\
\textsc{log}-\textsc{ntm}	3-husband 	way.being-\textsc{ins}	\textsc{compl}	3-word-\textsc{vblz}-\textsc{dur}	\textsc{compl}	3-\textsc{erg}\\
\glt ‘Then she was telling her about her husband's  way of being’\\
‘Depois ela ficou lhe contando sobre o jeito de ser do seu esposo’ 
\z

\newpage 

\ea kigekeha\\[.3em]
\gll kigeke=ha\\
1.2.go.\textsc{imp}=\textsc{ha}\\
\glt ‘“Let’s go!”’\\
‘“Vamos!”’
\z


\ea aminga akatsange uenhümingo eitigini einhani nügü iheke aminga \\[.3em]
\gll aminga	akatsange	u-e-nhümingo	e-itigi-ni	e-inha-ni	nügü  i-heke aminga\\
other.day	\textsc{int}		1-come-\textsc{fut}	2-\textsc{fin}-\textsc{pl}		2-\textsc{dat}-\textsc{pl}	\textsc{pnct}	3-\textsc{erg} other.day \\
\glt ‘“The day after tomorrow I’ll come to you, to get you,” she (the mother-in-law) said to her, “the day after tomorrow”’\\
“Depois de amanhã eu voltarei para vocês, para buscar vocês”, ela (sogra) disse para ela, “depois de amanhã”\\
\z

\ea kekeha egetüeha \\[.3em]
\gll keke=ha	egetüe=ha \\
1.2.go=\textsc{ha}	go.\textsc{imp.pl=ha} \\
\glt ‘“Let’s go,” (the woman said). “You can go!” (the mother-in-law replied)’ \\
‘“Vamos!”, (disse a mulher) “Podem ir”, (respondeu a sogra)’ \\
\z

\ea etelü hõhõ ihüsoho telü leha anha telü leha{\footnotemark} \\[.3em]
\gll e-te-lü		hõhõ	i-hüsoho	te-lü		leha	anha	te-lü	 	leha \\
3-go-\textsc{pnct} 	\textsc{emph}	3-mother.in.law	go-\textsc{pnct}	\textsc{compl}	dead	go-\textsc{pnct}	\textsc{compl} \\
\glt ‘She went away, her mother-in-law went away, the dead one went away (but would be back soon)’ \\
‘Ela foi embora, a sogra foi embora, a morta foi embora (mas iria voltar logo)’ \\
\footnotetext{As explained by the Kuikuro speakers, when someone says “\emph{etelü hõhõ}” (she/he went away \textsc{emphatic}), they are speaking about another person who went away with the intention of returning soon, the same day or the following day; when someone says “\emph{etelü leha}” (she/he went away \textsc{completive}), they are speaking about someone who went away not knowing if they would come back.}
\z

\newpage 
\ea anha hekisei ihoginhi leha isipe ihüsohope \\[.3em]
\gll anha=ha	ekise-i		i-hogi-nhi	leha	isi-pe		i-hüsoho-pe \\
dead=\textsc{ha} 	3.\textsc{dist-cop} 	3-find-\textsc{anmlz}	\textsc{compl}	mother-\textsc{ntm} 	3-mother.in.law-\textsc{ntm} \\
\glt ‘It was the dead one that found her, the one who had been the mother (of her husband), her deceased mother-in-law’ \\
‘Era a morta aquela que a encontrou, a que tinha sido a mãe (do seu esposo), sua finada sogra’ \\
\z

\ea lepe leha umm igiataka tünkgülü ihanügü iheke{\footnotemark} \\[.3em]
\gll üle-pe		leha	umm	igia=taka	t-ünkgü-lü		iha-nügü	i-heke \\
\textsc{log-ntm} 	\textsc{compl}	\textsc{itj} 	so=\textsc{ep} 		\textsc{refl}-sleep-\textsc{pnct}	tell-\textsc{pnct} 	3-\textsc{erg} \\
\glt ‘Then, (the woman remembered): “Umm, that was how she promised to sleep"’\\
‘Depois (a mulher lembrou): “Umm, foi assim que ela prometeu dormir”’ \\
\footnotetext{The storyteller showed the fingers of her hand counting the number three; to ‘sleep three' means a three day period of time.}
\z

\ea lepe leha anhü ika kigeke \\[.3em]
\gll üle-pe		leha	anhü	ika	kigeke \\
\textsc{log-ntm}	\textsc{compl}	son 	wood 	1.2.go.\textsc{imp} \\
\glt “Then, (she said): “My dears! Let’s go collect wood!” \\
‘Depois, (ela disse): “Queridos! Vamos buscar lenha!” \\
\z

\ea etelü leha{\footnotemark}{} \\[.3em]
\gll e-te-lü		leha \\
3-go-\textsc{pnct} 	\textsc{compl} \\
\glt‘She went away’ \\
‘Ela foi embora’ \\
\footnotetext{In a narrative, the expression of movement with the verb ‘go’ (root \emph{te}) often means the movement of the narrative itself, from one scene to the following one. }
\z

\newpage 
\ea etsutühügü ati leha inhügü \\[.3em]
\gll etsu-tühügü		ati	leha	i-nhügü \\
promise.return-\textsc{prf} 	when	\textsc{compl}	be-\textsc{pnct} \\
\glt ‘Then came the day she had promised to return’ \\
‘Chegou o dia em que ela tinha prometido voltar’ \\
\z

\ea indeha eitsüe tülimo ngondingalü leha iheke inhondingalüko \\[.3em]
\gll inde=ha	e-i-tsüe	tü-limo	ngondi-nga-lü		leha	i-heke  i-ngondi-nga-lü-ko \\
here=\textsc{ha}	2-be-\textsc{imp.pl} 	\textsc{refl}-children 	leave-\textsc{hab-pnct}	\textsc{compl}	3-\textsc{erg} 3-leave-\textsc{hab-pnct-pl} \\
\glt ‘“Stay here!" She used to leave her children, she often leaves them’ \\
‘“Fiquem aqui!” Ela costumava deixar seus filhos, ela costumava deixá-los' \\
\z

\ea indeha amanhetüe \\[.3em]
\gll inde=ha	amanhe-tüe \\
here=\textsc{ha} 	play-\textsc{imp.pl} \\
\glt ‘“Play here!”’ \\
‘“Brinquem por aqui!”’ \\
\z

\ea lepe leha \\[.3em]
\gll üle-pe		leha \\
\textsc{log-ntm}	\textsc{compl} \\
\glt ‘After this’ \\
‘Depois disso’ \\
\z

\ea etelü leha egei isinümbata gehale ihüsoho enhümbata \\[.3em]
\gll e-te-lü		leha	ege-i		is-i-nüN-hata		gehale	i-hüsoho 	eN-nhüN-hata \\
3-go-\textsc{pnct} 	\textsc{compl}	\textsc{dist-cop}	3-be-\textsc{pnct}-when		also 3-mother.in.law	come-\textsc{pnct}-when \\
\glt ‘She went away when she (the mother-in-law) was coming back again’ \\
‘Foi, quando ela (a sogra) estava vindo novamente’ \\
\z

\newpage 
\ea inhalü leha ingügijüi leha hüle iheke \\[.3em]
\gll inhalü{\footnotemark}{}	leha	ingügi-jü-i		leha	hüle	i-heke \\
\textsc{neg}	\textsc{compl}	give.lap-\textsc{pnct-cop} 	\textsc{compl}	\textsc{cntr}	3-\textsc{erg} \\
\footnotetext{Kuikuro has two free grammatical forms to mark negation having scope on verbal or nominal phrases: \emph{inhalü} is a kind of weak negation, and it always occurs with the non-verbal copula \emph{-i} suffixed to the negated verb or nominal. \emph{ahütü} is for stronger negations and it always occurs with the privative  \emph{-la} suffixed to the negated verb or nominal.}
\glt ‘However, she (the mother-in-law) did not circle around anymore’ \\
‘Ela (a sogra), porém, não deu mais voltas’ \\
\z

\ea uhunügü leha iheke \\[.3em]
\gll uhu-nügü	leha	i-heke \\
know-\textsc{pnct} 	\textsc{compl}	3-\textsc{erg} \\
\glt ‘She (the woman) already knew’ \\
‘Ela (a mulher) já sabia’ \\
\z

\ea lepe leha isinügü \\[.3em]
\gll üle-pe	leha	is-i-nügü \\
then	\textsc{compl}	3-be-\textsc{pnct} \\
\glt ‘Then, she (the mother-in-law) arrived’ \\
‘Então, ela (a sogra) chegou’ \\
\z

\ea ande taka uetsagü akihalükoinha \\[.3em]
\gll ande	taka	u-e-tsagü	akiha-lü-ko-inha \\
here	\textsc{ep}	1-arrive-\textsc{prog}	2.tell-\textsc{pnct-pl-dat} \\
\glt ‘“Here I come to warn you” (the mother-in-law said)’ \\
‘“Aqui chego para avisá-los” (disse a sogra)’ \\
\z

\ea aminga akatsange elimo uünko telüingo haguna aminga \\[.3em]
\gll aminga	akatsange	e-limo		uüN-ko	te-lü-ingo	hagu-na aminga \\
other.day 	\textsc{int}		2-children 	father-\textsc{pl} 	go-\textsc{pnct}-\textsc{fut} 	bayou-\textsc{all} 
other.day \\
\glt ‘“The day after tomorrow your children’s father will go on a fishing trip”’ \\
‘“Depois de amanhã o pai de seus filhos irá para a pescaria”’ \\
\z

\newpage 
\ea ülegote akatsange uenhümingo akihalükoinha ülegote \\[.3em]
\gll üle-gote	akatsange	u-e-nhümingo	akiha-lü-ko-inha	üle-gote \\
\textsc{log}-when	\textsc{int} 		1-arrive-\textsc{pnct.fut}	tell-\textsc{pnct-pl-dat}	\textsc{log}-when \\
\glt ‘“When this happens, I’ll come to warn you”’ \\
‘“Quando isso acontecer, eu virei avisá-la”’ \\
\z

  
\ea eitigini etimbelüko ingitomi \\[.3em]
\gll e-itigi-ni	etimbe-lü-ko	ingi-tomi \\
2-\textsc{fin-pl}		arrive-\textsc{pnct-pl}	see-\textsc{purp} \\
\glt ‘“To pick you up, so you’ll see their arrival (from the fishing trip)”’ \\
‘“Para buscar você, de modo que veja a chegada deles (da pescaria)”’ \\
\z

\ea esepe kae akatsange uenhümingo eitigini{\footnotemark}  \\[.3em]
\gll ese-pe		kae	akatsange	u-e-nhümingo		e-itigi-ni \\
3.\textsc{prox-ntm} 	\textsc{loc}	\textsc{int} 		1-arrive-\textsc{pnct.fut}	2-\textsc{fin-pl} \\
\glt ‘“After this (after four days), I’ll come back to pick you up”’ \\
‘“Depois deste (depois de quatro dias), eu virei lhe buscar”’ \\
\footnotetext{\emph{Esepe kae} (after this one): the mother-in-law counts three fingers and indicates the fourth finger, the one after the third finger. The system of Kuikuro numerals is base 5, with counting beginning with the thumb and progressing to the little finger, thus completing a unit of 5. From 6 to 9 the numbers from 1 to 4 are repeated with the addition of the expression “made to cross to the other side". The number 10 closes another unit of 5. The same logic operates for the numbers from 11 to 15 (on the foot) and from 16 to 20 (crossing to the other side [the other foot]).}
\z

\ea isakihatagü gehale üleki ügühütuki hagunaha etelükoingoki \\[.3em]
\gll is-akiha-tagü	gehale	üle-ki		ügühütu-ki	hagu-na=ha	e-te-lü-ko-ingo-ki \\
3-tell-\textsc{dur}	also	\textsc{log-ins}	custom-\textsc{ins}	bayou-\textsc{all=ha}	3-go-\textsc{pnct-fut-ins} \\
\glt ‘She (the mother-in-law) was also telling about that, about their way of being, about how they go on fishing trips’ \\
‘Ela (a sogra) também ficou contando sobre aquilo, sobre o jeito deles, sobre como eles vão para a pescaria’ \\
\z

\largerpage[2]
\ea kekeha nügü iheke etelüko leha \\[.3em]
\gll keke=ha	nügü	i-heke	e-te-lü-ko	leha \\
1.2.\textsc{imp=ha}	\textsc{pnct}	3-\textsc{erg}	3-go-\textsc{pnct-pl}	\textsc{compl} \\
\glt ‘“Let’s go!” she said, they went away’ \\
‘“Vamos!”, ela disse e foram embora’ \\
\z


\ea tetingugi leha atahaingalüko \\[.3em]
\gll tetingugi	leha	at-ahaiN-nga-lü-ko \\
each		\textsc{compl}	3\textsc{dtr}-separate-\textsc{hab-pnct-pl} \\
\glt ‘They separated, each walking to a different side at the same time’ \\
‘Separaram-se, cada uma indo para um lado ao mesmo tempo’ 
\z

 
\ea lepe leha igia isünkgüpügü atai \\[.3em]
\gll üle-pe		leha	igia	is-ünkgü-pügü	atai \\
\textsc{log-ntm}	\textsc{compl}	so	3-sleep-\textsc{prf} 		when \\
\glt ‘After sleeping this way’ (3){\footnotemark}{} \\
‘Depois de ter dormido assim’  (3) \\
\footnotetext{\emph{Ájahi} indicates with the hand the number 3, meaning that three days elapsed.}
\z

\ea ika kigeke ailehüle üle tohoingo hüle \\[.3em]
\gll ika	kige-ke		aileha=hüle	üle-toho-ingo		hüle \\
wood 	go-\textsc{imp} 		\textsc{compl}=\textsc{cntr}		\textsc{log-insnmlz-fut} \textsc{cntr} \\
\glt ‘“Let’s go to collect wood!” It will be then that it happened’ \\
‘“Vamos buscar lenha!” Será naquele momento que acontecerá’ \\
\z

\ea ijopenümi gehale isinügü \\[.3em]
\gll ijope-nümi		gehale	is-i-nügü \\
come.toward-\textsc{pnct.cop}	also	3-be-\textsc{pnct} \\
\glt ‘Once more she (the mother-in-law) came toward her’ \\
‘Novamente ela (a sogra) veio em sua direção’ \\
\z

\ea andetaka uetsagü akihalükoinha \\[.3em]
\gll ande=taka	u-e-tsagü	akiha-lü-ko-inha \\
now=\textsc{ep} 	1-come-\textsc{dur}	tell-\textsc{pnct-pl-dat} \\
\glt ‘“I came here today to warn you”’  \\
‘“Eu vim lhe avisar”’ \\
\z

\ea ehe elimo uünko akatsange leha haguna ihaki nügü iheke \\[.3em]
\gll ehe	e-limo		uüN-ko	akatsange	leha	hagu-na	ihaki 	 nügü	i-heke \\
\textsc{itj}	2-children	father-\textsc{pl} \textsc{int} \textsc{compl}	bayou-\textsc{all} 	far 	 
\textsc{pnct} 	3-\textsc{erg} \\
\glt ‘“All right, your children’s father (is) really far away on the fishing trip,” she said to her’ \\
‘“Certo, o pai dos seus filhos (está) mesmo longe na pescaria”, disse para ela’ \\
\z


\ea igia akatsange isüngülüko inhatüi inhatüi \\[.3em]
\gll igia	akatsange	is-üngü-lü-ko		inhatüi	inhatüi \\
so	\textsc{int} 		3-sleep-\textsc{pnct-pl} 	five 		five \\
\glt ‘“They will sleep five (days), five (during the fishing trip)”’ \\
‘“Eles vão dormir cinco (dias), cinco (na pescaria)”’ \\
\z

\ea esepe kae akatsange uenhümingo eitigini \\[.3em]
\gll ese-pe		kae	akatsange	u-e-nhümingo		e-itigi-ni \\
3.\textsc{prox-ntm}	\textsc{loc} 	\textsc{int}		1-come-\textsc{pnct.fut} 	2-\textsc{fin-pl} \\
\glt ‘“One after this, I’ll really come back to pick you up”’ \\
“Um depois deste, eu voltarei para lhe buscar”’ \\
\z

\ea ingigokomi hõhõ ehekeni \\[.3em]
\gll ingi-gokomi	hõhõ	e-heke-ni \\
3.see-\textsc{purp.pl} 	\textsc{emph}	2-\textsc{erg-pl} \\
\glt ‘“For you to see them”’ \\
‘“Para você vê-los”’ \\
\z

\ea igiaha tingakügüi etelüko hinhe \\[.3em]
\gll igia=ha		tingakügü-i	e-te-lü-ko=hinhe \\
so=\textsc{ha} 		weeping-\textsc{cop} 	2-go-\textsc{pnct-pl=npurp} \\
\glt ‘“So you do not go around weeping for him”’ \\
‘“Para você não mais andar pranteando por ele”’ \\
\z

\ea elimo uünko hüngüngü igelü hinhe ehekeni \\[.3em]
\gll e-limo		uüN-ko	hüngüngü	ige-lü=hinhe		e-heke-ni \\
2-children	father-\textsc{pl}	feeling.lack 	carry-\textsc{pnct=npurp} 	2-\textsc{erg-pl} \\
\glt ‘“For you to no longer carry such nostalgia for your children’s father” \\
‘“Para você não carregar mais a saudade do pai dos seus filhos”’ \\
\z

\ea epetsakinge inatagü sogitse tingakügüi gitse etengatohokoi gitse elimo uünko nügü iheke \\[.3em]
\gll epetsaki-nge	inata-gü	sogitse	tingakügü-i	gitse e-te-nga-toho-ko-i		gitse	e-limo	uüN-ko	nügü	i-heke \\
adorn-\textsc{nmlz}	noose-\textsc{poss} 	\textsc{em}	weeping-\textsc{cop}	\textsc{em} 2-go-\textsc{hab-insnmlz-pl-cop} 	\textsc{em}	2-children 	father-\textsc{pl} 	\textsc{pnct}	3-\textsc{erg} \\
\glt ‘“He is the first to adorn himself; don’t go around weeping for your children’s father,” she said to her’ \\
‘“Ele é o primeiro a se enfeitar; não ande sempre por aí pranteando o pai dos seus filhos”, ela lhe disse’ \\
\z

 
\ea tatoho ingakatalü sogitse elimo uünkoi gitse esei tingakugui eigengalüko heke \\[.3em]
\gll t-atoho	 ingakata-lü sogitse e-limo uüN-ko-i	gitse ese-i tingakügü-i e-ige-nga-lü-ko heke \\
\textsc{refl}-other.wife love-\textsc{pnct} em 2-children father-\textsc{pl-cop} em 3.\textsc{prox-cop} weeping-\textsc{cop} 2-take-\textsc{hab-pnct-pl} \textsc{erg} \\
\glt ‘“He makes love with his other wife, your children’s father is like this, while you are weeping”’\\
‘“Ele namora com a sua outra esposa (do outro mundo), assim é o pai dos seus filhos, enquanto você fica chorando”’ \\
\z


\ea tahekasasai gele tatoho itinhündelü heke \\[.3em]
\gll tahekasasa-i	gele	t-atoho		itinhünde-lü		heke \\
lying-\textsc{cop}	still	\textsc{refl}-other.wife	lie.hammock-\textsc{pnct}	\textsc{erg} \\
\glt ‘“He is always lying down in the hammock with his other wife”’ \\
‘“Ele está sempre deitado na rede com a sua outra esposa”’ \\
\z

\ea ilango gitse elimo uünkoi tühünitati ehekeni \\[.3em]
\gll ila-ngo		gitse	e-limo		uüN-ko-i	tü-hüni-tati	e-heke-ni \\
there-\textsc{nmlz}	\textsc{ep} 	2-children 	father-\textsc{pl-cop}	\textsc{refl}-feel.lack-? 	2-\textsc{erg-pl} \\
\glt ‘“There, your children’s father is like this, the one that you are missing”’ \\
‘“Lá, o pai dos seus filhos é assim, aquele de quem você sente falta”’ \\
\z

\ea eijatongoko itinhündelü heke gitse elimo uünko itsagü nügü iheke \\[.3em]
\gll eijatongo-ko	itinhünde-lü	heke	gitse	e-limo		uüN-ko i-tsagü	nügü i-heke \\
2nd.wife-\textsc{pl}	lie-\textsc{pnct}	\textsc{erg} 	\textsc{ep} 	2-children 	father-\textsc{pl} be-\textsc{dur} 	\textsc{pnct} 3-\textsc{erg} \\
\glt ‘“Your children’s father always lies down in the hammock with his second wife,” she said to her’ \\
‘“O pai dos seus filhos fica sempre deitado na rede com a sua segunda esposa”, ela disse’\\
\z

\newpage 
\ea üle heke leha  ihitsü ingüngingükijü \\[.3em]
\gll üle	heke	leha	i-hi-tsü	ingüN-ki-nguN-ki-jü \\
\textsc{log} 	{erg}	\textsc{compl}	3-wife-\textsc{poss}	eye-\textsc{ins-vblz-ins-pnct} \\
\glt‘This made his wife think’ \\
‘Isso fez a esposa dele pensar’ \\
\z

\ea lepe leha etelüko leha etelü leha \\[.3em]
\gll üle-pe	leha	e-te-lü-ko	 leha	e-te-lü		leha \\
\textsc{log-ntm}	\textsc{compl}	3-go-\textsc{compl-pl}	\textsc{compl}	3-go-\textsc{pnct}	\textsc{compl} \\
\glt ‘After this, they went away, she went away’ \\
‘Depois, elas foram, ela foi’ \\
\z

\ea lepe leha engü leha ijimo telü leha haguna leha \\[.3em]
\gll üle-pe	leha	engü	leha 	i-limo		 te-lü	leha	hagu-na	leha\\
\textsc{log-ntm}	\textsc{compl}	 then	\textsc{compl}	3-children 	go-\textsc{pnct} 	\textsc{compl}	bayou-\textsc{all} 	\textsc{compl}\\
\glt ‘Then, her (mother-in-law's) children went on the fishing trip’ \\
‘Então, os filhos dela (da sogra) foram para a pescaria’ \\
\z
 
\ea igeitaka isünkgülüingo igeitaka \\[.3em]
\gll ige-i=taka	is-ünkgü-lü-ingo	ige-i=taka \\
\textsc{prox-cop=ep} 	3-sleep-\textsc{pnct-fut}	\textsc{prox-cop=ep} \\
\glt ‘“Will she sleep so many nights?”’{\footnotemark} \\
‘“Será que ela irá dormir estas tantas noites?”’ \\
\footnotetext{The widowed woman asks herself how many days will pass until her mother-in-law comes back to get her.}
\z

\ea esepe kae itigi hüle isinügü \\[.3em]
\gll ese-pe		kae	itigi	hüle	is-i-nügü \\
\textsc{3.prox-ntm} 	\textsc{loc} 	to.seek	\textsc{cntr} 	3-come-\textsc{pnct} \\
\glt ‘But exactly on this day she (the mother-in-law) arrived to get her’ \\
‘Mas exatamente neste dia ela (a sogra) veio buscá-la’ \\
\z

\ea kekegeha nügü iheke \\[.3em]
\gll kekege=ha	nügü	i-heke \\
let’s.go=\textsc{ha}	\textsc{pnct}	3-\textsc{erg} \\
\glt ‘“Let’s go,” she said to her’ \\
‘“Vamos!”, ela lhe disse’ \\
\z

\ea ande akatsege elimo uünko etimbelüingo nügü iheke \\
\gll ande	akatsege	e-limo		uüN-ko	etimbe-lü-ingo	nügü	i-heke \\
now	\textsc{int}		2-children	father-\textsc{pl}	come-\textsc{pnct-fut} 	\textsc{pnct} 	3-\textsc{erg} \\
\glt ‘“Today your children’s father will come,” she said to her’ \\
‘“Hoje o pai dos seus filhos vai chegar”, disse a ela’ \\
\z

\ea lepe leha etelü leha \\[.3em]
\gll üle-pe		leha	e-te-lü		leha \\
\textsc{log-ntm}	\textsc{compl} 3-go-\textsc{pnct}	\textsc{compl} \\
\glt ‘Then she went away’ \\
‘Depois disso ela foi’ \\
\z

\ea inde atsange eitsüe nügü iheke tülimo heke leha \\[.3em]
\gll inde	atsange	e-i-tsüe	nügü	i-heke	tü-limo	heke	leha \\
here 	\textsc{int}		2-be-\textsc{imp.pl} 	say 	3-\textsc{erg}	\textsc{refl}- children \textsc{erg} 	\textsc{compl} \\ 
\glt ‘“Stay here!” she (the woman) said to her own children’ \\
‘“Fiquem aqui!”, ela (a mulher) disse para os seus filhos’ \\
\z

\ea ilá kohõtsige utehesundagü ige nügü iheke \\[.3em]
\gll ilá	kohõtsige	u-tehesuN-tagü	ige	nügü	i-heke \\
there  	walking.little 	1-walk-\textsc{prog} 		\textsc{prox}	say	3-\textsc{erg} \\
\glt ‘“I’m going there walking a little,” she said to them’ \\
‘“Eu vou para lá, passear um pouco”, ela disse para eles’ \\
\z

\ea lepe leha etelüko leha \\[.3em]
\gll üle-pe		leha	e-te-lü-ko	leha \\
\textsc{log-ntm}	\textsc{compl} 3-go-\textsc{pnct-pl}	\textsc{compl} \\
\glt ‘Then, they went away’ \\
‘Depois disso, eles foram’ \\
\z

\ea hakitsetse leha atamini ngika ngondilü ihekeni osiha ina eitsüe \\[.3em]
\gll haki-tsetse	leha	ata-mini	ngika		ngondi-lü	i-heke-ni osi=ha	ina e-i-tsüe \\
far-\textsc{dim}		\textsc{compl}	when-\textsc{pl} 	them	leave-\textsc{pnct} 	3-\textsc{erg-pl} well=\textsc{ha} 	here 2-be-\textsc{imp.pl} \\
\glt ‘When they were not so far from the village, they left them (the children): “Stay here!”’ \\
‘Quando estavam pouco longe da aldeia, os (filhos) deixaram: “Fiquem aqui!”’
\z



\ea tüatsagati leha tüilü iheke \\[.3em]
\gll tü-atsagati	leha	tüi-lü		i-heke \\
\textsc{refl}-in.front	\textsc{compl}	put-\textsc{pnct} 	3-\textsc{erg} \\
\glt ‘She (the mother-in-law) put her in front of herself’ \\
‘Ela (a sogra) a colocou em sua frente’ \\
\z

 
\ea itükanhenügü letüha iheke itükanhenügü itükanhenügü \\[.3em]
\gll itüka-nhe-nügü	üle=tü=ha	i-heke itüka-nhe-nügü itüka-nhe-nügü \\
3.move.up-\textsc{tr-pnct} 	\textsc{log=uncr}=\textsc{ha}	3-\textsc{erg} 3.move.up-\textsc{tr-pnct} 3.move.up\textsc{-tr-pnct} \\
\glt ‘Then, she moved her up, she moved her up, she moved her up’ \\
‘Ela a fez subir, fez subir, fez subir’{\footnotemark} \\
\footnotetext{\emph{tü(ha)} and \emph{tsügü(ha)} are clitics, epistemic/evidentials (EM), whose meaning is approximately: ‘I, the speaker, cannot assume the truth of this information; people say it happened'.}
\z

\ea inegetüha ihhh kahü ijatüna tsitsi letsügüha inhünkgo leha \\[.3em]
\gll inege=tü=ha		ihhh	kahü	ijatü-na	tsitsi	üle=tsügü=ha	i-nhüN-ko	leha \\
this.side=\textsc{uncr=ha} 	\textsc{id} 	sky 	armpit-\textsc{al} 	almost 	\textsc{log}=\textsc{uncr=ha} 3-be-\textsc{pnct-pl} \textsc{compl} \\
\glt ‘On this side, \textit{ihhh}, they reached almost to the limit of the sky (with the earth) and (there) they stayed’ \\
‘Deste lado, \textit{ihhh}, chegaram quase no limite do céu (com a terra) e (lá) ficaram’ \\
\z

\ea itükainjüko leha osiha etimükeĩtüe ah etimükeĩtüe \\[.3em]
\gll itükaiN-jü-ko		leha	osi=ha		et-imükeĩ-tüe ah	et-imükeĩ-tüe	 \\	
3-move.up-\textsc{pnct-pl} 	\textsc{compl}	well=\textsc{ha} 	2.\textsc{dtr}-turn.face-\textsc{imp.pl} \textsc{itj} 	2.\textsc{dtr}-turn.face-\textsc{imp.pl} \\
\glt ‘They moved up: “Well, turn your face (down), ah, turn your face (down)!” \\
‘Elas subiram; “Vire o rosto (para baixo), vire o rosto (para baixo)!” \\
\z

\ea tetimükeĩtü eh humbungaka leha \\[.3em]
\gll t-et-imükeĩ=tü			eh	humbungaka leha \\
\textsc{ptcp-dtr}-turn.face.\textsc{ptcp=uncr}	\textsc{itj} 	upside.down \textsc{compl} \\
\glt ‘With the face turned down, upside down’ \\
‘Com o rosto virado (para baixo), de cabeça para baixo’ \\
\z


\ea engü atühügü leha ngongoho atühügü leha \\[.3em]
\gll engü	a-tühügü	leha	ngongo-ho 	a-tühügü	leha \\
then	be-\textsc{prf}		\textsc{compl}	earth\textsc{-loc}	be-\textsc{prf}		\textsc{compl} \\
\glt ‘Then, the earth appeared (upside down)’ \\
‘Então, a terra apareceu (de cabeça para baixo)’ \\
\z

\ea ama üntepügü leha kahü alüpengine ige ige ugupongaha \\[.3em]
\gll ama	ünte-pügü	leha	kahü	alüpengine	ige 	ugupo-nga=ha \\
way 	down-\textsc{prf}	\textsc{compl}	\textsc{sky} 	\textsc{ine} 		\textsc{prox} 	above-all=\textsc{ha} \\
\glt ‘The way down from the sky to above here’ \\
‘O caminho que desce do céu até em cima daqui’ \\
\z

\ea igeha kungongogu uguponga leha \\[.3em]
\gll ige=ha ku-ngongo-gu ugupo-nga leha \\
\textsc{prox=ha} 1.2{\footnotemark}-earth-\textsc{poss} above-\textsc{all} \textsc{compl} \\
\glt ‘Here above our earth’ \\
‘Aqui em cima da nossa terra’ \\
\z

\ea lepe leha etelüko \\[.3em]
\gll lepe	leha	e-te-lü-ko \\
then	\textsc{compl} 3-go-\textsc{pnct-pl} \\
\glt ‘Then, they went away’ \\
‘Depois disso, elas foram’ \\
\z

\ea anha engübeha tanginhügü ẽgipügati leha inhügü \\[.3em]
\gll anha	engübeha	tanginhü-gü	ẽgipügati	leha	i-nhügü \\
dead 	\textsc{ep} 		main.path-\textsc{poss} 	at.top.head 	\textsc{compl}	be-\textsc{pnct} \\
\glt ‘They stayed right at the beginning of the main path of the dead’ \\
‘Ficaram bem no começo do caminho principal dos mortos’ \\
\z

\newpage 
\ea üle ama gae geletügüha inginügü iheke \\[.3em]
\gll üle	ama	gae	geletügü=ha	ingi-nügü	i-heke \\
\textsc{log} 	path 	on.edge	?=\textsc{ha} 		3.bring-\textsc{pnct}	3-\textsc{erg} \\
\glt ‘She (the mother-in-law) brought her right to the edge of the path’ \\
‘Ela (a sogra) a trouxe bem na beira do caminho’ \\
\z


\ea itsapügü itahiale leha itsapügü ingitüingiha anha heke\\[.3em]
\gll i-tapü-gü	itahi-ale	leha	i-tapü-gü	ingi-tüingi=ha	anha	heke\\
3-foot-\textsc{poss}	erase-\textsc{sim}	\textsc{compl}	3-foot-\textsc{poss}	see-\textsc{avd=ha} 	dead	\textsc{erg}\\
\glt ‘Erasing her footprints, for the dead not to see her footprints’ \\
‘Apagando as pegadas dela, para o morto não ver as pegadas dela’ \\
\z

\ea ihüsoho e-nhügü leha isingi \\[.3em]
\gll i-hüsoho	e-nhügü	leha	is-ingi \\
3-mother.in.law	come-\textsc{pnct} 	\textsc{compl}	3-after \\
\glt ‘The mother-in-law was coming after her’ \\
‘A sogra vinha atrás dela’ \\
\z

\ea inhalütüha tütüte isingi inginümi iheke \\[.3em]
\gll inhalü=tü=ha	tütüte	 is-ingi	ingi-nümi		i-heke \\
\textsc{neg=uncr=ha}	hidden	3-after 	bring-\textsc{pnct.cop} 	3-\textsc{erg} \\
\glt ‘She (the mother-in-law) didn’t bring her (daughter-in-law) hidden after her’ \\
‘Ela (a sogra) não trouxe (a nora) escondida atrás dela’ \\
\z

\ea teh titamingügi ekubetüha anha akapügüha \\[.3em]
\gll teh	titamingügi	ekube=tü=ha	anha	akapügü=ha \\
\textsc{itj}	drawn		good\textsc{-uncr=ha} 	dead 	proper=\textsc{ha} \\
\glt ‘Wow! (The path) was really beautiful, drawn properly for the dead’ \\
‘Poxa! Era bem bonito (o caminho) desenhado especialmente para os mortos’ \\ 
\z

\ea amaha simagüko tanginhü \\[.3em]
\gll ama=ha	is-ima-gü-ko	tanginhü \\
path=\textsc{ha}	3-path-\textsc{poss}-\textsc{pl}	main.path \\
\glt ‘The path, their path (of the dead), the main path’ \\
‘O caminho, o caminhos deles (dos mortos), o caminho principal’ \\
\z

\ea tange itamingügüi leha teh \\[.3em]
\gll tange	itamingü-gü-i		leha	teh \\
pot 	drawing-\textsc{poss-cop}	\textsc{compl}	\textsc{itj} \\
\glt ‘It looked like the drawing on the small pots, beautiful!’ \\
‘Parecia com a pintura da panelinha de barro, bem bonita!’ \\
\z

\ea ama tepügü \\[.3em]
\gll ama	te-pügü \\
path 	go-\textsc{prf} \\
\glt ‘The way of going (to the village of the dead)’ \\
‘O caminho da ida (para a aldeia dos mortos)’ \\
\z

\ea anha akapügütsügü \\[.3em]
\gll anha	akapügü=tsügü \\
dead 	proper=\textsc{uncr} \\
\glt ‘Done especially for the dead’ \\
‘Feito especialmente para os mortos’ \\
\z

\ea ingiale ekugu igia tsügü tihü heke gele ingiale isita gele\\[.3em]
\gll ingi-ale	ekugu	igia=tsügü tihü	heke	gele	ingi-ale	is-ita gele\\ 
see-\textsc{sim} 	really 	like.this=\textsc{uncr} living \textsc{erg} 	still 	see-\textsc{sim} 	3-be\textsc{-cont} still\\ 
\glt ‘The living one was coming like this, looking, looking’\\
‘A viva estava vindo assim olhando, olhando’{\footnotemark}\\
\footnotetext{Only one who is dead can pass on the path of the dead, but in this case the woman was alive.}
\z 

\ea anha imagü \\[.3em]
\gll anha ima-gü \\
dead path-\textsc{poss} \\
\glt ‘The path of the dead’ \\
‘O caminho dos mortos’ \\
\z

\ea lepe leha etelüko leha \\[.3em]
\gll üle-pe		leha		e-te-lü-ko		leha \\
log-\textsc{ntm}	\textsc{compl} 	3-go-\textsc{pnct-pl}	\textsc{compl} \\
\glt ‘Then, they went’ \\
‘Depois elas foram’ \\
\z

\ea ngikaho letüha sinünkgo \\[.3em]
\gll ngikaho	üle=tü=ha	is-i-nüN-ko \\
back.door	\textsc{log=uncr=ha}	3-be-\textsc{pnct-pl} \\
\glt ‘At the back of the houses, there they stayed’ \\
‘Atrás das casas, lá ficaram’ \\
\z

\ea enüngo titage leha ogo{\footnotemark} uguponga leha \\[.3em]
\gll e-nüN-ko titage leha ogo ugupo-nga leha \\
enter-\textsc{pnct-pl} straight \textsc{compl} platform on-\textsc{all} \textsc{compl} \\
\glt ‘They entered (into the house) directly, over the platform’ \\
‘Entraram (na casa) direto, por cima do jirau’ \\
\footnotetext{The cassava paste is placed on an \emph{ogo} – a platform built at the back of the house – to dry in the sun.}
\z

\ea timbuku{\footnotemark} tüha üle hujati leha \\[.3em]
\gll timbuku=tü=ha 	üle	huja-ti 		leha \\
cassava.piece=\textsc{uncr}=\textsc{ha} 	\textsc{log}	middle-\textsc{all} 	\textsc{compl} \\
\glt ‘Just in the middle of the pieces of dried cassava paste’ \\
‘No meio dos pedaços de massa seca de mandioca’ \\
\footnotetext{The word \emph{timbuku} refers to a particular form of the pieces of the dried cassava paste.}
\z

\ea ĩtsüi kuiginhu{\footnotemark} hagatepügü \\[.3em]
\gll ĩtsüi	kuiginhu	hagate-pügü \\
lot 	cassava.flour 	store-\textsc{prf} \\
\glt ‘There was a lot of cassava flour stored there’ \\
‘Tinha muito polvilho armazenado’ \\
\footnotetext{The word \emph{kuiginhu} refers to the cassava flour, the end product of women's long and heavy labor necessary to process the cassava (\emph{Manihot esculenta}), which begins in the gardens with the digging up of the roots and goes through successive phases of withdrawal of the hydrocyanic acid (poisonous to humans), until the cassava paste is left to dry in the sun.}
\z

\ea anha inhangoha kaküngi tsügüha anha inhango \\[.3em]
\gll anha	inhango=ha	kaküngi	tsügü=ha	anha	inhango \\
dead 	food=\textsc{ha} 	much 		\textsc{uncr=ha} 	dead 	food \\
\glt ‘There was a lot of the food of the dead, food of the dead’ \\
‘Tinha muita comida dos mortos, comida dos mortos’ \\
\z

\newpage 
\ea üle hata letü ihüsoho agapagatsita leha \\[.3em]
\gll üle 	hata	üle=tü		i-hüsoho		agapagatsi-ta	leha \\
\textsc{log} 	while 	\textsc{log=uncr}	3-mother.in.law	sweep-\textsc{dur}	\textsc{compl} \\
\glt ‘Meanwhile her mother-in-law was sweeping’ \\
‘Enquanto isso, a sogra dela varria’ \\
\z

\ea engüha egei uhupungetagü leha iheke uhutüingiha uhutüingi \\[.3em]
\gll engü=ha	ege-i		uhu-pu-nge-tagü{\footnotemark}{}	 	leha	i-heke \\ 
\textsc{con=ha}	\textsc{dist-cop} 	know\textsc{-neg-tr-dur} 	\textsc{compl} \textsc{3-erg}  \\
\gll uhu-tüingi=ha		uhu-tüingi \\
know-\textsc{avd=ha} 	know\textsc{-avd} \\
\glt ‘(The mother-in-law) was hiding (her daughter-in-law) so nobody knew, so nobody knew’ \\
‘(A sogra) estava escondendo (a nora) para ninguém saber, para ninguém saber’
\footnotetext{Literally, ‘(she) was making (her) unknown’.}
\z

\ea üle hata tsügü gehale \\[.3em]
\gll üle	hata 	tsügü 		gehale \\
\textsc{log} 	while 	\textsc{uncr} 	again \\
\glt ‘Meanwhile, again’ \\
‘Enquanto isso, novamente’ \\
\z

\ea túhagu{\footnotemark} ingete anha kitagü üngahingo{\footnotemark} kitagü \\[.3em]
\gll túhagu		iN-kete		anha	ki-tagü		üngahi-ngo		ki-tagü \\
strainer		bring-\textsc{imp} 	dead	say\textsc{-dur} 	circle.houses-\textsc{nmlz}	say-\textsc{dur} \\
\glt ‘“Bring \emph{túhagu} (a sieve)” the dead one was saying, the one of the other house was saying’ \\
‘“Traga \emph{túhagu} (peneira)!” dizia o morto, dizia o da outra casa’ \\
\fnminus
\footnotetext{In the language of the dead, words are different (for the same referent): \emph{túhagu} is the word of the dead for ‘strainer’, \emph{angagi} in the language of the living.}
\fnplus
\footnotetext{The adverb \emph{üngahi} means ‘along the circle of the houses’ (the Xinguan village is typically an oval circle of houses) and here it is nominalized by the suffix \emph{-ngo}, exclusive for adverbs and numerals.}
\z

\ea itsatüeha itsatüeha kakisükope uhitsa leha kupeheni \\[.3em]
\gll i-ta-tüe=ha	i-ta-tüe=ha k-aki-sü-ko-pe	uhi-tsa 	leha 	kupehe-ni \\
3-hear-\textsc{imp.pl=ha} 3-hear-\textsc{imp.pl=ha} \textsc{incl}-word-\textsc{poss-pl-ntm} 	search-\textsc{prog}	\textsc{compl}	1.2.\textsc{erg-pl} \\
 \glt ‘“Listen! Listen! They are trying to speak our former language”’ (the mother-in-law was saying) \\
‘“Ouça! Ouça! Eles estão tentando falar a nossa língua (quando vivos)”’ (a sogra dizia)  \\
\z

\ea egea akatsange kakisüko anügü leha \\[.3em]
\gll egea		akatsange	k-aki-sü-ko		a-nügü		leha \\
like.that 	\textsc{int} 		1.2-word-\textsc{poss-pl} 	be-\textsc{pnct} 	\textsc{compl}  \\
\glt ‘“That's how our language is here”’ \\
‘“É assim que é a nossa língua aqui”’  \\
\z

\ea kakisükope uhitsa leha igei kupeheni leha egea leha \\[.3em]
\gll k-aki-sü-ko-pe			uhi-tsa		leha	ige-i	kupehe-ni leha \\
\textsc{incl}-word-\textsc{poss}-\textsc{pl\textsc}-\textsc{ntm} 	search-\textsc{dur}	\textsc{compl} \textsc{prox} 1.2-\textsc{pl} \textsc{compl}  \\
\gll egea	leha \\
so 	\textsc{compl} \\
\glt ‘They were trying to speak our language (that was their language when alive)’ (comment by \textit{Ájahi}) \\
‘Eles estavam tentando falar a nossa língua (que era a língua deles quando vivos)’ (comentário de \textit{Ájahi}) \\
\z

\ea igiatsetse unkgu uigiholotogu{\footnotemark}{}  ingete \\[.3em]
\gll igia-tsetse	unkgu	u-igiholoto-gu	iN-kete \\
like.this-\textsc{dim} 	small	l-griddle\textsc{-pos}s	bring\textsc{-imp.cntp} \\
\glt ‘It did not take too long, (she heard): “Bring my \emph{igihitolo} (a clay griddle for cooking cassava bread)”’ \\
‘Não demorou muito, (ela ouviu): “Traga meu \emph{igihitolo} (tacho)!”’ \\
\footnotetext{\emph{Igiholoto} is the word for \emph{alato} (the griddle pan for cooking cassava bread), in the language of the dead.}
\z

\ea itsatüeha \\[.3em]
\gll i-ta-tüe=ha \\
3-hear-\textsc{imp.pl=ha} \\
\glt ‘Listen to this!’ (\textit{Ájahi} says to the researcher) \\
‘Ouça isto!’ (\textit{Ájahi} diz para o pesquisador) \\
\z

 
\ea alato heke akatsege tagü iheke \\[.3em]
\gll alato heke akatsege tagü i-heke \\
griddle \textsc{erg} \textsc{int} \textsc{dur} 3-\textsc{erg} \\
\glt ‘She said (referring to) \emph{alato} (the griddle for cooking cassava bread in the language of living)’ \\
‘Ela falou referindo-se a \emph{alato}’ (tacho na língua dos vivos)  \\
\z

\ea ekü hüle egei angagi heke túhagu ingete ta iheke \\[.3em]
\gll ekü hüle ege-i angagi heke túhagu iN-kete ta iheke \\
\textsc{con} \textsc{cntr} \textsc{prox-cop} strainer \textsc{erg} túhagu bring-\textsc{imp.ctp} \\
\glt ‘But before it was the sieve she was talking about, (when she said) “Bring \emph{túhagu}!”’ \\
‘Mas antes era da peneira que ela estava falando, (quando disse) “Traga \emph{túhagu}!”’ \\
\z

\ea angagi heke tetunetohongoi heke \\[.3em]
\gll angagi heke	t-et-une-toho-ngo-i				heke \\
strainer \textsc{erg}	\textsc{refl-dtr}-sift-\textsc{insnmlz-nmlz-cop}	\textsc{erg} \\
\glt ‘(Referring) to the sieve, that she used to sift (cassava paste) for herself’ \\
‘(Referindo-se) à peneira, aquilo que serve para ela peneirar para ela mesma’ \\
\z

\ea ikine ikitsomi tsügü hüle egei iheke tük \\[.3em]
\gll ikine		iki-tsomi			tsügü		hüle	ege-i		i-heke	tük \\
cassava.bread	make.cassava.bread-\textsc{purp} 	\textsc{uncr}	\textsc{cntr} 	\textsc{dist-cop} 	3-\textsc{erg}	\textsc{ideo} \\
\glt ‘But for her to cook cassava bread, \textit{tük}’ \\
‘Mas para para ela fazer beiju, \textit{tük}’ \\
\z

\ea uigiholotogu ingete tatohoi iheke \\[.3em]
\gll u-igiholoto-gu	iN-kete 		ta-toho-i 		i-heke \\
1-griddle-\textsc{pos} 	bring\textsc{-imp}.\textsc{cntp} 	say-\textsc{ins.nmlz-cop}	3-\textsc{erg} \\
\glt ‘“Bring my \emph{igiholoto} (griddle),” was what she meant to say’ \\
‘“Traga meu \emph{igiholoto} (tacho)!”, era para ela dizer’ \\
\z

 
\newpage  
\ea itsatüeha nügü iheke \\[.3em]
\gll i-ta-tüe=ha		nügü		i-heke \\
3-hear-\textsc{imp.pl=ha} 	say\textsc{.pnct} 	3-\textsc{erg} \\
\glt ‘“Listen!” she said’ \\
‘“Ouça!”, ela disse’ \\
\z

\ea egea akatsange leha kakisükope leha \\[.3em]
\gll egea		akatsange	leha	k-aki-sü-ko-pe		leha \\
like.this 	\textsc{int} 		\textsc{compl}	1.2-word-\textsc{poss-pl-ntm} 	\textsc{compl} \\
\glt ‘“This is what our language is like here”’ (the mother-in-law said) \\
‘“É assim que é a nossa língua aqui”’ (a sogra disse) \\
\z 

\ea kakisükope elükugigatühügü leha \\[.3em]
\gll k-aki-sü-ko-pe		elükugi-ga-tühügü	leha \\
1.2-word-\textsc{poss-ntm} 	reverse\textsc{-dur-prf} 	\textsc{compl} \\
\glt ‘“Our former language was reversed”’ \\
‘“A nossa língua foi sendo invertida”’ \\
\z 

\ea uhijüi leha kupehe ngiko itanügü kupehe \\[.3em]
\gll uhijü-i		leha	kupehe		ngiko	ita-nügü	kupehe \\
search-\textsc{cop} 	\textsc{compl} 1.2.\textsc{erg}	thing 	call-\textsc{pnct} 	1.2.\textsc{erg} \\
\glt ‘“We try to call things”’ \\
‘“Nós tentamos chamar as coisas”’ \\
\z

\ea ta tsügü iheke \\[.3em]
\gll ta	tsügü 		i-iheke \\
\textsc{dur} 	\textsc{uncr} 	3-\textsc{erg} \\
\glt ‘She was saying to her’ \\
‘Dizia para ela’ \\
\z

\ea tumukugu hitsü akihatagü iheke \\[.3em]
\gll tu-muku-gu		hi-tsü		aki-ha-tagü			i-heke \\
\textsc{refl}-son-\textsc{poss} 	wife-\textsc{poss} 	word\textsc{-vblz}-\textsc{dur}	3-\textsc{erg} \\
\glt ‘She was telling her son's wife’ \\
‘Ela contava para a esposa do seu filho’ \\
\z

 
\newpage 
\ea ülepe igia ünkgu tinho tataheti teh \\[.3em]
\gll üle-pe		igia		ünkgu	tü-nho			t-atahe-ti		teh \\
\textsc{log-ntm} 	like.this 	little 	\textsc{refl}-husband 	\textsc{ptcp}-spy-\textsc{ptcp}	\textsc{itj} \\
\glt ‘Shortly after, she spied on her husband (and said): “Wow! Beautiful!”’ \\
‘Pouco depois disso, ela espiou o esposo (e disse): “Nossa! Que bonito!”’ \\
\z

\ea tahisügi{\footnotemark}{} ekubekuletüha anha itu hugogo{\footnotemark} teh \\[.3em]
\gll tahisügi	ekubeku=letü=ha	anha	itu	hugogo	teh \\
red 		really\textsc{=uncr=ha}	dead 	village	plaza 		\textsc{itj} \\
\glt‘The plaza of the village of the dead was really reddish, beautiful!’ \\
‘A praça da aldeia dos mortos era bem avermelhada, muito bonita!’ \\
\fnminus
\footnotetext{The village plaza is qualified as reddish (\emph{tahisügi}, root \emph{hisu} ‘red’) because this is the typical color of most of the non-flood areas of central Brazil; the villages are always erected in these higher places.}
\fnplus
\footnotetext { \emph{Hugogo} is the village plaza, surrounded by the circle of houses.}
\z

\ea tatsajo ületü tomogokombeke üne tepügü \\[.3em]
\gll tatsajo		üle=tü		tomogokombeke	üne	te-pügü \\
all.together 	\textsc{log=uncr} 	stuck.in.the.other 	house	go-\textsc{prf} \\
\glt ‘The houses were one next to the other’ \\
‘As casas eram coladas umas às outras’ \\
\z

\ea tepugopeti \\[.3em]
\gll t-epugope-ti \\
\textsc{ptcp}-be.circle-\textsc{ptcp} \\
\glt ‘In a circle’ \\
‘Em círculo’ \\
\z

\ea hangakaki tajühe tsetsei kuakutu \\[.3em]
\gll hanga-ka-ki tajühe tsetse-i kuakutu \\
ear-big-\textsc{ins} chief.house almost.same-\textsc{cop} men.house \\
\glt ‘The \emph{kwakutu} (men’s house) was almost as the \emph{tajühe} (chief’s house) with big ears’ \\
‘O \emph{kwakutu} (casa dos homens) era quase do tamanho da \emph{tajühe} (casa do chefe) de orelhas grandes’ \\
\z

\newpage 
\ea tihü heke tsama ingitai iheke ahütüha apünguhügüila \\[.3em]
\gll tihü	heke 	tsama	ingi-tai		i-heke	ahütü=ha	apüngu-hügü-i-la \\
living 	\textsc{erg} 	? 	see-\textsc{fut.im} 	3-\textsc{erg}	\textsc{neg=ha} 	die-\textsc{prf-cop-priv} \\
\glt ‘The living one, it was she who could see, she was not really dead’ \\
‘A viva, era ela quem podia ver, não estava mesmo morta’ \\
\z

\ea ülepe leha igia ungku giti ikühagatilü \\[.3em]
\gll üle-pe		leha		igia	ungku	giti	ikühagati-lü \\
\textsc{log-ntm} 	\textsc{compl}	like.this little 	sun 	rise-\textsc{pnct} \\
\glt ‘Shortly after, the sun came up’ \\
‘Não demorou muito e o sol logo nasceu’ \\
\z

\ea hü hü hü{\footnotemark}{} aikobeha inhomo etimbelü \\[.3em]
\gll hü hü hü 	ai-ko-be=ha		i-nho-mo{\footnotemark}{} 		etimbe-lü \\
hü hü hü	\textsc{dem-pl-?=ha} 	3-husband-\textsc{coll} 	come-\textsc{pnct} \\
\glt ‘\textit{Hü hü hü}, those, her husband and brothers-in-law, came’ \\
‘\textit{Hü hü hü}, aqueles, seu esposo e os cunhados, chegaram’ \\
\addtocounter{footnote}{-1}
\footnotetext{The fishermen shout as they approach the village.} 
\stepcounter{footnote}
\footnotetext{The collective of “husband" refers to the group composed by the husband and his brothers. These are a woman's potential sexual partners and spouses. The suffix \emph{-mo} is a cognate of a common plural/collective suffix in other Carib languages.}
\z

\ea aindeko	akatsange \\[.3em]
\gll ainde-ko 	akatsange \\
\textsc{dem-pl} 	\textsc{int} \\
\glt ‘They are coming’ \\
‘Eles estão vindo’ \\
\z

\ea etihüĩtsüe etihüĩtsüe \\[.3em]
\gll et-ihüĩ-tsüe et-ihüĩ-tsüe \\
2.\textsc{dtr}-stay.quite-\textsc{imp.pl} 2\textsc{.dtr}-stay.quite-\textsc{imp.pl} \\
\glt ‘“Don’t move, don’t move!’” (the mother-in-law says to her daughter-in-law hidden among the cassava balls) \\
‘“Não se mexa, não se mexa!”’ (a sogra fala para a nora escondida entre as bolas de massa de mandioca) \\
\z

\newpage  
\ea elimo uünko ingilüpile atsange ketikaĩtsó \\[.3em]
\gll e-limo		uüN-ko	ingi-lü-pile 		atsange		ket-ikaĩ-tsó \\
2-children 	father-\textsc{pl} 	see\textsc{-pnct-conc} 	\textsc{int} 		\textsc{proh}-get.up-\textsc{proh} \\
\glt ‘“Even if you see the father of your children, don’t get up!”’ \\
‘“Mesmo se você ver o pai dos seus filhos, não se levante!”’ \\
\z

\ea ketikaĩtsó atsange \\[.3em]
\gll ke-tikaĩ-tsó		atsange \\
\textsc{proh}-get.up-\textsc{proh}	\textsc{int} \\
\glt ‘“Don’t really get up!”’ \\
‘“Não se levante mesmo!”’ \\
\z

\ea kakungakitüingi atsange elimo uünko enhügü kakungakitüingi \\[.3em]
\gll k-akunga-ki-tüingi		atsange		e-limo		uüN-ko 	e-nhügü  \\
1.2-soul-\textsc{vblz-avd} 	\textsc{int} 		2-children 	father-\textsc{pl}	come\textsc{-pnct} \\
\gll k-akunga-ki-tüngi \\
1.2-soul-\textsc{vblz-avd} \\
\glt ‘“Do not be alarmed by the arrival of the father of your children, do not be alarmed!”’ \\
‘“Não se assuste com a chegada do pai dos seus filhos, não se assuste!”’ \\
\z

\ea ülepe leha \\[.3em]
\gll üle-pe		leha \\
\textsc{log-ntm} 	\textsc{compl} \\
\glt ‘After this’ \\
‘Depois disso’ \\
\z

\ea ẽnünkgo leha ihinhanoko ẽnügü pokü \\[.3em]
\gll ẽ-nüN-ko	leha	i-hinhano-ko		ẽ-nügü		pokü{\footnotemark}{} \\
3.enter-\textsc{pnct-pl} 	\textsc{compl}	3-older.brother\textsc{-pl} 	enter-\textsc{pnct}	\textsc{ideo} \\
\glt ‘They entered (the house), their older brother entered, \textit{pokü}’ \\
‘Eles entraram, o irmão mais velho deles entrou, \textit{pokü}’ \\
\footnotetext{Ideophone for the act of unloading weight from the back or from the head to the floor.}
\z

\newpage 
\ea asankgu leha inhegikini itsangagüko \\[.3em]
\gll asankgu	leha	inhegiki-ni	i-kanga-gü-ko \\
basket 		\textsc{compl}	each.one\textsc{-pl} 	3-fish\textsc{-poss-pl} \\
\glt ‘Each had a basket (full of) fish’  \\
‘Cada um tinha um cesto de peixe’ \\
\z

\ea inhakagüki{\footnotemark}{} hegei ütepügüko totsonkgitohokoki \\[.3em]
\gll inhakagü-ki=ha	ege-i		ü-te-pügü-ko	t-ot-konkgi-toho-ko-ki \\
soap-\textsc{ins=ha} 	\textsc{dist-cop} 	3-go\textsc{-prf-pl}	\textsc{refl-dtr-insnmlz-pl-ins} \\
\glt ‘They had already gone to look for \emph{inhakagü} to wash themselves’ \\
‘Já tinham ido buscar \emph{inhakagü} para se lavar’\\
\footnotetext{Root of an unidentified plant that when rubbed with water produces foam; it was used before industrialized soap became available.}
\z

\ea igia tuhugu tsügü nhakagü kamisatühügü \\[.3em]
\gll igia		tuhugu=tsügü		inhakagü	kami-sa-tühügü \\
like.this 	amount=\textsc{uncr} 	soap 		tie-\textsc{dur-prf} \\
\glt ‘That's how they had tied the bundle of \emph{inhakagü} roots’ \\
‘Era assim que eles tinham amarrado o feixe de raízes de \emph{inhakagü}’ \\
\z

\ea isasankguguko ugupo itsangagüko ugupo{\footnotemark} \\[.3em]
\gll is-asankgu-gu-ko	ugupo	i-kanga-gü-ko		ugupo \\
3-basket-\textsc{poss-pl} 	on 	3-fish-\textsc{poss-pl} 	on \\
\glt ‘On their baskets, on their fishes’ \\
‘Em cima do cesto deles, em cima dos peixes deles’ \\ 
\footnotetext{\textit{Ájahi} remembered an old custom and showed to the listeners how old people used to manage and carry the traditional “soap".}
\z

\ea ai tüha ẽnünkgo \\[.3em]
\gll ai=tü=ha	ẽ-nüN-ko \\
then=\textsc{uncr=ha}	enter-\textsc{pnct-pl} \\
\glt ‘Then, they entered’ \\
‘Então, eles entraram’ \\
\z

 \largerpage
\ea ihinhanoko hotugui tüẽdinhüi \\[.3em]
\gll i-hinhano-ko		hotugu-i 	tü-ẽ-ti-nhü-i \\
3-older.brother\textsc{-pl} 	first\textsc{-cop} 	\textsc{ptcp}-enter\textsc{-ptcp-nanmlz-cop} \\
\glt ‘It was the older brother who entered first’ \\
‘Foi o irmão mais velho que entrou primeiro’ \\
\z

\ea ülepe ihisü ülepe isingingope ülepe aküpügüko tsügü hüle ekisei ihisükoi \\[.3em]
\gll üle-pe	i-hi-sü	üle-pe is-ingi-ngo-pe üle-pe	aküpügü-ko=tsügü	hüle	ekise-i i-hi-sü-ko-i \\
\textsc{log-ntm} 	3-younger.brother-\textsc{poss} 	\textsc{log-ntm} 	3-behind-\textsc{nmlz-ntm} \textsc{log-ntm} 3.youngest-\textsc{pl=uncr} \textsc{cntr} 3.\textsc{dist-cop} 3-younger.brother-\textsc{poss-pl-cop} \\
\glt ‘Then the younger brother, then the one who comes after him, then the last one, that one, their younger brother’ \\
‘Depois o irmão mais novo, depois o que vem atrás dele, depois o último, aquele, o irmão mais novo deles’ \\
\z

\ea ekiseiha ekisei ngisoi \\[.3em]
\gll ekise-i=ha		ekise-i	ngiso-i \\
\textsc{3.dist-cop=ha} 	\textsc{3.dist-cop} 	spouse\textsc{-cop} \\
\glt ‘That one was the husband of that (woman)’ \\
‘Aquele era o esposo daquela (mulher)’ \\
\z

\ea üẽnünkgo leha \\[.3em]
\gll ü-ẽ-nüN-ko		leha \\
3-enter-\textsc{pnct-pl} 	\textsc{compl}  \\
\glt ‘They entered’ \\
‘Eles entraram’ \\
\z

\ea tünho ingilütü iheke ikühagatilü ikühagatilü  \\[.3em]
\gll tü-nho			ingi-lü=tü		i-heke	ikühagati-lü 		ikühagati-lü \\
\textsc{refl-}husband	see-\textsc{pnct=uncr} 	\textsc{3-erg}	3.get.up.little-\textsc{pnct} 3.get.up.little-\textsc{pnct} \\
\glt ‘She saw her husband (and) got up a little, she got up a little’ \\
‘Ela viu o esposo e se levantou um pouco, se levantou um pouco’ \\
\z

\largerpage
\ea tãuguila letüha titaginhu imbüa geleha ihüsoho heke apenügü \\[.3em]
\gll tãuguila üle=tü=ha t-itaginhu	imbua gele=ha i-hüsoho heke	ape-nügü \\
speaking.high \textsc{log-uncr-ha} \textsc{refl}-converse in.middle	still\textsc{=ha} 3-mother-in-law \textsc{erg} tell.shut.up-\textsc{pnct} \\
\glt ‘While (her mother-in-law) was speaking loudly (with her son), still in the middle of the conversation, her mother-in-law told her to shut up’ \\
‘Enquanto (a sogra) falava alto (com o filho), ainda no meio da conversa, a sogra mandou ela se calar’ \\
\z

\ea üle lopenümi ületahüle ihitsü etikainjü itsikuoinjüinha \\[.3em]
\gll üle	lope-nümi 	ületa=hüle 	i-hi-tsü et-ikaiN-jü i-tikuoĩ-jü-inha \\
\textsc{log} stand.direction-\textsc{pnct.cop} \textsc{log=cntr} 3-wife\textsc{-poss} \textsc{3.dtr-}get.up\textsc{-pnct} 3-hug-\textsc{pnct-dat} \\
\glt‘Going to meet him (the dead), the wife (of the dead) got up to hug him’ \\
‘Indo ao seu (do morto) encontro, a esposa (do morto) se levantou para abraçá-lo’{\footnotemark} \\
\footnotetext{\emph{Ájahi} refers here to \emph{Itangitsegu}, the wife of the dead in the village of the dead.}
\z

\ea ikuilisale leha ipigagü kae \\[.3em]
\gll i-kuili-sale	leha	i-piga-gü kae \\
3-kiss-\textsc{sim} 	\textsc{compl} 3-cheek-\textsc{poss} on \\
\glt ‘She kissed him on the cheeks’ \\
‘Beijou-o nas bochechas’ \\
\z

\ea akehenügü ületüha iheke mbokü \\[.3em]
\gll akehe-nügü			üle=tü=ha 		i-heke	mbokü{\footnotemark}{} \\
take.house.corner-\textsc{pnct} 	\textsc{log=uncr=ha} 	\textsc{3-erg}	\textsc{ideo} \\
\glt ‘She took him to the corner of the house (and) lay on top of him’ \\
‘Ela o levou para o canto da casa (e) deitou-se em cima dele’ \\
\footnotetext{Ideophone that expresses the act of lying on someone (the sexual act).}
\z

 %avoid split footnote next page
\ea itsikaĩholü itsikaĩholü \\[.3em]
\gll i-tikaĩ-ho-lü		i-tikaĩ-ho-lü  \\
3-get.up-\textsc{cf-pnct} 3-get.up-\textsc{cf-pnct} \\
\glt ‘She (the living wife) almost got up, she almost got up’ \\
‘Ela (a esposa viva) quase se levantou, quase se levantou’ \\
\z

\ea itsinhulukijü tsügüha Itsangitsegu heke \\[.3em]
\gll i-kinhulu-ki-jü			tsügü=ha	Itsangitsegu	heke \\
3-jealousy-\textsc{vblz-pnct} 	\textsc{uncr=ha} 	itsangitsegu 	\textsc{erg} \\
\glt ‘She got jealous of \emph{Itsangitsegu}’ \\
‘Ela ficou com ciúmes de \emph{Itsangitsegu}’ \\
\z

\ea Itsangitsegu hekisei ihitsüi Itsangitsegu atühügü \\[.3em]
\gll Itsangitsegu=ha	ekise-i		i-hi-tsü-i	Itsangitsegu	a-tühügü \\
itsangitsegu=\textsc{ha} 	3.\textsc{dist-cop} 	3-wife-\textsc{poss-cop} 	itsangitsegu	\textsc{stay-prf} \\
\glt ‘That one was \emph{Itsangitsegu, Itsangitsegu} had become his wife’ \\
‘Aquela era \emph{Itsangitsegu, Itsangitsegu} tinha se tornado a esposa dele’ \\
\z

\ea ülehinhe hüle egei \\[.3em]
\gll üle-hinhe	hüle	ege-i \\
\textsc{log-npurp}	\textsc{cntr}	\textsc{dist-cop} \\
\glt ‘It was because of her (that the living wife had been jealous)’ \\
‘Era por causa dela (que a esposa viva tinha ficado com ciúme)’ \\
\z

\ea ketikaĩtsó atsange ketikaĩtsó \\[.3em]
\gll ke-tikaĩ-tsó		atsange		ke-tikaĩ-tsó	\\		
\textsc{proh}-get.up-\textsc{proh}	\textsc{int} 		\textsc{proh-}get.up-\textsc{proh} 	\\
 \glt ‘“Do not get up, do not get up!”’ (the mother-in-law said) \\
‘“Não se levante, não se levante!”’ (disse a sogra) \\
\z

 
\ea hum hum uãbeki kukanünkgo \\[.3em]
\gll hum hum	uã-beki		kuk-a-nüN-ko \\
\textsc{ideo} \textsc{ideo} 	\textsc{q-ep}		1.2-be-\textsc{pnct-pl} \\
\glt ‘“\textit{Hum hum}, what's happening to us?”’ (the dead said) \\
‘“\textit{Hum hum}, o que está acontecendo conosco?”’ (os mortos disseram) \\
\z

\ea hum hum tihühokolo giketilübe nügü leha ihekeni \\[.3em]
\gll hum hum{\footnotemark}{}	tihühokolo	gike-ti-lü=be 			nügü	leha	i-heke-ni \\ 
\textsc{ideo} \textsc{ideo}	living 		smell-\textsc{vblz-pnct=ep} 	say 	\textsc{compl} \textsc{3-erg-pl} \\
\glt ‘“\textit{Hum},  I can smell a living person,” they said’\\
‘“\textit{Hum }, estou sentindo cheiro de pessoa viva” eles disseram’ \\
\footnotetext{At this moment, \textit{Ájahi}, the storyteller, represents the dead character spitting on the ground, thus expressing nausea induced by the smell of the living. It is another example of inverted perspective, since the smell of the dead (rotten flesh) causes disgust in the living. In the next lines she repeats the gesture.}
\z

\newpage 
\ea tihühokolo giketilübe hum ahijunu giketilübe \\[.3em]
\gll tihühokolo	gike-ti-lü=be 			hum	ahijunu{\footnotemark}{}		gike-ti-lü=be \\
living 		smell-\textsc{vblz-pnct=ep} 	\textsc{ideo}	annatoo 	smell-\textsc{vblz-pnct=ep} \\
\glt ‘“A living person is giving off a smell, annatto is giving off a smell”’ \\
‘“Uma pessoa viva está exalando cheiro, urucum está exalando cheiro”’ \\
\footnotetext{The dead smell the living, who give off the scent of annatto (\emph{Bixa orellana}). A red pigment extracted from the seeds of this plant is used not only on ritual occasions, but almost daily, to paint the body and artifacts. In the language of the dead, however, annatto is called \emph{ahijunu}, while in the language of the living it is called \emph{umüngi}. Annatto is life.}
\z

\ea umüngi hekeha egei ta iheke umüngiha egei nhigatakoi ahijunui \\[.3em]
\gll umüngi heke=ha ege-i ta i-heke \\
annatto \textsc{erg=ha} \textsc{dist-cop} \textsc{dur} \textsc{3-erg} \\
\gll umüngi=ha	ege-i i-ng-iga-ta-ko-i	ahijunu-i \\
annatto=\textsc{ha} \textsc{dist-cop} 3-\textsc{obj}-name-\textsc{cont-pl-cop} annatto-\textsc{cop} \\
\glt ‘It was annatto he was talking about, it was annatto that they called \emph{ahijunu}' (clarification by \textit{Ájahi})’ \\
‘Era do urucum que ele estava falando, era urucum o que eles chamavam de \emph{ahijunu}’ (esclarecimento de \textit{Ájahi}) \\
\z

 
\ea hum tihühokolo giketilübe \\[.3em]
\gll hum	tihühokolo	gike-ti-lü=be \\
\textsc{ideo}	living 		smell\textsc{-vblz-pnct=ep} \\
\glt‘“Hum, a living person is giving off a smell
”’ \\
‘“Hum, uma pessoa viva está exalando cheiro”’ \\
\z

\ea ülepe		leha \\[.3em]
\gll üle-pe 		leha \\
\textsc{log-ntm}	\textsc{compl} \\
\glt ‘After this’ \\
‘Depois disso’ \\
\z

\ea tükangagüko inkgatilü leha ihekeni hugombonga \\[.3em]
\gll tü-kanga-gü-ko	inkgati-lü		leha		i-heke-ni	hugombo-nga \\
\textsc{refl}-fish-\textsc{poss-pl} 	go.share-\textsc{pnct} 	\textsc{compl} 	3\textsc{-erg-pl} 	plaza-all \\
\glt ‘They went to the middle of the plaza to share their fish’{\footnotemark}{} \\
‘Eles levaram peixe para o centro da aldeia’ \\
\footnotetext{When men return from a collective fishing trip, during the performance of a ritual, they take much of what they have caught to the central plaza of the village, in front of or inside the men's house, to be divided and distributed to all houses and to the men gathered in the center.}
\z

\ea etelüko \\[.3em]
\gll e-te-lü-ko \\
3-go-\textsc{pnct-pl} \\
\glt ‘They went’ \\
‘Eles foram’ \\
\z

\ea lepe enhügü leha \\[.3em]
\gll üle-pe		e-nhügü 	leha \\
\textsc{log-ntm} 	come-\textsc{pnct} 	\textsc{compl} \\
\glt ‘Then they came back’ \\
‘Depois voltaram’ \\
\z

\ea igia unkgu kigeke tuãka kigeke tuãka \\[.3em]
\gll igia		unkgu	kigeke		tuãka kigeke 	tuãka \\
like.this 	little 	1.2.go.\textsc{imp} 	water.all 1.2.go.\textsc{imp} water.\textsc{all} \\
\glt‘It did not take long: “Let's take a bath! Let's take a bath!”’ \\
‘Não demorou muito: “Vamos tomar banho! Vamos tomar banho!”’ \\
\z

\ea ese heke inhakagü igelü ese heke nhakagü igelü tüẽgikini \\[.3em]
\gll ese	heke	inhakagü	ige-lü		\\
\textsc{3.prox} 	\textsc{erg} 	soap 		carry-\textsc{pnct} \\
\gll ese		heke	inhakagü	ige-lü		tü-ẽgiki-ni \\
3.\textsc{prox}	\textsc{erg} 	soap 		carry-\textsc{pnct} 	\textsc{refl}-each-\textsc{pl} \\
\glt ‘He took \emph{inhakagü}, he took \emph{inhakagü}, each one for himself’ \\
‘Ele pegou \emph{inhakagü}, ele pegou \emph{inhakagü}, cada um deles para si mesmo’ \\
\z

\ea inhakagü inügü leha ihekeni \\[.3em]
\gll inhakagü i-nügü leha i-heke-ni \\
soap bring-\textsc{pnct} \textsc{compl} 3-\textsc{erg-pl} \\
\glt ‘They brought \emph{inhakagü}’ \\
‘Eles trouxeram \emph{inhakagü}’ \\
\z

\newpage 
\ea etelüko leha \\[.3em]
\gll e-te-lü-ko		leha \\
3-go-\textsc{pnct-pl} \textsc{compl} \\
\glt ‘They went’ \\
‘Eles foram’ \\
\z


\ea totsonkgilükoinha leha tüenkgügükope tijüinha tsügüha egei ütelüko \\[.3em]
\gll t-o-konkgi-lü-ko-inha		leha  \\
\textsc{refl-dtr}-wash-\textsc{pnct-pl-dat} 	\textsc{compl} \\
\gll tü-enkgü-gü-ko-pe 			tijü-inha 	tsügü=ha	ege-i  \\
\textsc{refl-}bad.smell-\textsc{poss-pl-ntm} 	take.off-\textsc{dat} 	\textsc{uncr=ha} 	\textsc{dist-cop}  \\
\gll ü-te-lü-ko \\
3-go-\textsc{pnct-pl} \\
\glt ‘They went to wash themselves, to get rid of their bad smell (of fish)’ \\
‘Foram para se lavar, para tirar o seu cheiro podre de peixe’ \\
\z

\ea atütüila kukugeko ai \\ [.6em]
\gll atütü-i-la 		kukuge-ko 	ai \\
good-\textsc{cop-priv} 	\textsc{1.2-pl} 		\textsc{ideo} \\
\glt‘“Unfortunately, we're not well!”’ (the mother-in-law said) \\
“‘Hélas, nós não (estamos) bem!”’ (a sogra disse) \\
\z

 
\ea Tisuge{\footnotemark}{} \\[.3em]
\gll Tisuge \\
1.3 \\
\glt “‘We’” \\
“‘Nós”’ 
\footnotetext{The dead mother-in-law alternates between use of first-person plural inclusive free pronoun \emph{kukugeko} (in the preceding line) and the first-person plural exclusive free pronoun \emph{tisuge}. This is an example of the shift of perspective from inclusive to exclusive pronouns (or vice-versa): with \emph{kukuge(ko)} the addressee is included because her body will rot inevitably when she becomes \emph{anha}; with \emph{tisuge}, the speaker excludes the addressee, opposing the dead to the living.}
\z

\ea igia agagenaha ketsüjenügü \\[.3em]
\gll igia		agage=naha	k-etsüje-nügü \\
like.this 	\textsc{as=ep} 		1.2-\textsc{die-pnct} \\
\glt ‘When we die’ \\
‘Quando morremos’ \\
\z

\newpage 
\ea ülepe inhalüma jahetüha kukenkgügü etijüi \\[.3em]
\gll üle-pe		inhalü-ma	jahe=tü=ha \\
\textsc{log-ntm} 	\textsc{neg-dub} 	quickly=\textsc{uncr=ha} \\
\gll kuk-enkgü-gü		et-ijü-i \\
1.2-smell\textsc{-poss} 	\textsc{dtr-}remove-\textsc{pnct-cop} \\
\glt ‘Our bad smell does not come out soon’ \\
‘Nosso mal cheiro não sai logo’ \\
\z

\ea itsatüe papa hõhõ ugikegü \\[.3em]
\gll i-ta{\footnotemark}{}-tüe 		papa	hõho	 u-gike-gü \\
3-smell-\textsc{imp.pl} 	\textsc{itj} 	\textsc{emph}	1-smell-\textsc{poss} \\
\glt ‘“So, smell me!”’ \\
‘“Então, cheire-me!”’ \\
\footnotetext{The verbal root \emph{ta} means all kind of perceptions through the senses, except for vision.} 
\z

\ea igia tühigüsi inatati \\[.3em]
\gll igia		tü-hi-gü-isi				inata-ati \\
like.this 	\textsc{refl}-grandson.\textsc{poss}-mother 	noose-\textsc{ill} \\
\glt ‘Like this, (she extended her hand) to the nose of the grandchildren's mother (her daughter-in-law)' \\
‘Desse jeito, (levou a mão) ao nariz da mãe dos netos (sua nora)’ \\
\z

\ea igia ige tüilü iheke \\[.3em]
\gll igia ige			tüi-lü		i-heke \\
like.this \textsc{prox} 	do-\textsc{pnct} 	\textsc{3-erg} \\
\glt ‘Like this, she did it’ \\
‘Desse jeito ela fez’ \\
\z


\ea tühüseki isikegü \\[.3em]
\gll tühüseki 	i-gike-gü{\footnotemark}{} \\
fetid 		3-smell-\textsc{poss} \\
\glt ‘Quite stinky, her smell’ \\
‘Bastante fétido, o cheiro dela’ \\
\footnotetext{After the high front vowel, at morphemic boundaries, the consonant /g/ is realized as [s] \citep{Franchetto1995}.}
\z

\newpage 
\ea etelüko kigekeha kigekeha \\ [.6em]
\gll e-te-lü-ko 		kigeke=ha	kigeke=ha \\
3-go-\textsc{pnct-pl} 	let.go=\textsc{ha} 	let.go=\textsc{ha} \\
\glt ‘They went away, “Let’s go! Let’s go!”’ \\
‘Elas foram, “Vamos! Vamos!’ \\
\z

\ea üle hata  tülimo ugutega hõhõ iheke \\[.3em]
\gll üle	hata	tü-limo 	ugu-te-ga		hõhõ	i-heke \\
\textsc{log} 	when 	\textsc{refl}-children 	flat.bread{\footnotemark}{}-\textsc{vblz-dur} 	\textsc{emph} 3-\textsc{erg} \\
\glt ‘Meanwhile she was making cassava flat bread for her own children’ \\
‘Enquanto isso, ela estava fazendo beiju para os seus filhos’ \\
\footnotetext{\emph{Ugu} refers to a specific food, the cassava flat bread, called “beiju" in Brazilian Portuguese.}
\z


\ea ikuguko ẽgiki tatute tülimo ugutelü iheke{\footnotemark}{} \\[.3em]
\gll iku-gu-ko	ẽgiki	tatute	tü-limo			ugu-te-lü		i-heke \\
3.beverage-\textsc{poss}	each 	all 	\textsc{refl}-children 	flat.bread-\textsc{vblz-pnct} 	\textsc{3-erg} \\
\glt ‘She made beiju for the cassava beverage of each of her own chidren’ \\
‘Ela fez beiju para a bebida de mandioca de cada um dos seus filhos’ \\
\footnotetext{\emph{Iku} refers to a beverage made  with thin, dried cassava flatbread mixed with water.}
\z

\ea üle onhati leha kanga hutita iheke ese oku uguponga ese oku uguponga tülimo ẽgiki{\footnotemark}{} \\ [.6em]
\gll üle	onhati 		leha		kanga	huti-ta	i-heke  \\
\textsc{log} 	inside.\textsc{ill} 	\textsc{compl} 	fish 	take\textsc{-dur} 	\textsc{3-erg}  \\
\gll ese 		oku	ugupo-nga	ese 		oku 	ugupo-nga \\		  
3\textsc{.prox}	drink 	on-\textsc{all} 	\textsc{3.prox}	drink 	on\textsc{-all}   \\
\gll tü-limo			ẽgiki \\
\textsc{refl}-children	 each  \\
\glt ‘Inside this (cassava flat bread), she was putting the fish, one by one, and (she was putting beiju) on top of the beverage of one, on top of the beverage of another, (for) each one of her own children’ \\
‘Ela colocou os peixes dentro do beiju, um por um, (e colocou beiju) sobre a bebida deste, sobre a bebida daquele, (para) cada um dos seus filhos’ \\
\footnotetext{The root \emph{huti} means ‘take one out of a set’.}
\z

\largerpage[2]
\ea lepene tü hüle ütelüko \\[.3em]
\gll lepene=tü	hüle 	ü-te-lü-ko \\
then=\textsc{uncr} 	\textsc{cntr} 	3-go\textsc{-pnct-pl}	 \\
\glt ‘Then, however, they went’ \\
‘Depois disso, contudo, elas foram’ \\
\z


\ea tülimo kangagü hutita letü iheke \\[.3em]
\gll tü-limo			kanga-gü	huti-ta		leha=tü		i-heke \\
\textsc{refl}-children	fish-\textsc{poss} 	take-\textsc{dur} 	\textsc{compl=uncr} 	\textsc{3-erg} \\
\glt‘She was taking the fish of her own children, one by one’ \\
‘Ela foi tirando os peixes dos seus filhos, um por um’ \\
\z


\ea ese kangagü ese kangagü ese kangagü \\[.3em]
\gll ese kanga-gü		ese kanga-gü		ese kanga-gü \\
\textsc{3.prox} fish-\textsc{poss} 	\textsc{3.prox} fish-\textsc{poss}	\textsc{3.prox} fish\textsc{-poss} \\
\glt‘The fish of this one, the fish of this one, the fish of this one’ \\
‘O peixe desse, o peixe desse, o peixe desse’ \\
\z

\ea inkgatingalü letüha iheke egena \\[.3em]
\gll  inkgati-nga-lü		üle=tü=ha		i-heke	egena \\
3.share\textsc{-hab-pnct}	\textsc{log=uncr=ha} 	\textsc{3-erg}	there.all \\
\glt ‘In this way, she used to share (food) there (in the middle of the village)’ \\
‘Ela sempre compartilhava (alimentos) para lá (no meio da aldeia)’  \\
\z

 
\ea tühigüsi kangagüingoha egei \\[.3em]
\gll tü-higü-isi kanga-gü-ingo=ha ege-i \\
\textsc{refl}-grandson-mother fish\textsc{-poss-fut=ha} \textsc{dist-cop} \\
\glt ‘That will be the fish of the mother of her (the mother-in-law's) own grandchildren’ \\
‘Aquilo será o peixe da mãe dos seus (da sogra) netos’ \\
\z

\ea nhigelüingoha nhingütelüingoha ina \\[.3em]
\gll i-ng-ige-lü-ingo=ha		i-ng-ingüte-lü-ingo=ha 		ina \\
3-\textsc{obj}-take-\textsc{pnct-fut=ha} 	3-\textsc{obj}-go.down-\textsc{pnct-fut=ha} 	here\textsc{.all} \\
\glt ‘That she (daughter-in-law) will take, that she will bring down here’ \\
‘Que ela (nora) levará, que ela trará aqui em baixo’ \\
\z

\ea kigekeha nügü iheke \\[.3em]
kigeke=ha	nügü	i-heke \\
let.go\textsc{=ha} 	say 	3-\textsc{erg} \\
\glt ‘“Let’s go!” she (the mother-in-law) said’ \\
‘“Vamos!”, ela (a sogra) disse’ \\
\z

\newpage 
\ea opü atsange elimo otomoko einhümingo opü \\[.3em]
\gll opü	atsange		e-limo		oto-mo-ko	ei-nhümingo		opü \\
\textsc{itj} 	\textsc{int} 		2-children 	master-\textsc{coll-pl} be-\textsc{pnct.fut} \textsc{itj} \\
\glt ‘“Pay attention! The parents of your children will stay like this”’ \\
‘“Preste atenção! Os pais dos seus filhos vão ficar assim”’ \\
\z

\ea lepe leha etelüko leha \\[.3em]
\gll üle-pe leha e-te-lü-ko leha \\
\textsc{log-ntm} \textsc{compl} 3-go-\textsc{pnct-pl} \textsc{compl} \\
\glt ‘Then, they went away’\\
‘Então, elas foram’\\
\z

\ea etelüko letü \\[.3em]
\gll e-te-lü-ko leha=tü \\
3-go-\textsc{pnct-pl} \textsc{compl=uncr} \\
\glt ‘They went away’ \\
‘Elas foram’ \\
\z


\ea tüimapüani itsapügü itahiale leha \\[.3em]
\gll tü-ima-püa-ni i-tapü-gü itahi-ale leha \\
\textsc{refl}-path-\textsc{ntm-pl} 3-foot-\textsc{poss} delete-\textsc{sim} \textsc{compl} \\
\glt ‘Along their former way (of coming), erasing their footprints’ \\
‘Por aquele que fora o seu caminho [de vinda], apagando as
suas pegadas’ \\
\z

\ea tanginhü ẽgipügati{\footnotemark}{} \\[.3em]
\gll tanginhü ẽgipügati \\
main.path on.top.head \\
\glt ‘On top of the head of the main path’ \\
‘No topo da cabeça do caminho principal’ \\
\footnotetext{‘On top of the head of the main path’ means: ‘Just at the end of the main path’.}
\z

\ea aibeha ina ama humbugakainjü \\[.3em]
\gll aibe=ha ina ama humbugakaiN-jü \\
\textsc{con=ha} here path be.head.down\textsc{-pnct}  \\
\glt‘Here, the path turns upside down’ \\
‘Aqui o caminho fica de cabeça para baixo’ \\
\z

\ea osiha inaha eitsüe \\[.3em]
\gll osi=ha ina=ha e-i-tsüe \\
all.right=\textsc{ha} her=\textsc{ha} 2-be-\textsc{imp.pl} \\
 \glt ‘“All right, stay here!”’ (the mother-in-law said to her daughter-in-law) \\
‘“Certo, fique aqui!”’ (a sogra disse para a nora) \\
\z

\ea tütüki letüha inginügü iheke inatsüha ina leha kungongoguhonga \\[.3em]
\gll tütüki üle=tü=ha ingi-nügü i-heke ina=tsü=ha ina leha ku-ngongo-gu-ho-nga \\
slowly \textsc{log=uncr=ha} 3.bring-\textsc{pnct} \textsc{3-erg} here=\textsc{uncr=ha} here 1.2-earth\textsc{-poss-loc-all} \\
\glt ‘Very slowly, she was bringing her here on our land’ \\
‘Bem devagar, ela a trazia aqui na nossa terra’ \\
\z

 
\ea aiha \\[.3em]
\gll aiha \\
done \\
\glt ‘Done’ \\
‘Feito’ \\
\z

\ea egetüeha egetüeha \\[.3em]
\gll ege-tüe=ha ege-tüe=ha \\
can.go-\textsc{imp.pl=ha} can.go\textsc{-imp.pl=ha} \\ 
\glt ‘“You can go, you can go!”’ (the mother-in-law said to her daughter-in-law) \\
‘“Você pode ir, pode ir!”’ (a sogra disse para a nora) \\
\z

\ea kegetimükeĩtó atsange kegetimükeĩtó atsange \\[.3em]
\gll keg-et-imükeĩ-tó atsange keg-et-imükeĩ-tó atsange \\
\textsc{proh-dtr-}turn.face-\textsc{proh} \textsc{int} \textsc{proh-dtr-}turn.face-\textsc{proh} \textsc{int} \\
\glt ‘“Do not turn your face back, do not turn your face back!”’ \\
‘“Não vire o rosto para trás, não vire o rosto para trás!”’ \\
 \z

\ea eitsamini geleha \\[.3em]
\gll e-i-tsa-mini gele=ha \\
2-be-\textsc{dur-purp.pl} still=\textsc{ha}  \\
\glt ‘“For you to stay alive”’ \\
‘“Para você permanecer viva”’ \\
\z

\ea jatsitsü üngelei atsütaka atehe egei uitigi sinügü üngelei \\[.3em]
\gll jatsitsü üngele-i atsütaka atehe ege-i  \\
poor 3. \textsc{log-cop} \textsc{ep} because \textsc{dist-cop} \\
\gll u-itigi is-i-nügü üngele-i \\
\textsc{1-fin} 3-come-\textsc{pnct} \textsc{3.log-cop} \\
\glt ‘“Poor thing! Because she is the only one who could come and get me, only she” (the woman, the daughter-in-law, was saying to herself)’ \\
‘“Coitada! Porque só ela é quem poderia vir me buscar, só ela” (a mulher, a nora) falava para si mesma’ \\
\z

 
\ea luale utimükeĩtai \\[.3em]
\gll luale ut-imükeiN-tai \\
sorry \textsc{1.dtr}-turn.face-\textsc{fut.im} \\
\glt ‘“Sorry! I will turn my face back”’ \\
‘“Desculpe! Eu vou virar meu rosto para trás”’ \\
\z 

\ea igiagage tetimükei uãhhhh     etinhapehikilü{\footnotemark}{} \\[.3em]
\gll igia=agage t-et-imükeiN. uãhhhh     et-inhapehiki-lü \\
like.this=same \textsc{ptcp-dtr}-face.turn.\textsc{ptcp} \textsc{itj} 3.\textsc{dtr}-wave.hand-\textsc{pnct} \\ 
\glt ‘When she (daugther-in-law) turned her face back, \textit{uãhhhh}, the mother-in-law waved her open hand’ \\
‘Quando ela (a nora) olhou para trás, \textit{uãhhhh}, a sogra acenou (para ela) com a mão aberta’ \\
\footnotetext{\emph{Etinhapehikilü means} ‘with the open hand’; with this gesture, the mother-in-law communicates to her daughter-in-law that she (the daughter-in-law) will die in five days time, after not many days, soon.}
\z

\ea kegetimükeĩtó ukita titakegei \\[.3em]
\gll keg-et-imükeiN-to u-ki-ta ti=taka=ege-i \\
\textsc{proh-dtr}-turn.face-\textsc{proh} 1-say-\textsc{dur} \textsc{cr=ep-dist-cop} \\
\glt ‘“Don’t turn your face back! I really meant it” (the mother-in-law said)’ \\
‘“Não vire seu rosto para trás! Eu eastava falando a verdade mesmo” (a sogra disse)’ \\
\z

\newpage 
\ea isünkgülü aküngiduingo ale hegei{\footnotemark}{} \\[.3em]
\gll is-ünkgü-lü aküngiN-tu-ingo ale=ha ege-i \\
3.sleep-\textsc{pnct} quantity-\textsc{nmlz-fut} ?=\textsc{ha} \textsc{dist-cop} \\
\footnotetext{\emph{Ájahi}, the story-teller, showed her open hand to mean the number five: the daughter-in-law, the living one, will die in a few days: her destiny is sealed.}
 \glt ‘This will be the number (of days) she was going to sleep’  \\
‘Este é o número (de dias) que ela iria dormir’ \\
\z

\ea ülepeha ahütüha elimo uünko etsote elimo uünko itajotelüingola ehekeni \\[.3em]
\gll üle-pe=ha ahütü=ha e-limo uüN-ko e-tsote  e-limo uüN-ko itajote-lü-ingo-la e-heke-ni\\
\textsc{log-ntm=ha} \textsc{neg=ha} 2-children father-\textsc{pl} come-when
2-children father-\textsc{pl} swear-\textsc{pnct-fut-priv} \textsc{2-erg-pl} \\
\glt ‘“Later, when the father of your children comes, you can not swear at him” (the mother-in-law said)’ \\
‘“Depois, quando o pai dos seus filhos vier, não poderá xingá-lo” (a sogra disse)’ \\
\z

\ea itaginkgügikümingola nügü hõhõ i-heke \\[.3em]
\gll itagi-nkgügi-kü-mingo-la nügü hõhõ iheke \\
speech-hard-?-\textsc{pnct.fut-priv} \textsc{pnct} \textsc{emph} \textsc{3-erg} \\
\glt ‘“Without you talking harshly,” she (the mother-in-law) said to her’ \\
‘“Sem falar duro”, (a sogra) disse a ela’ \\
\z

\ea üngele akatsange ekise tengalü heke  hokugeũ hokugeũ  hokugeũ itigiha \\[.3em]
\gll üngele akatsange ekise te-nga-lü heke hokugeũ hokugeũ hokugeũ itigi=ha \\
3.\textsc{log} \textsc{int} \textsc{3.dist} go\textsc{-hab-pnct} \textsc{erg} pauraque pauraque pauraque \textsc{3.fin=ha} \\
\glt ‘“It is he who always goes seeking (and saying): hokugeũ{\footnotemark}{}, hokugeũ, hokugeũ”’ \\
‘“É ele mesmo que sempre anda buscando (dizendo): “hokugeũ,hokugeũ, hokugeũ”’  \\
\footnotetext{\emph{Anha} (the dead) may return to the living, announcing himself, behind the houses, as a common pauraque bird (\emph{Nyctidromus albicollis}) . It has a brownish and greyish plumage, and its singing sounds like a piercing scream, and is repeated in regular intervals for hours after dusk. \emph{hokugeũ}, his name in Kuikuro, is an onomatopoeic noun; this is an ominous sign.}
\z

\ea aitüha isinügü kohotsi \\[.3em]
\gll ai=tü=ha is-i-nügü kohotsi \\
então=\textsc{uncr=ha} 3-come\textsc{-pnct} late.afternoon \\
\glt ‘So, she (the living daugther-in-law) came (to the village) in the late afternoon’  \\
‘Então, ela (a viva) chegou (na aldeia) no final da tarde’ \\
\z

 
\ea ingitühügüko atai leha kohotsi ko ko hokugeũ \\[.3em]
\gll ingi-tühügü-ko atai leha kohotsi ko ko  hokugeũ \\
3.bring-\textsc{prf-pl} when \textsc{compl} late.afternoon ko ko pauraque \\
\glt ‘When it had already been brought, in the late afternoon, pauraque (sang) “\textit{ko ko}”’ \\
‘Quando ela já tinha sido trazida, no final da tarde, bacurau (cantou) “\textit{ko ko}”’ \\
\z

\ea eteke tingibataha uheke ehitsü anügü \\[.3em]
\gll e-te-ke t-ingiN=hata=ha u-heke e-hi-tsü a-nügü \\
2-go-\textsc{imp} \textsc{ptcp}-see.\textsc{ptcp=ha} \textsc{1-erg} 2-esposa-\textsc{poss} be-\textsc{pnct} \\
\glt ‘“Go away! I’ve already seen how your wife is” (the living woman said) \\
“Vá embora! Eu já vi como é a sua esposa’ \\
\z

\ea heinongombe ihugu ehitsü heinongombe nhangatügü ehitsü \\[.3em]
\gll heinongombe i-hu-gu e-hi-tsü heinongombe inh-angatü-gü e-hi-tsü \\
with.half 3-ass-\textsc{poss} 2-wife-\textsc{poss} with.half 3-breast-\textsc{poss} 2-wife-\textsc{poss} \\
\glt ‘“Your wife (has) half an ass (one buttock), your wife has half (one) breast”’ \\
‘“A sua esposa (tem) meia (uma) nádega, sua esposa (tem) meia (uma) teta”’ \\
\z

\newpage 
\ea mbüu itsuhünkginügü{\footnotemark} leha etelü leha \\[.3em]
\footnotetext{\emph{Mbüu} is an ideophone, whose meaning is a sudden and abrupt interruption of some event or action. The verbal stem \emph{itsu-hüN-ki-} is formed by the roots \emph{itsu} (sound vocalized by non-humans and some musical instruments), and \emph{hüN} ‘emit', and by the verbalizer \emph{ki} ‘take off, stop'.}
\gll mbüu itsu-hüN-ki-nügü leha e-te-lü leha \\
\textsc{ideo} sound-emit-\textsc{vblz-pnct} \textsc{compl} 3-go\textsc{-pnct} \textsc{compl} \\
\glt ‘\textit{mbüu}, he stopped making the sound (of a pauraque) and went away’ \\
‘\textit{mbüu}, ele parou de emitir som (como bacurau) e foi embora’ \\
\z

 
\ea lepetü tüti ilüinha leha \\[.3em]
\gll üle-pe-tü tüti i-lü-inha leha \\
\textsc{log-ntm-uncr} \textsc{refl.mother} fight-\textsc{pnct-dat} \textsc{compl} \\
\glt ‘Shortly after, he (the dead husband) (arrived) to fight with his own mother’  \\
‘Logo depois, ele (o morto) (chegou) para brigar com a sua própria mãe’ \\
\z

\ea ehigüsi ingitühügü itsagü nika uãke{\footnotemark}{}  eheke egei \\[.3em]
\gll e-hi-gü-isi ingi-tühügü i-tsagü nika uãke e-heke ege-i  \\
2-grandson-\textsc{poss}-mother bring-\textsc{prf} 3.be-\textsc{dur} \textsc{ep} time.ago 2\textsc{-erg} \textsc{dist-cop} \\
\glt ‘“Did you really bring the mother of your grandchildren?” (the dead man said)’ \\
‘“Você trouxe mesmo a mãe dos seus netos?” (o morto disse)’ \\
\footnotetext{\emph{Uãke}, in this line and in the following ones, is an adverb with temporal and epistemic values: it determines the interpretation of the event/action as having occurred before the speech time (past tense), and has an epistemic value of strong authority. }
\z

\ea uhupüngekela{\footnotemark} hüle egei uãke tisitsagü \\[.3em]
\gll uhu-püngekela hüle ege-i uãke tis-i-tsagü \\
know-? \textsc{adv} \textsc{dist-cop} time.ago 1.3-be\textsc{-dur} \\
\glt ‘“We had realized this (the coming of the living woman)”’ \\
‘“Nós tínhamos percebido isso (a vinda da mulher viva)”’ \\
\footnotetext{We could not segment what follows the root \emph{uhu}.}
\z

\newpage 
\ea tütomima uãke ehigüsi itigi etepügü tütomi \\[.3em]
\gll tü-tomi=ma uãke e-hi-gü-isi itigi e-te-pügü tü-tomi \\
\textsc{q-purp=dub} time.ago 2-grandson\textsc{-poss-}mother 3\textsc{.fin} 2-go-\textsc{prf} \textsc{q-purp} \\
\glt ‘“Why did you go to get the mother of your grandchildren? Why?”’ \\
‘“Por que você foi buscar a mãe de seus netos? Por quê?”’ \\
\z

\ea ehinhão ingitahüngü ekuniküle uãke eheke nügü iheke \\
\gll e-hi-nhão ingi-ta-hüngü eku=niküle uãke e-heke nügü i-heke \\
2-grandson-\textsc{coll} see\textsc{-dur-neg} real\textsc{-ep} time.ago 2\textsc{-erg} \textsc{pnct} \textsc{3-erg} \\
\glt ‘“Do not you look after your grandchildren?” he said’ \\
‘“Você não pensa nos seus netos?”, ele disse’ \\
\z

\ea ehinhão inkgukitai atainipa hõhõ ehigüsi heke \\[.3em]
\gll e-hi-nhão inkguki-tai atai=nipa hõhõ e-hi-gü-isi heke \\
2-grandson-\textsc{coll} raise\textsc{-fut.im} ?=\textsc{ep} \textsc{emph} 2-grandson\textsc{-poss-mother} \textsc{erg} \\
\glt ‘“Let the mother of your grandchildren raise them!”’ \\
‘“Deixe a mãe dos seus netos criá-los!”’ \\
\z

\ea utelüingo akatsige itigi nügü leha iheke \\[.3em]
\gll u-te-lü-ingo akatsige itigi nügü leha i-heke \\
1-go-\textsc{pnct-fut} \textsc{int} \textsc{3.fin} \textsc{pnct} \textsc{compl} \textsc{3-erg} \\
\glt ‘“I will really go to look for her,” he said to her’ \\
‘“Eu mesmo irei buscá-la” ele disse a ela’ \\
\z

\ea tüti ilü leha iheke \\[.3em]
\gll  tüti i-lü leha i-heke \\
\textsc{refl.}mother fight-\textsc{pnct} \textsc{compl} \textsc{3-erg} \\
\glt ‘He fought with his own mother’ \\
‘Ele brigou com a sua própria mãe’ \\
\z

\ea ülepe leha isiko enhügü gehale \\[.3em]
\gll üle-pe leha isi-ko e-nhügü gehale \\
\textsc{log-ntm} \textsc{compl} 3.mother-\textsc{pl} come\textsc{-pnct} again \\
\glt ‘After this, their mother (deceased) came again (to the village of the daughter-in-law)’ \\
‘Depois disso, a mãe deles voltou de novo (à aldeia da nora)’ \\
\z

\ea ülepe ihatigi ngikahonga \\[.3em]
\gll üle-pe iha-tigi ngikaho-nga \\
\textsc{log-ntm} tell\textsc{-fin} back.house-\textsc{all} \\
\glt ‘To tell her behind the house’ \\
‘Para contar (a ela) atrás da casa’ \\
\z

\ea ukita heke ande akatsange uetsagü nügü iheke ande  akatsange uetsagü nügü iheke \\[.3em]
\gll u-ki-ta heke ande akatsange u-e-tsagü nügü i-heke  \\
1-say-\textsc{dur} \textsc{erg} here \textsc{int} 1-come-\textsc{dur} \textsc{pnct} 3-\textsc{erg} \\
\gll ande  akatsange uetsagü nügü iheke \\
here \textsc{int} 1-come-\textsc{dur} \textsc{pnct} 3-\textsc{erg} \\
\glt ‘“I’m saying that I really came here” she said to her (the daughter-in-law) “I came here” she said to her’ \\
‘“Estou dizendo que eu vim mesmo aqui”, ela disse para ela (para a nora), “eu vim mesmo aqui”, ela disse para ela’ \\
\z

\ea tütomima elimo uünko itaginkgugita ehekeni \\[.3em]
\gll tü-tomi=ma e-limo uüN-ko itaginkgugi-ta e-heke-ni \\
\textsc{q-purp=dub} 2-children father-\textsc{pl} speak.hard\textsc{-dur} \textsc{2-erg-pl} \\
\glt ‘“Why were you speaking harshly to the father of your children?”’  \\
‘“Por que você falou duro para o pai dos seus filhos?” \\
\z

\ea uita takege ihekeni leha uitagü eigepügüko hinhe \\[.3em]
\gll u-i-ta takege i-heke-ni leha u-i-tagü e-ige-pügü-ko=hinhe \\
1-fight-\textsc{dur} \textsc{ep} 3\textsc{-erg-pl} \textsc{compl} 2-fight-\textsc{dur} 2-take\textsc{-prf-pl=npurp} \\
\glt ‘“They are fighting (with me), fighting (with me), because I took you (to the village of dead)”’ \\
‘“Eles estão brigando comigo, brigando, por que eu levei você (para a aldeia dos mortos)”’ \\
\z

\ea isakisüpeko ihataleha egea leha ihata leha iheke \\[.3em]
\gll is-aki-sü-pe-ko iha-ta=leha egea leha iha-ta leha i-heke \\
3-word-\textsc{poss-ntm-pl} tell\textsc{-dur=compl} like.this \textsc{compl} tell\textsc{-dur} \textsc{compl} \textsc{3-erg} \\
\glt ‘She was reporting their words, she was telling in this way’ \\
‘Ela estava relatando as palavras deles, contando assim’ \\
\z

\ea ukipügüa leha ihata leha iheke \\[.3em]
\gll u-ki-pügü-a leha iha-ta leha i-heke \\
2-say-\textsc{prf}-like.this \textsc{compl} tell\textsc{-dur} \textsc{compl} 3\textsc{-erg} \\
\glt ‘The way I said it, she was telling (to the daughter-in-law)’ \\
‘Do jeito que eu falei, ela estava contando (para a nora)’ \\
\z


\ea tumukugu akisüpe ihata leha iheke \\[.3em]
\gll tu-muku-gu aki-sü-pe iha-ta leha i-heke \\
\textsc{refl-}-son-\textsc{poss} word\textsc{-poss-ntm} tell-\textsc{dur} \textsc{compl} \textsc{3-erg} \\
\glt ‘She was reporting the words of her son’ \\
‘Ela estava relatando as palavras do filho’ 
\z

 

\ea ami atsange isitote ketitaginkgugito \\[.3em]
\gll ami atsange is-i-tote ket-itaginkgugi-tó \\
other.day \textsc{int} 3-come-when \textsc{proh}-speak.hard\textsc{-proh} \\
\glt ‘“The next time he comes, do not talk harshly with him!”’ \\
‘“Da próxima vez que ele vier, não fale duro com ele!”’ \\
\z

\ea ami ami akatsange eitigini leha isinümingo tükotui tükotui \\[.3em]
\gll ami ami akatsange e-itigi-ni leha is-i-nümi-ingo tü-kotu-i tü-kotu-i \\
other.day other.day \textsc{int} \textsc{2-fin-pl} \textsc{compl} 3-come-\textsc{pnct-fut} \textsc{refl-}angry\textsc{-cop} \textsc{refl}-angry\textsc{-cop} \\
\glt ‘“Next time, next time, he will come get you angry, angry”’ \\
‘“Na próxima vez, na próxima vez, ele virá buscar você com raiva, com raiva”’ \\
 \z

\ea ehe lepei igei igei isünkgülü hata isünkgülü hata \\[.3em]
\gll ehe üle-pe-i ige-i ige-i is-ünkgü-lü hata is-ünkgü-lü hata \\
\textsc{aff} \textsc{log-ntm-cop} \textsc{prox-cop} \textsc{prox-cop} 3-sleep-\textsc{pnct} when 3-sleep-\textsc{pnct} when \\
\glt ‘Yes, then, she (the living woman) slept like this, she slept like this’{\footnotemark}{} \\
‘Sim, então, ela (a viva) dormiu assim, dormiu assim’ \\
\footnotetext{\emph{Igei} (‘it is this’): \emph{Ájahi} is showing her open hand to mean five days (or nights).}
\z

\newpage 
\ea teti \\[.3em]
\gll t-e-ti \\
\textsc{ptcp}-come-\textsc{ptcp} \\
\glt ‘(The dead husband) came’ \\
‘(O esposo morto) veio’ \\
\z

\ea kü kü hokugeũ etekebeha tüki nigei enhalü igei tüki nigei enhalü \\[.3em]
\gll kü kü hokugeũ e-te-ke=be=ha tü-ki nile-ige-i e-nha-lü igei tü-ki nige-i  e-nha-lü \\
\textsc{ontp} \textsc{ontp} pauraque 2-go-\textsc{imp=ep=ha} \textsc{q-ins} \textsc{ep-prox-cop} arrive-\textsc{hab-pnct} \textsc{prox-cop} \textsc{q-ins} \textsc{ep-prox-cop} arrive\textsc{-hab-pnct} \\
\glt ‘The pauraque \textit{kü kü} (sang). “Go away! Why do you always come? Why do you always come?" (the living woman said)’ \\
‘O bacurau \textit{kü kü} (cantou). “Vá embora! Por que você sempre vem? Por que você sempre vem?” (a esposa viva disse)’ \\
\z

\ea tüki \\[.3em]
\gll tü-ki \\
\textsc{q-ins} \\
\glt ‘“Why?”’ \\
‘“Por quê?”’\\
\z

\ea ehitsütsapa itinhündeta heinongo nhangatügü \\[.3em]
\gll e-hi-tsü=tsapa itinhüN-te-ta heino-ngo i-ngangatü-gü \\
2-wife\textsc{-poss=e} lay.down-\textsc{vblz-dur} half\textsc{-nmlz} 3-breast-\textsc{poss} \\
\glt ‘“Go to bed with your wife who only has one tit!”’ \\
‘“Vá lá deitar com a sua esposa que só tem uma teta!”’ \\
\z

\ea heinongo ihugu \\[.3em]
\gll heino-ngo i-hu-gu \\
half\textsc{-nmlz} 3-buttock-\textsc{poss} \\
\glt ‘“(And) just one buttock"’ \\
‘“(E) só uma nádega”’ \\
\z

\newpage 
\ea tüendi tsürürü gitsitoho{\footnotemark}{} atati uẽtigi leha \\[.3em]
\gll tü-eN-ti tsürürü gitsi-toho atati uẽ-tigi leha \\
\textsc{ptcp}-enter-\textsc{ptcp} \textsc{ideo} urinate-\textsc{insnr} \textsc{ill} wait\textsc{-fin} \textsc{compl} \\
\glt ‘He (the dead) entered: \textit{tsürürü!} the place where people urinate, to wait’ \\
‘Ele (o morto) entrou \textit{tsürürü!} no lugar onde se urina, para esperar’ \\
\footnotetext{\emph{Gitsi-toho}, ‘made for urinating’: in the old days, there were bits of bamboo inside the house, into which people urinated. The dead husband has already turned into a snake and hides in the old urinal.}
\z

\ea lepe leha itsagü leha \\[.3em]
\gll üle-pe leha i-tsagü leha \\
\textsc{log-ntm} \textsc{compl} 3\textsc{-erg} \textsc{compl} \\
\glt ‘Then, he stayed (there)’ \\
‘Aí, ele ficou (lá)’ \\
\z

 
\ea koko bela leha kahugutilü leha koko tsitsi leha \\[.3em]
\gll koko=bela leha k-ahuguti-lü leha koko tsitsi leha \\
night=\textsc{ep} \textsc{compl} 1.2-get.dark-\textsc{pnct} \textsc{compl} night \textsc{dim} \textsc{compl} \\
\glt ‘Already at dusk, in the early evening’ \\
‘Já de noite, anoiteceu, no começo da noite’ \\
\z

\ea tütikaĩsitü gitsitoho atati tügitsilüinha \\[.3em]
\gll tü-tikaiN-si=tü gitsi-toho atati tü-gitsi-lü-inha \\
\textsc{ptcp}-get.up-\textsc{ptcp=uncr} urinate-\textsc{insnmlz} \textsc{ill} \textsc{refl}-urinate-\textsc{pnct-dat} \\
\glt ‘Having got up, to go to the place to urinate ' \\
‘Tendo ela se levantado, para ir ao local de urinar’ \\
\z

\ea isitsilü hata tük titsimbe eke atati leha inhügü\\[.3em]
\gll i-gitsi-lü hata tük t-itsi=mbe eke atati leha i-nhügü\\ 
3-urinate-\textsc{pnct} when \textsc{ideo} \textsc{ptcp}-bite\textsc{.ptcp=ep} snake \textsc{ill} \textsc{compl} be\textsc{-pnct}\\
\glt ‘While she was urinating, \textit{tük}! he had turned into a snake and he bit her’\\
‘Enquanto ela estava urinando, \textit{tük}! transformado em cobra ele a mordeu’ \\
\z

\newpage 
\ea eke tük ige kaenga itsilü iheke \\[.3em]
\gll eke tük ige kae-nga itsi-lü i-heke \\
snake \textsc{ideo} \textsc{prox} \textsc{loc-all} 3.bite\textsc{-pnct} \textsc{3-erg} \\
\glt ‘As a snake, he bit her here’ (\textit{Ájahi} shows the place) \\
‘Como cobra, ele a mordeu aqui’ (\textit{Ájahi} mostra o local) \\
\z

\ea hum pok pok pok{\footnotemark}{} aletüha etelü leha \\[.3em]
\gll hum pok pok pok ale=tü=ha e-te-lü leha \\
\textsc{ideo} \textsc{ideo} \textsc{ideo} \textsc{ideo} \textsc{sim=uncr=ha} 3-go\textsc{-pnct} \textsc{compl} \\
\glt ‘“\textit{Hum},” \textit{pok pok pok}, she screamed and convulsed’ \\
‘“\textit{Hum}”, \textit{pok pok pok}, ela gritou e ficou se debatendo’ \\
\footnotetext{The ideophones transcribed as \emph{hum} and as \emph{pok} (repeated, iterative) is the cry of the woman and her spasms, respectively.}
\z 



\ea pok pok pok pok tük isotütitagü leha tütükibeletü apüngü leha apüngü \\[.3em]
\gll pok pok pok pok tük is-otüti-tagü leha tütüki=bele=tü apüngü leha apüngü \\
\textsc{ideo} \textsc{ideo} \textsc{ideo} \textsc{ideo}  \textsc{ideo} 3-convulse-\textsc{dur} \textsc{compl} slowly\textsc{=ep=uncr} 3.die\textsc{.pnct} \textsc{compl} 3.die\textsc{.pnct} \\
\glt ‘She was convulsing, and she died, she died slowly’ \\
‘Ficou tendo convulsões, aos poucos foi morrendo’ \\
\z

\ea itigi hegei inhotelü itigi \\[.3em]
\gll i-tigi=ha ege-i i-nho te-lü itigi \\
3-\textsc{fin=ha} \textsc{dist-cop} 3-husband go\textsc{-pnct} 3\textsc{.fin} \\
\glt ‘The husband went to get her’ \\
‘O esposo foi buscá-la’ \\
\z

\ea aiha{\footnotemark}{} \\[.3em]
\gll aiha \\
done \\
\glt ‘Ready’ \\
‘Pronto’ \\
\footnotetext{The last four lines contain the formulas that every good storyteller must use to close her narrative: \emph{áiha} (‘ready/done’); \emph{upügüha egei} (‘that was the last/the end’); \emph{uitsojigü}, an untranslatable word that the storytellers say use to frighten away sleep, since the listener would be in a state of sleep/dreaming, from which he must awaken. }
\z

\ea ai akatsange \\[.3em]
\gll ai akatsange \\
ready \textsc{int} \\
\glt ‘Truly ready’ \\
‘Pronto mesmo’ \\
\z

\ea upügü hegei \\[.3em]
\gll upügü=ha ege-i \\
last\textsc{=ha} \textsc{dist-cop} \\
\glt ‘That was the end’ \\
‘Aquilo foi o final’ \\
\z

 
\ea uitsojigü nika kitse \\[.3em]
\gll uitsojigü nika ki-tse \\
uitsojigü \textsc{ep} say-\textsc{imp} \\
\glt ‘Say: \emph{uitsojigü}!’ \\
‘Diga: \emph{uitsojigü}!’ \\
\z

\section*{Acknowledgments}
Our first debt is to the Kuikuro for their commitment, generosity and longstanding hospitality. The following Brazilian institutions have been a fundamental support for conducting research among the Kuikuro since 1976: Fundação Nacional de Apoio ao Índio (FUNAI), Conselho Nacional de Desenvolvimento Científico e Tecnológico (CNPq), Museu Nacional (Universidade Federal do Rio de Janeiro). The DoBeS Program financed the Project for the Documentation of the Upper Xingu Carib Language or Kuikuro from 2001 to 2005. We would also like to thank Gustavo Godoy for his technical expertise and suggestions, to Kristine Stenzel for her many and wise revisions, and to Luiz Costa for the English translation of the Introduction.

\section*{Non-standard abbreviations}

\begin{tabularx}{.5\textwidth}{lQ}
\textsc{1.2 } & 1st person plural inclusive \\
\textsc{1.3 } & 1st person plural exclusive \\
\textsc{3.dist } & 3rd person distal \\
\textsc{3.prox } & 3rd person proximal \\
\textsc{aff } & affirmative \\
\textsc{anmlz } & agent nominalizer \\
\textsc{avd } & avoidance \\
\textsc{cf } & counterfactual \\
\textsc{cntr } & contrastive \\
\textsc{con } & connective \\
\textsc{conc } & concessive \\
\textsc{cr } & certainty \\
\textsc{dim } & diminutive \\
\textsc{dtr } & detransitivizer \\
\textsc{dub } & dubitative \\
\textsc{emph } & emphatic \\
\textsc{ep } & epistemic \\ 
\textsc{fin } & finality \\
\textsc{fut.im } & imminent future \\
\textsc{ha } & ha particle \\
\textsc{hab } & habitual \\
\textsc{hort } & hortative \\
\textsc{hort.pl } & plural hortative \\
\textsc{ill } & illative \\
\textsc{imp.pl } & imperative plural \\
\textsc{imp.ctp } & centripetal imperative \\
\textsc{ideo } & ideophone \\
\end{tabularx}
\begin{tabularx}{.45\textwidth}{lQ}
\textsc{imp.ctp.pl } & plural centripetal imperative \\
\textsc{imp.ctf } & centrifugal imperative \\
\textsc{imp.ctf.pl } & plural centrifugal imperative \\ 
\textsc{ine } & inessive \\
\textsc{inel } & inelative \\
\textsc{insnmlz } & instrumental nominalizer \\
\textsc{itj } & interjection \\
\textsc{int } & intensifier \\
\textsc{log } & logophoric \\
\textsc{3.log } & 3rd person logophoric \\
\textsc{nanmlz } & non-agent nominalizer \\
\textsc{npurp } & negative purposive \\
\textsc{ntm } & nominal tense marker \\
\textsc{o } & object \\
\textsc{ontp } & onomatopoeia \\
\textsc{pnct } & punctual \\
\textsc{priv } & privative \\
\textsc{sim } & simultaneous \\
\textsc{uncr } & uncertainty \\
\textsc{vbzl } & verbalizer \\ 
\\
\end{tabularx}


 
% 
{\sloppy
\printbibliography[heading=subbibliography,notkeyword=this]
}
\end{document}
