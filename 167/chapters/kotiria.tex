\documentclass[output=paper,
modfonts,nonflat
]{langsci/langscibook} 
\author{Kristine Stenzel\affiliation{Federal University of Rio de Janeiro, Brazil}\and%
Teresinha Marques\and%
José Galvez Trindade%
\lastand Miguel Wacho Cabral%
}%
\title{Kotiria}
\ourchaptersubtitle{Mahsa khõ'akho'topori}
\ourchaptersubtitletrans{‘Kotiria sacred cemeteries'}  
\lehead{K.\ Stenzel, Teresinha Marques, José Galvez Trindade \& Miguel Wacho Cabral}
\ChapterDOI{10.5281/zenodo.1008783}
\maketitle

\begin{document}

% \chapter{Kotiria, East Tukano}
\section{Introduction} 

This narrative recounts the origin of the sacred cemeteries of the Kotiria people.\footnote{Although they are also identified as Wanano or Guanano, the traditional, self-determined name Kotiria ‘water people’ is used here at the request of the speakers.}  The Kotiria are one of the sixteen East Tukano groups,\footnote{These are the Kotiria, Bará (Waimajã), Barasana, Desano, Karapana, Kubeo, Makuna, Pisamira, Siriano, Taiwano (Eduuria), Tanimuka (Retuarã), Tatuyo, Tukano, Tuyuka, Wa’ikhana (Piratapuyo), and Yuruti.} living in the upper Rio Negro border region between Brazil and Colombia in northwestern Amazonia and whose total population is approximately twenty-six thousand.\footnote{According to information from the Instituto Socioambiental (ISA)-PIB online \url{<http://pib.socioambiental.org/en>}, and the Colombian 2005 and Brazilian 2010 national censuses.} There are some 2000 Kotiria, most of whom live in traditional communities located along the Vaupés river, a territory they have occupied for at least seven centuries \parencite[10]{Stenzel2013}. 

The Sacred Cemeteries narrative was recorded on September 20, 2005, during a community workshop on Kotiria geography and history organized by the \textit{Khumuno Wʉ’ʉ Kotiria} Indigenous School. Participants in this five-day workshop included students, teachers, family members, and elders from several different Kotiria villages, who gathered together in \textit{Koama Phoaye} (Carurú Cachoeira), the largest Kotiria community on the Brazilian side of the Vaupés (see Figure 2). Several non-indigenous outsiders were also present, including the organizer of this chapter, linguist Kristine Stenzel, and two pedagogical consultants, Dr. Marta Maria Azevedo and Lucia Alberta Andrade de Oliveira. At the time, Azevedo was one of the coordinators of the Educational Program of the Instituto Socioambiental, and Andrade had been working as a pedagogical aid on-site with the Kotiria school for many months (for more on the history of the school, see \cite{Oliveira2012}). Workshop activities included map-making, text-writing, research on the history of individual villages, and visits to important regional landmarks. Many of the written materials and illustrations produced during the workshop were later gathered in a book entitled \textit{Phanopʉ, Mipʉ Mahka Bu’erithu} ‘Past and Present, Studies of our Origins’, from which the illustration at the end of the narrative was taken. There were also a number of talks on different historical topics proffered by invited elders, one of whom was the much-respected author of our narrative, Teresinha Marques.

\def\oldIntextsep{\the\intextsep}
\setlength{\intextsep}{0.25\baselineskip}
\begin{wrapfigure}{l}{7.5 cm}
  \includegraphics[width=7.5 cm]{figures/teresinha.jpg}
  \caption{Teresinha Marques}
  \label{fig:teresinha}
\end{wrapfigure}
\setlength{\intextsep}{\oldIntextsep}

Teresinha and her family came to the workshop every day from the nearby village of \textit{Bʉhka Khopa} (Matapí, Colombia, see Figure 3), which is the traditional home of one of the highest ranked Kotiria sibs, the \textit{Biari Pho’na} (children/descendants of \textit{Biari}, one of \textit{Dianumia Yairo}'s sons).   
    
The knowledge Teresinha shares was passed down from her own father and was not widely known; indeed, many participants in the workshop were learning about this important episode in Kotiria history for the first time. Teresinha’s fifteen-minute narrative was filmed and later integrated into the Kotiria Linguistic and Cultural Archive.\footnote{All materials in this archive have been deposited at the Endangered Languages Archive, SOAS, University of London https://elar.soas.ac.uk/Collection/MPI132528 and at the PRODOCLIN Archive at the Muséu do Índio/FUNAI, Brazil, with open access granted by the community.} Co-author José Galvez Trindade introduces her to the audience at the beginning of the recording and interacts with her at several points during the narrative. He was later involved in the initial transcription and translation of the narrative in 2010, and additional detailed analysis was accomplished with the help of co-author Miguel Cabral in 2016.
    
\begin{figure}[t]
\includegraphics[width=\textwidth]{figures/mapa-1-urn.png}
\caption{The Upper Rio Negro Region, showing the traditional territory of the Kotiria people on the Vaupés river. (Source: \citealt[9]{Stenzel2013})\label{mapone}} 
\end{figure}

The series of dramatic and tragic events leading to the origin of the Kotiria sacred cemeteries unfold against a backdrop of significant features of Kotiria cosmology and social organization.\footnote{Comprehensive ethnographic analysis of the Kotiria can be found in works by \textcite{Chernela1983, Chernela1993, Chernela2004, Chernela2013}, among others. For an overview of aspects of shared regional culture, see \cite{Epps2013} and the references cited there.}   Like all Tukanoan groups, the Kotiria have an origin myth in which pre-human beings, still in a state of “transformation", travel underwater upriver in an anaconda canoe from the Milk Lake to their territory on the Vaupés. Stopping at many places along the way (“houses of transformation"), they slowly acquire knowledge, techniques, instruments, encantations or “blessings", dances, and adornments — essential elements that contribute to their transformation into a fully human state.\footnote{For similar accounts for other groups, see also \cite[165]{Cabalzar2008} and \cite{Andrello2012}.}  Once reaching the headwaters of the Vaupés, their anaconda canoe turns around, and the mythical ancestors of the different Kotiria kin groups emerge at the places where parts of its body surface. Those emerging closer to the anaconda’s head are considered the higher ranked “older" brothers, those closer to the tail are “younger" brothers. A third, “servant" group, the \textit{Wiroa}, originated separately from birds. Thus, the twenty-five or so Kotiria sibs are organized into three larger groupings with the symbolic roles of “chiefs", “dancers/masters of ceremonies", and “servants/cigar holders" (\citealt{Chernela1983}, \citealt[5-15, 51-59]{Chernela1993}; \citealt{Waltz1997} offers slightly different sib names, numbers and relative rankings). Each individual in Kotiria society inherits a fixed rank in the social hierarchy as a descendant from an ancestral sibling and is highly aware of the roles and responsibilities associated with that rank. 

The four main protagonists in Teresinha’s Sacred Cemeteries narrative belong to the highest ranked, \textit{Biari} group: they are \textit{Ñahori}, his older brother \textit{Diani}, and younger brother \textit{Yuhpi Diani}, sons of the great ancestor shaman \textit{Dianumia Yairo}. As is the case in many tales of betrayal, vengeance, and bloodshed, this one begins with a dispute over a woman. \textit{Ñahori}, living in \textit{Mukʉ Dʉhpʉri} (see Figure 3) with his own two wives and two children, has promised to capture a new bride for his younger brother \textit{Yuhpi Diani}. However, when \textit{Ñahori} returns with the woman, older brother \textit{Diani} lays claim to her. \textit{Ñahori} goes to \textit{Yuhpi Diani} in \textit{Khãnʉhko} to tell him what has happened, and an indignant \textit{Yuhpi Diani} tries to capture her back, but is unsuccessful. Angered and feeling betrayed, \textit{Yuhpi Diani} prepares an attack on \textit{Ñahori}, who manages to send his wives and children away to safety and bravely resists, but is eventually killed by \textit{Yuhpi Diani}’s men. They set fire to all the houses in \textit{Mukʉ Dʉhpʉri}, and the smoke is seen from afar by \textit{Diani} and the people of \textit{Bʉhka Khopa}. They go downriver to investigate and find the burned homes. Searching for survivors, they eventually entice the terrified wives and children out of hiding and then come across \textit{Ñahori}’s charred body. 

\largerpage
\textit{Diani} returns home and tells \textit{Dianumia Yairo} what has transpired. He ignores \textit{Dianumia Yairo}’s plea for the dispute not to escalate any further, and begins preparations to avenge \textit{Ñahori}’s death. \textit{Dianumia Yairo} reluctantly blesses \textit{Diani} and his warriers, embuing them with valor and violent spirit to ensure their victory in battle. They travel downstream and wage a furious attack on \textit{Yuhpi Diani} and his men in \textit{Khãnʉhko. Diani}’s men prevail, forcing \textit{Yuhpi Diani} to escape inland to \textit{Khãphotai}, where he and his men build a fortress with a high lookout platform from which\textit{Yuhpi Diani} hopes to be able to see his attackers approach. In the meantime, \textit{Dianumia Yairo} has come downriver and tries one last time to convince \textit{Diani} to call off the war, but to no avail. In the middle of the night, \textit{Diani}’s men transform themselves into worms and tunnel into the fortress. They trick \textit{Yuhpi Diani} into coming down from the platform, and subsequently capture and dismember him.

\textit{Diani} and his warriors go back to \textit{Bʉhka Khopa} to report their success and celebrate the victory. However, \textit{Dianumia Yairo} is saddened and disheartened by the terrible consequences of his sons’ failure to obey social norms — breaking promises and warring against each other — and so announces that he will remove himself to another world, taking with him sacred instruments, adornments, and knowledge. He gathers his sacred objects and together with his two jaguar-dogs, goes up a hill called \textit{Kharẽ Khutu} where he sits and slowly enters the ground alive. After a few days, tremendous thunder announces that he has entered completely, establishing \textit{Bu’i Kho’to}, the burial site for his own \textit{Biari} descendants. Teresinha explains that three other cemeteries were later established for the descendants of the other brothers, each group having its own proper burial place, as divided in death as their ancestors had become in life.\\ 

\begin{figure}[t]
\includegraphics[width=\textwidth]{figures/mapa-2-sacred-cemeteries.png}
\caption{Sites in the Sacred Cemeteries Narrative, by Miguel Cabral Junior.\label{maptwo}}
\end{figure}

Teresinha’s narrative takes us on a journey into Kotiria culture, and at the same time allows us to observe prominant features of Kotiria narrative discourse and grammatical structure.\footnote{Interesting examples and details of particular structures will be noted throughout the text; see \citealt{Stenzel2013} for a comprehensive descriptive grammatical analysis.} Like all Tukanoan languages, Kotiria is highly synthetic, agglutinative (except in certain verbal inflectional paradigms), and almost exclusively suffixing. It has nominative-accusative syntactic alignment and clear OV word order, with the position of subjects conditioned by discourse-pragmatic considerations. New, topical subjects (often full lexical noun phrases), tend to occur clause-initially, coinciding with the left-edge default focus position. Already known, continuing-topic subjects (commonly in pronominal form), tend to occur clause-finally, but any constituent whose identity is inferable from context can be a null element \parencite{Stenzel2015}. Clause-level grammatical relations are established by a combination of fairly rigid OV order, limited subject agreement morphology on verbs, and dependent-marking by means of a small case system. A single ‘objective case’ suffix \textit{–re} (glossed as \textsc{-obj}) occurs on all indirect objects and is differentially marked on direct objects; it is also found on many temporal and locative constituents. Referential status, interacting with distinctions of animacy and definiteness, is the key to understanding ‘objective case’ marking in this system \parencite{Stenzel2008b}. The other case markers are the locative suffixes \textit{-pʉ} or \textit{-i} and the clitic ={\textasciitilde}\textit{be’re}, marking NPs with commitative or instrumental semantic roles.

Kotiria has two basic word classes: nouns and verbs. Both adverbial and adjectival notions are formed from stative “quality" verbs that undergo nominalization in order to function as nominal predicates or as modifiers \parencite{Stenzelforthcoming}. Kotiria’s rich system of noun classification morphology — coding distinctions of countability, animacy, shape, and utility — permeates the grammar, performing a variety of concordial, derivational, and referential functions. Root serialization is extremely productive in verbal words, and is used to express a wide range of adverbial, aspectual, modal, and spatial distinctions \parencite{Stenzel2007a}. Verbal morphology includes optional polarity, modal, and aspectual markers, as well as obligatory inflection coding person, aspect, and “clause modality" for different sentential moods. All declarative (realis) statements must be marked by one of five categories of evidentiality: visual, non-visual, inference, assertion, or reported \parencite{Stenzel2008a,StenzelGomez2018}. 

Prominent characteristics of discourse include several types of linking mechanisms. Generic “summary-head" expressions, such as \textit{ãyoa} ‘so, then / doing that / because of that’ and \textit{ãni} ‘saying that’, respectively mark the close of event and dialog paragraphs, while full tail-head adverbial clause linking strategies create cohesion between sentences. These linking structures interact with a switch-reference marking system operating within the resulting complex sentences. Subordinate clauses with the same subject as the main clause are nominalized by cross-referencing noun class markers. If there is a change to a new subject accompanied by a shift in focus, the ‘different-subject’ suffix \textit{-chʉ} is used \parencite{Stenzel2015, Stenzel2016}. Teresinha’s discourse moreover shows that a specific nominalizer is used for broader “event" or “locational/situational" subordinate clauses.

There are several additional features of Teresinha’s discourse that deserve special mention. One of these is how she recreates events and dramatically underscores the fact that they actually occurred \textit{right there}, in the immediate surroundings, through her use of deictic elements (including a distal imperative (see line 110), spatial and motion expressions, onomatopoeia and gestures. It is also interesting to note how Teresinha interacts with her audience, stepping out of the role of narrator at several points to make comments, ask questions, or remind her listeners that they are themselves descendants of the story’s protagonists. The result is a mixture of epic narrative and highly personal commentary by a much esteemed and respected Kotiria elder. 

	Presentation of the narrative gives the Kotiria orthographic representation on the first line, and free translations in English and Portuguese on the final two lines. The second line gives the underlying, segmented representation that includes some important phonological information. Morphemic nasalization is indicated by a tilde [ {\textasciitilde} ] preceding an inherently [+nasal] morpheme, and an apostrophe indicates glottalization, which is perceived as a glottal stop and is often accompanied by laryngealization of vowels (see \citealt{Stenzel2007b}; \citealt{StenzelDemolin2013}). Tone is represented at the word level, with High tone indicated by the acute accent [ ´ ] and Low tone left unmarked. Phrase and sentence-level tonal phenomena, including sentence-final downstep patterns, are not represented. The third line gives corresponding glosses, with a list of non-standard abbreviations provided at the end of the text.

\newpage 
\subsection*{Introduction by José Galvez Trindade}
  
\ea
vinte hira, ã yoaro 20 ti khʉ'ma 2005 hichʉ̃.\\[.3em]
\gll vinte	hí-ra	{\textasciitilde}a=yóá-ro	ti={\textasciitilde}khʉ'bá	2005	{\textasciitilde}híchʉ\\
twenty	\textsc{cop-vis.ipfv.2/3}	so=do-\textsc{sg}	\textsc{anph}=year	2005	\textsc{temp}\\
\glt ‘It's the twentieth (of September) in the year 2005.’
\glt  ‘É dia vinte (de setembro) do ano 2005.’
\z

  
\ea sã a’ríkoro wamañokoro me'ne to durukuare thʉ'o yoana tana niha. \\[.3em]
\gll {\textasciitilde}sa=a’rí-koro	{\textasciitilde}wabáro-koro={\textasciitilde}be're	to=dú-ruku-a-re thʉ'ó	yoá-{\textasciitilde}dá	tá-{\textasciitilde}dá	{\textasciitilde}dí-ha\\
     \textsc{1pl.excl.poss=dem.prox-f.rsp}	father's.sister-\textsc{f.rsp=com}	\textsc{3sg.poss}=speak-stand-\textsc{pl-obj} hear	do-\textsc{pl}	come-\textsc{pl}	\textsc{prog-vis.ipfv.}1\\
\glt ‘We've come with our aunt{\footnotemark} (Teresinha) to listen to her stories (about our ancestors).’
\glt ‘Estamos com nossa tia (Teresinha) para ouvir suas histórias (dos ancestrais).’
\footnotetext{José uses the kinship term specifically for one's father's sister (a paternal aunt, real or classificatory).}  
\z 

\ea õi Carurui hiha.\\[.3em]
\gll {\textasciitilde}ó-í	caruru-i	hí-ha\\
     \textsc{deic.prox-loc.vis}	caruru-\textsc{loc.vis}	\textsc{cop-vis.ipfv.}1\\
\glt ‘We're here in Carurú Cachoeira.’
\glt ‘Estamos aqui em Carurú Cachoeira.’
\z

\largerpage
\section{Mahsa khõ'akho'topori}
\translatedtitle{‘Kotiria sacred cemeteries'}\\
\translatedtitle{‘Cemetérios sagrados dos Kotiria'}\footnote{Recordings of this story are available from \url{https://zenodo.org/record/997439}}

\ea  mahsa khõ'akho'topori\\[.3em]
\gll {\textasciitilde}basá {\textasciitilde}kho'á-kho'to-pori\\
     people bone-proper.place-\textsc{pl.}place\\
\glt ‘Sacred burial places/cemeteries’
\glt ‘Cemitérios sagrados’
\z 
  
\ea a’rina ñarananumia wi'i, yʉ durukuare thʉ'oduayu'ka.\\[.3em]
\gll a’rí-{\textasciitilde}da	{\textasciitilde}yará-{\textasciitilde}da-{\textasciitilde}dubia	wi'í	yʉ=dú-ruku-a-re	thʉ'ó-dua-yu'ka\\
     \textsc{dem.prox-pl}	white.people-\textsc{pl-pl.f}	arrive\textsc{.cis}	1\textsc{sg.poss}=speak-stand\textsc{-pl-obj}  	hear-\textsc{des-rep.quot}{\footnotemark}\\
\glt ‘These white ladies have come wanting to hear my stories.' (I was told)
\glt ‘Chegaram nossas assessoras brancas querendo ouvir minhas histórias.' (me disseram)
\footnotetext{\label{fn:kotiria:9}Use of reported evidentials is relatively rare in narratives. Teresinha employs the quotative evidential here, and in line 12, as a polite reported speech strategy indicating, but not directly identifying, the original speaker who had invited her to tell her stories. In fact, the invitation to speak came from José Trindade (Joselito), who was present and introduced her at the beginning of the recording. See lines 20-22, 26, and 173 for additional interesting uses of the reported evidential.}
\z 

 
\ea yʉ’ʉ tire michare ya'uko tako niha.\\[.3em]
\gll yʉ'ʉ́	tí-re	{\textasciitilde}bichá-ré	ya'ú-ko  	tá-ko	{\textasciitilde}dí-ha\\
     1\textsc{sg}	\textsc{anph-obj}  	today\textsc{-obj}  	tell-\textsc{f} 	come-\textsc{f}	\textsc{prog-vis.ipfv.}1\\
\glt ‘(So) today I'm coming to tell (them).’{\footnotemark}
\glt ‘(Então) hoje estou vindo contar.’
\footnotetext{This sentence contains two nominalized verbs, \textit{ya'u} ‘tell’, as the complement of the purposive construction ‘come to X’, and \textit{ta} ‘come’, as the complement of the progressive formed with the auxiliary copula \textit{ni}. This auxiliary copula is cognate to the primary copula found in many Tukanoan languages \parencite{StenzelGomez2018}, while the primary copula in Kotiria is \textit{hi}, seen in the next line. See lines 25, 289, and 299 for instances of \textit{ni} used with copular semantics, rather than as a component of the progressive construction.}
\z

\ea yʉ’ʉ hiha wãri khutiro biari pho'nakoro. \\[.3em]
\gll yʉ'ʉ́	hí-ha	{\textasciitilde}wárí khútíró bíári	{\textasciitilde}pho'dá-kó-ró \\
     1\textsc{sg}	\textsc{cop-vis.ipfv.}1	wãri khutiro biari	descendants{\footnotemark}\textsc{-f-sg}\\
\glt ‘I'm a descendant of \textit{wãri khutiro biari}.’
\glt ‘Sou descendente de \textit{wãri khutiro biari}.’
\footnotetext{The root \textit{pho’na} literally means ‘children’ or ‘offspring’ (e.g. in lines 75, 95, 97, and 117), but is used metaphorically here to refer to the descendants of a specific mythical ancestor, and in lines 52, 58 (and others) to refer to people over whom one has control, such as servants or warriers.}
\z 

\newpage
\ea yʉ wama Maria Teresinha Marques. \\[.3em]
\gll yʉ={\textasciitilde}wabá	Maria Teresinha Marques \\
     1\textsc{sg.poss}=name	Maria Teresinha Marques\\
\glt ‘My name (is) Maria Teresinha Marques.’
\glt ‘Meu nome (é) Maria Teresinha Marques.’
\z 

\ea yʉ wama so'toai hira yʉ’ʉre bu'sana phoko wamatiha yʉ’ʉ. \\[.3em]
\gll yʉ={\textasciitilde}wabá	so'tóá-í	hí-ra	yʉ’ʉ́-ré	bu'sáná phokó	{\textasciitilde}wabá-tí-há	yʉ’ʉ́ \\
     1\textsc{sg.poss}=name	end-\textsc{loc.vis}	\textsc{cop-vis.ipfv.}2/3	1\textsc{sg-obj}  	bu'sana phoko	name\textsc{-vbz-vis.ipfv.}1	1\textsc{sg} \\
\glt ‘(And){\footnotemark} this is my last (traditional) name, what I'm called is \textit{bu'sana phoko}.’
\glt ‘(E) meu sobrenome (nome tradicional), me chamam de \textit{bu'sana phoko}.’
\footnotetext{There are no overt conjunctions in Kotiria, so these and other elements understood from context are given in parentheses in the translation lines.}
\z 

\ea wãri khutiro biaripho'nakoro. \\[.3em]
\gll {\textasciitilde}wárí khútíró bíári-{\textasciitilde}pho'da-ko-ro \\
     wãri khutiro biari-descendants\textsc{-f-sg} \\
\glt ‘I'm a woman of the \textit{wãri khutiro biari} group.’
\glt ‘Sou mulher do grupo \textit{wãri khutiro biari}.’
\z 

\ea yʉ phʉkʉ yʉ’ʉre hire wĩhoa. \\[.3em]
\gll yʉ=phʉkʉ́	yʉ’ʉ́-ré	hí-re	{\textasciitilde}wihóá\\
     1\textsc{sg.poss}=father	1\textsc{sg}\textsc{-obj}  	\textsc{cop-}\textsc{vis.pfv.}2/3	wi͂hoa \\
\glt ‘For me, my father was \textit{wi͂hoa}.’ (I knew him as \textit{wi͂hoa}, his traditional name.)
\glt ‘Para mim, meu pai era \textit{wĩhoa}.’ (Eu o conheci como \textit{wi͂hoa}, seu nome tradicional.)
\z 
 
\ea tire a’rina thʉ'oduayu'ka. \\[.3em]
\gll tí-re	a’rí-{\textasciitilde}da	thʉ'ó-dua-yu'ka \\
     \textsc{anph-obj}  	\textsc{dem.prox-pl}	hear-\textsc{des-rep.quot} \\
\glt ‘These (visitors) want to hear (stories).’ (I’m told)
\glt ‘Esses (visitantes) querem ouvir (histórias).’ (Me disseram)
\z 

\ea ã yoako, yʉ’ʉ tire michadahchore noano me'ne a’rinakinare, yʉ mahsiapokane yʉ thʉ'otuare yʉ’ʉ durukukota. \\[.3em]
\gll {\textasciitilde}a=yóá-kó	yʉ’ʉ́	tí-re	{\textasciitilde}bichá-dáchó-ré	{\textasciitilde}dóá-ro={\textasciitilde}be're	a’rí-{\textasciitilde}da-{\textasciitilde}kida-re yʉ={\textasciitilde}basí-a-poka-re	yʉ=thʉ'ó-tu-a-re	yʉ’ʉ́	dú-ruku{\footnotemark}-ko-ta \\
     so=do-\textsc{f}	1\textsc{sg}	\textsc{anph-obj}  	today-day\textsc{-obj}  	good\textsc{-sg=com}	\textsc{dem.prox-pl-pl.rsp-obj} 1\textsc{sg.poss}=know-\textsc{pl}-little-\textsc{obj}  	1\textsc{sg.poss}=hear-think\textsc{-pl-obj}  	1\textsc{sg}	speak-stand-\textsc{f-intent} \\
\glt ‘So, today with pleasure I'm going to tell them a little of what I know, what I understand.’
\glt ‘Por isso vou contar hoje com prazer um pouco do que eu sei, do que entendo.’
\footnotetext{The serial verb construction \textit{du-ruku} ‘speak-stand’ means to talk for an extended period of time. It can be used to refer to a single speaker making a speech, bragging (see line 144), or offering a narrative, but can also indicate multiple speakers singing, chanting (see line 236), or having a conversation together. The posture verb \textit{duku} ‘stand’ (here [ruku], indicating lexicalization) contributes a continuative aspectual reading, and can occur with other verbs, e.g. ‘lie-stand' in line 131.}
\z

\ea a’ri hiri hire wa'manopʉre da'poto. \\[.3em]
\gll a’rí	hí-ri	hí-re	{\textasciitilde}wa'bá{\footnotemark}-ro-pʉ-re	da'pó-tó \\
     \textsc{dem.prox}	\textsc{cop-nmlz}	\textsc{cop-vis.pfv.}2/3	young/new\textsc{-sg-loc-obj}  	origin/roots-\textsc{nmlz.loc/evnt} \\
\glt (Sigh and moment of silence). ‘These are stories about what happened in the origin times.’
\glt (Suspiro e momento de silêncio). ‘Essas histórias são sobre o que aconteceu nos tempos de origem.’
\footnotetext{All “adjectival" notions in Kotiria are expressed by stative verbs, e.g. \textit{wa'ma} ‘to be young’ or ‘to be new’. To simplify the glosses, only the qualitative semantics, e.g. young/new/good/evil are given in the gloss line.} 
\z 

\ea pha'muri mahsa õre pha'muyohataa.  \\[.3em]
\gll {\textasciitilde}pha'bú-rí	{\textasciitilde}basá 	{\textasciitilde}ó-ré	{\textasciitilde}pha'bú-yóhá-tá-á \\
     originate-\textsc{nmlz}	people  	\textsc{deic.prox-obj}  	originate-go.upriver-come-\textsc{assert.pfv} \\
\glt ‘The origin beings appeared coming upriver here.’
\glt ‘Os seres de transformação apareceram subindo pelo rio aqui.’
\z 

 
\ea duhkusʉ̃ yʉ ñʉhchʉsʉmʉa wʉ'ʉsekho'topori khʉaa õre matapi. \\[.3em]
\gll dukú-{\textasciitilde}sʉ{\footnotemark}	yʉ={\textasciitilde}yʉchʉ́-{\textasciitilde}sʉ́bʉ́á	wʉ'ʉ́-sé(ri)-kho'to-pori	khʉá-a {\textasciitilde}ó-ré	matapi \\
     stand-arrive.\textsc{trns}	1\textsc{sg.poss}=grandfather-\textsc{pl}:kin	house-\textsc{pl:}row-proper.place-\textsc{pl}:place	have-\textsc{assert.pfv} \textsc{deic.prox-obj}  	matapi.village \\
\glt ‘Arriving, my ancestors established their houses (their rightful place) here in Matapi.’
\glt ‘Chegando, meus antepassados estabeleceram suas casas (seu lugar próprio) aqui em Matapi.’
\footnotetext{Kotiria has a number of motion verbs with specific deictic semantics, such as two “arrive" verbs: \textit{sʉ̃} ‘to arrive there (translocative motion)’, and \textit{wi’i} ‘to arrive here (cislocative motion)’ (some examples are lines 5, 16, and 31, among others). The distinction is indicated in the glosses. When used in serial verb constructions, as we see here, “arrive" verbs indicate the perfectivity or completedness of an event involving motion.}
\z 

\ea to hira, bʉhkakhopa wamatira.  \\[.3em]
\gll tó	hí-ra	bʉhkákhópá	{\textasciitilde}wabá-tí-rá \\
     \textsc{def}	\textsc{cop-}\textsc{vis.ipfv.}2/3	bʉhka.khopa	name-\textsc{vbz-vis.ipfv.}2/3 \\
\glt ‘That place is called \textit{bʉhkakhopa}.’
\glt ‘Esse lugar se chama \textit{bʉhkakhopa}.’
\z 
 
\ea ñarana nia ... matapi nina ti mahkare. \\[.3em]
\gll {\textasciitilde}yará-{\textasciitilde}dá	{\textasciitilde}dí-a	matapi	{\textasciitilde}dí-ra	ti={\textasciitilde}baká-re\\
     white.people-\textsc{pl}	say-\textsc{assert.pfv}	matapi.village	say-\textsc{vis.ipfv.}2/3	\textsc{anph}=village{\footnotemark}\textsc{-obj}  \\
\glt ‘White people say … (they) call that village Matapi.’
\glt ‘Os brancos dizem ... chamam essa aldeia de Matapi.’
\footnotetext{The nominal root \textit{mahka} indicates a ‘place of origin or belonging’, and when used with human referents, is understood to mean a ‘village'. When used with wild plants or animals, it may be understood to indicate the jungle or forest (as in line 66).}
\z 

\ea tina a’rina hiphitina, a’rina ñahoriapho'na wa'i khapea ñahoria hia sã ba'ana waro. \\[.3em]
\gll tí-{\textasciitilde}da	a’rí-{\textasciitilde}da	híphiti-{\textasciitilde}da	a’rí-{\textasciitilde}da	ñáhori-a-{\textasciitilde}pho'da wa'í khapea ñáhori-a	hí-a	{\textasciitilde}sa=ba'á-{\textasciitilde}da	wáró \\
     \textsc{anph-pl}	\textsc{dem.prox-pl}	everyone-\textsc{pl}	\textsc{dem.prox-pl}	ñahori{\footnotemark}-\textsc{pl}-descendants wa’i khapea ñahori-\textsc{pl} 	\textsc{cop-assert.pfv}	1\textsc{pl.excl.poss}=younger.brother-\textsc{pl}	\textsc{emph}\\
\glt ‘All of these, descendants (children), \textit{wa'i khapea ñahoria}, are our true younger brothers.’
\glt ‘Todos esses, descendentes (filhos), \textit{wa'i khapea ñahoria}, são nossos irmãos menores verdadeiros.’
\footnotetext{To facilitate recognition, the proper names of the main protagonists and places have the same underlying and surface representations, e.g. \textit{ñahori} rather than {\textasciitilde}\textit{yahori}.}
\z 

\ea biari, yabaina, ñahoripho'na hiyu'ka. \\[.3em]
\gll biári	yabá-{\textasciitilde}ida	ñáhori-{\textasciitilde}pho'da	hí-yu'ka \\
     biari 	\textsc{q:}what/how-\textsc{nmlz.pl}	ñahori-descendants	\textsc{cop-rep.quot} \\
\glt ‘(Younger brothers of the) \textit{biari} (Teresinha’s group), the \textit{ñahori} descendants are.’ (they’re saying)
\glt ‘(Irmãos menores dos) \textit{biari} (grupo da Teresinha), os \textit{ñahori} são.’ (estão dizendo)
\z

\ea tina sã dohka mahkarikurua hiyu'ka a’rina. \\[.3em]
\gll tí-{\textasciitilde}da	{\textasciitilde}sa=doká	{\textasciitilde}baká-ri-kuru-a	hí-yu'ka	a’rí-{\textasciitilde}da \\
     \textsc{anph-pl}	1\textsc{pl.excl.poss}=below	origin-\textsc{pl}-group-\textsc{pl}	\textsc{cop-rep.quot}	\textsc{dem.prox-pl} \\
\glt ‘They are the younger second group.’ (they’re saying){\footnotemark}
\glt ‘Eles são irmãos menores.’ (estão dizendo)
\footnotetext{Teresinha’s use of the quotative reported evidential in this sequence of sentences is another reported speech strategy, given that throughout the workshop there had been a great deal of discussion about different Kotiria groups. It is also likely a way of softening this series of statements related to group rankings, especially since the audience was mostly composed of people belonging to a group she states to be lower than her own.}
\z 

\newpage 
\ea tiro õi wʉ'ʉsekho'topori khʉayu'ka, mukʉdʉhpʉri a’rina sã ba'ana. \\[.3em]
\gll tí-ró	{\textasciitilde}ó-í	wʉ'ʉ́-sé(ri)-kho'to-pori	khʉá-yu'ka múkʉ.dʉhpʉri	a’rí-{\textasciitilde}da	{\textasciitilde}sa=ba'á-{\textasciitilde}da \\
     \textsc{anph-sg}	\textsc{deic.prox-loc.vis}	house-\textsc{pl.}row-proper.place-\textsc{pl}.place	have-\textsc{rep.quot} mukʉ.dʉhpʉri	\textsc{dem.prox-pl}	1\textsc{pl.excl.poss}=younger.brother-\textsc{pl} \\
\glt ‘He (\textit{Ñahori}) had his houses here in \textit{Mukʉ Dʉhpʉri} (and) these ones here (Carurú villagers are) our younger brothers.’ (they’re saying). 
\glt ‘Ele (\textit{Ñahori}) tinha suas casas aqui em \textit{Mukʉ Dʉhpʉri} (e) esses aqui (moradores de Carurú são) nossos irmãos menores.’ (estão dizendo)
\z 

\ea phʉaro numia, phʉaro numia ti phapʉre namotia tire himarebʉ, tiaro numiapʉ bʉhkʉthurupʉre. \\[.3em]
\gll phʉá-ro	{\textasciitilde}dúbí-á	phʉá-ro	{\textasciitilde}dúbí-á	ti=phá-pʉ-re	{\textasciitilde}dabó-tí-á  tí-re	hí-{\textasciitilde}bare-bʉ{\footnotemark}	tiá-ro	{\textasciitilde}dúbí-á-pʉ́	bʉkʉ́-thúrú-pʉ́-ré\\
     two\textsc{-sg}	woman\textsc{-pl}	two\textsc{-sg}	woman-\textsc{pl}	\textsc{anph}=time-\textsc{loc-obj}  	wife-\textsc{vbz-assert.pfv} \textsc{anph-obj}  	\textsc{cop-rem.ipfv-epis}	three\textsc{-sg}	woman-\textsc{pl-loc}	ancestor-times-\textsc{loc-obj}  \\
\glt ‘In those olden times, the custom was to marry two wives, two or even three.’
\glt ‘Na época antiga, era costume casar com duas mulheres, duas, ou até três.’
\footnotetext{The expression \textit{himarebʉ} occurs several times in Teresinha’s narrative, indicating ways of being or events as specifically related to “origin" times. These may be customs that contrast with current social norms (e.g. having more than one wife), or capabilities that humans nowadays no longer possess, such as the ability to transform themselves into other kinds of beings (see line 197). The expression is clearly formed with the copula \textit{hi}, followed by a morpheme \textit{-ma} or \textit{-mare}, which is analyzed here as ‘remote imperfective aspect’, but is possibly related to the morpheme \textit{ma} used in conversation to show respect for one’s interlocutor. The final morpheme -\textit{bʉ} is cognate to evidential markers in other Tukanoan languages \parencite{StenzelGomez2018} but is not regularly found in the Kotiria evidential system. The expression as a whole seems to indicate the speaker’s authoritative knowledge about such times and customs, and thus is interpreted as having epistemic value: ‘this was how it \textit{was} in those times’. In line 187 it occurs with the initial verb \textit{ni} ‘say’, indicating ‘this was what X was \textit{called} in those times’.} 
\z 

\newpage 
\ea namoti a’ri, õre matapi wʉ'ʉkho'tori tiro hiyu'ka bʉhkʉro, mari ñʉhchʉ dianumia yairo. \\[.3em]
\gll {\textasciitilde}dabó-tí	a’rí	{\textasciitilde}ó-ré	matapí	wʉ'ʉ́-kho'to-ri	tí-ró bʉkʉ́-ró	{\textasciitilde}bari={\textasciitilde}yʉchʉ́	diánúmíá.yáíró \\
     wife-\textsc{vbz}	\textsc{dem.prox}	\textsc{deic.prox-obj}  	matapi.village	house-proper.place-\textsc{pl}	\textsc{anph-sg} ancestor\textsc{-sg}	1\textsc{pl.incl.poss}=grandfather	dianumia.yairo \\
\glt ‘The old one, our grandfather (ancestor) \textit{Dianumia Yairo}, was married (and) he had houses there in Matapi.’ (they’re saying)
\glt ‘O velho, nosso avô (ancestral) \textit{Dianumia Yairo}, era casado (e) tinha casas alí em Matapi.’ (estão dizendo)
\z 

\ea phayʉ mahsa hia niatia. \\[.3em]
\gll phayʉ́	{\textasciitilde}basá	hí-a	{\textasciitilde}dí-ati-a \\
     many	people	\textsc{cop-pl}	\textsc{cop-ipfv-assert.pfv} \\
\glt ‘There were lots of people living there.’
\glt ‘Havia muita gente vivendo alí.’
\z 

\ea ã yoa õre hiyu'ka tiro ñahori, a’riro yuhpi dianine: \\[.3em]
\gll {\textasciitilde}a=yóá	{\textasciitilde}ó-ré	hí-yu'ka	tí-ró	ñáhórí	a’rí-ro	yuhpí.diáni-re \\
     so=do	\textsc{deic.prox-obj}	\textsc{cop-rep.quot}	\textsc{anph-sg}	ñahori	\textsc{dem.prox-sg}	yuhpi.diani\textsc{-obj} \\
\glt ‘So (they’re saying) that \textit{Ñahori} lived just there (in \textit{Mukʉ Dʉhpʉri} and said to) \textit{Yuhpi Diani}:
\glt ‘Então (estão dizendo que) \textit{Ñahori} vivia logo alí (em \textit{Mukʉ Dʉhpʉri} e disse ao) \textit{Yuhpi Diani}:
\z 

\ea “numia yʉ’ʉ́ nai wa'ai niha, yʉ buhibo.” \\[.3em]
\gll {\textasciitilde}dúbí-á	yʉ’ʉ́	{\textasciitilde}dá-i	wa'á-i	{\textasciitilde}dí-ha	yʉ=buhíbo \\
     woman-\textsc{pl}	1\textsc{sg}	get-\textsc{m}	go-\textsc{m}	\textsc{prog-vis.ipfv.}1	1\textsc{sg.poss}=sister.in.law \\
\glt ‘“I'm going to get women, my sisters-in-law.”’
\glt ‘“Estou indo pegar mulheres, minhas cunhadas.”’ 
\z

\newpage 
\ea nichʉ, tirota sinikaatia: \\[.3em]
\gll {\textasciitilde}dí-chʉ	tí-ró-ta	{\textasciitilde}sidí-ka’a-ati-a \\
     say-\textsc{sw.ref}	\textsc{anph-sg-emph}	ask.for-do.immediately-\textsc{ipfv-assert.pfv} \\
\glt ‘When (\textit{Ñahori}) said that, (\textit{Yuhpi Diani}) asked:’
\glt ‘Quando (\textit{Ñahori}) falou isso, (\textit{Yuhpi Diani}) pediu:’
\z 

\ea “yʉ’ʉkhʉre kʉ̃koro natanamoa” niatia a’rina ñahoriapho'na. \\[.3em]
\gll yʉ’ʉ́-khʉ́-ré	{\textasciitilde}kʉ́-kó-ró	{\textasciitilde}dá-ta-{\textasciitilde}dabo-a	{\textasciitilde}dí-ati-a	a’rí-{\textasciitilde}da	ñahóri-a-{\textasciitilde}pho'da \\
     1\textsc{sg-add-obj}  	one/a-\textsc{f-sg}	get-come-wife-\textsc{pl}	say-\textsc{ipfv-assert.pfv}	\textsc{dem.prox-pl}	ñáhórí-\textsc{pl}-descendants \\
\glt ‘“Bring one more wife for me too,” (he asked to \textit{Ñahori}, the father) of all these \textit{Ñahoria} descendants here.’
\glt ‘“Traga mais uma para mim também,” (pediu ao \textit{Ñahori}, pai) desses descendentes de \textit{Ñahoria} aqui.’
\z 

\ea ñariro tuakaro kha'mapha, hum ... hum. \\[.3em]
\gll {\textasciitilde}yá-rí-ró	túá-ká-ró	{\textasciitilde}kha'bá-pha	hum...hum \\
     bad-\textsc{nmlz-sg}	strong-\textsc{intens-sg}	want-\textsc{spec}	(laughs)\\
\glt ‘I guess the rascal (\textit{Yuhpi Diani}) really liked/needed (women), hum … hum.’ [Teresinha chuckles]
\glt ‘Parece que o danado (\textit{Yuhpi Diani}) gostava/precisava mesmo (de mulher), hum ... hum.’ [risos da Teresinha]
\z 
 
\ea ã yoa tiro nano wa'a, nawi'i, wehse wa'a te õpʉ. \\[.3em]
\gll {\textasciitilde}a=yóá	tí-ró	{\textasciitilde}dá-ro	wa'á	{\textasciitilde}dá-wi'i	wesé	wa'á	té	{\textasciitilde}ó-pʉ́ \\
     so=do	\textsc{anph-sg}	get\textsc{-sg}	go	get-arrive\textsc{.cis}	garden	go	until 	\textsc{deic.prox-loc} \\
\glt ‘So, he (\textit{Ñahori}) went to get them, brought (them) back, and went off to the gardens over there (in \textit{Mukʉ Dʉhpʉri}).’
\glt ‘Então ele (\textit{Ñahori}) foi pegar, trouxe de volta (e) foram lá para a roça (em \textit{Mukʉ Dʉhpʉri}).’
\z 

\newpage 
\ea a’rina ñahoria yaro wʉ'ʉkho'to hira õ mukʉ dʉhpʉri.  \\[.3em]
\gll a’rí-{\textasciitilde}da	ñáhórí-á	yá-ro	wʉ'ʉ́-khó'tó	hí-ra	{\textasciitilde}ó	múkʉ́.dʉhpʉri \\
     \textsc{dem.prox-pl}	ñahori-\textsc{pl}	\textsc{poss-sg}	house-proper.place	\textsc{cop-vis.ipfv.}2/3	\textsc{deic.prox}	mukʉ.dʉhpʉri \\
\glt ‘\textit{Mukʉ Dʉhpʉri} is the rightful place of these \textit{ñahoria}.’ 
\glt ‘\textit{Mukʉ Dʉhpʉri} é o lugar desses \textit{ñahoria}.’ 
\z 

 
\ea sekhoa ti yadi'ta te hiro nina to khũre. \\[.3em]
\gll (bu')sé-khoa	ti=yá-dí'tá	té	hí-ro	{\textasciitilde}dí-ra	tó	{\textasciitilde}khú-re \\
     side-half/part	3\textsc{pl.poss=poss}-ground	until	\textsc{cop-sg}	\textsc{prog-vis.ipfv.}2/3	\textsc{def}	place\textsc{-obj} \\
\glt ‘Their place is on the other side of the river, and that place is still theirs.’
\glt ‘O lugar deles fica no outro lado do rio (e) esse lugar ainda é deles.’
\z 

\ea no'o ti tachʉ khũre, noa a’rinare ñahoriapho'nare “do'se yoa ta hihari?” \\[.3em]
\gll {\textasciitilde}do'ó	ti=tá-chʉ	{\textasciitilde}khú-re	{\textasciitilde}doá	a’rí-{\textasciitilde}da-re	ñáhórí-a-{\textasciitilde}pho'da-re do'sé	yoá	ta	hí-hari \\
     \textsc{q}:where{\footnotemark}	3\textsc{pl.poss}=come-\textsc{sw.ref}	place\textsc{-obj}	\textsc{q:}who	\textsc{dem.prox-pl-obj}	ñahori-\textsc{pl}-descendants\textsc{-obj} \textsc{q}:how	do	come	\textsc{cop-q.ipfv} \\
\glt ‘If they were to come back there, no one could ask \textit{Ñahori}’s descendants “Why do you come (what are you doing) here?”’
\glt ‘Se um dia voltassem ao lugar, ninguém podia perguntar aos descendantes de \textit{Ñahori} “Porque vieram (o que fazem) aqui?”’
\footnotetext{This sentence contains three different question words glossed with their most common semantics, although most can occur with other morphology to derive other interrogative meanings. For example \textit{no'o} ‘where’, can combine with morphemes \textit{-pe} or \textit{-puro}, deriving quantity question forms ‘how much/many?’, \textit{do'se} ‘how’ also occurs in the expression \textit{do'se hichʉ̃} ‘when?’ and \textit{do'se yoa} ‘why?' (e.g. in line 273).} 
\z

\ea ne nito bahsioerara. \\[.3em]
\gll {\textasciitilde}dé	{\textasciitilde}dí-to	basio-éra-ra \\
     \textsc{neg}	say-\textsc{nmlz.loc/evnt}	correct-\textsc{neg-vis.ipfv.}2/3 \\
\glt ‘No one could say that.’ 
\glt ‘Ninguem poderia dizer isso.’
\z 

\newpage
\ea do'poto	to hiro	hia. \\[.3em]
\gll do'pó-to	to=hí-ro	hí-a \\
     origin/roots-\textsc{nmlz.loc/evnt}	3\textsc{sg.poss=cop-sg} 	\textsc{cop-assert.pfv} \\
\glt ‘It's his (\textit{Ñahori}’s) origin site.’
\glt ‘É o lugar de origem dele (\textit{Ñahori})’.
\z 


\ea ã yoa mahaa.  \\[.3em]
\gll {\textasciitilde}a=yóá	{\textasciitilde}bahá-a \\
     so=do	go.uphill-\textsc{assert.pfv} \\
\glt ‘So, (\textit{Ñahori} and the woman) came up (towards \textit{Yuhpi Diani}).
\glt ‘Então (eles, \textit{Ñahori} e a mulher) vinham subindo (na direção ao \textit{Yuhpi Diani}).’
\z 

\ea “a’ríkoro yʉ’ʉre numia na, yʉ’ʉ duhtirikoro tire a’rikoro” nia tiro. \\[.3em]
\gll a’rí-kó-ró	yʉ’ʉ́-ré	{\textasciitilde}dúbí-á	{\textasciitilde}dá yʉ’ʉ́	dutí-ri-ko-ro	tí-re	a’rí-kó-ró	{\textasciitilde}dí-a	tí-ró \\
     \textsc{dem.prox-f-sg}	\textsc{1sg-obj}	woman-\textsc{pl}	get 1\textsc{sg}	order-\textsc{nmlz-f-sg}	\textsc{anph-obj}	\textsc{dem.prox-f-sg}	say-\textsc{assert.pfv}	\textsc{anph-sg} \\
\glt ‘“This woman's for me, the one I asked you to get,” he (\textit{Yuhpi Diani}) said.’
\glt ‘“Essa mulher é minha, a que mandei pegar,” dizia ele (\textit{Yuhpi Diani}).’
\z 
 
\ea a’riro ñahori: “hierare. soropʉ diani yakoro hira” nia. \\[.3em]
\gll a’rí-ro	ñáhórí	hi-éra-re sóró-pʉ́	diáni	yá-kó-ró	hí-ra	{\textasciitilde}dí-a \\
     \textsc{dem.prox-sg}	ñahori	\textsc{cop-neg-vis.pfv.}2/3 different.one-\textsc{loc}	diani	\textsc{poss-f-sg}	\textsc{cop-vis.ipfv.}2/3	say-\textsc{assert.pfv} \\
\glt ‘\textit{Ñahori} (speaking): “Not so, this one is promised to \textit{Diani},” he said.’
\glt ‘O \textit{Ñahori}  (falando): “Não, essa ficou prometida para \textit{Diani},” respondeu.’
\z 

\ea “hierare. yʉ’ʉ duhtii” nia. \\[.3em]
\gll hi-éra-re	yʉ’ʉ́	dutí-i	{\textasciitilde}dí-a \\
     \textsc{cop-neg-vis.pfv.}2/3	1\textsc{sg}	order-\textsc{vis.pfv.}1	say-\textsc{assert.pfv} \\
\glt ‘“Not so! I ordered (her),” (\textit{Yuhpi Diani}) said.’
\glt ‘“Nada disso! Eu mandei pegar,” respondeu (\textit{Yuhpi Diani}).’
\z 

\newpage
\ea “yʉ’ʉ duhtii. yʉ yakoro hika” nia.  \\[.3em]
\gll yʉ’ʉ́	dutí-i	yʉ=yá-kó-ró	hí-ka	{\textasciitilde}dí-a\\
     1\textsc{sg} order-\textsc{vis.pfv.}1	1\textsc{sg.poss=poss-f-sg}	\textsc{cop-assert.ipfv}	say-\textsc{assert.pfv}{\footnotemark} \\
\glt ‘“I ordered (her). She's mine!” (\textit{Yuhpi Diani}) said.’ 
\glt ‘“Eu mandei (pegar). É minha mulher!” disse (\textit{Yuhpi Diani}).’ 
\footnotetext{The unusual use of the assertive evidential \textit{-a} in this sentence (rather than the expected visual form \textit{-i}, as in the previous line) is quite interesting. It is possible that use of the assertive form gives the statement greater legitimacy, coding it as already internalized “fact" or as collectively recognized knowledge.} 
\z

\ea “hierara” nia. \\[.3em]
\gll hi-éra-ra	{\textasciitilde}dí-a \\
     \textsc{cop-neg-vis.ipfv.}2/3	say-\textsc{assert.pfv} \\
\glt ‘“No (she's) not,” (\textit{Ñahori}) said.’
\glt ‘“Não é,” disse (\textit{Ñahori}).’
\z 

\ea ni, wehse wa'a, thuatarikorore ña'atia.  \\[.3em]
\gll {\textasciitilde}dí	wesé	wa'á	thuá-ta-ri-ko-ro-re	{\textasciitilde}ya'á-ti-a \\
     say	garden	go	return-come-\textsc{nmlz-f-sg-obj}	grab-\textsc{attrib-assert.pfv} \\
\glt ‘That said, (she) went to the garden (and) returning, was grabbed.’ 
\glt ‘Ditto isso, (ela) foi para a roça (e) voltando, foi agarrada.’
\z 

\ea “thuaga yʉ’ʉ me'ne” nia. \\[.3em]
\gll thúá-gá	yʉ’ʉ́={\textasciitilde}be're	{\textasciitilde}dí-a \\
     stay-\textsc{imp}	1\textsc{sg=com}	say-\textsc{assert.pfv} \\
\glt ‘“(You) stay with me,” said (\textit{Diani}).’
\glt ‘“Fica comigo,” disse (\textit{Diani}).’
\z 

\ea “hierara. khã'i nire. pakoro nabosaita.”  \\[.3em]
\gll hi-éra-ra	{\textasciitilde}kha'í	{\textasciitilde}dí-re pá-ko-ro	{\textasciitilde}dá-bosa-i-ta \\
     \textsc{cop-neg-vis.ipfv.}2/3	desire/love	\textsc{prog-vis.pfv.}2/3 \textsc{alt-f-sg}	get\textsc{-ben-m-intent}\\
\glt ‘“(She) isn’t (yours). You want (her but) I'll get another one for you,” (said \textit{Ñahori}).’
\newpage
\glt ‘“(Essa) não é (sua). Você quer ficar (com ela mas) vou trazer outra para você,” (disse \textit{Ñahori}).’
\z 

\ea “hierara” nia. \\[.3em]
\gll hi-éra-ra	{\textasciitilde}dí-a \\
     \textsc{cop-neg-vis.ipfv.}2/3	say-\textsc{assert.pfv} \\
\glt ‘“No,” said (\textit{Diani}).’ 
\glt ‘“Não,” respondia (\textit{Diani}).’
\z 

\ea ni, tikorore thũkukaa.  \\[.3em]
\gll {\textasciitilde}dí	tí-kó-ró-ré	{\textasciitilde}thúkuka-a \\
     say	\textsc{anph-f-sg-obj}	force-\textsc{assert.pfv} \\
\glt ‘Saying that, (\textit{Diani}) grabbed her.’ 
\glt ‘Dizendo isso, (\textit{Diani}) segurou a mulher.’
\z 

\ea de tikoro bʉhkoro kʉ̃korotha thuaa.\\[.3em]
\gll dé	tí-kó-ró	bʉkó-ro	{\textasciitilde}kʉ́-ko-ro-ta	thúa-a\\
     \textsc{intj}:poor.one!	\textsc{anph-f-sg}	ancestor.\textsc{f}\textsc{-sg} 	alone-\textsc{f-sg-emph}	stay-\textsc{assert.pfv}\\
\glt ‘Poor thing! The old gal was all alone.’
\glt ‘Coitada da velha, estava sozinha.’
\z 
\ea sʉ̃ to manʉnore yabare, khanʉhkore hiphato tiro Yuhpi Diani: \\[.3em]
\gll {\textasciitilde}sʉ́	to={\textasciitilde}badʉ́-ro-re	yabá-re	khá-{\textasciitilde}dʉ́kó-ré	hí-pha-to	tí-ró yuhpi.diani\\
     arrive.\textsc{trns}	3\textsc{sg.poss}=husband\textsc{-sg-obj}	\textsc{q:}what\textsc{-obj}	hawk-island\textsc{-obj}	\textsc{cop-}time-\textsc{nmlz.loc/evnt}	\textsc{anph-sg} yuhpi.diani\\
\glt ‘(\textit{Ñahori}) went there to her intended husband in \textit{Kha Nʉhko} (Hawk Island), where \textit{Yuhpi Diani} lived (and said):’
\glt ‘(\textit{Ñahori}) chegou ao marido pretendido dela em \textit{Kha Nʉhko} (Ilha de Inambú), onde \textit{Yuhpi Diani} morava (e disse):’
\z 

\largerpage[2]
\ea “mʉ namore yoatapʉ mʉ mahsawamino ña'a thũkukare” nia. \\[.3em]
\gll{\textasciitilde}bʉ={\textasciitilde}dabó-re	yoá-tá-pʉ́	{\textasciitilde}bʉ={\textasciitilde}basá-{\textasciitilde}wabi-ro	{\textasciitilde}ya'á {\textasciitilde}thúkuka-re	{\textasciitilde}dí-a\\
     2\textsc{sg.poss}=wife\textsc{-obj}	far-\textsc{emph-loc}	2\textsc{sg.poss}=people-older.brother\textsc{-sg}	grab force-\textsc{vis.pfv.}2/3	say-\textsc{assert.pfv} \\
\glt ‘“Your (intended) wife's gone, your older brother’s people captured her,” said (\textit{Ñahori}).’
\newpage 
\glt ‘“Sua esposa (prometida) já foi, o povo do seu irmão maior pegou,” disse (\textit{Ñahori}).’
\z 

%longdistance
\ea “kue! do'se yʉ’ʉ tirore ã yoa duhtierai yʉ’ʉ” nia. \\[.3em]
\gll kúé	do'sé	yʉ’ʉ́	tí-ró-ré	{\textasciitilde}a=yóá	duti-éra-i	yʉ’ʉ́	{\textasciitilde}dí-a \\
     \textsc{intj:}surprise	\textsc{q:}how	1\textsc{sg}	\textsc{anph-sg-obj}	so=do	order-\textsc{neg-vis.pfv.}1	1\textsc{sg} say-\textsc{assert.pfv}\\
 
\glt ‘“What? How? I forbade him do that!” (\textit{Yuhpi Diani}) said.’
\glt ‘“O que? Como? Eu o proibi a fazer isso!” (\textit{Yuhpi Diani}) falou.’
\z 

\ea ni to phosapho'nakãre, kʉ̃nʉmʉ ba'aro wa'arokaa, phosa pho'nakã phʉaro. \\[.3em]
\gll {\textasciitilde}dí	to=phosá-{\textasciitilde}pho'da-{\textasciitilde}ka-re	{\textasciitilde}kʉ́-{\textasciitilde}dʉ́bʉ́=ba'a-ro	wa'á-dóká-á phosá-{\textasciitilde}pho'da-{\textasciitilde}ka	phʉá-ro \\
     say	3\textsc{sg.poss}=maku.people-descendants\textsc{-dim-obj}	one/a-day=after\textsc{-sg}	go-\textsc{dist-assert.pfv} maku.people-descendants-\textsc{dim}	two\textsc{-sg}\\
\glt ‘Saying that to his servants,{\footnotemark} the next day two of them went (to check).’
\glt ‘Falando disso aos seus criados, no dia seguinte, os dois foram (verificar).’
\footnotetext{The “servants" referred to here are identified as people of one of the Makú ethnic groups, reflecting the unequal social relations between dominant riverine Tukano and Arawak groups and the forest-dwelling peoples, speakers of what are now referred to as languages of the Nadahup and Kákua-Nʉkak families (\citealt{Epps2008}; \citealt{Bolaños2016}). Use of the diminutive suffix -{\textasciitilde}\textit{ka} highlights their smaller physical stature when compared to the riverine peoples (see lines 58-59), but is also used metaphorically to indicate diminished social status.} 
\z 

\ea tikoro kowaro bu'atarikorore pharitaropʉ, tikorore ña'aa tina. \\[.3em]
\gll tí-kó-ró	kówa-ro	bu'á-ta-ri-ko-ro-re	pharí-taro-pʉ tí-kó-ró-ré	{\textasciitilde}ya'á-a	tí-{\textasciitilde}da \\
     \textsc{anph-f-sg}	fetch.water\textsc{-sg}	go.downhill-come-\textsc{nmlz-f-sg-obj}	form-\textsc{clf}:lake-\textsc{loc} \textsc{anph-f-sg-obj}	grab-\textsc{assert.pfv}	\textsc{anph-pl} \\
\glt ‘When the woman came down to fetch water from the pond, they grabbed her.’
\glt ‘Quando a mulher desceu para pegar água no poço, eles a agarraram.’
\z 

\ea thunua tikorokhʉ, “yʉ’ʉre ñaenatiga! soro mahsawa'mino yakoropʉ hiha” nia. \\[.3em]
\gll {\textasciitilde}thudú-a	tí-kó-ró-khʉ	yʉ’ʉ́-ré	{\textasciitilde}ya'a-éra-tiga sóro	{\textasciitilde}basá-{\textasciitilde}wa'bi-ro	yá-ko-ro-pʉ	hí-ha	{\textasciitilde}dí-a \\
     resist-\textsc{assert.pfv}	\textsc{anph-f-sg-add}	1\textsc{sg-obj}	grab-\textsc{neg-neg.imp} different.one	people-older.brother\textsc{-sg}	\textsc{poss-f-sg-loc}	\textsc{cop-vis.ipfv.}1	say-\textsc{assert.pfv} \\
\glt ‘She resisted them: “Let me go! I'm reserved for one from the higher clan,” she said.’
\glt ‘Ela resistiu: “Solta-me! Estou prometida para um do clã maior,” disse.’
\z 

\ea “mʉ'ʉre phikare khero mʉ manʉ” nima. \\[.3em]
\gll {\textasciitilde}bʉ'ʉ́-ré	phí-ka('a)-re	khé-ro	{\textasciitilde}bʉ={\textasciitilde}badʉ́	{\textasciitilde}dí-{\textasciitilde}ba \\
     2\textsc{sg-obj}	call-do.immediately-\textsc{vis.pfv.}2/3	fast\textsc{-sg}	2\textsc{sg.poss}=husband	say-\textsc{frus} \\
\glt ‘“Your (true) husband is calling you right now,” said (the servants, to no avail).’
\glt ‘“O seu marido (verdadeiro) está chamando rápido,” disseram (em vão, os criados).’
\z

\ea ne kha'maeraa.\\[.3em]
\gll {\textasciitilde}dé	{\textasciitilde}kha'ba-éra-a\\
     \textsc{neg}	want-\textsc{neg-assert.pfv}\\
\glt ‘[(She) didn't want (to go)!]’{\footnotemark}
\glt ‘[(Ela) não queria (ir)!]’
\footnotetext{[ ] indicates a personal comment or observation by Teresinha or Joselito.}
\z 

\ea kha'maera, thunuwihtika'aa ña'ano.\\[.3em]
\gll {\textasciitilde}kha'ba-éra	{\textasciitilde}thudú-witi-ka'a-a	{\textasciitilde}ya'á-ro{\footnotemark}\\
     want\textsc{-neg}	resist-escape-do.moving-\textsc{assert.pfv}	grab\textsc{-sg}\\
\glt ‘(She) didn't want (to go), resisted (and) escaped (their) embrace.’
\glt ‘Não queria ir, resistiu (e) escapou do agarro (deles).’
\footnotetext{The suffix \textit{-ro} derives a nominal count noun from a verbal root; here, the noun ‘embrace’ is derived from the verb for ‘grab’.}
\z 

\ea to pho'na õ ma'ainakã hia a’rina ñahoria khi'ti.\\[.3em]
\gll to={\textasciitilde}pho'dá	{\textasciitilde}ó	{\textasciitilde}ba'á-{\textasciitilde}ídá-{\textasciitilde}ká	hí-a	a’rí-{\textasciitilde}da	ñáhórí-á khí'ti\\
     3\textsc{sg.poss}=children	\textsc{deic.prox}	small-\textsc{nmlz.pl-dim}	\textsc{cop-assert.pfv}	\textsc{dem.prox-pl}	ñahori-\textsc{pl}	become\\
\glt ‘His (\textit{Yuhpi Diani}'s) servants, [the future \textit{Ñahoria}] were just this tall.’
\glt ‘Os criados dele (\textit{Yuhpi Diani}) [os \textit{Ñahoria} do futuro] eram deste tamanhinho.’
\z 

 
\ea phʉarokã hia.\\[.3em]
\gll phʉá-ro-{\textasciitilde}ka	hí-a\\
     two\textsc{-sg-dim}	\textsc{cop-assert.pfv}\\
\glt ‘There were two little guys.’
\glt ‘Eram dois pequenos.’
\z 

\ea mʉmʉ mʉhawa'aa tinakã:\\[.3em]
\gll {\textasciitilde}bʉb́ʉ	{\textasciitilde}bʉhá-wa'a-a	tí-{\textasciitilde}da-{\textasciitilde}ka\\
     run	\textsc{mov.}upward-go-\textsc{assert.pfv}	\textsc{anph-pl-dim}\\
\glt ‘They ran up (to tell \textit{Yuhpi Diani}):’
\glt ‘Foram correndo para cima (para avisar ao \textit{Yuhpi Diani}):’
\z 

\ea “de yoatapʉ mʉ namore ña'a tu'sʉre.”\\[.3em]
\gll dé	yoá-ta-pʉ	{\textasciitilde}bʉ={\textasciitilde}dabó-re	{\textasciitilde}ya'á-tu'sʉ-re\\
     \textsc{intj:}poor.one!	far-\textsc{emph-loc}	2\textsc{sg.poss}=wife\textsc{-obj}	grab-finish-\textsc{vis.pfv.}2/3\\
\glt ‘“It’s too late, your wife's already taken.”’
\glt ‘“Já era, pegaram sua mulher.”’ 
\z 

\ea ti ñari, tiro hi'na ta bʉea me'ne bʉepati hi'na.\\[.3em]
\gll tí-{\textasciitilde}ya-ri	tí-ró	{\textasciitilde}hí'da	tá	bʉé-a={\textasciitilde}be're	bʉé-(pa)ti{\footnotemark}	{\textasciitilde}hí'da\\
     \textsc{anph}-bad-\textsc{nmlz}	\textsc{anph-sg}	\textsc{emph}	come	arrow-\textsc{pl=com}	arrow-\textsc{vbz}	\textsc{emph}\\
\glt ‘Those servants (and) he (\textit{Yuhpi Diani}) himself came back with arrows and started shooting.’
\glt ‘Os criados (e) ele (\textit{Yuhpi Diani}) mesmo vieram de volta com flechas e começaram a flechar.’
\footnotetext{The verbalizing suffix \textit{-pati} may be an older, longer form of the now more commonly used verbalizer \textit{-ti}, seen in lines 9, 17, 23, 24, and 140, among others.}
\z

\newpage 
\ea mari ñʉhchʉsʉmʉa bʉerina phanamana himahana mari.\\[.3em]
\gll {\textasciitilde}bari={\textasciitilde}yʉchʉ́-{\textasciitilde}sʉ́bʉ́á	bʉé-ri-{\textasciitilde}da	{\textasciitilde}phadába-{\textasciitilde}da	hí-{\textasciitilde}ba-ha-({\textasciitilde}hi')da	{\textasciitilde}barí{\footnotemark}\\
     1\textsc{pl.incl.poss}=grandfather-\textsc{pl.}kin	arrow.shoot-\textsc{nmlz-pl}	grandchild-\textsc{pl}	\textsc{cop-rsp}-\textsc{vis.ipfv.1-emph}	1\textsc{pl.incl}\\
\glt  ‘[We're grandchildren (descendants) of grandparents (ancestors) who used arrows.]’
\glt ‘[Nós somos descendentes dos nossos avôs (ancestrais) que usavam flechas.]’ 
\footnotetext{Although marked tones reflect word-level patterns, there is tonal downstep on all sentence-final pronouns.} 
\z 

\ea bʉeato õbaroi, to pharokãre sã'aphaato.\\[.3em]
\gll bʉé-a-to	{\textasciitilde}ó-ba'ro-i	to=phá-ro-{\textasciitilde}ka{\footnotemark}-re	{\textasciitilde}sa'á-pha-to\\
     arrow.shoot-\textsc{pl-nmlz.loc/evnt}	\textsc{deic.prox-clf:}kind-\textsc{loc.vis}	3\textsc{sg.poss}=stomach\textsc{-sg-dim-obj}	\textsc{mov.}inside-stomach-\textsc{nmlz.loc/evnt}\\
\glt ‘(\textit{Diani}'s people) shot (\textit{Yuhpi Diani}'s servant) right here in the stomach.’
\glt ‘(O pessoal do \textit{Diani}) flecharam (o criado do\textit{Yuhpi Diani}) bem aqui na barriga.’
\footnotetext{In this instance and many others, the diminutive suffix \textit{-kã} is used for emphasis.} 
\z 

\ea tina bʉathuataa.\\[.3em]
\gll tí-{\textasciitilde}da	bʉá-thua-ta-a\\
     \textsc{anph-pl}	crawl/crouch-return-come-\textsc{assert.pfv}\\
\glt ‘They came crawling back.’ 
\glt ‘Eles vinham se arrastando.’
\z 

\ea mahkarokahore tirore dʉhte koaa.\\[.3em]
\gll {\textasciitilde}baká-dóká{\footnotemark}-hó-ré	tí-ró-ré	dʉté	koá-a\\
     forest-\textsc{dist}-banana\textsc{-obj}	\textsc{anph-sg-obj}	chop	cure-\textsc{assert.pfv}\\
\glt ‘(They) found wild bananas (and) chopped (them to extract the liquid) to make a cure for him (the wounded guy).’
\footnotetext{The second root in the compound is underlyingly \textit{doka} ‘throw’, grammaticalized as a marker of “distal" spatial relations: that the object is related to distant place (the case here), that the movement is toward the distance or that the action is occurring at a far off location. See also lines 52, 106, 138, and 153.}
\newpage
\glt ‘Encontraram banana-do-mato, cortaram (para tirar o líquido e) fizeram curativo para ele (o ferido).’
\z 

\ea nathua te ti phʉ'toro yuhpi diani ka'apʉ.\\[.3em]
\gll {\textasciitilde}dá-thúá-a	té	ti=phʉ'tó-ro	yuhpí.diáni	ka'á-pʉ\\
     get-return-\textsc{assert.pfv}	until	3\textsc{pl.poss}=master\textsc{-sg}	yuhpi.diani	beside-\textsc{loc}\\
\glt ‘(They) took (him) right up to their leader\textit{Yuhpi Diani}.’
\glt ‘Levaram até o chefe deles \textit{Yuhpi Diani}.’
\z 

\ea “mʉ'ʉ sãre ã wa'achʉ yoara” nia.\\[.3em]
\gll {\textasciitilde}bʉ'ʉ́	{\textasciitilde}sá-re	{\textasciitilde}a=wa'á-chʉ́	yoá-ra	{\textasciitilde}dí-a\\
     2\textsc{sg}	1\textsc{pl.excl-obj}	so=go-\textsc{sw.ref}	do-\textsc{vis.ipfv.}2/3	say-\textsc{assert.pfv}\\
\glt ‘“You made this happen to us,” they said.’
\glt ‘“Você fez isso acontecer conosco,” disseram.’
\z 

\ea “mʉ'ʉre wãhakãnata” nia.\\[.3em]
\gll {\textasciitilde}bʉ'ʉ́-ré	{\textasciitilde}wahá-ka’a-{\textasciitilde}da-ta	{\textasciitilde}dí-a\\
     2\textsc{sg-obj}	kill-do.immediately-\textsc{pl-intent}	say-\textsc{assert.pfv}\\
\glt ‘“Now we're going to kill you!” (they) said.’
\glt ‘“Agora vamos matar você!” disseram.’ 
\z

\ea “hierara, yʉ kasero. yʉ’ʉre yoenatiga.\\[.3em]
\gll hi-éra-ra	yʉ=kaséro{\footnotemark}	yʉ’ʉ́-ré	yoa-éra-tiga\\
     \textsc{cop-neg-vis.ipfv.}2/3	1\textsc{sg.poss}=servant	1\textsc{sg-obj}	do\textsc{-neg-neg.imp}\\
\glt ‘“No you won't, my servants. Don’t do that to me.” 
\glt ‘“Nada disso, meus criados. Não façam nada a mim.”
\footnotetext{The word \textit{kasero} is a borrowing from Portuguese “caseiro", a housekeeper or servant.}
\z 


\ea mʉhsa phʉ'toro hiha. yʉ’ʉbahsi mahsita” nia tiro.\\[.3em]
\gll {\textasciitilde}bʉsa=phʉ'tó-ro	hí-ha	yʉ’ʉ́-basi	{\textasciitilde}basí{\footnotemark}-ta	{\textasciitilde}dí-a tí-ró\\
     2\textsc{pl.poss}=master\textsc{-sg}	\textsc{cop-vis.ipfv.}1	1\textsc{sg-emph}	know-\textsc{intent}	say-\textsc{assert.pfv}	\textsc{anph-sg}\\
\glt ‘“I'm your leader. I myself can resolve things,” he (\textit{Yuhpi Diani}) said.’
\glt ‘“Sou o chefe. Eu mesmo posso resolver isso,” disse ele (\textit{Yuhpi Diani}).’
\footnotetext{The verb \textit{mahsi} ‘know’ is also used to indicate ability, to ‘know how’ to do something.} 
\z 

\ea ni, ñaina hi'na bʉea kha'noari ti(na).\\[.3em]
\gll {\textasciitilde}dí	{\textasciitilde}yá-{\textasciitilde}ida	{\textasciitilde}hí'da	bʉé-a	{\textasciitilde}kha'dó-wa'a-ri	tí-{\textasciitilde}da\\
     say	bad-\textsc{nmlz.pl}	\textsc{emph}	arrow-\textsc{pl}	prepare/organize-go-\textsc{nmlz}	\textsc{anph-pl}\\
\glt ‘Saying that, the servants started making arrows (preparing for war).’ 
\glt ‘Falando isso, os criados foram preparar (armas).’
\z 

\ea tiro a’rina ñʉhchʉno khi'to hia.\\[.3em]
\gll tí-ró	a’rí-{\textasciitilde}da	{\textasciitilde}yʉchʉ́-ro	khí'to	hí-a\\
     \textsc{anph-sg}	\textsc{dem.prox-pl}	grandfather\textsc{-sg}	become	\textsc{cop-assert.pfv}\\
\glt ‘[He (\textit{Yuhpi Diani} and) the ones who would become (your) grandfathers.]’\\
\glt ‘[Ele (\textit{Yuhpi Diani} e) os que seriam os avôs (de vocês).]'
\z 

\ea sã ba'ʉ, tiro ñahori kʉ̃irota hia, to namosãnumia, phʉaro numia ...\\[.3em]
\gll {\textasciitilde}sa=ba'ʉ́	tí-ró	ñáhórí	{\textasciitilde}kʉ́-iro-ta	hí-a to={\textasciitilde}dabó-{\textasciitilde}sadubia	phʉá-ro-{\textasciitilde}dubia\\
     1\textsc{pl.excl.poss}=younger.brother	\textsc{anph-sg}	ñahori	alone-\textsc{nmlz.sg-emph}	\textsc{cop-assert.pfv} 3\textsc{sg.poss}=wife-\textsc{pl.f}	two\textsc{-sg-pl.f}\\
\glt ‘Our younger brother, \textit{Ñahori}, lived alone with his wives, two women …’
\glt ‘Nosso irmão menor, \textit{Ñahori}, vivia sozinho com as mulheres, duas esposas ...’ 
\z 

\ea to pho'nakã phʉarokã, to phayoa phʉarokã, seista hia tina õre mukʉ dʉhpʉrire.\\[.3em]
\gll to={\textasciitilde}pho'dá-{\textasciitilde}ka	phʉá-ro-{\textasciitilde}ka	to=phayó-a	phʉá-ro-{\textasciitilde}ka seis{\footnotemark}=ta  		hí-a	tí-{\textasciitilde}da	{\textasciitilde}ó-ré	múkʉ.dʉhpʉri-re\\
     3\textsc{sg.poss}=children-\textsc{dim}	two\textsc{-sg-dim}	3\textsc{sg.poss}=servant-\textsc{pl}	two\textsc{-sg-dim} six=\textsc{emph}  		\textsc{cop-assert.pfv}	\textsc{anph-pl}	\textsc{deic.prox-obj}	mukʉ.dʉhpʉri\textsc{-obj}\\
\glt ‘two children and two servants, they were exactly six here in \textit{Mukʉ Dʉhpʉri}.’
\glt ‘dois filhos e dois criados, no total eram seis aqui em \textit{Mukʉ Dʉhpʉri}.’
\footnotetext{Although there are frequently used Kotiria terms for numbers one through five, nowadays it is quite common for both numbers and time expressions, such as \textit{semana} ‘week’ in line 76, to be borrowed from Portuguese. See also line 264.}
\z 

\newpage 
\ea no'oi tiasomana ba'aro, ñariro phiriaka bihsi yohata to kaserua me'ne.\\[.3em]
\gll {\textasciitilde}do'ó-i	tíá-semana=ba'a-ro {\textasciitilde}yá-ri-ro	phíríá-ká{\footnotemark} 	bisí	yóhá-tá	to=kaséro-a={\textasciitilde}be're\\
     \textsc{q:}when-\textsc{loc.vis}	three-weeks=after\textsc{-sg} bad\textsc{-nmlz-sg}	flute-\textsc{clf:}round	sound	go.upriver-come	3\textsc{sg.poss}=servant-\textsc{pl=com}\\
\glt ‘Three weeks later, the warrier (\textit{Yuhpi Diani}) came upstream with his servants, to the sound of \textit{piriaka} flutes.’
\glt ‘Três semanas depois, o guerreiro (\textit{Yuhpi Diani}) vinha subindo com seus criados, ao som da flauta \textit{piriaka}.’ 
\footnotetext{\textit{Piriaka} are small wooden flutes held horizontally and played through a single blow hole.}
\z 

 
\ea ti phapʉre, yarokã tina kha'mawãhaera himarebʉ, tutu ... tutu ... tutuuuuu ...\\[.3em]
\gll tí	phá-pʉ́-ré	yá-ró-{\textasciitilde}ká	tí-{\textasciitilde}da	{\textasciitilde}kha'bá-{\textasciitilde}waha-éra	hí-{\textasciitilde}bare-bʉ	tutu...tutu...tutuuuuu\\
     \textsc{anph}	time\textsc{-loc-obj}	secret\textsc{-sg-dim}	\textsc{anph-pl}	bring/do.together{\footnotemark}-kill\textsc{-neg}	\textsc{cop-rem.ipfv-epis}	\textsc{ontp:}flute.playing\\
\glt ‘In those times, it wasn't the custom to make war silently (but to play flutes) \textit{tutu ... tutu ... tutuuuuu} ...’
\glt ‘Naqueles tempos, não era costume guerrear em silêncio, (vinham tocando flautas) \textit{tutu ... tutu ... tutuuuuu} ...’
\footnotetext{The root \textit{khãbá} ‘bring or do X together' (see also \fnref{fn:kotiria:together}), when used in a serialization can indicate reflexive action, as in line 162.}
\z 

\ea õ wãhsipiria nʉ̃hkoi wa'asʉ̃ ñaina mahaa. te õi sʉ̃a.\\[.3em]
\gll {\textasciitilde}ó	{\textasciitilde}wahsípíríá	{\textasciitilde}dʉko-i	wa'á-{\textasciitilde}sʉ́	{\textasciitilde}yá-{\textasciitilde}ida	{\textasciitilde}bahá-a té	{\textasciitilde}ó-í	{\textasciitilde}sʉ́-a\\
     \textsc{deic.prox}	wãhsipiria	island-\textsc{loc.vis}	go-arrive.\textsc{trns}	bad-\textsc{nmlz.pl}	go.uphill-\textsc{assert.pfv} until	\textsc{deic.prox-loc.vis}	arrive.\textsc{trns}-\textsc{assert.pfv}\\
\glt ‘They arrived here in \textit{Wãhsipiria Nʉ̃hko} (island, and) the warriors came up the path until they arrived (at \textit{Ñahori}'s house).’
\glt ‘Chegaram aqui em \textit{Wãhsipiria Nʉ̃hko} (ilha, e) os guerreiros foram subindo pelo caminho até chegarem (na casa do \textit{Ñahori}).’
\z 

\ea tiro bi'asãmato tiro. khatarosohpakai duhia tiro bʉhkʉro.\\[.3em]
\gll tí-ró	bi'á-sã'a-{\textasciitilde}ba-to	tí-ró khatá-ró-sópáká-í	duhí-a	tí-ró	bʉkʉ́-ro\\
     \textsc{anph-sg}	close-\textsc{mov.}inside-\textsc{frus-nmlz.loc/evnt}	\textsc{anph-sg} flatbread.oven\textsc{-sg}-opening-\textsc{loc.vis} 	sit-\textsc{assert.pfv} 	\textsc{anph-sg} 	ancestor\textsc{-sg}\\
\glt ‘He (\textit{Ñahori}) barred (the entrance, in vain, and), the old guy sat next to the opening to the flatbread oven.’
\glt ‘Ele (\textit{Ñahori}) se-fechou dentro (em vão, e) o velho ficou sentado na boca do forno.’
\z 
 
\ea ñahori to wa'masitia me'neta, a’ri wʉhʉnihti, a’ri noano hiphitiro to dohkaa me'ne.\\[.3em]
\gll ñáhórí	to={\textasciitilde}wa'básítía={\textasciitilde}be're{\footnotemark}-ta a’rí	wʉhʉ́{\textasciitilde}dítí	a’rí	{\textasciitilde}dóá-ró	híphiti-ro	to=doká-a={\textasciitilde}be're\\
     ñahori	3\textsc{sg.poss}=adornments=\textsc{com-emph} \textsc{dem.prox}	ceremonial.perfume	\textsc{dem.prox}	good\textsc{-sg}	everything\textsc{-sg}	3\textsc{sg.poss}=spear-\textsc{pl=com}\\
\glt ‘\textit{Ñahori} with all his adornments, the ceremonial perfume (and) all of his weapons.’
\glt ‘\textit{Ñahori} com todos os enfeites do corpo, perfume ceremonial (e) com todas as sua armas.’
\footnotetext{Use of the comitative clitic \textit{me'ne} causes tonal downstep on the tonal element immediately before; the same phenomenon occurs in the second occurrence in this example.}
\z 

\ea tatu'sʉa tina. kʉ̃pho'na mahataa \\[.3em]
\gll tá-thu'sʉ-a	tí-{\textasciitilde}da	{\textasciitilde}kʉ́-{\textasciitilde}pho'da	{\textasciitilde}bahá-ta-a\\
     come-finish-\textsc{assert.pfv} \textsc{anph-pl}	one/a-\textsc{clf:}line	go.uphill-come-\textsc{assert.pfv}\\
\glt ‘They (\textit{Yuhpi Diani} and warriors) arrived. (They) were coming up in a line.’
\glt ‘Eles (\textit{Yuhpi Diani} e guerreiros) já estavam vindo numa fila.’
\z

\ea yaba? um ... hum, bo'teapũ, do'se nihari?\\[.3em]
\gll yabá{\footnotemark}	hum...hum	bo'téa{\textasciitilde}pu	do'sé	{\textasciitilde}dí-hárí\\
     \textsc{q:}what/how	hum...hum	embaúba.tree(sp)	\textsc{q:}how	say-\textsc{q.ipfv}\\
\glt ‘[What? um, um ... Embaúba tree, [is that what it's called?]’ 
\glt ‘[Como? um, um ... Embaúba, [é assim que chama?]’ 
\footnotetext{The question word \textit{yaba} ‘what/how’ is used as a filler in discourse, when the speaker has a doubt about something or needs a moment to think or reformulate.}
\z

\ea pho'ophĩni peri ti khãtarire khʉaa tinase'e.\\[.3em]
\gll pho'́o-{\textasciitilde}phí-rí	péri	ti={\textasciitilde}kháta-ri-re	khʉá-a	tí-{\textasciitilde}da-se'e\\
     molongó.tree-\textsc{clf:}bladelike-\textsc{pl}	many	3\textsc{pl.poss}=cut.separate-\textsc{nmlz-obj}	have-\textsc{assert.pfv}	\textsc{anph-pl-contr}\\
\glt ‘They had a lot of strips of molongó wood that they had sharpened …’
\glt ‘Eles tinham um monte de ripas de molongó que estavam bem afiados ...’
\z 

\ea ti bahtichʉ hipa tinase're.\\[.3em]
\gll tí	batíchʉ	hí-pha	tí-{\textasciitilde}da-se'e-re\\
     \textsc{anph}	shield	\textsc{cop-spec}	\textsc{anph-pl-contr-obj}\\
\glt ‘[That (I suppose) were like a shield for them.]’
\glt ‘[Que (suponho) era como se fosse escudo para eles.]’
\z 

\ea “tara sõ'o phʉ'toro” nia.\\[.3em]
\gll tá-rá	{\textasciitilde}so'ó	phʉ'tó-ró	{\textasciitilde}dí-a\\
     come-\textsc{vis.ipfv.}2/3	\textsc{deic.dist}	master\textsc{-sg}	say-\textsc{assert.pfv}\\
\glt ‘“(They're) coming, master!” (\textit{Ñahori}'s servants) said.’
\glt ‘“(Eles) já vem, chefe!” disseram (os servos do \textit{Ñahori}).’
\z 

\ea tinakã phʉarokã nia: “to ... to ... tooooooo, ñaina wahpana pho'nañari!”\\[.3em]
\gll tí-{\textasciitilde}da-{\textasciitilde}ka	phʉá-ro-{\textasciitilde}ka	{\textasciitilde}dí-a	to...to...tooooooo	{\textasciitilde}yá-{\textasciitilde}ida	wapá-{\textasciitilde}da	{\textasciitilde}pho'dá-{\textasciitilde}ya-ri\\
     \textsc{anph-pl-dim}	two\textsc{-sg-dim}	say-\textsc{assert.pfv}	\textsc{intj:}taunt	bad-\textsc{nmlz.pl}	enemy-\textsc{pl}	children-bad\textsc{-pl}\\
\glt ‘(\textit{Ñahori} 's) two poor servants taunted them: “\textit{To ... to ... tooooooo}, hated sons of our enemy!"’
\glt ‘Os dois criados coitados (de \textit{Ñahori} ) gritavam xingando: “\textit{To ... to ... tooooooo}, malditos filhos do inimigo!"’
\z 

\ea “thʉ̃rekhãnata” nimaati pakhuioina. \\[.3em]
\gll {\textasciitilde}thʉ́re-{\textasciitilde}kha-{\textasciitilde}da-ta	{\textasciitilde}dí-{\textasciitilde}ba-ati	pá-khui-o-{\textasciitilde}ida\\
     knock.down-chop-\textsc{pl-intent}	say-\textsc{frus-ipfv}	\textsc{alt}-afraid-\textsc{caus-nmlz.pl}\\
\glt ‘“We're going to mow you down!” the poor terrified ones yelled.’
\glt ‘“Vamos derrubar vocês!” gritavam os pobres apavorados.’
\z 

\ea ne bioera, tinase phayʉbia hia.\\[.3em]
\gll {\textasciitilde}dé	bio-éra	tí-{\textasciitilde}da-se	phayʉ́-bia	hí-a\\
     \textsc{neg}	defend/resist\textsc{-neg}	\textsc{anph-pl-contr}	many-\textsc{aum}	\textsc{cop-assert.pfv}\\
\glt ‘(But) they (\textit{Ñahori}'s servants) were overwhelmed, there were too many of the others.’
\glt ‘(Mas) não aguentaram (os criados de \textit{Ñahori}), os outros eram uma multidão.’
\z 

\ea tinakã, a’rí tho'ori khʉamaati pakhuioina, ñahori yainakã ...\\[.3em]
\gll tí-{\textasciitilde}da-{\textasciitilde}ka	a’rí	thó'ó-rí	khʉá-{\textasciitilde}ba-ati	pá-khui-o-{\textasciitilde}ida  ñáhórí	yá-{\textasciitilde}ida-{\textasciitilde}ka\\
     \textsc{anph-pl-dim}	\textsc{dem.prox}	spear-\textsc{pl}	have-\textsc{frus-ipfv}	\textsc{alt}-afraid-\textsc{caus-nmlz.pl} ñahori	\textsc{poss-nmlz.pl-dim}\\
\glt ‘Those poor guys who just had spears, were terrified, \textit{Ñahori}'s servants ...’\\
\glt ‘Os coitadinhos que só tinham lanças, ficaram com medo, os criados do \textit{Ñahori} ...’
\z

\ea duhia tina kho'taphisaa bahtirodita.\\[.3em]
\gll duhí-a	tí-{\textasciitilde}da	kho'tá-phísá{\footnotemark}-á	batí-ro-dita\\
     sit-\textsc{pl}	\textsc{anph-pl}	wait-be.on-\textsc{assert.pfv}	shield-\textsc{clf:}concave-\textsc{sol}\\
\glt ‘they sat up there waiting with their shields.’
\glt ‘ficaram em cima sentados só com escudos.’
\footnotetext{Many verb serializations contain stative positional roots such as \textit{phisa} ‘be on’, which contribute detailed perspective of the action. Here, we understand that the servants are cowering on higher ground, observing and dreading the arrival of the enemy approaching from below.}  
\z 

\ea bʉemati ne bioera: “phayʉbia hira. sã phʉ'toro, wihaga” nimaa.\\[.3em]
\gll bʉé-{\textasciitilde}ba-ati	{\textasciitilde}dé	bio-éra phayʉ́-bia	hí-ra	{\textasciitilde}sa=phʉ'tó-ro	wihá-ga	{\textasciitilde}dí-{\textasciitilde}ba-a\\
     arrow.shoot-\textsc{frus-ipfv}	\textsc{neg}	defend/resist\textsc{-neg} many-\textsc{aum}	\textsc{cop-vis.ipfv.}2/3	1\textsc{pl.excl.poss}=master\textsc{-sg}	\textsc{mov.}outward-\textsc{imp}	say-\textsc{frus-assert.pfv}\\
\glt ‘Shooting but unable to resist, (the servants) cried (in vain): “There are too many of them, master. Run away!”’
\glt ‘Flechando mas não resistindo, gritaram (os criados em vão): “São muitos, chefe. Fuja!”’ 
\z 

\ea  “wa'eraha yʉ’ʉ́se kʉ̃iro pha'ñohita tinare” nia tiro.\\[.3em]
\gll wa'a-éra-ha	yʉ’ʉ́-se	{\textasciitilde}kʉ́-író	{\textasciitilde}pha'yó-híta{\footnotemark}	tí-{\textasciitilde}da-re	{\textasciitilde}dí-a	tí-ró\\
     go\textsc{-neg-vis.ipfv.}1	1\textsc{sg-contr}	alone-\textsc{nmlz.sg}	complete\textsc{-sg.intent}	\textsc{anph-pl-obj}	say-\textsc{assert.pfv}	\textsc{anph-sg}\\
\glt ‘“I'm not going! I'm going to defeat them all!” he (\textit{Ñahori}) said.'
\glt‘“Não vou! Eu mesmo vou acabar com eles,” disse ele (\textit{Ñahori}).'
\footnotetext{The expression \textit{ta hita}, here in reduced form, shows the speaker’s intent to do something ‘by myself.’}
\z 

\ea “sã phʉ'toro wa'aga, wa'aga, sã phʉ'toro. mʉ'ʉre wãhakãka. sãma'chʉne noano yoahã'ka.”\\[.3em]
\gll {\textasciitilde}sa=phʉ'tó-ro	wa'á-gá	wa'á-gá	{\textasciitilde}sa=phʉ'tó-ro {\textasciitilde}bʉ'ʉ́-ré	{\textasciitilde}wahá-{\textasciitilde}ka-ka	{\textasciitilde}sá-{\textasciitilde}ba'chʉ-re	{\textasciitilde}dóá-ró	yoá-{\textasciitilde}ha{\footnotemark}-ka\\
     1\textsc{pl.excl.poss}=master\textsc{-sg}	go-\textsc{imp}	go-\textsc{imp}	1\textsc{pl.excl.poss}=master\textsc{-sg} 2\textsc{sg-obj}	kill-\textsc{dim-predict}	1\textsc{pl.excl-add-obj}	good\textsc{-sg}	do-\textsc{compl-predict}\\
\footnotetext{The morpheme -\textit{hã} is analyzed as a shortened form of \textit{phã’yo} (see line 105), which in a serial verb construction adds a ‘completive’ aspectual reading.} 
\glt ‘“Master, run away, run away, master! (They) will kill you. (They're) ready to kill us too.”’
\glt ‘“Chefe, fuja, fuja, chefe! Vão matar você. Nós também, já vão nos matar.”’
\z 

\ea “hierara. yʉ’ʉbahsi mahsita,” nia.\\[.3em]
\gll hi-éra-ra	yʉ’ʉ́-basi	{\textasciitilde}basí-ta	{\textasciitilde}dí-a\\
     \textsc{cop-neg-vis.ipfv.}2/3	1\textsc{sg-emph}	know-\textsc{intent}	say-\textsc{assert.pfv}\\
\glt ‘“No! I myself can (defeat them),” (\textit{Ñahori} ) said.’
\glt ‘“Não! Eu mesmo sei (acabar com eles),” disse (\textit{Ñahori}).’
\z 

\ea ni, tina ñahori pho'nakãre, to namosãnumia “a’rínakãre naaga” nia.\\[.3em]
\gll {\textasciitilde}dí	tí-{\textasciitilde}da	ñáhórí	{\textasciitilde}pho'dá-{\textasciitilde}ka-re	to={\textasciitilde}dabó-{\textasciitilde}sadubia a’rí-{\textasciitilde}da-{\textasciitilde}ka-re	{\textasciitilde}dá-wa'a-ga	{\textasciitilde}dí-a\\
     say	\textsc{anph-pl}	ñahori	children\textsc{-dim-obj}	3\textsc{sg.poss}=wife-\textsc{pl.f} \textsc{dem.prox-pl-dim-obj}	get-go-\textsc{imp}	say-\textsc{assert.pfv}\\
\glt ‘Saying that, Ñahori ordered his wives: “Take the children away!” he said.’\\
\glt ‘Falando disso, Ñahori mandou as esposas: “Levem as crianças embora!” disse.’
\z 

   
\ea “hai” ni. tinakãre na, ti yaipiripʉre naa. \\[.3em]
\gll hai	{\textasciitilde}dí	tí-{\textasciitilde}da-{\textasciitilde}ka-re	{\textasciitilde}dá	ti=yáí-pírí-pʉ-re	{\textasciitilde}dá-a \\
     \textsc{intj:}agree	say	\textsc{anph-pl-dim-obj}	get	3\textsc{pl.poss}=jaguar-teeth-\textsc{clf:}basket\textsc{-obj}	get-\textsc{assert.pfv} \\
\glt ‘“Yes!” (they) said. They got the children (and) put them in (their basket with) jaguar-teeth necklaces (and other sacred objects).’
 
\glt ‘“Sim!” disseram. Pegaram os filhos (e) colocaram no (cesto) de colares de dente-de-onça (e outros objetos sagrados).’
\z

\ea to namosãnumia pho'nakã phʉaro. \\[.3em]
\gll to={\textasciitilde}dabó-{\textasciitilde}sadubia	{\textasciitilde}pho'dá-{\textasciitilde}ka	phʉá-ro \\
     3\textsc{sg.poss}=wife-\textsc{pl.f}	children-\textsc{dim}	two\textsc{-sg} \\
\glt ‘[His wives’ two children.]’
\glt ‘[Os dois filhos das esposas dele.]’
\z 

\ea “toi, sã bioeraka. mʉ'ʉbahsi mahsiga sã phʉ'toro. mʉ'ʉre khã'imaha sã,” nia. \\[.3em]
\gll tó-i	{\textasciitilde}sá	bio-éra-ka {\textasciitilde}bʉ'ʉ́-basi	{\textasciitilde}basí-gá	{\textasciitilde}sa=phʉ'tó-ro  {\textasciitilde}bʉ'ʉ́-ré	{\textasciitilde}kha'í-{\textasciitilde}ba-ha	{\textasciitilde}sá	{\textasciitilde}dí-a\\
    \textsc{def-loc.vis}	\textsc{1pl.excl}	defend/resist\textsc{-neg-assert.ipfv} 2\textsc{sg-emph}	know-\textsc{imp}	1\textsc{pl.excl.poss}=master\textsc{-sg} 2\textsc{sg-obj}	love-\textsc{rsp-vis.ipfv.1}	1\textsc{pl.excl}	say-\textsc{assert.pfv} \\
\glt ‘“We can't resist anymore! You take over, master. We love you,” (\textit{Ñahori}’s servants) said.’
\glt ‘“Não aguentamos mais. Você pode resistir, chefe. Nós amamos você,” disseram (os criados do \textit{Ñahori}).’
\z 

\ea tiro tobahsi bʉawihatamaa bʉerotaro. \\[.3em]
\gll tí-ró	to-basí	bʉá-wiha-ta-{\textasciitilde}ba-a	bʉé-ro-ta-ro \\
     \textsc{anph-sg}	\textsc{def-emph}	crawl/crouch-\textsc{mov.}outward-come-\textsc{frus-assert.pfv}	arrow.shoot\textsc{-sg}-come\textsc{-sg} \\
\glt ‘He (\textit{Ñahori}) himself came out crouching down (and) shooting arrows.’
\glt ‘Ele (\textit{Ñahori}) mesmo veio saindo agachado (e) flechando.’
\z 

\ea tiata, quatrotari, tirore õ waroi bʉetu'sʉa tina, chʉʉʉ! tʉ ... tʉ .. .tʉʉʉʉʉʉʉʉʉ, phuhu! \\[.3em]
\gll tíá́-ta	quátro-ta-ri	tí-ró-ré	{\textasciitilde}ó	wáró-i	bʉé-thu'sʉ-a	tí-{\textasciitilde}da chʉʉʉ! tʉ...tʉ...tʉʉʉʉʉʉʉʉʉ	phuhu!\\
     three-\textsc{emph}	four-\textsc{emph-pl}	\textsc{anph-sg-obj}	\textsc{deic.prox}	\textsc{emph-loc.vis}	arrow.shoot-finish-\textsc{assert.pfv}	\textsc{anph-pl} \textsc{ontp:}arrows.being.shot \textsc{ontp:}arrows.flying	\textsc{ontp:}falling.on.ground \\
\glt ‘After three (or) four rounds, an arrow hit him (\textit{Ñahori}). \textit{Chʉʉʉ! tʉ ... tʉ ... tʉʉʉʉ! phuhu!}’ (sounds of arrows flying, striking \textit{Ñahori} and his falling down.)
\glt ‘Depois de três (ou) quatro vezes, acertaram nele (\textit{Ñahori}).\textit{Chʉʉʉ! tʉ ... tʉ ... tʉʉʉʉ! phuhu!}’ (som das flechas, do acerto nele e \textit{Ñahori} caindo.)
\z 

\ea thetereka khataro dʉ'tʉka'apʉ. \\[.3em]
\gll thetéré-ká	khatá-ró	dʉ'tʉ́ka'a-pʉ \\
     tremble-\textsc{assert.ipfv}	flatbread.oven-\textsc{clf:}concave	edge-\textsc{loc} \\
\glt ‘He collapsed trembling right next to the oven.’
\glt ‘Caiu tremendo bem do lado do forno.’
\z 

\ea kha'asʉ̃a.  \\[.3em]
\gll kha'á-{\textasciitilde}sʉ-a \\
     fall-arrive\textsc{.trns-assert.pfv} \\
\glt ‘He fell over.’
\glt ‘Ficou caido.’
\z 

\largerpage[2]
\ea to kha'asʉ̃chʉ, “noana mari wahpakʉro ñakãre” ni. \\[.3em]
\gll to=kha'á-{\textasciitilde}sʉ-chʉ	{\textasciitilde}dóádá	{\textasciitilde}bari=wapá-kʉ-ro	{\textasciitilde}yá-{\textasciitilde}ka-re	{\textasciitilde}dí \\
     3\textsc{sg.poss}=fall-arrive.\textsc{trns-sw.ref}	that's.good	1\textsc{pl.incl.poss}=enemy\textsc{-m-sg}	bad-\textsc{dim-obj}	say \\
\glt ‘After his fall{\footnotemark} (\textit{Diani}'s warriors cried): “You deserved it, evil enemy!”’
\footnotetext{These sentences contain a typical “tail-head" linking sequence, in which the finite predicate (or part of it) of one sentence is repeated in the following sentence, usually as an initial subordinate adverbial clause (though the subordinate clauses can occur at the end of the sentence, as in line 195). When the action involves a shift in subject between the subordinate and main clause, the subordinate clause is marked with the switch-reference suffix \textit{-chʉ} (see \citealt{Stenzel2016}).}
\glt ‘Depois que caiu, (os guerreiros do \textit{Diani} gritaram): “Bem feito, inimigo malvado!”’
\z

 
\ea doa, ni “hʉʉʉʉ ... bi'oha mari tirore,” ni thumaharekũa. \\[.3em]
\gll dóa	{\textasciitilde}dí	hʉʉʉʉ bi'ó-há	{\textasciitilde}barí	tí-ró-ré {\textasciitilde}dí	thú-{\textasciitilde}baha-re-{\textasciitilde}kua \\
     shout	say	\textsc{intj:}victory.cry	successful-\textsc{vis.ipfv.}1	1\textsc{pl.incl}	\textsc{anph-sg-obj} say	push-\textsc{mov.}upward\textsc{-obj}-leave.on.ground \\
\glt ‘(They) shouted: “\textit{Hʉʉʉʉ!} ... We got (killed) him!” (and then) rolled his body over.'
\glt ‘Gritaram: “\textit{Hʉʉʉʉ!} ... Nós conseguimos (matar) ele!” (e depois) viraram (o corpo dele) de costas.'
\z 

\ea wʉ'ʉ hʉ̃pha'yohã'a. \\[.3em]
\gll wʉ'ʉ́	{\textasciitilde}hʉ́-pha'yo-{\textasciitilde}ha-a  \\
     house	burn-\textsc{compl-compl-assert.pfv} \\
\glt ‘(They) burned down all the houses.’
\glt ‘(Eles) tocaram fogo em todas as casas.’
\z 

\ea ã yoa tina dʉhsetire ñʉrokaa nia. \\[.3em]
\gll {\textasciitilde}a=yóá	tí-{\textasciitilde}da	dʉsé-ti{\footnotemark}-re	{\textasciitilde}yʉ́-dóká-a	{\textasciitilde}dí-a \\
     so=do	\textsc{anph-pl}	mouth\textsc{-attrib-obj}	see/look-\textsc{dist-pl}	\textsc{prog-assert.pfv} \\
\glt ‘So, the others (from Matapi) were watching the confusion from afar.’
\glt ‘Então, os outros (de Matapi) ficaram observando a confusão de longe.’
\footnotetext{The notion of “confusion" is derived from the root for ‘mouth’ marked by the attributive suffix \textit{-ti}, something akin to ‘be mouthy’. It can also be used to refer to a discussion.}  
\z 

\ea a’ri ora hiro hiarĩto?{\footnotemark} \\[.3em]
\gll a’rí	hóra	hí-ro	hí-a-rito \\
     \textsc{dem.prox}	hour	\textsc{cop-sg}	\textsc{cop-}go-or.not \\
\glt ‘[Was that when it was?]’ 
\glt ‘[Era nessa hora ou não?]’ 
\footnotetext{The expression \textit{hiro hiarito} is used to express doubt: ‘Was that the way it went/was or not?’ The verb \textit{wa'a} ‘go' is often shortened to \textit{-a} when it occurs in serializations (see also line 292).}
\z 

 
\largerpage[2]
\ea ni ora hiro hiarito to phichanihti mʉhaa tiñʉchʉ. \\[.3em]
\gll {\textasciitilde}dí	hóra	hí-ro	hí-a-rito	to=phichá-{\textasciitilde}dítí	{\textasciitilde}bʉhá-a	ti={\textasciitilde}yʉ́-chʉ́ \\
     \textsc{q:}which	hour	\textsc{cop-sg}	\textsc{cop-}go-or.not	\textsc{rem}=fire-ash	\textsc{mov.}upward-\textsc{assert.pfv}	3\textsc{pl.poss}=see/look-\textsc{sw.ref} \\
\glt ‘[Was at this time or not when (the women in Matapi) saw the smoke rising?]’
\newpage
\glt ‘[Era essa hora ou não quando (as mulheres de Matapi) viram a fumaça subir?]’
\z 

\ea ã yoa ti manʉsʉmʉare ... sã numia khitiphayʉ nimahana. \\[.3em]
\gll {\textasciitilde}a=yóá	ti={\textasciitilde}badʉ́-{\textasciitilde}sʉbʉa-re	{\textasciitilde}sá	{\textasciitilde}dubía	khití-phayʉ	{\textasciitilde}dí-{\textasciitilde}bá-há={\textasciitilde}(hi')da \\
     so=do	3\textsc{pl.poss}=husband-\textsc{pl}.kin-\textsc{obj}	1\textsc{pl.excl}	women	story-many	say-\textsc{frus-vis.ipfv.1=emph} \\
\glt ‘So (they commented to) their husbands ... [we women are such gossips! (story-tellers)]’
\glt ‘Então (falaram) aos seus maridos ... [como nós mulheres somos fofoqueiras (contadoras de histórias)!]’
\z

\ea “ñʉhʉ̃! do'se mari ñahorire wãhahãpha sina.” \\[.3em]
\gll {\textasciitilde}yʉ́-{\textasciitilde}hʉ{\footnotemark}	do'sé	{\textasciitilde}bari=ñáhórí-ré	{\textasciitilde}wahá-{\textasciitilde}ha-pha	śi-{\textasciitilde}da \\
     see/look-\textsc{imp.deic}	\textsc{q:}what	1\textsc{pl.incl.poss}=ñahori\textsc{-obj}	kill-\textsc{compl-spec}	\textsc{dem.dist-pl} \\
\glt ‘“Look there! Have those people killed our \textit{Ñahori}?” (they speculated).’
\glt ‘“Olha lá! Será que aqueles mataram nosso \textit{Ñahori}? (especularam).’
\footnotetext{The imperative suffix -\textit{hʉ̃} has inherent distal deictic semantics.}
\z 

\ea “yoatapʉ wʉ'ʉma'chʉ hʉ̃mʉhana” nia. \\[.3em]
\gll yoá-tá-pʉ́	wʉ'ʉ́-{\textasciitilde}ba'chʉ	{\textasciitilde}hʉ́-{\textasciitilde}bʉha-ra	{\textasciitilde}dí-a  \\
     long\textsc{-emph-loc}	house-\textsc{add}	burn-\textsc{mov.}upward-\textsc{vis.ipfv.}2/3	say-\textsc{assert.pfv} \\
\glt ‘“The smoke from the houses has been rising for a long time!” (they) said.’
\glt ‘“Faz tempo que sobe fumaça das casas!” disseram.’
\z 

 
\ea “pa! ã thiharide ni ñʉrokaa. \\[.3em]
\gll pá	{\textasciitilde}á	thí-hari-de	{\textasciitilde}dí	{\textasciitilde}yʉ́-dóká-á \\
     \textsc{intj:}neg.surprise	so	true-\textsc{q.ipfv-emph}	say	see/look-\textsc{dist}-\textsc{assert.pfv} \\
\glt ‘“Oh no! Is it true?” (they wondered) looking from afar.'
\glt ‘“Puxa! Será verdade?” (especularam) olhando de longe.'
\z 

\ea “bahsañʉna!”  \\[.3em]
\gll basá-{\textasciitilde}yʉ́-({\textasciitilde}hí')da \\
     \textsc{exrt}-see/look-\textsc{exrt}{\footnotemark} \\
\glt ‘“Let's go see!”'
\glt ‘“Vamos ver!”' 
\footnotetext{The exhortative is composed of elements bracketing the verb: initial \textit{(ba)sa} and final \textit{(hí')na}, both of which are frequently shortened as indicated.}
\z 

\ea ti phare ma'ari hiatiri himarebʉ taati phaati ... pʉʉʉ ... a’ri mukʉ dʉhpʉripʉ. \\[.3em]
\gll tí-phá-ré	{\textasciitilde}ba'á-ri	hí-ati-ri	hí-{\textasciitilde}bare-bʉ tá-átí	phá-ati	pʉ́	a’rí	múkʉ.dʉhpʉri-pʉ \\
     \textsc{anph}-time\textsc{-obj}	path-\textsc{pl}	\textsc{cop-ipfv-nmlz}	\textsc{cop-rem.ipfv-epis} come-\textsc{ipfv}	time-\textsc{ipfv}	\textsc{loc}	\textsc{dem.prox}	mukʉ.dʉhpʉri-\textsc{loc} \\
\glt ‘[In those days it seems there were already paths coming all the way to \textit{Mukʉ Dʉhpʉri} (\textit{Ñahori}’s place).]’
\glt ‘[Naqueles tempos parece que já existiam as trilhas vindo até ao \textit{Mukʉ Dʉhpʉri} (lugar do \textit{Ñahori}).]’
\z 

\ea de ti wi'ichʉ wʉ'ʉ borawa'aa ... karaa! \\[.3em]
\gll dé	ti=wi'í-chʉ	wʉ'ʉ́	borá-wa'a-a	karaa \\
     \textsc{intj:}poor.one(s)!	3\textsc{pl.poss}=arrive\textsc{.cis-sw.ref}	house	slide/fall-go-\textsc{assert.pfv}	\textsc{ontp:}house.collapsing \\
\glt ‘Poor folks! When they got there, the houses were already collapsing ... \textit{Karaa!}’ (sound of houses falling down)
\glt ‘Coitados! Quando chegaram, as casas já estavam caindo ... \textit{Karaa!}’ (som das casas caindo)
\z 

\ea “kue! do'se wa'aride” ni, phisumahkamaa tina ñahorire. \\[.3em]
\gll kué	do'sé	wa'á-ri-de{\footnotemark}	{\textasciitilde}dí	phisú-{\textasciitilde}báká-{\textasciitilde}ba-a	tí-{\textasciitilde}da	ñáhórí-ré \\
     \textsc{intj}:surprise	\textsc{q:}what	go\textsc{-nmlz-neg.mir}	say	call-look.for-\textsc{frus-assert.pfv}	\textsc{anph-pl}	ñahori\textsc{-obj}\\
\footnotetext{The morpheme \textit{-de} seems to be a negative mirative, indicating both unexpected information and fear.}
\glt ‘“What in the world happened?” (they thought) calling out, looking (in vain) for \textit{Ñahori} (and his family).’
\newpage 
\glt ‘“O que será que aconteceu?" (pensaram) chamando e procurando (em vão) \textit{Ñahori} (e a familia).’
\z

\ea “ñahoriiii, ñahori pho'nakãããã, mʉhsare wãhapha'ñonohkaka,” nia tina biarise.\\[.3em]
\gll ñáhórí	ñáhórí	{\textasciitilde}pho'dá-{\textasciitilde}ká	{\textasciitilde}bʉsá-re	{\textasciitilde}wahá-{\textasciitilde}pha'yo-{\textasciitilde}doka-ka {\textasciitilde}dí-a	tí-{\textasciitilde}da	bíári-se\\
     ñahori	ñahori	children-\textsc{dim}	2\textsc{pl-obj}	kill-complete-together-\textsc{assert.ipfv} say-\textsc{assert.pfv}	\textsc{anph-pl}	biari-\textsc{contr}\\
\glt ‘“\textit{Ñahori} ... \textit{Ñahori} children ... (they've) killed all of you together!” the \textit{Biari} cried.’
\glt ‘“\textit{Ñahori} ... \textit{Ñahori} filhos ... mataram todos vocês juntos!” gritaram eles, os \textit{Biari}.’
\z 

\ea yʉ'tiera tina, yo'o khuinohkaa.\\[.3em]
\gll yʉ'ti-éra	tí-{\textasciitilde}da	yo'ó	khuí-{\textasciitilde}doka-a\\
     answer\textsc{-neg}	\textsc{anph-pl}	slaughter	afraid-together-\textsc{assert.pfv}\\
\glt ‘They (the ones who had escaped) didn't answer, fearful of being slaughtered.’
\glt ‘Não respondiam (os que tinham escapado), com pavor de serem mortos.’
\z 

\ea “marire pharituri wãhataa nina” nia.\\[.3em]
\gll {\textasciitilde}barí-re	pharí-thu(a)-ri	{\textasciitilde}wahá-ta-a	{\textasciitilde}dí-ra	{\textasciitilde}dí-a\\
     1\textsc{pl.incl-obj}	time-return-\textsc{nmlz}	kill-come-\textsc{pl}	\textsc{prog-vis.ipfv.}2/3	say-\textsc{assert.pfv}\\
\glt ‘“(They're) coming back to kill us!” (they) thought.’
\glt ‘“Vão vir de novo para nos matar!” pensaram.’
\z 

\largerpage
\ea ba'aro, “no'oi wa'ari mʉhsa ñahori pho'nakã” nia. \\[.3em]
\gll ba'á-ró{\footnotemark}	{\textasciitilde}do'ó-i	wa'á-ri	{\textasciitilde}bʉsá	ñáhórí	{\textasciitilde}pho'dá-{\textasciitilde}ká	{\textasciitilde}dí-a\\
     after\textsc{-sg}	\textsc{q:}where-\textsc{loc.vis}	go-\textsc{q.pfv}	2\textsc{pl}	ñahori	children-\textsc{dim}	say-\textsc{assert.pfv}\\
\glt ‘Later, (the \textit{Biari}) cried out, “Where have you gone, \textit{Ñahori} children?”’
\glt ‘Mais tarde, (os \textit{Biari}) gritaram, “Onde foram vocês, filhos do \textit{Ñahori}?”’
\footnotetext{Relative temporal reference is accomplished by stative roots ‘do/be after’ (also in line 76) or ‘do/be before’ (as in line 282), glossed with their basic semantics.} 
\z 

\ea “mʉhsare wãhapha'ñoboari chẽ” nia.\\[.3em]
\gll {\textasciitilde}bʉsá-re	{\textasciitilde}wahá-{\textasciitilde}pha'yo-bo-a-ri	{\textasciitilde}ché	{\textasciitilde}dí-a\\
     2\textsc{pl-obj}	kill-complete\textsc{-dub-affect-q.pfv}	\textsc{intj}:doubt	say-\textsc{assert.pfv}\\
\glt ‘“Could they have killed all of you?”’
\glt ‘“Será que mataram todos vocês?”’
\z 

\ea tina a’rina yuhpi diani tinakãre wãhapha'ñochʉre, a’rina ne ñahoria maniaboka. \\[.3em]
\gll tí-{\textasciitilde}da	a’rí-{\textasciitilde}da	yuhṕi.díání	tí-{\textasciitilde}da-{\textasciitilde}ka-re	{\textasciitilde}wahá-{\textasciitilde}pha'yo-chʉ-re a’rí-{\textasciitilde}da	{\textasciitilde}dé	{\textasciitilde}yáhórí-á	{\textasciitilde}badía-bo-ka\\
     \textsc{anph-pl}	\textsc{dem.prox-pl}	yuhpi.diani	\textsc{anph-pl-dim-obj}	kill-complete-\textsc{sw.ref-obj} \textsc{dem.prox-pl}	\textsc{neg}	ñahori-\textsc{pl}	not.exist-\textsc{dub-assert.ipfv}\\
\glt ‘[If \textit{Yuhpi Diani} had killed off the little ones, there wouldn't be all these \textit{Ñahoria}.]’ (referring to the audience)
\glt ‘[Se \textit{Yuhpi Diani} tivesse matado os pequenos, não existiriam esses \textit{Ñahoria}.]’ (falando da plateia assistindo)
\z 

 
\ea tinakã phʉaro mʉakã hia.  \\[.3em]
\gll tí-{\textasciitilde}da-{\textasciitilde}ka	phʉá-ro	{\textasciitilde}bʉ́-a-{\textasciitilde}ka	hí-a \\
     \textsc{anph-pl-dim}	two\textsc{-sg}	man/person-\textsc{pl-dim}	\textsc{cop-assert.pfv}\\
\glt ‘There were two little boys.’
\glt ‘Havia dois meninos.’
\z

\ea ã yoa mahkaphome yʉ'tia. “ʉʉʉʉʉʉʉʉ” nia. \\[.3em]
\gll {\textasciitilde}a=yóá	{\textasciitilde}baká-{\textasciitilde}phóbé	yʉ'tí-a	ʉʉʉʉʉʉʉʉ	{\textasciitilde}dí-a \\
     so=do	look.for-give.up	answer-\textsc{assert.pfv}	\textsc{intj:}here!	say-\textsc{assert.pfv}\\
\glt ‘When (the \textit{Biari}) were about to give up looking, (the boys) answered: “\textit{ʉʉʉʉʉʉ}.” (We’re here!)'
\glt ‘Então quando (os \textit{Biari}) já estavam cansandos de procurar, (os meninos) responderam: “\textit{ʉʉʉʉʉʉ}.” (Estamos aqui!)'
\z 

\newpage
\ea “phi'atakʉ mʉhsare, mahkaha” nia. \\[.3em]
\gll phi'á-tá-kʉ́	{\textasciitilde}bʉsá-re	{\textasciitilde}baká-há	{\textasciitilde}dí-a \\
     \textsc{mov}.out-come-\textsc{m}	2\textsc{pl-obj}	look.for-\textsc{vis.ipfv.}1	say-\textsc{assert.pfv} \\
\glt ‘“Come on out! We're looking for you,” (the \textit{Biari}) cried.’
\glt ‘“Venham! Estamos procurando vocês,” (os \textit{Biari}) gritaram.’
\z 

\ea de tinakã to namosãnumia ti wi'iphi'tichʉ, tinakãre “ne do'se wa'ari mʉhsa?” nia. \\[.3em]
\gll dé	tí-{\textasciitilde}da-{\textasciitilde}ka	to={\textasciitilde}dabó-{\textasciitilde}sadubia	ti=wi'í-phi'ti-chʉ	tí-{\textasciitilde}da-{\textasciitilde}ka-re {\textasciitilde}dé	do'sé	wa'á-ri	{\textasciitilde}bʉsá	{\textasciitilde}dí-a\\
     \textsc{intj}:poor.one(s)!	\textsc{anph-pl}-\textsc{dim}	3\textsc{sg.poss}=wife-\textsc{pl.f}	3\textsc{pl.poss}=arrive\textsc{.cis}-end-\textsc{sw.ref}	\textsc{anph-pl-dim-obj} \textsc{neg}	\textsc{q:}what	go-\textsc{q.pfv}	2\textsc{pl}	say-\textsc{assert.pfv}\\
\glt ‘When the poor little ones and \textit{Ñahori}'s wives arrived, (they asked) them: “What happened to you?”’
\glt ‘Quando os pequenos coitados (e) as esposas (do \textit{Ñahori}) chegaram, perguntaram a eles: “O que aconteceu com vocês?”’
\z 

\ea “yoatapʉ sã manʉre wãhanohkare” nia. \\[.3em]
\gll yoá-tá-pʉ́	{\textasciitilde}sa={\textasciitilde}badʉ́-ré	{\textasciitilde}wahá-{\textasciitilde}doka-re	{\textasciitilde}dí-a \\
     do-come-\textsc{loc}	1\textsc{pl.excl.poss}=husband\textsc{-obj}	kill-together-\textsc{vis.pfv.}2/3	say-\textsc{assert.pfv}\\
\glt ‘“(They) all came to kill our husband,” (the women) said.’
\glt ‘“Vieram matar nosso marido,” disseram (as mulheres).’
\z 

\ea “sãwaro a’rinakãre sã nai.” \\[.3em]
\gll {\textasciitilde}sá-waro	a’rí-{\textasciitilde}da-{\textasciitilde}ka-re	{\textasciitilde}sá	{\textasciitilde}dá-i \\
     1\textsc{pl.excl-emph}	\textsc{dem.prox-pl-dim-obj}	1\textsc{pl.excl}	get-\textsc{vis.pfv.}1\\
\glt ‘“It was us, we took the little ones away.”’
\glt ‘“Nós mesmo, nós levamos esses pequenos.”’
\z 

\newpage 
\ea “noana” nia. \\[.3em]
\gll {\textasciitilde}dóádá	{\textasciitilde}dí-a \\
     that's.good	say-\textsc{assert.pfv}\\
\glt ‘“Good thing!” (the \textit{Biari}) answered.’
\glt ‘“Que bom!” responderam (os \textit{Biari}).’
\z 

\ea “a’rinakã phʉaro a’ri yaipiripʉre tuaro sã khã'i” nia a’rina to mahsawamisʉmʉabʉ'se. \\[.3em]
\gll a’rí-{\textasciitilde}da-{\textasciitilde}ka	phʉá-ro	a’rí	yaí-pírí-pʉ-re	túá-ró	{\textasciitilde}sá {\textasciitilde}kha'í	{\textasciitilde}dí-a	a’rí-{\textasciitilde}da	to={\textasciitilde}basa-{\textasciitilde}wabi-{\textasciitilde}sʉbʉa-bʉ'se{\footnotemark}\\
     \textsc{dem.prox-pl-dim}	two\textsc{-sg}	\textsc{dem.prox}	jaguar-teeth-\textsc{clf:}basket\textsc{-obj}	strong\textsc{-sg}	1\textsc{pl.excl} care.for	say-\textsc{assert.pfv}	\textsc{dem.prox-pl}	3\textsc{sg.poss}=people-older.brother-\textsc{pl}.kin-side\\
\glt ‘“We will care for the two little ones (and) the jaguar-teeth basket (and other sacred objects),” his older brothers (the \textit{Biari}) said.’
\glt ‘“Nós cuidamos desses dois pequenos (e) o cesto de dentes-de-onça (e outros objetos sagrados),” diziam os irmão maiores dele (os \textit{Biari}).’
\footnotetext{Teresinha’s use of the noun \textit{bʉ'se} ‘side’ here indicates this as a possible lexical origin of the contrastive marker \textit{-se(’e)} (see lines 84, 165, 273, 275, among others), a grammaticalization path from ‘side’ to ‘contrastive other’.} 
\z

\largerpage[2]
\ea tinakãre nawa'a te ti ñʉwa'aka'aa no'oi hʉ̃phi'ti, de khataro dʉ'tʉka'ai, ti yiriwʉrʉ khõaduhkua. \\[.3em]
\gll tí-{\textasciitilde}da-{\textasciitilde}ka-re	{\textasciitilde}dá-wa'a	té	tí({\textasciitilde}da)	{\textasciitilde}yʉ́-wa'a-ka'a-a	{\textasciitilde}do'ó-i	{\textasciitilde}hʉ́-phi'ti dé	khatá-ró-dʉ'tʉka'a-i	ti=yírí-wʉ'rʉ	{\textasciitilde}khoá-duku{\footnotemark}-a\\
     \textsc{anph-pl-dim-obj}	get-go	until	\textsc{anph}	see/look-go-do.moving-\textsc{assert.pfv}	\textsc{q:}where-\textsc{loc.vis}	burn-end \textsc{intj:}poor.one!	flatbread.oven-\textsc{clf:}concave-edge-\textsc{loc.vis} 	3\textsc{pl.poss}=skull-\textsc{aug}	lie-stand-\textsc{assert.pfv}\\
\glt ‘Taking the little ones (they) went looking around where the fire was going out, (and saw) the poor guy fallen at the edge of the oven, his big skull lying on the ground.’
\footnotetext{Note the serialization of two posture verbs, in which the second root \textit{-duku} ‘stand’ indicates durative aspect. In this case, it emphasizes the complete immobility of the burned corpse lying there.} 
\newpage
\glt ‘Levando os pequenos, indo olhando onde o fogo estava acabando, (viram) o pobre corpo queimado do lado do forno, a caveira grande deitada no chão.’
\z 

\ea ñahorikiro hʉ̃ khõano.  \\[.3em]
\gll ñáhórí-kíró	{\textasciitilde}hʉ́	{\textasciitilde}khoá-ro \\
     ñahori-\textsc{m.rsp}	burn	lie\textsc{-sg} \\
\glt ‘It was \textit{Ñahori} lying there all burned up.’ 
\glt ‘Era \textit{Ñahori} deitado queimado.’ 
\z 

\ea sua yʉ'dʉa mʉawa'a ñaina. \\[.3em]
\gll súá	yʉ'dʉ́-á	{\textasciitilde}bʉ́-a-wa'a	{\textasciitilde}yá-{\textasciitilde}ida \\
     angry	\textsc{intens-assert.pfv}	man/person-\textsc{pl}-go	bad-\textsc{nmlz.pl}\\
\glt ‘They became furious.’
\glt ‘Ficaram enfurecidos.’
\z 

\ea “noana, sã ba'ʉre yoaka” nia.  \\[.3em]
\gll {\textasciitilde}dóádá	{\textasciitilde}sa=ba'ʉ́-re	yoá-ka{\footnotemark}	{\textasciitilde}dí-a \\
     very.well	1\textsc{pl.excl.poss}=younger.brother\textsc{-obj}	do-\textsc{assert.ipfv}	say-\textsc{assert.pfv} \\
\glt ‘“Very well, (they've) done this to our younger brother (killed him),” (the \textit{Biari}) said.’
\glt ‘“Tudo bem, já fizeram isso (mataram) nosso irmão,” disseram (os \textit{Biari}).’
\footnotetext{Use of the imperfective assertion marker \textit{–ka} (rather than the perfective \textit{–a}) suggests a shared conclusion based on evidence they can all see.}   
\z 

\ea thua wa'aa. \\[.3em]
\gll thúa-wa'a-a \\
     return-go-\textsc{assert.pfv}\\
\glt ‘They returned home.’
\glt ‘Voltaram.’
\z 

\ea hia ti phʉkʉro a’riro dianumia, mari ñʉhchʉ. \\[.3em]
\gll hí-a	ti=phʉkʉ́-ró	a’rí-ro	diánúmíá	{\textasciitilde}bari={\textasciitilde}yʉchʉ́ \\
     \textsc{cop-assert.pfv}	3\textsc{pl.poss}=father\textsc{-sg}	\textsc{dem.prox-sg}	dianumia	1\textsc{pl.incl.poss}=grandfather \\
\glt ‘Their father, \textit{Dianumia (Yairo)}, our grandfather was there.’
\glt ‘Lá estava o pai deles, \textit{Dianumia (Yairo)}, nosso avô.’
\z 

\ea “ne do'se wa'ari, pho'na.”  \\[.3em]
\gll {\textasciitilde}dé	do'sé	wa'á-ri	{\textasciitilde}pho'dá \\
     \textsc{intj:}well	\textsc{q:}what	go-\textsc{q.pfv}	children\\
\glt ‘“Well, what happened, sons?” (\textit{Dianumia Yairo} asked).’
\glt ‘“Então, o que aconteceu, filhos?” (\textit{Dianumia Yairo} perguntou).’
\z

\ea “yoatapʉ sã ba'ʉre hʉ̃khõarokare. wãhahare” nia. \\[.3em]
\gll yoá-tá-pʉ́	{\textasciitilde}sa=ba'ʉ́-ré	{\textasciitilde}hʉ́-{\textasciitilde}khoa-doka-re	{\textasciitilde}wahá-{\textasciitilde}ha-re	{\textasciitilde}dí-a \\
     do-\textsc{emph-loc}	1\textsc{pl.excl.poss}=younger.brother\textsc{-obj}	burn-lie-\textsc{dist-vis.pfv.}2/3	kill-\textsc{compl-vis.pfv.}2/3	say-\textsc{assert.pfv}\\
\glt ‘“Over there they burned our younger brother. (They) killed (him)!” (they) said.’
\glt ‘“Lá longe queimaram nosso irmão. Mataram!” relataram.’
\z 

\ea “noanoka yʉ pho'na, topuruta hiro, to yabu'iri hika.”  \\[.3em]
\gll {\textasciitilde}dóároka	yʉ={\textasciitilde}pho'dá	tópuru-ta	hí-ro	to=yá-bu'i-ri	hí-ka \\
     enough!	1\textsc{sg.poss}=children	that's.all-\textsc{emph}	\textsc{cop-sg}	3\textsc{sg.poss=poss}-cause-\textsc{nmlz}	\textsc{cop-assert.ipfv}\\
\glt ‘“Enough, my sons, it's over, it was his own fault.”’ 
\glt “Basta, meus filhos, já chega, foi culpa dele mesmo.”’
\z 

\largerpage
\ea “ya'uchʉ, thʉ'o wamatierare” tiro nia tota ti phʉkʉrose dianumiase. \\[.3em]
\gll ya'ú-chʉ́	thʉ'́ó	{\textasciitilde}waba-ti-éra-re	tí-ró	{\textasciitilde}dí-a 	to-ta{\footnotemark} ti=phʉkʉ́-ro-se	diánúmíá-sé\\
     warn-\textsc{sw.ref}	hear	word\textsc{-vbz-neg-vis.pfv.}2/3	\textsc{anph-sg}	say-\textsc{assert.pfv}	\textsc{anph(3sg)-emph} 3\textsc{pl.poss}=father\textsc{-sg-contr}	dianumia-\textsc{contr}\\
\footnotetext{Teresinha uses the expression \textit{tota} (a short form of \textit{tiro-ta} ‘he there’) for emphasis; indicating that \textit{Dianumia Yairo} was expecting this to happen and is angered and disgusted by the situation. See the plural form of the expression, \textit{tita} shortened from \textit{tina-ta} in line 142.}  
\glt ‘“I warned him, but he ignored (it),” he (\textit{Dianumia Yairo}) retorted [disgusted].”
\newpage 
\glt ‘“Eu aconselhei, mas ele ignorou,” ele (\textit{Dianumia Yairo}) respondeu [desapontado].”
\z 

\ea “hierara, kha'maeraha, sã phʉkʉ, (wãha)kha'makasininatha tirore.” \\[.3em]
\gll hi-éra-ra	{\textasciitilde}kha'ba-éra-ha	{\textasciitilde}sa=phʉkʉ́ ({\textasciitilde}waha){\textasciitilde}kha'báka{\footnotemark}-{\textasciitilde}sidi-{\textasciitilde}da-ta	tí-ró-ré\\
     \textsc{cop-neg-vis.ipfv.}2/3	want\textsc{-neg-vis.ipfv.}1	1\textsc{pl.excl.poss}=father (kill)\textsc{recp}-do.still-\textsc{pl-intent}	\textsc{anph-sg}\textsc{-obj}\\
\glt ‘“No, we won’t leave it at that, father. We're still going to avenge him.”’
\glt ‘“Não queremos deixar por isso mesmo, pai. Vamos ainda vingar ele).”’
\footnotetext{\label{fn:kotiria:together}The form \textit{kha'maka} is a shortened form of \textit{wãhakha'maka} ‘kill together', meaning ‘to avenge' (also in line 192).}
\z 


\ea “bu'irimaniano tiro buhtibokari” nia tita tikhʉre. \\[.3em]
\gll bu'í-rí-{\textasciitilde}badia-ro	tí-ró	butí-bo-kari	{\textasciitilde}dí-a	tí-ta	tí-khʉ-re \\
     cause-\textsc{nmlz-}not.exist\textsc{-sg}	\textsc{anph-sg}	disappear-\textsc{dub-q.spec}	say-\textsc{assert.pfv}	\textsc{anph(3pl)-emph}	\textsc{anph-add-obj}\\
\glt ‘“How could he just disappear without being avenged?” (they) added that too.’
\glt ‘“Como ele podia sumir sem ser vingado?” disseram bem assim também.’
\z 

\ea ã hi'na mʉhsa mʉa ã nika patere. \\[.3em]
\gll {\textasciitilde}a={\textasciitilde}hí'da	{\textasciitilde}bʉsá	{\textasciitilde}bʉ́-a	{\textasciitilde}a={\textasciitilde}dí-ka	pá-tere \\
     so=\textsc{emph}	2\textsc{pl}	man/person-\textsc{pl}	so=say-\textsc{assert.ipfv}	\textsc{alt}-time\\
\glt ‘[Yes, you men sometimes like to say things like that.]’
\glt ‘[Sim, vocês homens às vezes gostam de falar essas coisas.]’
\z 

 
\ea sã wãhahachʉ, sã ba'ʉre õse yoari hire nika, pairo duruku tuariro thʉ'omahsiriro. \\[.3em]
\gll {\textasciitilde}sá {\textasciitilde}wahá-{\textasciitilde}ha-chʉ	{\textasciitilde}sa=ba'ʉ́-re	{\textasciitilde}óse	yóá-rí	hí-re {\textasciitilde}dí-ka	pá-író	dú-rúkú	tuá-ri-ro	thʉ'ó-{\textasciitilde}basi-ri-ro\\
     1\textsc{pl.excl} kill-\textsc{compl-sw.ref}	1\textsc{pl.excl.poss}=younger.brother\textsc{-obj}	like.this	do\textsc{-nmlz}	\textsc{cop-obj} say-\textsc{assert.ipfv}	\textsc{alt}\textsc{-nmlz.sg}	speak-stand	strong\textsc{-nmlz-sg}	hear-know\textsc{-nmlz-sg} \\
\glt ‘[When one of us is killed, (you) tell our brother to do that (fight, seek revenge), (or) some other one who talks big, is a braggart.]’
\newpage 
\glt ‘[Quando um de nós é morto, (vocês) falam ao nosso irmão assim mesmo (para brigar, vingar), (ou) algum outro que fala muito, que se acha valente.]’
\z

\ea “hai, ya'umai, ya'ukʉse, noana.” \\[.3em]
\gll hái	ya'ú-{\textasciitilde}ba-i	ya'ú-kʉ-se	{\textasciitilde}dóádá \\
     \textsc{intj:}agree	warn-\textsc{frus-vis.pfv.}1	tell-\textsc{m-contr}	very.well\\
\glt ‘“Well then, I warned (you, in vain), I told (you) otherwise, (but) so be it.”’ (\textit{Dianumia Yairo} speaking)
\glt ‘“Então tá, eu avisei (em vão), aconselhei diferente, (mas) que seja assim.”’ (\textit{Dianumia Yairo} falando)
\z 

\ea “mʉhsabahsi mahsia, yʉ pho'na.”  \\[.3em]
\gll {\textasciitilde}bʉsá-básí	{\textasciitilde}basí-á	yʉ={\textasciitilde}pho'dá \\
     2\textsc{pl-emph}	know-\textsc{assert.pfv}	1\textsc{sg.poss}=children \\
\glt ‘“You take care of it yourselves, my sons.”’
\glt ‘“Vocês mesmo resolvem, meus filhos.”’
\z 

\ea “mʉhsa padahchopʉ mʉhsa õse ya'uchʉ, thʉ'oduerana nina.” \\[.3em]
\gll {\textasciitilde}bʉsá	pá-dacho-pʉ	{\textasciitilde}bʉsá	{\textasciitilde}óse	ya'ú-chʉ	thʉ'ó-dua-era-{\textasciitilde}da	{\textasciitilde}dí-ra \\
     2\textsc{pl}	\textsc{alt}-day-\textsc{loc}	2\textsc{pl}	like.this 	warn\textsc{-sw.ref}	hear-\textsc{des-neg-pl}	\textsc{prog-vis.ipfv.}2/3\\
\glt ‘“(I’m) warning you like this, (you) of this next (younger) generation, (but you) aren't listening.”’
\glt ‘“Eu estou aconselhando vocês assim, (vocês) dessa outra geração (mais nova), (mas vocês) não estão querendo ouvir.”’
\z 

\ea nisua, bu'sa wã'koa. \\[.3em]
\gll {\textasciitilde}dí-súá	bu'sá	{\textasciitilde}wa'kó-a \\
     say-angry	adornment	prepare-\textsc{assert.pfv} \\
\glt ‘Saying that angrily, (\textit{Dianumia Yairo}) prepared adornments.'
\glt ‘Dizendo assim chateado, (\textit{Dianumia Yairo}) preparou enfeites.'
\z 

\ea tiro bʉhkʉro, mʉ'nokʉ phʉadʉ noano ñariro phutisu'aa. \\[.3em]
\gll tí-ró	bʉkʉ́-ro	{\textasciitilde}bʉ'dó-kʉ́	phʉá-dʉ́	{\textasciitilde}dóá-ró	{\textasciitilde}yá-ri-ro	phutí-su'a-a \\
     \textsc{anph-sg}	ancestor\textsc{-sg}	tobacco-\textsc{clf:}cylindrical	two-\textsc{clf:}cylindrical	good\textsc{-sg}	bad\textsc{-nmlz-sg}	blow-penetrate-\textsc{assert.pfv}\\
\glt ‘The old guy made two cigars, blowing smoke on them (to bless, or imbue them with violence for success in battle).’
\glt ‘O velho fez dois bons cigarros e soprou neles (fez benzimento de força e violência para vencer na guerra).’
\z 

\ea tina, ã yoata ni, wamoare kha'notu'sʉa. \\[.3em]
\gll tí-{\textasciitilde}da	{\textasciitilde}a=yóá-ta	{\textasciitilde}dí	{\textasciitilde}wabóá-rí	{\textasciitilde}kha'dó-tu'sʉ-a \\
     \textsc{anph-pl}	so=do-\textsc{emph}	say	weapons-\textsc{pl}	prepare/organize-finish-\textsc{assert.pfv}\\
\glt ‘They (the \textit{Biari}), as soon as he said that, set straight away to preparing their weapons.’
\glt ‘Eles (os \textit{Biari}), assim que ele (\textit{Dianumia Yairo}) falou isso, já começaram a preparar as armas.’
\z 

\ea sõ'o sã mipʉ hirore, bʉeakhanʉ peri hiatimare, bʉea, bʉeakhanʉ õmahadohtori waro. \\[.3em]
\gll {\textasciitilde}so'ó	{\textasciitilde}sa={\textasciitilde}bí-pʉ́	hí-ro-re	bʉé-a-{\textasciitilde}khadʉ	péri	hí-ati-{\textasciitilde}bare bʉé-a	bʉé-a-{\textasciitilde}khadʉ	{\textasciitilde}ó-{\textasciitilde}báhá-dótó-rí	wáró\\
     \textsc{deic.dist}	1\textsc{pl.excl.poss}=now\textsc{-loc}	\textsc{cop-sg-obj}	arrow-\textsc{pl}-cane	many	\textsc{cop-ipfv-rem.ipfv} arrow-\textsc{pl}	arrow-\textsc{pl}-cane	\textsc{deic.prox-mov}.upward-bundle-\textsc{pl}	\textsc{emph} \\
\glt ‘[There, in the place we live now (Matapi), there used to be a lot of arrow-cane, arrows, arrow-cane, big bundles this tall.]
\glt ‘[Ali, no lugar onde moramos agora (Matapi), havia muita cana-de-flecha, flechas, cana-de-flecha, montes de feixes altos assim.]’
\z

\ea seis dohtori noano yoaa. \\[.3em]
\gll seis	dotó-rí	{\textasciitilde}dóá-ró	yoá-a \\
     six	bundle\textsc{-pl}	good\textsc{-sg}	do-\textsc{assert.pfv}\\
\glt ‘(Each warrier) made six good bundles (of arrows).’
\glt ‘(Cada guerreiro fez seis feixes (de flechas).’
\z 

\newpage 
\ea “kho'taharo sãre sã ba'ʉ yuhpi diani,” nirokaata. \\[.3em]
\gll kho'tá-ha-ro	{\textasciitilde}sá-re	{\textasciitilde}sa=ba'ʉ́	yuhpí.diáni	{\textasciitilde}dí-dóká-á-tá \\
     wait-\textsc{compl-sg}	1\textsc{pl.excl-obj}	1\textsc{pl.excl.poss}=younger.brother	yuhpi.diani	say-\textsc{dist-assert.pfv-emph}\\
\glt ‘“Tell our younger brother \textit{Yuhpi Diani} to be expecting us,” they sent off a message.’
\glt ‘“Avisa ao nosso irmão menor \textit{Yuhpi Diani} que vamos chegar,” mandaram recado.’
\z 

\ea mʉ'ʉ hii di'tamaniano wãhaboakʉ. \\[.3em]
\gll {\textasciitilde}bʉ'ʉ́	hí-i	di'tá-{\textasciitilde}badia{\footnotemark}-ro	{\textasciitilde}wahá-bo-a-kʉ \\
     2\textsc{sg}	\textsc{cop-m}	dirt-not.exist\textsc{-sg}	kill-\textsc{dub-affect-m}\\
\glt ‘[(But) if it were you, better to attack without letting them know!]’ (Teresinha jokes with Joselito)
\glt ‘[(Mas) se fosse você, melhor atacar sem avisar!]’ (Teresinha brinca com Joselito)
\footnotetext{The expression \textit{di'tamania} ‘no dirt’ means to ‘do something in silence'.}
\z 

 
\ea ñaina bʉrʉta õna ma'apʉ taa te wãhsipiria tũhkui. \\[.3em]
\gll {\textasciitilde}yá-{\textasciitilde}ida	bʉrʉ́-ta	{\textasciitilde}ó-({\textasciitilde}hí')da	{\textasciitilde}ba'á-pʉ́	tá-á té	wãhsipiria	{\textasciitilde}túkú-í\\
     bad\textsc{-nmlz.pl}	go.downriver-come	\textsc{deic.prox-emph}	path\textsc{-loc}	come-\textsc{assert.pfv} until	wãhsipiria	pond\textsc{-loc}.vis\\
\glt ‘The warriors came down right over here, on the river, coming right up to the \textit{Wãhsipiria} pond (river port).’
\glt ‘Os guerreiros vinham descendo bem aqui, no caminho (pelo rio), vindo até o poço de \textit{Wãhsipiria} (porto no rio).’
\z 

\ea ti bʉhsoka hia phiriaka'saribʉhsoka. \\[.3em]
\gll ti=bʉsó-ka	hí-a	phí-ria-ka'sa-ri-bʉso-ka	 \\
     3\textsc{pl.poss}=canoe-\textsc{clf:}round	\textsc{cop-assert.pfv}	big-\textsc{clf:}round-bark\textsc{-nmlz}-canoe-\textsc{clf:}round	\\
\glt ‘Their canoes were huge bark canoes.’
\glt ‘Suas canoas enormes eram feitas com casca de árvores.’
\z 

\ea do'se ka'saribʉhsokare yoari hi’na ñaina. \\[.3em]
\gll do'sé	ka'sá-rí-bʉ́só-ká-ré	yoá{\footnotemark}-ri	{\textasciitilde}hí'da	{\textasciitilde}yá-{\textasciitilde}ida \\
     \textsc{q:}how	bark\textsc{-nmlz}-canoe-\textsc{clf:}round\textsc{-obj}	do/make-\textsc{q.pfv}	\textsc{emph}	bad\textsc{-nmlz.pl}\\
\glt ‘[How did the warriors make those bark canoes?]’
\glt ‘[Como os guerreiros faziam essas canoas de casca?]’ 
\footnotetext{The root \textit{yoa} means ‘do’ or ‘make’, and is glossed with the meaning appropriate to the context.}
\z

\ea õna bʉrʉtaa.  \\[.3em]
\gll {\textasciitilde}ó-({\textasciitilde}hi')da	bʉrʉ́-ta-a \\
     \textsc{deic.prox-emph}	go.downriver-come-\textsc{assert.pfv}\\
\glt ‘Right down here (they) came.’
\glt ‘Desceram bem para cá.’
\z 

 
\ea õre wa'aa, õre yʉ'dopo'o. \\[.3em]
\gll {\textasciitilde}ó-ré	wa'á-a	{\textasciitilde}ó-ré	yʉ'dó-po'o \\
     \textsc{deic.prox-obj} 	go-\textsc{assert.pfv}	\textsc{deic.prox-obj} 	drag.canoe-place.floating\\
\glt ‘They went (through the rapids), dragging the canoes (on the rocks) over here.’{\footnotemark}
\glt ‘Passaram (a cachoeira), arrastando as canoas (pelas pedras) aqui.’
\footnotetext{Teresinha gestures to the locations she is referring to, all near the village where the narrative was recorded.}
\z 

\ea ñaina phiriaka bihsibʉrʉaduhkua noano me'neta pu ... pu .. .puuuuuuu, \\[.3em]
\gll {\textasciitilde}yá-{\textasciitilde}ida 	phíríá-ká	bisí-bʉ́rʉ́-á-dúkú-a {\textasciitilde}dóá-ró={\textasciitilde}be're-ta	pu...pu...puuuuuuu\\
     bad\textsc{-nmlz.pl}	flute-\textsc{clf:}round	sound-go.downriver-go-stand-\textsc{assert.pfv} good\textsc{-sg}=com-\textsc{emph}	\textsc{ontp:}flute.playing\\
\glt ‘The warriors came down to the constant sound of \textit{phiriaka} flutes: \textit{pu ... pu ... puuuuuuu!}’ 
\glt ‘Os guerreiros vinham descendo sempre ao som das flautas \textit{phiriaka}: \textit{pu ... pu ... puuuuuuu!}’ 
\z 

\newpage 
\ea kho'tathu'sʉmaa, tinakãkhʉ. sʉ̃pa tina bʉeri himarepha tinare. \\[.3em]
\gll kho'tá-thu'sʉ-{\textasciitilde}ba-a	tí-{\textasciitilde}da-{\textasciitilde}ka-khʉ	{\textasciitilde}sʉ́-pa	tí-{\textasciitilde}da bʉé-rí	hí-{\textasciitilde}bare-pha	tí-{\textasciitilde}da-re\\
     wait-finish-ready-\textsc{assert.pfv}	\textsc{anph-pl-dim-add}	arrive.\textsc{trns}-bring.to.shore	\textsc{anph-pl} arrow.shoot\textsc{-pl}	\textsc{cop-rem.ipfv}-time	\textsc{anph-pl-obj}\\
\glt ‘(\textit{Yuhpi Diani}'s men) were already there ready waiting, (and) when (the \textit{Biari}) landed, they starting right in firing arrows at them.’
\glt ‘Os (guerreiros de \textit{Yuhpi Diani}) também já estavam esperando de prontidão (e) quando (os \textit{Biari}) encostaram, já começaram a flechá-los.’
\z 

\ea tinakãkhʉ yoakha'ma, wãhañoeraa ñaina. \\[.3em]
\gll tí-{\textasciitilde}da-{\textasciitilde}ka-khʉ	yoá-{\textasciitilde}kha'ba	{\textasciitilde}waha-{\textasciitilde}yo-éra-a	{\textasciitilde}yá-{\textasciitilde}ida \\
     \textsc{anph-pl-dim-add}	do-\textsc{recp}	kill-do.immediately\textsc{-neg-assert.pfv}	bad\textsc{-nmlz.pl}\\
\glt ‘They too (\textit{Yuhpi Diani}'s men) returned fire, (but the \textit{Biari}) weren’t hit.’
\glt ‘Os (guerreiros do \textit{Yuhpi Diani}) também começaram a flechar de volta, (mas os \textit{Biari}) não morriam.’
\z 

\ea de to kaseroakã ... \\[.3em]
\gll dé	to=kaséro-a-{\textasciitilde}ka \\
     \textsc{intj:}poor.one(s)!	3\textsc{sg.poss}=servant\textsc{-pl-dim}\\
\glt ‘Poor guys, his (\textit{Yuhpi Diani}'s) servants …’
\glt ‘Coitados, os criados (do \textit{Yuhpi Diani}) ...’
\z 

\largerpage 
\ea tiata noano wi'ichʉ, de õita tiro du'tihã hi'na, bʉea tinare phi'tiawa'aa. \\[.3em]
\gll tíá-ta	{\textasciitilde}dóá-ró	wi'í-chʉ	dé	{\textasciitilde}ó-i-ta	tí-ró du'tí-{\textasciitilde}ha	{\textasciitilde}hí'da	bʉé-a	tí-{\textasciitilde}da-re{\footnotemark}	phi'tí-a-wa'a-a\\
     three-\textsc{emph}	good\textsc{-sg}	arrive\textsc{.cis-sw.ref}	\textsc{intj:}poor.one!	\textsc{deic.prox-loc.vis-emph}	\textsc{anph-sg} escape-\textsc{compl}	\textsc{emph}	arrow\textsc{-pl}	\textsc{anph-pl-obj}	end-\textsc{affect}-go-\textsc{assert.pfv}\\
\footnotetext{Use of the objective case suffix \textit{-re} on the pronoun \textit{tina} and the marker \textit{-a} on the verb emphasize how the subject is being negatively affected by the situation.}
\glt ‘After three rounds (of arrows) had arrived, with the next round, the poor guy (\textit{Yuhpi Diani}) ran off, the arrows running out on them.’
\newpage 
\glt ‘Depois de chegarem três levas (de flechas), na outra o coitado (\textit{Yuhpi Diani}) acabou fugindo, as flechas deles acabando.’
\z

  
\ea siesere ã yoa duhkuhãa nia painase. \\[.3em]
\gll sié-(bʉ')se-re	{\textasciitilde}a=yóá	dukú-{\textasciitilde}ha-a	{\textasciitilde}dí-a	pá-{\textasciitilde}ida-se \\
     front-side\textsc{-obj}	so=do	stand-\textsc{compl-pl}	\textsc{prog-assert.pfv}	\textsc{alt-nmlz.pl-contr}\\
\glt ‘(While) from the opposite side the others (\textit{Biari}) continued on (shooting).’
\glt ‘(Enquanto) do outro lado os outros (os \textit{Biari}) continuavam (a flechar).’
\z 

 
\ea phome, ba'aro utimu, ti sʉ̃chʉ, tinare tãa dohkare. \\[.3em]
\gll {\textasciitilde}phobé	ba'á-ró	utimu	ti={\textasciitilde}sʉ́-chʉ́	tí-{\textasciitilde}da-re	{\textasciitilde}tá-a	doká-re \\
     give.up	after\textsc{-sg}	last.one	3\textsc{pl.poss}=arrive.\textsc{trns-sw.ref}	\textsc{anph-pl-obj}	rock\textsc{-pl}	throw-\textsc{vis.pfv.}2/3\\
\glt ‘Already exhausted, after their (the \textit{Biari}’s) last (arrow) arrived, (\textit{Yuhpi Diani}'s servants) threw rocks (at the \textit{Biari}).’
\glt ‘Já cansados, depois que chegou a última (flecha dos \textit{Biari}), (os criados de \textit{Yuhpi Diani}) jogaram pedras (nos \textit{Biari}).’
\z 

\ea toi ni ã yoari himarebʉ sõ'o khã nʉhkore ...  \\[.3em]
\gll tó-i	{\textasciitilde}dí	{\textasciitilde}a=yóá-rí	hí-{\textasciitilde}bare-bʉ	{\textasciitilde}so'ó	{\textasciitilde}khá-{\textasciitilde}dʉko-re \\
     \textsc{rem-loc.vis}	say	so=do\textsc{-nmlz}	\textsc{cop-rem.ipfv-epis}	\textsc{deic.dist}	hawk-island\textsc{-obj}\\
\glt ‘There, it's said (that), in the old times, there in \textit{Khã Nʉhko} …’
\glt ‘Lá, dizem então, que lá nos tempos antigos, em \textit{Khã Nʉhko} ...’
\z 

\ea thu'sʉ, pitiasama sa'ari himarebʉta. \\[.3em]
\gll thu'sʉ́	pítia{\textasciitilde}saba	sa'á-ri	hí-{\textasciitilde}bare-bʉ-ta \\
     finish	war.trench	dig\textsc{-nmlz}	\textsc{cop-rem.ipfv-epis-emph} \\
\glt ‘in the end, (\textit{Yuhpi Diani}'s men) dug trenches, it seems that's just how it was.’
\glt ‘no final (os criados de \textit{Yuhpi Diani}) cavaram trincheiras, parece que foi assim mesmo.’
\z 

\ea “ã hiro to pitiasama to di khõarinʉhko hira, yʉ mahko.” \\[.3em]
\gll {\textasciitilde}a=hí-ro	to=ṕitia{\textasciitilde}saba	to=dií	{\textasciitilde}khoá-ri-{\textasciitilde}dʉko hí-ra	yʉ={\textasciitilde}bakó\\
     so=\textsc{cop-sg}	3\textsc{sg.poss}=war.trench	3\textsc{sg.poss}=blood	lie\textsc{-nmlz}-island \textsc{cop-vis.ipfv.}2/3	1\textsc{sg.poss}=daughter\\
\glt ‘[“That why it's an island (that has) trenches with their spilled blood, my daughter,” (my father said).{\footnotemark}]’
\glt ‘[“Por isso, é uma ilha (que tem) buracos com sangue deles derramado, minha filha," (dizia meu pai).]’
\footnotetext{Teresinha explicitly attributes this negative assessment of the neighboring island and its inhabitants to her father, perhaps as a way to sidestep responsibility for it. Her use of the epistemic expression \textit{himarebʉta} just two lines earlier may also signal that she is aware she is entering into delicate politico-narrative territory.} 
\z 

\ea “patere bʉhkʉthuru so'toapʉre kha'mawãha buhtiaka a’ri nʉhkore.” \\[.3em]
\gll pátere	bʉkʉ́-thúrú	so'tóa-pʉ-re	{\textasciitilde}kha'bá-{\textasciitilde}waha	butí-a-ka	a’rí	{\textasciitilde}dʉkó-re \\
     maybe	old-times	end\textsc{-loc-obj}	do/bring.together-kill	disappear\textsc{-pl-predict}	\textsc{dem.prox}	island\textsc{-obj}\\
\glt ‘[“Maybe, in the future they're all going to kill each other and disappear from the island.”]’
\glt ‘[“Talvez, no futuro vão se-matar e sumir todos dessa ilha.”]’
\z 

\ea niatire mai, yʉ ñʉhchʉmʉnano.  \\[.3em]
\gll {\textasciitilde}dí-átí-ré	{\textasciitilde}baí	yʉ={\textasciitilde}yʉchʉ́-{\textasciitilde}bʉ́dá-ró \\
     say\textsc{-ipfv-vis.pfv.}2/3	father	1\textsc{sg.poss}=grandfather-deceased\textsc{-sg}\\
\glt ‘[My dad (and) my late grandfather used to say that.]’
\glt ‘[Assim papai (e) meu avô
finado contavam.]’
\z

\ea yʉ phʉkʉ niya'ure, niya'utire. \\[.3em]
\gll yʉ=phʉkʉ́	{\textasciitilde}dí-yá'ú-ré	{\textasciitilde}dí-yá'ú-átí-ré \\
     1\textsc{sg.poss}=father	say-tell-\textsc{vis.pfv.}2/3	say-tell-\textsc{ipfv-vis.pfv.}2/3\\
\glt ‘[My father told (me), used to tell (me).]’
\glt ‘[Meu pai contou, contava.]’
\z 

\newpage 
\ea ã to dikhõano hiro niyu'ka to. ã hia tina kha'machʉa phayʉ nika tore. \\[.3em]
\gll {\textasciitilde}a=tó	dií-{\textasciitilde}khoa-ro	hí-ro	{\textasciitilde}dí-yu'ka{\footnotemark}	tó {\textasciitilde}a=hí-a	tí-{\textasciitilde}da	{\textasciitilde}kha'báchʉ-a	phayʉ́	{\textasciitilde}dí-ka	tó-re \\
     so=\textsc{rem}	blood-lie\textsc{-sg}	\textsc{cop-sg}	say-\textsc{rep.quot}	\textsc{rem} so=\textsc{cop-assert.pfv}	\textsc{anph-pl}	fight\textsc{-pl}	many	\textsc{prog-assert.ipfv}	\textsc{rem-obj}\\
\glt ‘[So they say it's a place that has spilled bood. That's why they're always fighting there.]’
\glt ‘[Dizem que é um lugar de sangue derramado. É por isso que estão sempre brigando lá.]’
\footnotetext{Use of the quotative evidential in these statements is another strategy by which Teresinha can attribute a negative opinion to others.} 
\z 

\ea ti(na) pimʉ'nonore bahsama'noeina, õse ti(na) bihsiawe boowema'noeno. \\[.3em]
\gll tí	pí-{\textasciitilde}bʉ'do-ro-re	basá{\footnotemark}-{\textasciitilde}ba'doé-{\textasciitilde}ida {\textasciitilde}óse	ti=bisí-a-we	bó-o-we-{\textasciitilde}ba'doé-ro\\
     \textsc{anph}	war-tobacco\textsc{-sg-obj}	bless-not.do\textsc{-nmlz.pl} like.this	3\textsc{pl.poss}=sweet-\textsc{affect-mov.}through	descend-\textsc{caus-mov.}through-not.do\textsc{-sg}\\
\glt ‘[They're ones who never received the (anti-)war blessing, to make them sweet (calm, non-violent).]’
\glt ‘[Eles que são não benzidos (contra) guerra, que os tornariam doces (calmos, não-violentos).]’
\footnotetext{The root \textit{bahsa} is used with a range of meanings, including ‘dance/sing' (see lines 235, 237 and ‘bless'.} 
\z

\ea bahsama'noeri nʉhko hikano nika, yo'o ti nʉhko. \\[.3em]
\gll basá-{\textasciitilde}ba'doé-ri	{\textasciitilde}dʉkó	hí-ka-ro	{\textasciitilde}dí-ka	yo'ó	tí	{\textasciitilde}dʉkó \\
     bless-not.do\textsc{-nmlz}	island	\textsc{cop-predict-sg}	\textsc{prog-assert.ipfv}	slaughter	\textsc{anph}	island\\
\glt ‘[It will always be an unblessed island, a violent island.]’
\glt ‘[Será sempre uma ilha que não foi benzida, uma ilha de violência.]’
\z 

\newpage 
\ea õse wamoa ti da'rasuariro. \\[.3em]
\gll {\textasciitilde}óse	{\textasciitilde}wabóá	tí	da'rá-sua{\footnotemark}-ri-ro \\
     like.this	violence	\textsc{anph}	work-angry\textsc{-nmlz-sg}\\
\glt ‘[So everyone is violent.]’
\glt ‘[Assim todos são violentos.]’
\footnotetext{The expression \textit{da’ra-sua} indicates it’s as if they always had weapons in their hands (instead of tools or other normal work instruments), see line 150.}
\z 

\ea ã yoa phomea.  \\[.3em]
\gll {\textasciitilde}a=yóá	{\textasciitilde}phobé-a \\
     so=do	give.up-\textsc{assert.pfv}\\
\glt ‘So, (they, \textit{Yuhpi Diani}'s servants) gave up.’ 
\glt ‘Então, (eles, os criados de \textit{Yuhpi Diani}) desistiram.’
\z 

\ea “marire phi'tiera a’ri marire” ni tinakã du'tika'aa te khã pho'tai. \\[.3em]
\gll {\textasciitilde}barí-re	phi'ti-éra	a’rí	{\textasciitilde}barí-re	{\textasciitilde}dí tí-{\textasciitilde}da-{\textasciitilde}ka	du'tí-ka'a-a	té	{\textasciitilde}khá-pho'ta-i\\
     1\textsc{pl.incl-obj}	end\textsc{-neg}	\textsc{dem.prox}	1\textsc{pl.incl-obj}	say \textsc{anph-pl-dim}	escape-do.moving-\textsc{assert.pfv}	until	hawk-headwater\textsc{-loc.vis}\\
)\glt ‘“This (attack) against us never ends!” (they) said (and escaped off to the headwaters of the Inambú stream (\textit{Khã Pho'tai}).’
\glt ‘“Esse (ataque) não acaba!” disseram (e) fugiram até a cabeceira do igarapé Inambú (\textit{Khã Pho'tai}).’
\z

\ea no'oi hiharita baraphoa? no'oi hikarita ti phoayeri? \\[.3em]
\gll {\textasciitilde}do'ó-i	hí-hari-ta	bará-phoa {\textasciitilde}do'ó-i	hí-kari-ta	tí	phoá-yeri \\
     \textsc{q:}where\textsc{-loc.vis}	\textsc{cop-q.ipfv-emph}	potato.sp-falls/rapids \textsc{q:}where\textsc{-loc.vis}	\textsc{cop-q.spec-emph}	\textsc{anph}	falls/rapids\textsc{-pl}\\
\glt ‘[Where are those ‘potato’ rapids?{\footnotemark} I wonder where those rapids are.]’
\glt ‘[Onde fica essa cachoeira de ‘batata’? Não sei onde fica essa cachoeira.]’
\footnotetext{Teresinha uses the name by which the rapids are currently known.}
\z 

\newpage 
\ea kha pho'tai, toi hia. \\[.3em]
\gll khá-pho'ta-i	tó-i	hí-a \\
     hawk-headwater\textsc{-loc.vis}	\textsc{rem-loc.vis}	\textsc{cop-assert.pfv}\\
\glt  (Joselito responds) ‘[The Inambú (hawk) headwaters are over there.]’
\glt  (Joselito responde) ‘[A cabeçeira do Inambú (gavião) é para lá.]’
\z 

\ea “ne,” nimato, de ne mania.  \\[.3em]
\gll {\textasciitilde}dé	{\textasciitilde}dí-{\textasciitilde}ba-to	dé	{\textasciitilde}dé	{\textasciitilde}badía \\
     \textsc{intj:}hello	say-\textsc{frus-nmlz.loc/evnt}	\textsc{intj:}poor.one(s)!	\textsc{neg}	not.exist\\
\glt ‘“Hello?” (\textit{Diani}'s warriors, the \textit{Biari}) called out (in vain), but poor guys, there was no one.’
\glt ‘“Alô?” (os guerreiros do Diani, os \textit{Biari}) chamaram (em vão), mas coitados, não havia ninguém.’
\z 

 
\ea mahka tini manieno: “no'oi wa'ari” ni. \\[.3em]
\gll {\textasciitilde}baká-{\textasciitilde}tidi	{\textasciitilde}badié-ro	{\textasciitilde}do'ó-i	wa'á-ri	{\textasciitilde}dí \\
     look.for-wander.around	not.have\textsc{-sg}	\textsc{q:}where\textsc{-loc.vis}	go-\textsc{q.pfv}	say\\
\glt ‘Looking all over (the island): “Where have they gone?” (they) wondered.’
\glt ‘Procurando por toda parte (da ilha): “Onde foram?” se perguntaram.’
\z 

\ea tia nʉmʉ hi, panʉmʉ ñami bʉrʉta ñaina. \\[.3em]
\gll tíá	{\textasciitilde}dʉbʉ́	hí	pá-{\textasciitilde}dʉbʉ	{\textasciitilde}yabí	bʉrʉ́-ta	{\textasciitilde}yá-{\textasciitilde}ida \\
     three	day	\textsc{cop}	\textsc{alt}-day	night	go.downriver-come	bad\textsc{-nmlz.pl}\\
\glt ‘Three days passed (and) the next night (\textit{Dianumia Yairo} and his warriors from Matapi) came downriver.’
\glt ‘Passaram três dias (e) no dia seguinte de noite (\textit{Dianumia Yairo} e os guerreiros de Matapi) vinham descendo.’
\z 

\ea de thʉ'oboropʉ tho! tho! tho! \\[.3em]
\gll dé	thʉ'ó-boro-pʉ	tho!...tho!...tho! \\
     \textsc{intj:}poor.one(s)!	hear-separate.into.pieces\textsc{-loc}	\textsc{ontp:}chopping\\
\glt ‘Poor guys (\textit{Yuhpi Diani}'s servants). (The \textit{Biari}) could hear from afar the sound of chopping: \textit{Tho! Tho! Tho!}’
\glt ‘Coitados (os criados de \textit{Yuhpi Diani}). (Os \textit{Biari}) ouviram de longe o som de batidas: \textit{Tho! Tho! Tho!}’ 
\z 

\ea yoati pakhuoina, a’ri wʉ'ʉ thu'sʉpha’yoa thu'sʉa. \\[.3em]
\gll yoá-ati	pá-khui-o{\footnotemark}-{\textasciitilde}ida	a’rí	wʉ'ʉ́	thu'sʉ́-phá'yó-á	thu'sʉ́-a \\
     do-\textsc{ipfv}	\textsc{alt}-afraid-\textsc{caus-nmlz.pl}	\textsc{dem.prox}	house	finish-complete-\textsc{affect}	finish-\textsc{assert.pfv}\\
\glt ‘The other frightened ones (\textit{Yuhpi Diani}’s servants) were finishing up (the construction) of a barricade.’
\glt ‘Os outros apavorados (os criados de \textit{Yuhpi Diani}) estavam acabando (de construir) uma barricada.’
\footnotetext{Some verbs in Kotiria have causative forms derived from the root+causative suffix \textit{-o}, as we see here in the nominalized form \textit{pakhuoina} ‘other frightened/terrified ones'.} 
\z

 
\ea a’ri cerca thu'sʉa to mʉhapisaato.\\[.3em]
\gll a’rí	cerca{\footnotemark}	thu'sʉ́-a	to={\textasciitilde}bʉhá-phísá-tó\\
     \textsc{dem.prox}	fence	finish-\textsc{assert.pfv}	3\textsc{sg.poss}=\textsc{mov.}upward-be.on-\textsc{nmlz.loc/evnt}\\
\glt ‘They finished the barricade, where they could be up high (on a platform).'
\glt ‘Acabaram a barricada, onde subiam bem no alto (numa plataforma).'
\footnotetext{Here Teresinha uses a borrowed word from Portuguese \textit{cerca} ‘fence' to refer to the barricade.}
\z 

\ea ‘khakhasario’ nimarebʉ tokhʉre, khakhasario ti yoario. \\[.3em]
\gll khá-khásá-ríó	{\textasciitilde}dí-{\textasciitilde}bare-bʉ	tó-khʉ-re	khá-khásá-ríó	ti=yoá-rio \\
     hawk-platform-\textsc{clf:}flat	say\textsc{-rem.ipfv-epis}	\textsc{def-add-obj}	hawk-platform-\textsc{clf:}flat	3\textsc{pl.poss}=do-\textsc{clf:}flat\\
\glt ‘It was called then \textit{khakhasario}, it was a “hawk's nest” that they built.’
\glt ‘Era chamada naquele tempo de \textit{khakhasario}, foi “ninho de gavião” que fizeram.’
\z 

\largerpage
\ea tiñami toi koatare nia, “ne hiharo khatiduaro” nika tiro.  \\[.3em]
\gll tí-{\textasciitilde}yábí	tó-i	koá-ta-re	{\textasciitilde}dí-a	{\textasciitilde}dé	hí-haro{\footnotemark}	khatí-dúá-ró {\textasciitilde}dí-ka	tí-ró\\
     \textsc{anph}-night	\textsc{rem-loc.vis}	make.noise-come\textsc{-obj}	say-\textsc{assert.pfv}	\textsc{neg}	\textsc{cop-imp.3}	live-\textsc{des-sg} say-\textsc{assert.ipfv}	\textsc{anph-sg} \\     
\footnotetext{The suffix \textit{-haro} is a third-person imperative ‘allow him to be one who lives’. See another instance of the same morpheme in line 234.}
\glt ‘That night, hearing the noise from up there (\textit{Dianumia Yairo}) said: “Don't (attack), let (\textit{Yuhpi Diani}) survive.”’ 
\glt ‘Naquela noite, ouvindo o som vindo de lá (\textit{Dianumia Yairo}) disse: “Não (ataca), deixa (\textit{Yuhpi Diani}) sobreviver.”’ 
\z

\ea “to yabu'iri, mʉhsa koiro thʉ'oerare.” \\[.3em]
\gll to=yá-bu'i-ri	{\textasciitilde}bʉsa=kó-iro	thʉ'o-éra-re \\
     3\textsc{sg.poss=poss}-cause\textsc{-nmlz}	2\textsc{pl.poss}=relative\textsc{-nmlz.sg}	hear\textsc{-neg-vis.pfv.}2/3\\
\glt ‘“He's at fault, your brother wouldn't listen.”’
\glt ‘“Ele é culpado, seu irmão não ouvia (meus conselhos).”’
\z 

\ea “hira to pho'nakã ñahoriakã, tina mahsaphutiaka” nimaati. \\[.3em]
\gll hí-ra	to={\textasciitilde}pho'dá-{\textasciitilde}ka	ñáhórí-á-{\textasciitilde}ká	tí-{\textasciitilde}da {\textasciitilde}basá-phú-tí-á-ká	{\textasciitilde}dí-{\textasciitilde}ba-ati\\
     \textsc{cop-vis.ipfv.}2/3	3\textsc{sg.poss}=children-\textsc{dim}	ñahori\textsc{-pl-dim}	\textsc{anph-pl} people-expand-\textsc{vbz-pl-predict}	say-\textsc{frus-ipfv}\\
\glt ‘“(But) there are two \textit{Ñahori} children, they will multiply (reproduce),” (\textit{Yuhpi Diani}) said (in vain, still trying to convince them).’
\glt ‘“(Mas) há dois pequenos filhos do \textit{Ñahori}, eles se multiplicarão,” disse (\textit{Yuhpi Diani} em vão, tentando convencê-los).’
\z 

\ea “kha'maeraha, kha'maeraha, kha'maeraha, kha'maerakãha” nia.\\[.3em]
\gll {\textasciitilde}kha'ba-éra-ha	{\textasciitilde}kha'ba-éra-ha	{\textasciitilde}kha'ba-éra-ha {\textasciitilde}kha'ba-éra-{\textasciitilde}ka-ha	{\textasciitilde}dí-a\\
     want\textsc{-neg-vis.ipfv.}1	want\textsc{-neg-vis.ipfv.}1	want\textsc{-neg-vis.ipfv.}1 want\textsc{-neg}-dim-\textsc{vis.ipfv.}1	say-\textsc{assert.pfv}\\
\glt ‘“We refuse! We refuse! We refuse! We absolutely refuse!” (they) insisted.’
\glt ‘“Não queremos! Não queremos! Não queremos! Não queremos mesmo!” insistiram.’
\z

\ea “(wãha)kha'makasininatha ñatiaro yoari hire,” niatia. \\[.3em]
\gll ({\textasciitilde}waha){\textasciitilde}kha'báka-{\textasciitilde}sidi-{\textasciitilde}da-ta{\footnotemark}	{\textasciitilde}yá-ti-a-ro	yoá-ri	hí-re	{\textasciitilde}dí-ati-a \\
     (kill)\textsc{recp}-do.still\textsc{-pl-emph}	bad-\textsc{attrib-affect-sg}	do\textsc{-nmlz(infer) }	\textsc{cop-vis.pfv.}2/3	say-\textsc{ipfv-assert.pfv}\\
\footnotetext{Here we see an example of the suffixes \textit{-na-ta} marking 1\textsc{pl} intent (see also lines 69 and 87) in contrast to 1\textsc{sg} intent (in lines 13, 45, and others). We also find a good example of inference evidential marking, used because the speakers only saw the result of what was done to their brother, but not the actual actions. The expression \textit{(wãha)kha'maka} is understood to mean ‘kill as was done to them’, in other words, to avenge.}
     
\glt ‘“We're still going to avenge (our brother) for the evil done to him,” they kept saying.’
\glt “Ainda vamos vingá-lo (nosso irmão) pelo mal feito a ele,” ficavam dizendo.’
\z

\ea “hai” ni. \\[.3em]
\gll hái	{\textasciitilde}dí \\
     \textsc{intj:}agree	say\\
\glt ‘“All right” (\textit{Dianumia Yairo}) said.’
\glt ‘“Está bom,” falou (\textit{Dianumia Yairo}).’
\z 
\ea thʉ'oa, ñami no'opeina taa nia ñaina, quatro ou seis hiari ti(na)?  \\[.3em]
\gll thʉ'ó-a	{\textasciitilde}yabí	{\textasciitilde}do'ó-pe-{\textasciitilde}ida	tá-a	{\textasciitilde}dí-a	{\textasciitilde}yá-{\textasciitilde}ida quatro ou seis	hí-a-ri	tí({\textasciitilde}da)\\
     hear\textsc{-pl}	night	\textsc{q-quant.c-nmlz.pl}	come\textsc{-pl}	\textsc{prog-assert.pfv}	bad\textsc{-nmlz.pl} four or six	\textsc{cop-pl-q.pfv}	\textsc{anph}\\
\glt ‘Hearing (the sounds from the barracade), at night … [how many warriors were there – four or six of them maybe?]’ 
\glt ‘Ouvindo (os barulhos da barricada), de noite … [quantos guerreiros eram – quatro ou seis talvez?]’ 
\z 

\ea wa'awa'atha ñamitha pʉ cerca. \\[.3em]
\gll wa'á-wa'a-a-ta	{\textasciitilde}yabí-ta	pʉ́	cerca \\
     go-go-\textsc{assert.pfv-emph}	night-\textsc{emph}	\textsc{dist} fence\\
\glt ‘(They) went right up to the barricade.’
\glt ‘Foram indo até perto da barricada.’
\z 

\ea noano õsekã yoamaati tinakã, cerca ne suhsueraro. \\[.3em]
\gll {\textasciitilde}dóá-ró	{\textasciitilde}óse-{\textasciitilde}ka	yoá-{\textasciitilde}ba-ati	tí-{\textasciitilde}da-{\textasciitilde}ka	cerca	{\textasciitilde}dé	susu-éra-ro \\
     good\textsc{-sg}	like.this-\textsc{dim}	do-\textsc{frus-ipfv}	\textsc{anph-pl-dim}	fence	\textsc{neg}	have.holes\textsc{-neg-sg}\\
\glt ‘They (\textit{Yuhpi Diani}'s servants) were making a good barricade, like this with no open spaces (to get through).’
\glt ‘Eles (os criados do \textit{Yuhpi Diani}) estavam fazendo a barricada bem feita assim, sem frestas (para alguem passar).’
\z 

  
\ea ñariro hũiro dohomʉaa.  \\[.3em]
\gll {\textasciitilde}yá-ri-ro	{\textasciitilde}hú-iro	dohó-{\textasciitilde}bʉ{\footnotemark}-(wa’)a-a \\
     bad\textsc{-nmlz-sg}	worm\textsc{-nmlz.sg}	transform-run-go-\textsc{assert.pfv}\\
\glt ‘\textit{Dianumia Yairo} (and his men) quickly transformed into worms.’
\glt ‘\textit{Dianumia Yairo} (e os guerreiros) rapidamente se transformaram em minhocas.’
\footnotetext{When the root \textit{mʉ} ‘run’ occurs in a serial verb construction it adds the adverbial notion to ‘do X quickly’.}
\z 

\ea bʉhkʉthurupʉ hiri himarero, doho sã'a phitiawa'aa. \\[.3em]
\gll bʉkʉ́-thúrú-pʉ́	hí-ri	hí-{\textasciitilde}bare-ro	dohó	{\textasciitilde}sa'á	phíti-a-wa'a-a \\
     ancestor-times\textsc{-loc}	\textsc{cop-nmlz}	\textsc{cop-rem.ipfv-sg}	transform	\textsc{mov.}inside	accompany\textsc{-pl}-go-\textsc{assert.pfv}\\
\glt ‘[That's the way it was in ancient times, (they) transformed (and) all of them went (into the ground).]’
\glt ‘[Era assim nos tempos antigos, se transformavam (e) entraram todos (na terra).]’
\z

\largerpage[2]
\ea ti sãsʉrʉka'achʉwaro, nia to kaseroa: \\[.3em]
\gll ti={\textasciitilde}sá('a)-sʉrʉ-ka'a-chʉ-waro	{\textasciitilde}dí-a	to=kaséro-a \\
     3\textsc{pl.poss}=\textsc{mov.}inside-pause-beside-\textsc{sw.ref-emph}	say-\textsc{assert.pfv}	3\textsc{sg.poss}=servant\textsc{-pl}\\
\glt ‘When they came out right beside (\textit{Yuhpi Diani}'s) servants, they cried out:’
\glt ‘Quando eles sairam bem perto dos criados (do \textit{Yuhpi Diani}), eles gritaram:’
\z 
\newpage 

\ea “sã phʉ'toro duhitaga mʉ'ʉ.” \\[.3em]
\gll {\textasciitilde}sa=phʉ'tó-ro	duhí-ta-ga	{\textasciitilde}bʉ'ʉ́ \\
     1\textsc{pl.excl.poss}=master\textsc{-sg}	descend-come-\textsc{imp}	2\textsc{sg}\\
\glt ‘“Come down, master!”’
\glt ‘“Desça, chefe!”’
\z 

 
\ea “mʉ'ʉ ‘dʉhkaboha nimeheta’, mʉ'ʉ sãre tire wã'kore.” \\[.3em]
\gll {\textasciitilde}bʉ'ʉ́	dʉka-bo-ha	{\textasciitilde}dí-{\textasciitilde}beheta	{\textasciitilde}bʉ'ʉ́	{\textasciitilde}sá-re	tí-re	{\textasciitilde}wa'kó-re \\
     2\textsc{sg}	begin-\textsc{dub-vis.ipfv.}1	say\textsc{-neg.intens}	2\textsc{sg}	1\textsc{pl.excl-obj}	\textsc{anph-obj}	cause.to.happen-\textsc{vis.pfv.}2/3\\
\glt ‘“You (decided) ‘I’m going to start (this war)’, it's your fault this is happening to us!”’
\glt ‘“Você (resolveu) ‘Eu vou começar (essa guerra)’, é sua culpa o que está acontecendo conosco!”’
\z 

\ea taga mʉ'ʉkhʉ sã da'rana yoakha'maha sã,” nia.  \\[.3em]
\gll tá-gá	{\textasciitilde}bʉ'ʉ́-khʉ	{\textasciitilde}sa=da'rá-{\textasciitilde}da	yoá-{\textasciitilde}kha'ba-ha	{\textasciitilde}sá	{\textasciitilde}dí-a \\
     come-\textsc{imp}	2\textsc{sg-add}	1\textsc{pl.excl.poss}=work\textsc{-pl}	do\textsc{-recp-vis.ipfv.}1	1\textsc{pl.excl}	say-\textsc{assert.pfv}\\
\glt ‘“You come too, to help with our work,” (the servants) said.’
\glt ‘“Venha você também ajudar o nosso trabalho,” falaram.'
\z 

\ea “hai” ni. \\[.3em]
\gll hái	{\textasciitilde}dí \\
     \textsc{intj:}agree	say\\
\glt ‘“All right,” said (\textit{Yuhpi Diani}).'
\glt ‘“Está bom,” disse (\textit{Yuhpi Diani}).'
\z 

\ea ñaina dohamʉaa ti(na)tha tirore. \\[.3em]
\gll {\textasciitilde}yá{\textasciitilde}-ida	dohá-{\textasciitilde}bʉa-a	tí-ta	tí-ró-ré \\
     bad\textsc{-nmlz.pl}	neg.curse-high-\textsc{assert.pfv}	\textsc{anph-emph}	\textsc{anph-sg-obj}\\
\glt ‘(The servants) sent up a curse on him (\textit{Yuhpi Diani}).’
\glt ‘(Os criados) amaldiçoaram o (\textit{Yuhpi Diani}).’
\z

\newpage 
\ea “hai” ni tirore sĩ'ariphĩ sĩ'a. \\[.3em]
\gll hái	{\textasciitilde}dí	tí-ró-ré	{\textasciitilde}si'á-ri-{\textasciitilde}phi	{\textasciitilde}si'á-a \\
     \textsc{intj:}agree	say	\textsc{anph-sg-obj}	set.fire.to\textsc{-nmlz-clf:}bladelike	set.fire.to-\textsc{assert.pfv}\\
\glt ‘“All right,” he said (and they) lit up a torch (to light the way) for him.’
\glt ‘“Está bom,” ele disse (e) acenderam uma tocha para (clarear o caminho) para ele.’
\z

\ea sĩ'a duhiato pakhuoriro … \\[.3em]
\gll {\textasciitilde}si'á	duhí-(wa')a-to	pá-khui-o-ri-ro \\
     set.fire.to	descend-go-\textsc{nmlz.loc/evnt}	\textsc{alt}-afraid\textsc{-caus-nmlz-sg}\\
\glt ‘With the light, the poor terrified guy (\textit{Yuhpi Diani}) started down ...’
\glt ‘Com a luz, o coitadinho assustado foi descendo ...’
\z 

\ea pʉ! “mʉ'ʉ ... yabari ... dahchomahkamahkari tañore dʉhte taga mʉ'ʉbahsi” nia. \\[.3em]
\gll pʉ́	{\textasciitilde}bʉ'ʉ́	yabá-rí	dachó{\textasciitilde}baka-{\textasciitilde}baka-ri	{\textasciitilde}tayó-re	dʉte-ta-ga {\textasciitilde}bʉ'ʉ́-básí	{\textasciitilde}dí-a\\
     \textsc{dist}	2\textsc{sg}	\textsc{q:}what/how-\textsc{q.pfv}	middle/center-origin\textsc{-nmlz}	beam\textsc{-obj}	chop-come-\textsc{imp} 2\textsc{sg-emph}	say-\textsc{assert.pfv}\\
\glt ‘Coming way down! “You [how was it?] you yourself cut the central beam (of the barricade),” (the servants) said.'
\glt ‘Descendo tudo! “Você [como foi?] você mesmo venha cortar o travessão do meio (da barricada),” disseram (os criados).’
\z 

\ea “hai” ni. sĩ'aborataa.  \\[.3em]
\gll hái	{\textasciitilde}dí	{\textasciitilde}si'á-bora-ta-a \\
     \textsc{intj:}agree	say	set.fire.to-slide/fall-come-\textsc{assert.pfv}\\
\glt ‘“OK” (\textit{Yuhpi Diani}) said (and) came sliding down.’
\glt ‘“Está bom,” respondeu (\textit{Yuhpi Diani}), e veio descendo escorregando.’
\z 

\ea to sĩ'aphĩri sĩ'a, yoariphĩ ña'a, to dʉhteka'achʉwaro, ñaina ña'atu'sʉa tirore. \\[.3em]
\gll to={\textasciitilde}si'á-{\textasciitilde}phi-ri	{\textasciitilde}si'á	yoá-rí-{\textasciitilde}phí	{\textasciitilde}ya'á to=dʉté-ká'á-chʉ́-wáró	{\textasciitilde}yá-{\textasciitilde}ida	{\textasciitilde}ya'á-thu'sʉ-a	tí-ró-ré\\
     3\textsc{sg.poss}=set.fire.to-\textsc{clf:}bladelike\textsc{-pl}	set.fire.to	long\textsc{-nmlz-clf:}bladelike	grab 3\textsc{sg.poss}=chop-do.moving-\textsc{sw.ref-emph}	bad\textsc{-nmlz.pl}	grab-finish-\textsc{assert.pfv}	\textsc{anph-sg-obj}\\
\glt ‘With his torch burning, he grabbed his machete (and) when he started to chop (the beam), the warriors (\textit{Biari}) captured him.’
\glt ‘Com a tocha acesa, pegou o terçado (e) quando começou a cortar (o travessão), os guerreiros (\textit{Biari}) o capturaram.’
\z 

 
\ea “kueeee! yʉ kha'makã, yʉ’ʉ́re ña'atu'sʉra ñaina.”  \\[.3em]
\gll kueeeee!	yʉ={\textasciitilde}kha'bá-{\textasciitilde}ka	yʉ’ʉ́-ré	{\textasciitilde}ya'á-thu'sʉ-ra	{\textasciitilde}yá-{\textasciitilde}ida \\
     \textsc{intj:}No!	1\textsc{sg.poss}=bring.together-\textsc{dim}	1\textsc{sg-obj}	grab-finish-\textsc{vis.ipfv.}2/3	bad\textsc{-nmlz.pl}\\
\glt ‘“No! Servants, the warriors have captured me!”’
\glt ‘“Não! Criados, os guerreiros me pegaram!”’
\z 

\ea “khero, mʉhsakhʉ bʉea nataga” nia. \\[.3em]
\gll khé-ro	{\textasciitilde}bʉsá-khʉ	bʉé-a	{\textasciitilde}dá-ta-ga	{\textasciitilde}dí-a \\
     fast\textsc{-sg}	2\textsc{pl-add}	arrow\textsc{-pl}	get-come-\textsc{imp}	say-\textsc{assert.pfv}\\
\glt ‘“Quickly, you get your arrows too!” (\textit{Yuhpi Diani}) cried.’
\glt ‘“Rápido, vocês pegam as flechas também!” disse (\textit{Yuhpi Diani}).’
\z 

\ea tina, ñamire, yo'o da'ramahkatinikaa. \\[.3em]
\gll tí-{\textasciitilde}da	{\textasciitilde}yabí-ré	yo'ó	da'rá-{\textasciitilde}baka-{\textasciitilde}tidi-ka('a)-a \\
     \textsc{anph-pl}	night\textsc{-obj}	in.contrast	work-look.for-wander.around-do.moving-\textsc{assert.pfv}\\
\glt ‘(But) in the dark, they just ran around like this looking for (things).’
\glt ‘(Mas), no escuro, eles só andaram assim para lá e para cá procurando.’
\z

 
\ea tirore ña'atu'sʉrasi, ña'adi'okãhãa. \\[.3em]
\gll tí-ró-ré	{\textasciitilde}ya'á-thu'sʉ́ra-{\textasciitilde}si(di)	{\textasciitilde}ya'á-di'o-{\textasciitilde}ka-{\textasciitilde}ha-a \\
     \textsc{anph-sg-obj}	grab-finish-do.now	grab-restrain-\textsc{dim-compl-assert.pfv}\\
\newpage    
\glt ‘(The warriors) had already captured him (\textit{Yuhpi Diani}), (and he) couldn't move.’
\glt ‘(Os guerreiros) já tinham capturado \textit{Yuhpi Diani}, (e ele) não conseguia se mexer.’
\z 

\ea ña'amahare kãku õi su'su. \\[.3em]
\gll {\textasciitilde}ya'á-{\textasciitilde}báhá-ré	{\textasciitilde}kákú	{\textasciitilde}ó-i	su'sú \\
     grab-\textsc{mov}.upward-\textsc{vis.pfv.}2/3	throw.on.ground	\textsc{deic.prox-loc.vis}	embrace\\
\glt ‘(They) threw him down on the ground, holding him here.’
\glt ‘Derrubaram jogando no chão, segurando aqui assim.’
\z 
\ea ti(na) to a’ri tañobaro topʉta du'upayoa to dʉhsore. \\[.3em]
\gll tí	to	a’rí	{\textasciitilde}tayó=ba'ro	tó-pʉ-ta	du'ú-páyó-á	to=dʉsó-re \\
     \textsc{anph}	\textsc{def}	\textsc{dem.prox}	beam=\textsc{clf:}kind	\textsc{def-loc-emph}	leave-put.on.top-\textsc{assert.pfv}	3\textsc{sg.poss}=thigh\textsc{-obj}\\
\glt ‘On a beam just like this one here, they (\textit{Dian}i's warriors) left his (\textit{Yuhpi Diani}'s) leg.’
\glt ‘Eles (os guerreiros do \textit{Diani}), num travessão tipo esse mesmo, deixaram a coxa dele (do \textit{Yuhpi Diani}) bem em cima.’
\z 

\ea yo'o tinapʉ (hi')na khomakhʉ, yaba hiri himarero? \\[.3em]
\gll yo'ó	tí-{\textasciitilde}da-pʉ	{\textasciitilde}(hi')da	{\textasciitilde}khobá-khʉ	yabá	hí-ri	hi-{\textasciitilde}bare-ro \\
     in.contrast	\textsc{anph-pl-loc}	\textsc{emph}	ax-\textsc{add}	\textsc{q:}what/how	\textsc{cop-nmlz}	\textsc{cop-ipfv.epis-sg}\\
\glt ‘[But those guys then had axes too — what were they?]’
\glt ‘[Mas naquela época eles também tinham machado — como é que era?]’
\z 

\largerpage[2]
\ea bookhoma! ʉ̃hʉ, ti khoma sioripha tinapʉre? \\[.3em]
\gll bóó-{\textasciitilde}khómá	ʉ̃hʉ	tí	{\textasciitilde}khóbá	sió-ri-pha	tí-{\textasciitilde}da-pʉ-re \\
     stone-axe	\textsc{intj:}yes	\textsc{anph}	ax	sharp\textsc{-nmlz-spec}	\textsc{anph-pl-loc-obj}\\
\glt ‘[\textit{Bookhoma}! Yes ... do you suppose to them that (kind of) axe{\footnotemark} was sharp?]’
\footnotetext{Teresinha is referring to axes with stone heads, still used in the region in the early twentieth century (\citealt[171-172]{Koch-Grünberg1995}).}
\newpage 
\glt ‘[\textit{Bookhoma}! Sim ... será que para eles aquele (tipo de) machado era afiado?]’
\z 

 
\ea mari (hi')na khoma hieramarero, tãphĩ, tã hia nimarero. \\[.3em]
\gll {\textasciitilde}barí	{\textasciitilde}(hi')da	{\textasciitilde}khobá	hi-éra-{\textasciitilde}bare-ro	{\textasciitilde}tá-{\textasciitilde}phi	{\textasciitilde}tá	hí-a	{\textasciitilde}dí-{\textasciitilde}bare-ro \\
     1\textsc{pl.incl}	\textsc{emph}	ax	\textsc{cop-neg-rem.ipfv-sg}	rock-\textsc{clf.}bladelike	rock	\textsc{cop-assert.pfv}	\textsc{cop-rem.ipfv-sg}\\
\glt ‘[(For) us (that old kind) aren't axes, (just) rocks, they were rocks.]’
\glt ‘[(Para) nós mesmo (aquele tipo) não seria machado, (só) pedra, era pedra.]’
\z 

\ea ʉ̃hʉ. \\[.3em]
\gll ʉ̃hʉ \\
     \textsc{intj:}yes\\
\glt (Joselito) ‘[Yes.]’
\glt (Joselito) ‘[Sim.]’
\z

\ea dʉhtetaroka, dohkapayoa te to ka'apʉ khakhasario ti ninopʉ. \\[.3em]
\gll dʉté-tá-dóká	doká-payo-a	té	to	ka'á-pʉ́ khá-khásá-ríó	ti={\textasciitilde}dí-ro-pʉ\\
     chop-separate-\textsc{dist}	throw-put.on.top-\textsc{assert.pfv}	until	\textsc{rem}	beside\textsc{-loc} hawk-platform-\textsc{clf}:flat	3\textsc{pl.poss}=say\textsc{-sg}\textsc{-loc}\\
\glt ‘(They) chopped off (his leg and) threw it all the way up near that (thing) they call the hawk's nest platform.’
\glt ‘Cortaram fora (a perna) e jogaram lá em cima perto (daquilo) que chamavam de ninho de gavião.’
\z 

\largerpage
\ea dohkapayoroka, “maa! bi'oha hi'na ʉʉʉʉʉʉʉ ... wiiiii!” \\[.3em]
\gll doká-payo-doka{\footnotemark}	{\textasciitilde}báa	bi'ó-há	{\textasciitilde}hí'da	“ʉʉʉʉʉʉʉ...wiiiii!” \\
     throw-put.on.top-\textsc{dist}	\textsc{intj}:done/ready!	successful-\textsc{vis.ipfv.}1	\textsc{emph}	\textsc{ontp:}cries...whistles\\
\footnotetext{Note the two instances of the root \textit{doka} ‘throw’, the first meaning literally ‘to throw’, and the second indicating “distal" motion.}
\glt ‘Throwing his leg way up there (they cried) “There! We've done it! \textit{ʉʉʉʉʉ ... wiiiiii}”' (cries and whistles)
\newpage
\glt ‘Jogando lá em cima (gritaram) “Acabou! Conseguimos mesmo! \textit{ʉʉʉʉʉ ... wiiiiii}”' (gritos e assobios)
\z 

\ea“wahpʉro, mahkʉnoñakãre, tirobahsitha, sã ba'ʉre to ã yoari bu'iri!” \\[.3em]
\gll wapʉ́-ro	{\textasciitilde}bakʉ́-ro-{\textasciitilde}ya-{\textasciitilde}ka-re	tí-ró-basi-ta	{\textasciitilde}sa=ba'ʉ́-re to={\textasciitilde}a=yóá-rí	bu'í-ri \\
     enemy\textsc{-sg}	son\textsc{-sg}-bad\textsc{-dim-obj}	\textsc{anph-sg-emph-emph}	1\textsc{pl.excl.poss}=younger.brother\textsc{-obj} 3\textsc{sg.poss}=so-do\textsc{-nmlz}	cause\textsc{-nmlz}\\
\glt ‘“Enemy, evil son, (we got) him for what he himself did to our brother!”’
\glt ‘“Inimigo, filho malvado, (pegamos) ele mesmo pelo o que fez ao nosso irmão!”’
\z 

\ea “bu'iriti wãhanona mʉ'ʉ ñaka si'ro.” \\[.3em]
\gll bu'í-rí-tí	{\textasciitilde}wahá-ro-({\textasciitilde}hi')da	{\textasciitilde}bʉ'ʉ́	{\textasciitilde}yá-ká	sí'ro \\
     cause\textsc{-nmlz-attrib}	kill\textsc{-sg-emph}	2\textsc{sg}	bad-\textsc{assert.ipfv}	bastard\\
\glt ‘“Guilty one! Murderer! You evil bastard!”’
\glt ‘“Culpado! Assassino! Você malvado!”’
\z 

\ea samu te yohaa te õpʉ mahasʉ̃, sʉ̃a.   \\[.3em]
\gll {\textasciitilde}sabú	té	yohá-a	té	{\textasciitilde}ó-pʉ́	{\textasciitilde}bahá-{\textasciitilde}sʉ	{\textasciitilde}sʉ́-a \\
     embark.in.canoe	until	go.upriver-\textsc{assert.pfv}	until	\textsc{deic.prox-loc}	go.uphill-arrive.\textsc{trns}	arrive.\textsc{trns}-\textsc{assert.pfv}\\
\glt ‘(They) got into their canoes (and) came upriver here, went up (the hill and) arrived home.’
\glt ‘Embarcaram nas canoas (e) subiram para cá, subiram até em cima (e) chegaram em casa.’
\z 

\ea “ne, yʉ pho'na bi’ori.”  \\[.3em]
\gll {\textasciitilde}dé	yʉ={\textasciitilde}pho'dá	bi'ó-ri \\
     so	1\textsc{sg.poss}=children	be.successful-\textsc{q.pfv}\\
\glt ‘(Their father \textit{Dianumia Yairo} asked): “Were you successful, sons?”’ 
\glt ‘(O pai deles \textit{Dianumia Yairo} perguntou): “Conseguiram, meus filhos?”’ 
\z 

\newpage 
\ea “bi’oi yʉ phʉkʉ,” nia. \\[.3em]
\gll bi’ó-i	yʉ=phʉkʉ́	{\textasciitilde}dí-a \\
     successful-\textsc{vis.pfv.}1	1\textsc{sg.poss}=father	say-\textsc{assert.pfv}\\
\glt ‘“We were, father,” (\textit{Diani}) responded.’
\glt ‘“Conseguimos, pai,” falou (\textit{Diani}).’
\z

\ea “bi’oi, to yabu'iri sã koirore to ñano yoari bu'iri,” nia. \\[.3em]
\gll bi’ó-i	to=yá=bu'i-ri	{\textasciitilde}sa=ko-iro-re to={\textasciitilde}yá-ró	yoá-rí	bu'í-ri	{\textasciitilde}dí-a\\
     successful-\textsc{vis.pfv.}1	3\textsc{sg.poss}=\textsc{poss}=cause\textsc{-nmlz}	1\textsc{pl.excl.poss}=relative\textsc{-nmlz.sg-obj} 3\textsc{sg.poss}=bad\textsc{-sg}	do\textsc{-nmlz}	cause\textsc{-nmlz}	say-\textsc{assert.pfv}\\
\glt ‘“We did it, (killed) the one responsible for the evil he did to our relative,” (\textit{Diani}) said.’
\glt ‘“Conseguimos (matar) o culpado, aquele que fez tanto mal ao nosso parente,” disse (\textit{Diani}).’
\z 

\ea “tina ti ñano sã koirore yoare.”  \\[.3em]
\gll tí-{\textasciitilde}da	ti={\textasciitilde}yá-ró	{\textasciitilde}sa=kó-iro-re	yoá-re \\
     \textsc{anph-pl}	\textsc{anph}=bad\textsc{-sg}	1\textsc{pl.excl.poss}=relative\textsc{-nmlz.sg-obj}	do-\textsc{vis.pfv.}2/3\\
\glt ‘“Those evil ones who did that to our brother.”’
\glt ‘“Aqueles malvados que fizeram mal ao nosso irmão.”’
\z 

\ea “tiro yoerarirota noano yʉ'dʉbohkari,” nia. \\[.3em]
\gll tí-ró	yoa-éra-ri-ro-ta	{\textasciitilde}dóá-ró	yʉ'dʉ́-boka-ri	{\textasciitilde}dí-a \\
     \textsc{anph-sg}	do\textsc{-neg-nmlz-sg-emph}	good\textsc{-sg}	\textsc{intens}-find\textsc{-nmlz}	say-\textsc{assert.pfv}\\
\glt ‘“He couldn't expect anything good to happen,” (\textit{Diani}) said.’
\glt ‘“Ele não podia esperar coisa boa,” disse (\textit{Diani}).’
\z 

\newpage
\ea “noana, yʉ pho'na mʉhsa yʉ’ʉ́ ba'arore, mʉhsa ne thʉ'oduerara,” nia. \\[.3em]
\gll {\textasciitilde}dóádá	yʉ={\textasciitilde}pho'dá	{\textasciitilde}bʉsá	yʉ’ʉ́	ba'á-ró-ré {\textasciitilde}bʉsá	{\textasciitilde}dé	thʉ'o-dua-éra-ra	{\textasciitilde}dí-a\\
     very.well	1\textsc{sg.poss}=children	2\textsc{pl}	1\textsc{sg}	after\textsc{-sg-obj} 2\textsc{pl}	\textsc{neg}	hear\textsc{-des-neg-vis.ipfv.}2/3	say-\textsc{assert.pfv}\\
\glt ‘“Very well, you, my children who come after me (of the next generation) just won't listen/obey,” (\textit{Dianumia Yairo}) said.’
\glt ‘“Bem, vocês, meus filhos que vem depois de mim (da outra geração) não querem mais ouvir/obedecer,” disse (\textit{Dianumia Yairo}).’
\z 

\ea “õse yoanakã nichʉ, mʉhsa thʉ'otina hierara,” nia.  \\[.3em]
\gll {\textasciitilde}óse	yoá-{\textasciitilde}da-{\textasciitilde}ka	{\textasciitilde}dí-chʉ́	{\textasciitilde}bʉsá	thʉ'ó-ti-{\textasciitilde}da	hi-éra-ra	{\textasciitilde}dí-a \\
     like.this	do\textsc{-pl-dim}	say\textsc{-sw.ref}	2\textsc{pl}	hear\textsc{-attrib-pl}	\textsc{cop-neg-vis.ipfv.}2/3	say-\textsc{assert.pfv}\\
\glt ‘“Just like now, you ignore (me, you aren't ones who listen),” said (\textit{Dianumia Yairo}).’
\glt ‘“Como agora, vocês ignoram (não são gente que escuta),” disse (\textit{Dianumia Yairo}).’
\z 

\ea “noanokã” ni, to suabu'sa to doharire tinare phãawe. \\[.3em]
\gll {\textasciitilde}dóá-ró-{\textasciitilde}ká	{\textasciitilde}dí	to=súá-bú'sá	to=dohá-ri-re	tí-{\textasciitilde}da-re	{\textasciitilde}phaá-wé \\
     good\textsc{-sg-dim}	say	3\textsc{sg.poss}=angry-adornment	3\textsc{sg.poss}=neg.curse\textsc{-nmlz-obj}	\textsc{anph-pl-obj}	remove.curse/blessing-\textsc{mov.}through\\
\glt ‘“So be it,” he said (and) took back his war adornments (and) his blessing for courage.’
\glt ‘“Bem,” disse (e) tirou deles os adornos de bravura e benzimento de coragem.’
\z 

\newpage 
\ea tinare mʉ'nophuti, tina mipʉse thʉ'otua, tina ne thʉ'omahsieraphati. \\[.3em]
\gll tí-{\textasciitilde}da-re	{\textasciitilde}bʉ'dó-phútí	tí-{\textasciitilde}da	{\textasciitilde}bí-pʉ-se	thʉ'ó-thu-a tí-{\textasciitilde}da	{\textasciitilde}dé	thʉ'ó-{\textasciitilde}basi-era-pha-ati{\footnotemark}\\
     \textsc{anph-pl-obj}	tobacco-blow	\textsc{anph-pl}	now\textsc{-loc-clf:}similar	hear-think-\textsc{assert.pfv} \textsc{anph-pl}	\textsc{neg}	hear-know\textsc{-neg-time-ipfv}\\
\glt ‘Blowing smoke on them, they became the way they are now (peaceful), no longer in that violent state.’
\glt ‘Soprando fumaça neles, se tornaram como hoje (calmos), não mais naquele estado de violência.’
\footnotetext{The construction \textit{thʉ'ómahsieraphaati} indicates a ‘state of violence’.}
\z

 
\ea “ã yoa ñamichakã bo'rearoi, khʉ naharo, mʉhsa namosãnumia” nia.\\[.3em]
\gll {\textasciitilde}a=yóá	{\textasciitilde}yabíchá-{\textasciitilde}ká	bo'ré-(wa')a-ro-i	khʉ́{\footnotemark}	{\textasciitilde}dá-haro {\textasciitilde}bʉsa={\textasciitilde}dabó-{\textasciitilde}sadubia	{\textasciitilde}dí-a\\
     so=do	tomorrow\textsc{-dim}	lighten-go\textsc{-sg-loc.vis}	manioc	get-\textsc{imp.3} 2\textsc{pl.poss}=wife\textsc{-pl.f}	say-\textsc{assert.pfv}\\
\glt ‘“So, the day after tomorrow, send your wives to get manioc,” he ordered.’
\glt ‘“Então, depois de amanhã, mande as suas esposas tirarem maniva,” ele disse.’
\footnotetext{Here we see an example of an unmarked object, \textit{khʉ} ‘maniva/manioc root’, that is nonreferential and therefore does not require use of the objective case marker \textit{-re}. In the following sentence, the same unmarked object is phonologically incorporated with the verb \textit{na} ‘get/pick’. A rare example of an unmarked pronominal object occurs in line 144, where the pronoun \textit{sã} '1\textsc{pl.excl'} is nonreferential and thus is also unmarked by \textit{-re}.}
\z 

\ea tina bahsabahtoa taro tina khʉna, da're.\\[.3em]
\gll tí-{\textasciitilde}da	basá-bátóá-ró	tá-ro	tí-{\textasciitilde}da	khʉ́-{\textasciitilde}da	da'ré \\
     \textsc{anph-pl}	sing/dance-last\textsc{-sg}	come\textsc{-sg}	\textsc{anph-pl}	manioc.root-get	prepare\\
\glt ‘For the last ceremony (celebrating their victory in war), they got manioc (and) prepared (caxiri).'{\footnotemark}
\glt ‘Para a última cerimonia (festejando a vitória na guerra), eles foram pegar maniva e prepararam (caxiri).’
\footnotetext{“Caxiri" (also referred to as “chicha" in some of the chapters in this volume) is a type of drink, usually made from toasted manioc flatbread diluted in water and left to ferment for a couple of days.}
\z 

\newpage 
\ea pharinʉmʉ tina bagapo durukua, nahubahsapore.\\[.3em]
\gll pha-rí-{\textasciitilde}dʉbʉ	tí-{\textasciitilde}da	bagápó	dú-rúkú-á	{\textasciitilde}dahú-básápó-ré \\
     time\textsc{-nmlz}-day	\textsc{anph-pl}	cerimonial.dance/chant	speak-stand-\textsc{assert.pfv}	flatbread-dance\textsc{-obj}\\
\glt ‘[On those celebration days, they chanted ceremonial dances (like) the flatbread dance.]’
\glt ‘[Nesses dias de festa, eles cantavam as danças (como) a dança do beijú.]’
\z 

 
\ea ti hika bahsabʉhkʉ nahubahsa, sa'waroa, [yaba] miniawahkʉ, wamo thi’biri.\\[.3em]
\gll tí	hí-ka	basá-bʉ́kʉ́	{\textasciitilde}dahú-básá	sa'wáró-á yabá	{\textasciitilde}bidíawakʉ	{\textasciitilde}wabó-thí'bi-ri\\
     \textsc{anph}	\textsc{cop-assert.ipfv}	sing/dance-ancestor	flatbread-sing/dance	brown.lizard\textsc{-pl} \textsc{q}:what/how	fruit.dance	arm/hand-intertwine\textsc{-nmlz}\\
\glt ‘[These are traditional (origin) dances: flatbread dance, lizard dance — what else? — fruit dance, peace (holding hands) dance.]’
\glt ‘[As danças originais são a dança do beijú, dança do calango marrom — que mais? — dança dos frutos, dança da paz (mãos dadas).]’
\z 

\ea tina bahsañopha'ño phʉanʉmʉtha, hi('na): \\[.3em]
\gll tí-{\textasciitilde}da	basá-{\textasciitilde}yo-{\textasciitilde}pha'yo	phʉá-{\textasciitilde}dʉbʉ-ta	{\textasciitilde}hí('da) \\
     \textsc{anph-pl}	sing/dance-show-complete	two-day-\textsc{emph}	\textsc{emph}\\
\glt ‘They performed all the dances for two days (and then \textit{Dianumia Yairo} said):'
\glt ‘Eles apresentaram todas as danças durante dois dias (e então \textit{Dianumia Yairo} disse):'
\z 

\ea “yʉ’ʉ wa'awa'aika hi'na” nia, mipʉ to duhitore. \\[.3em]
\gll yʉ’ʉ́	wa'á-wa'a-i-ka	{\textasciitilde}hí'da	{\textasciitilde}dí-a	{\textasciitilde}bí-pʉ́	to=duhí{\footnotemark}-a-to-re \\
     1\textsc{sg}	go-go-\textsc{m-predict}	\textsc{emph}	say-\textsc{assert.pfv}	now\textsc{-loc}	3\textsc{sg.poss}=sit\textsc{-pl-nmlz.loc/evnt-obj}\\
\glt ‘“I'm going away now.” [To the place he still is sitting.]’
\glt ‘“Agora eu vou indo.” [Ao lugar onde ele ainda está sentado.]’
\footnotetext{The verb \textit{duhi} ‘sit’ is used in the sense of ‘being’. This root is analyzed as a possible lexical origin of the Kotiria copula verb \textit{hi} in \cite{Stenzelforthcoming}.}
\z

\newpage 
\ea “wa'aika yʉ’ʉ, yʉ pho'na, mʉhsare ne khũsi” nia. \\[.3em]
\gll wa'á-i-ka	yʉ’ʉ́	yʉ={\textasciitilde}pho'dá	{\textasciitilde}bʉsá-ré	{\textasciitilde}dé	{\textasciitilde}khú-sí	{\textasciitilde}dí-a \\
     go-\textsc{m-predict}	1\textsc{sg}	1\textsc{sg.poss}=children	2\textsc{pl-obj}	\textsc{neg}	leave/place\textsc{-neg.irr}	say-\textsc{assert.pfv}\\
      
\glt ‘“I'm going (and) I'm not leaving you anything, my children,” \textit{Dianumia Yairo}) said.’
\glt ‘“Vou mesmo (e) não vou deixar nada para vocês, meus filhos,” disse (\textit{Dianumia Yairo}).’
\z

\ea “ne a’ri wĩho yaichʉawĩho, warimahsawĩho, khʉ'mawĩho, hʉka phiri ne mʉhsare khũsi” nia. \\[.3em]
\gll {\textasciitilde}dé	a’rí	{\textasciitilde}wihó{\footnotemark}	yaí-chʉ́-á-{\textasciitilde}wíhó	wári-{\textasciitilde}basa-{\textasciitilde}wiho {\textasciitilde}khʉ'bá-{\textasciitilde}wiho	hʉ́ká	phí-ri	{\textasciitilde}dé	{\textasciitilde}bʉsá-ré	{\textasciitilde}khú-sí	{\textasciitilde}dí-a\\
     \textsc{neg}	\textsc{dem.prox}	halluc.powder	jaguar-eat\textsc{-pl}-halluc.powder	kidnap-people-halluc.powder summer-halluc.powder	hungry	big\textsc{-nmlz}	\textsc{neg}	2\textsc{pl-obj}	leave/place\textsc{-neg.irr}	say-\textsc{assert.pfv}\\
\glt ‘“Not my hallucinogenic powder, nor the one that can transform you into jaguars to eat people, nor the one to kidnap people (to become invisible), nor the one that brings on summer, nor will I leave the one that causes hunger,” (\textit{Dianumia Yairo}) said.’
\glt ‘“Nem meu paricá, meu paricá de virar onça e comer gente, paricá de roubar gente (de ficar invisível), paricá de verão, nem vou deixar o de causar fome,” dizia (\textit{Dianumia Yairo}).’
\footnotetext{Note the various tones on the nominal root \textit{wiho} ‘hallucinogenic powder’. As an independent root, it has a LH melody, but when incorporated into compound words, it receives the final tone of the root to its left through processes of tonal spread.} 
\z 

\ea “hiphiti a’ri phinitare naita yʉ’ʉ" nia.  \\[.3em]
\gll híphiti	a’rí	{\textasciitilde}phídi-ta-re	{\textasciitilde}dá-i-ta	yʉ’ʉ́	{\textasciitilde}dí-a \\
     everything	\textsc{dem.prox}	right.here-\textsc{emph-obj}	get-\textsc{m-intent}	1\textsc{sg}	say-\textsc{assert.pfv}\\
\glt ‘“All of these things here I'm taking away,” (\textit{Dianumia Yairo}) said.’
\glt ‘“Vou levar todas essas coisas aqui,” disse (\textit{Dianumia Yairo}).’
\z 
\ea “mʉsa thʉ'omasiduerara pa thurupʉre." \\[.3em]
\gll {\textasciitilde}bʉsá	thʉ'ó-{\textasciitilde}basi-dua-era-ra	pá-thuru-pʉ-re \\
     2\textsc{pl}	hear-know-\textsc{des-neg-vis.ipfv.}2/3	\textsc{alt}-times\textsc{-loc-obj}\\
      
\glt ‘“In the future, you won't want (you won’t know \textit{how}) to use them appropriately.”’
\glt ‘“Vocês no futuro não vão querer (não vão saber \textit{como}) usar bem.”’
\z 

\ea “do'se mʉhsa yoaro, yoakãduara.” \\[.3em]
\gll do'sé	{\textasciitilde}bʉsá	yoá-ro	yoá-{\textasciitilde}ka-dua-ra \\
     \textsc{q:}how	2\textsc{pl}	do\textsc{-sg}	do-\textsc{dim-des-vis.ipfv.}2/3\\
\glt ‘“(Because) you will always just do what you want.”’
\glt ‘“(Porque) vocês sempre fazem como querem.”’
\z 

\ea mari suamʉa, wĩho wĩhi marire khãrirore ʉʉʉ ... yaichʉ khoakana. \\[.3em]
\gll {\textasciitilde}barí	súá-{\textasciitilde}bʉ́-á	{\textasciitilde}wihó	{\textasciitilde}wihí	{\textasciitilde}barí-re	{\textasciitilde}khári-ro-re	ʉʉʉ  yaí-chʉ	khoá-ka-{\textasciitilde}(hi')da \\
     1\textsc{pl.incl}	angry-man/person\textsc{-pl}	halluc.powder	sniff	1\textsc{pl.incl-obj}	offend\textsc{-sg-obj}	\textsc{ontp:}hurting jaguar-eat	finish-\textsc{assert.ipfv-emph}\\
\glt ‘[When we get angry, we sniff powder to turn into a jaguar and devour (fight with, or kill) the one who has offended us.]’
\glt ‘[Quando ficamos com raiva, cheiramos paricá para nós nos transformar em onça e devorar (brigar, ser capaz de matar) aquele que nos ofendeu.]’
\z 

\ea ã hia hima mari wĩho, kotiria wĩho.  \\[.3em]
\gll {\textasciitilde}a=hí-a	hí-{\textasciitilde}ba	{\textasciitilde}bari={\textasciitilde}wiho	kótiria	{\textasciitilde}wihó \\
     so=\textsc{cop-assert.pfv}	\textsc{cop-rem.ipfv}	1\textsc{pl.incl.poss}=halluc.powder	Kotiria	halluc.powder\\
\glt ‘[That's how our powder used to be, Kotiria powder.]’
\glt ‘[Assim era o nosso paricá, dos Kotiria.]’
\z 

\ea yo'omeheta wĩhi wahpʉatia hieraa a’ri.  \\[.3em]
\gll yo'ó-{\textasciitilde}beheta	{\textasciitilde}wihí	wapʉ́-ati-a	hi-éra-a	a’rí \\
     in.contrast\textsc{-neg.intens}	sniff	do.for.long.time-\textsc{ipfv-pl}	\textsc{cop-neg-assert.pfv}	\textsc{dem.prox}\\
\glt ‘[But people weren't supposed to keep sniffing it all the time.]’ 
\glt ‘[Mas não era para ficar cheirando muito isso.]’
\z

 
\ea a’ri do'beba'roñoa. \\[.3em]
\gll a’rí	do'bé-ba'ro-{\textasciitilde}yo-a \\
     \textsc{dem.prox}	paint.with.finger-\textsc{clf:}kind-show-\textsc{assert.pfv}\\
\glt ‘[It was just the kind for face-painting (which already has an effect).]’
\glt ‘[Era só para usar pintando (que já fazia efeito).]’
\z 

\ea tiro pharipʉta, tipʉre khʉaa. \\[.3em]
\gll tí-ró	pharí-pʉ-ta	tí-pʉ-re	khʉá-a \\
     \textsc{anph-sg}	form-\textsc{clf:}basket-\textsc{emph}	\textsc{anph}-\textsc{clf:}basket\textsc{-obj}	have-\textsc{assert.pfv}\\
\glt ‘The whole basket (with all the materials), (he) had that basket.’
\glt ‘O aturá inteiro (com todo o material), (ele) tinha aquele cesto.’
\z 

\ea to yaichʉre thuaka'a, to bahtichʉre sĩosuahã. \\[.3em]
\gll to=yaíchʉ-re	thuá-ka'a-a	to=batíchʉ-re	{\textasciitilde}siósua-{\textasciitilde}ha \\
     3\textsc{sg.poss}=shaman.staff\textsc{-obj}	lean.on-do.moving-\textsc{assert.pfv}	3\textsc{sg.poss}=shield\textsc{-obj}	place.around.arm-\textsc{compl}\\
\glt ‘(He) took up his staff (and) put his shield on his arm.’
\glt ‘Segurou no bastão (e) enfiou o escudo no braço.’
\z 

\ea to diero mʉta, toi hia to diero.  \\[.3em]
\gll to=dié-ró	{\textasciitilde}bʉ́tá	tó-i	hí-a	to=dié-ró\\
     3\textsc{sg.poss}=dog\textsc{-sg}	advance	\textsc{rem-loc.vis}	\textsc{cop-assert.pfv}	3\textsc{sg.poss}=dog\textsc{-sg}\\
\glt ‘His dog went up ahead, up there.’
\glt ‘O cachorrro ia lá, na frente.’
\z 

\ea tina hia yaiya wʉ'ʉ.  \\[.3em]
\gll tí-{\textasciitilde}da	hí-a	yaí-yá	wʉ'ʉ́ \\
     \textsc{anph-pl}	\textsc{cop-assert.pfv}	jaguar\textsc{-pl}	casa\\
\glt ‘[They were the house jaguars (guards).]’
\glt ‘[Eles eram as onças da casa (guardiões).]’
\z 

\ea tina hira mipʉre tore khuaina, sã hiromahanone. \\[.3em]
\gll tí-{\textasciitilde}da	hí-ra	{\textasciitilde}bí-pʉ́-ré	tó-re	khuá-{\textasciitilde}ida	{\textasciitilde}sa=hí-ro-{\textasciitilde}baha-ro-re \\
     \textsc{anph-pl}	\textsc{cop-vis.ipfv.}2/3	now\textsc{-loc}\textsc{-obj}	\textsc{rem-obj}	dangerous\textsc{-nmlz.pl}	1\textsc{pl.excl.poss=cop-sg}-go.uphill\textsc{-sg-obj}\\
\glt ‘[They're the ones (jaguars) that nowadays are a danger to us, there where we live (in Matapi).]’
\glt ‘[São esses (as onças) que hoje são perigosos, perto de onde nós moramos (em Matapi).]’
\z 

\ea wa'ato pʉʉʉʉ, to namonokoro nʉnʉti. \\[.3em]
\gll wa'á-to	pʉ́	to={\textasciitilde}dabó-ro-koro	{\textasciitilde}dʉdʉ́-atí \\
     go-\textsc{nmlz.loc/evnt}	\textsc{dist}	3\textsc{sg.poss}=wife\textsc{-sg-f.rsp}	follow-\textsc{ipfv} \\
\glt ‘He went way off (to the place he had selected), and his wife was following behind.’
\glt ‘Foi longe (até o lugar escolhido por ele), e a mulher foi indo atrás dele.’
\z

\ea sã hiromahanota himanaro to dʉruwero. \\[.3em]
\gll {\textasciitilde}sa=hí-ro-{\textasciitilde}baha-ro-ta	hí-{\textasciitilde}ba-{\textasciitilde}da-ro	tó	dʉrú-we-ro \\
     1\textsc{pl.excl.poss}=\textsc{cop-sg-}go.uphill\textsc{-sg-emph}	\textsc{cop-rem.ipfv-pl-sg}	\textsc{def}	thunder-\textsc{mov.}through\textsc{-sg}\\
\glt ‘It was our place up there where it thunders, the sacred place (called) \textit{Dʉrʉwero}.’{\footnotemark}
\glt ‘É nosso lugar bem lá em cima, lugar sagrado (chamado de) \textit{Dʉrʉwero}.’
\footnotetext{The placename \textit{Dʉrʉwero} means ‘where you can feel the thunder’.}
\z 

\ea toi tikorokoro tʉ̃kuñʉa õse thʉ! thʉ! thʉ!  \\[.3em]
\gll tó-i	tí-kó-ró-kóró	{\textasciitilde}tʉ́ku-{\textasciitilde}yʉ-a	{\textasciitilde}óse	thʉ!...thʉ!...thʉ! \\
     \textsc{rem-loc.vis}	\textsc{anph-f-sg-f.rsp} 	stomp-try-\textsc{assert.pfv}	like.this	\textsc{ontp:}thundering\\
\glt ‘There she stomped on the ground like this: \textit{Thʉ! Thʉ! Thʉ!}' (making a thundering sound)
\glt ‘Alí ela bateu com o pé assim: \textit{Thʉ! Thʉ! Thʉ!}' (fazendo ruido de trovão)
\z 

 
\ea “yaba hihari” ni ... mahareñʉ thuatera. \\[.3em]
\gll yabá	hí-hari	{\textasciitilde}dí	{\textasciitilde}baháré{\footnotemark}-{\textasciitilde}yʉ	thuá-te-ra \\
     \textsc{q:}what/how	\textsc{cop-q.ipfv}	say	turn.around-see/look	return\textsc{-neg-vis.ipfv.}2/3\\
\glt ‘“What's that?” she wondered (but \textit{Dianumia Yairo}) didn't even turn around.’
\glt “O que é isso?” ficou pensando (mas \textit{Dianumia Yairo}) nem olhou para trás.’
\footnotetext{The form \textit{maháré} is used for the actions of ‘turning around' or ‘going back and forth/going and returning', as in line 288.}
\z 

\ea do'se chõa mari buhtito warore yoarota niri hireto tiro bʉhkʉro ... ne mahare ñʉeraa, wa’awa’aa. \\[.3em]
\gll do'sé	{\textasciitilde}chóa	{\textasciitilde}bari=butí-to	waro-re	yoá-ro-ta {\textasciitilde}dí-ri	hí-re-to	tí-ró	bʉkʉ́-ro {\textasciitilde}dé	{\textasciitilde}baháré	{\textasciitilde}yʉ-éra-a 	wa'á-wa'a-a\\
     \textsc{q:}how	nephew	1\textsc{pl.incl.poss}=disappear-\textsc{nmlz.loc/evnt}	\textsc{emph-obj}	far\textsc{-sg-emph} \textsc{cop-nmlz}	\textsc{cop-clf:}gen-\textsc{nmlz.loc/evnt}	\textsc{anph-sg}	ancestor\textsc{-sg} \textsc{neg}	turn.around	look\textsc{-neg-assert.pfv}	go-go-\textsc{assert.pfv}\\
\glt (addressing Joselito) ‘[Because, nephew, (for him) it was just as it will be for us in the future (our disappearance, death, burial) — the old ancestor ... didn't even look back, off he went.]’
\glt (falando com Joselito) ‘[Porque, sobrinho, (para ele) era como vai ser para nós no futuro (nosso sumiço, morte, enterro) — o velho ... nem olhou para trás, foi embora.]’
\z 

\ea  te mipʉ to khã're khu'tu to nino nuhusʉ̃a. \\[.3em]
\gll  té	{\textasciitilde}bí-pʉ́	to	{\textasciitilde}kha'ŕe	khu'tú	to={\textasciitilde}dí-ro	{\textasciitilde}duhú-{\textasciitilde}sʉ-a\\
     until	now\textsc{-loc}	\textsc{def}	abiú/cucura.fruit	cemetery	3\textsc{sg.poss}=say-\textsc{sg}	accomodate.oneself-arrive.\textsc{trns}-\textsc{assert.pfv}\\
\glt ‘Until he (got to the place) now called \textit{Khãre Khu’tu} (and) made himself comfortable.’
\glt ‘Até chegar (no lugar) que hoje chamam de \textit{Khãre Khu’tu} (e) se acomodou.’
\z 

\ea to phosapho'nakã phʉarokã hia.  \\[.3em]
\gll to=phosá-{\textasciitilde}pho'da-{\textasciitilde}ka	phʉá-ro-{\textasciitilde}ka	hí-a \\
     3\textsc{sg.poss}=maku.people-children-\textsc{dim}	two\textsc{-sg-dim}	\textsc{cop-assert.pfv}\\
\glt ‘He had his two servants.’
\glt ‘Havia os seus dois criados.’
\z 

\ea khumuno naa, to muyaichʉ to bahtichʉ. \\[.3em]
\gll {\textasciitilde}khubú-ro	{\textasciitilde}dá-a	to={\textasciitilde}búyaichʉ	to=batíchʉ \\
     bench\textsc{-sg}	get-\textsc{assert.pfv}	3\textsc{sg.poss}=shaman.staff	3\textsc{sg.poss}=shield\\
\glt ‘(They) carried his bench, his shaman staff (and) shield.’
\glt ‘(Eles) carregaram seu banco do pajé, seu bastão de pajé (e) escudo.’
\z

\ea a’riase to diero, ba'arose to diero, sieseñʉroka yoaa. \\[.3em]
\gll a’ría-(bʉ')se	to=dié-ro	ba'á-ro-(bʉ')se	to=dié-ro	sié-(bʉ')sé-{\textasciitilde}yʉ́-dóká	yoá-a \\
     \textsc{dem.prox}-side	3\textsc{sg.poss}=dog\textsc{-sg}	after\textsc{-sg}-side	3\textsc{sg.poss}=dog\textsc{-sg}	front-side-see/look-\textsc{dist}	far-\textsc{assert.pfv}\\
\glt ‘Here (in front was) his dog/jaguar and behind his (other) dog/jaguar (and he sat) looking off straight ahead into the distance.’
\glt ‘Aqui (na frente ficou) um cachorro/onça e atrás seu (outro) cachorro/onça (e ele sentou) olhando para frente bem longe.’
\z 

\ea duhia phʉanʉmʉ, tia nʉmʉ, õpʉ hitu'sʉa.  \\[.3em]
\gll duhí-a	phʉá-{\textasciitilde}dʉbʉ	tíá-{\textasciitilde}dʉbʉ	{\textasciitilde}ó-pʉ	hí-thu'sʉ-a \\
     sit-\textsc{assert.pfv}	two-day	three-day	\textsc{deic.prox-loc}	\textsc{cop-}finish-\textsc{assert.pfv}\\
\glt ‘(He) sat for two days (and) on the third day, it was already up to here (his body entering into the ground).’
\glt ‘Sentou dois dias e no terceiro dia já estava até aqui (o corpo entrando dentro da terra).’
\z 


\largerpage
\ea “quatro, cinco, kʉ̃ somana ba'aro taga,” nia. \\[.3em]
\gll quatro	cinco {\textasciitilde}kʉ́-sómáná=ba'a-ro	tá-gá	{\textasciitilde}dí-a \\
     four	five	one/a-semana=after\textsc{-sg}	come-\textsc{imp}	say-\textsc{assert.pfv}\\
\glt ‘“Come back in four, five (days), a week,” (\textit{Dianumia Yairo}) said.’ 
\glt ‘“Daqui a quatro, cinco (dias), uma semana, venham,” (\textit{Dianumia Yairo}) disse.’
\z 
 

\ea “yʉ’ʉ, mʉhsa phʉkʉ, yʉ ñʉto bahsioro, ñamidahchomahka waroi sãika.” \\[.3em]
\gll yʉ’ʉ́	{\textasciitilde}bʉsa=phʉkʉ́	yʉ={\textasciitilde}yʉ́-tó	bahsí-o-ro {\textasciitilde}yabí-dáchó{\textasciitilde}báká	wáró-í	{\textasciitilde}sá-i-ka\\
     1\textsc{sg}	2\textsc{pl}.poss=father	1\textsc{sg.poss}=try-\textsc{nmlz.loc/evnt}	true\textsc{-caus-sg} night-middle/center	\textsc{emph}\textsc{-loc.vis}	inside-\textsc{m-predict}\\
\glt ‘“I, your father, will truly be going (to another world), right in the middle of the night.”’
\glt ‘“Eu, o pai de vocês, estarei indo de verdade (ao outro mundo), bem no meio da noite.”’
\z 

\ea “yʉ’ʉre bihsiroka, yʉ’ʉre thʉ'onaka mʉhsa yʉ pho'na” nia. \\[.3em]
\gll yʉ’ʉ́-re	bisí-doka	yʉ’ʉ́-ré	thʉ'ó-{\textasciitilde}da-ka	{\textasciitilde}bʉsá	yʉ={\textasciitilde}pho'dá	{\textasciitilde}dí-a \\
     1\textsc{sg-obj}	sound-\textsc{dist}	1\textsc{sg-obj}	hear\textsc{-pl-predict}	2\textsc{pl}	1\textsc{sg.poss}=children	say-\textsc{assert.pfv}\\
\glt ‘“I'll be going in (to the ground and) there will be a thunderous sound for me (and) you, my children will hear,” (\textit{Dianumia Yairo}) said.’
\glt ‘“Entrarei (na terra e) fará trovão bem forte para mim (e) todos vocês, meus filhos, vão ouvir,” disse (\textit{Dianumia Yairo}).’
\z 

\ea “hai” nia, dee ... topuro to niriba'ro bihsia thʉʉʉʉʉ ... \\[.3em]
\gll hái	{\textasciitilde}dí-a	dé	tó-puro	to={\textasciitilde}dí-rí-ba'ro	bisí-a	tʉʉʉʉʉ \\
     \textsc{intj:}agree	say-\textsc{assert.pfv}	\textsc{intj:}poor.one!	\textsc{def-quant.ms}	3\textsc{sg.poss}=say\textsc{-nmlz}-\textsc{clf:}kind	sound-go	\textsc{ontp:}thunder\\
\glt ‘“All right,” (they) answered (and) poor guy ... just at the time he indicated, there was a sound: \textit{Thʉʉʉʉʉ} ...' (thunder)
\glt ‘“Está bom,” responderam (e) coitado... bem na hora que ele indicou, deu ruído: \textit{Thʉʉʉʉʉ} ...' (trovoada)
\z 

\ea “de mari phʉkʉ wa'awa'aka” nia.  \\[.3em]
\gll dé	{\textasciitilde}bari=phʉkʉ́	wa'á-wa'a-ka	{\textasciitilde}dí-a \\
     \textsc{intj:}poor.one!	1\textsc{pl.incl.poss}=father	go-go-\textsc{assert.ipfv}	say-\textsc{assert.pfv}\\
\glt ‘“Poor guy! Our father's gone,” (they) said.’ 
\glt ‘“Coitado! Nosso pai foi embora,” falaram.’
\z

\newpage 
\ea “ñʉna wa'ahʉ̃ta” nia. ne mania.\\[.3em]
\gll {\textasciitilde}yʉ́-{\textasciitilde}dá	wa'á-{\textasciitilde}hʉ-ta	{\textasciitilde}dí-a	{\textasciitilde}dé	{\textasciitilde}badía \\
     see/look\textsc{-pl}	go-\textsc{imp.deic-emph}	say-\textsc{assert.pfv}	\textsc{neg}	not.exist\\
\glt ‘“Go there and see,” (they) said. (But) there was nothing there.’
\glt ‘“Vão lá ver,” disseram. (Mas) lá não havia nada.’
\z 
\ea ãta buhti sãwa'aa. \\[.3em]
\gll {\textasciitilde}áta	butí	{\textasciitilde}sa'á-wa'a-a \\
     because	disappear	\textsc{mov.}inside-go-\textsc{assert.pfv}\\
\glt ‘Because (\textit{Dianumia Yairo}) had disappeared into (the ground, alive).’ 
\glt ‘Porque (\textit{Dianumia Yairo}) havia sumiu entrando (na terra, vivo).’
\z 

\ea ã yoaroto, numia bu'iri, bu'iri nia tore, bu'ikho'to hiphakato. \\[.3em]
\gll {\textasciitilde}a=yóá-ró-tó	{\textasciitilde}dubía	bu'í-ri	bu'í-ri	{\textasciitilde}dí-a	tó-ré bu'í-kho'to	hí-phaka-to\\
     so=do-\textsc{sg-nmlz.loc/evnt}	women	cause\textsc{-nmlz}	cause\textsc{-nmlz}	say-\textsc{assert.pfv}	\textsc{def-obj} cause-proper.place	\textsc{cop-spec-nmlz.loc/evnt}\\
\glt ‘That's why, because of women, (that cemetery) is called \textit{Bu'i Kho’to} (place of guilt/problem), it seems so.’
\glt ‘Por isso, por causa de mulher, que (o cemetério) lá é chamado de \textit{Bu'i Kho’to} (lugar de culpa/problema), parece que é.’
\z 

\ea numia bu'iri õse yoa, sua to sãnuhʉriro ti khu'tu, mahsa yarikhu'tu.\\[.3em]
\gll {\textasciitilde}dubía	bu'í-ri	{\textasciitilde}óse	yoá	súá	to={\textasciitilde}sa'á-{\textasciitilde}duhʉ-ri-ro tí-khu'tu{\footnotemark}	{\textasciitilde}basá	yá-ri-khu'tu\\
     women	cause\textsc{-nmlz}	like.this	do	angry	3\textsc{sg.poss}=\textsc{mov.}inside-remain\textsc{-nmlz-sg} \textsc{anph}-cemetery	people	\textsc{poss-nmlz}-cemetery\\
\glt ‘It was because of women that he angrily went into that ground, the cemetery for our own bodies.’
\footnotetext{From this point on in the narrative, Teresinha switches to the form \textit{khu'tu}, which literally means ‘clearing', to refer to actual  cemeteries. This term is phonologically similar to \textit{kho'to}, ‘proper place' (see lines 16, 22, and 24), which is also used in the narrative in the names of specific cemeteries (e.g. line 271) and to refer to these as the ‘proper places' for burial, as in the title (line 4). It is likely the terms are etymologically related.}
\newpage 
\glt ‘Foi por causa das mulheres que ele foi com raiva dentro da terra, no cemetério para nossos corpos.’
\z

\ea bu'ikho'to hia. to se'reta sãse' bohkaerati, “do'se yoa bu'ikho'to niari” ni wãhkui. \\[.3em]
\gll bu'í-kho'to	hí-a	tó=se'e-re-ta	{\textasciitilde}sá-se'e	boka-éra-ti do'sé	yoá	bu'í-kho'to	{\textasciitilde}dí-á-rí	{\textasciitilde}dí	{\textasciitilde}wakú-i\\
     cause-proper.place	\textsc{cop-assert.pfv}	\textsc{def=contr-obj-emph}	1\textsc{pl.excl-contr}	find\textsc{-neg-attrib} \textsc{q:}how	do	cause-proper.place	say\textsc{-pl}\textsc{-nmlz}	say	wonder-\textsc{vis.pfv.}1 \\
\glt (Joselito speaking) ‘[(Yes) it's \textit{Bu'i Kho'to}. So that's it ... we didn't know why (people) called it that …’ 
\glt (Joselito falando) ‘[(Sim) é \textit{Bu'i Kho'to}. Então, é por isso ... não sabíamos por que (as pessoas) chamavam assim ...’ 
\z 

\ea ãni ya'uatire maikiro.  \\[.3em]
\gll {\textasciitilde}a={\textasciitilde}dí	ya'ú-átí-ré	{\textasciitilde}baí-kíró \\
     so-say	tell-\textsc{ipfv-vis.pfv.}2/3	father-\textsc{m.rsp}\\
\glt (Teresinha responding) ‘[That's the way my father used to tell it.]’ 
\glt (Teresinha respondendo) ‘ [Assim contava meu pai.]’
\z

\ea yʉ’ʉ ninose'ta mʉno hiko, noano sinitu thʉ'otuboa. \\[.3em]
\gll yʉ’ʉ́	{\textasciitilde}dí-ro-se'e-ta	{\textasciitilde}bʉ́-ro	hí-ko	{\textasciitilde}dóá-ró	sinítu	thʉ'ó-tu-bo-a \\
     1\textsc{sg}	say\textsc{-sg-contr-emph}	man\textsc{-sg}	\textsc{cop-f}	good\textsc{-sg}	ask	hear-think-\textsc{dub-assert.pfv}\\
\glt ‘[Like I said, if I were a man, I would have asked (and) understood more.{\footnotemark}]’
\glt ‘[Como disse, se fosse homem, teria perguntado (e) entendido mais.]’
\footnotetext{The composition of the conditional sentence is worth noting: the protasis is a nominalization with person marking \textit{-ko} ‘feminine’, and the apodosis contains the dubitative marker \textit{-bo}, showing that the result was not forthcoming.} 
\z 

\largerpage[2]
\ea õse ni to ya'uakãre thʉ'oatikʉrʉ, mahsihari? \\[.3em]
\gll {\textasciitilde}óse	{\textasciitilde}dí	to=ya'ú-a-{\textasciitilde}ka-re	thʉ'ó-ati-kʉrʉ	{\textasciitilde}basí-hárí \\
     like.this	say	3\textsc{sg.poss}=tell\textsc{-pl-dim-obj}	hear-\textsc{ipfv-advers}	know-\textsc{q.ipfv}\\
\glt ‘[(But) I unfortunately wasn’t paying enough attention to what little he said, you know?’]
\newpage
\glt ‘[(Mas) eu infelizamente não prestava atenção ao pouco que ele contava, sabe?]’
\z 

\ea õse hiatia mahko,” niatire.  \\[.3em]
\gll {\textasciitilde}óse	hí-ati-a	{\textasciitilde}bakó	{\textasciitilde}dí-ati-re \\
     like.this	\textsc{cop-ipfv-assert.pfv}	daughter	say-\textsc{ipfv-vis.pfv.}2/3\\
\glt ‘“That's what happened, daughter,” (father) used to say.’
\glt ‘“Era assim, filha” contava (meu pai).’ 
\z 

\ea “ãhia mahsapekururi yaaka” nia.\\[.3em]
\gll {\textasciitilde}a=hí-a	{\textasciitilde}basá-pé-kúrú-rí	yá-a-ka	{\textasciitilde}dí-a\\
so=\textsc{cop-pl}	people-\textsc{quant.c}-group\textsc{-nmlz}	bury\textsc{-pl-predict} say-\textsc{assert.pfv}\\
\glt ‘“So that all the (Kotiria) groups would have (a place) to be buried.”’
\glt ‘“Para que todos os grupos kotiria tivessem (o seu lugar) de ser enterrado.”’
\z 

\ea “a’ría nari siria, nari ya'saria, nari ñiria, nari ye'seria khõaroka” nia. \\[.3em]
\gll a’rí-á	{\textasciitilde}dá-rí	sí-ria	{\textasciitilde}dá-rí	ya'sá-ria {\textasciitilde}dá-rí	{\textasciitilde}yí-ria	{\textasciitilde}dá-rí	ye'sé-ria	{\textasciitilde}khoá-ro-ka	{\textasciitilde}dí-a\\
     \textsc{dem.prox-pl}	small.stone\textsc{-pl}	shiny-\textsc{clf:}round	small.stone\textsc{-pl}	green/blue-\textsc{clf:}round small.stone\textsc{-pl}	black-\textsc{clf:}round	small.stone\textsc{-pl}	white-\textsc{clf:}round	lie\textsc{-sg-predict}	say-\textsc{assert.pfv}\\
\glt ‘“There will be the place of Shiny Stones, the place of Green Stones, the place of Black Stones (and) the place of White Stones,” (father) said.’
\glt ‘“O lugar de Pedra Luminosa, lugar de Pedra Verde, lugar de Pedra Preta (e) lugar de Pedra Branca, esses vão ficar,” dizia (papai).’
\z 

\ea mari mipʉ hi ti kururire, õse nahu tha'rose. \\[.3em]
\gll {\textasciitilde}barí	{\textasciitilde}bí-pʉ́	hí	ti=kurú-ri-re	{\textasciitilde}óse	{\textasciitilde}dahú	tha'ró-sé \\
     1\textsc{pl.incl}	now\textsc{-loc}	\textsc{cop}	\textsc{anph}=group\textsc{-nmlz-obj}	like.this	flatbread	cut.in.quarters-\textsc{clf:}similar\\
\glt ‘(And) now we're divided into groups, like a flatbread cut into fourths.’
\glt ‘(E) nós agora ficamos dividos em grupos, como beijú cortado em quatro.’
\z 

 
\ea ã hiro õba'roi wa'masʉrore nari phichasiria khõaa, sã yakhu'tu. \\[.3em]
\gll {\textasciitilde}a=hí-ro	{\textasciitilde}ó=ba'ro-i	{\textasciitilde}wa'básʉ-ro-re {\textasciitilde}dá-rí	phichá-sí-ríá	{\textasciitilde}khoá	{\textasciitilde}sa=yá-khu'tu\\
     so=\textsc{cop-sg}	\textsc{deic.prox=clf:}kind\textsc{-loc.vis}	entrance\textsc{-sg-obj} small.stone\textsc{-pl}	fire-shiny-\textsc{clf:}round	lie	1\textsc{pl.excl.poss=poss}-cemetery\\
\glt ‘So, the entrance (first part) is for Shiny Stone burials, our grounds (for the highest clan).’
\glt ‘Então na primeira parte (na entrada), é lugar para Pedra Luminosa se enterrar, nosso lugar (do clã maior).’
\z

\ea ã hiro mahsawa'mino yariato, phano ñamidahchomahka bihsipakato nariwʉ'ʉ. \\[.3em]
\gll {\textasciitilde}a=hí-ro	{\textasciitilde}basá-{\textasciitilde}wa'bi-ro	yaríá-tó	{\textasciitilde}phádó	{\textasciitilde}yabí-dáchó{\textasciitilde}báká	bisí-pá-ká-tó	{\textasciitilde}dá-ri-wʉ'ʉ\\
     so=\textsc{cop-sg}	people-older.brother\textsc{-sg}	die-\textsc{nmlz.loc/evnt}	before	night-middle/center	sound-\textsc{alt}-do.moving-\textsc{nmlz.loc/evnt}	get\textsc{-nmlz}-house\\
\glt ‘That's why, before an older brother dies, in the middle of the night, a loud noise comes from the house/cemetery (where the person will be) taken.’
\glt ‘Por isso, antes de morrer um irmão maior, no meio da noite soa da casa/cemetério (para onde a pessoa vai ser) levada.’
\z 

\ea thʉʉʉʉʉʉ! patere khaaaa! bihsimarero. \\[.3em]
\gll tʉʉʉʉʉʉ	pá-tere	khaaaa	bisí-{\textasciitilde}bare-ro \\
     \textsc{ontp:}thunder	\textsc{alt}-time	\textsc{ontp:}thunder	sound-\textsc{rem.ipfv-sg}\\
\glt ‘\textit{Thʉʉʉʉ!} (or) sometimes \textit{Khaaaa!} is always the sound.’
\glt ‘\textit{Thʉʉʉʉ!} (ou) às vezes \textit{Khaaaa!} sempre soa assim.’
\z 

\ea a’riase khʉre ãta mʉhsa yakhu'tu a’riase nari ye'seria, ñahori yaro ya'saria. \\[.3em]
\gll a’rí-a-se	khʉ́-ré	{\textasciitilde}átá	{\textasciitilde}bʉsá	yá-khu'tu	a’rí-a-se {\textasciitilde}dá-rí	ye'sé-ria	ñáhórí	yá-ró	ya'sá-ria\\
     \textsc{dem.prox-pl-clf:}similar	\textsc{add-obj}	also	2\textsc{pl}	\textsc{poss}-cemetery	\textsc{dem.prox-pl-clf:}similar get\textsc{-nmlz}	white-\textsc{clf:}round	ñahori	\textsc{poss-sg}	green/blue-\textsc{clf:}round\\
\glt ‘It's the same in other places, your place, White Stones (for the \textit{Diani} and) for the \textit{Ñahori}, Green Stones.’
\glt ‘É assim no outro lado também, o lugar de vocês, Pedra Branca (dos \textit{Diani} e), o lugar dos \textit{Ñahori}, Pedra Verde.’
\z 

\ea sõ'o ñahoripho'na yaro ñiria khõaa.  \\[.3em]
\gll {\textasciitilde}so'ó	ñáhórí-{\textasciitilde}pho'da	yá-ró	{\textasciitilde}yí-ria	{\textasciitilde}khoá-a \\
     \textsc{deic.dist}	ñahori-descendants	\textsc{poss-sg}	black-\textsc{clf:}round	lie-\textsc{assert.pfv}\\
\glt ‘Over there in Black Stones is where \textit{Ñahori} children lie.’
\glt ‘Lá na Pedra Preta enterra-se os filhos de \textit{Ñahori}.’
\z 

\ea “ã hiro nahu tha'rose hira” niatire. \\[.3em]
\gll {\textasciitilde}a=hí-ró	{\textasciitilde}dahú	tha'ró-se	hí-ra	{\textasciitilde}dí-ati-re \\
     so=\textsc{cop-sg}	flatbread	cut.in.quarters-\textsc{clf:}similar	\textsc{cop-vis.ipfv.}2/3	say-\textsc{ipfv-vis.pfv.}2/3\\
\glt ‘“So, it's like a flatbread divided in fourths,” (my father) used to say.’
\glt ‘“Assim são como beijú dividido em quatro partes,” (meu pai) contava.’
\z 

\ea mahsape, tina yabaina puertu paloma, yabaro hihari? \\[.3em]
\gll {\textasciitilde}basá-pé	tí-{\textasciitilde}da	yabá-{\textasciitilde}ida	puertu paloma,	yabá-ro	hí-hari \\
     people-\textsc{quant.c}	\textsc{anph-pl}	\textsc{q:}what/how\textsc{-nmlz.pl}	puerto paloma,	\textsc{q:}what/how\textsc{-sg}	\textsc{cop-q.ipfv}\\
\glt ‘Everyone, those ones from Puerto Paloma (the last Kotiria village) — how’s it called?’
\glt ‘Todo mundo, aqueles de Puerto Paloma (última comunidade Kotiria) — como se chama?’
\z 


\ea ye'pua phitomahkainapʉ mahsa ya mahareatia, “kue,” nia niha. \\[.3em]
\gll ye'pú-á	phíto-{\textasciitilde}baka-{\textasciitilde}ida-pʉ	{\textasciitilde}basá	yá	{\textasciitilde}baháré-átí-á kué	{\textasciitilde}dí-a	{\textasciitilde}dí-ha\\
     cucura.fruit\textsc{-pl}	mouth.of.stream-village\textsc{-nmlz.pl-loc}	people	bury	go.and.return-\textsc{ipfv-assert.pfv} \textsc{intj}:surprise	say\textsc{-pl}	\textsc{prog-vis.ipfv.}1\\
\newpage 
\glt ‘\textit{Ye'pua Phito} (mouth of the cucura stream) villagers went back and forth to bury people, always saying “It's so far!”’
\glt ‘Moradores de \textit{Ye'pua Phito} (boca do igarapé cucura) vinham para cá e voltavam para enterrar gente, sempre dizendo “Como é longe!”’
\z

\ea paina ʉ̃rinapʉre namʉha, ti ma'a bʉhkʉma'a hiro nimanaro, to bu'iare ti ma'a. \\[.3em]
\gll pá-{\textasciitilde}ida	{\textasciitilde}ʉrí-{\textasciitilde}da-pʉ-re	{\textasciitilde}dá-{\textasciitilde}bʉha	tí-{\textasciitilde}ba'a	bʉkʉ́-{\textasciitilde}ba'a	hí-ro {\textasciitilde}dí-{\textasciitilde}ba-{\textasciitilde}da-ro	to=bu'ía-re	tí-{\textasciitilde}ba'a \\
     \textsc{alt-nmlz.pl}	smelly\textsc{-pl-loc-obj}	carry-\textsc{mov.}upward	\textsc{anph}-path	old-path	\textsc{cop-sg} \textsc{cop-rem.ipfv-pl-sg}	\textsc{def}=bu'ia.stream\textsc{-obj}	\textsc{anph}-path\\
\glt ‘They carried other rotting bodies up that old path that's always been there, the path from the \textit{Bu'ia} stream (leading to the cemetery).’
\glt ‘Carregavam outros corpos podres no caminho antigo, que sempre esteve ali, o caminho do igarapé \textit{Bu'ia}.’
\z 

\ea ti ma'aita mahareatii sã.  \\[.3em]
\gll tí-{\textasciitilde}ba'a-i-ta	{\textasciitilde}baháré-átí-í	{\textasciitilde}sá \\
     \textsc{anph}-path\textsc{-loc.vis-emph}	go.and.return-\textsc{ipfv-vis.pfv.}1	1\textsc{pl.excl}\\
\glt ‘It's the same path we always use to go back and forth (to our gardens).’
\glt ‘Nesse mesmo caminho nós sempre vamos e voltamos (da roça).’
\z 

\ea tina te ʉ̃rinapʉre ti khã'rekho'toi nawi'ika. \\[.3em]
\gll tí-{\textasciitilde}da	té	{\textasciitilde}ʉrí-{\textasciitilde}da-pʉ-re	ti={\textasciitilde}kha'ré-khu'tu-i	{\textasciitilde}dá-wi'i-ka \\
     \textsc{anph-pl}	until	smelly\textsc{-pl-empath-obj}	\textsc{anph}=abiú/cucura.fruit-cemetery\textsc{-loc.vis}	get-arrive\textsc{.cis-assert.ipfv}\\
\glt ‘They brought the poor decomposing bodies (on that path) to \textit{Khãre Kho’to}.’
\glt ‘Traziam os pobres fedendos (nesse caminho) até \textit{Khãre Kho’to}.’
\z 

\ea tina tia, tia, tara: “sã ñʉhchʉ, mʉ'ʉ kha'mana mahsane nana sã khã'inare.” \\[.3em]
\gll tí-{\textasciitilde}da	tí-a	tí-a	tá-rá {\textasciitilde}sa={\textasciitilde}yʉchʉ́	{\textasciitilde}bʉ'ʉ́	{\textasciitilde}kha'bá-rá	{\textasciitilde}basá-re	{\textasciitilde}dá-{\textasciitilde}da	{\textasciitilde}sa={\textasciitilde}kha'i-{\textasciitilde}ida-re\\
     \textsc{anph-pl}	cry-go	cry-go	come-\textsc{vis.ipfv.}2/3 1\textsc{pl.excl.poss}=grandfather	2\textsc{sg}	want-\textsc{vis.ipfv.}2/3	people\textsc{-obj}	get\textsc{-pl}	1\textsc{pl.excl.poss}=love\textsc{-nmlz.pl-obj}\\
\glt ‘They come weeping, weeping (saying): “You, our grandfather, want it, so we give you the body of the ones we love.”’
\glt ‘Vêm chorando, chorando (dizendo): “A você nosso avô, que quer, estamos entregando o corpo das pessoas que amamos.”’
\z 

\ea “mipʉre wi'boga hi'na” ni, phayʉ nikhʉ kha'machʉ tia. \\[.3em]
\gll {\textasciitilde}bí-pʉ́-ré	wi'bó-gá	{\textasciitilde}hí'da	{\textasciitilde}dí	phayʉ́	{\textasciitilde}dí-khʉ́	{\textasciitilde}kha'bá-chʉ	tí-a \\
     now\textsc{-loc-obj}	take.care-\textsc{imp}	\textsc{emph}	say	many	say-\textsc{add}	do.together-\textsc{sw.ref}	cry-\textsc{assert.pfv}\\
\glt ‘“Now take good care (of this body),” (they said), and said many other things, weeping together.’
\glt ‘“Agora guarda bem (o corpo),” diziam, e falavam muitas outras coisas, chorando juntos.’
\z 

\ea yo'omeheta nu'miriinadita, ti mahkari phaakãmahkanumiadita wa'aina topʉre. \\[.3em]
\gll yo'ó-{\textasciitilde}beheta	{\textasciitilde}du'bí-ri-{\textasciitilde}ida-dita	tí-{\textasciitilde}baka-ri phaá-{\textasciitilde}ka-{\textasciitilde}baka-{\textasciitilde}dubia-dita	wa'a-{\textasciitilde}ida	to-pʉ-re\\
     in.contrast\textsc{-neg}.intens	body.paint\textsc{-pl}\textsc{-nmlz.pl-sol}	\textsc{anph}-village\textsc{-nmlz} clan.members-\textsc{dim}-village-\textsc{pl.f-sol}	go\textsc{-nmlz.pl}	\textsc{def-loc-obj}\\
\glt ‘But (they) would only go painted, and only the women of that clan (family of that specific community) could go along.’
\glt ‘Mas iam só pintados, e só podiam ir as mulheres que faziam parte daquela clã (família da comunidade).’
\z 

\largerpage
\ea mahkari phaakãmahkono hieraro ti khu'tuita thuaa. \\[.3em]
\gll {\textasciitilde}baká-ri	phaá-{\textasciitilde}ka-{\textasciitilde}bako-ro	hi-éra-ro	tí-khu'tu-i-ta	thúa-a \\
     origin\textsc{-nmlz}	clan.members-\textsc{dim}-daughter\textsc{-sg}	\textsc{cop-neg-sg}	\textsc{anph}-cemetery\textsc{-loc.vis-emph}	return-\textsc{assert.pfv}\\
\glt ‘A woman who wasn’t part of the clan had to return before getting to the cemetery.’
\newpage 
\glt ‘A mulher que não fazia parte do clã tinha que retornar antes de chegar no cemetério.’
\z

\ea wi'i, thuawi'i ko toawe, ku'sʉwe, khõ'aroka, mahkachʉ ...  \\[.3em]
\gll wi'í	thúa-wi'i	kó	toá-we	ku'sʉ́-we	{\textasciitilde}kho’a-doka	{\textasciitilde}baká-chʉ \\
     arrive\textsc{.cis}	return-arrive\textsc{.cis}	medicine	vomit-\textsc{mov.}through	bathe-\textsc{mov.}through	throw-\textsc{dist}	look.for-eat\\
\glt ‘Arriving home, (they cleansed themselves with) vomit medicine, bathed to throw off (the effect of the burial), then had something to eat ...’ 
\glt ‘Quando chegavam (do enterro) faziam limpeza de estômago (vomitando), tomavam banho para jogar fora (o efeito do enterro), depois se alimentavam ...’
\z 

\ea “yoaro hiro nira,” niatire maimʉnano ti khu'ture. \\[.3em]
\gll yoá-ró	hí-ro	{\textasciitilde}dí-ra{\footnotemark}	{\textasciitilde}dí-átí-ré	{\textasciitilde}baí-{\textasciitilde}bʉda-ro	tí-khu'tu-re \\
     do\textsc{-sg}	\textsc{cop-sg}	\textsc{prog-vis.ipfv.}2/3	say-\textsc{ipfv-vis.pfv.}2/3	father-deceased\textsc{-sg}	\textsc{anph}-cemetery\textsc{-obj}\\
\glt ‘“This is how it's (always) done,” my late father used to tell (me) about the cemetery.’
\glt ‘“Assim é que se faz (sempre),” contava meu pai finado sobre o cemetério.’
\footnotetext{This sentence is a good example of how the visual imperfective suffix used on the copula is understood as a statement of fact, rather than having any type of specific temporal reference (see \citealt[281]{Stenzel2013}).} 
\z 

\ea ã yoa mahariro tiro bʉhkʉro toi duhika. \\[.3em]
\gll {\textasciitilde}a=yóá	{\textasciitilde}bahá-ri-ro	tí-ró	bʉkʉ́-ro	tó-i	duhí-ka \\
     so=do	go.uphill\textsc{-nmlz-sg}	\textsc{anph-sg}	ancestor\textsc{-sg}	\textsc{rem-loc.vis}	sit-\textsc{assert.ipfv}\\
\glt ‘Because our old ancestor (\textit{Dianumia Yairo}) who went up there is still sitting there (still exists, lives there).’
\glt ‘Porque o velho nosso avô (\textit{Dianumia Yairo}) que subiu ainda está sentado alí (ainda existe, vive alí).’
\z 

 
\ea a’ri phakʉi hirota nira.  \\[.3em]
\gll a’rí	phákʉ-i	hí-ro-ta	{\textasciitilde}dí-ra \\
     \textsc{dem.prox}	body\textsc{-loc.vis}	\textsc{cop-sg-emph}	\textsc{cop-vis.ipfv.}2/3\\
\glt ‘That body is there.’	
\glt ‘O corpo está lá.’
\z 

\ea tiro yariariro hierare. himarero tiro, a’ri phakʉi tiro.\\[.3em]
\gll  tí-ró	yaríá-rí-ró	hi-éra-re	hí-{\textasciitilde}bare-ro	tí-ró	a’rí	phákʉ-i	tí-ró \\
     \textsc{anph-sg}	die\textsc{-nmlz-sg}	\textsc{cop-neg-vis.pfv.}2/3	\textsc{cop-rem.ipfv-sg}	\textsc{anph-sg}	\textsc{dem.prox}	body\textsc{-loc.vis}	\textsc{anph-sg}\\
\glt ‘He isn't dead, he's there (alive), his body.’
\glt ‘Ele não está morto, está (vivo), o corpo dele.’
\z 

\ea “do'se tiro thíkari khõ'amahkʉ” ni dohoatiti patena. \\[.3em]
\gll do'sé	tí-ró	thí-kari	{\textasciitilde}khoabakʉ	{\textasciitilde}dí	dohó-ati-ti	páte{\textasciitilde}da \\
     \textsc{q:}how	\textsc{anph-sg}	true-\textsc{q.spec}	god	say	ask-\textsc{ipfv-refl}	sometimes\\
\glt ‘[Sometimes I ask myself: “Could he be God?”]’
\glt ‘[Às vezes me pergunto: “Será que é ele Deus?”]’
\z 

\ea tiro hira a’ri dahchore khʉariro. \\[.3em]
\gll tí-ró	hí-ra	a’rí dachó-ré	khʉá-ri-ro \\
     \textsc{anph-sg}	\textsc{cop-vis.ipfv.}2/3	\textsc{dem.prox} day\textsc{-obj}	have\textsc{-nmlz-sg}\\
\glt ‘He's the one who has/controls time.’
\glt ‘Ele é quem é o dono do tempo.’
\z

\ea ni maimʉnano ya'uatire ti khu'ture, ãhia wa'manopʉre. \\[.3em]
\gll {\textasciitilde}dí	{\textasciitilde}baí-{\textasciitilde}bʉda-ro	ya'ú-átí-ré	tí-khu'tu-re {\textasciitilde}a=hí-a	{\textasciitilde}wa'bá-ro-pʉ-re\\
     say	father-deceased\textsc{-sg}	tell-\textsc{ipfv-vis.pfv.}2/3	\textsc{anph}-cemetery\textsc{-obj} so=\textsc{cop-pl}	young/new\textsc{-sg}\textsc{-loc}\textsc{-obj}\\
\glt ‘That's what my late father told (me) about this cemetery, about how things were back then.’
\glt ‘Assim contava meu pai a respeito desse cemetério, assim como era antigamente.’
\z 

\begin{figure}[t]
  \includegraphics[width=1.0\textwidth]{figures/figure-bui-khoto.png}
  \caption{\textit{Bu'i Kho'to}. Illustration by Moisés Galvez Trindade and Auxiliadora Figueiredo}
\end{figure}

\section*{Acknowledgments}

I am deeply grateful to the Kotiria people who have welcomed me so warmly into their communities over our many years of work together. 

\def\oldIntextsep{\the\intextsep}
\setlength{\intextsep}{0.25\baselineskip}
\begin{wrapfigure}{l}{6.3 cm}
  \includegraphics[width=6.3 cm]{figures/Teresinha_Kris.jpg}
  \caption{Kris and Teresinha in 2017}
  \label{fig:kristeresinha}
\end{wrapfigure}
\setlength{\intextsep}{\oldIntextsep}

Special thanks are due to the wise and witty Teresinha Marques for offering this epic narrative and to José Galvez Trindade for his recognition of its importance and for his dedicated work on the initial transcription. This analysis also owes a great deal to Miguel Cabral and his vast knowledge of Kotiria culture and language. Finally, my thanks to Bruna Franchetto for her careful reading and help with the Portuguese translations.

 
\newpage  
My research on Kotiria and other East Tukano languages has received financial support from 
the National Science Foundation Linguistics Program (dissertation grant 0211206), 
the NSF/NEH Documenting Endangered Languages Program (FA-52150-05; BCS-1664348), 
ELDP/SOAS (MDP-0155), 
the Brazilian National Council for Scientific and Technological Development (CNPq, post-doctoral grant 2005-2007),
the Brazilian Ministry of Education’s Program for Continuing Academic Development (CAPES, post-doctoral ‘Estágio Senior' grant, 2014-2015), and 
the Federal University of Rio de Janeiro. 
Renata Alves of the Instituto Socioambiental is gratefully acknowledged for her design of \autoref{mapone}, as is Miguel Cabral Junior for his rendition of \autoref{maptwo}. 

\section*{Non-standard abbreviations}

\begin{tabularx}{.45\textwidth}{lQ}
\textsc{add} & additive \\
\textsc{aff} & affected \\
\textsc{alt} & alternate \\
\textsc{anph} & anaphoric \\
\textsc{assert} & assertion \\
\textsc{attrib} & attributive \\
\textsc{aum} & augmentative \\
\textsc{cis} & cislocative \\
\textsc{contr} & contrastive \\
\textsc{deic} & deictic \\
\textsc{des} & desiderative \\
\textsc{dim} & diminutive \\
\textsc{empath} & empathetic \\
\textsc{emph} & emphasis \\
\textsc{epis} & epistemic \\
\textsc{evnt} & event \\
\textsc{exrt} & exhortative \\
\textsc{frus} & frustrative \\
\textsc{infer} & inference \\
\end{tabularx}
\begin{tabularx}{.45\textwidth}{lQ}
\textsc{intent} & intention \\
\textsc{intens} & intensifier \\
\textsc{mir} & mirative \\
\textsc{predict} & prediction \\
\textsc{quant.c} & quantitative for count noun \\
\textsc{quant.ms} & quantitative for mass noun \\
\textsc{refl} & reflexive \\
\textsc{rem} & remote \\
\textsc{rep} & reported \\
\textsc{rsp} & respect \\
\textsc{spec} & speculative \\
\textsc{sol} & solitary \\
\textsc{sw.ref} & switch reference \\
\textsc{temp} & temporal \\
\textsc{trns} & translocative \\
\textsc{vbz} & verbalizer \\
\textsc{vis} & visual \\
\end{tabularx}

\newpage 
{\sloppy
\printbibliography[heading=subbibliography,notkeyword=this]
}

\end{document}
