\documentclass[output=paper,
modfonts,nonflat
]{langsci/langscibook} 
\author{Ana Vilacy Galucio\affiliation{Museu Paraense Emílio Goeldi}%
\and Mercedes Guaratira Sakyrabiar%
\and Manoel Ferreira Sakyrabiar%
\and Rosalina Guaratira Sakyrabiar%
\lastand Olimpio Ferreira Sakyrabiar%
}%
\title{Sakurabiat}
\lehead{A. Galucio, M. G. Sakyrabiar, M. F. Sakyrabiar, R. Sakyrabiar \& O. Sakyrabiar}
\ourchaptersubtitle{Kõtkõra asisi}
\ourchaptersubtitletrans{‘Kõtkõra's corn’}  
% \abstract{noabstract}
\ChapterDOI{10.5281/zenodo.1008787}

\maketitle

\begin{document}

\section{Introduction} 

This short narrative is a fragment of a mythological tale that describes the origin of maize and other crops, such as beans and manioc (yucca), among the Sakurabiat people.
Sakurabiat is pronounced [sa'kɨrabiat]. In the orthographic convention for the language the grapheme \textbf{<u>} represents the hight central vowel \textbf{[ɨ]}.The Sakurabiat are very reduced in number. In the last survey done in 2016, there were only 65 people living in the Rio Mekens Indigenous Land.

The \textit{Kõtkõra asisi} story is told by Mercedes Guaratira Sakyrabiar, one of the oldest speakers of Sakurabiat at the time of the recording. Sadly, she passed away in December of 2015. Mercedes's age was not known for certain, but she was believed to be more than 75 years old when she told this story in 2006. 
The story was recorded in audio as part of a long term project for the documentation and study of the Sakurabiat (Mekens) language, which had partial support from the Endangered Language Documentation Program, funded by the School of Oriental and African Studies (SOAS). 

Transcription and analysis was done by Galucio with the assistance of Rosalina Guaratira Sakyrabiar, Mercedes's daughter, and the two brothers who have been Galucio's main collaborators in the study of the Sakurabiat language: Manoel Ferreira Sakyrabiar, a very talented man and an enthusiast of the study of his language, who was brutally murdered in 2016, and his younger brother Olimpio Ferreira Sakyrabiar. All three are bilingual speakers of Sakurabiat and Portuguese. 

This text is one of the 25 mythological tales, recounted by some of the most distinguished Sakurabiat elders, that appear in the book \textit{Narrativas Tradicionais SAKURABIAT mayãp ebõ} (organized by Galucio 2006). Some of the illustrations used in that book, which were drawn by two Sakurabiat children at the time of publication, Lidia Sakyrabiar and Ozelio Sakyrabiar, are included here.

According to the account given in the narrative, the Sakurabiat were unfamiliar with maize and other edible crops until one day \textit{Arɨkʷajõ} discovered that \textit{Kõtkõra}, a shaman from a neighboring group, had maize. \textit{Arɨkʷajõ} then went to visit that shaman and stole the seeds. 

\begin{figure}
\includegraphics[width=\textwidth]{figures/mekens-illustration12.png}
  \caption{\textit{Arɨkʷajõ} following the black-fronted piping guan bird. Illustration by Lidia Sakyrabiar and Ozélio Sakyrabiar}
\end{figure}

\textit{Arɨkʷajõ} is the great mythological figure in Sakurabiat traditional stories. He is considered to be the great shaman with vast powers and wisdom. In the mythological narratives, he appears as the creator of many aspects of nature. For instance, he created the mountains and the valleys, and he had water and fire when no one else had them. He is also portrayed as the father of the sun, named \textit{Kiakop} ‘our warmth', and the moon, named \textit{Pakori}, and the one responsible for sending them both away from earth as punishment for inappropriate social behavior. \textit{Kiakop}, the sun, had set his sisters on fire, and \textit{Pakori}, the moon, had tricked his sister into having an incestuous sexual relationship with him. In the present narrative \textit{Arɨkʷajõ} is married to his second wife, \textit{Pãrãrẽkosa}. 

\newpage 
The other main character in the narrative, \textit{Kõtkõra}, which means ‘cicada’, is another mythological shaman, apparently from a distinct ethnic mythological group.  \textit{Kõtkõra}'s group had edible crops that \textit{Arɨkʷajõ}'s people, the Sakurabiat, did not have. 

The fragment of the story analyzed here focuses on how, after noticing that a black-fronted piping guan bird was defecating corn, \textit{Arɨkʷajõ} followed the bird to find out where it was eating corn. When he discovered that it was at \textit{Kõtkõra}'s village, he visited the other shaman's house with the intent to get corn for himself and his family. Thus, despite being well received by \textit{Kõtkõra}, \textit{Arɨkʷajõ} used his special powers to steal the seeds from \textit{Kõtkõra}'s house. After bringing the crops to his village, and starting to make his food from maize and yucca, he changed his children's teeth, adapting them to their new condition of eating crops rather than stones. 
That is, since they no longer had to eat stones and wild seeds, they could have softer teeth, teeth that would fall out, eventually. 
The complete narrative recalls the first meeting of these two shamans and how both ended up having crops to eat. 
After \textit{Arɨkʷajõ} was received in \textit{Kõtkõra}'s house, he took seeds for his village, but did not destroy \textit{Kõtkõra}'s fields. Thus, it is a mythological tale that touches upon relationships of hospitality and rivalry among neighbors. 

\begin{figure}
\includegraphics[width=\textwidth]{figures/mekens-illustration11.png}
  \caption{The arrival at \textit{Kõtkõra}'s house. Illustration by Lidia Sakyrabiar and Ozélio Sakyrabiar}
\end{figure}

The language of the Sakurabiat people has been traditionally referred to in the literature as “Mekens". In more recent years, it has also been referred to as Sakurabiat in an attempt to acknowledge the self denomination of the group, and we adopt this name in this work, to refer to the language, one of the five surviving languages of the Tuparian branch of the large Tupian family.

\newpage 
The Tupian family is composed of ten subfamilies \citep{hv:Rodrigues:Tupi-Guarani:Internas}, which include about 40-45 languages that are spread throughout the Amazon \citep{hel:Mooreetal:Amazonicas}.
Alongside Sakurabiat, the other four languages of the Tuparian branch are: Akuntsú, Makurap, Tupari, and Wayoro.
All five languages are spoken in the state of Rondônia, in the northwestern part of Brazil, near the Brazil-Bolivia border, and they are all highly endangered due to the greatly reduced number of speakers. 
According to information collected in 2016, from Galucio's  field work and from colleagues working with these specific languages, Tupari has about 300 speakers, Makurap has about 55-60 speakers, Akun\-tsú only 4, and Wayoro just 3 speakers. Sakurabiat has about 16 speakers, and they are all adults. 

\begin{figure} 
\includegraphics[width=\textwidth]{figures/mekens-map-mod}
  \caption{The Rio Mekens Indigenous Land, where the Sakurabiat live in the Brazilian state of Rondônia, is shown in yellow.}
  \label{fig:mekens:map}
\end{figure}

It is noteworthy that the current state of Rondônia houses representatives of six of the ten Tupian subfamilies, including five that are spoken exclusively there: Arikém, Puruborá, Mondé, Ramarama, and Tupari. The other five subfamilies are Juruna, Munduruku, Mawé, Aweti, and Tupi-Guarani, which is the largest and most widespread of the Tupian subfamilies.

Sakurabiat is a typical Tupian language. It is a primarily suffixing language, but it also has a few prefixes, such as the  pronominal person markers and valence changing morphemes (causative and intransitivizer). The language shows a head-marking profile, with locus of morphosyntactic marking on the head of the phrase. In clauses, the syntactic functions of subject and object are marked on the verb rather than on the nominal arguments. In simple transitive clauses with nominal arguments, both noun phrases tend to precede the verb, following basic SOV order. 
There are three types of lexical verbs: intransitive, transitive, and uninflectible or particle verbs. Transitive and intransitive verbs take person agreement and TAM inflectional markers. Only one argument is indexed on the verb by means of person prefixes. The intransitive verb indexes the subject and the transitive verb indexes the object.
The particle verbs do not inflect in that way. In order to take person and TAM inflection, they undergo derivation via the verb formatives (\textit{-ka}, \textit{-kwa}, \textit{e-}), which give as output transitive or intransitive verb stems. 

Based on the distribution of the person markers, the morphosyntactic alignment can be described as nominative-absolutive in simple main clauses, as proposed for some Cariban and Northern Jê languages by \citet{GildeaEtAl2010}. The set of prefixes marks the absolutive argument (S/P), while the set of free pronouns expresses the nominative (A/S).
In the case of transitive verbs, pronominal subjects obligatorily occur as free pronouns, except for third person, which can be left unmarked. With intransitive verbs, on the other hand, free pronouns are optionally used, co-occurring  with the subject verb agreement markers. For its part, the O argument is never expressed by a free pronoun. For an overall description of Sakurabiat verb agreement and argument structure, see \citet{Galucio2014}.

\largerpage
Auxiliaries also show person agreement and TAM inflections. Person indexation on auxiliaries follows a nominative pattern, always indexing the clause subject (A or S).
Auxiliaries and demonstratives are positional roots that indicate the body posture of the subject, in the case of auxiliaries, and of the referent, in the case of demonstratives. 
In addition to the positional demonstratives, there is a series of discourse anaphoric demonstratives or proforms that are used to replace a syntactic unit: they can replace a syntactic phrase, an entire clause, or even larger stretches of discourse. 

Nominalization is the main strategy used to form adverbial (temporal, conditional, causal, and final) clauses in Sakurabiat \citep{Galucio2011}. The adverbial modification is encoded by a nominalized verb form (with the nominalizer \textit{-ap} ‘instrumental; circumstantial’) or one of the demonstrative proforms followed by a postposition. The ablative postposition \textit{eri} is used for causal clauses, and the locative postposition \textit{ese} for temporal/conditional and also some causal clauses. 

Three dialects have been identified for Sakurabiat: Guaratira, which is the one spoken by the narrator of this story, Siokweriat, and Sakurabiat/Guarategayat. The major differences among them are phonological and lexical.

The \textit{Kõtkõra asisi} story is transcribed phonetically in the first line, and segmented phonologically and  morphologically in the second line. There is nasal harmony inside the word in Sakurabiat. 
Nasality spreads rightwards from a nasal consonant or vowel, and is blocked only by an obstruent in onset position.
Thus, in the second line, only the underlying nasal element is indicated as being nasal.
The third line gives a morpheme-by-morpheme gloss. English and Portuguese free translations are given in the fourth and fifth lines. The Portuguese translation attempts to maintain in as much as possible the structure of the original Sakurabiat narrative. 








\section{Kõtkõra asisi}
\translatedtitle{\textit{‘Kõtkõra's} corn’ or ‘The origin of maize and other crops among the Sakurabiat people’}\\

\translatedtitle{\hspace*{-3.5mm}‘O milho do \textit{Kõtkõra}’ ou ‘A origem do milho e outras plantas entre o povo Sakurabiat’}\footnote{Recordings of this story are available from \url{https://zenodo.org/record/997447}}
 

\ea Arɨkʷajõ asisi aapi ara sekoa kõnkõrã tegeri.\\[.3em]
\gll arɨkʷajõ asisi aapi at-a se-ko-a kõtkõra tek=eri\\
     Arɨkʷajõ corn crop.seed get-\textsc{thv} \textsc{3cor-aux.mov-thv} cicada house=\textsc{abl}\\
\glt ‘\textit{Arɨkʷajõ} got corn seed from \textit{Kõtkõrã}'s house.’ 
\glt ‘\textit{Arɨkʷajõ} arrumou semente de milho na casa da Cigarra.’
\z 

 
\newpage
\ea Kʷako sejẽrora ek piitse.\\[.3em]
\gll kʷako se-jẽt ot-a ek pi=ese\\
     black-fronted.piping.guan \textsc{3cor}-feces leave-\textsc{thv} house inside=\textsc{loc}\\
\glt ‘The black-fronted piping guan bird\footnote{The black-fronted piping guan bird (\textit{Aburria jacutinga}) is a large bird that is easily identified, since in almost all its area of occurrence it is the only cracid with a white spot on the wing. Its scientific name comes from \textit{burria, aburri, aburria} = Colombian Amerindian onomatopoeic name for birds generally called \textit{jacu}; and from Old Tupi \textit{jacú} = Jacu, and \textit{tinga} = white, in reference to the head, nape and wings of this bird that has feathers with white coloration (\url{http://www.wikiaves.com.br/jacutinga}).} (that was around) defecated inside the house.’
\glt ‘Jacutinga (filhote de jacutinga que estava andando por lá) defecou dentro da casa.' 
\z 

 

\ea Pɨ ke itoa{\footnotemark} enĩĩtse.\\[.3em]
\gll pɨ ke i-to-a eni=ese\\
     lying \textsc{dem} \textsc{3sg-aux.lie-thv} hammock=\textsc{loc}\\
\glt ‘He (\textit{Arɨkʷajõ}) was there just lying in the hammock.’
\glt ‘E ele (\textit{Arɨkʷajõ}) estava lá deitado na rede.'
\footnotetext{The third person singular prefix has two allomorphs: \textit{i-} before consonant-initial stems; and \textit{s-} before vowel-initial stems.}
\z 

\ea Sete itsoa ajẽẽri te kẽrã atsitsi ko?\\[.3em]
\gll sete i-so-a a-jẽ=eri te kẽrã asisi ko\\
     \textsc{3sg} \textsc{3sg}-see-\textsc{thv} \textsc{q-dem.prox}=\textsc{abl} \textsc{foc} \textsc{nassert} corn ingest\\
\glt ‘(Then) he looked, (and thought): “Where does he (the bird) eat corn?”’
\glt  ‘Aí ele olhou (e pensou): “Aonde será que ele come milho?”' 
\z 

\ea Sitõm{\footnotemark} nẽ pa õt, otagiat.\\[.3em]
\gll s-itõp  ne pa õt o-tag-iat\\
     \textsc{3sg}-follower \textsc{cop} \textsc{fut} \textsc{1sg} \textsc{1sg}-daughter-\textsc{col}\\
\glt ‘Then (he said to this daughters): “I will follow him, my daughters.”’
\glt ‘Aí (falou pras filharadas): “Eu vou atrás dele, minhas filhas.”'
\footnotetext{\textit{-itõp} is an adjective root that is reported by Sakurabiat speakers to mean something like ‘the follower, the one that follows someone or something, the one that accompanies someone’.}
\z 

\newpage 
\ea Soa kot kaap{\footnotemark} te pekaat soa kot.\\[.3em]
\gll so-a kot kaap te pe=kaat so-a kot\\
     see-\textsc{thv} \textsc{im.fut} \textsc{quot} \textsc{foc} \textsc{obl}=\textsc{dem} see-\textsc{thv} \textsc{im.fut}\\
\glt ‘“I will look to see, I will look at that.”’
\glt ‘“Eu vou atrás pra ver."'
\footnotetext{There are two third person quotative forms: \textit{kaap} and \textit{kaat}. The quotative in combination with the immediate future morpheme \textit{kot} derives a desiderative clause. Desideratives are grammaticalizing as future. The quotative form \textit{kaat} is also homophonous to the demonstrative \textit{kaat}, a proform that is used anaphorically to replace a stretch of discourse.}
\z 

\ea Pia{\footnotemark} setoa õẽm te kʷako setset nẽãrã.\\[.3em]
\gll pia se-to-a õẽp te kʷako se-set neara\\
     wait \textsc{3cor-aux.lie-thv} already \textsc{foc} black-fronted.piping.guan \textsc{3cor}-leave again\\
\glt ‘He (\textit{Arɨkʷajõ}) stayed there waiting for the black-fronted piping guan bird to come back.’
\glt ‘Ele (\textit{Arɨkʷajõ}) ficou esperando esse jacu para poder ver a hora que ele ia de novo lá (comer).'
\footnotetext{\textit{pia} ‘wait’ is the form in the Guaratira and Siokweriat dialects. It corresponds to \textit{pisa} in the Sakurabiat dialect.}
\z 

\largerpage
\ea  Atɨ{\footnotemark} sete sitõmnã tõpnã sekoa pɨbot Kõnkõrã taap.{\footnotemark}\\[.3em]
\gll atɨ sete s-itõp=na tõp=na se-ko-a pɨbot kõtkõra taap\\
     \textsc{intj} \textsc{3sg} \textsc{3sg}-follower=\textsc{vblz} follower=\textsc{vblz} \textsc{3cor}-\textsc{aux.mov}-\textsc{thv} arrive cicada village\\
\glt ‘Poor guy, he followed him all the way until he arrived at \textit{Kõtkõrã}'s (Cicada's) village.’ (\textit{Arɨkʷajõ} followed the bird all the way to \textit{Kõtkõra}'s house)
\glt ‘Aí ele coitado acompanhou (o jacu), acompanhou, acompanhou até que chegou na casa da Cigarra.'
\addtocounter{footnote}{-1}
\footnotetext{\textit{atɨ} is an interjection used as a negative exclamation expressing an emotional reaction, such as negative astonishment or bewilderment. It does not form a syntactic constituent to the rest of the sentence. It can be translated sometimes as ‘pitied guy’ or ‘pitied thing’. It is used to cast doubt about the proposition that one considers to be unlikely or absurd. For instance, if someone says to you \textit{osera kot ameko miapna} ‘I'm going over there to kill a jaguar’, you can respond with \textit{atɨ nop} ‘Poor you, no (you won't).’}
\stepcounter{footnote}
\footnotetext{\textit{taap} is a word that has several meanings, it could be translated as either ‘village’ or ‘house’, but  it contrasts with the regular word for house, which is \textit{ek}.}
\z 

\ea Kerep itegõ{\footnotemark} kõjẽ siko õpinã.\\[.3em]
\gll kerep i-tek=õ kõjẽ s-iko õp-pit=na\\
     enter \textsc{3sg}-house=\textsc{dat} sit \textsc{3sg}-food give-\textsc{nmlz}=\textsc{vblz}\\
\glt ‘He (\textit{Arɨkʷajõ}) entered the house, sat himself down, and was given food.’
\glt ‘(\textit{Arɨkʷajõ}) entrou na casa, sentou, aí começaram dar comida pra ele comer.'
\footnotetext{The dative postposition has two allomorphs: \textit{õ} after consonant-final stems; and \textit{bõ} after vowel-final stems. This postposition has a meaning that is broader than the usual datives. It can express the indirect object, but also the instrumental, the general locative, and the temporal locative.}
\z 

\ea Kõjẽ poget kop.\\[.3em]
\gll kõjẽ poget kop\\
     sit standing \textsc{aux.mov}\\
\glt ‘He sat down, then got up, and stayed around there.’
\glt ‘Ele sentou, depois levantou, ficou por ali.'
\z 

\ea Ma te kẽrã eke aose setserara, ke te Kõnkõrã tagiat.\\[.3em]
\gll ma te kẽra eke aose se-set-a-ra ke te Kõtkõra tak-iat\\
     when \textsc{foc} \textsc{nassert} \textsc{dem.n} man \textsc{3cor}-leave-\textsc{thv-rep} \textsc{dem} \textsc{foc} cicada daughter-\textsc{col}\\
\glt ‘“When is that man leaving?” said \textit{Kõtkõrã}'s daughters.’
\glt ‘“É esse homem não vai não embora, não, será?” Assim que a filharada do \textit{Kõtkõra} falou.' 
\z 

\ea Teeri ka aotse atsitsi 'ara nããn kop.\\[.3em]
\gll te-eri ka aose asisi 'at-a naat kop\\
     \textsc{3sg}=\textsc{abl} move man corn get-\textsc{thv} \textsc{cop} \textsc{aux.mov}\\
\glt ‘Through his mind, the man is carrying off the corn.’ (Lit. ‘It comes from him, the man is taking the corn.’)
\glt ‘Ele no pensamento dele tá carregando milho.'
\z 

\newpage 
\ea Tapsɨrõ i'ara atsitsibõ i'ara arakʷibõ i'ara komatabõ i'ara kaat naat kop aose.\\[.3em]
\gll tapsɨt=bõ i-'at-a asisi=bõ i-'at-a arakʷi=bõ i-'at-a komata=bõ i-'at-a kaat naat kop aose\\
     yucca=\textsc{dat} \textsc{3sg}-get-\textsc{thv} corn=\textsc{dat} \textsc{3sg}-get-\textsc{thv} peanut=\textsc{dat} \textsc{3sg}-get-\textsc{thv} beans=\textsc{dat} \textsc{3sg}-get-\textsc{thv} \textsc{dem} \textsc{cop} \textsc{aux.mov} man\\
\glt  ‘He got manioc, corn, peanuts, beans, he stayed there doing that (taking everything), the man (i.e., \textit{Arɨkʷajõ}).’
\glt ‘Levou mandioca, milho, amendoim, feijão, ficou carregando tudo, o homem.'
\z  

\ea Poret õẽm 'arabetse set nẽãrã.\\[.3em]
\gll poret õẽp 'ar-ap=ese set neara\\
     now already get-\textsc{nmlz}=\textsc{loc} leave again\\
\glt ‘Then, when he had already got it all, he left again.’
\glt ‘Aí quando já tinha carregado tudo, ele foi embora de novo.'
\z 

\ea Pɨbot nẽãrã setoabõ.{\footnotemark}\\[.3em]
\gll pɨbot neara se-top-ap=õ\\
     arrive again \textsc{3cor}-lying.down-\textsc{nmlz}=\textsc{dat}\\
\glt ‘He arrived again at his own hammock.’
\glt ‘E chegou na sua rede (na casa dele) novamente.'	
\footnotetext{The word \textit{toap} ‘hammock’ is a derived noun, formed by the auxiliary root \textit{top} ‘\textsc{aux.lie}' plus the circumstantial nominalizer \textit{-ap}; the final \textit{-p} is deleted before the vowel \textit{-a}: \textit{top+ap} → \textit{toap}.}
\z 

\ea Tamõ'ẽm porẽtsopega{\footnotemark} petsetagiat:\\[.3em]
\gll ta=bõ='ẽp porẽsopeg-a pe=se-tak-iat\\
     \textsc{dem.stand}=\textsc{dat}=\textsc{emph} ask-\textsc{thv} \textsc{obl}=\textsc{3cor}-daughter-\textsc{col}\\
\glt ‘He just got there and asked his daughters:’
\glt ‘Entrou, foi direto perguntar pra filharada dele:'
\footnotetext{We have not been able to do a morphological analysis of the word \textit{porẽsopega} ‘ask’, but it is clearly a complex word, where it is possible to identify the root \textit{pek} ‘call’. Monomorphemic words in Sakurabiat are usually shorter than four syllables.}
\z

\newpage 
\ea Aeke te ejatsi?\\[.3em]
\gll a-eke te ejat-si\\
     \textsc{q}-\textsc{dem.n} \textsc{foc} \textsc{2pl}-mother\\
\glt ‘“Where is your mother?”’
\glt ‘“Para onde foi a mãe de vocês?"'
\z 


\ea Osesi jãõmõ ka te ikãw taaga kit ara naat te kijkona taaga kit ara.\\[.3em]
\gll ose-si jãõ=bõ ka te ikão taaga kit at-a naat te ki-iko=na taaga kit at-a\\
     \textsc{1pl.excl}-mother \textsc{dem.dist}=\textsc{dat} move \textsc{foc} \textsc{dem.time} walking.palm seed get-\textsc{thv} \textsc{cop} \textsc{foc} \textsc{1pl.incl}-food=\textsc{vblz} walking.palm seed get-\textsc{thv}\\
\glt  ‘“Our mother went over there at that time to get walking palm's seed for us to eat, get walking palm's seed,” (they replied).’{\footnotemark}
\glt ‘“Nossa mãe foi por ali, buscar caroço de paxiúba pra nós comermos.”'
\footnotetext{The walking palm tree (\textit{Socratea exorrhiza}) is known as \textit{paxiúba} in Brazil. Its seeds are not edible.}
\z

\ea Poret ejarora{\footnotemark} ipegara taaga kit aratkʷa nõm pegat.{\footnotemark}\\[.3em]
\gll poret ejat-ot-a i-pek-a-ra taaga kit at-a-t-kʷa nop pegat\\
     now \textsc{2pl}-leave-\textsc{thv} \textsc{3sg}-call-\textsc{-thv-rep} walking.palm seed get-\textsc{thv}-\textsc{pst-pl.ev} \textsc{neg} \textsc{irr.fut}\\
\glt ‘“Then, go call her, it is no longer necessary to bring walking palm seed” (\textit{Arɨkʷajõ} told his daughters).’
\glt ‘Aí (\textit{Arɨkʷajõ} disse:) “Vão chamar ela, não era pra trazer mais semente de paxiúba, não."'
\addtocounter{footnote}{-1}
\footnotetext{Sakurabiat has two types of hortative constructions. The first type uses the special hortative verb \textit{soga} following the lexical verb. The other type of hortative construction, illustrated in this example, uses the verb \textit{ot} ‘leave, go’  prefixed by either first person plural or second person (plural) subject agreement, followed by the lexical verb.}
\stepcounter{footnote}
\footnotetext{The irrealis future morpheme \textit{pegat} seems to be a complex form that contains the morpheme \textit{pek} ‘\textsc{fut}' plus an allomorph of the past tense morpheme -(\textit{a})\textit{t}.}
\z 

\ea Kiopap ta eba jẽ ẽma kaareri imãã ke te kɨape, kieba mõtkʷa ke te sɨraamnã.{\footnotemark}\\[.3em]
\gll ki-opap ta eba jẽ eba kaat=eri i-ma-a ke te ki-ɨape kieba mot-kʷa ke te sɨraap=na\\
     \textsc{1pl.incl}-corn \textsc{dem.stand} \textsc{evid} \textsc{dem.sit} \textsc{evid} \textsc{dem}=\textsc{abl} \textsc{3sg}-make-\textsc{thv} \textsc{dem} \textsc{foc} \textsc{1pl.incl}-our.beverage tuber make-\textsc{pl.ev} \textsc{dem} \textsc{foc} \textit {massaco}=\textsc{vblz}\\
\glt ‘“Our corn is here, make our beverage from it, and prepare the manioc to make (our) \textit{massaco}.”’{\footnotemark}
\glt ‘“Nosso milho tá aqui pra fazer nossa chicha, e amassar macaxeira pra fazer massaco."'
\addtocounter{footnote}{-1}
\footnotetext {Two pieces of information are necessary here. First, the words \textit{ma} and \textit{mot(kwa)} are synonymous, both translate as ‘make, prepare, build’. The root \textit{mot} is generally used with the verb formative suffix \textit{-kwa} that is also a marker of event plurality. Secondly, the noun \textit{kieba} is a general word that can refer to any type of edible tuber, either manioc (yucca), sweet potato, \textit{cará}, etc.}
\stepcounter{footnote}
\footnotetext{\textit{Massaco} is a Portuguese word used to describe a dish that is made with cooked banana or yucca, pounded with a pestle. This dish is very popular among the Sakurabiat.}
\z 

  
\ea Ikʷaksoa te itagiat sɨrɨk=nẽ'ẽp pɨbot.\\[.3em]
\gll i-kʷak-so-a te i-tak-iat sɨrɨk=ne='ẽp pɨbot\\
     \textsc{3sg}-sound-see-\textsc{thv} \textsc{foc} \textsc{3sg}-daughter-\textsc{col} go.\textsc{pl.sbj}=?=\textsc{emph} arrive\\
\glt ‘They heard him, the daughters, then they left, went straight there and arrived (where their mother was).’
\glt ‘Escutaram o que ele falou, as filhas. Aí saíram foram até lá (onde a mãe estava).'
\z 

\ea Akʷa kɨp perek piora nããn kop.\\[.3em]
\gll akʷa kɨp perek piora naat kop\\
     \textit{cará}{\footnotemark} stick long dig \textsc{cop} \textsc{aux.mov}\\
\glt ‘She was digging for wild \textit{cará} tubers.’{\footnotemark}
\glt ‘Ela estava cavando cará do mato.'
\addtocounter{footnote}{-1}
\footnotetext{\textit{akwa} \textit{‘cará}' is a cultivated type of tuber. The wild, uncultivated \textit{cará} is called \textit{akwa kup perek} because is elongated (stick-like), unlike the cultivated one, which is more round.}
\stepcounter{footnote}
\footnotetext{This sentence is translated in the past tense, to agree with the rest of the text. However, the auxiliary form \textit{kop} is the present tense form; the past tense form would be \textit{koa}.}
\z

\ea Abitop{\footnotemark} epegarat, osi.\\[.3em]
\gll abi-top e-pek-a-ra-t o-si\\
     father-father \textsc{2sg}-call-\textsc{thv-rep-pst} \textsc{1sg}-mother\\
\glt ‘“Our father called you, mother.”’
\glt ‘“Nosso pai te chamou, mamãe."'
\footnotetext{This form is the special vocative for father for female egos. It combines the vocative \textit{abi} ‘my father' for male egos, and the referential stem \textit{-top} ‘father'.}
\z 

\ea Kɨape mã ke te kiopaberi ita kaat ikãw.\\[.3em]
\gll ki-ɨape ma ke te ki-opap=eri i-ta kaat ikão\\
     \textsc{1pl.incl}-beverage make \textsc{dem} \textsc{foc} \textsc{1pl.incl}-corn=\textsc{abl} \textsc{3sg-aux.stand} \textsc{quot} \textsc{dem.time}\\
\glt ‘“(Call her) to make our \textit{chicha}{\footnotemark} from our corn here, that (is what he) said at that time.”’
\glt ‘“(Chama ela) pra fazer chicha pra nós, do nosso milho que tá 
aqui.” Assim (ele disse) àquela hora.'
\footnotetext {\textit{Chicha} is a regional word in Brazilian Portuguese to refer to a fermented beverage. This beverage, which is very popular among several indigenous groups in Amazonia, can be made out of corn, yucca or any other kind of tuber.}
\z  

\ea Kieba mõtkʷa sɨraapnã kaat ikãw.\\[.3em]
\gll kieba mot-kʷa sɨraap=na kaat ikão\\
     tuber make-\textsc{pl.ev} \textit{massaco}=\textsc{vblz} \textsc{quot} \textsc{dem.time}\\
\glt ‘“To prepare manioc to make \textit{massaco},” that (is what he) said at that time.’
\glt ‘“Pra amassar macaxeira pra fazer massaco.” Assim ele disse àquela hora.'
\z 

\ea Erek tɨɨnã sitoabip etoabip tõẽn, ejattaɨbiat sara, aose igorerõp sete.\\[.3em]
\gll erek tɨɨ=na s-itoabip e-toabip tõet ejat-taɨp-iat sara aose igot-e-rõp sete\\
     speak \textsc{intj}=\textsc{cop} \textsc{3sg}-cultivated.field \textsc{2sg}-cultivated.field \textsc{dub} \textsc{2pl}-son-\textsc{col} pitied man possessor-?-\textsc{neg} \textsc{3sg}\\
\glt ‘She said: “Poor bastard, I doubt he has a crop. You poor children, he is a man who has nothing.”’  (Lit. ‘She said: “Pitied one, I doubt his crop (exists). Poor people of your children, this man has nothing.”’)
\glt ‘Aí ela falou: “Esse  coitado aí, plantação dele! Plantou nada! Coitados dos filhos de vocês, esse aí é homem que não tem nada."'
\z  
 

\ea Arẽm sɨrɨk nẽãrã te itagiat.\\[.3em]
\gll arẽp sɨrɨk neara te i-tak-iat\\
     then go.\textsc{pl.sbj} again \textsc{foc} \textsc{3sg}-daughter-\textsc{col}\\
\glt ‘They went again, his daughters.’
\glt ‘Foram embora de novo, as filhas dele.'
\z 

\newpage 
\ea Kaap tẽẽn te otsetsi.\\[.3em]
\gll kaap tẽet te ose-si\\
     \textsc{dem} only \textsc{foc} \textsc{1pl.excl}-mother\\
\glt ‘“Mom said just that.” (And, thus, the daughters told their father what their mother had said.)'
\glt ‘“Mamãe disse assim.” (Assim as filhas contaram pro pai o que a mãe falara.)'
\z 

  
\ea Ke ebõ te setaɨpkʷa paat te.\\[.3em]
\gll ke ebõ te se-taɨp-kʷa paat te\\
     \textsc{dem} really \textsc{foc} \textsc{3cor}-get.calm-\textsc{vblz} \textsc{fut}.3 \textsc{foc}\\
\glt ‘“It is like that now (she is angry at me), but she'll calm down."'
\glt ‘“Só agorinha que tá assim (brava comigo), vai se amansar.”'
\z

\ea Setoorekʷa mãjã ikoop sete.\\[.3em]
\gll se-toorekʷa maj-a i-koop sete\\
     \textsc{3cor}-laugh tell-\textsc{thv} \textsc{3sg}-\textsc{aux.mov} \textsc{3sg}\\
\glt ‘“She is still going to laugh,” he (said).’ (That is what \textit{Arɨkʷajõ} thought about his wife.)
\glt ‘“Ela ainda vai dar risada.” Ele disse.'
\z  

\ea Pia setoa arẽm te aramĩrã 'ibat nẽãrã.\\[.3em]
\gll pia se-to-a arẽp te aramira 'ip-a-t neara\\
     wait \textsc{3cor}-\textsc{aux.lie-thv} then \textsc{foc} woman come-\textsc{thv-pst} again\\
\glt ‘He stayed there waiting, then the woman came back again.’
\glt ‘Ele ficou esperando, aí a mulher chegou de novo.'
\z 
 

\ea Taibap sekẽrẽkʷa saraka te Pãrãrẽkotsa.\\[.3em]
\gll taib-ap se-e-kẽrẽ-kʷa sara-ka te Pãrarekosa\\
     gentle-\textsc{neg} \textsc{3cor}-\textsc{intrvz}-angry-\textsc{vblz}	 pitied-\textsc{vblz} \textsc{foc} Pãrarekosa\\
\glt ‘She was angry, poor \textit{Pãrarekosa}.’ 
\glt ‘Estava brava, coitada dela, a \textit{Pãrarekosa}.'
\z 

\newpage 
\ea Poget kop peropka pe akʷa kɨp perek\\[.3em]
\gll poget kop perop-ka pe=akʷa kɨp perek\\
     standing \textsc{aux.mov} cooked-\textsc{vblz} \textsc{obl}=\textit{cará} stick long\\
\glt ‘She stayed there, and then went to cook wild \textit{cará} tubers.’
\glt ‘Ficou por aí, e foi cozinhar cará do mato.'
\z 

\ea Kaa kaat ebõ nã sekoa ɨmẽ.\\[.3em]
\gll ko-a kaat ebõ=na se-ko-a 'ɨme\\
     ingest-\textsc{thv} \textsc{dem} really=\textsc{vblz} \textsc{3cor-aux.mov-thv} dark\\
\glt ‘She ate, and stayed there the way she was (angry), until it got dark.’
\glt ‘Comeu, ficou por ali assim (brava mesmo), até que escureceu.'
\z 

\ea Era kʷirik poret.\\[.3em]
\gll et-a kʷirik poret\\
     sleep-\textsc{thv} clear now\\
\glt ‘Then it dawned.’{\footnotemark}
\glt ‘Aí amanheceu.'
\footnotetext {\textit{era kʷirik}, which literally means ‘sleep and clear’ is the expression used to announce that the day dawned, and also to count how many nights/days have passed, in an iterative way. Thus, to say something like ‘two days after’ one would say \textit{era kʷirik} \textit{era kʷirik}.}
\z 

\ea Ejarɨape kaabõpkʷa{\footnotemark} kot.\\[.3em]
\gll ejat-ɨape	 kaabõp-kʷa kot\\
     \textsc{2pl}-beverage bless;heal-\textsc{pl.ev} \textsc{im.fut}\\
\glt ‘(After that \textit{Arɨkʷajõ} came and said to his children:) ‘“I will cure your beverage.”’
\glt ‘(Aí \textit{Arɨkʷajõ} veio e disse:) “Vou curar a chicha de vocês.”'
\footnotetext{The consultant that was helping with translation and morphemic analysis explained that the verb form should be \textit{kaabõa}, not \textit{kaabõp}, but he could not explain why. We chose to keep the form that was given by the narrator.}
\z 

\ea Ko soga!\\[.3em]
\gll ko soga\\
     ingest \textsc{hort}\\
\glt ‘“You can drink! Drink!”’
\glt ‘“Pode beber! Bebe!”'
\z 

\ea Ejatjãj sɨgɨka kotke{\footnotemark} õn.\\[.3em]
\gll ejat-jãj sɨgɨ-ka kot=ke=õt\\
     \textsc{2pl}-tooth drop-\textsc{vblz} \textsc{im.fut}=\textsc{quot}=\textsc{1sg}\\
\glt ‘“I am going to make your teeth fall out.”’
\glt ‘“Eu vou fazer o dente de vocês cair tudo.”'
\footnotetext{The first and second person desiderative construction is formed with a combination of the immediate future morpheme plus the quotative morpheme for first and second persons (\textit{kot+ke} ‘\textsc{im.fut + quot}'= \textsc{desiderative}).}
\z 

  
\ea Kaabõ'ẽm sɨgɨ tejatjãj.\\[.3em]
\gll kaap õ'ẽp sɨgɨ te-jat-jãj\\
     \textsc{dem} already drop \textsc{3sg-col}-tooth\\
\glt ‘It happened really that way, their teeth fell out.’
\glt ‘Fez mesmo, caiu todos os dentes delas.'
\z  

\ea Pẽrãm te otagiat ejariko pek.\\[.3em]
\gll pẽt-ap te o-tak-iat e-jat-iko pek\\
     hard-\textsc{nmlz} \textsc{foc} \textsc{1sg}-daughter-\textsc{col} \textsc{2sg}-\textsc{col}-food \textsc{fut}\\
\glt ‘“It won't be hard, your food, my daughters.”’ (This sentence continues the speech of \textit{Arɨkʷajõ} to his children.)
\glt ‘“Não vai ser duro, minhas filhas, a comida de vocês."'
\z 


\ea Kaanã'ẽp poret kap kapnã te kijãj pogeri poret.\\[.3em]
\gll kaat=na='ẽp poret ko-ap ko-ap=na te ki-jãj poget=i poret\\
     \textsc{dem}=\textsc{vblz}=\textsc{emph} now  ingest-\textsc{nmlz} ingest-\textsc{nmlz}=\textsc{vblz} \textsc{foc} \textsc{1pl.incl}-tooth standing=\textsc{aux.pl} now\\
\glt ‘They stayed that way, our teeth, in order for us to eat.’  (Lit. ‘They became that way in order to be (our) eating instrument, our teeth stayed that way.')
\glt ‘Ficou assim mesmo pra ser aquilo com que se come, os nossos dentes ficaram assim.'
\z 

\newpage 
\ea Kaannã{\footnotemark} te te kijãj ipẽnnã kenõm.\\[.3em]
\gll kaat=na te te ki-jãj i-pẽt=na ke nop\\
     \textsc{dem}=\textsc{vblz} really \textsc{foc} \textsc{1pl.incl}-tooth \textsc{3sg}-hard=\textsc{vblz} \textsc{dem} \textsc{neg}\\     
\footnotetext{One of the morphosyntactic strategies to express causal adverbial clauses in Sakurabiat is to use a derived verb phrase formed by the anaphoric demonstrative \textit{kaat} with the verbalizer \textit{nã}  (see \citet{Galucio2011}, for a thorough discussion of adverbial clauses in Sakurabiat.)}
\glt ‘That's why our teeth are not hard,’ (Lit. ‘In being that way really, our teeth are not hard.')
\glt ‘Por isso que o nosso dente (de hoje em dia) nao é duro.'
\z 

 
\ea Sesɨgɨka kʷaap tẽẽn.\\[.3em]
\gll se-sɨgɨ-ka kʷaap tẽet\\
     \textsc{3cor}-drop-\textsc{vblz} \textsc{hab} only\\
\glt ‘They just drop out (and grow again).’
\glt ‘Só cai (e nasce de novo).'
\z 

\ea Kɨrɨt sĩit jãj etsɨgɨka.\\[.3em]
\gll kɨrɨt sĩit jãj e-sɨgɨ-ka\\
     child \textsc{dim} tooth \textsc{intrvz}-drop-\textsc{vblz}\\
\glt ‘(That's why) kids' teeth drop out.’
\glt ‘(Por isso que agora) dente de criança cai tudo.'
\z 


\ea Kekʷaap nããm{\footnotemark} tẽẽn.\\[.3em]
\gll Ke kʷaap=na-ap tẽet\\
     \textsc{dem} \textsc{hab}=\textsc{vblz}-\textsc{nmlz} only\\
\glt ‘It is always just that way.’ (‘The cycle keeps repeating itself, it is always like that.’)
\glt ‘É todo tempo só assim.'
\footnotetext{It is possible to delete the nominalizer morpheme \textit{-ap}, with no apparent change in meaning. The consultant said that another way of saying the same thing was \textit{kekwaapnã tẽet}.}
\z 

\ea ke te kijãj.\\[.3em]
\gll ke te ki-jãj\\
     \textsc{dem} \textsc{foc} \textsc{1pl.incl}-tooth\\
\glt ‘That's how our teeth are.’
\glt ‘É assim nosso dente.'
\z 

\ea Kaabese nããn aapi õtsop te atsitsi.\\[.3em]
\gll kaap=ese naat aapi õ-sop te asisi\\
     \textsc{dem}=\textsc{loc} \textsc{cop} crop.seed \textsc{caus}-see \textsc{foc} corn\\
\glt ‘That is how (they) found corn seed.’
\glt ‘Foi assim que acharam semente de milho.'
\z 

 
\ea Kʷeet piro kiiko pek kɨ̃rẽp.\\[.3em]
\gll kʷeet piro ki-i-ko pek kɨ̃rep\\
     thing exist \textsc{1pl.incl}-\textsc{obj.nmlz}-ingest \textsc{fut} now\\
\glt ‘And that's how it appeared the things that we eat.’
\glt ‘Assim que apareceram as coisas de comer.'
\z 

\ea Kaap te eba nããriat.\\[.3em]
\gll kaap te eba naat=iat\\
     \textsc{dem} \textsc{foc} \textsc{evid} \textsc{cop}=\textsc{rem.pst}\\
\glt ‘(But) that's how it was at those times.’
\glt ‘Mas era assim antigamente.'
\z 


\ea Kõnkõrãrõpnã arobõ te atsitsi nããn eteet.\\[.3em]
\gll kõtkõra-rop=na arop-õ te asisi naat eteet\\
     cicada-\textsc{neg}=\textsc{vblz} thing-\textsc{neg} \textsc{foc} corn \textsc{cop} \textsc{hyp}\\
\glt ‘If it were not for \textit{Kõtkõra}, there would be nothing, no corn. (All edible things were first planted by \textit{Kõtkõra}, who was also a shaman).'
\glt ‘Se não fosse \textit{Kõtkõra}, não tinha nada, não tinha milho, não. (Tudo foi \textit{Kõtkõra} quem plantou, ele era \textit{kwamoa} (‘pajé') também.)' 

\z 

\ea Asisirõp.\\[.3em]
\gll asisi-rõp\\
     corn-\textsc{neg}\\
\glt ‘(There would be) no corn.’
\glt ‘Não tinha milho.'
\z 

\ea Arobõ te piro tapsɨt akʷa kʷaako piroap.\\[.3em]
\gll arop-õ te piro tapsɨt akʷa kʷaako piro-ap\\
     thing-\textsc{neg} \textsc{foc} exist yucca \textit{cará} sweet.potato exist-\textsc{neg}\\
\glt ‘There wouldn't be anything, no yucca, no \textit{cará} tuber, no sweet potato.’
\glt ‘Não tinha nada, nem mandioca, nem cará, nem batata, não tinha nada.'
\z 

\ea Arɨkʷajõ ekap sigot nẽnõã.\\[.3em]
\gll arɨkʷajõ ekap s-igot ne=no-a\\
     Arɨkʷajõ \textsc{sbjv} \textsc{3sg}-possessor \textsc{cop}=\textsc{neg}-\textsc{thv}\\
\glt ‘If it were up to \textit{Arɨkʷajõ}, he wouldn't own anything.’
\glt ‘\textit{Arɨkʷajõ} não tinha nada.'
\z 

\ea Sigot tɨɨɨ nẽnõã.\\[.3em]
\gll s-igot tɨɨ ne=no-a\\
     \textsc{3sg}-possessor \textsc{intj} \textsc{cop}=\textsc{neg}-\textsc{thv}\\
\glt ‘He possessed nothing.’
\glt ‘Ele não tinha nada, não.'
\z 


\ea Kʷai mariko kɨpkɨba 'a mariko{\footnotemark} sete.\\[.3em]
\gll kʷai mat i-ko kɨpkɨba 'a mat i-ko sete\\
     stone ? \textsc{obj.nmlz}-ingest tree fruit ? \textsc{obj.nmlz}-ingest \textsc{3sg}\\
\glt ‘He only eats stone and fruit (as if he were not human).’ (Lit. ‘Stone is what he eats, and fruit is what he eats.’)
\glt ‘Comida dele é pedra, é fruta de pau (como se não fosse gente).'
\footnotetext{The expression N \textit{mariko} is used when you want to refer to something that is someone's preferred choice of food. For instance, \textit{Kwe mariko õrõn.} ‘I only like to eat game meat.’/ ‘I only eat game meat.’ (\textit{kwe} ‘game meat’ – \textit{mat} (?) – \textit{i-ko} ‘\textsc{obj.nmlz}-ingest’ – \textit{õr-õn} \textsc{1sg-emph}).}
\z 

\ea Aose eteet.\\[.3em]
\gll aose eteet\\
     man \textsc{hyp}\\
\glt ‘If it were not for this man (\textit{Kõtkõrã}), (there would be nothing.)’
\glt ‘Se não fosse esse homem (\textit{Kõtkõra}), (não teria nada mesmo, não.)'
\z 

\ea Õẽm.\\[.3em]
\gll õẽp\\
     already\\
\glt ‘It's finished.’
\glt ‘Acabei.'
\z 

\begin{figure}
\includegraphics[width=\textwidth]{figures/mekens-illustration10.png}
  \caption{The old and the newly acquired crops. Illustration by Lidia Sakyrabiar and Ozélio Sakyrabiar}
\end{figure}

\newpage 
\section*{Acknowledgments}
Galucio would like to acknowledge the continued support of the Sakurabiat people, especially the elders who shared the wonders of their traditional narratives, and her collaborators in this article. A special token of gratitute goes to the late Mercedes Guaratira Sakyrabiar and Manoel Ferreira Sakyrabiar, \textit{in memoriam}. Manoel was a very good friend, and an enthusiast and great supporter of the study of Sakurabiat. Dona Mercedes was not only a supporter but also a grandmother figure in all the visits to the Sakurabiat village. They are both greatly missed. We thank Ana Carolina Alves for helping with the audio files, Ellison Santos for helping with the figures, and Hein van der Voort for facilitating permission of the map.
The present work was carried out with the support of CNPq, the National Council of Scientific and Technological Development, Brazil (Process number 308286/2016-2).

\section*{Non-standard abbreviations}

\begin{tabularx}{.5\textwidth}{lQ}
\textsc{col } & collective \\
\textsc{cor } & co-referential \\
\textsc{dub } & dubitative \\
\textsc{dem.time } & temporal demonstrative \\
\textsc{emph } & emphatic \\
\textsc{ev } & event \\
\textsc{evid } & evidential \\
\textsc{hab } & habitual \\
\textsc{hort } & hortative \\
\textsc{hyp } & hypothetical \\
\end{tabularx}
\begin{tabularx}{.45\textwidth}{lQ}
\textsc{im } & immediate \\
\textsc{intrvz } & intransitivizer \\
\textsc{lie } & lying \\
\textsc{mov } & moving \\
\textsc{nassert } & non-assertive \\
\textsc{rep } & repetitive \\
\textsc{sit } & sitting \\
\textsc{stand } & standing \\
\textsc{thv } & thematic vowel \\
\textsc{vbzl } & verbalizer \\
\end{tabularx}

{\sloppy
\printbibliography[heading=subbibliography,notkeyword=this]
}
\end{document}
