\documentclass[output=paper,
modfonts,nonflat
]{langsci/langscibook} 
\author{Cédric Yvinec\affiliation{CNRS/Mondes Américains, Paris, France}%
\lastand Agamenon Gamasakaka Suruí%
}%
\lehead{C. Yvinec \& Agamenon Gamasakaka Suruí}
\title{Suruí of Rondônia}
\ourchaptersubtitle{Ana omamõya \~{G}oxoraka ã}
\ourchaptersubtitletrans{‘This is the way my grandfather killed a Zoró’}
% \storytitle{Ana omamõya \~{G}oxoraka ã}
% \translatedstorytitle{‘This is the way my grandfather killed a Zoró’}
% \abstract{noabstract}
\ChapterDOI{10.5281/zenodo.1008795}

\maketitle

\begin{document}

\section{Introduction} 

This story relates one of the numerous skirmishes in the continuously warlike relations between the Suruí of Rondônia (or, according to their autodenomination, \textit{Paiter}) and their indigenous neighbors. It is representative of the Suruí narrative genre. Suruí is a language of the Mondé family of the Tupian stock. The Suruí live in the Sete de Setembro Indigenous Land, on the border between the Brazilian states of Rondônia and Mato Grosso, near the city of Cacoal. The speakers of Suruí now number about 1,200 individuals; this population has increased steadily since the 1970s, when measles, flu and tuberculosis epidemics broke out after their first peaceful contact with the Brazilian society in 1969, causing a demographic crisis. Although the Suruí traditionally lived in just one or two villages, they are now scattered among more than 20 settlements along the boundaries of their land, which remains an island of rainforest surrounded by cattle ranchers (see \figref{fig:surui:1}). Slash-and-burn horticulture and hunting have been replaced by coffee farming, illegal logging, and environmentalist projects. War with neighboring Indian groups and White settlers have ceased and numerous matrimonial bonds now tie the Suruí to their former enemies. Thus, although the events narrated here evoke a world vividly experienced by the generations born before the 1960s, this past is very different from the daily life of the young adults today, who form a large part of the audience of such narratives.\footnote{For general ethnographic information on the Suruí, see \citet{Mindlin1985,Mindlin1996} and \citet{Yvinec2011}. For information on the various ways of narrating past events among the Suruí, see \citet{Yvinec2016}.}

\begin{figure} [t]
\includegraphics[width=0.66\textwidth]{figures/surui-map-present.jpg}
\caption{Present location of the Suruí (Map: C. Yvinec)}
\label{fig:surui:1}
\end{figure}

The story is 9 minutes long. It was narrated in June 2013 by Agamenon \~{G}amasakaka Suruí, a fifty-year-old man, one of the most powerful and respected men in the village of Lapetanha (see Figure 2). It was told to me at night, after I explicitly asked to record it. That very afternoon, Agamenon had spontaneously narrated it to his two wives, surrounded by several of his sons, daughters and nephews, while we were enjoying a break in the tedious work of picking coffee berries. The aim of the narration was both to celebrate the ancestor of his political faction and to entertain the audience, especially by singing a beautiful war song. When closely related, politically friendly adults get together, for example while sitting around a fire at night or having a rest during a collective work, they often exchange such stories among themselves. Because they are narrated over and over, their content is rarely entirely unknown to the audience. 

\def\oldIntextsep{\the\intextsep}
\setlength{\intextsep}{0pt}
\begin{wrapfigure}{r}{0.31\textwidth}
  \begin{figure}[H]
  \includegraphics[width=\textwidth]{figures/agamenon1.jpg}
  \caption*{Figure 2: The narrator, Agamenon \raisebox{-.02mm}[0mm]{\~{G}}amasakaka Suruí (Photo C. Yvinec)}
  \end{figure}
\end{wrapfigure}
\setlength{\intextsep}{\oldIntextsep}

\addtocounter{figure}{1}

Here, Agamenon evokes an attack on a neighboring Zoró village -- a Mondé-speaking population of north-western Mato Grosso who were the Suruí’s preferred enemies -- that took place before the birth of all living Suruí, probably in the 1920s. The narration culminates with the song one of the warriors composed in celebration of his deeds. Indeed, it is through such songs that memory of historical events is passed down among the Suruí.

\largerpage
Several other stylistic features are worth noting. For the most part, the story is composed of embedded quotations of successive narrators of the event, so that it is rendered in direct first person speech, including the alleged inner discourse (thoughts) of actors. Thus, the narration paints a vivid picture of the events, intensified by the extensive use of ideophones. The evidential status of each embedded discourse is systematically marked, either as witnessed evidence (narration heard by the speaker from his father) or as non-witnessed evidence (narration heard by the latter from his own father, the protagonist of the event). However, in a kind of introductory summary (lines 5-8), and then again in the conclusion (lines 55--69), the narrator refers to the nature and general behavior of the main character, his grandfather.\footnote{The recording situation and the absence of a Suruí interlocutor prone to ask questions and open new stories led this narrator to develop these considerations further than most narrators would do in conversational situations, and gave the story a unity that spontaneously-occurring narrations do not often have.} Paradoxically, these parts of the story are marked as witnessed evidence, although the narrator was born years after his grandfather's death. Such deletion of non-witnessed evidentiality marking in the course of a story is common in various narrative genres, including myths. It is often due to a prioritized focus on the sequence of events rather than on the source of information. Here, on the other hand, the contrast is between a particular event (identified by a song), which needs to have its non-witnessed evidential status restated in each sentence of the description because it is painted as if it had been lived by the speaker, and general inferences that can be construed as witnessed on the basis of the lifelike description of the event. Finally, the difference between Suruí historical and mythological narratives needs to be pointed out: whereas myths are attributed to an indeterminate group of anonymous speakers, the chain of narrators of historical events is clearly established throughout the story.

The transcription convention used here is mostly similar to that developed by the SIL missionaries with a group of Suruí for textbooks and Bible translation. All orthographic symbols are to be pronounced as in Brazilian Portuguese, except for the following: \textit{\~{g}} is a velar nasal [\ng] ; \textit{h} indicates that the preceding vowel is long; \textit{s} is a voiceless velar [x] or dental fricative [θ]; and \textit{u} is a closed front rounded vowel [y]. The tilde, acute accent, and circumflex denote nasalization, high tone on oral vowels, and high tone on nasalized vowels, respectively. Vowels without any diacritic mark are oral vowels with low (or undetermined) tone. Suruí consonants vary in initial and final word positions, according to the preceding or following morpheme, especially in possessive noun phrases and in the construction of the object-verb group. The most frequent variations are 
$p\textasciitilde m, 
k\textasciitilde\~{g}, 
s\textasciitilde l\textasciitilde x, 
t\textasciitilde tx\textasciitilde n,
d\textasciitilde j,
m\textasciitilde\varnothing$, and
$w\textasciitilde\varnothing$. They are reproduced in the writing convention.

Suruí typological features can be summed up as follows. The basic word orders are genitive-noun; noun-adjective; and object-verb. The subject has no strictly determined position, but tends to precede the object-verb group, and is marked by aspectual and/or evidential suffixes. There are two classes of nouns: obligatorily possessed and non-obligatorily possessed; the latter can nonetheless occur in a possessive construction, marked by the possessive prefix \textit{-ma-}, positioned between the genitive and the noun. There are two class of verbs: transitive and intransitive. The latter can have a reflexive pronoun (which only differs from the regular pronoun in the third person) in the object position, thus aligning subjects of intransitive verbs with objects of transitive verbs. Indirect objects can appear with both classes of verbs and are marked by dative, benefactive, or ablative suffixes. Evidentiality is marked by suffixes on the subject and/or by sentence-final markers.\footnote{On some occasions (lines 8; 13; 60), sentence parsing contradicts the sentence-final markers, because prosody shows that the speaker extended and corrected his utterance after the sentence-final marker.}  Evidentiality can be witnessed, non-witnessed, or non-declarative (a subclass of non-witnessed).\footnote{Only preliminary studies are available on Suruí: on phonology, see \citet{LacerdaGuerra2004}; on syntax, see \citet{Bontkes1985,Bontkes1985}, and \citet{VanderMeer1985}, as well as the primary author's own work \citep[679--691]{Yvinec2011}.}


\newpage 
\section{Ana omamõya \~{G}oxoraka ã}
\translatedtitle{‘This is the way my grandfather killed a Zoró’}\\
\translatedtitle{‘Foi assim que meu avô matou um Zoró’}\footnote{Recordings of this story are available from \url{https://zenodo.org/record/997449}}


\ea Nem, a olobde merema o\~{g}ay ma e. \\[.3em]
\gll nem a o-sob-de pere-ma o-ka ma e \\
\textsc{intj} \textsc{dem.prox} \textsc{1sg}-father-\textsc{wit} \textsc{iter}-do{\footnotemark} \textsc{1sg-dat} \textsc{prf.pst} \textsc{sfm.wit}\\
\glt ‘Well, here is what my father recounted to me.’\\
\glt ‘Meu pai me contou isso.’\\
\footnotetext{The suffix \textit{-ma} used as a verb can have many meanings, including ‘to say’ or ‘to explain’.}
\z 

\ea Nem, a, ““ana oiyã” olobiyã,” olobesesob, denene.\\[.3em]
\gll nem a a-na o-ya o-sob-ya o-sob-e-sob \(\varnothing\)-de-ee-na-ee-na-e{\footnotemark}\\
\textsc{intj} \textsc{dem.prox} \textsc{dem.prox-foc} \textsc{1sg-nwit} \textsc{1sg}-father-\textsc{nwit} \textsc{1sg}-father-\textsc{nmlz}-father \textsc{3sg-wit-endo-foc-endo-foc-sfm.wit}\\
\glt ‘He said this, about my father’s father: “I heard that my father said: “I did this, they say.””’{\footnotemark}\\
\glt ‘Ele falou assim, do pai de meu pai: “Eu ouvi que meu pai falou assim: “Eu fiz isso, pessoas falam.””’
\fnminus
\footnotetext{The combination \textit{ee-na}, \textsc{endo-foc}, is an adverbial locution that is pervasive in narrative speech. Its meaning is very loose and almost expletive: ‘thus’, ‘this way’, ‘like this’. It emphasizes the inner links of the speech, contrasting with \textit{i-na}, \textsc{exo-foc}, which occurs in conversations that refer to physically present objects.}
\fnplus
\footnotetext{\label{fn:surui:multquot}From this point on, the whole narration is a quotation of Agamenon's father quoting a narration by his own father, except for a few sentences (lines 5-9; 43-47; 49-50; 55-64). This embedding of quoted speech is marked at the end of almost every sentence. However, we do not reproduce it systematically in the free translation in order to lighten it a little; quotation marks only will indicate the levels of embedding. The evidential value of each level will not be reproduced either, since it remains unchanged throughout the narration: Agamenon's father's speech was witnessed (\textit{de}, \textsc{wit}), his grandfather's was not (\textit{ya}, \textsc{nwit}).}
\z
 
\newpage 
\ea ““Nem, olobaka \~{G}oxoriyã” iyã” de.\\[.3em]
\gll nem o-sob-aka \~{G}oxor{\footnotemark}-ya i-ya \(\varnothing\)-de{\footnotemark}\\
\textsc{intj} \textsc{1sg}-father-kill Zoró-\textsc{nwit} \textsc{3sg-nwit} \textsc{3sg-wit}\\
\fnminus
\footnotetext{The word \textit{\~{G}oxor}, Zoró, is a lexicalization of the locution \textit{lahd\~{g}oesor}, \textit{lahd-koe-sor}, Indian.enemy-language-ugly, ‘enemy whose language is difficult to understand’ -- but not incomprehensible, by contrast with other neighbors, either from the Mondé family (Cinta-Larga) or not (Kawahib, Nambikwara). The Portuguese ethnonym “Zoró" is a corruption of the Suruí word. At the time of the narrated events, \textit{\~{G}oxor} referred to a single population, from which are descended the two ethnic groups nowadays called Zoró and Gavião of Rondônia, who split off in the 1940s, when the latter entered in contact with rubber tappers, while the former remained in voluntary isolation until 1977 (see \figref{fig:surui:3}). For ethnohistorical information about the Zoró, see \citet{Brunelli1987}.}
\fnplus
\footnotetext{Quotations are marked in Suruí by an isolated subject, that is, a noun or pronoun (or the absence of it, for third person singular) with an aspectual and/or evidential suffix, to which no verbal group corresponds. Moreover, an indirect object, specifying the addressee can appear after the subject. These markers usually appear after the quoted speech and can be strung together to indicate embedding of quotations, as we see here.}
\glt ‘““Well, a Zoró killed my father.””’\\
\glt ‘““Um Zoró matou meu pai.””’\\
\z

\ea ““Eebo oya okãyna oladeka sona olobepika sona yã” olobiyã” olobde.\\[.3em]
\gll  ee-bo{\footnotemark} o-ya o-kãyna o-sade-ee-ka{\footnotemark} sona o-sob-wepika sona a o-sob-ya o-sob-de\\
\textsc{endo-advers} \textsc{1sg-nwit} \textsc{1sg}-grow \textsc{1sg-prog.sim-endo-dat} often \textsc{1sg}-father-avenge often \textsc{sfm.nwit} \textsc{1sg}-father-\textsc{nwit} \textsc{1sg}-father-\textsc{wit}\\
\glt ‘““So as I grew up, I avenged my father many times.””’\\
\glt ‘““Então, quando eu cresci, vinguei meu pai muitas vezes.””’\\
\fnminus
\footnotetext{In Suruí narrative discourse, almost every sentence is introduced by an adverbial locution that connects it to the preceding one: \textit{ee-bo} (\textsc{endo-advers}), ‘and’, ‘so’, or ‘but’; \textit{ee-te} (\textsc{endo-adv}), ‘then’, ‘indeed’, ‘and’; \textit{ee-tiga-te} (\textsc{endo-sim-adv}), ‘then’, ‘at that time’; \textit{a-yab-} (\textsc{dem.prox-endo}), ‘and this’, ‘and the latter’. It is difficult to give a uniform translation of those discourse-linking expressions.}
\fnplus
\footnotetext{The combination \textit{ee-ka}, \textsc{endo-dat}, functions as a postposition that subordinates the preceding clause as an indirect object of the verbal group of the main clause. Its meaning is either temporal or causal.}
\z

\newpage 
\ea Eebo dena sona \~{G}oxoraka akah sone.\\[.3em]
\gll ee-bo \(\varnothing\)-de-ee-na sona \~{G}oxor-aka a-kah sona-e.\\
\textsc{endo-advers} \textsc{3sg-wit-endo-foc} often Zoró-kill \textsc{3.refl}-go often-\textsc{sfm.wit}\\
\glt ‘So he repeatedly set out to kill some Zoró.’\\
\glt ‘Então ele foi matar o Zoró muitas vezes.’\\
\z

\begin{figure}[b]
% \includegraphics[width=0.4\textwidth]{figures/surui-map-past.jpg}
\includegraphics[width=0.55\textwidth]{figures/surui-map-past.pdf}
\caption{Approximate location of the Suruí, Zoró, and neighboring groups of the Machado-Roosevelt interfluve in the first half of the 20th century. At that time, the stations of telegraph line built by Rondon in 1914 were the only permanent colonial presence in that region. (Map: C. Yvinec)}
\label{fig:surui:3}
\end{figure}

\newpage 
\ea Ayab\~{g}oy xibebiga, ikar iperedena, ihbahba\~{g}a iperedena, eyab\~{g}ar iperedekena.\\[.3em]
\gll a-yab-koy xi-ibeb-iga i-kar i-pere-de-na ihbahb-ma\~{g}a i-pere-de-na ee-yab-kar i-pere-de-kah-ee-na\\
\textsc{dem.prox-endo}-to \textsc{3sg}-track-pick \textsc{3sg}-search \textsc{3sg-iter-wit-foc} canoe-make \textsc{3sg-iter-wit-foc} \textsc{endo-endo}-search \textsc{3sg-iter-wit}-go-\textsc{endo-foc}\\
\glt ‘He repeatedly searched for them and followed their tracks, he repeatedly made a canoe and set out in search of some of them.’{\footnotemark}\\
\glt ‘Ele o procurava e seguia o rastro dele, muitas vezes ele construiu uma canoa e saiu para procura-lo.’\\
\footnotetext{Suruí uses the third person singular pronoun (\textit{xi-} or \textit{i-}) to refer to the targeted enemy, but this does not mean that the Suruí warrior would have gone searching for a particular individual among the Zoró, the killer of his father. Any individual supposedly belonging to that ethnic group would be a suitable victim, and one death was still an ambitious goal for a raid. By contrast, using the third person plural pronoun (\textit{ta-}) would have implied that the target was the whole ethnic group -- but the Zoró lived in a more scattered fashion than the Suruí did. We use the plural in the free translation when it is more relevant in English.}
\z

\ea Ayabmi aor te, ayabmi akah ihbahbtar ena te makaomi ikar.\\[.3em]
\gll a-yab-pi a-or te{\footnotemark} a-yab-pi a-akah ihbahb-tar ee-na te ma-kao-pi i-kar\\
 \textsc{dem.prox-endo-abl} \textsc{3.refl}-come \textsc{adv} \textsc{dem.prox-endo-abl} \textsc{3.refl}-go canoe-on \textsc{endo-foc} \textsc{adv} other-dry.season-\textsc{abl} \textsc{3sg}-search\\
\glt ‘Afterward he came back, and after that, during the next dry season, he would set out again on a canoe in search of some of them.’\\
\glt ‘Depois, ele voltou, e depois disso, na próxima estação seca, ele saía de novo numa canoa para procura-lo.’\\
\footnotetext{The adverbial particle \textit{te} is almost an expletive. It is probably a weak form of the intensive adverbial particle \textit{ter}, and often has only a prosodic function.}
\z
 
\ea Enaitxa te iperedena sona \~{G}oxorkar ena sone, mater e, asobaka deke, omamõperedenene.\\[.3em]
\gll ee-na-itxa te i-pere-de-na sona \~{G}oxor-kar ee-na sona-e mater e a-sob-aka \(\varnothing\)-de-ee-ka-e o-ma-amõ-pere-de-na-ee-na-e\\
\textsc{endo-foc}-with \textsc{adv} \textsc{3sg-iter-wit-foc} often Zoró-search \textsc{endo-foc} often-\textsc{sfm.wit} long.ago \textsc{sfm.wit} \textsc{3.refl}-father-kill \textsc{3sg-wit-endo-dat-sfm.wit} \textsc{1sg-poss}-grandfather-\textsc{iter-wit-foc-endo-foc-sfm.wit}\\
\glt ‘Thus, my grandfather was always in search of some Zoró, long ago, because one of them had killed his father.’\\
\glt ‘Assim, meu avô sempre procurava o Zoró, há muito tempo, porque aquele tinha matado o pai dele.’\\
\z

\largerpage
\ea Ena olobdena iwema.\\[.3em]
\gll ee-na o-sob-de-ee-na iwe-ma\\
\textsc{endo-foc} \textsc{1sg}-father-\textsc{wit-endo-foc} \textsc{dem.exo}-do\\
\glt ‘This is what my father said.’\\
\glt ‘Meu pai falou assim.’\\
\z

\ea ““Eebo oya xikin ã” iyã” de.\\[.3em]
\gll ee-bo o-ya xi-ikin a i-ya \(\varnothing\)-de\\
\textsc{endo-advers} \textsc{1sg-nwit} \textsc{3sg}-see \textsc{sfm.wit} \textsc{3sg-nwit} \textsc{3sg-wit}\\
\glt ‘He said: “[My father] said: “And once I saw him.””’ \\
\glt ‘Ele disse: “[Meu pai] disse: “Aí uma vez eu o vi.””’ \\
\z

\ea ““Eebo oya olob\~{g}armeyitxa yã” iyã” de.\\[.3em]
\gll ee-bo o-ya o-sob-karmey-itxa a i-ya \(\varnothing\)-de\\
\textsc{endo-advers} \textsc{1sg-nwit} \textsc{1sg}-father-younger.sibling-with \textsc{sfm.wit} \textsc{3sg-nwit} \textsc{3sg-wit}\\
\glt ‘““I was with my father’s younger brother.””’\\
\glt ‘““Eu estava com o irmão mais novo de meu pai.””’\\
\z

\ea ““Eebo oilud ena tar ã.””\\[.3em]
\gll ee-bo o-oilud ee-na tar ã\\
\textsc{endo-advers} \textsc{1sg}-young \textsc{endo-foc} \textsc{prf} \textsc{sfm.nwit}\\
\glt ‘““I was already a young man.””’{\footnotemark}\\
\glt ‘““Eu já estava moço.””’\\
\footnotetext{Men are \textit{oilud} when they are young adults, still single, approximately between 17 and 25 years old.}
\z

\ea ““Ete oya \~{G}oxoribebepemaã tar ã” iyã” de, ““mixa\~{g}ataga.””\\[.3em]
\gll ee-te o-ya \~{G}oxor-ibeb-e-pe-maã tar a i-ya \(\varnothing\)-de mixa\~{g}-mataga\\
\textsc{endo-adv} \textsc{1sg-nwit} Zoró-track-\textsc{nmlz}-path-take \textsc{prf} \textsc{sfm.nwit} \textsc{3sg-nwit} \textsc{3sg-wit} night-go.through\\
\glt ‘““That time I had followed the track of a Zoró through the night.””’\\
\glt ‘““Aquela vez, eu tinha seguido o rastro do Zoró durante a noite.””’\\
\z

\ea ““Oker õm a.””\\[.3em]
\gll o-ker õm a\\
\textsc{1sg}-sleep \textsc{neg} \textsc{sfm.nwit}\\
\glt ‘““I had not slept at all.””’\\
\glt ‘““Eu não tinha dormido nada.””’\\
\z
 
\ea ““Mokãyxibotorera oya xibebaã tar ã, xipemaã tar ã” iyã” de.\\[.3em]
\gll mokãy-xibo-tor-wera o-ya xi-ibeb-maã tar a xi-pe-maã tar a i-ya \(\varnothing\)-de\\
fire-flame-carry-walk \textsc{1sg-nwit} \textsc{3sg}-track-take \textsc{prf} \textsc{sfm.wit} \textsc{3sg}-path-take \textsc{prf} \textsc{sfm.nwit} \textsc{3sg-nwit} \textsc{3sg-wit}\\
\glt ‘““I had followed his track carrying a torch, I had followed his path.””’\\
\glt ‘““Eu tinha seguido o rastro dele, com uma tocha na mão, eu tinha seguido o caminho dele.””’\\
\z

\ea ““Ete oya paitereya adihr eka, “Ãtiga meykodaatẽ ma”, oya takay ena yã” iyã” de.\\[.3em]
\gll ee-te o-ya pa-iter-ey{\footnotemark}-ya a-dihr ee-ka ã-tiga mey-koda-aat-tẽ ma o-ya ta-ka ee-na a i-ya \(\varnothing\)-de\\
\textsc{endo-adv} \textsc{1sg-nwit} \textsc{1pl.incl}-very-\textsc{pl-nwit} \textsc{3.refl}-exhaust \textsc{endo-dat} \textsc{dem.prox-sim} \textsc{2pl}-sit.\textsc{pl}-ly-\textsc{inch} \textsc{imp} \textsc{1sg-nwit} \textsc{3pl-dat} \textsc{endo-foc} \textsc{sfm.nwit}  \textsc{3sg-nwit} \textsc{3sg-wit}\\
\footnotetext{The lexicalized locution \textit{pa-iter} (\textsc{1pl.incl}-very), ‘we (inclusive of addressee) ourselves’ is the ethnic autodenomination of the Suruí. Here it obviously does not refer to the whole ethnic group, but to the younger male individuals that went with the warriors to carry their provisions of food and arrows.}
\glt ‘““And, as our men were exhausted, I told them: “You all should sit down here.”””’\\
\newpage 
\glt ‘““Aí, como nosso pessoal não aguentava mais, eu lhes disse: “Vocês podem sentar aquí.”””’\\
\z

\ea ““Eebo oyakah metota osahra, bobobob.””\\[.3em]
\gll ee-bo o-ya-kah pe-tota o-sahr-a bobobob\\
\textsc{endo-advers} \textsc{1sg-nwit}-go path-along \textsc{1sg}-swift-\textsc{vblz} \textsc{id}:walk.quickly\\
\glt ‘““And I trotted away swiftly along the path.””’\\
\glt ‘““Aí eu me fui embora rapidinho, seguindo o caminho.””’\\
\z

\ea ““Ñokoy oyakah, nem, \~{G}oxormaarãyasade awaga ewepi yã” iyã” de.\\[.3em]
\gll ano-koy o-ya-kah nem \~{G}oxor-ma-arãya-sade a-awaga ewe-epi a i-ya \(\varnothing\)-de\\
\textsc{dem.dist}-to \textsc{1sg-nwit}-go \textsc{intj} Zoró-\textsc{poss}-chicken-\textsc{prog.sim} \textsc{3.refl}-cry \textsc{dem.endo}-hear \textsc{sfm.nwit} \textsc{3sg-nwit} \textsc{3sg-wit}\\ 
\glt ‘““I went there, and I heard the pet cock of the Zoró singing.””’{\footnotemark}\\
\glt ‘““Eu fui lá, aí ouvi o galo do Zoró que cantava.””’\\
\footnotetext{The Suruí and the Zoró used to raise various kinds of pets -- mainly dogs, curassows and guans (turkey-like forest birds of the genera \textit{Mitu}, \textit{Penelope}, and \textit{Pipile}) -- and trust them to warn of approaching enemies. According to the Suruí, before contact, the Zoró already bred chickens that they had caught in rubber tapper settlements.}
\z

\ea ““Bohb, oya osahrokabi ya” iyã” de.\\[.3em]
\gll bohb o-ya o-sahrokabi a i-ya \(\varnothing\)-de\\
\textsc{id}:run \textsc{1sg-nwit} \textsc{1sg}-swift.under.cover \textsc{sfm.nwit} \textsc{3sg-nwit} \textsc{3sg-wit}\\
\glt ‘““Quickly, I approached, ducking under cover.””’\\
\glt ‘““Eu me aproximei rapidinho, abaixando-me para esconder-me.””’\\
\z



\ea ““Ãter oya xixababetâhikin a” iyã” de.\\[.3em]
\gll ã-ter o-ya xi-sab-abe-tâh-ikin a i-ya \(\varnothing\)-de\\
\textsc{dem.prox}-very \textsc{1sg-nwit} \textsc{3sg}-house-outside-stand-see \textsc{sfm.nwit} \textsc{3sg-nwit} \textsc{3sg-wit}\\
\glt ‘““Here, I could see their house.””’{\footnotemark}\\
\glt ‘““Aí eu vi a maloca dele.””’\\
\footnotetext{The Suruí and Zoró “houses" (\textit{lab}, non-possessed form of \textit{-sab}) were huge, oblong, and vaulted constructions  thatched with palm leaves, about 30 meters long and 5 meters high (see \figref{fig:surui:4}). Zoró houses, in contrast to Suruí dwellings, had no bark walls but were thatched down to the ground, allowing arrows to be shot through the palm leaves.}
\z


\begin{figure}
\includegraphics[width=.65\textwidth]{figures/maloca1.jpg}
\caption{A traditional Suruí house, rather small, built in 2005 near Lapetanha (Photo C. Yvinec)}
\label{fig:surui:4}
\end{figure}

 

\ea ““Yeter oytxepo!””\\[.3em]
\gll ye-ter oytxepo\\
\textsc{dem.med}-very perfect\\
\glt ‘““That’s perfect!””’\\
\glt ‘““Ótimo!””’\\
\z

\ea ““Bohb, oya osahrokabi yã, “Ikaytxer akah ana i?”””\\[.3em]
\gll bohb o-ya o-sahrokabi a i-ka-ter a-kah a-na i\\
\textsc{id}:run \textsc{1sg-nwit} \textsc{1sg}-swift.under.cover \textsc{sfm.nwit} \textsc{3sg-dat}-very \textsc{3.refl}-go \textsc{dem.prox-foc} \textsc{sfm.ndecl}\\
\glt ‘““Quickly, I approached nearer, ducking under cover, and I thought: “So is he staying in this one?”””’{\footnotemark}\\
\glt ‘““Logo me aproximei, abaixado ainda, e pensei: “Será que ele fica nesta?”””’\\
\footnotetext{The Suruí, and their Indian neighbors as well, frequently left their villages for long treks in the forest, either in search of forest resources or out of fear of enemies.}
\z

\largerpage[2] %longdistance
\ea ““Etiga te tamaawuruya waohwaohwaoh awuruya tamanikesota o\~{g}ay txar ã” iyã” de.\\[.3em]
\gll ee-tiga te ta-ma-awuru-ya waohwaohwaoh awuru-ya ta-maniga-e-sor-ta o-ka tar a i-ya \(\varnothing\)-de\\
\textsc{endo-sim} \textsc{adv} \textsc{3pl-poss}-dog-\textsc{nwit} \textsc{id}:bark dog-\textsc{nwit} \textsc{3pl}-come.near.to-\textsc{nmlz}-hard-\textsc{vblz} \textsc{1sg-dat} \textsc{prf} \textsc{sfm.nwit} \textsc{3sg-nwit} \textsc{3sg-wit}\\
\glt ‘““But then: \textit{Woof! Woof! Woof!} Their dog barked and did not let me come nearer to them.””’\\
\glt ‘““Aí, de repente: \textit{Au-au! Au-au! Au-au!} O cachorro dele latiu, não deixando eu me aproximar mais.””’\\
\z

\largerpage[2]
\ea ““Atãr o\~{g}ay txar ã.””\\[.3em]
\gll a-tãr o-ka tar a\\
\textsc{3.refl}-fierce \textsc{1sg-dat} \textsc{prf} \textsc{sfm.nwit}\\
\glt ‘““It was already barking fiercely at me.””’\\
\glt ‘““Ele já estava brabo comigo, estava latindo demais.””’\\
\z

\ea ““Etiga te oya, nem, awurupami tar ã” iyã” de.\\[.3em]
\gll ee-tiga te o-ya nem awuru-pami tar a i-ya \(\varnothing\)-de\\
\textsc{endo-sim} \textsc{adv} \textsc{1sg-nwit} \textsc{intj} dog-fear \textsc{prf} \textsc{sfm.nwit} \textsc{3sg-nwit} \textsc{3sg-wit}\\
\glt ‘““And me, well, I was scared of the dog.””’\\
\glt ‘““Aí eu estava com medo do cachorro.””’\\
\z

\ea “““Eh méhk palana pagah i!” oya tar ã” iyã” de.\\[.3em]
\gll eh méhk pa-sa-a-na pa-agah i o-ya tar a i-ya \(\varnothing\)-de\\
oh daybreak \textsc{1pl.incl-prog-dem.prox-foc} \textsc{1pl.incl}-dawn \textsc{sfm.ndecl} \textsc{1sg-nwit} \textsc{prf} \textsc{sfm.nwit} \textsc{3sg-nwit} \textsc{3sg-wit}\\
\glt ‘“““Oh, I realized, daybreak is coming, isn’t it?”””’\\
\glt ‘“““Oh, eu percebi, já está amanhecendo!”””’\\
\z

\ea ““Etiga te oyakah olob\~{g}armeyka “One te ana iwepi ner e, ba,” oya ena ikay ã” iyã” de.\\[.3em]
\gll ee-tiga te o-ya-kah o-sob-karmey-ka one te a-na iwe-pi{\footnotemark} ter e ba o-ya ee-na i-ka a i-ya \(\varnothing\)-de\\
\textsc{endo-sim} \textsc{adv} \textsc{1sg-nwit}-go \textsc{1sg}-father-younger.sibling-\textsc{dat} \textsc{neg} \textsc{adv} \textsc{dem.prox-foc} \textsc{dem.exo-abl} very \textsc{sfm.wit} father{\footnotemark} \textsc{1sg-nwit} \textsc{endo-foc} \textsc{3sg-dat} \textsc{sfm.nwit} \textsc{3sg-nwit} \textsc{3sg-wit}\\
\fnminus
\footnotetext{The phrase \textit{one te ana iwepi}, literally ‘not this way out of that’, means ‘not easy’.}
\fnplus
\footnotetext{One's father's brothers are classificatory “fathers" in the Suruí kinship terminology.}
\newpage 
\glt ‘““So I went straight to my father's younger brother and said to him: “Father, the situation is not easy.”””’\\
\glt ‘““Então voltei logo para o irmão mais novo de meu pai, aí lhe disse: “Pai, não é muito fácil.”””’\\
\z

\ea “““Iye. “Payahrxid ewaba”, te elaye, paitereykãra ejeka aye ewemi\~{g}a paitereyitxa iter” olobiya o\~{g}ay a” iyã” de.\\[.3em]
\gll iye pa-mayahr-sid e-waba te e-sa-aye pa-iter-ey-kãra e-de-ee-ka aye ewe-pi\~{g}a pa-iter-ey-itxa ter o-sob-ya o-ka a i-ya \(\varnothing\)-de\\
all.right \textsc{1pl.incl}-go.away-\textsc{hort} \textsc{2sg-hort} \textsc{adv} \textsc{2sg-prog-fut} \textsc{1pl.incl}-very-\textsc{pl}-retaliate.against \textsc{2sg-wit-endo-dat} \textsc{fut} \textsc{dem.endo}-seize:worry.about \textsc{1pl.incl}-very-\textsc{pl}-with very \textsc{1sg}-father-\textsc{nwit} \textsc{1sg-dat} \textsc{sfm.nwit} \textsc{3sg-nwit} \textsc{3sg-wit}\\
\glt ‘“““All right, my father answered, you can say: “Let's go away.” I am worried about our people: because of what you did, there will be retaliations against us.”””’\\
\glt ‘“““Tá bom, disse meu pai, você pode falar assim: “Vamos embora.” Estou preocupado com o nosso povo: por causa do que você fez, vai ter represálias contra nossa gente.”””’\\
\z

\ea ““Ete awurusena aker õm a, waohwaohwaoh, awurusena atãr ã.””\\[.3em]
\gll ee-te awuru-sa-ee-na a-ker õm a waohwaohwaoh awuru-sa-ee-na a-tãr a\\
\textsc{endo-adv} dog-\textsc{prog-endo-foc} \textsc{3.refl}-sleep \textsc{neg} \textsc{sfm.nwit} \textsc{id}:bark.in.the.distance dog-\textsc{prog-endo-foc} \textsc{3.refl}-fierce \textsc{sfm.nwit}\\
\glt ‘““And in the distance, the dog had not fallen asleep: \textit{Woof! Woof!} It was still barking fiercely.””’\\
\glt ‘““E lá, o cachorro não adormeceu: \textit{Au-au! Au-au!} Ele ficava brabo.””’\\
\z

\largerpage[-2]
\ea ““Etiga te oya “Okahsidlii” oladeka, “Atemareh, ba,” oya olob\~{g}armeyka yã” iyã” de.\\[.3em]
\gll ee-tiga te o-ya o-kah-sid-sa-i o-sade-ee-ka a-ter-ma-reh{\footnotemark} ba o-ya o-sob-karmey-ka a i-ya \(\varnothing\)-de\\
\textsc{endo-sim} \textsc{adv} \textsc{1sg-nwit} \textsc{1sg}-go-\textsc{hort-prog-sfm.ndecl} \textsc{1sg-prog.sim-endo-dat} \textsc{dem.prox}-very-\textsc{imp-hort.pl} father \textsc{1sg-nwit} \textsc{1sg}-father-younger.sibling-\textsc{dat} \textsc{sfm.nwit} \textsc{3sg-nwit} \textsc{3sg-wit}\\
\footnotetext{This locution is lexicalized as an interjection that means ‘Wait here for me!’ The plural aspect of the hortative suffix \textit{-reh} refers to the multiple wills (those of the addressee and of the speaker) implicated in actions that require individuals to coordinate themselves.}
\glt ‘““But I was already thinking: “I shall go,” and I said to my father's younger brother: “Wait, father!”””’\\ 
\glt ‘““Contudo eu já estava pensando: “Vou lá,” e falei para o irmão mais novo do meu pai: “Espere aí, pai!”””’\\
\z

\ea “““Owena ite te bolakah ãsabtiga yedeiwayka mareh!” oya ikay ã” iyã” de.\\[.3em]
\gll o-e-na ter te bo-o-sa-kah a-sab-tiga{\footnotemark} yed-iway-ka ma-reh o-ya i-ka a i-ya \(\varnothing\)-de\\
\textsc{1sg-nmlz-foc} very \textsc{adv} \textsc{advers-1sg-prog}-go \textsc{dem.prox}-house-\textsc{sim} \textsc{rel}-master-\textsc{dat} \textsc{imp-hort.pl} \textsc{1sg-nwit} \textsc{3sg-dat} \textsc{sfm.nwit} \textsc{3sg-nwit} \textsc{3sg-wit}\\
\glt ‘“““Let me go and show the master of this house who I am!” I said to him.””’{\footnotemark}\\
\glt ‘“““Vou mostrar ao dono desta maloca quem eu sou!” eu lhe disse.””’\\
\fnminus
\footnotetext{The suffix \textit{-tiga} can have a spatial meaning, as well as a temporal one.}
\fnplus
\footnotetext{The phrase \textit{ã-sab-tiga yed-iway}, \textsc{dem.prox}-house-\textsc{sim} \textsc{rel}-master, literally translates as ‘the master of the place where this house stands’. This periphrastic expression can be a rhetorical device. However, it is also a way to get round the ambiguity to which the simpler construction \textit{ã-sab-iway}, \textsc{dem.prox}-house-master, ‘the master of this house,’ could have given rise: indeed the latter phrase is a lexicalized expression, \textit{labiway}, house-master, that refers to the political status of chief.}
\z

\largerpage[-2]
\ea ““Bohb, osahror.””\\[.3em]
\gll bohb o-sahr-or\\
\textsc{id}:run \textsc{1sg}-swift-come\\
\glt ‘““I approached it quickly.””’\\
\glt ‘““Aproximei-me rapidinho.””’\\
\z

\newpage
\ea ““Awurusade atãr eamaĩ te osahrokabi.””\\[.3em]
\gll awuru-sade a-tãr ee-amaĩ{\footnotemark} te o-sahrokabi\\
dog-\textsc{prog.sim} \textsc{3.refl}-fierce \textsc{endo}-in.front.of \textsc{adv} \textsc{1sg}-approach.under.cover\\
\glt ‘““Although the dog was still fierce, I approached under cover.””’\\
\glt ‘““Embora o cachorro estivesse ainda brabo, eu me aproximei abaixado.””’\\
\footnotetext{Here the spatial postposition \textit{-amaĩ} has an abstract meaning of concession.}
\z

\ea ““Etiga te xiway añuma okabesahra etiga ee, mĩhnaka ana mehkap, ““Nan ariwa awuru ma\~{g}a?” bola awuruka yã”, ya ana mehkapa yã” iyã” de.\\[.3em]
\gll ee-tiga te xi-iway añum o-kabe-sahr-a ee-tiga ee mĩhna-ka a-na mehkap nan a-ariwa awuru ma-e\~{g}a bo-o-sa awuru-ka a \(\varnothing\)-ya a-na mehkap-a a i-ya \(\varnothing\)-de\\
\textsc{endo-sim} \textsc{adv} \textsc{3sg}-master a.little \textsc{1sg}-stoop-swift-\textsc{vblz} \textsc{endo-sim} \textsc{endo} door-\textsc{dat} \textsc{dem.prox-foc} opening \textsc{q} \textsc{3.refl}-be.noisy dog \textsc{q-prs} \textsc{advers-1sg-prog} dog-\textsc{dat} \textsc{sfm.nwit} \textsc{3sg-nwit}{\footnotemark} \textsc{dem.prox-foc} opening-\textsc{vblz} \textsc{sfm.nwit} \textsc{3sg-nwit} \textsc{3sg-wit}\\
\glt ‘““As I was running up stooping, its master half-opened the door and thought about the dog: “What is making the dog bark?”””’\\
\glt ‘““Eu estava correndo curvado, o dono dele entreabriu a porta, e ele pensou sobre o cachorro: “O que está fazendo o cachorro latir?”””’\\
\footnotetext{This phrase shows complex clause embedding used to express thought as inner speech: [[[\textit{Nan ariwa awuru ma-e\~{g}a}] \textit{bo-o-sa awuru-ka a}] \(\varnothing\)-ya], ‘[[[What is making the dog bark?] I am saying this (to myself) about the dog] he said this (to himself)].}
\z

\ea “““Ah sehr awuru!” olahrikin ajeka, dik, apĩhnapoga.””\\[.3em]
\gll ah sehr awuru o-sahr-ikin a-de-ee-ka dik a-mĩhna-poga\\
ah \textsc{id}:look.and.see dog \textsc{1sg}-swift-see \textsc{3.refl-wit-endo-dat} \textsc{id}:close \textsc{3.refl}-door-close\\
\glt ‘“““Ah, I see, dog!” he said as he saw me running up, and: \textit{Slam!} He shut the door.””’\\
\glt ‘“““Ah, estou vendo, cachorro!” ele disse quando me viu correndo para ele, e aí: \textit{Slam!} Fechou a porta.””’\\
\z

\newpage 
\ea ““Turuk, awuruyakahekoy iya iõmaniga ya” iyã” de.\\[.3em]
\gll turuk awuru-sa-kah-e-koy i-ya i-õm-a-niga a i-ya \(\varnothing\)-de\\
\textsc{id}:dodge.in dog-\textsc{prog}-go-\textsc{nmlz}-to \textsc{3sg-nwit} \textsc{3sg-neg-vblz-sim} \textsc{sfm.nwit} \textsc{3sg-nwit} \textsc{3sg-wit}\\
\glt  ‘““\textit{Whoosh!} The dog dodged its way in and disappeared inside.””’\\
\glt  ‘““\textit{Whoosh!} o cachorro entrou e desapareceu.””’\\
\z

\ea ““Ahwob, sog, sog omador xixabapa i!””.\\[.3em]
\gll ahwob sog sog{\footnotemark} o-ma-de-or xi-sab-ma-apa i\\
\textsc{id}:blow \textsc{id}:set.fire \textsc{id}:set.fire \textsc{1sg-prf.pst-wit}-come \textsc{3sg}-house-\textsc{caus}-burn \textsc{sfm.ndecl}\\
\glt ‘““But I had come already and: \textit{Puff!} I blew on my torch and: \textit{Whoosh! Whoosh!} I set fire to his house on both sides!””’\\
\glt ‘““Mas eu chegou já, e: \textit{Puff!} Soprei na minha tocha, e: \textit{Whoosh! Whoosh!} Toquei fogo nos dois lados da maloca dele!””’\\
\footnotetext{The repetition of the ideophone conveys the idea that the action was done twice, that is, on both sides of the house (Zoró houses had two doors).}
\z

\ea ““Etiga te etrrrk amauraã o\~{g}ay ã” iyã” de.\\[.3em]
\gll ee-tiga te etrrrk a-ma-ur-maã o-ka a i-ya \(\varnothing\)-de\\
\textsc{endo-sim} \textsc{adv} \textsc{id}:catching.fire \textsc{3.refl-poss}-bow-take \textsc{1sg-dat} \textsc{sfm.nwit} \textsc{3sg-nwit} \textsc{3sg-wit}\\
\glt ‘““It immediately caught fire and, inside, they picked up their bows to shoot at me.””’\\
\glt ‘““A maloca pegou fogo logo, e, dentro, eles pegaram os seus arcos para me flechar.””’\\
\z

\largerpage[-2]
\ea “““Oeh, amauraã do\~{g}ewa i! Ã ãtigareh!” tak, tak, mãeya ñokoy, tak, mãeyka, eeerh mamugekoya” iyã” de.\\[.3em]
\gll oeh a-ma-ur-maã \(\varnothing\)-de-o-ka-wa i ã ã-tiga-reh{\footnotemark} tak tak ma-ey-ya ano-koy tak ma-ey-ka eeerh ma-pug-e-koe-ya i-ya \(\varnothing\)-de\\
ah \textsc{3.refl-poss}-bow-take \textsc{3-wit-1sg-dat-hort} \textsc{sfm.ndecl} \textsc{dem.prox} \textsc{dem.prox-sim-hort.pl} \textsc{id}:shoot.arrow \textsc{id}:shoot.arrow other-\textsc{pl-nwit} \textsc{dem.dist}-to \textsc{id}:shoot.arrow other-\textsc{pl-dat} \textsc{id}:mortally.wounded \textsc{indf}-child-\textsc{nmlz}-voice-\textsc{nwit} \textsc{3sg-nwit} \textsc{3sg-wit}\\
\footnotetext{This phrase that uses a plural hortative suffix (\textit{-reh}) to refer to a highly individual decision and action (shooting one's arrow at the enemy) is an idiomatic construction. The plural hortative is perhaps motivated because this action requires resoluteness and self-coordination.}
\glt ‘“““Ah, they are picking up theirs bows to shoot at me! Let's shoot now!” I thought and I shot twice, the enemies moved away, I shot once again, and: \textit{Arrh!} A child's voice cried out as I mortally wounded him.””’\\
\glt ‘“““Ah, estão procurando os seus arcos para me flechar! Vamos flechá-los!” pensei, e flechei duas vezes, os inimigos se afastaram, flechei mais uma vez, e aí: \textit{Arrh!} Uma criança gritou, mortalmente ferida.””’\\
\z

\ea ““Maya maã, xitiya mamugpiekoy manáh atar ã” iyã” de.\\[.3em]
\gll ma-ya maã xi-ti-ya ma-pug-pi-ee-koy{\footnotemark} manáh tar a i-ya \(\varnothing\)-de\\
other-\textsc{nwit} take \textsc{3sg}-mother-\textsc{nwit} \textsc{indf}-child-hear-\textsc{endo}-to insult \textsc{prf} \textsc{sfm.nwit} \textsc{3sg-nwit} \textsc{3sg-wit}\\
\glt ‘““Another had seized his bow already, and because she heard him crying, the child's mother was insulting me.””’{\footnotemark}\\
\glt ‘““Já um outro inimigo pegou o seu arco e, ao ouvir sua criança gritando, a mãe dela me xingou.””’
\fnminus
\footnotetext{The locution \textit{ee-koy}, \textsc{endo}-to, has a causal meaning.}
\fnplus
\footnotetext{In combat, just as in most other contexts, insults (\textit{manáh}) scoff at the physical appearance of the addressee, especially at his or her genitals.}
\z

\ea ““Oya etiga te onepotê tedne \~{G}oxorka tar ã” iyã” de.\\[.3em]
\gll o-ya ee-tiga te o-tepotê ted-te \~{G}oxor-ka tar a i-ya \(\varnothing\)-de\\
\textsc{1sg-nwit} \textsc{endo-sim} \textsc{adv} \textsc{1sg}-shoot.arrow only-\textsc{adv} Zoró-\textsc{dat} \textsc{prf} \textsc{sfm.nwit} \textsc{3sg-nwit} \textsc{3sg-wit}\\
\glt ‘““But I just kept on shooting arrows at the Zoró.””’\\
\glt ‘““Eu fiquei flechando o Zoró.””’\\
\z

\newpage 
\ea ““Eebo oyena \~{G}oxorsabapa yã” iyã” de.\\[.3em]
\gll ee-bo o-ya-ee-na \~{G}oxor-sab-ma-apa a i-ya \(\varnothing\)-de\\
\textsc{endo-advers} \textsc{1sg-nwit-endo-foc} Zoró-house-\textsc{caus}-burn \textsc{sfm.nwit} \textsc{3sg-nwit} \textsc{3sg-wit}\\
\glt ‘““Thus I burnt down the Zoró's house.””’\\
\glt ‘““Foi assim que eu queimei a maloca do Zoró.””’\\
\z

\ea Eebo omamõperedena ena \~{G}oxorka, xameomi ter denene, asobaka dekenene.\\[.3em]
\gll ee-bo o-ma-amõ-pere-de-na ee-na \~{G}oxor-ka xameomi ter \(\varnothing\)-de-ee-na-ee-na-e a-sob-aka \(\varnothing\)-de-ee-ka-ee-na-ee-na-e\\
\textsc{endo-advers} \textsc{1sg-poss}-grandfather-\textsc{iter-wit-foc} \textsc{endo-foc} Zoró-\textsc{dat} much very \textsc{3sg-wit-endo-foc-endo-foc-sfm.wit} \textsc{3.refl}-father-kill \textsc{3sg-wit-endo-dat-endo-foc-endo-foc-sfm.wit}\\
\glt ‘My grandfather did this several times to the Zoró, he did it many times, because his father had been killed by one of them.’\\
\glt ‘Várias vezes meu avô fez isso ao Zoró, muitas vezes, porque aquele tinha matado o pai dele.’\\
\z

\ea Oilud ena alaba dena ena maiter ikay e.\\[.3em]
\gll oilud ee-na a-saba \(\varnothing\)-de-ee-na ee-na ma-iter{\footnotemark} i-ka e\\
young \textsc{endo-foc} \textsc{3.refl-prog.pst} \textsc{3sg-wit-endo-foc} \textsc{endo-foc} other-very \textsc{3sg-dat} \textsc{sfm.wit}\\
\glt ‘He was young then, so he did it once again to them.’\\
\glt ‘Ele estava moço naquele tempo, aí lhes fez isso mais uma vez.’\\
\footnotetext{The locution \textit{ma-iter} (other-very) means ‘more’ or ‘once again’.}
\z

  
\ea Ayabmi dena maiter ikay e.\\[.3em]
\gll a-yab-pi \(\varnothing\)-de-ee-na ma-iter i-ka e\\
\textsc{dem.prox-endo-abl} \textsc{3sg-wit-endo-foc} other-very \textsc{3sg-dat} \textsc{sfm.wit}\\
\glt ‘And afterward, he did it once more to them.’\\
\glt ‘Aí depois, ele lhe fez isso mais uma vez.’\\
\z

\newpage 
\ea Ayabmi dena maiter xixabapa, xixabapa tedne iperedena sone.\\[.3em]
\gll a-yab-pi \(\varnothing\)-de-ee-na ma-iter xi-sab-ma-apa xi-sab-ma-apa ted-te i-pere-de-na sona-e.\\
\textsc{dem.prox-endo-abl} \textsc{3sg-wit-endo-foc} other-very \textsc{3sg}-house-\textsc{caus}-burn \textsc{3sg}-house-\textsc{caus}-burn only-\textsc{adv} \textsc{3sg-iter-wit-foc} often-\textsc{sfm.wit}\\
\glt ‘And afterward, he burnt one their houses down again, several times he just burnt a house down.’\\
\glt ‘Aí depois, ele queimou de novo  outra maloca dele, várias vezes ele só queimou uma maloca dele.’\\
\z

\ea Omamõperedene.\\[.3em]
\gll o-ma-amõ-pere-de-na-e\\
\textsc{1sg-poss}-grandfather-\textsc{iter-wit-foc-sfm.wit}\\
\glt ‘My grandfather did that again and again.’\\
 \glt ‘Meu avô fez isso várias vezes.’\\
\z

\ea ““Eebo oyena \~{G}oxoreaka, xixabapa, olobaka deka yã” olobiyã” olobde.\\[.3em]
\gll ee-bo o-ya-ee-na \~{G}oxor-e-aka xi-sab-ma-apa o-sob-aka \(\varnothing\)-de-ee-ka ã o-sob-ya o-sob-de\\
\textsc{endo-advers} \textsc{1sg-nwit-endo-foc} Zoró-\textsc{nmlz}-kill \textsc{3sg}-house-\textsc{caus}-burn \textsc{1sg}-father-kill \textsc{3sg-wit-endo-dat} \textsc{sfm.nwit} \textsc{1sg}-father-\textsc{nwit} \textsc{1sg}-father-\textsc{wit}\\
\glt ‘My father said this: “My father said this: “Thus I killed the Zoró, I burnt his house down, because he had killed my father.””’\\
\glt ‘Meu pai falou assim: “Meu pai contou isso: “Foi assim que eu matei o Zoró, queimei a maloca dele, porque ele tinha matada meu pai.””’\\
\z

  
\ea Ena.\\[.3em]
\gll ee-na\\
\textsc{endo-foc}\\
\glt ‘It happened like this.’\\
\glt ‘Aconteceu assim.’\\
\z

\newpage
\ea Ayabdena iwewá ikay e.\\[.3em]
\gll a-yab-de-ee-na iwe-ewá{\footnotemark} i-ka e\\
\textsc{dem.prox-endo-wit-endo-foc} \textsc{dem.exo}-say \textsc{3sg-dat} \textsc{sfm.wit}\\
\glt ‘And he sang to celebrate this event.’\\
\glt ‘Aí ele cantou para celebrar este acontecimento.’\\
\footnotetext{Although the verb \textit{-ewá} just means ‘to say’ or ‘to talk’ when it used intransitively (\textit{awewá}, \textit{a-we-ewá}, \textsc{3.refl-refl}-say, ‘they talk to each other’), when it is used transitively, like here, it always implies that the speech is sung.}
\z

\ea “““Awurutihma mamekoka o\~{g}ay omamibewẽtig, wẽtiga, wẽtiga ya.”””{\footnotemark}\\[.3em]
\gll awuru-tih{\footnotemark}-ma ma-meko-ka o-ka o-pami-be-wẽtiga wẽtiga wẽtiga a{\footnotemark}\\
dog-big-\textsc{prf.pst} \textsc{indf}-jaguar-\textsc{dat} \textsc{1sg-dat} \textsc{1sg}-fear-\textsc{nmlz}-sound sound sound \textsc{sfm.nwit} \\
\glt ‘“““The big dog sounded its fear of a jaguar, of me it sounded it, sounded it, they say.’\\
\glt ‘“““O cão grande soou seu medo da onça, de mim ele o soou, o soou, ouvi falar isso.’\\
\fnminus
\fnminus
\footnotetext{This and the two following lines were sung. Here Agamenon quoted only a sample of the song, which was actually far longer.}
\fnplus
\footnotetext{The suffix \textit{-tih} is often used to distinguish mythological or spiritual beings from their ordinary homonyms.}
\fnplus
\footnotetext{Non-witnessed evidentiality is an aesthetic rule with which all Suruí sung speeches comply.}
\z

\ea “““Awurutihma mamekoka o\~{g}ay,”””\\[.3em]
\gll awuru-tih-ma ma-meko-ka o-ka\\
dog-big-\textsc{prf.pst} \textsc{indf}-jaguar-\textsc{dat} \textsc{1sg-dat}\\
\glt ‘The big dog, of a jaguar, of me,’\\
\glt ‘O cão grande, da onça, de mim,’\\
\z

  
\ea “““Oikin nedne loykubeyaawurutihma mamekoka o\~{g}ay mamibewẽtig, wẽtiga, wẽtiga ya.”””\\[.3em]
\gll o-ikin ted-te loykub{\footnotemark}-ey-ma-awuru-tih-ma ma-meko-ka o-ka pami-be-wẽtiga wẽtiga wẽtiga a\\
\textsc{1sg}-see only-\textsc{adv} enemy-\textsc{pl-poss}-dog-big-\textsc{prf.pst} \textsc{indf}-jaguar-\textsc{dat} \textsc{1sg-dat} fear-\textsc{nmlz}-sound sound sound \textsc{sfm.nwit}\\
\footnotetext{This word is only used in sung speech, instead of the word \textit{lahd}, and always with the plural suffix \textit{-ey}.}
\glt ‘Just at seeing me, the big dog of the enemies sounded its fear of a jaguar, of me, sounded it, sounded it, they say.”’\\
\glt ‘Ao me ver, o cão grande dos inimigos soou seu medo da onça, de mim, o soou, o soou, ouvi falar isso.”’\\
\z

\ea ““Oya iwewá ya” iyã, “ximaawurumaĩ ojehwá yã” iyã, “ena.””\\[.3em]
\gll o-ya iwe-ewá a i-ya xi-ma-awuru-ma-aĩ o-de-ee-ewá a i-ya ee-na\\
\textsc{1sg-nwit} \textsc{dem.exo}-say \textsc{sfm.nwit} \textsc{3sg-nwit} \textsc{3sg-poss}-dog-big-\textsc{caus}-go.into \textsc{1sg-wit}-\textsc{endo}-say \textsc{sfm.nwit} \textsc{3sg-nwit} \textsc{endo-foc}\\
\glt ‘I sang this, I sang that I made his big dog run in, like this.””’\\
\glt ‘Eu cantei assim, eu cantei que eu fiz que o cão grande dele se esconder dentro da casa, assim.””’\\
\z

\ea Ayabmi maite te.\\[.3em]
\gll a-yab-pi ma-iter te\\
\textsc{dem.prox-endo-abl} other-very \textsc{adv}\\
\glt ‘Afterward, he did it once again.’\\
\glt ‘Depois disso, ele fê-lo de novo.’\\
\z

 
\ea Ayabmi denena te ena ikãyna alaba ena xixabapa akah \~{G}oxor ene.\\[.3em]
\gll a-yab-pi \(\varnothing\)-de-ee-na-ee-na te ee-na i-kãy-na a-saba ee-na xi-sab-ma-apa a-kah \~{G}oxor ee-na-e\\
\textsc{dem.prox-endo-abl} \textsc{3sg-wit-endo-foc-endo-foc} \textsc{adv} \textsc{endo-foc} \textsc{3sg}-old-\textsc{foc} \textsc{3.refl-prog.pst} \textsc{endo-foc} \textsc{3sg}-house-\textsc{caus}-burn \textsc{3.refl}-go Zoró \textsc{endo-foc-sfm.wit}\\
\glt ‘Afterward, he did it, when he grew up, he went and burnt down the Zoró’s house.’\\
\glt ‘Depois, ele o fez, quando ele cresceu, ele se foi queimar a maloca do Zoró.’\\
\z

\ea Ena asobaka eka te iperedena agõarih ikay e.\\[.3em]
\gll ee-na a-sob-aka ee-ka te i-pere-de-na a-agõa-arih{\footnotemark} i-ka e\\
\textsc{endo-foc} \textsc{3.refl}-father-kill \textsc{endo-dat} \textsc{adv} \textsc{3sg-iter-wit-foc} \textsc{3.refl}-heart-lazy \textsc{3sg-dat} \textsc{sfm.wit}\\
\glt ‘Thus, because his father had been murdered, he remained merciless toward them.’\\
\glt ‘Assim, porque seu pai tinha sido morto, ele ficou implacável contra aquele.’\\
\footnotetext{The word \textit{agõa}, ‘heart’, has physical and emotional meaning. When it is ‘unresponsive’ or ‘lazy’ (\textit{-arih}), it means that one feels no compassion for someone else -- which is an attitude that is not always praised, even toward enemies.}
\z 

\ea Eebo dena ikãyna alaba enene, xixabapa akah e.\\[.3em]
\gll ee-bo \(\varnothing\)-de-ee-na i-kãy-na a-saba ee-na-ee-na-e xi-sab-ma-apa a-kah e\\
\textsc{endo-advers} \textsc{3sg-wit-endo-foc} \textsc{3sg}-old-\textsc{foc} \textsc{3.refl-prog.pst} \textsc{endo-foc-endo-foc-sfm.wit} \textsc{3sg}-house-\textsc{caus}-burn \textsc{3.refl}-go \textsc{sfm.wit}\\
\glt ‘And he did it again when he grew up, he went and burnt down one of their houses.’\\
\glt ‘Ele o fez de novo quando cresceu, ele foi queimar a maloca daquele.’\\
\z

 
\ea Eebo dena epi xaka ene.\\[.3em]
\gll ee-bo \(\varnothing\)-de-ee-na ee-pi xi-aka ee-na-e\\
\textsc{endo-advers} \textsc{3sg-wit-endo-foc} \textsc{endo-abl} \textsc{3sg}-kill \textsc{endo-foc-sfm.wit}\\
\glt ‘And afterward he killed another one.’\\
\glt ‘Aí depois ele matou mais um Zoró.’\\
\z

  
\ea Xixabapa akah ñorĩ, ete “Xixabapa o\~{g}abi ma!” sadena mãeyka, ete epetima\~{g}a alaba.\\[.3em]
\gll xi-sab-ma-apa a-kah ñorĩ ee-te xi-sab-ma-apa o-kabi ma sade-ee-na ma-ey-ka ee-te ee-petima\~{g}a a-saba\\
\textsc{3sg}-house-\textsc{caus}-burn \textsc{3sg}-go stealthily \textsc{endo-adv} \textsc{3sg}-house-\textsc{caus}-burn \textsc{1sg-ben} \textsc{imp} \textsc{prog.sim-endo-foc} other-\textsc{pl-dat} \textsc{endo-adv} \textsc{endo}-ambush \textsc{3.refl-prog.pst}\\
\glt ‘He went away to burn down a Zoró house, he said to a few others: “Burn down their house for me!” and he lay in ambush.’\\
\glt ‘Ele saiu para queimar uma maloca do Zoró, ele falou para outros seus parentes: “Queimem a maloca dele para mim!”  e ele ficou emboscado.’\\
\z

\ea Eebo “Ka\~{g}oy ena mapãri ma, palodena sona i?” xixabapa adeke, ete yakadena asabalabiĩ soeydekena pãri amauraã yakena madane.\\[.3em]
\gll ee-bo ka-koy ee-na ma-pãri ma palo-de-ee-na sona i xi-sab-ma-apa sade-ee-ka-e ee-te i-sade-ee-na a-sab-alabiĩ so-ey-de-kah-ee-na pãri a-ma-ur-maã i-sa-ee-na ma-de-ani-e\\
\textsc{endo-advers} \textsc{q}-to \textsc{endo-foc} \textsc{indf}-make.noise \textsc{q} someone-\textsc{wit-endo-foc} often \textsc{sfm.ndecl} \textsc{3sg}-house-\textsc{caus}-burn \textsc{prog.sim-endo-dat-sfm.wit} \textsc{endo-adv} \textsc{3sg-prog.sim-endo-foc} \textsc{3.refl}-house-burning thing-\textsc{pl-wit}-go-\textsc{endo-foc} make.noise \textsc{3.refl-poss}-bow-take \textsc{3sg-prog-endo-foc} \textsc{indf-wit-gno-sfm.wit}\\
\glt ‘And while the house was burning up, he was watching and wondering: “Where is one making noise, is there someone?” because as one’s house is in flames, one moves things about and makes noise in search of one’s bow.’\\
\glt ‘Aí, quando a maloca estava queimando, ele observava-a pensando: “Onde está quem está fazendo barulho? Será que tem alguém aí dentro?” porque, quando a maloca de alguém está em chamas, este alguém mexe as coisas e faz barulho, procurando seu arco.’\\
\z

\ea “Ãtigareh!”, tap, sok, ena xaka ene.\\[.3em]
\gll ã-tiga-reh tap sok ee-na xi-aka ee-na-e\\
\textsc{dem.prox-sim-hort.pl} \textsc{id}:shoot.arrow \textsc{id}:hit.target \textsc{endo-foc} \textsc{3sg}-kill \textsc{endo-foc-sfm.wit}\\
\glt ‘“Let’s shoot now!” he thought, he fired his arrow and hit his target, that’s how he killed each one.’\\
\glt ‘“Vamos flechar agora mesmo!” ele pensava, ele flechava e atingia o seu alvo, era assim que ele matava aquele.’\\
\z

\ea Ayabmi dena ãtiga manopetima\~{g}a, ãtiga manopetima\~{g}a, ãtiga manopetima\~{g}a, ãtiga manopetima\~{g}a.\\[.3em]
\gll a-yab-pi \(\varnothing\)-de-ee-na ã-tiga ma-ano-petima\~{g}a ã-tiga ma-ano-petima\~{g}a ã-tiga ma-ano-petima\~{g}a ã-tiga ma-ano-petima\~{g}a\\
\textsc{dem.prox-endo-abl} \textsc{3sg-wit-endo-foc} \textsc{dem.prox-sim} other-standing.up-ambush \textsc{dem.prox-sim} other-standing.up-ambush \textsc{dem.prox-sim} other-standing.up-ambush \textsc{dem.prox-sim} other-standing.up-ambush\\
\glt ‘And next to him, there was another one standing in ambush, and there another one, and there another one, and there another one.’{\footnotemark}\\
\glt ‘E aí perto dele, outro ficava emboscado, e lá mais um, e lá mais um, e lá mais um.’\\
\footnotetext{Because of the size of their bows, which need to be held vertically, warriors in ambush had to wait standing upright, usually hiding themselves behind a tree trunk.}
\z

\ea Ete ãtiga manode mapa mokãyĩ, ãtiga mano, pãri dena amauraã yakade, masena, "Ãtigareh!", tap, enike.\\[.3em]
\gll ee-te ã-tiga ma-ano-de mapa mokãy-ĩ ã-tiga ma-ano pãri \(\varnothing\)-de-ee-na a-ma-ur-maã i-sade ma-sa-ee-na ã-tiga-reh tap ee-na-i-ka-e\\
\textsc{endo-adv} \textsc{dem.prox-sim} other-standing.up-\textsc{wit} shoot.arrow fire-inside \textsc{dem.prox-sim} other-standing.up noise \textsc{3sg-wit-endo-foc} \textsc{3.refl-poss}-bow-take \textsc{3sg-prog.sim} other-\textsc{prog-endo-foc} \textsc{dem.prox-sim-hort.pl} \textsc{id}:hit.target \textsc{endo-foc-3sg-dat-sfm.wit}\\
\glt ‘Then one of them shot an arrow in the fire, and, as someone made noise in search of his bow, another one standing there thought: “Let’s shoot now,” and did it to that one.’\\
\glt ‘Aí um deles flechava no fogo, aí, quando alguém fazia barulho procurando seu arco, outro que ficava lá pensava: “Vamos flechar agora mesmo,” e fazia isso contra aquele.’\\
\z

\ea ““Ete oyena ena yã” iyã” de.\\[.3em]
\gll ee-te o-ya-ee-na ee-na ã i-ya \(\varnothing\)-de\\
\textsc{endo-adv} \textsc{1sg-nwit-endo-foc} \textsc{endo-foc} \textsc{sfm.nwit} \textsc{3sg-nwit} \textsc{3sg-wit}\\
\glt ‘[My father] said: “He said: “So I did it that way.””’\\
\glt ‘[Meu pai] contou isso: “Ele contava: “Foi assim que eu fiz.””’\\
\z

\newpage
\ea ““Eebo labdena apa ya” iyã” de, ““ewewewaya, \~{G}oxorsade ana apabiar awerkar anokoy ewenam\~{g}a enikay ã’ iyã” de, ““ano agaap alap, tap””.\\[.3em]
\gll ee-bo sab-de-ee-na a-apa ã i-ya \(\varnothing\)-de ewewaya \~{G}oxor-sade a-na a-pabiar a-werkar ano-koy ewe-nam-ka ee-na-i-ka ã i-yã \(\varnothing\)-de ano a-agaa-ap a-alap tap\\
\textsc{endo-advers} house-\textsc{wit-endo-foc} \textsc{3.refl}-burn \textsc{sfm.nwit} \textsc{3sg-nwit} \textsc{3sg-wit} \textsc{id}:burning Zoró-\textsc{prog.sim} \textsc{dem.prox-foc} \textsc{3.refl}-on.all.fours \textsc{3.refl}-walk \textsc{dem.dist}-to \textsc{dem.endo}-quantity-\textsc{dat} \textsc{endo-foc-3sg-dat} \textsc{sfm.nwit} \textsc{3sg-nwit} \textsc{3sg-wit} \textsc{dem.dist} \textsc{3.refl}-belly-hole \textsc{3.refl}-stretch.out \textsc{id}:hit.target\\
\glt ‘““The house was burning up, the Zoró crawled on all fours and stretched out like this, and we shot at them.””{\footnotemark}\\
\glt ‘““A maloca estava queimando, o Zoró rastejando de quatro, assim, e o flechávamos.””
\footnotetext{The volume of the house, whose thatched roof burnt away very quickly, allowed the occupants to survive the fire, if they were not shot.}
\z

  
\ea Nem, “awaĩ ikay ã” iyã” de, ““eeerh!””\\[.3em]
\gll nem a-waĩ i-ka ã i-ya \(\varnothing\)-de eeerh\\
\textsc{intj} \textsc{3.refl}-shoot.arrow \textsc{3sg-dat} \textsc{sfm.nwit} \textsc{3sg-nwit} \textsc{3sg-wit} \textsc{id}:dying\\
\glt ‘““One of us shot one of them, and he was dying.””’\\
\glt ‘““Aí um de nós o flechou, e ele ficou morrendo.””’\\
\z

  
\ea Ena omamõdena lahd\~{g}a ena mater e.\\[.3em]
\gll ee-na o-ma-amõ-de-ee-na lahd-ka ee-na mater e\\
\textsc{endo-foc} \textsc{1sg-poss}-grandfather-\textsc{wit-endo-foc} Indian.enemy-\textsc{dat} \textsc{endo-foc} long.ago \textsc{sfm.wit}\\
\glt ‘That’s the way my grandfather treated the enemy long ago.’\\
\glt ‘Foi assim que meu avô tratou o inimigo há muito tempo.’\\
\z

\largerpage[2]
\ea Bo te.\\[.3em]
\gll bo te\\
\textsc{advers} \textsc{adv}\\
\glt ‘That’s it.’\\
\glt ‘Foi isso.’\\
\z
\newpage
\section*{Acknowledgments}
Fieldwork on Suruí in 2013 was funded by a postdoctoral fellowship from the Fondation Fyssen and by the ANR project Fabriq’Am (ANR-12-CULT-005). I would like to thank Kristine Stenzel for her careful revision of linguistic concepts and English translations.

\section*{Non-standard abbreviations}

\begin{tabularx}{.45\textwidth}{lQ}
\textsc{advers} & adversative \\
\textsc{endo} & endophoric \\
\textsc{exo } & exophoric \\
\textsc{gno} & gnomic \\
\textsc{hort } & hortative \\
\textsc{id } & ideophone \\
\textsc{inch } & inchoative\\
\textsc{iter } & iterative \\
\end{tabularx}
\begin{tabularx}{.45\textwidth}{lQ}
\textsc{med } & medial \\
\textsc{ndecl} & non-declarative \\
\textsc{nwit } & non-witnessed evidentiality \\
\textsc{sfm } & sentence final marker \\
\textsc{sim } & simultaneity \\
\textsc{vblz } & verbalizer \\
\textsc{wit } & witnessed evidentiality \\
\end{tabularx}
 

  
{\sloppy
\printbibliography[heading=subbibliography,notkeyword=this]
}
\end{document}
