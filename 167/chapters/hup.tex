\documentclass[output=paper,
modfonts,nonflat
]{langsci/langscibook} 
\author{Patience Epps\affiliation{University of Texas Austin, USA}%
\and Isabel Salustiano%
\and Jovino Monteiro%
\lastand Pedro Pires Dias%
}
\title{Hup}
\lehead{P.\ Epps, Isabel Salustiano, Jovino Monteiro \& Pedro Pires Dias}
\ourchaptersubtitle{Mohõ̀y wäd nɨ̀h pɨnɨ̀g}
\ourchaptersubtitletrans{‘Story of the Deer Spirit’}  
\ChapterDOI{10.5281/zenodo.1008785}

\maketitle

\begin{document}

\section{Introduction} 

This narrative tells the story of a liaison between a deer spirit and a woman, and the troubles that came of it. The tale is one of many stories told by the Hupd’äh, who live in the Vaupés region, straddling the border of Brazil and Colombia (\figref{fig:hup:1}). Hup, the language of the Hupd’äh (lit. ‘person-\textsc{pl}’) belongs to the small Naduhup family (formerly known as Makú; see \citealt{EppsandBolaños2017}); the speakers of the four Naduhup languages inhabit the interfluvial zones south of the Vaupés River and the middle Rio Negro. The approximately 2000 Hupd’äh live in communities ranging from a few families to several hundred people, located between the Vaupés and the Tiquié Rivers.
	
\begin{figure}[p]
%   \includegraphics[width=0.90\textwidth]{figures/fig-1-map.png}
  \includegraphics[width=0.90\textwidth]{figures/fig-1-map.pdf}
  \caption{Map of the Upper Rio Negro Region \citep{EppsStenz2013}}
\label{fig:hup:1}
\end{figure}
    \begin{figure}[p]
 \includegraphics[width=.90\textwidth]{figures/fig-2-tat-deh-community.jpg}
  \caption{Taracua Igarapé community}
\label{fig:hup:2}
\end{figure}

    The version of the Deer Story narrative presented here was recorded in November 2001 in the Hup community of Taracua Igarapé, known in Hup as \textit{Tát Dëh}, or ‘Ant (sp.) Creek’ (\figref{fig:hup:2}). Taracua Igarapé is located along the large creek that bears the same name, at about an hour’s walk into the forest from the banks of the Tiquié River, and is home to about 150 people. While the residents of Taracua Igarapé belong to a number of clans, the community is understood to be in the principal territory of the \textit{Sokw’ät Nok’öd Tẽ́hd’äh}, the ‘Descendents of the Toucan’s Beak’, and is itself the most recent of a series of communities associated with the Toucan’s Beak clan, which over the past six to eight generations have relocated incrementally from the more remote interfluvial zones toward the Tiquié River. 
    

    
	The Deer Story was performed by Isabel Salustiano, a talented storyteller with a vast repertoire of traditional stories and a masterful delivery. She is originally from the nearby community of Cabari do Japu (\textit{Pìj Dëh}), and is married to Américo Monteiro Socot, who is himself a Toucan’s Beak clansman and an influential figure in the community. The story was recorded outdoors in the community, in a central area near a cluster of houses, and in the presence of several of her children and other community members. The text was subsequently transcribed and checked over by Patience Epps, Jovino Monteiro Socot, and Pedro Pires Dias (of Barreira Alta). It appears in the book of Hup stories that was produced for the Middle Tiquié River Hup communities (ed. by Epps, 2005/2016), with illustrations drawn by Estevão Socot (Jovino’s son). Several of these illustrations are included here.
 
\begin{figure}[b]
  \includegraphics[width=.75\textwidth]{figures/fig-3-forest-stream.jpg}
  \caption{Forest creek in the region of Taracua Igarapé}
  \label{fig:hup:3}
\end{figure}  
	As the story begins, a widow, struggling to feed herself and her children, is following a forest stream, collecting tiny freshwater shrimp. This is poor fare, but the most she can manage without a husband to hunt and fish for her. As she moves upstream, she begins to find fish, freshly caught and set out on the bank (Figures \ref{fig:hup:3}--\ref{fig:hup:4}). In her desperation, she takes the fish, although she knows that by doing so she is entering into a relationship with an unknown and potentially dangerous other. Soon after, she hears a whistle, and looks up to see a deer spirit in man’s form, brilliant with red body paint, looking down at her from the bank. He tells her that he will come that evening to see her, and she agrees. The woman then returns home, feeds her children and puts them to bed, and waits for the deer spirit. He arrives, laden with game, and the two of them spend the evening eating, leaving none for the children. The deer spirit then sleeps with her, together in one hammock, and leaves just before dawn. 
        
 

\begin{figure}[t]
  \includegraphics[width=.90\textwidth]{figures/fig-4-woman-fishing.jpg}
  \caption{The woman finding fish set out for her (Estevão Socot)}
  \label{fig:hup:4}

\end{figure}

	The deer spirit continues to visit the woman nightly, always bringing large quantities of game, which they eat together without saving any for the children. During the day, the woman prepares special \textit{manicuera}, a drink made from boiled manioc juice mixed with tasty fruits, to offer her deer husband at night, while she gives only old sour manicuera to the children. Eventually the oldest boy, wondering why his mother always sends them to bed at night with such haste, resolves to stay up and watch. He hides in his hammock and peeks through the holes in the loosely woven palm-fiber mesh, and sees the deer spirit and his mother feasting on the game the spirit has brought (\figref{fig:hup:5}). Furious with this revelation, the boy tells his siblings their mother’s secret. Together they dig fish-poison root (\textit{Lonchocarpus} sp.), beat it to extract the poison, and squeeze it into the special manicuera that their mother had prepared for her husband.
    
\begin{figure}
  \includegraphics[width=0.90\textwidth]{figures/fig-5-watching-in-hammock.jpg}
  \caption{The boy watching his mother and the deer spirit (Estevão Socot)}
  \label{fig:hup:5}

\end{figure}
    \begin{figure}[b]
  \includegraphics[width=0.65\textwidth]{figures/fig-6-carrying-deer.jpg}
  \caption{\mbox{The woman carrying the body of the deer spirit (Estevão Socot)}}
  \label{fig:hup:6}
\end{figure}    
	That night, the boy lies watching again as the deer spirit and his mother feast and go to sleep. In the morning, the mother frantically shakes her husband to waken him, but finds him dead. Sending the children out of the house to bathe, she breaks up her husband’s body and squeezes it into a large burden-basket, which she carries into the sky for burial (\figref{fig:hup:6}). The spot where she leaves him becomes a formation known as the Deer’s Tomb, visible in the night sky. This formation is also recognized by the Tukano people, who call it by the same name in their own language, but its location has not yet been identified in the ethnoastronomical work carried out in the region (see \citealt{Cardoso2007,Oliveira2010}); it is probably one of the “constellations" that peoples of this region visualize in the dark spaces between clusters of stars, rather than in the stars themselves. 
	
    The mother returns, and before long she gives birth to the deer spirit’s child. She conceals the infant from her children by hanging it up in a bag of charcoal from the rafters of the house, and only takes it down twice a day to nurse it. However, her actions do not go unnoticed by her children, who become curious about the contents of the bag, and climb up one day to have a look while their mother is out in her manioc garden. They discover the baby with delight, and take it out with them into the overgrown swidden (garden areas that have been abandoned in the cycle of slash-and-burn farming) to play with it. There they feed it manioc leaves, potato leaves – all the garden plants to which deer help themselves today – and proceed to play with it by pushing it back and forth among them. As they do so, the baby deer rapidly gains strength, and suddenly it gives a snort, leaps over the children’s heads, and disappears into the forest. 
    
    
\begin{figure}[b]
  \includegraphics[width=0.70\textwidth]{figures/fig-7-currassow-wikipedia.jpg}
  \caption{Nocturnal Curassow \textit{Nothocrax urumutum} (Photo: Jelle Oostrom) \tiny \url{http://www.flickr.com/photos/jelle82/4823615464/};CC-BY-SA 2.0 \url{http://creativecommons.org/licenses/by-sa/2.0} )} 
  \label{fig:hup:7}
\end{figure}
    

	The loss of the baby deer is the final step in severing the children’s relationship with their mother. Fearful of her anger, they have already begun a transformation into curassow birds (\textit{Nothocrax urumutum}, \figref{fig:hup:7}) by the time she discovers the missing baby. Other birds have filled the children's skin with feathers and drawn circles around their eyes, and the children have dug holes in the ground like those of curassows. When their mother rushes into the house to beat them, the children scatter, flying off to hide in their holes. The mother tries in vain to catch one little girl by putting a basket over her hole, but the bird-child only tunnels away, comes up in another place, and flies off. The poor mother wanders crying after her children, and thus transforms herself into a \textit{bëbë} bird – a small brown bird, probably a type of antthrush (family \textit{Formicariidae}), that walks on the forest floor and whose call is reminiscent of crying. 

	A major theme of the Deer Story is one of transformation, carried out in the context of liminality of behavior and existence. Through their actions, the figures in the story occupy the zones between the human, animal, and spirit “worlds", and their engagement with the occupants of these other worlds ultimately propels them wholly into their domains. This theme is a familiar one in verbal art and cultural practice throughout the Amazon basin: humans, animals, and spirits are understood to share a similar conception of their worlds, but with a fundamental disconnect such that, for example, what a vulture sees as grilled fish appears to a person as maggots in rotting meat (\citealt{ViveirosdeCastro1998}, \textit{inter alia}). One must maintain one’s own position as the inhabitant of a particular subjective world by means of appropriate actions – in particular, appropriate social actions – while direct engagement with entities outside this domain is inherently perilous (see, e.g. \citealt{Santos-Granero2006,LondoñoSulkin2005,Uzendowski2005,Vilaça2000}). It is through this engagement that one may lose one’s own subjectivity and enter that of one’s interlocutor. Thus the widow, by accepting the fish presented by the deer spirit, opens herself to his “world", and in so doing takes a step out of her own. Her deepening relationship with the spirit and ongoing mistreatment of her own children represent this liminal space that she has entered. Similarly, the children’s withdrawal from their mother ultimately propels them into the domain of the curassows. Finally, the mother’s own liminality leads her to abandon her human speech and resort only to crying, such that she transforms into the ever-crying \textit{bëbë} bird.
    
	The Deer Story is also representative of the multilingual Vaupés region. The Hupd’äh, like the other peoples of the area, are participants in the regional melting pot of culture, discourse, and language, which has led to striking parallels in verbal art and other practices across many of the Upper Rio Negro peoples (see e.g. \citealt{EppsandStenzel2013}). The Deer Story, like many others, is told widely in the region, and (as noted above) the Deer’s Tomb constellation is recognized among other groups as well. Like many other peoples in the region, the Hupd’äh are multilingual, but avoid overt mixing of languages; code-switching is therefore tightly constrained, but is acceptable and even preferred for certain functions – in particular, marking the speech of entities who are treated as social “others" in narrative. Isabel’s telling of the Deer Story makes expert use of this device: when the deer spirit first comes to the woman, he inquires about the children in Tukano, the principal second language of the middle Tiquié River Hupd’äh. Here Isabel departs from her narrative momentarily to comment (perhaps for the benefit of the recording?) that the spirit apparently spoke in “River-Indian language" (Tukano). Later on, when the spirit makes the same inquiry, Isabel comments that he still is speaking in Tukano, but provides his quoted speech in Hup. Finally, at the end of the story, Isabel quotes the woman’s crying “my children! my children" – but notably this quote is given in a \textit{mix} of Hup and Tukano (\textit{nɨ põ’ra!} ‘my[Hup] children[Tukano]’). Her representation of this mixed-language cry appears to index its semi-human, transformative quality, i.e. a metaphorical use of code-switching. Moreover, the compound (serial) verb construction – which presents components of an event as a conceptually linked package – that Isabel uses to describe the event itself highlights that the transformation comes about \textit{through} this act of speaking: \textit{ʔɨd-ham-döhö-} [speak-go-transform-] ‘went saying and transformed’.
    
	In Isabel’s telling of the Deer Story, she inserts a number of comments in the narrative; some of these are directly relevant to the story, while others reflect on her own narration. She observes at several points that the events she describes – which are understood to have taken place in a distant, mythic past – set a sort of precedent that shaped the world as we currently know it; for example, that women who remarry sometimes do not treat the children of their first marriage well, and that the leaves fed to the baby deer were exactly those that deer now eat from gardens, thus damaging the crop. Among her more self-reflective comments, she stumbles slightly over the first Tukano utterance of the spirit, and laughs that she did not deliver it so well; later on she hesitates momentarily and comments that she is trying to remember the story line. I have moved this second type of comments out of the main text and into the footnotes, so as not to distract from the flow of the story.
    
	The text also makes use of a number of notable grammatical and discursive features that are characteristic of the Hup narrative genre more generally. The reported evidential (\textit{mah}) is heavily used throughout, normally at least once per main clause, while the inferred evidential marker (\textit{sud}) is mostly limited to quoted speech (such as, for example, the children are speculating about their mother’s actions). The nonvisual evidential (\textit{hõ}) occurs only in one of Isabel’s asides, where she is commenting on her memory of the story; her asides also include a number of instances of the inferred evidential. Hup’s second inferred or assumed evidential (\textit{ni}), which is restricted to past tense and is less dependent on tangible evidence, also occurs occasionally in the text. The distant past contrast marker (\textit{s'ã́h}) only appears sporadically, in keeping with its generally infrequent use in Hup discourse, although some speakers use it more regularly in traditional narrative to index the distant time of the events and/or of when they themselves learned the stories (both considerations seem to be relevant). Other discursively important grammatical resources include the compound (serial) verb constructions, which offer a neat conceptual packaging of associated events or sub-events, as in the example of speaking and transforming given above. Finally, the text provides ample illustration of the head-tail linkage strategy that is common in Hup narrative, such that preceding clauses are often briefly summarized in the first part of the following sentence, marked with the sequential suffix \textit{–yö́ʔ} (i.e., ‘having done [verb], ...’). These characteristics of Hup grammar and discourse are generally in keeping with those found in other Vaupés languages, while also exhibiting certain differences – for example, the very sporadic use of past-tense marking contrasts with its ubiquitous use in Tukanoan languages, and Stenzel (this volume) notes that Kotiria narrative makes much lighter use of reported evidential marking. Otherwise, closely similar evidential categories, compound/serial verb constructions, and head-tail linkage strategies are widely represented in the area, and Hup’s fairly rigid verb-final constituent order, sensitivity of object (non-subject) case marking to animacy and definiteness, use of nominal classifiers, and range of aspectual categories are likewise generally consistent with a wider Vaupés linguistic profile (e.g. \citealt{Gomez-Imbert1996,Aikhenvald2002,Epps2007,Stenzel2013}; see also Stenzel, this volume). More information on these and many other aspects of Hup grammar can be found in \citet{Epps2008}.
    
	The transcription conventions followed here make use of the Hup practical orthography, which has been adopted by Hup teachers in the local schools (see
\citealt{Ramirez2006}). The majority of symbols correspond to those found in the International Phonetic Alphabet, with the following exceptions. For vowels, orthographic <ë> = IPA /e/, <ä> = /ǝ/, <ö> = /o/, <e> = /æ/, and <o> = /ɔ/. For consonants, <s> = /c/ (palatal voiceless stop), with a word-initial allophone [ʃ], < ' > = /ʔ/, <j> = /ɟ/, and <y> = /j/. Hup’s phonological inventory contains voiced, voiceless, and glottalized consonants; while glottalized consonants do not contrast underlyingly for voicing, the practical orthography distinguishes the allophones <s’/j’> and <k’/g’>, respectively (as realized in syllable onset and coda positions). Nasalization is a morpheme-level prosody in Hup, as is the case in other Vaupés languages, but nasal and oral allophones of voiced obstruents (<m/b> and <n/d>) are distinguished depending on whether the context is oral or nasal; otherwise, a tilde on the vowel indicates that the entire syllable (in most cases, morpheme) is nasalized. Vowel-copying suffixes take their nasal/oral quality from the final element in the stem they attach to. Hup has two contrastive tones, which occur only on stressed syllables; these are marked via a diacritic on the vowel of the relevant syllable (v́ = high tone [of which a falling contour is an allophone], v̀ = rising tone).

    
	The first line of transcribed text follows the full set of conventions of word segmentation and phonemic representation current in the practical orthography. The second transcription line deviates from these conventions in several respects: it provides a morphological breakdown, and in so doing it indicates morpheme/clitic boundaries within the phonological word (via - and =, respectively), in keeping with the morphological analysis provided in \citet{Epps2008}, including where these are represented by spaces between etyma in the practical orthography. The second transcription line also includes morpheme-initial glottal stops (which are phonemic but are omitted in the practical orthography), since these help to clarify the distinction between consonant- and vowel-initial suffixes within phonological words; it also uses the IPA symbol <ʔ> for the glottal stop consonant in order to differentiate this phoneme from the glottalized consonants (represented as <C’>). The third line provides a morpheme-by-morpheme gloss, with a list of non-standard abbreviations provided at the end of the text.
    
	The line-by-line translations attempt to maintain a relatively literal reading that closely mirrors the discourse norms of the original, while balancing this goal with readability in English. In general, I have leaned toward transparency in the morpheme-by-morpheme glosses that correspond to relatively lexicalized multimorphemic constructions, while the meaning of the collocation as a whole is given in the translation line; e.g. \textit{s’äb-te-yɨ’} (night-still\textsc{-adv}) ‘morning’.
    
\newpage
\section{Mohõ̀y wäd nɨ̀h pɨnɨ̀g}
\translatedtitle{‘Story of the Deer Spirit’}\\
\translatedtitle{‘História do Espírito Veado’}\footnote{Recordings of this story are available from  \url{https://zenodo.org/record/999238}}

\ea  Sä̀’ ségep mah tɨh hámáh.\\ 
\gll sä̀ʔ ség-ep=mah tɨh hám-áh\\
     shrimp net-\textsc{dep=rep} 3\textsc{sg} go-\textsc{decl}\\
\glt ‘She went netting shrimp, it’s said.'
\glt ‘Ela foi pegando camarão, dizem.'
\z 
 
\ea  Hɨn’ɨ̀h pã̀ãp mah, tẽhíp pã̀ãp mah tɨh hámáh.\\ 
\gll hɨn’ɨ̀h pã̀-ãp=mah, tẽh=ʔíp pã̀-ãp=mah tɨh hám-áh\\
     what \textsc{neg.ex-dep=rep} child=father \textsc{neg.ex-dep=rep} 3\textsc{sg} go-\textsc{decl}\\
\glt ‘With nothing, it’s said, with no husband, it’s said, she went.'
\glt ‘Sem nada, dizem, sem marido, dizem, ela foi.'
\z 

\ea  Ham yö́’, dëh-míit sä́’ mah tɨh ségéh.\\ 
\gll ham-yö́ʔ, dëh=mí-it sä́ʔ=mah tɨh ség-éh\\
     go\textsc{-seq} water=course\textsc{-obl} shrimp\textsc{=rep} \textsc{3sg} net\textsc{-decl}\\
\glt ‘Having gone, it’s said, she was netting shrimp in the stream.'
\glt ‘Tendo ido, dizem, ela estava pegando camarão no igaparé.'
\z 

\ea Yúp mah tɨh seg péét mah, d’ö̀bn’àn tɨh käk w’öb pe níh, húpup ĩhĩ́h.\\
\gll yúp=mah tɨh seg-pé-ét=mah, d’ö̀b=n’àn tɨh käk-w’öb-pe-ní-h húp-up=ʔĩh-ĩ́h\\ 
     that\textsc{=rep} \textsc{3sg} net-go.upstream\textsc{-obl=rep} acará\textsc{=pl.obj} \textsc{3sg} pull-set-go.upstream\textsc{-infr2-decl} person\textsc{-dep=msc-decl}\\
\glt ‘So, it’s said, as she went upstream netting, it’s said, he was (also) going upstream, fishing acará (\textit{Pterophyllum} sp.) and setting them out (for her), a man.'
\glt ‘Aí, dizem, enquanto ela ia rio acima pegando camarão, dizem, ele (também) estava indo rio acima, pescando acará e deixando lá (para ela), um homem.'
\z

\newpage
\ea  Të́ käk w’öb pe yö́’, të́ tɨh-k’etd’ö́hanay mah, tiyì’ b’ay këy d’öb k’ët níayáh, mohòyóh.\\ 
\gll të́ käk-w’öb-pe-yö́ʔ, të́ tɨh=k’etd’ö́h-an-ay=mah, tiyìʔ=b’ay këy-d’öb-k’ët-ní-ay-áh, mohòy-óh.\\
     until pull-set-go.upstream\textsc{-seq} until \textsc{3sg=}end\textsc{-dir-inch=rep} man=again see-descend-stand-be\textsc{-inch-decl} deer\textsc{-decl}\\
\glt ‘Until, having gone upstream fishing and setting out (the fish), all the way to the headwaters (of the stream), it’s said, the man was standing (on the bank) looking down (at her), a deer.'
\glt ‘Até, tendo indo rio acima pescando e deixando (o peixe), até a cabeceira (do igarapé), dizem, o homem ficou (na beira) olhando para baixo para ela.'
\z 

\ea  Yúp mah, “Ùy sap ũ̀h àn hõ̀p käk w’öb pée’ páh?” tɨh nóop b’ay.\\ 
\gll yúp=mah, ʔùy sap ʔũ̀h ʔàn hõ̀p käk-w’öb-pé-e’ páh? tɨh nó-op=b’ay.\\
     that\textsc{=rep} who \textsc{ints} \textsc{epist} \textsc{1sg.obj} fish pull-set-go.upstream\textsc{-q} \textsc{prox.cntr} \textsc{3sg} say\textsc{-dep=}again\\
\glt ‘So, it’s said, “Who can it be, who has been going upstream fishing and setting out (fish) for me?” she said.'
\glt ‘Aí, dizem, “Quem pode ser, que estava rio acima para a cima e deixando (peixe) para mim?" ela falou.'
\z 

\ea  Yɨnɨh yö́’ mah, hõ̀p wed túup k’ṍhṍy nih, tẽhíp pã̀ãp k’ṍhṍy nih, hĩ́ tɨh d’ö’ pe yɨ́’ɨ́h.\\ 
\gll yɨ-nɨh-yö́ʔ=mah, hõ̀p wed-tú-up k’ṍh-ṍy=nih, tẽh=ʔíp pã̀-ãp k’ṍh-ṍy=nih, hĩ́ tɨh d’öʔ-pe-yɨ́ʔ-ɨ́h.\\
     \textsc{dem.itg-}be.like\textsc{-seq=rep} fish eat-want\textsc{-dep} be\textsc{-dynm=emph.co} child=father \textsc{neg.ex-dep} be\textsc{-dynm=emph.co} only \textsc{3sg} take-go.upstream\textsc{-tel-dynm}\\
\glt ‘So, it’s said, wanting to eat fish, being without a husband, she just went upstream taking the fish.'{\footnotemark}\footnotetext{Isabel uses the verb \textit{k'õh-} ‘be' throughout this text; this verb is a salient feature of the Japu dialect (whereas the middle Tiquié dialects use only the form \textit{ni-}), and is often a source of comment among speakers regarding dialectal differences.}
\glt ‘Aí, dizem, querendo comer peixe, sem marido, ela ia para rio acima pegando o peixe.'
\z 

\ea  D’ö’ pe yö́’ mah, “Ya’àp yɨ’ ãh wed të́h, nííy,” no yö́’ mah, tɨh hup käd b’ay yɨ’ kamí mah, tɨ́hàn tɨh wíçíy.\\ 
\gll d’ö’-pe-yö́ʔ=mah, ya’àp=yɨʔ ʔãh wed-të́-h, ní-íy, no-yö́ʔ=mah, tɨh hup-käd-b’ay-yɨʔ-kamí=mah, tɨ́h-àn tɨh wíç-íy.\\
     take-go.upstream\textsc{-seq=rep} all.gone\textsc{=adv} \textsc{1sg} eat\textsc{-fut-decl} be\textsc{-dynm} say\textsc{-seq=rep} \textsc{3sg} \textsc{refl-}pass-return\textsc{-tel}-moment.of\textsc{=rep} \textsc{3sg-obj} \textsc{3sg} whistle\textsc{-dynm}\\
\glt ‘Taking (the fish) as she went upstream, it’s said, saying, “Just this I’ll (take to) eat,” it’s said, just as she turned around to go back, he whistled to her.'
\glt ‘Pegando (o peixe), rio acima, dizem, falando, “Só isso vou (levar para) comer," no momento em que ela virou para voltar, ele assobiou para ela.'
\z 

\ea  Tɨ́hàn tɨh wíçíy këyö́’ mah, “Ùy sap ũ̀h, àn tiyì’ pã̀ãt, àn wiç k’ët k’ö́’ö’ páh?”{\footnotemark}\footnotetext{The particle \textit{páh} marks recent past, but is used primarily in a contrastive sense; it is the counterpart of the distant past contrast marker mentioned above in the Introduction.} no yö́’ mah.\\ 
\gll tɨ́hàn tɨh wíç-íy këyö́ʔ=mah, ʔùy sap ʔũ̀h, ʔàn tiyìʔ pã̀-ãt, ʔàn wiç-k’ët-k’ö́ʔ-öʔ páh? no-yö́ʔ=mah.\\
     \textsc{3sg.obj} \textsc{3sg} whistle\textsc{-dynm} because\textsc{=rep} who \textsc{ints} \textsc{epist} \textsc{1sg.obj} man \textsc{neg.ex-obl} \textsc{1sg.obj} whistle-stand-go.around\textsc{-q} \textsc{prox.cntr} say\textsc{-seq=rep}\\
\glt ‘As he whistled, it’s said, “Who could it be, (I being) without a husband, who could be going around whistling for me?” she said, it’s said.'
\glt ‘Como ele assobiou, dizem, “Quem pode ser, (eu) sem marido, quem está por aí assobiando para mim?" ela falou, dizem.'
\z 

\ea  Tɨh këy sop k’ë́të́h, tɨh këy sop k’ë́të́t mah, tɨh mè’ sój d’öb k’ët pö́ayáh.\\ 
\gll tɨh këy-sop-k’ë́t-ë́h, tɨh këy-sop-k’ë́t-ë́t=mah, tɨh m’èʔ sój d’öb-k’ët-pö́-ay-áh.\\
     \textsc{3sg} see-ascend-stand\textsc{-decl} \textsc{3sg} see-go.up-stand\textsc{-obl=rep} \textsc{3sg} carajuru brilliant descend-stand\textsc{-emph1-inch-decl}\\
\glt ‘She stood looking up (toward the bank), as she stood looking up, it’s said, he stood looking down, brilliant with \textit{carajuru}.{\footnotemark}\footnotetext{\textit{Carajuru} is the regional term for the \textit{Arrabidaea chica} plant and the red body paint made from its leaves.}
\glt ‘Ela ficou olhando para cima (na beira), e enquanto ficava olhando, dizem, ele olhava para baixo, brilhante com carajuru.'
\z 

\ea  Yup m’é’ sój d’öb k’ët yö́’ mah, tɨ́hàn tɨh ɨ́dɨ́h.\\ 
\gll yup m’éʔ sój d’öb-k’ët-yö́ʔ=mah, tɨ́h-àn tɨh ʔɨ́d-ɨ́h.\\
     that carajuru brilliant descend-stand\textsc{-seq=rep} \textsc{3sg-obj} \textsc{3sg} speak\textsc{-decl}\\
\glt ‘Standing there looking down, brilliant with \textit{carajuru}, it’s said, he spoke to her.'
\glt ‘Ficando lá olhando para ela, brilhante com carajuru, dizem, ele falou para ela.'
\z 

\ea  Yɨ́t páh, “Hõ̀p ámàn ãh käk w’öb péét, d’ö́’ö́y ám páh?” nóóy mah.\\ 
\gll yɨ́t páh, hõ̀p ʔám-àn ʔãh käk-w’öb-pé-ét, d’ö́ʔ-ö́y ʔám páh? nó-óy=mah.\\
     thus \textsc{prox.cntr} fish \textsc{2sg-obl} \textsc{1sg} pull-set-go.upstream\textsc{-decl} take\textsc{-dynm} \textsc{2sg} \textsc{prox.cntr} say\textsc{-dynm=rep}\\
\glt ‘And then, “Where I went upstream catching fish and setting them out for you; have you taken them?” he said, it’s said.'
\glt ‘Aí, "Lá onde fui rio acima, pescando e deixando peixe, você pegou?" ele falou, dizem.'
\z 

\ea  “D’ö́’ö́y páh ã́hã́h,” nóóy mah tɨ́hɨ́h.\\ 
\gll d’ö́ʔ-ö́y páh ʔã́h-ã́h, nó-óy=mah tɨ́h-ɨ́h.\\
     take\textsc{-dynm} \textsc{prox.cntr} \textsc{1sg-decl} say\textsc{-dynm=rep} \textsc{3sg-decl}\\
\glt ‘“I have taken them,” she said, it’s said.'
\glt ‘“Eu peguei," ela falou, dizem.'
\z 

\ea  “D’ö’ wéd, am máhan wed ay yö́’ páh ámàn käk w’öb pe nííy mah.\\ 
\gll	D’ö’-wéd, ʔam máh-an wed-ʔay-yö́ʔ páh ʔám-àn käk-w’öb-pe-ní-íy=mah\\
	take-eat.\textsc{imp} \textsc{2sg} near\textsc{-dir} eat-\textsc{vent-seq} \textsc{prox.cntr} \textsc{2sg-obj} pull-set-go.upstream-be-\textsc{dynm=rep}\\
\glt ‘“Take and eat them; having gone and eaten at your place I (will) be setting out fish for you,” it’s said.' 
\glt ‘“Leve-os para comer; depois de você ir e comer em casa, estarei deixando peixe para você", dizem.'
\z 

\newpage
\ea  “Tán, am máhan, d’ú’ ãh yë të́h,” nóóy mah.\\ 
\gll tán, ʔam máh-an, d’úʔ ʔãh yë-të́-h, nó-óy=mah.\\
     later \textsc{2sg} near\textsc{-dir} evening \textsc{1sg} enter\textsc{-fut-decl} say\textsc{-dynm=rep}\\
\glt ‘“Later, in the evening I will come to you,” he said, it’s said.'
\glt ‘“Depois, no final do dia, chegarei até você," ele falou, dizem.'
\z 

\ea  “Hä̀’,” nóóy mah tɨ́hɨ́h.\\ 
\gll hä̀ʔ, nó-óy=mah tɨ́h-ɨ́h.\\
     yes say\textsc{-dynm=rep} \textsc{3sg-decl}\\
\glt ‘“All right,” she said, it’s said.'
\glt ‘“Tá bom," ela falou, dizem.'
\z 

\ea  Yɨ̃́ no yö́’ mah tɨh b’ay yɨ́’ayáh.\\ 
\gll yɨ̃́-no-yö́ʔ=mah tɨh b’ay-yɨ́ʔ-ay-áh.\\
     \textsc{dem.itg-}say\textsc{-seq=rep} \textsc{3sg} return\textsc{-tel-inch-decl}\\
\glt ‘Having said that, it’s said, she went back.'
\glt ‘Tendo dito isso, dizem, ela voltou.'
\z 

\ea  B’ay yö́’, të́ wɨdb’ay yɨ’ nííy ni yö́’ mah, tɨh tẽ́hn’àn.\\ 
\gll b’ay-yö́ʔ, të́ wɨd-b’ay-yɨʔ ní-íy ni-yö́ʔ=mah, tɨh tẽ́h=n’àn.\\
     return\textsc{-seq} until arrive-return\textsc{-tel} be\textsc{-dynm} be-\textsc{seq=rep} \textsc{3sg} offspring=\textsc{pl.obj}\\
\glt ‘She went back, until she had arrived to where her children were.'
\glt ‘Ela voltou, até chegar onde estavam as suas crianças.'
\z 

\ea  Yúp sä́’ mehn’àn tɨh k’ët hipud yö́’ mah, tɨh k’ët wed yö́’ mah, tɨh k’ët õh yɨ́’ɨ́h.\\ 
\gll yúp sä́ʔ=meh=n’àn tɨh k’ët-hipud-yö́ʔ=mah, tɨh k’ët-wed-yö́ʔ=mah, tɨh k’ët-ʔõh-yɨ́ʔ-ɨ́h.\\
    \textsc{dem.itg} shrimp\textsc{=dim=pl.obj} \textsc{3sg} stand-mix.broth\textsc{-seq=rep} \textsc{3sg} stand-eat\textsc{-seq=rep} \textsc{3sg} stand-sleep\textsc{-tel-decl}\\
\glt ‘Then, having made \textit{mojica}{\footnotemark}\footnotetext{\textit{Mojica} is a stew, usually made with fish, flavored with hot pepper, and thickened with tapioca.} for them from the little shrimp, it’s said, having fed them, she put them to sleep.'
\glt ‘Aí, depois de fazer uma mojica de pequenos camarões para elas, dizem depois de tê-las alimentado, as colocou para dormir.'
\z

\ea  Yúp k’ët õh yö́’ mah yúp, “Nɨg õháy, hẽ́gyɨ’ nɨg õh hẽ́gyɨ’ áy, kayak dë̀h äg tu yö́’ nɨg õh hẽ́gyɨ’ áy!” tɨh nóóh.\\ 
\gll yúp k’ët-ʔõh-yö́ʔ=mah yúp, “nɨg ʔõh-áy, hẽ́g-yɨʔ nɨg ʔõh-hẽ́g-yɨʔ-ʔáy, kayak=dë̀h ʔäg-tu-yö́ʔ nɨg ʔõh-hẽ́g-yɨʔ-ʔáy! tɨh nó-óh.\\
     \textsc{dem.itg} stand-sleep-\textsc{seq=rep} \textsc{dem.itg} \textsc{2pl} sleep-\textsc{inch.imp} quick\textsc{-adv} \textsc{2pl} sleep-quick\textsc{-tel-vent.imp} manioc=liquid drink-immerse\textsc{-seq} \textsc{2pl} sleep-quick\textsc{-tel-vent.imp} \textsc{3sg} say\textsc{-decl}\\
\glt ‘Putting them to sleep, “Quick, you all go to sleep quickly, having drunk up your \textit{manicuera},{\footnotemark}\footnotetext{As noted above, \textit{manicuera} is a drink made from boiled manioc juice, often flavored with fruits.} you all go to sleep quickly!” she said.'
\glt ‘Mandando eles dormir, “Rápido, durmam rápido, depois de beber toda a manicuera de vocês, durmam logo!" ela falou.'
\z

\ea  Yɨ̃ nóóy këyö́’ mah yúp, yɨd’ä̀h mèhd’äh, tɨh-dö’ mèhd’äh mah, íp pã̀ mèhd’äh, hɨd õh yɨ́’ɨ́h.\\ 
\gll yɨ̃ nó-óy këyö́ʔ=mah yúp, yɨ-d’ä̀h mèh=d’äh, tɨh=döʔ=mèh=d’äh=mah, ʔíp pã̀ mèh=d’äh, hɨd ʔõh-yɨ́ʔ-ɨ́h.\\
     \textsc{dem.itg} say\textsc{-dynm} because\textsc{=rep} \textsc{dem.itg} \textsc{dem.itg-pl} \textsc{dim=pl} \textsc{3sg}=child\textsc{=dim=pl=rep} father \textsc{neg.ex} \textsc{dim=pl} \textsc{3pl} sleep\textsc{-tel-decl}\\
\glt ‘Upon her saying this, it’s said, those little ones, those little fatherless ones, they went to sleep.'
\glt ‘Com ela falando isso, dizem, esses pequenos, esses pequenos sem pai, eles dormiram.'
\z 

\ea  Yúp mah bɨ́g nonɨ́h mah tɨh yë yɨ́’ayáh.\\ 
\gll yúp=mah bɨ́g no-nɨ́h=mah tɨh yë-yɨ́ʔ-ay-áh.\\
     that\textsc{=rep} \textsc{hab} say\textsc{-neg=rep} \textsc{3sg} enter\textsc{-tel-dynm-decl}\\
\glt ‘Then, it’s said, it was not long before he came in.' 
\glt ‘Aí, dizem, não foi muito antes dele chegar.'
\z 

\newpage
\ea  Tɨh kètd’öh sö́’ö́y ṍy’, hũ̀ytu sö́’ö́y ṍy’, hakténéyd’äh ṍy’.\\ 
\gll tɨh kètd’öh sö́ʔ-ö́y ʔṍy’, hũ̀ytu sö́ʔ-ö́y ʔṍy’, haktén-éy=d’äh ʔṍy’.\\
     \textsc{3sg} end \textsc{loc-dynm} bunch behind \textsc{loc-dynm} bunch side\textsc{-dynm-pl} bunch\\
\glt ‘With a bunch of game at the end (of a pole) in front, a bunch of game in back, bunches of game on either side.'
\glt ‘Com umas caças no extremo (de um pau) na frente, umas caças atrás, caças dos dois lados.'
\z 

\ea Mòh ṍy’d’äh  k’ṍh maháh.  Hisɨhnɨ́h mah yɨd’ä̀hä́h.  Hõ̀pd’ähä́t yɨ’, mòhd’ähä́t yɨ’ mah tɨh k’õhníh.\\ 
\gll mòh ʔṍyʔ=d’äh  k’ṍh-mah-áh.  hisɨhnɨ́h=mah yɨ-d’ä̀h-ä́h.  hõ̀p=d’äh-ä́t=yɨʔ, mòh=d’äh-ä́t=yɨʔ=mah tɨh k’õh-ní-h.\\
     tinamou bunch\textsc{=pl} be\textsc{-rep-decl} many\textsc{=rep} \textsc{dem.itg-pl-decl} fish\textsc{=pl-obl=adv} tinamou\textsc{=pl-obl=adv=rep} \textsc{3sg} be\textsc{-infr2-decl}\\
\glt ‘They were bunches of tinamous.{\footnotemark}\footnotetext{These birds of the family \textit{Tinamidae} are a preferred type of game.} Lots of them. With fish, with tinamous he was thus (laden).'
\glt ‘Tinham inambus. Muitos. Com peixes, com inambus, ele estava (carregado).'
\z 

\newpage
\ea  Yɨnɨh yö́’ mah yúp, “Marĩ põ’ra, marĩ põ’ra karĩrã?” nóóy mah.\\ 
\gll yɨ-nɨh-yö́ʔ=mah yúp, marĩ põ’ra, marĩ põ’ra karĩ-rã? nó-óy=mah\\
     \textsc{dem.itg-}be.like\textsc{-seq=rep} \textsc{dem.itg} [\textsc{1pl} children \textsc{1pl} children sleep\textsc{-pl}] say\textsc{-dynm=rep}\\
\glt ‘Thus, it’s said, [in Tukano] “Are our children, our children asleep?” he said, it’s said.'\footnote{Isabel stumbled a little over the Tukano phrase, and added a further comment:\\ 
\begin{exe}
\ex Yúp ãh d’äh d’äh ham nɨ́h dɨ’ kodé, wòh ɨd mɨ̀’ sud ũhniy yẽ́h yúwúh.\\
\gll Yúp ʔãh d’äh-d’äh-ham-nɨ́h dɨʔ-kodé, wòh ʔɨd mɨ̀ʔ sud-ʔũhniy yẽ́h yúw-úh.\\
\textsc{dem.itg} \textsc{1sg} send-send-go-\textsc{neg} remain-\textsc{vdim} river.indian speech \textsc{under} \textsc{infr}-maybe \textsc{frust} \textsc{dem.itg-decl}\\
\glt ‘I didn’t say that very well, even though it was supposed to be Tukano.’
\glt ‘Não falei muito bem, mesmo que deveria ter sido em Tucano.’
\end{exe}}
\glt ‘Aí, dizem, [em Tukano] “Os nossos filhos, nossos filhos estão dormindo?" ele falou, dizem.' 
\z 

\ea  Wòh ĩh sud ũhníy.  “Marĩ põ’ra karĩrã?” nóóy mah.\\ 
\gll wòh=ʔĩh=sud ʔũhníy. Marĩ põ’ra karĩ-rã? nó-óy=mah.\\
     river.indian\textsc{=msc=infr1} maybe [\textsc{1pl} children sleep\textsc{-pl}] say\textsc{-dynm=rep}\\
\glt ‘He was apparently a River Indian, perhaps.{\footnotemark}\footnotetext{As discussed in the Introduction, the use of Tukano marks the deer spirit as an “Other". Here Isabel's meta-comment regarding his choice of language may have been motivated by the fact that her story was being recorded.} [In Tukano] “Are our children asleep?” he said.'
\glt ‘Era um indio do rio, parece. [em Tukano] “Nossos filhos estão dormindo?" ele falou.'
\z 

\newpage
\ea  \textit{Yúp mah, yɨno yö́’ mah yúp, yúp hõ̀p tɨh k’ët wédéh, hõ̀p tɨh k’ët wèd, mòh tɨh k’ët wèd, nííy mah.}\\ 
\gll yúp=mah, yɨ-no-yö́ʔ=mah yúp, yúp hõ̀p tɨh k’ët-wéd-éh,{\footnotemark} hõ̀p tɨh k’ët-wèd, mòh tɨh k’ët-wèd, ní-íy=mah.\\
     \textsc{dem.itg=rep} \textsc{dem.itg-}say\textsc{-seq=rep} \textsc{dem.itg} \textsc{dem.itg} fish \textsc{3sg} stand-eat\textsc{-decl} fish \textsc{3sg} stand-eat tinamou \textsc{3sg} stand-eat be\textsc{-dynm=rep}\\
\glt ‘Having said that, it’s said, he gave her fish to eat; he went on giving her fish to eat, to give her tinamous to eat, it’s said.'
\glt ‘Tendo falado assim, dizem, ele deu peixe para ela comer; ele continuou dando peixe, inambu, dizem.'
\z 
\footnotetext{\label{fn:hup:caus}Hup derives causative constructions by means of compounded verb roots. The verb \textit{d'öʔ-} ‘take' is used for direct causation; \textit{d'äh-} for less direct causation, and \textit{k'ët-} ‘stand' for indirect or ‘sociative' causation, as in this example.} 

\ea  Yɨ́t tɨh nɨ́hɨ́t yɨ’ tɨh k’ët hiwag yɨ́’ɨ́h.\\ 
\gll yɨ́t tɨh nɨ́h-ɨ́t=yɨʔ tɨh k’ët-hi-wag-yɨ́ʔ-ɨ́h.\\
     thus \textsc{3sg} be.like\textsc{-obl=adv} \textsc{3sg} stand\textsc{-fact-}day\textsc{-tel-decl}\\
\glt ‘Doing thus, he accompanied her until dawn.'
\glt ‘Assim, ele a acompanhou até amanhecer.'
\z 

\ea  Të sadakà’ õh säwä’ të́g kót’ah meh mah, tɨh tẽh-ínít tɨh säk te’ sak k’ã’ yɨ́’ayáh.\\ 
\gll të sadakàʔ ʔõh-säwäʔ-të́g kót’ah=meh=mah, tɨh tẽh=ʔín-ít tɨh säk teʔ-sak-k’ãʔ-yɨ́ʔ-ay-áh.\\
     until chicken sleep-wake\textsc{-fut} before\textsc{=dim=rep} \textsc{3sg} child=mother\textsc{-obl} \textsc{3sg} buttocks join.with-go.up-hang\textsc{-tel-inch-decl}\\
\glt ‘Until just before the time that the rooster wakes and crows, he lay together with his wife in the hammock.'
\glt ‘Até pouco antes do tempo do galo acordar e cantar, ele ficava deitado na rede com a mulher dele.'
\z 

\newpage
\ea  Yɨnɨh yö́’ mah yúp sadakà’ õh säwä’ kamí pɨ́d mah tɨh way yɨ́’ay pɨdɨp b’ay.\\ 
\gll yɨ-nɨh-yö́ʔ=mah yúp sadakàʔ ʔõh-säwäʔ-kamí pɨ́d=mah tɨh way-yɨ́ʔ-ay pɨd-ɨp=b’ay.\\
     \textsc{dem.itg-}be.like\textsc{-seq=rep} that chicken sleep-wake-moment.of \textsc{dist=rep} \textsc{3sg} go.out\textsc{-tel-inch} \textsc{distr-dep}=again\\\\
\glt ‘Thus, it’s said, at the time when the rooster awakes and crows, it’s said, he went out again.'
\glt ‘Assim, dizem, no momento em que o galo acorda e canta, dizem, ele foi embora de novo.'
\z 

\ea  Të́ way yö́’ mah tɨh s’ùg kakáh ham yɨ’ ni pɨ́dɨ́h, yup tiyì’íh.\\ 
\gll të́ way-yö́ʔ=mah tɨh s’ùg kakáh ham-yɨʔ-ni-pɨ́d-ɨ́h, yup tiyìʔ-íh.\\
     until go.out\textsc{-seq=rep} \textsc{3sg} forest among go\textsc{-tel-}be\textsc{-distr-decl} that man\textsc{-decl}\\
\glt ‘On going out, it’s said, he went off into the forest, that man.'
\glt ‘Saindo, dizem, ele foi embora no mato, esse homem.'
\z 

\ea  Yɨnɨh mɨ̀’ mah tɨh, “Nɨg s’om áyáy, tẽ́h!” no d’äh d’öb yɨ’ pɨ́dɨ́h.\\ 
\gll yɨ-nɨh mɨ̀ʔ=mah tɨh, “Nɨg s’om-ʔáy-áy, tẽ́h!” no-d’äh-d’öb-yɨʔ-pɨ́d-ɨ́h.\\
     \textsc{dem.itg-}be.like \textsc{under=rep} \textsc{3sg} \textsc{2pl} bathe\textsc{-vent-inch.imp} offspring say-send-descend\textsc{-tel-distr-decl}\\
\glt ‘With that, it’s said, “You all go bathe, children!” she said, sending them down to the water.'
\glt ‘Assim, dizem, “Vão tomar banho, filhos!" ela falou, mandando eles para o igarapé.'
\z 

\ea  Yɨnɨh yö́’, s’äbtéyɨ’ b’òt ham yö́’ mah, kayak dë̀h tɨh bɨ́’ɨ́h.\\ 
\gll yɨ-nɨh-yö́ʔ, s’äb-té-yɨʔ b’òt ham-yö́ʔ=mah, kayak dë̀h tɨh bɨ́ʔ-ɨ́h.\\
     \textsc{dem.itg-}be.like\textsc{-seq} night-still\textsc{-adv} swidden go\textsc{-seq=rep} manioc liquid \textsc{3sg} make\textsc{-decl}\\
\glt ‘Having done thus, having gone early in the morning to her garden, it’s said, she prepared \textit{manicuera}.'
\glt ‘Tendo feito assim, tendo ido de manhã cedo para a roça, dizem, ela preparou a manicuera.'
\z 

\ea  Kayak dë̀h bɨ’ yö́’ mah yúp, dɨ’ téyɨ’ pɨ́d mah, tɨh-dëhwàh mah tɨh tẽ̀hn’àn tɨh b’äh k’ët ṹhṹh.\\ 
\gll kayak dë̀h bɨʔ-yö́ʔ=mah yúp, dɨʔ té-yɨʔ pɨ́d=mah, tɨh=dëh-wàh=mah tɨh tẽ̀h=n’àn tɨh b’äh-k’ët-ʔṹh-ṹh.\\
     manioc liquid make\textsc{-seq=rep} that remain still\textsc{-adv} \textsc{distr=rep} \textsc{3sg=}liquid-old.food\textsc{=rep} \textsc{3sg} offspring\textsc{=pl.obj} \textsc{3sg} pour-stand\textsc{-appl-decl}\\
\glt ‘Having prepared \textit{manicuera}, it’s said, she would take a little that was left over, it’s said, the part that isn’t tasty, it’s said, and she would pour that out for her children.'
\glt ‘Tendo preparado a manicuera, dizem, ela tirou um pouco que sobrou, a parte sem gosto, e a despejou para os seus filhos'
\z 

\ea  Yɨnɨh mɨ̀’ mah yúp tɨh tẽhípàn b’ay tɨh-dë̀h húp b’ay, tɨh k’ä́h náw, sanàát hitú’úp náw, hipud y’et yɨ́’ɨ́h.\\ 
\gll yɨ-nɨh mɨ̀ʔ=mah yúp tɨh tẽh=íp-àn=b’ay tɨh=dë̀h húp=b’ay, tɨh=k’ä́h náw, sanà-át hi-túʔ-úp náw, hipud-y’et-yɨ́ʔ-ɨ́h.\\
     \textsc{dem.itg-}be.like \textsc{under=rep} that \textsc{3sg} child=father\textsc{-obj}=again \textsc{3sg=}liquid beautiful=again \textsc{3sg-}sweet good pineapple\textsc{-obl} \textsc{fact-}immerse\textsc{-dep} good mix.broth-lay\textsc{-tel-decl}\\
\glt ‘But, it’s said, for her husband, she would mix up good \textit{manicuera}, sweet, mixed nicely with pineapple.'
\glt ‘Mas, dizem, para o marido dela ela misturava a manicuera gostosa, doce, bem mixturada com abacaxí.'
\z 

\ea  Pë̀dë́t tɨh-kúút tɨh hipúdup, náw mah.\\ 
\gll pë̀d-ë́t tɨh=kú-út tɨh hipúd-up, náw=mah.\\
     cunuri\textsc{-obl} \textsc{3sg-}age.bury\textsc{-obl} \textsc{3sg} mix.broth\textsc{-dep} good\textsc{=rep}\\
\glt ‘She mixed it with aged \textit{cunuri};{\footnotemark}\footnotetext{The nuts of the \textit{cunuri} tree (\textit{Cunuria spruceana}) are prepared via a technique of burying them in the ground and leaving them for some time to ferment.} it was good, it's said.'
\glt ‘Ela misturou com cunuri enterrado, era muito boa, dizem.'
\z 

\newpage
\ea  Tɨh bɨ’ y’et yɨ’ pɨ́dɨ́h, yúp tɨh tẽhípànáh, mohòy wädànáh.{\footnotemark}\footnotetext{The ‘respected' marker \textit{wäd} is an honorific device used for male referents, derived from \textit{wähäd} ‘old (male)' (compare \textit{wa}, for old/respected female referents).}\\ 
\gll tɨh bɨʔ-y’et-yɨʔ-pɨ́d-ɨ́h, yúp tɨh tẽh=ʔíp-àn-áh, mohòy=wäd-àn-áh.\\
     \textsc{3sg} make-lay\textsc{-tel-dist-decl} that \textsc{3sg} child=father\textsc{-obj-decl} deer\textsc{=resp-obj-decl}\\
\glt ‘She would make it and set it down, for her husband, the deer.'
\glt ‘Ela fazia e colocava, dizem, para o marido dela, o veado.'
\z 

\ea  Yúp mah yúp tɨhpày mah yúp tɨh tẽ̀hn’àn tɨh bɨ’ nó’op b’ay.\\ 
\gll yúp=mah yúp tɨh=pày=mah yúp tɨh tẽ̀h=n’àn tɨh bɨʔ-nóʔ-op=b’ay.\\
     that\textsc{=rep} that \textsc{3sg=}bad\textsc{=rep} that \textsc{3sg} offspring\textsc{=pl.obj} \textsc{3sg} make-give\textsc{-dep}=again\\
\glt ‘Thus, it’s said, she did badly for her children.'
\glt ‘Assim, dizem, ela fez mal para seus filhos.'
\z 

\ea  Nutèn ã́yd’äh, ɨn tè̃̀hn’àn hitama’ nɨ́h, ɨn ní-tëg yɨ’ tɨh nɨ́hɨp mah yúp hiníp.\\ 
\gll nutèn ʔã́y=d’äh, ʔɨn tè̃̀h=n’àn hitamaʔ-nɨ́h, ʔɨn ní-tëg=yɨʔ tɨh nɨ́h-ɨp=mah yúp=hin-íp\\
     today woman\textsc{=pl} \textsc{1pl} offspring\textsc{-obj.pl} do.well.by\textsc{-neg} \textsc{1pl} be\textsc{-clf}:\textsc{thing=adv} \textsc{3sg} be.like\textsc{-dep=rep} that=also\textsc{-dep}\\
\glt ‘Women of today, (when) we (who remarry) don’t treat our children well, our way is as she did, it’s said, likewise.'
\glt ‘As mulheres de hoje, (quando casam de novo e) não tratam bem nossos filhos, esse jeito é como o jeito dela, dizem, assim mesmo.'
\z 

\ea  Yúwàn ùy d’äh këy d’äh hám b’ayáh, yúpyɨ' tɨh bɨ’ ni nɨh níh.\\  
\gll yúw-àn=ʔùy=d’äh këy-d’äh-hám-b’ay-áh, yúp-yɨʔ tɨh bɨʔ-ni-nɨh-ní-h.\\ 
	that\textsc{-obj}=who=\textsc{pl} see-send-go=again\textsc{-dep-decl} that\textsc{-adv} \textsc{3sg} make-be-be.like\textsc{-infr2-decl}\\
\glt ‘Because that's how it is for those people, thus in this way she behaved.' 
\glt ‘Por que é assim mesmo para essas pessoas, assim desse jeito ela fez.'
\z 

\newpage
\ea  Hitama’nɨ́h nutèn ã́yd’äh ɨn hiníh tíh.\\ 
\gll hitamaʔ-nɨ́h nutèn ʔã́y=d’äh ʔɨn=hin-íh tíh.\\
     do.well.by\textsc{-neg} today woman\textsc{=pl} \textsc{1pl=}also\textsc{-decl} \textsc{emph2}\\
\glt ‘We (women who remarry) of today likewise do not treat (our children) well.'
\glt ‘Nós (mulheres que casam outra vez) hoje em dia também não tratamos bem nossos filhos.'
\z 

\ea  Yɨnɨ́hɨ́y mah yup d’ú’ nénéy, nɨ́hɨ́y pɨ́d mah, “ɨn tẽ́h=d’äh õh yɨ’ sɨ̃́wɨ̃́y hɨ́d?” tɨh no wɨdyë pɨ́dɨ́h, yup wòh ɨ́dɨtɨ́h.\\ 
\gll yɨ-nɨ́h-ɨ́y=mah yup d’úʔ nén-éy, nɨ́h-ɨ́y pɨ́d=mah, ʔɨn tẽ́h=d’äh ʔõh-yɨʔ-sɨ̃́w-ɨ̃́y hɨ́d? tɨh no-wɨdyë-pɨ́d-ɨ́h, yup wòh ʔɨ́d-ɨt-ɨ́h.\\
     \textsc{dem.itg-}be.like\textsc{-dynm=rep} that evening come\textsc{-dynm} be.like\textsc{-dynm} \textsc{distr=rep} \textsc{1pl} offpring\textsc{=pl} sleep\textsc{-tel-compl-dynm} \textsc{3pl} \textsc{3sg} say-arrive.enter\textsc{-distr-decl} that river.indian language\textsc{-obl-decl}\\
\glt ‘Then like that, it’s said, the evening would arrive, it would go like this, it’s said: “Are our children already asleep?” he would say as he entered, he would speak in River Indian language.'{\footnotemark}\footnotetext{Here Isabel provides the Deer Spirit's quoted speech in Hup, but comments that he actually would have spoken in Tukano.}
\glt ‘Assim, dizem, no final do dia, era sempre assim, dizem: “Nossos filhos já estão dormindo?" ele dizia, entrando, dizia na língua dos indios do rio.'
\z 

\ea  Yɨnɨh yö́’ pɨ́d mah yúp, dɨ’ téyɨ’ pɨ́d, “nɨg õh yɨ́’, hẽ̀gay!” tɨh no pɨ́dɨ́h.\\ 
\gll yɨ-nɨh-yö́ʔ pɨ́d=mah yúp, dɨʔ té=yɨʔ pɨ́d, nɨg ʔõh-yɨ́ʔ, hẽ̀g-ay! tɨh no-pɨ́d-ɨ́h.\\
     \textsc{dem.itg-}be.like\textsc{-seq} \textsc{distr=rep} that remain still\textsc{=adv} \textsc{distr} \textsc{2pl} sleep\textsc{-tel.imp} quick\textsc{-inch} \textsc{3sg} say\textsc{-distr-decl}\\
\glt ‘Thus, it’s said, just before (he would come), “You all go to sleep, quickly!” she would say.'
\glt ‘Aí, dizem, pouco antes (dele chegar), “Vocês durmam logo!" ela falava."
\z 

\newpage
\ea  Yúp mah yúp, ya’ápyɨ’ pɨ́d mah yup tɨh d’ö’ níh, tɨh kètd’öh sö́’ö́y mòh õ̀y’, hũ̀ytu sö́’ö́y mòh õ̀y’, háktenéyd’äh hṹ sáp ni bahadnɨ́h pɨ́d mah tɨh yë́ë́h.\\ 
\gll yúp=mah yúp, yaʔáp=yɨʔ pɨ́d=mah yup tɨh d’öʔ-ní-h, tɨh kètd’öh sö́ʔ-ö́y mòh ʔõ̀y’, hũ̀ytu sö́ʔ-ö́y mòh ʔõ̀y’, hákten-éy=d’äh hṹ sáp ni-bahad-nɨ́h pɨ́d=mah tɨh yë́-ë́h.\\
     that\textsc{=rep} that all.that\textsc{=adv} \textsc{distr=rep} that \textsc{3sg} take\textsc{-infr2-decl} \textsc{3sg} end \textsc{loc-dynm} tinamou bunch behind \textsc{loc-dynm} tinamou bunch side\textsc{-dynm=pl} animal \textsc{ints} be-appear\textsc{-neg} \textsc{distr=rep} \textsc{3sg} enter\textsc{-decl}\\
\glt ‘So, it’s said, he would take all that, it’s said, a bunch of tinamou at the end (of the pole), a bunch of tinamou behind, (with so much game) on either side that he could hardly be seen, he would come in.'
\glt ‘Aí, dizem, ele sempre levava tudo isso, dizem, um monte de inambu no final (de um pau), uns inambus atrás, (com tanta caça) nos dois lados que o corpo dele quase não aparecia, ele entrava.'
\z 

\ea  Yɨ̃nɨ́hɨ́y pɨ́d mah yup d’ú’ tɨh k’ët wed wɨdyë́ë́p, të́ hiwag noh yet yɨ’ pɨ́dɨ́h, të́ sadakà’ õh säwä’ të́g kót’ah mah pɨ́d, hɨd yãhã́’ã́h, hɨd wédep.\\ 
\gll yɨ̃-nɨ́h-ɨ́y pɨ́d=mah yup d’úʔ tɨh k’ët-wed-wɨdyë́-ëp, të́ hi-wag-noh-yet-yɨʔ-pɨ́d-ɨ́h, të́ sadakàʔ ʔõh-säwäʔ-të́g kót’ah=mah pɨ́d, hɨd yãhã́ʔ-ã́h, hɨd wéd-ep.\\
     \textsc{dem.itg-}be.like\textsc{-dynm} \textsc{distr=rep} that evening \textsc{3sg} stand-eat-arrive.enter\textsc{-dep} until \textsc{fact}-day-fall-lie\textsc{-tel-distr-decl} until chicken sleep-wake\textsc{-fut} before\textsc{=rep} \textsc{distr} \textsc{3pl} stop\textsc{-decl} \textsc{3pl} eat\textsc{-dep}\\
\glt ‘Thus, it’s said, he would arrive in the evening with food for her, and they would eat, stopping only when day was breaking, just before the rooster crows, it’s said.'
\glt ‘Assim, dizem, ele chegava no final do dia com comida para ela, e eles comiam até amanhecer, parando só pouco antes de o galo cantar, dizem.'
\z 

\ea  Yɨnɨh yö́’ pɨ́d mah tɨh tẽ́hn’àn wèd dɨ’nɨ́h tɨh ni yɨ’ pɨ́dɨ́h.\\ 
\gll yɨ-nɨh-yö́ʔ pɨ́d=mah tɨh tẽ́h=n’àn wèd dɨʔ-nɨ́h tɨh ni-yɨʔ-pɨ́d-ɨ́h.\\
     \textsc{dem.itg-}be.like\textsc{-seq} \textsc{distr=rep} \textsc{3sg} offspring\textsc{=pl.obj} food remain\textsc{-neg} \textsc{3sg} be\textsc{-tel-distr-decl}\\
\glt ‘Always thus, it’s said, she/they would leave nothing for her children.'
\glt ‘Sempre era assim, dizem, não deixavam nada para os filhos.'
\z 

\newpage
\ea  Tɨh tẽ́hn’àn wèd dɨ’nɨ́h ni yö́’ pɨ́d, tɨh tẽhín máh tɨh sak k’ã’ yö́’ pɨ́d, hɨd këynɨ́h yɨ’ pɨ́d, tɨh way yɨ’ɨ́h.\\ 
\gll tɨh tẽ́h=n’àn wèd dɨʔ-nɨ́h ni-yö́ʔ pɨ́d, tɨh tẽh=ʔín máh tɨh sak-k’ãʔ-yö́ʔ pɨ́d, hɨd këy-nɨ́h=yɨʔ pɨ́d, tɨh way-yɨʔ-ɨ́h.\\
     \textsc{3sg} offpring\textsc{=pl.obj} food remain\textsc{-neg} be\textsc{-seq} \textsc{distr} \textsc{3sg} child=mother near \textsc{3sg} climb-hang\textsc{-seq} \textsc{distr} \textsc{3pl} see\textsc{-neg=adv} \textsc{distr} \textsc{3sg} go.out\textsc{-tel-decl}\\
\glt ‘Always leaving no food for her children, he would climb into the hammock with his wife, (and later) while they (the children) did not see, he would go out.'
\glt ‘Sem deixar nada para os filhos, ele sempre subia na rede com a mulher dele, e (depois), sem as crianças ver, ele sempre saía.'
\z 

\ea  Tẽ́h bɨ’ yö́’ pɨ́d mah yɨ́t tɨh way yɨ’ ni pɨ́dɨ́h.\\ 
\gll tẽ́h bɨʔ-yö́ʔ pɨ́d=mah yɨ́t tɨh way-yɨʔ-ni-pɨ́d-ɨ́h.\\
     offspring make\textsc{-seq} \textsc{distr=rep} thus \textsc{3sg} go.out\textsc{-tel-}be\textsc{-distr-decl}\\
\glt ‘After producing a child,{\footnotemark}\footnotetext{That is, they would make love, such that after a time his wife became pregnant. The wording here may refer to the model of conception in which repeated love-making events are understood to produce a child.} it’s said, he would go out.'
\glt ‘Depois de fazer um filho, dizem, ele sempre saiu.'
\z 

\ea  Të bɨ́gay mah yúp, “Hɨn’ɨ̀h yö́’ sáp ɨn ín, ɨ́nàn yɨ̃ no bɨáh tì, yã̀’ ɨ́nàn yɨ̃ no bɨáh?” no yö́’ mah, hɨd pɨb sákáy nih sud ũhníy, hɨ́dɨ́h.\\ 
\gll të bɨ́g-ay=mah yúp, hɨn’ɨ̀h-yö́ʔ sáp ʔɨn ʔín, ʔɨ́n-àn yɨ̃-no-bɨ-áh tì, yã̀ʔ ʔɨ́n-àn yɨ̃-no-bɨ-áh? no-yö́ʔ=mah, hɨd pɨb sák-áy=nih=sud ʔũhníy, hɨ́d-ɨ́h.\\
     until long.time\textsc{-inch=rep} that what\textsc{-seq} \textsc{ints} \textsc{1pl} mother \textsc{1pl-obj} \textsc{dem.itg-}say\textsc{-hab-foc} \textsc{q.emph} mom \textsc{1pl-obj} \textsc{dem.itg}-say-\textsc{hab-foc} say\textsc{-seq=rep} \textsc{3pl} strong go.up\textsc{-dynm=emph.co=infr} maybe \textsc{3pl-decl}\\
\glt ‘Until, after a long time, it’s said, (the children) said, “Why in the world does our mother always say this to us, does Mama always say this to us?” They were growing up, perhaps, those (children).'
\glt ‘Até, depois de muito tempo, “Por que será que a nossa mãe sempre fala assim para nós, Mamãe sempre nos fala assim?" Eles estavam crescendo, parece, essas (crianças).'
\z 

\ea  Yɨnɨ́hɨ́y mah yúp ayùp ĩh, tɨh-wàh dɨ́yɨ’, këy k’ã́’ayáh.\\ 
\gll yɨ-nɨ́h-ɨ́y=mah yúp ʔayùp=ĩh, tɨh=wàh dɨ́yɨʔ, këy-k’ã́ʔ-ay-áh.\\
     \textsc{dem.itg-}be.like\textsc{-dynm=rep} that one\textsc{=msc} \textsc{3sg=}mature \textsc{cpm} see-hang\textsc{-inch-decl}\\
\glt ‘So, it's said, one boy, the oldest one, (stayed awake) watching from his hammock.'
\glt ‘Aí, dizem, um rapaz, o mais velho, (ficou acordado) olhando da rede dele.'
\z 

\ea  Këy k’ã’ yö́’ mah yúp, “Hɨn’ɨ̀h pöy sáp bɨ́g yẽ́h tɨ́hah?!  D’ú’ ɨ́nàn ‘hẽ̀gyɨ’ nɨg õhyɨ’ áy tẽ́h, nɨg ápyɨ’ nɨg õh hẽ̀gyɨ’ áy!' ɨ́nàn tɨh no bɨ́ɨ’ s’ã́h?” no yö́’ mah,\\ 
\gll këy-k’ãʔ-yö́ʔ=mah yúp, hɨn’ɨ̀h pö-y sáp bɨ́g yẽ́h tɨ́h-ah?!  d’úʔ ʔɨ́n-àn hẽ̀g-yɨʔ nɨg ʔõh-yɨʔ-ʔáy tẽ́h, nɨg ʔápyɨʔ nɨg ʔõh-hẽ̀g-yɨʔ-ʔáy! ʔɨ́n-àn tɨh no-bɨ́-ɨʔ s’ã́h? no-yö́ʔ=mah,\\
     see-hang\textsc{-seq=rep} that what \textsc{emph-dynm} \textsc{ints} \textsc{hab} \textsc{frust} \textsc{3sg-foc} evening \textsc{1pl-obj} quick\textsc{-adv} \textsc{2pl} sleep\textsc{-tel-vent.imp} offspring \textsc{2pl} all \textsc{2pl} sleep-quick\textsc{-tel-vent.imp} \textsc{1pl-obj} \textsc{3sg} say\textsc{-hab-q} \textsc{dst.cntr}{\footnotemark} say\textsc{-seq=rep}\\
\glt ‘Watching from the hammock, saying, “What in the world is she always doing?! Why does she always say, in the evening, ‘Go quickly to sleep, children, all of you go quickly to sleep!'?”'
\glt ‘Olhando da rede dele, dizendo, “O que é que ela pode estar fazendo?! Porque ela sempre fala, no final do dia, ‘Vão dormir logo, filhos, vocês todos durmam logo!'?"'
\z 
\footnotetext{The ‘distant past contrast' marker (\textsc{dst.cntr}) \textit{s’ãh} in this context clarifies that the situation has been going on for a long time.}

\ea  yúp tɨh wãg yäd k’ã́’ayáh, yág seseg ë’ ní-íy mah, s’ámyɨ’ hä’;\\ 
\gll yúp tɨh wãg-yäd-k’ã́ʔ-ay-áh, yág seseg-ʔëʔ-ní=mah, s’ám=yɨʔ häʔ;\\
     that \textsc{3sg} spy-hide-hang\textsc{-inch-decl} hammock perforated\textsc{-pfv-infr2=rep} \textsc{dst.cntr=adv} \textsc{tag2}\\
\glt ‘He hung spying, hidden; it was a net-woven hammock, it’s said, (the kind from) the old days;'
\glt ‘Ele ficou lá espiando, escondido; era uma rede tecida (de fibra), dizem, de antigamente;'
\z 

\newpage
\ea  s’àk s’ó yág ë’ ní mah, s’ámyɨ’ɨy yágáh.\\ 
\gll s’àk s’ó yág-ʔëʔ-ní=mah, s’ám=yɨʔ-ɨy yág-áh.\\
     buriti flower hammock\textsc{-pfv-infr2=rep} \textsc{dst.cntr=adv-dynm} hammock\textsc{-decl}\\
\glt ‘in the old days they were buriti-fiber hammocks, those hammocks in the old days.'{\footnotemark}\footnotetext{Here Isabel offers an explanatory comment; today most indigenous people of the region use manufactured cotton hammocks bought or traded for from local merchants. Buriti is the regional name for the palm \textit{Mauritius flexuosa}.}
\glt ‘antigamente tinham redes de fibra de buriti, essas redes antigas.'
\z 

\ea Nutènep tëghṍd’äh nɨ̀h yágay, nutènep, yág húpútay nɨg k’ã́’ãhä̀’; páy mah ɨn pem k’ö’ ë́h, s’ámyɨ’ɨh.\\
\gll nutèn-ep tëghṍ=d’äh nɨ̀h yág-ay, nutèn-ep, yág húp-út-ay nɨg k’ã́ʔ-ãhä̀ʔ; páy=mah ʔɨn pem-k’ö’-ʔë́h, s’ám=yɨʔ-ɨh.\\
	today\textsc{-dep} non.indian\textsc{=pl} \textsc{poss} hammock\textsc{-inch} today\textsc{-dep} hammock beautiful\textsc{-obl-inch} \textsc{2pl} hang-\textsc{tag2} bad\textsc{=rep} \textsc{1pl} sit-go.around-\textsc{pfv} \textsc{dst.cntr=adv-decl}\\
\glt ‘Nowadays you all lie in the non-Indian people’s nice hammocks; we went badly in the old days, it's said.' 
\glt ‘Hoje em dia vocês deitam nas redes bonitas dos brancos; foi mal para nós antigamente.'
\z

\ea  Yɨn’ɨ̀h yág hitä́’äp mah yúp tɨh wãg yäd k’ã́’ã́h, yup tiyì’ mehéh.\\ 
\gll yɨ-n’ɨ̀h yág hitä́ʔ-äp=mah yúp tɨh wãg-yäd-k’ã́ʔ-ã́h, yup tiyìʔ=meh-éh.\\
     \textsc{dem.itg-nmlz} hammock covered\textsc{-dep=rep} that \textsc{3sg} spy-hide-hang\textsc{-decl} that man\textsc{=dim-decl}\\
\glt ‘So, covered by that hammock, it's said, he hung spying, hidden, that boy.'\hspace*{-4mm}
\glt ‘Assim, coberto pela rede, dizem, ele ficou lá espiando, escondido, aquele rapazinho.'
\z 

\newpage
\ea  Yúp yɨnɨh yö́’ mah yúp, “hɨn’ɨ̀h të́g ɨn yà?!  ɨ́nàn yɨnɨ́hɨ́y sud ɨn íníh, páy bɨ́’ɨ́y sud ɨ́n ín ɨ́nànáh!{\footnotemark}\footnotetext{The boy's comment makes use of the inferential evidential, in contrast to the reported evidential that is used more heavily throughout the narrative text.}\\
\gll yúp yɨ-nɨh-yö́ʔ=mah yúp, hɨn’ɨ̀h-të́g ʔɨn yà?!  ʔɨ́n-àn yɨ-nɨ́h-ɨ́y=sud ʔɨ́n ʔín-íh, páy bɨ́ʔ-ɨ́y=sud ʔɨ́n ʔín ʔɨ́n-àn-áh!\\
     that \textsc{dem.itg-}be.like\textsc{-seq=rep} that what\textsc{-fut} \textsc{1pl} \textsc{tag1} \textsc{1pl-obj} \textsc{dem.itg-}be.like\textsc{-dynm=infr}  \textsc{1pl} mother\textsc{-decl} bad work\textsc{-dynm=infr} \textsc{1pl} mother \textsc{1pl-obj-decl}\\
\glt ‘So, it’s said, “What can we do?! Our mother has been doing thus to us, it seems, our mother has been doing badly by us, it seems!'
\glt ‘Aí, dizem, “O que podemos fazer?! Parece que a nossa mãe está nos tratando assim, parece que a nossa mãe está nos fazendo mal!'
\z 

\ea  “Ya’áp s’ã́h hɨd wed bɨg súdúh!  Kë́yë́y s’ã́h ã́hã́h, méh!” tɨh nóayáh.{\footnotemark}\footnotetext{In this utterance, the distant past contrast marker (together with the inferential evidential) clarifies that the event must have been going on for a long time.}\\
\gll yaʔáp s’ã́h hɨd wed-bɨg-súd-úh!  kë́y-ë́y s’ã́h ʔã́h-ã́h, méh! tɨh nó-ay-áh.\\
     all.that \textsc{dst.cntr} \textsc{3pl} eat\textsc{-hab-infr-decl} see\textsc{-dynm} \textsc{dst.cntr} \textsc{1sg-decl} younger.sister \textsc{3sg} say\textsc{-inch-decl}\\
\glt ‘“They’ve been eating so much all this time, apparently! I’ve seen it, younger sister!” he said.'
\glt ‘Faz tempo que eles estão comendo tanto, parece! Eu vi, minha irmã menor!" ele falou.'
\z 

\newpage
\ea  Yɨnɨh yö́’ mah yúp, s’äbtéyɨ’ s’om d’ö̀b d’äh mah hɨd ũh ɨ́dɨ́h.{\footnotemark}\footnotetext{This utterance illustrates the use of the reciprocal prefix \textit{-ʔũh}, which is formally identical to several other morphemes in Hup (as evident in this text), including the applicative suffix and the epistemic modal particle. See \citet{Epps2010} for discussion of the historical connection among these forms.}\\ 
\gll yɨ-nɨh-yö́ʔ=mah yúp, s’äb-té=yɨʔ s’om-d’ö̀b=d’äh=mah hɨd ʔũh=ʔɨ́d-ɨ́h.\\
     \textsc{dem.itg-}be.like\textsc{-seq=rep} that night-still\textsc{=adv} bathe-descend\textsc{=pl=rep} \textsc{3pl} \textsc{recp=}speak\textsc{-decl}\\
\glt ‘So, it’s said, in the morning as they were going to bathe they spoke together.'
\glt ‘Aí, dizem, de manhã, quando estavam indo para tomar banho, eles falavam entre eles.'
\z 

\ea  Yúp ësáp b’ay mah tɨh b’òt ham yɨ́’ɨp b’ay.\\ 
\gll yúp ʔësáp=b’ay=mah tɨh b’òt ham-yɨ́ʔ-ɨp=b’ay.\\
     that tomorrow=again\textsc{=rep} \textsc{3sg} swidden go\textsc{-tel-dep}=again\\
\glt ‘So the next day, it’s said, she (their mother) went to her swidden garden.'
\glt ‘O dia depois, dizem, ela (a mãe) foi para a roça.'
\z 

\ea  Yúp tɨh b’òt hámap, yɨ́tyɨ’ pɨ́d, tɨh-dëhwàh yɨ’ pɨ́d mah hɨ́dàn tɨh b’äh k’et käsät ũ̀hṹh.\\ 
\gll yúp tɨh b’òt hám-ap, yɨ́t=yɨʔ pɨ́d, tɨh=dëh-wàh=yɨʔ pɨ́d=mah hɨ́d-àn tɨh b’äh-k’et-käsät-ʔũ̀h-ṹh.\\
     that \textsc{3sg} swidden go\textsc{-dep} thus\textsc{=adv} \textsc{distr} \textsc{3sg=}water-old.food\textsc{=cntr.emph} \textsc{distr=rep} \textsc{3pl-obj} \textsc{3sg} pour-stand-be.first\textsc{-appl-decl}\\
\glt ‘As she was going to the garden, as always, she poured out old tasteless \textit{manicuera} for (the children), it's said.'
\glt ‘Saindo para a roça, como sempre, ela deixou a manicuera ruim para (as crianças), dizem.'
\z 

\newpage
\ea  Yúp b’ay mah, tɨh tẽhípàn tɨh-dëh húp yɨ’ pɨ́d, sanàát tɨh hitú’up náw pɨ́d, tɨh hipud y’et yɨ́’ɨ́h, pë̀dë́t hitú’up.\\ 
\gll yúp=b’ay=mah, tɨh tẽh=ʔíp-àn tɨh=dëh húp=yɨʔ pɨ́d, sanà-át tɨh hitúʔ-up náw pɨ́d, tɨh hipud-y’et-yɨ́ʔ-ɨ́h, pë̀d-ë́t hitúʔ-up.\\
     that=again\textsc{=rep} \textsc{3sg} child=father\textsc{-obj} \textsc{3sg=}liquid beautiful\textsc{=cntr.emph} \textsc{distr} pineapple\textsc{-obl} \textsc{3sg} mix\textsc{-dep} good \textsc{distr} \textsc{3sg} mix.broth-lay\textsc{-tel-decl} cunuri\textsc{-obl} mix\textsc{-dep}\\
\glt ‘And again, it’s said, for her husband it was good \textit{manicuera}, nicely mixed with pineapple, that she mixed and set out, mixed with \textit{cunuri}.'
\glt ‘E como sempre, dizem, para o marido dela ela misturou e colocou a manicuera boa, bem misturada com abacaxí, misturada com cunuri.'
\z 

\ea  Yɨnɨh yö́’ mah yúp, “hɨn’ɨ̀h të́g ɨ́n, ɨ́nàn yúpyɨ’ bɨ’ nɨ́hɨ́y sud yúwúh?!” no yö́’ mah, hɨd tɨh-dö́’d’äh hɨd hámayáh.\\ 
\gll yɨ-nɨh-yö́ʔ=mah yúp, hɨn’ɨ̀h-të́g ʔɨ́n, ʔɨ́n-àn yúp=yɨʔ bɨʔ-nɨ́h-ɨ́y=sud yúw-úh?! no-yö́ʔ=mah, hɨd tɨh=dö́ʔ=d’äh hɨd hám-ay-áh.\\
     \textsc{dem.itg-}be.like\textsc{-seq=rep} that what\textsc{-fut} \textsc{1pl} \textsc{1pl-obj} thus\textsc{=adv} make-be.like\textsc{-dynm=infr} \textsc{dem.itg-decl} say\textsc{-seq=rep} \textsc{3pl} \textsc{3sg=}child\textsc{=pl} \textsc{3pl} go\textsc{-inch-decl}\\
\glt ‘So then, it’s said, saying, “What will we do, (since) she’s apparently treating us this way?!” it’s said, the children went off.'
\glt ‘Assim, dizem, falando “O que vamos fazer, com ela nos tratando assim?!" dizem, as crianças foram embora.'
\z 

\ea  Ham yö́’ mah, d’ùç hɨd hátáh.\\ 
\gll ham-yö́ʔ=mah, d’ùç hɨd hát-áh.\\
     go\textsc{-seq=rep} timbó \textsc{3pl} dig\textsc{-decl}\\
\glt ‘Having gone, it's said, they dug up fish-poison (root/vine).'{\footnotemark}\footnotetext{Fish-poison (regional name “timbó"; \textit{Lonchocarpus} sp.) is used to poison sections of streams in order to kill fish, but can also be used as a means of poisoning people. The root is beaten in water to release the poison.}
\glt ‘Foram, dizem, e desenterravam timbó.'
\z 

\newpage
\ea  D’ùç hat yö́’ mah, s’ómop tɨh kädd’öb mɨ̀’ mah, d’ùç hɨd tätäd d’ö́’ayáh.\\ 
\gll d’ùç hat-yö́ʔ=mah, s’óm-op tɨh kädd’öb-mɨ̀ʔ=mah, d’ùç hɨd tätäd-d’ö́ʔ-ay-áh.\\
     timbó dig\textsc{-seq=rep} bathe\textsc{-dep} \textsc{3sg} pass.descend\textsc{-under=rep} timbó \textsc{3pl} beat.timbó-take\textsc{-inch-decl}\\
\glt ‘Having dug fish-poison, as she (their mother) was on her way down (to the stream) to bathe, they beat the fish-poison (to release the poison).'
\glt ‘Depois de desenterrar o timbó, enquanto (a mãe) estava indo para tomar banho, eles baterem o timbó (para fazer o veneno sair).'
\z 

\ea  Tätäd d’ö’ yö́’, yúp tɨh-dëh húpút, tɨh hipud y’et yɨ́’ɨwɨ́t hɨd köw’öw’ tu’ y’et yɨ́’ayáh.\\ 
\gll tätäd-d’öʔ-yö́ʔ, yúp tɨh=dëh húp-út, tɨh hipud-y’et-yɨ́ʔ-ɨw-ɨ́t hɨd köw’öw’-tuʔ-y’et-yɨ́ʔ-ay-áh.\\
     beat.timbó-take\textsc{-seq} that \textsc{3sg=}liquid beautiful\textsc{-obl} \textsc{3sg} mix.broth-lay\textsc{-tel-flr-obl} \textsc{3pl} squeeze-immerse-lay\textsc{-tel-inch-decl}\\
\glt ‘Having beaten the fish-poison, they squeezed (the juice) into the tasty \textit{manicuera}, into the \textit{manicuera} that (their mother) had set out (for her husband).'
\glt ‘Depois de bater timbó, eles espremiam (o líquido) na manicuera boa, na manicuera que (a mãe) tinha colocado (para o marido).'
\z 

\ea  Köw’öw’ tu’ y’et yɨ’ yö́’ mah, hɨd yɨn’ɨ̀h nonɨ́h õh yɨ́’ayáh.\\ 
\gll köw’öw’-tuʔ-y’et-yɨʔ-yö́ʔ=mah, hɨd yɨ-n’ɨ̀h no-nɨ́h ʔõh-yɨ́ʔ-ay-áh.\\
     squeeze-immerse-lay\textsc{-tel-seq=rep} \textsc{3pl} \textsc{dem.itg-compl} say\textsc{-neg} sleep\textsc{-tel-inch-decl}\\
\glt ‘Having squeezed the juice into (it), it’s said, they went to sleep, saying nothing about it.'
\glt ‘Depois de espremer o líquido (na manicuera), dizem, eles dormiram, sem dizer nada.'
\z 

\newpage
\ea  Yúp mah ayùp ĩh, këy k’ã’ bɨ́gɨp ĩh yɨ’ pɨ́d, këy k’ã́’ b’ayáh.\\ 
\gll yúp=mah ʔayùp=ʔĩh, këy-k’ãʔ-bɨ́g-ɨp=ʔĩh=yɨʔ pɨ́d, këy-k’ã́ʔ-b’ay-áh.\\
     \textsc{dem.itg=rep} one\textsc{=msc} see-hang\textsc{-hab-dep=msc=cntr.emph} \textsc{distr} see-hang-again\textsc{-decl}\\
\glt ‘So, it’s said, one boy, the one who had been watching from his hammock, watched from his hammock again.'
\glt ‘Aí, dizem, um rapaz, aquele que estava olhando da rede, ficou olhando da rede de novo.'
\z 

\ea  Yúp mah tɨh wɨdyë yɨ́’ay b’ayáh.\\ 
\gll yúp=mah tɨh wɨdyë-yɨ́ʔ-ay=b’ay-áh.\\
     that\textsc{=rep} \textsc{3sg} arrive.enter\textsc{-tel-inch}=again\textsc{-decl}\\
\glt ‘Then, it's said, he (the deer) came in.'
\glt ‘Aí, dizem, o veado entrou.'
\z 

\ea  “Õh yɨ’ sɨ̃́wɨ̃́y hɨ́d, ɨ́n tẽ́hd’äh?” no wɨdyë́ë́y b’ay mah.\\ 
\gll ʔõh-yɨʔ-sɨ̃́w-ɨ̃́y hɨ́d, ʔɨ́n tẽ́h=d’äh? no-wɨdyë́-ë́y=b’ay=mah.\\
     sleep\textsc{-tel-compl-dynm} \textsc{3pl} \textsc{1pl} offspring\textsc{=pl} say-arrive.enter\textsc{-dynm=again=rep}\\
\glt ‘“Are our children asleep?” he said, entering, it’s said.'
\glt ‘“Nossos filhos estão dormindo?" ele falou entrando, dizem.'
\z 

\ea  “Õh yɨ’ sɨ̃́wɨ̃́y yɨd’ä̀hä́h, páhyɨ’ hɨd õh yɨ́’ɨ́h,” nóóy mah yúp, tɨh tẽhín waáh.\\ 
\gll ʔõh-yɨʔ-sɨ̃́w-ɨ̃́y yɨ-d’ä̀h-ä́h, páh=yɨʔ hɨd ʔõh-yɨ́ʔ-ɨ́h, nó-óy=mah yúp, tɨh tẽh=ʔín=wa-áh.\\
     sleep\textsc{-tel-compl-dynm} \textsc{dem.itg-pl-decl} \textsc{prox.cntr=adv} \textsc{3pl} sleep\textsc{-tel-decl} say\textsc{-dynm=rep} that \textsc{3sg} offspring=mother\textsc{=resp-decl}\\
\glt ‘“They’re already asleep, they went to sleep a short while ago,” she said, his wife.'
\glt ‘“Já dormiram, dormiram há pouco tempo," ela falou, a mulher dele.'
\z 

\newpage
\ea  Yɨ̃ no yö́’ mah yúp, tɨh ä́gayáh, yúwädä́h, wed hupsɨ̃̀p, yup hɨd kö’wöw’ tu’ y’et yɨ’ pög ë́wànáh.\\ 
\gll yɨ-no-yö́ʔ=mah yúp, tɨh ʔä́g-ay-áh, yú-wäd-ä́h, wed-hup-sɨ̃̀p, yup hɨd kö’wöw’-tuʔ-y’et-yɨʔ-pög-ʔë́-w-àn-áh.\\
     \textsc{dem.itg-}say\textsc{-seq=rep} \textsc{dem.itg} \textsc{3sg} drink\textsc{-inch-decl} \textsc{dem.itg-resp-decl} eat\textsc{-refl-compl} that \textsc{3pl} squeeze-immerse-lay\textsc{-tel-aug-pfv-flr-obj-decl}\\
\glt ‘Having said that, it’s said, he drank it, that respected one, after eating, that which they had squeezed (poison) into and left there.' 
\glt ‘Falando isso, dizem, ele tomou (a manicuera), esse (veado), depois de comer, aquela que eles tinham deixado com (o veneno) espremido.'
\z 

\ea  Yúp äg yö́’ mah tɨh sak k’ã’ yɨ́’ayáh, hɨd ka’àpd’äh.\\ 
\gll yúp ʔäg-yö́ʔ=mah tɨh sak-k’ãʔ-yɨ́ʔ-ay-áh, hɨd kaʔàp=d’äh.\\
     \textsc{dem.itg} drink\textsc{-seq=rep} \textsc{3sg} climb-hang\textsc{-tel-inch-decl} \textsc{3pl} two\textsc{=pl}\\
\glt ‘Having drunk, it’s said, he climbed into the hammock, the two of them (together).'\footnote{Here Isabel briefly lost her train of thought and commented:\\
\begin{exe}
\ex \textit{Hɨ̃ no pö́y ũh mah s’ã́h yúw? Ãh hipãhnɨ́h yúwàn, ã́hã́h, yúp, hã̀y, hɨ̃ no pö́y mah s’ã́h yúw? Yúwàn ãh hipãhnɨ́h hõ.}\\
\gll Hɨ̃ no-pö́-y ʔũh=mah s’ã́h yúw? ʔÃh hipãh-nɨ́h yúw-àn, ʔã́h-ã́h, yúp, hã̀y, hɨ̃ no-pö́-y=mah s’ã́h yúw? Yúw-àn ʔãh hipãh-nɨ́h=hõ.\\
	what say-\textsc{emph1-dynm} \textsc{epist=rep} \textsc{dst.cntr} \textsc{dem.itg} \textsc{1sg} know\textsc{-neg} \textsc{dem.itg-obj} \textsc{1sg-decl} \textsc{dem.itg} um what say\textsc{-emph1-dynm=rep} \textsc{dst.cntr} \textsc{dem.itg} \textsc{dem.itg-obj} \textsc{1sg} know\textsc{-neg=nonvis}\\
\glt ‘Now how does it (the story) go? I don’t remember this part, um, how does it go? I don’t remember this part.’
\glt ‘Então, como é essa parte (da história)? Não lembro essa parte, eh, como é? Não lembro essa parte.'
\end{exe}}
\glt ‘Depois de tomar, ele subiu na rede, os dois juntos.'
\z  

\newpage
\ea  Tɨh äg yö́’ tɨh na’ sak k’ã́’awayáh.\\
\gll Tɨh ʔäg-yö́ʔ tɨh naʔ-sak-k’ã́ʔ-aw-ay-áh.\\
	\textsc{3sg} drink\textsc{-seq} \textsc{3sg} lose.consciousness-climb-hang\textsc{-flr-inch-decl}\\
\glt ‘Having drunk, he climbed up drunkenly into the hammock.’
\glt ‘Depois de tomar, ele subiu, bêbado, na rede.'
\z

\ea  Yúp sak k’ã’ yö́’ mah yúp tɨh äg ná’awɨ́t yɨ’ mah, tɨh õh kädham yɨ’ níayáh.\\ 
\gll yúp sak-k’ãʔ-yö́ʔ=mah yúp tɨh äg-náʔ-aw-ɨ́t=yɨʔ=mah, tɨh ʔõh-kädham-yɨʔ-ní-ay-áh.\\
     \textsc{dem.itg} climb-hang\textsc{-seq=rep} \textsc{dem.itg} \textsc{3sg} drink-lose.consciousness\textsc{-flr-obl=adv=rep} \textsc{3sg} sleep-pass.go\textsc{-tel-}be\textsc{-inch-decl}\\
\glt ‘Having climbed into the hammock, it’s said, in his drunken(-like) state, he went directly to sleep.'
\glt ‘Tendo subido na rede, dizem, bêbado, ele dormiu direto.'
\z 

\ea  Yɨ́t mah tɨh na’ yɨ’ níayáh tɨh tẽhín hupáh máh, tɨh tawak k’ã’ pög níayáh.\\ 
\gll yɨ́t=mah tɨh naʔ-yɨʔ-ní-ay-áh tɨh tẽh=ʔín hupáh máh, tɨh tawak-k’ãʔ-pög-ní-ay-áh.\\
     thus\textsc{=rep} \textsc{3sg} lose.consciousness\textsc{-tel-}be\textsc{-inch-decl} \textsc{3sg} child=mother back near \textsc{3sg} stiff-hang\textsc{-aug-}be\textsc{-inch-decl}\\
\glt ‘Thus, it’s said, he died there against his wife’s back, he lay there stiff.'
\glt ‘Assim, dizem, ele morreu lá contra as costas da mulher dele, ele ficou lá rígido.'
\z 

\ea  Yúp mah yúp, “ɨn tẽ́hd’äh säwä́’ayáh!” noyö́’ mah, tɨh yũy’ yẽ́hayáh, tɨh yũy’ yẽ́hayáh, säwä’nɨ́hay mah,\\
\gll yúp=mah yúp, ʔɨn tẽ́h=d’äh säwä́ʔ-ay-áh! no-yö́ʔ=mah, tɨh yũy’-yẽ́h-ay-áh, tɨh yũy’-yẽ́h-ay-áh, säwäʔ-nɨ́h-ay=mah,\\
     \textsc{dem.itg=rep} \textsc{dem.itg} \textsc{1pl} offspring\textsc{=pl} awake\textsc{-inch-decl} say\textsc{-seq=rep} \textsc{3sg} shake\textsc{-frust-inch-decl} \textsc{3sg} shake\textsc{-frust-inch-decl} awake\textsc{-neg-inch=rep}\\
\glt ‘So, it’s said, saying “Our children are waking up!” she shook him and shook him in vain; he did not wake up.'
\glt ‘Aí, dizem, falando, “Nossos filhos estão acordando!" dizem, ela o sacudiu, o sacudiu, para nada; ele não acordou.'
\z 

\ea  Säwä’ huphipãhnɨ́h, tawak d’ak pö́ay mah, tɨh hupáh, tɨh tẽhín hupáh, mohòyṍh.\\
\gll säwäʔ-hup-hipãh-nɨ́h, tawak-d’ak-pö́-ay=mah, tɨh hupáh, tɨh tẽh=ʔín hupáh, mohòy-ṍh.\\
     awake\textsc{-refl-}know\textsc{-neg} stiff-be.against\textsc{-aug-inch=rep} \textsc{3sg} back \textsc{3sg} offspring=mother back deer\textsc{-decl}\\
\glt ‘He did not awake to consciousness, he lay there stiff against his wife's back, it’s said, the deer.'
\glt ‘Ele não acordou, ficou lá rígido contra as costas da mulher dele, dizem, o veado.'
\z 

\ea  “Nɨ́g s’õm áy ham áy tẽ́h! Hɨn’ɨ̀h nɨg k’ã́’ã́y nɨ́g?” no ë̀y mah yúp, hɨd ín waáh.\\
\gll nɨ́g s’õm-ʔáy ham-ʔáy tẽ́h! hɨn’ɨ̀h nɨg k’ã́ʔ-ã́y nɨ́g? no-ʔë̀-y=mah yúp, hɨd ʔín=wa-áh.\\
     \textsc{2pl} bathe\textsc{-vent.imp} go\textsc{-vent.imp}  offspring what \textsc{2pl} hang\textsc{-dynm} \textsc{2pl} say\textsc{-pfv-dynm=rep} \textsc{dem.itg} \textsc{3pl} mother\textsc{=resp-decl}\\
\glt ‘“You all go bathe, children! What are you doing still in your hammocks?” she said, it’s said, their mother.'
\glt ‘“Vão embora tomar banho, filhos! Por que vocês ficam ainda nas redes?" ela falou, dizem, a mãe deles.'
\z 

\ea  Yɨnɨ́hɨ́y këyö́’ sud’ũ̀h hɨd d’öb yɨ́’ay ĩh.\\
\gll yɨ-nɨ́h-ɨ́y këyö́ʔ sudʔũ̀h hɨd d’öb-yɨ́ʔ-ay=ʔĩh.\\
     \textsc{dem.itg-}be.like\textsc{-dynm} because \textsc{infr.epist} \textsc{3pl} descend\textsc{-tel-inch=msc}\\
\glt ‘So with that, apparently, they went down to the water.'
\glt ‘Assim, parece, eles foram para o igarapé.'
\z 

\ea  Yúp mah tɨh mɨ̀’ sud’ũ̀h të̀g b’ók pö̀g bug’ k’ët d’ö’ö’ĩh.\\
\gll yúp=mah tɨh mɨ̀’ sud’ũ̀h të̀g=b’ók pö̀g bug’-k’ët-döʔ-öʔĩh.\\
     \textsc{dem.itg=rep} \textsc{3sg} \textsc{under} \textsc{infr.epist} tree=bark big bundle-stand-take\textsc{-msc}\\
\glt ‘So, it’s said, while (they were out), apparently, she gathered up a big bundle of bark.'
\glt ‘Aí, dizem, enquanto (eles estavam fora), parece, ela juntou um feixe grande de casca de árvore.'
\z 

\newpage
\ea  Yɨ́t mah yúp tɨh päd hiyet yɨ’ pö́ayáh, tɨh tẽhíp pögàn, mohòy wädàn.\\
\gll yɨ́t=mah yúp tɨh päd-hi-yet-yɨʔ-pö́-ay-áh, tɨh tẽh=ʔíp=pög-àn, mohòy=wäd-àn.\\
     thus\textsc{=rep} \textsc{dem.itg} \textsc{3sg} encircle-\textsc{fact}-lie\textsc{-tel-aug-inch-decl} \textsc{3sg} offspring=father\textsc{=aug-obj} deer\textsc{=resp-obj}\\
\glt ‘So, it’s said, she laid (his body) encircled (in the bark), her husband, the deer.'
\glt ‘Aí, dizem, ela envolveu (o corpo dele na casca), o marido dela, o veado.'
\z 

\ea  Yúp päd hiyet yɨ’ yö́’ mah yúp tëg b’ók pögö́t yɨ́t tɨh m’am’an’ d’ö’ kädway yɨ́’ayáh,\\
\gll yúp päd-hi-yet-yɨʔ-yö́ʔ=mah yúp tëg=b’ók pög-ö́t yɨ́t tɨh m’am’an’-d’öʔ-kädway-yɨ́ʔ-ay-áh,\\
     \textsc{dem.itg} encircle-fact-lie\textsc{-tel-seq=rep} \textsc{dem.itg} tree=bark big\textsc{-obl} thus \textsc{3sg} roll.up-take-pass.go.out\textsc{-tel-inch-decl}\\
\glt ‘Having laid out (his body) out encircled, it’s said, she rolled (it) up in the bark and took it quickly out (of the house),'
\glt ‘Depois de envolver (o corpo dele), dizem, ela o enrolou na casca e levou fora da casa.'
\z 

\ea  täh sud d’ö’ kädway yɨ́’ayáh, tɨnɨ̀h máj pöö́t.\\
\gll täh-sud-d’öʔ-kädway-yɨ́ʔ-ay-áh, tɨnɨ̀h máj=pö-ö́t.\\
     break-be.inside-take-pass.go.out\textsc{-tel-inch-decl} \textsc{3sg.poss} basket\textsc{=aug-obl}\\
\glt ‘She broke up (his body, to fit) inside (the basket) and took it quickly out, in her basket.'
\glt ‘Ela quebrou (o corpo para fazer entrar) dentro (de uma cesta), e levou rapidamente fora, no aturá dela.'
\z 

\newpage
\ea  Täh sud d’ö’ kädway yö́’ mah, tɨh kẽ́’ay mah s’ã́h tí, pö́hö́y mòyan.\\
\gll täh-sud-d’öʔ-kädway-yö́ʔ=mah, tɨh kẽ́ʔ-ay=mah s’ã́h tí, pö́h-ö́y mòy-an.\\
     break-be.inside-take-pass.go.out\textsc{-seq=rep} \textsc{3sg} bury\textsc{-inch=rep} \textsc{dst.cntr} \textsc{emph.dep} high\textsc{-dynm} house\textsc{-dir}\\
\glt ‘Having broken it up inside and gone out quickly, it’s said, she buried him, it’s said, in a place high up (in the sky).'
\glt ‘Depois de quebrâ-lo dentro e sair rápido, dizem, ela enterrou ele, dizem, num lugar alto (no céu).'
\z 

\ea  Pö́hö́y mòyan s’ã́h, yɨ́d’ähä́h, nusá’áh yɨ́d’ähä́h, mohòy höd nóop bahad bɨ́ɨtíh, pö́hö́y sa’ah sö’ötíh, mohòy höd hɨd nóowóh.\\
\gll pö́h-ö́y mòy-an s’ã́h, yɨ́-d’äh-ä́h, nu-sáʔáh yɨ́-d’äh-ä́h, mohòy höd nó-op bahad-bɨ́-ɨtíh, pö́h-ö́y saʔah söʔ-ötíh mohòy höd hɨd nó-ow-óh.\\
     high\textsc{-dynm} house\textsc{-dir} \textsc{dst.cntr} \textsc{dem.itg-pl-decl} here-side \textsc{dem.itg-pl-decl}  deer hole say\textsc{-dep} appear\textsc{-hab-emph2} high\textsc{-dynm} side \textsc{loc-emph2} deer hole \textsc{3pl} say\textsc{-flr-decl}\\
\glt ‘In a place high up (in the sky), over here, people from here (say), that which they call the Deer’s Tomb always appears, up high (in the sky), they call (it) the Deer’s Tomb.'{\footnotemark}\footnotetext{As noted in the Introduction, the location of this formation is uncertain, but it appears to be one of the “constellations" represented by a gap among stars.}
\glt ‘Num lugar alto (no céu); sempre aparece para cá, gente daqui (dizem), aquele que chamam de Túmulo do Veado, bem alto, o que chamam de Túmulo do Veado.'
\z 

\ea Yúp mah yúwúh, mohòy höd hɨd nóowóh.\\
\gll yúp=mah yúw-úh, mohòy höd hɨd nó-ow-óh.\\
	\textsc{dem.itg=rep} \textsc{dem.itg-decl} deer hole \textsc{3pl} say\textsc{-flr-decl}\\
\glt ‘That’s it, it’s said, they call it the Deer’s Tomb.'
\glt ‘É isso, dizem, que chamam de Túmulo do Veado.'
\z

\newpage
\ea  Yɨnɨhyö́’ mah yúp tɨh wɨdyë ni yɨ́’ɨp b’ay, konnɨ́h ni yɨ́’ɨ́y mah, tɨh-tẽ́hn’àn.\\
\gll yɨ-nɨh-yö́ʔ=mah yúp tɨh wɨdyë-ni-yɨ́ʔ-ɨp=b’ay, kon-nɨ́h ni-yɨ́ʔ-ɨ́y=mah, tɨh tẽ́h=n’àn.\\
     \textsc{dem.itg-}be.like\textsc{-seq=rep} \textsc{dem.itg} \textsc{3sg} arrive.enter-be\textsc{-tel-dep}=again like\textsc{-neg} be\textsc{-tel-dynm=rep} \textsc{3sg} offspring\textsc{=pl.obj}\\
\glt ‘After that she came back, and there she stayed with dislike (unhappiness) towards her children.'
\glt ‘Depois disso ela voltou, e ficou lá infeliz com os filhos dela.'
\z 

\ea  Yúp konnɨ́h tɨh ni yɨ́’ɨ́y këyö́’ mah yúp, hɨ́d b’ay hipãh yɨ’ sɨ̃́wɨ̃́y b’ay.\\
\gll yúp kon-nɨ́h tɨh ni-yɨ́ʔ-ɨ́y këyö́ʔ=mah yúp, hɨ́d=b’ay hipãh-yɨʔ-sɨ̃́w-ɨ̃́y=b’ay.\\
     \textsc{dem.itg} like\textsc{-neg} \textsc{3sg} be\textsc{-tel-dynm} because\textsc{=rep} \textsc{dem.itg} \textsc{3pl}=again know\textsc{-tel-compl-dynm}=again\\
\glt ‘As she stayed there unhappy with them, they became aware of it.'
\glt ‘Como ela ficou lá infeliz com eles, eles já perceberam.'
\z 

\ea  “Hɨ̃n’ɨ̀h të́g ũ̀h ɨn ín ɨ́nàn páh?” nóóy mah hɨ́dɨ́h.\\
\gll hɨ̃n’ɨ̀h=të́g ʔũ̀h ʔɨn ʔín ʔɨ́n-àn páh?” nó-óy=mah hɨ́d-ɨ́h.\\
     what\textsc{=fut} \textsc{epist} \textsc{1pl} mother \textsc{1pl-obj} \textsc{prox.cntr} say\textsc{-dynm=rep} \textsc{3pl-decl}\\
\glt ‘“What will our mother do to us?” they said, it’s said.'
\glt ‘“O que é que a nossa mãe vai nos fazer?" eles falaram, dizem.'
\z 

\ea  Yɨnɨh yö́’ mah, tɨh b’òtan ham yɨ́’ɨ́y b’ay.\\
\gll yɨ-nɨh-yö́ʔ=mah, tɨh b’òt-an ham-yɨ́ʔ-ɨ́y=b’ay.\\
     \textsc{dem.itg-}be.like\textsc{-seq=rep} \textsc{3sg} swidden\textsc{-dir} go\textsc{-tel-dynm}=again\\
\glt ‘(One day) after that, she went to her swidden garden.'
\glt ‘Aí, (un dia) ela foi para a roça.'
\z 

\ea  Yúp tɨnɨh heyó kakah yɨ́’ b'ay mah tɨh-dö́’àn tɨh su’ ní b’ayáh, mohòy tẽ́hànáh.\\
\gll yúp tɨnɨh heyó kakah=yɨ́ʔ=b'ay=mah tɨh=dö́ʔ-àn tɨh suʔ-ní-b’ay-áh, mohòy tẽ́h-àn-áh.\\
     \textsc{dem.itg} \textsc{3sg.poss} middle among\textsc{=adv}=again\textsc{=rep} \textsc{3sg=}child\textsc{-obj} \textsc{3sg} catch-be-again\textsc{-decl} deer offspring\textsc{-obj-decl}\\
\glt ‘There in the middle (of the swidden) she had a child, the deer’s child.'
\glt ‘Lá no meio (da roça) ela teve filho, o filho do veado.'
\z 

\ea  Yúp mohòy tẽ́hàn sú’up mah yúp, pö́h, máját, sákuút tɨ́hàn tɨh yö k’ã’ ni b’ayáh, yúp tɨh tẽ́h mehànáh\\
\gll yúp mohòy tẽ́h-àn súʔ-up=mah yúp, pö́h, máj-át, sáku-út tɨ́h-àn tɨh yö-k’ãʔ-ni-b’ay-áh, yúp tɨh tẽ́h=meh-àn-áh.\\
     \textsc{dem.itg} deer offspring\textsc{-obj} catch\textsc{-seq=rep} \textsc{dem.itg} high basket\textsc{-obl} bag\textsc{-obl} \textsc{3sg-obj} \textsc{3sg} dangle-hang-be-again\textsc{-decl} \textsc{dem.itg} \textsc{3sg} offspring\textsc{=dim-obj-decl}\\
\glt ‘Having given birth to the deer’s child, she put it into a basket, a sack, and she hung it up high (in the house), her little child.'
\glt ‘Depois de ter o filho do veado, ela colocou (o nenê) em um aturá, em um saco, e pendurou no alto (da casa), o filhinho dela.'
\z 

\ea  Tëg-sä̀hä́t mone yö́’ mah, tɨh d’ö’ sud k’ã’ yɨ’ níh.\\
\gll tëg=sä̀h-ä́t mone-yö́ʔ=mah, tɨh d’öʔ-sud-k’ãʔ-yɨʔ-ní-h.\\
     wood=charcoal\textsc{-obl} mix\textsc{-seq=rep} \textsc{3sg} take-be.inside-hang\textsc{-tel-infr2-dynm}\\
\glt ‘Having mixed in charcoal (in order to conceal the child in the basket), she put it in (the basket) and hung it up.'
\glt ‘Misturando com carvão (para esconder o nenê), ela colocou (no aturá) e pendurou no alto.'
\z 

\ea  S’äbtéyɨ’ tɨh no’ púdup, b’òt wɨdyë́ë́p tɨh no’ púdup, ya’àp yɨ’ mah tɨh no’ pud pɨ́dɨ́h.\\
\gll s’äbtéyɨʔ tɨh noʔ-púd-up, b’òt wɨdyë́-ë́p tɨh noʔ-púd-up, yaʔàp=yɨʔ=mah tɨh noʔ-pud-pɨ́d-ɨ́h.\\
     morning \textsc{3sg} give-nurse\textsc{-dep} swidden arrive.enter\textsc{-dep} \textsc{3sg} give-nurse\textsc{-dep} all.that\textsc{=adv=rep} \textsc{3sg} give-nurse\textsc{-distr-decl}\\
\glt ‘She would nurse it in the early morning, she would nurse it when she came back from her swidden garden, those were the only (times) she would nurse it.'
\glt ‘De manhã, ela dava peito, chegando da roça ela dava peito, só nessas (vezes), dizem, ela dava peito (para ele).'
\z 

\newpage
\ea  “Hɨn’ɨ̀h bɨ́g yẽ́h, yã̀’ b’òt wɨdyë́ëp yɨkán käkäynɨ́h yɨ’ kädsak wög bɨg yẽ́hẽ’ yà?” no yö́’ mah, hɨd sákayáh, dö́’d’ähä́h.\\
\gll hɨn’ɨ̀h bɨ́g yẽ́h, yã̀ʔ b’òt wɨdyë́-ëp yɨkán käkäy-nɨ́h=yɨʔ kädsak-wög-bɨg-yẽ́h-ẽʔ yà? no-yö́ʔ=mah, hɨd sák-ay-áh, dö́ʔ=d’äh-ä́h.\\
     what \textsc{hab} \textsc{frust} mama swidden arrive.enter\textsc{-dep} there gap\textsc{-neg=adv} pass.climb\textsc{-aug-hab-frust-q} \textsc{qtag} say\textsc{-seq=rep} \textsc{3pl} climb\textsc{-dynm-decl} child\textsc{=pl-decl}\\
\glt ‘“What could it be, why does Mama always climb up there when she comes back from the swidden garden?” Saying this, it’s said, they climbed up (to see), the children.'
\glt ‘“O que será? Por que mamãe sempre sobe lá quando ela volta da roça?" Falando assim, dizem, eles subiram para ver, as crianças.'
\z 

\ea  Huphipãhnɨ́h yẽ́háh dö́’d’ähätíh, nutènéyd’äh hínitíh!\\
\gll hup-hipãh-nɨ́h yẽ́h-áh dö́ʔ=d’äh-ätíh, nutènéy=d’äh=hín-itíh\\
     \textsc{refl-}know\textsc{-neg} \textsc{frust-foc} child\textsc{=pl-emph2} today\textsc{=pl}=also-\textsc{emph2}\\
\glt ‘Those children did not know better,{\footnotemark}\footnotetext{That is, they lacked a sense of what is right and/or socially acceptable (\textit{hup-hipãh-nɨ́h} [\textsc{refl}-know\textsc{-neg}] lit. ‘did not know themselves’).} just like children of today!'
\glt ‘Essas crianças não entenderam, como as crianças de hoje em dia.'
\z 

\ea  Sak yö́’ mah hɨd kë́yayáh, pɨb dɨ́’ay nííy sud mah.\\
\gll sak-yö́ʔ=mah hɨd kë́y-ay-áh, pɨb dɨ́ʔ-ay ní-íy=sud=mah.\\
     climb\textsc{-seq=rep} \textsc{3pl} see\textsc{-inch-decl} strong remain\textsc{-inch} be\textsc{-dynm=infr2=rep}\\
\glt ‘Climbing up they saw it, it apparently was already growing strong, it's said.'
\glt ‘Subindo, eles viram, já estava crescendo forte, dizem.'
\z 

\newpage
\ea  Yɨ́t mah “Apá! ɨn ín-tẽ́h sud yúwúh, ã́y!” hɨd ũh nóayáh, “méh!” hɨd ũh nóayáh.\\
\gll yɨ́t=mah ʔapá! ʔɨn ʔín=tẽ́h=sud yúw-úh, ʔã́y!” hɨd ʔũh-nó-ay-áh, “méh!” hɨd ʔũh=nó-ay-áh.\\
     thus\textsc{=rep} \textsc{interj} \textsc{1pl} mother=offspring\textsc{=infr2} \textsc{dem.itg-decl} old.sister \textsc{3pl} \textsc{recp=}say\textsc{-inch-decl} younger.sister \textsc{3pl} \textsc{recp=}say\textsc{-inch-decl}\\
\glt ‘So, it’s said, “Ah, this must be our sibling, older sister!" they said to each other, “younger sister!” they said to each other.'{\footnotemark}\footnotetext{The children are using these terms of address to each other, as is common in Hup discourse.}
\glt ‘Aí, dizem, “Ô, deve ser o filho de nossa mãe, irmã maior!" eles se falavam, “irmã menor!" eles se falavam.'
\z 

\ea  Yɨno yö́’ mah yúp, tɨ́hàn hɨd dö’ híayáh.\\
\gll yɨ-no-yö́ʔ=mah yúp, tɨ́h-àn hɨd döʔ-hí-ay-áh.\\
     \textsc{dem.itg-}say\textsc{-seq=rep} \textsc{dem.itg} \textsc{3sg-obj} \textsc{3pl} take-descend\textsc{-inch-decl}\\
\glt ‘Saying thus, it’s said, they took (the baby deer) down.'
\glt ‘Falando assim, dizem, eles trouxeram (o nenê veado) para baixo.'
\z 

\ea  D’ö’ hi yö́’ mah, “Máy! n’ikán, kayak tìg k’et, pí’ k’et ɨn no’ k’ö́’ayáh, yɨ́’an!” no yö́’ mah,\\
\gll d’öʔ-hi-yö́ʔ=mah, máy! n’ikán, kayak=tìg=k’et, píʔ=k’et ʔɨn noʔ-k’ö́ʔ-ay-áh, yɨ́ʔ-an!” no-yö́ʔ=mah,\\
     take-descend\textsc{-seq=rep} let’s.go over.there manioc=stem=leaf potato=leaf \textsc{1pl} give-go.about\textsc{-inch-decl} capoeira\textsc{-dir} say\textsc{-seq=rep}\\
\glt ‘Taking (it) down, saying, “Come on! Let's go give it manioc and potato leaves out there in the \textit{capoeira} (overgrown swidden)!” it’s said,'
\glt ‘Depois de baixâ-lo, falando, “Bora! Vamos lá na capoeira para dar folhas de mandioca, folhas de batata para ele!" dizem,'
\z 

\ea  hɨd ton hámayáh, hɨd ín b’òtan ham yɨ́’ mɨ̀’, “ɨn ín-tẽ́h sud yúwúh!” no yö́’.\\
\gll hɨd ton-hám-ay-áh, hɨd ʔín b’òt-an ham-yɨ́ʔ-mɨ̀ʔ, ʔɨn ʔín=tẽ́h=sud yúw-úh! no-yö́ʔ.\\
     \textsc{3pl} hold-go\textsc{-inch-decl} \textsc{3pl} mother swidden\textsc{-dir} go\textsc{-tel-under} \textsc{1pl} mother=offspring\textsc{=infr} \textsc{dem.itg-decl} say\textsc{-seq}\\
\glt ‘they took (it) off, while their mother was away in the garden, saying, “It must be our sibling!”'
\glt ‘eles levaram (o nenê), enquanto que a mãe deles estava na roça, dizendo, “Deve ser o filho de nossa mãe!"'
\z 

\ea  Yɨkán mah kayak tìg k’et hɨd nó’óh, pí’ k’et mah hɨd nó’óh,\\
\gll yɨkán=mah kayak=tìg=k’et hɨd nóʔ-óh, píʔ=k’et=mah hɨd nóʔ-óh,\\
     out.there\textsc{=rep} manioc=stem=leaf \textsc{3pl} give\textsc{-decl} potato=leaf\textsc{=rep} \textsc{3pl} give\textsc{-decl}\\
\glt ‘Out there, it’s said, they gave it manioc leaves, they gave it potato leaves, it’s said,'
\glt ‘Lá, dizem, eles deram folhas de mandioca, folhas de batata, dizem,'
\z 

\ea  hɨd no’ ë’ àpyɨ’ mah, nutèn hin tɨh wéd b’ayáh, mohòyóh.\\
\gll hɨd noʔ-ʔëʔ ʔàpyɨʔ=mah, nutèn=hin tɨh wéd-b’ay-áh, mohòy-óh.\\
     \textsc{3pl} give\textsc{-pfv} all\textsc{=rep} today=also \textsc{3sg} eat-again\textsc{-decl} deer\textsc{-decl}\\
\glt ‘they gave it everything, it’s said, that the deer eats today.'
\glt ‘deram tudo, dizem, que o veado come hoje em dia.'
\z 

\ea  Yúp mah hɨd nó’óh, hɨd kakàh d’ö’ k’ët yö́’ mah, “ɨn ín-tẽ́h sud yúwúh!” no yö́’ pɨ́d,\\
\gll yúp=mah hɨd nóʔ-óh, hɨd kakàh d’öʔ-k’ët-yö́ʔ=mah, "ʔɨn ʔín=tẽ́h=sud yúw-úh!” no-yö́ʔ pɨ́d,\\
     \textsc{dem.itg=rep} \textsc{3pl} give\textsc{-decl} \textsc{3pl} middle take-stand\textsc{-seq=rep} \textsc{1pl} mother=offspring\textsc{=infr} \textsc{dem.itg-decl} say\textsc{-seq} \textsc{distr}\\
\glt ‘So they gave it (food), it’s said, (the children) put (the baby deer) in the middle (of the circle they formed), saying “It must be our sibling!”'
\glt ‘Aí (as crianças) deram (comida), dizem, e colocaram (o nenê veado) no meio deles, dizendo “Deve ser o filho de nossa mãe!"'
\z 

\ea  sã́’ãh mah pɨ́d mah hɨd tɨy d’äh ham muhú’úh.\\
\gll sã́ʔãh=mah pɨ́d=mah hɨd tɨy-d’äh-ham-muhúʔ-úh.\\
     other.side\textsc{=rep} \textsc{distr=rep} \textsc{3pl} push-send-go-play\textsc{-decl}\\
\glt ‘they playfully pushed it back and forth.'
\glt ‘eles brincaram empurrando-o de um lado a outro.'
\z 

\newpage
\ea  Yɨ́t hɨd nɨ́hɨ́t yɨ’, yɨ́t hɨd nɨ́hɨ́t yɨ’, tɨh m’em’em’ k’ët k’ö́’ö́t mah yúp, tɨh pɨb yɨ́’ɨ́h.\\
\gll yɨ́t hɨd nɨ́h-ɨ́t=yɨʔ, yɨ́t hɨd nɨ́h-ɨ́t=yɨʔ, tɨh m’em’em’-k’ët-k’ö́ʔ-ö́t=mah yúp, tɨh pɨb-yɨ́ʔ-ɨ́h.\\
     thus \textsc{3pl} be.like\textsc{-obl=adv} thus \textsc{3pl} be.like\textsc{-obl=adv} \textsc{3sg} weak-stand-go.about\textsc{-obl=rep} \textsc{dem.itg} \textsc{3sg} strong\textsc{-tel-decl}\\
\glt ‘As they did thus, as they did thus, as it went wobbling about, it's said, it grew strong.'
\glt ‘Enquanto eles foram assim, foram assim,(o nenê), balançando aqui e lá, dizem, cresceu forte.'
\z 

\ea  Yɨ̃ nɨhɨy mah yúp, tɨh kädham yɨ́’ayáh, tɨh s’äk kädham yɨ́’ayáh,\\
\gll yɨ̃-nɨh-ɨy=mah yúp, tɨh kädham-yɨ́ʔ-ay-áh, tɨh s’äk-kädham-yɨ́ʔ-ay-áh,\\
     \textsc{dem.itg}-be.like-\textsc{dynm=rep} \textsc{dem.itg} \textsc{3sg} pass.go\textsc{-tel-inch-decl} \textsc{3sg} jump-pass.go\textsc{-tel-inch-decl}\\
\glt ‘As they (playfully pushed the deer) thus, it's said, it took off, it leapt (over them) and took off;'
\glt ‘Fazendo assim (brincando com o veado), dizem, ele foi embora, pulou (por cima deles) e foi embora,'
\z 

\ea  hɨd kakàh yɨ’ mah yúp, s’ẽ́ç no kädham yɨ́’ay mah.\\
\gll hɨd kakàh=yɨʔ=mah yúp, s’ẽ́ç no-kädham-yɨ́ʔ-ay=mah.\\
     \textsc{3pl} middle\textsc{=adv=rep} \textsc{dem.itg} deer.snort say-pass.go\textsc{-tel-inch=rep}\\
\glt ‘it leapt out of the middle (of the circle of children) and took off, it gave a snort \textit{sẽ́ç!} and took off, it’s said.'
\glt ‘ele pulou do meio (das crianças) e foi embora, bufou \textit{sẽ́ç!} e foi embora, dizem.'
\z 

\ea  Yĩnóóy yẽ́h tɨ́h-ĩhĩtíh, sẽ́ç! no kädham yɨ́’ay mah.\\
\gll yĩ-nó-óy yẽ́h tɨ́h=ĩh-ĩtíh, sẽ́ç! no-kädham-yɨ́ʔ-ay=mah.\\
	\textsc{dem.itg-}say\textsc{-dynm} \textsc{frust} \textsc{3sg=msc-emph2} deer.snort say-pass.go\textsc{-tel-inch=rep}\\
\glt ‘That’s what it said: \textit{sẽ́ç!} and it took off, it’s said.’
\glt ‘Assim que ele falou: \textit{sẽ́ç!} e foi embora, dizem.'
\z 

\newpage
\ea  Yɨnɨh yö́’ mah yɨ́tyɨ’, bahadnɨ́h tɨh níayáh, yúp hɨd ín-tẽ́hayáh.\\
\gll yɨ-nɨh-yö́ʔ=mah yɨ́t=yɨʔ, bahad-nɨ́h tɨh ní-ay-áh, yúp hɨd ín=tẽ́h-ay-áh.\\
     \textsc{dem.itg-}be.like\textsc{-seq=rep} thus\textsc{=adv} appear\textsc{-neg} \textsc{3sg} be\textsc{-inch-decl} \textsc{dem.itg} \textsc{3pl} mother=offspring\textsc{-inch-decl}\\
\glt ‘So with that, it’s said, it disappeared, their sibling.'
\glt ‘Assim, dizem, ele desapareceu, o filho da mãe deles.'
\z 

\ea  Yúp won d’ak k’ö’ këy ë́y yẽ́h mah yɨ́d’ähä́h, won d’ak k’ö’ këy ë́y mah.\\
\gll yúp won-d’ak-k’öʔ-këy-ʔë́-y yẽ́h=mah yɨ́-d’äh-ä́h, won-d’ak-k’öʔ-këy-ʔë́-y=mah.\\
     \textsc{dem.itg} follow-be.against-go.about-see\textsc{-pfv-dynm} \textsc{frust=rep} \textsc{dem.itg-pl-decl} follow-be.against-go.about-see\textsc{-pfv-dynm=rep}\\
\glt ‘They went wandering around looking for it in vain, it’s said, those (children), wandering around looking for it in vain, it’s said.'
\glt ‘Eles andavam procurando-o em vão, dizem, essas (crianças), andavam procurando-o em vão, dizem.'
\z 

\ea  Hɨd yë yɨ́’ayáh, “Hɨn’ɨ̀h tëg ɨ́n?! ɨ́n ín ɨ́nàn meh të́g ɨ́nànáh!” no yö́’ mah,\\
\gll hɨd yë-yɨ́ʔ-ay-áh, “Hɨn’ɨ̀h-të́g ʔɨ́n?! ʔɨ́n ʔín ʔɨ́n-àn meh-të́g ʔɨ́n-àn-áh!” no-yö́ʔ=mah,\\
     \textsc{3pl} enter\textsc{-tel-inch-decl} what\textsc{-fut} \textsc{1pl} \textsc{1pl} mother \textsc{1pl-obj} beat\textsc{-fut} \textsc{1pl-obj-decl} say\textsc{-seq=rep}\\
\glt ‘They returned home, it’s said, saying, “What shall we do?! Our mother will beat us!”'
\glt ‘Eles voltarem, dizem, falando, “Como vamos fazer?! Nossa mãe vai nos bater!"'
\z 

\ea  tëg sä̀h b’ɨ́yɨ’ mah hɨd mug sud hitab k’ã’ yɨ́’ayáh, yúp sákuan b'ay.\\
\gll tëg=sä̀h b’ɨ́yɨʔ=mah hɨd mug-sud-hitab-k’ãʔ-yɨ́ʔ-ay-áh, yúp sáku-an-b'ay.\\
     wood=charcoal only\textsc{=rep} \textsc{3pl} scoop.by.hand-be.inside-fill-hang\textsc{-tel-inch-decl} \textsc{dem.itg} bag\textsc{-dir}-again\\
\glt ‘They filled that sack up with charcoal and hung it up again.'
\glt ‘Eles encherem o saco com carvão e penduraram de novo.'
\z

\ea  Yɨnɨ́hɨ́y mah hɨd s’omd'äh tu’ k’ö’ yɨ́’ɨ́h, hɨd yɨnɨh mɨ̀’, mòy hat hupsɨ̃p yɨ’ sɨ̃́wɨ̃́y sud mah, hɨ́dɨwɨ́h.\\
\gll yɨ-nɨ́h-ɨ́y=mah hɨd s’om=d'äh tuʔ-k’öʔ-yɨ́ʔ-ɨ́h, hɨd yɨ-nɨh mɨ̀ʔ, mòy hat-hupsɨ̃p-yɨʔ-sɨ̃́w-ɨ̃́y=sud=mah, hɨ́d-ɨw-ɨ́h.\\
     \textsc{dem.itg-}be.like\textsc{-dynm=rep} \textsc{3pl} bathe\textsc{=pl} immerse-go.about\textsc{-tel-decl} \textsc{3pl} \textsc{dem.itg-}be.like \textsc{under} dwelling.hole dig-finish\textsc{-tel-compl-dynm=infr=rep} \textsc{3pl-flr-decl}\\
\glt ‘Thus, it’s said, while they were going about bathing, they had apparently already dug (dwelling-)holes, it’s said.'
\glt ‘Aí, dizem, enquanto estavam indo tomar banho, eles já tinham cavado os buracos deles, dizem.'
\z

\ea  Yɨkán mah yúp, moytùd mòy hɨd nóowóh.\\
\gll yɨkán=mah yúp, moytùd mòy hɨd nó-ow-óh.\\
     there\textsc{=rep} \textsc{dem.itg} curassow  dwelling.hole \textsc{3pl} say\textsc{-flr-decl}\\
\glt ‘Out there, it’s said, curassow (\textit{Nothocrax urumutum}) holes, they call them.'
\glt ‘Para lá, dizem, buracos de urumutum, como chamam.'
\z

\ea  Yɨkán ũhníy yúp mòyóh.\\
\gll yɨkán ʔũhníy yúp mòy-óh.\\
     there maybe \textsc{dem.itg} dwelling.hole\textsc{-decl}\\
\glt ‘Those holes were out there, maybe [pointing].'
\glt ‘Esses buracos estavam para lá, talvez [apontando].'
\z

\ea  Yɨnɨ́hɨ́y mah, wɨdyë yɨ’ nííy ni yö́’, “huphipãh nɨ́h nɨg niníh!” no yö́’ mah,\\
\gll yɨ-nɨ́h-ɨ́y=mah, wɨdyë-yɨʔ ní-íy ni-yö́ʔ, “hup-hipãh-nɨ́h nɨg ni-ní-h!” no-yö́ʔ=mah,\\
     \textsc{dem.itg-}be.like\textsc{-dynm=rep} arrive.enter\textsc{-tel} be\textsc{-dynm} be\textsc{-seq} \textsc{refl-}know\textsc{-neg} \textsc{2pl} be\textsc{-infr-decl} say\textsc{-seq=rep}\\
\glt ‘So, it’s said, (their mother) having come home, saying, “You all don’t know what’s right (acted irresponsibly)!”'
\glt ‘Aí, dizem, (a mãe), tendo voltado para a casa, ficou dizendo, “Vocês não têm inteligência!"'
\z

\newpage
\ea  hɨ́dàn tɨh sɨwɨp sij d’äh way yɨ́’ɨ́h, hɨ́dàn tɨh méhéway, tɨh in, hɨd ínay.\\
\gll hɨ́d-àn tɨh sɨwɨp-sij-d’äh-way-yɨ́ʔ-ɨ́h, hɨ́d-àn tɨh méh-éw-ay, tɨh in, hɨd ʔín-ay.\\
     \textsc{3pl-obj} \textsc{3sg} whip-scatter-send-go.out\textsc{-tel-decl} \textsc{3pl-obj} \textsc{3sg} beat\textsc{-flr-inch} \textsc{3sg} mother \textsc{3pl} mother\textsc{-inch}\\
\glt ‘She whipped them until they (fled) scattering, she beat them, their mother.'
\glt ‘Ela os bateu até que eles voaram, espalhando-se, ela os bateu, a mãe.'
\z

\ea  Méhéy këyö́’ mah yúp, moytùdd’äh, hɨd hidöhö ham yɨ́’ayáh, hɨ́d b’ayáh, tɨh tẽ́hd’äh k’õh ë̀’d’äh b’ayáh.\\
\gll méh-éy këyö́ʔ=mah yúp, moytùd=d’äh, hɨd hidöhö-ham-yɨ́ʔ-ay-áh, hɨ́d=b’ay-áh, tɨh tẽ́h=d’äh k’õh-ʔë̀ʔ=d’äh=b’ay-áh.\\
     beat\textsc{-dynm} \textsc{cause=rep} \textsc{dem.itg} curassow\textsc{=pl} \textsc{3pl} transform-go\textsc{-tel-inch-decl} \textsc{3pl}=again\textsc{-decl} \textsc{3sg} offspring\textsc{=pl} be\textsc{-pfv=pl}=again\textsc{-decl}\\
\glt ‘Because she beat them, they transformed into curassows, they did, those who had been her children.'
\glt ‘Porque ela os bateu, eles se transformaram em urumutuns, eles, os que eram os filhos dela.'
\z

\ea  Hɨ́dàn tɨh-kë́ sĩy’ hũ’ sɨ̃́wɨ̃́y sud mah hɨ́d hiníh.\\
\gll hɨ́d-àn tɨh=kë́ sĩy’-hũʔ-sɨ̃́w-ɨ̃́y=sud=mah hɨ́d=hin-íh.\\
     \textsc{3pl-obj} \textsc{3sg=}wing poke.in-finish\textsc{-compl=infr=rep} \textsc{3pl}=also-\textsc{decl}\\
\glt ‘They (other birds) had already filled their wings (with feathers), apparently, it’s said.'
\glt ‘Eles (outros pássaros) já tinham enchido as asas deles (com penas), parece, dizem.'
\z

\ea  Hũtẽ́hd’äh nihṹ’ mah hɨ́dàn kë́ hɨd sĩy’níh.\\
\gll hũtẽ́h=d’äh ni-hṹʔ=mah hɨ́d-àn kë́ hɨd sĩy’-ní-h.\\
     bird\textsc{=pl} be-finish\textsc{=rep} \textsc{3pl-obj} wing \textsc{3pl} poke.in\textsc{-infr2-decl}\\
\glt ‘All the birds, it’s said, filled their wings (with feathers).'
\glt ‘Todos os pássaros, dizem, encheram as asas deles (com penas).'
\z

\newpage
\ea  Yúp tɨh kë́ sĩy’ hũ’ yɨ́’ɨway k’ṍhṍy nih mah yúp,\\
\gll yúp tɨh kë́ sĩy’-hũʔ-yɨ́ʔ-ɨw-ay k’ṍh-ṍy=nih=mah yúp,\\
     \textsc{dem.itg} \textsc{3sg} wing poke.in-finish\textsc{-tel-flr-inch} be\textsc{-dynm=emph.co=rep} \textsc{dem.itg}\\
\glt ‘Thus with their wings already filled up (with feathers),'
\glt ‘Assim com as asas já preenchidas (com penas),'
\z

\ea  hɨ́d hin b’ay do’kë́y, hɨd ín hɨ́dàn meh wɨdyë́ë́t, hɨd do’kë́y, hɨd waydö’ kädway yɨ́’ayáh.\\
\gll hɨ́d=hin=b’ay do’kë́y, hɨd ʔín hɨ́d-àn meh-wɨdyë́-ë́t, hɨd do’kë́y, hɨd waydöʔ-kädway-yɨ́ʔ-ay-áh.\\
     \textsc{3pl}=\textsc{also}=again correct \textsc{3pl} mother \textsc{3pl-obj} beat-arrive.enter\textsc{-obl} \textsc{3pl} correct \textsc{3pl} fly-pass.go.out\textsc{-tel-inch-decl}\\
\glt ‘straightaway, when their mother entered to beat them, straightaway they flew out (of the house).'
\glt ‘direto, quando a mãe deles entrou para bater neles,eles saíram voando direto (da casa).'
\z

\ea  Hɨdnɨ̀h käwä̀gä́t pɨ́d mah hɨd hɨ’ popot nihíh.\\
\gll hɨdnɨ̀h käwä̀g-ä́t pɨ́d=mah hɨd hɨʔ-popot=nih-íh.\\
     \textsc{3pl.poss} eye\textsc{-obl} \textsc{distr=rep} \textsc{3pl} draw-encircle\textsc{=emph.co-decl}\\
\glt ‘They (the birds) had also drawn circles around their eyes (as curassows have).'{\footnotemark}\footnotetext{The other birds assisted them in their transformation by filling their new wings with feathers and drawing circles around their eyes.}
\glt ‘Eles (os pássaros) também tinham desenhado círculos ao redor de seus olhos (como têm os urumutuns).'
\z

\ea  Yɨnɨ́hɨ́y mah yup do’kë́y hɨd moytùd hɨd hidöhö kädsak yɨ́’ayáh.\\
\gll yɨ-nɨ́h-ɨ́y=mah yúp do’kë́y hɨd moytùd hɨd hidöhö-kädsak-yɨ́ʔ-ay-áh.\\
     \textsc{dem.itg-}be.like\textsc{-dynm=rep} \textsc{dem.itg} correct \textsc{3pl} curassow \textsc{3pl} transform-pass.climb\textsc{-tel-inch-decl}\\
\glt ‘Thus, it’s said, straightaway they transformed into curassows and (flew) quickly up.'
\glt ‘Assim, dizem, transformaram-se imediatamente em urumutuns e subiram (voando).'
\z

\ea  Opɨ́d hɨd mòy hat ë́yay k’ṍhṍy nihíh.\\
\gll opɨ́d hɨd mòy hat-ʔë́y-ay k’ṍh-ṍy=nih-íh.\\
     right.away \textsc{3pl} dwelling.hole dig\textsc{-pfv-inch} be\textsc{-dynm=emph.co-decl}\\
\glt ‘They had already dug their dwelling-holes.'
\glt ‘Eles já tinham cavado os buracos deles.'
\z

\ea  Yɨnɨ́hɨ́y mah yúp hɨd ham sij yɨ́’ayáh, waydö’ ham sij yɨ́’ay mah.\\
\gll yɨ-nɨ́h-ɨ́y=mah yúp hɨd ham-sij-yɨ́ʔ-ay-áh, waydöʔ-ham-sij-yɨ́ʔ-ay=mah.\\
     \textsc{dem.itg-}be.like\textsc{-dynm=rep} \textsc{dem.itg} \textsc{3pl} go-scatter\textsc{-tel-inch-decl} fly-go-scatter\textsc{-tel-inch=rep}\\
\glt ‘So like that, it’s said, they went scattering off, flew scattering off, it's said.'
\glt ‘Assim, dizem, eles foram espalhando-se, voaram dispersando-se, dizem.'
\z

\ea  Tã’ã́y mehàn mah tɨh woy ë’ yẽ́hẽ́h, tã’ã́yàn mah tɨh hituk hiyet ë’ yẽ́hẽ́h.\\
\gll tãʔã́y=meh-àn=mah tɨh woy-ʔëʔ-yẽ́h-ẽ́h, tãʔã́y-àn=mah tɨh hi-tuk-hi-yet-ʔëʔ-yẽ́h-ẽ́h.\\
     woman\textsc{=dim-obj=rep} \textsc{3sg} be.stingy\textsc{-pfv-frust-decl} woman\textsc{-obj=rep} \textsc{3sg} \textsc{fact-}face.down\textsc{-fact-}lie\textsc{-pfv-frust-decl}\\
\glt ‘But she (the mother) tried in vain to keep one little girl, she overturned (a basket on the hole to catch) the girl, in vain.'
\glt ‘Mas ela (a mãe) tentou em vão segurar uma menina, ela virou (um aturá sobre o buraco para pegar) a menina, em vão.'
\z

\ea  Yɨnɨ́h mɨ̀’ mah, mɨ̀’ay, s'áh k’öd sö́’, tɨh hupkäd kädhi nííy yúwúh,\\
\gll yɨ-nɨ́h mɨ̀ʔ=mah, mɨ̀ʔ-ay, s'áh k’öd=sö́ʔ, tɨh hupkäd-kädhi-ní-íy yúw-úh,\\
     \textsc{dem.itg-}be.like \textsc{under=rep} \textsc{under-inch} earth inside\textsc{=loc} \textsc{3sg} turn.around-pass.descend-be\textsc{-dynm} \textsc{dem.itg-decl}\\
\glt ‘As she did this, it’s said, underneath, inside the hole, she (the child) turned around and quickly descended (digging deeper);'
\glt ‘Assim, ela fez, dizem, para baixo, dentro do buraco, ela (a menina) virou e desceu rapidamente (cavando mais ainda);'
\z

\newpage
\ea  tã’ã́y b'ay, yɨnɨ́hɨ́y mah nusö́’ b'ay tɨh bahad kädway yɨ́’ayáh.\\
\gll tãʔã́y=b'ay, yɨ-nɨ́h-ɨ́y=mah nu-sö́ʔ=b'ay tɨh bahad-kädway-yɨ́ʔ-ay-áh.\\
     woman=again \textsc{dem.itg-}be.like\textsc{-dynm=rep} this\textsc{-loc}=again \textsc{3sg} appear-pass.go.out\textsc{-tel-inch-dynm}\\
\glt ‘then like this, it’s said, the girl appeared over here (dug up to the surface in a different spot), and quickly went out (and flew away).'
\glt ‘e assim a menina, dizem, apareceu para cá (cavando para cima até a superfície em outro lugar) e saiu (voando).'
\z

\ea  Huphipö’ nɨ́hay nííy mah yɨ́d’ähä́h.\\
\gll hup-hipöʔ-nɨ́h-ay ní-íy=mah yɨ́-d’äh-ä́h.\\
     \textsc{refl-fact.}cover\textsc{-neg-inch} be\textsc{-dynm=rep} \textsc{dem.itg-pl-decl}\\
\glt ‘They would not be caught, it’s said.'
\glt ‘Eles não deixaram que fossem pegos, dizem.'
\z

\ea  Yɨn'ɨ̀h hɨd hidöhö́öway k’ṍhṍy nih.\\
\gll yɨ-n'ɨ̀h hɨd hidöhö́-öw-ay k’ṍh-ṍy=nih.\\
     \textsc{dem.itg-nmlz} \textsc{3pl} \textsc{fact.}transform\textsc{-flr-inch} be\textsc{-dynm=emph.co}\\
\glt ‘They had transformed into those (curassows).'
\glt ‘Eles tinham se transformado nesses (urumutuns).'
\z

\ea  Yɨnɨ́hɨ́y mah yúp hɨd ham yɨ́’ayáh.\\
\gll yɨ-nɨ́h-ɨ́y=mah yúp hɨd ham-yɨ́ʔ-ay-áh.\\
     \textsc{dem.itg-}be.like\textsc{-dynm=rep} \textsc{dem.itg} \textsc{3pl} go\textsc{-tel-inch-decl}\\
\glt ‘Thus, it’s said, they went away.'
\glt ‘Assim, dizem, eles foram embora.'
\z

\newpage
\ea  Yúp mah yúp hɨd ín b’ay ot d’ak k’ö́’öp b’ayáh, bëbë́ ɨn notë́gë́h, bëbë́ ɨn notë́gë́h.\\
\gll yúp=mah yúp hɨd ʔín=b’ay ʔot-d’ak-k’ö́ʔ-öp=b’ay-áh, bëbë́ ʔɨn no-të́g-ë́h, bëbë́ ʔɨn no-të́g-ë́h.\\
     \textsc{dem.itg=rep} \textsc{dem.itg} \textsc{3pl} mother=again cry-be.against-go.about\textsc{-dep}=again\textsc{-decl} bird.sp \textsc{1pl} say\textsc{fut-decl} bird.sp \textsc{1pl} say\textsc{fut-decl}\\
\glt ‘Then, it’s said, their mother went following after them crying, like what we call a \textit{bëbë} bird'.
\glt ‘Aí, dizem, a mãe deles andava atrás, chorando, como o que chamamos de pássaro \textit{bëbë}.'
\z

\ea  Yúp mah yúp tɨh tẽ́hn’àn tɨh ótayáh.\\
\gll yúp=mah yúp tɨh tẽ́h=n’àn tɨh ʔót-ay-áh.\\
     \textsc{dem.itg=rep} \textsc{dem.itg} \textsc{3sg} offspring\textsc{=pl.obj} \textsc{3sg} cry\textsc{-inch-decl}\\
\glt ‘So it’s said, she (went) crying for her children.'
\glt ‘Aí, dizem, ela foi chorando por causa dos filhos dela.'
\z

\ea  Tɨh tẽ́hn’àn tɨh ot ë́’ yɨ’, “nɨ̀ põ'ra, nɨ̀ põ'ra!” tɨh no ë́’ yɨ́’ mah, yúp ɨd ham döhö yɨ́’ayáh.\\
\gll tɨh tẽ́h=n’àn tɨh ʔot-ʔë́ʔ=yɨʔ, “nɨ̀ põ'ra, nɨ̀ põ'ra!” tɨh no-ʔë́ʔ=yɨʔ=mah, yúp ʔɨd-ham-döhö-yɨ́ʔ-ay-áh.\\
     \textsc{3sg} offspring\textsc{=pl.obj} \textsc{3sg} cry\textsc{-pfv=adv} \textsc{1sg.poss} [offspring\textsc{.pl}] \textsc{1sg.poss} [offspring\textsc{.pl}] \textsc{3sg} say\textsc{-pfv=adv=rep} \textsc{dem.itg} say-go-transform\textsc{-tel-inch-decl}\\
\glt ‘Crying for her children, saying, “My children, my children!” so saying, she transformed (into a \textit{bëbë} bird).'{\footnotemark}\footnotetext{This quoted speech combines two languages: the first word (‘my’) is in Hup, while the second word (‘children’) is in Tukano. As noted above, this multilingual quotation, together with the compound verb ‘say-do-transform' indexes the mother's transformation via the act of crying, i.e. speaking the “language" of the \textit{bëbë} bird.}
\glt ‘Chorando pelos filhos, dizendo, “Meus filhos! meus filhos!" falando assim, ela se transformou (em pássaro \textit{bëbë}).'
\z

\newpage
\ea  Ya’àpay nih s’ã́h yúp ɨ́dɨwɨ́h.\\
\gll yaʔàp-ay=nih s’ã́h yúp ʔɨ́d-ɨw-ɨ́h.\\
     that.much\textsc{-inch=emph.co} \textsc{dst.cntr} \textsc{dem.itg} speech\textsc{-flr-decl}\\
\glt ‘That’s all there is to this tale.'
\glt ‘Tem só isso nessa fala.'
\z



\section*{Acknowledgments}

Epps expresses her gratitude to the Hup people of Taracua Igarapé and other communities of the Tiquié River for welcoming her into their homes and villages, and for their ongoing friendship and collaboration. This work was supported by funding from Fulbright-Hayes, National Science Foundation, and the Max Planck Institute for Evolutionary Anthropology. Epps also thanks CNPq and FUNAI for the permission to work in the Upper Rio Negro Region, and the Museo Paraense Emilio Goeldi, the Instituto Socioambiental, and FOIRN for practical support in Brazil. Thanks to Tony Woodbury for comments on the text, and to Kristine Stenzel and Bruna Franchetto for the invitation to participate in this volume.

\section*{Non-standard abbreviations}
Several abbreviations in this list (\textsc{emph, infr, tag}) correspond to more than one morpheme; these cases are distinguished by numbers in the gloss lines (e.g. \textsc{emph1, emph2})

\medskip\noindent

\begin{tabularx}{.45\textwidth}{lQ}
\textsc{aug} & augmentative \\
\textsc{cntr} & contrast \\
\textsc{co} & coordinator \\
\textsc{cpm} & comparative \\
\textsc{dep} & dependent \\
\textsc{dim} & diminutive \\
\textsc{dir} & directional \\
\textsc{dst} & distant (past) \\
\textsc{dynm} & dynamic \\
\textsc{emph} & emphasis \\
\textsc{epist} & epistemic \\
\textsc{ex} & existential \\
\textsc{fact} & factitive \\
\textsc{flr} & filler \\
\end{tabularx}
\begin{tabularx}{.45\textwidth}{lQ}
\textsc{frus} & frustrative \\
\textsc{hab} & habitual \\
\textsc{inch} & inchoative \\
\textsc{infr} & inferential evidential \\
\textsc{interj} & interjection \\
\textsc{ints} & intensifier \\
\textsc{itg} & intangible \\
\textsc{nonvis} & nonvisual evidential \\
\textsc{qtag} & question tag \\
\textsc{ref} & reflexive \\
\textsc{rem} & remote \\
\textsc{rep} & reported evidential \\
\textsc{resp} & respect marker \\
\textsc{seq} & sequential \\
\end{tabularx}

\begin{tabularx}{.45\textwidth}{lQ}
\textsc{tag} & discourse tag \\
\textsc{tel} & telic \\
\textsc{under} & simultaneous/under \\
\end{tabularx}
\begin{tabularx}{.45\textwidth}{lQ}
\textsc{vdim} & verbal diminutive \\
\textsc{vent} & venitive \\
\\
\end{tabularx}
 
 
\nocite{ Aikhenvald2002, Cardoso2007, Epps2005/2016, Epps2007, Epps2008, Epps2010, EppsandBolaños2017, EppsandStenzel2013, Gomez-Imbert1996, LondoñoSulkin2005, Oliveira2010, Chernela1993, Santos-Granero2006, Stenzel2013, Uzendowski2005, Vilaça2000, ViveirosdeCastro1998}

{\sloppy
\printbibliography[heading=subbibliography,notkeyword=this]
}
\end{document}
