\documentclass[output=paper,draft,draftmode,colorlinks,citecolor=brown]{langscibook}
\ChapterDOI{10.5281/zenodo.7353627}
\author{Michael Everdell\affiliation{University of Texas at Austin} and
Gabriela García Salido
\affiliation{CEA-FCPyS-Universidad Nacional Autónoma de México}}
%\ORCIDs{}

\title{Existential negation in O'dam}

\shorttitlerunninghead{Existential negation in O'dam}

\renewcommand{\lsCollectionPaperFooterTitle}{Existential negation in O'dam}

\abstract{This chapter discusses the properties of existential constructions as well as standard and existential negation in the Uto-Aztecan language O’dam. In terms of the negative existential cycle, O’dam is a Type A language where existential constructions are negated by means of standard negation strategies. We also compare existential negation in O’dam to that of several other Southern Uto-Aztecan languages, most of which appear to be Type B languages. We find that standard negation and existential negation strategies have overall played very little role in each other's development in O'dam and across Southern Uto-Aztecan.

\keywords{O'dam, Southern Uto-Aztecan, existential and locative predication}
}

\IfFileExists{../localcommands.tex}{
   \usepackage{langsci-optional}
\usepackage{langsci-gb4e}
\usepackage{langsci-lgr}

\usepackage{listings}
\lstset{basicstyle=\ttfamily,tabsize=2,breaklines=true}

%added by author
% \usepackage{tipa}
\usepackage{multirow}
\graphicspath{{figures/}}
\usepackage{langsci-branding}

   
\newcommand{\sent}{\enumsentence}
\newcommand{\sents}{\eenumsentence}
\let\citeasnoun\citet

\renewcommand{\lsCoverTitleFont}[1]{\sffamily\addfontfeatures{Scale=MatchUppercase}\fontsize{44pt}{16mm}\selectfont #1}
  
   %% hyphenation points for line breaks
%% Normally, automatic hyphenation in LaTeX is very good
%% If a word is mis-hyphenated, add it to this file
%%
%% add information to TeX file before \begin{document} with:
%% %% hyphenation points for line breaks
%% Normally, automatic hyphenation in LaTeX is very good
%% If a word is mis-hyphenated, add it to this file
%%
%% add information to TeX file before \begin{document} with:
%% %% hyphenation points for line breaks
%% Normally, automatic hyphenation in LaTeX is very good
%% If a word is mis-hyphenated, add it to this file
%%
%% add information to TeX file before \begin{document} with:
%% \include{localhyphenation}
\hyphenation{
affri-ca-te
affri-ca-tes
an-no-tated
com-ple-ments
com-po-si-tio-na-li-ty
non-com-po-si-tio-na-li-ty
Gon-zá-lez
out-side
Ri-chárd
se-man-tics
STREU-SLE
Tie-de-mann
}
\hyphenation{
affri-ca-te
affri-ca-tes
an-no-tated
com-ple-ments
com-po-si-tio-na-li-ty
non-com-po-si-tio-na-li-ty
Gon-zá-lez
out-side
Ri-chárd
se-man-tics
STREU-SLE
Tie-de-mann
}
\hyphenation{
affri-ca-te
affri-ca-tes
an-no-tated
com-ple-ments
com-po-si-tio-na-li-ty
non-com-po-si-tio-na-li-ty
Gon-zá-lez
out-side
Ri-chárd
se-man-tics
STREU-SLE
Tie-de-mann
}
   \togglepaper[1]%%chapternumber
}{}

\begin{document}
\maketitle

\newcommand{\ɇ}{\stacktwo{/}{e}}
\section{Introduction} 
In this chapter we discuss existential negation in O’dam\il{O’dam|(} (Southeastern Tepehuan)\footnote{We use O’dam here in accordance with the community’s preferences.} and compare the negation strategies used to those of other Southern Uto-Aztecan languages. 
O’dam uses several strategies to express existential meaning: the existential predicate \emph{jai'ch}, as in \REF{ex:odam-introtep}, positionals and movement verbs such as \emph{daa} `sit' in \REF{ex:odam-intromom}, and copular constructions, as in \REF{ex:odam-introbask}. All existential predication strategies in O'dam are also compatible with definite subjects, where they express a locative meaning, rather than an existential one.

\begin{exe}
\ex\label{ex:odam-introtep}
\gll Ya’ jai’ch-am gu o’dam.\\
\textsc{dem.prox} \textsc{ex-3pl.sbj} \textsc{det} O'dam\\
\glt ‘There are O'dam.’ (Text\_072011\_PSC\_GG\_elcuidadodelamujer1, 15:37)
\end{exe}

\ea
\label{ex:odam-intromom}
\gll Añ 		na=\o-guʼ 		guiʼ-ñi 			mu’-ñi 	ja’k	daa gu 	dɨ’i’n.\\
	\textsc{1sg.sbj} 	\textsc{sub=3sg.sbj-adv} 	\textsc{dem.dist-vis} 	\textsc{dem.dist-vis}	\textsc{dir} 	sit.\textsc{sg} \textsc{det}	mother.\textsc{possd}\\
\glt ‘As for me, because the mother is over there (Lit. the mother sits over there).’
(Text\_102010\_CFC\_GGS\_Cuandolacuranderaeraniña, 19:57)
\z 
\ea
\label{ex:odam-introbask}
\gll para 	dhi 	balh-cha'm     		pai'  	ja'p 	pai'    	jɨ'k 	na	jir=ki$\sim$kcham\\
	for  	\textsc{dem.prox} 	basket-on 	where 	\textsc{dir} 	where  some 	\textsc{sub} 	\textsc{cop=pl}$\sim$house\\
\glt ‘for those in The Basket over there where there are houses’ (\citealt{garciaeinrev})
\z
Existential negation is largely attested as clausal negation, although preverbal (as opposed to postverbal) constituent negation is also attested. O’dam is a Type A language\footnote{\citet{Croft1991} describes 3 language types, relating to various stages in the development of existential negation. In Type A, the standard negation strategy is used to negate verbal and existential clauses. In Type B, existentials are only negated by a special strategy. In Type C, the standard negation strategy differs from the existential negation strategy, but the existential negator is regularly used for verbal negation.} because all negation is expressed through one of two particles: \emph{cham} and \emph{cham tu’}. O’dam contrasts with other Southern Uto-Aztecan languages, which are largely type B, except Pima Bajo and Guarijío, which are types A and A $\sim$ B, respectively. The existential negation type, as well as the standard and existential negation markers of each language examined here are shown in \tabref{tab:odam-suaexcyc}. We find that there is little evidence that either negation type played a role in the other's development. The etymological variety in the standard and existential negators suggests that both sets of markers emerge, evolve and are replaced along distinct pathways.

\begin{table}
\caption{Southern Uto-Aztecan Existential Negation Cycle}
\label{tab:odam-suaexcyc}
\begin{tabularx}{\textwidth}{lQQQ}
     \lsptoprule
     Language & Existential & Standard & Existential \\
     &  Negation Type &   Negation Marker &   Negation Marker\\
     \midrule
 O'dam    & A   &   \emph{ cham (tu')}    &  --- \\
 \tablevspace
 Northern & B & \emph{mai}/\emph{tomali} & \emph {tiípu(ka)} \\
 Tepehuan   &   &   &\\
 \tablevspace
 Pima Bajo & A  &   \emph{im}/\emph{kova} & --- \\
 \tablevspace
 Cora & B   &   \emph{ka}    &   \emph{ka} + \emph{me'e}\\
 \tablevspace
 Huichol & B    & \emph{ka-}    &   \emph{mawe} or \\
 & & & \emph{ka} + \emph{xuawe}\\
 \tablevspace
 Guarijío & A $\sim$ B &   \emph{ki=}  &   \emph{ki'te} or \\
    &   &   &   \emph{ki=maní}\\
 \lspbottomrule
\end{tabularx}
\end{table}

In \sectref{sec:odam-basicchar} we briefly lay out some of the characteristics of O’dam, focusing on constituent order and argument expression. In \sectref{sec:odam-excont} we describe the strategies that have been attested as expressing existential meaning. In \sectref{sec:odam-negation} we discuss negation strategies in the language, beginning with standard negation (\sectref{sec:odam-staneg}) and ending with existential negation (\sectref{sec:odam-exneg}). We then take a broader look at the place of other Uto-Aztecan languages on the negative existential cycle in \sectref{sec:odam-exnegsouth} and then discuss a possible pathway of change in standard and existential negation in O'dam in \sectref{sec:odam-protoexneg}.

\section{Basic characteristics of O’dam}
\label{sec:odam-basicchar}
O'dam is a Uto-Aztecan language and is a variety of Southern Tepehuan.
As of the last census, there are  approximately 36,543 speakers of Southern Tepehuan, which consists of three varieties O'dam, Audam (Southwestern Tepehuan) and Central Tepehuan. The majority of Southern Tepehuan speakers speak O'dam and live primarily in the Mexican state of Durango with smaller communities of speakers in Nayarit and Zacatecas \citep{inegi2010}.

An O’dam clause obligatorily consists of a verb, all other clausal constituents are optional and it is quite rare for multiple DPs to appear in a sentence \citep{willett1991, garcia2014}. The language is V-initial with S and O arguments being freely ordered following the verb, this is shown in \REF{ex:odam-vpos} and \REF{ex:odam-vspo} where the subject and primary object appear in opposite orders.\footnote{S and O orders are equally free in matrix and subordinate clauses.}
\ea
\label{ex:odam-vpos}
Verb-Primary Object-Subject\\
\gll Mummu ja-kukpa-am {\ob}gu ja’tkam{\cb}$_{\textsc{po}}$ {\ob}gu sandaarux{\cb}$_{\textsc{sbj}}$..\\
\textsc{dem.dist} \textsc{3pl.po}-lock.up-3\textsc{pl.sbj} \textsc{det} persons \textsc{det} 	soldiers\\
\glt `The soldiers lock up people there (in Santiago Teneraca).' (E1\_32011\_IA\_GGS)
\z
\ea
\label{ex:odam-vspo}
Verb-Subject-Primary Object\\
\gll Ya' sap pu=x-maax-ka’ na=m-pai’ daghia’ {\ob}gu chio’ñ{\cb}$_{\textsc{sbj}}$ {\ob}gu ubii{\cb}$_{\textsc{po}}$..\\
\textsc{dem.prox} \textsc{rprt.ui} \textsc{sens=cop}-know-\textsc{stat} \textsc{sub=3pl.sbj-adv} grab \textsc{det} man \textsc{det} woman\\
\glt ‘Here one could tell where they grab her, the man to the woman.’ (Text\_082011\_CRG\_GGS\_El mito, 00:11)
\z
DPs are not marked for case, instead grammatical roles are indicated through verbal argument affixes. Subjects are marked with a subject suffix, or preverbal free form as in \REF{ex:odam-ditrans}, and a prefix on the verb that agrees with the primary object. By primary object, we mean that only one object is marked on the verb even if the clause contains more than one object.\footnote{Note that the notion here of ``primary object language" is somewhat different from the primary object alignment system. \citet{dryer1986} defines a primary object marking language as that which treats the recipient of a ditransitive sentence in the same way as the object/patient of the monotransitive sentence. However, O'dam primary object marking is somewhat less consistent.} The object that is marked on the verb is generally the most prominent (i.e. human, animate), although the exact factors that determine primary objecthood are still unknown. Both verbs in \REF{ex:odam-trans} and \REF{ex:odam-ditrans} realize the same object marking, the \textsc{3pl} marker \emph{ja-}. The primary object marker in \REF{ex:odam-ditrans} refers to the plural recipient rather than the theme\footnote{Mass nouns like \emph{koi'} `food' are morphologically and syntactically singular so \textsc{3pl} \emph{ja-} can only refer to the overtly plural recipient.} because the recipient is more prominent.
\ea
\label{ex:odam-trans}
\gll Ya’ 	ja-ai-ch-dha’-iñ..\\
\textsc{dem.prox} 	\textsc{3pl.po}-arrive-\textsc{caus-cont-1sg.sbj}\\
\glt ‘I brought them.’ (Elicitation\_032011\_IA\_GGS)
\z 
\ea
\label{ex:odam-ditrans}
\gll Añ 		tu-ja-maa 			gu 	ta$\sim$toxkolh 	gu 	koi’..\\
\textsc{1sg.sbj} 	\textsc{dur-3pl.po}-give.\textsc{pfv} 	\textsc{det}	\textsc{pl}$\sim$pig 	\textsc{det} 	food\\
\glt ‘As for me, I gave food to the pigs.’ (E1\_032011\_IA\_GGS)
\z 
While O’dam currently exhibits verb-initial order, it maintains elements of the verb-final order of Proto Uto-Aztecan---these are shown in \tabref{tab:odam-odamfeat} (\citealt{garcia2014,garciar2015}, also see \citealt[24--26]{langacker1977}  for a reconstruction of Proto Uto-Aztecan word order).

\begin{table}
\caption{O’dam features with respect to order of constituents (\citeauthor{garcia2014} 2014, \citeauthor{garciar2015} 2015)}
\label{tab:odam-odamfeat}
\fittable{%
 \begin{tabular}{ clcc }
  \lsptoprule
 VO & \multicolumn{2}{l}{~~O'dam} & OV\\
  \midrule
  prepositions	&	&	X	&	postpositions\\
  \midrule
  initial question particle	&	&	X	&	final question particle\\
  \tablevspace
  verb – adpositional phrase	&	X	&	X	&	adpositional phrase – verb\\
  \tablevspace
  auxiliary verb – main verb	&	X	&	X	&	main verb – auxiliary verb\\
  \tablevspace
  main clause –	 subordinate clause&	X	&	X	&	subordinate clause –main clause\\
 	&	&	(temporal)	&	\\
  \tablevspace
  noun – genitive	&	X	&	&	genitive - noun\\
  \tablevspace
  initial adverbial subordinator	&	X	&	&	final adverbial subordinator\\
  \tablevspace
  initial complementizer	&	X	&	&	final complementizer\\
  \tablevspace
  noun – relative clause	&	X	&	&	relative clause - noun\\
  \lspbottomrule
 \end{tabular}}
\end{table}

In the next section we describe the attested strategies in O’dam for expressing existential meaning. First, we consider the non-verbal existential predicate \emph{jai’ch}, then we consider other strategies based on locative constructions and a copular construction.
\section{Existential constructions}
\label{sec:odam-excont}
Here we consider an \emph{existential construction/existential} to be a construction that expresses a proposition about the existence of some entity \citep[1829]{mcnally2011}. In many languages these constructions are atypical in one or more ways: non-canonical subject order, lack of agreement between the subject and predicate, special morphology, specialized negation, etc. However, as we discuss in this section, existential constructions in O'dam do not appear to be encoded differently from non-existentials, therefore we cannot turn to such diagnostics. We also follow other authors in this volume, as well as \citet{Veselinova2014,Veselinova2016} in assuming the definiteness restriction, where existential constructions are constrained to indefinite nominals, although see \citet{ziv1982,reulanm1987,abbott1997,beaveretal2006,McNally2016} for further discussion and criticisms. 

O’dam uses several strategies to encode existential meaning. The primary
strategy is the non-verbal existential predicate \emph{jai’ch}, shown in
(\ref{ex:odam-tepehuans}-\ref{ex:odam-eat}). \citet[93]{garcia2014}
analyses \emph{jai’ch} as a non-verbal predicate because it takes
morphology that otherwise only appears on non-verbal predicates, such as
the stative marker  \emph{-ka’} in \REF{ex:odam-eat}.%


\ea
\label{ex:odam-tepehuans}
\gll Ya’ jai’ch-am gu o’dam..\\
\textsc{dem.prox} \textsc{ex-3pl.sbj} \textsc{det} O'dam\\
\glt ‘Here, there are O'dam.’ (Text\_072011\_PSC\_GG\_elcuidadodelamujer1, 15:37)
\z
\ea
\label{ex:odam-custom}
\gll Na=\o=gu’ 	xib 	makam 	ba-jai’ch 	gu 	kostumbre..\\
\textsc{sub=3sg.sbj-adv} today	different	\textsc{compl-ex}	\textsc{det} 	custom\\
\glt ‘because now there is a different custom’ (Text\_072011\_PSC\_GGS\_elcuidadodelamujer1, 8:50)
\z 
\ea
\label{ex:odam-lime}
\gll Jai’ch=aa	gu	jabook	matai	mi’-ñi		bibiatam jup-kai’ch	gu	Juan pui’-ñ			dho		t{\ɇ}-k{\ɇ}{\ɇ}-ka’		na	sap		jai’ch jup-kai’ch	gu	Peegro..\\
\textsc{ex=q}	\textsc{det}	light	lime	\textsc{dem.med-vis}	spring \textsc{iter}-say		\textsc{det}	Juan \textsc{sens-1sg.sbj}	\textsc{direv}	\textsc{dur}-hear-\textsc{stat}	\textsc{sub}	\textsc{rprt.ui}	\textsc{ex} \textsc{iter}-say		\textsc{det}	Pedro\\
\glt ‘“Is there lime in the spring?” Juan asked. “I have heard that there is” said Pedro’ \citep[76]{willettw2015}
\z 
\ea
\label{ex:odam-eat}
\gll Cham 	jai’ch-ka’ 	na=m 			tu’ 		jugia’..\\
\textsc{neg}	\textsc{ex-stat} 	\textsc{sub=3pl.sbj} 	something 	eat\\
\glt ‘There was nothing to eat.’ (Text\_072011\_PSC\_GGS\_elcuidadodelamujer1, 9:40)
\z 
 The \emph{jai'ch} predicate is also used for locative predications, as in (\ref{ex:odam-checkus}-\ref{ex:odam-curesabi}). There is no clear syntactic difference between locative and existential \emph{jai'ch}. Both take standard subject marking, as in \REF{ex:odam-tepehuans} and \REF{ex:odam-checkus}, and standard V-initial word order. One possible difference is that in our data existential predications are only attested with overt DPs. In contrast, \emph{jai'ch} in locative contexts is attested without a DP referring to the subject. Posture seems to have a cultural significance---in our corpora women tend to be associated with sitting posture \emph{daa}, men with standing \emph{kɨɨk} and we believe that \emph{jai'ch} is possibly used here for things that are bad or taboo (i.e. they lack posture). In our experience, mestizo doctors \REF{ex:odam-checkus}, as opposed to Tepehuan \emph{curanderos}, are rarely talked about, and the second reference to \emph{animales} (from Spanish `animals') in \REF{ex:odam-curesabi} refers to animals under the influence of a demon. Thus both apparently postureless subjects here appear to be taboo or bad, although, we must admit that this is tentative and requires further invstatigation. 
\ea
\label{ex:odam-checkus}
\gll Mia’n 	jaich-am 		gui’ 	na=m jaroi’ 		jich-rebisar-ka’.\\
close	\textsc{ex-3pl.sbj} 	\textsc{dem.dist} 	\textsc{sub=3pl.sbj} who 	\textsc{1pl.po}-check-\textsc{stat}\\
\glt ‘They [mestizo doctors] are close, the ones that check us.’ (Text\_072011\_PSC\_GGS\_elcuidadodelamujer1, 04:27)
\z 
\ea
\label{ex:odam-curesabi}
\gll Kuantas animales bhɨjɨdɨr ja'p kantar-im-am gio jumai bhɨjɨ ja'p kɨɨk g{\ɇ} ja'ok' kuj-im na=\o-jɨ'k \textbf{jaich}-\textbf{am} gu animales bhai' ba-kujim-am.\\
\emph{cuántas} \emph{animales} \textsc{dir} \textsc{dir} sing-\textsc{prog-3pl.sbj} and other \textsc{dir} \textsc{dir} stand.\textsc{sg} great demon roar-\textsc{prog} \textsc{sub=3sg.sbj}-some \textsc{ex-3pl.sbj} \textsc{det} \emph{animales} \textsc{dir} \textsc{compl}-roar-\textsc{prog-3pl.sbj}\\
\glt `how many animals came singing...the other  was standing over there, a great demon came roaring, all the animals came roaring.'
(Text003\_Hipolito\_los2compadres, 03:13)
\z 

Positional verbs in O’dam are generally used for locative constructions \citep{garcia2017} but in (\ref{ex:odam-stand}-\ref{ex:odam-sit}) we see them used for existential meaning. Similarly, the verb \emph{oilhia’} ‘move’\footnote{This is a suppletive verb---\emph{oilhia’} is the form for singular subjects, while \emph{oipo} is the form for plural subjects.} can be used for existential meaning, as in \REF{ex:odam-move}.
\ea
\label{ex:odam-stand}
\gll Mi’	kɨɨk		ma’n 	gu 	tua 	bhai’=ñich		 ji 	dhaibu.\\
\textsc{dem.med}	stand.\textsc{sg}	one 	\textsc{det} 	tree 	\textsc{dir=1sg.sbj.pfv} 	\textsc{foc} 	sit\\
\glt ‘There was a tree (Lit. there stands a tree), and I climbed and sat there.’ (Text\_092010\_HSA\_GGS\_Los2compadres, 4:51)
\z 
\ea
\label{ex:odam-sit}
\gll Dai	sap	ja'm-ni 		gok 	am 		bha 	daraa gu 	u'$\sim$ub        		tɨ$\sim$tɨya.\\
only	\textsc{rprt} 	\textsc{prt-prec} 		two	3\textsc{pl.sbj} 	\textsc{dir} 	sit.\textsc{pl.sbj}  \textsc{det} 	\textsc{pl}$\sim$woman 	\textsc{pl}$\sim$young\\
\glt ‘but that there were only two there (sitting), two girls’ (\citealt{garciaeinrev})
\z 
\ea
\label{ex:odam-move}
\gll Mi 	oipo-’am 		quince gu	ja’tkam	mi 	piesta.\\
	\textsc{dem.med}	move.\textsc{pl-3pl.sbj}	quince	\textsc{det}	people		\textsc{dem.med}	party\\
\glt ‘Are there fifteen people at the party?’ (Elicitation\_082018\_MA\_ME)
\z 
Positional verbs and \emph{oilhia’} ‘move’ appear to be 
compatible with both definite and indefinite existential and 
non-existential locative meanings. The determiner \emph{gu} is
underspecified for definiteness and can be pragmatically 
linked to (in)def\-i\-nite\-ness based on context or the appearance
of certain quantifiers. Notice in \REF{ex:odam-lime}, reproduced below, that \emph{gu 
jabook} is not referring to a definite referent, only the 
existence of some referent. Later in the same utterance 
\emph{gu Juan} and \emph{gu Peegro} both have definite and 
specific referents. For further discussion of O'dam 
determiners and definiteness see \citet[25--28]{everdell2018}.
\ea
\gll Jai’ch=aa	gu	jabook	matai	mi’-ñi		bibiatam jup-kai’ch	gu	Juan pui’-ñ			dho		t{\ɇ}-k{\ɇ}{\ɇ}-ka’		na	sap		jai’ch jup-kai’ch	gu	Peegro.\\
\textsc{ex=q}	\textsc{det}	light	lime	\textsc{dem.med-vis}	spring \textsc{iter}-say		\textsc{det}	Juan \textsc{sens-1sg.sbj}	\textsc{direv}	\textsc{dur}-hear-\textsc{stat}	\textsc{sub}	\textsc{rprt.ui}	\textsc{ex} \textsc{iter}-say		\textsc{det}	Pedro\\
\glt ‘“Is there lime in the spring?” Juan asked. “I have heard that there is” said Pedro’ \citep[76]{willettw2015}
\z 

It is unsurprising that O’dam uses locative predicates for both locative
(\ref{ex:odam-momsit}-\ref{ex:odam-takecare}) and existential meaning
(\ref{ex:odam-stand}-\ref{ex:odam-move}), even though it also has a separate existential
predicate. The relationship between locatives and existentials has been
well documented, including from a diachronic perspective
\citep[e.g.][]{breivik1981,Gaeta2013-o}. The full set of positional verbs
is shown in \tabref{tab:odam-positverbs}, because they are suppletive for
number, we show their singular and plural forms.


\ea
\label{ex:odam-momsit}
\gll Añ 		na=\o-guʼ 		guiʼ-ñi 			mu’-ñi 	ja’k	daa gu 	dɨ’i’n.\\
	\textsc{1sg.sbj} 	\textsc{sub=3sg.sbj-adv} 	\textsc{dem.dist-vis} 	\textsc{dem.dist-vis}	\textsc{dir} 	sit.\textsc{sg} \textsc{det}	mother.\textsc{possd}\\
\glt ‘As for me, because the mother is over there (Lit. the mother sits over there).’
(Text\_102010\_CFC\_GGS\_Cuandolacuranderaeraniña, 19:57)
\z 
\ea
\label{ex:odam-takecare}
\gll Jum-kuidar-ka’ 		nai’ 	na=m 			tu-oipo.\\
\textsc{mid}-take.care-\textsc{stat} 	\textsc{dir} 	\textsc{sub=3pl.sbj} 	\textsc{dur}-move.\textsc{pl}\\
\glt ‘(They) need to take care of themselves where they are around.’
(Text\_072011\_PSC\_GGS\_elcuidadodelamujer1, 2:09)
\z 

\begin{table}
\caption{Positional verbs in O’dam \citep{garciaetal2019}}
\label{tab:odam-positverbs}
 \begin{tabularx}{.8\textwidth}{XXl}
  \lsptoprule
 \textsc{sg} & \textsc{pl} & Meaning\\ 
  \midrule
  \emph{kɨɨk}	&	\emph{guguk}	&	stand animate\\
  \emph{kɨɨk}	&	\emph{tut}	&	stand inanimate\\
  \emph{boo'}	&	\emph{bobuk}	&	lay down animate\\
  \emph{kat}	&	\emph{bit}	&	lay down inanimate\\
  \emph{daa}	&	\emph{daara}	&	sit\\
   {s{\ɇ}’}& \emph{s{\ɇ}s{\ɇ}’}&hang\\
  \lspbottomrule
 \end{tabularx}
\end{table}

The final existential strategy we find in O’dam is a copular construction with a PP or noun. Copular constructions in O'dam are formed by a copula that appears as a preclitic on the predicate expression. The element derived by the copula is treated as part of the predicate and is not treated as a syntactic object (i.e. it does not receive a coreferenced object prefix). The copula construction is limited to intransitive valency and the aspectual suffixes \emph{-ka} `stative' and \emph{-t} `imperfective'. The aspectual restriction \citet[88ff]{garcia2014} considers to be diagnostic of their status as non-verbal predicates.
\ea
\label{ex:odam-copconst}
\gll Na=p jir={\ob}xib-kam{\cb}-ka'.\\
\textsc{sub=2sg.sbj} \textsc{cop}=today-from-\textsc{stat}\\
\glt `When you were new.' \citep[89]{garcia2014}
\z
In existential copula constructions, the nominal appears as either a bare N, as in \REF{ex:odam-partysch}, or as derived with a postposition, as in \REF{ex:odam-thebask}. The most common copula used for existential predication is \emph{jir=}.\footnote{O’dam has a second copula \emph{jix=}, which is related to temporary states, while \emph{jir=} is used for permanent states \citep{martinez2016}.}
\ea
\label{ex:odam-partysch}
\gll Dhu 		sap 		buimuk 	mo 	bhai=r-piasta-ka’ ji 	bhai’-ñi 	dam-dɨr 		na-\o-pai’=r-iskuel.\\
\textsc{direv} 	\textsc{rprt.ui} 	tomorrow 	doubt 	\textsc{dir=cop}-fiesta-\textsc{stat} \textsc{foc} 	\textsc{dem.med-vis} 	up-from		\textsc{sub=3sg.sbj}-where=\textsc{cop}-school\\
\glt ‘Supposedly, tomorrow there is a party up here where there is a school.’ (Text\_092011\_MMC\_GGS\_Elborrachoylamuerte, 14:46)
\z 
\ea
\label{ex:odam-thebask}
\gll Para 	dhi 	balh-cha'm     		pai'  	ja'p 	pai'    	jɨ'k 	{na=\o}	{jir=ki$\sim$kcham}.\\
	for  	\textsc{dem.prox} 	basket-on 	where 	\textsc{dir} 	where  some 	\textsc{sub-3sg.sbj} 	\textsc{cop=pl}$\sim$house\\
\glt ‘For those in The Basket over there where there are houses.’ (\citealt{garciaeinrev})
\z 
While we generally find the copula construction being used for existential predication, we see in (\ref{ex:odam-mypartysch}-\ref{ex:odam-mybask}) that it is also compatible with locative predication. The sentences below are minimally changed from \REF{ex:odam-partysch} and \REF{ex:odam-thebask}, respectively. We use possessor prefixes to force a locative reading, because attributive possession presupposes possession and existence \citep{mithun2001}.
\ea
\label{ex:odam-mypartysch}
\gll {cham tu'} bhai ja'k \textbf{jir}=\text{jiñ-}\textbf{piasta} jir=bhammu-ñi ja'k na-pai’=\textbf{r-}\textbf{jum}-\textbf{iskuel}\\
\textsc{neg} \textsc{dir} \textsc{dir} \textsc{cop=1sg.poss-}party \textsc{cop=dir-vis} \textsc{dir} \textsc{sub}-where=\textsc{cop-2sg.poss}-school\\
\glt ‘My party is not up there, it is where your school is.’ (Elicitation\_082019\_WG\_MSE)
\z 
\ea
\label{ex:odam-mybask}
\gll Para 	dhi 	balh-cha'm     		pai'  	ja'p 	pai'    	jɨ'k 	{na=\o}	{\textbf{jir=}\textbf{jiñ-}\textbf{ki}\textbf{$\sim$kcham}}.\\
	for  	\textsc{dem.prox} 	basket-on 	where 	\textsc{dir} 	where  some 	\textsc{sub=3sg.sbj} 	\textsc{cop=1sg.poss-pl}$\sim$house\\
\glt ‘For those in The Basket over there where my houses are.’ (Elicitation\_082019\_WG\_MSE)
\z 

Now that we have discussed the expression of existential predication in
O'dam, we turn to negation. First we discuss standard negation strategies
in \sectref{sec:odam-staneg}, then we discuss the use of standard negation
in existential predications in \sectref{sec:odam-exneg} and an existential negation strategy that does not have an attested positive syntactic counterpart.
\section{Negation}
\label{sec:odam-negation}
\subsection{Standard negation}
\label{sec:odam-staneg}
\citet[1]{Miestamo2005} defines standard negation as the negation of "declarative verbal main clauses"; in the following subsections, we show that O’dam uses the same strategy for both standard negation and existential negation. Standard negation in O’dam is marked using the particle \emph{cham tu’} and its shortened form \emph{cham} \citep[109]{garcia2014}. The two negation strategies are distinguished by the position of the particle, but both can be used for clausal or constituent negation. For clausal negation, the negative particle precedes the verb, as in \REF{ex:odam-karabin} and \REF{ex:odam-happenap}. For constituent negation, the negation particle follows the negated elements (e.g. DPs), as in \REF{ex:odam-mestiza}. The negated element in these examples in underlined. It is rare but there are a few attested examples where \emph{cham tu’} precedes a negated element that is not a verb, like in (\ref{ex:odam-human}-\ref{ex:odam-notoday}), although this is not attested for \emph{cham}.

\ea
\label{ex:odam-karabin}
\gll Karabiñ-kɨ’n 		tɨi 		pu=p 		jiñ-ma’yasa na=ñich 		cham	\uline{oi}.\\
carabine-with 	\textsc{nrint} 	\textsc{sens=it} 	\textsc{1sg.po}-shoot \textsc{sub=1sg.sbj} 	\textsc{neg} 	go.\textsc{pfv}\\
\glt ‘With a rifle he wanted to shoot me because I did not go.’ (Text\_062011\_ESS\_GGS\_susamores, 04:51)
\z 
\ea
\label{ex:odam-happenap} 
\gll {Na=\o} 	{cham tu’} 	\uline{tu=x-pasarui-dha}.\\
	\textsc{sub=3sg.sbj}	\textsc{neg}		\textsc{dur=cop}-happen-\textsc{appl}\\
\glt ‘So that nothing happens to us.’ (Text028/Text\_102010\_MCC\_GGS\_Losmuchachosquebuscabancomida, 07:16)
\z 
\ea
\label{ex:odam-mestiza}
\gll Maʼnim	dhu 		gu 	siman	ji 	na=ñ chu-bos-ka’		\uline{gu}	\uline{nabat} 		cham	na=\o-jax		xia’lhi-dhaʼ.\\
one.time 	\textsc{direv} 	\textsc{det} 	week 	\textsc{foc} 	\textsc{sub=1sg.sbj} 	\textsc{dur}-sweep-\textsc{stat}	\textsc{det} 	mestizo 	\textsc{neg} 	\textsc{sub=3sg.sbj}-how 	dawn-\textsc{cont}\\
\glt ‘Once a week, I sweep, but the mestiza does not, she sweeps whenever she wakes up.’ (Text\_072011\_PSC\_GGS\_elcuidadodelamujer2, 08:10)
\z 
\ea
\label{ex:odam-human}
\gll {Cham tu'}	\uline{tu’} 		\uline{ja’tkam}	ja’pi 	xi’$\sim$xbulhi-k.\\
\textsc{neg} 		something 	people 		but 	\textsc{pl}$\sim$swirl-\textsc{pnct}\\
\glt ‘They were not human, they were swirls.’ (Text\_092011\_MMC\_GGS\_Lamujerquenopodiatenerhijos, 12:20)
\z 
\ea
\label{ex:odam-notoday}
\gll Dhu 	ji 	xib 	ji 	{cham tu'} 	kabuimuk.\\
\textsc{direv}	\textsc{foc}	today	\textsc{foc} 	\textsc{neg}		\uline{tomorrow}\\
\glt ‘Well today, not tomorrow.’ (Text\_092011\_Varios\_GGS\_pláticaenlacocina, 05:05)
\z 
In addition to clauses and noun phrases, \emph{cham} and \emph{cham tu’} are used to negate directionals and demonstratives \REF{ex:odam-teneraca} and pronouns (\ref{ex:odam-alwaystire}-\ref{ex:odam-anotherwom}).
\ea
\label{ex:odam-teneraca}
\gll gu 	chiatnarak 	ach 		\uline{ya’} 	cham 	ji\\
\textsc{det} 	Teneraca 	\textsc{1pl.sbj} 	\textsc{dem.prox} 	\textsc{neg} 	\textsc{foc}\\
\glt ‘The people from Teneraca, as for us, not (the ones from) here.’ (Text\_082011\_CRG\_GGS\_Conquistarmujer, 00:12)
\z 
\ea
\label{ex:odam-alwaystire}
\gll \uline{Ach} 		cham 	na=ch 			jir=o’dam 		na=ch-gu' 	jix=momgon-ka’\\
\textsc{1pl.sbj} 	\textsc{neg} 	\textsc{sub=1pl.sbj} 	\textsc{cop}=O'dam 	\textsc{sub=1pl.sbj-adv}	\textsc{cop}=tired-\textsc{stat}\\
\glt ‘We do not, the O’dam people, because we are always tired.’ (Text\_072011\_PSC\_GGS\_elcuidadodelamujer2, 08:34)
\z 
\ea
\label{ex:odam-anotherwom}
\gll Añ      		ubii  		ya'	ai-ch-dha jumai' 		{cham tu'}       \uline{ap}.\\
\textsc{1sg.sbj} 	woman		\textsc{dem.prox} 	arrive-\textsc{caus-appl}	another		\textsc{neg} \textsc{2sg.sbj}\\
\glt ‘I am going to bring another woman and not you.’
(Text045\_102010\_CFC\_GGS\_Cuandolacuranderaeraniña, 21:18)
\z 
Finally, when negating a dependent clause, \emph{cham} appears inside of the dependent clause but still precedes the verb, as in \REF{ex:odam-chamdep}, while \emph{cham tu'} always immediately precedes the subordinator \REF{ex:odam-chamtudep}.
\ea
\label{ex:odam-chamdep}
\gll no’=ñ git jir=alhii-ka’ cham bhammuk-da’-iñ git gio {\ob}\textbf{na=ñ} \textbf{cham} jiñ-lokiar-da’]\\
\textsc{cond=1sg.sbj} \textsc{subj} \textsc{cop}=little-\textsc{stat} \textsc{neg} angry-\textsc{cont-1sg.sbj} \textsc{subj} \textsc{coord} \textsc{sub=1sg.sbj} \textsc{neg} \textsc{1sg.mid}-crazy-\textsc{cont}\\
\glt `If I were a child, I could not be able to get angry or get crazy.'
(Text\_092010\_MSM\_GGS\_Lavidatepehuana)
\z
\ea
\label{ex:odam-chamtudep}
\gll Jix=kako’k-ka’-am {cham tu’} na=m tu’ jix=kɨkɨ’-ka’-am.\\
\textsc{cop}=sick-\textsc{stat-3pl.sbj} \textsc{neg} \textsc{sub=3pl.sbj} something \textsc{cop}=healthy-\textsc{stat-3pl.sbj}\\
\glt ‘They are ill, they are not in good health.’
(Text\_072011\_PSC\_GGS\_elcuidadodelamujer2, 13:42)
\z
Beyond positional differences, it is not clear what the differences in usage are between \emph{cham} and \emph{cham tu'}. The former may be somewhat more emphatic because \citet[136--140]{garcia2014} finds that negative commands are only formed with \emph{cham} + \emph{ap} `\textsc{2sg.sbj}', as in \REF{ex:odam-negcomm}. However, we do not have clear evidence that emphaticness distinguishes the two negators otherwise.
\ea
\label{ex:odam-negcomm}
\begin{xlist}
\item\gll gio sap bhai’=p ka-xi-juu \textbf{cha=p} dhu=ñ kua’da’ jiñ-jaduñ ja’p sap kai’ch\\
\textsc{coord} \textsc{rprt.ui} \textsc{dir=iter} \textsc{prf-imp}-eat \textsc{neg=2sg.sbj} \textsc{evid=1sg.sbj} eat-\textsc{cont} \textsc{1sg.poss}-brother \textsc{dir} \textsc{rprt.ui} say.\textsc{pfv}\\
\glt `And he ate again, do not eat me brother, he said.’ (Text\_072011\_PSC\_GGS\_Gokbhabomkox, 28:59)
\item\gll \textbf{Cha’=p} ñiok-da’ tɨɨ gu-m-taat na=t-jax dhoda\\
\textsc{neg=2sg.sbj} speak-\textsc{cont} \textsc{nrint} \textsc{det-2sg.poss}-father \textsc{sub=3sg.sbj.pfv}-how do.something.to.person\\
\glt ‘Shut up, you do not know what he did to your father.’ (Text\_092010\_HSA\_GGS\_Elcuento, 04:28)
\end{xlist}
\z
\subsection{Existential negation}
\label{sec:odam-exneg}
In terms of the negative existential cycle, O’dam is a Type A language, where standard negation strategies are used for existential constructions. Notice in (\ref{ex:odam-noexist}-\ref{ex:odam-bigplain}) \emph{jai’ch} is negated by an immediately preceding negation particle and that the negation strategy is the same regardless of whether the existential occurs in a matrix clause (\ref{ex:odam-noexist}-\ref{ex:odam-nowom}) or a subordinate clause (\ref{ex:odam-demon}-\ref{ex:odam-bigplain}). It seems that singular nouns are only used to negate the existence of a singular referent (e.g. the demon), while plural nouns are used to negate the existence of sets (e.g. women, plants). This appears to contrast with O’dam treatment of mass nouns, which are morphosyntactically singular but may have individuated units.\footnote{For example, tortillas, potatoes and apples are all mass nouns in O’dam but \citet{everdelld2019} find that they can trigger plural state marking on resultatives and statives.}
\ea
\label{ex:odam-noexist}
\gll Bajɨk	dɨr 	{cham tu'} 	jaich-ka' 	dhu.\\
before 	\textsc{dir} 	\textsc{neg}		\textsc{ex-stat}	\textsc{direv}\\
\glt ‘That did not exist before.’ (Text007/Text\_092010\_MSM\_GGS\_Lavidatepehuana, 11:03)
\z 
\ea
\label{ex:odam-nowom}
\gll Cham 	jai’ch-am-a’ 		ba’ 	gu 	u’$\sim$ub.\\
\textsc{neg}	\textsc{ex-3pl.sbj-irr}	\textsc{seq}	\textsc{det}	\textsc{pl}$\sim$woman\\
\glt ‘Then there are no women (and there will be no women).’ (Text\_082011\_MMC-MRS\_GGS\_Conversación, 00:52)
\z 
\ea
\label{ex:odam-demon}
\gll Ji 	chu'ul 		pu	jii     		na=\o-jax	cham	ka-jaich 	xib gu  	ji 	chu'ul.\\
\textsc{foc}	demon   	\textsc{sens} 	go.\textsc{pfv} 	\textsc{sub=3sg.sbj}-how 	\textsc{neg}  	\textsc{prf-ex} today \textsc{det} 	\textsc{foc}	demon\\
\glt ‘The demon went and since then there hasn’t been a demon.’ (Text013/Text\_092010\_HSA\_GGS\_Elcuento, 07:15)
\z 
\ea
\label{ex:odam-bigplain}
\gll G{\ɇ}' 	giotɨr 	pai' 	{na=\o} 	cham 	jai'ch 	gu 	u'$\sim$ux.\\
Big	Plains	where	\textsc{sub=3sg.sbj}	\textsc{neg}	\textsc{ex}	\textsc{det}	\textsc{pl}$\sim$plant\\
\glt ‘Llano Grande where there are no plants.’ (Text\_082011\_MMC\_GGS\_La estrelladelamañana3, 05:47)
\z 

Standard negation strategies are also used for existential constructions where the predicate element is other than \emph{jai’ch}. In 
(\ref{ex:odam-fuckme}-\ref{ex:odam-nodance}), the standard negation strategy is used for the copular existential construction in a subordinate and main clause, respectively.
\ea
\label{ex:odam-fuckme}
\gll Mi=ñ 	jodero 	no=ñ	jim na=Ø 	cham 	\uline{pai’}	\uline{jir=ki$\sim$kcham} ja’p 	sap 	tɨtda.\\
\textsc{dem.med=1sg.sbj}	fuck	\textsc{cond=1sg.sbj}	go	\textsc{sub=3sg.sbj}	\textsc{neg}	place	\textsc{cop=pl}$\sim$house		\textsc{dir}	\textsc{rprt.ui}	say\\
\glt ‘He's going to fuck me if I walk around where there are no people, he said.’ (lit. if I walk around in the place where there are no houses) (Text028\_102010\_MCC\_GGS\_Losmuchachosquebuscabancomida, 05:43)
\z 
\ea
\label{ex:odam-nodance}
\gll {Cham tu'} 	pɨk	mi' 	jap	\uline{jir=bailes-ka'}		mi'	ja’p	pai' dhi' 	juktɨr.\\
\textsc{neg}		\textsc{prt}	\textsc{dem.med}	\textsc{dir} 	\textsc{cop}=dances-\textsc{stat} 	\textsc{dem.med} 	\textsc{dir} 	where \textsc{dem.prox}  	Santa\_María\_de\_Ocotán\\
\glt ‘Now there are no dances in Santa María de Ocotán.’ (Text007\_092010\_MSM\_GGS\_Lavidatepehuana, 19:10)
\z 
While constituent negation is well attested in non-existential contexts, in existential constructions we have no attested cases of postverbal constituent negation. Instead, apparently constituent negation must take place before the verb, as in \REF{ex:odam-buyeat} where the demonstrative \emph{ya'} is negated. In \REF{ex:odam-nobody} and \REF{ex:odam-seeanything} we see examples of negation of preverbal indefinite pronouns, where DPs cannot appear. 
\ea
\label{ex:odam-buyeat}
\gll Na=\o-gu’	\uline{ya'}	cham pai'  jaich 		gu  	tu' na=ñ             		chu-tan-da-'	na-ñ            	chu-kua-da-'.\\
\textsc{sub=3g.sbj-adv}	\textsc{dem.prox} 	\textsc{neg}	place		\textsc{ex} 	\textsc{det}	something	 \textsc{sub=1sg.sbj}		\textsc{dur}-buy-\textsc{cont-irr}	\textsc{sub=1sg.sbj} \textsc{dur}-eat-\textsc{cont-irr}\\
\glt ‘Because this here is not what I’m going to buy to eat.’ (Text005\_092010\_TSC\_GGS\_Guasak, 05:56)
\z 
\ea
\label{ex:odam-nobody}
\gll Cham	jaroi’ 		bha=jim.\\
\textsc{neg}	someone	\textsc{dir}=go\\
\glt ‘Nobody is coming.’ (Elicitation\_082018\_MA\_ME)
\z 
\ea
\label{ex:odam-seeanything}
\gll Cham 	tu’ 		nɨi’ñ-iñ.\\
\textsc{neg}	something	see-1\textsc{sg.sbj}\\
\glt ‘I do not see anything.’ (Elicitation\_082018\_MA\_ME)
\z 
We only find examples of non-clausal negation of indefinite pronouns (no + somebody, no + thing, etc.) in existential constructions; we have no examples of negated preverbal subject pronouns. Thus, in O'dam clausal negation appears to rely on the definiteness of the verbal arguments. We show indefinite existential negation through the clausal negation strategy in (\ref{ex:odam-notheat}-\ref{ex:odam-noask}). In both examples, the subordinate clauses expressing the negated referent use the indefinite pronouns \emph{tu’} ‘something’ and \emph{jaroi’} ‘someone’, respectively.
\ea
\label{ex:odam-notheat}
\gll Cham 	jai’ch-ka’ 	na=m 			tu’ 		jugia’.\\
\textsc{neg} 	\textsc{ex-stat} 	\textsc{sub=3pl.sbj} 	something 	eat\\
\glt ‘There was nothing to eat.’ (Text\_072011\_PSC\_GGS\_elcuidadodelamujer1, 9:40)
\z 
\ea
\label{ex:odam-noask}
\gll Na=\o-gu' sap cham jai'ch-ka' na=ñ tu-tɨka-'.\\
\textsc{sub=3sg.sbj-adv} \textsc{rprt.ui} \textsc{neg} \textsc{ex-stat} \textsc{sub=1sg.sbj} \textsc{dur-}cover-\textsc{irr}\\
\glt ‘Because there was nothing to cover me with.’ (Text\_102010\_PSC\_GGS\_Lavidademiesposo, 43:00)
\z 
It may be that existential constructions in O’dam entirely disallow postverbal constituent negation or that it is simply unattested. Our findings for O'dam (negative) indefinite pronouns align with typological work showing that negative/ne\-gated indefinite pronouns can often function as direct negation markers \citep{Haspelmath1997,Veselinova2013,Alsenoy2016}. At this point, we do not find any difference in the use of negated \emph{jai'ch} + \textsc{indefinite pronoun} versus a negated indefinite pronoun, unlike in Swedish \citep{Bordal2017}. However, our current corpus is relatively small so we do not discount statistical tendencies.

In addition to the use of standard negation on attested existential construction types, we also find several cases where a negative existential meaning arises out of a construction that is not attested in positive existential contexts. The verb \emph{maax} ‘see, notice’ can express an existential meaning when negated. In \REF{ex:odam-footprints}, \emph{maax} is being used to express that there are no footprints but speakers report that the footprints discussed in the sentence are not visible because they do not exist.\footnote{Our
    consultants report that \REF{ex:odam-negsee} is quite odd if followed up with something like \REF{ex:odam-continuation} that contradicts the existential negation meaning of the original sentence.
    \ea\label{ex:odam-continuation}
    \gll ...pero mi=x jai'ch-am\\
    ...\emph{pero} \textsc{dem.med=cop} \textsc{ex-3pl.sbj}\\
    \glt  `but they are there.'
    \zlast
}
However, in positive contexts like \REF{ex:odam-possee} and some negative contexts like \REF{ex:odam-negsee}, the verb expresses visibility rather than existence.



\ea
\label{ex:odam-footprints}
\gll Na=m-gu' 			cham 	maax.\\
\textsc{sub=3pl.sbj-adv}	\textsc{neg} 	see\\
\glt ‘Because there are no footprints.’ (Text\_092011\_MMC\_GGS\_elseñorqueperdiósusanimales1, 03:49)
\z 
\ea
\label{ex:odam-aandb}
\begin{xlist}
\item\label{ex:odam-possee}\gll Ya	ja’p	bak	buus	gu	jaroi’		na	ba’	gamai’-ñi pɨx	maax	bɨix	a’nsap.\\
	\textsc{dem.prox}	\textsc{dir}	\textsc{infr}	pass	\textsc{det}	someone	\textsc{sub}	\textsc{seq}	\textsc{dir-vis} \textsc{mir}	see	along	descent\\
\glt ‘It seems that some people passed by here, you can see the tracks on the descent.’ \citep[120]{willettw2015}
\item\label{ex:odam-negsee}\gll Moo 	ja’p 	cham 	maax	jia 	na=\o-jax 	dhuu-ka-t 	tu-iipuñi-dha’		sia 	na=r tu’.\\
doubt	\textsc{dir}	\textsc{neg}	see	\textsc{ret}	\textsc{sub=3sg.sbj}-how	rain-\textsc{est-ipfv}	\textsc{dur}-grow-\textsc{cont}	\textsc{exps}	\textsc{sub=cop} something\\
\glt ‘See how you cannot tell when the plants are sprouting.’ \citep[120]{willettw2015}

\end{xlist}
\z 
In (\ref{ex:odam-nopine}-\ref{ex:odam-nopass}) we see two cases where the negated existential construction is expressed through zero-derived denominal verbs, \emph{juuk} ‘pine’ and \emph{busiñ} ‘pass’, respectively. This construction type is, thus far, unattested for positive existential meanings but is attested in negative predicative possession constructions, as in \REF{ex:odam-noeyes}. 
\ea
\label{ex:odam-nopine}
\gll {Cham tu'} 	pɨk 	mo 	ka-juku-'\\
\textsc{neg}		\textsc{prt}	doubt	\textsc{prf}-pine-\textsc{irr}\\
\glt ‘Then there are probably almost no pines.’ (Text007\_092010\_MSM\_GGS\_Lavidatepehuana, 10:17)
\z 
\ea
\label{ex:odam-nopass}
\gll {Cham tu'} 	ka-busiñ.\\
\textsc{neg}		\textsc{prf}-pass\\
\glt ‘There is no pass.’ (Text\_092011\_Varios\_GGS\_Platica, 05:31)
\z 
\ea
\label{ex:odam-noeyes}
\gll Gu 	jax	dhui 		na=\o-gu’ 			{cham tu'} bu$\sim$pui-ka-t 				jia.\\
\textsc{det} 	how 	\textsc{direv} 	\textsc{sub=3sg.sbj-adv} 	\textsc{neg} \textsc{pl}$\sim$eye-\textsc{est-ipfv} 	\textsc{ret}\\
\glt ‘Well, as he did not have eyes, right?’ (Text\_092010\_HSA\_GGS\_Los2compadres, 4:08)
\z 
In addition to overt negation, there appear to be attested cases where the negative sense is expressed, but there is no overt marker. This comes across with the adverb \emph{ampɨx} ‘only’ and the verb \emph{jugia'} ‘finish’ (\ref{ex:odam-waternoth}-\ref{ex:odam-notrain}). There does not appear to be a similar construction for positive contexts so that this construction appears to be restricted to negative meaning when used in existential contexts. However, this structure is attested outside of existential contexts, where \emph{ampɨx} appears to add the meaning that ‘everything’ will be finished, as in \REF{ex:odam-notrain}.
\ea
\label{ex:odam-waternoth}
\gll Ampɨx chu-ju’ 	mi’ 	sudai-chɨr 	apim 		chi-jix=bhio' ji	ja’p 	sap 		kai'ch.\\
only	\textsc{dur}-finish	\textsc{dir}	water-of	\textsc{2pl.sbj}	\textsc{dur-cop}=hungry \textsc{foc}	\textsc{dir}	\textsc{rprt.ui}	say\\
\glt ‘There is nothing in the water, you all will be hungry, he says.’ (Text033\_102010\_TMR\_GGS\_Los3hermanos1parte, 03:26)
\z 
\ea
\label{ex:odam-noththere}
\gll Nai' 	sap 		ba' 	pɨx 	ampɨx 	ba-tu-ju'.\\
\textsc{dir}	\textsc{rprt.ui}	\textsc{seq}	\textsc{mir}	only	\textsc{compl-dur}-finish\\
\glt ‘So there’s nothing there.’ (Text\_072011\_PSC\_GGS\_Gokbhabomkox, 08:50)
\z 
\ea
\label{ex:odam-notrain}
\gll Gio	   {na=\o} 		{ba=r-taabhak-ka’} 	ampɨx 	ji 	chu-m-jugia’.\\
\textsc{coord} \textsc{sub=3sg.sbj} 	\textsc{compl=cop}-rain-\textsc{stat} 	only 	\textsc{foc} 	\textsc{dur-mid}-finish\\
\glt ‘And when it does not rain (Lit. when rain is done), everything ends.’ (Text\_072011\_LRF\_GGS\_Lahistoriadelasmujeres1, 00:28)
\z 
Now that we have discussed existential negation in O’dam, we turn to standard and existential negation in several Southern Uto-Aztecan languages.
\section{Existential Negation in Southern Uto-Aztecan}
\label{sec:odam-exnegsouth}
O’dam is on the Tepiman branch of Uto-Aztecan, which is a subgroup of the
Southern Uto-Aztecan branch (a full tree is shown in the Appendix).
Northern Tepehuan is a Tepiman language spoken in Chihuahua and Northern
Durango. It appears to be a Type B language where there are distinct
strategies for standard and existential negation. Standard negation is
indicated through the negative particle \emph{mai} and the negative
adverbial \emph{tomali}, as shown in (\ref{ex:odam-ntother}-\ref{ex:odam-ntspan}). As we
discuss in \sectref{sec:odam-protoexneg}, it is plausible that \emph{mai} is cognate with the /m/ in O'dam \emph{cham}.
\ea
\label{ex:odam-ntother}
\gll Mai 	áágai 	aánɨ 		góóvai	áágai	aánɨ 		ɨgáa.\\
\textsc{neg} 	want	\textsc{1sg.sbj} 	\textsc{dem}	want 	\textsc{1sg.sbj}	other\\
\glt ‘I do not want those, I want the others.’ \citep[26]{bascom2003}
\z 
\ea
\label{ex:odam-ntspan}
\gll Tomali 	ɨmóóko 	go-ááli 	mai 	maátɨ 	ñioókai	oobáí-kɨ-dɨ.\\
not		one		\textsc{det}-children	\textsc{neg}	know	speak		spanish-\textsc{vblz-nmlz}\\
\glt ‘None of those children can speak Spanish.’ \citep[32]{bascom2003}
\z 
Existential negation is indicated by the negative existential \emph{tiípu(ka)}, shown in (\ref{ex:odam-nthouse}-\ref{ex:odam-ntcold}). The negative existential is apparently compatible with the negative particle \emph{mai}, as shown in \REF{ex:odam-hive}, although it is not clear if the construction in \REF{ex:odam-hive} is used for emphasis.
\ea
\label{ex:odam-nthouse}
\gll Tiipúka 	maáxi 	óodami 	kiiyɨ́rɨ.\\
\textsc{neg.ex}	seem	person		house.inside\\
\glt ‘There seems to be nobody in the house.’ \citep[264]{bascomm1998}
\z 
\ea
\label{ex:odam-ntcold}
\gll Alí 	ɨɨpídi	oidígi 		vai 	tiípu 		kuáági	ixtumá		naadá-gi dai 	gɨr-uukáda-gi.\\
	very	cold	weather	\textsc{cnj}	\textsc{neg.ex}	wood	thing		make.fire-\textsc{irr} \textsc{cnj}	\textsc{1pl.obj}-warm-\textsc{irr}\\
\glt ‘It’s very cold and there is no wood to put in the fire to warm us.’ \citep[264]{bascomm1998}
\z 
\ea
\label{ex:odam-hive}
\gll Tɨɨ́ 	aánɨ 		ɨmó	alí 	sáívuli 	ɨmó 	uuxí-ána 	dai ka 		mai	tiípu 		dɨɨ$\sim$dɨ́dɨ .\\
find	\textsc{1sg.sbj}	one	small	hive	one	tree-in		\textsc{cnj}	already		\textsc{neg}	\textsc{neg.ex}	\textsc{pl}$\sim$bee\\
\glt ‘I found a little hive in a tree and there were no more bees.’ \citep[15]{bascomm1998}
\z 
\citet{bascom1982} finds that positive existential predications in Northern Tepehuan are either expressed through juxtaposition (noun-noun, noun-pronoun, question word-noun, adjective-noun, or quantifier-noun), as in \REF{ex:odam-juxta}, or through the verb \emph{oid\textsuperscript{y}ága}, as in \REF{ex:odam-oidag}. Based on \citet{bascom1982}'s brief discussion, Northern Tepehuan may separate locative from existential predications. \citet{bascom1982} does not list examples of locative predications with the juxtaposition or the existential verb strategy. Instead, \citet{bascom1982} only gives examples of locative predications with positional verbs.
\ea
\label{ex:odam-juxta}
\begin{xlist}
\item\gll Múí-d\textsuperscript{y}u kií$\sim$ki.\\
many-some \textsc{pl}$\sim$house\\
\glt `There are many houses.' \citep[281]{bascom1982}
\item\gll Ši=ɨ́ɨ́ki-du-ka-tadai\\
\textsc{q}=how.many-\textsc{quant-stat-pst.cont}\\
\glt `How many were there?' \citep[282]{bascom1982}
\end{xlist}
\z 
\ea
\label{ex:odam-oidag}
\begin{xlist}
\item\gll Oid\textsuperscript{y}ága múí-d\textsuperscript{y} kií$\sim$ki.\\
there.are many-some \textsc{pl}$\sim$house\\
\glt `There are many houses.' \citep[282]{bascom1982}
\end{xlist}
\z
The positive existential predicate \emph{oid\textsuperscript{y}ága} is related to O'dam \emph{oilhia'} `move'. The Northern Tepehuan form is from Proto-Tepiman \emph{*oida} `to follow', while the O'dam form is from the related form \emph{*oimɨrai} `to walk about' \citep{hill2014}. \citet[281]{bascom1982} notes that the positive existential \emph{oid\textsuperscript{y}ága} can co-occur with the standard negation particle \emph{mai}. However, he does not offer examples nor does he explain possible differences between the negative existential \emph{tiipú(ka)} and \emph{mai} + \emph{oid\textsuperscript{y}ága}.

Pima Bajo, a Tepiman language spoken in Sonora, appears to be a Type A language. \citet[149]{estrada2014} finds that Pima Bajo has two suppletive existential forms that are probably historically related: one for existential predicates of plural or mass entities \REF{ex:odam-pbdeer} and another for singular existential predicates \REF{ex:odam-pbhouse}. In addition, \citet[154]{estrada2014} lists several other verbs used for existential meaning: \emph{maasi} `seem, be, exist', \emph{tu'ig} `stay', and \emph{is} `be'.
\ea
\label{ex:odam-pbdeer}
\gll	I’i 	si’ik 		amig.\\
\textsc{loc} 	\textsc{pl}$\sim$deer	exist.\textsc{pl}\\
\glt `Here, there are deer.’ \citep[149]{estrada2014}
\z
\ea
\label{ex:odam-pbhouse}
\gll	Ai-m 		kii 	in-ki-ga.\\
exist.\textsc{sg-cont}	house 	\textsc{1sg.nsbj}-house-\textsc{al}\\
\glt `My house is that.’ (lit. `There is a house, my house.’)
\citep[150]{estrada2014}
\z

Standard negation in Pima Bajo is expressed by means of the negative particle \emph{im} (\ref{ex:odam-pbcold}-\ref{ex:odam-hail}) or by the emphatic negative particle \emph{kova} \REF{ex:odam-kovain}, both in preverbal position \citep{estrada2014}. 
\ea
\label{ex:odam-pbcold}
\gll Im 	hɨɨp.\\
\textsc{neg} 	cold\\
\glt ‘There is no cold.’ \citep[162]{estrada2014}
\z 
\ea
\label{ex:odam-hail}
\gll Tia 		im 	gɨɨs-im.\\
hail 	\textsc{neg} 	fall-\textsc{cont}\\
\glt ‘It is not hailing.’ \citep[163]{estrada2014}		
\z 
\ea
\label{ex:odam-kovain}
\gll Kova-in 	gɨɨg-ia 	uus-kar	ha’a.\\
\textsc{neg.emph-imp} 	hit-\textsc{prob} 	stick-\textsc{ins} pot\\
\glt ‘Do not hit the pot with the stick!’ \citep[132]{estrada2014}
\z 		
Standard negation is also used for existential predications \citep[155]{estrada2014}. Notice in \REF{ex:odam-pbnoseem} that the standard negation marker \emph{im} is used so that existential negation is accomplished through the same means as standard negation.
\ea
\label{ex:odam-pbnoseem}
\gll As hɨgi im maasi ɨrav kuid-am.\\
\textsc{rprt} \textsc{3sg.sbj} \textsc{neg} seem inside below-\textsc{loc}\\
\glt `He said there does not seem to be anything down in there.' \citep[155]{estrada2014}
\z
Cora, a Corachol language spoken in Nayarit immediately to the south of O’dam, appears to be a Type B language. It uses the particle \emph{ka} to express standard negation \REF{ex:odam-pbeat}. This particle usually appears in first position in the clause and is followed by second position enclitics that encode subject (Vázquez Soto, p.c.).
\ea
\label{ex:odam-pbeat}
\gll ɨ́ Juan, ka pu wa-mɨ́'ɨ\\
\textsc{det} John \textsc{neg} \textsc{s3sg} \textsc{compl}-die.\textsc{sgs}\\
\glt `As for John, he did not die.' \citep[201]{vasquez2001}
\z 
Cora differentiates between a positive \REF{ex:odam-crposcop} existential copula\footnote{\citet{vazquez2013} shows that Cora can use the existential copula for locative constructions of definite referents, especially in questions.} that suppletes for number and a negative existential copula that does not supplete \REF{ex:odam-crnokids}. The standard negation particle \emph{ka} is also apparently obligatory in negative existential constructions.
\ea
\label{ex:odam-crposcop}
\gll hó’u-ni h-é’en tátsi'u?\\
\textsc{loc-inter} \textsc{3sg.sbj.anim-cop.ex.sg} rabbit\\
\glt `Where is the rabbit?' \citep[139]{vazquez2013}
\z
\ea
\label{ex:odam-crnokids}
\gll ká=pu mé’e pá’arih Chimaltita\\
\textsc{neg=3sbj} \textsc{cop.ex.neg} child Chimaltita\\
\glt ‘There are no children in Chimaltita.’ \citep[165]{vazquez2013}
\z
A verb of posture can co-occur with the existential copula in Cora, as in \REF{ex:odam-allinside}, however, it is unclear whether this co-occurrence is possible with indefinite subjects, see \citep[180ff]{vazquez2013}.
\ea 
\label{ex:odam-allinside}
\gll Núh 	náimi’i 	ma-tíh 		mána’a pwá’ame 	wi-ráa-uu.\\
\textsc{evid} all	\textsc{3pl.sbj-sub} \textsc{3pl.sbj.emph} \textsc{cop.ex.pl} \textsc{adh}:hole-inside-be.standing.\textsc{pl}\\
\glt ‘They say that all of them are inside.’
\citep[181]{vazquez2013}
\z 
\citet[181]{vazquez2013} argues that in the case of negative locative descriptions of the existential type the postural verb is ungrammatical, as in \REF{ex:odam-coraungram}. Instead only the negative existential copula may be used, as in \REF{ex:odam-corapig}. Thus, it seems that all negative existential constructions in Cora require both the standard negator \emph{ka} and the negative existential copula \emph{mé'e}.
\ea[*]{
\label{ex:odam-coraungram}
\gll Ká=pu 		wa-tá-ka 			tuíixu 	kuráh-ta’a.\\
\textsc{neg=3sg.sbj.anim} 	\textsc{compl.ext-sup}-sit.\textsc{sg} 	pig 	barnyard-\textsc{loc}\\ 
\glt Intended meaning: ‘There is not a pig in the corral.’ \citep[181]{vazquez2013}
}
\z 
\ea[]{
\label{ex:odam-corapig}
\gll Ká=pu 	mé’e 			tuíixu 		kuráh-ta’a.\\
\textsc{neg=3sg.sbj} 	\textsc{cop.ex.neg} 	pig 		barnyard-\textsc{loc}\\
\glt ‘There is not a pig in the corral.’ \citep[181]{vazquez2013}
}
\z 
Huichol, a Corachol language spoken in Nayarit just to the south of O’dam, appears to be a Type B language where a separate negative existential verb \emph{mawe} is used for existential negation,\footnote{The phonological and functional similarities between Northern Tepehuan \emph{mai}, Cora \emph{mé'e}, and Huichol \emph{mawe} are such that they may be cognate, although we hesitate to make a more definitive claim here because the exact vowel correspondences and the correspondence between Cora /'/ and Huichol /w/ are not otherwise attested \citep{stubbs2011}.} compare \REF{ex:odam-huichdisea} and \REF{ex:odam-huichyb}.
\ea
\label{ex:odam-huichdisea}
\gll Kwiniya	waniu 		mu-xuawe.\\
disease	\textsc{indir} 	\textsc{as2-ex}\\
\glt ‘…there are diseases…’ \citep[112]{bierge2017}
\z 
\ea
\label{ex:odam-huichyb}
\gll kumu ne-maine hepaɨ ’ukara-tsi pu-mawe-kai\\
how \textsc{1sg.sbj}-say how woman-\textsc{pl} \textsc{as1-neg.ex-ipfv}\\
\glt ‘…as I’m saying, there were no women…’ \citep[114]{bierge2017}
\z 
Unlike in the closely related Cora, in Huichol the negative existential apparently does not co-occur with the standard negation prefix \emph{ka-}, as in \REF{ex:odam-huichnew}. However, the negative existential \emph{mawe} can alternate with the standard negation prefix plus existential \emph{xuawe}, as shown in \REF{ex:odam-notpeo}. \citet[115]{bierge2017} notes that the \emph{ka-} + \emph{xuawe} construction is less frequently used than the negative existential \emph{mawe} and it is probably borrowed from Spanish \emph{no} + \emph{existir}. We tentatively do not consider Huichol to be intermediate between Type A and B, because the Type A strategy seems to be so marginal.
\ea
\label{ex:odam-huichnew}
\gll nee=ri kwatsie ’a-hetsie ne-p-e-tanua-ni ’a-papa\\
 \textsc{1sg}=already \textsc{aff} \textsc{2sg}-in \textsc{1sg.sbj-as1-ext}-defend-\textsc{fut} \textsc{2sg}-father\\
\glt ‘…I’ll defend you from your dad’

\gll \textbf{ka}-metsi-he-ku-waya-ni=ri\\
\textsc{neg-2sg.nsbj-ext-sp}-hit-\textsc{fut}=already\\
\glt ‘so that he does not hit you anymore…’ \citep[54]{bierge2017}
\z 
\ea
\label{ex:odam-notpeo}
\gll ne-kie teɨteri me-kwa-xuawe-kai\\
\textsc{1sg}-house people \textsc{3pl.sbj-neg-ex-ipfv}\\
\glt ‘There were no people at home.’ \citep[115]{bierge2017}
\z
Finally, Guarijío, a Taracahitan language located in the West Sierra Madre Mountains in Chihuahua and the border of Sonora, also appears to be a type A $\sim$ B language. For existential negation, speakers can choose to use a standard negation strategy with the positive existential predicate, Type A, or use a dedicated negative existential predicate without the standard negator, Type B. For standard negation Guarijío uses the clitic \emph{ki=}, which apparently attaches to the negated element (\ref{ex:odam-guarbuy}-\ref{ex:odam-guarnobuy}).
\ea
\label{ex:odam-guarbuy}
\gll Ki=tara-rú=ne			munní.\\
\textsc{neg}=buy-\textsc{pfv.evid=1sg.sbj}	beans\\
\glt ‘I didn’t buy beans.’ \citep[192]{armendariz2006}
\z 
\ea
\label{ex:odam-guarnobuy}
\gll Ki=amó	tara-ké-ru=ne			munní.\\
\textsc{neg=2sg.nsbj}	buy-\textsc{appl-pfv.evid=1sg.sbj}	beans\\
\glt ‘I didn’t buy beans for you.’ \citep[193]{armendariz2006}
\z 
The positive existential \emph{maní} \REF{ex:odam-guarbean} contrasts with the negative existential verb \emph{ki’te}, as in \REF{ex:odam-guarnobean}. However, the standard negation particle can also attach to the positive existential marker, as in \REF{ex:odam-guarwater}. \citet{armendariz2006} makes no comment on the different uses of the dedicated negative existential versus the negated positive existential.
\ea
\label{ex:odam-guarbean}
\gll Maní	munní.\\
\textsc{ex} 	beans\\
\glt ‘There are beans.’ \citep[191]{armendariz2006}
\z 
\ea
\label{ex:odam-guarnobean}
\gll Ki’té		munní.\\
\textsc{neg.ex}	beans\\
\glt ‘There are no beans.’ \citep[192]{armendariz2006}
\z 
\ea
\label{ex:odam-guarwater}
\gll Ki=maní-re nerói.\\
\textsc{neg=ex-pfv} water\\
\glt `There is no water.' \citep[115]{armendariz2006}
\z
In light of the other negation elements in Guarijío (\emph{kái} `negative answer', \emph{katé} `negative imperative'), it seems likely that the negative existential \emph{ki’té} consists of a fossilized form of the negation particle \emph{ki=} that fused with some element \emph{té}, although we do not discount the possibility that \emph{ki=} is a reduction of \emph{ki'té}.

To summarize the discussions here, we present the standard and existential negation strategies in our sample of Southern Uto-Aztecan languages in \tabref{tab:odam-snenstrat}. 

\begin{table}[t]
\caption{Negation strategies among Southern Uto-Aztecan languages}
\label{tab:odam-snenstrat}
\fittable{
 \begin{tabular}{ llll }
  \lsptoprule
 Language  (branch)& SN & NegEx & Source\\
  \midrule
  O'dam (Tepiman)	&	\emph{cham(tu')}	&	\emph{cham(tu')} & \\
  \tablevspace
 Northern (Tepiman)	&	\emph{mai}, \emph{tomali}	&	\emph{tiípu(ka)} & \citet{bascom2003}\\
 Tepehuan   &   &   &\\
  \tablevspace
Pima Bajo (Tepiman)	&	\emph{im}, \emph{kova}	&	\emph{im}, \emph{kova}? & \citet{estrada2014}\\
  \tablevspace
 Cora (Corachol)	&	\emph{ka}	&	\emph{ka} + \emph{mé'e} & \citet{vazquez2013}\\
  \tablevspace
Huichol (Corachol)	&	\emph{ka-}	&	\emph{mawe}, or & \citet{bierge2017}\\
& & \emph{ka} + \emph{xuawe} & \\
  \tablevspace
 Guarijío (Taracahitan) &	\emph{ki=}	&	\emph{ki'té} or & \citet{armendariz2006}\\
  & & \emph{ki=maní} & \\
  \lspbottomrule
 \end{tabular}
 }
\end{table}

Now that we have discussed the existential negation types in a sample of
Southern Uto-Aztecan languages, we take a historical view and posit a
developmental path for existential and standard negation in O'dam.%


\section{A possible pathway of change}
\label{sec:odam-protoexneg}
Southern Uto-Aztecan languages in general seem to be Type B languages, where existential constructions are negated by a special strategy. The exceptions are O'dam and Pima Bajo, which are both Type A, where standard negation is used in all cases, and Guarijío, which seems to be both Type A and B. In the standard and existential negation strategies in \tabref{tab:odam-snenstrat} we find a significant amount of replacement and change. This suggests that in Southern Uto-Aztecan languages, the development of standard and existential negation occurs along quite different paths. Especially in Tepiman languages (O'dam, Northern Tepehuan, Pima Bajo), both existential and standard negation appear to be highly susceptible to change, but their change cannot obviously be linked in any way.

 \citet[32--33]{langacker1977} reconstructs \emph{*ka} as the Proto Uto-Aztecan basic negative morpheme. Across Southern Uto-Aztecan, Tepiman appears to be unique in lacking reflexes of \emph{*ka}.\footnote{This includes the Tepiman language Tohono O'odham, which uses the negation particle \emph{pi}.} All of the non-Tepiman languages discussed in this chapter maintain a reflex of the particle, summarized in \tabref{tab:odam-snenstrat}. Aztecan languages, which form a subgroup with Corachol, appear to use cognates of Classical Nahuatl \emph{a'mo} \citep{launey1981} for standard negation and express existential negation through negated indefinite pronouns.\footnote{See Hausteca Nahuatl \citep{bellerb1979}, Mecayapan Nahuatl \citep{wolgemuth2002}, Michoacán Nahuatl \citep{sischo1979}, North Puebla Nahuatl \citep{brockway1979}, Tetelcingo Nahuatl \citep{tuggy1979}, Tlaxcala Nahuatl \citep{flores2019}, and Pipil \citep{campbell1985}} \citet{hill2014} and \citet{langacker1977} say that \emph{a'mo} is a reflex of the aforementioned Proto Uto-Aztecan \emph{*ka}. Pima Bajo is possibly the only Tepiman language that maintains \emph{*ka} in its negative emphatic marker \emph{kova}, which is likely derived from a combination of the basic negator \emph{*ka} plus \emph{*pa}, which \citet[32]{langacker1977} reconstructs as an emphatic affirmative.

Looking to Northern Tepehuan, \emph{mai}, if \emph{*tia=mai} was a negative Proto-Te\-pe\-huan construction, then \emph{cham} would be the expected reflex if O’dam speakers froze the full construction and Northern Tepehuan speakers only maintained the ending \emph{mai}. The initial consonant [ch] appears in O’dam due to palatalization when /t/ is immediately adjacent to /i/ and the Southwestern Tepehuan negation particle \emph{jiam}, suggests that \emph{*i} in the Proto-Tepehuan form followed the initial \emph{*t}. O’dam would have then placed stress on the initial syllable and deleted the final diphthong \citep{willett1982}.

\citet[33]{langacker1977} reconstructs \emph{***ta} as a Proto Uto-Aztecan emphatic particle that gained its negative meaning through its common use in negative expressions. This could be the source of the \emph{*tia} element in the possible Proto-Tepehuan construction, however it does not explain the high vowel. A possible source for the high vowel lies in Pima Bajo \emph{im}, which suggests there could have been a Proto-Tepiman construction \emph{**ta-imai}. It is then possible that Proto-Tepehuan or Southern Tepehuan metathesized the diphthong in the initial syllable, however, such \emph{**ai} > \emph{ia} metathesis is otherwise unattested in O’dam reflexes so this seems unlikely. Additionally, \emph{**imai} does not have a clear source as we do not find negative or emphatic morphemes in other Uto-Aztecan languages with a similar phonological shape. The possible pathway of developments is shown in \figref{fig:odam-pathchange}, however without an in-depth look at negative and emphatic particles (beyond the scope of this chapter), we can only speculate on the origins of O’dam \emph{cham}.

\begin{figure}[t]
\caption{Possible development of O'dam \emph{cham}}
\label{fig:odam-pathchange}

 \begin{tabular}{rrrr}
\emph{***ta} >	&	\emph{**ta-imai}	>	&	\emph{*tia-mai}	>	&	\emph{cham}\\
\textsc{neg.emph}	&	\textsc{neg}-??	&	\textsc{neg-neg?}	&	\textsc{neg}\\
Proto Uto-Aztecan	&	Proto-Tepiman	&	Proto-(Southern)	&		O’dam\\
& & Tepehuan & \\
\end{tabular}
\end{figure}

\citet{hill2014} finds that the Proto Uto-Aztecan negator \emph{*ka} was maintained in all subgroups as a negation marker, except Tepiman, which entirely lacks reflexes of the form. In contrast with the rest of Uto-Aztecan, it seems that Tepiman has undergone quite a bit of innovation specific to the standard negation particles. We can only speculate on the origins of the Tepiman negation particles. The possible proto-form \emph{*imai} is not attested in any other parts of the family and is only weakly constructible based on present evidence. Moreover, the Tohono O'odham negation particle \emph{pi} is not obviously connected to any  elements in any other Uto-Aztecan language.\footnote{While it is possible that it developed out of a compound of the Proto Uto-Aztecan emphatic negator \emph{**pa} and our Proto-Tepiman \emph{*imai}, we seriously doubt this. First, Proto Uto-Aztecan \emph{**p} became \emph{*v} in Proto-Tepiman, so that we would expect \emph{vi} rather than \emph{pi}. Second, \emph{*pa=imai} would have had to lose its final CV segment and completely assimilate /a/ > /i/. While final vowel devoicing and deletion is almost a universal Tepiman process, deletion of final consonants, or full CV segments, is not attested in Tepiman, let alone Tohono O'odham.} 

O'dam also uniquely innovated the negation particle \emph{cham tu'}. This particle almost certainly developed from the combination of the basic negator \emph{cham} plus the indefinite pronoun \emph{tu'} `something'. While \emph{tu'} seems to most often pronominalize nouns, it also seems to be able to have an irrealis non-specific function with dependent clauses, a property not unique to O'dam \citep{Haspelmath1997}. In \REF{ex:odam-sometheve}, \emph{tu'} is the head of the bracketed subordinate clauses and essentially makes their meaning irrealis and non-specific. This structure mirrors that of standard externally headed relative clauses, shown in \REF{ex:odam-exheadrel}, where the head immediately precedes the subordinator  (\citeauthor{garciasubm}; submitted).
\ea
\label{ex:odam-sometheve}
\begin{xlist}
\item\gll Tu’ na pix ba-ñ-pasaru-’.\\
something \textsc{sub} \textsc{mir} \textsc{compl-1sg.po}-pass-\textsc{irr}\\
\glt ‘Something is going to happen to me.’ (Text\_092010\_MSM\_GGS\_Lavidatepehuana, 27:32)
\item\gll Jiñ-alhii-chu-k dhiʼ tu’ {{\ob}}{na=\o} {pɨx} pasar-ka'{{\cb}} ora mui' chumiñ-kɨʼn na=ñ ba-dependero'.\\
\textsc{1sg.poss}-boy-\textsc{caus-pnct} \textsc{dem.prox} something \textsc{sub=3sg.sbj} \textsc{mir} happen-\textsc{stat} now a.lot money-with \textsc{sub=1sg.sbj} \textsc{compl}-depend\\
\glt ‘Something is going to happen to my son, now with a lot of money, I help him.’
(Text\_102010\_CFC\_GGS\_Lacostumbre, 02:11)
\end{xlist}
\z
\ea
\label{ex:odam-exheadrel}
\gll Gu chi$\sim$chio’ñ 	{{\ob}}na=m ba-nab-dhi-po’{{\cb}}.\\
	\textsc{det} \textsc{pl}$\sim$man	\textsc{sub=3pl.sbj} \textsc{compl}-hunt-\textsc{appl-mov}\\
\glt ‘The men who are going to hunt.’ (Text004/Text\_092010\_TSC\_GGS\_Elxiotahl, 00:31)
\z
Through frequent collocation, \emph{cham} + \emph{tu'} would become a
frozen \textsc{neg} + `indefinite head' construction.  We must caution that
the modern particle \emph{cham tu'} differs in many ways from its plausible
previous life as a negated external relative clause head construction. It
can be used in realis and specific contexts, \REF{ex:odam-chamspec} and currently we
do not know of any semantic constraints that \emph{cham tu'} places on the
negated element that would follow from it being a relative clause head. As
discussed in \sectref{sec:odam-staneg}, \emph{cham} can only precede negated verbs (clausal negation) and must follow all other negated constituents. In addition, it must occur inside of negated dependent clauses. Conversely, \emph{cham tu'} can precede negated verbs and constituents and occurs outside of dependent clauses. Thus, the position of the particle follows from its development from an external relative clause head.
\ea
\label{ex:odam-chamspec}
\gll Añ ubii ya’ ai-ch-dha’ jumai’ \textbf{cham tu’} \uline{ap}.\\
\textsc{1sg.sbj} woman \textsc{dir} arrive-\textsc{caus-appl} another \textsc{neg} \textsc{2sg.sbj}\\
\glt ‘As for me, I am going to bring another woman (for me), not you.’
(Text\_102010\_CFC\_GGS\_Cuandolacuranderaeraniña, 21:17)
\z

It seems that Tepiman languages as a whole, including O'dam, are particularly prone to elaborating and replacing negative particles within the Uto-Aztecan family. It is not clear whether the O'dam and Tepiman forms were taken from common sources (e.g. emphatic negative particles) or whether they were simply innovated separately. However, the key point is that the development of standard negation in O'dam, and Tepiman more widely, is unconnected to existential negation.

Turning to existential predication in Southern Uto-Aztecan, we find that
there is quite a bit of evidence for the (re-)emergence of negative
existentials. In \tabref{tab:odam-suaposnegex} we show the dedicated positive and negative existential predicates in each language in our sample. To our knowledge none of the forms are cognate with each other, and they do not appear to be reflexes of attested Proto Uto-Aztecan cognates.

\begin{table}
\caption{Positive and negative existential predicates in our Southern Uto-Aztecan sample}
\label{tab:odam-suaposnegex}
    \centering
    \begin{tabularx}{.8\textwidth}{lXl}
    \lsptoprule
         Language & PositiveEX & NegativeEX \\
         \midrule
         O'dam & \emph{jai'ch} & \emph{cham (tu')} + \emph{jai'ch} \\
         \tablevspace
         Northern Tepehuan & \emph{oid\textsuperscript{y}ága} & (mai +) tiípu(ka)\\
         \tablevspace
         Pima Bajo & \emph{ai} \textsc{sg} & \emph{im/kova?} + \emph{ai/amig}\\
         & \emph{amig} \textsc{pl} & \\
         \tablevspace
         Cora & \emph{é'en} \textsc{sg} & (\emph{ka} +) \emph{mé'e}\\
         & \emph{pwá'ame} \textsc{pl} & \\
         \tablevspace
         Huichol & \emph{xuawe} & \emph{mawe}\\
         \tablevspace
         Guarijío & \emph{maní} & \emph{ki'té} or\\
         & & \emph{ki=maní}\\
         \lspbottomrule
    \end{tabularx}
\end{table}

We see that the Corachol subgroup (Cora, Huichol) seems to have derived their positive and negative existentials from a common source, or possibly one from the other. However, it is unclear where this source would be or what the origin of the /m/ initial segment is. The possible Proto-Tepiman \emph{*imai} seems an unlikely source because it is unattested in Corachol and truncation of /imai/ > /m/ would be otherwise unattested in Corachol. 

The negative existential in Northern Tepehuan and Guarijío are completely unrelated to their positive counterpart. Most of the languages allow the negative marker to co-occur with either the positive or negative existential, and this is obligatory in O'dam and Pima Bajo. However, Guarijío is the only language that has plausible evidence for evolutionary interaction of standard and existential negation, because the /ki/ segment of \emph{ki'té} could plausibly be from the standard negation clitic. All others do not show any obvious reflex of the standard negation particle in the positive or negative existential forms. Thus, while we do not know the source of the negative, or positive, existential predicates in Southern Uto-Aztecan, only Guarijío seems to have any evolutionary interaction between standard and existential negation.

\section{Conclusion}
This chapter described the strategies that O’dam employs to express existential meaning and their negation. O’dam uses several types of constructions to express positive existential meaning; these include the non-verbal predicate \emph{jai’ch}, locative positional constructions and a copular construction. We also described standard negation, which is accomplished through the use of two particles \emph{cham} and \emph{cham tu’} that are used for both clausal and constituent negation. O’dam is a Type A language because it uses standard negation strategies to negate existential constructions. Clausal negation seems to be preferred for existential negation and we find no attested cases of postverbal constituent negation. The apparent exception to O’dam’s Type A status is the use of \emph{ampɨx} + ‘finish’, which does not appear with any overtly negative elements and seems limited to ‘there is nothing’. Finally, we discuss the place of other Southern Uto-Aztecan languages in the existential negation cycle, most of which appear to be Type B. O’dam and Pima Bajo appear to be unique as Type A languages. Standard negation particles and existential negators seem to be commonly replaced and emergent, especially in Tepiman. Thus, it seems that standard and existential negation in the history of O'dam, and likely Southern Uto-Aztecan, have not played roles in each other's development and evolution.\il{O’dam|)}


\section*{Abbreviations}
\begin{tabularx}{.5\textwidth}{lQ}
    \textsc{adv}& adverbializer\\
    \textsc{aff}& affirmative\\
    \textsc{al}& alienable\\
    \textsc{anim}& animate\\
    \textsc{appl}& applicative\\
    \textsc{as1}& primary assertion\\
    \textsc{as2}& secondary assertion\\
    \textsc{dem.prox}& demonstrative proximal\\
    \textsc{dem.dist}& demonstrative distal\\
    \textsc{dem.med}& demonstrative medial\\
    \textsc{direv}& direct evidential\\ 
    \textsc{det}& determiner\\
    \textsc{dur}& durative\\
    \textsc{caus}& causative\\
    \textsc{cnj}& conjunction\\
    \textsc{compl}& completive\\
\end{tabularx}
\begin{tabularx}{.45\textwidth}{lQ}
    \textsc{cond}& conditional\\
    \textsc{cont}& continuative\\
    \textsc{coord}& coordinator\\
    \textsc{cop}& copula\\
    \textsc{dir}& directional\\
    \textsc{dur}& durative\\
    \textsc{emph}& emphatic\\
    \textsc{evid}& evidential\\
    \textsc{ex}& affirmative existential\\
    \textsc{exps}& expository\\
    \textsc{ext}& extension\\
    \textsc{foc}& focus\\
    \textsc{fut}& future\\
    \textsc{imp}& imperative\\ 
    \textsc{indir}& indirect evidential\\ 
    \textsc{infr}& inferential\\
\end{tabularx}

\begin{tabularx}{.45\textwidth}{lQ}
    \textsc{ins}& instrumental\\
    \textsc{inter}& interrogative\\
    \textsc{ipfv}& imperfective\\
    \textsc{irr}& irrealis\\
    \textsc{iter}& iterative\\
    \textsc{loc}& locative\\
    \textsc{mid}& middle\\
    \textsc{mir}& mirative\\
    \textsc{mov}& movement\\
    \textsc{neg}& negation\\
    \textsc{neg.ex}& negative existential\\
    \textsc{nmlz}& nominalizer\\
    \textsc{nrint}& non-realized intention\\
    \textsc{nsbj}& non-subject\\
    \textsc{pfv}& perfective\\
    \textsc{pnct}& punctual\\
    \textsc{po}& primary object\\
    \textsc{possd}& possessed\\ 
    \textsc{prec}& precision\\ 
    \textsc{prf}& perfect\\ 
\end{tabularx}
\begin{tabularx}{.5\textwidth}{lQ}
    \textsc{prob}& probability\\ 
    \textsc{prog}& progressive\\ 
    \textsc{prt}& particle\\ 
    \textsc{q}& question marker\\ 
    \textsc{quant}& quantifier\\ 
    \textsc{rprt}& reportative\\
    \textsc{rprt.ui}& reportative unknown information\\
    \textsc{ret}& rhetorical\\ 
    \textsc{sbj}& subject\\ 
    \textsc{sens}& sensorial\\ 
    \textsc{seq}& sequential\\
    \textsc{sgs}& singular subject\\
    \textsc{stat}& stative\\ 
    \textsc{sub}& subordinator\\ 
    \textsc{subj}& subjunctive\\ 
    \textsc{sup}& support\\ 
    \textsc{vblz}& verbalizer\\
    \textsc{vis}& visual\\
    \\
    \end{tabularx}


\section*{Acknowledgements}
This work was supported by funding from the following sources 1) The Calota Smith Fellowship and Joel Sherzer Fellowship to the first author and 2) UNAM-PAPIIT-DGAPA IA401619 to the second author. We want to thank the editors of this volume (Ljuba Veselinova and Arja Hamari) for their great comments and
suggestions to improve this paper. Also, thank you to the anonymous reviewers for their extremely
useful comments. We would also like to thank Inocencia Arellano, Martha Arellano, Wendy Gurrola, Elizabeth Soto and Humberto Bautista for their help and insights on the language.

\newpage
\section*{Appendix: Uto-Aztecan family tree}
Uto-Aztecan family tree based on \citeauthor{haugen2019} (forthcoming). Some subgroups are controversial, these are indicated with a (?). 



\medskip

\renewcommand\DTstyle{\rmfamily}
\dirtree{%
.1 Uto-Aztecan.
.2 Northern Uto-Aztecan.
.3 Numic.
.3 Takic.
.4 Serran.
.4 Gab-Cupan.
.3 \emph{Tübatulabal (tub)}.
.3 \emph{Hopi (hop)}.
.2 Southern Uto-Aztecan (?).
.3 Tepiman.
.4 Piman.
.5 Upper Piman\\ \emph{Pima, Tohono O’odham (ood)}.
.5 Lower Piman\\ \emph{Pima Bajo (pia), Névome}.
.4 Tepehuan.
.5 \emph{Northern Tepehuan (ntp)}.
.5 Southern Tepehuan\\ \emph{O’dam (stp), Audam (tla), Central Tepehuan, Tepecano (tep)}.
.3 Taracahitan (?).
.4 Cahitan\\ \emph{Yaqui (yaq), Mayo (mfy)}.
.4 \emph{Ópata-Eudeve (opt)}.
.4 \emph{Tarahumara-Guarijío (var)}.
.3 \emph{Tubar (tbu)}.
.3 Corachol-Aztecan.
.4 Corachol\\ \emph{Cora (crn), Huichol (hch)}.
.4 Aztecan.
.5 \emph{Pochutec (xpo)}.
.5 General Aztecan\\ Nahuatl (many varieties), \emph{Pipil (ppl)}.
}
%
%
% \begin{forest}
%   for tree={
%     grow'=0,
%     child anchor=west,
%     parent anchor=south,
%     anchor=west,
%     calign=first,
%     edge path={
%       \noexpand\path [draw, \forestoption{edge}]
%       (!u.south west) +(7.5pt,0) |- node[fill,inner sep=1.25pt] {} (.child anchor)\forestoption{edge label};
%     },
%     before typesetting nodes={
%       if n=1
%         {insert before={[,phantom]}}
%         {}
%     },
%     fit=band,
%     before computing xy={l=15pt},
%   }
% [Uto-Aztecan
%     [Northern Uto-Aztecan
%         [Numic]
%         [Takic
%             [Serran]
%             [Gab-Cupan]
%         ]
%         [\emph{Tübatulabal} (tub)]
%         [\emph{Hopi} (hop)]
%     ]
%     [Southern Uto-Aztecan (?)
%         [Tepiman]
%           [Piman
%               [{Upper Piman\\ \emph{Pima, Tohono O’odham} (ood)}]
%               [{Lower Piman\\ \emph{Pima Bajo} (pia), \emph{Névome}}]
%           ]
%           [Tepehuan
%               [\emph{Northern Tepehuan} (ntp)]
%               [{Southern Tepehuan \emph{O’dam} (stp), \emph{Audam} (tla), \emph{Central Tepehuan}, \emph{Tepecano} (tep)}]
%           ]
%       [Taracahitan (?)
%           [Cahitan\\ \emph{Yaqui (yaq), Mayo (mfy)}]
%           [\emph{Ópata-Eudeve} (opt)]
%           [\emph{Tarahumara-Guarijío} (var)]
%       ]
%       [\emph{Tubar} (tbu)]
%       [Corachol-Aztecan
%           [{Corachol\\ \emph{Cora (crn), Huichol (hch)}}]
%           [Aztecan
%               [\emph{Pochutec} (xpo)]
%               [{General Aztecan\\ Nahuatl (many varieties), \emph{Pipil} (ppl)}]
%           ]
%       ]
%     ]
% ]
% \end{forest}
%
%

{\sloppy\printbibliography[heading=subbibliography, notkeyword=this]}
\end{document}
