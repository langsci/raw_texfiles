\documentclass[output=paper]{langsci/langscibook}
\title{Intertwining the negative cycles}
\author{Johan van der Auwera\affiliation{University of Antwerp} and
Olga Krasnoukhova\affiliation{University of Antwerp} and
Frens Vossen\affiliation{University of Antwerp}
}

\abstract{In the synchronic and diachronic typology of negation three
so-called ``cycles'' have been prominent: the Jespersen Cycle, the Negative
Existential Cycle and the Quantifier Cycle. This paper refines these
notions, sketches what is cyclical about them and shows how they relate to
one another. For the Jespersen Cycle it is argued that it crucially
involves a negator that is either contaminated by another item or fuses
with it. The Negative Existential Cycles comes in three subtypes, two of
which can be fit into a more general Jespersen Cycle frame. For the
Quantifier Cycle it is argued that the term had better be given a new
definition and we then show how it is similar to a Jespersen Cycle and
feeds into it.

\textbf{Keywords:} Generalized Jespersen Cycle, Negative Existential Cycle, Quantifier Cycle, Positive Existential Cycle
}
%
\maketitle
% vanhan kirkkoslaavin aineisto laitetaan erilliseen luetteloon
\addtocategory{online}{OCS}
\begin{document}

\section{A tale of three Cycles}\label{sec:int-1}

Both the synchronic and the diachronic typology of standard negation, i.e.,
the negation of a main clause affirmative verbal predicate, have been
described and explained in terms of at least two ``cycles'', i.e., hypotheses
about the nature and the development of negative markers. The ``cycle''
hypothesis that has been most prominent is, without any doubt, the
``Jespersen Cycle''. This hypothesis is associated with the Danish Anglicist
and general linguist Otto Jespersen, who drew attention to a ``curious
fluctuation'' \parencite[4]{Jespersen1917} in the renewal of negative markers,
with one negative marker first weakening, then being strengthened,
``generally'' by another word, not itself negative, but which in time becomes
a negator too and suppresses the original negator. The process is
schematized in \REF{ex:int-jespersen}. 
%
\begin{exe}\ex\label{ex:int-jespersen}
          The ``Jespersen Cycle'' \\[1ex]
    %\begin{tabularx}{\textwidth}{@{} Q @{}}
negation is expressed by one negator\\ 
→\\
this one negator is strengthened by another word\\ 
→\\ 
the ``other word'' is interpreted as part of the by now bipartite negator\\
→\\ 
negation is expressed by one negator again, but it is the word that was previously added to the old negator 
    %\end{tabularx}
    \end{exe}
%
This path is indeed a cycle, for the new negator can then also undergo this
process. The term ``Jespersen Cycle'' was introduced by \textcite{Dahl1979}, in
the variant ``Jespersen's Cycle'', with a possessive \textit{'s}. Other
terms are ``negation cycle'' \parencite[e.g.][]{Schwegler1983} or ``negative
cycle'' \parencites(e.g.)(){Gelderen2011}{Mithun2016}. The
phenomenon was extensively studied even before 1917: \textcite{Meillet1912}
used it to illustrate grammaticalization in the very paper in which he
introduced the term ``grammaticalization''. The textbook illustration
features French\il{French}. Specifically, earlier French negated a finite verb with a
preverbal \textit{ne}, modern French has this \textit{ne} in the company of
a postverbal \textit{pas}, the original and still surviving lexical meaning
of which is `step'. With this original meaning the reference was to
something small, which lent itself to a negative polarity use, which was
emphatic. In the context of negation \textit{pas} turned negative polarity
into a negation force and lost the emphatic sense. Now colloquial French
may negate with only \textit{pas}. \REF{ex:int-jespersen-french} is a four stage
representation of what happened in French.
%
\begin{exe}\ex\label{ex:int-jespersen-french}\il{French}
          The ``Jespersen Cycle'' in French\\[1ex]
    \makebox[.75in][l]{\textit{ne}}       `not'      \\
    →                      \\
    \makebox[.75in][l]{\textit{ne \ldots{} pas}}  `not even a step'  \\
    →                                          \\
    \makebox[.75in][l]{\textit{ne \ldots{} pas}}  `not'              \\
    →                                          \\
    \makebox[.75in][l]{\textit{pas}}            `not'              
  \end{exe}
%
The scheme in \REF{ex:int-jespersen-french} is, of course, a
language-specific illustration. In French\il{French}, both negators are syntactic
elements; the first one is preverbal, and the second one is postverbal
(relative to the finite verb) and it results from an emphatic minimizing
use of the word \textit{pas}. These properties are not essential, i.e., the
negators may be affixes, the order with respect to the verb may be
different and the origin of the new negator need not be a word that means
`step' or even a minimizer (the English ancestor to `not' and counterpart
to \textit{pas} was a pronoun meaning `nothing'). Furthermore, the
representations in \REF{ex:int-jespersen} and \REF{ex:int-jespersen-french}
are too simple, even for French\il{French}. Thus \REF{ex:int-jespersen} and
\REF{ex:int-jespersen-french} sketch the process in
terms of four stages. However, the `not at all' stage could be made
explicit and one can add two intermediate stages: a stage in which
\textit{pas} is not obligatory yet and another stage in which \textit{ne}
is no longer necessary. 

The second cycle is the ``Negative Existential Cycle'', so named by the
first linguist to focus on it, viz. \textcite[6]{Croft1991}, and later also
called ``Croft's Cycle'' \parencite[e.g.][73]{Kahrel1996}. The idea is that
a language may develop a special negator for existential clauses like
\REF{ex:int-swans}.
%
\begin{exe}\ex\label{ex:int-swans}
    \textit{There are black swans.}
    \end{exe}
%
The special existential negator may extend its use to standard negation and
ultimately replace the original standard negator. The Cycle is summarized
in \REF{ex:int-nec}.
%
\begin{exe}\ex\label{ex:int-nec}
          The ``Negative Existential Cycle'\\[1ex]
    %\begin{tabularx}{\textwidth}{@{} Q @{}}
one negator is used for both standard and existential negation  \\
→\\
one negator is used for standard negation and another one for existential
negation\\
→\\
one negator is used for both standard and existential negation, but it is
the one that was previously only used for existential negation and so it is
a ``new'' one
  %\end{tabularx}
  \end{exe}
%
There is no textbook illustration and we are not aware of a language in
which the full cycle is attested \parencite[see
also][]{Veselinova2014}. The scheme in
\REF{ex:int-nec-tuvaluan} takes us to Tuvaluan\il{Tuvaluan} (Polynesian),
based on \textcite[1345--1346]{Veselinova2014}; the third stage is
hypothetical.
%
\begin{exe}\ex\label{ex:int-nec-tuvaluan}\il{Tuvaluan}
          The ``Negative Existential Cycle'' in Tuvaluan [tvl]\\[1ex]
    %\begin{tabularx}{\textwidth}{@{} Q @{}}
\textit{see} is used both for standard and existential negation\\ 
→\\ 
\textit{see} is used for standard negation and \textit{seeai} (a fusion of
\textit{see} and an existence marker) for existential negation\\
→ \\
\textit{seeai} is also used for standard negation
%\end{tabularx}
\end{exe}
%
Similar to \REF{ex:int-jespersen}, the representation in \REF{ex:int-nec} is a
simplification, not least because one can add intermediate stages. The
stage between the first and second stage, for example, is the constellation
in which existential negation is not the exclusive terrain of the special
negative existential negator, because it allows constructions with the
standard negator too. 

In both cycles the last stage takes us back to the beginning.%
%
\footnote{The new first stage is not exactly the same as the old one,
though. For French the first single negator stage has \textit{ne} but the
next single negator stage starts with \textit{pas}. From this point of
view, the Gabelentz term ``spiral'' (\citeyear[251]{Gabelentz1891}), used by
\textcite[394]{Meillet1912}, is a better term.} %
%
We are dealing here with one notion of cyclicity. There is a second notion,
a wider one, in which it is sufficient that when the language has reached a
final stage, it can start a new cycle, but not necessarily with the negator
of the last cycle. This is the perspective taken by 
\textcite{Gelderen2011}. It is also the perspective under which yet a third
cycle comes up. This is the ``Quantifier Cycle''
\parencite[e.g.][36]{WillisLucas2013}, first described as the ``Jespersen argument Cycle'' by
\textcite[438]{Ladusaw1993} and subsumed under the Jespersen Cycle by
\textcite{Larrivee2011}. The phenomenon concerns the development of negative
indefinites out of constructions with a negator and a non-negative word,
via stages in which the latter becomes negatively polar and then negative. 
%
\begin{exe}\ex\label{ex:int-qc}
    The ``Quantifier Cycle'\\[1ex]
    %\begin{tabularx}{\textwidth}{@{} Q @{}}
a clausal negator combines with a non-negative word\\ 
→\\ 
the non-negative word which the clausal negator combines with becomes a
negatively polar indefinite \\
→\\
the negatively polar indefinite that the clausal negator combines with
becomes a negative indefinite\\ 
→\\
the negative indefinite occurs without the clausal negator
%\end{tabularx}
\end{exe}
%
A textbook illustration takes us to French\il{French} again. French \textit{personne}
`nobody' ultimately derives from a word meaning `person', which got
restricted to negatively polar contexts. In negative contexts it first
needs the support of the clausal negator and may change into a true
negative indefinite -- a pattern that has come to be known as ``negative
concord''.%
%
\footnote{The term became standard since \textcite{Giannakidou1998},
although we have to go back to Jespersen once more for an \textit{avant
la lettre} occurrence, not this time to
\textcite{Jespersen1917}, but to \textcite[352]{Jespersen1922}.} %
%
In colloquial
French\il{French} \textit{personne} is negative and can occur without \textit{ne}. A
four stage representation is given in \REF{ex:int-qc-french}.
%
\begin{exe}\ex\label{ex:int-qc-french}\il{French}
          The ``Quantifier Cycle'' in French\\[1ex]
    %\begin{tabularx}{\textwidth}{@{} l l @{}} 
  \makebox[1in][l]{\textit{personne}}     'person'\\ 
    →\\
    \makebox[1in][l]{\textit{ne \ldots{} personne}}  `not anybody'\\
    →\\ 
    \makebox[1in][l]{\textit{ne \ldots{} personne}}   `not nobody'\\
    →\\ 
    \makebox[1in][l]{\textit{personne}}               `nobody'
%\end{tabularx}
\end{exe}
%
Like the earlier sketches of cycles, the sketch in \REF{ex:int-qc-french} is
language-specific and too simple. As to the language-specificness, note
that the end stage has a pronoun that is semantically, but not
morphologically, negative. This is not necessary. The English word
\textit{nobody} underwent a variant of the ``Quantifier Cycle'' too but
\textit{nobody} is morphologically negative.%
%
\footnote{The notion of
morphological negativity is tricky, cp. \textcite[130--133]{Haspelmath1997} on
``dunno'' indefinites, i.e indefinites that have a negative component but are
not semantically negative in the way \textit{nobody} is.} %
%
Like the
Jespersen Cycle, the ``Quantifier Cycle'' has led to an abundance of
research. Since the ``Quantifier Cycle'' does not itself create standard
negators, we will not focus on it.  Importantly, the process shown in
\REF{ex:int-qc-french}  is not a cycle in the sense that the fourth stage takes us
back to beginning. But negative indefinites do show a real cycle, although
in the case of \textit{personne}, we have to look at a wider trajectory,
starting from Latin\il{Latin} \textit{nemo} `nobody'
\parencite[cp.][208]{Gianollo2018a}.
%
\begin{exe}\ex\label{ex:int-qc-latin}\il{French}\il{Latin}
A ``Quantifier Cycle'' in Latin and French\\[1ex]
% \makebox[1in][l]{Latin}    \textit{nemo} 'nobody'\\
% \makebox[1in][l]{} → \\
%   \makebox[1in][l]{Latin to French}  \parbox[t]{3in}{\textit{nemo} disappears,
%   perhaps replaced by \textit{nesun} ‘not one’, in turn replaced by a
%   construction with a negator and \textit{personne} 'not a
%   person'}\\[.4\baselineskip]
% \makebox[1in][l]{} →\\
% \makebox[1in][l]{French} \textit{personne} 'nobody'
%
\begin{tabularx}{.8\textwidth}{@{} l Q @{}}
Latin               &   \textit{nemo} 'nobody'\\
                    & → \\
Latin to French     &\textit{nemo} disappears,
  perhaps replaced by \textit{nesun} `not one', in turn replaced by a
  construction with a negator and \textit{personne} `not a
  person'\\
                    & →\\
French              & \textit{personne} 'nobody'
\end{tabularx}
\end{exe}

Here the first and the third stages have a negative pronoun. Curiously,
the term ``Quantifier Cycle'' is not, as far as we know, used for this
wider trajectory. Yet something like \REF{ex:int-qc-latin}, we propose, shows a
better use of this term.

In this paper we aim to improve our understanding of the three Cycles in so
far as they tell us something about the development of standard negation.
In \sectref{sec:int-2} we focus on the Jespersen Cycle, and in
\sectref{sec:int-3} on the Negative Existential Cycle. In
\sectref{sec:int-4} we aim to come to a generalized model of a Jespersen
Cycle. Subsections \sectref{sec:int-4.1} and \sectref{sec:int-4.2} discuss a few cases which illustrate the
interaction between the two Cycles. \sectref{sec:int-4.3} presents a
Positive Existential Cycle, which is another illustration of the
interaction of the Cycles, and \sectref{sec:int-4.4} brings all arguments
together in a model of a generalized Jespersen Cycle. \sectref{sec:int-5}
treats the relation between this Cycle and the ``Quantifier Cycle'', both the
classical version in \REF{ex:int-qc} and \REF{ex:int-qc-french} and its alternative
shown in \REF{ex:int-qc-latin}. \sectref{sec:int-6} is the conclusion.

\section{Refining the notion of the Jespersen Cycle}\label{sec:int-2}

In this section we show that analyses of the Jespersen Cycle encounter a
terminological dilemma due to two definitions, and we suggest a solution.
For most linguists, including ourselves, the most crucial stage in the
simplified model of the Jespersen Cycle has been the third one. In other
words, it is the switch from single to double negation that is crucial.
There are two important implications.

First, a final stage with a return to a single negator is not crucial.
Instead of this return to a single negator the language may get ``stuck''
in the doubling phase and never realize the potential of further
development. It may also enter a fourth stage with three negators. This
is illustrated in \REF{ex:int-brabantic}.
%
\begin{exe}\ex \label{ex:int-brabantic}\il{Dutch}
          Mid 20\textsuperscript{th} c. Brabantic Belgian Dutch [no code]
\langinfo{}{Indo-European}{\citealt[454]{Pauwels1958}}\\
\gll Pas op da ge \textbf{nie} \textbf{en} valt \textbf{nie} \\
fit  on  that  you  \textsc{neg} \textsc{neg}  fall  \textsc{neg} \\
    \glt `Take care that you don't fall!'  
    \end{exe}
%
Cross-linguistically the tripling of negation is rare. In
\textcite[344]{Vossen2016} tripling only shows up in 19 out of 1715 languages
investigated, as against 383 languages with doubling and 418 languages with
a postverbal negator that could be the result of a classic left to right
Jespersen Cycle. However, we don't know how many of these postverbal
negator languages really went through a Jespersen Cycle, nor do we know
that these cycles took the classical left to right direction. In any case,
in this paper we don't pursue tripling \parencite[see][]{DevosAuwera2013}
nor the even rarer quadrupling \parencite[only 3 languages
in][343]{Vossen2016}  or the very special quintupling 
(no languages in \citealt[][]{Vossen2016}, and only one in 
\citealt[42]{AuweraVossen2017}).

Second, it is not sufficient for an element to join the first negator to
fit into the second stage. This second element has to become a negator
too. In Latin\il{Latin} the negator \textit{non} is a fusion of the negator
\textit{ne} with \textit{oenum} `one' and the latter does not itself
become a negator -- this only happens to the univerbation. Thus
\textcite[14--15]{Jespersen1917} assumes that \textit{ne} was replaced by
\textit{non} without a doubling stage. Obviously, there is an
intermediate stage with two elements but the second one does not itself
become negative \REF{ex:int-a-cycle-in-latin}.
%
\begin{exe}\ex\label{ex:int-a-cycle-in-latin}\il{Latin}
          A Cycle in Latin [lat]\\[1ex]
\textit{ne} \hspace{5mm}  →  \hspace{5mm}  \textit{ne oenum}  \hspace{5mm}  →  \hspace{5mm}  \textit{non}
    \end{exe}
%
\textcite[14--15]{Jespersen1917} sets this trajectory apart from the ``curious fluctuation'' named after him later. 
%
\begin{quote}
Sometimes it seems as if the essential thing were only to increase the phonetic bulk of the adverb by the addition of no particular meaning, as when in Latin \textit{non} was preferred to \textit{ne}, \textit{non} being according to the explanation generally accepted compounded of \textit{ne} and \textit{oenum} (= \textit{unum}) `one' (neutr.).
\end{quote}
%
Of course, it is not because \textcite[14--15]{Jespersen1917} didn't see
\REF{ex:int-a-cycle-in-latin} as a manifestation of a Jespersen Cycle, that we, a century
later, are forced to do this too. There are many other things that
Jespersen didn't see and that we now recognize as a type of Jespersen
Cycle. Unknown to \textcite{Jespersen1917} is doubling with a second element
originating from a focus particle `also, even', as with Amharic\il{Amharic}
–\textit{mm}; it is now given a Jespersen Cycle treatment by
\textcite[305--306, 349--350, 2018: 341--343, 388--389]{Sjors2015}
 \parencite[cp. also][69 on the Loyalty Islands languages Drehu
and Nengone]{MoyseFaurieOzanneRivierre1999}.
%
\begin{exe}\ex\label{ex:int-amharic-breakfast}\il{Amharic}
Amharic [amh] \langinfo{}{Afro-Asiatic}{Fridman p.c.}\\
    \gll zare  ḳurs    \textbf{al}-bälla-\textbf{mm} \\
  today  breakfast  \textsc{neg}-eat.\textsc{pst.3m.sg-neg}  \\
    \glt `He didn't eat breakfast today.'
    \end{exe}
%
Jespersen was also not aware of the fact that negator status could accrue
to a subordinator -- as argued for the Arizona Tewa\il{Tewa} former subordinator
\textit{dí} by \textcite{Kroskrity1984} and explicitly integrated into the
Jespersen Cycle by \textcite[83]{Auwera2010}. 
%
\begin{exe}\ex\label{ex:int-tewa-man}\il{Tewa}
Arizona Tewa [tew] \langinfo{}{Kiowa-Tanoan}{\citealt[95]{Kroskrity1984}}
\begin{xlist}
\ex \gll he'i  sen  na-mεn-\textbf{dí} 'o-yohk'ó \\
    that  man  3.\textsc{stat}-go-\textsc{sub}  1.\textsc{stat}-be.asleep \\
    \glt `When that man went, I was asleep.'
\ex \gll sen  k\textsuperscript{w}iyó    \textbf{we}-mán-mun-\textbf{dí}\\
man  woman  \textsc{neg-3>3.act}-see-\textsc{neg}\\
\glt `The man did not see the woman.'
\end{xlist}
    \end{exe}
%
Negator status can also befall on the bareness of the lexical verb that
goes with a Finnish negative, the so-called ``connegative'' form, which in
dialectal Finnish\il{Finnish} \citep[238]{Miestamo2005} -- and dialectal
Estonian\il{Estonian}
\citep[425--426]{Tamm2015} -- can carry negation all by itself (the fourth
stage of a Jespersen Cycle).%
%
    \footnote{The Uralicist's term ``connegative'' may be taken to say that
    this form of the verb is ``not negative in itself''
    \parencites[82]{Miestamo2005}[56]{WagnerNagy2011}. It is indeed not
    morphologically negative, but neither is the French word \textit{pas},
    but like French \textit{pas} it has become strongly associated with
    negation. The association is not complete though, neither in French nor
    in Finnish: there is still a French word \textit{pas} meaning `step'
    and the connegative form is often the same as the second singular
    imperative.  Uralicists have not, to our knowledge, considered a
    connegative construction to illustrate Jespersenian doubling. The fact
    that in Finnish and Estonian dialects the connegative can mark negation
    by itself makes clear that a description in terms of a Jespersen Cycle
    is appropriate.} %
%
This is obviously quite different from the classical French type.%
%
    \footnote{Note that ``the French type'' is not only found in French.
    We find it in Italian dialects and a special case -- with a so-called
    ``partitive'' element or an element meaning `first' -- is found in
    Vanuatu \parencite[72--74]{VossenAuwera2014}.} %

%
Jespersen did not
include in his fluctuation hypothesis the repetition of a clausal negator
either, though he was aware of it \parencite[72--73]{Jespersen1917}.
One of his examples is Swedish\il{Swedish} \REF{ex:int-swedish-notice},
where the doubling is emphatic.
%
\begin{exe}\ex\label{ex:int-swedish-notice}\il{Swedish}
    Swedish [swe] \langinfo{}{Indo-European}{\citealt[72]{Jespersen1917}}  \\
    \gll \textbf{Inte}   märkte  han   mig   \textbf{inte}.  \\
  \textsc{neg}  noticed  he  me  \textsc{neg} \\
    \glt `He didn't notice me.'
    \end{exe}
%
\textcite[72]{Jespersen1917} called this ``resumptive negation''. However, in
the 35 years since \textcite{Dahl1979} it has become accepted practice to
consider the copying of an identical negator to be a part of a Jespersen
Cycle too -- and we follow that practice. Somewhat related to this
resumptive use -- and even called that by
\textcites[359]{Sjors2015}[399]{Sjors2018} for
South Arabian languages -- is the integration of a ``pro-sentence'', i.e., a
construction that corresponds to \textit{No!}.%
%
\footnote{Pro-sentences do
not only serve as holistic denials. As \textcite[111]{Veselinova2013} shows,
\textit{not} in \textit{Are you coming or
not} is also a pro-sentence, i.e., a ``sentence[s] with the same
propositional content as the utterance of the preceding context''
\parencite[89]{BerniniRamat1996}. However, for our purposes -- and for those
of Veselinova, as well as the authors in Hovdhaugen \& Mosel (eds.)
(\citeyear{HovdhaugenMosel1999}), for who pro-sentences are important (see
\sectref{sec:int-4.2}), only the
denial uses matter. \textcite[30]{Schwegler1988} calls the pro-sentence use an
``absolute negator'' use.} %
%
This type was not included in Jespersen's own
account either, but it is now. Example \REF{ex:int-lifunga-see} from the
Bantu\il{Bantu} language Lifunga\il{Lifunga} shows both a sentence-external
pro-sentential and a clause-internal use of a negator.
%
\begin{exe}\ex\label{ex:int-lifunga-see}\il{Lifunga}
Lifunga [bmg]
\langinfo{}{Atlantic-Congo}{\citealt[143]{Djamba1996}, \citealt[233]{DevosAuwera2013}}\\
    \gll \textbf{tɛ}  na-í-mo-wɛn-ɛ   \textbf{tɛ}\\
  no  \textsc{1sg-neg}-1-see-\textsc{prs}   \textsc{neg} \\
    \glt `No, I will not see him.'
    \end{exe}

Given that the term ``Jespersen Cycle'' now covers quite a few phenomena
that \textcite{Jespersen1917} did not associate with a French type cycle
that would later carry his name, we should return to Latin\il{Latin}. Should one
take the non-doubling \textit{ne oenum} trajectory to be part of
a Jespersen Cycle too? \textcites{Schwegler1983}{Schwegler1988} would, even though his term
was not ``Jespersen Cycle'' but ``negation cycle'' \parencite[cp. also]
[180]{Gianollo2018a}. It is interesting to bring in Greek\il{Greek}. The
fate of Classical Greek is similar to that of Latin. The modern Greek
standard negator is \textit{den} and it derives from \textit{ouden},
composed of the Classical Greek standard negator \textit{ou} followed by
a particle \textit{de} `even' and the numeral \textit{hen} `one'. The
change from \textit{ouden} to \textit{den} is apparently a phonetic one
\parencite[303]{Willmott2013} -- just like the development of Latin\il{Latin}
\textit{non} to French\il{French} \textit{ne}. It is the change from \textit{ou} to
\textit{ouden} that is relevant, for it is taken to have happened without
doubling.
%
\begin{exe}\ex\label{ex:int-a-cycle-in-greek}\il{Greek}
A Cycle in Classical Greek [grc]\\[1ex]
    \makebox[1in][l]{\textit{ou}} `not'\\
→\\
\makebox[1in][l]{\textit{ou de hen}}  `not even one'\\
→\\
\makebox[1in][l]{\textit{ou de hen}}  `not'\\
→\\
  \makebox[1in][l]{ouden}    `not'
    \end{exe}
%
Just like for Latin, the question is whether one should call this a
Jespersen Cycle. \textcite{Willmott2013} stresses the differences between the
Greek and French scenarios and decides against a Jespersen Cycle analysis,
though she is aware of the similarities. More or less simultaneously,
\textcite{Chatzopoulou2012}, later also
\textcites{Chatzopoulou2015}{Chatzopoulou2019}, discusses
the same data: her analysis is similar, but she prefers to redefine the
concept of the Jespersen Cycle, and she explicitly does this so as to
include both the Greek and the French scenario. 

Since doubling is not the defining characteristic for a Jespersen Cycle for
\textcites{Chatzopoulou2012}{Chatzopoulou2015}{Chatzopoulou2019} nor,
\textit{mutatis mutandis},
for \textcites{Schwegler1983}{Schwegler1988}, it is important to see what they do consider to
be crucial. For them, the defining features are emphasis, whether through
doubling or fusion, and the later bleaching. This is a perfectly good
definition,%
%
    \footnote{We gloss over the problem of describing the nature of
    emphasis. In the last decade it has been proposed that emphasis has to
    be replaced or explained by notions of discourse presupposition or
    activation.  Such accounts have been particularly prominent for
    resumptive negation, as in Brazilian Portuguese
    \parencite{Schwenter2006} or Palenquero \parencite{Schwegler1991}, but
    they have also been offered for the textbook case of French
    \parencites{MosegaardHansen2009}{Larrivee2010}. We offer three
    considerations. First, in case notions of discourse presupposition or
    activation are to replace emphasis, this is fully compatible with our
    insistence that the term ``Jespersen Cycle'' covers a variety of
    phenomena, a variety more compatible with a plural ``Jespersen Cycles''
    than with a singular \parencite{Auwera2009}. Second, it is no less
    possible that in some cases presupposition and activation will not so
    much replace emphasis but, to borrow Schwenter's term, ``fine-tune''
    it. Third, accounts downplaying emphasis are found more with resumptive
    negation, and this fact is interesting. Resumptive negation is a matter
    of repeating a marker and this could simply serve to make the meaning
    clearer, which is not the same as making a negation emphatic. This
    analysis was offered for resumptive negation in Brabantic Belgian Dutch
    by \textcite[52]{Auwera2009}, with reference to
    \textcite[76]{Pauwels1974}.} %
%
but then they don't include scenarios such as the doubling that appears
through the reinterpretation of subordination, as in Arizona Tewa\il{Tewa},
or non-finiteness (the connegative of dialectal Finnish\il{Finnish} and
Estonian\il{Estonian}). So we are left with a terminological
dilemma. A form-based definition of a Jespersen Cycle requires there to be
doubling, whether it goes with emphasis or not. A meaning-based account
requires emphasis, whether it goes with doubling or not. The embryo of the
dilemma is the fact that Jespersen's textbook example of French fits both
definitions. Arizona Tewa and Finnish as well as Latin and Greek only fit
one definition. The dilemma can be solved in more than one way. One
solution is simply to stick to one of two definitions. A second one is to
drop the term ``Jespersen Cycle'' altogether. After all, we now know more
about the ``curious fluctuation'' than in 1917 and Jespersen delivered
neither the first nor the best early description. \textcite{Meillet1912}, for
one, beat him, and he was not the first either. In a somewhat obscure paper
on Coptic\il{Coptic}, \textcite{Gardiner1904} makes a parallel between \textit{pas} and
Coptic \textit{ı͗wn'} `certainly'. Earlier still, in the book that launches
the term ``sémantique'' \textcite[22]{Breal1897} assures us that ``everybody
knows what happened to the words \textit{pas}, and \textit{point}''' [our
translation]. But then, the term ``Jespersen Cycle'' has been around for
close to 40 years, everyone more or less knows what it is all about.
However, there is an easy way to embrace both the meaning- and the
form-based account: a more general definition that allows both accounts.
What we then require of a Jespersen Cycle is that it deals with the genesis
of a standard negator  from a constellation that involves a standard
negator and another element `α', where α is either another negator (e.g. in
Swedish\il{Swedish}) or a non-negative element (e.g. a minimizer like in
French\il{French} or a
subordinator like in Arizona Tewa\il{Tewa}). This constellation can further develop
in two directions: (i) the negator and α fuse, the new element becomes a
negator and it may replace the original negator, or (ii) if there is no
fusion, then α, which is either negative from the start or has become
negative by contamination%
%
\footnote{The ``contamination'' metaphor goes
back to at least \textcite[221--226]{Breal1897}. It is better than the more
sober ``reinterpretation'' because reinterpretation can happen through a
range of language external or internal factors, while ``contamination''
nicely captures that the original meaning disappears under the influence of
another element in the clause, viz. the negator.} %
%
from the original
negator, could replace this original negator. These developments may be
prompted by emphasis or not. This is what we propose -- and we will come
back to it in \sectref{sec:int-4}, after we have discussed the Negative Existential
Cycle and we have seen whether the new definition could encompass this too. 

\section{Refining the notion of the Negative Existential
Cycle}\label{sec:int-3}

After its introduction in \citet{Croft1991} and a period where not much
happened to it, the Negative Existential Cycle came within the purview of
\textcites{Veselinova2010}{Veselinova2013}{Veselinova2014}{Veselinova2015}{Veselinova2016}. Veselinova made at least three
very important contributions. The first one is an endeavor to check the
hypothesis on a wide range of language families. The second one is her
finding that the Negative Existential Cycle is rarely completed. The third
contribution is her claim that the mere fact that a language uses an
existential strategy for both existential and standard negation does not
itself constitute evidence for the Negative Existential Cycle yet. Thus in
Bulgarian\il{Bulgarian} an invariable \textit{njama} `not.have' is used for both
existential negation and future tense standard negation.
%
\begin{exe}\ex\label{ex:int-bulgarian-cats-movies}\il{Bulgarian}
Bulgarian [bul] \parencites(Indo-European;)()[1333]{Veselinova2014}[204]{Veselinova2010}
\begin{xlist}
    \ex\gll \textbf{Njama}      div-i     kotk-i \\
not.have.\textsc{3sg.prs}  wild-\textsc{pl}   cat-\textsc{pl} \\
    \glt `There aren’t any wild cats.'
    \ex\gll \textbf{Njama}      da  xod-ja    na  kino.\\
not.have.\textsc{3sg.prs}  to  go-1\textsc{sg.prs}  to  movies\\
    \glt `I will not go to the movies.'
    \end{xlist}\end{exe}
%
But the use of \textit{njama} in standard negation is not due to an
extension of the use of the existential negator. In Old Church Slavonic the
positive future also availed itself of 'have' (as one option,
\href{https://lrc.la.utexas.edu/eieol/ocsol/50#grammar_1014}{https://lrc.la.utexas.edu/eieol/ocsol/50\#grammar\_1014})\nocite{OCS}. What we see
therefore is a decrease in the use of `have' for the future and a domain
expansion of `have' in the realm of expressions of existence
(\citealt[203--204]{Veselinova2010}, \cites(but compare )()[1336--1337]{Veselinova2014}
[157]{Veselinova2016}). So on top of the observation that a strategy is used
for both existential and standard negation, we ideally have diachronic
information on whether the construction originated in existential or in
standard negation. This information can be direct (language-internal) or
indirect (from comparing related languages) or even just etymological: a
negative existential that is a fusion of standard negator and an
existential marker and that is used for both existential and standard
negation is bound to have started in the existential domain.

In what follows, we focus on a problem with the third stage of the Negative
Existential Cycle. This is the stage in which the existential negator,
originally restricted to existential negation, has come to be used for
standard negation. Let us illustrate it with Tongan\il{Tongan} 
\parencites(cp.)()[12]{Croft1991}[1342]{Veselinova2014}. 
%
\begin{exe}\ex\label{ex:int-tongan-anything-laugh}\il{Tongan}
          Tongan [ton] \langinfo{}{Austronesian}{\cites[1342]{Veselinova2014}[101, 104]{Broschart1999}}
          \begin{xlist}
    \ex\label{ex:int-tongan-anything}
    \gll 'oku  \textbf{'ikai}  ha  me'a \\
    \textsc{prs}  ???  \textsc{nsp}  thing \\
    \glt `There is not anything'  
    \ex\label{ex:int-tongan-laugh}
    \gll na'e  \textbf{'ikai}  ke  kata  'a  Pita\\
\textsc{pst}  ???  \textsc{sub}  laugh  \textsc{abs}  Pita\\
    \glt `Pita didn't laugh' (`[It] was not that Pita laugh[ed]')
    \end{xlist}\end{exe}
%
We have purposely not yet glossed the occurrence of \textit{'ikai} in
both examples. Croft's gloss for the \REF{ex:int-tongan-anything} type of example is
`NEG.EX', which makes sense, for the sentence could not be more negative
existential. \textcite[101]{Broschart1999}, Veselinova's source linguist,
provides `It is not that there is anything' as the literal translation. For
the example of the \REF{ex:int-tongan-laugh} type Croft's gloss for the negator is
`NEG(EX)'. `NEG(EX)' is to indicate that we are dealing with a ``polysemy
between negative existential meaning and verbal negation''
\parencite[12]{Croft1991}. Since \REF{ex:int-tongan-laugh} illustrates verbal negation
one might argue that \textit{'ikai} permits the `NEG' gloss, i.e., the
gloss for the standard negator, and `NEG' is in fact the gloss that
\textcite[1342]{Veselinova2014}, following \textcite[104]{Broschart1999}, offers
for \REF{ex:int-tongan-laugh}. But her literal translation of this sentence `[It]
was not that Pita laugh[ed]' \parencite[in line again with][104]{Broschart1999}
 is a little confusing then, for it rather asks for a `NEG.EX' gloss.
To solve this problem we suggest that the third stage of the cycle should
be conceived of as the ``existentialization'' of standard negation.
\tabref{tab:int-three-analyses} represents the three analyses, in a three stage format.
We use underlining\todo{shading is not an option, hopefully this
solution is OK} to show that the negators of the third stage have the same
form as the NEG.EX of the second stage.
%
\begin{table}\begin{footnotesize}
\begin{tabularx}{\textwidth}{l l l l l l l}
\lsptoprule
&\multicolumn{2}{l}{\cite{Croft1991}}&\multicolumn{2}{l}{\cite{Veselinova2014}}&\multicolumn{2}{l}{this
paper}\\\midrule
construction & standard & existential & standard & existential & standard &
existential\\\midrule
marker  &NEG        &NEG        &NEG        &NEG        &NEG        &NEG\\
        &→          &→          &→          &→          &→          &→\\
        &NEG        &\underline{NEG.EX}     &NEG        &\underline{NEG.EX}
        &NEG &\underline{NEG.EX}\\
        &→          &→          &→          &→          &→          &→\\
        &\underline{NEG(EX)}    &\underline{NEG.EX}     &\underline{NEG}
        &\underline{NEG}     &\underline{NEG.EX} &\underline{NEG.EX}\\
\lspbottomrule
\end{tabularx}\end{footnotesize}
\caption{Comparison of three analyses (\citealt{Croft1991}, \citealt{Veselinova2014}, this paper)}
\label{tab:int-three-analyses}
\end{table}
%
In our view, the third stage has standard negation using an existential
negator. What has to happen now -- for the cycle to continue -- is that the
existentialized standard negation gets ``de-existentialized''. This is what
we arguably see in Spoken Kannada\il{Kannada}. In this language both types of negation
use \textit{illa,} but while this is a free form for existential negation,
it is a suffix for standard negation.
%
\begin{exe}\ex\label{ex:int-kannada-money-college}\il{Kannada}
           Spoken Kannada [kan] \langinfo{}{Dravidian}{\cites[144]{Veselinova2016}[111, 112]{Sridhar1990}}
    \begin{xlist}
    \ex\label{ex:int-kannada-money}
    \gll khaja:neyalli    haNa    \textbf{illa} \\
    treasury.\textsc{loc}    money  \textsc{neg.ex} \\
    \glt `There is no money in the treasury.'
    \ex\label{ex:int-kannada-college}
    \gll anil  ka:le:jige  ho:gu-vud-\textbf{illa}\\
    Anil  college.\textsc{dat}  go-\textsc{npst.ger-neg}\\
    \glt `Anil doesn't/won't go to college.'
    \end{xlist}\end{exe}
%
Note that we have glossed the free form with 'NEG.EX' and
the suffix with 'NEG', in agreement with Veselinova and
Sridhar and they do not  provide \REF{ex:int-kannada-college} with a literal gloss of
the type `it is not that Anil goes \slash{} will go to college'. At the risk of
overinterpretation of the glosses, we assume that there is nothing
existential about \REF{ex:int-kannada-college} and that it really just means `Anil
doesn't\slash{}won't go to college'. Suffixal \textit{–illa} has thus been
de-existentialized. The free form, however, is still existential. This
de-existentialization in the domain of standard negation is worthy of a
stage of its own. Thus, with application to Kannada\il{Kannada}, a fourth stage of
\REF{ex:int-fourth-stage} has suffixal \textit{–illa} as a standard negator and the
free form \textit{illa} as an existential negator. In a hypothetical fifth
stage existential negation could avail itself of \textit{–illa}, the
standard negator, together with some marker of existence.
%
\begin{exe}\begin{small}\ex\label{ex:int-fourth-stage}\begin{tabularx}{.5\textwidth}[t]{Q Q}
          standard & existential\\\midrule
        NEG                 &NEG            \\
→ & → \\
        NEG             &\underline{NEG.EX}  \\
→ & → \\
        \underline{NEG.EX}   &\underline{NEG.EX}  \\
→ & → \\
        \underline{NEG}   &\underline{NEG.EX}  \\
→ & → \\
        \underline{NEG}   &\underline{NEG}  
\end{tabularx}\end{small}\end{exe}

The claim that there are additional stages is a little tricky. Both Croft
and Veselinova have in fact included transitional stages in their stage
model. These are stages which have both NEG and NEG.EX for either standard
or existential negation, but they may not be equivalent: the choice could
depend on tense or the one option could carry emphasis. These kinds of
intermediate stages have to be accepted in the basically five stage model
of \REF{ex:int-fourth-stage} as well. Also, it does not follow that every standard
negation structure with a lexical verb and something like an auxiliary is a
negative existential structure. Finnish\il{Finnish} is a good example. Example
\REF{ex:int-finnish-kahvi} has a negative auxiliary and the so-called ``connegative'',
but this structure illustrates standard negation. So the negative auxiliary
is not a negative existential, though it might originate in one
\parencite[see][577 for references]{Veselinova2015}, and though it is also
used
for existential negation, it then combines with a `be' verb in the
connegative form.
%
\begin{exe}\ex\label{ex:int-finnish-kahvi}\il{Finnish}
          Finnish [fin] \langinfo{}{Uralic}{\citealt[476]{Vilkuna2015}}\\
    \gll Täällä  \textbf{ei}    \textbf{ole}    yhtään   kahvi-a.  \\
  here  \textsc{neg.3sg}  \textsc{ex.cng}  at.all    coffee-\textsc{part} \\
    \glt `There is no coffee here.'
    \end{exe} 

\section{Towards a generalized Jespersen Cycle}\label{sec:int-4}

In this section we look at the interaction of the Negative Existential and
Jespersen Cycles. First, we discuss to what extent a Negative Existential
Cycle can involve Jespersenian doubling and resumption
(\sectref{sec:int-4.1}). Then we look
at a specific claim about East Futuna (\sectref{sec:int-4.2}) and we pair the Negative
Existential Cycle with a Positive Existential Cycle
(\sectref{sec:int-4.3}). In (\sectref{sec:int-4.4}) we
offer a generalized Jespersen Cycle, even more general than what we ended
up with at the end of \sectref{sec:int-2}.

\subsection{Negative Existential Cycles with doubling}\label{sec:int-4.1}

The proposal for a Negative Existential Cycle came much later than that for
a Jespersen Cycle. It is appropriate therefore to check whether any
manifestation of the former is in fact a manifestation of the latter, under
either the form-based, the meaning-based or the general definition. We will
first discuss the original proposal by \textcite{Croft1991} and then the
detailed studies by
\textcites{Veselinova2010}{Veselinova2013}{Veselinova2014}{Veselinova2015}{Veselinova2016}.

In \citet{Croft1991} there is no explicit mentioning of the Jespersen
Cycle, but the implicit one is very strong and it concerns the French type.
A Negative Existential Cycle, so Croft claims, is that a special
existential negator may be used in combination with a standard negator.
According to him this is one of the two ways in which an existential
negator can enter the domain of standard negation. The other way is
replacing the standard negator partially or completely
\parencite[9--11]{Croft1991}.%
%
    \footnote{Partial and complete replacements are counted separately by
    \textcite{Croft1991}, so in that way he does not have two but three
    pathways of intrusion.} %
%
To judge from the later work by
Veselinova, who only discusses the replacement strategy, the latter would
seem to be the more important type of Negative Existential Cycle, but the
focus here is on the doubling type. 

The reason why, according to \textcite[13--14]{Croft1991}, an existential
negator may combine with a standard negator, is that this combination makes
the utterance emphatic. He illustrates this with two examples. One is from
the Australian language Mara\il{Mara} \parencite[289]{Heath1981}.
%
\begin{exe}\ex\label{ex:int-mara-see}\il{Mara}
          Mara [mec]
          \langinfo{}{Mangarrayi-Maran}{\cites[14]{Croft1991}
          [289]{Heath1981}} 
          \begin{xlist}
    \ex \gll \textbf{ganugu}  wunayi \\
\textsc{neg}    see.him \\
    \glt `He did not see him.'
    \ex\label{ex:int-mara-see-at-all}
    \gll \textbf{ganugu}  wunayi  \textbf{mal'uy}\\
\textsc{neg}    see.him  \textsc{neg.ex/emph}\\
    \glt `He did not see him at all.'
    \end{xlist}\end{exe}
%
Croft then goes on to say that the emphasis may bleach and that this
process ``is the same [\ldots] that has occurred in the evolution of the French
negative \textit{pas}'' \parencite[14]{Croft1991}, with reference to the
pre-Jespersen account of \citet{Meillet1912} as well to
\textcites{Schwegler1983}{Schwegler1988}. This counts as an acknowledgment
that this kind of Negative
Existential Cycle is a subtype of a Jespersen Cycle. More specifically,
with \REF{ex:int-mara-see-at-all} we are in the doubling stage of a Jespersen Cycle.
Interestingly, the Mara\il{Mara} form for the Negative Existential also serves as a
negative pro-sentence, a usage which, as \textcite[127]{Veselinova2013}
has shown, is cross-linguistically rather frequent. So it is not clear
whether the form that doubles is indeed the existential negator as such or
the negative pro-sentence. In the latter case Mara\il{Mara} joins languages like
Lifunga\il{Lifunga}, illustrated in \REF{ex:int-lifunga-see}, and it is again an
illustration of a Jespersen Cycle. 

The second example of Jespersenian doubling comes from the Wintuan language Wintu.
%
\begin{exe}\ex\label{ex:int-wintu-go}\il{Wintu}
Wintu [wit] \langinfo{}{Wintuan}{\cites[10]{Croft1991}[197]{Pitkin1984}}\\
    \gll \textbf{Ɂelew}-be:sken  hara:-wer-\textbf{mina} \\
\textsc{neg.ex}-you.\textsc{ipfv}  go-\textsc{fut}-\textsc{neg} \\
    \glt `You were not supposed to go.'
    \end{exe}
%
For our purposes, there are two interesting things about the Wintu
case.%
%
\footnote{\textcite[10]{Croft1991} points out that
\textcite{Pitkin1984}
has no example of a negative existential use, which is a bit problematic.
Also,  the second negator is itself also a negative existential in origin.
\textcite[10]{Croft1991} argues that it is older than the first one:
\textit{Ɂelew} is a separate word, one that is a finite verb furthermore,
and the second is morphological. This makes sense.}  %
%
First, the presence of the preverbal negator is said to ``reinforce''
\parencite[10]{Croft1991} the
original negator, but ``reinforce'' probably doesn't mean ``make emphatic''.
The translation in Croft and in the source figures non-emphatic negation.
This makes sense in a form-based Jespersen scenario, but no less in a
meaning-based one, for Wintu may illustrate what Mara\il{Mara} does not show: the
bleaching of the emphasis. Second, the source grammarian
\textcite[197]{Pitkin1984} makes clear that the negative existential also
serves as a negative pro-sentence. So, once more, there is a suspicion that
it is latter use that is crucial in this process of Jespersenian doubling.

As mentioned already, \textcites{Veselinova2010}{Veselinova2013}{Veselinova2014}{Veselinova2015}{Veselinova2016} does not
discuss the Mara -- Wintu scenario, and this strongly suggests that it is
relatively rare. Croft does not give any other languages either. We do,
however, find other candidates for a Jespersenian doubling analysis with a
negative existential in the Munda languages Juang\il{Juang}
\parencite[150--151]{Anderson2007} and Korku\il{Korku} \parencites[64--67]{Nagaraja1999}
[279--281]{Zide2008}, the isolate Urarina\il{Urarina} and also in the Takanan
language Tacana\il{Tacana}. For example, in Tacana standard negation with a finite
lexical verb almost always uses two negators.%
%
\footnote{\citeauthor{Guillaume2019} (this volume) signals only a handful of cases in his corpus with the postverbal negator
omitted.}
%
\begin{exe}\ex\label{ex:int-tacana-wind-firewood}\il{Tacana}
Tacana [tna] \langinfo{}{Takanan}{\cites{Guillaume2017}{Guillaume2016a}}
\begin{xlist}
    \ex\gll \textbf{Aimue}    e-juseute-ta=\textbf{mawe}/\textbf{mue}
    beni=ja \\
    \textsc{neg}    \textsc{fut-}fell-\textsc{a}3=\textsc{neg}    wind=\textsc{erg} \\
    \glt `The wind will not fell (the trees).'
    \ex\gll Kwati=mu    \textbf{aimue}    tsu'u\\
firewood=\textsc{cntr}  \textsc{neg.ex}  still\\
\glt `There is no firewood yet.'
    \end{xlist}\end{exe}
%
The postverbal negator (\textit{mawe/mue}) is the oldest one: it is
shorter, bound and phonologically dependent, it has variant forms and
occupies a rigid position in the construction (\citealt{Guillaume2016a},
\citeauthor{Guillaume2019} this volume). And it is also formally similar to
negators in the
other Takanan languages. The preverbal negator is an innovation in
Tacana\il{Tacana}
only (i.e., it is not found in the other Takanan languages). It is
identical to the existential negator and, in our view 
\parencite{AuweraKrasnoukhova2018} the etymology gives us `be.without',
which suggests that the negative existential use predates the standard
negator use. Its presence in standard negation, \textcite{Guillaume2016a}
suggests, was due to emphasis. Interestingly, this form, like in Mara and
Wintu, also serves as a pro-sentence \parencites{Guillaume2016a}{Guillaume2017}. And
even more interesting is the fact that the lexical verb may be non-finite,
in which case there is an optional finite auxiliary, and in this
construction the newer negator is the sole exponent of negation.
%
\begin{exe}\ex\label{ex:int-tacana-jaguar}\il{Tacana}
 Tacana [tna] \langinfo{}{Takanan}{\citealt{Guillaume2017}}  \\
    \gll    Biame    \textbf{aimue}=da  dia  {\op}a-ta-ina{\cp}  \\
    but    \textsc{neg=top}  eat  \textsc{aux.tr-a3-pst.hab}  \\
    \glt `But (the jaguar) would not eat it.'
    \end{exe}
%
We thus have a reasonably standard Jespersen Cycle with arguably
emphasis-driven doubling and even with the new negator forbidding the
company of the old negator, in one type of construction. And, importantly,
the new negator has the form of the existential negator, which is also the
negative pro-sentence.

In the isolate Urarina\il{Urarina}, standard (non-emphatic) negation is encoded by a
single postverbal negator, which has different allomorphs depending on
person, conjugation class and other factors \citep[484]{Olawsky2006}.
However, Urarina has two constructions which are regarded as ``emphatic''
standard negation. One of these constructions involves the negative
existential \textit{nijej} (\textit{ni-ji} `be-\textsc{neg}') before the
lexical verb that is already marked by a negator \REF{ex:int-urarina-wife}. And thus
we have doubling. The negative existential use is shown in
\REF{ex:int-urarina-fish}.%
%
\footnote{\textcite{Olawsky2006} uses two different
transcriptions of the negative existential in order to distinguish the
meanings; specifically, he notes that the distinction between
\textit{nijej} encoding emphasis `not at all' and the negated copula
\textit{ni-ji} encoding negative existence in the transcriptions is ``not
based on phonological differences, but in order to distinguish the two
meanings'' \parencite[555, footnote 65]{Olawsky2006}. Since there is no
difference in phonology, we reproduce the examples using one form
\textit{nijej}.} %
%
A negative reply in Urarina has a different form:
\textit{aji}, composed of an auxiliary \textit{aja} and a negative suffix
\textit{–i} \parencite[400]{Olawsky2006}. 
%
\begin{exe}\ex\label{ex:int-urarina-wife-fish}
Urarina [ura] (isolate; \citealt[554, 556]{Olawsky2006})
    \begin{xlist}
    \ex\label{ex:int-urarina-wife}
    \gll nii hãu̶ \textbf{nijej} beraj-ɲaa najɲ-\textbf{ene}    
    rai      komasaj \\
    that because \textsc{neg.ex} care.for-\textsc{inf}
    be.able-\textsc{neg.3e} \textsc{poss}   wife\\
    \glt `Therefore, his wife could not look after him at all.'
    \ex\label{ex:int-urarina-fish}
    \gll nukue  seti-aka=ne    \textbf{nijei}  ate  taba-j \\
    creek   fish-\textsc{1du=cond}       \textsc{neg.ex}  fish   
    be.big- \textsc{nmlz}\\
    \glt `When we fished in the creek, there were no big fish.'
    \end{xlist}\end{exe}

Veselinova, following leads by \textcite[21]{Croft1991} and
\textcite[38--39]{Schwegler1988}, also discusses the role of the negative
pro-sentence, but not in a scenario of first doubling up a standard negator
and later potentially being the sole exponent of negation, but in a
scenario of more directly replacing the standard negator. One of the
languages brought in to support this is Sino-Russian Pidgin\il{Sino-Russian
Pidgin}
\parencites[1337]{Veselinova2014}[155--156]{Veselinova2016}.%
%
    \footnote{The other one is the Austronesian language Kapingamarangi,
    but we only know the synchrony. For Sino-Russian Pidgin we do have some
    relevant diachrony, viz.  that of Russian.}  %
%
In this language the standard negator is \textit{netu},
which is related to Russian\il{Russian} \textit{net}. Russian \textit{net} is used both
as negative existential and as negative pro-sentence, with the latter use
being more prominent than the negative existential use, according to
\textcite[1337]{Veselinova2014}. The idea is that the greater prominence
of the pro-sentential use of \textit{net} could explain why it is the
related form \textit{netu}, rather than \textit{ne}, that functions as the
standard negator in Sino-Russian.
%
\begin{exe}\ex\label{ex:int-sinorussian-understand}\il{Sino-Russian Pidgin}
Sino-Russian Pidgin [no ISO; glottolog code: kjac1234] (Pidgin;
\cites[1337]{Veselinova2014}[19]{Stern2002})\\
    \gll naša    ego  ponimaj  \textbf{netu} \\
\textsc{1pl}    \textsc{3sg}  understand  \textsc{neg} \\
    \glt `We don't understand him.'
    \end{exe}
%
There are two problems with this hypothesis. First, Russian\il{Russian} has
\textit{netu}, too, in stylistically lower speech, but it is only used as
an existential one, not as a pro-sentential one. It is easier to assume
that Sino-Russian\il{Sino-Russian Pidgin} Pidgin borrowed \textit{netu}. Second, even if we grant
that the Russian input for the Sino-Russian Pidgin standard negator
\textit{netu} is indeed the pro-sentential \textit{net}, it is not clear
that it entered Sino-Russian Pidgin standard negation in its pro-sentential
role instead of just being a prominent exponent of negation in general. It
is interesting to compare Sino-Russian Pidgin with
English\il{English!English Creoles} Creoles. In the
overwhelming majority of the English Creoles the typical and sometimes the
only standard negator is \textit{no} rather than a form related to
\textit{do} combined with \textit{not} \parencite[140--141]{Auwera2017}.
\REF{ex:int-ghanaianpidginenglish-child} is an example from
Ghanaian\il{English!Ghanaian Pidgin English} Pidgin
English, nicely contrasting with Ghanaian English in
\REF{ex:int-ghanaianenglish-demonic}.
%
\begin{exe}
\ex\label{ex:int-ghanaianpidginenglish-child}\il{English!Ghanaian Pidgin
English}
Ghanaian Pidgin English [gpe] (Indo-European;
\cites[140]{Auwera2017}[398]{Huber2012a})\\
    \gll dε  pikin  \textbf{no}  dè  spik \\
    the  child  \textsc{neg}  \textsc{prog}  speak \\
    \glt `The child is not speaking.'
    \ex\label{ex:int-ghanaianenglish-demonic}\il{English!Ghanaian English}
Ghanaian English (\cites[140]{Auwera2017}[385]{Huber2012b})\\
    These demonic things …. I \textit{don't} believe it.
    \end{exe}
%
The Ghanaian Pidgin English speakers use \textit{no}, which has the same
form as pro-sentential \textit{No!} But what is so attractive about
pro-sentential \textit{no} to have it as a standard negator? Is it its
pro-sentential semantics or is it just its saliency and -- no doubt --
frequency as an exponent of negation? We propose the second answer.

\subsection{Interaction of the two Cycles in East
Futuna?}\label{sec:int-4.2}

The negation in Polynesian East\il{East Futuna} Futuna has given rise to a
claim on the interaction of the Negative Existential and Jespersen Cycles.
The original claim is explicit in \textcite[18]{Mosel1999}, it is implicit in
\textcite[122]{MoyseFaurie1999}, and the basic idea is endorsed by
\textcite[1359--1364]{Veselinova2014}. In what follows we start from
\textcite{Veselinova2014}.

In East\il{East Futuna} Futuna an existential negator \textit{le'ai} is made up of a
standard negator \textit{le} and an existential element
(\textit{i})\textit{ai.} There is also a reduced form \textit{le'e}.
\textit{le'ai} and another reduced form, \textit{e'ai}, function as
pro-sentences and \textit{le'e} has intruded the domain of standard
negation,%
%
    \footnote{\textcite[1364]{Veselinova2014} describes the intrusion
    only for forms with -\textit{se}, but the analysis also contains
    example \REF{ex:int-futuna-Stefano}, which is a standard negation use
    without -\textit{se.}} %
%
which qualifies the trajectory as an instance of the
Negative Existential Cycle. 
%
\begin{exe}\ex\label{ex:int-futuna-rice-Stefano}\il{East Futuna}
East Futuna [fud] \langinfo{}{Austronesian}{\cites[1362,
1361]{Veselinova2014}[117]{MoyseFaurie1999}[98]{MoyseFaurie1997}}
    \begin{xlist}
    \ex\label{ex:int-futuna-rice}
    \gll \textbf{e'a}   e   \textbf{le'e}     se   lāisi \\
    no  \textsc{tam}  \textsc{neg.ex}  \textsc{indf}  rice \\
    \glt `No, there is no rice.'
    \ex\label{ex:int-futuna-Stefano}
    \gll e \textbf{le'e} 'au a Setefano ki le fai o le ga'oi\\
      \textsc{tam}  \textsc{neg.ex}  come  \textsc{abs}  Stefano
      \textsc{obl} \textsc{det}   make    \textsc{poss}
      \textsc{det} work \\
    \glt `Stefano is still not coming to do the work.'
    \end{xlist}\end{exe}
%
What is special about East Futuna is that there are also the more complex
forms \textit{le'aise} and \textit{le'ese,} which function in the
existential domain and also intrude into the verbal domain. 
%
\begin{exe}\ex\label{ex:int-futuna-dance}\il{East Futuna}
 East Futuna [fud]   \langinfo{}{Austronesian}{\cites[126, 122]{MoyseFaurie1999}[1361]{Veselinova2014}}
    \begin{xlist}
    \ex\label{ex:int-futuna-dance-special}
    \gll ko  le  mako    ko  le  tapaki  e  \textbf{le'aise}  ko   se
    mako  tefua  ma  Futuna\\
    \textsc{pr}  \textsc{def}  dance    \textsc{pr}  \textsc{def}   tapaki
    \textsc{nsp}  \textsc{neg.ex}  \textsc{pr}    \textsc{ind} dance
    alone     for      Futuna \\
    \glt `The tapaki dance is not a special dance for Futuna.' 
    \ex\label{ex:int-futuna-dance-yesterday}
    \gll  na  \textbf{le'aise}  kau  ano  o  mako  i  nānafi\\
    \textsc{pst}  \textsc{neg.ex}  \textsc{1sg}  go  \textsc{comp}  dance
    \textsc{obl}  yesterday\\
    \glt `No, I didn't go dancing yesterday.'\footnotemark%
    %
    \footnotetext{The English translation has a pro-sentential \textit{No},
    but the East Futuna original does not. The \textit{No} must be meant to
    show that a negation with \textit{le'aise} is stronger
    \parencite[122]{MoyseFaurie1999} than one with \textit{le'ese.}}
    \end{xlist}\end{exe}
%
The element \textit{se}, which is added to the simple negators, is an
indefinite singular article \parencite[122]{MoyseFaurie1999}.%
%
    \footnote{It is not clear whether the article is indefinite or
    non-specific.  \textcite[18]{Mosel1999} and
    \textcite[1363]{Veselinova2014} call it ``non-specific''.
    \textcite[45]{MoyseFaurie1997} calls it ``non-specific'' too, but later
    in the grammar it is called ``indefinite''
    \parencite[88]{MoyseFaurie1997}.} %
%
But then there is also reduction, for
standard and existential negation allow the complex forms \textit{le'aise}
and \textit{le'ese} to reduce to \textit{se.} 
%
\begin{exe}\ex\label{ex:int-futuna-books-stone}\il{East Futuna}
 East Futuna [fud] \langinfo{}{Austronesian}{\cites[1360--1362]{Veselinova2014}[119, 122]{MoyseFaurie1999}}
    \begin{xlist}
    \ex\label{ex:int-futuna-books}
    \gll \ldots{} e \textbf{se} na'a ai se tosi \ldots{} \\
{} \textsc{genr} \textsc{neg.ex} be.there \textsc{anaph} \textsc{indf}
book\\
    \glt `\ldots{} there are no books \ldots{}' 
    \ex\label{ex:int-futuna-stone}
    \gll e  \textbf{se}  tio  a  tātou  ki  ke  fatu\\
    \textsc{genr}  \textsc{neg.ex}  see  \textsc{abs}  \textsc{1pl.incl}
    \textsc{obl}  \textsc{def}  stone\\
    \glt `We do not see the stone.'
    \end{xlist}\end{exe}
%
For Moyse-Faurie and Mosel, the fact that an erstwhile indefinite article
now functions as a negator shows that we are dealing with a Jespersen
Cycle. For Veselinova (p.c.) there is a Jespersen cycle because the
\textit{le'aise} and \textit{le'ese} are taken to carry emphasis, which
then got lost together with phonetic substance. But these claims are not
obvious. Much depends on what is meant with the notion of Jespersen Cycle
and this has to be made explicit. As argued in \sectref{sec:int-2}, most
linguists take a form-based approach of the Jespersen Cycle and require
doubling but in East\il{East Futuna} Futuna there is no doubling. The East
Futuna facts are thus similar to the Greek\il{Greek} ones. In Greek a
complex form \textit{ouden} lost the negative
morpheme and the emphasis, and it is the remains of a focus particle and a
numeral that now function as a negator. In East Futuna the complex forms
\textit{le'aise} and \textit{lé'se} lost the negative morpheme and the
emphasis, and it is the remains of an indefinite article that now function
as a negator. For East Futuna \textit{se} to count as the result of a
Jespersen Cycle one can thus use the semantically-based account, the one
that allows both doubling and fusion but requires an emphatic stage, or the
more general account, one that requires neither doubling nor emphasis. 

Of these two accounts, the general one seems better. The argument for the
extended notion has so far been, for both Schwegler\ia{Schwegler, Armin}
and Chatzopoulou\ia{Chatzopoulou, Katerina}, that
the second part of the fusion had an emphatic use. This is very clear in
Greek\il{Greek} as well as in Latin\il{Latin}. It is less clear in
East\il{East Futuna} Futuna. The
\textit{-se} part is an indefinite or non-specific article. The latter is
obligatory for noun phrases in the scope of negation and it is therefore ``a
frequent collocate of the existential negator''
\parencite[1348]{Veselinova2014}. In the fusion, \textit{se} then
``reinforced'' the original negator -- ``reinforce'' is the term in
\textcite[122]{MoyseFaurie1999} -- but it is not clear that it is meant in a
semantic sense. According to \textcite[18]{Mosel1999}, followed by
\textcite[1363]{Veselinova2014}, the reinforcement would indeed be
semantic: the reinforcement is to yield emphasis. But note that it is an
indefinite article that fuses, and not, for example, the numeral and
pronoun \textit{tasi} `one' \parencites[27,
35]{MoyseFaurie1997}[121]{MoyseFaurie1999}. A similar fusion is reported
for Cèmuhî\il{Cèmuhî} and Paicî\il{Paicî} \parencite[63]{MoyseFaurieOzanneRivierre1999} as well
as for Hawai'an\il{Hawai'an} \parencite[1348]{Veselinova2014}, each time with an
indefinite article. For
Hawai'ian the fusion does not appeal to emphasis: ``consequently,
\textit{a'ole} [the standard negator] must have become fused with
\textit{he} [the indefinite article] as a result of frequent
collocation'' \parencite[1348]{Veselinova2014}. In short, for the East Futuna
development of the \textit{se} negator to count as a Jespersen it can't be
the one embraced by Schwegler\ia{Schwegler, Armin} and
Chatzopoulou\ia{Chatzopoulou, Katerina}. The story of the
\textit{se} negator does, however, fit the general definition argued for in
\sectref{sec:int-2}: the development of a negator is a Jespersen Cycle, if it results
from the interaction of two elements, at least one of which is a negator.

\subsection{A Positive Existential Cycle?}\label{sec:int-4.3}

Before we clarify the general concept of a Jespersen Cycle more, it is
useful to point out that there is more to the interaction of existence and
negation than what has been sketched in the above. First of all, in the
Negative Existential Cycle proper, the one without doubling, we have so far
seen a negator fusing with something else, typically a positive
existential. A fusion of a negator and a positive existential is not,
however, the only strategy to make negative existentials, and it does not
seem to be the most frequent one. In a worldwide overview
\textcite[137]{Veselinova2013} points out that languages may recruit
negative existentials directly from the lexicon, more particularly from
words with a negative content, such as `absent' or `lack'.%
%
\footnote{The
development of standard negator out of a privative construction
(`without'), argued for Arawak by \textcite[285--288]{Michael2014}, could be
seen as a subtype. There could furthermore be a third origin, no doubt
rare, viz. a word of which the meaning was originally positive but which
got contaminated by a negator that later disappeared -- the typical
Jespersen scenario. At least in Kulina (Arawan; [cul]) the negative
pro-sentence, which derives from a negative existential, only utilizes a
word that originally meant `show' (\citealt[236]{Dienst2014}; p.c) and
which turned negative under the influence of a negator
\parencite{KrasnoukhovaAuweraXXXX}.} %
%
For the 42 languages for which she reports the
origin, 25 have this origin vs. 17 that involve fusion. We come back to
direct recruitment in \sectref{sec:int-4.4}.

Second, we have seen fusion in Latin\il{Latin} and Greek\il{Greek} Jespersen Cycles. These
Cycles are a little different from the French\il{French} one, in that the element that
combines with negation does not itself turn into a standard negator. It is
the fusion that turns into a standard negator. This begs the question of
whether there could be a cycle in which the positive existential and the
negator do not fuse, but in which the latter changes the meaning of the
former. What we are after is a constellation in which a negator turns an
existential marker into a negator, a new one, with the possibility of
ousting the old one. This is precisely what 
\textcite{AuweraVossen2017} have argued for in their study of negative
doubling in the Kiranti\il{Kiranti} languages.

In most of the Kiranti languages there is a preverbal negator with a solid
Tibeto-Burman ancestry, viz. \textit{ma}. In the eastern Kiranti languages
there is often a postverbal negator with the form \textit{ni} or a similar
form. It usually co-occurs with the preverbal \textit{ma} and it has no
clear negative etymology.
%
\begin{exe}\ex\label{ex:int-dumi-mother}\il{Dumi}
Dumi [dus] \langinfo{}{Tibeto-Burman}{\citealt[288]{Driem1993}}\\
    \gll i̶-mu-ʔa  tida:m-tida:m-mil  ryekbo
    \textbf{mə}-ti̶̶l-ni-\textbf{nə} \\
      their-mother-\textsc{erg}  child-child-\textsc{pl}  three 
    \textsc{neg.pst}-raise-\textsc{3pl-neg} \\
    \glt `Their mother did not raise the three little ones.'
    \end{exe}
%
Forms like \textit{ni}, however, do show up in Tibeto-Burman outside of
Eastern Kiranti as various sorts of `be' verbs \parencite{Lowes2006}, as in
Meithei\il{Meithei} \parencite[249--250, 297]{Chelliah1997}, with an ascriptive use in
\REF{ex:int-meithei-shirt} and an existential one in
\REF{ex:int-meithei-cloth}.

\begin{exe}\ex Meithei [mni]
\langinfo{}{Tibeto-Burman}{\citealt[297]{Chelliah1997}}
\begin{xlist}
    \ex\label{ex:int-meithei-shirt}
    \gll phurit-tu  ə-ŋəw-pə-\textbf{ni} \\
    shirt-\textsc{dist}  \textsc{att}-white-\textsc{nmlz-cop}\\
    \glt `That shirt is the white one.'
    \ex\label{ex:int-meithei-cloth}
    \gll əy-nə    phi  ə-du    ləŋ-thok-ləbə-\textbf{ni}\\
I-\textsc{cntr}  cloth  \textsc{att-dist}  throw-out-having-\textsc{cop}\\
    \glt `(It is that) I have thrown out that cloth.'
    \end{xlist}\end{exe}
%
In \textcite{AuweraVossen2017} it is argued that the \textit{ni} was
gradually reinterpreted as a negator. The semantics motivating the
reinterpretation is that the negative proposition was followed by an
emphatic \textit{so it is} phrase. This lost the emphasis
and got contaminated with negative meaning, first doubling up the earlier
negator with a potential of doing the negative work on its own. Given that
it is a positive `be' verb that will become a negator, one could call it as
``Positive Existential Cycle''.%
%
\footnote{The term is a bit misleading. The Positive Existential cycle is
still negative in the sense that it produces a new negator. The term
identifies the source as a positive existential, just like the term
``Negative Existential Cycle'' identifies the source as a negative
existential.} %
%
And given that it involves a progression from single to double and back to
single negation, it is no less of a Jespersen Cycle.

The case for a Positive Existential Jespersen cycle does not only rest on
the analysis of Kiranti\il{Kiranti} \textit{ni}. Within Kiranti itself there is more
evidence, the clearest case being a negative past verbal suffix
\textit{yuk/yukt} \parencite[163]{Doornenbal2009}, which co-occurs with a
negative prefix and which derives from a copula \parencite[276]{Doornenbal2009}
and still is one \parencite[119]{Doornenbal2009}. Outside of Kiranti candidates
for a Positive Existential Jespersen Cycle are the Oceanic language Lewo
spoken in Vanuatu \parencites[425--426]{Early1994a}[79--80]{Early1994b} and the
languages of the Awju group \parencite[127--140]{Wester2014} as well as Kaugel
\parencite[152--153]{Head1976}, spoken in New Guinea.

\subsection{A generalized Jespersen Cycle}\label{sec:int-4.4}

We are now ready to return to the most general conception of Jespersen
Cycle. The idea is that a standard negator may find itself co-occurring
with something `α' and then either fuse with it or contaminate it with
negativeness. If α is itself a negator, the same or another one, we get
doubling. In case α is not a negator, there are two alternatives with
respect to trajectories leading to a standard negator. Either the standard
negator turns α into another negator (i.e., the standard negator
contaminates α with negativeness) and we get doubling, or there is fusion.
The first trajectory, the doubling-after-reinterpretation, is the more
restricted form-based Jespersen Cycle. There may be emphasis and bleaching
(as in French\il{French}) or not (Arizona Tewa\il{Tewa}). In principle there is nothing
preventing the new standard negators to fuse and the result may then be a
third negator. We do not know, however, of any such cases and we use dotted
lines in the representation in \REF{ex:int-scenarios}.%
%
\footnote{Fusion of standard
negators is attested (\citealt[18]{Vossen2016} on the Austronesian
languages Lewo and Nese; \citealt{DevosTshibanda2010} on the Bantu language
Kanincin), but only in cases of tripling and quadrupling.} %
%
The second trajectory, the one involving fusion of the negator and α, has
two outcomes, depending on the nature of α. If α is an existential verb, we
get a (subtype of the) Negative Existential Cycle. If α is a minimizer --
the Latin\il{Latin} and Greek\il{Greek} case -- we get the more restricted
meaning-based Jespersen Cycle. The scenarios are represented in
\REF{ex:int-scenarios}.
%
\begin{figure}
\begin{exe}\ex\label{ex:int-scenarios}
\begin{small}
\begin{tikzpicture}[
baseline=(current bounding box.north),
every node/.style={
    rectangle,
    semithick,
    fill=white,
    align=center,
    },
solid/.append style={
    draw,
    },
theoretical/.append style={
    draw,
    densely dotted, % loosely/densely dotted
    },
blank/.append style={
    },
solidline/.style={thin, ->},
theoreticalline/.style={thin, densely dotted, ->},
node distance=5mm,
]

% vas. yläkulma:
\node (neg1_neg2) [blank] {\textsc{neg1} \textsc{neg2}};
% ylh. keskellä:
\node (neg1-neg2) [blank, right=1.2in of neg1_neg2.east]
    {\textsc{neg1.neg2}};
\draw [theoreticalline] (neg1_neg2) -- 
    node (fusion neg1_neg2) [theoretical, midway, above=-2mm]
    {fusion of \textsc{neg1} \\ and \textsc{neg2}} 
    (neg1-neg2);
% oik. yläkulma:
\node (neg3) [blank, right=1.2in of neg1-neg2.east] {\textsc{neg3}};
\draw [theoreticalline] (neg1-neg2) -- 
    node (reinterpret-neg3) [theoretical, midway]
    {reinterpretation \\ of \textsc{neg1.neg2} \\ as \textsc{neg3}} 
    (neg3);
% vas. alakulma:
\node (neg1_alpha) [blank, below=1.2in of neg1_neg2.south]
    {\textsc{neg1} $\alpha$};
\draw [solidline] (neg1_alpha) --
    node (reninterpret-alpha) [solid, midway]
    {reinterpretation of \\ $\alpha$ as \textsc{neg2}}
    (neg1_neg2);
% alh. keskellä:
\node (neg1alpha) [blank, right=1.3in of neg1_alpha.east]
    {\textsc{neg1}.$\alpha$};
\draw [solidline] (neg1_alpha) --
    node (fusion of neg1 and alpha) [solid, midway]
    {fusion of \textsc{neg1} \\ and $\alpha$}
    (neg1alpha);
% oik. alakulma:
\node (neg2) [blank, below=.9in of neg3.south] {\textsc{neg2}};
\draw [solidline] (neg1alpha) --
    node (reinternpret_neg1alpha) [solid, pos=.6, below=0mm]
    {reinterpretation of \\ \textsc{neg1}.$\alpha$ as \textsc{neg2}}
    (neg2);
\draw [solidline] (neg1_neg2) --
    node (elimination) [solid, pos=.55]
    {elimination of \\ \textsc{neg1}}
    (neg2);

\end{tikzpicture}\end{small}\end{exe}\end{figure}
        
Note that the figure in \REF{ex:int-scenarios} includes the end stages with one
new negator, but we do not require a language to have reached it for us to
claim that the language is involved in a Jespersen Cycle: the language may
get stuck in an intermediate stage or the end stage may show tripling. In
that sense \REF{ex:int-scenarios} does not say enough. In another sense, it may
say too much, for not every type of α has been attested with both a
reinterpretation and a fusion scenario. When α is an existential marker, we
do have both scenarios, i.e., a Positive Existential Cycle for
reinterpretation and a Negative Existential Cycle for fusion. The two
scenarios are schematized in \REF{ex:int-existential-scenarios}.

\begin{figure}
\begin{exe}\ex\label{ex:int-existential-scenarios}
\begin{small}
\begin{tikzpicture}[
baseline=(current bounding box.north),
every node/.style={
    rectangle,
    semithick,
    fill=white,
    align=center,
    },
solid/.append style={
    draw,
    },
blank/.append style={
    },
solidline/.style={thin, ->},
node distance=2mm,
]

\node (positive) [solid,font=\bfseries] 
    {\makebox[2.2in][c]{Positive Existential Cycle}};
\node (negative) [solid,font=\bfseries, below=1.8in of positive.south] 
    {\makebox[2.2in][c]{Negative Existential Cycle}};
\node (neg1_neg2) [blank, below=of positive.south]
    {\textsc{neg1 neg2}};
\node (neg1ex) [blank, above=of negative.north]
    {\textsc{neg1.ex}};
\node (centre) at ($ (positive)!.5!(negative) $) {};
\node (neg1_ex) [blank, left=1.5in of centre] {\textsc{neg1 ex}};
\node (neg2) [blank, right=1.5in of centre] {\textsc{neg2}};
\draw [solidline] (neg1_ex) -- 
    node [solid, midway]
    {reinterpretation \\ of \textsc{ex} as \textsc{neg2}}
    (neg1_neg2);
\draw [solidline] (neg1_ex) -- 
    node [solid, midway]
    {univerbation of \\ \textsc{neg1} and \textsc{ex}}
    (neg1ex);
\draw [solidline] (neg1_neg2) --
   node [solid, midway]
   {elimination of \\ \textsc{neg1}}
   (neg2);
\draw [solidline] (neg1ex) --
   node [solid, midway]
   {reinterpretation \\ of \textsc{neg1.ex} as \\ \textsc{neg2}}
   (neg2);

\end{tikzpicture}\end{small}\end{exe}\end{figure}

For most α's, however, only the reinterpretation scenario has been
attested. Thus in Arizona Tewa\il{Tewa} the subordinator \textit{dí} turned into a
negator through the influence of the negator \textit{we}, but we haven't
seen a language in which an original negator like \textit{we} is adjacent
to a subordinator like \textit{dí} and delivers a new negator
\textit{wedí}. So in this sense \REF{ex:int-scenarios} says too much. But in
another sense, \REF{ex:int-scenarios} -- or
\REF{ex:int-existential-scenarios} for that matter -- does
not say enough. For one thing, neither \REF{ex:int-scenarios} nor
\REF{ex:int-existential-scenarios}
show that a language may have negator doubling followed by tripling (and
even quadrupling and quintupling); these complications were barred from the
paper already in \sectref{sec:int-1}. For another thing, we do not expand
the simple 3 stage model of a classic French style Jespersen Cycle into a
model with more stages, nor do we include the five stages of the Negative
Existential Cycle, represented in \REF{ex:int-fourth-stage} in the above.
However, we need to come back to the Mara\il{Mara}, Wintu\il{Wintu},
Tacana\il{Tacana} and East\il{East Futuna} Futuna cases. They are also not
provided for in \REF{ex:int-scenarios} or
\REF{ex:int-existential-scenarios} yet. Like
in these simpler scenarios, Mara, Wintu, Urarina, Tacana and East Futuna
show doubling and fusion. For Mara, Wintu, Urarina and Tacana the negative
existential combines with a standard negator, it may become a standard
negator too with a further potential to oust the old standard negator. For
East Futuna, the negative existential combines with something else, viz. an
indefinite article. They fuse and combine to form a new negative
existential and later a new standard negator. In \REF{ex:int-nec-paths} the middle
lines show the simple Negative Existential Cycle, the ones on the top
represent Mara, Wintu, Urarina and Tacana and the ones below represent East
Futuna.

\begin{figure}
\begin{exe}\ex\label{ex:int-nec-paths}
\begin{footnotesize}
\begin{tikzpicture}[
baseline=(current bounding box.north),
every path/.style={thin, },
vertical/.append style={
    dashed,
    },
dashedline/.append style={
    densely dashed,
    ->,
    },
solidline/.append style={
    ->,
    },
every node/.style={
    rectangle,
    semithick,
    fill=white,
    align=center,
    },
solidbox/.append style={
    draw,
    },
blankbox/.append style={
    font=\scshape,
    },
node distance=2mm,
text depth=1ex,
]

\node (start) [blankbox] {[\textsc{neg1.ex}]1};
\node (end-neg2) [blankbox, right=3.5in of start.east] {\textsc{neg2}};
\draw [solidline] (start) -- (end-neg2);
% yläkerta:
\node (upperpath-1) [blankbox, above right=of start.north east] 
    {[neg1.ex]1 neg2};
\draw [solidline] (start) -- (upperpath-1);
\node (upperpath-2) [blankbox, right=.75in of upperpath-1.east]
    {neg3 neg2};
\draw [dashedline, name path=upperpathline1] (upperpath-1) -- (upperpath-2);
\coordinate (upperpathsect1) at ($ (upperpath-1) ! .5 ! (upperpath-2) $)
    {};
\node (upperpath-reinterpret) [solidbox, above=6mm of upperpathsect1]
    {reinterpretation of \\ {[\textsc{neg1.ex}] as \textsc{neg3}}};
\draw [vertical] (upperpath-reinterpret) -- (upperpathsect1);
%
\node (upperpath-end) [blankbox, right=.75in of upperpath-2.east]
    {neg3};
\draw [dashedline, name path=upperpathline2] 
    (upperpath-2) -- (upperpath-end);
\coordinate (upperpathsect2) at ($ (upperpath-2) ! .5 ! (upperpath-end) $)
    {};
\node (upperpath-elimination) [solidbox, above=6mm of upperpathsect2]
    {elimination of \\ {\textsc{neg2}}};
\draw [vertical] (upperpath-elimination) -- (upperpathsect2);
% alakerta:
\node (lowerpath-1) [blankbox, below right=of start.south east]
    {[neg1.ex]1 indf};
\draw [solidline] (start) -- (lowerpath-1);
\node (lowerpath-2) [blankbox, right=5mm of lowerpath-1.east]
    {[neg1.ex]1.indf};
\draw [solidline] (lowerpath-1) -- (lowerpath-2);
\coordinate (lowerpathsect1) at ($ (lowerpath-1) !.5! (lowerpath-2) $) {};
\node (lowerpath-fusion) [solidbox, below left=6mm and -10mm of lowerpathsect1]
    {fusion of {[\textsc{neg1.ex}]}1 \\ and \textsc{indf}};
\draw [vertical] (lowerpathsect1) -- (lowerpath-fusion);
\node (lowerpath-3) [blankbox, right=5mm of lowerpath-2.east]
    {[neg1.ex]2};
\draw [solidline] (lowerpath-2) -- (lowerpath-3);
\coordinate (lowerpathsect2) at ($ (lowerpath-2) !.5! (lowerpath-3) $) {};
\node (lowerpath-reinterpret1) [solidbox, below=6mm of lowerpathsect2]
    {reinterpretation of \\ {[\textsc{neg1.ex}]1.\textsc{indf}} \\ as
    [\textsc{neg1.ex}]2};
\draw [vertical] (lowerpathsect2) -- (lowerpath-reinterpret1);
\draw [solidline] (lowerpath-3.east) -- (end-neg2.south west);
\coordinate (lowerpathsect3) at ($ (lowerpath-3.east) !.5! (end-neg2.south west)
    $) {};
\node (lowerpath-reinterpret2) [solidbox, below right=6mm and -2mm of
lowerpath-3.east]
    {reinterpretation of \\ {[\textsc{neg1.ex}]2 as} \\ \textsc{neg2}};
\draw [vertical] (lowerpathsect3) -- (lowerpath-reinterpret2);

\end{tikzpicture}\end{footnotesize}\end{exe}\end{figure}

Finally, these schemas do not exhaust the paths that languages dispose of
to make negators. First, a negator may not only arise through the influence
of a negator that is already in place, whether through contamination or
fusion. It may be borrowed or calqued from other languages – and to the
extent that what is borrowed or calqued is negative doubling,
distinguishing this from a Jespersen scenario can be difficult 
\parencite[cp.][]{AuweraVossen2015}. Second, we have also assumed that the
negator
that will fuse or contaminate and thus yield a new negator is a standard
negator. In the cases discussed in the literature this seems to be the
case, but what could prevent a standard negator from arising from, say, a
contamination of a minimizer through a non-standard negator like a
derivational negator. Third, a negator may also be recruited directly from
the lexicon \parencite[cp.][74]{Auwera2010}. The source will be a
word with negative content and the outcome could in principle be a standard
negator, although we cannot give a good example
\parencite[cp.][75, 90--91]{Auwera2010}: the literature
\parencite[e.g.][292--339]{Gelderen2011} only shows cases which yield special
negators, such as prohibitives or negators of relative, focus or cleft
constructions \citep[917]{Givon1973} or, to wit, existential negators. As
mentioned already, in \citegen[137]{Veselinova2013} cross-linguistic survey of
the origin of existential negators, the majority of languages for which the
origin is known derive from a negative word and not from a fusion of the
standard negator and some existential marker. For these negative
existentials the dynamics described by Croft and Veselinova, and in
\sectref{sec:int-3} of this paper, are just as valid as for the negative
existentials that derive from fusion. And, importantly, these Negative
Existential Cycles are not part of Jespersen Cycles, for the simple reason
that they do not involve two things, at least one of which is a standard
negator. The Negative Existential Cycle may thus serve inside the
generalized Jespersen Cycle in the sense that we get from one standard
negator to another one with fusion, but it need not.

\section{The ``Quantifier Cycle'', similarities and links}\label{sec:int-5}

We now turn to the ``Quantifier Cycle'', not for a full analysis but for
describing the similarities and the links with the cycles that yield
standard negators. As the introduction of the ``Quantifier Cycle'' as a
``Jespersen Argument Cycle'' by \textcite[438]{Ladusaw1993} already suggests,
the ``Quantifier Cycle'' and the classical Jespersen Cycle are very similar.
What Ladusaw had in mind was the similarity between French\il{French} \textit{pas} and
\textit{personne}, shown in \REF{ex:int-pas-personne} in a four stage
format 
(cp. \citealt[][263]{Gianollo2018a}, \citeyear{Gianollo2018b}). 
%
\begin{exe}\ex\label{ex:int-pas-personne}
\begin{tabularx}{.9\textwidth}[t]{@{} l l @{}}
    \textit{ne} `not'       &\textit{personne} `person'\\ 
    →                           &→                      \\ 
    \textit{ne \ldots{} pas} `not any step'&     
    \textit{ne \ldots{} personne} `not any person'\\ 
    →                       &→                          \\ 
    \textit{ne \ldots{} pas} `not'  &
    \textit{ne personne} `nobody'           \\ 
    →                       &            →      \\ 
    \textit{pas} `not'      &\textit{personne} `nobody' \\
\end{tabularx}\end{exe}
%
A first similarity is that both French\il{French} \textit{pas} and
\textit{personne} were once polarity neutral nouns -- and these uses
prevail until today -- and they both turned into negative polarity
expressions on their way to becoming negative expressions (a process
finished for \textit{pas}).  Second, these reinterpretations are mirrored
by fusions. Different from French \textit{pas}, Latin\il{Latin}
\textit{non} involved fusion. Likewise, different from French
\textit{personne}, English\il{English} \textit{nobody} involved
fusion. Third, the third stage is in both cases a kind of doubling, i.e.,
classical Jespersenian doubling for standard negation and so-called
``negative concord'' for the negative indefinites. Fourth, in both cases the
doubling can get undone. Fifth, the undoubling stage need not be a final
stage. \textit{pas} can be the beginning of a new Jespersen Cycle and we
are back at stage 1. For the pronouns there is cyclicity too, but in the
version of the Cycle shown in \REF{ex:int-pas-personne} we go back to the
preceding stage: a standard negator is added and we return to negative
concord.  Interestingly, in the well-known cases of Canadian French and
Brabantic Belgian Dutch \parencite[e.g.][499]{AuweraAlsenoy2016} the
standard negator that is added now is not the one that fell in disuse.
\REF{ex:int-canadian-town} is an example from Canadian French, brought into
the literature since at least \citet[262--263]{Muller1991}.
%
\begin{exe}\ex\label{ex:int-canadian-town}\il{French}
Canadian French [no ISO code] 
\langinfo{}{Indo-European}{\citealt[262]{Muller1991}}\\
    \gll \ldots{} y a \textbf{pas} \textbf{personne} en ville \\
    {} there  has  \textsc{neg}  nobody    in  town \\
    \glt `[…] there is nobody in town'
    \end{exe} 

\REF{ex:int-pas-personne-cycle} shows the cyclicity based on the modelling
of \REF{ex:int-canadian-town}.
%
\begin{figure}
\begin{exe}\ex\label{ex:int-pas-personne-cycle}
\begin{tikzpicture}[
baseline=(not),
every path/.style={semithick, },
% solidline/.append style={
%     ->,
%     },
every node/.style={
    align=left,
    },
box/.append style={
    rectangle,
    semithick,
    fill=white,
    draw,
    },
]
\matrix[row sep={3.5ex,between origins},
cells={anchor=base west},
column sep=1cm,]{
    \node (not) [box] {\textit{ne} `not'};    & \\
    \node {→};   &       \\
    \node (not-any-step) {\textit{ne \ldots{} pas} `not any step'};   &
        \node (person) {\textit{personne} `person'}; \\
    \node {→};           &\node {→};       \\
    \node (not2) {\textit{ne \ldots{} pas} `not'};   &
        \node (nobody) [box] {\textit{ne personne} `nobody'};    \\
    \node {→};           &\node {→};          \\
    \node (not3) [box] {\textit{pas} `not'}; &
        \node (nobody2) {\textit{personne} `nobody'};    \\
                        &\node{→};               \\
    &\node (nobody3) [box] {\textit{pas personne} `nobody'};\\
};
\draw (not.west) .. 
    controls ($ (not.west) +(-3mm,0) $) and
    ($ (not3.west) +(-3mm,0) $) .. (not3.west);
\draw (nobody.west) .. controls
    ($ (nobody.west) +(-7mm,3mm) $) and
    ($ (nobody3.west) +(-7mm,-3mm) $) 
    .. (nobody3.west);
\end{tikzpicture}\end{exe}\end{figure}
%
But this representation can be improved. As already argued in
\sectref{sec:int-1} if we add Latin \textit{nemo} `nobody', there is more
cyclicity and if we do not mention the lexical component \textit{personne}
`person', just like the Jespersen Cycle does not have a separate stage
for \textit{pas} `step', the similarity becomes more transparent. We add
\textit{X} as the as yet unfulfilled ``doubler'' of \textit{pas}. 
%
\begin{figure}
\begin{exe}\ex\label{ex:int-pas-personne-cycle-improved}
\begin{tikzpicture}[
baseline=(not),
every path/.style={semithick, },
% solidline/.append style={
%     ->,
%     },
every node/.style={
    align=left,
    },
box/.append style={
    rectangle,
    semithick,
    fill=white,
    draw,
    },
]
\matrix[row sep={3.5ex,between origins},
cells={anchor=base west},
column sep=1cm,]{
    \node (not) [box] {\textit{ne} `not'};    & 
        \node (nemo) [box] {\textit{nemo} `nobody'};    \\
    \node {→};   &\node {→};       \\
    \node (not-any-step) {\textit{ne \ldots{} pas} `not any step'};   &
        \node (person) {\textit{ne \ldots{} personne} `not any person'}; \\
    \node {→};           &\node {→};       \\
    \node (not2) [box,] 
        {\makebox[.85in][l]{\textit{ne \ldots{} pas} `not'}};   &
        \node (nobody) [box] {\textit{ne personne} `nobody'};    \\
    \node {→};           &\node {→};          \\
    \node (not3) [box] {\textit{pas} `not'}; &
        \node [box] (nobody2) {\textit{personne} `nobody'};    \\
    \node {→};          &\node {→};               \\
    \node (x) [box,]
    {\makebox[.85in][l]{\textit{pas} X `not'}}; &
        \node (nobody3) [box] {\textit{pas personne} `nobody'};\\
};
\draw (not.west) .. 
    controls ($ (not.west) +(-3mm,0) $) and
    ($ (not3.west) +(-3mm,0) $) .. (not3.west);
\draw (nobody.east) .. controls
    ($ (nobody.east) +(11mm,2mm) $) and
    ($ (nobody3.east) +(7mm,-2mm) $) 
    .. (nobody3.east);
\draw (not2.east) .. controls
    ($ (not2.east) +(5mm,1mm) $) and
    ($ (x.east) +(5mm,-1mm) $)
    .. (x.east);
\draw (nemo.west) .. controls
    ($ (nemo.west) +(-5mm,3mm) $) and
    ($ (nobody2.west) + (-5mm,-3mm) $) 
    .. (nobody2.west);
\end{tikzpicture}\end{exe}\end{figure}
%
Of course, the motivation to redouble for \textit{personne} is not the
complex Jespersen Cycle trajectory. A plausible explanation, we find, is
the one offered by \citet[203]{Haspelmath1997}, echoing
\citet[99]{Heidolph1970}: standard negation is clause-level negation and
when it is marked on a participant there is a tendency to remedy this
construction and to add a standard negator. So this is a significant
difference between the two cycles. There are more differences. First of
all, the doubling illustrated by Canadian French\il{French} is not the only
additional stage in the ``Quantifier Cycle''. In another scenario, the
negative indefinite may trade its negativity for negative polarity. This is
taken to have happened to e.g. French \textit{nul} `no (one)'
\parencites(see)()[113--114]{Catalani2001}[135--137]{Buridant2000}[327]{AuweraAlsenoy2011}[211--213]{Gianollo2018a}
and \textit{jamais} \citep{MosegaardHansen2012}, as well as in Jamaican
Creole.  \REF{ex:int-oldfrench-bribe} is an Old French non-negative
example, culled from a fable by Marie de France by
\citet[167]{Buridant2000}.  \REF{ex:int-jamaicancr-kill} shows two
Jamaican\il{Jamaican Creole} Creole examples, taken from
\citet{DiTestament2012} and discussed in \citet{AuweraLisser2019}.
%
\begin{exe}\ex\label{ex:int-oldfrench-bribe}\il{French}
Old French [fro] \langinfo{}{Indo-European}{\citealt[118]{Brucker1998}}\\
    \gll Si  \textbf{nuls}    l'en  veut  doner  lüer \ldots \\
      If  anyone  him  wants  give  reward \\
    \glt `If anyone wants to bribe him \ldots'
\ex\label{ex:int-jamaicancr-kill}\il{Jamaican Creole}
   Jamaican Creole [jam] (Indo-European)
   \begin{xlist}
   \ex\label{ex:int-jamaicancr-kill-a}
   \gll \ob\ldots{\cb} \textbf{nobadi} we kil \textbf{nobadi}, dem a-go
   go a kuot ous \ob\ldots{\cb} \\
  {} nobody \textsc{rel} kill anybody \textsc{3pl}
  \textsc{prog}-\textsc{prosp} go to  court  house \\
    \glt `\ldots{} anybody who kills anybody is going to go to court
    [\ldots]' (\textit{Matthew} 5: 21)
    \ex\gll Bot muo dan \textbf{notn} els, Gad gud an kain tu wi.\\ 
    but more than anything else God good and kind to \textsc{1pl}\\
    \glt `But more than anything else, God is good and kind to us.'
    (\textit{James} 4: 6)
    \end{xlist}\end{exe}
%
In yet another scenario, the negative indefinite loses a marker of
negativity. This has been argued by \citet{AuweraCuypere2006} for a
small area within Brabantic Belgian Dutch\il{Dutch} in which the negative indefinite
\textit{niemand} `nobody' of the negative concord pattern in
\REF{ex:int-brabantic-nobody-a} has lost its initial nasal, thus resulting in
\textit{iemand}, the positive indefinite (`someone').
%
\begin{exe}\ex\label{ex:int-brabantic-nobody}\il{Dutch}
Brabantic Belgian Dutch [no code] (Indo-European)
    \begin{xlist}
    \ex\label{ex:int-brabantic-nobody-a}
    \gll Ik  heb  \textbf{niemand}  \textbf{nie}  gezien.  \\
    I  have  nobody  \textsc{neg}  seen \\
    \glt `I have seen nobody.'
    \ex\gll Ik  heb  \textbf{iemand}  \textbf{nie}  gezien.\\
    I  have  somebody  \textsc{neg}  seen\\ 
    \glt `I have seen nobody.'
    \end{xlist}\end{exe} 

All in all, the differences between the standard Jespersen Cycle and the
Quantifier Cycle are substantial%
%
    \footnote{No wonder also that \citet{Larrivee2011}, whose notion of
    Jespersen Cycle is narrower than ours but which subsumes the
    ``Quantifier Cycle'', concludes that what is going on is too diverse to
    continue using the term ``Jespersen Cycle''.} %
%
and, we propose, this is mirrored by the fact that not that many languages
seem to have undergone both the Quantifier and Jespersen Cycles. Or, put
differently, Jespersenian doubling probably seldom co-occurs with negative
concord \parencites{AlsenoyAuwera2014}[182--195]{Alsenoy2016}.%
%
\footnote{In Van Alsenoy's sample of 179 languages only 6 languages have
both Jespersenian doubling and negative concord
\parencite[187]{Alsenoy2016}.} %
%
But in languages like French\il{French} and English\il{English} the two
cycles do co-occur. In both English and French we see that a new standard
negator is recruited from the set of negative
indefinites and the resulting pattern is a doubling pattern, not unlike the
negative concord of the negative indefinites. In Latin\il{Latin} and
Greek\il{Greek} the new
standard negator also derives from a negative indefinite, but this time it
does not come from a doubling pattern but from one in which the negative
indefinite is not accompanied by a standard negator.%
%
\footnote{Different
from Latin, in Greek the indefinite that became a standard negator had
negative concord, but it was the non-strict type and it is from the
preverbal concord-free use of the negative indefinite that the standard
negator must have developed \citep[294--295]{Chatzopoulou2012}.} %
%
We also see
that when doubling disappears in standard negation, negative concord
disappears as well, and one may assume that the loss of the old standard
negator in one construction influences its loss in the other
pattern.%
%
\footnote{There is no claim here that the two processes are in sync
or it is invariably the same process that leads. Thus
\citet[152]{Ingham2011} argues that in Anglo-Norman the old negator
disappears in indefinites before it does in standard negation, but
\citet[176]{Jager2013} holds the opposite view for Middle High German.} %

\section{Conclusion}\label{sec:int-6}

In this paper we aimed to increase the understanding of each of the three
Negative Cycles individually and of the links between them. We focused on
the interaction between the Jespersen and the Negative Existential Cycles.
We argued for a wide definition of the Jespersen Cycle, which solves the
currently existing terminological dilemma. The new definition allows
elements not only to be contaminated by negators, and thus become negators
themselves, but also to fuse with negators and thus also make new negators.
Fusion can also yield negative existentials, and to that extent the
Negative Existential Cycle is part of a Jespersen Cycle, as are the
instances where Negative Existential Cycles allow negator doubling. We
integrated a Positive Existential Cycle, i.e., a scenario in which an
existential marker does not fuse with a negator but is contaminated by it.
Finally, we described the similarities and differences between Jespersen
and Quantifier Cycles and the way ``Quantifier Cycle'' output can be inserted
into a Jespersen Cycle. We also proposed a more enlightening modelling of
what goes in the ``Quantifier Cycle''.

\section*{Acknowledgements} 

Thanks are due to Franck Floricic (Paris), Josif Fridman (Moscow), Armin
Schwegler (Irvine), Anne Vanderheyden (Antwerp), Ljuba Veselinova
(Stockholm) and an anonymous reviewer. The latter advised caution with
respect to the Wintu and Sino-Russian Pidgin examples and also prompted us
to use the term ``Quantifier Cycle'' in a new sense. Further gratitude is
due the Research Foundation Flanders for its financial support.

\section*{Abbreviations}
\begin{minipage}{\textwidth}
\begin{tabularx}{.5\textwidth}[t]{@{} l Q }
1				&1\textsuperscript{st} person\\
3 				&3\textsuperscript{rd} person\\
3>3				&3\textsuperscript{rd} person agent + 3\textsuperscript{rd}
person patient\\
\textsc{a}	&agent\\
\textsc{abs}		&absolutive\\
\textsc{act}		&active\\
\textsc{anaph}		&anaphoric\\
\textsc{att}	&attributive\\
\textsc{aux}		&auxiliary\\
\textsc{cond}		&conditional\\
\textsc{cng}		&connegative\\
\textsc{cntr}		&contrastive\\
\textsc{comp}       &complementizer\\
\textsc{cop}		&copula\\
\textsc{ex}		& affirmative existential\\
\textsc{dat} 	&dative\\
\textsc{def}		&definite\\
\textsc{det}	&determiner\\
\textsc{dist} 		&distal\\
\textsc{du}         &dual\\
\textsc{e}          &E-type inflection class\\
\textsc{emph}       &emphatic\\
\textsc{erg}        &ergative\\
\textsc{fut} 		&future\\
\textsc{genr}		&general tense-aspect-mood\\
\textsc{ger}		&gerund\\
\end{tabularx}
% \settowidth{\colabbrchadic}{\textsc{neg.ex}}
% \settowidth{\colglosschadic}{Non-human/locative pronoun}
\begin{tabularx}{.5\textwidth}[t]{ l Q @{}}
\textsc{hab}		&habitual\\
\textsc{ipfv}		&imperfective\\
\textsc{incl}  &inclusive\\
\textsc{indf}	    &indefinite\\
\textsc{inf}        &infinitive\\
\textsc{loc}		&locative\\
\textsc{m} 		&masculine\\
\textsc{neg}		&negation\\
\textsc{nmlz}       &nominalizer\\
\textsc{npst}       &non-past\\
\textsc{nsp}        &non-specific tense-aspect\\
\textsc{obl}		&oblique\\
\textsc{part}		&partitive\\
\textsc{pl}		&plural\\
\textsc{poss}		&possessive\\
\textsc{pr}		&presentative\\
\textsc{prog}		&progressive\\
\textsc{prosp}		&prospective\\
\textsc{prs}		&present\\
\textsc{pst}	&past\\
\textsc{sg}		&singular\\
\textsc{stat}		&stative\\
\textsc{sub}		&subordinator\\
\textsc{ta}		&tense aspect\\
\textsc{tam}		&tense aspect mood\\
\textsc{top}		&topic\\
\textsc{tr}         &transitive\\
\textsc{v}          &verb\\
\end{tabularx}
\end{minipage}

\section*{List of languages}
\setlength\LTleft{0pt}
\setlength\LTright{0pt}
\begin{longtable}{@{} p{\textwidth} @{}}
Amharic [amh]\\
Arizona Tewa [tew] (NB: Arizona Tewa seems not to have its own ISO 693-3
code. The ISO code given here is the one for ``Rio Grande Tewa'', which is
one of at least varieties of Arizona Tewa. However, we do have geo
coordinates for Arizona Tewa: Latitude: 35,84; Longitude: -110,38 (source:
Glottolog)).\\
Awju [ahh] (NB: Awju is a group of 4 languages, we mention the group in the
text, not an individual language, here we give an ISO code of just one of
four languages)\\
Bulgarian [bul]\\
Cèmuhî\\
Drehu[dhv]\\
Dutch, Brabantic Belgian [no ISO code, glottolog code: brab1243]\\
Dumi [dus]\\
East Futuna [fud]\\
English [eng]\\
English, Ghanaian [no ISO code]\\
English, dialectal [no ISO code]\\
Finnish [fin]\\
Finnish, dialectal [no ISO code]\\
French [fra]\\
French, Anglo-Norman [xno]\\
French, Canadian [no ISO code]\\
French, Old [fro]\\
German, Middle High German [gpe]\\
Greek, Classical [grc]\\
Greek, Modern [ell]\\
Hawai'an [haw]\\
Juang [jun]\\
Kannada [kan]\\
Kanincin [rnd]\\
Kapingamarangi [kpg]\\
Kaugel [ubu]\\
Korku [kfq]\\
Kulina [cul]\\
Latin [lat]\\
Lewo [lww]\\
Lifunga [bmg] (NB: Ethnologue gives Lifunga as one of the dialects of
Bamwe [bmg]. There seems to be no separate iso-code for Lifunga, so we give
the iso-code fot Bamwe here.)\\
Mara [mec]\\
Meithei [mni]\\
Nese [no ISO code; glottolog code: nese1235]\\
Nengone [nen]\\
Sino-Russian Pidgin [no ISO code; glottolog code: kjac1234]\\
Swedish [swe]\\
Tacana [tna]\\
Tongan [ton]\\
Tuvaluan [tvl]\\
Urarina [ura]\\
Wintu [wit]
\end{longtable}

%%%
\printbibliography[heading=subbibliography,notcategory=online]
\printbibliography[heading=subbibliography,category=online,title={Sources}]
\end{document}
