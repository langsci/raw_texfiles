\chapter{Empirical investigation}\label{sec:experiments}
In this chapter I present empirical studies that focus on   the pragmatic effects of attributive \emph{que}. Very little has been said to date regarding the interpretation of these effects, meaning that more systematic research has proved necessary. Since there are no previous  experimental or corpus analyses that can be used as a basis for predictions,  the empirical studies that will be described in the following sections approach the phenomena with an exploratory mindset and toolkit. The  pragmatic effect of attributive \emph{que} is  strongly context-dependent. This instance of \emph{que} is hardly ever obligatory and often co-occurs with other context-dependent expressions.  Pairing corpus-based and experimental methods appears to be a suitable means of gaining an initial impression of the possible pragmatic effects involved in the constructions. Given these properties, it is necessary to take  native speaker's judgments on the acceptability and felicitousness of the constructions into account.  The approach I present in this chapter is also in keeping with the move towards empirical studies  that is currently changing the face of theoretical linguistics. This  dynamic and ever evolving field  invites experimentation and creativity in order to  test and implement  new methodological approaches. Exploring empirical methods to investigate the types of meaning involved in attributive \emph{que} will therefore hopefully also prove useful for future researchers.

This chapter starts with a general introduction to the empirical and statistical methodology employed. \sectref{sec:expcorp} presents the experimental method used. The experiments are a version of acceptability judgment experiments that differ from the traditional type in that the experimental stimuli are not constructed by the experimenter but are sampled from corpus data. In \sectref{sec:expstats}, I provide some background information on the statistical tools that I used to model the data. \sectref{sec:expdeic} presents three empirical studies focusing on the interpretation of \emph{que} in AdvC constructions. The first study is a corpus analysis that compares the distribution of AdvC to two other constructions in Spanish and Portuguese. The other two studies are corpus-based experiments on the same construction, one on Spanish and one on Catalan.  In \sectref{sec:expbias}, I present a corpus-based experiment that focuses on attributive \emph{que} in Catalan \isi{polar question}s and compares their acceptability to their \emph{que}-less counterparts. The aim is again to draw conclusions about the interpretation of \emph{que}. The chapter concludes with \sectref{sec:empgeneraldiss}, which provides a general discussion of how exploratory empirical research can inform our theoretic choices.

\section{Corpus-based experiments}\label{sec:expcorp}

For the experiments I use a method that is  inspired by the work described in \citet{Degen2015}. It differs from more traditional experimental approaches in that the stimuli are not constructed by the experimenter but are sampled from corpora. This means that the stimuli are heterogeneous by design. This design proves particularly useful in pragmatics research, because the stimuli are embedded in felicitous contexts. This reduces the risk of unintended outcomes provoked by accidental artifacts in the data. A further advantage of randomly sampling the stimuli is that a wide range of statistical tools can be employed to analyze the data,  enabling us to learn from the data and find patterns in a way that would not have been possible if the stimuli had been constructed with a specific type of analysis in mind. Phenomena that are strongly context-dependent are often affected by  multiple factors and more traditional experiments  concentrate on only a few of these. This can  lead to very clear results with respect to the effects of these few factors in the precise experimental contexts. It is however not always clear whether and how these results can be generalized to natural data. One risk is therefore that reducing the variation in the data creates results with a  limited predictive power  because the complexity of the phenomenon is not represented. Instead, corpus-based experimental methods acknowledge and invite  the complexity of linguistic data.

Developing this method means moving away from the established methods, which does carry some risks, but offers the significant advantage of finding new approaches to the data and  achieving novel results. In a way, this method combines the best of both worlds: Corpus-based research  expands our knowledge of the phenomenon and experimental research enhances this knowledge by including the native speakers' perspectives. In the best case scenario, it provides us with fine grained information on a phenomenon that cannot be extracted from the corpora alone. We therefore gain  a more complete picture of the issues under investigation.

In \citet{Degen2015}, all occurrences containing the relevant pattern she investigated were sampled and presented to native speakers of English, who were asked to judge how similar the stimulus was to a paraphrase provided by the experimenter. I use a different design. In my experiments a small random sample of the relevant patterns is drawn from a corpus, which is then carefully modified to accommodate the questions that I am trying to answer in this study. The judgments that I elicit are not based on  similarity to a paraphrase. The reason for this is that similarity judgments rely on the idea that the participants have a relatively clear understanding of the interpretation of the constructions, which did prove to be the case  in \citeauthor{Degen2015}'s study, but cannot be assumed to be true for the constructions involving attributive \emph{que}. I  elicit judgments on the interpretation of the constructions in a more indirect way. My primary interest is in studying how constructions containing attributive \emph{que} are different from similar constructions that lack the complementizer. 

I follow the example of \citeauthor{Degen2015} and use no fillers in my experiment. The stimuli are already very heterogeneous, so there is no risk of habituation due to  monotonous input. Additionally, the tasks I designed are tailored specifically to elicit judgments on very precise aspects of the interpretation of the constructions. Finding fillers that can be matched with the same tasks is therefore nearly impossible. Finally, one of the functions of fillers in traditional experiments is to make sure that the participants are unaware of the  goal of the investigation. Ensuring that the participants do not know the aims of the study is  necessary  in many areas of linguistic research, for instance when  a normative bias could influence the judgments. However, in the area that I am investigating here, the participants are unlikely to have explicit knowledge about what does and what does not correspond to the norm. Designing the experiment without fillers might therefore  mean that  the participants become aware of the goal of the investigation. Although the results of my experiments indicate that this was not actually the case, in principle a conscious participant can be a valuable informant for the questions that I am trying to answer.



\section{Statistical modeling}\label{sec:expstats}
I follow \citet{Baayen2008} in subscribing to a modern type of exploratory data analysis. In this approach we allow for the possibility that not all of the patterns in the data can be explained by an a priori formulated theory. Instead, data exploration is carried out with an open mind, inviting the possibility that there is more to find in the data than the theory can predict. Therefore, in the following studies, I pair tools and strategies from descriptive and hypothesis-testing approaches to see what we can learn from the data beyond what we may expect based on theory-driven hypotheses.



Conditional inference trees are classification trees that are  based on binary recursive partition (see  \citealt{Hothorn2006}, \citealt{Strobl2009}). They are robust against data sparseness and  make no assumptions about the distribution of the population from which the data was sampled. They can be used to detect high-order interactions and work well even  if the predictors are highly correlated. All these aspects make them useful for analyses of linguistic data that often have a high degree of variability (see \citealt{Tagliamonte2012}). 

Conditional inference trees, rather like regression models,  predict the response variable based on a series of predictor variables. The prediction is based on binary recursive splits. Conditional inference trees test whether a predictor is associated with a response variable and choose the predictor variable that has the strongest association with this response variable.  The dataset is then split into two subsets  with significantly different distributions of the response variable.  These steps are repeated on the subsets until there is no predictor variable left that is associated with the response variable at the level of statistical significance. The statistical significance is determined by permutation (see \citealt{Tagliamonte2012}, \citealt{Levshina2015}).



\begin{figure}[p]
\includegraphics[width=\textwidth]{figures/auxtree.pdf}
\caption{Conditional inference tree; \label{fig:auxtree}The tree was fit to the built-in dataset \texttt{auxiliaries}  from the \texttt{languageR} package (\citealt{Baayen2019}). It shows the influence of the number of synonyms (VerbalSynsets) and the regularity of the lexical verb (Regularity) on the choice of the auxiliary in the Dutch perfect tense.}  
\end{figure}

\begin{figure}[p]
\includegraphics[width=0.75\textwidth]{figures/auxvarimp.pdf}\caption{Dot plot of variable importance \label{fig:auxvarimp}}
\end{figure}

The example tree  in \figref{fig:auxtree} as well as all the other conditional inference trees in this book are modeled using the \verb|party| package in \verb|R| (\citealt{Hothorn2022}). In \figref{fig:auxtree} the choice of the Dutch past perfect auxiliary (\textit{hebben} `to have', \textit{zijn} `to be' or `both', i.e. 
 `have' and `be') is predicted by the number of synonyms (\textsc{VerbalSynsets}) and the regularity of the lexical verb (\textsc{Regularity}). Both predictors are significant. The strongest association is with the number of  synonyms that each verb has. This means that splitting the dataset into two groups based on the number of synonyms creates two datasets in which the distribution of the auxiliaries is  significantly different. In the left branch the number of verbal synonyms is 6 or higher. The right branch contains the complementary set with fewer than 6 verbal synonyms. Within the subset of the verbs with high numbers of synonyms, there is a further split on regularity. This  means that within this group, the choice of the auxiliary is significantly different depending on whether the verb is regular or not. 
 
 Random forests  are a useful tool to measure the importance of the predictors in tree models. A random forest is computed on a large number  of conditional inference trees, each of which is calculated based on a randomly generated subset of the data. The importance of the predictors is once again determined through permutation (cf. \citealt{Breiman2001}). \figref{fig:auxvarimp} is based on a random forest calculated for the same model as the conditional inference tree in \figref{fig:auxtree}. The Figure shows the relative importance of the two predictors in the model. The number of verbal synonyms is  identified as the most important  predictor.  This is also evident from the conditional inference tree in \figref{fig:auxtree}. The dot plot in \figref{fig:auxvarimp} additionally shows that its importance is far greater than that of the second predictor.

\clearpage\section{Deictic centers of epistemic and evidential modifiers}\label{sec:expdeic}
In this section, I discuss three empirical studies,  all of which focus on the interpretation of \isi{epistemic and evidential modifier}s in different constructions.   In principle, the studies are intended to investigate the function of \textit{que} in AdvC. To do this, I have designed corpus studies and experiments that compare AdvC to two similar constructions,  all of which contain \isi{epistemic and evidential modifier}s. The three constructions are given in \eqref{ex:epevsp}.

\ea\label{ex:epevsp} 	Spanish\\
	\ea
		
			(AdvC) \\
	 Seguro que Juan viene. 		
	 \ex (Adv)  \\
				Seguramente Juan viene. 
		\ex 	 (EsAdjC)\\
		Es seguro que Juan viene.  \\
		`Sure \textsc{que} / Surely / It is sure that John will come.'
	\z
\z

I am interested in determining the different readings that the modifiers receive  in these constructions (see also \sectref{sec:presupadvc} and \sectref{sec:presupprag} for a characterization of the constructions and their interpretation). Generally epistemic modals and evidentials express an evaluation of a proposition. This evaluation can be interpreted from the perspective of different \isi{deictic center}s made up either of  individual speech participants or of a joint perspective between multiple speech participants. The function that I propose for attributive \emph{que} is that it attributes a commitment to the proposition  to the hearer. A consequence is that the presence of  \emph{que} in AdvC usually establishes a shared perspective between the speaker and the interlocutor. I  proposed in \sectref{sec:presupprag} that by way of an implicature,  the  reading of the modifiers in AdvC is also centered on this shared perspective. In EsAdjC  the commitment to the propositions is added to the set of general discourse commitments. Again through an implicature, the perspective, on which   the evaluation is centered should therefore be that of the speaker and some contextually relevant authority or group of people. The evaluation presented in Adv is unmarked with respect to which perspective it is centered on. Therefore, whichever perspective is most plausible in the context is adopted. 

The first approach to the phenomenon is a corpus analysis that compares the distribution of the constructions in Spanish and Portuguese in different text types. The assumption is that different text types tend to invite different perspectives. For instance, in oral texts, which often include dialog sections, it is easier to refer to or present assumptions about the interlocutor's perspective than it is in academic texts, in which the interlocutor is often not present. One expectation is therefore that AdvC, which centers the evaluation of the modifier on the speaker and the interlocutors, should be more frequent in informal and oral than in formal texts. 


The second approach is composed of two experiments that build on the insights of the corpus study. Both elicit acceptability judgments and investigate the effect of the factors \textsc{construction}, \textsc{deictic center} and \textsc{modifier} on   the acceptability of the stimuli.  The first experiment deals with Spanish and the second  with Catalan.

As stated above, I adopt an exploratory approach in the design and the statistical analysis of the following studies, since it offers the potential to learn from the data and draw novel conclusions that could not have been formulated a priori.  
\subsection{Corpus study on Spanish and Portuguese}\label{sec:corpussppt}
In this section, I describe an exploratory corpus analysis through which I aim to determine the factors that influence the distribution of the three constructions. In this study I compare Spanish and Portuguese data. These languages were chosen because there are two very  similar publicly available corpora, namely the 2006 Genre/Historical \textit{Corpus do Português} (CdP) and the 2001 Genre/Historical \textit{Corpus del Espa\~{n}ol} (CdE). They have a comparably moderate size but unlike large contemporary web-crawled corpora, they have the advantage of consisting of different text types. \textsc{Text type} is a variable in the study that proved to be a significant predictor. A subset of the data has been used in a previous study reported in \citet{Kocher2018}. The results presented here are   novel, because  the database and the statistical approach are new. This study also functions as preparation for the following experiments. The implications of the results form the basis for the formulation of the hypotheses that drive the development of the experiments. 


\subsubsection{Corpus}
I used the 20th century subcorpus of the CdE and the CdP. Both corpora are publicly available and can be queried through an online interface. They cover data from the 12th century up to the 20th century and contain data from different dialects of both languages. The Spanish 20th century subcorpus makes up a quarter of the full corpus and the Portuguese one a little under half of it. These two subcorpora were chosen because they are comparable in their composition and their size (cf. Table \ref{tab:corpuscomparison}). Sample size was considered as a potential predictor, but showed no significant effect.

\begin{table}
\begin{tabular}{lcccccc}
\lsptoprule
& \multicolumn{5}{c}{Text type}\\\cmidrule(lr){2-6}
& academic & press & oral & fiction & total\\ 
\midrule 
		Corpus del Espa\~{n}ol & 5.1 & 5.1  & 5.1 & 5.1  &20.4\\ 
		Corpus do Português    & 5.9 & 6.6  & 2.2 &   6\phantom{.1}  &20.7\\ 
\lspbottomrule
\end{tabular}
\caption{Number of tokens in million words per \textsc{text type} in CdE and CdP \label{tab:corpuscomparison}}
\end{table}

\subsubsection{Sampling}
I collected the sample through the online interface and then exported it to \verb|R| (\citealt{RCT2013}) in order to add annotations and perform analyses. 
The first step was to determine the most frequent \isi{epistemic and evidential modifier}s in the three constructions in the corpora. These turned out to be the following: \textit{cierto}/\textit{certo},\footnote{In the following I will use the Spanish cognate to refer to all  the modifiers.} \textit{claro}, \textit{evidente}, \textit{natural},\footnote{For a discussion of the evidential nature of \textit{natural} in Spanish and Portuguese, see \citet{Kocher2014}.} \textit{probable} and \textit{seguro}. I then selected all the occurrences of the modifiers in the three constructions in sentence-initial position. The corpora are not annotated for sentence position; instead, this  was approximated by querying all occurrences of the relevant pattern following a punctuation mark that indicates the end of a sentence.  For AdvC, I included all cases of modifiers that end in \textit{-mente} and those that do not (cf. Table \ref{tab:corpusmente}). For EsAdjC, I included  versions with  copulas \textit{ser} and \textit{estar} (cf. Table \ref{tab:corpusserestar}).  Neither of these variables showed a significant effect in the modeling. 

\begin{table}

\begin{tabular}{lrrrrrrrrrrrr}
\lsptoprule
& \multicolumn{2}{c}{\itshape cierto} & \multicolumn{2}{c}{\itshape claro}& \multicolumn{2}{c}{\itshape evidente} & \multicolumn{2}{c}{\itshape natural} & \multicolumn{2}{c}{\itshape probable} & \multicolumn{2}{c}{\itshape seguro} \\\cmidrule(lr){2-3}\cmidrule(lr){4-5}\cmidrule(lr){6-7}\cmidrule(lr){8-9}\cmidrule(lr){10-11}\cmidrule(lr){12-13}
& sp & pt & sp & pt &sp & pt &sp & pt &sp & pt &sp & pt \\\midrule 
		\textit{-mente} &  14 & 8 &  0&0  & 18&  32 &   30  &   36 &   2&   0&   11&   3 \\ 
		\textit{-0} &34&9  &576 & 306 &0 &      11  & 2 &   4 & 0 &     0&82&    0\\
\lspbottomrule
	\end{tabular}
	\caption{Distribution of derived and underived adverbs per modifier in AdvC\label{tab:corpusmente}}
\end{table}
                        
\begin{table}
\begin{tabular}{lrrrrrrrrrrrr}
\lsptoprule 
& \multicolumn{2}{c}{\itshape cierto} & \multicolumn{2}{c}{\itshape claro}& \multicolumn{2}{c}{\itshape evidente} & \multicolumn{2}{c}{\itshape natural} & \multicolumn{2}{c}{\itshape probable} & \multicolumn{2}{c}{\itshape seguro} \\\cmidrule(lr){2-3}\cmidrule(lr){4-5}\cmidrule(lr){6-7}\cmidrule(lr){8-9}\cmidrule(lr){10-11}\cmidrule(lr){12-13}
	& sp & pt & sp & pt &sp & pt &sp & pt &sp & pt &sp & pt \\
\midrule 
		\textit{estar} &  0& 8 &  34&17 & 0&  0 &   0 &   0 &   0&   0&   0&   0 \\ 
		\textit{ser} &126&81  &24 & 190 &64 &      96  & 11 &    31  & 85&     25 & 8&     1\\
\lspbottomrule
	\end{tabular}
\caption{Distribution of \textit{ser} and \textit{estar} per modifier in EsAdjC\label{tab:corpusserestar}}
\end{table}

\subsubsection{Description of the data and  predictors}
The conditional inference tree that I will describe in the following section models the influence of the \textsc{modifier}, the \textsc{\textsc{text type}} and the \textsc{language} on the distribution of the three \textsc{constructions}. In this section I describe the distribution of these variables in the sample.

\begin{table}
\begin{tabular}{l *4{r}}
\lsptoprule
	& Adv & AdvC & EsAdjC& total\\ 
\midrule
	pt & 511 & 409 & 449&1369  \\ 
	sp & 839 & 769 & 352 &1960\\ 
\midrule 
	total	&1350 &  1178  &  801&3329\\
\lspbottomrule
	\end{tabular}
\caption{\textsc{construction} by \textsc{language}\label{tab:corpconstlang}}
\end{table}

Table \ref{tab:corpconstlang} shows the occurrence of each \textsc{construction} per \textsc{language}. In general there  are more  propositions containing an epistemic or evidential modifier in Spanish than in Portuguese. The difference in frequency between the \textsc{construction}s is strongest in Adv and AdvC, which both occur far more frequently in Spanish than in Portuguese. The differences  between the \textsc{construction}s are small in Portuguese, but are relatively large in Spanish. The variable \textsc{language} has only two levels and does not take dialectal variation into account. In the process of model fitting and selection, I considered dialectal variation as a possible predictor. Each data point in the corpus contains information about the country of origin of the author or speaker. For Portuguese, only two varieties (European and Brazilian Portuguese) are part of the corpus. For Spanish there is a larger range of countries of origin represented.  I grouped these into  seven dialectal varieties in accordance with \citet{Hualde2009} resulting in   Andean, Canarian, Caribbean, Central American, Chilean,  European, Mexican and Rioplatense Spanish. The variable was significant, but the increase in accuracy was minimal (2\%) and the resulting patterns were not straightforwardly interpretable. I hence chose to use a more simple and insightful model containing  the variable \textsc{language} with  only   two values, Spanish and Portuguese.

\begin{table}
\fittable{\begin{tabular}{l@{~}rrrrrrrrrrrr}
\lsptoprule
Modifier & \multicolumn{2}{c}{\itshape cierto} & \multicolumn{2}{c}{\itshape claro}& \multicolumn{2}{c}{\itshape evidente} & \multicolumn{2}{c}{\itshape natural} & \multicolumn{2}{c}{\itshape probable} & \multicolumn{2}{c}{\itshape seguro} \\\cmidrule(lr){2-3}\cmidrule(lr){4-5}\cmidrule(lr){6-7}\cmidrule(lr){8-9}\cmidrule(lr){10-11}\cmidrule(lr){12-13} 
	& sp & pt & sp & pt &sp & pt &sp & pt &sp & pt &sp & pt \\
\midrule 
		Adv &96 & 91&13 &13 &136 &106 &175 &144 &189 &139 &230 &18\\ 
		AdvC &48 & 17&576 &306 &18 &43 &32 &40 &2 &0 &93 &3\\
		EsAdjC &126 &89 &58 &207 &64 &96 &11 &31 &85 &25 &8 &1\\
\midrule
		 & 270 &197 &       647 & 526&      218 &245  &     218  &215  &       276      &164 &133 &22 \\\cmidrule(lr){2-3}\cmidrule(lr){4-5}\cmidrule(lr){6-7}\cmidrule(lr){8-9}\cmidrule(lr){10-11}\cmidrule(lr){12-13} 
total   & \multicolumn{2}{c}{467} & \multicolumn{2}{c}{1173} & \multicolumn{2}{c}{463} & \multicolumn{2}{c}{433} & \multicolumn{2}{c}{440} & \multicolumn{2}{c}{353} \\
\lspbottomrule
\end{tabular}}
                              
\caption{\textsc{construction} by \textsc{modifier}\label{tab:corpconstmod}}
\end{table}

Table \ref{tab:corpconstmod} shows the frequency of each \textsc{construction} per \textsc{modifier} for each \textsc{language}. The strongest contrast is between \textit{seguro} and \textit{probable}, which are both rarer in Portuguese than in Spanish.  \textit{Claro} is mostly used in AdvC in both \textsc{language}s. There is also a considerable number of instances of this \textsc{modifier} in EsAdjC, but only in Portuguese.

\begin{table}
	\begin{tabular}{l *8{r}}
\lsptoprule
Text type	& \multicolumn{2}{c}{academic} & \multicolumn{2}{c}{fiction }& \multicolumn{2}{c}{news} & \multicolumn{2}{c}{oral} \\
		\cmidrule(lr){2-3}\cmidrule(lr){4-5}\cmidrule(lr){6-7}\cmidrule(lr){8-9}
	& sp & pt & sp & pt &sp & pt &sp & pt  \\
\midrule
	Adv & 57 &104 &267 & 160& 126&98 &389 &149 \\ 
	AdvC & 2&10 &296 &152 &77 &127 &394 &120 \\ 
	EsAdjC &75 &33 &79 &110 &129 &162 &69 &144 \\ 
\midrule

		&134  &147 &642  &422 &332 &387 &852 &413  \\\cmidrule(lr){2-3}\cmidrule(lr){4-5}\cmidrule(lr){6-7}\cmidrule(lr){8-9}
	total   & \multicolumn{2}{c}{281} & \multicolumn{2}{c}{1064} & \multicolumn{2}{c}{719} & \multicolumn{2}{c}{1265}  \\
\lspbottomrule
	\end{tabular}
\caption{\textsc{construction} by \textsc{text type}\label{tab:corpconsttext}}
\end{table}

Table \ref{tab:corpconsttext} shows the frequency of each \textsc{construction} per \textsc{text type} per \textsc{language}. The frequency per \textsc{text type} is comparable in the two \textsc{language}s,\footnote{While there appears to be a  difference in oral texts, there is in reality none to speak of. The tokens per \textsc{text type} in Portuguese are not equally distributed. In particular, the Portuguese oral subcorpus is roughly half the size of the Spanish one (cf. Table \ref{tab:corpuscomparison}).} with the exception that regardless of which \textsc{construction} the \textsc{modifier}s appear in, they are less frequent in Portuguese fiction texts than in the Spanish equivalents.   There is a lower frequency of \textsc{modifier}s in academic and news texts, which can be explained as being due to the fact that these \textsc{text type}s often require the authors to maintain or at least simulate an objective perspective. Some assumptions can also be  made based on the distribution of the \textsc{construction}s in the different \textsc{text type}s. AdvC is very rare in academic texts and most frequent in fiction and oral texts. The high number of AdvC in oral texts, in which an interlocutor's perspective can be addressed most directly, is  an expected result. Fiction texts have more heterogeneous properties than the other \textsc{text type}s. One reason for the similar distribution of AdvC in fiction and oral texts could be that the former also often contain imitations of orality.  

\subsubsection{Results}
In this section, I present the results of the tree model that I fitted to the corpus data. Model selection was carried out in a exploratory manner. I fitted a number of models with a varying degree of complexity and settled on the present model based on objective measurements such as the accuracy of the models and also based on hypotheses-driven criteria such as the plausibility and interpretability of the predicted effects. 
Figure \ref{fig:corpctree} visualizes the conditional inference tree model. 
Three variables are significant in the model: \textsc{modifier}, which shows the greatest effect, \textsc{text type} and \textsc{language}. The dot plot in \figref{fig:corpvarimp} shows the importance of each of these variables, which was determined through a random forest.  Finally, the heat map in \figref{fig:corpheatmap} plots the observed vs. predicted values from Table \ref{tab:corppredictions}. It suggests that the model has a high degree of accuracy in its predictions: In fact it has an accuracy of 70\%. Accuracy was calculated by dividing the sum of the correctly predicted values, i.e. the values in the diagonal of \tabref{tab:corppredictionsnew} ($1137+790+409=2336$), by the sum of all of the predicted values ($2336/3329=0.702$).


\begin{table}
	\begin{tabular}{lrrr}
\lsptoprule
		& observed Adv & observed AdvC & observed EsAdjC \\ 
\midrule
		predicted	Adv & 1137 & 225 & 232 \\ 
		predicted	AdvC &  12 & 790 & 160 \\ 
		predicted	EsAdjC & 201 & 163 & 409 \\ 
\lspbottomrule
	\end{tabular}
	\caption{Predicted categories \label{tab:corppredictionsnew}}
\end{table}


The first and most significant split in the tree model separates all the data containing \textit{claro} from the data containing the remaining \textsc{modifier}s. This suggests that \textit{claro} displays particular behavior that is unlike the others. On the left branch, containing the \textit{claro}-data (cf. Figure \ref{fig:branch1}), Node 2 separates Portuguese and Spanish. Within the Portuguese data,  oral and academic texts are paired together and are significantly different from fiction and news texts. In the terminal Nodes (4,5) the distribution of \textit{claro} in the Portuguese \textsc{text type}s is plotted. In academic and oral texts AdvC and EsAdjC have a similar frequency (more than 40\%). In fiction and news texts, however, AdvC is more frequent. In the Spanish subset (Node 6), there are two subsequent splits on \textsc{text type}. The first one splits academic texts from the rest. In the corresponding terminal Node 10, the plot shows that in this \textsc{text type} \textit{claro} appears frequently in EsAdjC but also in Adv. The second split separates news from fiction and oral texts. In the latter group, AdvC makes up the overwhelming majority (cf. Node 9). In news texts AdvC is also the most frequent \textsc{construction} in which \textit{claro} appears, but there is also a substantial percentage (more than 20\%) of EsAdjC.  


\begin{figure}
\includegraphics[angle=90,height=.9\textwidth,origin=c]{figures/corpusfulltree.pdf}
\caption{Conditional inference tree with splits on \textsc{modifier}, \textsc{text type} and \textsc{language}\label{fig:corpctree}}	
\end{figure}	


\begin{figure}
\begin{floatrow}
\captionsetup{margin=.05\linewidth}
\ffigbox
	{\includegraphics[width=\linewidth]{figures/corpusvarimp.pdf}}
	{\caption{Dot plot of variable importance\label{fig:corpvarimp}}}
\ffigbox
	{\includegraphics[width=\linewidth]{figures/corpusheatmap.pdf}}
	{\caption{Heat map of predicted versus observed values; Darker  shades   correspond to larger counts.\label{fig:corpheatmap}}}
\end{floatrow}
\end{figure}	

\begin{figure}
	\subfigure[Branch 1.\label{fig:branch1}]{
		\includegraphics[width=0.5\textwidth]{figures/subtree1.pdf}}
	
	
	\subfigure[Branch 2. \label{fig:branch2}]{
		\includegraphics[height=0.48\textheight]{figures/subtree2.pdf}}
	\caption{Individual branches of the conditional inference tree in \figref{fig:corpctree} \label{fig:corptreesplit}}
\end{figure}
\begin{figure}
	\subfigure[Branch 3.\label{fig:branch3}]{\includegraphics[width=\textwidth]{figures/subtree3.pdf}}\smallskip\\
	\subfigure[Branch 4.\label{fig:branch4}]{\includegraphics[width=0.5\textwidth]{figures/subtree4.pdf}}
	\caption{Individual branches of the conditional inference tree in \figref{fig:corpctree}}
\end{figure}


In the subset of data containing all \textsc{modifier}s but \textit{claro} (cf. Figure \ref{fig:branch2}), the first split separates \textit{natural} and \textit{seguro} from \textit{cierto}, \textit{evidente} and \textit{probable} (Node 11). Within the \textit{natural}/\textit{seguro} subset, the next most important difference is based on \textsc{language}. In the subset that includes Portuguese data with \textit{natural} and \textit{seguro}, the only significant split is on \textsc{text type} (Node 18). In news texts the \textsc{modifier}s appear in all three \textsc{construction}s. They are most frequent in Adv but the differences are relatively small (Node 20). In fiction, oral and academic texts, the contrasts are bigger. Adv is by far the most frequent \textsc{construction} for Portuguese \textit{seguro} and \textit{natural} in these \textsc{text type}s (Node 19). In the Spanish subset, there is a split on \textsc{modifier} (Node 13). Text type does not play a role in the distribution of the \textsc{construction}s in which Spanish \textit{natural} appears. For Spanish \textit{seguro} there is a significant difference in  distribution depending on \textsc{text type}s. In news texts it appears in  all three \textsc{construction}s but is most frequent in Adv (Node 15). In the other three \textsc{text type}s, it only appears in Adv and AdvC (Node 16).

The first split  in the last set of \textsc{modifier}s   splits \textit{cierto}  and \textit{evidente} from \textit{probable}. In the subset containing \textit{cierto} and \textit{evidente} the first split is on \textsc{text type} (Node 21, \figref{fig:branch3}). There is a significant difference between news texts and the other three \textsc{text type}s. In news texts there are no further significant splits. The two \textsc{modifier}s appear in all three \textsc{construction}s, but by far most frequently in EsAdjC (Node 40). In the other \textsc{text type}s, there is a significant difference between the two \textsc{modifier}s (Node 23). The \textit{cierto} subset is further split on \textsc{text type} (Node 35) and  \textsc{language} (Node 37). In academic texts, \textit{cierto} appears most frequently in Adv (Node 44) in both languages. In Portuguese it appears almost exclusively as an adverb, whereas in Spanish, AdvC and EsAdjC are also found. In fiction and oral texts there is no language specific difference. \textit{Cierto} appears in all three \textsc{construction}s, but less frequently in AdvC (Node 36).  In the \textit{evidente} subset, there is a split separating academic from fiction and oral texts (Node 24). There is again  a split on \textsc{language}. In academic texts, \textit{evidente} is only found in EsAdjC in Spanish while in Portuguese it is found in all three constructions but mostly in Adv and EsAdjC. In fiction and oral texts, there is a split on \textsc{language} (Node 25). The \textsc{text type} is significant in both \textsc{language}s. Spanish \textit{evidente} appears by far most frequently in Adv in both \textsc{text type}s (Node 30, 31). In fiction texts there are no cases of the \textsc{modifier} in AdvC. In Portuguese, the \textsc{modifier} is also most frequent in Adv in both \textsc{text type}s, but the contrast between the \textsc{construction}s is less pronounced (Node 27, 28).\largerpage

Within the \textit{probable} subset (Node 21, Figure \ref{fig:branch4}), there is a significant difference in fiction and oral texts on the one hand and academic and news texts on the other hand (Node 41). In fiction and oral texts, \textsc{language} does not play a role. The \textsc{modifier} appears by far most frequently in Adv. Lastly, the subset containing the \textsc{modifier} in academic and news texts is  split on \textsc{language}. In Spanish, \textit{probable} has roughly the same frequency in Adv and EsAdjC. In Portuguese it has a far higher frequency in Adv than in EsAdjC.  There are practically no occurrences of \textit{probable} in AdvC.\footnote{There are two occurrences in Spanish, see Table~\ref{tab:corpconstmod}.} 


\subsubsection{Discussion}
The analysis shows that the \textsc{modifier} \textit{claro} has a very particular distribution that differs from  that of the other \textsc{modifier}s. In Spanish, it appears that \textit{claro} is specialized for a use in the AdvC \textsc{construction}. It is possible that Spanish \textit{claro que} has already grammaticalized or is at least in the process of turning into a fixed pragmatic marker.  \textit{Claro} has a high frequency in AdvC in Portuguese too,  but also appears in EsAdjC.  Given that the two \textsc{construction}s have a different meaning, the distribution of \textit{claro} in these \textsc{construction}s indicates that  the meaning of the \textsc{modifier}   may  have acquired different nuances in the two \textsc{language}s. In Portuguese  \textit{claro} seems to be specialized for \textsc{construction}s that take perspectives other than the speaker's into account (AdvC and EsAdjC), while in Spanish its meaning  is specialized for \textsc{construction}s that take the interlocutor's perspective into account (AdvC).


The results in general suggest a greater proximity between AdvC and EsAdjC than between AdvC and Adv, which is consistent with the idea that EsAdjC and AdvC – but not Adv – are attributive (cf. \sectref{sec:presupprag}). In the cases of Adv and EsAdjC, the wrongly predicted data points (EsAdjCs predicted as either Adv or AdvC, and Adv predicted as either EsAdjC or AdvC) are distributed relatively equally across the two  categories. In contrast, observed AdvCs are significantly more likely to be predicted to be EsAdjC (160) than Adv (12) (cf. Table \ref{tab:corppredictionsnew} repeated in \tabref{tab:corppredictions}). 

\begin{table}
	\begin{tabular}{lrrr}
		\lsptoprule
		& observed Adv & observed AdvC & observed EsAdjC \\\midrule
		predicted	Adv & 1137 & 225 & 232 \\ 
		predicted	AdvC &  12 & 790 & 160 \\ 
		predicted	EsAdjC & 201 & 163 & 409 \\ 
\lspbottomrule
	\end{tabular}
	\caption{Predicted categories \label{tab:corppredictions}}
\end{table}
 


Aside from  \textit{claro}, the other modifiers  appear most frequently in Adv. This \textsc{construction} is unmarked with respect to which speech participant the epistemic and evidential evaluation is centered on. It does not have a preferred reading but permits any type of \isi{deictic center} that is plausible in a given context. It  therefore follows that the other \textsc{modifier}s investigated in this study are less specialized for one reading.  


Apart from \textit{probable}, all the \textsc{modifier}s are attested in all  three \textsc{construction}s. \textit{Probable} is most frequent in Adv and does not appear in AdvC. This suggests that the meaning of this \textsc{modifier} does not easily adapt to a reading that takes the perspective of the interlocutor into account.

The relation between \textsc{text type} and \textsc{construction} supports the assumption that AdvC refers to an interlocutor's perspective, hence its high frequency in oral and fiction texts. EsAdjC peaks in academic and news texts, in which an authoritarian and apparently objective perspective is often employed and in which referring to the interlocutor's perspective directly is uncommon. Adv is the most common \textsc{construction} in all texts types. This supports the idea that it is unmarked and therefore adapts easily to all possible \isi{deictic center}s.

\textsc{language} plays  different roles in connection with the different \textsc{modifier}s. An interesting pattern is that the splits on \textsc{language} are at a high level of the tree for \textit{claro}, \textit{natural} and \textit{seguro}. This indicates that the meaning of the \textsc{modifier}s differs between Portuguese and Spanish. In contrast, the splits are at a deeply embedded level for \textit{cierto}, \textit{evidente} and \textit{probable},  suggesting that these three \textsc{modifier}s are less idiosyncratic. For \textit{cierto}, \textsc{language} is not a significant predictor at all suggesting that, with respect to the tested \textsc{construction}s,  the \textsc{modifier}s have the same  meaning in Spanish and Portuguese.

\subsection{Acceptability judgment experiment on Spanish}
In this section I describe the design and the results of an experiment to determine primarily the influence of the variables \textsc{deictic center} and \textsc{construction}, but also other factors,   on the acceptability of Spanish epistemic and evidential \textsc{modifier}s in their respective contexts.

\subsubsection{Corpus}
The stimuli are taken from the 2016 Web/Dialects CdE. The corpus comprises  2 billion tokens and contains data from 21 different Spanish speaking countries. The corpus was built based on data acquired using Google search from  20 million randomly sampled web pages and blogs from all 21 countries.\footnote{See \href{https://www.corpusdelespanol.org/web-dial/help/textsm.asp}{https://www.corpusdelespanol.org/web-dial/help/textsm.asp {[\today]}} for a detailed description of how the data were acquired and processed.} I chose this corpus because of its large size and its informal register that approximates orality,  which is essential when eliciting judgments on natural language pragmatics.

\subsubsection{Data acquisition and selection}
I first determined the four most frequent \isi{epistemic and evidential modifier}s in the corpus which are \emph{cierto, claro, evidente} and \emph{seguro}. For each of these, I extracted all sentence-initial occurrences of the three \textsc{construction}s under investigation. Since sentence position is not annotated in the corpus,  I  approximated this by extracting all cases of the relevant patterns following a sentence-final punctuation mark. 

The final experimental stimuli were identified through stepwise random sampling and controlled selection. The full sample from the corpus was exported to \verb|R| in order to  automatically exclude certain data points and to  draw random samples. All the occurrences of Adv followed by a punctuation mark (the adverbs in isolation can function as affirmative particles) and AdvC preceding the affirmative particle \textit{sí} or the negative particle \textit{no} followed by a punctuation mark were excluded from the sample. From the new reduced sample, I drew  random samples of 20 items per \textsc{modifier} per \textsc{construction}.  These random samples were then inspected individually. Further items had to be excluded based on my subjective selection. Items were excluded if the \textsc{construction}s did not modify declaratives, if the sentence and the contexts were not cohesive or if they contained offensive, sexual or religious content. I then drew another random sample of 3 items for each \textsc{modifier} and \textsc{construction}, yielding the final count of 36 experimental stimuli. 

\begin{table}
	\begin{tabular}{lccccc}
	\lsptoprule
		& \textit{cierto} & \textit{claro} & \textit{evidente} & \textit{seguro}\\\midrule
		Adv & 3 & 3 &3 &3 \\
		Adv-\textit{mente}C/Adv-0 C& 3/1 & 3/0 & 1/2 & 3/0 \\ 
		EsAdjC/EstáAdjC & 3/0 & 0/3 & 3/0 & 3/0\\
	\lspbottomrule
	\end{tabular}
	\caption{Derived vs. underived adverbs in AdvC in the experimental stimuli \label{tab:spall}}
\end{table}


In EsAdjC, the \textsc{modifier}s can appear with the copulas \textit{ser} and \textit{estar}. In AdvC the \textsc{modifier}s are sometimes derived and sometimes  not. The whole sample extracted from the corpus contained all the cases of each \textsc{modifier} in both varieties for each \textsc{construction}.  The random sampling resulted in the patterns given in Table \ref{tab:spall}. Neither of the two variables had a significant effect in the model.

\subsubsection{Data modification}
 The critical target sentences in the scope of the modifiers were shortened so that all  of them only  constituted  simple main sentences. The aim of this shortening was to obtain a more homogeneous set of stimuli and to reduce the number of words, thereby  rendering the experiment shorter overall. The size of the preceding context was selected individually for each target sentence. The selection was carried out based on my subjective judgment. The goal was to maintain the minimum amount of context necessary to make sense of the critical sentence and its modification. I opted for this qualitative criterion because the high heterogeneity of the fragments in the corpus both in length and content  made it impossible to apply a quantitative criterion.

Each target sentence was presented in three different conditions that introduce, or more precisely negate, certain readings of the modifiers. These conditions were created by adding a concessive clause before the critical sentence. 
Each stimulus thus consists of a complex target sentence made up of the concessive clause and the epistemically or evidentially modified sentence extracted from the corpus along with the corresponding preceding context.

\subsubsection{Experimental design}
The experiment was run on Ibexfarm. It started with two simple practice items that explained the task and illustrated how to use the scale. The practice items were followed by 36 experimental stimuli. The experiment was in a Latin square design. Each stimulus was presented to one participant in only one of the three conditions, with the stimuli presented in a random order. The participants were asked to provide a judgment on the acceptability of the concessive, which was underlined, in the relevant context. The judgment was elicited on a five-point Likert scale, with 1 translating to the lowest degree of acceptability  and 5 to the highest degree. 
The participants were instructed to provide judgments that corresponded to their intuitions regarding the naturalness and acceptability of the critical sentences. They were informed that the stimuli originated from a text corpus of an informal and therefore non-standard register. They were asked to disregard anomalies in orthography and other aspects that might not correspond to the norm. The  experiment was estimated to take 15 minutes (mean duration 13.8 minutes). No instructions were given as to whether the participants should provide their judgments  quickly or slowly.

\subsubsection{Participants}
The participants were recruited through social media by sharing the link with personal contacts and in Facebook groups of Spanish native speakers. In total 61 people participated in the experiment, 35 female and 26 male. The vast majority of them came from Spain (47), followed by Colombia (6) and Argentina (5). One participant each came from Germany, Uruguay and Mexico. The age of the participants ranged from 20 to 66 with a mean of 39.21 years. Contrary to what one might expect from a recruitment process that relies solely on social media,  the age distribution  shows that this approach can also reach older participants: 14.75\% are 60 or older. 




\subsubsection{Conditions}
In the experiment the target sentences are presented in three conditions. Each condition negates certain readings of the \textsc{modifier}. 

\ea 
\ea\label{ex:c1claramente} (\textit{yo}-\textsc{deictic center}) \\ \gll 
Aunque yo no lo crea, claramente tiene los ojos de su papá. \\
although I not it believe.\textsc{1sg.sbjv.prs} clearly have.\textsc{3sg.prs} the eye.\textsc{pl} of his father\\
\glt `Although I don't believe it, he clearly has his father's eyes.'
\ex\label{ex:c2claramente} (\textit{tú}-\textsc{deictic center})\\ \gll 
Aunque tú no lo creas, claramente tiene los ojos de su papá. \\
although you not it believe.\textsc{2sg.sbjv.prs} clearly have.\textsc{3sg.prs} the eye.\textsc{pl} of his father\\
\glt `Although you don't believe it, he clearly has his father's eyes.'
\ex\label{ex:c3claramente}  (\textit{gente}-\textsc{deictic center}) \\ \gll 
Aunque la gente en general no lo crea, claramente tiene los ojos de su papá.\\
although the people in general not it believe.\textsc{3sg.sbjv.prs} clearly have.\textsc{3sg.prs} the eye.\textsc{pl} of his father\\
\glt `Although people in general don't believe it, he clearly has his father's eyes.'
\z
\z

In the first condition (\textit{yo}-\textsc{deictic center}) the reading of the \textsc{modifier} centered on  the speaker is negated \eqref{ex:c1claramente}. In the second condition (\textit{tú}-\textsc{deictic center}), the hearer-centered reading is negated \eqref{ex:c2claramente}. In the third condition (\textit{gente}-\textsc{deictic center}), a reading where the \textsc{modifier} is centered on a more general group of people is negated \eqref{ex:c3claramente}.\largerpage

The aim of using these conditions is to determine whether the three \textsc{construction}s  make certain readings of the \textsc{modifier}s more prominent. If this turns out to be the case, we expect that the target sentences should be judged low on the acceptability scale in a condition that negates precisely this reading. The \textit{yo}-\textsc{deictic center} condition functions as a control condition. The expectation is that the speaker is always part of the \isi{deictic center} present in the \textsc{modifier}s that were selected for this experiment, because all of them imply a strong commitment of the speaker towards the truth of the proposition. The \textit{tú}-\textsc{deictic center} relates to the reading of the \textsc{modifier}s in AdvC.  Given the assumption that in this \textsc{construction} the evaluation of the \textsc{modifier} is centered on the speaker and the hearer, negating that the hearer believes \emph{p} should not be acceptable. Based on this we would expect low acceptability for AdvC in the \textit{tú}-\textsc{deictic center}. The third condition targets EsAdjC. In this \textsc{construction} the evaluation is centered on a contextually relevant group of people or to an unspecified authority. In the \textit{gente}-\textsc{deictic center}, the \textsc{deictic center}  ``the people'' is negated, hence EsAdjC should be judged lower in this condition.

\subsubsection{Description of the data and predictors}

Table \ref{tab:judgesepev}  summarizes the percentage of judgments for each value on the rating scale. There are  larger values at the extremes of the scale, suggesting that in most cases the participants had strong intuitions about the acceptability and unacceptability of the stimuli and they chose intermediate ratings, translating to unclear intuitions, less frequently.

\begin{table}
\begin{tabular}{l c c c c c}
\lsptoprule
rating & 1& 2& 3& 4& 5\\
percentage of judgments & 29.55 & 11.96 & 14.05& 18.61 & 25.83\\
\lspbottomrule
\end{tabular}
\caption{Percentage of judgments per rating\label{tab:judgesepev}}
\end{table}

There is always the potential of variation owed to the speaker or to properties of the items that are not captured by the factors one defined. In order to keep this variation in check,   the conditional inference tree and random forest  model the standardized rather than the raw judgments.

There is  considerable variation in the data by design, because the stimuli are taken from a corpus. The length (measured in number of words) is one aspect of this variation. The number of words per context ranges from 62 to 502 (mean 292). The length of the modified sentences ranges from 20 to 145 words (mean: 62). The examples in \eqref{ex:ncontextsp} illustrate two extremes in terms of context length.  In spite of these differences, the length of the contexts did not  provoke a significant effect. Furthermore, although \textsc{reaction time} is not a concern in this experiment, the correlation between \textsc{reaction time} and \textsc{item length} (sum of the length of the  context plus the length of the targets sentence) is very low (0.07).

\ea \label{ex:ncontextsp}
\ea\label{ex:ncontextspsmall} \textit{Context}: Hablante A – ¿Qué cosas tiene Erin de ti? \\`Speaker A – What  does Erin have from you?'\\
\textit{Target sentence}: Hablante B – Aunque tú no lo creas, claramente tiene los ojos de su papá. \\`Speaker B – Although you don't believe it, clearly she has her father's eyes.'
\ex\label{ex:ncontextsplarge} \textit{Context}: Hablante A – El objetivo de este seminario es reflexionar sobre aquellos conflictos del mundo (guerras, catástrofes, terrorismo) que en las primeras horas de producirse provocan un intenso seguimiento mediático pero semanas después desaparecen de las páginas de los diarios y de nuestro recuerdo. Decía Kapuscinzky que no entendía por qué los enviados especiales se iban cuando realmente en ese momento había que empezar a contar las historias... \\
`Speaker A – The goal of this seminar is to reflect on conflicts in the world (wars, catastrophes, terrorism) which, in the first hours after they take place, provoke an intense media following, but weeks after disappear from the pages of the newspapers and from our memories. Kapuscinzky said that he didn't understand why the special envoys left when it was actually at that point that it was time to start telling the stories.' \\
\textit{Target sentence}: Hablante B – Aunque yo no lo crea, es evidente que Kapuscinzky no logró convencer a los directores de los medios.' \\
 `Speaker B – Although I don't believe it, it is evident that Kapuscinzky didn't manage to convince the directors of the media.'
\z
\z

\begin{figure}[p]
\caption{Individual bar charts for judgments per \textsc{deictic center} per \textsc{construction}. The panels in the first row show the judgments for EsAdjC in the different \textsc{deictic center} conditions. The second row shows AdvC and the third row Adv.\label{fig:spgroupconst}}
\includegraphics[width=\textwidth]{figures/spgroupandconstructionhist.pdf}
\end{figure}

\begin{figure}[p]
\caption{Individual bar charts for judgments per \textsc{deictic center} per \textsc{modifier}. The panels in the first row show the judgments for \textit{seguro} in the different \textsc{deictic center} conditions. The second row shows \textit{evidente}, the third row \textit{claro} and the last row \textit{cierto}.\label{fig:spgroupmod}}
\includegraphics[width=\textwidth]{figures/spmodifierandgrouphist.pdf}
% % % \caption{Bar charts for the distribution of the judgments across different variables\label{fig:spbarcharta}}
\end{figure}


The bar charts in \figref{fig:spgroupconst}, made using the  \verb|lattice| \verb|R|-package (\citealt{Sarkar2021}),  show the percentage of judgments by \textsc{deictic center} by \textsc{construction}. The charts suggest that the participants were sensitive to the  \textsc{\isi{deictic center}}s but not to the \textsc{construction}s. This is clear from the fact that the plots on the horizontal axes show different patterns, while there is practically no difference on the vertical axes. The judgments for the \textit{yo}-\textsc{deictic center} are far lower than those for the other \textsc{deictic center}s. Roughly 60 percent of the judgments provided for the \textit{yo}-\textsc{deictic center} fall into the lowest category. The judgments for the other two \textsc{deictic center}s  have a similar distribution: They are spread across all categories and have the highest percentages at the upper edge of the scale. There is a slight difference between the \textit{tú}- and the \textit{gente}-\textsc{deictic center}. The judgments are more varied for the \textit{gente}-\textsc{deictic center}, which suggests that in some cases the participants were less certain about the acceptability of a reading negating this \textsc{deictic center} but are certain about the acceptability of the \textit{tú}-\textsc{deictic center}.  The bar charts in \figref{fig:spgroupmod} show the percentage of judgments by \textsc{deictic center} by \textsc{modifier}. They once again suggest  a \textsc{deictic center} effect. The differences between the \textsc{modifier}s are minimal.

\begin{figure}[p]
	\centering
	\includegraphics[width=\textwidth]{figures/spmodifierandconstructionhist.pdf}
	\caption{Individual bar charts for judgments per \textsc{construction} per \textsc{modifier}. The panels in the first row show the judgments for EsAdjC in the different \textsc{deictic center} conditions. The second row shows AdvC and the third row Adv.\label{fig:spbarchart}}
\end{figure}

The bar charts in \figref{fig:spbarchart} plot the percentage of judgments by \textsc{construction} by \textsc{modifier}. The charts support the idea that the participants had preferences for certain \textsc{modifier}s in  certain \textsc{construction}s. \textit{Cierto} is less acceptable in EsAdjC and most acceptable in Adv. \textit{Claro} has a high percentage of judgments at the lowest end of the scale. Interestingly, the \textsc{modifier} in AdvC has high percentages for the lowest and the highest values on the scale. This suggests that the relation between \textit{claro} and AdvC, which was strongly supported by the results from the corpus analysis, actually plays out differently in an acceptability judgment experiment.  \textit{Evidente} is most acceptable in EsAdjC and shows similar  patterns as \textit{claro} in AdvC and Adv. \textit{Seguro} has the highest percentage of  ratings at the lower extreme of the scale  in AdvC. In general it has high percentages for the lowest and the highest values on the scale. This suggests that there is no direct relation between the  \textsc{modifier} and the \textsc{construction}. 

\subsubsection{Results}

A conditional inference tree  was used to model the data. Model selection was based on the accuracy of the model and  on the plausibility and interpretability of the effects. Figure \ref{fig:spepevctree} is a visual representation of the model. Three variables are significant: \textsc{deictic center}, \textsc{age} and \textsc{country-participant} (nationality of the participants). Other  linguistic variables such as \textsc{construction} and \textsc{modifier} did not show a significant effect. 
The dot plot in \figref{fig:spepvevvarimp} shows the importance of each variable that entered the calculation of the model. It was determined using a random forest.  \textsc{deictic center} is by far the most important variable in this model.  The heat map in \figref{fig:spepevheatmap} plots the observed vs. the predicted values. The accuracy was calculated by dividing the sum of the correctly-predicted values by the sum of all the predicted values  (cf. Figure \ref{fig:spepevheatmap}). The accuracy  is moderate (38\%).\largerpage[2]

The first and most important split in the tree in \figref{fig:spepevctree}  shows  that there is a significant difference between the \textit{yo}-\textsc{deictic center} and the other two. The tree identifies complex interactions between the variables \textsc{deictic center} and \textsc{age} and \textsc{deictic center} and \textsc{country-participant}. In  the \textit{yo}-\textsc{deictic center}-group, the one speaker  from Uruguay gave significantly higher ratings than all the other speakers. In the group of data containing the \emph{tú}- and the \emph{gente}-\textsc{\isi{deictic center}} there is  an interaction with \textsc{age}. Younger participants found both readings equally acceptable, while older speakers judged  the \emph{tú}-\textsc{deictic center}  significantly more acceptable than the \emph{gente}-\textsc{deictic center}.\pagebreak

\begin{figure}
	\subfigure[conditional inference tree]{\label{fig:spepevctree}\includegraphics[width=\textwidth]{figures/sptree.pdf}}\smallskip\\
	\subfigure[variable importance]{\label{fig:spepvevvarimp}\includegraphics[width=0.6\textwidth]{figures/spepevvarimp.pdf}}
	\caption{Conditional inference tree with splits on \textsc{deictic center}, \textsc{age} and \textsc{country-participant} and a dot plot showing the importance of each variable \label{fig:spevepctree}}
\end{figure}


\begin{figure}
	\centering
	\includegraphics[width=0.6\textwidth]{figures/heatmapspepev.pdf}
	\caption{Heat map of 5 predicted and 5 observed values; Darker  shades  correspond to
		larger counts.\label{fig:spepevheatmap}}
\end{figure}

\subsubsection{Discussion}
The results show a  solid effect for \textsc{\isi{deictic center}}.  Negating the speaker's perspective is always judged low on the acceptability scale in combination with the \textsc{modifier}s chosen in this experiment. Although no significant interaction with \textsc{construction} could be detected in the model,  \figref{fig:spgroupconst} suggests that negating the speaker's perspective is even worse in AdvC.  The model also shows that there is a significant difference between the negation of the perspective of the interlocutor vs. the negation of the perspective of ``the people'' in the group of older speakers. Negating the interlocutor's perspective is more acceptable in the present experiment. The perspective of the interlocutor is active in all of the items, because all the stimuli are dialogs. My interpretation of these results is that it is easier to address a perspective that is already active, while it is more infelicitous to negate the perspective of ``the people'', if this perspective was not presented in the context.

 
The model does not support an effect of \textsc{construction} nor an interaction between \textsc{deictic center} and \textsc{construction}. One possible explanation could be that linguistic expressions of the variable \textsc{deictic center} stood  out more than the other variables.  The task required the participants to judge the acceptability of  concessives, which were underlined, in the contexts they appeared in. This means that the participants were instructed to focus on the \textsc{deictic center}, which might have led them to  disregard the other properties.  \textsc{\isi{deictic center}} is also the only strictly linguistic variable that was created by modifying the corpus data. It was  introduced in the form of concessive clauses that were not contained in the original data.  So another reason why this variable gives rise to a larger effect is potentially the fact that the concessives did not adapt easily to some contexts.  In the following experiment on Catalan, in order to counteract this issue, the  acquisition of data was carried out differently: Contexts with concessives expressing doubt or disbelief were extracted and the data were manipulated by the addition of the \textsc{modifier} and \textsc{construction}.
 
 
 
 
\subsection{Acceptability judgment experiment on Catalan}\label{sec:catepevexp}
The  Catalan experiment that will be described in this section again has the aim of determining the factors that influence the acceptability of the \textsc{modifier}s in the three \textsc{construction}s.

\subsubsection{Corpus} The stimuli were sampled from the caWac corpus, which is among the largest corpora of contemporary Catalan. It comprises 780 million tokens and was built by a web crawl from the top-level .cat domain in 2013. The fragments can be considered comparable to those used for Spanish  in their approximation of an oral register and style, because both are taken from corpora built using web data. 
\subsubsection{Data acquisition and selection} The caWac corpus does not have an online interface, but the data can be downloaded. The file contains xml code, tagging paragraphs and sentences, but no further annotation is provided. The final experimental stimuli were found through stepwise sampling, random sampling and controlled selection.

I  first split the files into subfiles that could be handled by a computer with an average RAM. I then wrote Python scripts to select fragments with  the critical sequences of words within a context of 500 words on the left and right side.  I extracted all the occurrences of sentence-initial \textit{encara que \emph{{[}0-3 intervening words{]}} sembli\textsubscript{subj}/sembla\textsubscript{ind}}  `although {[}0-3 intervening words{]} may seem/seems'. The selected sentences were chosen to approximate the concessives \textit{Encara que no t'ho creguis/Encara que a gent no s'ho cregui} `Although you/people in general don't believe it', since the target concessives were absent in the corpus. The data were then processed  in R. I kept all the items where the intervening words were a personal pronoun, the negative particle \textit{no}, the neutral pronoun \textit{ho} and combinations thereof, and discarded the rest. From the cases with zero intervening words I kept those in which \textit{sembli/sembla}  was followed by one of the following: \textit{mentida} `lie', \textit{increïble} `unbelievable', \textit{contradictori} `contradictory', \textit{estrany} `strange', \textit{impossible} `impossible', \textit{erroni} `wrong', \textit{el contrari} `the contrary', \textit{una paradoxa} `a paradox', \textit{paradoxal} `paradoxical', \textit{difícil de creure} `hard to believe', \textit{un contrasentit} `a misunderstanding', \textit{de bojos} `of crazy people', \textit{rar} `weird', \textit{un deliri} `a delirium', \textit{un tòpic} `a prejudice',  \textit{poc probable} `unlikely', \textit{una contradicció} `a contradiction', \textit{incomprensible} `incomprehensible', \textit{il·lògic} `illogical', \textit{tot el contrari} `all the contrary', \textit{que no pot ser} `that it can't be', \textit{subrealista} `unrealistic', \textit{que no pugui ser} `that it couldn't be', \textit{una incongruència} `an incongruence', \textit{una digressió} `a digression', \textit{absurd} `absurd'. The rest of the data was discarded. 

Out of the clean dataset I drew a random sample of 60 items. I manually excluded all the items that did not have a full sentence following the concessive or contained sensitive  sexual, religious or political content. From the remaining data I drew another random sample of 36 items, which constitute the final experimental stimuli. 

\subsubsection{Data modification}\largerpage
The only counterbalanced variable   is once again \textsc{deictic center}. \textsc{Modifier} and \textsc{construction} were introduced by randomly assigning  each item to one of four equal groups for the four different \textsc{modifier}s (\textit{cert}, \textit{clar}, \textit{evident} and \textit{segur}). Each of these groups was then randomly split into three equal subsets for the \textsc{construction} (Adv, AdvC and EsAdjC). The result is a set of 12 groups for all the \textsc{modifier} and \textsc{construction} combinations,  each of which contains three data points. The  \textsc{construction} and \textsc{modifier} were  added to the data at the beginning of the sentence that follows the concessive. 

Just as in Spanish,  the Catalan \textsc{modifier}s can appear with \textit{ser} and \textit{estar} in EsAdjC and with or without the derivational morpheme \textit{-ment} in AdvC. In the present experiment, this variation was added to the stimuli respecting the proportion of their distribution in the caWac corpus (cf. Tables~\ref{tab:catesser} and~\ref{tab:catment}). They were randomly assigned to the experimental stimuli. Neither of the two variables had any effect in the model.

\begin{table}
\begin{tabular}{rcccc}
	\lsptoprule
	& \textit{cert} & \textit{clar} & \textit{evident} & \textit{segur} \\\midrule
	Adv-\textit{ment} C  & 1 & 0 & 3 & 0 \\ 
	Adv-\textit{0} C & 2 & 3 & 0 & 3 \\
	\lspbottomrule
\end{tabular}
\caption{Derived vs. underived adverbs in AdvC in the experimental stimuli  \label{tab:catesser}}
\end{table}

\begin{table}
	\begin{tabular}{rcccc}
	\lsptoprule
	& \textit{cert} & \textit{clar} & \textit{evident} & \textit{segur}\\\midrule
	\textit{ser} & 3 & 2 & 3 & 2 \\ 
	\textit{estar} & 0 & 1 & 0 & 1 \\ 
	\lspbottomrule
	\end{tabular}
\caption{\textit{ser} vs. \textit{estar} in EsAdjC in the experimental stimuli\label{tab:catment}}
\end{table}


To create the two conditions, the concessives were substituted by \textit{Encara que tu no t'ho creguis} `Although you don't believe it' for the \textit{tú}-\textsc{deictic center} and \textit{Encara que la gent no s'ho cregui} `Although the people don't believe it' for the \textit{gente}-\textsc{deictic center}. The \textit{yo}-\textsc{deictic center} was not included since the experiment on Spanish had already shown that this \textsc{deictic center} is generally unacceptable with all the \textsc{modifier}s tested.


The sentence following the concessive was left at  its full length, i.e. the right context ends at the full stop. The length of the left part of the context was selected individually for each stimulus. The selection was carried out with the aim of maintaining the minimum amount of context necessary to make sense of the target sentences.
Each stimulus once again consists of a target sentence (concessive+modified sentence) and its context.

\subsubsection{Experimental design}
The experimental design was the same as in the previous experiment. It was run on Ibexfarm. It started with two practice items followed by 36 experimental stimuli. The experiment was in a Latin square design. Each stimulus was presented to each participant in only one of the two conditions, with the stimuli presented in a random order. The acceptability judgments were elicited on a ten-point Likert scale, with 1 translating to the lowest and 10 to the highest degree of acceptability. This more granular scale (with 10 instead of 5 points)  was employed to see whether this would lead to more nuanced judgments. The results, however, show that this was not the case (cf. \sectref{sec:rescatempev}).

The participants were instructed to provide judgments on the naturalness of the underlined concessives in the relevant contexts. They were also informed that the data were taken from online corpora and were asked to disregard aspects that did not correspond to the norm. The experiment was estimated to take 15 minutes (mean duration 16.8 minutes). No instructions were given as to whether the participants should provide the judgments quickly or slowly.
\subsubsection{Participants} The participants were recruited through social media by sharing the link with personal contacts and in Facebook groups for Catalan native speakers.  A total of 24 participants took part in the experiment, 15  female and 9 male. The majority of the participants came from Catalonia (16),  6 from Valencia and  2 from the Balearic Islands. The age of the participants ranges from 19 to 71 with a mean age of 38,9. The percentage of participants over the age of 60 (21\%) is even higher than in the previous experiment. 
\subsubsection{Conditions} The stimuli are presented in two conditions. Each condition negates one reading of the \textsc{modifier}. The first condition \eqref{ex:catcond1} negates the reading in which the \textsc{modifier} is centered on the interlocutor (\textit{tú}-\textsc{deictic center}). The second condition \eqref{ex:catcond2} negates the reading in which the \textsc{modifier} is centered on a more general group of people (\textit{gente}-\textsc{deictic center}). The third condition from the previous experiment, in which the \textsc{deictic center} on the speaker is negated, is not included in the present experiment. 

\ea	
\ea\label{ex:catcond1} (\textit{tú}-\textsc{deictic center})\\ \gll Encara que no t'ho creguis, és evident que aquest equip també pot perdre. \\
Although that not you-it believe.\textsc{2sg.sbjv.prs} be.\textsc{3sg.prs} evident that this team also can.\textsc{3sg.prs} lose\\
\glt `Although you don't believe it, it's evident that this team can also lose.'
\ex\label{ex:catcond2} (\textit{gente}-\textsc{deictic center}) \\ \gll Encara que  la gent no s'ho cregui, és evident que aquest equip també pot perdre. \\
Although that not  the people them-it believe.\textsc{3sg.sbjv.prs} be.\textsc{3sg.prs} evident that this team also can.\textsc{3sg.prs} lose\\
\glt `Although the people don't believe it, it's evident that this team can also lose.'  
\z
\z

The aim of the study is to determine whether the acceptability of these \textsc{deictic center}s differs depending on the \textsc{modifier} and \textsc{construction}. The experiment on Spanish provided strong evidence that \textsc{deictic center} has an effect, but no interaction between \textsc{deictic center} and \textsc{construction} or \textsc{modifier} could be identified. 
\subsubsection{Description of the data and predictors}
The judgments of the participants were provided on a ten-point Likert scale. Table \ref{tab:judgcatepev} shows the percentage of the participants' judgments per value of the scale. The judgments fall mostly at the higher end of the scale.  46.14\% of the elicited judgments are of a value of 8 or higher. This suggests that the participants found most of the stimuli felicitous.

\begin{table}
\begin{tabular}{l c c c c c}
\lsptoprule
		rating & 1& 2& 3& 4& 5 \\
		percentage of judgments & 12.68 & 9.66 & 7.6&  6.16 &  6.16 \\
	\midrule
			rating& 6 & 7 & 8 & 9& 10\\
		percentage of judgments  &3.62&7.97&13.29&13.53&19.32\\
		
\lspbottomrule
	\end{tabular}
	\caption{Percentage of judgments per rating\label{tab:judgcatepev}}
\end{table}


Again, in the statistical analysis I used standardized and not raw judgments.

Apart from the strictly linguistic variables, there are other variables found in the stimuli, that have their origins in the fact that the data were sourced from a corpus. One of them is the length of the text fragments. Length is  measured in number of words. Of the variables encoding length, only \textsc{context length} turned out to be a significant predictor in the model. The range of the variable is from 49 (cf. \ref{ex:cat49}) to 751 (cf. \ref{ex:cat751}) words per context.



\ea
\ea\label{ex:cat49} \textit{Context:} Per què les crispetes són tan cares en el cinema?\\
`Why is popcorn so expensive at the cinema?'\\
\textit{Target sentence:} Encara que la gent no s'ho creguis, clar que la resolució d'aquesta pregunta és un dels problemes recurrents que es plantejen en economia.  \\
`Although  people in general don't believe it, clearly the solution to this question is one of the current issues in economics.' 
\ex\label{ex:cat751} \textit{Context:} Imagineu-vos una empresa de serveis amb diversos centres de producció. L'empresa entra en pèrdues i el consell d'administració no té més diners per a invertir.  Ordenen al director executiu reducció de despeses.  Empresa A: es redueixen els costos d'estructura rebaixant personal improductiu.  Es fa un estudi de reorganització administrativa que estalvïi processos, i es procura produir dintre tot el que, fins aleshores, es donava a fer a empreses exteriors. Això fa augmentar la productivitat i reduir les despeses.  Empresa B: Es rebaixa la producció interna.  Es manté el personal improductiu i així es redueix la productivitat del “productiu”.  Es continua amb la mateixa gestió administrativa, i es manté el donar feina fora a altres tallers.  \\
`Imagine a company in the service sector with various production centers. The company starts to suffer losses and the board of directors doesn't have any money in order to intervene. They order the executive director to reduce expenses. Company A: The structural costs are reduced by letting go of unproductive personnel. They make a study and reorganize the administration in order to economize processes and they try to keep up production, including in the areas that used to be given to external companies. This makes the productivity increase and reduces the expenses. Company B: Reduces  in-house production. The unproductive personnel are kept and thus what is reduced is the productivity. They stick to the same administrative management and they keep on giving work to external factories.' \\
\textit{Target sentence:}
Encara que tu no t'ho creguis, segur que l'Empresa B és la sanitat catalana, aquesta és la realitat dels directors dels hospitals catalans.\\
`Although you don't believe it, surely Company B is the Catalan Health Department and this is the reality of many  Catalan hospital directors.' 
\z
\z


The participants were not instructed to move quickly or slowly through the experiment.  \textsc{reaction time} did not turn out to be a significant predictor in the model. Again, the correlation between the overall \textsc{length of the item} and the \textsc{reaction time} is small (0.18).\largerpage

The bar charts in \figref{fig:catgroupconst} plot the judgments for the \textsc{construction}s in the two \textsc{deictic center} conditions. They show  patterns that differ from those observed in the Spanish results (cf.  \figref{fig:spgroupconst}). They   suggest no  \textsc{deictic center} effect. The distribution of the data on the horizontal axes is nearly identical. There is a clear difference between the judgments provided for the lowest and highest acceptability. For EsAdjC, there is a large difference between the percentages for  the highest value and those for the lowest. This suggests that the participants were fairly certain about the acceptability of this \textsc{construction}. The contrast is even bigger in the \textit{tú}-\textsc{deictic center}. The difference between the high and the low ratings is far smaller in Adv and AdvC. 

The plots in \figref{fig:catgroupmod} show the judgments for each \textsc{modifier} in the two \textsc{deictic center} conditions. They  indicate again that \textsc{deictic center} does not have a strong effect on the judgments in the Catalan experiment. Most of the bar charts are symmetrical with high percentages at the extremes of the scale and low percentages at the center. The bar charts for \textit{segur} are very similar in the two groups. There are higher percentages at the right extreme of the scale, which means that the participants found \textit{segur} acceptable in both conditions. The pattern is similar for \textit{cert}, but here the percentages of the highest ratings are larger in the \textit{gente}-\textsc{deictic center} than in the \textit{tú}-\textsc{deictic center}. This could suggest that \textit{cert} is commonly interpreted as centered on speech participants, and therefore it is less acceptable to negate the \textit{tú}-\textsc{deictic center} than the \textit{gente}-\textsc{deictic center}. \textit{Evident} shows a very similar distribution for the \textit{gente}-\textsc{\isi{deictic center}}, so the perspective of ``the people'' does not appear to be encoded in the meaning of this \textsc{modifier}. The bar chart for the \textit{tú}-\textsc{deictic center} is practically symmetrical, which indicates that the participants are uncertain about the acceptability of \textit{evident} in a reading that negates the interlocutor's perspective.  The judgments for \textit{clar} peak at the lower extreme in the \textit{gente}-\textsc{deictic center}, while in the \textit{tú}-\textsc{deictic center} the lowest and the highest ratings have the same percentage. 


\begin{figure}
		\includegraphics[width=\textwidth]{figures/catgroupconstructionhist.pdf}
		\caption{Individual bar charts for judgments per \textsc{deictic center} per \textsc{construction}. The panels in the first row show the judgments for EsAdjC in the different \textsc{deictic center} conditions. The second row shows AdvC and the third row Adv.\label{fig:catgroupconst}}
\end{figure}
 
\begin{figure}
	\caption{Individual bar charts for judgments per \textsc{deictic center} per \textsc{modifier}. The panels in the first row show the judgments for \textit{segur} in the different \textsc{deictic center} conditions. The second row shows \textit{evident}, the third row \textit{clar} and the last row Cert.\label{fig:catgroupmod}}
	\includegraphics[width=\textwidth]{figures/catgroupmodifierhist.pdf}
\end{figure}
\begin{figure}
	\caption{Individual bar charts for judgments per \textsc{construction} per \textsc{modifier}. The panels in the first row show the judgments for EsAdjC in the different \textsc{deictic center} conditions. The second row shows AdvC and the third row Adv.\label{tab:catconstmod}}
	\includegraphics[width=\textwidth]{figures/catmodifierconstructionhist.pdf}
% % % 	\caption{Bar charts for the distribution of the judgments across the different variables\label{fig:catbarchart}}
\end{figure}




Finally, the last set of bar charts in \figref{tab:catconstmod} shows the judgments for each \textsc{modifier} in the different \textsc{construction}s. They indicate that some \textsc{modifier}-\textsc{con\-struc\-tion} combinations are preferred  (\textit{segur} in Adv, \textit{cert} and \textit{evident} in EsAdjC) while others are dispreferred  (\textit{evident} and \textit{clar} in Adv).





\subsubsection{Results}\label{sec:rescatempev}\largerpage
This section presents the conditional inference tree that was modeled on the basis of the experimental data. Model fitting and selection was  exploratory. The model is shown in 
\figref{fig:catepevctree}. There are three variables that show a significant effect: \textsc{construction}, \textsc{modifier} and \textsc{ncontext}, the length of the context measured in number of words. \textsc{deictic center} is not a significant predictor in this model. The dot plot in \figref{fig:catepvevvarimp}, based on a random forest, shows that \textsc{construction} is  the most important variable. 

The accuracy (10.14\%) is  low. This is  a result of  the way in which accuracy was calculated and the fact that the model predicts a reduced scale. It reduces the 10 values of the observed variables to just a 5 level opposition in the predicted values as can be seen in the heat map in \figref{fig:catepevheatmap}. 

The first split separates the data containing the modifier \emph{clar} from the rest. \textsc{modifier} \emph{clar} interacts with \textsc{ncontext}: A larger number of words in the context leads to a significantly higher acceptability. 

\begin{sloppypar}
Within the subset of data containing all modifiers but \emph{clar}, the model's strongest variable \textsc{construction},  splits EsAdjC from Adv and AdvC.  The conditional inference tree also reveals a three-way interaction  between \textsc{modifier}, \textsc{construction} and \textsc{ncontext}. For Adv and AdvC there is again an effect of \textsc{ncontext}. In much the same way as before, a larger number of words in the context results in higher acceptability.
\end{sloppypar}

\begin{figure}
	\subfigure[conditional inference tree]{\label{fig:catepevctree}\includegraphics[width=\textwidth]{figures/cateveptree.pdf}}
	\subfigure[variable importance]{\label{fig:catepvevvarimp}\includegraphics[width=.6\textwidth]{figures/catevepvarimp.pdf}}
	\caption{Conditional inference tree with splits on \textsc{construction}, \textsc{modifier} and \textsc{ncontext} and a  dot plot showing the importance of each variable \label{fig:catevepctree}}
\end{figure}



\begin{figure}
	\includegraphics[width=0.6\textwidth]{figures/catevepheatmap.pdf}
	\caption{Heat map of 5 predicted and 10 observed values; Darker  shades  correspond to
		larger counts.\label{fig:catepevheatmap}}
\end{figure}

\subsubsection{Discussion}\largerpage
The model  suggests that the {modifier} \textit{clar} is different from the others.  The results show that \textit{clar} is less acceptable than the other {modifier}s in all the contexts tested. Just as in the previous two studies, this suggests that \textit{clar(o)} is unusual in some respect.  In the present experiment, the highest degree of acceptability for the \textsc{modifier} is in Adv (mean judgment: 5.55), followed by EsAdjC (mean judgment: 5.25) and the lowest is  in AdvC (mean judgment: 4.91). Both conditions tested in the experiment negate the perspective of the interlocutor, which is predicted to be less acceptable with AdvC in the first place but also with the other attributive construction EsAdjC.  Finally, Adv should permit all readings of the modifier and therefore a high acceptability in all conditions is expected. This means that the judgments provided for stimuli containing \emph{clar} are in line with the theoretical assumptions.  In the corpus data \emph{clar} in AdvC is  very frequent in caWac (2303 cases), while  in comparison, it  occurs as a sentence-initial adverb (Adv) only 387 times in caWac. The highest frequency  for \textit{clar} is in EsAdjC (7410, of which 5062 contain the copula \textit{ser}), though it is possible that some of these 5062 are  actually  AdvC rather then EsAdjC. In recent years \emph{esclar} has emerged as a new evidential modifier resulting from a univerbation of the copula construction \emph{és clar} (see the discussion in \sectref{sec:presupadvc}). 
\textit{Esclar} does not yet appear in DIEC (\emph{Diccionarí de la llengua catalana de l'Institut d'Estudis Catalans}), but  has been adopted as a popular norm. It appears in the writing of some Catalan authors and is used by various newspapers, which demonstrates   the difference in function and interpretation between \textit{és clar} and \textit{esclar}.\footnote{See for instance \url{https://www.ara.cat/cronica/Aglutinacio-clar-esclar_0_444555554.html}.} There is a substantial number of cases (3579) of \textit{esclar} in the corpus, of which 33\% are cases of \textit{esclar que}. Some speakers, however, refrain from adopting the spelling \textit{esclar}.\footnote{\url{http://www.elpuntavui.cat/article/7-vista/23-lectorescriu/201549-esclar-o-es-clar.html}.} One can therefore assume that a number of the occurrences of the evidential adverb \textit{esclar} still appear in the more conservative spelling \textit{és clar} in the caWac corpus.  Finally, it is also possible that some of the participants in the experiment interpreted the sequence \textit{és clar}, intended as EsAdjC, as cases of AdvC.

The positive effect of context length on the judgments suggest that more information facilitates the interpretation of the epistemic and evidential \textsc{modifier}s. Or, conversely, it could mean that epistemically or evidentially modified statements are bad in out-of-the-blue contexts. 

In the present experiment the model  predicts fewer values for the response variable than the original scale. It is not clear whether this means that the par\-ti\-ci\-pants were  certain about the (un)acceptability  of the stimuli, making the intermediate levels obsolete, or whether the intermediate levels were not interpretable for the participants.

The  moderate fit of the model and the low number of significant variables is  also due to  the low number of participants. Attempts to reach a larger number of participants in a reasonable amount of time unfortunately failed.

Neither of the two experiments provided results that support direct interaction between \textsc{construction} and \textsc{deictic center}. The expectation based on the meaning of \textit{que} in AdvC would have been that this \textsc{construction} should be dispreferred in the \textit{tú}-\textsc{deictic center}, which denies that the interlocutor shares the evaluation. EsAdjC should have been dispreferred in the \textit{\textit{gente}}-\textsc{\isi{deictic center}}, which negates the perspective of a contextually-relevant group of people. A relation between the \textsc{deictic center} and the \textsc{construction} is however  supported by the corpus data. This shows that there is a striking mismatch between production and comprehension. While there might be a preferred \textsc{deictic center} for the \textsc{modifier}s in each \textsc{construction}, which is suggested by the corpus data, other readings can easily be accommodated, which makes it hard to isolate them in an experimental setting.   One of the findings is therefore that other methods that take into account the possibility of accommodation need to be developed in order to test the preferred \textsc{deictic center}s.


\section{Bias in polar questions}\label{sec:expbias}
This section focuses on  \emph{que}-initial \isi{polar question}s in Catalan. A prediction of the analysis in \sectref{sec:presupprag} is that the presence of attributive \emph{que} expresses  that the speaker is \isi{bias}ed and expects a positive answer from the hearer. The motivation behind the empirical investigation that I will describe here, is to determine how the presence and absence of \emph{que}, along with other linguistic and contextual factors, impact the \isi{bias}.

Although \emph{que}-initial \isi{polar question}s are also attested in Spanish (cf. \sectref{sec:presuppqC}), I have only focused  on Catalan here. Corpus data showed that \emph{que}-initial questions are allowed in seemingly the same contexts as in Catalan. However, unlike  Catalan, this feature of Spanish grammar does  not appear in descriptive grammars of the language, suggesting that there is little awareness of this construction. This could distort the results of an experiment that relies on written stimuli, because, even if the participants use  \emph{que}-initial questions in oral registers, they might not accept  them in written form.


\subsection{Corpus}
Just as in the experiment described in \sectref{sec:catepevexp}, the stimuli were sampled from the caWac corpus, which comprises 780 million tokens and was built through a web crawl. 
\subsection{Data acquisition and selection}
The caWac corpus if freely available to download. Apart from sentence and paragraph splits, the data  contain no  annotations. The final experimental stimuli were again found through stepwise sampling, random sampling and controlled selection.  

The preparation of the dataset and the sampling was carried out using Python. I extracted all the questions (defined as sentences ending in a question mark) and their preceding 400 words from the corpus. I then split the dataset containing all the questions introduced by \emph{que}  from the rest of the data. The remaining data were further reduced by discarding \textit{wh}-questions (defined as questions starting with an interrogative pronoun). 
I drew a random sample of 30 items from each dataset. The random samples were then  manually checked and items were discarded  if  they constituted negative questions, were  not formally complete, contained sensitive content or if they lacked cohesion with the previous context. From the remaining data I drew a random sample of 20 items per dataset, which  constitute the experimental stimuli.
\subsection{Data modification}
The experiment contains two counterbalanced variables that were introduced through modification of the corpus data. The first is the presence and absence of sentence-initial \emph{que}. This variable was created by preparing two versions of each stimulus, one with and one without the complementizer. 
The way this variable is set up makes it possible to investigate the import of \emph{que}  isolated from the corresponding question and the  context in which it appears. For every question that  originally contained \emph{que}, there is a also \emph{que}-less counterpart in the experimental stimuli and vice versa. 

\begin{sloppypar}
The second counterbalanced variable  was created by complementing each item with a ``yes'' or ``no'' answer to the question. The answer was not just the answer particle on its own, but repeated the prejacent (cf. \ref{ex:catmodification}). The motivation was to prevent ambiguity with syntactically complex questions where a simple particle answer could be interpreted as affirming or negating either the main or the embedded content. 
\end{sloppypar}

\ea\label{ex:catmodification}
\textit{Context:} \\
A  – Estàs pensant en marxar a estudiar a l'estranger però en tens alguns dubtes?  \\
`A – Are you thinking about leaving to study in a foreign country but you have some doubts?'\\
\textit{Target Question:}\\
Version 1: A –  Que t'agradaria estar més informat sobre els programes de mobilitat? \\
Version 2: A –  T'agradaria estar més informat sobre els programes de mobilitat? \\
`A – Would you like to be more informed about mobility programs?'\\
\textit{Answer:}\\
Version 1: B – Sí, m’agradaria.\\
`B – Yes, I would like to be.\\
Version 2: B – No, no m’agradaria.\\
`B – No, I wouldn't like to be.\\
\z



Each stimulus is a short dialog that consists of the context, the target question asked by one speaker and an answer provided by another speaker. The context is either  the sentences uttered by the first speaker before asking the question or an interaction between  two or more speakers. The contexts were  again chosen by qualitative rather then quantitative criteria with the aim of maintaining the minimum number of words necessary to make sense of the question.

\subsection{Experimental design}
The experiment was run online on Ibexfarm. It started with three practice items and was followed by 40 experimental items. The experiment is in a Latin square design, which means that each participant was only presented with each stimulus in one of the four conditions. The stimuli were presented in a random order. The participants provided  judgments on a five-point Likert scale, with 1 translating to the lowest degree and 5 to the highest degree.  The task was to judge the expectedness of a ``yes''/``no'' answer taking into consideration the form of the question and the context. In a pilot phase I tested two versions of the experiment. In one of them, the participants were asked to judge the degree of expectedness and in the other to judge the degree of surprise. In the pilots, the participants showed more difficulty in  judging the degree of surprise than the degree of expectedness, which is why I settled on the second option.  The mean duration of the experiment is 18 minutes.

\subsection{Participant}
A total of 46 native Catalan speakers participated in the experiment. Recruitment through social media, which I relied on in the previous experiments, was less successful in this case (it only resulted in the recruitment of 12 participants). The additional 34 participants were hence recruited via Prolific,\footnote{\href{prolific.co}{https://www.prolific.co/}} a platform similar to Amazon Mechanical Turk, which enables researchers to recruit participants online and compensate them for their time. The payment was between \pounds 1   and \pounds 3. The number of participants is relatively modest given the complexity of the experimental design. Although the experiment was online for several months and I distributed the link widely, a larger number of participants could not be achieved, in part because the number of native Catalan speakers registered on Prolific is very small.  There are 21 female and 25 male participants. The vast majority (42) are from Catalonia and 2 each are from the Balearic Islands and Valencia. The  age ranges from 18 to 61 years (mean\,=\,31). There is  only one participant above the age of 60, and 8.6\% are above the age of 50. This means that the participants are younger than in the other experiments. This is probably because the bulk of the recruitment was carried out via Prolific compared to the Facebook-based recruitment  in the other two experiments. Prolific attracted a younger group of people, while Facebook has members of all age groups, and the older cohorts proved to be particularly active in the Facebook groups I relied on for distributing the experiments.


\subsection{Conditions}
The stimuli are presented in four conditions resulting from the combinations of the two counterbalanced variables \textsc{presence of que} and \textsc{answer} (cf. Table \ref{tab:excatbiascond}).  

\begin{table}
	\centering
	\begin{tabular}{l c c }
\lsptoprule
		& \emph{que} & no \emph{que}\\
\midrule
		affirmative answer	&  C1 & C2 \\
		negative answer & C3 & C4 \\
\lspbottomrule
	\end{tabular}
\caption{Variables and conditions\label{tab:excatbiascond}}	
\end{table}
There are four versions of each item. This is illustrated in \tabref{tab:excatbiascondex} for the stimuli in \eqref{ex:catmodification}. In condition 1, the question is introduced by \emph{que} and the answer is affirmative. In condition 2, the question is not introduced by \emph{que} and the answer is also affirmative. In condition 3, the question is introduced by \emph{que} and followed by a negative answer. In condition 4, the question is not introduced by \emph{que} and the answer is negative. 

\begin{table}
\begin{tabularx}{\textwidth}{l X X }
	\lsptoprule
	& \emph{que} & no \emph{que}\\
	\midrule
	affirmative answer	&  {C1}\newline A –  Que t'agradaria estar més informat sobre els programes de mobilitat?\newline   B – Sí, m’agradaria. 		 &  {C2}\newline  A –  T'agradaria estar més informat sobre els programes de mobilitat?\newline   B – Sí, m’agradaria. \\
	negative answer  &  {C3}\newline   A –  Que t'agradaria estar més informat sobre els programes de mobilitat?\newline  B – No, no m’agradaria. &  {C4} \newline A –  T'agradaria estar més informat sobre els programes de mobilitat?\newline  B – No, no m’agradaria.\\ 
 \midrule
 
 translation & \multicolumn{2}{>{\hsize=\dimexpr2\hsize+2\tabcolsep+\arrayrulewidth\relax}X}{ A  –  (\textsc{que}) Would you like to have more information about mobility programs?  \newline B - Yes, I'd like to. / No, I wouldn't like to.} \\
	\lspbottomrule
\end{tabularx}	
\caption{Exemplified conditions \label{tab:excatbiascondex}}	
\end{table}


The motivation for introducing \textsc{answer} as a variable is to investigate the \isi{bias} that results from different configurations. The expectation based on the theoretical analysis is that questions introduced by \emph{que} express that the speaker expects a positive answer. This means that C1, where a \emph{que}-question is answered affirmatively, should lead to high acceptability judgments, reflecting the fact that a positive answer is expected. However, in C3, where the answer is negative, we should expect a low degree of expectedness on the part of speaker A, which should translate to low acceptability. In C2 and C4,  the \emph{que}-less questions are neutral and should not per se indicate that the speaker has a \isi{bias} towards a positive or negative answer.   C2 and C4 should not differ because the form of the question does not encode a speaker \isi{bias}, and so neither a positive nor a negative answer should be expected.   I should stress that these are the outcomes that could be expected in a highly controlled hypothesis-driven experiment, but the exploratory design that I have employed here is unlikely to work in the same way.  Given  that the stimuli are taken from naturally occurring data with  inherent variation, the present experiment  allows for the possibility that further properties, such as  word order, the presence of  modifiers or  contextual factors,  can influence the judgments and give rise to different readings of the questions. 

These factors and their interplay with the complementizer can be tested because every target question appears with and without \emph{que}. This means that we can investigate whether the differences between the tested questions go beyond the presence or absence of \emph{que}. In turn, this allows us to study, for instance, whether an original \emph{que}-question remains \isi{bias}ed even when \emph{que} is not present.

\subsection{Description of the data and predictors}

The experiment again elicits acceptability judgments on a five-point Likert scale. As in the previous experiments, the largest values are at the extremes of the scale. In this case, they are skewed towards to the top end (cf. \tabref{tab:judgcatbias}). 


\begin{table}
	\begin{tabular}{l c c c c c}
		\lsptoprule
rating & 1&2&3&4&5\\
percentage of judgments & 23.21 & 12.45 & 12.61  & 13.75 & 37.99 \\ 
		\lspbottomrule
	\end{tabular}
\caption{Percentage of judgments per rating}\label{tab:judgcatbias}
\end{table}

The response variable in the model discussed below is a standardized version of the raw judgments.

The participants were instructed to choose the value   5 if the answer to the question was completely expected by the speaker asking the question. The extreme degree to which this value was chosen, however, could indicate that the participants did not stick to the instruction. It is possible that they used  value 5 to express that the answer was not-unexpected, grouping together cases that are truly expected and cases where there were no prior expectations. 

The stimuli are drawn from corpus data and are hence heterogenous by design. There is a large degree of variation in \textsc{context length} (between 19 and 1003, mean\,=\,536), but this did not turn out to be a significant predictor. Moreover, the correlation between \textsc{reaction time} and \textsc{stimuli length} (sum of context, target sentence and answer) is low (10.2\%). Neither \textsc{reaction time} nor \textsc{stimuli length} is a significant predictor in the models.

Different predictors were considered for the statistical model. 
The counterbalanced variable, \textsc{presence of que}, is not a significant predictor. The variable \textsc{answer} is significant. The bar charts in \figref{fig:catqueans} show the percentage of the raw judgments by \textsc{presence of que} by  \textsc{answer}. The charts suggest that we might find a positive \isi{bias}, i.e. an expectation of a positive answer, with both types of \isi{polar question}s. The expectation of a positive answer appears to be stronger in the absence of \emph{que}. The unexpectedness of a negative answer, i.e. the percentage of judgments of  value 1 for negative answers,  point in a direction predicted by the theory: Although negative answers are judged to be unexpected in more than a quarter of  cases for both types of \isi{polar question}s, the percentage of these judgments is higher when \emph{que} is present (28.7\%). 

\begin{figure}[p] 
	\caption{Individual bar charts for judgments per \textsc{presence of que} per \textsc{answer}. The panels in the first row show the judgments when the answer is positive depending on the presence of \emph{que}. The second row shows the judgments when the answer is negative.\label{fig:catqueans}}
	\includegraphics[width=\textwidth]{figures/queanswerhist.pdf}
\end{figure}

\begin{figure}[p]
 	\caption{Individual bar charts for judgments per \textsc{original} per \textsc{answer}. The panels in the first row show the judgments when the answer is positive depending on whether or not the original version contained  \emph{que}. The second row shows the judgments when the answer is negative.\label{fig:catorigans}}
 	\includegraphics[width=\textwidth]{figures/origanswerhist.pdf}
% % % % 	\caption{Bar charts for the distribution of the judgments across the different variables\label{fig:catchart1}}
\end{figure}

The observed contrasts are even more pronounced in the bar charts in \figref{fig:catorigans}, which plot the percentages of  judgments per \textsc{answer} by \textsc{original}, instead of \textsc{presence of que}. The variable \textsc{original} has two values: \textsc{original-que}, if the original question contained \emph{que} and \textsc{original-noque}, if it did not. This variable is a significant predictor in the model described below. The bar charts indicate that the positive \isi{bias} is most pronounced for contexts with questions that did not originally contain \emph{que}. However,  negative answers are also judged highly in \textsc{original-noque} contexts. This might suggest again that the participants employed  value 5 if they considered the answer not-unexpected. The positive answer also appears highly expected in \textsc{original-que} contexts. The high degree of unexpectedness of a negative answer in the \textsc{original-que} contexts is once again notable.

The bar charts in \figref{fig:catorigque} plot the percentage of judgments per category by  \textsc{presence of que} by \textsc{original}. They show once more that \textsc{original} has a stronger impact than \textsc{presence of que}: The contrast between the two bar charts on the vertical axis is greater than that between the two bar charts on the horizontal axis. 


\begin{figure}[p]
	\includegraphics[width=\textwidth]{figures/origquehist.pdf}
		\caption{Individual bar charts for judgments per \textsc{presence of que} per \textsc{original version}. The panels in the first row show the judgments when the answer is positive depending on whether or not the original version contained  \emph{que}. The second row shows the judgments when the answer is negative.\label{fig:catorigque}}
\end{figure}

There are three further variables that turned out to be significant predictors in the model. The first  is called \textsc{realspeaker} and  was created post hoc. It has the value \textsc{real} if the interlocutors are introduced by a proper name, and \textsc{not real} if they are encoded as ``speaker A'', ``speaker B''. When creating the stimuli, I used the latter  in the cases where  no interlocutors were addressed directly, which  was the majority of the examples. In the contexts where the interlocutors were introduced with proper names, I adopted those.   



The second variable is termed \textsc{bias marking}. It has three values: \textsc{none}, if there is no marking, \textsc{wo} for word order, if the subject-verb inversion typical of un\isi{bias}ed \isi{polar question}s is not observed, and \textsc{mod} for modal, if the question contains modal expressions. Both of these properties can give rise to a bias. In \eqref{ex:wo} a stimulus carrying the value \textsc{wo}  is illustrated. The word order is that of a declarative and not that of a \isi{polar question}. In the pilot phase of the experiment, informants suggested that this word order is used when expressing a positive \isi{bias}.\largerpage[2]

\ea\label{ex:wo}
(\textsc{original-noque})\\
\gll 
(Que) l'Asia parla igual català que polonès?\\
\textsc{que} {the Asia} speak.\textsc{3sg.prs} same Catalan as Polish \\
\glt `Asia speaks Catalan and Polish equally well? 
\z

Similar suggestions were made for modal expressions. In \eqref{ex:mod}, the modal adverb \emph{potser} could itself signal a \isi{bias}.

\ea\label{ex:mod} (\textsc{original-que})\\ \gll  (Que) potser s' hagués pogut reconduir la cosa?\\
\textsc{que} maybe \textsc{cl.refl} \textsc{aux.3sg.sbjv.pst} can.\textsc{ptcp} re-route the issue\\
\glt `Could the issue maybe have been resolved otherwise?'
\z

The distribution of the values in the original data does not suggest a clear preference for these alternative  \isi{bias} markings in \isi{polar question}s with and without \emph{que}. The majority (9 \textsc{original-noque}, 12 \textsc{original-que}) of cases in the original data did not have a special \isi{bias} marking. Non-neutral word orders were found in 4 examples for each group and modal expressions were found in 7 \textsc{original-noque} and 4 \textsc{original-que} questions.

The last variable is \textsc{dialog}. Its values encode the number of extra interlocutors present  in addition to the addressee. The values are 1, 2 and 3.


\subsection{Results}

The model described in this section, like the previous models, was established through exploratory selection based on objective measurements and the theoretical plausibility of the model's effects. The conditional inference tree (\figref{fig:biastree}) shows significant effects of the variables \textsc{original}, \textsc{answer}, \textsc{real speaker}, \textsc{dialog} and \textsc{bias marking}.  

The dot plot in \figref{fig:biasvarimp} plots the variable importance calculated with a random forest model. It shows that \textsc{original} and \textsc{answer} are the strongest predictors.  The variables \textsc{real speaker}, \textsc{bias marking} and \textsc{dialog} play a minor role. The importance of the variable \textsc{presence of \emph{que}}, which also entered the calculation, is virtually non-existent. 


The heatmap in \figref{fig:biasheatmap} plots the observed vs. predicted values. The model predicts more categories than were observed.  This again has  an effect on the accuracy, which is relatively low (16\%) for this model.

\begin{figure}[ph]
	\subfigure[conditional inference tree]{\label{fig:biastree}\includegraphics[width=\textwidth]{figures/catquetree.pdf}}
	\subfigure[variable importance]{\label{fig:biasvarimp}\includegraphics[width=0.6\textwidth]{figures/catquevarimp.pdf}}
	\caption{Conditional inference tree with splits on \textsc{original}, \textsc{answer}, \textsc{realspeaker}, \textsc{dialog} and \textsc{bias marking} and a dot plot showing the importance of each variable.}
\end{figure}

\begin{figure}[h]
	\centering
	\includegraphics[width=0.6\textwidth]{figures/catqueheatmap.pdf}
	\caption{Heat map of  8 predicted values versus 5 observed values; Darker  shades  correspond to
		larger counts.\label{fig:biasheatmap}}
\end{figure}

The highest split in the model in Figure~\ref{fig:biastree} on the most important variable \textsc{original}, shows that there is a significant difference between the contexts in which the original version of the target question contained  \emph{que}  and those in which it  did not.   The judgments for the \textsc{original-que} contexts are significantly lower. The effect  is independent of whether \emph{que}  is actually present in the stimulus or not and also independent of what the answer is. 
The model shows complex three-way interactions. In the stimuli that  originally contained no \emph{que}, there is a significant difference between positive and negative answers. Both answers are expected, but when the answer is positive, the variable \textsc{real speaker} plays an additional role. The positive \isi{bias} is significantly stronger when the speakers are encoded by proper names.  In the stimuli that originally  contained \emph{que}-initial \isi{polar question}s, \textsc{answer} is also the most important predictor. In the contexts where the answer is ``yes'', the number of interlocutors plays a role. The positive answer is significantly less expected when there are three interlocutors interacting. When the answer is negative the presence of additional \isi{bias} marking boosts the expectation of the negative answer. Node 13 shows that a negative answer to a question without additional \isi{bias} marking is unexpected in the \textsc{original-que} contexts.

\subsection{Discussion}
The model shows a solid effect for \textsc{original}. The main insight that I draw from these  results is that  the differences between \isi{bias} in  \isi{polar question}s goes far beyond the mere presence or absence of \emph{que}.  The variable \textsc{original} has no obvious direct expression in the stimuli. I therefore conclude that the effect must be a result of multiple factors.  The post hoc variables (\textsc{bias marking}, \textsc{dialog}, \textsc{real speaker}) showed some  effects. However, the low accuracy and the fact that \textsc{original} remains the strongest effect suggest  that there are further factors yet to be discovered in the contexts.
Another novel insight comes from the importance of the variable \textsc{answer}. The results suggest that there is a general positive \isi{bias} irrespective of the type and properties of the \isi{polar question}s.

My interpretation of the  importance of the variable \textsc{real speaker} is that the use of proper names strengthens existing effects and can therefore be viewed as a tool to create more natural stimuli. In general, the data used as a basis here were not optimal because the caWac corpus is not made up of  dialog data. Since there are no Catalan dialog data freely available, there was no other choice but to work with these imperfect data. However, the use of proper names, appears to be a reasonable  means of achieving  more authentic stimuli in the future.


The  model shows that in \textsc{original-que} contexts with no \isi{bias} marking, the negative answer is unexpected. If a negative answer is unexpected, it conversely  suggests that the speakers expected  a positive one. Following this reasoning, the results can be taken to suggest that the contexts in which \emph{que}-initial questions normally appear might in fact carry a positive \isi{bias}. This is  in line with the theoretical analysis I presented in \sectref{sec:presupprag}.

\section{General discussion}\label{sec:empgeneraldiss}

To conclude this chapter, I would like to summarize a number of issues that I faced in the empirical investigations described here. I begin with problems that arise from the nature of the languages under investigation and then turn to challenges related to the empirical methods employed. Finally, I describe how my view on the relation between theory and empirical research was informed by the results reported in this chapter.

One  challenge in all three experiments was the recruitment of participants. It was extremely time consuming and despite the long process involved, I only managed to achieve a relatively moderate number of participants. This was particularly true of the experiments on Catalan. Prolific or other platforms that pool potential participants and allow them to be compensated  could become  useful tools. In the present study, however,  even relying on Prolific did not allow me to reach enough  Catalan speakers. While the numbers of Spanish and Portuguese speakers registered on the platform are definitely higher,  attempting to focus on  speakers of a particular Spanish or Portuguese variety will likely result in similar problems.

Another issue related to the languages being studied here is the fact that the publicly available resources are somewhat limited. This resulted in  a database that was not optimal for the experiments. The results  clearly showed that, while corpus data in general are surely a very useful basis for pragmatic experiments, it is not the case that all types of corpus data are adequate. On the contrary, each  phenomenon requires  specific types of data. The corpora I used for the experiments were web-based. These data represent an informal register and carry oral traits,  which is necessary for a study of the pragmatics of phenomena typical of oral speech. The dataset proved somewhat imperfect nonetheless. In particular, the last experiment would have benefited greatly from  true dialog data. The caWac corpus is very large. This allows quantitative analysis to be carried out even for less frequent phenomena. The usefulness of the corpus is limited, however, because it does not contain metadata on the fragments or authors. This means that studying  variation linked to extra-grammatical factors  is simply not possible on the basis of the available data. The same is true for the web-based parts of the CdE and CdP that I also relied on in this monograph.  They do contain information on the country of origin of a fragment. However, these are  taken from the domain of the homepage, which is insufficient to identify the country of origin of the author of a posting.  


Turning now to the experiments themselves where the tasks appear to have provoked some  difficulties for the participants. In the first two experiments, the focus of the participants might have been drawn to the parts of the stimuli that introduced the condition rather than to the  sentences containing the target construction. In the last experiment there are indications that the participants used the scale in a different way  than was intended and not as it was explained in the practice phase of the experiment. The tasks I designed were complex and went beyond simple acceptability. The problems I faced show the need  to test and develop further tasks  that  allow the investigation of pragmatic properties and are accessible to participants.  It also shows that attentiveness and sensitivity in the analysis and interpretation of patterns in the results  are an indispensable precondition for drawing conclusions even from imperfect data.

The previous point also touches on the issue of the appropriate means of measuring elicited judgments. I employed a scale, because an ordinal response variable allows for  a larger range of statistical modeling than if the elicited judgments were nominal. The  experiments carried out here, however, have also shown that there is a need to be aware of certain pitfalls. In all the experiments,  the scales were underused, which resulted in models that predicted a different number of categories than were observed. Although this can be a telling result, it also restricts the data modeling possibilities and makes it hard to determine the goodness of fit of models based on the accuracy of the predictions. 

Finally, it is important to determine how the empirical method used and described here can inform  theory. The  approach was exploratory, meaning that  I did not set out to confirm or reject a particular theory, but to broaden my understanding of the phenomenon.  In my view, the method proved successful in achieving this goal. For instance, one interesting result from the first set of investigations on different \isi{deictic center}s in AdvC and other constructions is the mismatch between production (which is in accordance with the theoretical predictions) and perception (which does not confirm them). These results suggest that while there are preferred readings, it is always possible to accommodate  other interpretations. A novel insight from the experiment on the Catalan \isi{polar question}s is that it is not sufficient to consider only the presence or absence of \emph{que} when trying to understand the \isi{bias} involved. In fact, it seems that it is the context in which each type of question appears that has the greatest predictive power with regard to the \isi{bias}. These results would not have been possible if the empirical approach had been more traditional and had not relied on corpus data. 
