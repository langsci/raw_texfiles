
\chapter{The syntax and pragmatics of commitment-attribution}\label{sec:presupint}
This chapter explores the syntactic behavior and  discourse contribution of the complementizer in the constructions exemplified in \eqref{ex:referentialque}. 

\ea \label{ex:referentialque}
	\ea\label{ex:referentialquea}
			Spanish\\
	\gll Tranquilo, tío.  Que no muerdo. \\
		calm man  \textsc{que}  not bite.\textsc{1sg.prs}\\
		\glt `Chill, man. I don't bite.'
		\ex\label{ex:referentialqueb}
				Spanish\\
		\gll ¿Que no muerdes? \\
		\textsc{que} not bite.\textsc{2sg.prs}\\
		\glt `You don't bite?'
			\ex\label{ex:referentialquec}
			Portuguese\\ 
			\gll Que chato que é. \\
		how annoying \textsc{que}  be.\textsc{3sg.prs}\\
		\glt `How annoying this is.'
			\ex\label{ex:referentialqued}
			Catalan\\
			\gll Segur que son amics. \\
		sure \textsc{que}     be.\textsc{3pl.prs} friend.\textsc{pl}\\
		\glt `Surely, they are friends.'
		\ex\label{ex:referentialquee}
		Catalan\\
		\gll Sí que son amics. \\
		\textsc{\isi{verum}} \textsc{que}   be.\textsc{3pl.prs} friend.\textsc{pl}\\
		\glt `They \textsc{are} friends.'
	\z
\z

In \eqref{ex:referentialquea}, the complementizer appears sentence-initially in a declarative and in a \isi{polar question} in \eqref{ex:referentialqueb}. In \eqref{ex:referentialquec}, the complementizer surfaces below the \textit{wh}-ex\-pres\-sion of a \is{wh-exclamative@\textit{wh}-exclamative}\textit{wh}-exclamative. In \eqref{ex:referentialqued}, for which I use the term AdvC, it follows an epistemic modifier\is{epistemic and evidential modifier} and in \eqref{ex:referentialquee}, which I will call AffC, the complementizer follows the  particle \emph{sí} in a \isi{verum} construction.



\ea \label{ex:gras9}\judgewidth{C:}
Spanish (adapted from \citealt[121: ex 9]{Gras2016})\footnote{The original example contained additional encoding relevant for conversational analyses but not for the current purpose.}\\\relax[Context: Family conversation. B and C are
married. They’re discussing where to invest their money. Bancaja is a local
bank in Valencia, Spain.]

\exi{}[C:]{\gll antes de sacarlo de la Bancaja preguntaré  si me dan más lo dejoo en la
Bancaja […]\\
before to take.\textsc{cl.m.sg} of the Bancaja {will ask} if me give.\textsc{3pl.prs} more it leave.\textsc{1sg.prs} in the Bancaja\\}
\exi{}[B:]{\gll ¿la Bancaja? que no conocemos a nadie ahora te vas a dar de...\\
the Bancaja \textsc{que} not know.\textsc{1pl.prs} \textsc{acc} \textsc{nobody} now  you go to give of  \\}
\exi{}[C:]{\gll  ¡que conozco yo al director!\\
 \textsc{que} know.\textsc{1sg.prs} I the manager\\}
\glt `C: Before I take it out from Bancaja I will ask them if they give me more
 interest if I leave it in Bancaja. B: Bancaja? We don’t know anybody now you’re going to fall flat on... C: I know the manager!'\\(Val.Es.Co. VC.117.A.1:  2–15.)
\z 

In \eqref{ex:gras9} two cases of \emph{que}-initial declaratives are illustrated in conversational data. In this fragment, both speakers use \emph{que}-sentences to introduce information that they think the other speech participant should have already been  aware of. This becomes clear in this context where a couple discuss where to invest their money. In speaker B's  \emph{que}-initial sentence she says that \emph{no conocemos a nadie} `we don't know anyone'. Through the use of the first person plural, speaker B establishes  a common perspective between herself and  speaker C, and she states that they share the knowledge she is presenting.  Speaker C  reacts with another \emph{que} sentence, saying that he does in fact know the manager of the bank. The omission of \emph{que} in both of these sentences would not result in ungrammaticality, but it does have a clear effect on the meaning. Both speakers present  information not as new, but as something each of them considers to be part of the shared \isi{common ground}. The conversational effect of \emph{que} is that the disagreement between the two speech participants is  highlighted.  Not only do they disagree on the facts: Speaker C believes they do not know anyone working at that particular bank whereas speaker B says he does know the manager. In addition, each of them also believes that the other should have known his/her belief to be true.


 Data like these are interesting in the light of the discussion regarding where   the boundary lies  between syntax and pragmatics. Once again the complementizer does not exhibit its prototypical function, but its presence has a distinct pragmatic effect.  

In the various sections of this chapter, I will argue that in all of these constructions  the complementizer is merged at the rightmost edge of the left periphery and reaches its surface position through head movement. The merge position is therefore distinct from  the position that hosts subordinate \emph{que} involved in \emph{que}-initial reportatives in my analysis.  Furthermore, I defend the idea that the low-merged complementizer makes the same  contribution to the discourse in all the constructions. The effect is that a \isi{commitment} to the proposition in its scope is ascribed to the hearer. This type of meaning has been termed \emph{attributive} by \citet{Poschmann2008}, which is reflected in my decision to use the  label \emph{attributive} \emph{que} (see \sectref{sec:presupprag} for a more detailed description of the meaning involved).

This chapter is organized as follows: In \sectref{sec:presupeval}, I discuss the previous analyses that have been proposed in the literature for the  constructions containing attributive \emph{que}. In \sectref{sec:presupanal}, I outline my own analysis and compare it to previous approaches. In \sectref{sec:presupsyn}, I report the empirical evidence behind the syntactic analysis and discuss further syntactic properties of the individual constructions. Finally, \sectref{sec:presupprag} focuses on the discourse contribution of attributive \emph{que} and the different nuanced effects that it creates in the constructions under discussion. 


 


\section{Previous analyses}\label{sec:presupeval} 
The principal idea of this chapter is that the different constructions exemplified in \eqref{ex:referentialque} all involve an instance of the complementizer merged in the same position and valued with the same feature. This is a novel insight: To date, a connection between all these constructions has not been identified  in the literature; consequently, the analyses that I discuss in the present section do not account for the whole phenomenon at once but  deal  separately with the individual constructions. While I mention all of the most important analyses, I only provide the finer details of those that are most comparable in terms of theoretical assumptions with the analysis developed in this book. This section first discusses the accounts for \emph{que}-initial declaratives with a non-reportative interpretation, followed by a review of how  \emph{que} in \isi{polar question}s has been treated in the literature. Then I present the previous analyses of \emph{que} in \is{wh-exclamative@\textit{wh}-exclamative}\textit{wh}-exclamatives. Finally, I summarize the main proposal  for \emph{que} when it follows \isi{epistemic and evidential modifier}s and  for \emph{que} in \isi{verum} marked sentences.

\subsection{\citet{Corr2016}}
The most extensive study of \emph{que}-initial sentences with a non-reportative meaning is found in \citet{Corr2016}. The author distinguishes between two types:  exclamative \textsc{que} and conjunctive \textsc{que}. The distinction is made  on an interpretive and structural level.  Exclamative \textsc{que} is illustrated in \eqref{ex:exclcorr}. 

\ea \label{ex:exclcorr}
\ea
Catalan (\citealt[88: ex 9]{Corr2016})\\
\gll (Ai) que t' atrapo! \\
\textsc{dm} \textsc{excl} \textsc{cl.2sg} catch.\textsc{1sg.prs}\\
\glt `I'm coming to get you!' 
\ex
Portuguese  (\citealt[88: ex 11]{Corr2016})\\ 
\gll Ai, quo o gato se me foi ao peixe! \\
\textsc{dm} \textsc{excl} the cat \textsc{cl.refl} \textsc{cl.1sg} go.\textsc{3sg.prf.pst} {to the} fish\\
\glt `The cat went off after the fish!'
\ex
		Spanish (\citealt[92: ex 31]{Corr2016})\\
\gll  ¡Que hemos salido en la radio, oiga! \\
\textsc{excl} \textsc{aux.1pl.prf.prs} go.\textsc{ptcp} in the radio \textsc{dm}\\
\glt `We’re on the radio, look!' 
\z
\z

Similar data have been studied by \citet{Biezma2008} on Spanish and \citet{Ledgeway2012} on Romance in general. Otherwise, the use of a complementizer in these types of exclamatives  has passed  mostly without comment in the formal literature  on  Ibero-Romance, although similar phenomena  found in other languages have been mentioned by some authors (cf. \citealt{Saeboe2005} on French,  \citealt{Schwabe2006} on German and  \citealt{Delsing2010} on Scandinavian languages). 

 \citet{Corr2016} treats Ibero-Romance exclamative \textsc{que} sentences as  exclamatives on the grounds that they are expressive: They convey the speaker's mental state or attitude with respect to the propositional content of the sentence, they are formally independent of interrogatives, they give rise to a degree interpretation and they  are potentially also \isi{factive} (on the central properties of exclamatives see also for instance \citealt{Gutierrez-Rexach2001}, \citealt{Zanuttini2003}, \citealt{Castroviejo2006}). 
To map out the position of exclamative \textsc{que}, the author relies again on her CP-external performative structure termed Utterance Phrase (UP) illustrated in \eqref{struc:alicereprep} (repeated from \eqref{struc:alice} see \sectref{sec:insubexistan} for a characterization of the UP). 

\ea \label{struc:alicereprep} {\ob}\textsubscript{MoodP}  {\ob}\textsubscript{SAlow}  {\ob}\textsubscript{EvalP} {\ob}\textsubscript{EvidP} {\ob}\textsubscript{DeclP} \dots ]]]]]
\z 

The data in \eqref{ex:exclcorrvoc} show that exclamative \textsc{que} follows \isi{discourse marker}s and vocatives. 

\ea Catalan \label{ex:exclcorrvoc}
\ea
 (\citealt[132: ex 118]{Corr2016})\\
\gll  (Ai/ apa) que (*ai/ apa) em poso vermella. \\
\textsc{dm} \textsc{dm} \textsc{excl} \textsc{dm} \textsc{dm} \textsc{cl.refl} put red\\
\glt `Ohh/gosh, I've gone red!' 
\ex (\citealt[134: ex 122]{Corr2016})\\

\gll (Amor) que (*amor) em poso vermella. \\
love \textsc{excl} love \textsc{cl.refl} put.\textsc{1sg.prs} red\\
\glt `Darling, you make me blush!' 
\z
\z

 To account for the syntactic properties, \citet{Corr2016} proposes the analysis in \figref{struc:exclir}.
Exclamative \textsc{que} is assumed to be merged in Eval$^0$ where it picks up an evaluative feature and moves to SALow, in which it receives the speech act feature and takes on its performative function. EvaluativeP, according to \citet{Corr2016}, hosts certain types of performative particles like \emph{mira} and \emph{anda}, which express a speaker's attitude and ``involve gradability: A key constitutive property of exclamatives'' (\citealt[136]{Corr2016}).  \citet{Corr2016} furthermore considers the fact that exclamative \textsc{que} is incompatible with the performative particles  merged in this same position as evidence  that they are in complementary distribution, cf. \eqref{ex:fofinha}.\footnote{In my analysis, one possible explanation for the ungrammaticality of \emph{que} above certain \isi{discourse marker}s in examples such as \eqref{ex:fofinha}  is that we are dealing with two instances of attributive  \emph{que}. The grammatical word order in which \emph{que} surfaces below \emph{olha}  leads to two possible conclusions: either  \emph{olha} is merged CP-externally, or, alternatively, it is merged in ModP and treated as an evaluative modifier.}



\ea\label{ex:fofinha}
Portuguese (\citealt[137: ex 132]{Corr2016})\\ 
\gll Ai, fofinha (*que) olha que estás com sorte! \\
\textsc{dm} cute.\textsc{dm} \textsc{excl} \textsc{dm} \textsc{que} be.\textsc{2sg.prs} with luck\\
\glt `Oh, darling, gosh aren't you lucky.' 
\z

The addition of a +SA, \emph{speech act} (standing in for \emph{performative} in \citeauthor{Corr2016}'s \citeyear{Corr2016} account) feature, is justified on the basis that an exclamative \textsc{que} ``performs an expression of one's attitude towards the proposition'' (\citealt[108]{Corr2016}).\largerpage[-2]


Although all the Ibero-Romance varieties that \citet{Corr2016} focuses on accept constructions involving exclamative \textsc{que}, the author claims that  in European Portuguese they are restricted to declaratives. This is why she assumes that in this language exclamative \textsc{que} is  merged lower, in Decl$^0$ where it picks up the additional declarative feature (as in the structure in \figref{struc:syncheadexcl}).{\interfootnotelinepenalty=10000\footnote{The general assumption made in \citet{Corr2016} is that many differences between European Portuguese and other Ibero-Romance varieties result from the fact that in the former  the declarative, evidential and evaluative  features are bundled on one functional head whereas they are scattered across three heads in the other varieties (see also \sectref{sec:insubexistan}). It is thus surprising that the author presents an analysis for exclamative \textsc{que}  in European Portuguese that does not involve EvidP.}}\largerpage

\begin{figure}
\caption{\label{struc:exclir}Exclamative \textsc{que}  in \citet[137, ex 134]{Corr2016}}
\begin{forest}
			[SAHighP 
			[DM] 
			[SAHigh' 
			[que,name=sahigh]  
			[SALowP
			[VocP] 
			[SALow'
			[que\\\textsc{[+sa]}, name=salow]
			[EvalP
			[~] 
			[Eval'
			[\sout{que}\\\textsc{[+eval]}, name=eval]
			[\dots]
			{
				\draw[->] (eval)  to[out=south west, in=south] (salow);
			} 	
			]]]]]]
\end{forest}
\end{figure}

\begin{figure}
\caption{\label{struc:syncheadexcl}Exclamative \textsc{que} in European Portuguese in \citet[236, ex 112]{Corr2016}}
	\begin{forest}
		[SAHighP 
		[DM] 
		[SAHigh' 
		[que,name=sahigh]  
		[SALowP
		[VocP] 
		[SALow'
		[que\\\textsc{[+sa]}, name=salow]
		[EvalP
		[~] 
		[Eval'
		[\sout{que}\\\textsc{[+eval]}, name=eval]
		[DeclP
		[~] 
		[Dec'
		[\sout{que}\\\textsc{[+decl]}, name=decl]
		[\dots]
		{			
			\draw[->] (decl)  to[out=south west, in=south] (eval);
			\draw[->] (eval)  to[out=south west, in=south] (salow);
		} 	
		]]]]]]]]
	\end{forest}
\end{figure}


The second type of non-reportative \emph{que}-initial construction that \citet{Corr2016} deals with involves a particle she calls conjunctive \textsc{que}. In the previous literature, it has been proposed that in these cases the complementizer establishes a causal\is{causal connective} (see \citealt{Alarcos1994}, \citealt{Ballesteros2000}, \citealt{Peres2006}, \citealt{Etxepare2013},  \citealt{Wheeler1999}, \citealt{Cunha1984},  \citealt{Lobo2003}, \citealt{Lopes2012}) 
or explicative (\citealt{Colaco2016}) relation between a previous sentence and the sentence introduced by \emph{que}. The use of an element  homophonous with a complementizer  as a clausal connective is also  documented in the history of the Ibero-Romance languages (\citealt{Martinez1978}, \citealt{Carrera1982},  
\citealt{Hernandez1988}, \citealt{Batllori2000},  \citealt{Batllori2005}).

An example from contemporary Spanish is given in  \eqref{ex:corrconjill}. Conjunctive \textsc{que} sentences often follow imperatives, but not always (see the example in \eqref{ex:quevacall}).\largerpage 
 
\ea\label{ex:corrconjill}
		Spanish (adapted from \citealt[229: ex 90]{Corr2016})\\
\gll  No me pises, que llevo chanclas. \\
not \textsc{cl.1sg} step.\textsc{2sg.sbjv.prs} \textsc{conj} wear.\textsc{1sg.prs} {flipflop.\textsc{pl}}\\
\glt `Don’t step on me, I’m wearing flipflops.'
\z


Despite the focus on the causal\is{causal connective} function in the literature,  \citet{Corr2016} shows that conjunctive \textsc{que} is semantically not the same as a causal\is{causal connective} connective.\footnote{For more arguments against a primary causal\is{causal connective} or explicative nature of these complementizers, see \sectref{sec:presupprag}.} She illustrates this with the example in \eqref{ex:quevacall}.


\ea\label{ex:quevacall} {Catalan} (\citealt[226: ex 84]{Corr2016})\label{ex:carvac}\\ \relax[Context: The addressee glances at some boarding passes on the speaker’s desk and the speaker notices what the addressee is looking at.]
\ea[] 
{
\gll Que me ’n vaig de vacances.   \\
\textsc{que} \textsc{cl.refl} \textsc{cl.part} go.\textsc{1sg.prs} of holiday.\textsc{pl}\\
\glt `I’m going on holiday.'\label{ex:quevac}}
\ex[*]{\gll  Perquè/ car me’ n vaig de vacances. \\
because for \textsc{cl.refl} \textsc{cl.part} go.\textsc{1sg.prs} of holiday.\textsc{pl}\\
\glt }
\z
\z

Conjunctive \textsc{que}, but not a true causal\is{causal connective} connective like \emph{perquè/car} (cf. \ref{ex:carvac}), is felicitous without a previous utterance and can serve to explain a non-linguistic situation (cf. \ref{ex:quevac}).  According to \citet{Corr2016}, these two categories are distinct because  conjunctive \textsc{que} introduces a syntactically independent clause, while the relevant causal\is{causal connective} connectives introduce a syntactically dependent clause. 
\citet[207]{Corr2016} argues that the primary function of conjunctive \textsc{que} is not to establish a causal\is{causal connective} link but  to maintain or  improve the conversational flow.

\citet{Corr2016}  maps out the syntactic position against the backdrop of her UP. She uses the data in \eqref{ex:corrconjpos} to show   that  conjunctive \textsc{que} cannot co-occur with discourse markers or vocatives. 
\ea Spanish \label{ex:corrconjpos}
\ea (\citealt[235: ex 109]{Corr2016})\\
\gll ¡Escúchame, (*oye) que (*oye) vamos a llegar tarde! \\
listen.\textsc{imp}=\textsc{cl.1sg} \textsc{dm} \textsc{conj} \textsc{dm} go.\textsc{1pl.prs} to arrive late\\
\glt `Listen, *hey we’re going to arrive late!' 
\ex (\citealt[235: ex 110]{Corr2016})\\
\gll ¡Escúchame, (*María) que (*María) vamos a llegar tarde! \\
listen.\textsc{imp}=\textsc{cl.1sg} María \textsc{conj} María go.\textsc{1pl.prs} to arrive late\\
\glt `Listen, *María we’re going to arrive late!' 
\z
\z

\begin{sloppypar}\citet{Corr2016} proposes the analysis given in \figref{struc:moveconj}.  Conjunctive \textsc{que} reaches SAHigh$^0$, the highest head in the UP. The complementizer is assumed to be merged in Decl$^0$, the lowest head of her split ForceP. On its way  to SAHigh$^0$, it passes through Evid$^0$ and Eval$^0$. Conjunctive \textsc{que} is valued with the features hosted by the individual heads.\footnote{As an alternative to the movement analysis,  \citet{Corr2016} also proposes a structure in which the features are valued simultaneously  by a syncretic head (cf. \citealt[236: ex 111]{Corr2016}).}  \citet{Corr2016} assumes that the merger occurs in DeclP because conjunctive \textsc{que} is restricted to declaratives. The evidential feature picked up in Evid$^0$ is explained on the grounds that conjunctive \textsc{que} shows a parallel behavior to certain evidential complementizers merged in this projection.  The evaluative feature in EvalP guarantees that the constructions are assertive and express a speaker's point of view. Finally the SA or performative feature ensures the performative nature of the constructions involving conjunctive \textsc{que}.\end{sloppypar}


\begin{figure}
\caption{\label{struc:moveconj}Head movement analysis of  conjunctive \textsc{que} in \citet[236, ex 111]{Corr2016}}
\begin{forest}
	[SAHighP 
	[~] 
	[SAHigh' 
	[que,name=sahigh]  
	[SALowP
	[~] 
	[SALow'
	[SALow]
	[EvalP
	[~] 
	[Eval'
	[\sout{que}, name=eval]
	[EvidP
	[~] 
	[Evid'
	[\sout{que}, name=evid]
	[DeclP
	[~] 
	[Decl'
	[\sout{que}, name=decl]
	[\dots]
	{
		\draw[->] (decl)  to[out=south west, in=south] (evid); 
		\draw[->] (evid)  to[out=south west, in=south] (eval);		
		\draw[->] (eval)  to[out=south west, in=south] (sahigh);
	} 	
	]]]]]]]]]]
\end{forest}
\end{figure}


The main issue I have with the analysis proposed in \citet{Corr2016} is that there  is no  empirical motivation relating to word order restrictions to support the idea that   the complementizers reach a position in the CP-external UP. The explanation presented in \citet{Corr2016} is based only  on the theoretical assumptions made by the author when she relates certain interpretative properties of the constructions 
to the  effects of the features in her assumed UP. This is a potential problem because, as in her analysis of  reportative \emph{que} discussed in \sectref{sec:insubexistan}, the question of  how the abstract features give rise to the specific meaning is not answered  satisfactorily.  In principle, the empirical data are also compatible with an analysis that assumes that in both  constructions \emph{que} is merged in  the highest CP head SubP (structurally equivalent to ForceP in the original hierarchy) and therefore below \citeauthor{Corr2016}'s UP. The data in \eqref{ex:exclcorrvoc} show that exclamative \textsc{que}  follows phrases like vocatives and \isi{discourse marker}s that are analyzed as being merged in the UP. These data are compatible with an analysis like that proposed by \citet{Corr2016} that places the complementizer in the lowest UP head but they are also in line with my alternative analysis that locates them  in the highest CP head. The data in \eqref{ex:corrconjpos} show that conjunctive \textsc{que} is incompatible with vocatives and \isi{discourse marker}s;  the assumption that the complementizer here reaches a high UP projection lacks therefore compelling empirical support. However, the conclusion that the complementizer simply occupies the highest CP head is  in line with the data. Pending further empirical evidence of movement to a CP-external projection, the position I adopt in the present monograph is that the complementizer in these contexts is merged in SubP, the top left projection of the split CP. 

\subsection{\citet{PrietoRigau2007}}

The syntactic behavior and the pragmatic effect of  \textit{que} in \isi{polar question}s have so far  been addressed almost exclusively as a feature of Catalan grammar (cf. \citealt{Rigau1984}, \citealt{MascaroiPons1986}, \citealt{Cuenca1997}, \citealt{Prieto1997, Prieto2002}, \citealt{Payrato2002}, \citealt{Celdran2005}, \citealt{HernanzRigau2006}). However, I will show in \sectref{sec:presuppqC} that it is also attested in Spanish. To the best of  my knowledge, this has been widely disregarded in the literature. One exception is \citet{Hualde1992}, where it is mentioned briefly and characterized as a case of transfer from Catalan. To date, the most extensive studies of \emph{que} in Catalan \isi{polar question}s are those carried out by \citet{Rigau2005} and \citet{PrietoRigau2007}. This section summarizes the main points of their analysis. 

One central finding in \citet{PrietoRigau2007} is that the presence of \emph{que} coincides with a falling question intonation. The authors furthermore suggest that there is dialectal  variation  with regard to the presence and absence of \emph{que} in  different pragmatic contexts. They argue that \emph{que} is  virtually unrestricted in Minorcan Catalan.  All  varieties accept the presence of \emph{que} in anti-expectational\is{surprise} contexts, i.e. contexts in which the facts or the situation are in disagreement with the speaker's expectations. In these contexts, \isi{polar question}s can  be used to express the speaker's \isi{surprise} or astonishment. An example is given in \eqref{ex:antiexnorth}.

\ea\label{ex:antiexnorth}
Catalan (\citealt[15: ex 30a]{PrietoRigau2007})\\
	\gll Que vindràs a Barcelona? No em pensava pas que ens acompanyessis. \\
	\textsc{que} {come}.\textsc{fut.2sg} to Barcelona not \textsc{cl.1sg} thought \textsc{neg} that \textsc{cl.1pl} accompany.\textsc{subj.2sg}\\
	\glt `Are you coming to Barcelona? I didn’t think you were coming with us.'
\z

The anti-expectational\is{surprise} nature of the context becomes evident from the statement that follows the \emph{que}-initial \isi{polar question}  where the speaker explicitly asserts that the fact that the hearer is going to Barcelona was not part of her prior beliefs. 

The complementizer is furthermore accepted  in confirmatory questions in all dialects. According to \citet{PrietoRigau2007}, these are questions where the speaker expects an affirmative answer. In Catalan, confirmatory \emph{que}-questions are often preceded by a question particle, cf. \eqref{ex:confirmatoryall}. According to the authors, the choice of the particle depends on the dialect. 

\ea\label{ex:confirmatoryall}
Catalan (\citealt[17: 35 a-f]{PrietoRigau2007})\\
	\gll {Oi / Eh / Veritat / No / Fa / És ver} que vindràs? \\
	\textsc{particle} \textsc{que} {come}.\textsc{fut.2sg}\\
	\glt `You're coming, aren't you?'
\z

\citet{Castroviejo2018} makes a compelling case that, at least in sentence-final positions like in \eqref{ex:cabells}, a subset of these particles are not synonymous but encode different meanings (see also \sectref{sec:presupprag}).
 
\ea\label{ex:cabells}
Catalan (\citealt[ex 19]{Castroviejo2018})\\
\gll T' has tallat els cabells, oi?/ eh? \\
\textsc{cl.2sg} \textsc{aux.3sg.prf.prs} cut.\textsc{ptcp} the hair.\textsc{pl} \textsc{oi} \textsc{eh}\\
\glt `You had your hair cut, right?/ huh?'
\z


Other non-neutral \isi{polar question}s that permit \emph{que} are rhetorical questions, as in \eqref{ex:rhetorical}.\footnote{Although this is not central to the present discussion, in \citet{Kocher2017a} I propose that questions like \eqref{ex:rhetorical} are better characterized as hyperbolic rather than rhetorical questions. Hyperbolic and rhetorical questions both  stand in for another utterance. However, a true rhetorical question stands in for an assertion, while these hyperbolic questions actually stand in for another \isi{polar question} as an exaggerated  version of it.}  

\ea\label{ex:rhetorical} Catalan \citep[18: 36b]{PrietoRigau2007}\\
  \gll 	Que et penses que tinc quatre mans, jo?\\
	\textsc{que} you think.\textsc{2sg.prs} that have.\textsc{1sg.prs} four hand.\textsc{pl} I\\
	\glt `Do you think I have four hands?' 
\z


 \citet{PrietoRigau2007} state that in   Northwestern, Central and Balearic Catalan, \emph{que} is furthermore allowed in what they term polite \isi{polar question}s. They are considered polite by the authors because the speaker  uses them when  they only require a low cost action by the hearer. A low cost action always implies that the speaker was certain that the hearer would answer positively.  
 
\ea\label{ex:platja}
Catalan (\citealt[4: ex 8a]{PrietoRigau2007})\\
\gll Que em deixes el teu apartament de la platja, aquest {cap de setmana}?\\
	\textsc{que} me leave.\textsc{2sg.prs} the your apartment of the beach this weekend\\
	\glt `Would you let me use your apartment by the beach this weekend?'
	\z
 
The use of \emph{que} in \eqref{ex:platja} is  felicitous only if the hearer has offered the  apartment to the speaker previously. In a context where this is not the case, \emph{que}-initial polar questions are not felicitous according to \citet{PrietoRigau2007}. 

\ea\label{ex:fumar} Catalan (\citealt[4: ex 9a]{PrietoRigau2007})\\
	\gll Que puc fumar?\\
	\textsc{que} can.\textsc{1sg.prs} smoke\\
	\glt `Can I smoke?'
\z

Similarly, \eqref{ex:fumar} is only felicitous if the speaker can assume that her smoking is not going to bother the hearer but is not felicitous if she expects it will or if she 
has no expectations in this regard.

\citet{PrietoRigau2007} adopt a cartographic approach and  assume that \textit{que} is merged at the lowest edge of the left periphery in {FinP}.   They propose the structures in \eqref{struc:prietorigau}.\largerpage

\ea Catalan\label{struc:prietorigau} 
\ea (\citealt[25: ex 56a]{PrietoRigau2007})\label{struc:prietorigaua}\\
\gll {\ob}\textsubscript{ForceP} {\ob}\textsubscript{Operator} Oi] {\ob}\textsubscript{Force} \textsubscript{+confirmative interrogative} {\ob}\textsubscript{FinP} que {\ob}\textsubscript{IP} en Pere no va a Barcelona?]]]]\\
{} {}  \textsc{oi} {} {} {}  \textsc{que} {} the Pere not go.\textsc{2sg.prs} to Barcelona\\
\glt `Pere isn't going to Barcelona, right?' 
\ex\label{struc:prietorigaub}(\citealt[25: ex 56b]{PrietoRigau2007})\\
\gll {\ob}\textsubscript{ForceP} {\ob}Operator] {\ob}\textsubscript{Force} \textsubscript{+anti-expect./neutral interrogative} {\ob}\textsubscript{FinP} que {\ob}\textsubscript{IP} no volies un collaret?]]]]\\
{} {}  {} {} {} \textsc{que} {} not want.\textsc{2sg.ipvf.pst} a necklace\\
\glt `Didn't you want a necklace?'
\z
\z

In their analysis, the non-neutral interpretation of \textit{que}-initial \isi{polar question}s is attributed to the presence of an interrogative operator in Force that is realized by  prosodic means. \textit{Que} is considered optional and does not contribute  any meaning of its own. 

 I will now turn to my evaluation of   this account. The prediction of the structural analysis proposed in \citet{PrietoRigau2007} is that, given the low position of \emph{que}, other left-peripheral material should precede rather than follow the complementizer. 


\ea \label{ex:clldpqrep} 	Catalan (\citealt[49: ex 98a]{Kocher2017a})\\ $[$Context: Marta finds a bag of oranges in the kitchen. She asks her roommate:$]$\\
	\gll Que les taronges$_j$  les$_j$ vas comprar tu? \\
	\textsc{que}  the orange\textsc{pl}  \textsc{cl.f.pl} \textsc{aux.2sg.prf.pst} buy you\\ 
	\glt `The oranges, you bought them, didn't you?'
\z



The data in \eqref{ex:clldpqrep} pose a problem for the analysis proposed in \eqref{struc:prietorigaub} because in \eqref{ex:clldpqrep}, a clitic left dislocated \isi{topic} follows rather than precedes \emph{que}. Furthermore, based on their analysis for particle questions in \eqref{struc:prietorigaua}, we would expect that certain phrases should be able to intervene between the particle and the complementizer. However, the data in \eqref{ex:oinotoprep} show that the particle and the complementizer must be adjacent (more data are discussed in \sectref{sec:presuppqC}). 


\ea Catalan \label{ex:oinotoprep}\\
\gll * Oi en Jordi que l' has convidat tu?\\
	 \phantom{*} \textsc{oi} the Jordi \textsc{que} \textsc{cl.m.sg} have.\textsc{aux.2sg.prf.prs} invite.\textsc{ptcp} tu\\
	\glt \phantom{*} Intended: `You are the one that invited Jordi, right?'
\z 

To reconcile these data I will propose a revised analysis in \sectref{sec:presuppqC}. I adopt the idea from \citet{PrietoRigau2007} that \emph{que} is merged in FinP but in my analysis it does not remain in this position;  instead, it moves through the left periphery and ends up in the head of the highest projection of the left periphery.

Turning to pragmatic considerations, \citet{PrietoRigau2007} offer a detailed characterization of different  contexts that license \emph{que} in \isi{polar question}s. Based on this characterization, I propose a generalization that, as will be seen in \sectref{sec:presupprag},   ultimately makes it possible to assume that \emph{que} has  a uniform discourse contribution in all the different constructions. What all the contexts that license \emph{que} in the majority of dialects have in common is that the speaker expects a positive answer from the hearer. In an anti-expectational\is{surprise} context like \eqref{ex:antiexnorth}, while the speaker's belief was the opposite, contextual evidence  suggests that the answer is going to be positive (see \citealt{Kocher2017a} and \sectref{sec:presupprag} for a revised definition of the notion of contextual evidence presented in \citealt{Buering2000}). In confirmatory contexts like \eqref{ex:confirmatoryall}, the speaker's belief itself makes her expect a positive answer. The polite \isi{polar question}s in \eqref{ex:platja} and \eqref{ex:fumar} can also be subsumed readily under the label of confirmatory questions because, as \citet{PrietoRigau2007} state, \emph{que} is only felicitous when the speaker has a hunch that the answer is going to be affirmative. Finally, rhetorical questions like \eqref{ex:rhetorical} are more challenging because they  do not   have the illocutionary force of a question, hence the speaker does not necessarily expect an answer. However, I believe they can be accounted for if they are treated in  the way that I propose for \emph{que}-initial assertions, in the sense that a \isi{commitment} to them is attributed to the hearer (cf. \sectref{sec:presupprag}).\footnote{\citet{Kocher2017a} offers  a more extensive discussion of Catalan \isi{bias}ed \isi{polar question}s. There, I  propose a slightly different generalization that relies on the typology of question \isi{bias}es in \citet{Sudo2013}. My basic idea was that the presence of \emph{que} is licensed when there is positive evidence in the context.}



\subsection{\citet{Ambar2003}, \citet{Castroviejo2006} and \citet{DemonteSoriano2009}}

 The  analyses of \is{wh-exclamative@\textit{wh}-exclamative}\textit{wh}-exclamatives involving \emph{que} that I will discuss here are  proposed by \citet{DemonteSoriano2009} for Spanish, \citet{Ambar2003} for Portuguese and \citet{Castroviejo2006} for Catalan (but cf. also \citealt{Bosque1984}, \citealt{Brucart1993},  \citealt{Villalba2008} and \citealt{Gutierrez-Rexach2001}). 
   Relevant accounts of \is{wh-exclamative@\textit{wh}-exclamative}\textit{wh}-exclamatives in other languages are put forward in  
\citet{Milner1978}, \citet{Radford1982}, \citet{Beninca1996}, \citet{Zanuttini2003},  \citet{Cruschina2015}, among others.\largerpage

    

\citet{Castroviejo2006} investigates \is{wh-exclamative@\textit{wh}-exclamative}\textit{wh}-exclamatives with \emph{que} in Catalan along with other types of exclamatives.  In her proposal, the complementizer is characterized as semantically vacuous and its presence is deemed  optional.  \citet{Castroviejo2006} does not adopt a split CP. She  proposes the analysis in \eqref{struc:castro}. The \textit{wh}-phrase is merged vP internally and moves through the specifier of the TP to the specifier of the only CP projection in the structure. The complementizer is realized as the head of the same CP.

\ea\label{struc:castro}
Catalan (\citealt[50: ex 123b]{Castroviejo2006})\\
\gll {\ob}\textsubscript{CP} {\ob}\textsubscript{SpecCP} Quins ingredients tan bons$_i$] {\ob}$_{C'}$ {\ob}$_{C^0}$ que] {\ob}\textsubscript{TP} t$_i$ té$_j$ {\ob}\textsubscript{vP}  aquesta sopa t$_j$ t$_i$.]]]] \\
	{} {}	which ingredient.\textsc{pl} so good {} {}  \textsc{que} {} {}  have.\textsc{2sg.prs} {} this soup\\ 
	\glt `What great ingredients this soup has!' 
\z


\citet{Ambar2003} offers an account of Portuguese \is{wh-exclamative@\textit{wh}-exclamative}\textit{wh}-exclamatives. She adopts a structured left periphery;  however, her proposed structure differs from the (minimally revised) structure based on \citet{Rizzi1997} that I adopt here. 
XP is  conceived of as a landing site for dislocated elements and can be considered parallel to a Rizzian TopP. WhP is an operator projection hosting \textit{wh}-phrases. AssertiveP is projected when assertive properties are involved in the constructions and is also linked to a \isi{factive} interpretation. EvaluativeP encodes the speaker's evaluation and hosts phrases that contain evaluative elements.

\ea\label{struc:ambar} {\ob}\textsubscript{XP}  {\ob}\textsubscript{EvaluativeP}  {\ob}\textsubscript{Evaluative'} {\ob}\textsubscript{AssertiveP} {\ob}$_{\text{Assertive}'}$ {\ob}\textsubscript{XP}  {\ob}\textsubscript{WhP}{\ob}$_{\text{Wh}'}$ {\ob}\textsubscript{FocusP}  {\ob}$_{\text{Focus}'}$ {\ob}\textsubscript{XP} {\ob}\textsubscript{IP}  \\ (\citealt[211: ex 1]{Ambar2003})
\z

The analysis that \citet{Ambar2003} assumes for \is{wh-exclamative@\textit{wh}-exclamative}\textit{wh}-exclamatives involves AssertiveP and EvaluativeP.  The \isi{factive} interpretation of \is{wh-exclamative@\textit{wh}-exclamative}\textit{wh}-exclamatives (see \sectref{sec:presupprag}) is  attributed to the feature in AssertiveP that can be checked either by the \textit{wh}-ex\-pres\-sion or, when present, by \emph{que}. In both cases, the \textit{wh}-ex\-pres\-sion is \textit{wh}-moved from an IP-internal position passing through WhP, FocusP and in \emph{que}-less \is{wh-exclamative@\textit{wh}-exclamative}\textit{wh}-exclamatives also through AssertiveP. It ends up in EvaluativeP where it checks an evaluative feature.
\ea Portuguese (\citealt[238--239: ex 88--89]{Ambar2003})\\ 
\gll  {\ob}\textsubscript{EvaluativeP} que livro$_i$ {\ob}\textsubscript{Evaluative'} {\ob}\textsubscript{AssertiveP} {\ob}$_{\text{Assertive}'}$ que/t$_i$ {\ob}\textsubscript{XP} o João$_j$ {\ob}\textsubscript{WhP} t$_i$ {\ob}$_{\text{Wh}'}$ {\ob}\textsubscript{FocusP} t$_i$ {\ob}$_{\text{Focus}'}$ {\ob}\textsubscript{XP} {\ob}\textsubscript{IP} t$_j$ leu t$_i$]]]]]]]]]]] \\
{} what book {} {} {} \textsc{que} {} the João {} {} {} {} {} {} {} {} {} read\\ 
\glt `What a book (that) John read!' 
\z 

\citet{DemonteSoriano2009} propose an account of Spanish \is{wh-exclamative@\textit{wh}-exclamative}\textit{wh}-ex\-clam\-a\-tives that relies on a split CP à la \citet{Rizzi1997}. The presence of \emph{que} is considered to be optional. In their analysis, \emph{que} is merged  in {FinP} and the \textit{wh}-expression in  FocP (cf. \ref{struc:demsorexcl}). 


\ea\label{struc:demsorexcl}
		Spanish (\citealt[33 : ex 19a]{DemonteSoriano2009}, analysis added by the author)\\
\gll  {\ob}\textsubscript{FocP} ¡Qué rico] ... {\ob}\textsubscript{FinP} (que)] {\ob}\textsubscript{IP} está!] \\
{} how good {} {} \textsc{que} {}  be.\textsc{3sg.prs}\\
\glt `How good this is!' 
\z


All these analyses are based on different theoretical assumptions, making it difficult to compare them.  Only \citet{DemonteSoriano2009} assume a \citeauthor{Rizzi1997}an style structure of the left periphery that makes it comparable to my own proposal.  
One potential issue with their analysis,  however, is that it cannot account for the data in \eqref{ex:exclprobs}.

\ea {Spanish}\label{ex:exclprobs}
\ea[*]{
\gll  Qué raro  a Juana$_i$ que  la$_i$ has invitado pero no a María. 
	\\
	how strange \textsc{dom} Juana \textsc{que} \textsc{cl.fs} have.\textsc{2s} invited but not \textsc{dom} María\\\label{ex:exclint}}
		\ex[]{\gll A Juana$_i$ qué raro que  la$_i$ has invitado pero no a María.\\
		\textsc{dom} Juana how strange  \textsc{que} \textsc{cl.f.sg} \textsc{aux.2sg.prf.prs} invite.\textsc{ptcp} but not \textsc{dom} María\\ \label{ex:exclprec}}
		\ex[]{\gll Qué raro que a Juana$_i$  la$_i$ has invitado pero no a María.\\
		how strange  \textsc{que}	\textsc{dom} Juana  \textsc{cl.f.sg} \textsc{aux.2sg.prf.prs} invite.\textsc{ptcp} but not \textsc{dom} María\\
		\glt `How strange that you invited Juana but not Maria.'\label{ex:exclfol}}
	\z
\z


The example in \eqref{ex:exclint} shows that a dislocated \isi{topic} cannot intervene between the \textit{wh}-expression and the complementizer, even though the account in \citet{DemonteSoriano2009} predicts that it should be able to. \eqref{ex:exclprec} shows that the \textit{wh}-expression can be preceded by a dislocated \isi{topic}, which is in line with their analysis. However, a dislocated \isi{topic} can also follow \emph{que},  which is again not predicted by the analysis because the complementizer occupies the lowest position in the left periphery. In my revised analysis, which is similar to that proposed in  \citet{DemonteSoriano2009}, the problematic data are accounted for by assuming that the complementizer moves from its initial merge position in FinP through the left periphery and ends up adjacent to the \textit{wh}-expression in the head of FocP. 





\subsection{\citet{Cruschina2017a, Cruschina2018}}
The Ibero-Romance construction whereby a complementizer follows an epistemic, evidential or to a lesser extent, evaluative modifier has been mentioned by various authors in the literature (\citealt{MartinZorraquino1998},  \citealt{Etxepare1997},  \citealt{Hummel2000,Hummel2014,Hummel2017}, \citealt{Gutierrez-Rexach2001}, \citealt{FreitesBarros2006}, \citealt{HernanzRigau2006}, \citealt{Ocampo2006}, \citealt{Rodriguez-Ramalle2007,Rodriguez-Ramalle2008,Rodriguez-Ramalle2015}, \citealt{Gras2010}, \citealt{Sansinenaetal2015}). Apart from my own analysis (\citealt{Kocher2014, Kocher2017}), the previous more extensive accounts of the construction focus primarily on other Romance languages  and not on the three Ibero-Romance varieties under discussion  here. Especially influential is the analysis by \citet{Hill2007b} (see also the review of this work in \citealt{Lupsa2011}) of this construction in Romanian that directly inspired the analysis in \citet{Cruschina2013} for Italian and Sicilian. The analysis in \citet{Cruschina2017a, Cruschina2018} is also applied to the  construction in Spanish along with other Romance languages.
What these  latter analyses have in common is that they adopt the neo-performative hypothesis by \citet{SpeasTenny2003} which postulates a functional field above the split CP mediating the interface between syntax and discourse (see \sectref{sec:performativehyp}). All the accounts assume that the complementizer is merged in the highest projection of the split CP, i.e. Force, which is structurally equivalent to my SubP. They assume that the modifier is located in a CP-external projection. In the structure in \figref{struc:advccrus}, for example, I illustrate the analysis proposed in \citet{Cruschina2018},  based on \citet{SpeasTenny2003}.
 

One potential issue with an analysis along these lines are data such as those in \eqref{ex:advcembed}.\largerpage

\ea \label{ex:advcembed}
\ea\label{ex:advcembeda}
European Portuguese\\ 
\gll  Disse  que certamente que iria ver logo resultados. \\
		say.\textsc{3sg.ipfv.pst}   that  certainly \textsc{que} go.\textsc{1sg.cond} see soon results\\
		\glt `S/he said that certainly I would  see results soon.' (CdP)
		\ex
		\label{ex:advcembedb} 
				Spanish\\
		\gll  Otra canción que claro {que} escuchamos todos y que podría parecer muy buena, es ``Realmente no estoy tan solo''. \\ 
		other song that claro \textsc{que} listen-to.\textsc{1pl.prs} all and that can.\textsc{3sg.cond} seem very good be.\textsc{3sg.prs} Realmente no estoy tan solo\\
		\glt `Another song that clearly we all listened to and that could seem very good is ``Realmente no estoy tan solo''.' (CdE)
		\pagebreak\ex\label{ex:advcembedc} 
		Catalan\\
		\gll I  {per això} us hem preparat un article que segur que us serà útil un moment o altre. \\
		and therefore \textsc{cl.2pl} \textsc{aux.1pl.prf.prs} prepare.\textsc{ptcp} an article that sure \textsc{que} \textsc{cl.2pl} {will-be} useful one moment or other\\
		\glt `And therefore we have prepared an article for you that surely will be useful for you at some point or another.' (caWac)
	\z
\z

\begin{figure}
	\caption{\label{struc:advccrus}AdvC in \citet[350: ex 23]{Cruschina2018} (adapted)}
	\begin{forest}
		[SAP
		[Spec] 
		[SA' 
		[SA$^0$]  
		[SentienceP
		[modifier] 
		[Sen'
		[Sen$^0$]
		[ForceP
		[Spec] 
		[Force'
		[Force$^0$\\que]
		[\dots]
		]]]]]] 
	\end{forest}
\end{figure}

These examples show that the construction is not restricted to root contexts  but can also appear in  embedded contexts, and in particular also in relative sentences (cf. an appositive relative in \eqref{ex:advcembedb} and a restrictive relative in \eqref{ex:advcembedc}). In this respect, the Ibero-Romance languages under investigation here appear to contrast with Italian, which according to \citet{Cruschina2018}  only permits the construction in complements of verbs of saying like in \eqref{ex:advcembeda} but not in relatives. The analysis I propose for the construction assumes a surface position within the split CP and is therefore able to account for the data in \eqref{ex:advcembed}.

\subsection{\citet{Hernanz2007}}
The construction in which a complementizer follows the \isi{verum} marker \emph{sí} is attested in Spanish and Catalan but not in Portuguese. It has been explored in the literature, notably  in \citet{Martins2006, Martins2013}, \citet{GonzalezRodriguez2008, GonzalezRodriguez2009,  GonzalezRodriguez2016} \citet{Escandell-Vidal2009}, \citet{Escandell-Vidal2009a}, \citet{Escandell-Vidal2011}, \citet{RodriguezMolina2014} and  \citet{VillaGarcia2020b, VillaGarcia2020a}. The most widely adopted analysis of the \isi{verum} construction in Spanish is developed in   \citet{Hernanz2007} (see also \citealt{Batllori2008} where the  analysis is applied to diachronic data). The analysis has also been extended to Catalan in \citet{Batllori2013}.  


\begin{figure}
  \caption{\label{struc:hernsique}AffC in \citet[144: ex 87]{Hernanz2007} (adapted)}
  \begin{forest}
	[ForceP
	[\emph{sí}]
	[Force' 
	[Force$^0$ 
	[que]]
	[IP
	[...] 	
	]]]
  \end{forest}
\end{figure}

\citet{Hernanz2007}  proposes the structure in \figref{struc:hernsique}. The complementizer is merged as  the head of ForceP and \textit{sí} is assumed to be in its specifier. 

Verum\is{verum} sentences do not require the presence of \emph{que} in either language. \citet{Hernanz2007} compares \emph{sí que}-sentences, as in  \eqref{ex:hernanzsique}, with their \emph{que}-less equivalents, as in \eqref{ex:hernanzsi}. 

\ea Spanish 
\ea\label{ex:hernanzsique}
		(\citealt[134: ex 3a]{Hernanz2007})\\
\gll Sí que ha llovido hoy. \\
yes that \textsc{aux.3sg.prf.prs} rain.\textsc{ptcp} today\\
\glt `It \textsc{has} indeed rained today.' 
\ex\label{ex:hernanzsi} 
(\citealt[134: ex 1a]{Hernanz2007})\\
\gll Sí ha llovido hoy. \\
yes \textsc{aux.3sg.prf.prs} rain.\textsc{ptcp} today\\
\glt `It \textsc{has} rained today.' 
\z
\z


She notes that they differ in that the version with the complementizer  emphasizes a proposition that has already been mentioned in the discourse. She adopts an idea from \citet{Etxepare1997} and states that a proposition introduced by the \isi{verum} marker requires a linguistic antecedent. \citet{Hernanz2007} suggests that this function should be attributed  to ForceP. Since this aspect of meaning, according to the author, does not occur when  \emph{que} is absent, she adopts a different syntactic analysis for  \emph{que}-less \isi{verum} sentences, which is exemplified in \eqref{struc:hernsi}.

\ea\label{struc:hernsi} (adapted from \citealt[129: ex 48]{Hernanz2007})\\ 	{\ob}\textsubscript{ForceP} {\ob}\textsubscript{TopicP} {\ob}\textsubscript{FocusP} sí$_i$ {\ob}\textsubscript{PolP} t$_i$ {\ob}\textsubscript{IP} ...]]]]] 
  \z

In this analysis \emph{sí} starts out in a polarity position termed PolP that is sandwiched between FinP and IP. It is the same polarity position in which the sentence negation particle \emph{no} is located  (see \citealt{Laka1990}). \citet{Hernanz2007} argues that the \emph{sí} in these contexts has focal properties, which is the reason behind her assumption that it moves  to the left-peripheral FocP.


There are some empirical data that cannot be accounted for straightforwardly  by the analysis proposed in \citet{Hernanz2007}, which  ultimately leads me to argue in favor of a different account (see \sectref{sec:presupaffc} and \citealt{Kocher2017} for a more detailed discussion). \citeauthor{Hernanz2007}'s analysis for \emph{sí que}-sentences fails to account for data such as in \eqref{ex:probshernanz}  which show that \emph{sí que} is also attested in embedded sentences, see \eqref{ex:probshernanza} and \eqref{ex:probshernanzb}, and can be preceded by a clitic left dislocated \isi{topic}, see \eqref{ex:probshernanzc}.


\ea \label{ex:probshernanz}
\ea\label{ex:probshernanza}
		Spanish (\citealt[94: ex 32b]{Kocher2017} from caWac)\\
\gll  En el bar de la Confederació General del Treball (CG\is{common ground}T) confirman 	que sí que hay
			huelga. \\
			in the bar of the Confederació General del Treball (CG\is{common ground}T) confirm.\textsc{3pl.prs} that \textsc{verum} \textsc{que} there.be.\textsc{3sg.prs} strike\\
			\glt `In the bar of the CG\is{common ground}T they confirm that there is a strike going on.' 
		\ex \label{ex:probshernanzb}
		Catalan (\citealt[45: ex 88b]{Kocher2017a} from caWac)\\
		\gll A banda d'aquest dissentiment inicial, hi ha dos partits de l'oposició que sí que la votaran afirmativament. \\
		at side {of this} disagreement initial there.be.\textsc{3sg.prs} two party.\textsc{pl} of {the opposition} that \textsc{verum} \textsc{que} it vote.\textsc{3pl.fut} affirmative\\
		\glt `Concerning this initial disagreement there are two parties of the opposition  that \textsc{will} vote in favour of it.'
\pagebreak\ex \label{ex:probshernanzc}
		Spanish (\citealt[94: ex 32a]{Kocher2017a} from caWac)\\
\gll	A López$_i$ sí que le$_i$ he visto agredir a dos de mis jugadores.\\
\textsc{dom} López \textsc{verum} \textsc{que} \textsc{cl.m.sg}  \textsc{aux.1sg.prf.prs} see.\textsc{ptcp} attack at two of my player.\textsc{pl}\\
\glt `I \textsc{have} seen López attack two of my players.'
		\z
\z


Finally, there is also a piece of data that is problematic for \citeauthor{Hernanz2007}'s analysis for the  \emph{que}-less \emph{sí}-sentences. Given  that they are assumed to start out in PolP, examples like \eqref{ex:esosino} are unexpected.\footnote{These data contradict \citet[139: fn 8]{Hernanz2007}, where it is claimed that in these configurations \emph{sí} cannot co-occur with negative particles.}

\ea \label{ex:esosino} 
		Spanish (\citealt[94: 31a]{Kocher2017} from CdE) \\
\gll Eso sí no podía faltar en ninguna casa. \\
this \textsc{verum} no can.\textsc{3sg.ipfv.pst} miss in any house\\
\glt `This could \textsc{not} be missing in any house.' 
\z

The example shows that \emph{sí} can co-occur with the sentence negation particle \emph{no}. This particle is  assumed to occupy PolP,   the same position in which \citet{Hernanz2007} assumes \emph{sí}  to be originally merged.
The alternative analysis  I adopt and will be presented in \sectref{sec:presupaffc} can account for these data.  \emph{Sí} always occupies the same left-peripheral position. The only difference between the two alternative means of expressing \isi{verum} lies in the presence of \emph{que}.





\section{Outline of the present analysis}\label{sec:presupanal}
 In this section I present my analysis that uniformly accounts for all the different constructions subsumed under the label of attributive \emph{que}. This book strives to present a uniform analysis that is compositional on a structural and interpretive level  whenever possible. In very simple terms, what I mean by this is that my goal is  to develop an account in which, unless there is empirical evidence to the contrary, each element involved in the construction  is merged where it always is and does whatever it always does. 
 
 The syntactic position of the complementizer in the relevant constructions will again be mapped out in the revised version of the split CP in \eqref{struc:myrizznew} (repeated from \eqref{struc:myrizz}) that I assume for the present investigation. To repeat, the main differences between the split CP used here and the one developed by \citeauthor{Rizzi1997} are that I adopt SubP instead of ForceP and assume a lower projection termed MoodP which takes over the functions originally associated with ForceP (cf. \sectref{sec:intconcept}).

\ea\label{struc:myrizznew} {[} SubP [ TopP* [ IntP [ TopP* [ FocP [ ModP* [ TopP* [ MoodP [TopP* [ FinP [ IP ]]]]]]]]]]] 
\z

The point of departure of my analysis is the idea that in all the relevant constructions the presence of \emph{que} has the same pragmatic impact (cf. \sectref{sec:presupprag}). However, the word order suggests that the complementizer does not occupy the same position in all the constructions (see \sectref{sec:presupinitialC} to \sectref{sec:presupaffc}). One possible way of accounting for these facts, which will not be adopted here, would be to assume that a lexical item with a dedicated function is inserted directly into the different positions. This option could  account for the shared meaning and explain why the complementizer surfaces in different positions, but it comes at the cost of potential overgeneralization, and additional motivations would be required to explain why the complementizer surfaces  in these exact positions  and in combination with these exact expressions. 

I argue in this chapter for an alternative explanation, whereby attributive \emph{que} is always merged in the same position in which it is valued with the interface feature responsible for its meaning. This explanation is in line with the general idea defended in this book that there is only one lexical item \emph{que} whose meaning is determined by the syntactic position in which it is externally merged. This facilitates a unified account of the constructions involving attributive \emph{que} and the \emph{que}-initial reportatives discussed in \chapref{sec:insubint}.  

\begin{figure}
\caption{\label{struc:refquegen}Complementizer movement}
\begin{forest}
	[SubP
	[~] 
	[Sub' 
	[Sub$^0$\\que\textsubscript{attributive}, name=force]  
	[IntP
	[~] 
	[Int' 
	[Int$^0$\\\sout{que}, name=int] 
	[FocP
	[~] 
	[Foc' 
	[Foc$^0$\\\sout{que}, name=foc]  
	[ModP
	[~] 
	[Mod' 
	[Mod$^0$\\\sout{que}, name=mod] 
	[MoodP
	[~] 
	[Mood' 
	[Mood$^0$\\\sout{que}, name=mood]  
	[FinP
	[~] 
	[Fin' 
	[Fin$^0$\\\sout{que\textsubscript{attributive}}, name=fin] 
	[IP
	[\dots,roof]
	{
		\draw[->] (fin)  to[out=south west, in=south] (mood); 
		\draw[->] (mood)  to[out=south west, in=south] (mod);		
		\draw[->] (mod)  to[out=south west, in=south] (foc);
		\draw[->] (foc)  to[out=south west, in=south] (int); 
		\draw[->] (int)  to[out=south west, in=south] (force);		
	} 
	]]]]]]]]]]]]]	
\end{forest}
\end{figure}

The position  hosting the attributive feature  is at the right edge of the left periphery. This feature attracts the complementizer to its head. It does not remain in FinP but reaches its final surface position  through head-to-head movement. It will be shown over the course of the following sections that this movement is restricted by a syntactic condition: The complementizer cannot cross a phrase that is externally merged in the left periphery. The structure in \figref{struc:refquegen} illustrates the basic ideas. In what follows, I briefly present  theoretical support for each of the basic assumptions in the analysis. The detailed empirical basis is  given in \sectref{sec:presupsyn} and \sectref{sec:presupprag}.

The assumption that the complementizer starts out in the lowest projection of the left periphery, FinP, has its origin in independent observations in the literature that this projection hosts finite complementizers (cf. for instance \citealt{Belletti2009}, \citeyear{Belletti2013}, \citealt{Ledgeway2005}).  Moreover, some Romance dialects have morphologically distinct forms to express low and high merged complementizers (see e.g. \citealt{Ledgeway2005} on Southern Italian dialects and \citealt{DAlessandro2015} on Abruzzese). The data from the Corsican variety reported in \citet{Ledgeway2012} are particularly interesting  for the present investigation because sentences containing the different morphological forms have different readings. \emph{Chì}, the complementizer merged in ForceP, introduces a declarative. An example is given in \eqref{ex:corchi} where it heads a sentence embedded under a verb of saying\is{verbum dicendi}.  In contrast, \emph{chè} is merged in FinP and according to \citet{Ledgeway2012} the proposition that it introduces receives an exclamative reading. At present, I am unable to  determine whether what the author calls an exclamative reading is similar to  the interpretation that is identified for low \emph{que} in Ibero-Romance. An  interesting parallel is that they are also able to introduce main clauses, illustrated in \eqref{ex:corche}.\largerpage[-2]



\ea Corsican
\ea\label{ex:corchi}   (\citealt[175: ex i.a]{Ledgeway2012})\\ \gll Dì  a Caccara chì, à ott’ ore sì Diu vole, saremu in casa \\
		tell.\textsc{imp.2sg} to Grandma that at eight hour.\textsc{pl} if God want.\textsc{3sg.prs} we.shall.be.\textsc{1pl} in house\\
		\glt `Tell Grandma that, God willing, we shall be home by eight o’clock' 
		\ex\label{ex:corche}  (\citealt[175: ex i.b]{Ledgeway2012}) \\\gll Chè vo un caschete! \\
		that you not fall.\textsc{3pl.prs}\\
		\glt `Watch you don’t fall.'  
	\z
	\z

The empirical distribution of the Ibero-Romance complementizer in the relevant constructions (cf. \sectref{sec:presupsyn}) shows that it always occupies positions above FinP, including the position immediately above it. These  facts are explained  by assuming that the complementizer is base-generated  below the lowest head that it  surfaces in, namely FinP.  
In this position, the complementizer is valued with an interface feature that has an impact on the interpretation of the sentence in its scope. I propose  a feature that I call \emph{attributive}. My analysis is also inspired by \citet{Cuba2013}. The authors show that complement clauses of \isi{factive} and non-\isi{factive} verbs have a different interpretation and also a different structure. Another  analysis that assumes structural differences in the CP between different types of complement clauses  was  proposed before \citet{Cuba2013} by  \citet{Haegeman2004, Haegeman2006}. She focuses on the contrasts between \eqref{ex:centraladv} and \eqref{ex:peripheraladv}.\largerpage

\ea[*]{I haven’t seen Mary since she probably left her job.\\  (\citealt[1653: ex 2b]{Haegeman2006}) \label{ex:centraladv}}
\ex[]{I won’t be seeing Mary, since she probably will be leaving early today. \\(\citealt[1653: ex 3b]{Haegeman2006})\label{ex:peripheraladv}} 
\z

\citet{Haegeman2006} observes word-order restrictions in the left periphery of these sentences, illustrated here by the ungrammaticality of \emph{probably}. She adopts a cartographic framework and proposes that adverbial clauses like \eqref{ex:centraladv} are integrated at the IP-level and have a more reduced structure than adverbial clauses like \eqref{ex:peripheraladv}, which she considers to be adjoined to the host clause at a later stage in the derivation.  The latter types start with a SubP and contain essentially the full set of projections that are also found in root clauses; the former type, integrated at the IP-level,  only project a Sub and a Fin head.


\Citet{Cuba2013} make  empirical observations similar to those in \citet{Haegeman2004,Haegeman2006} for the contrast between \isi{factive} and non-\isi{factive} complement clauses.


\ea \label{ex:thinkregret}
\ea(\textit{non-referential}) \\ John thinks that this book Mary read. \\(\citealt[8: ex 10a]{Cuba2013})\label{ex:think}
		\ex (\textit{referential})  \\ * John regrets that this book Mary read. \\ (\citealt[8: ex 9a]{Cuba2013})\label{ex:regret} 
	\z
\z

The proposal by \citet{Cuba2013}  builds on a contrast in the interpretation of the two types of complement clauses.
The complement clause of a non-\isi{factive} verb like \emph{think} is interpreted as a new proposition. In contrast,  the complement clause of the \isi{factive} verb \emph{regret} is presented as part of the \isi{common ground}. A property that follows from this contrast is that  the propositional content of the complement of a \isi{factive} verb remains constant under negation of the matrix verb, while the complement of a  non-\isi{factive} verb does not. In the negated version of (grammatical) \eqref{ex:regret} \emph{John doesn't regret that Mary read this book.}, it is still presupposed that Mary read this book. In the negated version of \eqref{ex:think}, \emph{John doesn't think that Mary read this book.}, the fact that Mary read this book is not presupposed. 

According to \citet{Cuba2013}, the two types of complement sentences  differ not only in their interpretation but also in their structure. The examples in \eqref{ex:thinkregret} show that while non-\isi{factive} sentences admit a left dislocated \isi{topic}alization of the object \emph{this book} (cf. \ref{ex:think}), the same dislocation	  is ungrammatical in \isi{factive} complements (cf. \ref{ex:regret}). The authors draw the conclusion that the CP of a \isi{factive} complement  is smaller. 
In \citet{Cuba2013},  Spanish examples are also discussed in order to demonstrate that the difference in size also holds for this language. While a clitic left dislocated \isi{topic} is grammatical in a non-\isi{factive} complement clause (cf. \ref{ex:cree}), it is ungrammatical in a \isi{factive} complement clause (cf. \ref{ex:sabe}). 
 
\ea
\ea
		Spanish (\textit{non-referential}) (\citealt[9--10: ex 11c]{Cuba2013})\label{ex:cree}\\
\gll  Juan cree que ese libro$_i$ ya se lo$_i$ había leído. \\
 		Juan believe.\textsc{3sg.prs} that that book already \textsc{cl.refl} \textsc{cl.akk} \textsc{aux.3sg.ipfv.pst} read.\textsc{ptcp}\\
 	\glt `Juan believed that that book he had already read.' 
\ex
 				Spanish  (\textit{referential}) (\citealt[10: ex 12a]{Cuba2013})\label{ex:sabe}\\
 		\gll * Sabía a Juan$_i$ qué le$_i$ había prometido el decano. \\
 		~ know.\textsc{3sg.ipfv.pst} to Juan what him \textsc{aux.3sg.ipfv.pst} promise.\textsc{ptcp} the dean\\
 \glt 	`I knew what the dean had promised John.' 
 	\z
 \z
 
 
The critical example of a complement sentence under the \isi{factive} verb \emph{saber} `to know' in \eqref{ex:sabe} does not contain a complementizer in standard Spanish.  Some dialects, however, do permit the co-occurrence of a \textit{wh}-pronoun and a complementizer  below \isi{factive} verbs as in \eqref{ex:informatico}, in which case  the complementizer follows \textit{qué}. This is an indication that the complementizer must be located lower than FocP.

\ea\label{ex:informatico} Peruvian Spanish\\ 
\gll ¿Sabes qué que le dice una madre a su hijo informático?  \\
know.\textsc{2sg.prs} what \textsc{que} him tell.\textsc{3sg.prs} a mother to her son {computer scientist} \\
\glt `You know what (that) a mother tells her computer scientist son?' (CdE)
\z

\Citet{Cuba2013} use the terms \emph{referential} vs. \emph{non-referential}  to distinguish between the two types of complement sentences. The logic behind these terms is that a sentence introduced by a \isi{factive} verb is considered by the authors to be referential in the sense that there is  a sentence in the \isi{common ground} that it refers to. The \isi{common ground} can be understood as the set of propositions to which all the speech participants are committed. This means that by marking a proposition as part of the \isi{common ground}, the speaker also claims that the hearer shares this \isi{commitment}. In the  constructions that are at the center of the present chapter,  it is precisely the attributive  \isi{commitment} to the hearer that is highlighted (see \sectref{sec:presupprag}). I therefore choose to call the feature \emph{attributive} rather than \emph{referential} to reflect this fact.    \Citet{Cuba2013} do not adopt a cartographic approach. The larger structure of non-referential sentences is modeled by assuming  that the CP is selected by a small cP, paralleling the vP-shell analysis (cf. \citealt{Chomsky1955}, \citealt{Larson1988, Larson1990}).  

\begin{sloppypar}
\citet{VillaGarcia2015} proposes a cartographic adaptation of the insights presented in \citet{Cuba2013} that is reminiscent of the account provided in \citet{Haegeman2004, Haegeman2006} for adverbial complements. He suggests that the difference in size is represented in the number (and nature) of functional heads projected in each type of complement clause. In his view, the left periphery of a  \isi{factive} complement clause  consists only of a FocP and a FinP while a  non-\isi{factive} complement clause has an additional ForceP and a TopicP.
\end{sloppypar}


\begin{table}[htb]
	\centering
	\begin{tabularx}{\textwidth}{l  XX}
\lsptoprule
	&referential 	& non-referential \\
&($\sim$ factive) & ($\sim$ non factive)\\
\midrule 

interpretation & new proposition & common grounded\\
\citet{Cuba2013} &  {[cP ${-ref}$ [CP ]]} & {[CP ${+ref}$ ]}\\
 \citet{VillaGarcia2015}&  {[ ForceP [ TopicP \newline [ FocusP [ FinP ]]]]} & {[ FocusP [ FinP ]]} \\
\lspbottomrule
\end{tabularx}
	\caption{Properties of referential and non-referential complements in  \citet{Cuba2013} and \citet{VillaGarcia2015}}
\end{table}

\Citeauthor{Cuba2013}'s goal is to capture the contrast in embedded sentences, but I propose that their insights can also be adapted to account for unembedded sentences, which are the main focus of this book. The novel aspect of my  approach is therefore that the property of referentiality, understood in the sense of \citet{Cuba2013}, is not restricted to embedded sentences but also plays a role in the interpretation of unembedded sentences.  I furthermore suggest that the  feature responsible for the interpretation is anchored within the  cartographic structure. This means that \emph{attributive} is not a feature linked to an entire CP, as \emph{referential} is in \citet{Cuba2013}, but  a feature hosted by a single functional head that is part of  the split CP.  My proposal, in line with the discussion above, is that the feature is hosted in the lowest head of the left periphery, FinP. The presence of the feature requires the merger of a complementizer. As a consequence, the sentence in the scope of the complementizer is interpreted as a non-discourse new proposition.  If the feature is not present,  no complementizer is merged in FinP and the sentence does not receive a non-discourse new interpretation. The exact pragmatic contribution of attributive \emph{que} will be discussed in \sectref{sec:presupprag}. 


Another assumption that I make in my analysis is that the complementizer does not remain in FinP but moves through the left periphery. This idea is mainly based on the empirical facts given in \sectref{sec:presupsyn}, which show that the complementizer surfaces at different positions in the different constructions. It is by no means unprecedented, however, for a complementizer to move: \citet{Rizzi1997}, \citet{Poletto2000}, \citet{Roberts2001}, \citet{Ledgeway2005} and  \citet{Belletti2009, Belletti2013} all independently assume the  merger of a complementizer in a lower CP position and movement to a higher one. 
In order to explain this movement, I adopt the idea presented in \citet{Belletti2009, Belletti2013}, who states that complementizer movement always obtains in languages where the same C-element realizes the content of Fin and Force (equivalent to Sub in my terminology), which is the case in all three languages under investigation. How this insight should be  modeled, perhaps  via  a feature-driven conception of movement, is left open for future research.  
  	

As a consequence of the complementizer movement, my analysis predicts that \emph{que} can surface in any of the split CP heads. The idea that a complementizer can occupy positions other than ForceP or FinP is  implicit in the proposals by \citet{Haegeman2004, Haegeman2006} and \citet{VillaGarcia2015}  and finds further support in a number of different works (cf. for instance \citealt{Roussou2000,Roussou2010, Gutierrez-Rexach2001, Brovetto2002, Rodriguez-Ramalle2003, DemonteSoriano2009, Ledgeway2005, Villa-Garci2012,Villa-Garci2012a, VillaGarcia2015, Corr2016}). 

The complementizer-movement analysis I propose predicts that \emph{que} can in principle move all the way up to the left edge of the functional field, SubP.\footnote{In \citet{Corr2016} it is suggested that the complementizer can even reach a position in her performative field above the CP.} However, the word order observed in the different constructions shows that the complementizer follows rather than precedes certain left-peripheral material, which would not be expected if attributive \emph{que} always reached SubP. Furthermore, the data that will be provided in \sectref{sec:presupsyn} show that  the complementizer surfaces at different heights in the different constructions. This suggests that the movement is conditioned in some way. In light of the empirical facts, my generalization is that the movement is inhibited by base-generated material. In other words, a complementizer cannot move on to the next projection if the specifier of the projection that currently hosts the complementizer is occupied by an externally-merged phrase. As a consequence, I assume that all the elements that surface immediately above attributive \emph{que} are externally merged in-situ. The reasoning behind this idea is given  in \sectref{sec:presupsyn}. The following examples show the analyses I assume for the different constructions.  

For attributive \emph{que} in declaratives \eqref{ex:initialsub} and \isi{polar question}s \eqref{ex:pqsub}, since there is no material hindering it, the movement of complementizer is unrestricted and consequently \emph{que} does in fact reach the highest projection SubP.

\ea 
\ea\label{ex:initialsub}
		Spanish\\
\gll Tranquilo, tío. {\ob}\textsubscript{SubP} Que\textsubscript{attributive}] \dots  {\ob}\textsubscript{FinP} \sout{que\textsubscript{attributive}}] {\ob}\textsubscript{IP} no muerdo.] \\
calm man {} \textsc{que} {} \textsc{que} {} not bite.\textsc{1sg.prs}\\
\glt `Chill, man. I don't bite.'
\ex\label{ex:pqsub}

\gll {\ob}\textsubscript{SubP} Que\textsubscript{attributive}] \dots  {\ob}\textsubscript{FinP} \sout{que\textsubscript{attributive}}] {\ob}\textsubscript{IP} no muerdes?] \\
{} \textsc{que} {} \textsc{que} {} not bite.\textsc{2sg.prs}\\
\glt `You don't bite?'
\z
\z

In \is{wh-exclamative@\textit{wh}-exclamative}\textit{wh}-exclamatives,  the \textit{wh}-expression is analyzed as being merged directly in FocP and the complementizer movement comes to a halt at the head of this phrase (cf. \ref{ex:exclfoc}).
\ea\label{ex:exclfoc}
Portuguese\\ 
\gll {\ob}\textsubscript{FocP} Que chato que\textsubscript{attributive}] \dots  {\ob}\textsubscript{FinP} \sout{que\textsubscript{attributive}}] {\ob}\textsubscript{IP} é.] \\
 {}	how annoying \textsc{que} {}  \textsc{que} {}  be.\textsc{3sg.prs}\\
		\glt `How annoying this is.'
\z

Similarly, \isi{epistemic and evidential modifier}s in AdvC as well as the \isi{verum} particle \emph{sí} in AffC are assumed to be merged directly in the left-peripheral position  in which they appear in the surface structure. The analyses I assume for these constructions are given in \eqref{ex:advcmod} and \eqref{ex:simood}. 

\ea \label{ex:advsi}
	\ea\label{ex:advcmod}
	Catalan\\
	\gll {\ob}\textsubscript{ModP} Segur que\textsubscript{attributive}] \dots  {\ob}\textsubscript{FinP} \sout{que\textsubscript{attributive}}] {\ob}\textsubscript{IP} son amics.] \\
		{}	sure \textsc{que} {}  \textsc{que} {}  be.\textsc{3pl.prs} friend.\textsc{pl}\\
		\glt `Surely, they are friends.'
				\ex\label{ex:simood}

				\gll {\ob}\textsubscript{MoodP} Sí que\textsubscript{attributive}] \dots  {\ob}\textsubscript{FinP} \sout{que\textsubscript{attributive}}] {\ob}\textsubscript{IP} son amics.] \\
		{}	\textsc{\isi{verum}} \textsc{que} {}  \textsc{que} {}  be.\textsc{3pl.prs} friend.\textsc{pl}\\
		\glt `They \textsc{are} friends.'
	\z
\z

To repeat the main idea, I assume that in (\ref{ex:exclfoc}--\ref{ex:advsi}), intervening material blocks the movement of the complementizer to the next phrase, which results in the surface word order observed in each of the constructions. The example in \eqref{ex:movec} is more complex because in addition to the evidential modifier\is{epistemic and evidential modifier} preceding \emph{que}, there is also a clitic left-dislocated \isi{topic} below the attributive complementizer.

\ea\label{ex:movec}
Spanish\\
\gll  Claro que a Juan$_i$ lo$_i$ invitaron. \\
	clear \textsc{que} to Juan \textsc{cl.m.sg} invite.\textsc{3pl.prf.pst}\\
	\glt `Clearly, they invited Juan.'
\z

The analysis for examples like \eqref{ex:movec} is illustrated in the structure in \figref{struc:movec}. The complementizer starts out in Fin$^0$ where it is valued with the attributive feature. It then moves from head to head. The complementizer is able to cross the filled specifier of TopP because \emph{a Juan} is a clitic left-dislocated \isi{topic} that is moved from the IP to its current position. In turn, the complementizer cannot cross the filled specifier of  ModP because \emph{claro} is directly merged in the left periphery. This is why  any further head movement of the complementizer is blocked at this point.

\begin{figure}
\caption{\label{struc:movec}Analysis of \eqref{ex:movec}; crossed out lines represent a potential  movement that does not take place}
	\resizebox{\textwidth}{!}{\begin{forest}
		[SubP
		[~] 
		[Sub' 
		[Sub$^0$, name=force]  
		[\dots, name=dots
	[,phantom ]
		[FocP
	[~] 
	[Foc', s sep={5em}
	[Foc$^0$, name=foc]  
		[ModP, s sep={5em}
		[Claro, draw, name=spmod,label={below:\emph{moved}}] 
		[Mod' 
		[Mod$^0$\\que\textsubscript{attributive}, name=mod] 
		[ToP, s sep={3em}
		[a Juan$_i$, name=sptop, draw, label={below:\emph{merged}}] 
		[Top' 
		[Top$^0$\\\sout{que}, name=top] 
		[MoodP
		[~] 
		[Mood' 
		[Mood$^0$\\\sout{que}, name=mood]  
		[FinP
		[~] 
		[Fin' 
		[Fin$^0$\\\sout{que\textsubscript{attributive}}, name=fin] 
		[IP
		[lo invitaron t$_i$.,roof, name=juan]
		{   \begin{scope}[decoration={crosses,shape size=1.5mm}]
			\draw[->] (fin)  to[out=west, in=270] (mood); 
			\draw[->] (mood) to[out=west, in=270] (top);		
			\draw[->] (top)  to[out=west, in=270] (mod);
			\draw[->] (mod)  to[out=west, in=270] (foc);	
			\draw[decorate] (mod)  to[out=west, in=270] (foc);	
% 			\draw[<-, dotted] (sptop)--++(-2em,-0ex)--++(-3em,0pt)
% 			node[anchor=east,align=center]{\emph{moved}};
% 			\draw[<-, dotted] (spmod)--++(-2em,-0ex)--++(-3em,0pt)
% 			node[anchor=east,align=center]{\emph{merged}};
			\end{scope}     
		}
		]]]]]]]]]]]]]]	
	\end{forest}}
\end{figure}


To conclude this section, I compare my own analysis to those discussed in \sectref{sec:presupeval}. In Table \ref{tab:myanalpresup}, I summarize the main aspects of my  analysis and in Table \ref{tab:companalpresup} the main aspects of the central analyses from the literature. 

In my analysis,  I treat \emph{que}  as a complementizer. In this book, a  complementizer in Ibero-Romance is defined as an underspecified element of the form \emph{que} that occupies a left-peripheral head position and that acquires its functional interpretation from a feature in the location in which the complementizer is merged  (see \sectref{sec:intconcept}). This definition allows us to maintain  that in the current constructions we are dealing with the same lexical item as in unembedded \emph{que}-initial reportatives and in any regular embedded sentences. 

\begin{table}
	\begin{tabularx}{\textwidth}{l X X X XX}
	\lsptoprule
			& \emph{que}-initial\newline declaratives & polar \newline questions & \textit{wh}-exclamatives & AdvC & AffC\\
	\midrule
			nature of \emph{que} & \multicolumn{5}{c}{\leavevmode\leaders\hrule depth-2pt height 2.4pt\hfill\kern0pt\ complementizer \leavevmode\leaders\hrule depth-2pt height 2.4pt\hfill\kern0pt}\\
			location& FinP>...\newline>SubP & FinP>...\newline>SubP  & FinP>...\newline>FocP & FinP>...\newline>ModP & FinP>...\newline >MoodP\\
			interpretation &\multicolumn{5}{c}{\leavevmode\leaders\hrule depth-2pt height 2.4pt\hfill\kern0pt\ attributive feature in FinP \leavevmode\leaders\hrule depth-2pt height 2.4pt\hfill\kern0pt}\\
	\lspbottomrule
	\end{tabularx}
	\caption{\label{tab:myanalpresup} Attributive \emph{que}  in the present analysis}
\end{table}
\begin{table}
	\begin{tabularx}{\textwidth}{l X X X }
	\lsptoprule
		& \multicolumn{2}{c}{\emph{que}-initial declaratives} & polar questions
		\\
		\midrule
		author &  \multicolumn{2}{c}{{\citet{Corr2016}}} & {\citet{PrietoRigau2007}} \\
		nature of \emph{que} & \multicolumn{2}{c}{\leavevmode\leaders\hrule depth-2pt height 2.4pt\hfill\kern0pt\ illocutionary complementizer \leavevmode\leaders\hrule depth-2pt height 2.4pt\hfill\kern0pt} & complementizer \\
		location & exclamative: EvalP>SAlowP  & conjunctive:\newline DeclP>EvidP> EvalP>SAHighP & FinP \\
		interpretation & \multicolumn{2}{c}{\leavevmode\leaders\hrule depth-2pt height 2.4pt\hfill\kern0pt\ features in UP \leavevmode\leaders\hrule depth-2pt height 2.4pt\hfill\kern0pt}& operator in Force  \\
		\midrule
		& \textit{wh}-exclamatives & AdvC & AffC \\
		\midrule
		author &   \citet{DemonteSoriano2014} & {\citet{Cruschina2018}} & {\citet{Hernanz2007}}\\
		nature of \emph{que} &  complementizer & complementizer & complementizer\\
		location & FinP & ForceP	& ForceP\\
		interpretation & none & attributed to \newline  active performative structure & attributed to \newline ForceP  \\
	\lspbottomrule
	\end{tabularx}
	\caption{\label{tab:companalpresup}Attributive \emph{que} in the previous  analyses}
\end{table}



The different functions and  syntactic behaviors follow from the assumption that the complementizer is merged in different syntactic projections where it receives  its featural values. In the case of attributive \emph{que}, we furthermore observe a greater syntactic mobility than in subordinate \emph{que}. This is guaranteed by the low merge position, which enables  upward movement through the left periphery. This movement is not possible in the case of subordinate \emph{que} because it  is already merged  in the highest position of the functional field.
Most previous analyses are in agreement with my assumption that \emph{que} is a complementizer. However,  some authors further state that this complementizer is different from the element that introduces subordinate clauses. \citet{Corr2016} considers it an illocutionary complementizer that is homophonous with other complementizers.     \citet{DemonteSoriano2009}  assume that the complementizer involved in \is{wh-exclamative@\textit{wh}-exclamative}\textit{wh}-exclamatives is distinct from other homophonous elements of the form \emph{que}; in their approach the difference is expressed in terms of different merge positions. Notably, this is  very close to what  I assume in this book.

In my  analysis, the final landing site of the complementizer in the different constructions is derived straightforwardly by  head movement that is inhibited by base-generated material. This means that the complementizer, while always merged in FinP, ends up at different heights within the left periphery in each construction. \citet{Corr2016} also assumes a movement derivation for  \emph{que}-initial declaratives  in her analysis. The complementizer, however, starts out in a  higher position in one head of her split ForceP (DeclP, EvidP, EvalP) and reaches positions in her performative UP. As noted in \sectref{sec:presupeval}, the motivation for this movement is mainly theoretical. My analysis in which \emph{que} reaches  SubP also captures the empirical data.  \citet{PrietoRigau2007}  propose that \emph{que} is merged in FinP in polar questions and do not assume further movement. I showed in \sectref{sec:presupeval} that the analysis provided in \citet{PrietoRigau2007} fails to account for  some critical data.\largerpage[1] My analysis, on the other hand, can account for these data because I assume that  just as in declaratives, in the absence of intervening base-generated material,  the complementizer reaches SubP in polar questions. In the account proposed for \is{wh-exclamative@\textit{wh}-exclamative}\textit{wh}-exclamatives in \citet{DemonteSoriano2009}, \emph{que} is also analyzed as located in FinP. The empirical facts again pose some problems for this analysis, but support my own assumption that the complementizer movement stops in the head of FocP, adjacent to the \textit{wh}-expression (see \sectref{sec:presupeval}). The analyses  for \emph{que} with epistemic and evidential  modifiers  by \citet{Cruschina2018}  and for \emph{que} in \isi{verum} constructions put forward by \citet{Hernanz2007} both assume a very high position for the complementizer in ForceP. Both of these analyses, among others, are hard to reconcile  with the data that show that the constructions can appear in embedded relative clauses (see \sectref{sec:presupeval}). Again, my own analysis makes the correct predictions with respect to these data.

Finally, the backbone of my analysis is  that in all the constructions under investigation in this chapter, the presence of the complementizer has the same basic effect on the interpretation of the sentence in its scope. In my account this common interpretation is linked to the interface feature \emph{attributive}, with which the complementizer is valued. A more detailed characterization of the interpretation of the constructions is given in \sectref{sec:presupprag}, but the basic idea is that the attributive complementizer attributes to the hearer a \isi{commitment} to the proposition. Some of the previous analyses do not assume that \emph{que} has any  special meaning. Others represent the apparently different sorts of \emph{que} in a similar way as I do here, namely as different features in the syntactic structure. The accounts differ in whether they link these features to the complementizer or to other material or properties of the constructions. In \citeauthor{Hernanz2007}'s analysis of \isi{verum} constructions, an interpretative impact similar to factivity is attributed to the functional head that hosts the complementizer, which is ForceP in her analysis. Among the accounts for \is{wh-exclamative@\textit{wh}-exclamative}\textit{wh}-exclamatives,  \citet{Castroviejo2006} and \citet{DemonteSoriano2009} do not claim that \emph{que} gives rise to a special interpretation. In turn,  \citet{Ambar2003} states that the syntactic position that \emph{que} is  merged in guarantees a \isi{factive} interpretation of the content of the \is{wh-exclamative@\textit{wh}-exclamative}\textit{wh}-exclamatives. The feature responsible for this interpretation is also present and checked in her derivation of \emph{que}-less \is{wh-exclamative@\textit{wh}-exclamative}\textit{wh}-exclamatives. It therefore appears  that there is  no particular interpretive function attributed directly to \emph{que}  in \citet{Ambar2003} either.  In the analysis of \emph{que}-initial sentences by \citet{Corr2016}, a combination of multiple UP-features checked by the complementizer  gives rise to the interpretation. In \citet{PrietoRigau2007}, while \emph{que} itself is not equipped with interpretive features and is considered optional,  an operator in ForceP expressed through prosodic means is  responsible for the different interpretations. In \citet{Cruschina2018}, the special interpretation is also not linked to the position in which \emph{que} is merged, but to the activation of the performative structure. Thus it appears that the authors connect the interpretive effect more strongly  to the modifier, merged directly in the CP-external structure, than to the complementizer itself located in ForceP.

In  \sectref{sec:presupsyn}, I outline the empirical evidence in support of my syntactic analysis. I deal with each construction individually and show  how the predictions made by my analysis are confirmed by the word order that we observe. 
\section{The syntax of attributive \emph{que}}\label{sec:presupsyn}
The analysis proposed in \sectref{sec:presupanal} is based on three assumptions. First,   the complementizer in the  constructions involving attributive \emph{que} is always merged in the same position: FinP. Second,  the complementizer moves from head to head through the left periphery. Third,  the movement of the complementizer  is inhibited by a base-generated phrase in the specifier of the  projection that the complementizer currently occupies. 

The first two assumptions are  based on  theoretical and empirical considerations. A central  observation, in this context, is that the pragmatic impact of  the complementizer is the same in all the different constructions. A \isi{commitment} to the proposition in the scope of attributive \emph{que} is ascribed to the hearer (cf. \sectref{sec:presupprag}). Since the complementizer surfaces in different positions in the constructions, there are  two options. The first option would be to assume one lexical item that is externally merged in different left-peripheral positions. The second  would be to assume that the complementizer is always merged in one and the same projection and that the surface  positions in the different constructions are reached through complementizer movement (see also \citealt{Rizzi1997}, \citealt{Poletto2000}, \citealt{Roberts2001}, \citealt{Ledgeway2005}). In this book, I argue in favor of the second option, which  is a better fit in light of the general ideas proposed here. 




The following sections contain the empirical support for my analysis. I map out the position in which the complementizer appears in the cartographic structure and   show that its inability  to cross a base-generated phrase explains  the word order we observe in the different constructions. Moreover, I present further syntactic properties of the  constructions and show how these can be accounted for in the present analysis. In \sectref{sec:presupinitialC}, I focus on the syntactic properties of attributive \emph{que}-initial declaratives and in \sectref{sec:presuppqC} on attributive \emph{que}-initial \isi{polar question}s. The following sections are dedicated to the structure of attributive \emph{que} when following certain left-peripheral phrases. In \sectref{sec:presupexclc}, I deal with  \is{wh-exclamative@\textit{wh}-exclamative}\textit{wh}-exclamatives, in \sectref{sec:presupadvc} with \emph{que} following \isi{epistemic and evidential modifier}s (AdvC) and in \sectref{sec:presupaffc} with \emph{que} in \isi{verum} sentences (AffC).

\subsection{\emph{Que}-initial declaratives}\label{sec:presupinitialC}

The basic idea of the attributive \emph{que}  analysis  is that its movement is only blocked by externally merged material in the specifier of a projection. One prediction that follows from this is that if  nothing intervenes, the complementizer  can  reach the highest projection of the functional field, i.e. SubP. In this section, I show  that this is precisely where it surfaces in attributive \emph{que}-initial declaratives.  As a consequence, on a superficial level,  these \emph{que}-initial sentences show a  structure parallel to that of \emph{que}-initial reportatives. Ultimately, it is  only the context that disambiguates between the two readings of the complementizer.
The precise discourse contribution of attributive \emph{que} is discussed in greater depth in \sectref{sec:presupprag}. Examples \eqref{ex:nouanysref} and \eqref{ex:nouanysrep} illustrate the difference between \emph{que}-initial reportatives and attributive \emph{que}-declaratives.
\ea  \label{ex:nouanysref} 
Catalan\\
	\gll Pare: És dolent demanar a un fill que llegeixi un llibre?\\
	father be.\textsc{3sg.prs} bad demand.\textsc{inf} of a son that read.\textsc{3sg.sbjv.prs} a book\\
	\gll Mare: Que té nou anys.\\
	mother \textsc{que} have.\textsc{3sg.prs} nine year.\textsc{pl}\\
	\glt `Father: Is it a bad thing to ask your son to read a book? Mother: He's only nine!' (ebook-cat) 
\ex  \label{ex:nouanysrep}
Catalan\\
 	\gll Mare: Té nou anys. \\
	mother have.\textsc{3sg.prs} nine year.\textsc{pl}\\
	\gll Pare: Eh?\\
	father huh\\
	\gll Mare: Que té nou anys.\\
	mother \textsc{que} have.\textsc{3sg.prs} nine year.\textsc{pl}\\
	\glt `Mother: S/he is nine years old. Father: Huh? Mother: [reportative:] S/he is nine years old.' 
\z 

In \eqref{ex:nouanysref} the reaction of the mother that their son is only nine years old does not contain new information for the father, who can be expected to be aware of the age of his own son. The mother's motive was therefore not to inform the father of the age of his son. The \isi{commitment} to the proposition introduced by \emph{que} is attributed to the hearer, the father. With her utterance, the mother  communicates that she does not consider  the books that the father gave to the son to be age appropriate.
In \eqref{ex:nouanysrep}, the same utterance by the mother results in a different reading. In this context, the father's reaction suggests that he did not understand the mother's utterance. The function of \emph{que} at the beginning of the last sentence of this mini-dialog is therefore to mark that the sentence is a reported version of the previous statement.  

The decision to analyze attributive \emph{que} as located in SubP is not  based on theoretical considerations alone. There are also  empirical facts that lead to the same conclusion. First, attributive \emph{que} precedes  left-dislocated \isi{topic}s (cf. \ref{ex:initialquecllda}). This is expected because moved phrases, as these \isi{topic}s are, do not constitute an obstacle for the head movement of the attributive complementizer.

\ea\label{ex:initialqueclld}Spanish
\ea\label{ex:initialquecllda}
\gll ¡Pues has sido muy rápida! {\ob}\textsubscript{SubP} Que] {\ob}\textsubscript{TopP} el vídeo$_i$] lo$_i$ puso esta misma semana, jajaja. \\
		well \textsc{aux.2sg.prf.prs} be.\textsc{ptcp} very fast {} \textsc{que} {} the video \textsc{cl.m.sg} put.\textsc{1sg.prf.pst} this same week hahaha\\
		\glt `Well, you've been very fast! The video, he just put it up this same week, haha.' (CdE)
		\ex 

		\gll Chicas, si son fashionistas, abran la mente! {\ob}\textsubscript{SubP} Que] {\ob}\textsubscript{TopP} la moda$_i$] la$_i$ hacen todos! \\
		girl.\textsc{pl} if be.\textsc{3pl.prs} fashionista.\textsc{pl} open.\textsc{3pl.imp} the mind {} \textsc{que} {} the fashion \textsc{cl.f.sg} make.\textsc{3pl.prs} everyone\\
		\glt `Girls, if you consider yourselves  fashionistas, open your mind! Fashion is made by everyone.' (CdE)
	\z
\z	


A schematic derivation of the example in \eqref{ex:initialquecllda} is given in \figref{struc:initialquecllda}. For ease of exposition, the intermediate projections through which the complementizer moves are not displayed.

\begin{figure}
	\caption{\label{struc:initialquecllda}Analysis of \eqref{ex:initialquecllda}}
	\begin{forest}
		[SubP, name=spsub
		[~]
		[Sub' 
		[Sub$^0$\\Que\textsubscript{attributive}, name=sub]
		[TopP
		[el vídeo$_j$, draw, name=sptop, label={below:\emph{moved}}] 
		[Top' 
		[Top$^0$\\\sout{que}, name=top] 
		[\dots, name=dots
		[,phantom ]
		[FinP
		[~] 
		[Fin' 
		[Fin$^0$\\\sout{que\textsubscript{attributive}}, name=fin] 
		[IP,		
		[lo puso t$_j$ esta\\ misma semana.]
		{
			\draw[-] (fin)  to[in=south west, out=south west]	(1.8,-8); 
			\draw[-] (1.8,-8)  to[in=south west, out=south west]	(1.8,-7); 
			\draw[->] (1.8,-7) to[in=south west, out=south west]	(top); 
			\draw[->] (top)  to[out=south west, in=south ] (sub);		
				}
		]]]]]]]]	
	\end{forest}
\end{figure}


A further piece of evidence in favor of the assumption that attributive \emph{que} reaches the highest position in the left periphery is that it is impossible to embed attributive \emph{que}-initial  declaratives (cf. also \citealt{Corr2016}), hence the ungrammaticality of \eqref{ex:doubleque}.

\ea\label{ex:doubleque}
Catalan\\
\gll La mare va dir que (*que) té nou anys. \\
	the mother \textsc{aux.3sg.prf.pst} say that \textsc{que} have\textsc{3sg.prs} nine year.\textsc{pl}\\
 \glt `The mother said that (*that) s/he is nine years old.'
\z

Notably, it is not because attributive \emph{que} is disallowed in embedded contexts, as can be seen in \eqref{ex:advembed} where AdvC is embedded (see also \sectref{sec:presupexclc}, \sectref{sec:presupadvc} and \sectref{sec:presupaffc} where further examples illustrating embedded attributive \emph{que} constructions are discussed).
\ea\label{ex:advembed}
European Portuguese\\ 
\gll Só queria dizer que obviamente que, por falta de informação, podemos, por vezes, fazer dietas mais tontas. \\
	just want.\textsc{1sg.ipfv.pst} say that obviously \textsc{que} for lack of information can.\textsc{1pl.prs} at times  do diet.\textsc{pl} more stupid\\
	\glt `I just wanted to say that obviously, for lack of information, at times we can end up doing stupider diets.' (CdP)
\z



There are different theoretical alternatives to account for the fact that a sequence of two \emph{que}s is disallowed in embedded contexts. One option is to assume that one of the complementizers, including its value, is deleted at PF. It is most likely that the  complementizer with the attributive feature would be deleted, since the subordinate feature needs to be visible in syntactically selected sentences (cf. \sectref{sec:insubanalysis}). A different way of accounting for the facts  is to assume that one \emph{que} can carry multiple values. In this case, the  complementizer that is visible in the structure could at the same time be valued with the attributive feature, picked up in FinP, and the subordinate feature, picked up in SubP. Which of these alternatives proves more  adequate is not a central concern here and is left aside for future research.


Attributive \emph{que} can furthermore introduce the answer particles Catalan and Spanish \emph{sí}, Portuguese \emph{sim} and Catalan and Spanish \emph{no},  Portuguese \emph{n\~{a}o} \eqref{ex:emphshort}. 

\ea \label{ex:emphshort} 
\ea Brazilian Portuguese\\ 
\gll -- Solfieri, não é um conto, isso tudo? -- Pelo inferno, que não! \\
 {} Solfieri not be.\textsc{3sg.prs} a swindle this all {} for.the hell \textsc{que} no \\
 \glt ` -- Solfieri, isn't all this a swindle? --- Hell no!' (CdP)

\ex
		Spanish\\
\gll La multitud responde: Que sí, que sí. \\
the crowd answer.\textsc{3sg.prs} \textsc{que} yes \textsc{que} yes\\
\glt `The crowd answers: Yes. Yes.' (CdE)
\pagebreak\ex
Catalan\\
\gll CARLES: Home no, tampoc no cal. PEP: Que sí, home. Que sí.  \\
Carles man no either not {is necessary} Pep \textsc{que} yes man \textsc{que} yes \\
\glt `Carles: Man, don't, it's not worth it either. Pep: Yes it is, man. Yes it is.' (caWac)
\z
\z

The  effect is that the  affirmation or negation is emphatic and in  particular the fact that the hearer shares the \isi{commitment} is stressed. This is in line with the general contribution of attributive \emph{que} assumed in this book (cf. also \sectref{sec:presupprag}).  Concerning their syntactic structure, I adopt the idea that these answer particles are merged in a polarity position sandwiched between the lowest head of the left periphery, FinP, and the IP (cf. \citealt{Laka1990}, \citealt{Zanuttini1997}, \citealt{Martins2006, Martins2007,Martins2013},  \citealt{Hernanz2007}, \citealt{Batllori2008}, among many others), for which, in line with \citet{Batllori2008}, I use the label Pol(arity)P.

\ea\label{struc:answerpart} {\ob}\textsubscript{SubP} Que$_i$] ...{\ob}\textsubscript{FinP} t$_i$] {\ob}\textsubscript{PolP} sí/sim/no/não.]
\z

\begin{sloppypar}
There are no strong theoretical considerations behind this terminological choice and I am open to adopting other notations such as \citeauthor{Laka1990}'s $\Sigma$P used by \citet{Martins2013}, if they prove more adequate. 
\end{sloppypar}

In my analysis given in \eqref{struc:answerpart}, I assume that  attributive \emph{que} moves to SubP in these cases too. This  should be taken as a cautious proposal because further investigation into the size and internal make-up of the syntactic structure of  short answers could show that some adjustment is necessary. The underlying assumption of the structure in \eqref{struc:answerpart} is  that answer particles appear in a structure that projects a left periphery. One prediction that follows is that   the answer particles should be preceded by left-peripheral material. Additionally, if attributive \emph{que} is present,  it should be subject to the same restrictions as are observed in full sentences. That this is indeed the case will be demonstrated in \sectref{sec:presupadvc} and \sectref{sec:presupaffc}, where I will show that the answer particles are also compatible with AdvC and AffC.

\emph{Que}-inial declaratives of the attributive type are attested in all three  languages, although my (subjective) impression is that they are less frequent in Portuguese than in Spanish and Catalan. An even stronger contrast arises  with respect to the occurrence of attributive \emph{que} in \isi{polar question}s, which is covered in  \sectref{sec:presuppqC}. This is a very common construction in  Catalan, but less frequent in Spanish and virtually unattested in Portuguese.

\subsection{\emph{Que}-initial \isi{polar question}s}\label{sec:presuppqC}
For  \emph{que} in \isi{polar question}s, the same reasoning holds as for \emph{que} in declaratives: The complementizer moves from head to head unrestricted and reaches the highest projection of the left periphery, if there is no externally-merged phrase in the specifier of any of the intermediate projections. Since this is generally the case in the \emph{que}-initial \isi{polar question}s under investigation, it follows that the position I assume for the complementizer here coincides with the position of  \emph{que} in the declaratives discussed in \sectref{sec:presupinitialC}.  The observed word order again results from the fact that the complementizer movement is not inhibited and thus reaches SubP in both \emph{que}-initial declaratives and \emph{que}-initial \isi{polar question}s. This permits a unified account that makes use of the same  mechanisms presented in \sectref{sec:presupanal} without requiring further stipulations. The relevant points are shown in \figref{struc:pqque}, where again the intermediate positions through which \emph{que} passes are omitted in the structure.

\begin{figure}
	\caption{\label{struc:pqque}Analysis of a \emph{que}-initial polar question}
	\begin{forest}
		[SubP, name=spsub
		[~]
		[Sub' 
		[Sub$^0$\\Que\textsubscript{attributive}, name=sub]
		[\dots, name=dots
		[,phantom ]
		[FinP
		[~] 
		[Fin' 
		[Fin$^0$\\\sout{que\textsubscript{attributive}}, name=fin] 
		[IP,		
		[hi es la Lola?]
		{\draw[-] (fin)  to[out=south west, in=south west]	(-0.7,-5.8); 
			\draw[-] (-0.7,-5.8) to[out=south west, in=south west]	(-1.2,-4.8); 
			\draw[->] (-1.2,-4.8)  to[out=south west, in=south west]	(sub);}
		]]]]]]	
	\end{forest}
\end{figure}


It is important to note here that I am not  saying that the syntactic structure of interrogatives and declaratives is the same. There are syntactic and prosodic differences that suggest that the two clause types have different derivations; however, these differences are not a central concern for the present discussion. The focus of this section is merely the syntactic behavior of attributive \emph{que}. The argument I put forward is only that it behaves in  essentially the same way in \isi{polar question}s and declaratives.

There are no obvious differences between the syntactic properties of \isi{polar question}s with  \emph{que} and those without. It therefore seems that \emph{que} is not  required  for syntactic reasons. My proposal is instead that attributive \emph{que} has a pragmatic function in polar questions too. These questions are not neutral: The speaker in fact uses them when she is \isi{bias}ed toward a positive answer (cf. also \citealt{Kocher2017a}). \emph{Que} in \isi{polar question}s  is often employed when there is  contextual evidence that makes the speaker suspect that the answer to the question is going to be ``yes''.

\ea\label{ex:queplou}
Catalan\\
\gll La Caterina va entrar i va córrer cap al lavabo amb el paraigua que regalimava. -- Que plou? \\
the Caterina \textsc{aux.3sg.prf.pst} enter and \textsc{aux.3sg.prf.pst} run {in direction} {to the} bathroom with the umbrella that drip.\textsc{3sg.ipfv.pst} {} \textsc{que} rains\\
\glt `Catarina entered and ran to bathroom with a dripping umbrella. -- Is it raining?' (ebook-cat)
\z

This can be seen in example  \eqref{ex:queplou}. Here the dripping umbrella functions as (indirect) contextual evidence  that it is raining. This  leads the speaker to expect a positive answer to her question and makes  the use of \emph{que} acceptable. Further discussion of the interpretation of \emph{que} in \isi{polar question}s follows in \sectref{sec:presupprag}, where  I  offer an explanation that allows a unified account of the discourse contribution across all constructions involving attributive \emph{que}.  The seemingly different effect in \isi{polar question}s is shown to result from more general pragmatic differences between  questions and assertions.

\emph{Que}-initial \isi{polar question}s have been mainly discussed in connection with Catalan. While many authors consider them disallowed in Spanish (for instance \citealt{Mioto2003}, \citealt{PrietoRigau2007},  \citealt{Rodriguez-Ramalle2007}, \citealt{Etxepare2008}, \citealt{DemonteSoriano2009}, \citealt{Gras2010},    \citealt{GonzalesPlanas2014}, \citealt{VillaGarcia2015}, \citealt{Corr2016}),  I did find attestations in corpora that point to their existence in this language.
\ea\label{ex:quepqsp} Madrileño Spanish\\ \gll  B: Son cinco bloques, con {pista de tenis} colectiva ¿no? Entonces, ahí jugamos. A: ¿Que tenéis apartamento no? B: No, es un piso en un bloque.    \\
 {} be.\textsc{3pl.prs} five {block of flats.\textsc{pl}} with {tennis court} collective no then there play.\textsc{1pl.prs} {}  \textsc{que} have.\textsc{2pl.prs} apartment no  {} no be.\textsc{3sg.prs} a flat in a {block of flats}  \\
\glt `A: It's five blocks of flats with a shared tennis court. So, this is where we play. B: You have an apartment, right? B: No, it's a flat in a block.' (CdE)
\z

The  example in \eqref{ex:quepqsp} can be characterized in a similar way to the Catalan example in \eqref{ex:queplou}. It also constitutes a \isi{bias}ed \isi{polar question} that encodes the speaker's suspicion that the answer is going to be positive. In the Spanish example, the \isi{bias} of speaker A toward a positive answer is motivated by  what speaker B is saying about her living situation. This makes speaker A suspect that she is living in an apartment. Therefore, with her question,  speaker A intends to confirm her belief.  In this case, the non-neutral nature of the \isi{polar question}s is marked through the initial \emph{que} but also through the question tag \emph{no}. 

Although examples such as \eqref{ex:quepqsp} therefore suggest that attributive \emph{que} is also licensed in Spanish \isi{polar question}s, in what follows I will still  rely mostly on Catalan examples to illustrate the core syntactic properties. I will  return to Spanish and to issues of cross-linguistic variation at the end of the section.

Attributive \emph{que} is restricted to \isi{polar question}s and is not attested in \textit{wh}-ques\-tions, hence the ungrammaticality of B's answer in \eqref{ex:constique} when \emph{que} is intended  as attributive. 
\ea 
\ea \label{ex:constique}
Catalan\\
		\gll A: M' he d' anar, tinc un  camí llarg a casa. \\
			{} \textsc{cl.1sg} have.\textsc{1sg.prs} to leave have.\textsc{1sg.prs} a  way long to home \\
		\exi{}\gll   B:  (*Que\textsubscript{attributive}) on vius?\\
	{}	{\phantom{(*}\textsc{que}} where live.\textsc{2sg.prs}\\
		\glt `A: I have to leave, I have a long way home. B: (*Que) where do you live?'
	\z
\z

A superficially equivalent version of the critical sentence is however perfectly acceptable with a subordinate complementizer giving rise to a reportative interpretation (cf. \ref{ex:queinwh}).
\ea \label{ex:queinwh} 
Catalan\\
\gll A: On vius? \\
		{} where  live.\textsc{2sg.prs}\\
\exi{}\gll  B: Què has dit? \\
 {} what \textsc{aux.2sg.prf.prs} say.\textsc{ptcp} \\
 \exi{}\gll A: Que\textsubscript{subordinate} on vius.\\
 { } 	\textsc{que} where live.\textsc{2sg.prs}\\
	\glt `A: Where do you live? B: What did you say? A: [reportative:] Where do you live?'
\z

The attributive  and the subordinate-valued complementizers also result in  different structural and interpretative properties within \isi{polar question}s. In a \isi{polar question} introduced by a subordinate \emph{que}, there is an additional interrogative complementizer present merged below it (cf. \ref{ex:quesitesi}). Furthermore, it does not have the illocutionary force nor the prosodic make-up of a question.  This means that the speaker does not expect an answer. Therefore, the interjection ``uf'' is a natural reaction on the part of the hearer who thereby expresses his negative emotions towards his supervisor's insistence. 
\ea\label{ex:quesitesi} 
Catalan\\
\gll B: M' he trobat amb la teva {directora de tesi} l' altre dia i m' ha preguntat per tu. \\
{} \textsc{cl.1sg} \textsc{aux.1sg.prf.pst} meet.\textsc{ptcp} with the your {thesis supervisor} the other day and \textsc{cl.1sg} \textsc{aux.3sg.prf.prs} ask.\textsc{ptcp} about you\\
\exi{} \gll  A: Què t' ha preguntat?\\
{} what \textsc{cl.2sg} \textsc{aux.3sg.prf.prs} ask.\textsc{ptcp}\\
\exi{} \gll B: Que si has acabat la tesi.\\
{} \textsc{que} if \textsc{aux.2sg.prf.prs} finish.\textsc{ptcp} the thesis\\
\exi{} \gll 
A: Uf...\\
{} Ugh...\\
\glt `A: I met your thesis supervisor the other day and she  asked about you. B: What did  she ask? A: Whether you'd finished your thesis. B: Ugh... '
\z

The situation with an attributive \emph{que}-initial \isi{polar question} is different, as illustrated in \eqref{ex:quetesis}. It appears without the additional interrogative complementizer. It furthermore has the illocutionary force of a question and, unless it is used as a rhetorical question, the speaker does expect an answer. Speaker A's statement that he now has time functions as evidence based on which speaker B can conjecture that the answer to her questions, whether he finished his thesis, will be positive. Therefore the use of \emph{que} is felicitous.\largerpage

\ea\label{ex:quetesis}
Catalan\\
\gll A: Fem un cafe la setmana que ve? Ara tinc temps. \\
{} make.\textsc{1pl.prs} a coffee the week that come.\textsc{3sg.prs} now have.\textsc{1sg.prs} time\\
\exi{} \gll B: Que has acabat la tesi?\\
{} \textsc{que} \textsc{aux.2sg.prf.prs} finish.\textsc{ptcp} the thesis \\
\exi{} \gll A: Sí! \\
{} yes\\
\glt `A: Should we have a coffee next week? I have time now. B: Have you finished your thesis? A: Yes!'
\z



With regard to the word order, the analysis  in \figref{struc:pqque} predicts that the complementizer should be followed rather then preceded by left-peripheral material. Evidence for the high position of \emph{que} in \isi{polar question}s is provided by data like \eqref{ex:clldpq} (repeated from \eqref{ex:clldpqrep}), which show that the complementizer precedes a  clitic left-dislocated \isi{topic}. 

\ea \label{ex:clldpq} Catalan\\ {[Context: Marta finds a bag of oranges in the kitchen. She asks her roommate:]}\\
	\exi{} 
	
	\gll {\ob}\textsubscript{SubP} Que$_i$] {\ob}\textsubscript{TopP} les taronges$_j$ t$_i$]  \dots {\ob}\textsubscript{FinP} t$_i$] {\ob}\textsubscript{IP} les$_j$ vas comprar tu?] \\
	{} \textsc{que} {} the orange.\textsc{pl} {} {} {}{} {} \textsc{cl.f.pl} \textsc{aux.2sg.prf.prs} buy you {}{}{}\\
	\glt `The oranges, did you buy them?' (\citealt[49: ex 98a]{Kocher2017a})
\z

As shown in \sectref{sec:presupeval}, a \isi{topic} can precede \emph{que} in \isi{polar question}s. However, these \isi{topic}s exhibit properties typical of hanging rather than  clitic left-dislocated \isi{topic}s. They are followed by an intonational break and can be resumed not only by a clitic but also by a full pronoun, a DP or an  epithet as the example in \eqref{ex:aquestidiota} shows. 

\ea \label{ex:aquestidiota} 		Catalan\\
 {[Context: Marta and Maria are at a party. Marta sees that Maria's  colleague Jordi is also there. Marta had a fight with Jordi recently and is not pleased about his presence. She is also sure that nobody at the party knows him but Maria. She asks her:]}\\
		\exi{} 
	\gll  {\ob}$_{\alpha P}$ En Jordi$_i$,] {\ob}\textsubscript{SubP} que]  {\ob}\textsubscript{TopP} aquest idiota$_i$ \dots  {\ob}\textsubscript{IP} l$_i$' has convidat tu?] \\
	{} the Jordi  {} \textsc{que} {}   that idiot {}  \textsc{cl.m.sg} \textsc{aux.2sg.prf.prs} invite.\textsc{ptcp} you \\
	\glt `Did you invite that idiot Jordi?'
\z

Additionally, according to my informants, it is not sufficient  to have a co-referential epithet or DP: The example is in fact only grammatical with an  additional co-referential clitic pronoun.  These properties suggest that the   \isi{topic} that precedes \emph{que} is merged outside of the core structure in a CP-external position. Therefore, examples like \eqref{ex:aquestidiota} do not constitute evidence against my  analysis. In the structure in \eqref{ex:aquestidiota}  \emph{en Jordi} is merged  directly  in a CP-external position and  \emph{aquest idiota} is a co-referential clitic left-dislocated \isi{topic} that is moved to the left periphery.

Catalan \emph{que}-initial \isi{polar question}s can furthermore be introduced by  pragmatic particles such as \emph{oi} and \emph{eh}, which according to \citet{PrietoRigau2007} are present when the speaker wants to achieve a confirmatory reading of the question. I briefly return to their pragmatic function in \sectref{sec:presupprag}.  
\ea\label{ex:oique}
Catalan\\
\gll  Oi que ens entenem? \\
	\textsc{oi} \textsc{que} \textsc{cl.2pl} understand.\textsc{2pl.prs}\\
	\glt `We understand each other, right?' (caWac)
\z

\citet{PrietoRigau2007} propose that these particles are merged in the specifier of ForceP, which  structurally coincides with SubP in the cartographic structure assumed in this book. The authors assume that \emph{que} is in FinP, hence at the lower edge of the left periphery. Based on this analysis, one would expect that  material would be able to intervene between  \emph{oi} and \emph{que}. The data in \eqref{ex:oinotop}, however, suggest precisely the opposite.   The ungrammaticality of a \isi{topic} intervening between \emph{oi} and \emph{que} shows that these two words need to be adjacent to each other.  I take this as evidence  that  they are located in the specifier and head of the same projection.  

\ea Catalan \label{ex:oinotop}\\
\gll * Oi en Jordi que l' has convidat tu?\\
	~ \textsc{oi} the Jordi \textsc{que} \textsc{cl.m.sg} \textsc{aux.2sg.prf.prs} invite.\textsc{ptcp} tu\\
	\glt Intended: `You are the one that invited Jordi, right?'
\z 

The example in \eqref{ex:oinotop} is consistent with the analysis that follows from the general assumption presented in this chapter. Just as in regular \emph{que}-initial \isi{polar question}s, the complementizer moves all the way up through the left-peripheral heads and ends up in SubP. Further support for the high position of \emph{(oi) que} is given in example \eqref{ex:oiqueclldpotser}, which shows that it precedes a clitic left-dislocated \isi{topic}.\largerpage[2]

\ea \label{ex:oiqueclldpotser}
Catalan\\
\gll  {\ob}\textsubscript{SubP} Oi que$_i$] {\ob}\textsubscript{TopP} aquesta pregunta$_j$ t$_i$] \dots {\ob}\textsubscript{FinP} t$_i$] {\ob}\textsubscript{IP} no se l$_j$' havien fet mai?] \\
{} \textsc{oi} \textsc{que} {} this question {} {}  {} {} not \textsc{cl.refl} \textsc{cl.f.sg} \textsc{aux.3pl.prf.prs} made never\\
\glt `They never asked themselves this question, right?' (caWac)
\z

An additional piece of evidence that  \emph{que} reaches a high position  in \isi{polar question}s is that, just as in the case of \emph{que}-initial declaratives discussed in \sectref{sec:presupinitialC}, \emph{que}-initial \isi{polar question}s are not found in embedded contexts (cf. \ref{ex:embedqueque}, \ref{ex:embedoique}). This is predicted by the present analysis  because the two instances of \emph{que} would compete for the same projection, SubP (see  the discussion around  \eqref{ex:doubleque} for some suggestions of how to account for this theoretically).
\ea Catalan
\ea\label{ex:embedqueque}\gll La Maria va preguntar que (si) (*que) l‘ he convidat jo. \\
the Maria \textsc{aux.3sg.prf.pst} ask that whether \textsc{que} \textsc{cl.3sg} \textsc{aux.1sg.prf.pst} invite.\textsc{ptcp} I\\
\glt `Maria asked whether I invited him.'
\ex\label{ex:embedoique}\gll La Maria va preguntar que (si) (*oi que) hi es la Lola.\\
the Maria \textsc{aux.3sg.prf.pst} ask that whether \textsc{oi} \textsc{que} \textsc{cl.loc} is the Lola\\
\glt `Maria asked whether Lola is there.'
\z
\z

As stated above, \emph{que}-initial \isi{polar question}s have, to date, primarily been discussed as a feature of Catalan grammar. To the best of my knowledge, apart from \citet{Hualde1992}, its existence in Spanish has so far been  disregarded. The review of corpus data, however, shows that Spanish does have \emph{que}-initial \isi{polar question}s as well. Moreover,  they  have the same function, namely, they express that the speaker is \isi{bias}ed towards a positive answer (cf. \ref{ex:quepqsp}). Another example that illustrates this is given in \eqref{ex:quepqhorst}.

\ea \label{ex:quepqhorst}  Chilean Spanish \\\gll -- Creo que no se le pagan 15 millones mensuales para que ande haciendo proselitismo político junto a Horst Golborne.  -- ¿Qué? -- ¿Que no se llama Horst? Ah perdón, me confundí... \\
{} believe.\textsc{1sg.prs} that not \textsc{cl.refl} \textsc{cl.dat} pay.\textsc{3pl.prs} 15 million.\textsc{pl} monthly so that go.\textsc{3sg.prs} do.\textsc{ptcp.prs} proselytism politic together to Horst Golborne {} what {} \textsc{que} not \textsc{cl.refl} call.\textsc{3sg.prs} Horst ah sorry \textsc{cl.1sg} confuse.\textsc{1sg.prf.pst} \\
\glt `-- I think that he doesn't get 15 million a month to wander around doing political proselytism along with Horst Golborne. -- What? -- So he's not called Horst? Ah, sorry, I confused the name.' (CdE)
\z

The speaker who utters the \emph{que}-initial \isi{polar question} takes her interlocutor's reaction to mean that she used the wrong name. She therefore expects the answer to her question \emph{He is not called Horst?} to be affirmative, and consequently  is \isi{bias}ed towards a positive answer. The origin of the two examples moreover shows  that \emph{que}-initial \isi{polar question}s in Spanish cannot be the result of Catalan influence, contra \citet[2]{Hualde1992} where it is stated that ``the use of \emph{que} in questions when transferred to Spanish, is stereotypical of a Catalan background''. This can hardly be the case given the attestations from varieties that are not in contact with Catalan (cf. \eqref{ex:quepqsp} from Madrileño and \eqref{ex:quepqhorst} from Chilean Spanish). 

The example in  \eqref{ex:sppqclld}, in which \emph{que} precedes a clitic left-dislocated \isi{topic}, suggests that  Spanish \emph{que}-initial \isi{polar question}s also have the same syntactic properties as their Catalan equivalents. Therefore, my analysis can be extended to Spanish.

\ea\label{ex:sppqclld}
US-Spanish \\
\gll  {\ob}\textsubscript{SubP} ¿Que$_i$] {\ob}\textsubscript{TopP} la respuesta$_j$ t$_i$] {\ob}\textsubscript{FinP} t$_i$ {\ob}\textsubscript{IP} la$_i$ publicamos en periódicos de provincias? \\
{} \textsc{que} {} the answer {} {} {} {} \textsc{cl.f.sg} publish.\textsc{1pl.prs} in newspaper.\textsc{pl} of province.\textsc{pl}\\
\glt `Are we publishing the answer in provincial newspapers?' (CdE)
\z

The data presented here suggest that there is no evidence for a systematic syntactic  difference between Spanish and Catalan \emph{que}-initial \isi{polar question}s. The only difference is that Spanish does not appear to allow particles like \emph{oi} or \emph{eh}.  My conclusion is therefore that both languages permit attributive \emph{que} in \isi{polar question}s. 

Although a more extensive investigation is yet to be completed, a brief review of Spanish corpus data shows that attributive \emph{que}  is actually quite frequent (cf. Table \ref{tab:quepqsp}). For the purpose of this corpus study, I relied on the data from the modern Spanish subportion of the 2001 version of the CdE, which are annotated for text type and country of origin. I used this smaller subcorpus of CdE for this inquiry because the large web-based corpus contains more orthographic irregularities. A particularly common misspelling is the omission of  diacritics, leading to the loss of the formal distinction between the interrogative pronoun \emph{qué} `what' and the complementizer. Using these imperfect data  would  risk  wrongly including a great proportion of \textit{wh}-questions in the sample. In total there are 180 occurrences (total tokens in the subcorpus: 20.4 million) of \emph{que}-initial \isi{polar question}s in the subcorpus. In comparison, the Catalan caWac contains 3124 occurrences (total tokens in the corpus: 780 million). This means that the relative number of occurrences in the CdE is actually larger than in the caWac. However, given the differences in the composition of the corpora, I am hesitant about  drawing  conclusions from this result. The CdE contains dialog data that favors the use of questions, as opposed to the  web-based caWac which might be less likely to include questions. The majority of the occurrences in CdE stem from oral and fiction data. Unfortunately, there is no information on the number and proportion of tokens from the different dialectal varieties in the corpus, so the absolute numbers of occurrences  presented in the Table cannot be  systematically compared in this respect. What the data  once again confirm, however, is that \emph{que}-initial \isi{polar question}s are unlikely to result from contact with Catalan, since there are attestations from various non-European varieties.

\begin{table}
\begin{tabular}{l*4{r}}
	\lsptoprule
	& fiction & press & oral & total \\ 
	\midrule
	Argentina &  21 &   1 &   7 & 29 \\ 
	Bolivia &   0 &   0 &   4 &4 \\ 
	Chile&  19 &   0 &  5&  24  \\ 
	Colombia&   2 &   3 &   1 &  6\\  
	Cuba &   1 &   1 &   0 &  2\\ 
	Gran Canaries &   0 &   0 &   1 &  1 \\
	Guatemala  &   2 &   1 &   1 &  4 \\ 
	Honduras &   0 &   1 &   0 &  1\\ 
	Mexico &  13 &   0 &  19&  32 \\ 
	Paraguay &  14 &   0 &   0&  14 \\ 
	Peru  &  11 &   1 &   0 &  12\\
	Puerto Rico &   1 &   0 &   1 &  2\\  
	Spain&  22 &   7 &  77 & 106 \\
	Venezuela &   0 &   0 &   8&8 \\ 
	N/A &  20 &   0 &   0 &20 \\\midrule
	total &  126  &  15 &  124& 180\\
\lspbottomrule
\end{tabular}
	\caption{Absolute numbers of \emph{que}-initial \isi{polar question}s in the 2001 contemporary subcorpus of CdE (N/A: not available)}\label{tab:quepqsp}
\end{table}

Finally, in Portuguese there are no cases  of \emph{que}-initial \isi{polar question}s attested. This suggests that in this respect there is in fact a systematic syntactic  contrast  between Portuguese on the one hand and  Spanish and Catalan on the other.\pagebreak

\ea 
\ea\label{ex:quesia}
Portuguese\\ 
\gll  A Joana pergunta (*que) se vens. \\
the Joana ask.\textsc{3sg.prs} that if come.\textsc{2sg.prs}\\
\ex\label{ex:quesib}
		Spanish\\
\gll Juana pregunta que si vienes. \\
Juana ask.\textsc{3sg.prs} that if come.\textsc{2sg.prs}\\
\ex\label{ex:quesic}
Catalan\\
\gll La Joana pregunta que si vens. \\
the Joana ask.\textsc{3sg.prs} that if come.\textsc{2sg.prs}\\
\glt `Jo/uana asks if you are coming.'
\z
\z 


 \citet{Corr2016} offers an explanation for this discrepancy. The author hypothesizes that  Portuguese  \emph{que} is more restricted than its cognates  in Spanish and Catalan in that it is specialized for declaratives and cannot appear in other clause types.
 One of the reasons behind  this hypothesis is the ungrammaticality of  \emph{que} in embedded and reported questions above an interrogative complementizer \emph{se} in Portuguese (see  \ref{ex:quesia}), which contrasts with the other two languages (see \ref{ex:quesib}, \ref{ex:quesic}; cf. \sectref{sec:insubcross} for further discussion on this contrast). 





\subsection{\emph{Que} in \is{wh-exclamative@\textit{wh}-exclamative}\textit{wh}-exclamatives}\label{sec:presupexclc}
The present subsection deals with the syntactic properties of attributive \emph{que} in \is{wh-exclamative@\textit{wh}-exclamative}\textit{wh}-exclamatives attested in all three languages under investigation.  The complementizer, when present,  always appears adjacent to the \textit{wh}-expression. I adopt the proposal made in \citet{DemonteSoriano2009} that the \textit{wh}-ex\-pres\-sion is located in FocP. In the cartographic literature, this projection has been identified as the host of foci and \textit{wh}-phrases (cf. \citealt{Rizzi1997}). The derivation I assume for \eqref{ex:demsorexclrep} (repeated from \eqref{struc:demsorexcl}) is given in \figref{struc:exclc}. 

\ea\label{ex:demsorexclrep}
		Spanish (\citealt[33 : ex 19a]{DemonteSoriano2009})\\
\gll   ¡Qué rico que está! \\
	 how good \textsc{que}   be.\textsc{3sg.prs}\\
	\glt `How good this is!'
\z

\begin{sloppypar}
The intermediate positions through which the complementizer passes are again omitted in the present structure.  Attributive \emph{que}  is as always assumed to be merged in FinP and moves from head to head until it reaches FocP, where the movement comes to a halt. 
\end{sloppypar}\largerpage[-2]

This analysis assumes that the \textit{wh}-expression is merged in the left periphery rather than moved to it. This allows us to maintain the idea that the movement of the attributive complementizer is inhibited by one simple condition that disallows the crossing of base-generated material. In what follows, I will show that in addition to the theoretical plausibility of this account,  there is also empirical evidence in favor of the assumption that the \textit{wh}-expression is base-generated in the left periphery. 

\begin{figure}
	\caption{\label{struc:exclc}Analysis of \eqref{ex:demsorexclrep}; crossed out lines represent a potential  movement that does not take place}
	\begin{forest}
		[FocP,name=focp
		[¡Qué rico, draw, name=spfoc,label={below:\emph{merged}}] 
		[Foc' 
		[Foc$^0$\\que\textsubscript{attributive}, name=foc]
		[\dots, name=dots
		[,phantom ] 
		[FinP
		[~] 
		[Fin' 
		[Fin$^0$\\\sout{que\textsubscript{attributive}}, name=fin] 
		[IP,		
		[está!,roof]
		{\begin{scope}[decoration={crosses,shape size=1.5mm}]
			\draw[-] (fin)  to[in=south west, out=south west]	(-0.3,-5.6);
			\draw[-] (-0.3,-5.6)  to[in=south west , out=south west]	(-0.3,-4.4);  
			\draw[->] (-0.3,-4.4)  to[in=south west , out=south west]	(foc);  
		\draw[->] (-0.9,-3)  to[out=south west, in=south]	(-2,0.4); 
			\draw[decorate] (-0.9,-3)  to[out=south west, in=south]	(-2,0.4); 
		\end{scope}
		}
		]]]]]]	
	\end{forest}
\end{figure}


If I adopt a derivation with a left-peripheral base position for the \textit{wh}-ex\-pres\-sion, one crucial aspect that requires explanation is how the  \textit{wh}-expression ends up being interpreted as an argument or adjunct dependent on the IP-internal verb. A non-local theta- and even case-assignment as must be assumed here are not an unprecedented idea in the literature (see  \citealt{Boskovic2007}, \citealt[168--170]{VillaGarcia2015}, \citealt{Saab2015}).  A similar  issue is encountered in \citet{VillaGarcia2015}. He analyzes recomplementation configurations such as that in \eqref{ex:recomp}, where the clitic left-dislocated \isi{topic} \emph{a los alumnos} is sandwiched between two complementizers. Assuming non-local case-assignment in these cases becomes necessary because \citet{VillaGarcia2015} shows that the \isi{topic}s are generated directly in the left periphery. 

\ea \label{ex:recomp}
		Spanish (\citealt[18: ex 23]{VillaGarcia2015})\\
\gll Susi dice que a los alumnos, que les van a dar regalos. \\
	Susi say.\textsc{3sg,prs} that \textsc{dat} the student.\textsc{pl} that \textsc{cl.dat.3pl} go.\textsc{3pl.prs} to give presents\\
	\glt `Susi says that they are going to give the students presents.' 
\z


In order to explain how these \isi{topic}s are assigned their  case, \citet{VillaGarcia2015} adopts the agreement mechanism described in  \citet{Boskovic2007}, building on the principle of \emph{Greed} introduced in \citet{Chomsky1993}. The basic idea of  \citeauthor{Boskovic2007}'s system is that the standard assumption that the \emph{v} is the probe and the case-marked DP is the goal is reversed.\footnote{In the minimalist program, syntactic agreement is perceived of as a matching relation between a probe and a goal (see \citealt{Chomsky2000}), where matching can be defined as  feature identity.} In this system, the DP is the probe and moves to a position from which it c-commands the goal \emph{v}. The DP probes  \emph{v} to license its case. \citet{VillaGarcia2015} applies this system to the recomplementation data. The derivation is even simpler in this case because according to the author the probe \emph{a los alumnos} is base-generated in the left-peripheral position sandwiched between two complementizers. It is therefore never lower than its case-licensor (\emph{v}) and no movement of the DP is required. The DP probe c-commands its goal \emph{v} from its base-generated location and is therefore in a position to check off its case feature (see \citealt[168--170]{VillaGarcia2015}).

Case assignment from a left-peripheral position, as proposed in \citet{VillaGarcia2015}, does not seem to be active in the construction under investigation. As  can be seen in \eqref{ex:premiob}, the left-peripheral \textit{wh}-expressions cannot bear case marking when followed by attributive \emph{que}. 

\ea Spanish\\ 
\ea[]{ \gll  Han dado un premio a una estudiante inteligente.\\
 \textsc{aux.3pl.prf.prs} give.\textsc{ptcp} the prize \textsc{dat} a student  intelligent\\
\glt `They gave the prize to an intelligent student.' \label{ex:premioa}}
\ex[*]{ \gll   ¡A qué estudiante más inteligente que han dado el premio!\\
       \textsc{dat} which student more intelligent \textsc{que}  \textsc{aux.3pl.prf.prs} give.\textsc{ptcp} the prize\\ \label{ex:premiob}}
\ex[*]{ \gll  ¡Qué estudiante más inteligente que han dado el premio!\\
             which student more intelligent \textsc{que} \textsc{aux.3pl.prf.prs} give.\textsc{ptcp} the prize\\
\glt Intended: `What an intelligent student they gave the prize to!'\label{ex:premioc}}
\z
\z


In the declarative equivalent to the \is{wh-exclamative@\textit{wh}-exclamative}\textit{wh}-exclamative in \eqref{ex:premioa}, the object of the verb is introduced by the dative marker \emph{a}. The corresponding \is{wh-exclamative@\textit{wh}-exclamative}\textit{wh}-exclamative in \eqref{ex:premiob}, in which the dative object is a left-peripheral \textit{wh}-expression, is judged ungrammatical by my informants. Furthermore, example \eqref{ex:premioc}  shows that an unmarked \textit{wh}-expression is  not grammatical either.\footnote{There appears to be a small degree of variation involved, as a minority of my informants judged examples like \eqref{ex:premiob} marginally acceptable. Additionally, some of the informants found \eqref{ex:premiono} only grammatical with a dative clitic co-referent to the \textit{wh}-expression.}

These data contrast with the example in  \eqref{ex:premiono}, which shows that in the absence of attributive \emph{que} a dative-marked object can be a left-peripheral \textit{wh}-expression. 

\ea Spanish\label{ex:premiono}\\
\gll  ¡A qué estudiante más inteligente (le) han dado el premio! \\
\textsc{dat} which student more intelligent \textsc{cl.dat} \textsc{aux.3pl.prf.prs} give.\textsc{ptcp} the prize\\
\glt `What an intelligent student they gave the prize to!' 
\z

This example provides crucial evidence for my idea, which I will return to below, that \is{wh-exclamative@\textit{wh}-exclamative}\textit{wh}-exclamatives with and without \emph{que} differ in that in the former case the \textit{wh}-expression is base-generated in the left periphery, while in the latter, it reaches the surface  position through movement.


The next question I address is how left-peripheral elements end up being interpreted as dependent on the verb.\largerpage

\ea 
\ea\label{ex:whtheta}
European Portuguese\\ 
\gll  Que coisa mais idiota que fazem aos animais. \\
what thing more stupid \textsc{que} do.\textsc{3pl.prs} to.the animal.\textsc{pl}\\
\glt `What stupid things people do to animals.' (CdP)
\ex \label{ex:whmod}
Catalan\\
\gll Que malament que anem. \\
how badly \textsc{que} go.\textsc{1pl.prs}\\
\glt `How badly it is going for us.' (caWac)
\z
\z


The potentially more complex  cases are those in which the \textit{wh}-expression appears to be an argument of the verb, as in \eqref{ex:whtheta}. In line with \citet{VillaGarcia2015}, I adopt the idea that for theta-role assignment, clausematehood is a sufficiently local configuration. In other words, a verb can assign its theta-role not only to the elements in its argument positions, but  also to elements that are externally merged in different positions, as long as they are contained in the same clause.  \citet{Saab2015} presents a formal account of long distance theta assignment. According to him, there are two central principles  that must be met in theta assignment: locality and activity (see also \citealt{Chomsky2000, Chomsky2001}).  \citet{Saab2015} states that a thematic head can assign a theta-role to a given argument if and only if the argument is active and local with respect to the thematic head (cf. \citealt[2]{Saab2015}).  Activity is conceived of as an unvalued K-feature at the point of derivation when the theta-role is assigned. The crucial point for the present argument is that in \citeauthor{Saab2015}'s proposal locality is not based on merge. This contrasts with previous accounts, such as that proposed by \citet{Sheehan2012} who states that theta-roles are assigned via internal or external merge with a thematic head. On the contrary, in \citet{Saab2015},   a local argument is simply defined as the closest argument to the thematic head.  
 
The principles of activity and locality required for theta-role assignment in the system outlined in \citet{Saab2015}  are both met by  the \textit{wh}-expressions in \is{wh-exclamative@\textit{wh}-exclamative}\textit{wh}-exclamatives like \eqref{ex:whtheta}. The object \emph{os animais} is local but not active because it has been assigned its case by the preposition \emph{a}  and thus does not contain an unvalued K-feature. In turn, the object \emph{que coisa mais idiota} is both active, i.e. not case-marked, and local in relation to the thematic head \emph{fazem} because there is no other potential active argument  closer  to the thematic head (i.e. c-commanded by \emph{que coisa mais idiota}). 
 
There are also cases of \textit{wh}-exclamatives such as \eqref{ex:whmod}, in which the \textit{wh}-ex\-pres\-sion appears to be a modifier of a verb. Non-local structural  positions yet local  interpretations of  modifiers have also been addressed previously in the literature. It has been proposed independently that adverbial modifiers can be base-generated at the edge of the clause in which they are interpreted  (\citealt[46--51]{Rizzi1990}, \citealt{Uriagereka1988}). 
   In a similar vein, it is maintained by some authors that the counterpart of \emph{why} in various languages is base-generated in its left-peripheral surface position, even though its interpretation is dependent on an element deeper in the structure (\citealt{Hornstein1995}, \citealt{Rizzi1990, Rizzi2001}, \citealt{Shlonsky2011}).




Having discussed how the IP-dependent interpretation of the \textit{wh}-expressions can be accounted for,  I will now  evaluate the predictions of the analysis outlined in \figref{struc:exclc}. I show that the word order of the \textit{wh}-expression and the complementizer with respect to other left-peripheral material confirms the analysis. As stated at the start of this section, \emph{que} and the \textit{wh}-expression must observe strict adjacency, meaning that no left-peripheral material can intervene, hence the ungrammaticality of \eqref{ex:excllda}. In the analysis, this is captured by postulating that they appear in the specifier and head of the same projection.

\ea\label{ex:excllda}
   Catalan (adapted from \citealt[48: ex 95a]{Kocher2017a})\\
    \gll  
	* Que estrany  al Jordi$_i$ que li$_i$ hagin trucat. \\
	~ how  strange \textsc{dat}.the Jordi  \textsc{que} \textsc{cl.dat} \textsc{aux.3pl.sbjv.prf.prs} call.\textsc{ptcp}\\
 \glt Intended: `How strange that they called Jordi of all people.'
\z 


The analysis furthermore predicts that it should be possible for the sequence of  \textit{wh}-expression plus complementizer to be preceded and followed by a clitic left-dislocated \isi{topic}. This prediction also holds true, as can be seen in  \eqref{ex:exclldb} and \eqref{ex:exclldc}.
\ea Catalan 
\ea  \label{ex:exclldb}  
		 (adapted from \citealt[48: ex 95b]{Kocher2017a})\\
\gll {\ob}\textsubscript{FocP} Que estrany que$_i$] {\ob}\textsubscript{TopP} al Jordi$_j$ t$_i$] {\ob}\textsubscript{FinP} t$_j$] {\ob}\textsubscript{IP} li$_j$ hagin trucat. $]$\\
			{} how  strange  \textsc{que} {} \textsc{dat}.the Jordi {} {} {} {} \textsc{cl.dat} \textsc{aux.3pl.sbjv.prf.prs} call.\textsc{ptcp} \\

	
		\ex \label{ex:exclldc} (adapted from \citealt[48: ex 95c]{Kocher2017a}) \\
		\gll  {\ob}\textsubscript{TopP} Al Jordi$_i$,] {\ob}\textsubscript{FocP} que estrany que$_j$]  {\ob}\textsubscript{FinP} t$_j$] {\ob}\textsubscript{IP} li$_i$ hagin trucat. $]$ \\
		{} \textsc{dat}.the Jordi {} how  strange  \textsc{que}  {} {} {} \textsc{cl.dat} \textsc{aux.3pl.sbjv.prf.prs} call.\textsc{ptcp}\\ 
		\glt `How strange that they called Jordi of all people.' 
	\z
\z

Moreover, since the complementizer  only moves  up to FocP but not higher,  \is{wh-exclamative@\textit{wh}-exclamative}\textit{wh}-exclamatives with attributive \emph{que} should appear in embedded contexts as well as in \emph{que}-initial reportatives. Both of them contain  the high complementizer merged directly in SubP  which is valued  with a subordinate feature in my analysis. Examples \eqref{ex:exclembed} and \eqref{ex:exclreport} confirm that this is indeed the case.

\ea Spanish
\ea\label{ex:exclembed}
		 \gll María dijo {\ob}\textsubscript{SubP} que\textsubscript{subordinate}] {\ob}\textsubscript{FocP} qué suerte que$_i$] {\ob}\textsubscript{FinP} t$_i$] {\ob}\textsubscript{IP} no salió de {fin de semana}.] \\ 
María say.\textsc{3sg.prf.pst} {} that {} what luck \textsc{que} {} {} {} not go-out.\textsc{3sg.prf.pst} for weekend\\ 
\glt `María said how lucky that she didn't go out on the weekend.' (CdE)
\ex \label{ex:exclreport} 
\gll   {\ob}\textsubscript{SubP} Que\textsubscript{subordinate}] {\ob}\textsubscript{FocP} qué susto que$_i$] {\ob}\textsubscript{FinP} t$_i$] {\ob}\textsubscript{IP}   no se aguante mis {pataletas de treceañera} que no puedo controlar. ] \\
{} \textsc{que} {} what shock \textsc{que} {} {} {} not \textsc{cl.refl} bear.\textsc{3sg.sbjv.prs} my {teenage tantrums} that not can.\textsc{1sg.prs} control\\
\glt `[reportative:] What a shock that s/he cannot bear my tantrums that I cannot control.' (CdE)
\z
\z 



 
Having established that the analysis makes the right predictions with respect to the word order in the left periphery, I now return to the idea that \textit{wh}-ex\-pres\-sions that precede \emph{que} are merged in,  rather than moved to, the left periphery. As mentioned above, this assumption is necessary  in order to maintain the uniform analysis defended in this chapter. The relative position of attributive \emph{que} with respect to other left-peripheral material follows straightforwardly from a simple restriction on an otherwise unconditioned complementizer movement, which disallows movement across material that is merged in the left periphery. I will now show that this is not just a theoretical necessity; on the contrary, it is also supported by empirical evidence. 

In previous  works on   \is{wh-exclamative@\textit{wh}-exclamative}\textit{wh}-exclamatives, the presence of \emph{que} has been considered optional (see for instance \citealt{Villalba2003}, \citealt{Castroviejo2006}). The reasons for this is that the Ibero-Romance languages under investigation allow \textit{wh}-ex\-clam\-a\-tives   with and  without a low complementizer.\footnote{There might be some cross-linguistic variation with respect to the possibility of omitting the complementizer. \citet{GonzalezPlanas2010} states that the complementizer is obligatory in Catalan \is{wh-exclamative@\textit{wh}-exclamative}\textit{wh}-ex\-clam\-a\-tives in many contexts where it can be omitted in Spanish.} The  existence of parallel examples like \eqref{ex:quinnasa} and \eqref{ex:quinnasb} leads the authors to conclude that they have the same underlying structure and optionally allow the merger of a complementizer. The complementizer itself has  been described as semantically vacuous (\citealt{Castroviejo2006}). 

\ea\label{ex:quinnas} Catalan
\ea\label{ex:quinnasa}
\gll Quin nas més gros que tens. \\
what nose more big \textsc{que} have.\textsc{2sg.prs}\\
\ex\label{ex:quinnasb}

\gll Quin nas més gros tens. \\
what nose more big  have.\textsc{2sg.prs}\\
\glt `What a big nose you have.'
\z
\z



Previous accounts furthermore assume that  \is{wh-exclamative@\textit{wh}-exclamative}\textit{wh}-exclamatives involve movement of the \textit{wh}-expression to a left-peripheral position (see for instance \citealt{Beninca1996}, \citealt{Gutierrez-Rexach2001}, \citealt{Zanuttini2003}, \citealt{Ambar2003}, \citealt{Villalba2003,Villalba2008},  \citealt{Castroviejo2006}, \citealt{DemonteSoriano2009}, \citealt{Gutierrez-Rexach2011}, \citealt{Cruschina2015a}). In contrast, I believe that there are  two different derivations for \is{wh-exclamative@\textit{wh}-exclamative}\textit{wh}-exclamatives in Ibero-Romance.  In one, the \textit{wh}-expression is moved to the left periphery from an IP-internal position; and in the other,  it is merged directly in the left periphery. The two examples in \eqref{ex:quinnasa} and \eqref{ex:quinnasb} are then not structurally equivalent.  Furthermore, \emph{que} only appears to be optional, but this is not  in fact  the case. The contrast between \eqref{ex:premiobrep} and \eqref{ex:premionorep} repeated from \eqref{ex:premiob} and \eqref{ex:premiono} shows that in a \is{wh-exclamative@\textit{wh}-exclamative}\textit{wh}-exclamative with attributive \emph{que}, a case-marked \textit{wh}-expression is impossible. On the contrary, the same case-marked \textit{wh}-expression is grammatical in \emph{que}-less exclamatives.

\ea Spanish
\ea[*]{\gll   ¡A qué estudiante más inteligente que han dado el premio!\\
	          \textsc{dat} which student more intelligent \textsc{que}  \textsc{aux.3pl.prf.prs} give.\textsc{ptcp} the prize\\
\glt intended: `What an intelligent student they gave the prize to!'\label{ex:premiobrep}}
\ex[]{\gll  ¡A qué estudiante más inteligente (le) han dado el premio! \\
\textsc{dat} which student more intelligent \textsc{cl.dat} \textsc{aux.3pl.prf.prs} give.\textsc{ptcp} the prize\\
\glt `What an intelligent student they gave the prize to!'\label{ex:premionorep}} 
\z	
\z

Another key piece of evidence for two different derivations is the existence of \is{wh-exclamative@\textit{wh}-exclamative}\textit{wh}-exclamatives like \eqref{ex:exclnonoptque}, where crucially, the complementizer cannot be omitted (cf. also \citealt{GonzalezPlanas2010})


\ea \label{ex:exclnonoptque}
\ea\label{ex:sortneva} Catalan\\
\gll  Quina sort *(que) neva. \\
what luck \textsc{que} snow.\textsc{3sg.prs}\\
\glt `What  luck that it is snowing!'
\ex\label{ex:quepassou}
Portuguese\\ 

\gll Que mau *(que) isto se tenha passado.\\
how bad \textsc{que} this \textsc{cl.refl} \textsc{aux.3sg.subj.prf.prs} happen.\textsc{ptcp}\\
\glt `How bad that this  happened.'
\ex\label{ex:lastima1}
		Spanish\\
\gll Qué lástima *(que) no vienes. \\
what pity \textsc{que} not come.\textsc{2sg.prs}\\
\glt `What a pity that you're not coming.'
\z
\z

At this point I have to briefly digress to acknowledge that there is an alternative view to the one maintained here, namely that these examples are not monoclausal exclamatives but biclausal clefts with an omitted copula. In this view, the structure of \eqref{ex:sortneva} can be represented by \eqref{struc:sortneva}, adapting an analysis of clefts proposed in \citet{Belletti2009, Belletti2013}. 
	
\ea\label{struc:sortneva} {\ob}\textsubscript{SubP1} \dots {\ob}\textsubscript{IP1}] \sout{es} {\ob}\textsubscript{SubP2} \dots {\ob}\textsubscript{FocP} Quina sorte {que}$_i$ \dots {\ob}\textsubscript{FinP} t$_i$ {\ob}\textsubscript{IP2}
		neva.]]]]]
		\z

Although my claim is that cases such as \eqref{ex:exclnonoptque}  do in fact constitute exclamatives and I provide  evidence for this below,  the alternative view is not necessarily in complete disagreement with my general proposal. Adopting a cleft-analysis à la \citeauthor{Belletti2009}, firstly, places the complementizer in FinP and assumes movement from left-peripheral head to left-peripheral head that comes to a halt in FocP where the clefted phrase is merged. This parallels the syntactic analysis  I assume for all attributive \emph{que} constructions addressed in the present chapter. Secondly, it has been observed that clefts carry  an existential presupposition (among many others, \citealt{Buering2013}). A sentence like \emph{It is Fred she invited.}  carries the presupposition  \emph{She invited someone}. This is very close to the meaning I identify for attributive \emph{que}. I will not go into detail here. Note, however, that if, as is commonly assumed, what it means to presuppose a proposition is  that this proposition is part of the speaker's and the hearer's \isi{common ground}, then the presence of the existential presupposition  can be reconciled with the meaning contribution I identify for attributive \emph{que} (see \sectref{sec:presupprag}). This could allow us to explain why we have the same resulting interpretation even though these could be viewed as embedded and not root sentences. The idea would be  that a complementizer merged in FinP, irrespective of whether we are dealing with an embedded or unembedded context, picks up an attributive feature which has the described effect on the interpretation. That \eqref{ex:sortneva} presupposes  \emph{It snows} then  boils down to the simple fact that the feature linked to the FinP-merged complementizer as usual  attributes a \isi{commitment} to the  proposition in its scope (in this case: \emph{It snows}) to the hearer.

Returning now to the  discussion at hand, the difference between the examples in \eqref{ex:quinnas} and in \eqref{ex:exclnonoptque} is that in the former  the \textit{wh}-expression can be reconstructed in the core sentence, while in the latter it cannot. Consider the examples in which the \textit{wh}-expression appears to be syntactically dependent on the verb: It is a predicate dependent on a copula verb in \eqref{ex:chato},  an argument in \eqref{ex:quinnas} and an adjunct in  \eqref{ex:irbien}.

As a side note, a contrast also seems to be present in English between \is{wh-exclamative@\textit{wh}-exclamative}\textit{wh}-exclamatives where the \textit{wh}-expression appears to originate in the IP and \is{wh-exclamative@\textit{wh}-exclamative}\textit{wh}-exclamatives where  it does not. This can be seen in the translations given below the relevant examples: Only those where the \textit{wh}-expression does not  have a function in the IP (for instance \eqref{ex:exclnonoptque} vs. \eqref{ex:chatoa}) allow the presence of a complementizer below the \textit{wh}-expression.\largerpage


\ea \label{ex:chato}
\ea\label{ex:chatoa}
Portuguese\\ 
\gll Que chato que és. \\
		how annoying \textsc{que} be.\textsc{2sg.prs}\\
		\ex\label{ex:chatob}\gll Que chato és. \\
		how annoying be.\textsc{2sg.prs}\\
		\glt `How annoying you are.' 
	\z
\ex \label{ex:irbien}
\ea\label{ex:irbiena}
		Spanish\\
\gll Qué bien que nos ha ido. \\
		how well \textsc{que} \textsc{cl.1pl} \textsc{aux.3sg.prf.prs} go.\textsc{ptcp}\\
		\ex\label{ex:irbienb}
		\gll  Qué bien nos ha ido. \\
		how well  \textsc{cl.1pl} \textsc{aux.3sg.prf.prs} go.\textsc{ptcp}\\
		\glt `How well it went for us.'
	\z
\z


In sum, while parallel examples  like those illustrated in \eqref{ex:quinnas}, \eqref{ex:chato} and \eqref{ex:irbien} could lead to the conclusion that a derivation by movement is a tenable analysis, examples such as those in \eqref{ex:exclnonoptque}, where the \textit{wh}-expression does not have a function in the IP, are not consistent with an account in which their left-peripheral position is  derived by movement. I propose that the contrast is not between cases where the \textit{wh}-expression has a function in the IP and those where it does not, but  between \is{wh-exclamative@\textit{wh}-exclamative}\textit{wh}-exclamatives with and without a complementizer. I suggest that these  are two types of \is{wh-exclamative@\textit{wh}-exclamative}\textit{wh}-exclamatives that are structurally distinct. In \textit{wh}-exclamatives with  \emph{que} the \textit{wh}-expression is merged  in the left periphery (cf. \figref{struc:exclmerged}),  while in those without \emph{que}, the \textit{wh}-expression is promoted to the left periphery through movement from an IP-internal position (cf. \figref{struc:exclmoved}).


\begin{figure}
\begin{floatrow}
  \captionsetup{margin=.05\linewidth}
  \ffigbox
    {\begin{forest}
			[FocP,name=focp
			[Que chato, draw, name=spfoc,label={below:\emph{moved}}] 
			[Foc' 
			[Foc$^0$, name=foc]
			[\dots, name=dots
			[,phantom ] 
			[FinP
			[~] 
			[Fin' 
			[Fin$^0$, name=fin] 
			[IP,		
			[és \sout{que chato}.,roof, name=ip]
			{
			\draw[->, overlay] (ip) to[in=south west, out=west] (spfoc);
			}
			]]]]]]
		\end{forest}}
	{\caption{\label{struc:exclmoved}Analysis of \eqref{ex:chatob}}}
  \ffigbox
    {\begin{forest}
			[FocP,name=focp
			[Que chato, draw, name=spfoc, label={below:\emph{merged}}] 
			[Foc', l sep+=\baselineskip
			[Foc$^0$\\que\textsubscript{attributive}, name=foc]
			[\dots, name=dots
			[,phantom ] 
			[FinP
			[~] 
			[Fin' 
			[Fin$^0$\\\sout{que\textsubscript{attributive}}, name=fin] 
			[IP,		
			[és.,roof]
			{
			\draw[-, overlay] (fin)  to[in=south west, out=west]	(-0.3,-5.7);
			\draw[-] (-0.3,-5.7)  to[in=south west , out=south west]	(-0.3,-4.5);  
			\draw[->] (-0.3,-4.5)  to[in=south west , out=south west]	(foc);  
					}
			]]]]]]	
		\end{forest}}
    {\caption{\label{struc:exclmerged}Analysis of \eqref{ex:chatoa}}}
\end{floatrow}
\end{figure}

Therefore, examples like \eqref{ex:quinnasa}, \eqref{ex:chatoa}, \eqref{ex:irbiena} and \eqref{ex:exclnonoptque} have the same underlying structure in the CP, while examples like \eqref{ex:quinnasb}, \eqref{ex:chatob} and \eqref{ex:irbienb} differ structurally. 
Further motivation behind the assumption that there are distinct derivations comes from the position of subjects. At least in Spanish and Catalan, the complementizer-less \is{wh-exclamative@\textit{wh}-exclamative}\textit{wh}-exclamatives do not allow a preverbal subject: The subject necessarily  follows  the verb (see \ref{ex:whexclverb}).  \citet{Castroviejo2006} takes this as evidence for the movement derivation of \is{wh-exclamative@\textit{wh}-exclamative}\textit{wh}-exclamatives, suggesting that the moved \textit{wh}-expression passes through a position at the edge of the IP which would otherwise be occupied by the preverbal subject.
 

\ea \label{ex:whexclverb}
\ea
Catalan\\
\gll Que blanca i bonica es presenta {des d'} allí la ciutat. \\
	how white and pretty \textsc{cl.refl} present.\textsc{3sg.prs} from there the city\\
	\glt `How white and pretty the city looks from there.' (caWac)
\ex
		Spanish\\
\gll  Qué suerte tenemos los lectores de tener quien nos escriba. \\
what luck have.\textsc{1pl.prs} the reader.\textsc{pl} to have who \textsc{cl.1pl} write.\textsc{3sg.sbjv.prs}\\
\glt `What  luck we readers have that someone writes to us.' (CdE)
\z 
\z 

Portuguese permits both word orders (cf. \ref{ex:whexclverbpt}).

\ea
\label{ex:whexclverbpt}  Brazilian Portuguese
\ea \gll Que sorte tive eu do ter conhecido. \\
what luck have.\textsc{1sg.prf.pst} I to-\textsc{cl.m.sg} \textsc{aux.inf.prf.prs} know.\textsc{ptcp}\\
\glt `What luck I had to get to have known him!' (CdP)
\ex \gll Que sorte eu tive de conseguir me despedir de você. \\
what luck I have.\textsc{1sg.prf.pst}  to manage \textsc{cl.1sg} {say goodbye} to you\\
\glt `What  luck I had to have managed to say goodbye to you.' (CdP) 
\z
\z


All three languages allow preverbal subjects in \is{wh-exclamative@\textit{wh}-exclamative}\textit{wh}-exclamatives with an attributive complementizer (as in \ref{ex:presubj}). Crucially, this is the case irrespective of whether the \textit{wh}-expression appears to have a function in the IP as in \eqref{ex:presubjb}  or  does not as in \eqref{ex:presubja}, \eqref{ex:presubjc}.

\ea\label{ex:presubj}
\ea\label{ex:presubja}
Catalan\\
\gll  Que trist que tu no sàpigues ni escriure ni expressar -te correctament. \\
how sad \textsc{que} you not  know.\textsc{2sg.sbjv.pst} neither write nor express \textsc{cl.2sg} correctly\\
\glt `How sad that you know neither how to write  nor how to express yourself correctly.' (caWac)
\ex\label{ex:presubjb}
Brazilian and European Portuguese\\ 
\gll Que parva (que) eu sou!  \\
what fool \textsc{que} I be.\textsc{1sg.prs}\\
\glt `What a fool I am.' (CdP)
\ex\label{ex:presubjc}
		Spanish\\
\gll  Qué lástima que la vida sea tan corta y las dictaduras tan largas. \\
What pity \textsc{que} the life be.\textsc{1sg.subj.prs} so short and the dictatorships so long\\
\glt `What a pity that life is so short and dictatorships are so long.' (CdE)
\z
\z




In the examples in \eqref{ex:presubja} and \eqref{ex:presubjc}, the verb  is inflected for subjunctive and not indicative as in the rest of the examples. In both \is{wh-exclamative@\textit{wh}-exclamative}\textit{wh}-exclamatives the \textit{wh}-expression is an evaluative term. The choice of mood, however, is not consistent but varies to a large degree. For instance,  example \eqref{ex:presubjc} contrasts with \eqref{ex:lastima1}, which contains the same evaluative term, \emph{lástima} `pity', but the  verb is inflected for indicative.  As the subjunctive is most commonly encountered in complement clauses, one might be led to the conclusion that these are elliptical structures ([\textsubscript{CP1} Qué lástima \sout{(es)} [\textsubscript{CP1} que la vida sea tan corta.]]). However, as can be seen in the example in \eqref{ex:ojala}, the subjunctive is not restricted to complement clauses but also appears in main clauses. This is why, for now, I stick to the idea that, despite the subjunctive mood,  these \is{wh-exclamative@\textit{wh}-exclamative}\textit{wh}-exclamatives can also be analyzed as main clauses.

\ea\label{ex:ojala}
Spanish\\
\gll Ojalá nos dejen. \\
hopefully \textsc{cl.1pl} leave.\textsc{3pl.sbjv.prs}\\
\glt `Hopefully they will leave us alone.' (CdE)
\z

One final argument in support of the move vs. merge distinction comes from data reported in \citet{Corr2016}. The author presents examples in which a complementizer precedes a \textit{wh}-expression in a \is{wh-exclamative@\textit{wh}-exclamative}\textit{wh}-exclamative such as \eqref{ex:irish}.

\ea\label{ex:irish} Galician (\citealt[90: ex 24]{Corr2016})\\ \gll Hala, que que ben (*que) fala a irlandesa!\\
wow \textsc{que} what well \textsc{que} speak.\textsc{3sg.prs} the Irish\\
\glt `Wow, the Irish girl speaks so well!' 
\ex\label{struc:irish} {\ob}\textsubscript{SubP} Que$_{j,\text{attributive}}$] ... {\ob}\textsubscript{FocP} que ben$_i$ t$_j$] ... {\ob}\textsubscript{FinP} t$_j$]{\ob}\textsubscript{IP}  fala t$_i$ a  irlandesa!]
\z

Incidentally, in all the examples of this type discussed in \citet{Corr2016},  a complementizer below the \textit{wh}-expression is ungrammatical and   all examples are cases where the \textit{wh}-expression can be reconstructed in the IP. This behavior is expected based on the assumptions made here. My  analysis is given in \eqref{struc:irish}. The \textit{wh}-expression is merged in the IP and moves to FocP.  The complementizer is merged in FinP and valued with an attributive feature. It moves from head to head, eventually reaching SubP. It is able to cross the \textit{wh}-expression in the specifier of FocP because the moved phrase does not constitute an obstacle to the complementizer movement. 


To conclude this section, I presented evidence in favor of my argument that the contrast between the \is{wh-exclamative@\textit{wh}-exclamative}\textit{wh}-exclamatives with and without the complementizer boils down to a move vs. merge distinction. As predicted by the general idea maintained in this book, it is precisely the  \textit{wh}-expressions that are merged in the left periphery that are followed by a complementizer.


\subsection{\emph{Que} with \isi{epistemic and evidential modifier}s}\label{sec:presupadvc}

This section deals with attributive \emph{que}  when it appears adjacent to \isi{epistemic and evidential modifier}s. I use the term AdvC to refer to this pattern. The analysis I propose is that the modifier is merged directly in the specifier of ModP. The complementizer starts out in FinP, where it receives the attributive feature and subsequently moves from head to head. The movement comes to a halt in the head of ModP because the complementizer cannot cross the base-generated modifier.

\begin{figure}
	\caption{\label{struc:advc}AdvC; crossed out lines represent a potential  movement that does not take place}
	\begin{forest}
		[ModP,name=modp
		[Claro, draw, name=spmod,label={below:\emph{merged}}] 
		[Mod', l sep+=\baselineskip
		[Mod$^0$\\que\textsubscript{attributive}, name=mod] 
		[MoodP
		[~] 
		[Mood' 
		[Mood$^0$\\\sout{que}, name=mood]  
		[FinP
		[~] 
		[Fin' 
		[Fin$^0$\\\sout{que\textsubscript{attributive}}, name=fin] 
		[IP,		
		[\dots,roof]
		{
		\begin{scope}[decoration={crosses,shape size=1.5mm}]
			\draw[->] (fin)  to[out=south west, in=south] (mood); 
			\draw[->] (mood)  to[out=south west, in=south] (mod);		
			\draw[->] (mod)  to[out=south west, in=south]	(-2,0.4); 
					\draw[decorate] (mod)  to[out=south west, in=south]	(-2,0.4); 
		\end{scope}
		}
		]]]]]]]	
	\end{forest}
\end{figure}


Independent of the construction at hand, the merger  of these types of modifiers  in the left periphery has been proposed previously: ModP was identified by \citet{Giorgi2010} and \citet{vanGelderen2011} as  the position in which \isi{epistemic and evidential modifier}s are base-generated.\footnote{The proposal that sentence adverbs are merged in the CP    is different from the analysis in \citet{Cinque1999}, who assumes that the  projections dedicated to Mod are below FinP. The conception of ModP implicit in \citet{Giorgi2010} and \citet{vanGelderen2011}  furthermore differs from that initially proposed by \citet{Rizzi2004a},  where ModP  is considered to be the position targeted by preposed adverbs which, as he claims, are distinct from left-peripheral \isi{topic}alized and focalized adverbs.} Based on the universal word order restrictions of co-occurring modifiers described in \citet{Cinque1999}, \citet{Giorgi2010} and \citet{vanGelderen2011}  propose that ModP consists of three sub-projections dedicated to  evaluative, evidential and epistemic meanings. For ease of exposition, I still assume that there is just one functional projection, ModP, since co-occurring modifiers expressing  these types of meanings are not a concern at  present.  

Although  the current focus is on  \isi{epistemic and evidential modifier}s, since they are most frequent, there are also some attestations of evaluative adverbs  in AdvC (cf. \ref{ex:evala}, \ref{ex:evalb}). This is in line with what the proposal in \citet{Giorgi2010} and \citet{vanGelderen2011} would suggest. Note that these examples, as well as most of the ones I will discuss in this section, are grammatical when attributive \emph{que} is absent.  However, the attributive meaning is then lost (cf. \sectref{sec:presupprag} and \sectref{sec:experiments}).

\ea\label{ex:evala}
European Portuguese\\ 
\gll {\ob}\textsubscript{ModP} Felizmente que] na Suiça se faz imensas coisas sem referendos. \\
{} fortunately \textsc{que} {in the} Switzerland \textsc{cl.refl} make.\textsc{3sg.prs} many thing.\textsc{pl} without referendum.\textsc{pl}\\
\glt `Fortunately there are many things happening in Switzerland without a referendum.' (CdP)
\ex \label{ex:evalb}
		Spanish\\
\gll
{\ob}\textsubscript{ModP} Lamentablemente que] no va a ocurrir. \\
{} lamentably \textsc{que} not go.\textsc{3sg.prs} to occur\\
\glt `Unfortunately, it won't occur.' (CdE)
\z




The idea introduced above allows a  compositional analysis, in which the modifiers are merged in the same place  irrespective of whether \emph{que} is present or not.  

Assuming that epistemic, evidential (and evaluative) modifiers are merged in, rather than moved to, the left periphery explains why attributive \emph{que} follows precisely these types of modifiers but precedes other  types. This can be seen in \eqref{ex:obviclar} in which the complementizer follows the evidential modifier\is{epistemic and evidential modifier}s \emph{obviamente} and \emph{claro} but precedes \emph{inmediatamente} and \emph{rapidamente}. This word order is expected under the analysis here because the lower adverbs are preposed from an IP-internal position. As moved elements, they do not constitute an obstacle to complementizer movement.
\ea\label{ex:obviclar}
\ea \label{ex:obvim}
		Spanish\\
\gll  {\ob}\textsubscript{ModP} Obviamente que] inmediatamente le surgirá la duda. \\
{} obviously \textsc{que} immediately him arise.\textsc{3sg.fut} the doubt\\
\glt  `Obviously,  doubt will arise immediately.' (CdE)
\ex \label{ex:clarrap}
European Portuguese\\ 
\gll  {\ob}\textsubscript{ModP} Claro que] rapidamente {deixa de} ser confortável fazer chamadas com o tablet ao ouvido. \\
{} clear \textsc{que} rapidly stop.\textsc{3sg.prs} be comfortable make call.\textsc{pl} with the table at.the ear\\
\glt `Clearly, it quickly stops being comfortable to make calls with a tablet at the ear.' (CdP)
\z
\z  


As   mentioned  in \sectref{sec:presupinitialC}, \isi{epistemic and evidential modifier}s can  appear in  short emphatic affirmations and negations introduced by attributive \emph{que} (cf. \ref{ex:emphshort}). In this case, the modifiers precede the complementizer, as expected. My  analysis for these patterns adopts  a polarity projection PolP sandwiched between FinP and IP. 
 
\ea 
\ea
Catalan\\
\gll  M' havia venjat jo mai de ningú? {\ob}\textsubscript{ModP} Segur que$_i$] {\ob}\textsubscript{FinP} t$_i$] {\ob}\textsubscript{PolP} sí.] \\
\textsc{cl.1sg} \textsc{aux.3sg.ipfv.pst} avenge.\textsc{ptcp} I never of {no one} {} sure \textsc{que} {} {} {} yes\\
\glt `Had I never sought revenge from anyone?  Sure I did.' (ebook-cat)
\ex
		Spanish\\
\gll Me preguntas si veo todo eso en tus ojos. {\ob}\textsubscript{ModP} Naturalmente que$_i$] {\ob}\textsubscript{FinP} t$_i$] {\ob}\textsubscript{PolP} sí],  amada mía.  \\
\textsc{cl.1sg} ask.\textsc{2sg.prs} if see.\textsc{1sg.prs} all this in your eye.\textsc{pl} {} naturally \textsc{que} {} {} {} yes beloved mine\\
\glt `You ask whether I see all of this in your eyes. Naturally I do, my beloved.' (CdE)
\ex Brazilian Portuguese\\
\gll  Mas isso não significa que o veganismo é uma política a se abandonar. {\ob}\textsubscript{ModP} Claro que$_i$] {\ob}\textsubscript{FinP} t$_i$] {\ob}\textsubscript{PolP}  não.] \\
but this not mean.\textsc{3sg.prs} that the veganism be.\textsc{3sg.prs} a politics to \textsc{cl.refl} abandon {} clear \textsc{que} {} {} {} not\\
\glt `But this does not mean that veganism is a policy to abandon. Of course not.'  (CdP) 
\z
\z








Since the modifier and the complementizer  occupy the specifier and the head of the same projection, the present analysis predicts that it should not be grammatical for any  left-peripheral material to intervene between them. The ungrammaticality of the intervening \isi{topic} in \eqref{ex:clldadvca} shows that this prediction holds true. A further prediction is that  AdvC can be followed and preceded by a \isi{topic} since here it is  merged in neither  the lowest nor highest position of the left periphery. The grammaticality of  \eqref{ex:clldadvcb} with a preceding \isi{topic} and  \eqref{ex:clldadvcc} with a \isi{topic} following the complementizer shows that this is indeed the case.




\ea \label{ex:clldadvc} Catalan
\ea\label{ex:clldadvca}

 \gll   {\ob}\textsubscript{ModP} *Segur]  {\ob}\textsubscript{TopP}  aquest llibre$_i$] {que}  l'$_i$ ha llegit. \\
		{} sure  {} this book \textsc{que} \textsc{cl.m.sg} \textsc{aux.3sg.prf.prs} read.\textsc{ptcp}\\
		\ex \label{ex:clldadvcb}

		\gll   {\ob}\textsubscript{TopP} Aquest llibre$_i$]  {\ob}\textsubscript{ModP} segur  que]  l'$_i$ ha llegit. \\
		{} this book  {} sure \textsc{que} \textsc{cl.m.sg} \textsc{aux.3sg.prf.prs} read.\textsc{ptcp}\\
	
		\ex\label{ex:clldadvcc} 

		\gll   {\ob}\textsubscript{ModP} Segur  {que}] {\ob}\textsubscript{TopP}  aquest llibre$_i$] l'$_i$ ha llegit. \\
		{} sure \textsc{que} {} this book \textsc{cl.m.sg} \textsc{aux.3sg.prf.prs} read.\textsc{ptcp}\\
		\glt `Surely, he indeed has read this book.'
		\z
\z




Furthermore, AdvC appears in embedded contexts. In \eqref{ex:advqueembeda}, it appears in the complement of a verb of saying\is{verbum dicendi}, in \eqref{ex:advqueembedb}, it appears in an appositive relative clause and in \eqref{ex:advqueembedc} in a restrictive relative clause.  This behavior  is also predicted by the analysis proposed for AdvC, and moreover constitutes  evidence in favor of the assumption that there is more than one  merge position accessible to the  complementizer in Ibero-Romance. 

\ea 
\ea\label{ex:advqueembeda} European Portuguese \\
\gll  Disse {\ob}\textsubscript{SubP} que]  {\ob}\textsubscript{ModP} certamente que] iria ver logo resultados. \\
say.\textsc{3sg.prf.pst}  {} that {} certainly \textsc{que} go.\textsc{1sg.cond} see soon result.\textsc{pl}\\
\glt `S/he said that certainly I would  see results soon.' (CdP)
	\ex\label{ex:advqueembedb}
			Spanish\\
			 \gll  Otra canción {\ob}\textsubscript{SubP} que]  {\ob}\textsubscript{ModP} claro {que}] escuchamos todos y que podría parecer muy buena, es ``Realmente no estoy tan solo''. \\ 
 other song {} that {} claro \textsc{que} listen-to.\textsc{1pl.prs} all and that could seem.\textsc{3sg.cond} very good be.\textsc{3sg.prs} Realmente no estoy tan solo\\
\glt `Another song that clearly we all listened to and that could seem very good is ``Realmente no estoy tan solo''.' (CdE)
\ex\label{ex:advqueembedc}
Catalan\\
\gll És gent {\ob}\textsubscript{SubP} que] {\ob}\textsubscript{ModP} segur que] has vist pero mai has passat un {cap de  setmana} amb ells. \\
are people {} that {} sure \textsc{que} \textsc{aux.2sg.prf.prs} see.\textsc{ptcp} but never \textsc{aux.2sg.prf.prs} spend.\textsc{ptcp} a weekend with them\\
\glt `These are people that you surely met but with whom you never spent a whole weekend.' (caWac)
\z
\z

Moreover,  AdvC can  appear in \emph{que}-initial reportatives as illustrated in  example \eqref{ex:multquebreprep} repeated  from \eqref{ex:multqueb} in \sectref{sec:insubwlp}. This again supports the idea that the two instances of \emph{que} have different functions and are merged in distinct positions.

\ea\label{ex:multquebreprep}
		Spanish\\
\gll {\ob}\textsubscript{SubP} Que]  {\ob}\textsubscript{ModP} claro que$_i$] {\ob}\textsubscript{FinP} t$_i$] le viene bien, que qué alegría, que dónde. \\
{} \textsc{que} {} clear \textsc{que} {} {} \textsc{cl.3sg} come.\textsc{3sg.prs} good \textsc{que} what joy \textsc{que} where \\
\glt `[reportative:] Of course it was no inconvenience. (She said) what a joy! (She said) where?' (CdE)
\z

Having presented the motivation for the analysis, I will now  describe further  properties of the construction; for a more complete characterization see \citet{Kocher2014, Kocher2017, Kocher2018}. One significant feature of AdvC is that it admits both underived modifiers (cf. \ref{ex:advunderived}), which  formally coincide with adjectives, and  derived modifiers (cf. \ref{ex:advderived}), which take the adverbial derivational suffix \emph{mente} (cf. also \citealt{Cruschina2017a,Cruschina2018}).

\ea Spanish
\ea\label{ex:advunderived}
		\gll Cierto que la culpa no es suya. \\
	certain \textsc{que} the fault not be.\textsc{3sg.prs} his\\
 	\glt `Certainly, it's not his fault.' (CdE)
\ex\label{ex:advderived}
\gll Ciertamente que la culpa no es suya. \\
certainly \textsc{que} the fault not be.\textsc{3sg.prs} his\\
\z
\z

 In \citet{Kocher2014, Kocher2018}, I presented evidence for the claim that despite these distinct forms, the underlying structure is the same. One central finding outlined in \citet{Kocher2014} is that the variation between the two different morphological forms depends on the language and the modifier (see also \sectref{sec:expdeic}). I  propose that they are all  sentential adverbs. This is not a controversial assumption for examples like \eqref{ex:advderived}, but requires further justification for examples like \eqref{ex:advunderived}.  In \citet{Kocher2014, Kocher2017}, I showed that, contra  \citet{MartinZorraquino1998}, \citet{FreitesBarros2006}, and  \citet{Ocampo2006}, it is not tenable to assume  that cases like \eqref{ex:advunderived} result from the \isi{ellipsis} of a copula construction (\emph{\sout{Es} cierto que...}), which I term EsAdjC.   First of all, Spanish, Catalan and Portuguese all allow underived modifiers to express adverbial functions (\citealt{Hummel2000, Hummel2017}). What is more, when the construction emerged in the 16th century (\citealt{Kocher2017}), the underived modifiers were in fact the dominant means of encoding adverbial functions (cf. \citealt{Hummel2017}).  A second argument is that, irrespective of whether the modifier is derived or not, they exhibit a parallel syntactic behavior,  suggesting that the underlying structure must be the same. Finally, the \isi{ellipsis} analysis is also not tenable from an interpretive point of view. It will be shown in \sectref{sec:presupprag} and \sectref{sec:expdeic} that the copula construction has a different interpretation than the AdvC construction, which is  consistent  regardless of  the morphological form of the modifier.
 
 AdvC displays further properties that show that it is not  merely an elliptical version  of a copula construction. First and foremost,  derived adverbs would not be expected to be attested. Moreover, the analysis proposed here predicts that any epistemic and evidential  modifier merged in ModP can appear in AdvC. Consequently,  not only derived and underived adverbs should be allowed, but also epistemic and evidential adverbial locutions. The examples in \eqref{ex:certeza}, \eqref{ex:duvida} and \eqref{ex:decerto} show that this prediction holds true.\largerpage[-2]
 
\ea
\ea\label{ex:certeza}
		Spanish\\
\gll  Con certeza que crearemos productos muy interesantes en esta nueva etapa. \\
 		with certainty \textsc{que} create.\textsc{1pl.fut} product.\textsc{pl} very interesting in this new phase\\
 		\glt `Certainly, we will create very interesting products in this new phase.' (CdE)

 		\ex\label{ex:duvida} European Portuguese\\ \gll   Sem dúvida que nos Açores há espaço e público para este tipo de eventos! \\
 		without doubt \textsc{que} {in the} Azores there.be.\textsc{3sg.prs} space and audience for this type of event.\textsc{pl}\\
 		\glt `Without a doubt, there is a space and an audience for this type of event in the Azores.' (CdP)
 		\ex\label{ex:decerto}  
 		\gll Decerto que o condutor adormeceu. \\
 		of-certain \textsc{que} the driver fall-asleep.\textsc{3sg.prf.pst}\\
 		\glt `Certainly the driver fell asleep.' (CdP)
 	\z
 \z
 
Additionally, the fact that only epistemic and evidential (and evaluative) modifiers are permitted constitutes an argument in favor of the non-elliptical nature of the structure. If this were merely a copula construction,  other  modifiers would be expected. For instance,  \emph{fundamental} in \eqref{ex:fundamentala}  is found frequently in copula constructions but is  ungrammatical in AdvC (see \ref{ex:fundamentalb}).\largerpage[2]

\ea Brazilian Portuguese
\ea[]{
\gll É fundamental que a criança explore o seu mundo. \\
be.\textsc{3sg.prs} essential that a child explore.\textsc{3sg.prs} the its world\\
\glt `It is essential that a child explores its world.' (CdP)\label{ex:fundamentala}}
\ex[*]{
\gll Fundamentalmente/ Fundamental que a criança explore o seu mundo. \\
     essentially essential that a child explore.\textsc{3sg.prs} the its world\\
\glt Intended: `It is essential that a child explores its world.'\label{ex:fundamentalb}}
\z
\z 



One additional argument is that the modifier in AdvC cannot be itself modified (compare \eqref{ex:bemcertoa} and \eqref{ex:bemcertob}; see also \citealt{Cruschina2017a, Cruschina2018}). This would once again not be expected if it were just a version of a copula construction. 
 
\ea {European Portuguese}
\ea[]{ \gll  É bem certo que as rivalidades se semeiam. \\
	be.\textsc{3sg.prs} good certain that the rivalry.\textsc{pl} \textsc{cl.refl} spread.\textsc{3pl.prs}\\
	\glt `It is pretty certain that the rivalries are spreading.' (CdP)\label{ex:bemcertoa}}
\ex[*]{ \gll  Bem  certamente/ certo que as rivalidades se semeiam. \\
 good  certainly  certain that the rivalry.\textsc{pl} \textsc{cl.refl} spread.\textsc{3pl.prs}\\\label{ex:bemcertob}}
\z
	\z

Furthermore, AdvC is incompatible with morphological negation (cf. \ref{ex:bellugina} and \ref{ex:belluginb}), which further supports the idea that  it is more than just an elliptical version of a copula construction. The same fact is also observed in \citet{Cruschina2017a, Cruschina2018}.  

 \ea Catalan
 \ea[]{ 
 \gll  És impossible que es belluguin tant sense la meva voluntat. \\
 be.\textsc{3sg.prs} impossible that \textsc{cl.refl} move.\textsc{3pl.sbjv.prs} so without the my will\\
\glt `It is impossible that they move this much against my will.' (ebook-cat)\label{ex:bellugina}}
\ex[*]{ \gll  Impossiblement que es belluguin tant sense la meva voluntat.\\
		 impossibly that \textsc{cl.refl} fight.\textsc{3pl.sbjv.prs} so without the my will\\
\label{ex:belluginb}}
\z 
 \z



With regard to its cross-linguistic distribution, AdvC is  attested in all three languages, although there is some variation in the frequency of individual modifiers and whether derived or underived adverbs are preferred (cf. \citealt{Kocher2014, Kocher2017} and \sectref{sec:expdeic}). 

To close this section, I discuss the  development of a new  adverb in Catalan that illustrates the productivity of AdvC. Recently, \textit{esclar} ($<$ \textit{és clar}) has emerged as a new evidential adverb (see \sectref{sec:expdeic} where I  interpret  experimental results  in light of this new adverb).
Catalan displays a tendency to grammaticalize new words on the basis of univerbation. A prominent example is \textit{sisplau} `please', which results from the grammaticalization of conditional  \textit{si us plau} `if you\textsubscript{2p} please' ( see \citealt{Alturo2009} on the grammaticalization of \textit{sisplau}). The process features typical traits  of grammaticalization, namely phonological  ([siwsplaw] $>$ [sisplaw]) and morphological reduction (loss of second person plural agreement), syntactic reanalysis (protasis of conditional clause $>$ adverb) and semantic change (cf. \citealt{HopperTraugott2003}).
Similarly, the Catalan evidential adverb \textit{esclar} resulted from univerbation, originating from  EsAdjC (cf. \ref{ex:esclarstruc}).
\ea\label{ex:esclarstruc} {\ob}\textsubscript{CP1} És clar {\ob}\textsubscript{CP2} (que) {\ob}\textsubscript{IP} \ldots ]]]\\
\z

At some point in the development, the parenthetical use of the sequence \textit{és clar} enabled  grammaticalization towards an adverb, see \eqref{ex:esclarparent}.


\ea \label{ex:esclarparent}Catalan
\ea 
\gll
		Jo, és clar, en això no sé què aconsellar-te. \\
		I is clear in that not know.\textsc{1sg.prs} what recommend-you\\
		\glt `I -that is clear- don't know what to recommend you in this case.' (caWac)
		\ex

		\gll Jo no conec mai els meus clients. Ni ells a mi, és clar. \\
		I not know.\textsc{1sg.prs} ever the my client.\textsc{pl} nor they to me be.\textsc{3sg.prs} clear\\
		\glt `I never know my clients, nor they know me - that is clear.' (caWac)
	\z
\z

Typical features of grammaticalization can also be observed in the case of \textit{esclar} such  as phonological reduction ([eskla] $>$ [əskla]) and syntactic reanalysis (clause $>$ adverb). 

\ea\label{ex:esclarmobile} Catalan
\ea
\gll Sola no hi he estat mai, {esclar}, però sé què vull dir. \\
		alone not there \textsc{aux.1sg.prf.prs} be.\textsc{ptcp} ever clearly but know.\textsc{1sg.prs} what want.\textsc{1sg.prs} say\\
		\glt `I've never been there on my own, evidently, but I know what it means.' (caWac)
	
		\ex
		 \gll No es tracta, {esclar}, de desmentir l'existència d'una organització social. \\  
		not \textsc{cl.refl} treat.\textsc{3sg.prs} clearly to deny {the existence} {of a} organization social\\
		\glt `Evidently, it's not about denying the existence of a social organization.' (caWac)
	\z
\z

The examples in \eqref{ex:esclarmobile} illustrate the syntactic mobility of \textit{esclar}, which parallels the syntactic mobility of epistemic and evidential adverbs.

\ea \label{ex:esclarque}Catalan\\
\gll  ¿Com era possible que una {obra mestra}  {[...]} tingués un {final feliç} com aquell? -{Esclar que} {per a} Espert el suïcidi de la protagonista també es pot entendre com un final feliç. \\
		how be.\textsc{3sg.ipfv.pst} possible that a masterpiece { } have.\textsc{3sg.ipfv.pst} a {happy ending} like that {clearly \textsc{que}} for E. the suicide of the protagonist also \textsc{cl.refl} can.\textsc{3sg.prs} understand as a happy ending.\\
		\glt `How was it possible that a masterpiece had a happy ending like that? - Evidently for Espert the suicide of the protagonist can also be interpreted as a happy ending.' (caWac)
\z 

One final piece of evidence demonstrating the adverb-hood of \textit{esclar} is the fact that it can appear  in AdvC \eqref{ex:esclarque}. This  development consequently  constitutes compelling evidence for the compositional structure of the constructions that I argue for in this chapter. The reasoning is that, if these were fixed grammaticalized expressions,  a productive extension of the construction to novel modifiers would not be expected.

\subsection{\emph{Que} in \isi{verum} sentences}\label{sec:presupaffc}
In this section I focus on a construction in which attributive \emph{que} appears jointly with the \isi{verum} marker \emph{sí}.  I will use  the term AffC to refer to this pattern. The construction is not attested in European Portuguese, so the data illustrating it are  drawn exclusively from Spanish and Catalan. Explanations for this cross-linguistic contrast will be addressed at the end of this section.

The analysis I adopt here is very similar to the analysis outlined above for AdvC and is illustrated in \figref{struc:affc}. The assumption is once again that the complementizer is merged in FinP where it receives the attributive interface feature. The complementizer then moves to the next head in the hierarchy, the sentence mood head MoodP. The \isi{verum} marker is directly merged in the specifier of this head. This idea is inspired by \citet{Lohnstein2015}, who argues that the \isi{verum} interpretation results from stress on sentence mood. Consequently, since the \isi{verum} marker \emph{sí} is merged rather than moved, the complementizer cannot cross it, and therefore the surface word order in which \emph{que} follows \emph{sí} obtains.
 
 
\begin{figure}
	\caption{\label{struc:affc}AffC; crossed out lines represent a potential  movement that does not take place}
	\begin{forest}
		[MoodP,name=moodp
		[Sí, draw, name=spmood, label={below:\emph{merged}}] 
		[Mood',l sep+=\baselineskip
		[Mood$^0$\\que\textsubscript{attributive}, name=mood] 
		[FinP
		[~] 
		[Fin' 
		[Fin$^0$\\\sout{que\textsubscript{attributive}}, name=fin] 
		[IP,		
		[\dots,roof]
		{
		 \begin{scope}[decoration={crosses,shape size=1.5mm}]
			\draw[->] (fin)  to[out=south west, in=south] (mood); 
		    \draw[->] (mood)  to[out=south west, in=south]	(-2,0.4); 
			\draw[decorate] (mood)  to[out=south west, in=south]	(-2,0.4); 
		\end{scope}
		}
		]]]]]
	\end{forest}
\end{figure}

Previous syntactic analyses of \isi{verum} focus and other polarity related phenomena also propose a projection in the same area of the left periphery. The difference is that in these accounts the projection was specialized for polarity or \isi{verum} (\citealt{Laka1990}, \citealt{Martins2006}, \citealt{Hernanz2007}, \citealt{Batllori2008}, \citealt{RodriguezMolina2014}, \citealt{VillaGarcia2020b,VillaGarcia2020a}).  Since the present analysis is grounded in   \citeauthor{Lohnstein2015}'s sentence mood theory of \isi{verum} focus, which argues that the interpretive effect of \isi{verum} results from focusing sentence mood, a dedicated projection for \isi{verum} becomes obsolete.\largerpage
 
\citet{Lohnstein2015} uses German data to support his proposal. German is a V2 language, meaning that the finite verb always occupies the second position in a root declarative. This is often explained as the result of  movement of the finite verb  to a left-peripheral position (cf. \citealt{Roberts2001} and references therein). \citet{Lohnstein2015} proposes that the position targeted is a sentence mood head that he calls MoodP.  
As stated above, \citeauthor{Lohnstein2015}'s  basic idea is that the effect of \isi{verum} results from \isi{focus} on sentence mood. A relation between sentence mood and  \isi{verum} has already been observed by \citet{Hohle1992}  who offers the first extensive study of \isi{verum} focus. 
The main empirical evidence for \citeauthor{Lohnstein2015}'s idea is given in \eqref{ex:loh}. In German declaratives, \isi{verum} is marked through stress on the finite verb as in \eqref{ex:lohdec}.  In embedded declaratives, in which the finite verb is sentence-final, the stress does not fall on the verb, but on the complementizer \eqref{ex:lohembed}. \citet{Lohnstein2015} takes this as evidence that the complementizer occupies MoodP, hence why the \isi{focus} feature is expressed as stress on this element.

\ea \label{ex:loh} German
\ea\label{ex:lohdec}
		 (\citealt[1: ex 1a]{Lohnstein2015})\\
\gll Karl \textsc{hat} den Hund gefüttert. \\
		Karl \textsc{aux.3sg.prf.prs} the dog feed.\textsc{ptcp}\\
		\glt `Karl \textsc{did} feed the dog.' 
		\ex \label{ex:lohembed} (\citealt[2: ex 1e]{Lohnstein2015})\\

		\gll Aber Maria glaubt \textsc{dass} Karl in Urlaub gefahren ist. \\
but Maria believe.\textsc{3sg.prs} that Karl in holiday drive.\textsc{ptcp} \textsc{aux.3sg.prf.prs}\\
\glt `But Maria believes that Karl \textsc{did} go on holiday.'
	\z
\z\largerpage


In \citet{Kocher2017,Kocher2018b}, I show that the analysis proposed by \citet{Lohnstein2015} to account for the German data can be extended to account for Spanish. Given  that  Spanish and Catalan employ the same \isi{verum} strategy, it is plausible to assume that the analysis can  be extended to  account for Catalan as well. The basic idea is that the underlying structure is the same for the two Ibero-Romance languages and German.  I model this by assuming  a \isi{focus} feature in MoodP. The finite verb  remains in the IP in Spanish and Catalan; the \isi{focus} feature on sentence mood requires lexical material to be expressed, and \emph{sí} is merged directly in the specifier of the projection to satisfy this requirement. 

This analysis  offers a straightforward explanation for why the \isi{verum} marker \emph{sí} does not occur in imperatives (cf. \ref{ex:siimp} vs. \ref{ex:nosiimp}). To construct this argument a short digression is required to account for my analysis of  directives. The properties of the Ibero-Romance directives are illustrated in \eqref{ex:allimps}. The central contrast is between verb-  and \emph{que}-initial directives  (\eqref{ex:imperativearep} repeated from \eqref{ex:imperativea},  \eqref{ex:jusnoquearep} repeated from \eqref{ex:jusnoquea}  in  \sectref{sec:insubwlp} vs. \eqref{ex:imperativecrep} repeated from \eqref{ex:imperativec} in \sectref{sec:insubwlp}). They differ in their verbal mood (imperative vs. subjunctive), the position of pronominal clitics (enclisis vs. proclisis) and subjects (post- vs. preverbal).  A detailed discussion of their empirical properties can be found in \sectref{sec:insubwlp}.
\ea\label{ex:allimps} Spanish
\ea		
\gll ¡Vete! \\
	go.\textsc{imp}-\textsc{cl.2sg}\\
	\glt `Go away!'\label{ex:imperativearep}
	\ex 
\gll ¡Que se vaya! \\
	\textsc{que} \textsc{cl.3sg} go.\textsc{3sg.sbjv.prs}\\
	\glt `He/She should go!'\label{ex:imperativecrep}
\ex
\gll ¡Váyase! \\
	\textsc{que} \textsc{cl.3sg} go.\textsc{3sg.sbjv.prs}\\
	\glt `He/She should go!'\label{ex:jusnoquearep}
\z
\z

In order to account for third-person \emph{que}-initial directives in Ibero-Romance, I build on the claim from \citet{DemonteSoriano2009} and  \citeauthor{Rivero1995} (\citeyear{Rivero1995}) that the complementizer is merged in a left-peripheral position.  A  difference between \citet{DemonteSoriano2009} and the  analysis I propose is the location in which \emph{que} is assumed to be merged. \citet{DemonteSoriano2014} propose that \emph{que} is merged in FinP to license an imperative feature. I, however, consider  \citeauthor{Lohnstein2015}'s \isi{clause typing} projection MoodP, located immediately above FinP, to be the more appropriate candidate because  FinP is not associated with \isi{clause typing} in the present analysis. There is also empirical proof exemplified in \eqref{ex:clldjuss} that shows that \emph{que} can be followed by a clitic left dislocated \isi{topic} that is assumed to be merged in a left-peripheral \isi{topic} position. Crucially,  this word order is correctly predicted by the structure proposed here, but would not be expected if the complementizer occupied the lowest projection of the left periphery FinP, which does not have a \isi{topic} projection below it. 

\ea  \label{ex:clldjuss}
		Spanish\\
\gll {\ob}\textsubscript{MoodP} Que  {\ob}\textsubscript{TopP} los libros$_i$ {\ob}\textsubscript{FinP} {\ob}\textsubscript{IP} los$_i$ lean los  ni\~nos  y no las revistas.{\cb}{\cb}{\cb}{\cb} \\
{} \textsc{que} {} the book.\textsc{pl}  {} {} \textsc{cl.3pl} read.\textsc{3pl.sbjv.prs} the children and not  the magazines\\
\glt`The children must read the books, not the magazines.' 
\z

\begin{sloppypar}
The basic idea of the analysis   is that MoodP encodes the clause type of each sentence through a feature (see \citealt{Lohnstein2015} and \citealt{Kocher2018b}). For the present purposes, I assume the minimal repertoire of a declarative, an interrogative and an imperative feature.\footnote{While these three clause types are generally accepted in the literature, some authors argue that there are additional types. See for instance \citet{Zanuttini2003} who  consider exclamatives as a fourth clause type.} While the  verbal mood of \emph{que}-initial directives  is subjunctive, the clause type  is nevertheless imperative (cf. \citealt{Portner2004}). In line with the empirical data, I suggest that imperative sentences in Ibero-Romance require that the feature is checked directly in MoodP. In \emph{que}-less imperatives the imperative verb itself checks the feature. The data illustrated in \eqref{ex:allimps} and discussed in greater detail in \sectref{sec:insubwlp}  show that the imperative verb occupies a higher position than  indicative verbs in Catalan and Spanish  and than subjunctive verbs in all three languages. Unstressed pronouns are enclitic and neither subjects nor negative particles can appear before the imperative verb. 
\end{sloppypar}

To account for the merger of \emph{que} in \emph{que}-initial third person directives, I adopt  the last resort explanation from \citet{Rivero1995} and \citet{DemonteSoriano2014}.  In regular imperatives, the verb has the necessary morphological features (=imperative inflection) to check the imperative feature. This  triggers the movement of the verb to the projection. In  third-person \emph{que}-directives, the subjunctive verb is not equipped with the necessary feature; its movement to the left periphery is therefore inhibited  and the verb is unable to check the feature in Mood$^0$. Since imperatives require a local checking of the feature, as a last resort, the underspecified complementizer is merged directly in MoodP to satisfy this requirement. 

Returning now to the issue of   the incompatibility of \emph{sí que} and imperatives, the main point is  that directives  require local checking of the sentence mood feature and that the imperative verb moves to Mood$^0$.  What this means for the \isi{verum} construction is that in imperatives, there is no need for the merger of \emph{sí} in Mood because the \isi{focus} feature can be expressed on the verb itself.

\ea Spanish (\citealt[27: ex 49]{Kocher2018b})
\ea[*]{\label{ex:siimp}
\gll  ¡Sí que cógete una silla {de una vez}! \\
\textsc{verum} \textsc{que} grab.\textsc{imp}-\textsc{cl.2sg} a chair {for once}\\}

\ex[]{\gll ¡\textsc{cógete} una silla {de una vez}! \\
grab.\textsc{imp}-\textsc{cl.2sg} a chair {for once}\\
\glt `\textsc{grab} a chair, at once!'\label{ex:nosiimp}}
\z
\z



The presence of \emph{que} is not required for Spanish and Catalan \isi{verum} sentences (cf. \ref{ex:facque}). 

\ea\label{ex:facque}
Catalan\\
\gll Sí (que) ve a la festa. \\
\textsc{\isi{verum}} \textsc{que} come.\textsc{3sg.prs} to the party\\
\glt `S/he \textsc{does} come to the party.'
\z

I discuss  some interpretive differences between \isi{verum} sentences with and without \emph{que} in Spanish and Catalan in \sectref{sec:presupprag}. With regard to the syntactic structure, the idea of the  analysis presented here is again that \emph{sí} is merged in the same place irrespective of whether \emph{que} is present or not. This allows for a compositional analysis. In the literature focusing on \emph{sí} and \emph{sí que}, it was often proposed that \emph{sí} without \emph{que} is merged in the same IP-internal polarity position as sentential negation \emph{no}  (see \citealt{Hernanz2007}, \citealt{VillaGarcia2020b, VillaGarcia2020a}). The data in \eqref{ex:sino} constitute the principal evidence against this assumption, showing that \emph{sí} cannot occupy this position because it can co-occur with a sentential negation particle.\pagebreak

\ea\label{ex:sino}
\ea\label{ex:sinosp}
		Spanish  (\citealt[94: ex 31a]{Kocher2017} from CdE)\\
\gll {\ob}\textsubscript{TopP} Eso] {\ob}\textsubscript{MoodP} sí] {\ob}\textsubscript{PolP} no podía faltar en ninguna casa.] \\
{} this {} \textsc{verum} {} not can\textsc{3sg.ipfv.pst} miss in any house\\
\glt `This could \textsc{not} be missed in any house.'
\ex\label{ex:sinocat} Catalan\\
\gll {\ob}\textsubscript{TopP} Aixó] {\ob}\textsubscript{MoodP} sí] {\ob}\textsubscript{PolP}  no sé fins quan durarà amb l' ajuntament que tenim.]\\
{} this {} \textsc{verum} {} not know.\textsc{1sg.prs} until when take.\textsc{3sg.fut} with the {city hall} that have.\textsc{1pl.prs}\\
\glt `I do \textsc{not} know how long this will take with the city hall we have.' (caWac)
\z
\z


One prediction that follows from the analysis I have presented here is once again that it should not be possible for \emph{sí} and \emph{que} to be separated by other left-peripheral material. The example in \eqref{ex:sitopque} shows that this prediction holds true. A clitic left-dislocated \isi{topic} cannot intervene between \emph{sí} and \emph{que}.

\ea\label{ex:sitopque}
Catalan\\
\gll * Sí aquesta noia$_i$ que la$_i$ conec. \\ 
{} \textsc{\isi{verum}} this girl \textsc{que} \textsc{cl.f.sg} know.\textsc{1sg.prs}\\
\glt Intended `This girl I \textsc{do} know.'
\z


Because of its  position within the left periphery,  it should be possible for the sequence of \emph{sí} and \emph{que} to be both followed and preceded by \isi{topic}s. The examples in  \eqref{ex:sitopa} and \eqref{ex:sitopb} show that this is indeed the case.\largerpage[2]


\ea 
\ea \label{ex:sitopa} Catalan \\ \gll {\ob}\textsubscript{TopP} Aquesta noia$_i$] {\ob}\textsubscript{MoodP} sí que] la$_i$ conec.\\
	{} this girl {}	\textsc{\isi{verum}}  \textsc{que} \textsc{cl.f.sg} know.\textsc{1sg.prs}\\
	\glt `This girl I \textsc{do} know.'
		\ex\label{ex:sitopb}\gll  {\ob}\textsubscript{MoodP} Sí que] {\ob}\textsubscript{TopP} aquesta noia$_i$] la$_i$ conec.\\
		 {}	\textsc{\isi{verum}}  \textsc{que} {} this girl \textsc{cl.f.sg} know.\textsc{1sg.prs}\\
	\glt `This girl I \textsc{do} know.'	
	\z
\z

The analysis moreover predicts that \emph{sí que} should follow modifiers merged in ModP, which is also what is found in the data.\footnote{There are some examples in which the reverse order obtains. Although a definite analysis of these cases is pending further investigation, it  is notable that  examples of \emph{sí}  following AdvC are restricted to cases involving \emph{claro que}. There are indications that  \emph{claro que} is undergoing a process of grammaticalization and turning into a fixed expression (\citealt{Kocher2014,Kocher2018}, see also \sectref{sec:expdeic}). It  might therefore no longer be analyzed compositionally by some speakers, which could be a potential explanation for these examples.}

\ea\label{ex:seguramentsi}
Catalan\\
\gll {\ob}\textsubscript{ModP} Segurament] {\ob}\textsubscript{MoodP} sí que] és el millor dissenyat a nivell funcional (o això crec jo). \\
{} surely {} \textsc{verum} \textsc{que} be.\textsc{3sg.prs} the best design at level functional or this believe.\textsc{1sg.prs} I\\
\glt `Surely this \textsc{is} the best design in terms of functionality (or at least I think so).' (caWac)
\z


Furthermore, AffC, like AdvC, can appear in embedded contexts. In \eqref{ex:embedsiquea} and \eqref{ex:embedsiqueb}, it follows a verb of saying\is{verbum dicendi}. In \eqref{ex:embedsiquec}, it appears  in an appositive relative clause. In \eqref{ex:embedsiqueb}, a \isi{topic} and an epistemic modifier\is{epistemic and evidential modifier} are merged above AffC,  demonstrating the availability of structural positions above it.\largerpage


\ea
\ea \label{ex:embedsiquea}
		Spanish\\
\gll  Ella le dijo {\ob}\textsubscript{SubP} que]  {\ob}\textsubscript{MoodP} sí que] le extrañaba. \\
		she \textsc{cl.m.sg} tell.\textsc{3sg.prf.pst} {} that {} \textsc{verum} \textsc{que} \textsc{cl.m.sg} miss.\textsc{3sg.ipfv.pst}\\
		\glt `She told him that she \textsc{did} miss him.' (CdE)
		
		\ex \label{ex:embedsiqueb}		
		Catalan\\
		\gll[...] i ha pensat {\ob}\textsubscript{SubP} que]  {\ob}\textsubscript{ModP} potser] {\ob}\textsubscript{TopP} d’ aquesta dona$_i$]  {\ob}\textsubscript{MoodP} sí que] se ’n$_i$ podria enamorar. \\
		{}   and \textsc{aux.3sg.prf.prs} think.\textsc{ptcp} {} that {} maybe {}  of this woman {} \textsc{verum} \textsc{que} \textsc{cl.refl} \textsc{cl.part} can.\textsc{3sg.cond} {fall in love}\\
		\glt `And he has thought that maybe it was true that he could fall in love with this woman.' (ebook-cat)
		\ex \label{ex:embedsiquec}
		 \gll Les tecnologies no són dolentes, el {\ob}\textsubscript{SubP} que]  {\ob}\textsubscript{MoodP} sí que] pot ser perjudicial és l' ús que en fem. \\
		the technology.\textsc{pl} be.\textsc{3pl.prs} not bad the {} that {} \textsc{verum} \textsc{que} can.\textsc{3sg.prs} be prejudicial is the use that \textsc{cl.part} make.\textsc{1pl.prs}\\
		\glt `Technologies themselves are not bad, what \textsc{can} be prejudicial is our use of them.' (caWac) 
			\z 
\z

\eqref{ex:quesique} illustrates an example of AffC in a \emph{que}-initial reported sentence. These data show again that the complementizer involved in \emph{que}-initial reportatives has a different function and is merged in a different position than the complementizer present in AffC. 

\ea\label{ex:quesique} Spanish \\ \gll  A: No vas a venir, {¿pues no?}\\
{} not go.\textsc{2sg.prs} to come right\\
\exi{}\gll B: Sí que vengo.\\
{} \textsc{verum} \textsc{que} come.\textsc{1sg.prs}\\
\exi{}\gll A: ¿Eh?\\
{} huh\\
\exi{}\gll B: {\ob}\textsubscript{SubP} Que] \dots {\ob}\textsubscript{MoodP} sí que] vengo.\\
{} {} \textsc{que} {} \textsc{verum} \textsc{que} come.\textsc{1sg.prs}\\
\glt `A: You're not coming, right? B: I \textsc{am} coming. A: Huh? B: [reportative:] I \textsc{am} coming.'
\z


Finally, the data shown in \eqref{ex:siquesi}, in which  \emph{sí que} appears in short emphatic affirmations and negations, constitute further evidence that when expressing \isi{verum}, \emph{sí} targets a left-peripheral position and is not merged in the lower PolP.

\ea\label{ex:siquesi}
\ea
		Spanish\\
\gll ¡Eso {\ob}\textsubscript{MoodP} sí que$_i$] {\ob}\textsubscript{FinP} t$_i$] {\ob}\textsubscript{PolP} no!] \\ 
this {} \textsc{verum} \textsc{que} {} {} {} no\\
\glt `This:  no way!'(CdE)
\ex
Catalan\\
\gll Ara {\ob}\textsubscript{MoodP} sí que$_i$] {\ob}\textsubscript{FinP} t$_i$] {\ob}\textsubscript{PolP} sí!] \\
now {} \textsc{verum} \textsc{que} {} {} {} yes\\
\glt `Now: absolutely yes!' (caWac)
\z
\z


I will now turn to the question of  cross-linguistic variation. As  stated at the start of this section, AffC is not available in  Portuguese.  According to \citet{Martins2013}, there are two alternative strategies to express \isi{verum} in European Portuguese:  Verb reduplication and final or post-verbal \emph{sim}. The verb reduplication strategy, in which the finite verb is doubled, is illustrated in \eqref{ex:comproucarro}.

\ea\label{ex:comproucarro}Portuguese (\citealt[97: ex 1]{Martins2013})\\  \gll A: O João não comprou o carro. \\
{} the João not buy.\textsc{3sg.prf.pst} the car\\
\exi{}\gll B: O João comprou o carro comprou.\\
{} the João  buy.\textsc{3sg.prf.pst} the car buy.\textsc{3sg.prf.pst}\\
\glt `A: João didn't buy the car. B: João \textsc{did} buy the car.' 
\z



The analysis proposed in \citet{Martins2013} is given in \eqref{struc:comproucarro}. 

\ea\label{struc:comproucarro} {\ob}\textsubscript{ToP} [ele comprou$_i$ o carro]$_k$ {\ob}\textsubscript{Top'} {\ob}\textsubscript{CP} {\ob}$_{C'}$ {\ob}$_{C}$ comprou$_i$] 
	\sout{{\ob}$_{\Sigma P}$ ele$_j${\ob}$_{\Sigma'}$ comprou$_i$ {\ob}\textsubscript{IP} {\ob}$_{I'}$ comprou$_i$ {\ob}\textsubscript{VP} ele$_j$ comprou$_i$ o carro]]]]]$_k$} ]]]] (\citealt[101: 8c]{Martins2013})
\z



Her proposal is that verb reduplication results from the phonetic realization of two copies of the verbal movement chain. The finite verb moves from the IP to $\Sigma$P, a polarity position sandwiched between IP and the left periphery and can thus be identified with PolP. The verb  then moves to a CP position, the nature of which is not further specified in \citet{Martins2013}.  As a final step in the derivation, she assumes remnant movement of the $\Sigma$P to the specifier of Top, which she analyzes as an example of IP-\isi{topic}alization.  

\citet{Martins2013} draws a comparison between this and the Spanish and Catalan AffC construction. She proposes that the contrasting strategies result from the allowance vs. disallowance of verbal movement to $\Sigma^0$ and the (unspecified) C-position. In her account, this is why the verb reduplication that relies on movement of the verb to these positions is only available in European Portuguese, which allows verb movement of this sort, but not in the other two languages, which do not.
 
Although I will not offer a detailed account at this point, it seems that this analysis can be integrated fairly straightforwardly into the theory developed in this chapter, simply by proposing that the unspecified C-position  in \citet{Martins2013} can be identified with MoodP.  It is not only involved in the Spanish and Catalan verum construction: As stated above, this clause-typing projection has been identified as the host of the finite verb in V2 languages and is also the landing site of the high-merged verbs in Ibero-Romance imperatives. 


The second alternative strategy that European Portuguese employs according to \citet{Martins2013} is one in which \emph{sim} appears post-verbally.   This is exemplified in \eqref{ex:finalsim}.
 
\ea\label{ex:finalsim} Portuguese (\citealt[117: ex 57b]{Martins2013})\\ \gll Comprou sim. \\
 buy.\textsc{3sg.prf.pst} \textsc{aff}\\
 \glt `Yes, he \textsc{did} buy it.' 
 \z
 
Final \emph{sim} is not restricted to European Portuguese; its attestations in CdP actually show a greater preference in Brazilian than in European Portuguese. The pattern is furthermore attested in Spanish, and   here too, to a larger degree in Latin American varieties than in European Spanish.
 
 

\citet{Martins2013} proposes an analysis of final \emph{sim} that makes use of the same mechanisms assumed for verb reduplication. She proposes that \emph{sim} is directly merged in the C-position,  the verb moves to $\Sigma^0$, and the whole $\Sigma$P is again moved to a left periphery  \isi{topic} position resulting in the observed word order in which the finite verb precedes \emph{sim}. 


 I am cautious about adopting this analysis,   because there are multiple attestations of the post-verbal \emph{sim} pattern that cannot be accounted for. The examples in \eqref{ex:adjasim} show that \emph{sí/sim} appears adjacent to the finite verb and crucially, precedes the arguments of the verb. This word order is not expected based on the analysis proposed by \citet{Martins2013}.


\ea\label{ex:adjasim}
\ea Brazilian Portuguese\\ \gll E ainda esqueceu-se de consultar um dicionário para ver que a palavra presidenta existe sim na língua portuguesa. \\
and yet forget.\textsc{3sg.prf.pst}-\textsc{cl.refl} to consult a dictionary to see that the word presidenta exist.\textsc{3sg.prs} \textsc{sim} {in the} language Portuguese\\
\glt `And yet it was forgotten to consult a dictionary to see that the word `presidenta' \textsc{does} exist in the Portuguese language.'  (CdP) 
 \ex Columbian Spanish \\ \gll  Escribir novelas tiene sí algo de riesgo. \\
 write novels have.\textsc{3sg.prs} \textsc{si} something of risk\\
 \glt `Writing novels \textsc{does} carry some risks.' (CdE)
\z
\z


Final \emph{sí/sim} is furthermore compatible with attributive \emph{que}. In \eqref{ex:finalsimadv} this is illustrated by  an example where it co-occurs with AdvC.\largerpage[2]

\ea\label{ex:finalsimadv}
\ea Brazilian Portuguese \\ \gll  Claro que existe sim uma relação.\\
clear \textsc{que} exist.\textsc{3sg.prs} \textsc{sim} a relation\\
\glt `Clearly, there \textsc{does} exist a relation.'  (CdP) 
\ex Spanish \\ \gll  Claro que la creatividad debe sí girar entorno a la música. \\
clear \textsc{que} the creativity should.\textsc{3sg.prs} \textsc{si} revolve around of the music\\
\glt `Clearly, the creativity \textsc{should} revolve around music.' (CdE)
\z
\z


 
There are data that suggest that an altogether different way of accounting for post-verbal \emph{sim} might be called for. There are multiple instances of  \emph{sim} surfacing adjacent to different types of XPs. What \emph{sim} appears to mark in these examples is a contrast with a salient (negative) alternative. In \eqref{ex:jovem} the days the young woman works are contrasted with those she does not work. In \eqref{ex:pais} the contrast is between the father who is protective and the mother who is not.
\ea
\ea\label{ex:jovem} Brazilian Portuguese\\
\gll 
A jovem trabalha dia sim, dia não. \\
the young-woman work.\textsc{3sg.prs} day \textsc{sim} day \textsc{n\~ao}\\
\glt `The young woman works every other day.' (CdP)
\ex\label{ex:pais} Angolan Portuguese\\
\gll Os seus pais são muito protectores? O meu pai sim, a minha mãe não.\\
the your parents be.\textsc{3pl.prs} very protective the my father \textsc{sim} the my mother \textsc{n\~ao}\\
\glt `Are your parents very protective? My father is, my mother isn't.' (CdP)
\z 
\z 

I believe that the meaning of post-verbal \emph{sim} can be captured in this way as well. The examples in \eqref{ex:ansmartins} illustrate a  contrast noted in  \citet{Martins2013} between an answer with \emph{sim}  as a regular answer particle and an answer containing post-verbal \emph{sim}. While \emph{sim} followed by a pause is  used as an answer to a neutral, un\isi{bias}ed question \eqref{ex:ansinitialsim}, in \eqref{ex:ansfinalsim} the final \emph{sim} is used to rebut the salient negative alternative \emph{não comprou} `he didn't buy'.

\ea\label{ex:ansmartins} Portuguese
\ea\label{ex:ansinitialsim}
 (\citealt[117: ex 56a-b]{Martins2013})\\ 
\gll A: Ele comprou o carro vermelho? \\
{} he buy.\textsc{3sg.prf.pst} the car red\\
\exi{}\gll B: Sim, comprou.\\
{} \textsc{aff} buy.\textsc{3sg.prf.pst}\\
\glt `A: Did he buy the red car? B: Yes, he did.' 
\ex\label{ex:ansfinalsim} (\citealt[117: ex 57a-b]{Martins2013})\\
\gll A: Ele não comprou o carro vermelho, (pois não)? \\
{} he not buy.\textsc{3sg.prf.pst} the car red \textsc{pois} not\\
\exi{}\gll B: Comprou sim.\\
{} buy.\textsc{3sg.prf.pst} \textsc{aff}\\
\glt `A: He didn't buy the car, did he? B: Yes, he \textsc{did}.'
\z
\z

Post-verbal \emph{sim} and Spanish/Catalan \emph{sí (que)} appear in overlapping contexts but I believe that  they are not the same. Generally, speakers use \isi{verum} for different purposes: It emphasizes the truth value of a proposition. As such, it can also be used to emphasize agreement with the hearer or to rebut a hearer's previous utterance. Before the seminal work of \citet{Hohle1992}, \citet{Watters1979} introduced the term  \emph{polar focus} to refer to  both of these functions. This type of \isi{focus} is therefore functionally equivalent to what I call \isi{verum} in this book. \citet{Watters1979}  furthermore introduces the term \emph{counter-assertive focus} to refer to the sub-function of \isi{verum} that serves to rebut a hearer's previous utterance.   Spanish and Catalan \emph{sí (que)}  can express both functions. This is illustrated with Spanish data in \eqref{ex:polarvscontra}. In \eqref{ex:polarfocus}, speaker A states that the interlocutors disagreed in some aspect and speaker B emphatically stresses the truth of this statement. In \eqref{ex:counterassertive} the sentence preceding the one headed by AffC states that what follows contradicts the interlocutor's belief.



\ea\label{ex:polarvscontra} Spanish
\ea\label{ex:polarfocus}  \gll A: Recuerdo perfectamente que discrepamos en este aspecto. B: Sí que discrepamos. \\
{} remember.\textsc{1sg.prs} perfectly that disagree.\textsc{1pl.prs} in this aspect {} \textsc{verum} \textsc{que} disagree.\textsc{1pl.prs}\\
\glt `A: I remember perfectly that we disagreed in this respect. B: We \textsc{did} in fact disagree.' (CdE)
\ex\label{ex:counterassertive}
\gll No estoy de acuerdo con tí. Sí que hay sistemas de pensiones que son sostenibles.\\
not be.\textsc{1sg.prs} in agreement with you \textsc{verum} \textsc{que} there.be.\textsc{3sg.prs} system\textsc{pl} of pension.\textsc{pl} that be.\textsc{3pl.prs} sustainable\\
\glt `I don't agree with you. There \textsc{are} systems of pensions that are sustainable.' (CdE)
\z 
\z

It is precisely in   counter-assertive contexts like \eqref{ex:counterassertive}, where a negative version of the finite verb contained in the proposition is salient, that both \isi{verum} and post-verbal \emph{sim} are predicted to be felicitous. In both cases the effect is that the salient alternative is contrasted and rejected. 

To conclude the discussion of the Portuguese data, I now turn to the properties of post-XP \emph{sim} in general. The sequence XP-\emph{sim} can occupy various positions in the syntactic structure, among them  a high,  potentially left periphery position, as in \eqref{ex:feliz}. In this position, it superficially  parallels the \isi{verum} structure in Spanish and Catalan.


\ea\label{ex:feliz}
Brazilian Portuguese\\
\gll Isto sim  é ser feliz. \\
 this  \textsc{sim} be.\textsc{3sg.prs} be happy\\
\glt `This is what being happy means.' (CdP)
\z


Even more curious are the cases where \emph{sim} is followed by a complementizer, giving rise to the sequence \emph{sim que}.
\ea\label{ex:errada} Brazilian Portuguese \\ \gll Isso sim que é uma idéia completamente errada.  \\
 that \textsc{sim} \textsc{que} be.\textsc{3sg.prs} an idea completely wrong\\
\glt `It is that idea that is completely wrong.' (CdP)
\z

This could lead to the assumption that these are  cases of AffC (\citealt{Kocher2019}). However, I believe that these examples are  more adequately captured as instances of post-XP-\emph{sim} structures. They do not give rise to a \isi{verum} interpretation of the sentence. As an anonymous reviewer suggested,   the meaning of cases like these is best paraphrased not as stress on the truth of the proposition but as a contrast of the XP with a salient alternative. This is very much in line with what I propose above for post-XP \emph{sim} in general.  Additionally, if it were a \isi{verum} structure, we would expect  a similar distribution as  in Spanish and Catalan. For one thing, we would expect to find \emph{sim que} at the beginning of a sentence. However, this pattern is virtually absent in the corpus (CdP), whereas there are numerous examples of the sentence-initial sequence of a nominal XP followed by \emph{sím que}.


Although a thorough analysis is pending, the sequence  \emph{sim que} in examples like \eqref{ex:errada} could be the result of an elision\is{ellipsis} process.\footnote{I am grateful to an anonymous reviewer who made this insightful proposal during the reviewing process of this book.}  \citet{Mioto2016} mention  that in Brazilian but not European Portuguese the copula in an inverted cleft can be dropped, as shown in \eqref{ex:mioto}.

\ea\label{ex:mioto}
\ea\label{ex:miotofull} European/Brazilian Portuguese (\citealt[287: ex 42c]{Mioto2016})\\  \gll
João é que pescou esse peixe.\\
João be.\textsc{3sg.prs} that fish.\textsc{3sg.prf.pst} this fish\\
\glt `It was João that fished this fish.' 
\ex\label{ex:miotodrop} Brazilian Portuguese (\citealt[287: ex 43a]{Mioto2016})\\  \gll
João  que pescou esse peixe.\\
João  that fish.\textsc{3sg.prf.pst} this fish\\
\glt `It was João that fished this fish.' 
\z 
\z 

An explanation for the \emph{sim que} examples  in \eqref{ex:errada}  is  then that they result from a similar case of copula drop.  What happens is that the XP is focused via an inverted cleft construction and additionally a contrast is marked with the particle \emph{sim}. This ``full'' structure is  illustrated in \eqref{ex:noticia}.  As expected, it is more frequent in European Portuguese  than in Brazilian Portuguese in the corpus data.

\ea\label{ex:noticia}
\ea European Portuguese \\ \gll 
Isso sim é que é uma grande notícia. \\
that \textsc{sim} be.\textsc{3sg.prs} that be.\textsc{3sg.prs} a big news\\
\glt `That is what big news is.' (CdP)
\ex Brazilian Portuguese\\
\gll 
 Isto sim é que é justiça.\\
this \textsc{sim} be.\textsc{3sg.prs} that be.\textsc{3sg.prs} justice\\
\glt  `This is what justice means.' (CdP)
\z 
\z 



Finally, the sequence \emph{sim que}  emerges  via the elision\is{ellipsis} of the copula. Although this pattern is in fact more frequent in Brazilian Portuguese \eqref{ex:simquebp}, there are still examples  in European Portugueses \eqref{ex:simqueep} in the corpus.

\ea
\ea\label{ex:simquebp} 
Brazilian Portuguese \\
\gll Isso sim \sout{é} que é capitalismo tardio.\\
that \textsc{sim} be.\textsc{3sg.prs} that be.\textsc{3sg.prs} capitalism late\\
\glt `That is what late capitalism is.'  (CdP)
\ex\label{ex:simqueep}
European Portuguese \\
\gll 
 Isso sim  \sout{é} que era trabalhar. \\
that \textsc{sim} be.\textsc{3sg.prs} that be.\textsc{3sg.ipfv.pst} work\\
\glt `That is what work was.' (CdP)
\z
\z 



To conclude the discussion of the syntactic properties of attributive \emph{que}, I briefly summarize the major points. In the preceding sections, I provided the empirical evidence in support of my unified analysis of the different constructions involving attributive \emph{que}. I showed that when the complementizer appears sentence-initially it is plausible to assume that it reaches the highest position in the left periphery SubP (see \sectref{sec:presupinitialC}, \sectref{sec:presuppqC}). In \is{wh-exclamative@\textit{wh}-exclamative}\textit{wh}-exclamatives, AdvC and AffC, the complementizer and the preceding material always observe strict adjacency, suggesting that the complementizer moves to the head of the respective projection.  I studied the  word order of attributive \emph{que} with respect to other left-peripheral material. These data corroborated the idea  that the movement of the complementizer is only inhibited by material merged in the left periphery but can cross moved phrases.  In \sectref{sec:presupprag}, I will present a more concise characterization of the discourse contribution of attributive \emph{que}. The main argument is that there is a shared   meaning present in all the different constructions. Different interpretive effects depend on the clause type and the meaning of the elements with which attributive \emph{que} co-occurs in each construction.
 
\section{The pragmatics of attributive \emph{que}}\label{sec:presupprag}



In this section I focus on the meaning of attributive \emph{que}. All the constructions are attributive in the sense of \citet{Poschmann2008}. This means that the \isi{commitment} expressed is attributed to someone  other than  the speaker. In the present case,  the  central function of attributive \emph{que} is to ascribe to the hearer a \isi{commitment} to the proposition in the scope of the complementizer. In other words, the proposition is presented as something that in the view of the speaker, the hearer should believe to be true (cf. also \citealt{Gras2016} for a similar idea).  This is illustrated by the contrast between the examples in \eqref{ex:falsebelief} and \eqref{ex:saying}.

\ea\label{ex:falsebelief}Spanish \judgewidth{\#}
\begin{xlist}
\exi{A:}[]{\gll Cual es la falsa idea que tienen los doctorandos al inicio de sus estudios?\\
what be.\textsc{3sg.prs} the false idea that have.\textsc{3pl.prs} the {PhD student.\textsc{pl}} at.the beginning of their study.\textsc{pl}\\}
\exi{B:}[]{\gll  Que seguramente acabarán su tesis a tiempo.\\
 that surely finish.\textsc{3pl.fut} their thesis on time\\}
\exi{B':}[\#]{\gll Que seguro que acabarán su tesis a tiempo.\\
				   that sure \textsc{que} finish.\textsc{3pl.fut} their thesis on time\\}
\end{xlist}
\glt `A: What is the false belief PhD students have at the beginning of their studies? B: That surely (\#que) they will  finish their thesis on time.'
\z

In \eqref{ex:falsebelief},  version B' of the answer involving AdvC is not acceptable while  version B, with the same epistemic modifier\is{epistemic and evidential modifier} but no attributive \emph{que}, is perfectly acceptable.  Since  attributive \emph{que} marks the proposition as something that is considered to be true by the hearer, it is incompatible  in the context of this example in which the proposition is declared as a false belief by interlocutor A. Therefore, the belief of \emph{p}  cannot be attributed to interlocutor A, the hearer because he has just stated that he does not consider \emph{p}  to be true.  Version B of the answer is fine because the epistemic modifier\is{epistemic and evidential modifier}  is not attributive in itself.

\ea\label{ex:saying}Spanish 
\begin{xlist}
\exi{A:}{
\gll Qué dicen los doctorandos al inicio de sus estudios?\\
	 what say.\textsc{3pl.prs} the {PhD students} at.the beginning of their study.\textsc{pl}\\}
\exi{B:}{\gll
Que seguramente acabarán su tesis a tiempo.\\
that surely finish.\textsc{3pl.fut} their thesis on time\\}
\exi{B':}{\gll Que seguro que acabarán su tesis a tiempo.\\
That sure \textsc{que} finish.\textsc{3pl.fut} their thesis on time\\}
\end{xlist}
\glt `A: What do PhD students say at the beginning of their studies? B: That surely they will finish their thesis on time.'
\z

In the context presented in \eqref{ex:saying}, both versions of the answer are acceptable because speaker A inquires about what PhD students say at the beginning of their studies, but does not declare or retract his \isi{commitment} to the truth of the proposition. 

In what follows, I show that this function of attributing the belief of the unmodified sentence, called the prejacent, to the hearer is consistent in all the constructions containing attributive \emph{que}. There  are different nuanced effects in the various constructions which, I argue, result from the clause type and the function of the other elements involved in them.  

Before going into the detail of my account, I will briefly mention that  similar observations have been made in the literature.  \citet{Bianchi2017}, for instance, note that the propositional content of certain embedded clauses, for instance the complements of \isi{factive} verbs, is imposed on the \isi{common ground}. They use  the term  \emph{informative presupposition}, going back to \citet{Prince1978}.   

Another concept that might seem relevant to the phenomena being discussed here is  presupposition accommodation (\citealt{Beaver2001},
\citealt{Simons2003}, \citealt{Beaver2007},  \citealt{Fintel2008} and references therein). However, propositions that are added to the \isi{common ground} by way of accommodation are usually not at issue in contrast to attributive  \emph{que}-initial propositions (cf. \ref{ex:atrasado}).  

\citet{AnderBois2010, AnderBois2015} offer an analysis that seeks to  capture the contrast between referential and appositive relative clauses. Similar to the observation by \citet{Bianchi2017}, they propose that the propositional content of appositives is imposed on the \isi{common ground}. \citet{Lohiniva2017} links this function to the complementizer in the French constructions involving subordinating \emph{bien que} `although' and coins the term \emph{impositive complementizer}.   A core aspect of  the analysis by \citet{AnderBois2015} is that these propositions  are  not at issue. Among other things, one consequence of this information status is that the proposition imposed on the \isi{common ground} cannot be denied by the hearer. This explains the  oddness of \eqref{ex:lungcancer} as a reaction to \eqref{ex:prostatecancer} because it directly denies the content of the appositive clause.
   
 
\ea\judgewidth{??}
\ea[]{\label{ex:prostatecancer}His husband, who had prostate cancer, was being treated at the Dominican Hospital. (\citealt[116: ex 53a]{AnderBois2015})}
\ex[??]{\label{ex:lungcancer}No, he had lung cancer. (\citealt[116: ex 53b]{AnderBois2015})}
\ex[]{No, he was being treated at the Standford Hospital.\\ (\citealt[116: ex 53c]{AnderBois2015})}
\z
\z


The same is not true for  propositions headed by attributive \emph{que}. They  are at issue and thus can be denied. Therefore the reaction of B,  rejecting the AdvC-proposition introduced by A, is perfectly acceptable. 
 
\ea\label{ex:atrasado}Brazilian Portuguese
\begin{xlist}
\exi{A:}{
\gll Seguro que o Pedro chegou à hora. \\
     sure \textsc{que} the Pedro arrive.\textsc{3sg.prf.pst} on.the time\\}
\exi{B:}{
  \gll Não, chegou atrasado.\\
       no  arrive.\textsc{3sg.prf.pst} late\\}
\end{xlist}
  \glt `A: Surely Peter arrived on time. B: No, he arrived late.'
 \z
 


At present, this crucial contrast between presupposition accommodation or imposition such as in  the phenomena explored by \citet{AnderBois2015} and the phenomena under investigation here prevents me from systematically relating  this analysis to my own proposal. Future research will show whether and how these  approaches can be reconciled. 




As stated above, the shared meaning that I propose for all attributive \emph{que} constructions  is that the speaker attributes a \isi{commitment} to the proposition to the hearer.  In many cases, in an  attributive \emph{que} sentence the speaker revisits a proposition that has been asserted previously. An example of this type  is given in \eqref{ex:aposta}. Speaker E asserts the same  content that is then taken up again by speaker M in the proposition introduced by AdvC. It is used as a tool to emphasize that the \isi{commitment} to the proposition is shared by both interlocutors.\pagebreak


\ea\label{ex:aposta}
European Portuguese\\ 
\gll  E. - Isso é uma aposta e o governo pode sair perdendo.  M. - Claro que é uma aposta. \\
{} {} this be.\textsc{3sg.prs} a gamble and the government can.\textsc{3sg.prs} leave losing {} {} clearly \textsc{que} be.\textsc{3sg.prs} a gamble\\
\glt `E. - This is a gamble and the government could end up losing. M. Clearly, this is indeed a gamble.' (CdP)
\z 

Attributive \textit{que} is however not limited to propositions to which  the hearer has expressed a previous \isi{commitment}. It can also be a way for the speaker to persuade the hearer to accommodate a proposition. Attributive \emph{que} can thus be used to speed up the conversation or  to present    propositions with potentially controversial content as if they were uncontroversial. The speaker, thereby, anticipates the hearer's \isi{commitment} and leaves him less room to debate or reject it. This is what happens in \eqref{ex:valores}, where  AffC introduces a proposition that contains a potentially controversial opinion the speaker holds about the hearer.
\ea\label{ex:valores}
Catalan\\
\gll No em valores prou. No em fas cas. A ell sí que n' hi faries.   \\
not \textsc{cl.1sg} appreciate.\textsc{2sg.prs} enough not \textsc{cl.1sg} make.\textsc{2sg.prs} case to he  \textsc{verum} \textsc{que} \textsc{cl.part} \textsc{cl.3sg} make.\textsc{2.sg.cond} \\
\glt `You don't appreciate me enough. You don't pay heed to me. To him, you  \textsc{do} pay heed.'
(ebook-cat) 
\z


In other contexts,  attributive \emph{que} is used despite the fact that the hearer has explicitly expressed an opposing \isi{commitment}. The effect is that the speaker  communicates that she was under the impression that the proposition was part of the \isi{common ground}, but the hearer's recent linguistic or extra-linguistic behavior  contradicts the idea that the hearer was aware of  the proposition. In these contexts, attributive \emph{que} can serve to  express the speaker's \isi{surprise} regarding the hearer's ignorance or rejection of the proposition, as  the speaker considered it to be obviously true.

A context in which the attributive \emph{que}-initial sentence rebuts an opposing hearer-\isi{commitment} is exemplified in \eqref{ex:pies}. In this case, the hearer has explicitly stated that he is committed to the fact that there are six feet. María's use of attributive \emph{que} twice in her last enunciation expresses her \isi{surprise} about the hearer's incorrect belief and she emphatically insists on the truth that there are actually four feet and that,  in her opinion, the hearer should have been aware of this fact.

\ea\label{ex:pies}
		Spanish\\
\gll “Oye, María. Yo creo que aquí hay seis pies.” “¿Pero qué dices?” “Yo estoy convencido de que aquí hay seis pies.” “¡Que no, hombre, que sólo hay cuatro!” \\
listen María I believe.\textsc{1sg.prs} that here be.\textsc{3pl.prs} six foot.\textsc{pl} but what say.\textsc{2sg.prs} I be.\textsc{1sg.prs} convinced of that here be.\textsc{3pl.prs} six foot.\textsc{pl} \textsc{que} no man \textsc{que} only be.\textsc{3pl.prs} four.\\
\glt `“Listen, María. I think that here are six feet.” “But what are you saying?” “I am convinced that here are six feet.” “Absolutely not, man! There are only four!”' (CdE)
\z



For the characterization of the pragmatic effect that the presence of attributive \emph{que} has in the different constructions, I adopt the implementation  presented in \citet{Malamud2015} of a  conversational scoreboard in the style of \citet{Lewis1979}. Their version constitutes a modification of the model proposed in  \citet{Farkas2010}  that builds on \citet{Hamblin1971}, \citet{Gunlogson2003}, \citet{Ginzburg2012} and others, and is further developed in \citet{Roelofsen2015}. Scoreboards permit a dynamic  modeling of conversations.  In these models,  the speaker keeps track of information states. When making a conversational move, like asserting a proposition, certain aspects of the informational states on the scoreboard change. In the original version in \citet{Farkas2010}, the conversational states consist of  discourse \isi{commitment}s (DC$_X$) for each conversational participant X, which are sets of propositions that participant X is committed to. The concept of a \isi{common ground} (CG\is{common ground}) is adopted from \citet{Stalnaker1978} and refers to sets of propositions that the participants share a \isi{commitment} to. The CG\is{common ground} can therefore be viewed as the intersection of the DC of all the contextually relevant participants. There is furthermore a table, which is a similar concept to \emph{question under discussion} (\citealt{Ginzburg1996}, \citealt{Roberts1996}, \citealt{Engdahl2006}). It contains an ordered stack of propositions or issues to be resolved in the conversation. There is also a projected CG\is{common ground}, which is a set of potential future CG\is{common ground}s given the possible resolutions of the top issue on the table. 
The central modification of \citet{Malamud2015} is that  there are projected versions not only of the CG but also of other parts of the scoreboard. Particularly relevant to the  phenomenon under discussion   are the projected DCs for each participant. This  allows the speaker to give tentative \isi{commitment}s by adding a proposition to the projected rather than the current DC. The projected DC of the speaker, according to \citet{Malamud2015}, coincides roughly with the concept of contingent \isi{commitment} in \citet{Gunlogson2008}. In \citeauthor{Gunlogson2008}'s system, however, there is no correspondence with the projected hearer \isi{commitment}. In \citet{Malamud2015}, there  is also a projected table, which allows an issue to be tentatively raised for resolution in future moves.
 
 
To exemplify how these models work, the conversational scoreboard of an assertion is represented in Table \ref{tab:scoreboardassertion}.  In the system  proposed by \citet{Malamud2015}, when  a proposition is asserted, it is added, along with its presuppositions,  to the top of the stack on the table, the speaker's DC and the projected CG\is{common ground}. What this means is that when asserting a proposition, the speaker raises it as an issue she seeks to resolve, signals her \isi{commitment} to its truth and proposes to add it to the CG\is{common ground}. 

\begin{table}
\begin{tabular}{l l  l l}
\lsptoprule
\multicolumn{2}{c}{current}  & \multicolumn{2}{c}{projected}\\\cmidrule(lr){1-2}\cmidrule(lr){3-4}
	CG\is{common ground}{} & $\{\,\}$ & CG\is{common ground}* & $\{\{p\}\}$\\
	DC\textsubscript{Speaker} & $\{p\}$ & DC*\textsubscript{Speaker} & $\{\{\,\}\}$ \\
	DC\textsubscript{Hearer} & $\{\,\}$  & DC*\textsubscript{Hearer} & $\{\{\,\}\}$\\
	Table & $\langle p\rangle$ & Table* & $\langle\langle\,\rangle\rangle$ \\\lspbottomrule
\end{tabular}
\caption{Conversational scoreboard when asserting {p}}\label{tab:scoreboardassertion}
\end{table}


With these general ideas in place, I now turn to the constructions under investigation and show how the discourse contribution of attributive \emph{que} can be modeled in a conversational scoreboard à la \citet{Malamud2015}. 
As stated at the beginning of this section, the function of \emph{que} is attributive in the sense of \citet{Poschmann2008}, meaning that  a  \isi{commitment} to the proposition is ascribed to the hearer. Translating this into the scoreboard model, I propose that in all the different constructions containing attributive \emph{que}, the proposition \emph{p} is added to the hearer's  DC. The contribution of \emph{que} is thus the same in all the constructions.  What is different in \isi{polar question}s  is that the proposition  is added to the projected DC of the hearer and not the current one as is the case in declaratives.

\begin{sloppypar}
Table \ref{tab:scoreboardqueassertion} illustrates the basic case of  attributive \emph{que}-initial declaratives. They have all the  same conversational states as normal assertions. Therefore, the proposition is added to the speaker's DC, to the table and to the projected CG\is{common ground}. The presence of attributive   \emph{que} means that the proposition is furthermore added to the hearer's DC. The resulting effect is that the speaker conveys that, in her view,  the \isi{commitment} to the proposition is  shared by the hearer.
\end{sloppypar}

\begin{table}
	\begin{tabular}{l l l l}
	\lsptoprule
		\multicolumn{2}{c}{current}  & \multicolumn{2}{c}{projected}\\\cmidrule(lr){1-2}\cmidrule(lr){3-4}
	CG\is{common ground}{} &$\{\,\}$ & CG\is{common ground}* &$\{\{p\}\}$\\
	DC\textsubscript{Speaker}& $\{p\}$ & DC*\textsubscript{Speaker}& $\{\{\,\}\}$ \\
	DC\textsubscript{Hearer}& $\{p\}$  & DC*\textsubscript{Hearer} &$\{\{\,\}\}$\\
	Table& $\langle p\rangle$ & Table* & $\langle\langle\,\rangle\rangle$ \\
	\lspbottomrule
\end{tabular}
\caption{Conversational scoreboard when asserting attributive \emph{Que p}}\label{tab:scoreboardqueassertion}
\end{table}



Attributive \emph{que}-initial declaratives often appear juxtaposed with directive sentences.  In these contexts,  the sentence headed by \textit{que} is sometimes interpreted as the reason or explanation for the  previous assertion or command. This accounts for the assumption made by some authors in the literature that the function of \emph{que} in these contexts is to express a causal\is{causal connective} relation (cf. \citealt{Alarcos1994}, \citealt{Ballesteros2000}, \citealt{Peres2006}, \citealt{Etxepare2013},  \citealt{Wheeler1999}, \citealt{Cunha1984},  \citealt{Lobo2003}, \citealt{Lopes2012}, \citealt{Colaco2016}). However, I refrain from relating this causal\is{causal connective} meaning to the complementizer itself because a causal\is{causal connective}ity interpretation is available even in the absence of \emph{que}. Even without a special marker, the  interpretation that two juxtaposed sentences are causal\is{causal connective}ly related is available.

\ea 
\ea\label{ex:causallyque} 
Catalan\\
\gll No menges pipas. (Que) fa castellà. \\
	not eat.\textsc{2sg.prs} {sunflower seed.\textsc{pl}} \textsc{que} make.\textsc{3sg.prs} Castillian\\
	\glt `Don't eat sunflower seeds. That makes you (look) Spanish.' (ebook-cat)

	\ex\label{ex:quehoje}
	Portuguese\\ 
	\gll  Keep calm. (Que) hoje é sexta-feira. \\
	keep calm \textsc{que} today be.\textsc{3sg.prs} Friday\\
	\glt `Keep calm. Today is Friday.'  
\z
\z 

In \eqref{ex:causallyque} the second sentence, irrespective of whether \emph{que} is present or not, can be interpreted as an explanation for why you should not eat sunflower seeds. Similarly, in \eqref{ex:quehoje}, both versions of the second sentence  could be interpreted as a reason why you should keep calm.  
Therefore, the causal function is independent of attributive \emph{que}. If attributive \emph{que} is present in an example with a causal interpretation, nothing changes about \emph{que}'s general function: The proposition introduced by the complementizer is  presented as uncontroversial information that, according to the speaker,  both speaker and hearer are committed to. 

There are cases  where the \emph{que}-initial declarative follows a directive, in which a causal\is{causal connective} interpretation is not  consistent. One such example is given in \eqref{ex:salid}. 

\ea Spanish\\
\label{ex:salid}\gll  Salid. Que no os mataré. \\
	leave.\textsc{imp.2pl} \textsc{que} not \textsc{cl.2pl} kill.\textsc{1sg.fut}\\
	\glt `Come out. I won't kill you.'
\z


In this case, the causal\is{causal connective} interpretation seems to be lost. A direct causal\is{causal connective}  relation between following the order by coming out of a hiding place and not getting killed cannot be constructed.  It can also hardly be understood as a causal\is{causal connective} relation on a higher level,  where not killing the hearers would be  the reason for saying \emph{Salid}.\footnote{An explicative interpretation, however,  whereby not killing the hearers could be the explanation for why they should come out, is  possible.} The motivation for the second sentence is  to reassure the hearers and make them believe that there is no imminent threat to their lives. By adding  attributive \emph{que} at the beginning of the  sentence, the speaker  anticipates the hearers' agreement with it, which makes it sound more  persuasive and less threatening.



My proposal for AdvC-declaratives is that they involve the same conversational states as attributive \emph{que}-initial declaratives, illustrated in Table \ref{tab:scoreboardclaroque}. The only difference is that the speaker's DC contains the modified rather than the bare proposition. In turn, the unmodified proposition is  added to the hearer's DC, the table and the projected CG\is{common ground}. The reason for this discrepancy is that the speaker's \isi{commitment} to the prejacent depends on the modifier. Different \isi{epistemic and evidential modifier}s imply different degrees of \isi{commitment} towards  the truth of the proposition. Weak epistemic modals like \emph{probably} allow the speaker to retract her \isi{commitment} to the proposition that it modifies. Similarly, certain types of evidential modifier\is{epistemic and evidential modifier}s like \emph{clearly} imply that the speaker is committed to the truth of the proposition while others like \emph{allegedly} do not.

\begin{table}
	\begin{tabular}{l l  l l}
	\lsptoprule
		\multicolumn{2}{c}{current}  & \multicolumn{2}{c}{projected}\\\cmidrule(lr){1-2}\cmidrule(lr){3-4}
		CG\is{common ground}{}& $\{\,\}$ & CG\is{common ground}*& $\{\{p\}\}$\\
		DC\textsubscript{Speaker} &$\{claro(p)\}$ & DC*\textsubscript{Speaker}& $\{\{\,\}\}$ \\
		DC\textsubscript{Hearer}& $\{p\}$  & DC*\textsubscript{Hearer}& $\{\{\,\}\}$\\
		Table &$\langle p\rangle$ & Table*&  $\langle\langle\,\rangle\rangle$ \\\lspbottomrule
	\end{tabular}
	\caption{Conversational scoreboard when uttering \emph{Claro que p}}\label{tab:scoreboardclaroque}
\end{table}

To understand the effect of attributive \emph{que} in combination with evidential and epistemic modifier\is{epistemic and evidential modifier}s, it is useful to contrast AdvC with similar constructions that involve the same epistemic and evidential  modifiers but no attributive complementizer (see \citealt{Kocher2018} and \sectref{sec:expdeic} for experimental and corpus studies focusing on the three constructions; see also \citealt{Cruschina2018} for a similar contrastive characterization of the three constructions).  The examples in \eqref{ex:epevconst} illustrate the  constructions for comparison here.

\ea \label{ex:epevconst}
\ea\label{ex:advc} Spanish\\  \gll Obviamente/ Ciertamente {que} Pedro viene a la fiesta. \\
		obviously/ certainly  \textsc{que} Pedro come.\textsc{3sg.prs} to the party\\
		\glt `Obviously/Certainly, Pedro  will in fact come to the party.'
		\ex\label{ex:adv}
			Catalan\\
		\gll  Obviament/ Certament en Pere ve a la festa. \\
		obviously/ certainly the Pere come.\textsc{3sg.prs} to the party\\
		\glt `Pere will {obviously/certainly} come to the party.'
		\ex  \label{ex:esadjc}   Portuguese \\ \gll É óbvio/ certo que o Pedro vem à festa. \\
		is obvious/ certain that the Pedro come.\textsc{3sg.prs} to.the party\\
		\glt `It is {obvious/certain} that  Pedro will come to the party.'	
	\z
\z


In \eqref{ex:advc}, the modifier appears in AdvC, containing attributive \emph{que}. In \eqref{ex:adv} it functions as a simple adverb and in \eqref{ex:esadjc}, which I call EsAdjC, the modifier appears as a predicative adjective in a copula construction.

The contrast that I consider central for my argument is the different readings of the modifiers that are rendered prominent. \Citet{Fintel2011} introduce the term \emph{bare epistemic modals} to refer to epistemic modals that permit  different readings depending on their modal base. The modal bases are built  on the information states  of different speech participants and are grounded in  knowledge, beliefs or evidence. The key point here is that with bare epistemic modals the  evaluation expressed by the modal can be anchored to different \isi{deictic center}s. The  Ibero-Romance epistemic modifier\is{epistemic and evidential modifier}s involved in the three constructions can be classified as bare epistemic modals. This is illustrated for the cognate of \emph{probably} in \eqref{ex:peterprobably}. A statement like \emph{Peter will probably come} can be interpreted  from the perspective of different speech participants and thus  anchored to different \isi{deictic center}s.\footnote{A similar concept is ``(inter)subjectivity'' (\citealt{Nuyts2001}, \citealt{Traugott2010}, going back to \citealt{Lyons1977}), which is also  used to refer to the fact that certain constructions or expressions are anchored to different speech participants. See also \citet{Kocher2018} for an application of this concept to Spanish and Portuguese modifiers.}

\ea \label{ex:peterprobably}
\ea (\textit{deictic center=speaker})\\
    (In the view of what I know) Peter will probably come.
\ex (\textit{deictic center=hearer})\\
	(In the view of what you know) Peter will {probably} come.
\ex\label{ex:probyouandI}(\textit{deictic center=speaker and hearer})\\
	(In the view of what you and I know) Peter will {probably} come.
\ex\label{ex:probgeneral}(\textit{deictic center=a (contextually relevant) group or authority})\\
	(In the view of what is generally known) Peter will {probably} come.
\z
\z

The idea maintained in \citet{Fintel2011} is that bare epistemic modals are ambiguous by design. This means that when uttering \textit{probable}(\textit{p}), the speaker puts into play all contextually relevant readings of \textit{probable}(\textit{p}) that can then be either taken up or rejected. This is illustrated in example \eqref{ex:alex}, where Alex uses the bare epistemic modal \emph{might}, and the two acceptable continuations that follow Alex's statement pick up different readings of the modal. In the reaction in \eqref{ex:billy1}, Billy takes up the reading of the modal anchored to the speaker Alex, while in the reaction in \eqref{ex:billy2}, he takes up and rejects a reading anchored to a perspective that involves Billy, the hearer, i.e. himself. 


\ea\label{ex:alex}  Alex is aiding Billy in the search for her keys:\\
Alex: You might have left them in the car. 
\ea\label{ex:billy1} Billy: You’re right. Let me check. 
	\ex\label{ex:billy2} Billy: No, I still had them when we came into the house. \\ (\citealt[114--115: ex 12--14]{Fintel2011})
\z
\z


While \citet{Fintel2011} are only concerned with different readings of epistemic modals, there is reason to believe that evidential evaluation can similarly be tied to different speech participants.\footnote{ I make no claims here about the relation between epistemic modality and evidentiality, but see  \citet{Auwera1998}, \citet{Haan1999}, \citet{Aikhenvald2004}, \citet{Palmer2001},  \citet{Rooryck2001}, \citet{Faller2006}, \citet{Diewald2010}, \citet{Fintel2010}, \citet{Boye2012},  \citet{Matthewson2015}  for some prominent ideas, and see \citet{Cornillie2009}  and \citet{Kocher2014} for an overview of the different positions.} I therefore also propose that  when the speaker utters \textit{obvious}(\textit{p}) all the contextually relevant readings of \textit{obvious}(\textit{p}) are put into play (cf. \ref{ex:peterobviously}).\largerpage

\ea\label{ex:peterobviously}
\ea(In the view of what I know) Peter will {obviously} come. \\
 (\textit{deictic center=speaker})
	\ex (In the view of what you know) Peter will {obviously} come. \\ (\textit{deictic center=hearer})
	\ex\label{ex:obviyouandI} (In the view of what you and I know) Peter will {obviously} come.  \\ (\textit{deictic center=speaker and hearer})
	\ex \label{ex:obvigeneral}(In the view of what is generally known) Peter will {obviously} come. \\(\textit{deictic center=a (contextually relevant) group or authority})
\z
\z



As stated above, I consider  the \isi{epistemic and evidential modifier}s involved in the relevant constructions to be bare in the sense of \citet{Fintel2011}, in that they can be anchored to different \isi{deictic center}s. The ambiguity is retained when they appear as adverbs as in \eqref{ex:adv}, but specific readings are made prominent when they appear in either AdvC or EsAdjC.    My proposal is that the prominent reading of the modifiers in AdvC can be linked to the discourse contribution made by attributive \emph{que}. Since the proposition is presented as something the hearer is committed to,  by way of an implicature the interpretation arises that he also shares the epistemic or evidential evaluation. This results in the reading of the modifier that is characterized in \eqref{ex:probyouandI} and \eqref{ex:obviyouandI}.
 In the simple adverbial construction, the \isi{commitment} to \emph{p}  is not attributed to the hearer nor to any participant other than the speaker and  therefore the  implicature does not arise automatically. The adverbs do not have a prominent reading, meaning that different readings are available depending on the context.
  Finally, EsAdjC is also attributive. However, the \isi{commitment} to the proposition is not attributed to the hearer, but to a general group of people or an authority. In this case, again, an implicature arises  resulting in  the evaluation being presented as shared by the speaker and a general group of people, i.e. the reading of the modifier exemplified in \eqref{ex:probgeneral} and \eqref{ex:obvigeneral}. It is important to stress that the readings are not set in stone. In fact, even if a construction makes a specific reading  prominent, speakers are able to accommodate other readings. This is strongly suggested by the results of the experiment in \sectref{sec:expdeic}, which indicate that the prominent reading of the modifiers can be overridden if contextual factors make other readings available. 
 
 
 
To conclude this section on AdvC, I will briefly discuss the  conversational scoreboards that I assume for pure adverbial modification and for EsAdjC and point out how they contrast with the conversational scoreboard assumed for AdvC.
 Table \ref{tab:scoreboardclaramente} shows the conversational scoreboard  when uttering a proposition modified by an evidential adverb. It parallels the scoreboard assumed for AdvC. The speaker is committed to the modified proposition and the prejacent, i.e. the unmodified proposition, is put on the table and added to the projected CG\is{common ground}. In the absence of attributive \emph{que},  the attributive addition of the proposition to the hearer's DC is not part of the scoreboard. 


\begin{table}
	\begin{tabular}{l l  l l}
		\lsptoprule
		\multicolumn{2}{c}{current}  & \multicolumn{2}{c}{projected}\\\cmidrule(lr){1-2}\cmidrule(lr){3-4}
		CG\is{common ground}{}& $\{\,\}$ & CG\is{common ground}*& $\{\{p\}\}$\\
		DC\textsubscript{Speaker} &$\{claro(p)\}$ & DC*\textsubscript{Speaker}& $\{\{\,\}\}$ \\
		DC\textsubscript{Hearer}& $\{\,\}$  & DC*\textsubscript{Hearer}& $\{\{\,\}\}$\\
		Table &$\langle p\rangle$ & Table*&  $\langle\langle\,\rangle\rangle$ \\\lspbottomrule
	\end{tabular}
	\caption{Conversational scoreboard when uttering \emph{Claramente p}}\label{tab:scoreboardclaramente}
\end{table}

Table \ref{tab:scoreboardesclaro} shows the conversational scoreboard that I assume for EsAdjC, the equivalent biclausal copula construction in which a proposition is modified by a predicative adjective. I propose that these are also attributive constructions, with the \isi{commitment}  not ascribed to the hearer but to a general group of people or an authority. The states involved are much the same as in AdvC, illustrated in Table \ref{tab:scoreboardclaroque}, except that the proposition is not added to the hearer's DC. Instead, I propose that the speaker also keeps track of a set of general discourse \isi{commitment}s (DC\textsubscript{General}), to which the proposition is added in these constructions.  


\begin{table}
	\begin{tabular}{l l l l}
		\lsptoprule
		\multicolumn{2}{c}{current}  & \multicolumn{2}{c}{projected}\\\cmidrule(lr){1-2}\cmidrule(lr){3-4}
		CG\is{common ground}{}& $\{\,\}$ & CG\is{common ground}*& $\{\{p\}\}$\\
		DC\textsubscript{Speaker} &$\{claro(p)\}$ & DC*\textsubscript{Speaker}& $\{\{\,\}\}$ \\
		DC\textsubscript{Hearer}& $\{\,\}$  & DC*\textsubscript{Hearer}& $\{\{\,\}\}$\\
		DC\textsubscript{General}& $\{p\}$  & DC*\textsubscript{General}& $\{\{\,\}\}$\\
		Table &$\langle p\rangle$ & Table*&  $\langle\langle\,\rangle\rangle$ \\\lspbottomrule
	\end{tabular}
	\caption{Conversational scoreboard when uttering \emph{Está claro p}}\label{tab:scoreboardesclaro}
\end{table}



In what follows I characterize the discourse effects of attributive \emph{que} in \is{wh-exclamative@\textit{wh}-exclamative}\textit{wh}-exclamatives and \isi{verum} sentences. Before going into detail,  I should stress that the conversational scoreboards that I will shortly present might  be subject to further refinement  in the future. This is because the information states involved in \isi{verum} sentences and exclamatives are more complex  than in attributive \emph{que}-initial declaratives, \isi{polar question}s and  AdvC. Further research is certainly required to be able to offer  all encompassing analyses.


Exclamatives are said to be \isi{factive} (cf. \citealt{Grimshaw1979}, \citealt{Zanuttini2003}, \citealt{Castroviejo2006}). This means that the information they contain is presupposed to be true. They  furthermore cannot be used as answers to a question (cf. \ref{ex:paualt1}) but they can be confirmed (cf. \ref{ex:paualt2}). 

\ea  Catalan
\ea\label{ex:paualt1}
 \gll A: Com és d'alt en Pau? B1: Molt alt. B2: Fa 1.90m. B3: \#Que alt que és! \\
{} how be.\textsc{3sg.prs} {of tall} the Pau {} very tall {} makes 1.90m {} how tall \textsc{que} be.\textsc{3sg.prs}\\
\glt `A: How tall is Pau? B1: Very tall. B2: He's 1.90m tall. B3: \#How tall he is!.'
\ex\label{ex:paualt2} \gll A: Que alt que és en Pau! B1: \#Fa 1.90m. B2: I tant!\\
{} how tall \textsc{que} be.\textsc{3sg.prs} the Pau {} makes 1.90m {} and so\\
\glt `A: How tall Pau is! B1: \#He's 1.90m tall. B2: Indeed!'  
\z
\z

According to \citet{Gunlogson2003} and \citet{Castroviejo2006}, the speaker commits herself to the descriptive content of the sentence, but does not assert it. In \citet{Castroviejo2006} it is stated that exclamatives  denote the fact that an individual has the property expressed by the \textit{wh}-expression to a high degree. The speaker furthermore expresses her attitude towards the degree and, according to \citet{Castroviejo2006}, it is precisely this information that is used to update the CG\is{common ground}. The  scoreboard I propose for a \emph{que}-less \is{wh-exclamative@\textit{wh}-exclamative}\textit{wh}-exclamative like \eqref{ex:chatobrep} repeated from \eqref{ex:chatob} is given in Table \ref{tab:scoreboardexcl}.



\ea \label{ex:chatorep} Portuguese
\ea\label{ex:chatoarep}
 \gll Que chato que és. \\
		how annoying \textsc{que} be.\textsc{2sg.prs}\\
		\ex\label{ex:chatobrep}\gll Que chato és. \\
		how annoying be.\textsc{2sg.prs}\\
		\glt `How annoying you are.' 
	\z
\z

\begin{table}
	\begin{tabular}{l l  l l}
		\lsptoprule
		\multicolumn{2}{c}{current}  & \multicolumn{2}{c}{projected}\\\cmidrule(lr){1-2}\cmidrule(lr){3-4}
		CG\is{common ground}{} &$\{p\}$ & CG\is{common ground}*& $\{\{\,\}\}$\\
		DC\textsubscript{Speaker}& $\{Excl(p)\}$ & DC*\textsubscript{Speaker} &$\{\{\,\}\}$ \\
		DC\textsubscript{Hearer}& $\{\,\}$  & DC*\textsubscript{Hearer} &$\{\{\,\}\}$\\
		Table& $\langle\,\rangle$ & Table*&  $\langle\langle\,\rangle\rangle$ \\
		\lspbottomrule
	\end{tabular}
	\caption{Conversational scoreboard when uttering \emph{Qué x  p}}\label{tab:scoreboardexcl}
\end{table}


 The proposition and the exclamative import are added to the speaker's DC.\footnote{The semantic derivation of \is{wh-exclamative@\textit{wh}-exclamative}\textit{wh}-exclamatives is not a concern here, but cf. \citet{Zanuttini2003},  \citet{Castroviejo2006}, \citet{Villalba2008}, \citet{Gutierrez-Rexach2011, Gutierrez-Rexach2016} and references therein.} As stated above, exclamatives do not assert the prejacent, i.e. the underlying proposition. In the  model, I represent this by not putting the proposition  on the table nor into the projected CG\is{common ground}. Exclamatives are \isi{factive} and hence presuppose the truth of the prejacent.  I propose to model this by imposing the proposition on the current CG\is{common ground}.

\begin{sloppypar}
The main argument of this section is that  the discourse contribution of the attributive complementizer is the same in all the constructions it involves. Therefore, in \is{wh-exclamative@\textit{wh}-exclamative}\textit{wh}-exclamatives with \emph{que} like \eqref{ex:chatoarep} repeated from \eqref{ex:chatoa}, the consequence of the presence of \emph{que} is again  that the proposition is  added to the hearer's DC. Apart from that, the scoreboard illustrated in Table \ref{tab:scoreboardexclque} is the same as the one that I assume for \emph{que}-less exclamatives.
\end{sloppypar}

\begin{table}
	\begin{tabular}{l l  l l}
	\lsptoprule
		\multicolumn{2}{c}{current}  & \multicolumn{2}{c}{projected}\\\cmidrule(lr){1-2}\cmidrule(lr){3-4}
		CG\is{common ground}{} &$\{p\}$ & CG\is{common ground}*& $\{\{\,\}\}$\\
		DC\textsubscript{Speaker}& $\{Excl(p)\}$ & DC*\textsubscript{Speaker} &$\{\{\,\}\}$ \\
		DC\textsubscript{Hearer}& $\{p\}$  & DC*\textsubscript{Hearer} &$\{\{\,\}\}$\\
		Table& $\langle\,\rangle$ & Table*&  $\langle\langle\,\rangle\rangle$ \\
		\lspbottomrule
	\end{tabular}
	\caption{Conversational scoreboard when uttering \emph{Qué x que p}}\label{tab:scoreboardexclque}
\end{table}


The CG\is{common ground} is by definition the set of all discourse \isi{commitment}s shared by the interlocutors. Consequently,  attributive \emph{que} imposes something that is already part of the meaning of a \is{wh-exclamative@\textit{wh}-exclamative}\textit{wh}-exclamative even in the absence of \emph{que}. This might explain why many authors consider  \emph{que} to be optional and to have no effect on the interpretation of exclamatives (see for instance \citealt{Villalba2003}, \citealt{Castroviejo2006}).  In those \is{wh-exclamative@\textit{wh}-exclamative}\textit{wh}-exclamatives that permit versions without the complementizer, what we  expect is  that the version with \emph{que} should place more emphasis on the hearer's \isi{commitment}. 

Verum\is{verum} sentences are similar in certain respects. Unlike exclamatives, they are asserted and can function as answers to questions (cf. \ref{ex:sipau}).

\ea\label{ex:sipau}
Catalan\\
\gll A: És alt en Pau? B: Sí (que) ho és! \\
{} be.\textsc{3sg.prs} tall the Pau {} \textsc{verum} {} \textsc{cl.n} be.\textsc{3sg.prs}\\
\glt `A: Is Pau tall? B: He \textsc{is} (indeed)!'
\z

 In the  model,  this means that the speaker does put the proposition on the table. However, like exclamatives, verum sentences are \isi{factive} and thus presuppose the truth of the prejacent. In my  proposal this is modeled as an imposition of the proposition on the CG\is{common ground}.
 
\ea Spanish
\ea \label{ex:pablo1}
		\gll
Pablo sí que es alto. \\
Pablo \textsc{verum} \textsc{que} be.\textsc{3sg.prs} tall\\
 \ex\label{ex:pablo2}

 \gll Pablo sí es alto. \\
 Pablo \textsc{verum}  be.\textsc{3sg.prs} tall\\
\glt `Pablo \textsc{is} tall.'
 \z
 \z
 
  The  scoreboard for a \emph{que}-less \isi{verum} sentence like \eqref{ex:pablo2} is given in Table \ref{tab:scoreboardsi}. The \isi{verum} marked proposition is part of the speaker's DC. The prejacent is put on the table and imposed on the current CG\is{common ground}.

\begin{table}
	\begin{tabular}{l l  l l}
	\lsptoprule
		\multicolumn{2}{c}{current}  & \multicolumn{2}{c}{projected}\\\cmidrule(lr){1-2}\cmidrule(lr){3-4}
		CG\is{common ground}{} & $\{p\}$ & CG\is{common ground}* &$\{\{\,\}\}$\\
		DC\textsubscript{Speaker}& $\{\isi{verum}(p)\}$ & DC*\textsubscript{Speaker}& $\{\{\,\}\}$ \\
		DC\textsubscript{Hearer}& $\{\,\}$  & DC*\textsubscript{Hearer}& $\{\{\,\}\}$\\
		Table& $\langle p\rangle$ & Table*&  $\langle\langle\,\rangle\rangle$ \\\lspbottomrule
	\end{tabular}
	\caption{Conversational scoreboard when uttering \emph{Sí p}}\label{tab:scoreboardsi}
\end{table}

Table \ref{tab:scoreboardsique} illustrates the conversational scoreboard that I assume for \emph{sí que}-sentences like \eqref{ex:pablo1}. The presence of the attributive complementizer again highlights an aspect that is already present in the interpretation of the \isi{verum} sentence by virtue of the fact that whatever is part of the CG\is{common ground} is also part of the hearer's DC.   Again, a consequence of this proposal is that sentences with \emph{que} should be interpreted as being more insistent and emphatic. 



\begin{table}
	\begin{tabular}{l l l l}
	\lsptoprule
		\multicolumn{2}{c}{current}  & \multicolumn{2}{c}{projected}\\\cmidrule(lr){1-2}\cmidrule(lr){3-4}
		CG\is{common ground}{} & $\{p\}$ & CG\is{common ground}* &$\{\{\,\}\}$\\
		DC\textsubscript{Speaker}& $\{\isi{verum}(p)\}$ & DC*\textsubscript{Speaker}& $\{\{\,\}\}$ \\
		DC\textsubscript{Hearer}& $\{p\}$  & DC*\textsubscript{Hearer}& $\{\{\,\}\}$\\
		Table& $\langle p\rangle$ & Table*&  $\langle\langle\,\rangle\rangle$ \\
		\lspbottomrule
	\end{tabular}
	\caption{Conversational scoreboard when uttering \emph{Sí que p}}\label{tab:scoreboardsique}
\end{table}


The final part of this section is dedicated to the discourse contribution of attributive \emph{que} in \isi{polar question}s like \eqref{ex:pressa}.

\ea\label{ex:pressa}
Catalan\\
\gll Que tens pressa? \\
\textsc{que} have.\textsc{2sg.prs} stress\\
\glt `Are you stressed out?' (ebook-cat)
\z



 My proposal is that  here too the function of attributive \emph{que} is to   attribute a \isi{commitment} to  a proposition to the hearer. Other than in declaratives, the proposition is not added directly to the current DC  but to the projected DC of the hearer.  
Table \ref{tab:scoreboardquestion} illustrates the conversational scoreboard proposed for neutral \isi{polar question}s in \citet{Malamud2015}. The prejacent, i.e. the declarative equivalent of the \isi{polar question},  is added to the top of the table.\footnote{For an explanation of why it is the positive version of the proposition that is put on the table, see the concept of \emph{highlighting} introduced in \citet{Roelofsen2015}.} The positive and  negative versions of the prejacent are added to the projected CG\is{common ground}. In the following moves, the answer adds the hearer's \isi{commitment} to either \emph{p}  or $\neg p$ which will then become part of the CG\is{common ground}. 

\begin{table}
	\begin{tabular}{l l  l l}
	\lsptoprule
		\multicolumn{2}{c}{current}  & \multicolumn{2}{c}{projected}\\\cmidrule(lr){1-2}\cmidrule(lr){3-4}
		CG\is{common ground}{} &$\{\,\}$ & CG\is{common ground}*& $\{\{p\}, \{\neg p\}\}$\\
		DC\textsubscript{Speaker}& $\{\,\}$ & DC*\textsubscript{Speaker}& $\{\{\,\}\}$ \\
		DC\textsubscript{Hearer}& $\{\,\}$  & DC*\textsubscript{Hearer} &$\{\{\,\}\}$\\
		Table& $\langle p\rangle$ & Table*&  $\langle\langle\,\rangle\rangle$ \\\lspbottomrule
	\end{tabular}
	\caption{Conversational scoreboard when asking \emph{p?}}\label{tab:scoreboardquestion}
\end{table}


For \emph{que}-initial \isi{polar question}s, I assume again that they contain all the same conversational states as neutral \isi{polar question}s, along with  an additional state that results from the presence of the attributive complementizer (cf. Table \ref{tab:scoreboardquequestion}). This means that the prejacent is put on the table and the positive and negative versions are added to the projected CG\is{common ground}. In \sectref{sec:presupeval}, I proposed the generalization that  \emph{que} in \isi{polar question}s signals that the speaker expects a positive answer from the hearer. I model this by proposing that the prejacent is added to the hearer's projected DC.  By adding the prejacent to the projected rather than the current DC, the speaker shows that she is not absolutely sure that the hearer will answer positively, but she suspects it.  If she were sure that the hearer was committed to the proposition, there would be no need to ask a question in the first place; instead the speaker could directly assert the bare or attributive \emph{que}-initial proposition.   What this means is that the contribution of the complementizer in  \emph{que}-initial declarative and \emph{que}-initial \isi{polar question}s only differs in that in the latter the attributive \isi{commitment}  is tentative whereas in the former  it is presented as definite. Importantly, delaying the attribution to the projected DC is not proposed ad hoc here, but rather, as I showed, it follows from  general properties of the information states involved in questions.


\begin{table}
	\begin{tabular}{l l  l l}
	\lsptoprule
		\multicolumn{2}{c}{current}  & \multicolumn{2}{c}{projected}\\\cmidrule(lr){1-2}\cmidrule(lr){3-4}
		CG\is{common ground}{} &$\{\,\}$ & CG\is{common ground}*& $\{\{p\}, \{\neg p\}\}$\\
		DC\textsubscript{Speaker}& $\{\,\}$ & DC*\textsubscript{Speaker}& $\{\{\,\}\}$ \\
		DC\textsubscript{Hearer}& $\{\,\}$  & DC*\textsubscript{Hearer}& $\{\{p\}\}$\\
		Table& $\langle p\rangle$ & Table*&  $\langle\langle\,\rangle\rangle$ \\\lspbottomrule
	\end{tabular}
	\caption{Conversational scoreboard when asking \emph{Que p?}}\label{tab:scoreboardquequestion}
\end{table}

In \sectref{sec:presupeval}, I presented the different contexts in which \emph{que}-initial \isi{polar question}s appear,  which were teased apart  by \citet{PrietoRigau2007}. For \isi{polar question}s in an anti-expectational\is{surprise} context,  the speaker held a previous belief that $\neg$ \emph{p}  was the case (see the scoreboard for an anti-expectational\is{surprise} \emph{que}-initial \isi{polar question}s in Table \ref{tab:scoreboardantiexquequestion}), but contextual evidence contradicts her belief. This is illustrated in the example in \eqref{ex:antiexnorthrep}  repeated from \eqref{ex:antiexnorth}.


\ea Catalan 	(\citealt[15]{PrietoRigau2007})\label{ex:antiexnorthrep}\\
	\gll Que vindràs a Barcelona? No em pensava pas que ens acompanyessis.\\
	\textsc{que} come.\textsc{2sg.fut} to B. not me think\textsc{1sg.ipfv.pst} \textsc{neg} that us acompany.\textsc{2sg.ipfv.pst}\\
	\glt `Are you coming to Barcelona? I didn’t think you were coming with us.'\\
\z

I define contextual evidence as evidence accessible to the speaker  in the current discourse situation or  in a previous situation (\citealt{Kocher2017a}).\footnote{This definition contrasts with that proposed by \citet{Buering2000} and adopted by \citet{Sudo2013}, where it is stated that the contextual evidence has to be mutually accessible to the speaker and the hearer. In \citet{Kocher2017a}, I show that to make \emph{que} in Catalan \isi{polar question}s felicitous the evidence does not have to be mutually accessible as long as the hearer can accommodate the fact that the speaker is \isi{bias}ed.}  Importantly,  given the contextual evidence and despite the fact that it  is in disagreement with her own previous belief, the speaker expects that the hearer will answer her question in the positive. This licenses the use of attributive \emph{que} in these contexts, which again places the  prejacent in the  projected DC of the hearer. 
 
 \begin{table}
 	\begin{tabular}{ll  l l l l}
 	\lsptoprule
 		\multicolumn{2}{c}{previous} &	\multicolumn{2}{c}{current}  & \multicolumn{2}{c}{projected}\\\cmidrule(lr){1-2}\cmidrule(lr){3-4}\cmidrule(lr){5-6}
 	CG\is{common ground}{} &$\{\,\}$ &	CG\is{common ground}{} &$\{\,\}$ & CG\is{common ground}*& $\{\{p\}, \{\neg p\}\}$\\
 		DC\textsubscript{Speaker}& $\{\neg p\}$ &	DC\textsubscript{Speaker}& $\{\,\}$ & DC*\textsubscript{Speaker}& $\{\{\,\}\}$ \\
 		DC\textsubscript{Hearer}& $\{\,\}$  &	DC\textsubscript{Hearer}& $\{\,\}$  & DC*\textsubscript{Hearer}& $\{\{p\}\}$\\
 		Table& $\langle\,\rangle$ &	Table& $\langle p\rangle$ & Table*&  $\langle\langle\,\rangle\rangle$ \\
 		\lspbottomrule
 	\end{tabular}
 	\caption{Conversational scoreboard when asking an anti-expectational\is{surprise} \emph{Que p?}}\label{tab:scoreboardantiexquequestion}
 \end{table}


Confirmatory \isi{polar question}s such as \eqref{ex:vacancesquerep} present a different picture.
\ea\label{ex:vacancesquerep} Catalan 	(\citealt[19: ex 40a]{Kocher2017a})\\ $[$Context: Anna meets her friend Carles. He is tanned and seems relaxed. Anna asks:$]$
	\exi{}\gll Que has estat de vacances?\\
	\textsc{que} \textsc{au.2sg.prf.prs} be.\textsc{ptcp} on vacation\\
	\glt `Have you been on  vacation?'
\z



 In these cases,  there is no contextual evidence that contradicts the speaker's previous belief; rather, the speaker herself  suspects that \emph{p}  is the case and she  asks the question in order to confirm this suspicion. Given her belief, regardless of what it is founded on -- in the case of \eqref{ex:vacancesquerep}, indirect contextual evidence --, she expects that the hearer's answer  will be positive. She uses attributive \emph{que} to express the fact that  she considers the prejacent to be part of the projected discourse \isi{commitment} of the hearer. 


 \begin{table}
	\begin{tabular}{l l  l l}
	\lsptoprule
		\multicolumn{2}{c}{current}  & \multicolumn{2}{c}{projected}\\\cmidrule(lr){1-2}\cmidrule(lr){3-4}
		CG\is{common ground}{} &$\{\,\}$ & CG\is{common ground}*& $\{\{p\}, \{\neg p\}\}$\\
		DC\textsubscript{Speaker}& $\{\,\}$ & DC*\textsubscript{Speaker}& $\{\{p\}\}$ \\
		DC\textsubscript{Hearer}& $\{\,\}$  & DC*\textsubscript{Hearer}& $\{\{p\}\}$\\
		Table& $\langle p\rangle$ & Table*&  $\langle\langle\,\rangle\rangle$ \\
		\lspbottomrule
	\end{tabular}
	\caption{Conversational scoreboard when asking a confirmatory \emph{Que p?}}\label{tab:scoreboardconfirmatoryquequestion}
\end{table}


To conclude the discussion on \emph{que} in \isi{polar question}s, I will now provide a  characterization of the scoreboard  of these questions when they contain the question particles \emph{oi} or \emph{eh}. For the following line of argument, one crucial point is that according to \citet{PrietoRigau2007}, these constructions only appear in confirmatory questions. Furthermore, they are not restricted to full questions but can also appear in reduced confirmatory tags (cf. \ref{ex:oiehquesino}).

\ea\label{ex:oiehquesino}
\ea
\label{ex:ehquesi}
Catalan\\
\gll
 Sense fer res es fa més llarga l' estona. Eh que sí? \\
 without do nothing \textsc{cl.imp} make.\textsc{3sg.prs} more long the {time period} \textsc{eh} \textsc{que} yes\\
\glt `When doing nothing time grows longer. Right?' (caWac)
\ex\label{ex:oiqueno} 

\gll No tots els partits estan acusats d' haver -se finançat il·legalment, oi que no? \\
not all the party.\textsc{pl} be.\textsc{3pl.prs} accused of \textsc{aux.inf.prf.prs} \textsc{cl.refl} finance.\textsc{ptcp} illegally \textsc{oi} \textsc{que} no\\
\glt `Not all parties are accused of being funded illegally, right?' (caWac)
\z
\z


The particles also appear as final tags on declaratives where they function as  requests for confirmation. \citet{Castroviejo2018} argues that in these contexts, \emph{eh}  and \emph{oi} have slightly different discourse contributions.  With \emph{oi?} the speaker double-checks the truth of the prejacent, while with \emph{eh?} she requests that the hearer voices his \isi{commitment}. 
While  the particles are interchangeable in many contexts, they do not behave the same way when  a  confirmation of facts is requested, as  in \eqref{ex:rodona}: These sentences are infelicitous with \emph{eh?} but felicitous with \emph{oi?}. On the other hand, in requests for confirmation of  opinions like in \eqref{ex:cabellsrep} repeated from \eqref{ex:cabells}, both particles are allowed.

\ea Catalan
\ea\label{ex:rodona}
 (\citealt[ex 18]{Castroviejo2018})\\
\gll La Terra és rodona, oi?/ \#eh? \\
the earth be.\textsc{3sg.prs} round \textsc{oi} \textsc{eh}\\
\glt `The Earth is round, right/ \#huh?' 
\ex\label{ex:cabellsrep}  (\citealt[ex 19]{Castroviejo2018})\\

\gll T' has tallat els cabells, oi?/ eh? \\
\textsc{cl.2sg} \textsc{aux.2sg.prf.prs} cut.\textsc{ptcp} the hair \textsc{oi} \textsc{eh}\\
\glt `You had your hair cut, right?/ huh?'
\z
\z

\citet{Castroviejo2018}  proposes the following conversational scoreboards to model the discourse contribution of the sentence-final particles. 
Table \ref{tab:declarativeoi} shows the scoreboard assumed in \citet{Castroviejo2018} for \emph{p, oi?}.





 It has all the information states of an assertion, i.e. the prejacent is added to the table and the projected CG\is{common ground}. The contribution of the particle comes by way of adding the prejacent  to the speaker's DC because the \isi{commitment} to the truth of the proposition is at issue. 
  
 \begin{table}
 	\begin{tabular}{l l  l l}
 	\lsptoprule
 		\multicolumn{2}{c}{current}  & \multicolumn{2}{c}{projected}\\\cmidrule(lr){1-2}\cmidrule(lr){3-4}
 		CG\is{common ground}{} &$\{\,\}$ & CG\is{common ground}*& $\{\{p\}\}$\\
 		DC\textsubscript{Speaker}& $\{\,\}$ & DC*\textsubscript{Speaker}& $\{\{p\}\}$ \\
 		DC\textsubscript{Hearer}& $\{\,\}$  & DC*\textsubscript{Hearer}& $\{\{\,\}\}$\\
 		Table& $\langle p\rangle$ & Table*&  $\langle\langle\,\rangle\rangle$ \\
 		\lspbottomrule
 	\end{tabular}
 	\caption{Conversational scoreboard when uttering \emph{p, oi?} adopted from \citet{Castroviejo2018}}\label{tab:declarativeoi}
 \end{table}
 
 \citet{Castroviejo2018} proposes a different contribution for  \emph{eh}. The corresponding scoreboard is given in Table \ref{tab:declarativeeh}.  It again contains all the information states of an assertion but unlike with \emph{oi}, the prejacent is not added to the projected DC of the speaker. The \isi{commitment} to the truth is not at issue, which is why \emph{eh} is infelicitous with facts (cf. \ref{ex:rodona}). The proposition is  attributed to the hearer's projected DC because \emph{eh} is used when the speaker seeks a confirmation of the tentative \isi{commitment} she attributes to the hearer.
 
\begin{table}
	\begin{tabular}{l l  l l}
	\lsptoprule
		\multicolumn{2}{c}{current}  & \multicolumn{2}{c}{projected}\\\cmidrule(lr){1-2}\cmidrule(lr){3-4}
		CG\is{common ground}{} &$\{\,\}$ & CG\is{common ground}*& $\{\{p\}\}$\\
		DC\textsubscript{Speaker}& $\{\,\}$ & DC*\textsubscript{Speaker}& $\{\{\,\}\}$ \\
		DC\textsubscript{Hearer}& $\{\,\}$  & DC*\textsubscript{Hearer}& $\{\{p\}\}$\\
		Table& $\langle p\rangle$ & Table*&  $\langle\langle\,\rangle\rangle$ \\
		\lspbottomrule
	\end{tabular}
	\caption{Conversational scoreboard when uttering \emph{p, eh?} adopted from \citet{Castroviejo2018}}\label{tab:declarativeeh}
\end{table}

The contrast between the two particles is lost when they introduce a \emph{que}-initial \isi{polar question}. In these,  the presence of either particle coincides with a  confirmatory reading. 
\ea\label{ex:feina}
Catalan (\citealt[ex 15]{Castroviejo2018})\\
\gll Oi/ Eh que acabaràs la feina? \\
\textsc{oi} \textsc{eh} \textsc{que} finish.\textsc{2sg.fut} the work\\
\glt `You'll finish your work, right?' 
\z


The  loss of the different interpretations, in my view, constitutes evidence  that the contribution made by the particles and the attributive complementizer is compositional. To show this, I propose the  scoreboard in Table \ref{tab:quequestionsehoi}. 

 
\begin{table}
	\begin{tabular}{l l  l l}
	\lsptoprule
		\multicolumn{2}{c}{current}  & \multicolumn{2}{c}{projected}\\\cmidrule(lr){1-2}\cmidrule(lr){3-4}
		CG\is{common ground}{} &$\{\,\}$ & CG\is{common ground}*& $\{\{p\}\}$, $\{\{ \neg p\}\}$ \\
		DC\textsubscript{Speaker}& $\{\,\}$ & DC*\textsubscript{Speaker}& $\{\{p\}\}$ \\
		DC\textsubscript{Hearer}& $\{\,\}$  & DC*\textsubscript{Hearer}& $\{\{p\}\}$\\
		Table& $\langle p\rangle$ & Table*&  $\langle\langle\,\rangle\rangle$ \\\lspbottomrule
	\end{tabular}
	\caption{Conversational scoreboard when uttering \emph{Oi/Eh p?}}\label{tab:quequestionsehoi}
\end{table}

In both cases, the \isi{polar question}s introduced by the particles have a confirmatory reading.  My assumption is therefore that  they have all the same information states as confirmatory \emph{que}-initial   \isi{polar question}s. This means that just as in a regular \isi{polar question}, the prejacent is added to the table and the positive and negative version of the prejacent are added to the projected CG\is{common ground}. As proposed above, I conceive of a confirmatory reading as a tentative \isi{commitment} on the part  of the speaker. This is modeled as an addition of the prejacent to the speaker's projected DC. The contribution of attributive  \emph{que} is once again modeled as an attributive \isi{commitment} to the hearer, i.e. the addition of the prejacent to the hearer's DC. Given this setup, the contribution of each of the particles does not give rise to any change in the information states. In the view of \citet{Castroviejo2018}, the contribution of \emph{eh} is that the prejacent is added to the hearer's DC, which coincides with the contribution of \emph{que}. In the alternative case with \emph{oi}, its contribution that adds  the prejacent  to the speaker's projected DC is  also already present because of the confirmatory reading of the question.  As a result, the scoreboards for questions containing either of the particles are identical, which explains why the contrast in their contribution following declaratives observed by \citet{Castroviejo2018}  is lost when they introduce \emph{que}-initial \isi{polar question}s.   

To sum up, in this section I have modeled  the pragmatic contribution of attributive \emph{que} by making use of  conversational scoreboards as proposed by \citet{Malamud2015}. My point of departure was that all the relevant constructions  are attributive in the sense of \citet{Poschmann2008}. This means that a \isi{commitment} to the prejacent is attributed to the hearer. It was proposed that this can be modeled in a system à la \citet{Malamud2015} by adding the prejacent to the current DC of the hearer in assertions and to the projected DC of the hearer in \isi{polar question}s. I applied this idea to the individual constructions and suggested that  further pragmatic effects that arise are a result of the interplay between the attributive meaning  and the properties of the other elements involved in the constructions.

\section{Summary}
This chapter has explored a variety of different constructions involving attributive \emph{que}.  My proposal was that in all the constructions  the complementizer is merged in the lowest projection of the left periphery, in FinP, where it is valued with an attributive feature.  The surface position of the complementizer in the different constructions is reached through head-to-head movement. In the different sections of the chapter I demonstrated empirically that this movement is conditioned by the presence of externally-merged material in a specifier. It was shown that this simple mechanism allows the correct predictions to be made with regard  to word orders involving the attributive complementizer.  The final section of this chapter was dedicated to the discourse contribution of attributive \emph{que}. My proposal was that it attributes a \isi{commitment} to the   proposition to the hearer. Different pragmatic effects were explained as the result of an interplay between the attributive meaning and the other elements contained in the constructions. 

Concerning the interaction of syntax and pragmatics, this chapter again provides evidence for a decoupling of the two components of grammar. The presence of an attributive  feature  as well as the  distributional properties of a complementizer with this value are syntax internal and therefore rightly treated within its domain. The consequences of the attributive feature for the interpretation of the sentences as well as its interaction with other elements involved in them are not syntactic but context-dependent and therefore part of pragmatics. 
