
\chapter{Introduction}\label{sec:1}


This monograph explores the question of how the interface between syntax and pragmatics is organized in Ibero-Romance. The principal aim is to investigate where the boundary between these two components of grammar lies, through the use of tools from generative syntax and formal  pragmatics. 

The empirical focus is on what I will call root-clause complementizer constructions.\footnote{In this book, I use the term \emph{construction} in a theory-neutral way to refer to a pairing of a productive abstract structure with a compositional meaning.}  With this term I refer to cases in which the complementizer \emph{que} shows an unexpected behavior and appears in apparently unembedded contexts. The constructions prove particularly interesting with respect to the question I address in this book. Complementizers are traditionally considered to serve the  syntactic function of marking a sentence as embedded. However, in the contexts that I investigate, they do not carry out this function; instead their presence has an impact on the pragmatics of the sentence that contains them. 

A complementizer can generally be defined as the element that identifies a sentential complement  and  that  has the   function of subordinating a dependent sentence (see for instance \citealt{Noonan2007}, \citealt{Kehayov2016}). This means that we typically expect to encounter complementizers only in embedded contexts.  The data in \eqref{ex:queobj} and \eqref{ex:quesubj} show that based on this definition the Ibero-Romance word \emph{que} qualifies as a complementizer. The sentences following \emph{que} are all sentential  arguments dependent on the matrix clause verb. In \eqref{ex:queobj} the sentences introduced by \emph{que} function as the object of the verb `say' and  in \eqref{ex:quesubj}, as the subject of the copula verb `be'.


\ea\label{ex:queobj}
\ea\label{ex:queobjcat}
Catalan\\
\gll La Joana diu que la Maria és lingüista. \\
the Joana say.\textsc{3sg.prs} that the Maria be.\textsc{3sg.prs} linguist\\
\ex\label{ex:queobjpt} Portuguese\\
\gll A Joana diz que a Maria é linguista. \\
the Joana say.\textsc{3sg.prs} that the Maria be.\textsc{3sg.prs} linguist\\
\ex\label{ex:queobjes}
\gll Juana dice que María es lingüista. \\
Juana say.\textsc{3sg.prs} that  Maria be.\textsc{3sg.prs} linguist\\
\glt`Jo/uana says that Maria is a linguist.'
\z
\ex
\label{ex:quesubj}
\ea\label{ex:quesubjcat}
Catalan\\
\gll Que la Maria sigui lingüista és fantàstic. \\
that the Maria be\textsc{.3sg.sbjv.prs} linguist be.\textsc{3sg.prs} fantastic\\
\ex\label{ex:quesubjpt}
Portuguese\\
\gll Que a Maria seja linguista é fantástico. \\
that the Maria be\textsc{.3sg.sbjv.prs} linguist be.\textsc{3sg.prs} fantastic\\

\ex\label{ex:quesubjes}
Spanish\\
\gll Que María sea lingüista es fantástico. \\
that  Maria be.\textsc{3sg.sbjv.prs} linguist be.\textsc{3sg.prs} fantastic\\
\glt`That Maria is a linguist is fantastic.'
\z
\z


The data in \eqref{ex:querel} and \eqref{ex:queadv} show that the element \emph{que} is furthermore present in modifiers. In \eqref{ex:querel} it introduces a relative sentence that modifies a DP.\footnote{There are opposing views regarding whether \emph{que} is best analyzed as a relative operator  or a complementizer  in certain relative sentences. On Spanish, see for instance \citet{Rivero1982}, who considers it a relative operator, and  \citet{Arregi1998}, \citet{Brucart1992} who consider it a complementizer.} The examples in \eqref{ex:queadv} show that \emph{que} can co-occur with adverbs when introducing sentential adjuncts.\footnote{The presence of the complementizer  in sentential adjuncts is subject to adverb-, language- and register-dependent variation. } 


\ea\label{ex:querel}
\ea
Catalan\\
\gll  La Joana parla d' un llibre que està llegint. \\
 the Joana talk.\textsc{3sg.prs} of a book 	that \textsc{aux.3sg.prog.prs} reading\\
	\ex
	Portuguese\\
	\gll A Joana está a  falar dum livro que está a ler. \\
	 the Joana \textsc{aux.3sg.prog.prs} to talk {of a} book 	that \textsc{aux.3sg.prog.prs} to read\\
\ex
Spanish\\
\gll Juana habla de un libro que esta leyendo. \\
Juana talk.\textsc{3sg.prs} of a book 	that \textsc{aux.3sg.prog.prs} reading\\
\glt`Jo/uana is talking about a book that she is reading.'
\z
\ex\label{ex:queadv} 
\ea
Catalan\\
\gll Has de netejar el teu dormitori abans que arribi jo. \\
have.\textsc{2sg.prs} to clean the your room before that arrive.\textsc{1sg.sbjv.prs} I\\
\ex
Portuguese\\
\gll Tens de limpar o teu quarto antes que eu chegar. \\
have.\textsc{2sg.prs} to clean the your room before that I arrive.\textsc{1sg.sbjv.fut} \\
\ex
Spanish\\
\gll Has de limpiar tu cuarto antes que  llegue yo. \\
have.\textsc{2sg.prs} to clean  your room before that arrive.\textsc{1sg.sbjv.prs}  I\\
\glt`You have to clean your room before I arrive.'
\z
\z


The phenomena that are the central concern of the present monograph pose a problem for a traditional notion of complementizer. In  these constructions \emph{que} does not seem to exhibit its typical function of identifying a complement clause but appears in root contexts  without being syntactically embedded.
The presence of the complementizer, contrary to the examples above, is not obligatory in these sentences. It  is not required strictly for syntactic reasons but the presence of the complementizer has an impact on the interpretation of the sentence that it introduces. 
 
\ea\label{ex:quereprep}
\ea Catalan\\
\gll  (Que) la Maria ve a la festa. \\
\textsc{que} the Maria come.\textsc{3sg.prs} to the party\\
\ex
Portuguese\\
\gll (Que) a Maria vai à  festa. \\
\textsc{que} the Maria go.\textsc{3sg.prs} {to the} party\\
\ex
Spanish\\
\gll  (Que) María viene a la fiesta. \\
\textsc{que}  Maria come.\textsc{3sg.prs} to the party\\
\glt`[reportative:] Maria is coming to the party.'
\z
\z

\begin{sloppypar}
One interpretation  of the examples in  \eqref{ex:quereprep}  is that they constitute reported speech. This interpretation is usually supported by the contextual presence of a \emph{\isi{verbum dicendi}}.  In the sentences in (\ref{ex:advque}--\ref{ex:sique}) the presence of  \emph{que} has a different effect. 
\end{sloppypar}

\ea\label{ex:advque}
\ea
Catalan\\
\gll Certament (que) és un arros bó. \\
certainly \textsc{que} be.\textsc{3sg.prs} a rice good\\
\ex
Portuguese\\
\gll Certamente (que) é um arroz bom. \\
certainly \textsc{que} be.\textsc{3sg.prs} a rice good\\
\ex
Spanish\\
\gll Ciertamente (que) es un arroz bueno. \\
certainly \textsc{que} be.\textsc{3sg.prs} a rice good\\
\glt`Certainly, this is  a great rice.'
\z
\ex\label{ex:exclque}
\ea
Catalan\\
\gll Que bó (que) és aquest arròs! \\
	how good \textsc{que} be.\textsc{3sg.prs} this rice\\
\ex
Portuguese\\
\gll Que bom (que) é este arroz! \\
	how good \textsc{que} be.\textsc{3sg.prs} this rice\\
\ex
Spanish\\
\gll ¡Qué bueno (que) es este arroz! \\
	how good \textsc{que} be.\textsc{3sg.prs} this rice\\
\glt`How great this rice is!'
\z
\ex\label{ex:sique}
\ea
Catalan\\
\gll Sí (que) és un arròs bó. \\
\textsc{verum} \textsc{que} be.\textsc{3sg.prs} a rice good\\
\ex
Spanish\\
\gll Sí (que) es un arroz bueno. \\
\textsc{verum} \textsc{que} be.\textsc{3sg.prs} a rice good\\
\glt`This \textsc{is} a great rice.'
\z
\z

In all of these examples the proposition introduced by \emph{que} is marked as  information the hearer should already know.   In the constructions that give rise to this interpretation, the complementizer often co-occurs adjacent to other left-peripheral material. For instance in \eqref{ex:advque}, the complementizer follows the epistemic adverb `certainly', in \eqref{ex:exclque} it follows the \textit{wh}-expression of a \textit{wh}-exclamative and in \eqref{ex:sique}, it follows the verum marker \emph{sí}.\footnote{European Portuguese makes use of a different structure to express verum. The different strategies are discussed  in \sectref{sec:presupaffc}.} In these constructions the complementizer shows a different syntactic distribution than in reportatives, where these same expressions are preceded rather than followed by \emph{que}. 

The examples presented so far suggest that there are at least three different functions expressed by \emph{que}: It functions as a typical subordinating complementizer in \eqref{ex:queobj}, \eqref{ex:quesubj}, \eqref{ex:querel} and \eqref{ex:queadv}, it marks a sentence as reported speech in \eqref{ex:quereprep} and indicates that the content of the sentence constitutes known information in \eqref{ex:advque},  \eqref{ex:exclque} and \eqref{ex:sique}.  This could lead to the conclusion that we are dealing with three different lexical items: One \emph{que} that is a subordinator and two \emph{que}s that are pragmatic markers, one of which gives rise to a reportative interpretation while the other  imposes on the hearer a commitment to the proposition in the scope of \emph{que}. While this is doubtless a valid line of reasoning, this book will present arguments  for  precisely the opposite view. In the following chapters, I will present evidence in support of an analysis that does not propose  multiple lexical items with dedicated functional meanings, but rather assumes only one lexical item  with an underspecified meaning.



\section{Aims,  scope and motivation}\label{sec:intscope}
\subsection{Aims}
The central  aims of this book are to determine where the boundaries between syntax and pragmatics lie, how these components of grammar interact and how the interaction is  most adequately modeled within a formal approach to grammar. These questions will be addressed in the light of root clause complementizer constructions. A further  goal is therefore also to investigate the structure of these constructions  and develop an analysis that can account for their empirical distribution. The results of this investigation  provide  information on the interaction of \emph{que} with other left-peripheral material and  hence lead to a broader understanding of the left periphery in Ibero-Romance languages.

An additional goal is to gain insights into the properties of the complementizer itself. As stated above, the empirical focus of this book is those examples in which we encounter an item that looks like a complementizer but does not behave in accordance with the traditional notion of  complementizer. In principle this leaves us with  two choices. The first option would be to conclude that these are not complementizers and  that there are multiple lexical items that are spelled out as \emph{que}, each encoding a specific meaning. The alternative option I propose  is to assume that there is only one \emph{que}  and that it is the notion of complementizer that needs to be adjusted.  In other words,  \emph{que}  is a complementizer but, crucially, a complementizer is something other than what we thought it was. One of the objectives of this book is therefore to find empirical evidence in favor of this second option and to show that, a revised notion of complementizer allows us to account for  apparently atypical behavior like that illustrated in (\ref{ex:quereprep}--\ref{ex:sique}).  
One  empirical point that motivates the assumption that there is only one \emph{que} is that there is no  formal distinction that would suggest that there are multiple types of \emph{que}. Independent of its function, it is always spelled out the same way. Furthermore, although it does surface at different points in the functional field, \emph{que} is restricted to appearing in the left periphery of a sentence,  the natural habitat of a complementizer. If the item had gained a different function as a pragmatic marker through a process of grammaticalization, for instance,  greater syntactic mobility might be expected.
 


\subsection{Scope}

The examples I discuss in this book are taken from Catalan,  Spanish and Portuguese. Most generalizations I present hold for European and non-European varieties of the latter two languages. The Portuguese data that stem from a corpus are identified as either European or Brazilian. Where relevant, comparisons are drawn with similar phenomena in other Romance and non-Romance languages. The examples are taken from  a range of sources. Some evidence  is drawn from corpus data. For Spanish, I consulted the Corpus del Espa\~nol (henceforth CdE), making use of the contemporary portion of the 2001,	100 million token, Genre/Historical subcorpus and the entire 2016, 2 billion token, Web/Dialects subcorpus.  For Portuguese, I consulted the equivalent Corpus do Português (henceforth CdP), again making use of the contemporary portion of the 2006,  45 million token, Genre/Historical subcorpus and the entire 2016, 1.1 billion token,   Web/Dialects subcorpus. These corpora are useful in studying the phenomena  under investigation in this book: They  are typically employed in informal registers and the databases contain oral data (Genre/Historical subcorpus)  and web data (Web/Dialects subcorpus) in which informal registers usually prevail. The corpora have the additional advantage of being annotated,  which facilitates the query. For Catalan, there are no comparable annotated corpora that are publicly available. I  mainly relied on the 2014 780 million token Catalan Web as Corpus  (henceforth caWac). This corpus  does not have an online interface but can be  downloaded freely. The Catalan database also included a small self-compiled ebook corpus (400,000 tokens) (henceforth ebook-cat). In addition to the corpus data, I elicited judgments on grammaticality and acceptability  of constructed examples. My informants were predominantly  native speakers of the European varieties of Portuguese and Spanish, and  for Catalan, speakers of Central and Balearic Catalan. The experimental stimuli  in \chapref{sec:experiments} stem from the corpora listed above. The judgments in the experiments are elicited from native speakers  mostly of Central Catalan and  of the European variety of Spanish.

\subsection{Motivation}\largerpage
There are a number of considerations that motivated the development of a new analysis to account for the phenomena under investigation, despite the fact that most of them  have already been explored in the literature. First, a global goal of this monograph is  to adopt a unified perspective and to focus on the shared properties of data that have previously only been examined separately.  I attempt to achieve a  broader empirical coverage than previous accounts and  show how different phenomena are related on an underlying level.   I can thus contribute new insights that deepen our understanding of the nature of the complementizer and its interaction at the left periphery,  which in turn allows conclusions drawn regarding the central question of how syntax and pragmatics are related.    The analyses presented  in this book adopt a different modeling of the two relevant components  of grammar. The most drastic difference is that my proposal does not rely on a neo-performative hypothesis, while most previous analyses have adopted versions of this approach.  Neo-performative hypotheses propose that illocutionary forces and related pragmatic concepts are encoded syntactically, without treating them as deleted performative clauses as in the traditional performative hypothesis presented in \citet{Ross1970}. In the neo-performative hypotheses -- the most influential of which is developed in \citet{SpeasTenny2003} -- the performative structure is a part of the 
 architecture of the clause envisioned as a functional domain above the CP.  An overview of the central assumptions of (neo-)performative hypotheses as well as some criticism of these found in the literature is given in \sectref{sec:performativehyp}. 


Further motivating factors behind my adoption of this position will emerge over the course of the following chapters, in particular in  \sectref{sec:insubexistan} and \sectref{sec:presupeval}, where I review the  analyses that have been put forward in the literature and point out  potential limitations in some, though not all, of these proposals. I want to stress at this point that  the  weaknesses of the other accounts  are  by no means  dramatic enough to warrant rejection out-of-hand.  The motivation of this investigation is not to argue that the  explanation provided in this book is the only valid one. It is instead an attempt to show that the phenomena can be accounted for with a less inflated syntactic structure. A more limited amount of structure could be brought in as an argument in favor of economy. However, I refrain from presenting my analysis as the more economical alternative,  primarily because I do adopt a fairly rich cartographic structure  and therefore cannot claim that the assumed structure is minimal in any serious way. A second reason is that assuming less structure in the present account comes at a cost, namely, the attribution of a more dominant role to pragmatic mechanisms;   whether this is truly more economical cannot currently be determined.   Ultimately,  in a theoretical discipline as rich in conceptual alternatives as linguistics, the choice between multiple suitable and convincing analyses comes down to personal preference to some degree. This fairly  mundane factor has undeniably played a non-negligible  role in  developing the present proposal.

\section{Theoretical background}\label{sec:intconcept}\largerpage
In this section, I briefly summarize the theoretical background of my analysis. I outline  my re-conception of what an Ibero-Romance complementizer constitutes and give a brief introduction to my main assumptions and  the minimal adaptations to the cartographic approach to the left periphery, which provides the theoretical framework for my analysis.

\subsection{The nature of \emph{que}}
\begin{sloppypar}
In traditional conceptions of complementizers, the subordinating function is dominant (cf. the definition I cite at the beginning of this chapter). These conceptions, however, do not seem adequate to capture  the behavior of Ibero-Romance \emph{que} exemplified in (\ref{ex:quereprep}--\ref{ex:sique}). It is therefore necessary to redefine what an Ibero-Romance complementizer is in order to comply with one of the goals of the book: Namely, to maintain that there is only one lexical element \emph{que} and that this element is in fact a complementizer. My proposal is influenced by  what  \citet{Bayer2002,Bayer2004} concludes about German \emph{was} and its Bavarian cognate \emph{wos}. This word is notoriously polyfunctional, as illustrated by the examples in \eqref{ex:bayer}. They show that \emph{was} can take up nominal functions, for instance as an indefinite pronoun in \eqref{ex:bayera} and \eqref{ex:bayerb}, and  a \textit{wh}-pronoun in \eqref{ex:bayerc}. 
\end{sloppypar}

\ea\label{ex:bayer}German 
\ea\label{ex:bayera}  (\citealt[288: ex 23a]{Bayer2002})\\
\gll Ich hab da was gesehen. \\
I \textsc{aux.1sg.prf.prs} there \textsc{was} see.\textsc{ptcp}\\
\glt`I have seen something.' 
\ex\label{ex:bayerb} 
(\citealt[288: ex 20b]{Bayer2002})\\
\gll Was auf dem Oberdeck saß, war deutsch und trank {Sekt. }\\
\textsc{was} on the upper-deck sit.\textsc{3sg.ipfv.pst} be.\textsc{3sg.ipvf.pst} German and drink.\textsc{3sg.ipfv.pst} {sparkling wine}  \\
\glt`The people that sat on the upper deck, were German and drank sparkling wine.'  
\ex\label{ex:bayerc}
(\citealt[288: ex 19b]{Bayer2002}) \\
\gll  Was hast du gegessen? \\
\textsc{was} \textsc{aux.2sg.prf.prs} you eat.\textsc{ptcp}\\
\glt`What did you eat?' 
\z
\z

Additionally,   Austro-Bavarian \emph{wos} appears in embedded contexts and can acquire a subordinating function, for instance in the relative sentence in \eqref{ex:bayerd}, where \emph{wos} is preceded by a relative pronoun. 
 
\ea\label{ex:bayerd}  
Austro-Bavarian (\citealt[290: ex 26a]{Bayer2002})\\
\gll  die Frau (die) wos am Eck Wiaschtln vakauft\\
the woman who \textsc{was} at.the corner sausage.\textsc{pl} sell.\textsc{3sg.prs}\\
\glt`The woman who sells sausages at the corner' 
\z


The examples in \eqref{ex:bayerdoubly} suggest that \emph{wos} can be a subordinating item and can simultaneously  function as a \textit{wh}-pronoun in \eqref{ex:bayerdoublya}. This is supported by the fact that in this example, a doubly filled CP -- which could suggest that  the two functions are distributed  across two items and which is grammatical otherwise (cf.~\ref{ex:bayerdoublyc}) --  is not grammatical here (cf. \ref{ex:bayerdoublyb}).

\ea\label{ex:bayerdoubly}Austro-Bavarian (\citealt[4: 9a-c]{Bayer2004})
\ea[]{\gll I woaß, wos–a gern trinkt. \\
I know.\textsc{1sg.prs} \textsc{was}-he preferably drink.\textsc{3sg.prs}\\
\glt `I know what he likes to drink.'  \label{ex:bayerdoublya}}
\ex[*]{\gll 
I woaß, wos dass–a gern trinkt. \\
I know.\textsc{1sg.prs} \textsc{was} that-he preferably drink.\textsc{3sg.prs}\\
\glt \label{ex:bayerdoublyb}}
\ex[]{
\gll 
I woaß, wos fiar–a Bier dass-a gern trinkt.\\
I know.\textsc{1sg.prs} \textsc{was} for-a Bier that-he preferably drink.\textsc{3sg.prs}\\
\glt`I know which beer he likes to drink.' \label{ex:bayerdoublyc}}
	\z
\z

\citeauthor{Bayer2004}'s solution to the puzzle is that  \emph{wa/os} is a maximally underspecified item which acquires  its function contextually.
My proposal for Ibero-Romance \emph{que} is very similar: I also consider it to be  underspecified. In my analysis this is translated to mean that it carries an unvalued feature. The Ibero-Romance complementizer is therefore simply a lexical item with the form \emph{que}  that has an unvalued  feature and that is merged in a position in the CP. Depending on its merge position, a different value and consequently a different functional meaning are acquired.  In order to account for the data that are the  core of the  investigation, I assume that \emph{que} is valued with a \emph{subordinate} feature when it is merged in the highest projection of the left periphery and with an \emph{attributive} feature when it is merged in the lowest projection of the left periphery.  It will be shown over the course of this book that these features are not stipulated to account for the  unembedded constructions exemplified in (\ref{ex:quereprep}--\ref{ex:sique});  on the contrary, they are  the same features that \emph{que} acquires in syntactically embedded contexts like (\ref{ex:queobj}--\ref{ex:queadv}). This of course strengthens the claim that there is only one  \emph{que}.

This re-conception of \emph{que} could be put to use when accounting for the  interrogative pronouns Catalan \emph{què}, Portuguese \emph{que/quê}, Spanish \emph{qué}. One idea that could be developed on this basis would be that these are again expressions of an underspecified \emph{que} which receives a focus feature which has  consequences for its prosodic make-up  but also for its interpretation. 


\subsection{Cartographic approach}\is{cartographic approach|(}
\begin{sloppypar}
The theoretical approach of this book is that of generative linguistics, which determines the aims, argumentation, analysis and diagnostics that are developed and employed.  More precisely, the analysis is formulated within a cartographic approach, and  I adopt the assumption that the complementizer phrase (CP) (\citealt{Chomsky1986}) is split into a universal hierarchy of functional projections (\citealt{Rizzi1997}). The functional projections populating the left periphery mediate the interface between syntactic structure, interpretation and prosody. One assumption  of the cartographic project is that the interpretative and prosodic properties are directly read off the syntactic structure (see \citealt{Belletti2004}, \citealt{Bocci2009}). The proposal of a split CP is related to and inspired by similar ideas that motivated a splitting of phrases into hierarchically ordered projections  in the nominal  domain (\citealt{Abney1987}, \citealt{Cinque1994}, \citealt{Longobardi1996}) and in the inflectional domain (\citealt{Pollock1989}, \citealt{Belletti1990}, \citealt{Ouhalla1991}, \citealt{Cinque1999}). The  development of a richer and more articulate functional field above the IP is supported by empirical findings relating to word order restrictions on sentence peripheral material such as complementizers, interrogative pronouns, topics and foci.  The particular implementation of the idea that I adopt was initially proposed by \citet{Rizzi1997} and was later further refined (see \citealt{Rizzi2001,Rizzi2004a,Rizzi2013} and references therein).\footnote{\citeauthor{Rizzi1997}'s account was influenced by previous works proposing multiple functional projections in the left periphery,  for instance by \citet{Reinhart1981}, \citet{Uriagereka1988, Uriagereka1995},   \citet{Brody1990, Brody1995} and \citet{Culicover1992}.}
While the empirical focus was initially on Romance languages,  support for the universal nature of the hierarchy also stems  from cross-linguistic evidence, for instance from  Germanic (\citealt{Grewendorf2002}, \citealt{Haegeman2004,Haegeman2006,Haegeman2012}), Japanese (\citealt{Saito2012}, \citealt{Belletti2013}), Semitic (\citealt{Shlonsky2000, Shlonsky2014}) and Niger-Congo languages (\citealt{Aboh2004,Aboh2010}, \citealt{Torrence2013}).  
The full hierarchy of the functional heads is given in \eqref{struc:rizzall}.
\end{sloppypar}

\ea\label{struc:rizzall} {[} Force [ Top* [ Int [ Top* [ Foc [ Mod*  [ Top* [ Fin [ IP ]]]]]]]]]]\\ (\citealt{Rizzi2013})
\z 

ForceP, at the left edge of the functional field, is established in \citet{Rizzi1997} as  a projection that encodes the clause type of a sentence.  Subordinating complementizers are assumed to occupy the head of this projection, cf. \eqref{ex:rizforcea} and \eqref{ex:rizforceb}. The examples show that a clitic left dislocated topic targeting a Top-position is  only grammatical below the complementizer (as in \ref{ex:rizforcea}) but not  above it as in \eqref{ex:rizforceb}. This is in line with the idea that the complementizer occupies ForceP, the highest position in the left periphery that has no Top-projection above it.    In \citet{Rizzi1997}  subordinating complementizers express the clause type of a sentence. In more recent publications, some  authors have re-purposed ForceP, or decompositions thereof,  as a projection mediating clause types and illocutionary force (\citealt{SpeasTenny2003}, \citealt{Coniglio2010}, \citealt{Corr2016}, among many others).

\ea Italian (\citealt[288: ex 10a,b]{Rizzi1997})
	\ea[]{  \gll  Credo {{\ob}}\textsubscript{ForceP} che{{\cb}} {{\ob}}\textsubscript{TopP} il tuo libro$_i${{\cb}}, loro lo$_i$ aprezzerebbero molto. \\
believe.\textsc{1sg.prs} {} that {} the your book they \textsc{cl.akk} appreciate.\textsc{3pl.cond} much\\
\glt`I believe that your book, they would appreciate it a lot.' \label{ex:rizforcea}}
\ex[*]{\gll  Credo,  il tuo libro$_i$,  che  loro lo$_i$ aprezzerebbero molto. \\
			 believe.\textsc{1sg.prs}   the your book that they \textsc{cl.akk} appreciate.\textsc{3pl.cond} much\\ 
\glt \label{ex:rizforceb}}
	\z
\z

FinP delimits the functional field at the lower edge.  \citet{Rizzi1997}   proposes that this projection is related to finiteness, and  is the host of the non-finite counterpart of the finite complementizer (cf. \ref{ex:rizfina} and \ref{ex:rizfinb}). The examples apply the same diagnostic as above with Force. A clitic left dislocated topic is grammatical above the non-finite complementizer \eqref{ex:rizfinb} but not below it \eqref{ex:rizfinb}, which is in keeping with the assumption that \emph{di} occupies the lowest left-peripheral position. FinP has been proposed as the projection that is targeted by the finite verb in Germanic verb-second configurations (\citealt{Roberts2004}).\footnote{For a recent account, see \citet{Lohnstein2015} and \citet{Kocher2018b} who propose that the German finite verb in verb second configurations targets the \isi{clause typing} head MoodP.} An important observation in the context of this book is that in certain constructions a finite complementizer can also be merged in this position (see for instance \citealt{Belletti2009, Belletti2013},  \citealt{Ledgeway2005}). 

\protectedex{\ea Italian (\citealt[288: ex 11a,b]{Rizzi1997})
\ea[*]{
\gll Credo  di  il tuo libro$_i$  aprezzarlo$_i$ molto. \\
	 believe.\textsc{1sg.prs}  to the your book appreciate.\textsc{cl.m.sg} much\\
\glt \label{ex:rizfina}}
\ex[]{
\gll 
Credo  {\ob}\textsubscript{TopP} il tuo libro$_i${\cb} {\ob}\textsubscript{FinP}  di{\cb} aprezzarlo$_i$ molto. \\
believe.\textsc{1sg.prs} {} the your book   {} to appreciate.\textsc{cl.m.sg} much\\
\glt`I believe to appreciate your book a lot.' \label{ex:rizfinb}}
\z
\z}

IntP is postulated as the location in which the interrogative complementizer is merged. Evidence for this comes  from Spanish embedded polar questions like \eqref{ex:rizint}, which permit the co-occurence of the finite and the interrogative complementizer and in which, crucially, the interrogative complementizer \emph{si} follows \emph{que}. IntP is furthermore the location of expressions like Italian \emph{perché} `why'  (cf. \citealt{Shlonsky2011}). Moreover, IntP has been proposed  as the host of the complementizer in complementizer-initial polar questions in Sicilian (\citealt{Cruschin2012}) and Catalan (\citealt{Kocher2017a}) (but see \sectref{sec:presuppqC} for my revised take on complementizer-initial polar questions). 

\ea\label{ex:rizint}
Spanish (\citealt[349: ex 30b]{Suner1994})\\
\gll Me preguntaron {\ob}\textsubscript{ForceP} que{\cb} {\ob}\textsubscript{IntP} si{{\cb}} tus amigos ya te visitaron en Granada.\\
\textsc{cl.1sg} ask.\textsc{3pl.prf.pst} {} that {} whether your friends already \textsc{cl.2sg} visit.\textsc{3pl.prf.pst} in Granada\\
\glt`They asked me (that) whether your friends already visited you in Granada.' 
\z

FocP is the projection that hosts fronted foci and \textit{wh}-pronouns (cf. \ref{ex:focrizz}).  Much work has been dedicated to studying the structure, prosody and interpretation of foci within a cartographic framework (see for instance \citealt{Belletti2004},    \citealt{Cruschina2015a, Cruschina2017}, \citealt{Bianchi2016}).\largerpage[2]


\ea\label{ex:focrizz} Italian (adapted from \citealt[203: ex 5]{Rizzi2013}) \\ \gll Credo {\ob}\textsubscript{ForceP} che{\cb} {\ob}\textsubscript{TopP} a Gianni$_i${\cb} {\ob}\textsubscript{FocP} IL MIO LIBRO{\cb}  Piero gli$_i$ doverebbe dare. \\
believe.\textsc{1sg.prs} {} that {} to Gianni {} the my book  Piero \textsc{cl.3sg} should.\textsc{3sg.cond} give\\
\glt`I believe that to Gianni, Piero should give MY BOOK.' 
\z


According to \citet{Rizzi1997}, while   there is only one FocP per clause,  multiple  TopPs, hosting topics, are sandwiched between each of the other projections. This is motivated empirically by examples like \eqref{ex:toprizz} that illustrate the grammaticality of multiple clitic left dislocated topics in one sentence. \citet{Rizzi1997} does not elaborate on whether these topic positions are distinct in any way. \citet{Frascarelli2007a}  propose that the different positions correlate with different interpretations.

\ea\label{ex:toprizz}Italian (adapted from \citealt[290: ex 21]{Rizzi1997}) \\ \gll {\ob}\textsubscript{TopP} Il libro$_i${\cb}, {\ob}\textsubscript{TopP} a Gianni$_j${\cb}, glielo$_{i,j}$ darò senz'altro. \\
{} the book {} to Gianni \textsc{cl.dat}.\textsc{cl.akk} give.\textsc{1sg.fut} {for sure}\\
\glt`The book, to John, I'll give for sure.' 
\z

ModP is introduced as an additional projection in \citet{Rizzi2004} as the locus of  high sentential  modifiers (cf. \ref{ex:modrizz}). Some authors (cf. for instance \citealt[84]{Giorgi2010}, \citealt[248]{vanGelderen2011}) propose that ModP is  split into an Evaluative, Evidential and Epistemic Phrase to accommodate the insights drawn from \citet{Cinque1999}, which show that  modifiers expressing these meanings follow  ordering restrictions.

\ea\label{ex:modrizz}
Italian (\citealt[77: ex 38]{Giorgi2010})\\
\gll {\ob}\textsubscript{FocP} A PARIGI{\cb} {\ob}\textsubscript{ModP} probabilmente{\cb} Paolo è già stato (non a Londra). \\
{} to Paris {} probably Paolo \textsc{aux.3sg.prf.pst} already be.\textsc{ptcp} not to London\\
\glt`To PARIS probably Paolo has already been (not to London).' 
\z

\subsection{Adaptations}
In this book, I assume the slightly adapted version in \eqref{struc:myrizz} of the original functional hierarchy  presented in \eqref{struc:rizzall}. 

\ea\label{struc:myrizz} {[} SubP [ TopP* [ IntP [ TopP* [ FocP [ ModP* [ TopP* [ MoodP [TopP* [ FinP [ IP ]]]]]]]]]]] 
\z


The two adaptations  are inspired by \citet{Haegeman2006} (see also  \citealt{Roussou2010}). The first change is the replacement of ForceP by SubP, a functional projection that hosts subordinating conjunctions. According to \citet[1661]{Haegeman2006}, Sub is identified as the functional head that subordinates a clause and makes it available for selection. Contrary to \citeauthor{Rizzi1997}'s ForceP, this function is independent of the sentential force of the clause. That these two functions can be expressed by independent items follows  from data such as  \eqref{ex:rizint}. In this example, the sentential force is directly encoded through the interrogative complementizer \emph{si}, located in IntP. Sentential force is therefore clearly separate from the subordinating function linked to \emph{que}, which occupies the highest projection of the left periphery, SubP. 


The second change I make is the introduction of a \isi{clause typing} head, not  at the left edge, but in the lower section of the left periphery. 
I adopt  (sentence) MoodP from \citet{Lohnstein2015} as the functional projection responsible for \isi{clause typing}.  Notably, the position of MoodP is identical to  the position in which \citet{Haegeman2006} relocates ForceP, which overlaps in its function. I use the term Mood instead of Force in order  to prevent the unintended conflation of this functional projection with a recent conception of ForceP which links it to illocutionary force (\citealt{SpeasTenny2003}, \citealt{Coniglio2010}, \citealt{Corr2016}, among many others). 


\is{cartographic approach|)}
\section{The analysis in a nutshell}


In this section, I provide a brief preview of the main points of the analysis, with theoretical and empirical support for this analysis outlined in the following chapters. There are three main assumptions that determine the analysis. First, that there is only one item \emph{que} in the Ibero-Romance lexicon. Second, that this element is underspecified in the sense outlined in \sectref{sec:intconcept}. And third, that the syntactic position in which the complementizer is merged  has an impact on its meaning. The three assumptions are not independent from each other. The first assumption is founded on the fact that there is no formal distinction between the instances of \emph{que} in the different constructions I investigate.  The theoretical solution I propose is that there is only one  underspecified lexical item. This means that different interpretations of \emph{que} are  not encoded lexically. They must nevertheless have an explanation. This is where  the third assumption comes in, which states that there is a relation between the syntactic structure and the resulting interpretation.  This book provides the  empirical evidence for these assumptions.  



The gist of the analysis is that the complementizer is valued with an interface feature in the position in which it is merged. This feature  has an impact on the interpretation of the proposition in the scope of the complementizer. In order to account for the data that are central to this book, I propose that there are two positions that the complementizer is externally merged in: one  at the lower edge of the left periphery (FinP) and one at the higher edge of the left periphery (SubP). Additionally, there is evidence for at least a third left-peripheral position (MoodP) where the complementizer can be externally merged in. Although complementizers merged in MoodP are not a central concern of this book, I briefly return to the core properties of  the construction involving a complementizer  merged in MoodP in \sectref{sec:insubwlp} and \sectref{sec:presupaffc}. A scheme of the relevant functional projections and their corresponding features is given in \figref{struc:general}.


The complementizer in \emph{que}-initial reportative examples like \eqref{ex:repgen} is analyzed as appearing in SubP, the highest projection of the left periphery, cf. \eqref{struc:repgen}.

\ea Spanish
\ea\label{ex:repgen}
\gll  Que Juan viene. \\
\textsc{que} Juan come.\textsc{3sg.prs}\\
\glt`[reportative] Juan is coming.'
\ex\label{struc:repgen} {\ob}\textsubscript{SubP} Que\textsubscript{subordinate} \dots {\ob}\textsubscript{IP} Juan viene. ]]
\z
\z

The functional head provides a subordinate feature. The consequence is that sentences introduced by a complementizer valued with this feature are interpreted as subordinate. This feature is not postulated specifically to account for the phenomenon at hand but is assumed to be present in all (finite)   subordinate sentences, cf. \eqref{ex:repembed} and \eqref{struc:repembed}.

\ea Spanish
\ea\label{ex:repembed}
\gll  María ha dicho que Juan viene. \\
	María \textsc{aux.3sg.prf.prs} say.\textsc{ptcp} \textsc{que} Juan come.\textsc{3sg.prs}\\
	\glt`María said that Juan is coming.'
	\ex\label{struc:repembed} {\ob}\textsubscript{SubP} \dots {\ob}\textsubscript{IP} María ha dicho {\ob}\textsubscript{SubP} que\textsubscript{subordinate} \dots {\ob}\textsubscript{IP} Juan viene. {\cb}{\cb}{\cb}{\cb}
\z
\z

While both  sentences are introduced by a complementizer carrying a subordinate feature, the difference between the unembedded sentence in \eqref{ex:repgen} and the embedded sentence in \eqref{ex:repembed}   is that the latter  is selected by a matrix clause while  the former  remains unselected. My claim is that sentences like \eqref{ex:repgen} receive a reportative interpretation because a verb of saying\is{verbum dicendi} can be pragmatically reconstruct\is{reconstruction}ed from the  context.\largerpage[-1]
 
\begin{figure}
  \caption{\label{struc:general}SubP and MoodP and their corresponding features}
\begin{forest}
	[SubP
	[~] 
	[Sub' 
	[Sub$^0_{\text{subordinate}}$, name=sub]  
	[\dots
	[,phantom]
	[MoodP
	[~] 
	[Mood'
	[Mood$^0_{\text{declarative/imperative/interrogative}}$, name=mood] 
	[FinP
	[~] 
	[Fin' 
	[Fin$^0_{\text{attributive}}$, name=fin] 
	[IP
		]]]]]]]]	
\end{forest}
\end{figure} 

The complementizer in the other root-complementizer constructions examined here is analyzed as being merged in FinP. One example and a sketch of the corresponding analysis are given in \eqref{ex:refgen} and \eqref{struc:refgen}.

\ea Spanish
\ea \label{ex:refgen} 
\gll Ciertamente que Juan viene. \\
	certainly \textsc{que} Juan come.\textsc{3sg.prs}\\
	\glt`Certainly, Juan is coming.'
	\ex\label{struc:refgen} {\ob}\textsubscript{SubP} \dots {\ob}\textsubscript{ModP} Ciertamente que\textsubscript{attributive} {\ob}\textsubscript{FinP} t {\ob}\textsubscript{IP} Juan viene. {\cb}{\cb}{\cb}{\cb}
\z
\z

FinP provides an attributive feature. Once again the feature has a consequence for the interpretation of the sentence. I propose that a commitment to the proposition in the scope of the attributive complementizer  is attributed to the hearer.  Again, this feature is not stipulated  solely  for the specific phenomenon under investigation here. It is influenced by \citet{Cuba2013} who assume a similar     feature which they call \emph{referential}, to account for the structural and interpretive difference between factive and non-factive complement clauses.

One final fact that requires further explanation is that attributive \emph{que} surfaces at different points in the functional field in the different constructions. The proposal I put forward  is that the surface positions are reached through head-to-head movement of the complementizer. The word order we observe in the constructions is predicted correctly if we assume that the movement of the complementizer is conditioned by  its inability to cross a specifier that contains  material that was externally merged in its current position.  
 
\section{Performative and neo-performative hypotheses}\label{sec:performativehyp}
\is{performative hypotheses|(}

In recent years, neo-performative accounts have   received considerable attention.  Despite the popularity of a neo-performative vision, this book does not subscribe to this approach.  This section outlines the central ideas of (neo)-performative hypotheses and some criticism from the literature.  

One shared approach of neo-performative hypotheses is that certain pragmatic aspects  are treated within syntax. The modern adaptations of this view of pragmatics are grounded in the classical performative hypothesis formulated in \citet{Ross1970}. Central to \citeauthor{Ross1970}’s hypothesis is that every sentence is a performative utterance,
 in which the illocutionary force is directly encoded in the deep structure (DS) component of the transformational grammar (see also \citealt{Katz1963}, \citealt{Sadock1969, Sadock1974}). The illocutionary force is expressed
by a performative verb that embeds the main clause. This performative verb is later deleted through \emph{performative deletion} yielding the surface structure (SS) that we observe.

\ea	
	\ea $[_{\text{DS}}$ I tell you that I read Ross 1970.$]$
\ex  $[_{\text{SS}}$ \sout{I tell you that} I read Ross 1970.$]$ (via \emph{performative} \emph{deletion})
\z
\z
\ea
	\ea$[_{\text{DS}}$ I ask you whether Q you have read Ross 1970.$]$
		\ex  $[_{\text{SS}}$ \sout{I ask you whether Q } Have you  read Ross 1970?$]$ \\(via \emph{performative} \emph{deletion} and \emph{subject} \emph{auxiliary} \emph{inversion})
	\z
\z

The motivation for a performative hypothesis is to make  a pragmatic theory of illocutionary force à la \citet{Austin1961} obsolete by pushing the burden onto syntax and truth-conditional semantics. The following paragraphs go through some of the arguments cited in favor of Ross’s hypothesis and contrast them with some of the criticism
formulated in \citet[246--276]{Levinson1983}. 

One of the arguments in favor of assuming a performative structure is the fact that
first-person \eqref{ex:rossb} and second-person \eqref{ex:rossd} reflexives are licensed in contexts where
their third-person counterparts are ungrammatical. Proponents of the performative hypotheses explain this contrast syntactically: Reflexives have to be bound by
 an antecedent in their local domain, cf.  \eqref{ex:rossa}, \eqref{ex:rossc}. They conclude that these data constitute evidence that the speaker and the addressee must be encoded syntactically within an implicit
performative clause.
\ea
\ea\label{ex:rossa} Tom believed that the paper had been written by Ann and himself.
\\
(\citealt[226: ex 11b]{Ross1970})
			\ex \label{ex:rossb} This paper was written by Ann and myself/him*self.\\
				(adapted from \citealt[228: ex 21a]{Ross1970})
			\ex \label{ex:rossc} Herbert told Susan that people like herself are rare. \\
			(\citealt[248: ex 33]{Levinson1983})
		\ex \label{ex:rossd} People like yourself/her*self are rare.	\\
		(adapted from \citealt[248: ex 34]{Levinson1983})
	\z
\z

\citet{Levinson1983} offers a different explanation. According to him, the licensing of speech-act-participant vs. non-speech-act-participant is a pragmatic rather than a
syntactic issue. He states that \emph{himself/herself} is only infelicitous at the beginning of a
 discourse but is felicitous in other contexts. An example of this can be found in \eqref{ex:zapp}.

\ea \label{ex:zapp}
He [Zapp] sat down at the desk and opened the drawers. In the top right-hand
one was an envelope addressed to himself. \\(\citealt[716: ex 65]{Zribi-Hertz1989})
\z

The reasoning behind the pragmatic explanation is that speaker and hearer are always active
and can be addressed, whereas third-person participants need to be salient in order for it to be possible to refer to them using a reflexive, cf. also  \sectref{sec:insubsemprop}. 

\begin{sloppypar}
Another argument typically raised in favor of a performative structure is speech
act adverbs like \emph{frankly} in \eqref{ex:franklya}. The idea is that they modify an implicit performative
verb.
\end{sloppypar}

\ea
	\ea \label{ex:franklya} Frankly, I don’t care.
	\ex \label{ex:franklyb} \sout{I tell you} frankly \sout{that} I don’t care.
	\z
\z

One problem for this argument is that speech act adverbs also appear in a syntactic location
where they cannot be trivially analyzed as modifying the high performative clause.
For instance, they can  modify certain types of embedded clauses as in \eqref{ex:levinsona}.
One attempt to rescue the performative hypothesis is to propose a second implicit
 performative clause preceding the embedded clause, which is then in turn modified
by the speech act adverb. This, however, derives the wrong meaning for the \emph{because}-clause: Clearly ``I tell you something because I tell you something else.'' is not the
intuitive meaning of \eqref{ex:levinsonb}.

\ea
\ea\label{ex:levinsona}  I voted for Labour because, frankly, I don’t trust the Conservatives.\\
		(\citealt[262: ex 85]{Levinson1983})
\ex \label{ex:levinsonb}  I tell you that I voted for Labour because I tell you frankly I don’t trust
the Conservatives. (\citealt[262: ex 86]{Levinson1983})
\z
\z


Another issue for the speech-act-adverb argument is the fact that some of them only appear with explicit performatives \eqref{ex:levinsonperfa}, hence the infelicity of  \eqref{ex:levinsonperfb}. This is not expected given
the proposal put forward by proponents of the classical performative hypothesis.

\ea
\ea[]{\label{ex:levinsonperfa}I hereby order you to polish your shoes. \\
		(\citealt[255: ex 53]{Levinson1983})}
		\ex[?]{\label{ex:levinsonperfb}Hereby polish your shoes. \\
		(\citealt[255: ex 54]{Levinson1983})}
	\z
\z

In some cases, the adverb does not appear to modify the relevant implicit performative. \eqref{ex:levinsonperf1a} is most adequately paraphrased by \eqref{ex:levinsonperf1c} rather than \eqref{ex:levinsonperf1b}. This is
an issue because the performative hypothesis assumes a one-to-one mapping between
illocutionary force and performative verb. Therefore, for questions like  \eqref{ex:levinsonperf1a}, the
performative verb should be of asking \eqref{ex:levinsonperf1b} rather than of answering \eqref{ex:levinsonperf1c}.
 
\ea
\ea \label{ex:levinsonperf1a} Briefly, who do you think will win the gold medal? \\
(\citealt[256: ex 60]{Levinson1983})
\ex \label{ex:levinsonperf1b} I ask you briefly, who do you think will win the gold medal?
 \\
(\citealt[256: ex 61]{Levinson1983})
\ex \label{ex:levinsonperf1c} Tell me briefly, who do you think will win the gold medal?
 \\
(\citealt[256: ex 62]{Levinson1983})
	\z
	\z

One core problem of the performative hypothesis is that it predicts that \eqref{ex:flatworlda} and
\eqref{ex:flatworldb} have the same truth conditions.

\ea
\ea \label{ex:flatworlda} The world is flat. (\citealt[252: ex 42]{Levinson1983})
	\ex \label{ex:flatworldb}  I stated to you that the world is flat. (\citealt[252: ex 43]{Levinson1983})
	\z
\z

This does not do justice to the intuition that \eqref{ex:flatworlda} is a false statement about the
actual round world we are inhabiting. In contrast, \eqref{ex:flatworldb} is true if in fact I made this
 statement. This is irrespective of whether the world is flat or not (cf. \sectref{sec:insubsemprop} where
this fact is picked up again).

Another fundamental issue relates to the assumed direct mapping from clause type
to illocutionary force via a performative verb. This results in problems when dealing with indirect\is{indirect speech} speech acts. For
instance, the interrogative in \eqref{ex:homework} can be interpreted as a question, a command or
even a threat. 
\ea\label{ex:homework} Will you do your homework?
\z

Finally, assuming an implicit performative syntactic structure that is interpreted semantically makes the prediction that every sentence should be assigned a truth value.
However, not all meaningful sentences express statements that are either true or false. Obviously, questions, commands and exclamatives cannot be evaluated
in this way.

These points -- merely a selection was presented here, but for further details see \citet{Levinson1983} -- show that a classical performative hypothesis faces a number of serious issues. One attempt at rescuing the insights from \citet{Ross1970} is in the
recent developments of neo-performative hypotheses. In these, the classical implicit
performative verb is replaced by abstract functional categories.\footnote{But see the light performative hypothesis of \citet{Alcazar2014}, which returns to a more
	classical version of the performative hypothesis with the difference that the implicit performative clause
	does not contain a lexical verb (\citealt{Ross1970})  but a functional light verb.}
Neo-performative
 hypotheses  encode aspects of pragmatics such as illocutionary forces in syntax without treating them as deleted performative clauses. In the neo-performative hypotheses, the performative structure is a part of the  architecture of the
clause envisioned as another functional domain above the CP (cf. for instance the contributions made by \citealt{Beninca2001}, \citealt{Garzonio2004},  \citealt{Hill2006, Hill2007a, Hill2007b}, \citealt{SpeasTenny2003}, \citealt{Speas2004}, \citealt{Tenny2006}, \citealt{Poletto2003}, \citealt{Zanuttini2008}, \citealt{Zanuttini2012}, \citealt{Krifka2013}, \citealt{Haegeman2014}, \citealt{Wiltschko2014}).  

One very influential proposal is by \citet{SpeasTenny2003}, who assume that pragmatic roles are encoded syntactically. They treat declarative, interrogative,
imperative, subjunctive and quotative  as the universal types of speech acts. In their system, these speech acts are modeled via different configurations of the
pragmatic roles and the utterance content by following universal syntactic principles.

\citet{SpeasTenny2003} restrict the scope of their analysis to direct illocutionary forces,
leaving aside the complications brought by indirect illocutionary forces as illustrated in  example
\eqref{ex:homework}. Perhaps because they only focus on cases in which the clause type is mapped directly to an illocutionary force, they do not draw a terminological
distinction between the clause types and their corresponding illocutionary forces. I
follow this terminological imprecision when illustrating their proposal here. \citet{SpeasTenny2003}  postulate an enriched revision of \citeauthor{Rizzi1997}'s  ForceP in the form of two
projections above the CP called Speech Act Phrase (SAP) and Sentience Phrase (SenP).
The SAP has three arguments: the pragmatic  roles  \emph{speaker} and \emph{hearer},  and the
 \emph{utterance content}. The structure is illustrated in \figref{struc:speastennya}. The structure of the SAP and
the SenP are parallel to the vP shell. According to the authors, the lower projections
can furthermore be iterated, which is indicated by the asterisk in the structures.

\begin{figure}\small
\begin{floatrow}\captionsetup{margin=.05\linewidth}
\ffigbox[.35\textwidth]
{\begin{forest}
	[SAP 
	[\emph{speaker}]
	[SA' 
	[SA]
	[SA*P
	[\emph{utterance}\\\emph{content}]
	[SA*'
	[SA*]
	[\emph{hearer}]] 	
	]]]
\end{forest}}
{\caption{\label{struc:speastennya}The structure of SAP (\citealt[320: ex 9]{SpeasTenny2003}) }}
\ffigbox[.65\textwidth]
{\begin{forest}
	[{EvalP (SenP)}
	[\emph{seat of knowledge}]
	[{Eval' (Sen')} 
	[{Eval (Sen)} ]
	[{EvidP (Sen*P)}
	[\emph{evidence}]
	[{Evid' (Sen*')}
	[{Evid (Sen*)}]
	[{S (episP)}]] 	
	]]]
\end{forest}}
{\caption{\label{struc:speastennyb}The structure of EvalP/SenP (\citealt[334: ex 34]{SpeasTenny2003})}}
\end{floatrow}
\end{figure} 

The third pragmatic role represented syntactically in \citet{SpeasTenny2003} is the \emph{seat of knowledge} that encodes epistemic authority and evaluation of truth. It is located in
SenP, the upper structural layer of the utterance content in the scope of
SAP, cf. \figref{struc:speastennyb}. It consists of an EvaluativeP hosting \emph{seat of knowledge} and EvidentialP
hosting \emph{evidence}. Both projections are adopted from \citet{Cinque1990}.

While in the classical performative hypothesis the illocutionary force is encoded directly
through the semantic content of the performative verb, the head of the SAP is not
 considered to be a proper verb. Instead, in \citet{SpeasTenny2003}, different illocutionary forces are derived syntactically through the interplay of two parameters. The first is a feature that marks
the utterance content as + or −finite. The second parameter concerns the interaction of the pragmatic roles of \emph{speaker}, \emph{hearer} and  \emph{seat of knowledge}. In this model,  \emph{seat of knowledge} is controlled by the closest c-commanding pragmatic role. In the default configuration, this role is the \emph{speaker}. In a declarative clause, therefore, \emph{speaker} and \emph{seat of knowledge} coincide. This means  that the speaker takes the epistemic
authority and evaluates the truth of the utterance content. The basic structure in \figref{struc:speastennya}
represents the configuration of a declarative.

Questions are derived through a movement operation in which the \emph{hearer} is promoted to
the specifier of the iterated lower SAP, cf. \figref{struc:sapinterr}. From this position the \emph{hearer} controls
the \emph{seat of knowledge} and takes up epistemic authority.

A similar configuration is proposed for imperatives illustrated in \figref{struc:sapimp}. The \emph{hearer} also controls the \emph{seat of knowledge} in this case. In order to achieve this,  \emph{hearer} is once again promoted to a higher
position. The structure of imperatives differs from interrogatives in that the utterance
content carries a −finite feature. Furthermore the orientation of the SA* projection
dominating the \emph{utterance content} is reversed in imperatives. However, no motivation for this  is
found in \citet{SpeasTenny2003}.

The −finite equivalent of the declarative structure in \figref{struc:speastennya}, illustrated in \figref{struc:sapsubj}, is
the analysis \citet{SpeasTenny2003} assume for subjunctives.

\begin{figure}
\begin{floatrow}
\captionsetup{margin=.05\linewidth}
\ffigbox
	{\begin{forest}
		[SAP 
		[\emph{speaker}]
		[SA 
		[SA]
		[SA*
		[\emph{hearer}, name=hearer]
		[SA*
		[\emph{uc}\\\emph{+finite}]
		[SA*
		[SA*]
		[t, name=t]] 
				{
			\draw[->] (t)  to[out=south west, in=south west] (hearer);
		} 	
		]]]]
	\end{forest}}
	{\caption{Interrogative (\citealt[321: ex 10]{SpeasTenny2003}) \label{struc:sapinterr}}}

\ffigbox
	{\begin{forest}
		[SAP 
		[\emph{speaker}]
		[SA 
		[SA]
		[SA*
		[\emph{hearer}, name=hearer]
		[SA*
		[SA*
		[SA*]
		[t, name=t]] 
		[\emph{uc}\\\emph{−finite}] 
		{
			\draw[->] (t)  to[out= south west, in=south west] (hearer);
		} 	
		]]]]
	\end{forest}}
	{\caption{\label{struc:sapimp}Imperative (\citealt[322: ex 11]{SpeasTenny2003})}}
\end{floatrow}\medskip
\begin{floatrow}
\captionsetup{margin=.05\linewidth}
\ffigbox
	{\begin{forest}
		[SAP 
		[\emph{speaker}]
		[SA 
		[SA]
		[SA*
		[\emph{uc}\\\emph{−finite}]
		[SA*
		[SA]
		[\emph{hearer}]] 
		]]]
	\end{forest}}
	{\caption{\label{struc:sapsubj}Subjunctive (\cite[323: ex 13]{SpeasTenny2003})}}


\ffigbox
	{\begin{forest}
		[SAP 
		[\emph{expletive}]
		[SA 
		[SA]
		[SA*
		[\emph{uc}\\\emph{+finite}]
		[SA*
		[SA]
		[\emph{hearer}]] 
		]]]
	\end{forest}}
	{\caption{\label{struc:sapalcazar}Quotative (\citealt[97: ex 24]{Alcazar2014})}}
\end{floatrow}
\end{figure}

Finally, quotatives are treated as declaratives in which the speaker is absent but is replaced by an expletive subject.  \citet{SpeasTenny2003} do not offer a structure for this
configuration in their article, but in their review of the proposal, \citet{Alcazar2014} do, cf. \figref{struc:sapalcazar}.



Although I do not adopt the framework of  \citet{SpeasTenny2003} for my analysis,  it could nonetheless be useful for the analysis of
reportative \emph{que} constructions that are the topic of \chapref{sec:insubint}. However, additional stipulations are required. For one thing, marking the \emph{speaker} as an expletive might run
into problems since  the \emph{que}-initial reportative construction, for instance, also permits
self-reports  in which the actual speaker does in fact coincide with the speaker
of the report.

\citeauthor{SpeasTenny2003}'s proposal, as well a similar proposal by \citet{Haegeman2014},   has been widely adopted
by authors working on  phenomena related to the interface between syntax and pragmatics. The proposal,
however,  faces a number of issues. \citet{Gaertner2006}  are critical of \citeauthor{SpeasTenny2003}'s claim that the structure they put forward in \figref{struc:speastennya} and its derivations in Figures~\ref{struc:sapinterr}--\ref{struc:sapalcazar} are universal; they also  point out that there are other structures
derivable by universal syntactic principles that are disregarded by the authors without
a convincing explanation for their omission. The issue is not only of a theoretical nature: These other
structures, \citet{Gaertner2006} argue, would give rise to different illocutionary
forces that are either not universal or do not exist at all.

Further problems are identified by \citet{Alcazar2014}. They take issue with the fact that only \emph{speaker}, \emph{hearer} and \emph{seat of knowledge} are treated as indexicals.
\emph{Speech location} and \emph{speech time} are absent in \citeauthor{SpeasTenny2003}'s framework. It should be noted, however, that
in later adaptations of \citet{SpeasTenny2003}, this is accommodated. For instance, \citeauthor{Corr2016} (\citeyear[193--194]{Corr2016})  assumes with \citet{Sigurdsson2010} that location and time are encoded
in the syntactic structure. She proposes that EvidP hosts locative features, and EvalP hosts
speech time features (see  \sectref{sec:insubexistan} and \sectref{sec:presupeval} for a more detailed discussion of \citealt{Corr2016}).


Another aspect criticized by \citet{Alcazar2014} is that the two parameters assumed in \citet{SpeasTenny2003} are
not sufficient to differentiate between different subtypes of certain illocutionary forces.
They argue that the configuration proposed for questions, where the \emph{hearer} controls the \emph{seat
of knowledge}, only works for genuine but not for rhetorical questions, where one would
assume that the \emph{speaker} controls the \emph{seat of knowledge}. In their defense, it  should be noted though that  rhetorical questions fall into the category of utterances with an indirect illocutionary force and
therefore do not fall into the scope of \citeauthor{SpeasTenny2003}'s analysis.

The main criticism put forward by \citet{Alcazar2014} is that subjunctives and quotatives
are postulated as universal clause types. They conjecture that this happens purely out of a theoretical necessity, because the
two parameters result in four possible configurations. The typological literature (for instance \citealt{Sadock1985}, \citealt{Koenig2007}), however, does not support this postulation. In fact,
there is only agreement on the existence of three universal clause types: declaratives,
interrogatives and imperatives.

\is{performative hypotheses|)}
\section{Organization of the book}

This book is structured as follows: \chapref{sec:insubint} focuses on \emph{que}-initial reportative sentences such as those illustrated in \eqref{ex:quereprep}, in which \emph{que} is merged in the top left projection of the split CP and is valued with a subordinate feature. In \sectref{sec:insubexistan}, I  discuss the three main analyses advocated in the literature by  \citet{Etxepare2007,Etxepare2010,Etxepare2013,DemonteSoriano2014} and \citet{Corr2016}.  In  \sectref{sec:insubanalysis}, I present  my own analysis and compare it to the previous approaches, while in \sectref{sec:insubsynprop}, I provide the empirical support for my analysis by focusing on the syntactic properties of the construction. In \sectref{sec:insubcross}, I discuss some cross-linguistic differences and show that while  the basic underlying syntactic principles are the same in all three languages, there is a pragmatic difference between \emph{que}-initial reportatives in Portuguese on the one hand and Spanish and Catalan  on the other hand. \sectref{sec:insubsemprop} focuses on the pragmatic requirements to felicitously utter a \emph{que}-initial reportative. The chapter concludes with \sectref{sec:beyondrep},  in which I discuss how the analysis can be extended to non-reportative \emph{que}-initial sentences. 

\chapref{sec:presupint} focuses on  constructions like (\ref{ex:advque}--\ref{ex:sique}), in which \emph{que} is merged at the right edge of the left periphery and  receives an attributive value. In \sectref{sec:presupeval}, I introduce the  main analyses presented in the literature for the different constructions by \citet{Ambar2003,Castroviejo2006,Hernanz2007,PrietoRigau2007,DemonteSoriano2009,Corr2016} and \citet{Cruschina2018}. \sectref{sec:presupanal}  describes the details of my own proposal for the constructions and compares it to the previous analyses. The syntactic properties that support my analysis are explored in depth  in \sectref{sec:presupsyn}, in which I also discuss some cross-lin\-guis\-tic differences and offer explanations for the contrasts. The last section, \sectref{sec:presupprag}, is dedicated to the pragmatical properties of the constructions.

\chapref{sec:experiments} presents  empirical studies that further explore the pragmatic contribution of attributive \emph{que} in two constructions. The chapter begins with the methodological and statistical background laid out in   \sectref{sec:expcorp} and \sectref{sec:expstats}. In \sectref{sec:expdeic}, I discuss three of my studies, focusing  on the interpretation of attributive \emph{que} following \isi{epistemic and evidential modifier}s. In \sectref{sec:expbias}, I present an experimental study investigating attributive \emph{que} in Catalan polar questions. In the final section, \sectref{sec:empgeneraldiss}, I reflect on the usefulness of exploratory empirical methods in generative linguistics. Finally, \chapref{sec:conclusion} concludes this book with a summary of the  main points and results.
