\chapter{Conclusion}\label{sec:conclusion}

This monograph set out to investigate the properties of syntactically unselected complementizers in  Ibero-Romance. The aim was to show that, despite their atypical behavior, it is possible to maintain the idea that there is only one lexical  item \emph{que} in these languages if we adopt a new notion of complementizer. I proposed a definition that treats the complementizer as an underspecified item that adopts its functional interpretation in the syntactic position in which it is merged.
Over the course of the book,  I have shown that there is a correlation between the syntactic position, the behavior of the complementizer and the interpretation of the proposition in its scope. This has ultimately demonstrated that this new conception of complementizer is indeed adequate. The analysis I proposed to account for the central empirical phenomena  was that there are two merge positions in the left periphery that provide different interface features as values for the underspecified complementizer. The position at the left edge of the periphery provides a subordinate feature that results in a subordinate interpretation of the sentence. The lowest position in the left periphery provides an attributive feature that results in the interpretation that  a commitment to the proposition is ascribed to the hearer.

The overarching question that I addressed  was how the boundary between syntax and pragmatics is organized in Ibero-Romance. In the chapters of this book I argued in favor of a distribution of labor between these two components of grammar and against versions of neo-performative hypotheses that propose that  pragmatics should be treated in syntax. The motivation behind this is my conviction that only aspects that are demonstrably impacted by structural factors should also be modeled syntactically.
When no syntactic factors appear to be active,  pragmatics must kick in.  This view grants greater autonomy  to each of the grammatical components and was   supported empirically in this book. 

\chapref{sec:insubint} was dedicated to  \emph{que}-initial reportative sentences. The contribution of syntax and pragmatics was shown to be clearly divided. The information read off  the syntactic structure is that the sentence is to be interpreted as  subordinate. What it is  subordinate to depends on the context and is therefore something   contributed by pragmatics. In the chapter, I drew a parallel between \emph{que}-initial reportatives and embedded sentences and showed that the syntactic behavior of the complementizer in both cases is virtually the same.  The empirical distribution is in favor of an  analysis of the complementizer as being merged in the highest position in the left periphery, valued  with a subordinate feature. The difference between the complementizer in  embedded and unembedded contexts is merely that it is syntactically selected in the first case and remains unselected in the latter.  I furthermore proposed that the reportative interpretation results from reconstruction. However, since the requirements for syntactic reconstruction are not met, I argued that the  type of reconstruction we are dealing with in these cases must be pragmatic. Finally, on the basis of the empirical data I put forward the generalization that \emph{que}-initial reportatives require a salient verb of saying in the context. Portuguese \emph{que}-initial reportatives have a stronger requirement than  Spanish or Catalan: It is not sufficient for a verb of saying to be salient, it must be given. 


\chapref{sec:presupint} dealt with a variety of different constructions. In my analysis, all of them involve a low merged complementizer that is valued with an attributive feature. The contributions of syntax and pragmatics were again shown to be well within their domains: Syntax provides an attributive feature but the consequences for the interpretation of the  sentences valued in this way, as well as the interaction with other types of meanings, are dealt with in pragmatics. I developed a  unified analysis to account for a range of different constructions. This was motivated by the observation that they share a core interpretation which I ascribe to the presence of the attributive feature.  I proposed that the  feature is linked to the lowest projection of the left periphery. Evidence for this assumption was taken from the fact that in certain embedded contexts a low merged complementizer gives rise to a similar interpretation of the proposition in its scope. I showed that the empirical distribution of the complementizer in the different constructions  can be  correctly predicted by assuming that the low merged complementizer moves from head to head through the left periphery and only stops if the specifier of the relevant projection is filled by externally merged material. I furthermore showed that while the discourse contribution of attributive \emph{que}, which  ascribes a commitment to the proposition to the hearer, is always present, further interpretive effects that arise  can be explained on the basis of the sentence type and the interaction with other expressions involved in the particular construction. 

\begin{sloppypar}
\chapref{sec:experiments} presented  corpus-based empirical investigations that aimed to broaden the understanding of the meaning of some of the constructions involving attributive \emph{que}. The approach  was  exploratory rather than  hypotheses-driven. This  means that, while the results can tell us about the adequacy of the theoretical analysis, above all they serve to uncover new patterns and relations that would otherwise have remained undetected. In the case of the empirical investigations focusing on attributive \emph{que} with epistemic and evidential  modifiers, one novel insight is that there is a mismatch between production and  comprehension. The corpus study suggested that each construction gives rise to a preferred reading of the modifiers. The results of the experiments indicate that speakers are willing to accommodate different readings of the modifiers even if  they do not correspond to those made salient by the relevant constructions. With regard to the experiment on Catalan polar questions, the results  provide  evidence in favor of the analysis proposed. The participants judged negative answers particularly unexpected in the contexts that originally contained \emph{que}-initial polar questions. Consequently,  a positive answer was likely expected in these contexts. The exploratory approach made it possible to identify further interesting relations. Two findings were particularly  revealing.  First, the results suggest that speakers are generally biased toward a positive answer,  irrespective of the presence of  \emph{que} and other potential bias markers. Second,  a bias in polar questions is expressed simultaneously by multiple means, and  goes far beyond the mere presence or absence of \emph{que}.
\end{sloppypar}
