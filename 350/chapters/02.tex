\judgewidth{\#}
\chapter{The syntax and pragmatics of reportatives}\label{sec:insubint}
This chapter deals with the phenomenon  for which the term in\isi{subordination} was coined in functional literature  (\citealt{Evans2007}), namely   embedded sentences that appear without a matrix clause. The  empirical focus of this chapter is \emph{que}-initial sentences that receive a  reportative interpretation. 

The example in \eqref{ex:gras6} illustrates a typical case of what my analysis refers to as reportative \emph{que}. In this example speaker G introduces a sentence with \emph{que} to mark that he reiterates (part of) his previous utterance which the other speech participant, L, appears to not have fully understood or heard. In the discussion of this example, \citet{Gras2016} says that the omission of \emph{que} in this context, while not ungrammatical, would appear strange to the ears of a European Spanish speaker.

\ea Spanish (\citealt[119: ex 6]{Gras2016}) \label{ex:gras6} \\
$[$Context: Three friends talking about the route to pick
up a fourth friend.$]$

\exi{}  \gll G: (bue)no ¿y ahora por dónde nos vamos a ir?
\\
 {} well and now for where us go.\textsc{1pl.prs} to go\\
\exi{}\gll  L: ¿adónde?
\\
{} where\\
\exi{} \gll G: ¿que por dónde nos vamos ir?
\\
{} \textsc{que} for where \textsc{cl.refl} go.\textsc{1pl.prs} go
\\
\glt `G: Well and now which way should we take? L: Where? G: I said which way should we take?'\\
(Val.Es.Co.\footnote{Val.Es.Co. is a corpus of Spanish colloquial conversations \citep{Briz2002}.} L.15.A.2: 103–107.)
\z 


The phenomenon of reportative \emph{que} is at the heart of the question regarding how the boundaries between syntax and pragmatics are organized. While most previous accounts opted  for a syntactic explanation when deriving the reportative interpretation,  the central argument I develop here is that the reportative interpretation results from pragmatic rather than syntactic reconstruct\is{reconstruction}ion. Strong empirical support for this claim comes from the fact that the requirements for syntactic reconstruct\is{reconstruction}ion are not met in the contexts where \emph{que}-initial reportatives are grammatical. However, there is a pragmatic condition that is found in all the contexts, namely that a verb of saying\is{verbum dicendi} needs to be salient in order to felicitously utter a \emph{que}-initial reportative.

In the generative syntactic literature, these \emph{que}-initial sentences have sometimes been called \emph{quotatives};\is{quotation} I use the term \emph{reportative} instead,  because they do not behave like prototypical direct quotations that repeat an expression verbatim (cf. \citealt[2]{Coulmas1986}).\footnote{There is  discussion in the literature on different types of quotatives and  to what extent  the quoted expression must be matched verbatim (cf. \citealt{Davidson1968,Cappelen1997,Cappelen2003,Cappelen2005,Maldonada1999,Abbott2003}, and references therein).} The example in \eqref{ex:mentida} illustrates that the reportative sentence does not consist of the same words as the original sentence: In the reportative version, the  speaker paraphrases and attenuates her initial statement.

\ea\label{ex:mentida} Catalan\\ 
\gll A: Mentida. \\
	{} lie\\
	\exi{} \gll B: Què dius?\\
	{} what say.\textsc{2sg.prs}\\
	\exi{}\gll A: Perdoni. Que no hi {estic d’acord}.\\
	{} excuse.\textsc{2sg.imp} \textsc{que} not \textsc{cl.loc} agree.\textsc{1sg.prs} \\
	\glt `A: That's a lie. B: What did you say? A: Sorry. [reportative:] I don't agree.' (ebook-cat)\footnote{ebook-cat: a small self-compiled e-book corpus (400,000 tokens).}
\z

Another property that distinguishes  \emph{que}-initial reportatives from \isi{quotation}s is that deictic expressions  typically undergo an origo switch. In direct quotations, the speaker that quotes an expression adopts the original speaker's perspective (cf. \citealt[2]{Coulmas1986}). The origo switch that takes place in \emph{que}-initial reportatives indicates  that the \isi{deictic center} is transferred from the original speaker to the external speaker who reports the sentence.  The original sentence in \eqref{ex:origoswa}, in which \emph{Juan} is the \isi{deictic center}, is reported by the original hearer \emph{Maria} in \eqref{ex:origoswb}. One result of the reporting is the change in the clitic pronoun from second to first person and the change in verbal morphology from first to third person. It is these adaptations, which are typical for reported but not quoted speech, that have led me to choose the term \textit{reportative} for these constructions.\largerpage[-1]

\ea\label{ex:origosw} Catalan
\ea\label{ex:origoswa} 
\gll {\ob}Juan$_j$:] T'$_m$ espero$_j$ a la porta. \\
		Juan \textsc{cl.2sg} wait.\textsc{1sg.prs} at the door\\
		\glt `{\ob}Juan:] I will wait for you at the door.'
		\ex\label{ex:origoswb}
		\gll {\ob}Maria$_m$:] Que m'$_m$ espera$_j$ a la porta. \\ 
		Maria \textsc{que} \textsc{cl.1sg} wait.\textsc{3sg.prs} at the door\\
		\glt `{\ob}Maria:] [reportative:] He will wait for me at the door.'
	\z
\z



This chapter is structured as follows. In \sectref{sec:insubexistan}, I present the previous analyses of this construction  by \citet{DemonteSoriano2014}, \citet{Etxepare2007,Etxepare2010,Etxepare2013} and \citet{Corr2016}.  In \sectref{sec:insubanalysis}, I introduce my own proposal, the gist of which is that the complementizer is analyzed as being merged in SubP, the highest projection of the split CP, where it is  valued with the feature \emph{subordinate}. This is conceived of as an interface feature that primarily has  consequences for the interpretation of the  sentences valued with this feature.  My principal claim is that the CP of \emph{que}-initial sentences does not differ from its embedded counterparts. In \sectref{sec:insubanalysis}, I  compare my own proposal to the previous accounts developed in the literature. One of the main differences is that my analysis does not allude to a hidden syntactic structure or elide\is{ellipsis}d material. I show that a simple and transparent structure is possible if we assume that a complementizer-initial sentence can remain syntactically unselected.   Support for the present analysis is provided in \sectref{sec:insubsynprop},  which shows  that the complementizer in \emph{que}-initial sentences surfaces in the same syntactic position as in their embedded, i.e. selected, counterparts.  In \sectref{sec:insubcross}, I show that the apparent  syntactic  differences between \emph{que}-initial reportatives in Portuguese on the one hand and  Spanish and Catalan  on the other are not related to the phenomenon under discussion but are a reflex of a more general difference. This leads me to conclude that \emph{que}-initial reportatives in all three languages can be treated with the same basic syntactic analysis. There is, however, a pragmatic difference:  In Spanish and Catalan  \emph{que}-initial reportatives are felicitous if a host expression is  salient in the general context; this means that the host expression  is accessible or activated in some way. Crucially,  the expression can have been, but does not have to have been, explicitly mentioned in the linguistic context. In Portuguese, however, the expression must be given, i.e. mentioned, in the linguistic context. In \sectref{sec:insubsemprop}, I draw up a unified characterization, by proposing that the reportative interpretation results from pragmatic rather than syntactic reconstruct\is{reconstruction}ion.  Finally, this chapter focuses primarily on \emph{que}-initial reportatives; in \sectref{sec:beyondrep} I propose that in principle the analysis can be extended to
account for other types of unembedded sentences.

\section{Previous analyses}\label{sec:insubexistan}\largerpage[2]

\begin{sloppypar}
In this section I summarize how \emph{que}-initial reportative sentences have been treated in the generative syntactic literature.\footnote{There is also considerable discussion of the phenomenon in Spanish by authors relying on a functional grammar framework (see for instance \citealt{Ballesteros2000}, \citealt{PonsBorderia2003},  \citealt{Gras2010, Gras2016},  \citealt{Sansinenaetal2015}).}  I focus on the analyses presented in \citet{Corr2016}, \citet{Etxepare2007,Etxepare2010,Etxepare2013} and \citet{DemonteSoriano2014}. The three analyses  differ in their empirical coverage: \citet{Etxepare2013} and \citet{DemonteSoriano2014} only look at Spanish data, while \citeauthor{Corr2016}'s analysis extends to other standard varieties and dialects of Ibero-Romance  and also accounts for the cross-linguistic variation. All three analyses treat \emph{que} as a complementizer. But the authors each assume that it occupies a different  syntactic position and fulfills a different function. It is treated as the head of ForceP in \citet{DemonteSoriano2014}, the head of a LinkerP in \citet{Etxepare2013} and the head of EvidP, one of the subheads of her split ForceP,  in \citet{Corr2016}. The biggest difference is how each analysis accounts for the reportative interpretation. \citet{Etxepare2013} and \citet{DemonteSoriano2014} propose silent verbs/nouns of saying. \citet{Corr2016} presents a neo-performative hypothesis  in which pragmatic roles and functions are encoded in a syntactic layer above the CP (for details on performative and neo-performative hypotheses see \sectref{sec:performativehyp}). The reportative interpretation does not arise through the meaning of an  elide\is{ellipsis}d verb or noun of saying, but is encoded as a feature on the functional head in which the complementizer is merged. 
\end{sloppypar}

\subsection{\citet{Etxepare2013} and \citet{DemonteSoriano2014}}


\citet{Etxepare2013}, building on his previous work (\citeyear{Etxepare2007,Etxepare2010}), proposes that re\-por\-ta\-tive-com\-ple\-men\-ti\-zer constructions are the visible part of a larger structure. The whole structure is a small clause in which an elide\is{ellipsis}d event noun of saying functions as the predicate.  The author assumes a silent noun of saying -- the silent equivalent of \textsc{a saying} -- rather than a silent verb of saying\is{verbum dicendi} because coordinated reported sentences can trigger plural  agreement, cf. \eqref{ex:etxnumber}. In his theory,  number is restricted to nominals, so  he concludes that the silent predicate has to be a noun.

\ea\label{ex:etxnumber} Spanish \\\gll Que la lasaña estaba buena y que el vino estaba extraordinario resonaron en todo el restaurante\\
\textsc{que} the lasagne be.\textsc{3sg.ipfv.pst} good and \textsc{que} the wine was great resound.\textsc{3pl.prf.pst} in all the restaurant\\
\glt `The saying that the lasagne was good and a saying that the vine was great
resounded in the whole restaurant.' (\citealt[620: ex 59]{Etxepare2010})
\z

The sentence introduced by \emph{que} is analyzed as a ForceP that is the subject of the small clause. \emph{Que} functions as the linker occupying the head of the dedicated LinkerP dominating the ForceP. Predicate inversion obtains and leaves the predicate noun in the specifier of LinkerP, see \eqref{ex:etxepareanalysis}.

\ea\label{ex:etxepareanalysis} {\ob}\textsubscript{LinkerP} {\ob} A SAYING {\cb} {\ob}\textsubscript{Linker} \emph{que} {\cb} {\ob}\textsubscript{ForceP} {\ob} \dots {\cb}{\cb}{\cb} \\(adapted from \citealt[98: ex 14]{Etxepare2013}, details omitted)
\z



\citet{DemonteSoriano2014}  revisit \citeauthor{Etxepare2007}'s data and identify two distinct types of \emph{que}.  The first type is a proper complementizer selected by a silent verb of saying\is{verbum dicendi}, whose  properties mostly overlap with \citeauthor{Etxepare2007}'s description. The second type, however, is   not  a complementizer  in \citeauthor{DemonteSoriano2014}'s view, but a homophonous reportative evidential marker. 

I will now examine the core data from \citeauthor{DemonteSoriano2014}'s study  and propose that it is still possible to  maintain the position that there is only one type of \emph{que}: the complementizer. The second type, which \citet{DemonteSoriano2014} posit as  an evidential marker, can be interpreted as a version of the former with  different pragmatic restrictions because it is employed discourse-initially\is{out-of-the-blue context}. In the following paragraphs, I will call this version of \emph{que}-initial reportatives out-of-the-blue\is{out-of-the-blue context} reportatives.

In what follows, I will show that  the particular behavior that \citet{DemonteSoriano2014} observe and that motivates their postulation of a second type, can be explained by the context in which these \emph{que}-initial utterances are found. They are discussed as cases of  out-of-the-blue\is{out-of-the-blue context} reportatives, meaning they appear in a context where there is no previous utterance that the \emph{que}-initial reportative refers to.  In general, in order for an utterance to work at the beginning of a conversation\is{out-of-the-blue context}, it must be possible for the  addressee to  accommodate  the information that is not explicitly stated. What a speaker can assume her hearer to accommodate depends on the shared \isi{common ground}  of the speech participants. Intuitively, more general information that can be assumed to be cultural, universal or otherwise common knowledge,  is expected to  be accommodated more easily than specific and context-dependent information.

 While  different types of clauses are allowed in a context in which the reported utterance is salient, \citet{DemonteSoriano2014} show that out-of-the-blue\is{out-of-the-blue context} reportatives are restricted to declaratives. Starting a conversation with a reportative  always requires some guess-work. A cooperative hearer is usually prepared to accommodate  absent information when he is faced with an assertion, but making sense of a reported non-declarative clause is more difficult.
 
\ea\label{ex:demonteclase} Spanish \\ 
\gll  (Oye), que ma\~nana no hay clase.\\
	  listen.\textsc{2sg.imp} that tomorrow not there.be.\textsc{3sg.prs} class\\
\glt `Listen, there will be no class tomorrow (someone said\slash I just heard).' (\citealt[16: ex 12a]{DemonteSoriano2014})
\z
 
 To accept an out-of-the-blue\is{out-of-the-blue context} reportative like \eqref{ex:demonteclase}, the hearer needs to accommodate that the speaker heard the statement that there will be no class the next day. The hearer can accommodate this reportative statement. He might  conjecture that the source of the utterance is irrelevant. Based on shared knowledge, he might also conjecture who the source could have been. For instance, the hearer could conjecture that the speaker has talked to a classmate or the professor who informed her.
 
 The situation is trickier with reported questions, imperatives and exclamatives. Even if not reportative, they must follow a number of requirements in order to be felicitously uttered in an out-of-the-blue context. Questions are typical conversation starters; but  in true out-of-the-blue contexts only very general questions like  \emph{What's new?} or \emph{How are you?} are felicitous. Questions like \emph{How's your mum doing?}, that have a similar function, namely to initiate a conversation, are arguably not out-of-the-blue\is{out-of-the-blue context} as they refer to previous knowledge of the hearer's mum's well-being or health.  In any case, starting a conversation with \emph{Hey, someone asked what's new?} is odd.
 
\ea Spanish (\citealt[18, ex 16b]{DemonteSoriano2014})\label{ex:demonteganado}\\
\gll  \# Oye, ¿que hemos ganado la liga?\\
      \phantom{\#} listen.\textsc{2sg.imp} that \textsc{aux.1pl.prf.prs} win.\textsc{ptcp} the league\\
\glt  \phantom{\#} Intended: `Listen, have we won the league (I just heard)?'
\z
 
A question like \emph{Have we won the league?}  requires a very specific type of \isi{common ground} in order to be uttered felicitously at the beginning of a conversation\is{out-of-the-blue context}: One has to imagine a setting in which the games of the specific league that is referred to are salient to such an extent that they can be talked about without further contextualization.  A reported version of this question exemplified in \eqref{ex:demonteganado} is  judged infelicitous according to  \citet{DemonteSoriano2014}. A hearer confronted with an out-of-the-blue\is{out-of-the-blue context} question like \eqref{ex:demonteganado} would very likely be puzzled and uncertain about what is expected of him. \emph{Que}-initial questions do not have the illocutionary force of a question but rather that of an assertion. This means that the speaker does not require an answer of the hearer. The hearer must then assume that the speaker intends to convey information with her statement. However, retrieving information out of this out-of-the-blue\is{out-of-the-blue context}  \emph{que}-initial question is not easy. In brief, given that there is no context that the hearer can rely on, a cooperative speaker is unlikely to use a   \emph{que}-initial question  at the start of a conversation.


Regular, i.e. non-reported,  exclamatives are adequate conversation starters just as questions are. In these contexts they usually express an emotion towards some property of the immediate context (for instance \emph{What a beautiful day it is!}, \emph{How nice to run into you!}). They are again awkward as a reported version for similar  reasons that reported questions are awkward out-of-the-blue.  A hearer would have trouble making sense of an utterance like \eqref{ex:demontebonito} because he would expect a reported version of an exclamative only if the source and the original exclamative were salient. Reported exclamatives, just like questions, have the illocutionary force of assertions. A cooperative speaker, however, would not use a reported exclamative to convey information at the beginning of a conversation\is{out-of-the-blue context} because crucial information necessary to understand the utterance, for instance who produced the original exclamative, is absent.

\ea Spanish (\citealt[18: ex 16a]{DemonteSoriano2014})\label{ex:demontebonito}\\
\gll  \# Oye, que ¡qué bonito día hace!\\
      \phantom{\#}   listen.\textsc{2sg.imp} that what nice day make.\textsc{3sg.prs}\\
\glt \phantom{\#} Intended: `Listen, what nice day it is (I just heard)!'
\z

Finally, imperatives can readily be used at the beginning of a conversation\is{out-of-the-blue context} as well. Consider for instance a context where someone calls out an offender (\emph{Put on your mask!}). In order to sensibly use the imperative in a reportative version, the original order needs to be salient. Again, a reported order like \eqref{ex:mascarilla} is infelicitous as a conversation starter without a salient antecedent because more questions (\emph{Who said that?}, \textit{Why is this reported?}, \textit{Does he/she expect me to put on a mask?}) are generated than are answered.

\ea Spanish\label{ex:mascarilla}\\
\gll \#  Oye, que te pongas la mascarilla.\\
     \phantom{\#} listen.\textsc{2sg.imp} that \textsc{cl.refl} {put on}.\textsc{2sg.sbjv.prs} the mask\\
\glt \phantom{\#} Intended: `Listen, put the mask on (I just heard).'
\z

A second observation made by \citet{DemonteSoriano2014}  is  that out-of-the-blue\is{out-of-the-blue context} reportatives are infelicitous with an explicit source.  Starting a conversation with a reportative declarative can be felicitous, as I showed in the discussion around \eqref{ex:demonteclase}, but expressing its source as in \eqref{ex:profeclase} is predicted  to be infelicitous by \citet{DemonteSoriano2014}.  

\ea Spanish\judgewidth{\#}
\ea[\#]{\gll   (Oye), el profe, que ma\~nana no hay clase.\\
               listen.\textsc{2sg.imp} the professor that tomorrow not there.be.\textsc{3sg.prs} class\\
\glt `Listen, the professor was like there will be no class tomorrow.'\label{ex:profeclase}}
\ex[]{\gll  (Oye), acabo de encontrar  nuestro profe. Y él, que ma\~nana no hay clase.\\
	listen.\textsc{2sg.imp} end.\textsc{1sg.prs} to meet our professor and he that tomorrow not there.be.\textsc{3sg.prs} class\\
	\glt `Listen, I just ran into our professor. He was like there will be no class tomorrow.'\label{ex:profeclasecont}}
\z
\z

The example in \eqref{ex:profeclasecont} shows that the same utterance is perfectly felicitous when the previous context makes the source, the professor, salient. This shows once again, that this property has to do more with the out-of-the-blue\is{out-of-the-blue context} nature of the context than with the construction itself.


A further distinction drawn by \citet{DemonteSoriano2014}  is that out-of-the-blue\is{out-of-the-blue context} reportatives do not permit a speech participant to be the source of the reported utterance. For \citet{DemonteSoriano2014}, \eqref{ex:guerra} is infelicitous because the president cannot report his/her own declaration.

\ea Spanish\label{ex:guerra} \citep[154, ex 15]{DemonteSoriano2014}\\
\sn[\#]{\gll  Ciudadanos, que se ha / que hemos declarado la guerra. \\
	 citizen.\textsc{pl}	that \textsc{cl.refl} \textsc{aux.3sg.prf.prs} {} that  \textsc{aux.1pl.prf.prs} declare.\textsc{ptcp} the war\\
	\glt `Citizens, someone said that {one has/we have} declared war.'  }
\z

\citet{Corr2016}, however, convincingly shows that this example is actually infelicitous because a reported sentence introduced by \emph{que}  is inappropriate in the formal register required in a situation where \eqref{ex:guerra} could be uttered. This is demonstrated by the fact that, in an informal setting, such as the president making the same report to his/her significant other at home, \emph{que} is felicitous, as  in \eqref{ex:guerraalice}. 


\ea\label{ex:guerraalice} Spanisch \citep[154: ex 16]{Corr2016}\\
\gll Bill, que hemos declarado la guerra. \\
	Bill	that \textsc{aux.1pl.prf.prs} declare.\textsc{ptcp} the war\\
	\glt `Bill, [I said] we’ve declared war.' 
\z

\citet{DemonteSoriano2014} also state  that out-of-the-blue\is{out-of-the-blue context} reportatives cannot be fragments, nor foreign words or onomatopeias. This can easily be explained on the basis of my general argument: In order to be a fragmented version of something, a full version needs to be salient. This requires a shared linguistic context that is not present at the beginning of a conversation\is{out-of-the-blue context}. Likewise, starting a conversation with a foreign word or onomatopoetic  expression is awkward since these require a context that makes them felicitous as a single-word utterance.

This brief discussion suggests that   all these properties observed in \citet{DemonteSoriano2014} are related  to the restrictions that limit the options at the start of a conversation.  Thus, the fact that certain properties are not observed when \emph{que}-initial reportatives are used out-of-the-blue\is{out-of-the-blue context}  can be explained as a result of the  pragmatic requirements of discourse-initial\is{out-of-the-blue context} statements. In my view, then, there is no need to assume two distinct types of \emph{que}-initial reportative constructions.  In agreement with \citet{Corr2016}, I assume that there is only one type of \emph{que}-initial reportative construction. Postulating a  distinct syntactic object with the function of an evidential marker in the sense of \citet{DemonteSoriano2014}, in light of the review of the data above, does not appear necessary. The  analysis proposed by \citet[39: ex 53]{DemonteSoriano2014}, that I will discuss henceforth,  is repeated in \eqref{ex:demonteanal}.

\ea\label{ex:demonteanal} (V) [ ForceP [que ... [ IP ]]]
\z

In the proposals by \citet{Etxepare2013} and \citet{DemonteSoriano2014} \emph{que} is analyzed as a  complementizer that heads a clause that is subordinate to silent material. The silent material -- a verb in \citet{DemonteSoriano2014} and a noun in \citet{Etxepare2013} -- contributes the reportative meaning. 

\citet{DemonteSoriano2014} do not explicitly describe   how the silencing of the verb is licensed. According to \citet{Etxepare2013}, the elision of the predicate of saying marks that its denotation is given in the \isi{common ground}; however, no explanation is given as to how the predicate of saying entered the \isi{common ground}. Since a previously mentioned predicate of saying  is not required to render reportative complementizer constructions felicitous   (see \sectref{sec:insubsemprop}), it is not immediately evident how this issue is resolved in \citeauthor{Etxepare2013}'s analysis.






\subsection{\citet{Corr2016}}



The analysis proposed in \citet{Corr2016} is based on  the author's specific assumptions about the general structure of a sentence. She proposes the extension  of the highest layers of the clausal structure that is illustrated in \eqref{struc:alice}. 

\ea \label{struc:alice} {\ob}\textsubscript{SAHigh}  {\ob}\textsubscript{SALow}  {\ob}\textsubscript{EvalP} {\ob}\textsubscript{EvidP} {\ob}\textsubscript{DeclP} \dots {\cb}{\cb}{\cb}{\cb}{\cb}
\z 

She assumes  a dedicated utterance domain above the CP (similar ideas have been developed in   \citealt{Beninca2001, Garzonio2004, Hill2006, Hill2007b,Hill2007a, SpeasTenny2003, Speas2004, Tenny2006, Poletto2003, Zanuttini2008, Zanuttini2012, Krifka2013, Haegeman2014, Wiltschko2014}).  This so-called Utterance Phrase (hence UP) is split into a high layer termed  SAHigh (SpeechAct\-high) and a low layer termed SALow (SpeechAct\-low). The higher layer is  oriented toward utterance external aspects that are encoded through an activation feature. The lower level is  oriented towards utterance-internal aspects that are encoded through a bonding feature. SALow is furthermore decomposed into  projections dedicated to the addressee and the speaker. 
\citeauthor{Corr2016}'s motivation for this structure is  the observation that CP-external elements  like vocatives and certain \isi{discourse marker}s  co-occur and that  they  follow a hierarchical order (see also  \citealt{Moro2003}, \citealt{Hill2007a,Hill2013,Hill2014}, \citealt{Moreira2013}, \citealt{Espinal2013}, \citealt{Carvalho2013}, \citealt{Stavrou2013}, among others, on vocatives,  and 
\citealt{Munaro2009},   \citealt{Coniglio2010}, \citealt{Poletto2010}, \citealt{Bayer2011}, \citealt{Haegeman2014}, \citealt{Bayer2015}, \citealt{DelGobbo2015}, among others, on \isi{discourse marker}s). 
\citet{Corr2016} furthermore argues that the  highest CP head ForceP (\citealt{Rizzi1997}) is split into three projections. The higher two, EvalP (EvaluativeP) and EvidP (EvidentialP), are adopted from \citet{Cinque1990}  and find further support in \citet{SpeasTenny2003}. DeclP (DeclarativeP) is adopted from \citet{Ledgeway2012}  and is associated with \isi{clause typing}.




With regard to the construction under investigation here, \citet{Corr2016} draws a distinction  between  reportative constructions  where a potential matrix clause  can be reconstruct\is{reconstruction}ed from the context (Portuguese) and those where it cannot (Spanish, Catalan). Those with a reconstruct\is{reconstruction}ible matrix clause are not a central concern in \citet{Corr2016}, but she proposes an analysis and treats them as cases of elision. One of  \citeauthor{Corr2016}'s examples, along with her corresponding analysis, is given in  \eqref{ex:rotsen} and \eqref{ex:rotsenana}.
 
\ea Portuguese 
\ea\label{ex:rotsen} (\citealt[149: ex 6]{Corr2016})\\
\gll Rotsen, sabes {o que} me disseram?! Que a época iria começar a 3 de Dezembro. \\
		Rotsen know.\textsc{2sg.prs} what \textsc{cl.1sg} tell.\textsc{3pl.prf.pst} \textsc{que} the season go.\textsc{3sg.cond} begin on 3 of December\\
	\glt 	`Rotsen, do you know what they told me?! That the season was going to begin on 3
		December.' 
\ex\label{ex:rotsenana}(\citealt[149: ex 7]{Corr2016}) \\ $[$CP1 \sout{Disseram-me} $[$CP2 que a época iria começar a 3 de Dezembro$]]$ 
  \z
\z

 \citet{Corr2016} only treats instances that lack a  performative verb or performative  clause in the contexts as true cases of reportative \emph{que}, which she therefore considers  to only be found in Spanish and Catalan. In these cases, \emph{que} is analyzed  as a dedicated evidential complementizer that is merged in the head of her Evidential Phrase. Her analysis of \eqref{ex:pallissaa} is given in \eqref{ex:pallissab}.

\ea Catalan
\ea\label{ex:pallissaa}
  (\citealt[161: ex 26]{Corr2016})\\
\gll Que quina pallisa que els van clavar.  \\
		\textsc{quot} what battering that they \textsc{aux.3pl.prf.pst} get\\
	\glt	`[I said] what a battering they got.' 
		\ex\label{ex:pallissab} (\citealt[188: ex 82]{Corr2016})\\ {\ob}\textsubscript{SAHigh} {\ob}\textsubscript{SALow} {\ob}\textsubscript{Eval} {\ob}\textsubscript{Evid} QUE\textsubscript{quot} {\ob}\textsubscript{Decl} {\ob}\textsubscript{Topic} {\ob}\textsubscript{Pol-int} {\ob}\textsubscript{Excl’} quina pallissa {\ob}\textsubscript{Excl} que {\ob}\textsubscript{Wh-int} {\ob}\textsubscript{Focus} {\ob}\textsubscript{Fin} {\ob}\textsubscript{IP} els van clavar {\cb}{\cb}{\cb}{\cb}{\cb}{\cb}{\cb}{\cb}{\cb}{\cb} 
	\z
\z

The difference between Spanish and Catalan  on the one hand and Portuguese on the other  is explained in \citet{Corr2016} as a case of feature scattering versus feature bundling. In Portuguese, there are three features bundled in only one Force head (as in \figref{struc:bundpt}), while in Spanish and Catalan the features are scattered across three heads (EvaluativeP, EvidentialP and DeclarativeP) (as in \figref{struc:scatsp}). \citet{Corr2016} assumes that in the true cases of reportative \emph{que} found in Spanish and Catalan the complementizer spells out only the evidential feature. Since the evaluative, evidential and declarative features  are bundled in Portuguese,  it  is impossible to spell out only one of them. In \citeauthor{Corr2016}'s account, this  is the reason for the differences observed in the reportative constructions in Portuguese compared to Spanish and Catalan. Although it is not stated explicitly in \citeauthor{Corr2016}'s (\citeyear{Corr2016}) book, this analysis might also be able to explain why   non-declarative clause types are excluded in  Portuguese \emph{que}-initial reportatives (cf. \sectref{sec:insubwlp}) because the three features bundled on one head include the declarative feature. 

\begin{figure}
\caption{\label{struc:scatsp}Spanish/Catalan (\citealt[187: ex 81]{Corr2016}) }
\begin{forest}
	[FP1 
	[~] 
	[FP1' 
	[+SA]  
	[FP2 
	[~] 
	[FP2'
	[+EVAL]
	[FP3
	[~] 
	[FP3'
	[+EVID]
	[FP4 
	[~] 
	[FP4'
	[+DECL]
	[\dots 
	]]]]]]]]]	
\end{forest}
\end{figure}

\begin{figure}
\caption{\label{struc:bundpt}Portuguese (\citealt[187: ex 80]{Corr2016})}
\begin{forest}
	[FP1 
	[~] 
	[FP1' 
	[+SA]  
	[FP2 
	[~] 
	[FP2'
	[+EVAL\\+EVID\\+DECL]
	[\dots 
	]]]]]	
\end{forest}
\end{figure}

With the central points of \citeauthor{Corr2016}'s analysis in place, I now turn to  potential issues in her approach. As stated above, at the core of the  analysis is the idea that the Portuguese type of reportative construction  differs from the  Spanish/Catalan version in that the former requires a reconstruct\is{reconstruction}ible matrix clause and the latter does not. One possible problem with this distinction is the fact that  in Spanish and Catalan  the construction can also appear in a context  with an explicitly mentioned verb of saying\is{verbum dicendi}. Crucially, these \emph{que}-initial reportatives with a  contextually recoverable \emph{\isi{verbum dicendi}}  have the same syntactic properties and  surface in the same position as those without a  recoverable \emph{\isi{verbum dicendi}} (cf. my take on this in \sectref{sec:insubsynprop} and \sectref{sec:insubcross}). In the example in \eqref{ex:selfquotrep},  a \emph{\isi{verbum dicendi}} can be recovered from the context. In  the \emph{que}-initial sentence the complementizer precedes a \textit{wh}-phrase just as  in \eqref{ex:pallissaa}, an example with a non-reconstruct\is{reconstruction}ible matrix clause that should be distinct based on  \citeauthor{Corr2016}'s analysis.
  
\ea\label{ex:selfquotrep} 
Catalan\\ 
	\gll A: Què fa la Tecla? B: Què m' has preguntat? A: Que què fa la Tecla? \\
	    { } what do.\textsc{3sg.prs} the Tecla { } what \textsc{cl.1sg} \textsc{aux.2sg.prf.prs} ask.\textsc{ptcp} { } \textsc{que} what does the Tecla\\
	\glt `A: What does Tecla do? B: What did you say? A: [reportative:] What does Tecla do?' (ebook-cat)
\z


Another  issue  relates to  the interpretation of  \emph{que}-initial sentences without a reconstruct\is{reconstruction}ible matrix clause.   According to \citet{Corr2016} ``Ibero-Romance quotative\is{quotation} \textsc{que} 	constructions are reported speech clauses introduced by the item \emph{que} which,
crucially, do not rely on a retrievable \emph{\isi{verbum dicendi}} to be felicitous.'' (\citealt[145]{Corr2016}). While the author assumes that \emph{que}-initial reportatives receive a reportative interpretation,  how this interpretation arises in her approach is not immediately evident. It is also not obvious to me whether the author considers \emph{que} to be a reportative evidential marker or a complementizer. The first option could mean that \emph{que} is perceived to be  a lexical item although homophonous  but still distinct from the default complementizer.  In the second option, the  reportative interpretation could be assumed to be   encoded syntactically. With regard to the interpretation, the author states that the complementizer is merged in an evidential phrase and carries an evidential feature, indicating that the speaker has some sort of evidence for his/her statement.  \citet[159--169]{Corr2016} adopts the concept of a presentative force (cf. \citealt{Dechaine2017}) as the most basic type of illocutionary force that  places a proposition in the \isi{common ground} without committing to its truth. She  states  that the sentences headed by reportative \emph{que} are presentative rather than asserted (cf. \sectref{sec:insubsemprop}). However, she does not make explicit which mechanisms ensure that these presentatives with an evidential feature are interpreted  as reported sentences.


\section{Outline of the present analysis}\label{sec:insubanalysis}\largerpage
In this section, I present an outline of my own account. 
The syntactic analysis that I adopt for \emph{que}-initial reportatives builds on the observation that in these apparently unembedded sentences,  \emph{que} surfaces in the same location as it does in their embedded equivalents. As a consequence, I propose   that  \emph{que}-initial sentences and their counterparts should be analyzed in the same way.  In this view, the fact that   \emph{que} looks and behaves like a complementizer is not a mere coincidence:   It is in  fact a complementizer.  Similar proposals have  been made in the literature  and have been reviewed in  \sectref{sec:insubexistan}. The crucial difference is that, contrary to the other accounts,  the analysis put forward here does not resort to a hidden performative structure nor to an elide\is{ellipsis}d matrix predicate. 

The syntactic apparatus is very simple. This simplicity, however, comes at a price: It requires a new conception of what \isi{subordination} means.  The present analysis places the burden more on pragmatic mechanisms than  previous analyses have. In a nutshell, the claim I put forward is that the syntactic structure does not encode anything other than that the sentence is subordinate.  There are, however, pragmatic requirements for uttering a subordinate sentence. Its use is only felicitous if there is a salient linguistic expression that the sentence can be subordinate to.  These pragmatic conditions are described  in greater details in \sectref{sec:insubsemprop}.


The basic syntactic idea is illustrated in the structure   in \figref{struc:barca}, which  corresponds to the analysis I propose for \eqref{ex:barca}. The complementizer is  merged directly in the highest head of the split CP SubP where its underspecified feature is valued as \emph{subordinate}.\largerpage


 \ea \label{ex:barca} 
 Catalan (adapted from \citealt[34]{DemonteSoriano2014})\\ 
 \gll Que el Barça ha guanyat la Champeons. \\
	\textsc{que} the Barcelona  \textsc{aux.3sg.prf.prs} win.\textsc{ptcp} the Championsleague  \\
	\glt `[reportative:] Barcelona has won the Champions League.'
\z

\begin{figure}
\caption{\label{struc:barca}Analysis of \eqref{ex:barca}}
\begin{forest}
	[SubP
	[~] 
	[Sub' 
	[Sub$^0$\\Que\textsubscript{subordinate}, name=sub]  
	[\dots
	[,phantom]
	[FinP
	[~] 
	[Fin' 
	[Fin$^0$, name=fin] 
	[IP
	[el Barça ha guanyat\\ la Champeons.,roof]
	]]]]]]
\end{forest}
\end{figure}

In what follows I present the reconceptualization of \isi{subordination} that I have in mind. The central idea  is that the subordinate feature assumed to be located in SubP primarily has  consequences for the interpretation of a sentence.  Thus, in my conception,  \emph{subordinate} is not a syntax-internal feature but an interface feature.  A sentence can therefore  be  marked as subordinate without being selected by a matrix clause. This idea relies on a   separation between the syntactic and semantic aspects of \isi{subordination}. Syntactic \isi{subordination} is defined as selection by a  matrix clause. This means that a syntactically subordinate sentence depends  on a matrix clause in the sense that it occupies the position of an argument or an adjunct within this matrix clause. A semantically subordinate sentence is simply interpreted as subordinate. While all syntactically selected subordinate sentences are at the same time semantically subordinate, i.e. interpreted as subordinates,  the reverse entailment does not hold: Not all semantically subordinate sentences must be selected by a matrix clause. Applying this to the issue at hand,  I propose that the \emph{que}-initial reported sentences under investigation are unselected subordinate sentences. They are  interpreted as subordinates but are not syntactically subordinate in the sense mentioned above. Henceforth, to maintain a  consistent terminological distinction, I will use the term \emph{embedded} to refer to  syntactically selected subordinate sentences and \emph{unembedded} to refer to syntactically unselected subordinate sentences.\largerpage
 
 
The  theoretical assumption  outlined above can account for those unembedded sentences introduced by a complementizer that occur in contexts where its presence cannot be linked to a reconstruct\is{reconstruction}ible matrix clause (cf. \sectref{sec:insubsynprop}). Consequently, I argue that the complementizer is part of the structure for semantic and not for syntactic reasons. 
An additional empirical  motivation is that  we also find  sentences that are syntactically subordinate but lack an overt complementizer, as in the examples in \eqref{ex:jo} below.    This constitutes evidence for the assumption that a complementizer has primarily semantic functions since  if its presence were required  for purely syntactic reasons, then it should not be possible to omit it.



Another theoretical prerequisite for my analysis is a distinction between indirect\is{indirect speech} speech and direct \isi{quotation}. \citet{Cappelen1997} illustrate the difference with the examples in \eqref{ex:alice}. If Alice uttered \eqref{ex:alicea}, then \eqref{ex:aliceb} is a direct \isi{quotation} of \eqref{ex:alicea}. It repeats the exact  words she uttered. \eqref{ex:alicec} is an indirect\is{indirect speech} quote or report. In this example,  the indirect\is{indirect speech} quote also repeats Alice's exact words. According to \citet{Cappelen1997}, the difference boils down to the fact that \eqref{ex:aliceb} is only true if Alice uttered the exact words that are presented as a \isi{quotation} whereas \eqref{ex:alicec} is also true if she didn't. 

\ea\label{ex:alice} 
\ea\label{ex:alicea} Life is difficult to understand.
\ex\label{ex:aliceb}  Alice said ``Life is difficult to understand''.
\ex\label{ex:alicec}  Alice said that life is difficult to understand.\\
(\citealt[429: ex 1–3]{Cappelen1997})
\z
\z 

In written texts, direct \isi{quotation}s often receive a special orthographic marking like the \isi{quotation} marks in \eqref{ex:aliceb} that indicate the start and end of a quote, while indirect\is{indirect speech} speech is not marked orthographically. There are reasons to believe that  indirect\is{indirect speech} speech, unlike direct \isi{quotation}, is syntactically embedded. Typically, only indirect\is{indirect speech} speech is introduced by a complementizer, cf. \eqref{ex:alicec}.
However, a complementizer is not always present in  indirect\is{indirect speech} speech constructions in English and German,  because bridge verbs (cf.  \citealt{Erteschik1973}) such as \emph{say} permit 
 \isi{complementizer deletion} in declarative complement clauses.\footnote{On \isi{complementizer deletion} see for instance \citet{Erteschik1973}, \citet{Kayne1981,Kayne1984}, \citet{Stowell1981}, \citet{Pesetsky1995}, \citet{Boskovic2003}, \citet{Bianchi2017}.}
 This  leads to a superficial oral (yet normally not written) ambiguity  between a direct \isi{quotation} and an indirect\is{indirect speech} speech reading, exemplified in \eqref{ex:aliced}.

\ea\label{ex:aliced} Alice said (that) life is hard.
\z

Similar examples are illustrated for English in \eqref{ex:joa} and German in \eqref{ex:job}. There is no complementizer, but the examples can nevertheless receive  an indirect\is{indirect speech} speech interpretation. They are then treated as embedded sentences affected by \isi{complementizer deletion}.

An important contrast between indirect\is{indirect speech} speech and direct \isi{quotation} is that they have different  \isi{deictic center}s (see also  the discussion in the introduction to this chapter). In direct  \isi{quotation}s it is the original speaker, \emph{John} in \eqref{ex:joa} and \eqref{ex:job},  who  said that someone has to leave,  while in indirect\is{indirect speech}  quotations it is the external speaker, \emph{Mary},  who reports that John said that someone has to leave. The first part of the examples in \eqref{ex:jo} can in principle be interpreted as either a direct quotation or an indirect\is{indirect speech} speech report. However, in its most natural reading indicated by the indexes (see also \citealt{Gutzmann2011a}), it is an indirect\is{indirect speech} speech report, in which the indexical pronoun \emph{I} refers to the external speaker \emph{Mary} and not to \emph{John}. This reading is supported by the continuation, where once again \emph{I} refers to \emph{Mary} and she is the one who has to leave John's house. The intended interpretation is that Mary is unwelcome at John's house and therefore he tells her to leave.\footnote{There is an alternative interpretation in which having Mary as an unwelcome guest leads to John's decision to leave his own house. In this scenario,  a direct quote reading is possible  and the continuations in \eqref{ex:jounema}, \eqref{ex:jounemb} and \eqref{ex:joc}  are felicitous.}

\ea\label{ex:jo}
\ea\label{ex:joa} [Mary$_m$:] John$_j$  said I$_m$ have to leave. Apparently I'$_m$ m not welcome at his$_j$ house.
	
	\ex \label{ex:job}
	German\\
	\gll [Maria$_m$:] Hans$_h$ hat gesagt, ich$_m$ muss gehen. Scheinbar bin ich$_m$ in seinem$_h$ Haus nicht willkommen. \\
	 Maria Hans \textsc{aux.3sg.prf.prs} say.\textsc{ptcp} I must.\textsc{1sg.prs} go apparently be.\textsc{1sg.prs} I in his house not welcome\\
	\z	
\z

The parts of the first sentences in \eqref{ex:joa}, \eqref{ex:job} are not merely juxtaposed but must truly be syntactically subordinate in order to receive the indirect\is{indirect speech} speech interpretation. Consequently, the  reading is lost when the reported sentence is not embedded. In the examples in \eqref{ex:jounem} the internal argument position of the verb \emph{say} is filled by the pronoun \emph{it}, meaning that the reported sentence cannot be syntactically embedded.  A natural  interpretation of this sentence is that of a direct quote with the indicated referents, rendering the continuation infelicitous.

\ea \label{ex:jounem}
\ea\label{ex:jounema} [Mary$_m$:] John$_j$ said it again: I$_j$ have to leave. \#Apparently I$_m$ am not welcome at his$_j$ house.

\ex\label{ex:jounemb}
German\\
\gll [Maria$_m$:] Hans$_h$ hat es wieder gesagt:  Ich$_h$ muss gehen. \#Scheinbar bin ich$_m$ in seinem$_h$ Haus nicht willkommen. \\
Maria Hans  \textsc{aux.3sg.prf.prs} it again say.\textsc{ptcp} I must.\textsc{1sg.prs} go apparently be.\textsc{1sg.prs} I in his house not welcome\\
\z
\z

For German, an indirect\is{indirect speech} speech reading of \eqref{ex:jounemb} is acceptable. Potentially, however, we are dealing with a different structure involving  extraposition and \isi{complementizer deletion}. Evidence for this hypothesis comes from the data below that show that the \emph{dass}-initial embedded sentence is extraposed from the DP \emph{die Drohung}. According to \citet{Bianchi2017} an equivalent  structure is not possible in English.

\ea
German\\
\gll $[$Maria$_m$:$]$ Hans$_h$ hat die Drohung wiederholt dass ich$_m$/*$_h$ gehen muss. \\
Maria Hans \textsc{aux.3sg.prf.prs} the threat repeat.\textsc{ptcp} that I  leave must.\textsc{1sg.prs}\\
\glt `$[$Maria:$]$ Hans repeated his threat by saying that I have to leave.'
\z 

The Ibero-Romance languages under investigation  exhibit a structure that is superficially equivalent to \eqref{ex:joa} and \eqref{ex:job} and that also lacks a complementizer. This is exemplified for Spanish in \eqref{ex:joc}. Crucially, however, the interpretation as an indirect\is{indirect speech} speech report  with the external speaker as the \isi{deictic center}, as in the German and English examples in \eqref{ex:jo}, is not possible. It must be interpreted as a direct quotation: It is Juan who says  that he himself has to leave. Therefore, in the interpretation indicated by the indexes,  the continuation is  infelicitous. In order to achieve an indirect\is{indirect speech} speech interpretation,  an overt complementizer is necessary (as in  \ref{ex:jod}). This shows that   \isi{complementizer deletion} is not an option in these contexts in Ibero-Romance.

\ea Spanish
	\ea\label{ex:joc}	
	\gll  [María$_m$:] Juan$_j$ ha dicho tengo$_{j/*m}$ que irme$_{j/*m}$. \#Aparentemente no estoy$_m$ bienvenida en su$_j$ casa. \\
	María Juan \textsc{aux.3sg.prf.prs} say.\textsc{ptcp} have.\textsc{1sg.prs} that go.\textsc{cl.refl} apparently not be.\textsc{1sg.prs} welcome.\textsc{f.sg} in his house\\
		\glt `[María$_m$:] Juan$_j$ said I$_{j/*m}$ have to leave. \#Apparently I$_m$ am not welcome at his$_j$ house.'
	\ex\label{ex:jod}
	\gll [María$_m$:] Juan$_j$ ha dicho que tengo$_{*j/m}$ que irme$_{*j/m}$. Aparentemente no estoy$_m$ bienvenida en su$_j$ casa. \\
		María Juan \textsc{aux.3sg.prf.prs} that say.\textsc{ptcp} have.\textsc{1sg.prs} that go.\textsc{cl.refl} apparently not be.\textsc{1sg.prs} welcome.\textsc{f.sg} in his house\\
		\glt `[María$_m$:] Juan$_j$ said that I$_{*j/m}$ have to leave. Apparently I$_m$ am not welcome at his$_j$ house.'	
\z	
\z\largerpage


What these facts demonstrate is that in the relevant contexts, the Ibero-Ro\-mance languages require an overt complementizer to interpret a sentence as subordinate where German and English do not. Conversely, this also means that the presence of a complementizer in German and English is not sufficient to identify a sentence as subordinate, since in contexts such as that in \eqref{ex:jo} the sentence is syntactically subordinate, hence selected by a matrix clause, but there is no overt marker. This might be one  reason why in the right context, a complementizer heading a matrix sentence is enough to indicate that the sentence is  indirect\is{indirect speech} reported speech in Ibero-Romance languages, while in German and English more explicit strategies are called for, as in \eqref{ex:jm}.

\ea\label{ex:jm}
\ea
\label{ex:jms}
Spanish\\ Juan: Tengo que irme. \\
 María: ¿Eh?\\
 Juan: Que tengo que irme. 
\ex\label{ex:jme} John: I have to leave.\\
Mary: Huh?\\
John: *That I have to leave.
\z
\z

In the Spanish example in \eqref{ex:jms}, the sentence headed by \emph{que} is understood as a report without the need for additional lexical material. In the equivalent version in English in \eqref{ex:jme}, \emph{that I have to leave} on its own is not sufficient.  A \emph{that}-initial reported sentence is only acceptable  in English when a verb of saying\is{verbum dicendi} is given in the context: 

\ea
    John: I have to leave.\\
 	Mary: What did you say?\\
 	John: That I have to leave.
\z
 	
 	
Summing up, the main idea is that the presence of a complementizer has the same impact in embedded, i.e. selected, and unembedded, i.e. unselected, reported sentences. In both cases it ensures that the sentence following \emph{que} is interpreted as a subordinate.  Consequently, it is only logical that they should also surface in the same syntactic position. I analyze the complementizer as being merged in the highest projection in the split CP and  valued with  a subordinate feature. 
In the following sections I offer empirical support for this idea and show how this simple syntactic analysis  can also account for more complex cases than that illustrated in \eqref{ex:barca}.  





I end this section with a  brief comparison of the analyses of reportative \emph{que} proposed by \citet{Etxepare2013}, \citet{DemonteSoriano2014}  and \citet{Corr2016} (see \sectref{sec:insubexistan})  with my own analysis developed here. A summary of the main points is given in Table \ref{tab:companalinsub}.




\begin{table}[H]
		\begin{tabular}{llll}
\lsptoprule
			& nature of \emph{que} &  location & reportative  interpretation \\
\midrule
			\citet{Etxepare2013} & \textsc{comp} & LinkerP & silent noun of saying \\
			D\&F (2014)\footnote{\citet{DemonteSoriano2014}} & \textsc{comp} & ForceP & silent verb of saying\is{verbum dicendi} \\
			\citet{Corr2016} & evidential \textsc{comp}	 & EvidP & feature in EvidP \\
			present analysis & \textsc{comp} & ForceP & pragmatic reconstruct\is{reconstruction}ion \\
\lspbottomrule
		\end{tabular}
		\caption{\label{tab:companalinsub}Subordinate \emph{que} in different analyses}
\end{table}

With regard to the nature of \emph{que}, my analysis is consistent with those of \citet{Etxepare2013} and \citet{DemonteSoriano2014} in treating it as a  complementizer. \citet{Corr2016} unfortunately does not explicitly state what kind of element \emph{que} constitutes and what the label \emph{evidential complementizer} entails.

My analysis is also in  agreement with \citet{DemonteSoriano2014} in assuming that the syntactic position in which \emph{que} surfaces is ForceP. Arguably, \citeauthor{Corr2016}'s analysis is also in agreement, since  EvidP constitutes a subhead of her split ForceP. However, in \citet{Corr2016}, the function that my analysis proposes for \emph{que} in reportatives, namely to mark a sentence as subordinate, is not associated with EvidP but with  another  subphrase: DeclP. The motivation in \citet{Corr2016} for the splitting of ForceP into subheads is  conceptual rather than empirical, and  is driven by the goal of developing a system in which there is  a one-to-one correspondence between a syntactic projection and an abstract pragmatic feature. This, however, makes it difficult  to empirically test which of the two adjacent projections \emph{que} actually occupies in reportatives, since  \citet{Corr2016} does not discuss any cases in which the two projections are filled at the same time. 
\citet{Etxepare2013} proposes a small clause analysis: Although \emph{que} is treated as a  complementizer, its function differs from that proposed in \citet{DemonteSoriano2014} and from the function that I attribute to \emph{que} in my analysis. This is made clear by the syntactic positions \emph{que} occupies in \citet{Etxepare2013}. The complementizer has the function  of establishing a link between a silent predicate and the \emph{que}-initial sentence. It occupies a  Linker Phrase. In the structure proposed by \citet{Etxepare2013},  LinkerP and ForceP are adjacent, so, as with \citet{Corr2016}, it is difficult to test which of the two projections the complementizer truly occupies. 


Turning finally to the question of   how the reportative interpretation arises, there are three different options presented. The first option is to assume silent or elide\is{ellipsis}d material like \citet{Etxepare2013} and \citet{DemonteSoriano2014}.  \citet{Etxepare2013} adopts an approach that relies on elision, but how this elision is licensed  is not immediately  evident in his account. Syntactic elision relies on the possibility of reconstruct\is{reconstruction}ion; however, not all contexts allow the reconstruct\is{reconstruction}ion of the relevant material.  The second option proposed by  \citet{Corr2016} is inspired by neo-performative hypotheses. The author assumes that there is a  detailed  structure above the CP that  contains a dedicated evidential projection  which, she proposes, hosts \emph{que} in reportative constructions. In the current version of the analysis offered in \citet{Corr2016}, however, the relation  between the evidential marker and the reportative interpretation does not seem completely transparent.

The third option, which I adopt in this book is presented in detail in \sectref{sec:insubsemprop}. It is based on a reconceptualization of  \isi{subordination} which  allows a reduced syntactic analysis to be proposed without the need to allude to silent material and maintaining the view that \emph{que} is in fact  a complementizer. The basic idea is that the only information encoded syntactically is that the sentence is subordinate. The incomplete information provided by the  syntactic structure leads the hearer to look for possible matrix material in the context. The  reportative interpretation therefore results from  pragmatic rather than  syntactic reconstruct\is{reconstruction}ion. Unlike in previous accounts, the burden of deriving the interpretation is pushed toward pragmatics rather than syntax. 





The analysis I lay out  in this book  has the advantage of wider empirical coverage. It does not  need to make a distinction between  reportatives with a  recoverable versus a non-recoverable \emph{\isi{verbum dicendi}} that \citet{Corr2016} proposes. Instead, my approach consists of a very simple syntactic analysis that builds on the parallels between the  unembedded sentences and their embedded counterparts. They   differ in that embedded sentences are selected by a matrix clause while  unembedded sentences are unselected. As a consequence, the analysis  does not need to postulate any additional structure or features  in order to account for unembedded reportatives.
This fact also gives my analysis a theoretical advantage: It works  without assuming hidden syntactic layers. Analyses of similar phenomena often rely on a neo-performative approach, for instance the  widely adopted structure developed in \citet{SpeasTenny2003} (adapted in \citealt{Corr2016}). 
As reported in \sectref{sec:performativehyp}, these types of structures, however, are disputed in the literature (see for instance \citealt{Gaertner2006}, \citealt{Alcazar2014}). The syntactic simplicity  of my analysis is only possible because I grant a more dominant role to pragmatic mechanisms, as described in \sectref{sec:insubsemprop}.




\pagebreak\section{The syntax of \emph{que}-initial reportatives}\label{sec:insubsynprop}\largerpage[2]


The central assumption of my analysis is that the complementizer in \emph{que}-initial reportatives is no different from a ``normal'' complementizer that appears in syntactically subordinate reported speech. In this section, I discuss the syntactic evidence in favor of this assumption. \sectref{sec:insubwlp} maps out the position of the complementizer relative to left-peripheral, i.e. CP-internal, material. \sectref{sec:insubalp} deals with its position  relative to CP-external material. The empirical evidence laid out in these sections stems from Spanish and Catalan. \sectref{sec:insubcross} focuses on Portuguese and identifies the common and diverging properties.

\subsection{Location within the left periphery}\label{sec:insubwlp}

The analysis  in \sectref{sec:insubanalysis} locates the complementizer in the construction under investigation at the left edge of the  periphery. 

\ea\label{struc:rizz} {\ob}\textsubscript{SubP} {\ob}\textsubscript{TopP} {\ob}\textsubscript{IntP} {\ob}\textsubscript{TopP} {\ob}\textsubscript{ForceP} {\ob}\textsubscript{TopP} {\ob}\textsubscript{ModP} {\ob}\textsubscript{TopP} {\ob}\textsubscript{MoodP} {\ob}\textsubscript{TopP} {\ob}\textsubscript{FinP} {\cb}{\cb}{\cb}{\cb}{\cb}{\cb}{\cb}{\cb}{\cb}{\cb}{\cb}
\z


 \eqref{struc:rizz} shows the cartographic structure of the split CP adopted in this book. 
There are two adaptations to the structure proposed by \citet{Rizzi1997, Rizzi2004, Rizzi2013} that are inspired by \citet{Haegeman2004, Haegeman2006} and \citet{Lohnstein2015} (but also  \citealt{Roussou2010}). The first change, most relevant for the present analysis, is the replacement of ForceP by SubP. In this structure, SubP is the dedicated functional projection that hosts subordinating conjunctions. 
The second change is MoodP in the lower section of the left periphery,  adopted from \citet{Lohnstein2015}, and  responsible for \isi{clause typing}. I give more details on the cartographic structure and motivate these adaptations in   \sectref{sec:intconcept}. 

In the analysis I propose for \emph{que}-initial reportatives,  \emph{que} occupies the highest head SubP. It is therefore predicted that the complementizer should precede elements that occupy any of the other left-peripheral positions. This expectation is confirmed by the data in  (\ref{ex:preceedaa}--\ref{ex:preceedff}). 
In \eqref{ex:preceeda} the left dislocated \isi{topic} \emph{la tinta}, which is resumed by a clitic pronoun, follows the complementizer. In Ibero-Romance, clitic left dislocated \isi{topic}s can occupy different positions within the left periphery. This is why \citet{Rizzi1997}  proposes that a \isi{topic} position is sandwiched between each of the other left-peripheral projections.\footnote{See \citet{Frascarelli2007a} who propose that different \isi{topic} projections result in different \isi{topic} interpretations.}  Crucially, all the positions that  clitic left dislocated \isi{topic}s can occupy are below SubP. The reverse word order illustrated in \eqref{ex:followa}, where a clitic left dislocated \isi{topic} precedes the complementizer, is not grammatical.\largerpage[-1]

\ea\label{ex:preceedaa} Spanish
	\ea 
	\gll
	{\ob}\textsubscript{SubP} Que{\cb} {\ob}\textsubscript{TopP} la tinta$_i${\cb} la$_i$ hemos de hacer {por semanas}. \\
	{} \textsc{que} {} the ink \textsc{cl.akk} have.\textsc{1pl.prs} to make weekly\\
	\glt `(Somebody said) that we have to make the ink weekly.' (CdE)\label{ex:preceeda}
	\ex * La tinta$_i$ que la$_i$ hemos de hacer por semanas. \label{ex:followa}
	\z
\z

As a side note, \eqref{ex:followa} would be grammatical if \emph{la tinta} were followed by an intonational break. This   prosodic pattern  is typical for  hanging \isi{topic}s, which are usually analyzed in a CP-external position (cf. \citealt{Bianchi2010}). I briefly return to these CP-external hanging \isi{topic}s  in \sectref{sec:insubalp}.

In \eqref{ex:preceedb}, \emph{que} precedes a \isi{polar question} introduced by the interrogative complementizer \emph{si}.  The relative position and behavior of \emph{si}  are the reason behind  the introduction of IntP as an additional left-peripheral head (cf. \citealt{Rizzi2001}), assumed to be lower than SubP. The empirical facts  support the proposed analysis. Once again, the reverse word order in which  \emph{que} follows \emph{si} is ungrammatical, cf. \eqref{ex:followb}.


\ea Spanish
\ea  \gll 	{\ob}\textsubscript{SubP} Que{\cb} {\ob}\textsubscript{IntP} si{\cb} eres feliz. \\
{} \textsc{que} {} if be.\textsc{2sg.prs} happy\\
\glt `[reportative:] Are you happy.'\label{ex:preceedb}
\ex * Si que eres feliz. \label{ex:followb} 
\z
\z


In \eqref{ex:preceedc}, the complementizer precedes a \textit{wh}-pronoun. In \eqref{ex:preceedd}, it precedes the \textit{wh}-expression of a \is{wh-exclamative@\textit{wh}-exclamative}\textit{wh}-exclamative.\footnote{ On embedded exclamatives see \citet{Zanuttini2003},  \citet{Saeboe2010} and  \citet{Gutierrez-Rexach2016}.} In accordance with their interpretation and the fact that they cannot co-occur with left dislocated foci\is{focus}, these elements have been analyzed as located in FocP, which is  again assumed to be located  below SubP. It is once more ungrammatical to reverse the order of the \textit{wh}-pronoun and the complementizer  (see \ref{ex:followc}).  The example in \eqref{ex:followd} with the \textit{wh}-expression followed by \emph{que} in a \is{wh-exclamative@\textit{wh}-exclamative}\textit{wh}-exclamative is  ungrammatical with the relevant reportative interpretation. The sequence is  grammatical, however, if it is interpreted as a \is{wh-exclamative@\textit{wh}-exclamative}\textit{wh}-exclamative with a lower merged attributive complementizer (cf. \sectref{sec:presupint}  for my analysis of these constructions).\largerpage[-1]\pagebreak
 

\ea Catalan 
\ea
 \gll A: Què fa la Tecla? B: Què m' has preguntat? A: 	{\ob}\textsubscript{SubP} Que{\cb}  {\ob}\textsubscript{FocP} què{\cb} fa la Tecla? \\
	{ } what do.\textsc{3sg.prs} the Tecla { } what \textsc{cl.1sg} \textsc{aux.2sg.prf.prs} ask.\textsc{ptcp} { } { } \textsc{que} {} what do.\textsc{3sg.prs} the Tecla\\
	\glt `A: What does Tecla do? B: What did you say? A: [reportative:] What does Tecla do?' (ebook-cat) \label{ex:preceedc} 
	\ex  * Què que fa la Tecla?\label{ex:followc} 
\z
\ex Spanish
	\ea \gll El abogado, la secretaria y yo empezábamos a hablar. Muy amigable el abogado, solo sonrisas era. {\ob}\textsubscript{SubP} Que{\cb} {\ob}\textsubscript{FocP} qué bonito{\cb} tenía el terreno. \\
	the lawyer the secretary and I start.\textsc{1pl.ipfv.pst} to talk very amicable the lawyer only smiles be.\textsc{3sg.ipfv.pst} {} \textsc{que} {} how pretty have.\textsc{1sg.ipfv.pst} the terrain  \\
	\glt `The lawyer, the secretary and I started to talk. The lawyer: very friendly and all smiles. [reportative:] How pretty I kept the terrain.' (CdE)\label{ex:preceedd}
\ex * Qué bonito que\textsubscript{subordinate} tenía el terreno. \label{ex:followd}
\z
\z



Finally, the examples in  \eqref{ex:preceede} and \eqref{ex:preceedf} show that as predicted the  epistemic adverb \emph{seguramente} and the \isi{verum} marker \emph{sí} are also preceded by subordinate \emph{que}.  Their respective positions in the left periphery are explained  in \citet{Kocher2017} and the constructions involving them are discussed extensively in \sectref{sec:presupint}. The second \emph{que}-initial reportative in \eqref{ex:preceedf} shows that   multiple left-peripheral projections can also be  occupied in \emph{que}-initial reportatives. Here, \emph{que} and \emph{sí} are interrupted by the DP \emph{Martha} which is analyzed as a topicalized subject moved to one of the TopP projections between SubP and ModP  (see \citealt{Kocher2017} for similar data and analyses). Again, the reverse order, with the subordinate complementizer following the adverb or \isi{verum} marker is not grammatical in the interpretation relevant here, cf. \eqref{ex:followe}, \eqref{ex:followf}. Just as in the example above, however, the word order is grammatical in a different interpretation where the complementizer carries an attributive feature and is merged low, which is the topic of \sectref{sec:presupint}.


\ea Spanish 
\ea[]{
\gll ¿Hay mucha corrupción en el gobierno nacional y popular? Los kirchneristas dirán que no. {\ob}\textsubscript{SubP} Que{\cb} {\ob}\textsubscript{ModP} seguramente{\cb} es una sola. \\
there.be.\textsc{3sg.prs} {a lot} corruption in the government national and popular the kirchnerista.\textsc{pl} say.\textsc{3pl.cond} that no {} \textsc{que} {} surely be.\textsc{3sg.prs} one only\\
\glt `Is there a lot of corruption in the national and popular government? The supporters of Kirchner would answer that there isn't. [reportative:] There is only one type of corruption.' (CdE)\label{ex:preceede}}
\ex[*]{Seguramente que\textsubscript{subordinate} es una sola. \label{ex:followe} }
\z
\ex \label{ex:preceedff}Spanish
\ea[]{\gll Por ustedes dijo también para luego agregar: {\ob}\textsubscript{SubP} Que{\cb} no se enreden: {\ob}\textsubscript{SubP} Que{\cb} {\ob}\textsubscript{TopP} Martha{\cb} {\ob}\textsubscript{MoodP} sí{\cb}  va. {\ob}\textsubscript{SubP} Que{\cb} {\ob}\textsubscript{MoodP} sí{\cb} va porque ha demostrado que sí puede, con el apoyo de todos. \\
for you.\textsc{pl} tell.\textsc{3sg.prf.pst} also to afterwards add {}  \textsc{que} not \textsc{cl.refl}  tangle.\textsc{2pl.imp} {}  \textsc{que} {} Martha {} \textsc{sí} go.\textsc{3sg.prs} {} \textsc{que} {}  \textsc{sí} go.\textsc{3sg.prs} because \textsc{aux.3sg.prf.prs} demonstrate.\textsc{ptcp} that \textsc{sí} can.\textsc{3sg.prs} with the help of all\\ 
\glt `For you she also spoke and later added: [reportative:] Don't get all tangled up. [reportative:] Martha \textsc{does} go. [reportative:] She \textsc{does} come because she demonstrated that with everybody's help she \textsc{can}.'\label{ex:preceedf}
(CdE)}
\ex [*]{Sí que\textsubscript{subordinate} va.\label{ex:followf} }
\z
\z


Multiple \emph{que} constructions, in which a \emph{que}-initial reported sentence  contains another instance of \emph{que},  are also possible, cf. \eqref{ex:multque}. The second complementizer is valued with an attributive feature and  observes strict adjacency to a \textit{wh}-ex\-pres\-sion in the  \is{wh-exclamative@\textit{wh}-exclamative}\textit{wh}-exclamative in \eqref{ex:multquea}, an evidential modifier\is{epistemic and evidential modifier} in AdvC in \eqref{ex:multqueb} and the \isi{verum} marker \emph{sí} in  AffC in \eqref{ex:multquec}. The structure and interpretation of sentences with a low-merged complementizer are dealt with in depth in \sectref{sec:presupint}. Regardless of the specific details, what the data in \eqref{ex:multque} demonstrate is that there is more than one position within the left periphery that can host a complementizer. A central argument of  this work is that the position in which the complementizer is merged has an effect on the interpretation of the sentence. The claim is that while the higher position marks the sentence as subordinate, the lower position attributes to the hearer a commitment to the proposition in the scope of the complementizer. In the present analysis, this is modeled by assuming that the different projections provide different interface features. Briefly,  the lower complementizer is merged in FinP but does not remain in this position. Instead, it moves from head to head through the left periphery until it encounters an element in the specifier of a projection that blocks its movement.




\ea\label{ex:multque}
\ea\label{ex:multquea}
Catalan  (\citealt[161: ex 26]{Corr2016})\\ 
\gll {\ob}\textsubscript{SubP} Que{\cb} {\ob}\textsubscript{FocP} quina pallissa que$_i${\cb} {\ob}\textsubscript{FinP} t$_i${\cb} els	van clavar. \\
{} \textsc{que} {} what	battering \textsc{que} {} {} they \textsc{aux.3pl.prf.pst} get\\
\glt `(I said) what a battering they got.'
\ex\label{ex:multqueb}Spanish\\\gll {\ob}\textsubscript{SubP} Que{\cb}  {\ob}\textsubscript{ModP} claro que$_i${\cb} {\ob}\textsubscript{FinP} t$_i${\cb} le viene bien, que qué alegría, que dónde. \\
{} \textsc{que} {} clear \textsc{que} {} {} \textsc{cl.refl} come.\textsc{3sg.prs} good \textsc{que} what joy \textsc{que} where \\
\glt `[reportative:] Of course it was no inconvenience. (She said) what a joy! (She said) where?' (CdE)
\ex\label{ex:multquec}
Catalan\\ 
\gll {\ob}\textsubscript{SubP} Que{\cb} {\ob}\textsubscript{MoodP} sí que$_i${\cb} {\ob}\textsubscript{FinP} t$_i${\cb} heu vingut!  \\
{} \textsc{que} {} \textsc{sí} \textsc{que} {} {} \textsc{aux.2pl.prf.prs} come.\textsc{ptcp}\\
\glt `[reportative:] You \textsc{have} come.' (caWac)
\z
\z

My analysis proposes that \emph{que}-initial reportatives are essentially subordinate sentences that are unselected and therefore lack a matrix clause. An expectation that follows from this is that  the subordinate complementizer should surface in the same position in embedded and unembedded reportatives. The following paragraphs demonstrate that this expectation is borne out. There are two versions of embedded questions in Spanish and Catalan: One in which the complementizer is present and one it which it is absent (see the discussion of the  examples \eqref{ex:deresp} in \sectref{sec:insubcross}  and  see also \citealt{Etxepare2010}, \citealt{GonzalesPlanas2014} and \citealt{Corr2016}). Crucially, if the complementizer is present, as predicted, it surfaces in the same position as in unembedded reportatives (cf. \ref{ex:embedreportsa}--\ref{ex:embedreportsd}).  My proposal  is that the syntactic structure of the embedded sentence CP in \eqref{ex:embedreportsa} is equivalent to the structure of the unembedded sentence CP in \eqref{ex:preceedc}. This is supported by the fact that \emph{que} precedes the interrogative complementizer \emph{si} in both cases.\pagebreak


\ea\label{ex:embedreportsa}
Spanish\\ 
\gll Me acerqué  a una {oficina de turismo}, pregunté {\ob}\textsubscript{SubP} que{\cb}  {\ob}\textsubscript{IntP} si{\cb}  estaba abierto el camino hacia Florencia. \\
	\textsc{cl.refl} approach.\textsc{1sg.prf.pst} to a {tourist office} ask.\textsc{1sg.prf.pst} {} that {} whether be.\textsc{3sg.ipfv.pst} open the road to Florence\\
	\glt `I went to a tourist office and asked (that) whether the road to Florence was open.' (CdE)
\z

Example \eqref{ex:embedreportsb} parallels \eqref{ex:preceedc}. A \textit{wh}-pronoun, analyzed in FocP, is preceded by the complementizer \emph{que}.\largerpage

\ea\label{ex:embedreportsb}
 Spanish\\ 
\gll Me encontré con un amigo y el amigo me dijo  {\ob}\textsubscript{SubP} que{\cb}  {\ob}\textsubscript{FocP} qué{\cb}  estaba haciendo.  \\
\textsc{cl.refl} meet.\textsc{1sg.prf.pst} with a friend and the friend \textsc{cl.1sg} say.\textsc{3sg.prf.pst} {} that {} what be.\textsc{3g.prog.prs} doing\\
\glt `I ran into  a friend and the friend asked (that) what I was doing.' (CdE)
\z

The examples in \eqref{ex:embedreportsc} and \eqref{ex:embedreportsd} show that  syntactically embedded reportatives also permit the occurrence of  multiple instances of \emph{que}, just as  their unembedded counterparts do in \eqref{ex:multque}. The example in \eqref{ex:embedreportsc} demonstrates that the AdvC construction, like in the unembedded sentence in \eqref{ex:multqueb}, can be part of an embedded reportative and occupies a position lower than SubP. Example \eqref{ex:embedreportsd} shows that AffC also surfaces below the subordinating complementizer, similar to  \eqref{ex:multquec}.  The example in \eqref{ex:embedreportsd} furthermore shows that a clitic left dislocated \isi{topic} and an epistemic modifier\is{epistemic and evidential modifier} can intervene between the high instance of \emph{que} and the sequence \emph{sí que}. The position of the \isi{topic} and the modifier is also predicted by my analysis. The high \emph{que} is analyzed in  SubP while the lower one is merged in the lowest projection FinP before moving to the second-lowest MoodP. There are therefore a number of intervening projections available that can host further phrases (cf. the structure in \ref{struc:rizz}).

\ea\label{ex:embedreportsc} 
Catalan (\citealt[35: ex 57a]{Kocher2017a} from ebook-cat)\\ 
\gll A nivell de relació de parella us haig de dir {\ob}\textsubscript{SubP} que{\cb}  {\ob}\textsubscript{ModP} evidentment que$_i${\cb} {\ob}\textsubscript{FinP} t$_i${\cb} les coses canvien.  \\
at level of relationship of couple you have.\textsc{1sg.prs} to say {} that {} evidently \textsc{que} {} {} the thing.\textsc{pl} change.\textsc{3pl.prs}\\
\glt `Concerning couples' relationships I have to tell you that evidently things change.' 
\ex\label{ex:embedreportsd} 
Catalan (\citealt[35: ex 57c]{Kocher2017a} from ebook-cat)\\ 
\gll Al museu, mirant un dibuix de Manolo Hugué ha conegut una dona d’ulls nets, profunds i amb una punta de malícia, i ha pensat  {\ob}\textsubscript{SubP} que{\cb}  {\ob}\textsubscript{ModP} potser{\cb} {\ob}\textsubscript{TopP}  d’aquesta dona$_j${\cb} {\ob}\textsubscript{MoodP} sí que$_i${\cb} {\ob}\textsubscript{FinP} t$_i${\cb}  se’n$_j$ podria enamorar.  \\
{at the} museum looking a drawing of Manolo Hugué \textsc{aux.3sg.prf.prs} meet.\textsc{ptcp} a woman {of eyes} clean deep and with a hint of malice and \textsc{aux.3sg.prf.prs} think.\textsc{ptcp} {} that {} maybe {} {of this} woman {} \textsc{sí} \textsc{que} {} {} {\textsc{cl.refl}.\textsc{cl.part}} can.\textsc{3sg.cond} {fall in love}\\
\glt `At the museum, while looking at a drawing of Manolo Hugué he met a woman with honest, deep eyes and with a hint of malice and thought that maybe with this woman he \textsc{could} fall in love.' 
\z




A characteristic that has been highlighted by \citet{DemonteSoriano2014} is the fact that what follows \emph{que}  in reportatives does not have to be a full sentence.   This is illustrated in \eqref{ex:multquebrep}  repeated from \eqref{ex:multqueb}. At the end of the fragment, \emph{que} is followed only by the \textit{wh}-phrase \emph{qué alegría} and the \textit{wh}-pronoun \emph{dónde} `where' while the rest of the sentence is omitted. 


\ea
\label{ex:multquebrep}
Spanish\\ 
\gll {\ob}\textsubscript{SubP} Que{\cb}  {\ob}\textsubscript{ModP} claro que$_i${\cb} {\ob}\textsubscript{FinP} t$_i${\cb} le viene bien, {\ob}\textsubscript{SubP} que{\cb}  {\ob}\textsubscript{FocP} qué alegría{\cb}, {\ob}\textsubscript{SubP} que{\cb}  {\ob}\textsubscript{FocP} dónde.{\cb} \\
{} \textsc{que} {} clear \textsc{que} {} {} \textsc{cl.refl} come.\textsc{3sg.prs} good {} \textsc{que} {} what joy {} \textsc{que} {} where \\
\glt `[reportative:] Of course it was no inconvenience. (She said) what a joy! (She said) where?' (CdE)
\z



The data below show that permitting fragmentation is not a special property of unembedded sentences, but can also be observed in syntactically subordinated indirect reportatives. In \eqref{ex:fraga}  the reportative only consists of a \textit{wh}-pronoun, and in \eqref{ex:fragb}  it consists of the adverb \emph{ahora} `now'. This again supports my proposal that unembedded \emph{que}-initial reportatives are  unselected but essentially do not otherwise differ from their embedded counterparts. 

\ea Spanish
\ea \label{ex:fraga}
\gll  Todos {volvieron a preguntar} que qué.  \\
all ask.again\textsc{3p.prf.pst}  that what\\
\glt `Everybody asked ``what'' again.' (CdE)

\ex \label{ex:fragb}
\gll ¿Cuándo no hay condenación? ¡La Palabra de Dios dice que AHORA! \\
when not there.be.\textsc{3sg.prs} damnation the word of God say.\textsc{3sg.prs} that now\\
\glt `When is the moment of no damnation? The word of God says that the moment is now.' 	(CdE)
\z
\z





Further support for the idea that the complementizer surfaces in the same position in \emph{que}-initial reportatives as in their syntactically embedded counterparts stems from their behavior with respect to non-declarative sentences. Instances of \textit{wh}- and \isi{polar question}s were exemplified and discussed  in \eqref{ex:preceedb} and \eqref{ex:preceedc}. Another example is shown in \eqref{ex:whandpolarq}.

\ea\label{ex:whandpolarq}
 Spanish\\ 
\gll Pues como te iba diciendo, ella me preguntó por ti. {\ob}\textsubscript{SubP} Que{\cb} {\ob}\textsubscript{FocP} {qué{\cb} tal te va},  {\ob}\textsubscript{SubP} que{\cb} {\ob}\textsubscript{IntP} si{\cb} tienes trabajo y así. \\ 
well as \textsc{cl.2sg} go.\textsc{1sg.ipfv.pst} telling she \textsc{cl.1sg} ask.\textsc{3sg.prf.pst} about you  {} \textsc{que} {} {how be.\textsc{2sg.prs} you doing} {} \textsc{que} {} whether \textsc{2sg.prs} work and so\\
\glt `As I was telling you, she asked about you. [reportative:] How you are doing and do you have a job and so on.' (CdE)
\z



 What is important is that in both cases  the complementizer is in a higher position than the \textit{wh}-pronoun (in FocP)  or the interrogative complementizer \emph{si} (in IntP). 
 




One substantial argument in favor of the assumption that \emph{que} in unembedded sentences is truly a marker of \isi{subordination}  comes from imperatives. 
The examples in \eqref{ex:imperatives} show that in order to report an imperative, its structure and morphological makeup need to be adapted. Thus \eqref{ex:imperativeb}, in which the imperative is directly headed by the \isi{subordination} complementizer, is ungrammatical. In the grammatical version in \eqref{ex:imperativea}, the verbal morphology is no longer imperative but subjunctive. The verb furthermore occupies a higher position in the imperative in \eqref{ex:imperativea} than in the reported version in \eqref{ex:imperativec}. This is shown by the position of the clitic pronoun, which is enclitic in the true imperative and proclitic in  reported imperatives.\largerpage

\ea\label{ex:imperatives}Spanish
\ea[]{
\gll ¡Vete! \\
go.\textsc{2sg.imp}.\textsc{cl.refl}\\
\glt `Go away!'\label{ex:imperativea}}
\ex[*]{ \gll  ¡Que vete! \\
\textsc{que} go.\textsc{2sg.imp}.\textsc{cl.refl}\\
\glt Intended `[reportative:] Go away!'\label{ex:imperativeb}}
\ex[]{ \gll ¡Que te vayas! \\
\textsc{que} \textsc{cl.refl} go.\textsc{2sg.sbjv.prs}\\
\glt `[reportative:] You should go!'\label{ex:imperativec}}
\z
\z

A  comparison  with  syntactically embedded imperatives once again shows that they exhibit  the same behavior (on embedded imperatives see \citealt{Portner2007}, \citealt{Kaufmann2011}, \citealt{Kaufmann2013}). The example in  \eqref{ex:embedimperativeb}, in which the verb is inflected for imperative mood and is directly embedded under a verb expressing an order, is ungrammatical. The grammatical structure contains a verb inflected for subjunctive mood. The proclisis shows that the subjunctive verb surfaces lower in \eqref{ex:embedimperativeb} than the imperative verb in \eqref{ex:imperativea}.


\ea\label{ex:embedimperatives}Spanish
\ea[]{  
\gll Te ordeno que te vayas. \\
\textsc{cl.2sg} order.\textsc{1sg.prs} that \textsc{cl.refl} go\textsc{.2sg.sbjv.prs}\\
\glt `I order that you leave.'\label{embedimperativea}}
\ex[*]{\gll  Te ordeno que vete. \\
\textsc{cl.2sg} order.\textsc{1sg.prs} that  go\textsc{.3sg.imp}.\textsc{cl.refl}\\
\glt Intended: `I order that you leave.'\label{ex:embedimperativeb}}
\z
\z

The data in \eqref{ex:imperatives} and \eqref{ex:embedimperatives} are strong evidence in favor of the subordinate nature of \emph{que}-initial reportatives and against the assumption that \emph{que} is merely a pragmatic marker. 



I now turn to a  short digression on \emph{que}-initial directives. These share a number of properties with reported imperatives but  must nonetheless be kept separate. \emph{Que}-initial directives encode commands directed at a  third person who is not currently participating in the conversation. This type of directive is called a jussive. Examples of  \emph{que}-initial jussives are given in \eqref{ex:quejus}. They contrast with a (non-reported) command directed at a speech participant which is never introduced by the complementizer. The contrast between these \emph{que}-initial directives and the other \emph{que}-initial sentences I investigate in this book is also notable: With \emph{que}-initial jussives,  the omission of the complementizer results in ungrammaticality (see also \cite[117]{Gras2016}) which is not the case with the reportative and attributive \emph{que} constructions. Note that there is also a \emph{que}-less jussive in Ibero-Romance. However, it differs from jussives like those in \eqref{ex:quejus} in more than the mere presence or absence of \emph{que} and shares the syntactic properties of second-person directives (see the discussion around example \eqref{ex:jusnoque}). 
	\ea\label{ex:quejus}
\ea\label{ex:quejusa}
 Spanish\\ 
\gll Que se vaya. \\
\textsc{que} \textsc{cl.refl} go.\textsc{3sg.sbjv.prs}\\	
\ex
Catalan\\ 
\gll Que es vagi. \\
\textsc{que} \textsc{cl.refl} go.\textsc{3sg.sbjv.prs}\\	
\ex\gll Que  vá embora.\\
\textsc{que}  go.\textsc{3sg.sbjv.prs} away\\
\glt`He/She should go away.' 
\z
\z



From a formal point of view, the comparison  illustrated in \eqref{ex:impjuss}  between  directives addressing  a second and a third person is particularly revealing. First of all, the second- and third-person directives in these examples differ in their verbal inflection.  The verb in the third-person directive in \eqref{ex:jusscat} is inflected for subjunctive while the verb in the second-person directive in \eqref{ex:impcat} is inflected for imperative mood.  Furthermore, the unstressed pronouns are proclitic in \eqref{ex:jusscat} but enclitic in \eqref{ex:impcat}. These structural differences have led  authors  to conclude that the  verb reaches a higher position in imperatives than in third-person directives.  Some (for instance  \citealt{Rivero1994}, \citealt{Rivero1995}, \citealt{DemonteSoriano2009}, \citealt{Alcazar2014}) propose that  the  imperative verb targets a left-peripheral position.

\ea\label{ex:impjuss}Catalan
\ea[*]{\label{ex:jusscat} 
 \gll  ({Que}) es renti les mans. \\
\textsc{que} \textsc{cl.refl} wash.\textsc{3sg.sbjv.prs} the hands\\
\glt`He/she should wash her/his hands!'}

\ex[]{\label{ex:impcat}  \gll Renta't les mans. \\
{wash.\textsc{2sg.imp}.\textsc{cl.refl}} the hands\\
\glt`Wash your hands!'}
\z
\z

The examples in \eqref{ex:impjussdec} moreover show that the third-person directives (cf. \ref{ex:jusssp}, \ref{ex:juspt}) behave similarly to declaratives (cf. \ref{ex:decsp}), reported imperatives (cf. \ref{ex:imperativec}) and embedded imperatives (cf. \ref{ex:embedimperatives}) in licensing preverbal negation and subjects, both of which are ungrammatical in imperatives \eqref{ex:impsp}, \eqref{ex:imppt}. This is cited as  further  evidence for the idea that  an imperative verb reaches a higher position than  a subjunctive or indicative verb.

\ea\label{ex:impjussdec} Spanish\\ 
\ea\label{ex:jusssp} 
\gll Que él no lo compre. \\
\textsc{que} he not \textsc{cl.akk} buy.\textsc{3sg.sbjv.prs}\\
\glt`He/She should not buy it.'
\ex\label{ex:decsp} 

\gll Tú no lo compras. \\
you not \textsc{cl.akk} buy\textsc{2sg.ind.prs}\\
\glt`You don't buy it.'
\ex\label{ex:impsp}    \gll (*Tú) (*No) Cómpralo tú.  \\
you not {buy.\textsc{2sg.imp}.\textsc{cl.akk}} you\\
\z 
\z 
\ea
Portuguese\\ 
\ea \label{ex:juspt} 
\gll  Que ele n\~ao o compre. \\
\textsc{que} he not  \textsc{cl.akk} buy.\textsc{3sg.sbjv.prs}\\
\glt`He should not buy it.'
\ex\label{ex:imppt}
\gll (*Tu) (*N\~ao) compra-o. \\
you not buy.\textsc{2sg.imp}-\textsc{cl.akk}\\
\glt`Buy it!' 
\z
\z

These structural patterns hold in Spanish, Catalan and Portuguese.\footnote{As an aside, it must be mentioned that in  European Portuguese  clitic placement in declaratives in general requires enclisis (although there are a number of properties that induce proclisis). This means that  with respect to clitic placement, European Portuguese indicative verbs  pattern with imperative and not subjunctive verbs.  Some authors have argued that this is due to the fact that the finite indicative verb moves higher than the IP in Portuguese (cf. \citealt{Raposo2000}, \citealt{Galves2005}).  Famously, Brazilian Portuguese patterns differently and shows proclisis in declaratives, like Spanish and Catalan (cf. \citealt{Galves2005}).  Unlike imperatives, declaratives do permit preverbal subjects in both Portuguese varieties. This suggests that while clitic placement is the same, the structure of declaratives and imperatives is nevertheless different.
} As stated above, in addition to the \emph{que}-initial jussives, there is also a grammatical  \emph{que}-less counterpart in all three languages. These behave on  par with the second-person directives in triggering enclisis and postverbal subjects. They are illustrated in \eqref{ex:jusnoque}.

\ea\label{ex:jusnoque}
\ea\label{ex:jusnoquea}
 Spanish\\ 
\gll Váyase. \\
go.\textsc{3sg.sbjv.prs}.\textsc{cl.refl} \\	
\ex
Catalan\\ 
\gll Vagi-se. \\
go.\textsc{3sg.sbjv.prs}-\textsc{cl.refl} \\	
\ex
Portuguese\\ 
\gll Vá-se embora. \\
go.\textsc{3sg.sbjv.prs}-\textsc{cl.refl}  away\\
\glt`He/She should go away.' 
\z
\z


 To close, I want to mention a construction that is superficially identical to  \emph{que}-initial directives and that is employed to express optatives. These are defined in \citet{Grosz2012} as  sentences that express  a wish, regret, hope or desire in the absence of an overt lexical item
that encodes \emph{wish}, \emph{regret}, \emph{hope} or \emph{desire}.  Ibero-Romance optatives require an initial complementizer when they are directed at a (formally marked) third person  but also when they are directed  at a second person, cf. \eqref{ex:opt2}. An optative expressing a wish dedicated to  a non-speech participant coincides  with the form used when expressing a wish dedicated to a speech participant who is addressed with the honorific (cf. \ref{ex:opt3}). Sometimes the second-person optatives are employed as polite forms of an imperative (cf. \citealt{DemonteSoriano2014}, who even call them independent imperatives). 

\ea Catalan
\ea\label{ex:opt2}
\gll  Que tinguis un dia magnífic. \\
\textsc{que} have.\textsc{3sg.sbjv.prs} a day magnificent\\
\glt`May you have a magnificent day!'
\ex\label{ex:opt3}

\gll  Que tingui un dia magnífic. \\
\textsc{que} have.\textsc{3sg.sbjv.prs} a day magnificent\\
\glt`$[$honorific:$]$ May you have a magnificent day!'
\glt`May he/she have a magnificent day!'
\z
\z





Summing up this section, I have shown that  \emph{que} precedes all other types of phrases that are assumed to be merged in the split CP. The complementizer must therefore be located at the left edge of the functional field. This analysis  consequently predicts  that  \emph{que} must follow linguistic material merged above the CP. In the following section, I discuss data that show that this prediction  holds true.  




\subsection{Above the left periphery}\label{sec:insubalp}
This section discusses empirical evidence  that confirms the prediction that the complementizer in \emph{que}-initial reportatives follows CP-external material. First,  I show that, as expected,  \emph{que} follows certain \isi{discourse marker}s, vocatives and hanging \isi{topic}s.  I then discuss specific phrases that precede \emph{que} and that have been analyzed in the literature as expressing the original speaker of the reported sentence.  I offer an alternative analysis that treats these phrases as frame setters.


\citet{Corr2016} shows that \isi{discourse marker}s and vocatives are external to the left periphery and follow ordering restrictions (see also the syntactic analysis of vocatives for instance in \citealt{Moro2003}, \citealt{Hill2007a,Hill2013,Hill2014}, \citealt{Moreira2013}, \citealt{Espinal2013}, \citealt{Carvalho2013}, \citealt{Stavrou2013},  and of \isi{discourse marker}s for instance in 
 \citealt{Munaro2009},   \citealt{Coniglio2010}, \citealt{Poletto2010}, \citealt{Bayer2011}, \citealt{Haegeman2014}, \citealt{Bayer2015}, \citealt{DelGobbo2015}). This leads her to propose an extended  speech act structure above the split CP (as in \ref{struc:alicerep}).\footnote{For a detailed discussion of \citeauthor{Corr2016}'s analysis see \sectref{sec:insubexistan}.} 


\ea \label{struc:alicerep} {\ob}\textsubscript{SAHigh} \emph{\isi{discourse marker}s} {\ob}\textsubscript{SALow} \emph{vocatives} {\ob}\textsubscript{EvalP} {\ob}\textsubscript{EvidP} {\ob}\textsubscript{SpecEvid} \emph{source} {\cb} {\ob}\textsubscript{EvidP'} {\ob}$_{\text{Evid}^0}$ \emph{que\textsubscript{reportative}}{\cb}{\cb} {\ob}\textsubscript{DeclP} \dots {\cb}{\cb}{\cb}{\cb}{\cb} (\citealt{Corr2016})
\z 

\citet{Corr2016} proposes two speech act phrases above the split CP. According to her, the higher phrase hosts outward-oriented  and discourse-activating markers like \emph{oye} `listen', while the lower one  hosts inward-oriented and discourse-bonding markers like vocatives.  As expected, the complementizer follows both types of expressions in  \emph{que}-initial reportatives.

The data in \eqref{ex:alicedms} once again confirm that the complementizer in unembedded reportatives surfaces at the left edge of the split CP, above 
other CP-internal material, but crucially below CP-external material.
In particular, the example in \eqref{ex:alicedma} shows that the \isi{discourse marker} \emph{oye}, analyzed as located in SAHigh in \citet{Corr2016}, can only precede but not follow \emph{que} in reportatives. The same goes for the vocative \emph{Irene} in \eqref{ex:alicedmb}, analyzed as located in SALow in \citet{Corr2016}. Finally, the example in  \eqref{ex:alicedmc} shows that a \isi{discourse marker} and a vocative can co-occur and that they  both precede \emph{que}. This word order is predicted by my analysis.
In \eqref{ex:alicedmc}, in addition to the \isi{discourse marker} and the vocative, there is  a DP immediately above \emph{que}. The DP  \emph{la peinadora} is interpreted as the source of the report; according to \citet[152]{Corr2016} it  encodes the {``original
	interlocutor''}. She analyzes the phrase as being merged in the specifier of the projection occupied by the complementizer, hence the ungrammaticality of a complementizer preceding \emph{la peinadora} in \eqref{ex:alicedmc}.  

\ea \label{ex:alicedms} Spanish
\ea\label{ex:alicedma}
  (\citealt[182: ex 66]{Corr2016})\\ 
\gll  {\ob}\textsubscript{SAHigh} Oye{\cb}, que  (*oye) si  has visto mis llaves. \\
{} \textsc{dm} \textsc{que} \textsc{dm} \textsc{if} \textsc{aux.2sg.prf.prs} see.\textsc{ptcp} my keys\\
\glt `Hey, (I asked) have you seen my keys?'
\ex\label{ex:alicedmb}
\citet[182: ex 67]{Corr2016}\\
\gll {\ob}\textsubscript{SALow} Irene{\cb}, que (*Irene) si has visto mis llaves. \\
 {} \textsc{voc} \textsc{que} \textsc{voc} if \textsc{aux.2sg.prf.prs} see.\textsc{ptcp} my keys\\
\glt `Irene, (I asked) have you seen my keys?'  
\ex\label{ex:alicedmc}
(\citealt[182–183: ex 70]{Corr2016})\\
\gll {\ob}\textsubscript{SAHigh} Oiga{\cb}, {\ob}\textsubscript{SALow} señora Marquesa{\cb}, (*que) la peinadora que no puede esperar. \\
{} \textsc{dm} {}  \textsc{voc} {} \textsc{que} the hairdresser \textsc{que} not can.\textsc{3sg.prs} wait\\
\glt `Listen, Lady Marquis, the hairdresser says she cannot wait.'
\z
\z

In the example in \eqref{ex:alicedmc}, the referent of \emph{la peinadora} has a dual role: She is the original speaker but  she is also interpreted as the agent of the action  described in the proposition and thus coincides with the subject of the sentence.  Still, in the version  in \eqref{ex:alicedmc}, \emph{la peinadora} cannot be a dislocated and \isi{topic}alized subject because if it were, it would target a position lower than \emph{que} (cf. an equivalent version with a clitic left dislocated object in \eqref{ex:preceeda}). The only possible analysis for this example, then, is that there is a null subject agreeing with the finite verb \emph{esperar} `wait'.\largerpage

Analyzing \emph{la peinadora} as a  hanging topic is also not possible. While hanging topics can  precede a reportative \emph{que} (see   \sectref{sec:insubwlp}), the sentence would have different properties. Hanging \isi{topic}s differ from CP-internal \isi{topic}s in that they are always resumed by a clitic, a full pronoun or an epithet. Furthermore,  CP-internal \isi{topic}s are usually analyzed as being generated in an IP-internal position and  moved to the left periphery,\footnote{But cf. \citet{VillaGarcia2015} for a special type of \isi{topic} construction where the \isi{topic} is analyzed as being merged directly in a CP-internal position.} while hanging \isi{topic}s are analyzed as being merged directly in the CP-external position (cf. \citealt[165]{VillaGarcia2015} for a concise comparison).  
\ea\label{ex:impaciente} Spanish \\ \gll La peinadora$_i$, que esta impaciente$_i$ no puede esperar.\\
the hairdresser \textsc{que} this impatient not can.\textsc{3sg.prs} wait\\
\glt `[reportative:] The hairdresser, this impatient person cannot wait.' 
\z
 
The example in \eqref{ex:impaciente} illustrates the typical properties of hanging \isi{topic}s. \emph{La peinadora} is followed by an intonational break  indicated by the comma and  is resumed by the co-referential epithet \emph{esta impaciente}.{\interfootnotelinepenalty=10000\footnote{Some of the native speakers I consulted additionally accepted a version of \eqref{ex:impaciente} where \emph{la peinadora} is sandwiched between two \emph{ques}. Although further investigation is necessary, this is potentially a different phenomenon similar to the recomplementation  analyzed by \citet{VillaGarcia2015}.} }


The phrases that precede \emph{que} that have so far been analyzed as expressing an original speaker are sometimes also followed by an intonational break -- although, according to some of my informants, the break is shorter than the break that follows the hanging \isi{topic}  in \eqref{ex:impaciente}.  These nevertheless do not qualify as hanging \isi{topic}s. Importantly, they cannot be resumed because they are not  co-ref\-er\-en\-tial with an argument or an adjunct of the verb. This can be seen clearly in the example in  \eqref{ex:quotativeque}.

\ea
	\label{ex:quotativeque}
	
		\ea \label{ex:source}  
		Catalan\\ 
		         \gll  La mare, que què t' has cregut. \\
		the mother \textsc{que} what \textsc{cl.refl} \textsc{aux.2sg.prf.prs} think.\textsc{ptcp}\\
		\glt `[reportative:] The mother:   What were you thinking.’ (ebook-cat)
	\ex \label{ex:source1} Brazilian Portuguese \\
	\gll  E eu que não, que não. \\
	and I \textsc{que} no \textsc{no}\\
	\glt `[reportative] and I: no, no.' (CdP)
	\z
\z

In previous accounts, the phrases that precede \emph{que} have been used as supporting evidence for a special underlying performative (\citealt{DemonteSoriano2014}) or evidential (\citealt{Corr2016}) structure in which these phrases are claimed to  occupy a dedicated source or speaker position.  
 In the following,   however, I show that these preceding phrases do not always encode the original speaker or source of the reported sentence. And more importantly: even when the phrases can be interpreted as expressing an original speaker or source,   they do not show any  particular  behavior that suggests that they  have an especially local relation to the complementizer.   My proposal is that these \emph{que}-preceding phrases are frame setters. These are elements that set the frame in which the following expressions should be interpreted (\citealt{Krifka2008}, cf. also \citealt{Chafe1976} and \citealt{Jacobs2001}).   

The \emph{que}-preceding phrases can be DPs like in \eqref{ex:quotativeque}, which makes the conclusion that they express an original speaker appear plausible. However, \emph{que} can also  be preceded by PPs.  This is illustrated by  data  taken from \citet{Etxepare2013}.  

\ea\label{ex:etxaljaz}
 Spanish (\citealt[98: ex 11]{Etxepare2013})\\ 
 \gll Oye, en Al-Jazeera, que Obama va a atacar Iran. \\
	Hey in Al-Jazeera \textsc{que} Obama go.\textsc{3sg.prs} to attack Iran\\
	\glt `Hey, there’s a saying on Al Jazeera that Obama is going to attack Iran.' 
\z

 In the example in \eqref{ex:etxaljaz},  the phrase that precedes \emph{que} is a locative PP. In \citet{Etxepare2013} \emph{en Al-Jazeera} is analyzed as an adverbial modifier of the elide\is{ellipsis}d predicate of saying. I want to mention here that, although data like these are not elaborated on in \citet{Corr2016}, they could be reconciled with  her analysis by assuming that phrases preceded by \emph{que} do not express an agent (as she states at various points) but merely an information source.
 The examples in (\ref{ex:aljaza}--\ref{ex:aljazd}), however,   pose a problem for her analysis. In \eqref{ex:aljaza}, I have adapted  \citeauthor{Etxepare2013}'s example and present it in a context in which the locative PP \emph{en Al-Jazeera} is contrasted with \emph{en BBC}. 



\ea\label{ex:aljaza} 
 Spanish\\ 
\gll Ultimadamente uno no  se puede fiar de las noticias: En Al-Jazeera, que Obama va a atacar Iran. Pero en BBC, que va a retirar sus tropas. \\
lately one  not \textsc{cl.refl} can.\textsc{3sg.prs} trust in the news  in Al-Jazeera \textsc{que} Obama go.\textsc{3sg.prs} to attack Iran but in BBC \textsc{que} go.\textsc{3sg.prs} to retreat his troops\\
\glt `Lately you cannot trust the news: [reportative:] On Al Jazeera, Obama is going to attack Iran. [reportative:] But  on the BBC, he is going to retreat his troops.'
\z

In \eqref{ex:aljazb}, in the same context, the temporal PPs \emph{hoy por la mañana} and \emph{por la tarde} are contrasted. These temporal PPs can hardly be taken to constitute the source of a report. Once again, these data do not contradict \citeauthor{Etxepare2013}'s analysis but  they are not  expected under the analysis in \citet{Corr2016}. 

\ea\label{ex:aljazb}
 Spanish\\ 
\gll  Ultimadamente uno no se puede fiar de las noticias: Hoy por la mañana, que Obama va a atacar Iran. Pero por la tarde, que va a retirar sus tropas. \\
lately one not \textsc{cl.refl} can.\textsc{3sg.prs} trust in the news  today at the morning \textsc{que} Obama go.\textsc{3sg.prs} to attack Iran but at the afternoon \textsc{que} go.\textsc{3sg.prs} to retreat his troops\\
\glt `Lately you cannot trust the news: [reportative:] In the morning, Obama is going to attack Iran. [reportative:] But in the afternoon, he is going to retreat his troops.'
\z

Moreover, \eqref{ex:aljazc} and \eqref{ex:aljazd} show that more than one phrase can precede reportative \emph{que}. In \eqref{ex:aljazc}, there are two adjacent PPs. Crucially, in this example the element that potentially expresses a source, i.e. \emph{en Al-Jazeera}\slash\emph{en BBC},  is not adjacent to \emph{que}. In \eqref{ex:aljazd}, there are DPs, \emph{el reportero de Al-Jazeera} and \emph{el reportero de BBC} which qualify as sources, and they are followed by a temporal PP. Importantly, this sequence is not predicted by  \citeauthor{Corr2016}'s analysis since she proposes that the phrase encoding the source and \emph{que} occupy the specifier and head of the same phrase (see \citealt[183]{Corr2016}). The data show  that a PP can intervene between a potential source or original speaker and \emph{que}.


\ea\label{ex:aljazc}
 Spanish\\ 
\gll Ultimadamente uno no se puede fiar de las noticias: En Al-Jazeera, hoy por la mañana, que Obama va a atacar Iran. Pero, en BBC, por la tarde, que va a retirar sus tropas. \\
lately one not \textsc{cl.refl} can.\textsc{3sg.prs} trust in the news in Al-Jazeera today at the morning \textsc{que} Obama go.\textsc{3sg.prs} to attack Iran but in BBC at the afternoon \textsc{que} go.\textsc{3sg.prs} to retreat his troops\\
\glt `Lately you cannot trust the news: [reportative:]  On Al-Jazeera in the morning, Obama is going to attack Iran. [reportative:] But  on the BBC in the afternoon, he is going to retreat his troops.'
 
\ex\label{ex:aljazd}
 Spanish\\ 
\gll Ultimadamente uno no se puede fiar de las noticias: El reportero de  Al-Jazeera, hoy por la mañana, que Obama va a atacar Iran. Pero, el reportero de  BBC, por la tarde, que va a retirar sus tropas. \\
lately one not \textsc{cl.refl} can.\textsc{3sg.prs} trust in the news the reporter of Al-Jazeera today at the morning \textsc{que} Obama go.\textsc{3sg.prs} to attack Iran but the reporter of BBC at the afternoon \textsc{que} go.\textsc{3sg.prs} to retreat his troops\\
\glt `Lately you cannot trust the news: [reportative:] Al-Jazeera's reporter, in the morning: Obama is going to attack Iran. [reportative:]  But BBC's reporter, in the afternoon: he is going to retreat his troops.'
\z 


In sum, the  examples in (\ref{ex:etxaljaz}--\ref{ex:aljazd}) show that information source is not  the only possible interpretation for the preceding phrase and that, even if a phrase appears to express an information source, it does not have to be adjacent to \emph{que}, contrary to what we expect based on \citet{Corr2016}. As stated above, the data shown here do not contradict the analysis presented in \citet{Etxepare2013}, where the preceding phrases are treated as modifiers of an elide\is{ellipsis}d noun of saying. Since my aim is to present an analysis that does not need to assume elide\is{ellipsis}d material that could be modified, an alternative explanation is required. I characterize the  phrases that precede \emph{que} as instances of  \citeauthor{Krifka2008}'s frame setters (going back to similar concepts presented in \citealt{Chafe1976} and \citealt{Jacobs2001}). In what follows I discuss the evidence in favor of this idea focusing mainly on the interpretation of the phrases that precede \emph{que}. Although a syntactic analysis of  frame setters will not be fully fleshed out here,  their high syntactic position is an additional indicator that the proposal is on the right track, since  \citet{Beninca2004} showed that frame setters are always CP-external.

The first thing that \citet{Krifka2008} establishes is that frame setters are not \isi{topic}s. In particular, he underlines the contrast between frame setters and aboutness \isi{topic}s, which, according to \citet{Jacobs2001}, have not been separated clearly in the literature. The sentences that are introduced by frame setters   are not \emph{about} the frame. \citet{Krifka2008} illustrates this with the example in \eqref{ex:daimler}. In B's answer the \isi{topic} of the sentences is Daimler-Chrysler and not Germany or America. 

\ea\label{ex:daimler} A: How is business going for Daimler-Chrysler?\\
	B:
	{\ob}In Germany{\cb}\textsubscript{Frame} the prospects are good, but {\ob}in America{\cb}\textsubscript{Frame} they are losing money. (\citealt[269: ex 48]{Krifka2008})
\z


Although there are apparent similarities, \citet{Krifka2008} argues that frame setters  also do not constitute contrastive \isi{topic}s. In  his theory, contrastive \isi{topic}s are defined as aboutness \isi{topic}s  that contain a \isi{focus} that generates alternative aboutness \isi{topic}s. Consequently, from a theoretical standpoint,  the fact that frame setters are simply not what the sentence is about, is sufficient to not consider them contrastive \isi{topic}s in \citeauthor{Krifka2008}'s conception.   Contrastive \isi{topic}s and frame setters, however, also differ in their communicative functions. According to \citet{Krifka2008}, when a contrastive \isi{topic} is uttered, the current \isi{common ground}  contains an expectation that  information about a more comprehensive or distinct entity will be addressed in the conversation. Contrastive \isi{topic}s are used to show that the \isi{topic} of the sentence diverges from this expectation. In \eqref{ex:portbro}, speaker A's question establishes \emph{your sister} as a \isi{topic}. This leads to the expectation of receiving information about her, in particular about whether or not she speaks Portuguese. Speaker B's answer introduces \emph{my brother}  as a \isi{topic} that contrasts with the one introduced by speaker A and thereby shows that the expectations based on A's questions are not met.


\ea\label{ex:portbro} A: Does your sister speak Portuguese?\\
B: {\ob}My brother{\cb}\textsubscript{Contrastive topic} does. (adapted from \citealt[268: ex 46]{Krifka2008})
\z



 With frame setters, \citet{Krifka2008} states that the current \isi{common ground} also contains an expectation that more comprehensive or distinct information will be given. The frame setter indicates that the information is  less comprehensive and is just restricted to a particular dimension specified by the frame.  \citet{Krifka2008} illustrates this function with the example in \eqref{ex:healthwise}. Speaker A's question  leads to the expectation of an answer containing comprehensive information on John's general well-being. Speaker B's answer, however, provides information restricted to one particular area of well-being.  This is possible because the predicate \emph{fine} can be evaluated within different dimensions. John could also be doing \emph{fine}   financially, at work or with his love life, to name just a few. However, the frame restricts the information given about his well-being to the dimension of health. 


\ea\label{ex:healthwise} A: How is John?\\
	B:
	{\ob}Healthwise{\cb}\textsubscript{Frame} he is fine. (\citealt[269: ex 47]{Krifka2008})
\z


The \emph{que}-initial phrases behave like frame setters. In the examples in \eqref{ex:etxaljaz} and \eqref{ex:source} repeated in \eqref{ex:etxaljazrep} and \eqref{ex:sourcerep},  \emph{en Al-Jazeera} and \emph{la mare} are not what the sentence is about, i.e. they are not \isi{topic}s, but they establish a frame in which the proposition should be evaluated. The frame in \eqref{ex:etxaljazrep} gives us the origin  of the statement and in  \eqref{ex:sourcerep} the perspective or speaker that the statement is attributed to.

\ea\label{ex:aljazrepall}
\ea\label{ex:etxaljazrep}
 Spanish (\citealt[98: ex 11]{Etxepare2013})\\ 
\gll Oye, {\ob}en Al-Jazeera{\cb}\textsubscript{Frame}, que Obama va a atacar Iran. \\
		Hey  in Al-Jazeera \textsc{que} Obama go.\textsc{3sg.prs} to attack Iran\\
		\glt `Hey, there’s a saying on Al Jazeera that Obama is going to attack Iran.' 
		\ex \label{ex:sourcerep}  
		Catalan\\ 
		      \gll  {\ob}La mare{\cb}\textsubscript{Frame}, que què t' has cregut. \\
		the mother \textsc{que} what \textsc{cl.refl} \textsc{aux.2sg.prf.prs} think.\textsc{ptcp}\\
		\glt `[reportative:] The mother: what were you thinking-’ (ebook-cat)
	\z
\z


\citet{Krifka2008} furthermore observes that explicit frame setters always allude to alternative frames. This can be seen in \eqref{ex:daimler} where the explicit frames \emph{in Germany} and \emph{in America} are contrasted.  The examples in (\ref{ex:aljaza}--\ref{ex:aljazc}) show that the same is true for \emph{que}-preceding adverbials.  In \eqref{ex:mareijo} the two frames introduce alternative speakers.\largerpage

\ea\label{ex:mareijo}
Catalan\\ 
\gll {\ob}La mare{\cb}\textsubscript{Frame}, que què t’ has cregut. {\ob}Jo{\cb}\textsubscript{Frame}, que no volia tornar-ho a provar. \\
	the mother \textsc{que} what \textsc{cl.refl} \textsc{aux.2sg.prf.prs} think.\textsc{ptcp} I \textsc{que} not want.\textsc{1sg.ipfv.pst} do.again-\textsc{cl.n} to try\\
	\glt `[reportative:] The mother: what were you thinking. [reportative:] I: I didn't want to try it again.’ (ebook-cat)
\z

Finally, the interpretation of these frame setters is   context dependent. Even material that is not mentioned explicitly can play a role.  In   \eqref{ex:etxaljazrep} the frame is interpreted as the origin of a reported statement and in   \eqref{ex:sourcerep} as the speaker of a reported statement. Therefore,  the denotation of a \emph{\isi{verbum dicendi}} clearly plays a role in the interpretation of the frame, even though in neither case  is it part of the words that make up the current sentences. 

To conclude, in this section I have discussed the syntactic properties of unembedded \emph{que}-initial reportatives. I  showed that in these sentences, \emph{que}   precedes  CP-internal material but follows CP-external material. This is in line with the analysis that locates the complementizer in SubP, the highest projection of the split CP. The section has furthermore drawn  parallels with syntactically subordinate indirect\is{indirect speech} reportatives and has demonstrated that they exhibit the same behavior with respect to CP-internal peripheral material and non-declarative sentences and in permitting reportatives that are not full sentences. This corroborates my idea that unembedded \emph{que}-initial sentences are  structurally the same as their embedded counterparts, the only difference being that there is no matrix sentence.  Finally, in the last part of the section, I have  shown that phrases that precede \emph{que} do not always express an original speaker or information source, which is predicted by \citet{Corr2016}, and  have suggested an analysis of these phrases as frame setters.


\section{Contrasting \emph{que}-initial reportatives in Portuguese}\label{sec:insubcross}\largerpage



The empirical evidence in the previous section was drawn from Spanish and Catalan. Portuguese also exhibits instances of \emph{que}-initial sentences  that are interpreted as reported speech. While they do differ in some aspects,  I  claim that  they are still one and the same syntactic phenomenon as in Spanish and Catalan.  In this section I outline my reasons for this claim. I show that the slightly different properties of Portuguese unembedded reportatives can be explained as resulting from independent syntactic properties and  additional pragmatic requirements. The example in \eqref{ex:ptqrep} illustrates a \emph{que}-initial reportative in Portuguese. The sentence that follows the complementizer reports what the gynecologist said to her patient. This is a plausible interpretation because in the preceding context the act of talking to the gynecologist is made salient. 


\ea\label{ex:ptqrep}
 European Portuguese \\
\gll
Fiz anestesia geral mas só lá fiquei uma noite, porque falei logo com o meu ginec. Que não {valia a pena} lá ficar se depois não era vista por mais nenhum médico.\\
make.\textsc{1sg.prf.pst} anesthetic general  but only there stay.\textsc{1sg.prf.pst} one night because talk.\textsc{1sg.prf.pst} then with the my gynecologist \textsc{que} not be.worthwhile.\textsc{3sg.ipfv.pst} there stay if afterwards not be.\textsc{1sg.ipfv.pst} seen by more any doctor\\
\glt `I got a general anesthetic but I only stayed there for one night because I  talked to my gynecologist. [reportative:]  It was not worthwhile staying if in the end I wasn't seen by any other doctor.' (CdP)
\ex\label{ex:highqpt}
\ea\label{ex:highqpta}
 European  Portuguese \\
\gll {O Sr.} gosta de chafurdar na merda ideológica, filosófica, propagandística. Gosta de dizer que é muito bom, sabe muito bem, que possui uma viscosidade suave e um calor aconchegante. {\ob}\textsubscript{SubP} Que{\cb}  {\ob}\textsubscript{ModP} claro que$_i${\cb} {\ob}\textsubscript{FinP} t$_i${\cb} podia ser melhor, e será {uma vez} que estivermos todos afogados nela, e que um gajo até se habitua ao cheiro rapidamente. \\
you like.\textsc{3sg.prs} to wallow in.the shit ideological philosophical propagandistic like.\textsc{3sg.prs} to say that be.\textsc{3sg.prs} very good taste.\textsc{3sg.prs} very well \textsc{que} possess.\textsc{3sg.prs} a viscosity soft and a heat cozy {} \textsc{que} {} claro \textsc{que} {} {}  can.\textsc{3sg.cond} be better and be.\textsc{3sg.fut} once that be.\textsc{1pl.sbjv.fut} all drowned {in her} and \textsc{que} a guy even \textsc{cl.refl} habituate.\textsc{3sg.prs} to.the smell quickly\\
\glt `You like to wallow in ideological, philosophical and propagandistic shit. You like to say that it is very good, it tastes nice. [reportative:]  It has a soft viscosity and a cozy warmth.  [reportative:] Of course, it could be better and it will indeed be once we have all drowned in it. [reportative:] And one even gets used to the smell quickly.' (CdP)
\ex\label{ex:highqptb}
 Brazilian Portuguese \\
\gll Perguntei se ele lembrava de mim e ele disse {\ob}\textsubscript{SubP} que{\cb}  {\ob}\textsubscript{ModP} claro que$_i${\cb} {\ob}\textsubscript{FinP} t$_i${\cb} lembrava.  \\
ask.\textsc{1sg.prf.pst} whether he remember.\textsc{3sg.ipfv.pst}  of me and he say.\textsc{3sg.prf.pst} {}  that {} claro \textsc{que} {} {} remember.\textsc{3sg.ipfv.pst}\\
\glt `I asked whether he remembered me and he said that of course he remembered.' (CdP)
\z
\z

With regard to its syntactic properties, the complementizer in Portuguese unembedded reportatives occupies the same position as in their syntactically embedded counterparts. Furthermore, this position can also be identified with SubP, the highest position in the split CP. Evidence for this is given in \eqref{ex:highqpt}. The examples illustrate that in embedded \eqref{ex:highqptb} and unembedded \eqref{ex:highqpta} reportatives alike the complementizer surfaces in a high position, preceding for instance AdvC, which I analyze as being in ModP. Importantly,  this is the same behavior that we saw for  Spanish and Catalan reportative sentences.






The examples in \eqref{ex:fragpt} show that   Portuguese again behaves in the same way as Spanish and Catalan as far as sentence fragments are concerned. They are permitted in syntactically embedded \eqref{ex:fragptb} and unembedded reportatives \eqref{ex:fragpta}. 

\ea\label{ex:fragpt}
\ea\label{ex:fragpta}
 European Portuguese \\
 \gll
Quando cheguei a casa perguntei à minha mãe se agora as mulheres bebiam cerveja no café. Que sim. E muito. \\
when arrived.\textsc{1sg.prf.pst} at home ask.\textsc{1sg.prf.pst} to.the my other if now the woman.\textsc{pl} drank beer at.the coffeehouse \textsc{que} yes and loads\\
\glt `When I arrived at home I asked my mother whether nowadays women drank beer at the coffeehouses. [reportative:] yes: loads!' (CdP)

\ex\label{ex:fragptb}
 Brazilian Portuguese \\
\gll 40 alunos o total de 71,4\% responderam que o conteúdo associado à aula prática facilita o aprendizado; 6 (10,7\%) responderam que não e 10 (17,9\%) disseram que talvez.  \\
40 student.\textsc{pl} the total of 71.4\% answer.\textsc{3pl.prf.pst} that the content associated to.the class practical facilitat.\textsc{3sg.prs} the learning 6 (10.7\%) answer.\textsc{3pl.prf.pst} that no and 10 (17.9\%) say.\textsc{3pl.prf.pst} that maybe\\
\glt `40 students, a total of 71.4\%, answered that the content of the practical class facilitated their  learning; 6 (10.7\%) answered that it didn't and 10 (17.9\%) said maybe.' (CdP)
\z
 \z


\pagebreak

These data confirm that, with regard to their principal syntactic properties,  unembedded reportatives  in Spanish, Catalan and Portuguese are not distinct from one another. In all three languages, the complementizer occupies the same position, identified  as SubP in the present analysis, in embedded and unembedded reportatives.


 There are two  respects, however, in which Portuguese unembedded reportatives do differ from the other languages. 
The first relates to reported questions. While Spanish and Catalan admit \textit{wh}-pronouns and the interrogative complementizer below \emph{que}, Portuguese does not.  This is illustrated in \eqref{ex:ptonde} for the \textit{wh}-pronoun \emph{onde} and in \eqref{ex:ptvens} for the interrogative complementizer \emph{se} (see also \citealt[178: ex 61]{Corr2016}). Note that the culprit is not the unembedded question introduced by a subordinating element, as shown by  example in \eqref{ex:ptvens}. A's repetition is different from  the initial  question:  It is introduced by the interrogative complementizer \emph{se}, marking the sentence as subordinate.
	

\ea Portuguese 
\ea\label{ex:ptonde}
\gll A: Onde estás? \\
	{} where be.\textsc{2sg.prs}\\
	\exi{}\gll B: {O que} é que disseste? \\
	{} what be.\textsc{3sg.prs} that say.\textsc{2sg.prf.pst}\\
	\exi{}\gll A: (*Que) onde estás.\\
	{} \textsc{que} where be.\textsc{2sg.prs}\\
	\glt `A: Where are you? B: What did you say? A: [reportative:]  where are you.
\ex\label{ex:ptvens}
(adapted from \citealt[178: ex 61]{Corr2016})\\ 
\gll  A: Vens? \\
{} come.\textsc{2sg.prs}\\
\exi{}\gll B: {O que} é que disseste? \\
{} what be.\textsc{3sg.prs} that say.\textsc{2sg.prf.pst}\\
\exi{} \gll A: (*Que) se vens. \\
{} \textsc{que} if come.\textsc{2sg.prs}\\
\glt `A: Are you coming? B: What did you say? A: [reportative:]  are you  coming.' 
\z
\z
	
What is important is that  the inability of \emph{que} to precede a \textit{wh}-pronoun or the interrogative complementizer is  not unique to unembedded sentences but also extends to syntactically embedded reportatives (see \eqref{ex:ptondeemb} and \eqref{ex:ptvensemb}). 

\ea Portuguese 
\ea\label{ex:ptondeemb}
\gll A Joana  pergunta (*que) onde estás. \\
  the Joana ask.\textsc{3sg.prs} that where be.\textsc{2sg.prs}\\
  \glt `Joana asks (that) where you are.'
  \ex\label{ex:ptvensemb}
  \gll A Joana  pergunta (*que) se vens. \\
    the Joana ask.\textsc{3sg.prs} that if come.\textsc{2sg.prs}\\
  \glt `Joana asks (that) if you are coming.'
\z
\z

This differs from what is observed in Catalan and Spanish, where the complementizer \emph{que} can introduce \textit{wh}- and polar interrogatives irrespective of whether the sentence is syntactically subordinate or not (cf. \ref{ex:preceedb}, \ref{ex:preceedc} and \ref{ex:embedreportsa}, \ref{ex:embedreportsb}). 


As briefly mentioned in \sectref{sec:insubwlp}, Spanish and Catalan also permit embedded questions without an initial complementizer \eqref{ex:derespb}, paralleling the Portuguese structure in \eqref{ex:ptondeemb}. In \eqref{ex:deresp} the two structures are contrasted (see also \citealt{GonzalesPlanas2014} for more discussion on these contrasts, and also on embedded \isi{polar question}s and exclamatives). The two sentences are not equivalent:  \eqref{ex:derespa} is an embedded \textit{wh}-question, while \eqref{ex:derespb} is a declarative in which the \textit{wh}-pronoun  is referential. This is demonstrated by the fact that \eqref{ex:derespb}  can admit a continuation that spells out the referent, while the same continuation is infelicitous following \eqref{ex:derespa}. 

\ea\label{ex:deresp} Spanish
\ea\label{ex:derespa}
 \gll Juana repetió que quien vivia en la Rua da Saudade, \#era  Pereira. \\
Juana repeat.\textsc{3sg.prf.pst} that who live.\textsc{3sg.ipfv.pst} in the Rua da Saudade is.\textsc{3sg.ipfv.pst}  Pereira\\
\glt `Juana repeated: who lived in Rua da Saudade: \#it was Pereira.'
\ex \label{ex:derespb}

\gll
Juana repetió quien vivia en la Rua da Saudade: era Pereira. \\
Juana repeat.\textsc{3sg.prf.pst}  who live.\textsc{3sg.ipfv.pst} in the Rua da Saudade is.\textsc{3sg.ipfv.pst}  Pereira\\
\glt `Joana repeated who lived in Rua da Saudade: it was Pereira.'	
\z
\z

The example in \eqref{ex:derept} shows that the  structure  is ambiguous in spoken Portuguese. It admits a continuation that spells out the referent but also a continuation that makes the question-reading prominent. It should be noted that in written text, this ambiguity is resolved through orthography. The question reading would be marked with symbols signaling direct speech. Similarly the Spanish version in \eqref{ex:derespb} would be presented differently in written form.

\ea\label{ex:derept}
Portuguese\\ 
\gll
	A Joana repetiu quem morava na Rua da Saudade: \\
the Juana repeat.\textsc{3sg.prf.pst}  who live.\textsc{3sg.ipfv.pst} in.the Rua da Saudade \\
	\exi{}\gll {\ob}continuation 1:{\cb} era  Pereira. \\
{} {}	is.\textsc{3sg.ipfv.pst}  Pereira\\
	\exi{}\gll {\ob}continuation 2:{\cb} mas ninguem o sabia. \\
{} {}	but nobody it know.\textsc{3sg.ipfv.pst}\\
\glt `Joana repeated who lived in Rua da Saudade: it was  Pereira/but nobody knew it.'
\z



The important empirical generalization is  that Portuguese does not allow \emph{que} to appear before a \textit{wh}-pronoun or an interrogative complementizer in unembedded reportatives, but crucially also not in embedded reportatives, while Spanish and Catalan  allow this ordering in both contexts.  This once again supports   my conclusion that \emph{que}-initial reportatives are simply syntactically unselected reportatives with essentially the same properties as their embedded counterparts. 
What is fundamental to my argumentation is that these facts  permit a uniform syntactic treatment of the Portuguese unembedded reportatives and the equivalent structures in the other two languages.  The behavior of Portuguese \emph{que}-initial reportatives is  expected based on the proposal outlined in this book, which assumes the same structure for unembedded and embedded reported sentences. Indeed, it  actually constitutes  an argument in its favor. 
 Within the general analyses proposed in this book,  the syntactic contrasts between Portuguese on the one hand and Catalan and Spanish on the other,  can be explained as a reflex of a  difference in the feature specification of \textit{wh}-pronouns and interrogative complementizers. I  propose that the contrast boils down to the need to check a feature of a particular head. In  Portuguese \textit{wh}-pronouns and interrogative complementizers necessarily check the subordinate feature in SubP, while in Spanish and Catalan, there is one version where the feature is checked by the \textit{wh}-pronoun and one version where it is not. If the feature is checked by the \textit{wh}-pronoun, the result is the structure in \eqref{ex:derespb}, but if  the feature is not checked by the \textit{wh}-pronoun, a  complementizer is merged to check the feature which then results in the structure in \eqref{ex:derespa}. 






	


The second difference that distinguishes Portuguese \emph{que}-initial reportatives from those in Spanish and Catalan  relates to the contextual requirements that render these sentences felicitous.  The central contrast is illustrated in \eqref{ex:oyeouve}.


\ea \label{ex:oyeouve}
\ea \label{ex:oye}
 Spanish
 \begin{xlist}
\exi{A:}[]{\gll No se oye bien.\\
                not \textsc{cl.refl} hear.\textsc{3sg.prs} well\\}
\exi{B:}[]{\gll  ¿Qué? ¿Eh? \\
				what huh\\}
\exi{A:}[]{\gll  Que no se oye bien. \\
			    \textsc{que} not \textsc{cl.refl} hear\textsc{3sg.prs} well\\}
\end{xlist}
\glt `A: You can't hear it well. B: Huh? A: (I said) you can't hear it well.'
\ex \label{ex:ouve}
Portuguese (adapted from \citealt[148: ex 5]{Corr2016})
\begin{xlist}
\exi{A:}[]{\gll Não se ouve bem.  \\
				not \textsc{cl.refl} hear.\textsc{3sg.prs} well\\}
\exi{B:}[]{\gll  {O que}? Hein? \\
				what huh\\}
\exi{A:}[*]{\gll Que não se ouve bem. \\
				 \textsc{que} not \textsc{cl.refl} hear.\textsc{3sg.prs} well\\}
\exi{B':}[]{\gll  {O que} é que disseste? \\
				 what be.\textsc{3sg.prs} that say.\textsc{2sg.prf.pst}\\}
\exi{A':}[]{\gll Que não se ouve bem.\\
				 \textsc{que} not \textsc{cl.refl} hear.\textsc{3sg.prs} well\\}
\end{xlist}
\glt  `A: You can't hear it well. B': What did you say? A': That you can't hear it well.'
\z
\z




While  \emph{que}-initial reportatives are felicitous  without an explicit \emph{\isi{verbum dicendi}}  in the context  in Spanish and Catalan  \eqref{ex:oye}, they are not in Portuguese where a  \emph{\isi{verbum dicendi}} in the context is required \eqref{ex:ouve}.
 \citet{Corr2016} considers this difference substantial enough   to propose two very different syntactic analyses for Portuguese and the latter two languages (cf. \sectref{sec:insubexistan}). 

One empirical problem faced by \citeauthor{Corr2016}'s proposal  is that    Spanish and Catalan also exhibit examples of the Portuguese type of reportatives, i.e. of \emph{que}-initial sentences where a \emph{\isi{verbum dicendi}} can be recovered from the context, cf. \eqref{ex:oyenew}.

\ea \label{ex:oyenew}
 Spanish\\ 
\gll A: No se oye bien. \\
{} not \textsc{cl.refl} hear.\textsc{3sg.prs} well\\
\exi{}\gll B: ¿Qué dices?  \\
{} what say.\textsc{2sg.prs}\\
\exi{}\gll A: Que no se oye bien. \\
{} \textsc{que} not \textsc{cl.refl} hear.\textsc{3sg.prs} well\\
\glt `A: You can't hear it well. B: What did you say? A: That you can't hear it well.'
\z



The important point is that the only difference between   \eqref{ex:oye} and \eqref{ex:oyenew} is that in the latter  a verb of saying\is{verbum dicendi} is previously mentioned in the context while in the former it is not. Crucially,  no apparent syntactic differences are observed (cf. also the discussion of example \eqref{ex:selfquotrep} in \sectref{sec:insubexistan}).   \citet{Corr2016} unfortunately does not  address how these facts are dealt with in her account, nor is it clear  whether she assumes  two different analyses for \eqref{ex:oye} and \eqref{ex:oyenew}. 

Another aspect that makes  the assumption of two different structures less convincing is that Portuguese unembedded reportatives do not  exhibit any evident  syntactic differences compared to the same construction in Spanish and Catalan. The only exception  is that  \textit{wh}-pronouns and interrogative complementizers cannot follow \emph{que}, which I showed is not  unique to the phenomenon under investigation. On the contrary, the data presented in this section confirm that, just as in Spanish and Catalan, the complementizer behaves in the same way  in syntactically embedded and unembedded sentences. Portuguese reportatives are therefore compatible with an analysis that assumes that the complementizer is merged in the highest CP position. 

Given  the absence  of any real syntactic difference, I consider that the  true contrast between Portuguese on the one hand and Spanish and Catalan on the other resides purely in the pragmatic conditions imposed on the context. These pragmatic conditions are the topic of  \sectref{sec:insubsemprop}. In a nutshell, the difference is that while all three languages require salient material that can function as a host for the subordinate sentence, Portuguese has a stronger requirement: The host material must  be given in the context. This means that in Portuguese a verb of saying\is{verbum dicendi} must have been mentioned previously. While givenness, i.e. a previous mention, is one way to render an expression salient, it is  not the only way to achieve this. Extra-linguistic factors, for instance,  can also come into play. Consequently, in this case, while a given expression is always salient, the reverse implication does not hold: A salient expression is not always given. This is an  important fact that plays a role in the contrast between Portuguese on the one hand and Spanish and Catalan on the other.  The empirical data suggest that  in a conversational exchange such as that in \eqref{ex:oye}  a \emph{\isi{verbum dicendi}} is salient. This  is sufficient to render a \emph{que}-initial reportative sentence felicitous in Spanish and Catalan. In Portuguese, on the contrary, a conversational exchange is not enough: A \emph{\isi{verbum dicendi}} truly must be  given in order to felicitously utter a \emph{que}-initial reportative.



To conclude, in  this section I have shown that Portuguese \emph{que}-initial reportatives share crucial properties with their equivalents in Spanish and Catalan, and I have explored in detail the aspects in which they differ from each other. In line with the empirical facts, I argued that Portuguese reportative \emph{que} constructions are syntactically not distinct from those in Spanish and Catalan. In all three languages they constitute sentences that are marked as subordinate by a complementizer merged in the highest CP projection SubP. Based on this, they are  expected  to behave in the same way as syntactically embedded reportatives. I have shown that this is indeed the case. Furthermore, I  argued that the apparent syntactic differences between Spanish and Catalan on the one hand and Portuguese on the other are not unique to unembedded reportatives  but are also found in embedded reportatives, which ultimately constitutes an argument in favor of the present analysis. Finally, I defended the claim that the true differences  are related to pragmatics. In all three languages, there is a requirement for  contextually salient  material that can function as a host for the subordinate sentence. The difference  is that in Portuguese, the material must be mentioned previously while in Spanish and Catalan, mere saliency of the relevant material is sufficient.


\section{The pragmatics of \emph{que}-initial reportatives}\label{sec:insubsemprop}

The previous sections have focused on the syntactic properties of unembedded reportative sentences. The empirical facts supported my claim that the syntactic behavior of \emph{que} in these sentences does not differ from that of their syntactically embedded counterparts. I furthermore proposed that the  data can be analyzed without needing to resort to an underlying performative or evidential syntactic structure to account  for the reportative interpretation.  The claim I make is that the only  information that the syntactic structure provides  is that the sentence is subordinate.\footnote{I am unaware of any study on the prosody of \emph{que}-initial reportatives. However, potentially, intonation might play an additional role in encoding the meaning of these sentences.} This section presents the pragmatic properties of \emph{que}-initial reporatives and lays out how I assume that the reportative interpretation arises. I propose that this is not encoded in a hidden syntactic structure or a dedicated syntactic feature but instead results from the contexts in which the sentences appear. 
I will argue that when uttering a proposition introduced by \emph{que}, the speaker asserts that the proposition is subordinate. This leads the hearer to infer that there must be a salient linguistic expression that can function as a host for the subordinate proposition.

 As \citet{Corr2016} observes, in \emph{que}-initial reportatives the speaker does not assert the proposition corresponding to the sentence headed by the complementizer  and hence the speaker is not committed to its truth.  \citeauthor{Corr2016} illustrates this by the fact that  with an unmarked  declarative in \eqref{ex:juanaaiquea}, the speaker cannot negate that she believes that the proposition is true, but in \eqref{ex:juanaaiqueb}, with a \emph{que}-initial declarative, she can.

\ea Spanish
\ea\label{ex:juanaaiquea}
  (adapted from \citealt[158: ex 23]{Corr2016})\\ 
\gll Juana y Aique están casados, \#pero no es verdad. \\
		Juana and Aique be.\textsc{3pl.prs} marry.\textsc{ptcp} but not be.\textsc{3sg.prs} truth\\
		\glt `Juana and Aique are married, but it's not actually true.'
		\ex \label{ex:juanaaiqueb} (adapted from \citealt[158: ex 24]{Corr2016})\\
		\gll Que Juana y Aique están casados, pero no es verdad.  \\
		\textsc{que}	Juana and Aique be.\textsc{3pl.prs} marry.\textsc{ptcp} but not be.\textsc{3sg.prs} truth\\
		\glt `Someone said Juana and Aique are married, but it's not actually true.' 
	\z
\z

In this respect, \emph{que}-initial reportatives behave just like indirect\is{indirect speech} embedded reportatives. In \eqref{ex:galileio}  the embedded indirect\is{indirect speech} reportative is felicitous even though it is followed by a continuation that clarifies that the speaker made a statement he does not consider to be true.  

\ea\label{ex:galileio} {\ob}Galileio:{\cb} I said that the earth is flat. But I don't actually believe that it is.
\z



\emph{Que}-initial sentences  do not have an independent illocutionary force (see \citealt{Etxepare2010}). For instance, \emph{que}-initial questions do not have the illocutionary force of a question but that of a declarative.  
The reported question in \eqref{ex:insubilloc} does not require an answer from the hearer. Therefore a continuation where the hearer provides an answer appears odd. However, a reply from the hearer that does not answer the question but expresses his delight about the fact that someone asked about his well-being is perfectly acceptable.

\ea\label{ex:insubilloc}
 Spanish\\ 
\gll María: Antes de salir he hablado con mi madre. Le he dicho que estoy {a punto de} encontrarme contigo. Y ella, que ¿{qué tal} Jorge? \\
	Maria before of leave \textsc{aux.1sg.prf.prs} talk.\textsc{ptcp} with my mother \textsc{cl.3sg} \textsc{aux.1sg.prf.prs} tell.\textsc{ptcp} that be.\textsc{1sg.prs} {about to} meet.\textsc{cl.refl} with.you and she \textsc{que} {how's} Jorge\\
	\exi{}\gll Jorge: \#Estoy bien.\\
	Jorge be.\textsc{3sg.prs} fine\\
	\exi{}\gll Jorge': ¡Qué amable!\\
	Jorge how nice\\
	\glt `Maria: Before I left I talked to my mother. I told her I am about to meet with you. And she was like how is Jorge. Jorge: \#I am fine. Jorge': How nice of her.' 
\z


Another concern  in the literature has been the information status of the reported sentence. \citet{Etxepare2010} and \citet{Corr2016} claim  that the original sentence, which is the basis of the report, needs to be traceable. Traceability in \citet[613]{Etxepare2010}  means that a reported \emph{que}-sentence refers ``to a
contextually identified utterance preceding the report''.  Consequently, in this conception, a sentence is traceable if there is a past speech event that is salient, i.e. accessible to the speaker and the hearer,  in which that sentence was first uttered. \citet{Etxepare2010} illustrates this with the example in \eqref{ex:etxeparetrans}. 

\ea\label{ex:etxeparetrans}  Spanish (adapted from \citealt[613: ex 34]{Etxepare2010})\\  $[$Context: A and B share an office at a bank. B asks A about a particular transaction. A asks C in another office about the transaction. A tells B:$]$\\
\gll	Oye, que ya be.\textsc{3sg.prs} hecho. \\
\textsc{dm} \textsc{que} already be.\textsc{3sg.prs} done\\
\glt `Hey, it's already done.' 
\z

The reportative sentence introduced by \emph{que} is traceable in the context because it can be traced back to the previous speech event in which speaker A asked speaker C about the transaction.

  
There are, however, examples that show that the concept of traceability in its current formulation might be too strong. The  examples in \eqref{ex:futrep} and \eqref{ex:hyprep} show that it is possible to introduce future or  even  hypothetical  utterances with \emph{que}. In these cases, no previous uttering of the sentence could have taken place and thus it is not traceable in the sense of \citet{Etxepare2010}.


\ea\label{ex:futrep}
Catalan\\ 
\gll Avisa el comissari. Que ja pot venir. \\
notify.\textsc{2sg.imp} the inspector \textsc{que} already can.\textsc{3sg.prs} come\\
\glt `Notify the inspector. [reportative:]  He can already come.' (ebook-cat)
\z

In \eqref{ex:futrep} \emph{que} introduces a future utterance. The speaker proposes  the \emph{que}-initial  sentence  to the hearer for him to utter at a future speech event.  In \eqref{ex:hyprep} the \emph{que}-initial sentence is hypothetical.  The speaker thereby introduces a question that  she anticipates before the hearer has even asked it.  
 
\ea\label{ex:hyprep} 
Catalan\\ 
\gll És una cosa que no t’ havia explicat mai, estimada. No vaig gosar dir-t’ho. Que {per què} ho vaig fer? Perquè jo no sóc el meu pare. \\
be.\textsc{3sg.prs} a thing that not \textsc{cl.2sg} \textsc{aux.1sg.ipfv.pst} tell.\textsc{ptcp} ever {my love} not \textsc{aux.1sg.prf.pst} {have courage} tell-\textsc{cl.2sg}.\textsc{cl.n} \textsc{que} why \textsc{cl.n} \textsc{aux.1sg.prf.pst}  do because I not be.\textsc{1sg.prs} the my father\\
\glt `It is a thing that I never told you, my love. I didn't have the courage to tell you. [reportative:]  why I did it. Because I am not my father.' (ebook-cat)
\z






These data show that traceability, in its current definition, does not appear to be a requirement for \emph{que}-initial sentences. Sentences can be introduced by \emph{que} even before the ``original'' sentence was  uttered and even in cases when  it will never be uttered at all. This means that, contrary to what  \citet{Etxepare2010} and \citet{Corr2016} assume, for a \emph{que}-initial reportative to be felicitous, a previous speech event in which the original sentence was first uttered is not a necessary prerequisite.  The data discussed above furthermore show that there are no clear conditions on the information status of the \emph{que}-initial sentence: It can be given as in \eqref{ex:etxeparetrans} but also new  as in \eqref{ex:futrep} and \eqref{ex:hyprep}. 



My own proposal does not focus on the information status of the \emph{que}-initial sentence itself but on other  contextual requirements. There are two important concepts already introduced in \sectref{sec:insubcross}, which are key to the following discussion: Givenness and salience. Given is used here in the sense of \citet{Schwarzschild1999} (but cf. also \citealt{Rochemont2016} for a recent comprehensive definition of givenness). Descriptively, it can be thought of as ``previously mentioned'' (\citealt{Buering2003}). In a straightforward case, an expression is given if there is a previously uttered expression that is identical to it (\emph{woman}–\emph{woman}). However, an expression  can also be characterized as given if there is a hyponym (\emph{woman}–\emph{human}), a co-referent expression (\emph{a woman}–\emph{the woman/she}), or a  semantically vacuous expression (\emph{someone}) that was  mentioned previously.

Salience, in turn, refers  to expressions that are prominent (\citealt{Chiarcos2011}),  activated (\citealt{Chafe1976}) or accessible (\citealt{Ariel1990})  in a specific speech event. An expression can be salient because it was previously mentioned, i.e. is given.  Givenness is however not a necessary requirement for saliency. An expression can also  be salient for other reasons, for instance because it constitutes general knowledge or because extra-linguistic factors make it prominent.

With these concepts in place, I will now show that a felicitous \emph{que}-initial reportative requires a salient \emph{\isi{verbum dicendi}}. As an aside, this requirement also provides support for the argument  that no reportative information is encoded in the syntactic structure of the \emph{que}-initial sentence itself because if it were, the need  for a contextually salient verb of saying\is{verbum dicendi} would not be expected.

In the examples from \eqref{ex:insubcont1} to \eqref{ex:insubcont4}, I exemplify cases of \emph{que}-initial reportatives in different contexts to illustrate my claim. In \eqref{ex:insubcont1} an entire matrix clause can be reconstruct\is{reconstruction}ed from the context. The matrix clause from the previous sentence, \emph{(la mare) va dir} `(the mother) said', is an adequate host for the following \emph{que}-initial reportative sentence and contains the \emph{\isi{verbum dicendi}} \emph{dir} `say'.

\ea \label{ex:insubcont1} 
Catalan\\ 
\gll 
	{Tenia ganes} de     començar o amb el rus o amb     l’arameu, però la mare va entrar a     l’habitació i va dir ni parlar-ne.     Que ja estava bé amb aquelles     llengües que sabia. \\
	want.\textsc{1sg.ipfv.pst} to start either with the Russian or with {the Aramenaic} but the mother \textsc{aux.3sg.prf.pst} enter to {the room} and \textsc{aux.3sg.prf.pst} say {not even} talk.about-\textsc{cl.part} \textsc{que} already be.\textsc{3sg.ipfv.pst} well with these languages that know.\textsc{3sg.ipfv.pst} \\
	\glt `I wanted to start with  Russian or Aramaic but my mother came into my room and said: ``That's out of the question''. That the languages I already knew were enough.'  (ebook-cat)
\z

In \eqref{ex:insubcont2} repeated from \eqref{ex:ouve}, B's question introduces the host material. This example shows that there is no identity requirement: The host material is of a different clause type (interrogative vs. declarative) and has different verbal agreement (second vs. first person singular) than what is expected as a host for the unembedded sentence. Importantly, however, a \emph{\isi{verbum dicendi}} is involved.

\ea \label{ex:insubcont2} Portuguese (adapted from \citealt[148: ex 5]{Corr2016})\\ 
\begin{xlist}
\exi{A:} \gll  Não se ouve bem.  B: {O que} é que disseste? A: Que não se ouve bem.  \\
	not \textsc{cl.refl} hear.\textsc{3sg.prs} well
	{} what be.\textsc{3sg.prs} that say.\textsc{2sg.prf.pst} {} \textsc{que} not \textsc{cl.refl} hear.\textsc{3sg.prs} well\\
	\glt `A: You can't hear it well. B: What did you say? A: That you can't hear it well.'
\end{xlist}
\z

The example in \eqref{ex:insubcont3}  is different because no \emph{\isi{verbum dicendi}} is given. The relevant \emph{que}-initial reportative is uttered within a conversational exchange. This seems to be sufficient to make unembedded reportatives felicitous  in Spanish and Catalan. The conclusion within the present explanation must be that a \emph{\isi{verbum dicendi}} is salient in a conversational exchange.


\ea\label{ex:insubcont3}

 Spanish\\ 
	\gll   Inf.a - Hoy es el día de salida. Inf.b - ¿Eh? Inf.a. - Que hoy es el día de salida. \\ 
	{}  { } today be.\textsc{3sg.prs} the day of departure { } {} Hm? {} {} \textsc{que} today be.\textsc{3sg.prs} the day of depature\\          
	\glt `A: Today is the day of  departure. B: Hm? A: [reportative:]  Today is the day of  departure.' (CdE)
\z

Finally, in the example in \eqref{ex:insubcont4} repeated from \eqref{ex:hyprep}, the \emph{que}-initial reportative appears in a context without a given \emph{\isi{verbum dicendi}} and without an active interlocutor. Just as in the case of \eqref{ex:insubcont3}, in such a context an unembedded reportative is only possible in Spanish and Catalan. In the fragment, there is only one speaker present. However, a hearer is addressed with  \emph{estimada}. This  suggests that activating a hearer perspective is enough for a \emph{\isi{verbum dicendi}} to become salient.


\ea\label{ex:insubcont4}
Catalan\\ 
\gll És una cosa que no t’ havia explicat mai, estimada. {[...]} Que {per què} ho vaig fer?  \\
	be.\textsc{3sg.prs} a thing that not \textsc{cl.2sg} \textsc{aux.1sg.ipfv.pst} tell.\textsc{ptcp} ever {my love} {}  \textsc{que} why \textsc{cl.n} \textsc{aux.1sg.prf.pst} do \\
	\glt `It is a thing that I never told you, my love. {[...]} [reportative:]  why I did it.'	(ebook-cat)
\z



The latter examples, in particular, show that an \isi{ellipsis} analysis is not a useful means of accounting for unembedded reportatives because it would require syntactic reconstruct\is{reconstruction}ion and the preconditions (deletion under identity) are not met. 
These examples are therefore a central piece of evidence illustrating the need for a distinction between syntactic reconstruct\is{reconstruction}ion, required  for \isi{ellipsis}, and pragmatic reconstruct\is{reconstruction}ion, required for the phenomenon at hand. 

How the precise mechanism of pragmatic reconstruct\is{reconstruction}ion works will not be explored in detail here. However, there are other phenomena that show that a  theory of how meaning is picked up  pragmatically is necessary. \citet{Hankamer1976} illustrate that the interpretation of anaphoric pronouns in some cases requires syntactic control (surface anaphora in their terminology), yet  in  other cases pragmatic control (deep anaphora in their terminology) is sufficient. A case of pragmatic control is illustrated in \eqref{ex:saghank}.

\ea\label{ex:saghank} {\ob}Scenario: Sag produces a cleaver and prepares to hack off his left hand]\\
Hankamer: ... he never actually does it. (\citealt[392: ex 6b]{Hankamer1976})
\z

The referents for the pronouns \emph{he} and \emph{it} are not given in the  linguistic context. The pronouns are therefore not controlled syntactically. The referents  can be inferred from the general context because they are salient. According to \citet{Hankamer1976}, they are controlled pragmatically.  \citet{Jacobson2012} introduces the example in \eqref{ex:jacoba}, illustrating the same issue. The pronoun \emph{he} is pragmatically inferred to be \emph{Tony}, a referent who is salient but not given. In \eqref{ex:jacobb}, \emph{it} is understood to mean \emph{diving off the high diving board}, once again a salient yet not given expression.



\ea 
\ea\label{ex:jacoba} {\ob}Scenario:  We are at a party, and a very obnoxious guy named Tony comes in. No one  likes  Tony,  so  no  one  talks  to  him  all  night, and  no  one  mentions  his  name. He leaves, and I turn to you and say:] \\
Thanks goodness he left. (\citealt[4: ex 13]{Jacobson2012})
\ex\label{ex:jacobb} {\ob}Scenario:  I know that for years my friend Chris  has wanted to dive off the high diving board, but every time he gets up there he gets scared and climbs down.  I see him on the high diving board one afternoon, and I turn to you and say:]\\
 Poor Chris, I don’t think he’ll do it.   (\citealt[4: ex 14]{Jacobson2012})
 \z
\z

 \citet{Jacobson2012} makes a  more radical proposal than \citet{Hankamer1976}. She suggests  that the interpretation of anaphoric pronouns always  involves  pragmatically picking up a salient referent. In this sense, syntactic control is an illusion that appears to exist because  one very efficient means of making  things salient is to explicitly mention them. Notably, this is very close to what the present approach assumes for the reconstruct\is{reconstruction}ion of the host material of the \emph{que}-initial sentences. The minimal requirement to allow an unembedded sentence to be felicitously uttered is a salient  expression that can function as its host. The  expression can be salient in the general (extra-linguistic) context (cf. examples \ref{ex:insubcont3}--\ref{ex:insubcont4}) or it can be made salient  through an explicit mention (cf. examples \ref{ex:insubcont1}, \ref{ex:insubcont2}). 





To conclude, in  \sectref{sec:insubsynprop} and \sectref{sec:insubcross}, I argued that there are no real syntactic differences between \emph{que}-initial reportatives in Spanish and Catalan on the one hand and Portuguese on the other. I furthermore suggested that the behavior of \emph{que} matches that of a typical complementizer, which goes against the assumption that  \emph{que} is an evidential marker in these sentences. The conclusion I draw is that unembedded reportatives are essentially the same in all three languages and can be analyzed  uniformly. This section  has explored  the pragmatics of \emph{que}-initial reportatives. I argued that the minimal contextual requirement for a \emph{que}-initial reportative is a salient \emph{\isi{verbum dicendi}} that can function as a host for the subordinate sentence.  Portuguese differs from Spanish and Catalan in that a \emph{\isi{verbum dicendi}} needs to be given while in Spanish and Catalan givenness is not required.  

\section{Beyond reportatives}\label{sec:beyondrep}
The focus of this section is to present further evidence for the analysis laid out over the course of this chapter by showing that there is no one-to-one-relation between a \emph{que}-initial
sentence and a reportative interpretation. On the contrary, I illustrate in this section that depending
on the context a \emph{que}-initial sentence can receive a variety of different interpretations. Since the
approach developed here does not allude directly to any reportative meaning, this fact does not come as a surprise. In\isi{subordination} is not specialized for reports but it is a more pervasive
phenomenon in Ibero-Romance languages. This section shows how these data can be accounted for in the present analysis. The central idea that my analysis is built upon is that the syntactic structure
of unembedded sentences is essentially the same  as that  of their embedded counterparts. The complementizer surfaces in the highest position of the left periphery and receives
a subordinate feature that has implications for the interpretation of the sentence: Namely
that the sentence is semantically subordinate to a salient linguistic expression. There are two
 important empirical facts that support this analysis. Firstly, \emph{que}-initial sentences show the
same syntactic properties as their embedded counterparts. Secondly, they are only felicitous
in contexts in which a plausible host is salient.


In what follows, I present a number of examples of \emph{que}-initial sentences that are not interpreted
as reports. The first set of data are  sentences in contexts of mental predicates
and as answers to questions. The second set are unembedded relative sentences.

In the examples in \eqref{ex:mentalp}, a \emph{que}-initial sentence appears in the context of a given mental
predicate. These examples are very similar to the unembedded reported sentences presented
in the previous sections,  some of which also appeared in the context of a given verb
of saying. In \eqref{ex:mentalpa} the juxtaposed \emph{que}-initial sentences are understood to be subordinate to
the mental predicate \emph{mentalizar} `internalize', which appears in the matrix clause of the first
sentence of the fragment. In \eqref{ex:mentalpb}, the unembedded sentence appears in the context of the given mental predicate \emph{pensar} `think'. This case differs from that illustrated in \eqref{ex:mentalpa} in that the mental predicate is uttered by a different speaker than the one who utters the
\emph{que}-initial sentence.
\ea 
	\label{ex:mentalp}
\ea	\label{ex:mentalpa} 
 Brazilian Portuguese \\
 \gll Mentalize que você criou
seu filho bem. Que você {escolheu a dedo} a
sua escola. Que você sempre pode criar um laço de amizade com a professora. \\
mentalize.\textsc{2sg.imp} that you {bring.up}.\textsc{3sg.prf.pst} your child well \textsc{que} you handpick.\textsc{3sg.prf.pst}
the
his school \textsc{que} you always can.\textsc{3sg.prs} create a tie of friendship with the
teacher
\\
\glt  `Internalize that you brought your child up well. (Internalize) that you handpicked
his school. (Internalize) that you can always create a bond of friendship with the
teacher.' (CdP) 
\ex \label{ex:mentalpb} 
 Spanish\\ 
\gll 	 Enc.
¿qué piensa
de su trabajo? Inf. Que es agobiador.
\\
interviewer what think.\textsc{3sg.prs} of your work
informant \textsc{que} be.\textsc{3sg.prs} exhausting
\\
\glt `Interviewer: What do you think about your job? Informant: (That) it is exhausting.' (CdE)
	\z
\z

The analysis proposed here does not explicitly allude to \emph{verba dicendi}, so  \emph{que}-initial sentences in contexts of mental predicates like \eqref{ex:mentalpa} can be easily accommodated. The salient expression that is pragmatically reconstruct\is{reconstruction}ed as the host of the subordinate sentence is simply a mental predicate. 

Answers to \textit{wh}-questions,
as illustrated in \eqref{ex:mentalpb}, are also typical contexts for \emph{que}-initial sentences. Often the question contains a mental predicate like in \eqref{ex:mentalpb} or a verb of saying\is{verbum dicendi} as in \eqref{ex:oquea}. However, there are also examples
like \eqref{ex:oqueb}, where neither is the case.
\ea \label{ex:oque}
\ea \label{ex:oquea} 
 Brazilian Portuguese \\
\gll 
		Então o período de escravidão no
Brasil é amplo, diversificado, mas o 
livro didático, durante muito tempo disse o quê? Que você era escravo, propriedade de alguém, que apanhava, que  era inferior, que aceitava a escravidão. \\
so the period of slavery in.the Brazil be.\textsc{3sg.prs} ample diversified
but the
book didactic during much time say.\textsc{3sg.prf.pst} the what \textsc{que} you be.\textsc{3sg.ipfv.pst} slave property of someone \textsc{que} {take.a.beating}.\textsc{3sg.ipfv.pst} \textsc{que} be.\textsc{3sg.ipfv.pst}  inferior \textsc{que} accept.\textsc{3sg.ipfv.pst}  the slavery
\\
\glt `So the period of slavery in Brazil was ample and diversified. Still, what did school
books say for a long time? (That) you are a slave, somebody’s property. (A slave
that) took a beating, (that) was inferior and (that) accepted slavery.' (CdP)

\ex \label{ex:oqueb}
 Spanish\\ 
\gll  Cuáles son las falsas leyendas en la vida de George Sand? - Que sus dos grandes amores fueran Alfred de Musset y Chopin, cuando en realidad han
contado poco. \\
which be.\textsc{3pl.prs} the false legend.\textsc{pl} in the life of George Sand {} \textsc{que} her two
big love.\textsc{pl} be.\textsc{3pl.prf.pst}  Alfred de Musset and Chopin when in reality \textsc{aux.3pl.prf.prs} count.\textsc{ptcp} little\\
\glt `What are the false legends of George Sand’s life? - That her two big love affairs
were with Alfred de Musset and Chopin when really they were of little importance.' (CdE)
	\z
\z


These examples could be viewed as cases of fragmented answers. Many analyses of these
 types of reduced answers assume deletion under identity stating that parts of the
answer can be elide\is{ellipsis}d because they are semantically identical to the content of the question.\footnote{On formal and semantic identity see \citet[353--356]{Jacobson2016}.}
\citet{Jacobson2016}, however, disagrees with these analyses. She shows that, contrary to what these types of analyses predict, short, fragmented and long, full answers are not semantically identical.
 One point of divergence is that in the short answer by B in \eqref{ex:jill}, the speaker is committed
to the belief that Jill is a mathematics professor. This makes the continuation in which the
speaker withdraws the commitment infelicitous. With the long answer in B’ the speaker offers
the best she can do, but does not commit herself to the belief that Jill is a mathematics
professor. Therefore the continuation is felicitous.

\ea \label{ex:jill}
	A: Which mathematics professor left the party at midnight?\\
	B: Jill. \#Whether she is a mathematics professor I’m not sure.\\
	B’: Jill left the party at midnight. Whether she is a mathematics professor I’m not sure.
	(adapted from \citealt[342: ex 14]{Jacobson2016})
\z

Similarly, the long and short answers in \eqref{ex:esclavos} are not identical. In the short answer the speaker
is committed to the fact that what the book says about slavery in Brazil is that the slaves
are someone’s property. In the long answer she is not committed. She might just offer her best
guess by saying something the book says about slavery in general but not necessarily about
the particular case of slavery in Brazil.

\ea\label{ex:esclavos} 
 Spanish\\ 
\gll  A: Qué dice el libro sobre esclavitud en Brasil? 
\\
	{} What say.\textsc{3sg.prs} the book about slavery
	in Brazil\\
\exi{}  \gll B: Que los esclavos eran propiedad de alguien.
\\
{} \textsc{que} the slave.\textsc{pl} be.\textsc{3pl.ipfv.pst} property of someone
\\
\exi{} \gll B’: El libro dice que los esclavos eran propiedad de alguien.
\\
{} the book say.\textsc{3sg.prs} that the slave.\textsc{pl} be.\textsc{3pl.ipfv.pst} property of someone
\\
\glt  `A: What does the book say about slavery in Brazil? B: That slaves were someone’s
property. B’: The book says that slaves were someone’s property.' (CdE)
\z


Based on these facts, \citet{Jacobson2016} develops an analysis that accounts for short answers
without the need to allude to silent material. Without going into the details of her analysis, one important
prediction it makes is that a short answer can only be correctly interpreted when there is
a salient question in the context. Importantly, this is very close to the felicity conditions I
identified for \emph{que}-initial sentences.

Finally, there are also unembedded relative clauses. One example is given in \eqref{ex:gentque}. The  \emph{que}-initial sentence  is understood as a modifier  that encodes yet another property of the NP \emph{gent} `people' from
the preceding matrix sentence. The interpretation of the \emph{que}-initial sentence is possible once again because there is a salient linguistic expression in the context that can
function as a host for the subordinate sentence.
\ea \label{ex:gentque} 
Catalan (\citealt[52: ex 103]{Kocher2017a})\\ 
\gll  És gent que segur que has vist, però mai has passat un {cap de setmana} amb ells. Que segur que conèixes però que mai te ha convitat a casa seva. \\
	be.\textsc{3sg.prs} people that sure que \textsc{aux.2sg.prf.prs} see.\textsc{ptcp} but never \textsc{aux.2sg.prf.prs} pass.\textsc{ptcp} a weekend
	with
them \textsc{que} sure \textsc{que} know.\textsc{2sg.prs}
	but that never \textsc{cl.2sg} \textsc{aux.3sg.prf.prs} invite.\textsc{ptcp} to house their
\\
	\glt `These are people who surely you have met but never have passed a weekend with. 	Who surely you know but who never have invited you over to their house.' 
\z


The example in \eqref{ex:gentque} shows that unembedded relatives share core properties that we have identified
above. Here too, \emph{que} shows the same syntactic behavior in embedded and unembedded
sentences. The complementizer occupies the same high position in both cases. In both
 sentences it precedes \emph{segur que}, which is analyzed as being in ModP. This is compatible with the analysis
presented in this chapter that \emph{que} occupies a high position in the left periphery, which is potentially also  SubP in this case.
The example also indicates that \emph{que}-initial relatives, like the other types of unembedded
sentences, require a salient linguistic expression that can function as a host for the subordinate
sentence. In the case of relative clauses this is the modified noun phrase. In \eqref{ex:gentque}, a parallel structure consisting of  an entire matrix clause and a relative clause can be recovered from the linguistic context.  However, there are also examples where this is not the case. In \eqref{ex:ellaa} and \eqref{ex:ellab} the \emph{que}-initial sentences are understood as relative clauses modifying the noun phrases \emph{ella} and  \emph{gente}, without a parallel relative structure in the context. These examples show that, just as with \emph{que}-initial reportatives,  a salient potential host expression is sufficient to render a \emph{que}-initial sentence  felicitous. This suggests  that an analysis that does not allude to syntactic \isi{ellipsis}, is not just tenable but actually required to account for these data.


\ea 
\ea\label{ex:ellaa} 
 Spanish\\ 
\gll  Era extraño interrumpir sus labores y dedicarle
un minuto a ella. Que
había pertenecido a un mundo tan diferente. Que había pertenecido, pero que ya
no pertenecía. ¡Que había dado un paso tan difícil de explicar! \\
		be.\textsc{3sg.ipfv.pst} strange interrupt his works and dedicate.\textsc{cl.3sg} a minute to her \textsc{que} \textsc{aux.3sg.ipfv.pst} pertain.\textsc{ptcp} to a world so different \textsc{que} \textsc{aux.3sg.ipfv.pst} pertain.\textsc{ptcp} but \textsc{que} anymore not pertain.\textsc{ptcp} \textsc{que} \textsc{aux.3sg.ipfv.pst} do.\textsc{ptcp} a step so hard to describe\\
		\glt `It was strange for him to interrupt his work and dedicate a minute to her. Who
belonged to a world so different. Who used to belong there but didn’t belong
		there anymore. Who took a step so difficult to explain.' (CdE)
		\ex\label{ex:ellab} 
		 Brazilian Portuguese \\
		\gll  Há gente decente neste país. Que trabalha. Que não depende de benesses,
padrinhos ou tutores. Que ouve e não aceita, mesmo em silêncio. \\
		there.be.\textsc{3sg.prs} people decent in.this country \textsc{que} work.\textsc{3sg.prs}
\textsc{que} not depend.\textsc{3sg.prs} on benefit.\textsc{pl} patron.\textsc{pf} or guardian.\textsc{pl} \textsc{que} hear.\textsc{3sg.prs} and not accept.\textsc{3sg.prs} even in silence
\\
		\glt `There are decent people in this country. Who work. Who don’t depend on benefits,
	patrons or guardians. Who hear but don’t accept even if only in silence.' (CdP)
	\z
\z

A core argument that we are  dealing with pragmatic rather
than syntactic reconstruct\is{reconstruction}ions  comes from examples like that in \eqref{ex:volver}.


\ea \label{ex:volver}
Catalan\\ 
 \gll \emph{Volver} es una de las pel·lícules que vaig veure amb en Jordi. Que tota la gent ens recomanava però que no ens va agradar gaire. \\
	\emph{Volver} be.\textsc{3sg.prs} one of the movies
	that \textsc{aux.1sg.prf.pst} see with the Jordi \textsc{que} all the people 	\textsc{cl.2pl} recommend.\textsc{3sg.ipfv.pst} but \textsc{que} not \textsc{cl.2pl} \textsc{aux.3g.prf.pst} please {at all}
\\
	\glt `\emph{Volver} is one of the movies that I saw with Jordi. (That) everybody recommended to
us but (that) we didn’t like at all.'
	\ea\label{ex:volvera} Everybody recommended a number of movies, one of them \emph{Volver}, but we didn’t
	like them at all.
	\ex\label{ex:volverb} Everybody recommended \emph{Volver} but we didn’t like it at all.
\z
\z 

\eqref{ex:volver} is an example of an unembedded relative in the context of another relative clause. \eqref{ex:volvera}
and \eqref{ex:volverb} paraphrase two different readings of the unembedded sentence. If the entire matrix
clause were syntactically reconstruct\is{reconstruction}ed we would expect a reading of the \emph{que}-initial
sentence  like \eqref{ex:volvera}. However, this is not the typical reading  for \eqref{ex:volver}. On the contrary, the \emph{que}-initial sentence can be paraphrased by \eqref{ex:volverb}, in which it functions as a relative clause modifying
one salient expression: \emph{Volver}. This shows that even  when it is possible to
syntactically reconstruct\is{reconstruction} a matrix clause, it does not happen. This is  strong evidence that
something other than syntactic reconstruct\is{reconstruction}ion is taking place, namely pragmatic reconstruct\is{reconstruction}ion, as  in my proposal.


This section showed that the central predictions following from the analysis developed for \emph{que}-initial reports are also fulfilled by non-reportative \emph{que}-initial sentences. This suggests
that we are dealing with a relatively general phenomenon in which sentences  are marked
as subordinate and require a salient linguistic expression in order to be interpreted. 

Finally, it is interesting to note that  the translations of the examples discussed along the course of this section show that this phenomenon is not restricted to Ibero-Romance varieties. It appears to extend at least to English. At this stage, it would be premature  to claim  that we are dealing with the exact same phenomenon, since this would require a detailed comparison and thorough analysis. However,  it does hint at the fact that my new conception of \isi{subordination} and its implications is not only supported by Ibero-Romance but also by other languages.   

\section{Summary}
This chapter explored   \emph{que}-initial reported sentences. I examined their properties and developed an analysis to account for them. As a  new theoretical concept, I established a distinction between  selected and  unselected subordinate sentences. While both are interpreted as subordinate, only  selected subordinate sentences are syntactically dependent on a matrix clause.  This theoretical redefinition  makes it possible to treat \emph{que} as a normal complementizer, and makes it unnecessary to  stipulate a new linguistic category to account for its atypical behavior. It furthermore permits an analysis that does not rely on the \isi{ellipsis} of a matrix clause nor the assumption of a hidden performative structure.
Moreover, I showed that the complementizer in unembedded \emph{que}-initial  sentences surfaces in the same  position as the complementizer  in embedded contexts, suggesting that these complementizers can be analyzed in the same way, namely as heads of the highest left-peripheral position, SubP,  valued with the interface feature \emph{subordinate}. 
The syntactic analysis is  simple and, contrary to  previous accounts,  as mentioned, does not assume a hidden syntactic structure that  contributes the special reportative meaning. 
In my approach, the interpretation results from pragmatic rather than syntactic reconstruct\is{reconstruction}ion. This assumption is supported by the fact that the types of \emph{que}-initial sentences discussed in this chapter require a salient linguistic expression that can function as a host for the subordinate sentence. 


This chapter also presented a description of the pragmatic requirements. In order to formalize the generalizations in the future,  a number of questions need to be addressed. One of them is the concept of pragmatic reconstruct\is{reconstruction}ion. In order to reach a useful formal definition further research is required  to determine what it is that is reconstruct\is{reconstruction}ed (salient expression, entire matrix clause, etc.). Furthermore,  we need to investigate the questions of what being a host for a subordinate sentence means and what properties  the salient host has to have to be identified as such. 

The simplicity and
generality of the analysis predicts that we should expect cases of unembedded sentences that
have an interpretation other than that of a reportative. This is why,  in the last section of this chapter, I turned to non-reportative unembedded sentences and showed that their properties can indeed
be accounted for in principle by the same analysis developed for \emph{que}-initial reportatives. In other words,
they are also  marked as subordinate and are only felicitous in a context where a
 host expression can be recovered from the context.

This chapter provided evidence for a clear distribution of labor between syntax and pragmatics. I argued that the information that is read off  of the syntactic structure strictly pertains to this component of grammar. The additional information that has an effect on the interpretation of the sentence in its context is contributed by pragmatics. 

