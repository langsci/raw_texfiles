\chapter{Morphology}\label{sec-morh-nominal}
This chapter covers the basics of inflectional morphology and addresses some aspects of derivation morphology. 
We will first consider the inflectional morphology of nouns and pronouns (Sections~\ref{sec-morh-noun-class} and~\ref{sec-pronouns}). 
We then proceed with the inflectional categories of the verb (Section~\ref{sec-morh-verb}). 
Further sections are dedicated to adjectives (Section~\ref{sec-adjectives}) and adverbs (Section~\ref{sec-adverbs}). 
Finally, Section~\ref{morpho-derivation} discusses some aspects of derivational morphology.


\section{Nouns} \label{sec-morh-noun-class}

This section provides an outline of the morphological structure of Ru\-ruu\-li\hyp{}Lu\-nya\-la nouns. 
It describes the noun class system and discusses issues of number and meaning relating to the noun classes and noun class pairs. 
It also addresses some aspects of nominal derivation. 

As in other Bantu languages, most nouns in Ru\-ruu\-li\hyp{}Lu\-nya\-la consist of at least a nominal prefix and a stem. 
Every word form of a noun belongs to a particular noun class and in many cases the prefix taken by a word form of a noun signals the noun class to which it belongs. 
However, some nouns do not have such noun prefixes, and ultimately the allocation of a noun to a noun class is determined by the agreement prefixes (or concord) on various targets within a clause, e.g.\,on verbs (Section \ref{sec-verb-indexing}) or adjectives (Section~\ref{sec-adjectives}) and not by the prefix on the noun itelf.

It is common to describe Bantu nouns in pairs of two classes so that one represents the word-form in the singular and the other one represents the word-form in the plural.\footnote{For a recent critique of the traditional treatment of gender systems in Niger-Congo and some alternative suggestions, see \citet{Gldemannetal2019Niger-Congo}.}
These pairs of singular and plural forms are occasionally referred to as \textit{gender} in Bantu linguistics (e.g.\,in \citealt[103]{Katamba2003Bantu}) and we will use this term as well as the term \textit{singular-plural pairing}.in this grammar.  
Some nouns do not undergo such a pairing, i.e.\,they have only one form, either the one which corresponds to a singular or the one which corresponds to a plural form. 
These are so called mono-classes, which are discussed in Section~\ref{sec-morh-unpaired}. 
Which gender a noun belongs to (i.e.\,singular-plural noun class pair or mono-class, if a noun has only one word-form)  is lexically determined, though the allocation to some genders can be semantically motivated. 
The Dictionary part of this book (Part~\ref{sec-Dictionary}) indicates the singular-plural pair of noun classes for each Ruruuli-Lunyala noun.

Noun classes in the Bantu languages are traditionally numbered from 1 to 23 depending on the assumed relationship to the respective noun classes in Proto-Bantu (see e.g.\,\citealt{Vandeveldeetal2019Nominal}). 
Individual Bantu languages regularly have fewer than 23 noun classes, that is, some noun classes are unattested in individual languages. 
In Ru\-ruu\-li-Lu\-nya\-la, there are 18 non-locative noun classes, as well as four locative ones. Correspondences to Proto-Bantu classes 19 and 21 are not attested. 

As in some other Bantu languages, in addition to the noun class prefixes, the nominal prefix in Ru\-ruu\-li\hyp{}Lu\-nya\-la is frequently preceded by a vowel prefix traditionally referred to as \textit{augment}, \textit{pre-prefix} or \textit{initial vowel}. 
However, if the stem itself begins with an identical vowel, as is occasionally the case with loanwords and proper names, the augment cannot be prefixed, as e.g.\,\textit{akaivu} (*\textit{eakaivu}) `archive (9)' or \textit{Alena} (*\textit{oAlena}) `Alena (1a)'.

Below, we first provide an overview of the individual classes and discuss the various allomorphs (Section~\ref{sec-morh-noun-classes-overview}). 
We  then proceed with the presentation of noun class pairings (Section~\ref{sec-morh-pairings}) and discuss unpaired groups of nouns (Section~\ref{sec-morh-unpaired}). 
Next, the issue of noun class reassignment and diminutive and augmentative derivation are discussed in Section~\ref{sec-morh-diminutive}. 
Section~\ref{sec-morh-locative} discusses locative noun classes.


\subsection{Overview of the noun class system}\label{sec-morh-noun-classes-overview}

Most nouns have segmentable noun class prefixes cognate to the ones found in many other Bantu languages. Ru\-ruu\-li\hyp{}Lu\-nya\-la noun class prefixes are very similar to the ones reconstructed for Proto-Bantu (cf.\,e.g.\,the overview in \citealt[247]{Maho1999Comparative}). 
Table~\ref{tab-NC-morphology} gives an overview of the class prefixes, the respective augments, and  examples. 
The locative noun classes, viz.\,classes 16, 17, 18, and 23, are treated separately in Section~\ref{sec-morh-locative}. 
Only one noun in the dictionary (\textit{(a)waikirwa} `reception') was identified as inherently belonging to class 16.


\begin{table}
\caption{Noun classes and noun class prefixes}\label{tab-NC-morphology}
\begin{tabularx}{\textwidth}{X l X l }
\lsptoprule
 
Noun & Augment & Prefix & Examples\\
class &  &  & \\
\midrule
1	& \textit{o-} & \textit{mu-} &  \textit{(o)mukoli} `worker',\\
	 & 		& 		 & 	\textit{(o)mweryana} `daughter-in-law'\\
1a	& \textit{o-} & \textit{∅-} & \textit{(o)Kaawa} `Eve', \textit{(o)gaasi} `gas'\\

2	& \textit{a-} & \textit{ba-} & \textit{(a)bakoli} `workers',\\
	& 	& 	& 		\textit{(a)beryana} `daughters-in-law'\\

3 & \textit{o-} & \textit{mu-} &  \textit{(o)mufuko} `bag', \textit{(o)mwezi} `month'\\
3a & \textit{o-} & \textit{∅-}  &  \textit{(o)gumoso} `left hand'\\
4 & \textit{e-} & \textit{mi-} &  \textit{(e)mifuko} `bags', \textit{(e)myezi} `months'\\

5 & \textit{e-} & \textit{i-} & \textit{(e)ibbaale} `stone'\\
5a & \textit{e-} & \textit{∅-} &  \textit{(e)dagala} `medicine'\\
5b & \textit{e-} & \textit{li- (ri-)} &  \textit{(e)riiso} `eye'\\
6 & \textit{a-} & \textit{ma-} &  \textit{(a)mabbaale} `stones', \textit{(a)maiso} `eyes'\\
7 & \textit{e-} & \textit{ki-} & \textit{(e)kibbambo} `bundle'\\
8 & \textit{e-} & \textit{bi-} &    \textit{(e)bibbambo} `bundles'\\

9 & \textit{e-}	& \textit{n-}	& 	 \textit{(e)nsansa} `palm leaf',  \textit{(e)mpagi} `mast, pole'\\
9a & \textit{e-}	& \textit{∅-} 	&   \textit{(e)bbaasi} `bus'\\
10 & \textit{e-}	& \textit{n-}	&  \textit{(e)nsansa} `palm leaves', \\
 	& 		& 			&  \textit{(e)mpagi} `masts, poles'\\
10a & \textit{e-} & \textit{∅-} &   \textit{(e)bbaasi} `buses'\\ 

11 & \textit{o-} & \textit{lu-} &   \textit{(o)lubaju} `side'\\ 

12 & \textit{a-} & \textit{ka-} &     \textit{(a)kabbaani} `incense', \textit{(a)katiko} ‘mushroom’\\

13 & \textit{o-} & \textit{tu-} & \textit{(o)tuceere} ‘little rice’\\

14 & \textit{o-} & \textit{bu-} &  \textit{(o)bubbaani} `incenses', \textit{(o)butiko} ‘mushrooms’\\
 &  &  &  \textit{(o)bworo} `poverty'\\
15 & \textit{o-} & \textit{ku-} &  \textit{(o)kugulu} `footstep', \\
 & 			 & 		 & \textit{(o)kwerema} `independence'\\
16 & \textit{a-} & \textit{wa-} &     \textit{(a)waikirwa} `reception'\\
20 & \textit{o-} & \textit{gu-} &    \textit{(o)gusolo} `large animal'\\
22 & \textit{a-} & \textit{ga-} &      \textit{(a)gaitimba} `large pythons'\\

\lspbottomrule
\end{tabularx}
\end{table}

A number of nouns, however, deviate from this pattern and do not have any prefix in the singular form. 
When nouns in the singular form have varying prefixes (including zero), but have the same plural form and trigger the same agreement on various targets, it is common to allocate them to subclasses labelled as the respective class plus a letter, e.g.\,1a (cf.\,\citealt{Vandeveldeetal2019Nominal}). 
Whereas most of such subclasses and their labels are language specific, the subclass commonly labelled class 1a following \citet{Doke1927Textbook} occurs throughout Bantu. 
The nouns of noun class 1a are characterised by the absence of any noun class prefix, but otherwise behave like regular class 1 nouns. 
In Ru\-ruu\-li-Lu\-nya\-la, as in other Bantu languages, this class typically contains proper names, kinship terms, personified animals and borrowings. 
This subclass is explicitly indicated as 1a in the dictionary (see Part~\ref{sec-Dictionary} Dictionary).

In Ru\-ruu\-li\hyp{}Lu\-nya\-la several further subclasses can be postulated. 
Similarly to class 1a, these contain nouns which lack a noun class prefix but trigger agreement of the respective noun class. 
Thus, in addition to a large subclass 1a, there is a sizeable subclass 9a, as well as small subclasses 3a, 5a, 10a, 12a, and 14a with just a few words each. 
For instance, only two lexemes belong to class 3a viz.\,\textit{gulyo} `right side, right hand' and \textit{gumoso} `left side, left hand'.

No noun in the dictionary inherently belongs to the noun classes 20 and 22. 
The respective prefixes are used derivationally to form augmentative nouns. 
They substitute the inherent noun class prefixes, as discussed in Section~\ref{sec-morh-diminutive}.
Furthermore, only one noun in the dictionary belongs to class 13, viz.\,\textit{tulo} `sleep'. Otherwise the respective prefix \textit{tu-} is used derivationally for diminutive formation. 

Most noun class prefixes have phonologically conditioned allomorphs. 
Thus, before vowels \textit{mu-} `1' and `3' is realised as \textit{mw-}, \textit{lu-} `11' as \textit{lw-}, \textit{tu-} `13' as \textit{tw-}, \textit{bu-} `14' as \textit{bw-}, \textit{ku-} `15' as \textit{kw-}, and \textit{gu-} `20' as \textit{gw-} (see Section~\ref{sec-phonology-glide-formation} on glide formation). 
Following the same rule, \textit{mi-} `4' is realised as \textit{my-}, \textit{li-} (or \textit{ri-}) `5b' as \textit{ly-} (or \textit{ry-}), \textit{ki-} `7' as \textit{ky-}, and \textit{bi-} `8' as \textit{by-}.

The process of nasal assimilation discussed in Section~\ref{sec-phonology-nasal-assimilation} determines the realisation of the
noun class 9 and 10 prefix \textit{n-}: it is realised as \textit{m-} before the labials /b/ and /p/ and as [ŋ] (<n> in the orthography) before the velar consonants /k/ and /g/.


\subsection{Singular-plural pairings}\label{sec-morh-pairings}

Most nouns (slightly over 70\%) have both a singular and a plural form, which belong to two different classes. 
As in many other Bantu languages, nouns from a specific class in the singular form take a plural form from a corresponding noun class in the plural. 
However, there are exceptions to this generalisation. 
Of over 4400 nouns in paired noun classes the most common pairs (in descending order) are 7/8, 9/10, 1/2, 1a/2, 1a/10, and 3/4. 
When counted together, the pairs 1/2 and 1a/2 comprise the most frequent singular-plural pairing.  
Table~\ref{tab-pairings} lists attested pairings and provides examples. 
The size of these classes is large: all but the pairings 5a/6 and 5b/6 have over a hundred nouns.


\begin{table}
\caption{Singular-plural pairings}
\begin{tabular}{l l l p{6.5cm} }
\lsptoprule
Pair	& Prefix &  			& Examples\\
	& singular	 	& plural	& 	\\
\midrule

1/2	& \textit{mu-}	& \textit{ba-} & \textit{(o)mukoli}/\textit{(a)bakoli} `worker(s)'\\
1a/2	& \textit{∅-}	& \textit{ba-} & \textit{(o)fundi}/\textit{(a)bafundi} `tailor(s)' 	\\
1a/10	& \textit{∅-}	& \textit{n-} & \textit{(o)njango}/\textit{(e)njango} `cat(s)' 	\\

3/4	& \textit{mu-}	& \textit{mi-} & \textit{(o)muganda}/\textit{(e)miganda} ‘bundle(s)’	\\

5/6	& \textit{i-}	& \textit{ma-} & \textit{(e)ibeere}/\textit{(a)mabeere} ‘breasts(s)’	\\
5a/6	& \textit{∅-}	& \textit{ma-} & \textit{(e)dinisa}/\textit{(a)madinisa} ‘window(s)’	\\
5b/6	& \textit{li- (ri-)}	& \textit{ma-} & \textit{(e)riiso}/\textit{(a)maiso} ‘eye(s)’	\\
7/8	& \textit{ki-}	& \textit{bi-} & \textit{(e)kibbambo}/\textit{(e)bibbambo} ‘bundle(s)’	\\
9/10	& \textit{n-}	& \textit{n-} & \textit{(e)nsansa}/\textit{(e)nsansa} ‘palm leaf (leaves)’	\\
9a/10a	& \textit{∅-}	& \textit{∅-} & \textit{(e)bbaasi}/\textit{(e)bbaasi} ‘bus(es)’	\\

11/10	& \textit{lu-}	& \textit{n-} & \textit{(o)lulimi}/\textit{(e)ndimi} ‘flame(s)’	\\

12/14	& \textit{ka-}	& \textit{bu-} & \textit{(a)katiko}/\textit{(o)butiko} ‘mushroom(s)’	\\
\lspbottomrule
\end{tabular}
\label{tab-pairings}
\end{table}

\subsection{Unpaired noun classes}\label{sec-morh-unpaired}
Whereas most nouns have both a singular and plural word-form, quite a few nouns occur just in one class and lack either their singular or their plural counterpart. 
The respective classes and some examples are given in~(\ref{ex-NC-unpaired}).

\ea  \label{ex-NC-unpaired}
\begin{xlist}
\ex Class 1: \textit{(o)mu-} \textit{(o)mukama} ‘measles (1)’ (only a couple of nouns)
\ex Class 1a: \textit{(o)∅-} \textit{(o)bbungu} `alcoholic beverage made from cassava and sorghum (1a)' (over 250 nouns)
\ex Class 3: \textit{(o)mu-} \textit{(o)mukyamu} ‘neighbourhood (3)', \textit{(o)mulikiirya} ‘darkness (3), \textit{(o)mwika} ‘smoke, gas (3)’ (about 80 nouns)
\ex Class 4: \textit{(e)mi-} \textit{(e)misinde} ‘speed, running (4)’, \textit{(e)misinga} ‘copper (4)’ (a few nouns)
\ex Class 5: \textit{(e)i-} \textit{(e)iyali} `envy, jealousy (5)', \textit{(e)idembe} `freedom (5)' (about 90 nouns)
\ex Class 6:  \textit{(a)mananu} `truth (6)' (about 180 nouns)
\ex Class 7: \textit{(e)ki-} \textit{(e)kyamukisa} ‘fortune (7)’, \textit{(e)kibbaluka} ‘sudden death (7)’ (about 100 nouns)
\ex Class 8: \textit{(e)bi-} \textit{(e)bimira} ‘mucus (8)’ (about 50 nouns)
\ex Class 9: \textit{(e)n-} \textit{(e)caaka} `heartburn' (about 200 nouns)
\ex Class 10: \textit{(e)n-} \textit{(e)fiizi} `school fees' (about 20 nouns)
\ex Class 11: \textit{(o)lu-} \textit{(o)lufurofuro} `foam, froth' (about 50 nouns)
\ex Class 12: \textit{(a)ka-} \textit{(a)kabango} `disobedience, misbehaviour' (about 170 nouns)
\ex Class 14: \textit{(o)bu-} \textit{(o)bugaite} ‘sum (14)’, \textit{(o)bukama} ‘kingdom (14)’, \textit{(o)bwenkani} ‘fairness (14)’ (about 430 nouns)
\ex Class 15: \textit{(o)ku-} \textit{(o)kusaaga} ‘joke (15) (about 50 nouns in the dictionary; all infinitives when used as nouns also belong to this class)
\end{xlist}	
\z


\subsection{Diminutive and augmentative derivation}\label{sec-morh-diminutive}

Nouns from various noun classes can be reassigned to classes 12, 13, 14, 20, and 22 to derive diminutives and augmentatives. 
In this case, the inherent noun class prefix, if there is one, is replaced by the prefix of one of these five classes. 
Whereas no noun in the dictionary inherently belongs to classes 20 and 22, many nouns inherently belong to classes 12 and 14. 
Only one noun (\textit{o-tulo} `sleep', a loan word from Luganda) inherently belongs to class 13.

Classes 12, 13, and 14 are used to form diminutive nouns. 
Count nouns form the diminutive by being reassigned to class 12 in the singular form and to class 14 in the plural form, as in (\ref {ex-diminutive-12}) and (\ref{ex-diminutive-14}) respectively.

\ea Diminutive of singular count nouns: class 12 \label{ex-diminutive-12}
\begin{xlist}
\ex \textit{a-kaana} ‘a small child (12)’  from \textit{a-mwana} ‘child (1)’
\ex \textit{a-kantu} ‘a little thing (12)’ from \textit{e-kintu} ‘thing (7)’
\ex \textit{a-kasente} ‘little money (12) from \textit{e-sente} ‘money (9)’
\end{xlist}	
\z

\ea Diminutive of plural count nouns: class 14 \label{ex-diminutive-14}
\begin{xlist}
\ex \textit{o-bwana} ‘small children (14)’ from \textit{a-baana} ‘children (2)’
\ex \textit{o-busaale} ‘small trees (14)’  from \textit{e-misaale} ‘trees (4)’
\end{xlist}	
\z

\hspace*{-2pt}Mass nouns form the diminutive with the meaning of scarcity or a small amount by being reassigned primarily to class 13, as in (\ref{ex-diminutive-13}).
The diminutive of mass nouns primarily with affective values of endearment is obtained by the reassignment to classes 12 or 14, as in  (\ref{ex-diminutive-mass-12}). 

\newpage
\ea Diminutive of mass nouns: Class 13 \label{ex-diminutive-13}
\begin{xlist}	
	\ex \textit{o-tuceere}  ‘a little rice (13)’ from \textit{o-muceere} ‘rice (3)’
	\ex \textit{o-tusukaali} ‘a little sugar (13)’ from \textit{o-sukaali} ‘sugar (1a)’
	\ex \textit{o-tumere} ‘a little food (13)’ from \textit{e-mere} ‘food (9)’
\end{xlist}	
\z

\ea Diminutive of mass nouns: Classes 12 and 14 \label{ex-diminutive-mass-12}
\begin{xlist}	
	\ex \textit{a-kasukaali} ‘a little sugar (12)’ from \textit{o-sukaali} ‘sugar (1a)’
	\ex \textit{o-bu-sukaali} ‘a little sugar (14)’ from \textit{o-sukaali} ‘sugar (1a)’ 
	\ex \textit{a-kamere} ‘a little food (12)’ from \textit{e-mere} ‘food (9)’
	\ex \textit{o-bumere} ‘a little food (14)’ from \textit{e-mere} ‘food (9)’
\end{xlist}	
\z

Classes 20 and 22 are used to form augmentative nouns. 
Singular count nouns are reassigned to class 20, whereas plural count nouns are reassigned to class 22, as in (\ref{ex-augmentative-20}) and (\ref{ex-augmentative-22}) respectively. 
Mass nouns form the augmentative by being reassigned to class 22, as in (\ref{ex-augmentative-mass-22}).

\ea Augmentative of singular count nouns: Class 20 \label{ex-augmentative-20}
\begin{xlist}	
	\ex \textit{o-gusolo} `large animal (20)’ from \textit{e-kisolo} `animal (7)’
	\ex \textit{o-gutimba} `large python (20)’ from \textit{o-itimba} `python (1a)’
\end{xlist}	
\z

\ea Augmentative of plural count nouns: Class 22 \label{ex-augmentative-22}
\begin{xlist}	
\ex \textit{a-gasolo} `large animals (22)’ from \textit{e-kisolo} `animal (7)’
\ex \textit{a-gaitimba} `large pythons (22)’ from \textit{o-itimba} `python (1a)’
\end{xlist}	
\z

\ea Augmentative of mass nouns: Class 22 \label{ex-augmentative-mass-22}
\begin{xlist}	
	\ex \textit{a-gasukaali} ‘a lot of sugar (22)’ from \textit{o-sukaali} ‘sugar (1a)’
	\ex \textit{a-gamere} ‘a lot of food (22)’ from \textit{e-mere} ‘food (9)’

\end{xlist}	
\z

The nouns with the augmentative and diminutive noun class prefixes often have affective values: the diminutive is used to express endearment and affection, whereas the augmentative is used to express repulsion or fear. 
For instance,  in addition to meaning `a large animal', \textit{a-gusolo} in~(\ref{ex-augmentative-20}) can also mean `a bad animal', `an ugly animal' or `a dangerous animal'. 
With affective values the respective classes seem to be even more productive than with the purely  augmentative and diminutive semantics.

\subsection{Noun class assignment of loanwords}\label{sec-morh-NCloans}

Ruruuli-Lunyala has borrowed quite a number of nouns from other languages, primarily from Luganda and English or from other languages via Luganda and English. 
Loan nouns are modified to fit into the language structure. 
Most loan nouns from English are assigned to class 9 in the singular form and to class 10 in the plural form, if there is a plural counterpart. Some are assigned to class 1a or 1 in the singular and to class 2 in the plural if there is a plural counterpart. 
Some examples are provided in~(\ref{sec-morh-NCloans-English}).

\ea \label{sec-morh-NCloans-English}
\begin{xlist}	
	\ex Class 9: \textit{(e)saati} ‘shirt’, \textit{(e)motoka} ‘motorcar’, \textit{alipoota} ‘report’,  \textit{agenda} ‘report’,  \textit{alijebula} ‘algebra’ 
	\ex Class 1a: \textit{(o)kalifoomu} ‘chloroform’, \textit{(o)vanisi} `varnish',\\\textit{(o)meeya} `mayor'
	\ex Class 1: \textit{(o)mupolotesitanti} ‘Protestant’
\end{xlist}	
\z

Nouns borrowed from Luganda are assigned to various noun  classes in Ru\-ruu\-li-Lu\-nya\-la. 
The assignment is mostly conditioned by the respective noun class prefix already present in Luganda. 
Some examples are provided in~(\ref{sec-morh-NCloans-Luganda}).

\ea \label{sec-morh-NCloans-Luganda}
\begin{xlist}	
	\ex Class 1: \textit{(o)mufumbo} `married person’
	\ex Class 7: \textit{(e)kibonerezo} ‘punishment’
	\ex Class 14: \textit{(o)bumanyirivu} ‘experience’
\end{xlist}	
\z


\subsection{Locative noun classes}\label{sec-morh-locative}

In addition to the classes discussed in the previous sections, Bantu languages also have what are traditionally called \textit{locative classes}. 
The respective markers are additive, \ie they do not substitute the inherent noun class prefix but precede it. 
They are typically used to mark syntactic adjuncts, but also mark arguments of some verbs with non-canonical argument marking valency frames. 
In Ru\-ruu\-li\hyp{}Lu\-nya\-la, the locative classes 16, 17, 18 and 23 are attested.
In contrast to most other nominal prefixes, locative class prefixes are written as separate orthographic words. 
Locative markers are preceded by their respective augments, and the noun itself does not take an augment, as in~(\ref{ex-NC-locative-all}).

Class 17 is the semantically least specific locative class; it denotes a general location in space and time. 
When used with motion verbs, it often translates as `towards' or `from', as in~(\ref{ex-NC-locative-ku}).
Class 18 expresses interiority, as in~(\ref{ex-NC-locative-mu}). 
Class 23 is typically used with proper place names, as in~(\ref{ex-NC-locative-e}). 
Furthermore, the locative classes can be subcategorised for by individual verbs, e.g.\,the intransitive bivalent verb \textit{singaanisya} `to compete' takes two arguments: the participant of the competition and the kind of competition itself. 
This second argument is marked by the locative class 18, as in~(\ref{ex-NC-locative-mu2}).
	
\ea \label{ex-NC-locative-all}
\begin{xlist}
\ex \label{ex-NC-locative-ku}
  \glll Ayabire oku igombolola.\\
a-a-ab-ire o-ku igombolola\\
\textsc{1sbj}-\textsc{pst}-go-\textsc{pfv} \textsc{aug}-17.\textsc{loc} 7.subcounty\\
\glt `He has gone to the sub-county.'

\ex \label{ex-NC-locative-mu}
  \glll Yaboine oitimba omu kisiko.\\
a-a-boine o-itimba o-mu kisiko\\
\textsc{1sbj}-\textsc{pst}-see:\textsc{pfv} \textsc{aug}-1.python \textsc{aug}-18.\textsc{loc} 7.bush\\
\glt `He saw a python in the bush.'

\ex \label{ex-NC-locative-e}
  \glll Twayaba e Kampala.\\
tu-a-ab-a e Kampala\\
1\textsc{pl.sbj}-\textsc{fut}-go-\textsc{fv} 23.\textsc{loc} 1.Kampala\\
\glt `We will go to Kampala.'

\ex \label{ex-NC-locative-mu2}
  \glll Bakusingaanisya mu misinde.\\
ba-ku-singaanisy-a mu misinde\\
\textsc{1sbj}-\textsc{prog}-compete-\textsc{fv} 18.\textsc{loc} 4.athletics\\
\glt `They are competing in athletics.'
\end{xlist}
\z
	
In contrast to the locative classes 17, 18, and 23, the class 16 prefix is not used with nouns.\footnote{Only one noun inherently belongs to class 16, viz.\,\textit{(a)waikirwa} `reception'.} 
Instead, the class 16 agreement prefix is commonly used with demonstratives (see Section~\ref{sec-morph-demonstrpro}), as well as in existential and locative constructions with the copula \textit{li} translated as `there is/are', as in~(\ref{ex-NC-locative}) (see Section~\ref{sec-non-verbal-predication} on non-verbal predication for more examples).

\ea \label{ex-NC-locative}
\label{ex-ruulilocstrange}
  \glll Waliwo enjawulo.\\
wa-li=wo e-njawulo.\\
	1\textsc{6sbj}-\textsc{cop}=16.\textsc{loc} \textsc{aug}-9.difference\\
\glt `There is a difference.'
\z


\section{Pronouns}\label{sec-pronouns}

In this section we present the morphology of Ru\-ruu\-li\hyp{}Lu\-nya\-la pronouns.
Section~\ref{sec-pronouns-perspro} discusses personal pronouns. 
Section~\ref{sec-morph-demonstrpro} presents the demonstrative pronouns. 
The morphology of the possessive pronouns is outlined in Section~\ref{sec-morph-posspro}. 
The way in which these pronouns are used in noun phrases is discussed in Section~\ref{sec-syntax-NP}.


\subsection{Personal pronouns}\label{sec-pronouns-perspro}

Personal pronouns of the first and second person are given in Table~\ref{tab-personal-pronouns}.

\begin{table}
\caption{Personal pronouns of the first and second person}
\begin{tabular}{l l l }
\lsptoprule
& \multicolumn{2}{c}{Number}\\
Person & singular & plural\\
\midrule

1 & \textit{nje} & \textit{iswe}\\
2 & \textit{we} & \textit{nywe}\\

\lspbottomrule
\end{tabular}
\label{tab-personal-pronouns}
\end{table}

Independent personal pronouns are not frequently used in Ru\-ruu\-li\hyp{}Lu\-nya\-la. 
The most common context for their usage involves topicalisation and in this case they are frequently left-dislocated.\footnote{Independent personal pronouns are uncommon in cases of non-verbal predication; instead a copula with a respective subject index is obligatorily used instead, see Section~\ref{sec-non-verbal-predication}.} 
Speakers regularly translate such constructions into English with a \textit{for}-phrase, as in~(\ref{ex-perspro-topic}). 
In most cases the personal pronouns are coreferential with the subject argument of the clause, as in (\ref{ex-perspro-topic-a}) and (\ref{ex-perspro-topic-b}). 
However they can be coreferential with other arguments and non-arguments, e.g.\,a possessor, as in (\ref{ex-perspro-topic-c}). 

\ea \label{ex-perspro-topic}
\begin{xlist}	
    \ex \label{ex-perspro-topic-a}
    \glll Naye 	nje		eisumu 	nalizwireku.\\
         naye 	nje		e-isumu 	n-a-li-zu-ire=ku.\\
	but 	1\textsc{sg}	\textsc{aug}-5.spear	\textsc{1sg.sbj}-\textsc{pst}-\textsc{5obj}-abandon-\textsc{pfv}=17.\textsc{loc}\\
    \glt  ‘But as for me, I abandoned the spear.’\\
	(Speaker’s translation: `But for me, I abandoned the spear.')

   \ex  \label{ex-perspro-topic-b}
    \glll Iswe 	tuliire 	bunyonyi 	n' obusolo.\\
         iswe 	tu-li-ire 	bunyonyi 	na 	o-busolo\\
	1\textsc{pl}	1\textsc{pl.sbj}-eat-\textsc{pfv}	14.bird 	\textsc{com}	 \textsc{aug}-14.animal\\
    \glt  ‘As for us, we have eaten birds and animals.’\\ (Speaker’s translation: `For us, we have eaten birds and animals.')
 
    \ex  \label{ex-perspro-topic-c}
    \glll Nje abaana bange tibamaite.\\
         nje a-baana ba-a-nge ti-ba-maite\\
	1\textsc{sg}	\textsc{aug}-2.child 2-\textsc{assoc}-\textsc{1sg} \textsc{neg}-\textsc{2sbj}-know:\textsc{pfv}\\
    \glt  ‘As for me, my children don't know.'
\end{xlist}	
\z

The forms of third person personal pronouns are mostly systematic and are given in Table~\ref{tab-perspro-3}. They are  composed of the respective noun class prefix followed by the stem \textit{o}. 
The only exception is the pronoun for the noun class 1, which has the form \textit{ye}.
These forms are infrequent in discourse. 
An example of their usage is given in~(\ref{ex-perspro-3}).

\ea \label{ex-perspro-3}
    \glll Musaale	gukooto. Gwo	tegutwala		emyaka	zingi.\\
         musaale	gu-kooto gwo	te-gu-twal-a		e-myaka	zi-ingi\\
	3.tree		 3-big  	3	\textsc{neg}-\textsc{3sbj}-take-\textsc{fv}	\textsc{aug}-4.year	4-many\\
    \glt  ‘It is a big tree. It doesn't take many years (to grow).’
\z

\begin{table}
\caption{Personal pronouns of the third person}
\begin{tabular}{l l l l l l l }
\lsptoprule
 Class & Form & Class & Form\\
\midrule
1 & \textit{ye}   &     11 & \textit{lwo}\\
2 & \textit{bo}   &     12 & \textit{ko}\\
3 & \textit{gwo}  &     13 & \textit{two}\\
4 & \textit{gyo}  &     14 & \textit{bwo}\\
5 & \textit{lyo}  &     15 & \textit{kwo}\\
6 & \textit{go}   &      20 & \textit{gwo}\\
7 & \textit{kyo}  &     22 & \textit{go}\\
8 & \textit{byo}\\
9 & \textit{yo/gyo}\\
10 & \textit{zo}\\
\lspbottomrule
\end{tabular}
\label{tab-perspro-3}
\end{table}



The pronominal forms discussed in this section are frequently followed by the additive focus marker \textit{-na}. 
Two examples are provided in~(\ref{ex-perspro-additive}).

\ea \label{ex-perspro-additive}
\begin{xlist}	
 
    \ex  \label{ex-perspro-additive-a}
    \textit{Njena nkukoba kini.}\\
    \gll    nje-na n-ku-kob-a ki-ni\\
	1\textsc{sg}-\textsc{add.foc} \textsc{1sg.sbj}-\textsc{prog}-say-\textsc{fv} 7-\textsc{prox}\\
    \glt  ‘I also say this.’
    
    \ex  \label{ex-perspro-additive-b}
    \textit{Iswena nitukyuka.}\\
    \gll     iswe-na ni-tu-kyuk-a\\
	1\textsc{pl}-\textsc{add.foc} \textsc{nar}-\textsc{1pl.sbj}-change-\textsc{fv}\\
    \glt  `We also changed.’
\end{xlist}	
\z


\subsection{Demonstrative pronouns}\label{sec-morph-demonstrpro}

Ruruuli-Lunyala distinguishes three series of demonstratives which indicate different degrees of spatial distance from the speaker, viz.\,proximal, medial and distal locations. 
The demonstratives also have a range of non-spatial deictic uses. 
The proximal demonstrative has the stem \textit{ni} in Ruruuli and \textit{nu} in Lunyala. 
The form of the medial demonstrative is \textit{o}. 
The form of the distal demonstrative is \textit{di}. 
The agreeing forms are listed in Table~\ref{tab-demonstratives}. 

\begin{table}[!h]
\caption{Demonstrative pronouns}
\begin{tabular}{l l l l}
\lsptoprule
Class & Proximal & Medial & Distal\\
\midrule
1 & \textit{oni} & \textit{oyo} & \textit{odi}\\
2 & \textit{bani} & \textit{abo} & \textit{badi}\\
3 & \textit{guni} & \textit{ogwo} & \textit{gudi}\\
4 & \textit{gini} & \textit{egyo} & \textit{gidi}\\
5 & \textit{lini} & \textit{eryo} & \textit{lidi}\\
6 & \textit{gani} & \textit{ago} & \textit{gadi}\\
7 & \textit{kini} & \textit{ekyo} & \textit{kidi}\\
8 & \textit{bini} & \textit{ebyo} & \textit{bidi}\\
9 & \textit{eni/gini} & \textit{eyo/egyo} & \textit{edi}\\
10 & \textit{zini} & \textit{ezo} & \textit{zidi}\\
11 & \textit{luni} & \textit{olwo} & \textit{ludi}\\
12 & \textit{kani} & \textit{ako} & \textit{kadi}\\
13 & \textit{tuni} & \textit{otwo} & \textit{tudi}\\
14 & \textit{buni} & \textit{obwo} & \textit{budi}\\
15 & \textit{kuni} & \textit{okwo} & \textit{kudi}\\
16 & \textit{wani/ani} & \textit{awo} & \textit{wadi/adi}\\
17 & \textit{kuni} & \textit{okwo} & \textit{kudi}\\
18 & \textit{muni} & \textit{omwo} & \textit{mudi}\\
20 & \textit{guni} & \textit{ogwo} & \textit{gudi}\\
22 & \textit{gani} & \textit{ago} & \textit{gadi}\\
23 & \textit{eni} & \textit{eyo} & \textit{edi}\\
\lspbottomrule
\end{tabular}
\label{tab-demonstratives}
\end{table}

The forms are partially predictable: demonstrative stems take noun class agreement prefixes of the same form as on other hosts (e.g.\,on adjectives). 
In addition, some demonstratives – including all of the medial demonstratives – have an augment-like prefix identical in form to the regular augment of the respective class. 
As these word-forms do not occur without an augment, we do not segment it in the glosses.
Some examples are provided in~(\ref{ex-demonstrative}). 

\ea \label{ex-demonstrative}
\begin{multicols}{2}
\begin{xlist}	

    \ex  \label{ex-demonstrative-a}
    \textit{omusaiza oni}\\
    \gll   o-musaiza o-ni\\
	\textsc{aug}-1.man 1-\textsc{prox}\\
    \glt  ‘this man’
    
    \ex  \label{ex-demonstrative-b}
    \textit{ekisolo kini}\\
    \gll     e-kisolo ki-ni\\
	\textsc{aug}-7.animal 7-\textsc{prox}\\
    \glt  `this animal’
    
\ex  \label{ex-demonstrative-c}
    \textit{omusaiza oyo}\\
    \gll     o-musaiza oy-o\\
	\textsc{aug}-1.man 1-\textsc{med}\\
    \glt  `that man’

\ex  \label{ex-demonstrative-d}
    \textit{ekisolo ekyo}\\
    \gll     e-kisolo eki-o\\
	\textsc{aug}-7.animal 7-\textsc{med}\\
    \glt  `that animal’
    
\ex  \label{ex-demonstrative-e}
    \textit{omusaiza odi}\\
    \gll     o-musaiza o-di\\
	\textsc{aug}-1.man 1-\textsc{dist}\\
    \glt  `that man’

\ex  \label{ex-demonstrative-f}
    \textit{ekisolo kidi}\\
    \gll     e-kisolo ki-di\\
	\textsc{aug}-7.animal 7-\textsc{dist}\\
    \glt  `that animal’
\end{xlist}	
\end{multicols}

\z


\subsection{Possessive pronouns}\label{sec-morph-posspro}

This section discusses the morphological properties of  possessive pronouns. 
Their function in the noun phrase is discussed in Section~\ref{sec-NP-possessive-pronoun}. 
Possessive pronouns are formed with three morphemes. 
The first morpheme is the prefix which agrees with the possessee in person, number, and noun class. 
For the first and second person the respective forms are \textit{-nge} `1\textsc{sg}, \textit{-iswe} `1\textsc{pl}', \textit{-mu} `2\textsc{sg}', and \textit{-nywe} `2\textsc{pl}'. 
It is followed by the stem \textit{a}, known as the \textit{associative marker} in Bantu linguistics (see also Section~\ref{sec-NP-genitive}). 
The third element encodes the person and number of the possessor in the case of first and second person possessors or the noun class of third person possessors. 

Table~\ref{tab-proposs} lists the forms of the possessive pronouns in the orthographic representation for the first and second person possessors, as well as for third person possessors of noun classes 1 and 2. 
The last two forms (noun class 1 and 2) are the most commonly used forms of the third person possessive pronouns. 
Their formation deviates from the pattern found in other third person possessive pronouns (see below). 

\begin{table}
\caption{Possessive pronouns}
\fittable{
\begin{tabular}{l l l l l l l l}
\lsptoprule
 
	&  & 1\textsc{sg} & 1\textsc{pl} & 2\textsc{sg} & 2\textsc{pl} & 3\textsc{sg} (NC1) & 3\textsc{pl} (NC2)\\
NC 	& Prefix & `my' & `our' & `your (sg.)' & `your (pl.)' & `his/her' & `their'\\
\midrule
	& & \textbf{\textit{nge}} & \textbf{\textit{iswe}} & \textbf{\textit{mu}} & \textbf{\textit{nywe}} & \textbf{\textit{mwe(i)}} & \textbf{\textit{bwe}}\\
\midrule
1 & \textit{wa-} & \textit{wange} & \textit{waiswe} & \textit{waamu} & \textit{waanywe} & \textit{waamwe(i)} & \textit{waabwe}\\
2 & \textit{ba-}& \textit{bange} & \textit{baiswe} & \textit{baamu} & \textit{baanywe} & \textit{baamwe(i)} & \textit{baabwe}\\
3 & \textit{gu-}& \textit{gwange} & \textit{gwaiswe} & \textit{gwamu} & \textit{gwanywe} & \textit{gwamwe(i)} & \textit{gwabwe}\\
4 & \textit{gi-} & \textit{gyange} & \textit{gyaiswe} & \textit{gyamu} & \textit{gyanywe} & \textit{gyamwe(i)} & \textit{gyabwe}\\
& \textit{za-} & \textit{(zange)} & \textit{(zaiswe)} & \textit{(zaamu)} & \textit{(zaanywe)} & \textit{(zaamwe(i))} & \textit{(zaabwe)}\\

5 & \textit{li-}& \textit{lyange} & \textit{lyaiswe} & \textit{lyamu} & \textit{lyanywe} & \textit{lyamwe(i)} & \textit{lyabwe}\\
6 & \textit{ga-}& \textit{gange} & \textit{gaiswe} & \textit{gaamu} & \textit{gaanywe} & \textit{gaamwe(i)} & \textit{gaabwe}\\
7 & \textit{ki-}& \textit{kyange} & \textit{kyaiswe} & \textit{kyamu} & \textit{kyanywe} & \textit{kyamwe(i)} & \textit{kyabwe}\\
8 & \textit{bi-}& \textit{byange} & \textit{byaiswe} & \textit{byamu} & \textit{byanywe} & \textit{byamwe(i)} & \textit{byabwe}\\
9 & \textit{ya-}& \textit{yange} & \textit{yaiswe} & \textit{yaamu} & \textit{yaanywe} & \textit{yaamwe(i)} & \textit{yaabwe}\\
10 & \textit{za-}& \textit{zange} & \textit{zaiswe} & \textit{zaamu} & \textit{zaanywe} & \textit{zaamwe(i)} & \textit{zaabwe}\\
11 & \textit{lu-}& \textit{lwange} & \textit{lwaiswe} & \textit{lwamu} & \textit{lwanywe} & \textit{lwamwe(i)} & \textit{lwabwe}\\
12 & \textit{ka-}& \textit{kange} & \textit{kaiswe} & \textit{kaamu} & \textit{kaanywe} & \textit{kaamwe(i)} & \textit{kaabwe}\\
13 & \textit{tu-}& \textit{twange} & \textit{twaiswe} & \textit{twamu} & \textit{twanywe} & \textit{twamwe(i)} & \textit{twabwe}\\
14& \textit{bu-} & \textit{bwange} & \textit{bwaiswe} & \textit{bwamu} & \textit{bwanywe} & \textit{bwamwe(i)} & \textit{bwabwe}\\
15 & \textit{ku-}& \textit{kwange} & \textit{kwaiswe} & \textit{kwamu} & \textit{kwanywe} & \textit{kwamwe(i)} & \textit{kwabwe}\\
20 & \textit{gu-}& \textit{gwange} & \textit{gwaiswe} & \textit{gwamu} & \textit{gwanywe} & \textit{gwamwe(i)} & \textit{gwabwe}\\
22 & \textit{ga-}& \textit{gange} & \textit{gaiswe} & \textit{gaamu} & \textit{gaanywe} & \textit{gaamwe(i)} & \textit{gaabwe}\\
\lspbottomrule
\end{tabular}
}
\label{tab-proposs}
\end{table}

As Table~\ref{tab-proposs}  shows, for the third person singular noun class 1 possessor, two variants are common: in Lunyala we find the form \textit{mwei}, whereas in Ruruuli the form \textit{mwe} is found. 
Both forms are used in the examples in the dictionary (see Part~\ref{sec-Dictionary}).
Furthermore, the possessive pronouns of the class 4 possessor have two dialectally-conditioned variants, though the exact distribution of these two forms is not yet well-understood.

A few examples of noun phrases with possessive pronouns are given below. 
The examples in~(\ref{ex-posspro-NC1}) illustrate a possessee of class 1, whereas the examples in~(\ref{ex-posspro-NC6}) illustrate a possessee of class 6. 

\ea \label{ex-posspro-NC1}
\begin{xlist}
\begin{multicols}{2}
    \ex 
    \glll   omwana wange\\
            o-mwana wa-a-nge \\
 	\textsc{aug}-1.child 1-\textsc{assoc}-\textsc{1sg}\\
    \glt  ‘my child’ 

    \ex 
    \glll   omwana waiswe\\
            o-mwana wa-a-iswe\\
 	 \textsc{aug}-1.child 1-\textsc{assoc}-\textsc{1pl}\\
    \glt  ‘our child’ 
    
  \ex 
    \glll   omwana waamu\\
            o-mwana wa-a-mu\\
 	 \textsc{aug}-1.child 1-\textsc{assoc}-\textsc{2sg}\\
    \glt  ‘your (sg.) child’ 

  \ex
    \glll   omwana waanywe\\
        o-mwana wa-a-nywe\\
 	 \textsc{aug}-1.child 1-\textsc{assoc}-\textsc{2pl}\\
    \glt  ‘your (pl.) child’ 

\ex
    \glll   omwana waamwe\\
            o-mwana wa-a-mwe\\
 	 \textsc{aug}-1.child 1-\textsc{assoc}-1\\
    \glt  ‘his/her child’ 

\ex 
    \glll   omwana waabwe\\
            o-mwana wa-a-bwe\\
 	 \textsc{aug}-1.child 1-\textsc{assoc}-2\\
    \glt  ‘their child’ 
        
\end{multicols}
\end{xlist}	
\z
  
\ea \label{ex-posspro-NC6}
\begin{xlist}	
\begin{multicols}{2}
\ex 
    \glll   amaizi gange\\
        a-maizi ga-a-nge\\
 	\textsc{aug}-6.water 6-\textsc{assoc}-\textsc{1sg}\\
    \glt  ‘my water’ 
		
\ex 
    \glll   amaido gaiswe\\
    a-maido ga-a-iswe\\
	\textsc{aug}-6.groundnuts 6-\textsc{assoc}-\textsc{1pl}\\
    \glt `our groundnuts'

\ex 
    \glll   amaka gaamu\\
            a-maka ga-a-mu\\
	\textsc{aug}-6.home 6-\textsc{assoc}-2\textsc{sg}\\
    \glt `your (sg.) home'    

\ex 
    \glll   amatai gaanywe\\
            a-matai ga-a-nywe\\
	\textsc{aug}-6.milk 6-\textsc{assoc}-\textsc{2pl}\\
    \glt  ‘your (pl.) milk’
    
\ex 
    \glll   amabeere gaamwe\\
            a-mabeere ga-a-mwe\\
	\textsc{aug}-6.breast 6-\textsc{assoc}-1\\
    \glt  ‘his/her breasts’    

\ex
    \glll amaaba gaabwe\\
    a-maaba ga-a-bwe\\
	\textsc{aug}-6.breast 6-\textsc{assoc}-2\\
    \glt  ‘their departure’
\end{multicols}
\end{xlist}	    
\z

As the examples in~(\ref{ex-posspro-NC1}) and~(\ref{ex-posspro-NC6}) show, the possessive pronoun always follows the head noun and does not carry an augment. 
The use of the augment on the possessor is conditioned by its larger syntactic environment.


The forms of third person possessors of other noun classes (i.e.\,of noun classes 3–22) are built in a different way: 
The possessor agreement prefix attaches to the associative stem \textit{a}, which is in turn followed by the agreement prefix for the possessee and the marker \textit{o}. 
The respective forms are given in Tables~\ref{tab-proposs3a}–\ref{tab-proposs3b}. 
The use of the possessive pronouns of a third person possessor of noun classes 9 and 10 is illustrated in~(\ref{ex-posspro-other-possessors}). 
Like other possessive pronouns, they follow the head noun and do not take an augment.
\newpage %longdistance

\ea \label{ex-posspro-other-possessors}
\begin{xlist}	
    \ex \textit{entikko yaagwo}\\
    \gll     e-ntikko ya-a-gwo \\
	\textsc{aug}-9.culmination 9-\textsc{assoc}-3\\
    \glt  ‘its culmination' (e.g.\,of a function (\textit{mukolo} `function (3)')

   \ex \textit{engeso zaalyo}\\
    \gll     e-ngeso za-a-lyo\\
	\textsc{aug}-10.norm 10-\textsc{assoc}-5\\
    \glt  ‘its norms' (e.g.\,of a tribe (\textit{iyanga} `tribe (5)')
\end{xlist}	
\z

\begin{table}
\caption{Possessive pronouns of the third person possessor of noun class 3–22\\ (possessee of noun class 1–8)}
\fittable{
\begin{tabular}{l l l l l l l l l}
\lsptoprule

 & 1 & 2 & 3 & 4 & 5 & 6 & 7 & 8\\ 
 & \textit{wa-} & \textit{ba-} & \textit{gu-} & \textit{gi-} & \textit{li-} & \textit{ga-} & \textit{ki} & \textit{bi-}\\
\midrule

3 & \textit{waagwo} & \textit{baagwo} & \textit{gwagwo} & \textit{gyagwo} & \textit{lyagwo} & \textit{gaagwo} & \textit{kyagwo} & \textit{byagwo}\\
4 & \textit{waagyo} & \textit{baagyo} & \textit{gwagyo} & \textit{gyagyo} & \textit{lyagyo} & \textit{gaagyo} & \textit{kyagyo} & \textit{byagyo}\\
5 & \textit{waalyo} & \textit{baalyo} & \textit{gwalyo} & \textit{gyalyo} & \textit{lyalyo} & \textit{gaalyo} & \textit{kyalyo} & \textit{byalyo}\\
6 & \textit{waago} & \textit{baago} & \textit{gwago} & \textit{gyago} & \textit{lyago} & \textit{gaago} & \textit{kyago} & \textit{byago}\\
7 & \textit{waakyo} & \textit{baakyo} & \textit{gwakyo} & \textit{gyakyo} & \textit{lyakyo} & \textit{gaakyo} & \textit{kyakyo} & \textit{byakyo}\\
8 & \textit{waabyo} & \textit{baabyo} & \textit{gwabyo} & \textit{gyabyo} & \textit{lyabyo} & \textit{gaabyo} & \textit{kyabyo} & \textit{byabyo}\\
9 & \textit{waayo} & \textit{baayo} & \textit{gwayo} & \textit{gyayo} & \textit{lyayo} & \textit{gaayo} & \textit{kyayo} & \textit{byayo}\\
10 & \textit{waazo} & \textit{baazo} & \textit{gwazo} & \textit{gyazo} & \textit{lyazo} & \textit{gaazo} & \textit{kyazo} & \textit{byazo}\\
11 & \textit{waalwo} & \textit{baalwo} & \textit{gwalwo} & \textit{gyalwo} & \textit{lyalwo} & \textit{gaalwo} & \textit{kyalwo} & \textit{byalwo}\\
12 & \textit{waako} & \textit{baako} & \textit{gwako} & \textit{gyako} & \textit{lyako} & \textit{gaako} & \textit{kyako} & \textit{byako}\\
13 & \textit{waatwo} & \textit{baatwo} & \textit{gwatwo} & \textit{gyatwo} & \textit{lyatwo} & \textit{gaatwo} & \textit{kyatwo} & \textit{byatwo}\\
14 & \textit{waabwo} & \textit{baabwo} & \textit{gwabwo} & \textit{gyabwo} & \textit{lyabwo} & \textit{gaabwo} & \textit{kyabwo} & \textit{byabwo}\\
15 & \textit{waakwo} & \textit{baakwo} & \textit{gwakwo} & \textit{gyakwo} & \textit{lyakwo} & \textit{gaakwo} & \textit{kyakwo} & \textit{byakwo}\\
20 & \textit{waago} & \textit{baago} & \textit{gwago} & \textit{gyago} & \textit{lyago} & \textit{gaago} & \textit{kyago} & \textit{byago}\\
22 & \textit{waagwo} & \textit{baagwo} & \textit{gwagwo} & \textit{gyagwo} & \textit{lyagwo} & \textit{gaagwo} & \textit{kyagwo} & \textit{byagwo}\\
\lspbottomrule
\end{tabular}
}
\label{tab-proposs3a}
\end{table}

\begin{sidewaystable}
\caption{Possessive pronouns of the third person possessor of noun class 3–22 (possessee of noun class 9–22)}
\begin{tabular}{l l l l l l l l l l}
\lsptoprule
 & 9 & 10 & 11 & 12 & 13 & 14 & 15 & 20 & 22\\
  & \textit{ya-} &  \textit{za-} &  \textit{lu-} &  \textit{ka-} &  \textit{tu-} &  \textit{bu-} &  \textit{ku-} &  \textit{ga-} &  \textit{gu-}\\

\midrule
 
3 & \textit{yaagwo} & \textit{zaagwo} & \textit{lwagwo} & \textit{kaagwo} & \textit{twagwo} & \textit{bwagwo} & \textit{kwagwo} & \textit{gaagwo} & \textit{gwagwo}\\
4 & \textit{yaagyo} & \textit{zaagyo} & \textit{lwagyo} & \textit{kaagyo} & \textit{twagyo} & \textit{bwagyo} & \textit{kwagyo} & \textit{gaagyo} & \textit{gwagyo}\\
5 & \textit{yaalyo} & \textit{zaalyo} & \textit{lwalyo} & \textit{kaalyo} & \textit{twalyo} & \textit{bwaalyo} & \textit{kwalyo} & \textit{gaalyo} & \textit{gwalyo}\\
6 & \textit{yaago} & \textit{zaago} & \textit{lwago} & \textit{kaago} & \textit{twago} & \textit{bwago} & \textit{kwago} & \textit{gaago} & \textit{gwago}\\
7 & \textit{yaakyo} & \textit{zaakyo} & \textit{lwakyo} & \textit{kaakyo} & \textit{twakyo} & \textit{bwakyo} & \textit{kwakyo} & \textit{gaakyo} & \textit{gwakyo}\\
8 & \textit{yaabyo} & \textit{zaabyo} & \textit{lwabyo} & \textit{kaabyo} & \textit{twabyo} & \textit{bwabyo} & \textit{kwabyo} & \textit{gaabyo} & \textit{gwabyo}\\
9 & \textit{yaayo} & \textit{zaayo} & \textit{lwayo} & \textit{kaayo} & \textit{twayo} & \textit{bwayo} & \textit{kwayo} & \textit{gaayo} & \textit{gwayo}\\
10 & \textit{yaazo} & \textit{zaazo} & \textit{lwazo} & \textit{kaazo} & \textit{twazo} & \textit{bwazo} & \textit{kwazo} & \textit{gaazo} & \textit{gwazo}\\
11 & \textit{yaalwo} & \textit{zaalwo} & \textit{lwalwo} & \textit{kaalwo} & \textit{twalwo} & \textit{bwalwo} & \textit{kwalwo} & \textit{gaalwo} & \textit{gwalwo}\\
12 & \textit{yaako} & \textit{zaako} & \textit{lwako} & \textit{kaako} & \textit{twako} & \textit{bwako} & \textit{kwako} & \textit{gaako} & \textit{gwako}\\
13 & \textit{yaatwo} & \textit{zaatwo} & \textit{lwatwo} & \textit{kaatwo} & \textit{twatwo} & \textit{bwatwo} & \textit{kwatwo} & \textit{gaatwo} & \textit{gwatwo}\\
14 & \textit{yaabwo} & \textit{zaabwo} & \textit{lwabwo} & \textit{kaabwo} & \textit{twabwo} & \textit{bwabwo} & \textit{kwabwo} & \textit{gaabwo} & \textit{gwabwo}\\
15 & \textit{yaakwo} & \textit{zaakwo} & \textit{lwakwo} & \textit{kaakwo} & \textit{twakwo} & \textit{bwakwo} & \textit{kwakwo} & \textit{gaakwo} & \textit{gwakwo}\\
20 & \textit{yaago} & \textit{zaago} & \textit{lwago} & \textit{kaago} & \textit{twago} & \textit{bwago} & \textit{kwago} & \textit{gaago} & \textit{gwago}\\
22 & \textit{yaagwo} & \textit{zaagwo} & \textit{lwagwo} & \textit{kaagwo} & \textit{twagwo} & \textit{bwagwo} & \textit{kwagwo} & \textit{gaagwo} & \textit{gwagwo}\\
\lspbottomrule
\end{tabular}
\label{tab-proposs3b}
\end{sidewaystable}

\newpage
\section{Verbs}\label{sec-morh-verb}

This section discusses some aspects of verbal morphology. 
The structure of the finite inflected verb in Ru\-ruu\-li\hyp{}Lu\-nya\-la is given in~(\ref{ex-verb-slots}). 

\ea \label{ex-verb-slots}
TA1 - (\textsc{neg}1) – \textsc{sbj} – (\textsc{neg}2) – TA2 – (\textsc{obj}) – root – (TA3) – Final – Post-final
\z

The indexing\footnote{Unless specified otherwise, we follow \citet{Haspelmath2013Argument} and use the term \textit{(argument) indexing} and \textit{index} to refer to bound person forms on the verb. The question of whether these indexes are cases of grammatical agreement or cases of pronominal agreement will be discussed elsewhere.} of the subject argument is obligatory and occurs in the position S of the scheme in~(\ref{ex-verb-slots}), 
the indexing of the objects is in the position O and is optional, see Section~\ref{sec-verb-indexing}. 
The argument indexing on copulas in clauses is discussed in Section~\ref{sec-morpho-copulas}. 
The two slots for the negation prefixes can be occupied by either the standard negation prefix \textit{ti-} (\textsc{neg}1), which precedes the subject prefix, 
or the prefix \textit{ta-} (\textsc{neg}2), which directly follows it and is used in prohibitive (Sections~\ref{sec-prohibitive}) and negative hortative and jussive constructions (Section~\ref{sec-hortative}). 
Tense and aspect categories are expressed as either prefixes or suffixes. 
The slot TA1 can only be occupied by the narrative prefix.  The slot TA3 is occupied by the perfective suffix. 
The final can be occupied by either the final vowel or the subjunctive. 
There is no final if the verb carries the perfective suffix. 
The post-final slot can only be occupied by the habitual suffix.

For the sake of simplicity, the scheme in~(\ref{ex-verb-slots}) does not include the various voice-related affixes (the so-called extensions and the reflexive prefix). 
Their morphology is discussed in Section~\ref{sec-morph-verb-extensions}. 
The reflexive prefix \mbox{\textit{ee-}} immediately precedes the verb stem.  
The reflexive is the only diathetical operation in Ru\-ruu\-li\hyp{}Lu\-nya\-la marked by a prefix and not by a suffix and is thus not considered an extension in Bantu studies. 
The applicative suffix \textit{-ir} follows the root but precedes the perfective suffix. 
The causative suffix \textit{-isy}, the passive suffix \textit{-ibw}, and the reciprocal suffix \textit{-angan} follow the root. 
They merge into a portmanteau suffix with the perfective suffix, if there is one. 
Multiple extensions can occur on one word-form. 
The restrictions on their respective order, as well as the effect of the order on the meaning of the word-form need further studies.

\subsection{Argument indexing on verbs}\label{sec-verb-indexing}

The verb in Ru\-ruu\-li\hyp{}Lu\-nya\-la can index up to four arguments. 
The indexing of the subject 
is obligatory for all finite verb forms. 
The indexing of the direct object is conditioned by the properties of this argument. 
Below in Table~\ref{tab-verb-indexes} we list the respective prefixes and then provide some examples.
Most indices do not distinguish between different argument roles, i.e.\,they are the same for subjects and objects. 
The only differences are found between \textit{o-} ‘2\textsc{sg.sbj}’ and  \textit{ku-} ‘2\textsc{sg.obj}’,  \textit{mu-} ‘2\textsc{pl.sbj}’ and  \textit{ba-} ‘2\textsc{pl.obj}’, as well as between \textit{a-} ‘1\textsc{sbj}’ and  \textit{mu-} ‘1\textsc{obj}’. 

Table~\ref{tab-verb-indexes} does not list the phonologically-conditioned allomorphs of the subject and object prefixes. They are discussed in multiple subsections in Section~\ref{sec-morphophonological-process}. 
See in particular the subsections on hiatus resolution (Section \ref{sec-phonology-vowel-hiatus}), fortition (Section \ref{sec-phonology-fortition}), and nasal assimilation (Section \ref{sec-phonology-nasal-assimilation}).

\begin{table}
\caption{The subject and object prefixes}
\label{tab-verb-indexes}
	\begin{tabular}{lll}
\lsptoprule
Class & Subject & Objects\\
\midrule
1\textsc{sg} & \textit{n-} & \textit{n-}\\
1\textsc{pl} & \textit{tu-} & \textit{tu-}\\
2\textsc{sg} & \textit{o-} & \textit{ku-}\\
2\textsc{pl} & \textit{mu-} & \textit{ba-}\\
1 & \textit{a-} & \textit{mu-}\\
2 & \textit{ba-} & \textit{ba-}\\
3 & \textit{gu-} & \textit{gu-}\\
4 & \textit{gi-} & \textit{gi-}\\
5 & \textit{li-} & \textit{li-}\\
6 & \textit{ga-} & \textit{ga-}\\
7 & \textit{ki-} & \textit{ki-}\\
8 & \textit{bi-} & \textit{bi-}\\
9 & \textit{e-}/\textit{gi-}& \textit{gi-}\\
10 & \textit{zi-} & \textit{zi-}\\
11 & \textit{lu-} & \textit{lu-}\\
12 & \textit{ka-} & \textit{ka-}\\
13 & \textit{tu-} & \textit{tu-}\\
14 & \textit{bu-} & \textit{bu-}\\
15 & \textit{ku-} & \textit{ku-}\\
16 & \textit{wa-} &\\
17 & \textit{ku-} &\\
18 & \textit{mu-} &\\
20 & \textit{gu-} & \textit{gu-}\\
22 & \textit{ga-} & \textit{ga-}\\
23 & \textit{e-} &\\
\lspbottomrule
	\end{tabular}
\end{table}

The examples in~(\ref{ex-verb-indexing-subject}) illustrate the use of the subject indexes and show that in the majority of cases they are the first prefix of the verb form.

\ea \label{ex-verb-indexing-subject}
    \begin{xlist}	
\ex	\label{ex-verb-indexing-subject1}
	\glll Nkyasona mukeeka gwange.\\
		n-kya-son-a mukeeka gu-a-nge\\
 	\textsc{1sg.sbj}-\textsc{pers}-weave-\textsc{fv} 3.mat 3-\textsc{assoc}-\textsc{1sg}\\
	\glt  ‘I am still weaving my mat.'
        
\ex	\label{ex-verb-indexing-subject2}
    	\glll Oyendya kulya ki?\\
	o-yendy-a ku-li-a ki\\
	 2\textsc{sg.sbj}-want-\textsc{fv} \textsc{inf}-eat-\textsc{fv} what\\
    	\glt ‘What do you want to eat?'
    
\ex \label{ex-verb-indexing-subject3}
    	\glll Abantu bakusomoka enyanja.\\
	a-bantu ba-ku-somok-a e-nyanja\\
	\textsc{aug}-2.man 2\textsc{sg.sbj}-\textsc{prog}-cross-\textsc{fv} \textsc{aug}-9.lake\\
   	 \glt ‘The people are crossing the lake.'
    
\ex	\label{ex-verb-indexing-subject4}
    	\glll Tokwaba.\\
	  ti-o-ku-ab-a\\
		\textsc{neg}-2\textsc{sg.sbj}-\textsc{prog}-go-\textsc{fv}\\
	\glt ‘You are not going.' 
        \end{xlist}	
\z

The indexing of the direct object is conditioned by the properties of this argument.  
Objects of the first and second person are always indexed on the verb. 
The indexing of third person objects depends on their position in the clause – preverbal objects tend to be indexed – and ultimately on their referential properties, as shown in~(\ref{ex-verb-indexing-obj1}) and~(\ref{ex-verb-indexing-obj5}). 
Furthermore, ditransitive verbs can index both object, i.e.\,the theme and the recipient/goal.  
The indexing of the recipient argument is illustrated in~(\ref{ex-verb-indexing-obj4}). 
The indexing of the theme argument is illustrated in~(\ref{ex-verb-indexing-obj5}).
Finally, applicative verbs can index applied objects, as in~(\ref{ex-verb-indexing-obj3}), and it is possible to have two applicative suffixes which introduce two applied objects (see Section~\ref{sec-applicative} on the use of the applicative).

\ea \label{ex-verb-indexing-obj}
    \begin{xlist}	
\ex	\label{ex-verb-indexing-obj1}
	\glll 	Naye	nje		eisumu 	nalizwireku.\\
	naye	nje		e-isumu 	n-a-li-zu-ire=ku\\
 		but	1\textsc{sg}	\textsc{aug}-5.spear	\textsc{1sg.sbj}-\textsc{pst}-\textsc{5obj}-abandon-\textsc{pfv}=17.\textsc{loc}\\
	\glt  ‘But as for me, I abandoned the spear.’
        
\ex	\label{ex-verb-indexing-obj2}
	\glll 	Mubatumire oku iduuka.\\
	mu-ba-tum-ire o-ku iduuka\\
	2\textsc{sg.sbj}-\textsc{2obj}-send-\textsc{pfv} \textsc{aug}-17.\textsc{loc} 5.shop\\
	\glt ‘You have sent them to the shop.'

\ex	\label{ex-verb-indexing-obj4}
	\glll Amuwaire esaati.\\
	a-mu-wa-ire e-saati\\
	\textsc{1sbj}-\textsc{1obj}-give-\textsc{pfv} \textsc{aug}-9.shirt\\
	\glt ‘He has given him a shirt.'
    
\ex	\label{ex-verb-indexing-obj5}
    	\glll	Ekisunsuzo nkiwaire maama.\\
    	e-kisunsuzo n-ki-wa-ire maama\\
	\textsc{aug}-7.comb \textsc{1sg.sbj}-\textsc{7obj}-give-\textsc{pfv} 1.mother\\
    	\glt ‘I have given the comb to mother.'

\ex	\label{ex-verb-indexing-obj3}
    	\glll Tansomere kitabo.\\
    	ti-a-n-som-ere kitabo\\
	\textsc{neg}-\textsc{1sbj}-\textsc{1sg.obj}-read-\textsc{appl}:\textsc{pfv} 7.book\\
    	\glt ‘He has not read for me the book.'
\end{xlist}	    
\z

\subsection{Argument indexing on copulas}\label{sec-morpho-copulas}
The copulas \textit{li},\footnote{The vowel of the copula \textit{li} often undergoes progressive assimilation when it is followed by the enclitic \textit{=ku} or \textit{=mu} and is realised as \textit{lu}, as e.g.\,in \textit{Oluku akantu?} `Do you have a mental problem?' This assimilation is not reflected in the orthography.}
\textit{ta}, \textit{bba} take the regular subject agreement prefixes. 
Agreement in noun class is indicated by prefixes that in form are identical with the subject agreement on verbal predicates, see Table~\ref{tab-cop-agreement}. 
The usage of copulas is discussed in more detail in Section~\ref{sec-non-verbal-predication}.

The copula \textit{ni} is used when the subject of the clause is focused. 
In contrast to the other copulas, \textit{ni} takes a subject agreement suffix. 
The respective word-forms are given in the rightmost column of Table~\ref{tab-cop-agreement}. 

\begin{table}[!hbt]
\caption{Subject indexing on the copulas \textit{li}, \textit{ta}, \textit{bba}, and \textit{ni}}
\label{tab-cop-agreement}
	\begin{tabular}{llll l}
\lsptoprule
	person / &\textit{li}	& \textit{ta}	& \textit{bba} & \textit{ni}\\
	noun class	 		&			&		&	 & \\
\midrule
	1\textsc{sg}&\textit{n-li}&\textit{n-ta}&\textit{n-bba}		&\textit{ni-nje}\\
	1\textsc{pl}&\textit{tu-li}&\textit{tu-ta}&\textit{tu-bba}		&\textit{ni-swe}\\
	2\textsc{sg}&\textit{o-li}&\textit{o-ta}&\textit{o-bba}	&\textit{ni-iwe}\\
	2\textsc{pl}&\textit{mu-li}&\textit{mu-ta}&\textit{mu-bba}	&\textit{ni-nywe}\\
	1&\textit{a-li}&\textit{a-ta}&\textit{a-bba}			&\textit{ni-ye}\\
	2&\textit{ba-li}&\textit{ba-ta}&\textit{ba-bba}		&\textit{ni-bo}\\
	3&\textit{gu-li}&\textit{gu-ta}&\textit{gu-bba}		&\textit{ni-gwo}\\
	4&\textit{gi-li}&\textit{gi-ta}&\textit{gi-bba}			&\textit{ni-gyo}\\
	5&\textit{li-li}&\textit{li-ta}&\textit{li-bba}			&\textit{ni-ryo}\\
	6&\textit{ga-li}&\textit{ga-ta}&\textit{ga-bba}		&\textit{ni-go}\\
	7&\textit{ki-li}&\textit{ki-ta}&\textit{ki-bba}		&\textit{ni-kyo}\\
	8&\textit{bi-li}&\textit{bi-ta}&\textit{bi-bba}		&\textit{ni-byo}\\
	9&\textit{e-li}&\textit{e-ta}&\textit{e-bba}			&\textit{ni-yo}\\
	10&\textit{zi-li}&\textit{zi-ta}&\textit{zi-bba}		&\textit{ni-zo}\\
	11&\textit{lu-li}&\textit{lu-ta}&\textit{lu-bba}		&\textit{ni-rwo}\\
	12&\textit{ka-li}&\textit{ka-ta}&\textit{ka-bba}		&\textit{ni-ko}\\
	13&\textit{tu-li}&\textit{tu-ta}&\textit{tu-bba}		&\textit{ni-to}\\
	14&\textit{bu-li}&\textit{bu-ta}&\textit{bu-bba}		&\textit{ni-bwo}\\
	15&\textit{ku-li}&\textit{ku-ta}&\textit{ku-bba}		&\textit{ni-kwo}\\
        16&\textit{wa-li}& – &\textit{wa-bba}				& –\\
        17&\textit{ku-li}& – &\textit{ku-bba}				& –\\
        18&\textit{mu-li}& – &\textit{mu-bba}				& –\\      
        20 & \textit{gu-li} & \textit{gu-ta} & \textit{gu-bba}	&\textit{ni-gwo}\\
	22 & \textit{ga-li} & \textit{ga-ta} & \textit{ga-bba}	&\textit{ni-go}\\
        23&\textit{e-li}& – &\textit{e-bba}				& –\\             
\lspbottomrule
	\end{tabular}
\label{tab:subjectagreementlitaabba}
\end{table}


\subsection{Tense}\label{sec-tense}

In this section and the following two sections (Sections \ref{sec-aspect} and \ref{sec-mood}), we discuss the morphology of tense-aspect-mood (henceforth TAM) forms in Ru\-ruu\-li\hyp{}Lu\-nya\-la. 
Like other Bantu languages (see, for instance, \citealt{Nurse2008Tense}), Ru\-ruu\-li\hyp{}Lu\-nya\-la is rich in TAM forms. 
The description of the TAM system of Ru\-ruu\-li\hyp{}Lu\-nya\-la is still in progress and these sections  present only the most common TAM affixes and illustrate their use in independent clauses. 
Furthermore, we only consider TAM categories formed synthetically. The forms and functions of analytically formed tense-aspect categories – including the forms with the auxiliaries \textit{li}, \textit{iza}, and \textit{bba} – are another topic for future research. 

The term \textit{tense} is used following Comrie’s definition according to which ``tense is grammaticalised expression of location in time" \citep[9]{Comrie1985Tense}. 
Table~\ref{tab-verb-tense} lists the most common tense forms in Ru\-ruu\-li\hyp{}Lu\-nya\-la. 
The tenses are marked by prefixes excluding the present tense, which is unmarked. 
The Past tense and the Near Future tense are marked by the  prefixes \textit{á-} and \textit{à-}, respectively, which are differentiated by tone only. 
All these forms take the final vowel \textit{-a}.

\begin{table}[!h]
\caption{Tense forms (\textsc{sbj} stands for subject agreement prefix)}
\label{tab-verb-tense}
	\begin{tabular}{llll}
\lsptoprule

Tense  & Prefix  & Suffix  & Form template\\
\midrule
Present tense  & Ø- & \textit{-a} (\textsc{fv}) & \textsc{sbj}-Ø-\textit{sumb-a}\\
Past tense  & \textit{á-} (\textsc{pst}) & \textit{-a} (\textsc{fv}) & \textsc{sbj}-\textit{á-sumb-a}\\
Near Future  & \textit{à-} (\textsc{fut}) & \textit{-a} (\textsc{fv}) & \textsc{sbj}-\textit{à-sumb-a}\\
Remote Future  & \textit{li-} (\textsc{rfut}) & \textit{-a} (\textsc{fv}) &\textsc{sbj}-\textit{li-sumb-a}\\
\lspbottomrule
	\end{tabular}
\end{table}

\subsubsection{Present}
The Present tense is not overtly marked and there is no other tense or aspect marker on the verb in this form. 
Such so-called `null forms' are found in many Bantu languages (see \citealt[118]{Nurse2008Tense}).
In Ru\-ruu\-li\hyp{}Lu\-nya\-la, the unmarked form refers to the general present. 
The form describes general situations, general truths, or habits and situations which occur frequently. 
Example (\ref{ex-tense-present1}) describes such a general situation, i.e.\,teaching girls the etiquette of greeting. (\ref{ex-tense-present2}) illustrates the habit of eating sweet plantain in a particular way. 

\ea \label{ex-tense-present}
\begin{xlist}
\ex	\label{ex-tense-present1}
	\glll Omwala tumwegesya okulamuca.\\
	o-mwala      tu-mu-egesy-a     o-ku-lamuc-a\\
		\textsc{aug}-1.girl     1\textsc{pl.sbj}-\textsc{1obj}-teach-\textsc{fv}     \textsc{aug}-\textsc{inf}-greet-\textsc{fv}\\
	\glt ‘We teach the girl to greet.’
	
\ex	\label{ex-tense-present2}
	\glll Ogaramba bamulya mwokye.\\
	o-garamba ba-mu-li-a mu-okye\\
		\textsc{aug}-1.sweet\_plantain \textsc{2sbj}-\textsc{1obj}-eat-\textsc{fv} 1-roasted\\
	\glt  ‘Sweet plantain is usually eaten after being roasted.’
\end{xlist}
\z


\subsubsection{Past}

The Past tense is marked by the prefix \textit{á-} and indicates the events which happened  in the past.

\ea \label{ex-tense-past}
\begin{xlist}
	
\ex	\label{ex-tense-past1}
	\glll Yatusuubiza ebbaasi yaiswe.\\
	 a-a-tu-suubiz-a         e-bbaasi     ya-a-iswe\\
		\textsc{1sbj}-\textsc{pst}-\textsc{1pl.obj}-promise-\textsc{fv} \textsc{aug}-9.bus     9-\textsc{assoc}-\textsc{1pl}\\
	\glt ‘He promised us our bus.’
	
\ex	\label{ex-tense-past2}
	\glll Yaali mukama.\\
	 a-a-li mukama\\
		\textsc{1sbj}-\textsc{pst}-\textsc{cop} 1.leader\\
	\glt     ‘He was a leader.’
\end{xlist}
\z

\subsubsection{Near Future and Remote Future}

Ruruuli-Lunyala uses two future tenses which express near and far future. 
The Near Future is marked by the prefix \textit{à-} and expresses situations which happen within a day, as in (\ref{ex-tense-fut1}). 
The Remote Future illustrated in~(\ref{ex-tense-fut2}) is marked by the prefix \textit{li-} and is used for events which will happen any time from tomorrow to an indefinite future. 

\ea \label{ex-tense-fut}
\begin{xlist}
\ex	\label{ex-tense-fut1}
	\glll Abantu baabinuka oku biici emambya.\\
	a-bantu ba-a-binuk-a o-ku biici e-mambya\\
		\textsc{aug}-2.person \textsc{2sbj}-\textsc{fut}-have\_fun-\textsc{fv} \textsc{aug}-17.\textsc{loc} 9.beach \textsc{aug}-tomorrow\\
	\glt ‘The people will have fun at the beach tomorrow.'
	
		
\ex	\label{ex-tense-fut2}
	\glll Ndiiza mu magula.\\
	n-li-iz-a mu magula\\
		\textsc{1sg.sbj}-\textsc{fut}-come-\textsc{fv} 18.\textsc{loc} 6.buying\_season\\
	\glt   `I will come during the buying season.'
\end{xlist}
\z


\subsection{Aspect}\label{sec-aspect}
\largerpage[-1]
The term \textit{aspect} is used here following Comrie's (\citeyear[3]{Comrie1985Tense}) definition, which states that ``aspects are different ways of viewing the internal temporal constituency of a situation". 
Like many other Bantu languages, Ru\-ruu\-li\hyp{}Lu\-nya\-la has the habitual, progressive, persistive, and perfective aspects (on aspectual categories in other Bantu languages, see \citealt{Nurse2008Tense}). 
This section outlines the morphological structure of these aspectual categories and illustrates their usage.


\subsubsection{Perfective}\label{sec-aspect-perfective}

The perfective in Ru\-ruu\-li\hyp{}Lu\-nya\-la  is marked by the suffix /ire/ or its allomorphs. 
Apart from some exceptions, the distribution of the allomorphs is predictable, though several rules which determine the  word-form might be involved. 
Furthermore, with some verbs two alternative verb-forms are possible. 
The Dictionary part of this book (see Part~\ref{sec-Dictionary}) lists the perfective form for each verb.  
The account below is not exhaustive and discusses only the most frequent patterns. 
The use of the perfective is illustrated at the end of this section.  

The distribution of the two most common allomorphs [ire] and [ere] is conditioned by vowel height harmony (see Section~\ref{sec-phonology-vowel-harmony}): 
The allomorph [ire] is used when the stem vowels are /i/, /a/, or /u/, and some examples are given in~(\ref{ex-aspect-perfective-ire}). 
The allomorph [ere] is used when the stem vowel is /o/ or /e/, as in~(\ref{ex-aspect-perfective-ere}).

\ea \label{ex-aspect-perfective-ire}
\begin{tabbing}
xxxxxxxxxxx \= xxxxxxxxxxxxxxxxx \=`xxxxxxxxxxxxxxx'\kill
    \textit{samba}	\>`to kick' \>	\textit{sambire}\\
    \textit{simba}	\>`to make a line'\>	\textit{simbire}\\
    \textit{sumba}\>	`to cook'\>	\textit{sumbire}\\
    \textit{aba}		\>`to go'\> \textit{abire}\\
    \textit{tata}	 	\>`to make a net'\> \textit{tatire}\\
    \textit{semba}	\> `to come last'\> \textit{sembere}
\end{tabbing}
\z

\ea \label{ex-aspect-perfective-ere}
\begin{tabbing}
xxxxxxxxxxx \= xxxxxxxxxxxxxxxxx \=`xxxxxxxxxxxxxxx'\kill
    \textit{loota}	\>`to dream'\>	\textit{lootere}\\
    \textit{nena}	\>	`to bite'\>	\textit{nenere}\\
    \textit{somba}	\>`to collect'\> \textit{sombere}\\
    \textit{semba}	\> `to come last'\> \textit{sembere}
\end{tabbing}
\z

In addition to the above pattern, 
like in other Bantu languages, the formation of the perfective in Ruruuli-Lunyala involves imbrication. 
In Bantu studies, the term \textit{imbrication} is used to refer to the phenomenon of morpheme interweaving under certain morphophonological conditions (see \citealt{Bastin1983Finale, Hyman1995Minimality}, as well as \citet{Hymanetal2017Multiple} for a description of imbrication in the closely-related language Lusoga). 
The various imbrication patterns involved in the formation of the perfective word-forms are described below.

\newpage
Imbrication is common with longer stems and stems ending in certain consonants. 
In one imbrication subtype, the final consonant of the stem is truncated and the perfective suffix /ire/ fuses with it.
This leads to hiatus and various strategies of hiatus resolution are employed (see Section~\ref{sec-phonology-vowel-hiatus}). 
The examples in~(\ref{ex-aspect-perfective-lateral}) and~(\ref{ex-aspect-perfective-rhotic}) illustrate this process with stems of various length ending in /r/ and /l/. 
The examples in~(\ref{ex-aspect-perfective-exceptions}) illustrate the imbricated perfective forms of roots ending in other consonants. 

\ea \label{ex-aspect-perfective-lateral}
\begin{tabbing}
xxxxxxxxxxx \= xxxxxxxxxxxxxxxxx \=`xxxxxxxxxxxxxxx'\kill
\textit{bala} \>`to yield'\> \textit{baare}\\
\textit{kola} \>`to work'\>  \textit{koore}\\
\textit{bula} \>`to get lost'\> \textit{buure}\\
\textit{byala}	\>`to give birth'\>		\textit{byaire}\\
\textit{ebbala}	\>`to avoid'\>			\textit{ebbaire}\\
\textit{dukula}	\>`to hit hard'\>		\textit{dukwire}\\
\textit{duula}	\>`to brag'\>			\textit{dwire}
\end{tabbing}
\z
\ea \label{ex-aspect-perfective-rhotic}
\begin{tabbing}
xxxxxxxxxxx \= xxxxxxxxxxxxxxxxx \=`xxxxxxxxxxxxxxx'\kill
\textit{cwekera}	\>`to vanish, end'\> 	\textit{cwekeire}\\
\textit{bendeera}	\>`to wait'\>			\textit{bendeire}\\
\textit{bulira}	\>`to preach'\>		\textit{buliire}\\
\textit{walira}	\>`to resist, refuse'\>	\textit{waliire}\\
\textit{galamira}	\>`to lie down'\>		\textit{galamiire}\\
\textit{ebabiira}	\>`to shriek'\>		\textit{ebabiriire}
\end{tabbing}
\z

\ea \label{ex-aspect-perfective-exceptions}
\begin{tabbing}
xxxxxxxxxxx \= xxxxxxxxxxxxxxxxx \=`xxxxxxxxxxxxxxx'\kill
\textit{kwata}	\>`to hold'\>	\textit{kwaite}\\
\textit{bona}	\> `to see'\> \textit{boine}
\end{tabbing}
\z

The same imbrication pattern is found with more complex bases, e.g.\,when in addition to the monomoraic root there is one of the fossilised enclitics, such as /ku/, or the fossilised reflexive prefix /e/, or when the root is  reduplicated. 
Some examples are given in~(\ref{ex-aspect-perfective-monomoraicX}).

\ea \label{ex-aspect-perfective-monomoraicX}
\begin{tabbing}
xxxxxxxxxxx \= xxxxxxxxxxxxxxxxx \=`xxxxxxxxxxxxxxx'\kill
\textit{kolaku} \>`to treat'\> \textit{kooreku}\label{ex-aspect-perfective-monomoraic-clitic}\\
\textit{emala} \>`to be self-sufficient'\> \textit{emaare}\label{ex-aspect-perfective-monomoraic-refl}\\
\textit{emalaimala} \>`to fake innocence'\> \textit{emaareimaare}\label{ex-aspect-perfective-monomoraic-redupl}
\end{tabbing}
\z

With other bases, the perfective allomorph [i…e] ``interweaves" with the root and results in forms ending in [iCe], where C is the final consonant of the base. 
This type of imbrication is used with verbs ending in [an-a]: they form the perfective by changing it to [ain-e], as in~(\ref{ex-extensions-perfective-associative}).\footnote{For many of these verbs two alternative forms of the perfective are entered in the dictionary (see Part~\ref{sec-Dictionary} Dictionary): In addition to the form with imbrication, some verbs in /ana/ allow a regular perfective suffix as well, e.g.\,\textit{wakana} `to argue, disagree' listed in~(\ref{ex-extensions-perfective-associative}) allows the form \textit{wakanere} in addition to \textit{wakaine}. 
It remains a topic for future research to identify what determines the distribution of the two forms.}
In many, if not all relevant cases, [an] is the lexicalised associative extension (cf.\,Proto-Bantu \textit{*-an}, \citealt[173]{Schadebergetal2019Bantu}, see Section~\ref{sec-extension-associative}). 

\ea \label{ex-extensions-perfective-associative}
\begin{tabbing}
xxxxxxxxxxx \= xxxxxxxxxxxxxxxxxxxxxxxxx \=`xxxxxxxxxxxxxxx'\kill
\textit{awukana} \>`to differ' \>\textit{awukaine}\\
\textit{fukana} \>`to wrestle, fight'   \>   \textit{fukaine}\\
\textit{gugulana} \>`to trade (with each other)'   \>   \textit{gugulaine}\\
\textit{kumbaana} \>`to assemble, come together'   \>   \textit{kumbaine}\\
\textit{kyankalana} \>`to be annoyed'   \>   \textit{kyankalaine}\\
\textit{liraana} \>`to neighbour, be near to'   \>   \textit{liraine}\\
\textit{swoitana} \>`to quarrel'   \>   \textit{swoitaine}\\
\textit{wakana} \>`to argue, disagree'   \>   \textit{wakaine}\\
\textit{walana} \>`to hate, dislike'   \> \textit{walaine}
\end{tabbing}
\z

The historically complex reciprocal suffix \textit{-angan} was formed with the associative suffix \textit{-an} (see Section~\ref{sec-extension-associative}). 
Parallel to the way the perfective of the verbs with the associative suffix are formed, the perfective form of the verbs with this suffix is formed by the same process of imbrication of the perfective allomorph [i…e] with the reciprocal suffix resulting in the form [angaine], as in~(\ref{ex-aspect-reciprocal-pfv}).  

\ea
\label{ex-aspect-reciprocal-pfv}
\begin{tabbing}
xxxxxxxxxxxxx \= xxxxxxxxxxxxxxxxxxxxx \=`xxxxxxxxxxxxxxx'\kill
\textit{izukiry-angan-a} \>`to remind each other' \> \textit{izukiry-angine}\\
\textit{siim-angan-a} \>`to appreciate each other' \> \textit{siim-angaine}\\
\textit{swer-angan-a} \>`to marry each other' \> \textit{swer-angaine}
\end{tabbing}
\z

Another variant of imbrication is found with bases which end in a glide.  
These are often the lexicalised or productive causative extension \textit{-isy}/\textit{-y} (often accompanied by the applicative extension) or the passive extensions \textit{-ibw}/\textit{-w}, see the respective descriptions in Sections~\ref{sec-extension-causative} and~\ref{sec-extension-passive}). 
With bases ending in a liquid followed by a glide, the liquid is truncated and the perfective allomorph [ir…e] is interwoven with the glide resulting in [irGe], as in~(\ref{ex-aspect-imbrication-glide-Ay}) and (\ref{ex-aspect-imbrication-glide-Aw}). 
With bases ending in other consonants followed by a glide, there is no truncation of the consonant, but the perfective allomorph [ir…e] also interweaves with the glide, as illustrated in~(\ref{ex-aspect-imbrication-glide-By}) and~(\ref{ex-aspect-imbrication-glide-Bw}).

\ea
\label{ex-aspect-imbrication-glide-Ay}
\begin{tabbing}
xxxxxxxxxxxxx \= xxxxxxxxxxxxxxxxxxxxx \=`xxxxxxxxxxxxxxx'\kill
\textit{babulya} \>`to singe' \> \textit{babwirye}\\
\textit{boolya} \>`to cultivate' \> \textit{boirye}\\
\textit{toolya} \>`to look for' \> \textit{toirye}\\
\textit{lisiirya} \>`to dine with' \> \textit{lisiirye}\\
\textit{keperya} \>`to limp' \> \textit{kepeirye}\\
\textit{neneerya} \>`to grind (one's teeth)' \> \textit{neneirye}
\end{tabbing}
\z

\ea
\label{ex-aspect-imbrication-glide-Aw}
\begin{tabbing}
xxxxxxxxxxxxx \= xxxxxxxxxxxxxxxxxxxxx \=`xxxxxxxxxxxxxxx'\kill

\textit{lumiirwa} \>`to sympathise' \> \textit{lumiirwe}\\
\textit{ziriirwa} \>`to not hatch' \> \textit{ziriirwe}\\
\textit{gasirwa} \>`to benefit from' \> \textit{gasiirwe}\\
\textit{bboberwa} \>`to cover (millet dough)' \> \textit{bbobeirwe}\\
\textit{kerwa} \>`to be tired' \> \textit{keerwe}

\end{tabbing}
\z

\ea
\label{ex-aspect-imbrication-glide-By}
\begin{tabbing}
xxxxxxxxxxx \= xxxxxxxxxxxxxxxxxxxxxxx \=`xxxxxxxxxxxxxxx'\kill
\textit{bbandisya} \>`to summon spirits' \> \textit{bbandisirye}\\
\textit{bbuubya} \>`to wade (through)' \> \textit{bbuubirye}\\
\textit{atya} \>`to split' \> \textit{atirye}\\
\textit{deedya} \>`to carry (a heavy burden)' \> \textit{deederye}
\end{tabbing}
\z

\ea
\label{ex-aspect-imbrication-glide-Bw}
\begin{tabbing}
xxxxxxxxxxx \= xxxxxxxxxxxxxxxxxxxxxxx \=`xxxxxxxxxxxxxxx'\kill
\textit{subwa} \>`to miss (an opportunity)' \> \textit{subirwe}\\
\textit{bandwa} \>`to worship' \> \textit{bandirwe}\\
\textit{etamwa} \>`to worship' \> \textit{etamirwe}\\
\textit{nugwa} \>`to reject' \> \textit{nugirwe}
\end{tabbing}
\z

The only exception to the patterns outlined above is the weakly suppletive form of the verb `to know' given in~(\ref{ex-aspect-perfective-exceptions-know}).

\ea \label{ex-aspect-perfective-exceptions-know}
\begin{tabbing}
xxxxxxxxxxx \= xxxxxxxxxxxxxxxxx \=`xxxxxxxxxxxxxxx'\kill
\textit{manya}	\> `to know'\> \textit{maite}
%\ex \textit{(oku)weerya}	`togive':	\textit{waire}
\end{tabbing}
\z

The use of the Perfective aspect is illustrated in~(\ref{ex-TAM-perfective}). Many further examples can be found throughout this grammar sketch.
\ea \label{ex-TAM-perfective}
\begin{xlist}
\ex	
	\glll Twanenibwe emida enjunzai.\\
	  tu-a-nen-eibwe e-mida e-njunzai\\
		1\textsc{pl.sbj}-\textsc{pst}-bite-\textsc{pass}:\textsc{pfv} \textsc{aug}-4.louse \textsc{aug}-10.jigger\\
	\glt ‘We were bitten by lice, jiggers.’ 

\ex 		\glll Yakangiire mu irembo.\\
	a-a-kang-ir-ire	mu	irembo\\
		\textsc{1sbj}-\textsc{pst}-stop-\textsc{appl}-\textsc{pfv} 18.\textsc{loc} 5.house\\
\glt ‘He has stopped in front of the house.' 
\end{xlist}
\z


\subsubsection{Progressive} \label{sec-aspect-progressive}

The Progressive aspect is used to express actions in progress.
It is marked by the prefix \textit{ku-} and only occurs with the Present tense. 
It is illustrated in~(\ref{ex-TAM-progressive}). 

\ea \label{ex-TAM-progressive}
\begin{xlist}
\ex	
	\glll Onkoko akumaamira mawuli.\\
	o-nkoko a-ku-maamir-a mawuli\\
		\textsc{aug}-1.hen \textsc{1sbj}-\textsc{prog}-incubate-\textsc{fv} 6.egg\\
\glt  ‘The hen is sitting on eggs.’

\ex 		\glll Ndi ani nkunyumirwa nyembo.\\
	n-li ani n-ku-nyumirw-a nyembo\\
		\textsc{1sg.sbj}-\textsc{cop} here \textsc{1sg.sbj}-\textsc{prog}-enjoy-\textsc{fv} 9.music\\
\glt ‘I am here enjoying the music.'

\end{xlist}
\z
 
\subsubsection{Habitual} \label{sec-aspect-habitual}
The Habitual aspect is used to encode an action or state performed usually, ordinarily, or customarily. 
It is particularly common to describe traditional habits and events of the past and in this case the speakers regularly translate it with `used to' into English.
The aspect is marked by the suffix \textit{-nga} and is illustrated in~(\ref{ex-TAM-habitual}). 

\ea \label{ex-TAM-habitual}
\begin{xlist}

\ex 		\glll Abantu bairai baazwalanga endoobe.\\
	a-bantu ba-irai ba-a-zwal-a-nga e-ndoobe\\
	\textsc{aug}-2.person 2-ancient  \textsc{2sbj}-\textsc{pst}-wear-\textsc{fv}-\textsc{hab} \textsc{aug}-9.loin\_cloth\\
\glt ‘The people in the past used to wear loin cloths.'

\ex 		\glll Toyabanga mu kisiko nga tokwaite mwigo.\\
	ti-o-ab-a-nga mu kisiko nga ti-o-kwaite mwigo\\
	\textsc{neg}-2\textsc{sg.sbj}-go-\textsc{fv}-\textsc{hab} 18.\textsc{loc} 7.bush	while \textsc{neg}-2\textsc{sg.sbj}-hold:\textsc{pfv} 3.stick\\
\glt ‘Never go to the bush without a stick.'
\end{xlist}
\z
 
\subsubsection{Persistive} \label{sec-aspect-persisitive}

The major function of the persistive prefix \textit{kya-} is to express three of the concepts of phasal polarity, viz.\,\textsc{still}, \textsc{not yet} and \textsc{no longer}. 
When the main verb carries the prefix \textit{kya-}, the construction encodes the concept of \textsc{still}, as in~(\ref{ex-TAM-persistive-STILL}). 
The negation of this form yields the phasal polarity concept of \textsc{no longer}, as in~(\ref{ex-TAM-persistive-NOLONGER}) (see also Section~\ref{sec-negation} and \citealt{Molochieva2018Phasal} for further details). 


\ea \label{ex-TAM-persistive}
\begin{xlist}
\ex	\label{ex-TAM-persistive-STILL}
	\glll Abantu		baingi	bakyakolesya	emole 		okuomboka	enyumba	zaabwe.\\
	a-bantu		ba-ingi	ba-kya-kolesy-a 	e.mole 		o-ku-ombok-a	e-nyumba	z-a-abwe	\\
	\textsc{aug}-2.people	2-many	\textsc{2sbj}-\textsc{pers}-use-\textsc{fv}	  10.reed 	\textsc{aug}-\textsc{inf}-build-\textsc{fv}	\textsc{aug}-10.house	10-\textsc{assoc}-2\\
\glt  ‘Many people still use reeds to construct their houses.’

\ex 		\label{ex-TAM-persistive-NOLONGER}
	\glll Abantu	baingi	tibakyakolesya	emole { }{ } okwomboka	enyumba	zaabwe.\\
	a-bantu	ba-ingi	ti-ba-kya-kolesy-a	e-mole { }{ } o-ku-ombok-a	enyumba	za-a-abwe\\
	\textsc{aug}-2.people	2-many	\textsc{neg}-\textsc{2sbj}-\textsc{pers}-use-\textsc{fv}	\textsc{aug}-10.reed { }{ } \textsc{aug}-\textsc{inf}-build-\textsc{fv}	10.house	10-\textsc{assoc}-2\\
\glt ‘Many people no longer use reeds to construct their houses.'

\ex 	\label{ex-TAM-persistive-NOTYET}
	\glll Abantu	baingi	bakyali	kukolesya	emole.\\
	a-bantu	ba-ingi	ba-kya-li	ku-kolesy-a	e-mole\\
	\textsc{aug}-2.people	2-many	\textsc{2sbj}-\textsc{pers}-\textsc{aux}	\textsc{inf}-use-\textsc{fv}	\textsc{aug}-10.reed\\
\glt ‘Many people do not yet use reeds (to construct their houses.)'
\end{xlist}
\z


\subsection{Mood}\label{sec-mood}

Following \citet[1190]{Hengeveld2004Illocution}, we define the morphological category of mood as encompassing all grammatical elements operating on a proposition and not directly concerned with situating an event in the actual world, that is, these are morphological elements that are not tense, aspect or negation. 
Ruruuli-Lunyala has two morphologically distinguished moods viz.\,the indicative and the subjunctive. 
These two  mood categories have a wide range of applications and among other functions are involved in the expression of various illocutions discussed in Section~\ref{sec-syntax-illocutions}.

The indicative is indicated by the final vowel \textit{-a} with the majority of verb forms. However, the perfective suffix blocks the appearance of the final vowel. 
The final vowel is in complimentary distribution with the subjunctive suffix \textit{-e}. 
Both the form and the function of this suffix are similar to the cognate ones in the closely related Great Lakes Bantu languages \citep{Nurseetal1999Tense}, as well as in many other Bantu languages \citep[44, 192]{Nurse2008Tense}. 
The final \textit{*-e} has been reconstructed for Proto-Bantu \citep{Meeussen1967Bantu}. 

In Ru\-ruu\-li\hyp{}Lu\-nya\-la, subjunctive verb forms show subject agreement but no marking for either tense or aspect. 
When the verb root is reduplicated, the subjunctive suffix is attached after each reduplicant. The subjunctive form of verbs with the reciprocal suffix \textit{-angan} (see Sections~\ref{sec-extension-reciprocal} and \ref{sec-reciprocal}) is \textit{-engan-e}, that is, in addition to the regular subjunctive suffix \textit{-e}, the first vowel of the reciprocal suffix also changes to [e], as shown  in~(\ref{ex-mood-reciprocal-sbjv}).

 
\ea
\label{ex-mood-reciprocal-sbjv}
\begin{xlist}
\ex\textit{nyweger-angan-a} `kiss each other', \textsc{sbjv}: \textit{nyweger-engan-e}
\ex\textit{it-angan-a} `kill each other', \textsc{sbjv}: \textit{it-engan-e}
\ex\textit{akaly-angan-a} `go past each other', \textsc{sbjv}: \textit{akaly-engan-e}
\end{xlist}
\z


\subsection{Negation}\label{sec-negation}

All tense-aspect combinations are negated by means of the prefix \textit{ti-}; 
see~(\ref{ex-ruulistandarddrop}) for an illustration of negation in the present progressive. 
The prefix is realised as /t/ before vowels, e.g.\,before the second person subject prefix \textit{o-} in~(\ref{ex-ruulipresprog2neg}).

\ea 
\label{ex-ruulistandarddrop}
\begin{xlist}

\ex
\label{ex-ruulipresprog1}
	\glll Nkwaba.\\
	n-ku-ab-a\\
	\textsc{1sg.sbj}-\textsc{prog}-go-\textsc{fv}\\
	\glt ‘I am going.’ 
	
\ex \label{ex-ruulipresprog1neg}
	\glll Tinkwaba.\\
	  ti-n-ku-ab-a\\
		\textsc{neg}-\textsc{1sg.sbj}-\textsc{prog}-go-\textsc{fv}\\
	\glt ‘I am not going.’ 	
	
\ex \label{ex-ruulipresprog2}
	\glll Okwaba.\\
	 	o-ku-ab-a\\
		2\textsc{sg.sbj}-\textsc{prog}-go-\textsc{fv}\\
	\glt ‘You are going.’ 
	
\ex \label{ex-ruulipresprog2neg}
	\glll Tokwaba.\\
	  	ti-o-ku-ab-a\\
		\textsc{neg}-2\textsc{sg.sbj}-\textsc{prog}-go-\textsc{fv}\\
	\glt ‘You are not going.' 
\end{xlist}
\z

Except for the near future tense, negation constructions for all tense and aspect combinations in Ru\-ruu\-li\hyp{}Lu\-nya\-la are symmetric. 
In the negative near future, the indicative suffix \textit{-a} – glossed as \textsc{fv} for final vowel, as is common in the descriptions of Bantu languages, see e.g.\,\citet[44]{Nurse2008Tense} – is replaced by the subjunctive suffix \textit{-e}, as in~(\ref{ex-ruulistandardasyconstr}).
Thus, the negative near future construction shows A/Cat/TAM asymmetry. 
This is not the case for any of the other tense and aspect combinations. 

\ea 
\label{ex-ruulistandardasyconstr}
\begin{xlist}
\ex
\label{ex-ruulinear.future}
	\glll  Nayaba.\\
	n-a-ab-a\\
	\textsc{1sg.sbj}-\textsc{fut}-go-\textsc{fv}\\
	\glt ‘I will go (soon).’ 
\ex
\label{ex-ruulinear.futureneg}
	\glll Tinayabe.\\
		ti-n-a-ab-e.\\
		\textsc{neg}-\textsc{1sg.sbj}-\textsc{fut}-go-\textsc{sbjv}\\
	\glt ‘I will not go (soon).’ 	
\ex
\label{ex-ruulinear.futureasterix}
	*\gll ti-n-a-ab-a\\
		\textsc{neg}-\textsc{1sg.sbj}-\textsc{fut}-go-\textsc{fv}\\
\glt Intended: ‘I will not go (soon).’
\end{xlist}
\z

Ruruuli-Lunyala makes use of the persistive prefix \textit{kya-} (see Section~\ref{sec-aspect-persisitive}) to express the phasal polarity concepts \textsc{still}, \textsc{not yet}, and \textsc{no longer}. 
When attached to a main verb, the prefix \textit{kya-} expresses \textsc{still}, as in~(\ref{ex-ruulistill}). 
By adding the standard negator \textit{ti-} to a verb that is marked by the persistive, the concept \textsc{no longer} is expressed, as in~(\ref{ex-ruulinolonger}). 
The construction denoting \textsc{not yet} is formed by the combination of a positive verbal form and the auxiliary \textit{li} which carries the prefix \textit{kya-}, as in~(\ref{ex-ruulipers}).  
This construction seems to be the only tense-aspect combination in Ru\-ruu\-li\hyp{}Lu\-nya\-la that cannot be negated. 
This indicates paradigmatic asymmetry since a distinction that is made in the positive is not made in the negative. 
However, note that the persistive in the affirmative implies negative semantics. 
The construction of a negative persistive is formally possible and has a different meaning than its positive counterpart, but was rejected by the speakers probably because of semantic issues with the English translation.

\ea 
\label{ex-ruulistandardpersistive}
\begin{xlist}
	\ex \label{ex-ruulistill}
	\glll  OKato akyali e Kampala.\\
		o-Kato a-kya-li e Kampala\\
		\textsc{aug}-1.Kato \textsc{1sg.sbj}-\textsc{pers}-\textsc{cop} 23.\textsc{loc} 1.Kampala\\
\glt ‘Kato is still in Kampala.’ 

\ex \label{ex-ruulinolonger}
	\glll OKato takyali e Kampala.\\
	 oKato ti-a-kya-li e Kampala\\
		\textsc{aug}-1.Kato \textsc{neg}-\textsc{1sg.sbj}-\textsc{pers}-\textsc{cop} 23.\textsc{loc} 1.Kampala\\
\glt ‘Kato is no longer in Kampala.’ 
	\ex \label{ex-ruulipers}
	\glll  Nkyali kwaba. \\
	n-kya-li ku-ab-a\\
		\textsc{1sg.sbj}-\textsc{pers}-\textsc{aux} \textsc{inf}-go-\textsc{fv}\\
\glt ‘I have not gone yet.’  	
\end{xlist}
\z

Some tense-aspect forms consist of a combination of one of the auxiliaries and the main verb, e.g.\,the recent past progressive. 
In these forms, the standard negation marker \textit{ti-} (\textsc{neg1} in the scheme in~(\ref{ex-verb-slots})) is prefixed to the main verb. 
Compare the two examples in~(\ref{ex-complexTA}). 

\ea  \label{ex-complexTA}
\begin{xlist}
\ex	\label{ex-complexTA-pos}
	\glll  Abbaire aikala mpani obwomi bwamwe bwona.\\
		a-bba-ire a-ikal-a mpani obwomi bu-a-mwe bu-ona\\
		\textsc{1sbj}-\textsc{aux}-\textsc{pfv} \textsc{1sbj}-live-\textsc{fv} here 14.life 14-\textsc{assoc}-1 14-all\\
\glt ‘He has been living here for all his life.' 

\ex \label{ex-complexTA-neg}
 
	\glll	Emabega edi aBanyala babbaire tibataka kusoma.\\
	  e-mabega edi a-Banyala ba-bba-ire ti-ba-tak-a kusoma\\
		\textsc{aug}-6.past 23.\textsc{loc}.\textsc{dist} \textsc{aug}-2.Banyala \textsc{2sbj}-\textsc{aux}-\textsc{pfv} \textsc{neg}-\textsc{2sbj}-want-\textsc{fv} 15.education\\
\glt  `In the past, the Banyala there did not want education.’
\end{xlist}
\z

\subsection{Extensions}\label{sec-morph-verb-extensions}

In Bantu scholarship the term \textit{extension} refers to certain suffixes which follow the root of the verb. 
The canonical Bantu extension has the shape -VC \citep[173]{Schadebergetal2019Bantu}. 
An extension may be analysable as having a specific meaning, in which case one may call it a suffix. 
On the other hand, segmentation may be done on purely formal grounds, so that the analysis yields a formal radical and an “expansion” or formal suffix (see e.g.\,\citealt[\,172–173]{Schadebergetal2019Bantu}). 
Bantu extensions include applicative, causative, passive and reciprocal, as well as a range of others, such as impositive, neuter or tentive.
The inventory of Bantu extensions, as well as their functions and properties vary from language to language. 
This section provides an overview of the morphology of Ru\-ruu\-li\hyp{}Lu\-nya\-la extensions, \viz applicative (Section~\ref{sec-extension-applicative}), causative (Section~\ref{sec-extension-causative}), passive (Section~\ref{sec-extension-passive}), reciprocal (Section~\ref{sec-extension-reciprocal}), and associative (Section~\ref{sec-extension-associative}). 

Note that the reflexive form is not considered an extension in Bantu scholarship (e.g.\,it is missing from the list of extensions in \citealt[173]{Schadebergetal2019Bantu}) and is not discussed in this section, see instead Section~\ref{sec-reflexive}. 
The syntactic properties of the Ru\-ruu\-li\hyp{}Lu\-nya\-la extensions are discussed in Section~\ref{sec-syntax-extensions}. 


\subsubsection{Applicative}\label{sec-extension-applicative}

The applicative suffix \textit{-ir} derives from Pro\-to-Bantu \textit{*-ıd} \citep[173]{Schadebergetal2019Bantu} and many Bantu languages are reported to still possess a variant of this suffix (for a recent study, see \citealt{Pacchiarotti2017Bantu}). 
The distribution of the two most common applicative allomorphs [ir] and [er] is conditioned by vowel height harmony (see Section~\ref{sec-phonology-vowel-harmony}, see also \citealt{Atuhairwe2019Applicative} for further examples).  
The suffix surfaces as [ir] when the last vowel of the root is  /i, u, a/. 
It is realised as [er] when preceded by /e/ or /o/ in the root, as the examples in~(\ref{extension-appl-ir-er}) show.

\ea \label{extension-appl-ir-er}
The applicative allomorphs [ir] vs.\,[er]
\begin{xlist}
\ex
\textit{kubb-a}	`to beat':	\textit{kubb-ir-a} (beat-\textsc{appl}-\textsc{fv}) `to beat for’\\
\textit{sumb-a}	`to cook':	\textit{sumb-ir-a} (cook-\textsc{appl}-\textsc{fv}) `to cook for’\\
\textit{iruk-a}	 `to run':	 \textit{iruk-ir-a} (run-\textsc{appl}-\textsc{fv}) `to run for’\\
\textit{kang-a}	 `to stop':	 \textit{kang-ir-a} (stop-\textsc{appl}-\textsc{fv}) `to stop for’

\ex
\textit{leet-a}	`to bring':	\textit{leet-er-a} (beat-\textsc{appl}-\textsc{fv}) `to bring for’\\
\textit{kol-a} 	`to do': \textit{kol-er-a} (do-\textsc{appl}-\textsc{fv}) `to do for’\\
\textit{kob-a} 	`to say':  \textit{kob-er-a} (say-\textsc{appl}-\textsc{fv}) `to say for’\\
\textit{som-a} 	`to read': \textit{som-er-a} (read-\textsc{appl}-\textsc{fv}) `to read for’
\end{xlist}
\z

In the perfective verb forms, the liquid is deleted and the applicative allomorph has the shape of either [i] or [e], as in (\ref{extension-appl-pfv}). 
The perfective suffix is \textit{-ire} following both allomorphs and does not undergo the vowel height harmony in this context.

\ea \label{extension-appl-pfv}
\begin{xlist}
\ex
\textit{kubb-a}	`to beat':	\textit{kubb-iire}  (beat-\textsc{appl}:\textsc{pfv})  `to beat for’\\
\textit{sumb-a}	`to cook':	\textit{sumb-iire}  (cook-\textsc{appl}:\textsc{pfv})  `to cook for’\\
\textit{iruk-a}	 `to run':	 \textit{iruk-iire} (run-\textsc{appl}:\textsc{pfv})  `to run for’\\
\textit{kang-a}	 `to stop':	 \textit{kang-iire}  (stop-\textsc{appl}:\textsc{pfv})  `to stop for’

\ex
\textit{leet-a}	`to bring':	\textit{leet-eire} (read-\textsc{appl}:\textsc{pfv})  `to bring for’\\
\textit{kol-a} `to do': \textit{kol-eire} (do-\textsc{appl}:\textsc{pfv})  `to do for’\\
\textit{kob-a} `to say':  \textit{kob-eire} (say-\textsc{appl}:\textsc{pfv})  `to say for’\\
\textit{som-a} `to read': \textit{som-eire} (read-\textsc{appl}:\textsc{pfv})  `to read for’
\end{xlist}
\z

In addition to the pattern outlined above, as with the formation of the perfective forms discussed in Section \ref{sec-aspect-perfective}, one also finds various types of imbrication in the formation of the applicative forms. 
Verb roots with two and more syllables ending in /z/ form the applicative form by inserting the applicative allomorphs [ir] and [er] before /z/, as in~(\ref{extension-appl-z-infix}). 
Monosyllabic roots ending in /z/ form the applicative in the regular way, as in~(\ref{extension-appl-z-regular}). 

\ea \label{extension-appl-z}
\begin{xlist}

\ex \label{extension-appl-z-infix}
\textit{bigiz-a}	`to moderate': 	\textit{bigiriz-a} (moderate:\textsc{appl}-\textsc{fv}) `to moderate for'\\
\textit{tereez-a}	`to straighten': 	\textit{tereerez-a} (straighten:\textsc{appl}-\textsc{fv}) `to straighten for'

\ex \label{extension-appl-z-regular}
\textit{iz-a}	`to come':  \textit{iz-ir-a} (come-\textsc{appl}-\textsc{fv})  `to come for'\\
\textit{oz-a}	`to wash (laundry)': 	\textit{oz-er-a} (wash-\textsc{appl}-\textsc{fv})  `to wash for'
\end{xlist}
\z

Also verb stems in /j/ (<y> in the orthographic representation) insert the applicative suffix before /j/, as in~(\ref{extension-appl-j}). 
These include cases where synchronically /j/ is part of the root, as in~(\ref{extension-appl-j-root}), as well as where it is a causative suffix (see Section~\ref{sec-extension-causative} on causatives), as in~(\ref{extension-appl-j-caus}).

\ea \label{extension-appl-j}
\begin{xlist}

\ex \label{extension-appl-j-root}
\textit{bbasy-a}	`to sleep': 	\textit{bbasiry-a} (sleep:\textsc{appl}-\textsc{fv}) `to sleep for'\\
\textit{rapy-a}	`to wander':  	\textit{rapiry-a} (wander:\textsc{appl}-\textsc{fv}) `to wander for'

\ex \label{extension-appl-j-caus}
\textit{naaby-a} `to wash':  	\textit{naabiry-a} (wash:\textsc{appl}-\textsc{fv})  `to wash for'\\
\textit{bungeety-a}	`to spread (rumours)':  	\textit{bungeetery-a} (spread:\textsc{appl}-\textsc{fv})  `to spread (rumours) for'
\end{xlist}
\z

With roots ending in the liquid /l/, the liquid is truncated, as in~(\ref{extension-appl-l}). 

\ea \label{extension-appl-l}
\textit{kol-a}	`to work': 	\textit{koor-a} (work:\textsc{appl}-\textsc{fv}) `to work for'\\ 
\textit{twal-a}	`to take': 	\textit{twar-a} (take:\textsc{appl}-\textsc{fv}) `to take for'\\ 
\textit{simool-a}	`to speak':  	\textit{simoor-a} (speak:\textsc{appl}-\textsc{fv}) `to speak for 
\z

The form of the applicative with monosyllabic stems is synchronically not always predictable. 
Some forms can be predicted by considering the root vowel in the proto-forms, whereas other forms seem to be idiosyncratic. A few examples are given in~(\ref{extension-appl-monosyllabic}). 

\ea \label{extension-appl-monosyllabic}
Applicative with monosyllabic verbs
\begin{xlist}
\ex \textit{sya}	`to grind':  \textit{sy-er-a} (grind-\textsc{appl}-\textsc{fv}) `to grind for'
\ex \textit{bbwa}	`to bind': \textit{bbw-er-a}  (bind-\textsc{appl}-\textsc{fv}) `to bind for'
\ex \textit{lya}	 `to eat': 	\textit{leera} (eat-\textsc{appl}-\textsc{fv}) `to eat for'
\ex \textit{nia}	 `to defecate':	\textit{ni-er-a} (defecate-\textsc{appl}-\textsc{fv}) `to defecate for'
\end{xlist}
\z


\subsubsection{Causative}\label{sec-extension-causative}

Two causative allomorphs \textit{*-i/-ici} have been reconstructed for Proto-Bantu \citep[173]{Schadebergetal2019Bantu} and many Bantu languages are reported to still possess two variants of the suffix albeit with variable productivity and distribution. 
Ru\-ruu\-li\hyp{}Lu\-nya\-la also has two causative suffixes viz.\,\textit{-isy} /isj/ and \textit{-y} /j/.\footnote{Underlyingly, the two suffixes are /isi/ and /i/, but as they are always followed by a vowel and thus undergo vowel coalescence and glide formation (see Section~\ref{sec-phonology-vowel-hiatus}), we cite them as \textit{-isy} /isj/ and \textit{-y} /j/.\label{fn-isi-caus}}

The suffix \textit{-isy} /isj/ has two phonologically conditioned allomorphs: The allomorph \textit{-esy} [esj] is used after roots with the vowels /e/ or /o/, as in~(\ref{ex-extensions-causative-esy}), 
whereas the allomorph \textit{-isy} [isj] is used with all other roots, as in~(\ref{ex-extensions-causative-isy}) (see Section~\ref{sec-phonology-vowel-harmony} on vowel height harmony). 

The examples below also list the perfective form of the causative verbs: these are built by imbricating the two suffix, which results in \textit{-isirye} [isirje]/\textit{-esirye} [eserje] (see Section~\ref{sec-aspect-perfective}).

\ea \label{ex-extensions-causative-esy}
\begin{xlist}
\ex\textit{bon-a} `to see':  \textit{bon-esy-a} (see-\textsc{caus}-\textsc{fv}) `to cause to see, show', 
\textit{bon-eserye} (see-\textsc{caus}:\textsc{pfv})
\ex\textit{emb-a} `to sing':  \textit{emb-esy-a} (sing-\textsc{caus}-\textsc{fv}) `to cause to sing',  \textit{emb-eserye}
(sing-\textsc{caus}:\textsc{pfv})
\ex\textit{komb-a} `to lick':   \textit{komb-esy-a} (lick-\textsc{caus}-\textsc{fv}) `to cause to lick', \textit{komb-eserye} 
(lick-\textsc{caus}:\textsc{pfv})
\ex\textit{ombok-a} `to build':  \textit{ombok-esy-a} (build-\textsc{caus}-\textsc{fv}) `to cause to build', \textit{ombok-eserye} (build-\textsc{caus}:\textsc{pfv})
\ex\textit{sos-a} `to examine':   \textit{sos-esy-a} (examine-\textsc{caus}-\textsc{fv}) `to cause to examine', \textsc{pfv}: \textit{sos-eserye} (examine-\textsc{caus}:\textsc{pfv})
\end{xlist}
\z

\ea \label{ex-extensions-causative-isy}
\begin{xlist}
\ex\textit{ita-a} `to kill':  \textit{it-isy-a} (kill-\textsc{caus}-\textsc{fv}) `to cause to kill', \textit{it-isirye} (kill-\textsc{caus}:\textsc{pfv})
\ex\textit{kul-a} `to grow, get bigger':  \textit{kul-isy-a} (grow-\textsc{caus}-\textsc{fv}) `to cause to grow, bring up', \textit{kul-isirye} (grow-\textsc{caus}:\textsc{pfv})
\ex\textit{wandiik-a} `to write':  \textit{wandiik-isy-a} (write-\textsc{caus}-\textsc{fv}) `to cause to write', \textit{wandiik-isirye} (write-\textsc{caus}:\textsc{pfv})
\end{xlist}
\z

The causative suffix \textit{-y} /j/ illustrated in~(\ref{ex-extensions-causative-y-lex}) occurs only with some verbs. 
Its perfective forms is also built by imbricating the two affixes into [irye]/[erye]. 
Many of these verbs also allow the use of the more common causative suffix /isj/ instead, as in~(\ref{ex-extensions-causative-y-lex-iruk}), though further research is needed to account for the preferences and possibilities of the use of the allomorphs with individual verbs.

\ea \label{ex-extensions-causative-y-lex}
\begin{xlist}

\ex\textit{naab-a} `to wash (intr.), get washed':  \textit{naab-y-a} (wash-\textsc{caus}-\textsc{fv}) `to wash (tr.), cause to get washed',  \textit{naab-irye} (wash-\textsc{caus}:\textsc{pfv})

\ex\textit{emeer-a} `to stand':  \textit{emeer-y-a} (stand-\textsc{caus}-\textsc{fv}) `to stop, halt', \textit{mere-irye} (stand-\textsc{caus}:\textsc{pfv})

\ex \label{ex-extensions-causative-y-lex-iruk}
\textit{iruk-a} `to run':  \textit{iruk-y-a} or \textit{iruk-isy-a} (run-\textsc{caus}-\textsc{fv}) `to make run, chase', \textsc{pfv}: \textit{iruk-irye} or \textit{iruk-isirye} (run-\textsc{caus}:\textsc{pfv})

\end{xlist}
\z

Furthermore, the causative suffix \textit{-y} is used with all verbs which also have either a lexicalised or a productive applicative suffix, as in~(\ref{ex-extensions-causative-y-appl}).

\ea \label{ex-extensions-causative-y-appl}
\begin{xlist}
\ex \textit{gonger-a} `to be annoyed, be upset' (lexicalised applicative):  \textit{gonger-y-a} (be\_annoyed-\textsc{caus}-\textsc{fv})  `to annoy, upset', \textit{gong-eirye} (be\_annoyed-\textsc{appl}:\textsc{caus}:\textsc{pfv})
\ex\textit{ikiir-a} `to be complete, be achieved' (lexicalised applicative):  \textit{ikiir-y-a} (be\_complete-\textsc{caus}-\textsc{fv})  `to fulfil, make complete', \textit{ikiriirye} (be\_complete:\textsc{caus}:\textsc{pfv})
\ex\textit{ituka-a} `to recur, reappear':  \textit{ituk-ir-y-a} (recur-\textsc{appl}-\textsc{caus}-\textsc{fv}) `to revisit, reopen (a case)', \textit{ituk-iirye} (recur-\textsc{appl}:\textsc{caus}:\textsc{pfv})
\end{xlist}
\z

\subsubsection{Passive}\label{sec-extension-passive}
\largerpage
Ruruuli-Lunyala has a productive passive construction, whose syntax is dis\-cussed in detail in Section~\ref{sec-passive}. 
It is marked by the post-radical/ pre-final suffix \textit{-ibw/} (/ibw/) or \textit{-w} (/w/).\footnote{Underlyingly, the two passive suffixes are /ibu/ and /u/, but as they are always followed by a vowel and hence undergo vowel coalescence and glide formation (Section~\ref{sec-phonology-vowel-hiatus}), we cite them with the final /w/, similarly to our approach to the representation of the causative suffix outlined in Footnote~\ref{fn-isi-caus} in Section~\ref{sec-extension-causative}.}  
/ibw/ has two allomorphs, viz.\,[ibw] and [ebw]. 
The distribution of the two allomorphs is conditioned by vowel harmony (see Section~\ref{sec-phonology-vowel-harmony}): the suffix is realised as [ibw] when preceded by /i, u, a/, as in~(\ref{ex-extensions-passive-ibw}), and as [ebw] when preceded by /e/ or /o/ or when the stem has no vowel, as illustrated in~(\ref{ex-extensions-passive-ebw}). 
The suffix \textit{-w} seems to be in free variation with \textit{-ibw}, as shown in~(\ref{ex-extensions-passive-w}).

\ea \label{ex-extensions-passive-ibw}
\begin{xlist}
\ex \textit{siig-a} `to smear': \textit{siig-ibw-a} (smear-\textsc{pass}-\textsc{fv})
\ex \textit{sumb-a} `to cook': \textit{sumb-ibw-a} (cook-\textsc{pass}-\textsc{fv})
\ex \textit{tak-a} `to want':	\textit{tak-ibw-a} (want-\textsc{pass}-\textsc{fv})
\ex \textit{ly-a} `to eat':	\textit{li-ibw-a} (eat-\textsc{pass}-\textsc{fv})
\end{xlist}
\z

\ea \label{ex-extensions-passive-ebw}
\begin{xlist}
\ex \textit{et-a}	`to call':	\textit{et-ebw-a} (call-\textsc{pass}-\textsc{fv})
\ex \textit{tendek-a}	`to train':	\textit{tendek-ebw-a} (train-\textsc{pass}-\textsc{fv})
\ex \textit{bbw-a}	`to detain':	\textit{bbw-ebw-a} (detain-\textsc{pass}-\textsc{fv})
\end{xlist}
\z

\ea \label{ex-extensions-passive-w}
\begin{xlist}
\ex \textit{kol-a}	`to do':		\textit{kol-ebw-a} or \textit{kol-w-a} (do-\textsc{pass}-\textsc{fv})
\ex \textit{som-a}	`to read':	\textit{som-ebw-a} or \textit{som-w-a} (read-\textsc{pass}-\textsc{fv})
\end{xlist}
\z

In the perfective, the passive and the perfective are realised as a portmanteau morpheme \textit{-iibwe/-eibwe}, as in~(\ref{ex-extensions-passive-perfective}). 
This is another instance of so-called imbrication, discussed in Section~\ref{sec-aspect-perfective}.  

\ea \label{ex-extensions-passive-perfective}
\begin{xlist}
\ex\textit{tund-a}	`to sell':\\
\textit{tund-ire} (sell-\textsc{pfv}),  \textit{tund-iibwe} (sell-\textsc{pass}:\textsc{pfv})
\ex\textit{let-a}	`to see':\\
\textit{letere} (see-\textsc{pfv}), \textit{let-eibwe} (see-\textsc{pass}:\textsc{pfv})
\end{xlist}
\z


\subsubsection{Reciprocal}\label{sec-extension-reciprocal}

This section briefly discusses the morphological properties of the reciprocal extension, 
its syntactic properties and functions are discussed in Section~\ref{sec-reciprocal}. 
The reciprocal in Ru\-ruu\-li\hyp{}Lu\-nya\-la is marked by the suffix \textit{-angan}, as in~(\ref{ex-extensions-reciprocal}). 
 This is a historically complex suffix made up of the repetitive \textit{*-ag/-ang} and the associative \textit{*-an} (\citealt[173]{Schadebergetal2019Bantu}, see also \citealt[378]{Dometal2015Antipassive} on the origin of the similar complex reciprocal suffix in Cilubà).  

\ea \label{ex-extensions-reciprocal}
\begin{xlist}
\ex\textit{nyweger-a}	`kiss':\\
\textit{nyweger-angan-a} (kiss-\textsc{recp}-\textsc{fv}) `kiss each other'
\ex\textit{it-a}	`kill': \\
\textit{it-angan-a} (kill-\textsc{recp}-\textsc{fv}) `kill each other'
\ex\textit{akaly-a}	`go past': \\
\textit{akaly-angan-a} (go\_past-\textsc{recp}-\textsc{fv}) `go past each other'
\end{xlist}
\z

The perfective of the reciprocal verbs is formed by imbricating the regular \textit{-ire} suffix with the reciprocal suffix, which results in the suffix \mbox{\textit{-angaine}}, as in~(\ref{ex-extensions-reciprocal-pfv}) (see Section~\ref{sec-aspect-perfective}). 
This pattern is identical to the way the perfective of the verbs with the lexicalised associative suffix are formed (see Section~\ref{sec-extension-associative}).  
The form of the subjunctive is \textit{-engane}, that is, in addition to the regular subjunctive suffix \textit{-e}, the first vowel of the reciprocal suffix also changes to [e]. Some examples are provided in~(\ref{ex-extensions-reciprocal-pfv}).
 
\ea
\label{ex-extensions-reciprocal-pfv}
\begin{xlist}

\ex\textit{nyweger-angan-a} (kiss-\textsc{recp-fv}) `kiss each other':\\
\textit{nyweger-angaine} (kiss-\textsc{recp:pfv}), \\
\textit{nyweger-engane} (kiss-\textsc{recp:sbjv})

\ex\textit{it-angan-a} (kill-\textsc{recp-fv}) `kill each other':\\
\textit{it-angaine} (kill-\textsc{recp:pfv}), \\
\textit{it-engane} (kill-\textsc{recp:sbjv})

\ex\textit{akaly-angan-a} (go\_past-\textsc{recp-fv}) `go past each other': \\ 
\textit{akaly-angaine} (go\_past-\textsc{recp:pfv}), \\
\textit{akaly-engane} (go\_past-\textsc{recp:sbjv})
\end{xlist}
\z

\subsubsection{Associative}\label{sec-extension-associative}

Another extension reconstructed for Proto-Bantu is the associative extension \textit{*-an} \citep[173, 182–184]{Schadebergetal2019Bantu}. 
In the Bantu languages the most productive meaning of this extension is reciprocal. 

The associative extension \textit{-an} is not productive in Ru\-ruu\-li\hyp{}Lu\-nya\-la, but it is lexicalised with dozens of verbs. 
Some associative verbs have a corresponding verb without the associative extension, though they differ in terms of valency, as in~(\ref{ex-extensions-associative-counterpart}). 
Others do not have a corresponding non-associative verb, e.g.\,the ones in~(\ref{ex-extensions-associative-nocounterpart}). 
Many of the associative verbs refer to natural reciprocal situations and naturally collective actions
(cf.\,\citealt[267–270]{Kemmer1993Middle}).

\ea \label{ex-extensions-associative-counterpart}
\begin{xlist}
\ex \textit{awukana} `to differ, be different; to separate (in a fight or relationship)', \textsc{pfv}: \textit{awukaine}; 
from \textit{awuka} `to branch off, diverge from the main route; to differ (from)' 
\ex \textit{gugulana} `to trade (with each other)' (intr.),  \textsc{pfv}: \textit{gugulaine};  from \textit{gula} `to buy, purchase' (trans.)
\end{xlist}
\z

\ea \label{ex-extensions-associative-nocounterpart}
\begin{xlist}
\ex \textit{fukana} `to wrestle, fight' (intr.), \textsc{pfv}: \textit{fukaine}
\ex \textit{kumbaana} `to assemble, come together' (intr.), \textsc{pfv}: \textit{kumbaine}
\ex \textit{kyankalana} `to be annoyed' (intr.), \textsc{pfv}: \textit{kyankalaine}
\ex \textit{liraana} `to neighbour, be near to' (trans.), \textsc{pfv}: \textit{liraine}
\ex \textit{swoitana} `to quarrel' (intr.), \textsc{pfv}: \textit{swoitaine}
\ex \textit{wakana} `to argue, disagree; compete' (intr.), 
(\textsc{pfv}: \textit{wakaine})
\ex \textit{walana} `to hate, dislike' (trans.), \textsc{pfv}: \textit{walaine}
\end{xlist}
\z

Verbs with the associative extension form the perfective form by imbricating the perfective suffix \textit{-ire/-ere} with the extension \textit{-an}. This results in the form \textit{-aine}, as in~(\ref{ex-extensions-associative-nocounterpart}) (cf.\,Section~\ref{sec-aspect-perfective} on the morphology of the perfective).

\section{Adjectives}\label{sec-adjectives}

An adjective can be defined as a word which can be used to modify a nominal head without any further marking or modification, see e.g.\,\citet[58]{Hengeveld1992Non-verbal}. 
Bantu adjectives agree with the head nouns in noun class using adjectival prefixes. 
The prefix paradigm is fully or nearly identical to that of the nominal prefixes. 
Bantu languages vary as to the number of adjectives they have: some have only a handful (e.g.\,eight in Rwanda JD61), some have none at all (e.g.\,Eton A71, Mongo C61), whereas other languages have an open class of adjectives. 
Many of these adjectives are derived from change-of-state verbs \citep[258]{Vandeveldeetal2019Nominal}. 
Many closely-related Great Lakes Bantu languages are reported to have a small number of true adjectives: Ha (haq, Western Lakes Bantu; \citealt[76]{Harjula2004Ha}) is said to have ``a few adjectives” and the grammar gives a list of twelve adjectives, Kinyarwanda has very few adjectives (kin, Western Lakes Bantu; \citealt[231]{Kimenyi1980Relational}), Chiga has “twenty or so” true adjectives (cgg, \citealt[49]{Taylor1985Nkore-Kiga}), see e.g.\,\citet{Segerer2008Closed} for an overview. 
Ru\-ruu\-li\hyp{}Lu\-nya\-la has a small number of underived adjectives and a relatively large number of derived ones. Below is the list of the most common underived ones. 
Derived adjectives are discussed in Section~\ref{sec-derivation-adjective}.

Adjectives in Ru\-ruu\-li\hyp{}Lu\-nya\-la agree with the head noun in noun class (see Table~\ref{tab-adjectives}).
The adjective agreement prefixes are similar to the ones which occur on other agreement targets and are thus distinct from the noun class prefixes on nouns in classes 3, 4, 5, 6, and 10.\footnote{Class 4 agreement prefixes have two dialectally-conditioned variants viz.\,\textit{gi-} and \textit{zi-}.}
The same prefixes are used on adjectives both in the predicative and in the attributive function. 
Adjectives can also agree with the argument in the first and second person when they are used predicatively. 

\begin{table}[]
\caption{Noun class agreement prefixes on adjective}
\label{tab-adjectives}
\begin{tabularx}{\textwidth}{XXl}
\lsptoprule
Class & Prefix & Examples\\
\midrule
1  & \textit{mu-} & \textit{o-musaiza o-mu-sai}\\
  & 		     & \textsc{aug}-1.man \textsc{aug}-1-good ‘a good man’\\
2 & \textit{ba-} & \textit{a-basaiza a-ba-sai}\\
   & 			& \textsc{aug}-2.man \textsc{aug}-2-good ‘good men’\\
3 & \textit{gu-} & \textit{o-musaale o-gu-sai}\\
   & 			&\textsc{aug}-3.tree \textsc{aug}-3-good ‘a good tree’\\
4 & \textit{gi-} (\textit{zi-})& \textit{e-misaale e-gi-sai}\\
   & 			&\textsc{aug}-4.tree \textsc{aug}-4-good ‘good trees’\\
5 & \textit{li-/ri-} & \textit{e-itakali e-li-sai}\\
   & 			&\textsc{aug}-5.plot \textsc{aug}-5-good ‘a small plot’\\
6  & \textit{ga-} & \textit{a-magezi a-ga-sai}\\
   & 			&\textsc{aug}-6.advice \textsc{aug}-6-good ‘good advice’\\
7 & \textit{ki-} & \textit{e-kibuuzo e-ki-sai}\\
   & 		      &\textsc{aug}-7.question \textsc{aug}-7-good ‘a good question’\\
8  & \textit{bi-} & \textit{e-biseera e-bi-sai}\\
   & 			&\textsc{aug}-8.time \textsc{aug}-8-good ‘good times’\\
9 & \textit{n-}  & \textit{e-mpisa e-n-sai}\\
 & or \textit{gi-}&\textsc{aug}-9.manner \textsc{aug}-9-good ‘good manner’\\
10 & \textit{zi-} & \textit{e-mpisa e-zi-sai}\\
   & 			& \textsc{aug}-10.manner \textsc{aug}-10-good ‘good manners’\\
11 & \textit{lu-} & \textit{o-lugendo o-lu-sai}\\
   & 			& \textsc{aug}-11.trip \textsc{aug}-11-good ‘a good trip’\\
12 & \textit{ka-} & \textit{a-kafo a-ka-sai}\\
   & 			& \textsc{aug}-12.place \textsc{aug}-12-good ‘a good place’\\
13 & \textit{tu-} &\textit{o-tumere o-kusai}\\
		&& \textsc{aug}-13.little.food \textsc{aug}-13-good `little good food’\\
14 & \textit{bu-} & \textit{o-bubonero bu-sai}\\
   & 			&14.mark \textsc{aug}-14-good  `good grades’\\
15 & \textit{ku-} & \textit{o-kusomoozebwa o-ku-sai}\\
 &  & \textsc{aug}-15.challenge \textsc{aug}-15-good ‘a good big challenge’\\
20 & \textit{go-} & \textit{o-gusolo  o-gu-sai}\\
 &  & \textsc{aug}-20.animal \textsc{aug}-20-good ‘a good large animal’\\
22 & \textit{ga-} & \textit{o-gasukaali o-ga-kusai}\\
 &  & \textsc{aug}-22.sugar \textsc{aug}-22-good ‘a lot of good sugar’\\
\lspbottomrule
	\end{tabularx}
\end{table}

The form of the agreement prefixes has the same allomorphs as on other hosts, as illustrated with the examples in~(\ref{ex-adj-allomorphs}).

\ea \label{ex-adj-allomorphs}
\begin{xlist}
\ex 	\label{ex-adj-allomorphs1}
	\glll omusaiza	omugezi\\
	o-musaiza	o-mu-gezi\\
		\textsc{aug}-1.man	\textsc{aug}-1-intelligent\\
	\glt ‘an intelligent man’
	
\ex 	\label{ex-adj-allomorphs2}
	\glll onyana	omweru\\
	o-nyana	o-mu-eru\\
	\textsc{aug}-1.calf \textsc{aug}-1-white\\
	\glt ‘a white calf’

\ex 	\label{ex-adj-allomorphs3}
	\glll onyana	omwiragazu\\
    	o-nyana	o-mu-iragazu\\
		\textsc{aug}-1.calf	\textsc{aug}-1-black\\
\glt ‘a black calf’
\end{xlist}
\z

When used predicatively adjectives agree in noun class with the subject noun and do not take the augment, as~(\ref{ex-morph-adjpred}) shows.

\ea \label{ex-morph-adjpred}
\begin{xlist}
\ex
	\gll  Boona	balimu		basai.\\
	    ba-ona	ba-li=mu		ba-sai\\
		2-all	\textsc{2sbj}-be=18.\textsc{loc}	 2-good\\
	\glt ‘They are all fine.’	
\ex 
	
	\gll	Omuyembe ogubbisi ti gusai kulya.\\
	o-muyembe o-gu-bbisi ti gu-sai ku-li-a\\
	\textsc{aug}-3.mango \textsc{aug}-3-raw \textsc{neg}.\textsc{cop} 3-good \textsc{inf}-eat-\textsc{fv}\\
	\glt ‘A raw mango is not good for eating.’

\end{xlist}
\z


\section{Adverbs}\label{sec-adverbs}

The term \textit{adverb} refers to a word class (or part of speech) used to specify the circumstances of the verbal or sentential referent (cf.\,\citealt{Maienbornetal2011Adverbs}). 
In a language, this function can also be  fulfilled by other types of adverbials, e.g.\,by adpositional phrases. 
Such expressions are not discussed in this section, but see Section \ref{sec-morh-locative} for some examples.  
Typically, adverbs are restricted to a set of semantically limited usages: they specify time, place, or manner. 

Morphologically, adverbs in Ru\-ruu\-li\hyp{}Lu\-nya\-la are either underived, e.g.\,\textit{(o)luusi} `sometimes' and \textit{dala} ‘really, indeed', or derived from other word classes, e.g.\,from nouns, adjectives and verbs. 
The majority of Ru\-ruu\-li\hyp{}Lu\-nya\-la adverbs seems to be derived, 
recurrent patterns are illustrated in (\ref{ex-adv-derivation}).\footnote{Some of the prefixes are formally identical to noun class prefixes on nouns and various agreement targets, e.g.\,\textit{ma-} class 6, \textit{bu-} class 14, \textit{ku-} class 15.}  
Currently we have not been able to identify the respective roots in case of all adverbs, though they often carry the same prefixes as clearly derived adverbs.
Some examples are given in~(\ref{ex-adv-unclear-derivation}).

\ea  \label{ex-adv-derivation}
\begin{xlist}

\ex adverbs with the augment prefix \textit{a-}:\footnote{In some contexts the prefix \textit{a-} can be dropped and further research is needed to determine whether the same conditions which determine the use of augments elsewhere apply to adverbs and whether all adverbs in \textit{a-} behave consistently in this respect.}\textit{ampi} `near, at least, almost' < \textit{mpi} `short, small', \textit{amwei} `together' < \textit{mwei} `one', \textit{ansi} `down, to the ground' < \textit{nsi} `ground, land'

\ex adverbs with the prefix \textit{ma-}: \textit{mabberege} `like a pig' < \textit{mberege} `pig' (1a)', \textit{mabbwene} `like a dog' < \textit{mbwene} `dog (1a)', \textit{magalama} `on one’s back (e.g.\,of falling)' < \textit{galama} `to lie on one's back', \textit{mamiima} `tightly, closely' < \textit{miima} `to fit tightly'  \label{ex-adv-ma}

\ex adverbs with the prefix \textit{bu-}: \textit{bujuumuke} `upside down' < \textit{juumuka}	 `to turn upside down', 
\textit{bwomwei} `alone' < \textit{mwei} `one', \textit{buyaaka} `again' < \textit{yaaka} `new'
\ex  adverbs with the prefix \textit{ku-}: \textit{kusai} `well' < \textit{sai} `good', \textit{kubbi} `badly' < \textit{bbi} `bad'

\ex adverbs with the prefix \textit{ka-}: \textit{kadooli} `nearly, almost' < \textit{dooli} `small, little', \textit{kabiri} `twice' < \textit{biri} `two', \textit{kacuucu} `speedily, quickly' < \textit{cuucu} `dust' 
\ex adverbs with the prefix \textit{ki-}: \textit{kimwei} `totally, completely' < \textit{mwei} `one'
\end{xlist}
\z

\ea  \label{ex-adv-unclear-derivation}
\begin{xlist}
\ex adverbs starting with in \textit{ma-}: \textit{mangu} `fast'
\ex adverbs starting with in \textit{bu-}: \textit{bwemi} `still, without moving', \textit{bulebe} `somewhere', \textit{bwereere} `for no reason'
\ex adverbs with reduplication: \textit{biralebirale} `carelessly', \textit{cekeceke} `(of rain) continuously', \textit{fukufuku} `heavily, excessively'
\end{xlist}
\z

Most adverbs in Ru\-ruu\-li\hyp{}Lu\-nya\-la occur in the clause-final position but they may also appear in the clause-initial position. 
Section~\ref{sec-syntax-adverbs} provides examples and discusses the most common classes of adverbs grouped according to semantic categories. 


\section{Derivation}\label{morpho-derivation}

This section briefly outlines the major derivational strategies of Ru\-ruu\-li\hyp{}Lu\-nya\-la. 
We discuss nominal derivation in Section~\ref{sec-derivation-noun}. Section~\ref{sec-derivation-adjective} outlines the strategies of adjective derivation. 
This short overview does not exhaust the derivational strategies. 
Derivation of verbs and adverbs is attested but is less common and goes beyond the scope of this grammar sketch.


\subsection{Nominal derivation}\label{sec-derivation-noun}
Most derived  nouns in Ru\-ruu\-li\hyp{}Lu\-nya\-la are derived from verbs. 
There also a few nouns which are derived from other nouns. 
The verb to noun derivations include agentive, locative, instrumental and result nouns. 
Several patterns can be identified, which are presented below.

\subsubsection{Verb-to-noun derivation of abstract nouns}\label{sec-morphology-derivation-abstract}

Deverbal abstracts nouns are often assigned to noun class 14 and indeed most nouns of the class 14 are deverbal nouns. 
Other common class assignment options are noun classes 4, 5, 6, and 7 (see Section~\ref{sec-morh-noun-classes-overview} on noun morphology and noun classes). 
These nouns mostly do not have plural counterparts (i.e.\,they are assigned to the so-called monoclasses, see Section~\ref{sec-morh-unpaired}). 
Deverbal abstract nouns often end in the vowels /a/, /e/, /o/. 
This pattern is illustrated in~(\ref{ex-verb-to-noun-derivation}). 
Some of these nouns have further derivational affixes,  e.g.\,\textit{bubaziro} `freedom of speech (14)’ in~(\ref{ex-verb-to-noun-derivation-baziro}) or \textit{butasoma} `illiteracy (14)’ in~(\ref{ex-verb-to-noun-derivation-butasoma}).

\ea \label{ex-verb-to-noun-derivation}
\begin{xlist}
\ex Abstract nouns ending in /e/:\\
\textit{bubbwe} ‘imprisonment (14)’	 < \textit{bbwa} ‘to imprison’,\\
\textit{buganyire}  ‘forgiveness  (14)’	< \textit{ganyira}  ‘to forgive’,\\
\textit{bugaite} ‘sum  (14)’		< \textit{gaita}       ‘to add’

\ex Abstract nouns with the suffix \textit{-i}:\\
\textit{bubeyegi}  `deceit  (14)’ < \textit{beyega} ‘to deceive’,\\
\textit{buculeeri} ‘calmness  (14)’  < \textit{culeera}	 ‘to be calm’,\\
\textit{bufugi}     ‘rule  (14)’	< \textit{fuga} ‘to rule’

\ex  \label{ex-verb-to-noun-derivation-baziro}
Abstract nouns with the suffix \textit{-o}:\\
\textit{buganyuro} `benefit  (14)’ <  \textit{ganyura} ‘to benefit’,\\
\textit{bubaziro} `freedom of speech  (14)’ <  \textit{baza} ‘to speak’

\ex  \label{ex-verb-to-noun-derivation-butasoma}
Abstract nouns with the suffix \textit{-a}:\\
\textit{butasoma} `illiteracy  (14)’ < \textit{soma} ‘to study’
\end{xlist}
\z

\subsubsection{Verb-to-noun derivation of agentive nouns}\label{sec-morphology-derivation-agentive}

Similar to the process described in  Section~\ref{sec-morphology-derivation-abstract} above, 
agentive nouns are regularly derived from verbs through the suffixation of the derivational suffixes \textit{-i} 
to the verb root and simultaneous prefixation of the noun class 1 or 2 prefix. Some examples are given in (\ref{ex-morphology-derivation-agentive}).

\newpage
\ea  Agentive nouns derived with the suffix \textit{-i} \label{ex-morphology-derivation-agentive}
\begin{xlist}
\ex  \textit{mukoli} ‘worker (1)’ <  \textit{kola} ‘to work’ 
\ex  \textit{musumbi} ‘cook (1)’ <  \textit{sumba} ‘to cook’ 
\ex  \textit{mutegi} ‘fisherman (1)’ <  \textit{tega} ‘to fish’
\ex  \textit{mutambi} ‘herbalist/healer (1)’ <  \textit{tamba} ‘to heal’
\ex  \textit{mutayiiri} ‘circumcision surgeon (1)’ <  \textit{tayiira} ‘to circumcise’ 
\end{xlist}
\z


\subsubsection{Verb-to-noun derivation of patient nouns}

Patient nouns are derived from verbs through the suffixation of the derivational suffixes  \textit{-e}  
to the verb root and simultaneous prefixation of the noun class 1 or 2 prefix, as illustrated in~(\ref{ex-derivation-patient-nouns}).

\ea  Patient nouns derived with the suffix \textit{-e}\label{ex-derivation-patient-nouns}
\begin{xlist}
\ex 	 \textit{muwambe} ‘abductee (1)’ <   \textit{wamba} ‘to abduct’
\ex 	 \textit{mubyale}  ‘native (1)' < 	  \textit{byala} ‘to give birth’
\ex 	  \textit{munyage}  ‘captive (1)’ <   \textit{nyaga} ‘to capture’
\ex 	  \textit{mukomole} ‘circumcised person (1)’  <	 \textit{komola} ‘to circumcise’  
\end{xlist}
\z


\subsubsection{Verb-to-noun derivation of result nouns}
Result nouns refer to entities that come into existence during the event denoted by the base verb. 
We find it challenging to make a clear distinction between some abstract nouns discussed in Section~\ref{sec-morphology-derivation-abstract} and result nouns at the moment. 
In~(\ref{ex-derivation-result-nouns}), only examples of derived nouns referring to tangible objects are presented. 
The majority of result nouns seem to be formed with the suffix  \textit{-e} and are assigned to noun class 7. 
Only a small number of examples have been identified so far.

\ea \label{ex-derivation-result-nouns}
\begin{xlist}
\ex Result nouns of  noun class 7 derived with the suffix  \textit{-e}:\\
	 \textit{kibaize} ‘piece of furniture (7)’ <  \textit{baiza} ‘to do woodworking’ \\ 
	 \textit{kibumbe} ‘piece of pottery (7)’ <  \textit{bumba} ‘to mould with clay’

\ex Result nouns of noun  class 3 derived with the suffix  \textit{-e}:\\
	 \textit{musale} ‘cut, laceration (3)’ <  \textit{sala} ‘to cut, slice’
\end{xlist}
\z

\subsubsection{Verb-to-noun derivation of instrumental nouns}
Instrumental nouns are used to refer to physical objects or abstract concepts used as instruments. 
Derived instrumental nouns in Ru\-ruu\-li\hyp{}Lu\-nya\-la are derived using a number of strategies illustrated in~(\ref{ex-derivid-instr}). 
Some are derived by attaching a noun class prefix to the basic form in  \textit{-a}, while others use the causative stem, as in~(\ref{ex-derivid-instr-caus2})–(\ref{ex-derivid-instr-caus4}).
 
\ea \label{ex-derivid-instr}
\begin{xlist}
	\ex  \textit{kivuga} `musical instrument (7)' \\ <  \textit{vuga} `to play (an instrument)'
	
	\ex  \label{ex-derivid-instr-caus2}
	 \textit{kisindabulyo} `key, opener (7)' \\ <  \textit{sindabulya} `open (\textsc{caus})' (<  \textit{sindabula} `to open (tr.)' )
	
	\ex   \textit{kawandikisyo} `pen (12)' \\ <   \textit{wandiikisya} `write (\textsc{caus})' (<  \textit{wandiika} `to write')
	
	\ex 	\label{ex-derivid-instr-caus4}
	 \textit{kamanyisyo} `signpost, notification (12)' \\
	<  \textit{manyisya} `know (\textsc{caus})' (<  \textit{manya} `to know')
\end{xlist}
\z


\subsubsection{Deverbal location nouns} 
Some nouns referring to places are derived with the suffix  \textit{-o} from applicative verbs, as in~(\ref{ex-deverbal-location-nouns}). 
(See Section~\ref{sec-applicative} on the use of the applicative with locative adjuncts.) 
These nouns are assigned to noun class 5.

\ea \label{ex-deverbal-location-nouns}
\begin{xlist}
	\ex \textit{icumbiro} ‘kitchen (5)’ \\
	<   \textit{cumbira} `cook (\textsc{appl})' ( \textit{cuma} ‘to cook’)
	
	\ex  \textit{(e)ikumbaaniro} ‘social centre (5)'\\ 
	<  \textit{kumbaanira}  `gather (\textsc{appl})' ( \textit{kumbaana} ‘to gather (intr.)’)
	
	\ex  \textit{irwaro} ‘hospital (5)' \\
	<  \textit{lwalira}  `fall ill (\textsc{appl})' (<  \textit{lwala} ‘to fall ill’)
\end{xlist}
\z


\subsection{Adjectival derivation} \label{sec-derivation-adjective}

Ruruuli-Lunyala makes use of different processes to derive adjectives from other parts of speech. 
Most derived adjectives are derived via suffixation from change-of-state verbs. 
The most common suffix is  \textit{-i}. 
Much less frequent are  \textit{-u},  \textit{-e},  \textit{-a}, and  \textit{-o} in this order. 
Some examples are listed in~(\ref{ex-derivation-adjectives}). 
The choice of the suffix seems to be conditioned by the argument structure of the respective verb. 
For instance, the suffix  \textit{-e} attaches primarily to bivalent verbs. 

\ea \label{ex-derivation-adjectives}
\begin{xlist}
\ex	Suffix  \textit{-i}:

 \textit{kairiki} `old' <  \textit{kairika} `to grow old'\\
	\textit{tukuliki} `red' < \textit{tukulika} `to turn red'\\
	\textit{cucuki} `faded' < \textit{cucuka} `to fade'

\ex Suffix \textit{-u}:

\textit{kalu} `dry' < \textit{kala} `to become dry'\\
\textit{gumu} `strong, rigid' < \textit{guma} `to become strong'\\
\textit{gondu} `to become soft' < \textit{gonda} `to become soft' 

\ex Suffix \textit{-e}:

\textit{tabule} `mixed, prepared (of a food)' < \textit{tabula} `to mix, combine' 

\ex Suffix \textit{-a}:

\textit{doma} `stupid' < \textit{doma} `to become stupid'

\ex Suffix \textit{-o}:

\textit{yonjo} `clean' < \textit{yonja} `to make clean' 
\end{xlist}
\z

There are also cases of adjectival derivations using reduplication. 
Some examples are given in~(\ref{ex-derivation-adjectives-morepatterns}). 
Some of these adjectives are derived from verbs by one of the suffixes discussed above, whereas in other cases no verbal counterpart is attested. 
The use of reduplication often results in attenuation of the meaning.

\ea \label{ex-derivation-adjectives-morepatterns}
\begin{xlist}
\ex \textit{labukirabuki} `sharp and keen' <  \textit{labuka} `to become aware, to be on the alert' 
\ex \textit{myukimyuki} `reddish' < \textit{myuki} `red'
\ex  \textit{gezigezi} `cunning, sly' < \textit{gezi} `intelligent, wise'
\end{xlist}
\z


\subsection{Negative derivation} \label{sec-derivation-negative}

The preposition \textit{nambula} `without' can be compounded with nouns to indicate the lack of the compounded noun. 
For example, \textit{buculeeri} `peace (14)' is compounded with the preposition \textit{nambula}, which results in the adjective \textit{nambulabuculeeri} `restless, insecure'. 
The noun \textit{kinambulamugaso} `useless thing (7)' consists of the noun \textit{mugaso} `3.importance', 
and the preposition \textit{nambula} `without'. 
This noun is assigned to noun class 7 and receives the prefix \textit{ki-} 
(the noun class 7 also includes the generic noun \textit{kintu} `thing (7)').
At present, it is difficult to assess the productivity of this process. 
However, only a  small number of negative derivations of the above type is registered in the dictionary and no examples are found in the corpus.

\ea 	\label{ex-ruuliwithout}
	\glll Ensi eginambulabuculeeri tekulaakulana.\\
  e-nsi e-gi-nambulabuculeeri ti-e-kulaakulan-a\\
		\textsc{aug}-9.country \textsc{aug}-9-insecure \textsc{neg}-\textsc{1sbj}-develop-\textsc{fv}\\
\glt ‘An insecure country does not develop.'
\z
