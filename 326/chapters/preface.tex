\addchap{\lsPrefaceTitle}
\section*{Preface by Saudah Namyalo}\label{sec-preface}
\addcontentsline{toc}{section}{Preface by Saudah Namyalo}
This publication presents the first edited compilation of selected lemmas and a grammar sketch of Ru\-ruu\-li\hyp{}Lu\-nya\-la. 
It is a product of a Volkswagen Foundation funded project entitled ‘A comprehensive bilingual talking Luruuli/Lunyala-English\footnote{The spelling of the language name as Luruuli/Lunyala was used in the project title before the community of speakers agreed that the spelling of the language name should be Ru\-ruu\-li\hyp{}Lu\-nya\-la. 
In the rest of the book we use exclusive the spelling Ru\-ruu\-li\hyp{}Lu\-nya\-la.} dictionary with a descriptive basic grammar for language revitalisation and enhancement of mother-tongue based education’.
The aim of the project was to contribute towards the revitalisation of the endangered and mainly oral Ru\-ruu\-li\hyp{}Lu\-nya\-la language (ISO 639-3: ruc). 
Ru\-ruu\-li\hyp{}Lu\-nya\-la is the native language of the Baruuli/Banyala. 
It is spoken in the central region of Uganda. 
The ethnic groups of the Baruuli and Banyala are estimated to be about 160,000 (140,000 Baruuli, 21,000 Banyala, \citealt{Uganda2016National}). 
Ruruuli-Lunyala is one of the most endangered of Uganda’s indigenous languages and was categorised as threatened by \citet{ethnologue2016}, though later surveys categorise it as vigorous \citep{ethnologue2019, ethnologue2020}.
This status change is attributable to the ongoing language and cultural related activities that have been going on in the last ten years. 
For the last ten years, the Baruuli-Banyala people have been struggling for their right to be recognised as a distinct tribe with a distinct language and culture. 
 As part of this struggle, they have put in place kingdoms known as Buruuli and Bunyala Kingdoms. 
Cultural groups have been formed and are working towards language and cultural revitalisation. % (Namyalo 2012).   %TODO fix reference
 These efforts have resulted into series of self-published booklets on mainly the Ru\-ruu\-li\hyp{}Lu\-nya\-la culture, the on-going translation of the Bible into Ru\-ruu\-li\hyp{}Lu\-nya\-la by SIL, as well as the publication of basic literacy books.

The production of the dictionary was preceded by a compilation of both mainly the spoken and to a lesser extent written corpus of Ru\-ruu\-li\hyp{}Lu\-nya\-la. 
The corpus provided frequency information about lexemes, senses and collocations, as well as some of the authentic examples which were used in this dictionary. 
Additionally, the corpus was exploited to obtain basic lexical grammatical information on most of the entries in the dictionary. 

This volume has about 12,000 entries with ample examples to meet the needs of most users, including the native speakers, students, scholars and researchers. 
The present dictionary, in addition to being the first to be published on this language, is the initial volume of a two-volume set which describes the Ru\-ruu\-li\hyp{}Lu\-nya\-la language.  
There will also be a standalone grammar describing the grammatical structure thus offering significantly more detailed grammatical information than the grammatical sketch included into to the present volume. 
Additionally, this dictionary will have a counterpart of an online talking dictionary and a mobile app of an adapted version of the dictionary. 


\section*{Foreword from Isabaruuli of the Buruuli Kingdom}\label{sec-foreword-1}
\addcontentsline{toc}{section}{Foreword from Isabaruuli of the Buruuli Kingdom}

Ruruuli-Lunyala language is the means with which the Baruuli-Banyala people express the creative arts of orature and literature and it is one of the items of Buruuli cultural heritage. Like many other Uganda’s minority languages, Ru\-ruu\-li\hyp{}Lu\-nya\-la is an underdeveloped language. It still lacks written literature and its use in formal domains, such as in schools, print media, and law among others is yet to be achieved. For decades, Ru\-ruu\-li\hyp{}Lu\-nya\-la survived under the shadow of dominant languages, such as Luganda, Runyoro, and Lusoga. 
This among other things affected its development, as it was reduced to a ‘home language’. 
However, the coming of the National Resistance Movement (NRM) government in power in 1986 opened space for all indigenous communities in Uganda to promote and safeguard their languages and cultures. 
One of priority areas of Uganda National Culture Policy 2006 is the development and promotion of languages of indigenous communities in Uganda with one of its key interventions being ensuring the development of dictionaries in local languages. 
Thus, as the Isabaruuli of Buruuli and custodian of Buruuli cultural heritage, I appreciate the Government of Uganda under the leadership of His Excellency Gen.\,Yoweri Kaguta Museveni for allowing all ethnic communities in Uganda to promote and safeguard their local languages. 
In addition, I would like to thank the development partners particularly The Volkswagen Foundation (Germany) for funding the Ru\-ruu\-li\hyp{}Lu\-nya\-la-English dictionary project.  
I also thank Dr.\,Saudah Namyalo of the Department of African Languages from Makerere University and her entire research team for compiling this dictionary with a grammar sketch of Ru\-ruu\-li\hyp{}Lu\-nya\-la. 

I call upon all stakeholders particularly the Baruuli and Banyala people to always be proud of their language and culture. 
The only way to preserve and promote the language, necessitates that you use it in your day to day interactions. I also urge you to continue doing research and documenting it. I am optimistic that with this dictionary, Ru\-ruu\-li\hyp{}Lu\-nya\-la shall continue to be promoted and safeguarded from extinction.\bigskip

\noindent \textbf{Okulwa Nyamwanga Obumwei N’okweyomboka. (For God, Unity, and Development.)}

\noindent (Isabarongo Mwatyansozi Mwogezi Butamanya)


\section*{Foreword from Isabanyala of the Bunyala Kingdom}\label{sec-foreword-2}
\addcontentsline{toc}{section}{Foreword from Isabanyala of the Bunyala Kingdom}

Welcome to the first edition of the Ru\-ruu\-li\hyp{}Lu\-nya\-la-English dictionary. This dictionary comes at the time when a large number of children and youths in the Buruuli and Bunyala kingdoms cannot read, write nor speak their mother tongue. 
The continued use of English as the sole language of instruction in schools, the dominance of area languages such as Runyoro, Luganda, and Lusoga, and the multiplicity of languages in Buruuli-Bunyala Kingdom has been one of the major challenges that has affected the growth and development of Ru\-ruu\-li\hyp{}Lu\-nya\-la for decades. 

It is therefore with great pleasure that I welcome the first edition of Ru\-ruu\-li\hyp{}Lu\-nya\-la-English dictionary. 
This dictionary will with no doubt lead to increased acquisition of Ru\-ruu\-li\hyp{}Lu\-nya\-la among all the Baruuli and Banyala people as well as others who may wish to learn the language. 
All the definitions in this dictionary are written in clear, simple Ru\-ruu\-li\hyp{}Lu\-nya\-la with English translations that are easy to understand. 
The dictionary has a section on the Ru\-ruu\-li\hyp{}Lu\-nya\-la orthography, as well as a section on the grammar. I encourage you to pick interest and read it. This will help to learn how to write the language in addition to knowing its grammar. 
Lastly, allow me to thank Dr.\,Saudah Namyalo and her entire research team for the job well done. Writing a dictionary of an endangered language has always been challenging. 
I am glad that amidst all challenges you managed to write such a voluminous book which will go a long way towards developing and revitalising our cherished language. I also thank all the men and women particularly from the Bunyala kingdom for welcoming the project and working closely with the research team to ensure that the project succeeds. Last but not least, I thank the Volkswagen Foundation for funding this project.\bigskip

\noindent (Rtd.\,Maj.\,Baker Kimeze Mpagi II Byarufu)
\noindent (HRH Obukama bwa Bunyala)
