\chapter{Dictionary user’s guide}\label{sec-dictionary-users-guide}

\largerpage[2]
\section{The structure of the dictionary}
The dictionary is divided into three parts viz.\,the introduction, dictionary proper and the Ru\-ruu\-li\hyp{}Lu\-nya\-la grammar sketch. 
The introduction on the one hand provides an overview of the Ru\-ruu\-li\hyp{}Lu\-nya\-la language and the Baruuli-Banyala ethnic groups. 
On the other hand, it provides the user’s guide on how to look for entries in the dictionary. It ends with orthography of Ru\-ruu\-li\hyp{}Lu\-nya\-la language. 
The dictionary proper provides about 10,000 dictionary entries. Each entry has linguistic information in addition to the sentence example in Ru\-ruu\-li\hyp{}Lu\-nya\-la and a translation in English. The last section provides a grammar sketch of Ru\-ruu\-li\hyp{}Lu\-nya\-la. 

\section{The structure of the dictionary entries}

By convention, all dictionary entries are entered as roots or stems. 
The root is followed by the citation form of the word in italics. 
This is followed by the grammatical information of each word class. 
The grammatical information is followed by a translation of the entry into English. 
The meaning is presented in the following way: 
if the headword has two or more senses, the senses are numbered with (a), (b), (c), etc., as in Figures~\ref{fig-nouns}. 
In case of homographs, they are treated as discrete entries and are listed as separate headwords. Furthermore, they are numbered with superscript number. 


\subsection{Nouns}

Figure~\ref{fig-nouns} shows a dictionary entry of a noun. 
All headwords are given in bold font. 
No augment is indicated on head\-words. Nouns are alphabetised according to the first letter of the stem and not according to their augments. 
The headword is followed by the augmented form of the noun in italics with the augment in brackets. 
The form with the augment is followed by the indication of the word class (i.e.\,\emph{n.} for noun), as well as the number of the noun class. 
These numbers correspond to the standard noun class numbering employed in the research on Bantu languages. For an overview, see Section~\ref{sec-morh-noun-classes-overview} of the grammar sketch.

\begin{figure}[htb]
\begin{center}
 \includegraphics[width=0.5\linewidth]{figures/dictionary-noun.jpg}
\caption{A dictionary entry of a noun}\label{fig-nouns}
\end{center}
\end{figure}

\noindent Most nouns occur in singular/plural pairings (see Section~\ref{sec-morh-pairings}). 
The headword is given in the singular form, the specification of the noun class pairing for the singular and plural forms is provided, e.g.\,\emph{(e)kaada} `card' is indicated as belonging to class 9/10. 
The bullet $\bullet$ separates the examples from the rest of the entry. 
A Ru\-ruu\-li\hyp{}Lu\-nya\-la example is given in italics, its translation into English follows in a regular font.

The example in Figure~\ref{fig-nouns} also illustrates how 
fixed and semi-fixed phrases are treated. 
They are numbered as further senses and the whole fixed phrase is given in bold font.


\subsection{Pronouns}
Ruruuli-Lunyala personal pronouns have various forms. 
They are discussed in Section~\ref{sec-pronouns}. 
All pronoun which exist as free morphemes are entered as full forms and are arranged alphabetically, e.g.\,\emph{nje} `I, me' in Figure~\ref{fig-pronoun}. 
As the citation form is identical to the headword form, it is not listed separately.

\begin{figure}[htb]
\begin{center}
 \includegraphics[width=0.5\linewidth]{figures/dictionary-pronoun.jpg}
\caption{A dictionary entry of a pronoun}\label{fig-pronoun}
\end{center}
\end{figure}


As with other parts of speech, The bullet $\bullet$ separates the examples from the rest of the entry. An Ru\-ruu\-li\hyp{}Lu\-nya\-la example is given in italics, its translation into English follows in a regular font.


\subsection{Verbs}

The verbs are listed under the first letter of the stem e.g.\,\emph{soma} `to read, to study' is listed under <S> and \emph{galaala} `to be lazy' is listed under <G>, as in Figure~\ref{fig-verb}.
The headword is followed by the citation form, which for verbs is the form with the corresponding augment \emph{o-} and the noun class 15 prefix \emph{ku-}, e.g.\,\emph{(oku)galaala} in Figure~\ref{fig-verb}. 
This information is followed by the abbreviation of the word class (or part of speech, i.e.\,v for verb), as well as by the verb-specific grammatical information about the transitivity, e.g.\,v.tr.\,(for a transitive verb, i.e.\,a verb which can take an object) or v.intr.\,(for an intransitive verb, i.e.\,a verb which cannot take an object). 
Next, the perfective form of the verb is provided in italics, e.g.\,\emph{gulaire} (see Section~\ref{sec-aspect-perfective}). 
The grammatical information is followed by translations, as well as an example formatted in the way already discussed for nouns and pronouns. 

\begin{figure}[htb]
\begin{center}

 \includegraphics[width=0.5\linewidth]{figures/dictionary-verb.jpg}

 \caption{A dictionary entry of a verb}\label{fig-verb}
\end{center}
\end{figure}


\subsection{Adjectives, adverbs, and other parts of speech}

The conventions outlined above apply with some modifications to other parts of speech. 


All adjectives are listed as roots and are accompanied by examples with common agreeing prefixes, as in Figure~\ref{fig-adjective} (see Section~\ref{sec-adjectives}).

\begin{figure}[htb]
\begin{center}

 \includegraphics[width=0.5\linewidth]{figures/dictionary-adjective.jpg}

 \caption{A dictionary entry of an adjective}\label{fig-adjective}
\end{center}
\end{figure}

Most Ru\-ruu\-li\hyp{}Lu\-nya\-la adverbs have only one word-form and do not have any noun class agreement prefixes (see Section~\ref{sec-adverbs}). 
Thus, they are entered as full forms arranged alphabetically by the first letter as illustrated in Figure~\ref{fig-adverb}. 
Some adverbs take an augment, in these cases the augmented citation form follows the word class abbreviation adv.

\begin{figure}[htb]
\begin{center}

 \includegraphics[width=0.5\linewidth]{figures/dictionary-adverb.jpg}

 \caption{A dictionary entry of an adverb}\label{fig-adverb}
\end{center}
\end{figure}

The dictionary also lists a number of affixes, conjunctions and ideophones. 
Their representation follows the conventions – with the necessary adjustments – as applied for the representation of other word classes.

\subsection{Cross-references and dialectal and etymological information}\label{sec-userguide-dialect}

Some entries have further information. 
If a word is used only in Ruruuli or Lunyala or in the varieties spoken in Buyende or
Kiryandongo, this is indicated after the headword. 
Furthermore, many words have different pronunciations and are spelled differently respectively. 
In this case the most common pronunciation is chosen for the entry, other pronunciations are entered and provided with cross-references to the respective detailed entry, as in illustrated with the headword \emph{kaadi} at the bottom of  Figure~\ref{fig-nouns}.

In case of loanwords (i.e.\, words adopted from a different language), the source languages is given in angle brackets, as <from En.> (i.e.\,from English) in Figure~\ref{fig-nouns}.
