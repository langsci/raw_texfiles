\chapter{Phonology and orthography}\label{sec-phonology}

In this section we provide an overview of the segmental inventory of Ru\-ruu\-li\hyp{}Lu\-nya\-la. 
Section~\ref{sec-phonology-consonants} presents consonants. 
A note on tone is provided in Section~\ref{sec-phonology-tone} Section~\ref{sec-phonology-vowels} discusses vowels. 
Syllable structure is discussed in Section~\ref{sec-phonology-syllable}. 
An over\-view of a number of morphophonological processes is given in Section~\ref{sec-morphophonological-process}. 
Finally, Section~\ref{sec-phonology-orthogrpahy} discusses the relationship between phonological representation and orthography.


\section{Consonants}\label{sec-phonology-consonants}

Ruruuli-Lunyala has 21 consonants listed in Table~\ref{tab-consonants} with their corresponding graphemes. 
All voiceless stops and most fricatives have a voiced counterpart. 
All nasals in Ru\-ruu\-li\hyp{}Lu\-nya\-la are voiced. 
A selection of minimal pairs illustrating the phonemic contrasts are provided in~(\ref{ex-phon-consonants}).

\begin{table}
\caption{Consonants and corresponding graphemes (+v voiced, -v voiceless)}
\begin{tabularx}{\textwidth}{l X l X X X l}
\lsptoprule
 &  & Bilabial  & Labio-  & Alveolar  & Palatal  & Velar  \\
  &  &   & dental  &   &   &   \\
\midrule
Plosive  & +v & b <bb> &  & d <d>  & ɟ <j> & g <g> \\
 & -v & p <p> &  & t  <t> & c  <c> & k <k>\\
Fricative  & +v &  β  <b> & v  <v> & z  <z>&  &  \\
 & -v &  & f  <f> & s  <s>&  &  \\
Trill & +v &  &  & r  <r>&  &  \\
Lateral  & +v &  &  & l  <l>&  &  \\
Nasal & +v & m <m>&  & n  <n>& ɲ   <ny>& ŋ  <ŋ>\\
Approximant & +v & w <w>&  &  & j <y> &  \\
\lspbottomrule
\end{tabularx}
\label{tab-consonants}
\end{table}

\newpage
\ea \label{ex-phon-consonants}
\begin{xlist}
	\ex /b/:  \textit{bbuba} /buβa/ `to be possessive’ vs.\, \textit{vuba} /vuβa/ `to fish';  \textit{sobba} /soba/ `to move slowly' vs.\, \textit{soba} /soβa/ `to be wrong'
	\ex /p/:  \textit{paaka} /paːka/ `to praise' vs.\, \textit{bbaaka} /baːka/ `to prepare (millet) for fermentation';  \textit{pika} /pika/ `to pump up’ vs.\, \textit{bika} /βika/ `to announce the death (of a person)' 

	\ex /β/:  \textit{aba} /aβa/  `to go’ vs.\, \textit{abba} /aba/ `to work in exchange'
	\ex /c/:  \textit{tiica} /tiːca/ `to tear’ vs.\, \textit{tiika} /tiːka/ `to load';  \textit{cunda} /cunda/ `to shake' vs.\, \textit{junda} /ɟuːnda/ `to rot'
	\ex /d/:  \textit{duma} /duma/  `to run, rush’ vs.\, \textit{tuma} /tuma/ ‘to send’ vs.\,  \textit{luma}  /luma/ `to dig' 
	\ex /f/:   \textit{mufugi} /mufugi/ ‘ruler’ vs.\, \textit{muvugi}  /muvugi/ ‘driver’
	\ex /g/:   \textit{gaali} /gaːli/ ‘bicycle’ vs.\, \textit{kaali} /kaːli/ ‘crib’  
	\ex /ɟ/:	 \textit{janga} /ɟanga/ `to have fun' vs.\,canga /canga/ `to confuse'
	\ex /k/:	 \textit{kanya} /kaɲa/ `to gather' vs.\, \textit{ganya} /gaɲa/ `to allow'
	\ex /l/:	\label{ex-phono-l}
	 \textit{luma} /luma/  ‘to bite’ vs.\, \textit{duma} /duma/ ‘to run hurriedly’,  \textit{alaalya} /alaːlja/ `to spread out' vs.\, \textit{araalya} /araːlja/ `to scatter',  \textit{ibala} /iβala/ `stain, spot' vs.\, \textit{ibara} /iβara/ `name',  \textit{rubogo} /ruβogo/ `brownish bull or he-goat' vs.\, \textit{lubogo} /luβogo/ `bark cloth',  \textit{kinyolo} /kiɲolo/ `door bolt' vs.\, \textit{Kinyoro} /kiɲoro/ `the Kinyoro way'
	\ex /m/:	  \textit{mala}  /mala/  ‘to finish’ vs.\, \textit{sala} /sala/ ‘to cut’
	\ex /n/:	 \textit{naga}  /naga/  ‘to throw away’ vs.\,  \textit{nyaga} /ɲaga/ `to kidnap' vs.\,  \textit{maga} /maga/ ‘pied kingfisher’
	\ex /ŋ/:  \textit{kiŋaaŋaali} /kiŋaːŋaːli/ ‘jerry can without a top’ vs.\,  \textit{kima\-maa\-li} /kimamaːli/ ‘long lasting (7)'
 \ex /ɲ/:	 \textit{saanya} /saːɲa/ `to swim' vs.\, \textit{saana} /saːna/ `to deserve, be worthy of' 
\ex \label{ex-phono-r}
/r/:	 \textit{mworo} /muoro/  ‘poor person’ vs.\, \textit{mwolo}  /muolo/ ‘scythe’
\ex /s/:	  \textit{siga} /siga/ ‘to plant’ vs.\, \textit{ziga} /ziga/ ‘to trail’
\ex /t/:	  \textit{tena}  /tena/ ‘to be soggy’ vs.\,  \textit{sena}  /sena/ ‘to be fed up’
\ex /v/:	 \textit{vuga}  /vuga/  ‘to drive’ vs.\,/nuga/  \textit{nuga} ‘to be disgusted’
\ex /w/:	 \textit{laawa} /laːwa/ `to castrate' vs.\, \textit{laaya} /laːja/ `to grow tall'
\ex /j/:   \textit{yuga} /juga/  `to mingle'  vs.\,  \textit{juga} /ɟuga/	`to moo when aroused' 
\ex /z/:	  \textit{ziza} /ziza/ ‘to patch’ vs.\,  \textit{ziga}  /ziga/ ‘to trail’
\end{xlist}	
\z

\subsection{The phonemes /β/ and /b/}  This phoneme is realised as [β] in the intervocalic position (also across word boundaries). 
It is realised as [b] word-initially after a pause, as well as after nasals. 
The phoneme /b/ is realised as [b] in all contexts.


\subsection{The phonemes /l/ and /r/} According to our current analysis, /l/ and /r/ are two distinct phonemes in Ruruuli\hyp{}Lunyala and a number of minimal pairs were identified, see some examples in~(\ref{ex-phono-l}) and~(\ref{ex-phono-r}). 
Many of the speakers we worked with are consistent in preserving the contrast. 
Specifically, the contrast is preserved word-initially and after the non-front vowels /a/, /o/, and /u/. 
The distinction is neutralised after the front vowels /i/ and /e/: 
The underlying /l/ is realised as [r] after /e/ and /i/ and this alternation is reflected in the orthography (see Section~\ref{sec-orthography-lr}).

Furthermore, we observe dialectal variation in the realisation of these phonemes, which is reflected in the dictionary. 
Specifically, a larger number of lexical items have [r] in Ruruuli and [l] in Lunyala. In the dictionary we used the Ruruuli spelling for headwords, Lunyala variants are provided and cross-referenced with their Ruruuli counterparts. 

\subsection{The phoneme /ŋ/}
The phoneme /ŋ/ is infrequent in Ru\-ruu\-li\hyp{}Lu\-nya\-la. 
It is attested in a small number of nouns and verbs listed in~(\ref{ex-phono-ŋ}), as well as in a number of ideophonic words listed in~(\ref{ex-phono-ŋ-ideo}).

\ea \label{ex-phono-ŋ}
\begin{tabbing}
	 \textit{xxxxxxxxxxxxx}  \=`xxxxxxxxxx (15)'\kill
	 \textit{eŋaala} \>`to sit with legs wide apart'\\
	 \textit{eŋaŋa}  \>`to become furious, growl'\\
	 \textit{(e)kiŋaaŋaali} \> `jerry can or pot without a top (7)'\\
	 \textit{(e)kiriŋaliŋa} \> `snail (7)'\\
	 \textit{(a)marakwaŋa} \>`canna lily (6)'\\
	 \textit{(o)Mulaŋo} \>`Langi person (1)'\\
	 \textit{(o)Nakalaŋo} \> `Langi woman (1)'\\
	 \textit{(o)namuŋoona}  \> `pied crow (1)'\\
	 \textit{(e)ŋŋombe} \> `horn (9)'
\end{tabbing}	
\z

\ea \label{ex-phono-ŋ-ideo}
\begin{tabbing}
	 \textit{xxxxxxxxxxxxx}  \=`xxxxxxxxxx (15)'\kill	
	 \textit{ŋŋa} \> `cry of a baby' (ideo.)\\
	 \textit{ŋaŋala}  \> `to yelp'\\
	 \textit{ŋoŋoolika}  \> `to speak through one's nose'\\
	 \textit{ŋooŋoola}  \> `to speak through one's nose'\\
	 \textit{ŋooŋa}   \>`(of flies) to buzz'\\
	 \textit{ŋooŋa}  \>`to moo'
\end{tabbing}	
\z
 
\subsection{Geminate consonants}\label{sec-phonology-geminate}
Unlike in some neighbouring languages, such as Luganda, where most singletons contrast with geminates (see e.g.\,\citealt{Ashtonetal1954Luganda} and \citealt{Hymanetal1999Syllable}), there are no underlying geminates in Ru\-ruu\-li\hyp{}Lu\-nya\-la.  
Surface gemination emerges in several situations. 
First, surface gemination emerges from the deletion of the infinitive prefix  \textit{ku- } in rapid speech of many Ru\-ruu\-li\hyp{}Lu\-nya\-la speakers. 
When the prefix is deleted, it is compensated for by lengthening the root initial consonant, as in~(\ref{ex-phono-geminates-ku}).
 
\ea \label{ex-phono-geminates-ku}
\begin{xlist}	
	\ex  \textit{ku-canga} `to confuse': /kucanga/:  [kucaːnga] or [cːaːnga]
	\ex  \textit{ku-deedya} `to swing': /kudeːdja/:  [kudeːdja] or [dːeːdja]
	\ex  \textit{ku-kola} `to do': /kukola/:  [kukola] or [kːola]
	\ex  \textit{ku-naaba} `to bathe': /kunaːβa/:  [kunaːβa] or [nːaːβa]
	\ex  \textit{ku-zaawa} `to perish': /kuzaːwa/:  [kuzaːwa] or [zːaːwa]
\end{xlist}	
\z

Surface geminate consonants also occur as a result of prefixation of the first person singular prefix  \textit{n-} to verb stems beginning with a nasal, as in~(\ref{ex-phono-geminates-nn}).

\ea \label{ex-phono-geminates-nn}
\begin{xlist}	
	\ex \label{ex-phono-geminates-nn-nounA}
	 \textit{annenere} `\textsc{1sbj}-\textsc{1sg.obj}-bite-\textsc{pfv}': /a-n-nen-ire/ [anːenere]
	\ex \label{ex-phono-geminates-nn-nounB}
	 \textit{timmaite} `\textsc{neg}-\textsc{1sg.sbj}-know:\textsc{pfv}': /ti-n-maite/ [timːaite]
\end{xlist}	
\z

Furthermore, geminate consonants are preserved in a small number of loanwords, as in~(\ref{ex-phono-geminates-loanwords}).
\ea \label{ex-phono-geminates-loanwords}
\begin{xlist}	
	\ex  \textit{butto} /βutːo/ [butːo] `cooking oil, vegetable oil (1)' (from Luganda)
	\ex  \textit{nvujjo} /nvuɟːo/ [nvuɟːo] `land tax (9)' (from Luganda)
	\ex  \textit{munaddiini} /munadːiːni/ [munadːiːni] `devotee (1)' (from Luganda)
\end{xlist}	
\z

\section{Vowels}\label{sec-phonology-vowels}
Ruruuli-Lunyala has a system of five phonemic vowel qualities. 
All five vowel qualities occur as short and long. 
There are two high vowels /i/, /u/, and two mid vowels /e/, /o/ and one low vowel /a/.  
The phonemic status of the short vowels is illustrated with the minimal pairs in~(\ref{ex-phono-vowels}).

\ea \label{ex-phono-vowels}
\begin{xlist}	
	\ex /a/ vs.\,/e/:  \textit{kama} /kama/ ‘to milk’ 		vs.\, \textit{kema}  /kema/ ‘to tempt’
	\ex /a/ vs.\,/i/:   \textit{bamba} /βamba/ ‘to crucify’ 	vs.\, \textit{bimba} /βimba/ ‘to boil over’
	\ex /a/ vs.\,/o/:   \textit{mukali} /mukali/  ‘woman (1)’ 	vs.\, \textit{mukoli}  /mukoli/ ‘worker (1)’
	\ex /a/ vs.\,/u/:  \textit{ibara} /iβara/  ‘name (5)’		 vs.\, \textit{ibura} /iβura/  ‘scarcity (5)’ 
	\ex /e/ vs.\,/i/:   \textit{mega} /mega/	 ‘to plate’ 		vs.\, \textit{miga} /miga/  ‘to press’ 
	\ex /e/ vs.\,/o/:   \textit{kema} /kema/ ‘to groan’ 		vs.\, \textit{koma} /koma/  ‘to stop'
	\ex /e/ vs.\,/u/:   \textit{gema} /gema/ ‘to vaccinate’ 	vs.\,  \textit{guma} /guma/ ‘to be hard’
	\ex /i/ vs.\,/o/:  \textit{kifi} /kifi/ ‘piece of meat (7)’ 	 vs.\, \textit{kifo} /kifo/  ‘place (7)’
	\ex /i/ vs.\,/u/:  \textit{mibiri} /miβiri/ 	‘bodies (8)’ vs.\, \textit{mubiri} /muβiri/  ‘body (7)’
	\ex /o/ vs.\,/u/:   \textit{isomo} /isomo/ ‘lesson (5)’ 	vs.\, \textit{isumo}  /isumo/ ‘spear (5)’
\end{xlist}	
\z

As was mentioned above, all five vowel qualities occur as short and long. 
Long vowels can be lexically specified, as illustrated in~(\ref{ex-phono-vowels-long}), or can arise through morphophonemic processes, e.g.\,due to the contact between heteromorphemic vowels, see Section~\ref{sec-phonology-vowel-hiatus}. 

\ea \label{ex-phono-vowels-long}
\begin{xlist}	
\ex /a/ vs.\,/aː/\\
 \textit{sama} /sama/  ‘to make an incision’ vs.\, \textit{saama} /saːma/  ‘to process (bark cloth)’\\
 \textit{sala} /sala/  ‘to cut’ vs.\, \textit{saala} /saːla/ ‘(of moslems) to pray’ 

\ex /e/ and /eː/\\
 \textit{sera} /sera/  ‘to practice cannibalism' vs.\, \textit{seera} /seːra/  ‘to sell expensively’\\
  \textit{teba} /teβa/ ‘to turn around’ vs.\, \textit{teeba} /teːβa/  ‘to guess’
		
\ex /i/ vs.\,/iː/\\
 \textit{pika} /pika/  ‘to pump’ vs.\, \textit{piika} /piːka/ ‘to demand instantly’\\
  \textit{bika} /βika/  ‘to announce the death’ vs.\,  \textit{biika} /βiːka/ ‘to store’

\ex /o/ vs.\,/oː/\\
  \textit{doma} /doma/ ‘to be stupid’ vs.\, \textit{dooma} /doːma/  ‘to cause’\\
 \textit{goda} /goda/  ‘to bend’ vs.\,  \textit{gooda}  /goːda/ ‘to sit in a circle’

\ex /u/ vs.\,/uː/\\	
 \textit{tuma}  /tuma/  ‘to send’ vs.\, \textit{tuuma} /tuːma/  ‘to heap’\\
 \textit{sula} /sula/  ‘to spend a night’ vs.\, \textit{suula}  /suːla/  ‘chapter (9)’
\end{xlist}	
\z

As the examples above illustrate, vowel length is contrastive in Ru\-ruu\-li\hyp{}Lu\-nya\-la. However, this contrast is not  available in all contexts. 
Specifically, before word-internal nasal + obstruent sequences all vowels are realised long (see Section~\ref{sec-phonology-vowel-FVShortening}, see \citealt{Hyman1992Moraic, Downing2005Ambiguous} for the discussion of this phenomenon in other Bantu languages).

Ruruuli-Lunyala also has three vowel sequences viz.\,/ai/, /ei/, and /oi/.\footnote{A small number of lexical items listed in the dictionary are spelled with <nia>, <nio>, and <nie>. These are not cases of further vowel sequences but a variation in the realisation of the nasal for which we currently do not have a conclusive analysis.} 
These sequences can occur word initially, medially or in the final position. 
They can be tautosyllabic (i.e.\,diphthongs), as in~(\ref{ex-phono-tautosyllabic-sequences}), or heterosyllabic, as in (\ref{ex-phono-heterosyllabic-sequences}). 
Note that the vowel sequences cannot be tautosyllabic in the word initial position. 

\ea \label{ex-phono-tautosyllabic-sequences}
\begin{tabbing}
	 \textit{xxxxxxxxxxxxx} \= /xxxxxxxxxxxxxxxxxxx/ \=`xxxxxxxxxx (15)'\kill
	 \textit{omusaiza} \> /omusaiza/ [o.mu.sai.za] \> `man (1)'\\
	 \textit{geita}  \> /geita/ [gei.ta]  \>`to backbite'\\
	 \textit{goita}  \> /goita/ [goi.ta]  \>`to pound'\\
	 \textit{matai} \> /matai/[ma.tai]  \>`milk (6)'\\ 	
	 \textit{kimwei} \> /kimwei/ [ci.mwei]  \>`one'\\
\end{tabbing}
\z

\ea \label{ex-phono-heterosyllabic-sequences}
\begin{tabbing}
	 \textit{xxxxxxxxxxxxx} \= /xxxxxxxxxxxxxxxxxxx/ \=`xxxxxxxxxx (15)'\kill	
	 \textit{eisomero} \> /eisomero/ [e.i.so.me.ro] \>`\textsc{aug} + school (5)'\\
	 \textit{eibbaale}\> /eibaːle/ [e.i.baː.le]\> `\textsc{aug} + stone (5)'\\
	 \textit{maizi} \>/maizi/ [ma.i.zi] \>`water (6)'\\
	 \textit{bairu}\>/βairu/ [ba.i.ru] \>`slave (2)'\\ 
	 \textit{bbaire} \>/bbaire/ [ba.i.re] \>`to be (\textsc{pfv})'
\end{tabbing}
\z

\section{Phonological segments in loanwords}  

A number of phonological segment not attested as phonemes in Ruuli are occasionally preserved in loanwords. Due to their marginal status, they are not listed in Table~\ref{tab-consonants} and we refer to them as phonological segments and not as phonemes here. 

The phonological segment /h/ appears in a handful of loanwords, e.g.\, \textit{mahadi}  ‘disagreement’ (a borrowing from Runyoro) and  \textit{swahabba} [swahaba] `friend' (from Arabic). 
In the speech of some speakers, it undergoes phonological adaptation and is either realised as [w] or elided, e.g.\, \textit{mahadi} ‘disagreement’ can be realised as [mawadi] or [maadi].  
In such cases, all forms are entered in the dictionary, i.e.\, \textit{mahadi},  \textit{mawadi}, and  \textit{maadi}. 
With some words, the phonological segment  /h/ is retained in the speech of all speakers, as in  \textit{swahabba} [swahaba] `friend' and thus only this form is entered in the dictionary. 
The word  \textit{Allah} is conventionally written with <h> but realised as [alaː].

The phonological segment /ʃ/ is attested in several loanwords and is represented as <sh> in the orthography, as e.g.\,in  \textit{gaalimooshi} `train' and  \textit{shariya} `sharia'. 
The phonological segment /tʃ/ is attested in several loanword and is represented as <ch> in the orthography, as e.g.\, \textit{muchwezi} `demigods who occupied the Bunyoro kingdom'.

\section{A note on tone}\label{sec-phonology-tone}

Ruruuli-Lunyala is a tonal language: tone is contrastive, i.e.\,it plays a role in distinguishing lexical items. 
A comprehensive study of the Ru\-ruu\-li\hyp{}Lu\-nya\-la tone system is underway. 
The tone bearing unit is the mora (see also \citealt{deBrito2019Tonology}).

Preliminary findings suggest that Ru\-ruu\-li-Lu\-nya\-la has H, L and Ø tonal values whereby H and L are phonologically activated. 
Based on the three tone values, several surface tones are realised, they include: HL e.g.\, \textit{kifi} [kífì] ‘piece of meat (7)’, falling contour LH͡LL, e.g.\, \textit{musale} [mùsáàlè] ‘tree (3)’, downstep HH{\small \textsuperscript{↓}}H{\small \textsuperscript{↓}}HL, e.g.\, \textit{busigazi} [óβú{\small \textsuperscript{↓}}sí{\small \textsuperscript{↓}}gázì] ‘teenage years (14)’, downdrift HHL{\small \textsuperscript{↓}}HL, e.g.\, ómúlà{\small \textsuperscript{↑}}múlì ‘judge’, and raised low HHH{\small \textsuperscript{↑}}L{\small \textsuperscript{↓}}HL tone, e.g.\, éíbá{\small \textsuperscript{↑}}ì{\small \textsuperscript{↓}}zírò ‘carpenter’s workshop’. 
The falling contour tone is realised on either a long vowel, e.g.\,[mùsáàlè] ‘tree’ or a vowel followed by a syllabic nasal, e.g.\,[kìbáǹja] ‘untitled piece of land’.


\section{Syllable structure}\label{sec-phonology-syllable}

The following syllable types are attested in Ru\-ruu\-li\hyp{}Lu\-nya\-la V, CV, CVV, N, CGV, 
% Is CGGV possivle, some pFV anjwire?
NCV, NCVV, NCGV, CCVV, NGV, and in rare cases CCV. 
The variety of syllable structures in Ru\-ruu\-li\hyp{}Lu\-nya\-la is illustrated in~(\ref{ex-phono-syllable}), with the relevant syllables given in bold font. 

\ea \label{ex-phono-syllable}
\begin{xlist}
\ex V:  \textit{abasaiza} [\textbf{a}.βa.sai.za] ‘man (2)’,  \textit{ani} [\textbf{a}.ni] `here' \label{ex-phono-syllable-V}
\ex CV:  \textit{luma} [\textbf{lu.ma}] ‘to bite’,  \textit{mukira} [\textbf{mu.ki.ra}] ‘tail (3)’
\ex CVV:  \textit{munaala} [mu.\textbf{naː.}la] ‘tower (3)’,  \textit{teeba} [\textbf{teː.}βa] ‘to guess’,  \textit{matai} [ma.\textbf{tai}] ‘milk (6)’
\ex N:  \textit{mbuli} [\textbf{m.}bu.li] ‘goat (1)’,  \textit{nte} [\textbf{n}.te] ‘cow (1)’\label{ex-phono-syllable-N}
\ex CGV:  \textit{mulekwa} [mu.le.\textbf{kwa}] ‘orphan (1)’,  \textit{amya} [a.\textbf{mja}] ‘to ask formally’
\ex NCV:  \textit{onka} [o.\textbf{ŋka}] ‘to suck (milk)',  \textit{onte} [o.\textbf{nte}] ‘AUG + cow (1)’,  \textit{empaka} [eː.\textbf{mpa.}ka] ‘AUG + competition (9)’,  \textit{ansabire} [a.\textbf{nsa.}βi.re] ‘he has asked me’
\ex NCVV:  \textit{condooza} [co.\textbf{ndoː.}za] `to investigate',\\ \textit{kibunguura}   [ki.bu.\textbf{ŋguː.}ra] `caterpillar (7)'
\ex NCGV:  \textit{onkya} [o.\textbf{ŋkja}] `to breastfeed`,  \textit{sumantya} [su.ma.\textbf{ntja}] `to smoothen'
% The CCV realization is phonetic and not part of the phonology and thus not part of the underlying syllable structure
\ex CCV:  \textit{kusoma} [\textbf{sːo.}ma]\footnote{The alternative pronunciation is [ku.so.ma], see Section~\ref{sec-phonology-geminate}.} `to read',  \textit{kukola} [\textbf{kːo.}la]\footnote{The alternative pronunciation is [ku.ko.la], see Section~\ref{sec-phonology-geminate}.} `to do’,  \textit{mmumaite} [\textbf{mːu.}mai.te] ‘I know him’,  \textit{annenere} [a.\textbf{nːe.}ne.re] ‘he has bitten me’ 
\end{xlist}	
\z

The most common syllable type is the CV type. 
This syllable type occurs in all word positions viz.\,word initially, medially and finally. 
The second most frequent syllable type is the NCV type. 
Single vowel syllables occur only word initially.
A nasal that is immediately followed by an obstruent is analysed as a prenasalised consonant [NC] and as a part of the onset except in word-initial position, when the nasal is syllabic (see \citealt{deBrito2019Tonology}). 

Ruruuli-Lunyala noun stems can have a length of up to four syllables. 
The majority of noun stems are monosyllabic or disyllabic, though there are a number of trisyllabic noun stems and a rarer number of tetrasyllabic noun stems. 
Monosyllabic nouns and nouns with greater than five syllables are rare. 

Most common underived verb root structure is disyllabic and trisyllabic. 
Furthermore, many verbal stems in the dictionary have lexicalised extensions and are thus longer (trisyllabic or tetrasyllabic). 
As other Bantu languages, Ruruuli-Lunyala has about a dozen monosyllabic verb roots with the underlying CV structure, such as  \textit{sya} /sia/ `to grind',  \textit{gwa} /gua/ `to fall',  \textit{zwa} /zua/ `to go away', and  \textit{fa} /fua/ `to die'.


\section{Phonological and morpho-phonological processes}\label{sec-morphophonological-process}

The processes described in this section are primarily morpho-pho\-no\-lo\-gi\-cal, since they occur when certain morphemes are combined into words. 
In the following sections we will consider some of the most common processes. 
The overview below is not exhaustive and a range of further processes can be covered only in a full-length reference grammar. 
We first consider processes which affect vowels. 
These are vowel height harmony discussed in Section~\ref{sec-phonology-vowel-harmony}, a range of processes involved in hiatus resolution discussed in Section~\ref{sec-phonology-vowel-hiatus}, as well as some other morpho-phonological processes addressed in Section~\ref{sec-phonology-vowel-elision}. 
We then proceed with processes affecting consonants and discuss palatalisation (Section~\ref{sec-phonology-palatisalition}), fortition (Section~\ref{sec-phonology-fortition}), as well as nasal assimilation (Section~\ref{sec-phonology-nasal-assimilation}).

\subsection{Vowel height harmony}\label{sec-phonology-vowel-harmony}

Vowel harmony involves agreement between all vowels within a certain phonological domain (typically a prosodic word or a morphologically defined domain) in some phonological property, such as height or tongue root position (see e.g.\,\citealt{vanderHulstetal1995Vowel}). 
This agreement is evidenced in allomorphic alternations when morphemes are combined into morphologically complex words.
Vowel height harmony is a widespread characteristic of Bantu phonology (see e.g.\,\citealt{Hyman1999Historical}, \citeyear{Hyman2019Segmental}; \citealt{Odden2015Bantu}; \citealt[70]{Downingetal2017Phonology}; \citealt{Lionnetetal2018Languages}). 
Bantu vowel height harmony is characterised by a restriction imposed on the vowels composing the stem, such that peripheral vowels /i/, /u/, and /a/ combine only with other peripheral vowels, while the non-peripheral vowels /e/ and /o/ combine with other mid vowels. 
The root initial vowel determines the harmony for the rest of the stem (excluding the final vowel on verbs). 

Ruruuli-Lunyala roots obey the vowel height harmony (see Part~\ref{sec-Dictionary} Dictionary). 
Furthermore, in Ru\-ruu\-li\hyp{}Lu\-nya\-la the process of vowel height harmony operates in several morphological contexts determining the quality of the vowels in certain affixes.  
We first discuss the effects of vowel height harmony on the augments of nouns, as well as other hosts. We then proceed with a description of how vowel height harmony conditions the realisation of a range of verbal suffixes. 

The augment  \textit{o-} harmonises with /u/ in the noun class prefixes of classes 1, 3, 11, 12, 13, 14, 15, and 20, as in~(\ref{ex-vowel-harmony-augmentO}), while the augment  \textit{e-} harmonises with /i/ of the noun class prefixes of classes 4, 5, 7, and 8, as in~(\ref{ex-vowel-harmony-augmentE}), see Section~\ref{sec-morh-noun-classes-overview} for an overview of nominal morphology.
With the noun classes 2, 6, 12, and 22, the augment is realised as  \textit{a-}, as illustrated in~(\ref{ex-vowel-harmony-augmentA}). 

\ea \label{ex-vowel-harmony-augment}
\begin{xlist}

\ex \label{ex-vowel-harmony-augmentO}
 Augment  \textit{o-}
 \begin{tabbing}
	 \textit{xxxxxxxxxxxxx} \= /xxxxxxxxxxxxx/ \=`xxxxxxxxxx (15)'\kill
	 \textit{o-mu-leri} \> /omuleri/ \> `babysitter (1)'\\
	 \textit{o-mu-saale} \> /omusaːle/ \>  `tree (3)'\\
	 \textit{o-lu-baju} \>/oluβaɟu/ \>`side (11)'\\
	 \textit{o-tu-ceere}\> /otuceːre/ \>`little rice (12)'\\
	 \textit{o-bu-cwa} \>/oβucwa/ \>`poison (14)'\\
	 \textit{o-ku-dukana} \> /okudukana/ \>`diarrhoea (15)'\\
	 \textit{o-gu-solo} \>/ogusolo/ \>`large/huge/ugly animal (20)'	
 \end{tabbing}
\ex \label{ex-vowel-harmony-augmentE}
Augment  \textit{e-}
 \begin{tabbing}
	 \textit{xxxxxxxxxxxxx} \= /xxxxxxxxxxxxx/ \=`xxxxxxxxxx (15)'\kill
	 \textit{e-mi-saale} \> /emisaːle/ \> `tree (4)'\\
	 \textit{e-i-bbaale} \>/eibaːle/ \>`stone (5)'\\
	 \textit{e-ki-mole} \> /ekimole/ \>`traditional lantern (7)'\\
	 \textit{e-bi-saale} \>/eβisaːle/ \>`tree (8)'
\end{tabbing}	

\ex \label{ex-vowel-harmony-augmentA}
Augment  \textit{a-}
\begin{tabbing}
	 \textit{xxxxxxxxxxxxx} \= /xxxxxxxxxxxxx/ \=`xxxxxxxxxx (15)'\kill
 \textit{a-ba-ana} \> /aβaːna/  \>`children (2)'\\
	 \textit{a-ma-bbaale} \> /amabaːle/  \>`stone (6)'\\
	 \textit{a-kaali}  \>/akaːli/  \>`crib (12)'\\
	 \textit{a-ga-itimba}  \>/agaitimba/  \> ‘large  pythons (22)’
\end{tabbing}	
\end{xlist}
\z

The discussion above does not mention the noun classes 9 and 10, which take the augment  \textit{e-}. 
\citet[265–266]{Katamba1984Nonlinear} discusses the similar distribution of augments in Luganda and observes that with nouns of the noun classes 9 and 10 the vowel harmony rule was probably morphologically conditioned from the start and was never phonologically conditioned, as no Bantu language is known to have a front vowel in that class prefix and no convincing evidence has been provided for an alternative view. 

In addition to the augments discussed above, a range of verbal suffixes harmonise with the root in Ru\-ruu\-li-Lunyala. 
They are the perfective suffix \textit{-ire/-ere}, as well as several extensions (see Section~\ref{sec-morph-verb-extensions}), namely, the applicative \textit{-ir/-er}, causative \textit{-isy/-esy}, passive \mbox{\textit{-ibw}/}\mbox{\textit{-ebw},} and stative \textit{-ik/-ek}. 
The respective suffixes surface with the vowel [i] when preceded by /i, u, a/ in the verb root, but with [e] when preceded by /e/ or /o/. 
These suffixes are discussed individually below.

The two vowel height-conditioned allomorphs of the perfective suffix are \textit{-ire} [ire] and \textit{-ere} [ere]. They are illustrated in~(\ref{ex-phonolo-harmony-perfective}). 
Further allomorphs of this suffix are discussed in Section~\ref{sec-aspect-perfective}.

\ea \label{ex-phonolo-harmony-perfective}
Perfective \textit{-ire} [ire] vs.\,\textit{-ere} [ere]
\begin{xlist}

\ex The perfective allomorph \textit{-ire} [ire]
\begin{tabbing}
 \textit{xxxxxxxxxxxxx} \= /xxxxxxxxxxxxx/ \=`xxxxxxxxxx (15)'\kill
	  \textit{siig-a} \> `to smear':	\>  \textit{siig-ire}\\
	  \textit{sumb-a}\> 	`to cook':	\>  \textit{sumb-ire}\\
	  \textit{tak-a}	\>  `to want':	\>   \textit{tak-ire}\\
	  \textit{tamb-a}	\>  `to medicate':\> 	  \textit{tamb-ire}\\
\end{tabbing}	

\ex	The perfective allomorph \textit{-ere} [ere]
\begin{tabbing}
 	 \textit{xxxxxxxxxxxxx} \= /xxxxxxxxxxxxx/ \=`xxxxxxxxxx (15)'\kill
	 \textit{et-a} \> `to call':	\> \textit{et-ere}\\
	 \textit{kol-a} \>`to do': \> \textit{kol-ere}\\
	 \textit{meg-a} \>`to serve':\>   \textit{meg-ere}\\
	 \textit{som-a} \>`to read': \> \textit{som-ere}
	\end{tabbing}	
\end{xlist}
\z

The examples in~(\ref{ex-phonolo-harmony-applicative}) illustrate the distribution of the two applicative allomorphs \textit{-ir} [ir] and \textit{-er} [er]. Section~\ref{sec-extension-applicative} discusses further allomorphs, whereas the functions of the applicative are addressed in Section~\ref{sec-applicative}.

\ea \label{ex-phonolo-harmony-applicative}
Applicative \textit{-ir} [ir] vs.\,\textit{-er} [er]
\begin{xlist}
\ex The applicative allomorph \textit{-ir} [ir]
\begin{tabbing}
 \textit{xxxxxxxxxxxxx} \= /xxxxxxxxxxxxx/ \=`xxxxxxxxxx (15)'\kill
	 \textit{siig-a}	\>`to smear':	\> \textit{siig-ir-a}\\
	 \textit{sumb-a}\>	`to cook':\>	 \textit{sumb-ir-a}\\
	 \textit{tak-a}	\> `to want':	\>  \textit{tak-ir-a}\\
	 \textit{tamb-a}	\> `to medicate':\>	  \textit{tamb-ir-a} 
\end{tabbing}	

\ex The applicative allomorph \textit{-er} [er]
\begin{tabbing}
 \textit{xxxxxxxxxxxxx} \= /xxxxxxxxxxxxx/ \=`xxxxxxxxxx (15)'\kill
	 \textit{et-a}	\>`to call':	\> \textit{et-er-a}\\
	 \textit{kol-a} \>`to do': \> \textit{kol-ere-a}\\
	 \textit{meg-a} \>`to serve': \>  \textit{meg-er-a}\\
	 \textit{som-a} \>`to read':\>  \textit{som-er-a}
\end{tabbing}	
\end{xlist}
\z

The examples in~(\ref{ex-phonolo-harmony-passive}) illustrate the distribution of the two passive allomorphs \textit{-ibw} [ibw] and \textit{-ebw} [ebw]. 
For further details on the passive see Sections~\ref{sec-extension-passive} and \ref{sec-passive}.

\ea Passive \textit{-ibw} [ibw] vs.\,\textit{-ebw} [ebw] \label{ex-phonolo-harmony-passive}
\begin{xlist}

\ex The passive allomorph  \textit{-ibw} [ibw]\label{ex-phonolo-harmony-passive-ibw}
\begin{tabbing}
 	 \textit{xxxxxxxxxxxxx} \= /xxxxxxxxxxxxx/ \=`xxxxxxxxxx (15)'\kill
	 \textit{siig-a}  \>`to smear': \>  \textit{siig-ibw-a}\\
	 \textit{sumb-a} \> `to cook':  \> \textit{sumb-ibw-a}\\
	 \textit{tak-a}  \>`to want':  \>  \textit{tak-ibw-a}\\
	 \textit{tamb-a}  \>`to medicate':  \>  \textit{tamb-ibw-a}
\end{tabbing}	

\ex The passive allomorph \textit{-ebw} [ebw]\label{ex-phonolo-harmony-passive-ebw}
\begin{tabbing}
 	 \textit{xxxxxxxxxxxxx} \= /xxxxxxxxxxxxx/ \=`xxxxxxxxxx (15)'\kill
	 \textit{et-a} \>`to call': \> \textit{et-ebw-a}\\
	 \textit{kol-a}\> `to do': \> \textit{kol-ebw-a}\\
	 \textit{meg-a} \>`to serve': \>  \textit{meg-ebw-a}\\
	 \textit{som-a}\> `to read': \> \textit{som-ebw-a}
\end{tabbing}
\end{xlist}
\z

The vowel-height conditioned allomorphy of the  causative suffix is illustrated in~(\ref{ex-phonolo-harmony-causative}). 
Further allomorphs, as well as the various functions of the causative suffix are discussed in Sections~\ref{sec-extension-causative} and~\ref{sec-causative} respectively.

\ea Causative \textit{-isy} [isj] vs.\,\textit{-esj} [esj] \label{ex-phonolo-harmony-causative}
\begin{xlist}

\ex The causative allomorph \textit{-isy} [isj] \label{ex-phonolo-harmony-causative-isy}
\begin{tabbing}
 \textit{xxxxxxxxxxxxx} \= /xxxxxxxxxxxxx/ \=`xxxxxxxxxx (15)'\kill
	 \textit{siig-a} \>`to smear': \> \textit{siig-isy-a}\\
	 \textit{sumb-a} \>`to cook':\>  \textit{sumb-isy-a}\\
	 \textit{tak-a}\> `to want': \>  \textit{tak-isy-a}\\
	 \textit{tamb-a} \>`to medicate':\>   \textit{tamb-isy-a}
\end{tabbing}

\ex The causative allomorph \textit{-esj} [esj] \label{ex-phonolo-harmony-causative-esy}
\begin{tabbing}
	 \textit{xxxxxxxxxxxxx} \= /xxxxxxxxxxxxx/ \=`xxxxxxxxxx (15)'\kill
	 \textit{et-a} \>`to call':\>  \textit{et-esy-a}\\
	 \textit{kol-a} \>`to do':\> \textit{kol-esy-a}\\
	 \textit{meg-a} \>`to serve': \>  \textit{meg-esy-a}\\
	 \textit{som-a}\> `to read':\>  \textit{som-esy-a}
\end{tabbing}
\end{xlist}
\z

\subsection{Hiatus resolution}\label{sec-phonology-vowel-hiatus}

It is very common in Bantu languages for two vowels belonging to different morphemes to enter into direct contact and thus produce sequences of vowels. 
The term  \textit{vowel hiatus} is commonly used to refer to a sequence of adjacent vowels belonging to separate syllables \citep{Casali2011Hiatus}. 
Many languages have restrictions as to the contexts in which heterosyllabic vowel sequences can occur, while some languages eliminate them completely at the surface.  
Ruruuli-Lunyala is one of the languages which disallows V1.V2 sequences, and a range of repair strategies are available. 
They include glide formation (Section~\ref{sec-phonology-glide-formation}), vowel lengthening (Section~\ref{sec-phonology-vowel-assimilation}), and vowel coalescence (Section~\ref{sec-phonology-vowel-coalescence}).

\subsubsection{Glide formation}\label{sec-phonology-glide-formation}

When the vowel /i/, /u/ and /o/ are followed by certain vowels they are realised as glides. 
Furthermore, when the first vowel is preceded by a consonant, the second vowel undergoes compensatory lengthening word medially (see Section~\ref{sec-phonology-vowel-FVShortening} on word-final vowel shortening). 
The front vowel /i/ is realised as the palatal glide [j] when it precedes /a/, /e/, and /o/. 
The back vowels /u/ and /o/ are realised as the labial-velar glide [w] before /i/, /o/, /e/, and /a/.

Examples of glide formation in nouns resulting from the vowel hiatus between a prefix and a vowel-initial stem are provided in~(\ref{sec-phonology-glide-formation-nouns}).

\ea \label{sec-phonology-glide-formation-nouns}
\begin{xlist}	
	\ex 	/mu-ana/ →  \textit{mwana} [mwaːna] `child (1)'
	\ex 	/mu-oɟo/ →  \textit{mwojo} [mwoːɟo] `boy (1)'
	\ex 	/mu-iga/ →  \textit{mwiga} [mwiːga] `river (3)'
	\ex 	/mi-ezi/ →  \textit{myezi}  [mjeːzi] `month (4)' 
	\ex 	/ki-ole/ →  \textit{kyole} [coːle] `cave (7)'\footnote{In case of glide formation with the prefix  \textit{ki-}, the palatalisation rule applies to the output of the glide formation and results in surface forms without glides (see Section~\ref{sec-phonology-palatisalition}).}
	\ex 	/βu-ire/ →  \textit{bwire} [bwiːre] `time (14)'
\end{xlist}	
\z

Hiatus in verbs is common and occurs in a range of morphological contexts. 
The V1 is occupied by the vowel of one of the following prefixes:  
the subject prefixes with the vowels /u/ and /o/, such as  \textit{tu-} `1\textsc{pl.sbj}' (\ref{sec-phonology-glide-formation-verbs-TUEE}), 
 \textit{o-} `2\textsc{sg.sbj}' (\ref{sec-phonology-glide-formation-verbs-OW}), 
 \textit{mu-} `2\textsc{pl.sbj}',  \textit{gu-} `3\textsc{sbj}', etc., 
as well as the subject prefixes with the vowels /i/ and /e/, such as  \textit{ki-} `7\textsc{sbj}' and  \textit{e-} `9\textsc{sbj}' \,(see Table \ref{tab-verb-indexes} in Section~\ref{sec-verb-indexing}). 

The V2 can be the past prefix  \textit{a-}, as in (\ref{sec-phonology-glide-formation-verbs-TWA})–(\ref{sec-phonology-glide-formation-verbs-OW}), 
the future prefix  \textit{a-}, and the reflexive prefix  \textit{ee-}, as in~(\ref{sec-phonology-glide-formation-verbs-EE})–(\ref{sec-phonology-glide-formation-verbs-TUEE}).
If the first prefix has the shape CV, there is a compensatory vowel lengthening, 
but if the first prefix has the shape V, as e.g.\,in case of the prefix  \textit{o-} `2\textsc{sg.sbj}', there is no compensatory vowel lengthening. 
Note that the reflexive prefix  \textit{ee-} is itself long and remains long after the glide formation.

\ea \label{sec-phonology-glide-formation-verbs}
\begin{xlist}	
	\ex 	 \textit{tu-} `2\textsc{sg.sbj}' +  \textit{a-} `\textsc{pst}':\\
	/tu-a-koβ-a/\\→  \textit{twakoba}  [twaːkoβa] `we say' 
	\label{sec-phonology-glide-formation-verbs-TWA}
	\ex 	 \textit{o-} `2\textsc{sg.sbj}' +  \textit{a-} `\textsc{pst}':\\
	/o-a-koβ-a/\\→  \textit{wakoba}  [wakoβa] `you said' 
	\label{sec-phonology-glide-formation-verbs-OW}
	\ex 	 \textit{o-} `2\textsc{sg.sbj}' +  \textit{ee-} `\textsc{refl}':\\
	/o-eː-tongan-a/\\→  \textit{weetongana}  [weːtoːŋgana] `you defend yourself' 
	\label{sec-phonology-glide-formation-verbs-EE}
	\ex 	 \textit{a-} `1S' +  \textit{ee-} `\textsc{refl}':\\
	/a-eː-king-a/\\→  \textit{yeekinga}  [jeːkiːŋga] `he covers himself' 
	\label{sec-phonology-glide-formation-verbs-AEE}
	\ex 	 \textit{tu-} `1plS' +  \textit{ee-} `\textsc{refl}':\\
	/tu-ee-yimb-a/\\→  \textit{tweyimba}  [tweːjiːmba] `we unite ourselves' 
	\label{sec-phonology-glide-formation-verbs-TUEE}
	\end{xlist}	
\z

Further contexts of vowel hiatus on the verb are created when the vowel-initial stem is preceded by a prefix with a vowel. 
The relevant prefixes are the ones already discussed above, as well as the object prefixes, and some TAM prefixes, such as the progressive  \textit{ku-}. 
A few examples are given in~(\ref{sec-phonology-glide-formation-verbs2}).

\ea \label{sec-phonology-glide-formation-verbs2}
\begin{xlist}
	
	\ex 	object prefix + a vowel-initial stem:\\/tu-lu-aβj-a/ (1\textsc{pl.sbj}-1\textsc{1obj}-perform.last.funeral.rites-\textsc{fv})\\→  \textit{tulwabya}  [tulwaːβja] `we perform them (the last funeral rites)' 	\label{sec-phonology-glide-formation-verbs-OBJA}
	\ex 	 \textit{ku-} `\textsc{inf}' + a vowel-initial stem:\\/ku-aβj-a/ (\textsc{inf}-perform.last.funeral.rites-\textsc{fv})\\→  \textit{kwabya}  [kwaːβja] `we performed them (the last funeral rites)' 	\label{sec-phonology-glide-formation-verbs-OBJB}
	
	\ex 	 \textit{ku-} `\textsc{prog}' + a vowel-initial stem:\\
	/gu-ku-ak-a/ (\textsc{3sbj}-\textsc{prog}-burn-\textsc{fv})\\→  \textit{gukwaka} [gukwaːka] `it (3) is burning' 
	\label{sec-phonology-glide-formation-verbs-KU}
	\end{xlist}	
\z

Glide formation is also possible when suffixes attach to a monosyllabic verb stem, as in~(\ref{sec-phonology-glide-formation-suffix-monosyllabic}), as well as when vocalic suffixes, such as the causative \textit{-y} /i/ and the passive \textit{-w} /u/ attach to a verb stem and precede the final vowel \textit{-a}, as in (\ref{sec-phonology-glide-formation-suffix-causative}) (see Section~\ref{sec-morph-verb-extensions}). 
Note that vowel lengthening does not take place word finally, as in~(\ref{sec-phonology-glide-formation-suffix-monosyllabic-a}) and~(\ref{sec-phonology-glide-formation-suffix-causative-a}) (see Section~\ref{sec-phonology-vowel-FVShortening}).

\ea \label{sec-phonology-glide-formation-suffix-monosyllabic}
\begin{xlist}
	
	\ex \label{sec-phonology-glide-formation-suffix-monosyllabic-a}
	/a-li-a/ (\textsc{1sbj}-eat-\textsc{fv}) →  \textit{alya}  [alja] `she/he eats' 
	\ex 	/a-gu-ire/  (\textsc{1sbj}-fall-\textsc{pfv}) →  \textit{agwire}  [agwiːre] `she/he has fallen' 
	\end{xlist}	
\z

\ea \label{sec-phonology-glide-formation-suffix-causative}
\begin{xlist}
	
	\ex \label{sec-phonology-glide-formation-suffix-causative-a}
	/iruk-i-a/ (iruk-\textsc{caus}-\textsc{fv}) →  \textit{irukya} [irukja] ‘to chase, cause to run’
	\ex 	/kol-u-a/  (do-\textsc{pass}-\textsc{pfv}) →  \textit{kolwa}  [kolwa] `to be done' 
	\end{xlist}	
\z

Finally, we follow the analysis by \citet{Hymanetal1999Syllable} proposed for Luganda and assume that glide formation and compensatory lengthening also apply morpheme internally, as in~(\ref{sec-phonology-glide-formation-morphemeinternally}).

\ea \label{sec-phonology-glide-formation-morphemeinternally}
\begin{xlist}
	
	\ex 	/kasiono/ →  \textit{kasyono}  [kasjoːno] ` \textit{Acacia hockii} (12)'
	\ex 	/zualika/  →  \textit{zwalika}  [zwaːlika] `to sacrifice'
	\end{xlist}	
\z

\largerpage
\subsubsection{Vowel elision and compensatory lengthening}\label{sec-phonology-vowel-assimilation}

When a non-high vowel precedes another vowel, the first vowel is deleted and the second one is lengthened. 
When the two vowels are identical, the combination results in the long vowel, as e.g.\,in~(\ref{ex-vowel-lengthening-noun-aa}). 
The common contexts for this process are the concatenation of the noun class prefix with the nominal or verbal root, as in~(\ref{ex-vowel-lengthening-noun}) and~(\ref{ex-vowel-assimilation-verb}) respectively. 

\ea \label{ex-vowel-lengthening-noun}
\begin{xlist}	
	\ex /a-βa-egi/	→	\textit{abeegi}      [aβ\textbf{eː}gi]	‘student (2)’  
	\ex /a-βa-oro/	→	\textit{abooro}    [aβ\textbf{oː}ro] ‘poor man (2)’
       \ex /a-βa-ana/	→	\textit{abaana}      [aβ\textbf{aː}na]	‘child (2)’  \label{ex-vowel-lengthening-noun-aa}
\end{xlist}	
\zlast
\clearpage

\ea \label{ex-vowel-assimilation-verb}
	/βa-et-a/	→	\textit{beeta} 	     [b\textbf{eː}ta]	‘they call’ 
\z

Furthermore, in the same fashion as outlined above vowel elision and the subsequent compensatory lengthening can occur across word boundaries. This is common when a noun is followed by a demonstrative, as in~(\ref{ex-vowel-assimilation-across}).

\ea \label{ex-vowel-assimilation-across}
\begin{xlist}	
	\ex /mwaːla oni/	→ \textit{mwala oni} 	     [mwaːl\textbf{oː}ni]	‘this girl’ 
	\ex /emikisa edi/	→	\textit{emikisa edi} 	     [emikis\textbf{eː}di]	‘these blessings'
\end{xlist}	
\z

Vowel lengthening also takes place when the perfective suffix  \textit{-ire} or the applicative suffix  \textit{-ir} are imbricated with the verbal root, as described below in Sections \ref{sec-aspect-perfective} and~\ref{sec-applicative} and illustrated in~(\ref{ex-vowel-assimilation-imbrication}).

\ea \label{ex-vowel-assimilation-imbrication}
\begin{xlist}	
	\ex /a-mal-ire/	(\textsc{1sbj}-finish-\textsc{pfv})\\→	 \textit{amaare} 	     [am\textbf{aː}re]	‘s/he has finished’\\
		/a-kul-ire/	(\textsc{1sbj}-uproot-\textsc{pfv})\\→	 \textit{akuure} 	     [ak\textbf{uː}re]	‘s/he has uprooted’ 
	\ex  /a-kol-ir-a/ 	(\textsc{1sbj}-work-\textsc{appl}-\textsc{fv})\\→  \textit{akoora}	 [ak\textbf{oː}ra] `s/he works for'
\end{xlist}	
\z

\subsubsection{Vowel coalescence}\label{sec-phonology-vowel-coalescence}

Vowel coalescence or vowel fusion is a process by which two vowels in hiatus merge into a single vowel in a way which can be considered an ‘articulatory compromise’ between the two input segments (see \citealt[84]{deHaas1988Formal}). 
This means that adjoining segments fuse into one element such that the new segment is phonologically distinct from the ones in the input.  
In Ru\-ruu\-li\hyp{}Lu\-nya\-la vowel coalescence occurs only in a very limited context, namely, between the root vowel of monosyllabic verbs and the applicative suffix  \textit{-ir}, as in~(\ref{ex-vowel-coalescence}).

\ea \label{ex-vowel-coalescence}
\begin{xlist}	
	\ex /ta-ir-a/	→	 \textit{teera} 	     [t\textbf{eː}ra]	‘to put (\textsc{appl})’ 
	\ex /wa-ir-a/	→	 \textit{weera} 	     [w\textbf{eː}ra]	‘to give (\textsc{appl})’ 
\end{xlist}	
\z

\subsection{Other vowel phonological processes}\label{sec-phonology-vowel-elision}\label{sec-phonology-vowel-FVShortening}

A range of other less common phonological processes occur in Ru\-ruu\-li\hyp{}Lu\-nya\-la. 
This section addresses only the most common ones and discusses the final vowel shortening, the vowel lengthening before nasal clusters, as well as the shortening of trimoraic sequences of vowels.

Similarly to what has been reported for closely related languages, such as Luganda (see e.g.\,\citealt{Hymanetal1987Luganda, Hymanetal1990Final}), also in Ru\-ruu\-li\hyp{}Lu\-nya\-la no long vowels occur at the end of a phonological word. 
The examples in~(\ref{ex-vowel-FVS}) illustrates the application of this process: 
The application of the compensatory lengthening as a result of glide formation (see Section~\ref{sec-phonology-vowel-hiatus}) should have produced a long final vowel, however, the surface realisation has a short vowel. 
It remains a topic of further studies to account for the intricacies of this process when applied to clitic groups and various parts of speech.

\ea \label{ex-vowel-FVS}
\begin{xlist}	
	\ex /ku-baz-ibu-a/ (\textsc{inf}-talk-\textsc{pass}-\textsc{fv})	
	\\→	 \textit{kubazibwa} 	     [kubazibwa]	‘to be talked about’ 
	\ex /ku-baz-isi-a/ (\textsc{inf}-talk-\textsc{caus}-\textsc{fv})	\\
	→	 \textit{kubazisya} 	     [kubazisja]	‘to cause to talk’ 
\end{xlist}	
\z

All vowels are automatically lengthened before a nasal followed by obstruent sequence. 
Some examples are provided in~(\ref{ex-vowel-nasal-cluster-long}), further examples can be found throughout the grammar sketch where the surface realization is provided.

\ea \label{ex-vowel-nasal-cluster-long}
\begin{xlist}	
	\ex  \textit{musengi} /musengi/ [museːŋgi] ‘new settler (1)’ 
	\ex  \textit{lusimbo} /lusimbo/ [lusiːmbo] `planting season (11)'
\end{xlist}	
\z

Trimoraic sequences of vowels within a word are shortened to bimoraic sequences. 
This happens when a combination of prefixes or of prefixes and a root result in a trimoraic sequences, as in~(\ref{sec-phonology-shortening-prefix}). 
Another context of vowel shortening arises 
when the perfective suffix  \textit{-ire} is attached to a verb root with a long vowel in the root final syllable and ending in a liquid. When this happens, the liquid is deleted leading to a sequence of three vowels, as in~(\ref{sec-phonology-shortening-ire}).  
The trimoraic sequence is then also shortened to a bimoraic sequence.

\ea \label{sec-phonology-shortening}
\begin{xlist}
	
	\ex 	/βa-eːβalj-a/ (\textsc{2sbj}-appreciate-\textsc{fv})\\→  \textit{beebalya}  [beːβalja] `they appreciate' 	
	\label{sec-phonology-shortening-prefix}	
	\ex 	/gwaːraːr-ire/ (be.at.peace-\textsc{pfv})\\→ [gwaːraː-ire]\\→  \textit{gwaaraire}  [gwaːraire] `be at peace (\textsc{pfv})' 	\label{sec-phonology-shortening-ire}	
	\end{xlist}	
\z

 
\subsection{Palatalisation}\label{sec-phonology-palatisalition}

Palatalisation occurs when the palatal characteristics of the assimilator are transferred to the assimilee.  
Place-changing palatalisation affects the phonemes /k/ and /g/: in certain contexts they are realised as [c] and [ɟ] respectively when they precede the high vowel /i/ or the glide /j/, as in~(\ref{ex-phonology-palatisalition-k}) and~(\ref{ex-phonology-palatisalition-g}). 
This process seems to apply without exceptions in the cases of prefixes, but it is less common in other contexts, \cf (\ref{ex-phonology-nopalatisalition}). 
Note that the orthography does not reflect the palatalisation (see Section~\ref{sec-orthogr-gjkc}). 

\ea \label{ex-phonology-palatisalition-k}
\begin{xlist}	
	\ex 	/ki-ntu/ →  \textit{kintu} [ciːntu] `thing (7)'
	\ex 	/ki-aka/ →   \textit{kyaka} [kyaaka] → [caːka] `heartburn (7)' 
	\ex 	/a-kya-li/ (\textsc{1sbj}-\textsc{pers}-be) →  \textit{akyali} [acaːli] `s/he is still'
\end{xlist}	
\z

\ea \label{ex-phonology-palatisalition-g}
\begin{xlist}	
	\ex 	/gi-sai/ →  \textit{gisai} [ɟisai] `good (4)'
	\ex 	/gi-a-eːre/ (\textsc{4sbj}-\textsc{pst}-yield:\textsc{pfv})\\→ [gjaːjeːre] →  [ɟaːjeːre] /gyayere/ `it (4) yielded' 
\end{xlist}	
\z

\ea \label{ex-phonology-nopalatisalition}
\begin{xlist}	
	\ex 	/mu-sengi/ →  \textit{musengi} [museːŋgi] ‘new settler (1)’
	\ex 	/ki-bbaaki/ →   \textit{kibbaaki} [kibaːki] → [cibaːki] `shoulder bone (7)'
\end{xlist}	
\z


\subsection{Postnasal fortition}\label{sec-phonology-fortition}

Fortition as a phonological process is based on the notion of “strength” (or degree of stricture) of consonants.
The strength hierarchy is as follows: voiceless plosives > voiced plosives > voiceless fricatives > voiced fricatives > nasals > liquids > glides >low vowels >mid vowels >high vowels \citep{Vennemann1988Preference}. 
In other words, fortition involves a change where a relatively weak sound moves from right to left along the scale of strength. 
In Ru\-ruu\-li\hyp{}Lu\-nya\-la liquids /l/ and /r/ and other approximants, such as /j/ and /w/, change into stops after nasals, as in~(\ref{ex-phonology-fortition}).

\ea \label{ex-phonology-fortition}
\begin{xlist}	
	\ex 	/n-li/ (\textsc{1sg.sbj}-be)\\→  \textit{ndi} [ndi] `I am'
	\ex 	/n-li-tegesj-a/ (\textsc{1sg.sbj}-\textsc{5obj}-use-\textsc{fv})\\→  \textit{nditegesya}  [nditegesja]  `I use it (5)' 
	\ex 	/n-weːrj-a/ (\textsc{1sg.obj}-give-\textsc{fv})\\→  \textit{mpeerya} [mpeːrja] `give me'
\end{xlist}	
\z


\subsection{Nasal place assimilation}\label{sec-phonology-nasal-assimilation}

In Ru\-ruu\-li\hyp{}Lu\-nya\-la a nasal assimilates to the place of articulation of the immediately following consonant. 
The context for this process is common with noun class 9 and 10 noun prefixes  \textit{n-}, as in~(\ref{sec-phonology-nasal-9}), and the first person singular subject and object prefix  \textit{n-}, as in~(\ref{sec-phonology-nasal-1sg}). 
In these nasal consonant sequences, the nasal and the following consonant share the same place of articulation as a result of place feature assimilation of the nasal to the following consonant. 
Thus, the respective prefixes are realised as the bilabial nasal [m] before the labial /p/ and /b/, as the velar nasal [ŋ] before velar consonants /k/ and /g/, and as the palatal nasal [ɲ] before /ɟ/, /c/, and /j/. 
The alveolar [n] is found in all other contexts.

\ea \label{sec-phonology-nasal-9}
\begin{xlist}	
\ex /n-diga/ →  \textit{ndiga}  [ndiga] `xylophone (9)'
\ex /n-buli/ →  \textit{mbuli}  [mbuli] `goat (9)'
\ex /n-puku/ →  \textit{mpuku} [mpuku] `cave (9)'
\ex /n-ɟagi/ →  \textit{njagi}  [ɲɟagi] `tango eggplant (9)'
\ex /n-goma/ →  \textit{ngoma}  [ŋgoma] `gizzard (9)'

\end{xlist}	
\z

\ea \label{sec-phonology-nasal-1sg}
\begin{xlist}
\ex /n-som-a/ (\textsc{1sg.sbj}-read-\textsc{fv}) →  \textit{nsoma}  [nsoma] `I read'

\ex /n-wa-a/  (\textsc{1sg.sbj}-give-\textsc{fv}) →  \textit{mpa}  [mpa] `I give'
\ex /n-cuːw-a/  (\textsc{1sg.sbj}-be\_satisfied-\textsc{fv}) →   \textit{ncuːwa}  [ɲcuwa] `I am satisfied'
\ex /n-βal-a/ (\textsc{1sg.sbj}-count-\textsc{fv}) →  \textit{mbala}  [mbala] `I count'
\ex  /n-kam-a/ (\textsc{1sg.sbj}-milk-\textsc{fv}) →  \textit{nkama}  [ŋkama] `I milk'
\end{xlist}	
\z

Morpheme-internally, one only finds homorganic nasal consonant sequences, non-homorganic sequences are not attested. 

\section{Orthography}\label{sec-phonology-orthogrpahy}
The Ru\-ruu\-li\hyp{}Lu\-nya\-la orthography is being developed by the speakers' community with the assistance of SIL and in this section we summarise the rules developed by this collaboration and adopted in this dictionary and grammar sketch. 
We start with phoneme-grapheme correspondences in Section~\ref{sec-ortho-phoneme-grapheme}. 
We then discuss a range of special cases, which might be challenging and are less intuitive to native speakers, including the use of double consonants (Section~\ref{sec-orthogr-double}),  the use of the graphemes <g> vs.\,<j> and <k> vs.\,<c> (Section~\ref{sec-orthogr-gjkc}), the use of <l> vs.\,<r> (Section~\ref{sec-orthography-lr}), as well as the spelling of loanwords (Section~\ref{sec-orthography-loans}). 
The use of capital letters is discussed in Section~\ref{sec-orthogrpaphy-capitalisation}. 
We then proceed with the practical orthographic conventions going beyond the use of graphemes and discuss the word division (Section~\ref{sec-orthography-worddivision}) and the use of the apostrophe (Section~\ref{sec-orthography-apostrophe}).

\subsection{Phoneme-grapheme correspondence}\label{sec-ortho-phoneme-grapheme}

The letters (graphemes and digraphs) listed in~(\ref{tab-graphemes}) are used in the orthography to represent the sounds of Ru\-ruu\-li\hyp{}Lu\-nya\-la. 
The overview also provides the underlying phonological representation in slashes /…/, as well as the surface realisation (i.e.\,the actual pronunciation) in square brackets […] and thus highlights some of the differences between these three levels of representation.
The orthographic representation follows a number of further rules discussed and illustrated in what follows. 

\ea Phoneme-grapheme correspondences with examples\label{tab-graphemes}
 \begin{tabbing}
 <aaa>: \= /aaaaaa/\= \textit{baaaaba} ‘elder sibling’\kill
<a>: \> /a/ \>  \textit{isaza} /isaza/ [isaza] ‘country (5)’\\
\> \>  \textit{mwana} /muana/ [mwaːna] ‘child (1)’\\
<aa>:  \>  /aː/ \>  \textit{kaamuje} /kaːmuje/ [kaːmuje] ‘squirrel (1)’\\
<ai>:  \>  /ai/  \>   \textit{nzai} /nzai/ [nzai] ‘outside’\\
<b>:  \>  /β/  \>   \textit{mugaba} /mugaβa/ [mugaβa] ‘stretch mark (11)’\\
<bb>:  \>  /b/  \>   \textit{mwibbi} /muibi/ [mwiːbi] ‘thief (1)’\\
<c>:  \>  /c/  \>   \textit{mwica} /muica/ [mwiːca] ‘sausage tree (3)’\\
<d>:  \>  /d/  \>   \textit{daani} /daani/ [daːni] ‘mother-in-law (1)’\\
<e>:  \>  /e/  \>   \textit{musegwe} /musegwe/ [musegwe] ‘wolf (3)’\\
	  \>	 \>  		 \textit{musengi} /musengi/ [museːŋgi] ‘new settler (1)’\\
<ee>:  \>  /eː/  \>   \textit{munyeeto} /muɲeːto/ [muɲeːto] ‘generation (3)’\\
<ei>:  \>  /ei/  \>   \textit{lukeito} /lukeito/ [lukeito] ‘beach (11)’\\
<f>:  \>  /f/  \>   \textit{mufugi} /mufugi/ [mufugi] ‘leader (1)’\\
<g>:  \>  /g/  \>   \textit{mugaaga} /mugaːga/ [mugaːga] ‘bangle (3)’\\

<i>:  \>  /i/ \>   \textit{mulimo} /mulimo/ [mulimo] ‘job (3)’\\
  	\>		  \>	 \textit{lusimbo} /lusimbo/ [lusiːmbo] `planting season (11)\\
<ii>:  \>  /iː/  \>   \textit{muliiti} /muliːti/ [muliːti] ‘bird trap (3)’\\
<j>:  \>  /ɟ/  \>   \textit{mwijo} /muiɟo/ [mwiːɟo]  ‘generation (3)’\\
<k>:  \>  /k/  \>   \textit{mwaka} /muaka/ [mwaːka] ‘year (3)’\\
<l>:  \>  /l/  \>   \textit{mwala} /muala/ [mwaːla] ‘girl (1)'\\
<m>:  \>  /m/  \>   \textit{mala} /mala/  [mala] ‘to finish’\\
<n>:  \>  /n/  \>   \textit{naka} /naka/ [naka] ‘to chase’\\
<ny>:  \>  /ɲ/  \>   \textit{isanyu} /isaɲu/ [isaɲu] ‘joy (5)’\\
<ŋ>:  \>  /ŋ/  \>   \textit{ŋooŋa} /ŋoːŋa/ [ŋoːŋa] ‘to moo’\\
<o>:  \>  /o/  \>   \textit{isomo} /isomo/ [isomo] ‘lesson (5)’ \\
	\> 	\>  \textit{magombe} /magombe/ [magoːmbe] `tomb (6)'\\
<oo>:  \>  /oː/  \>   \textit{itooke} /itoːke/ [itoːke] ‘banana (5)’\\
<oi>:  \>  /oi/  \>   \textit{mukoi} /mukoi/ [mukoi] ‘brother-in-law (1)’\\
<p>:  \>  /p/  \>   \textit{mupaapaali} /mupaːpaːli/ [mupaːpaːli] ‘pawpaw (3)’\\
<r>:  \>  /r/  \>   \textit{maara} /maːra/ [maːra] ‘pus (6)’\\
<s>:  \>  /s/  \>   \textit{musaiza} /musaiza/ [musaiza] ‘man (1)’\\
<t>:  \>  /t/  \>   \textit{mutala} /mutala/ [mutala] ‘settlement (3)’\\
<u>:  \>  /u/  \>   \textit{katungulu} /katungulu/ [katuːŋgulu] ‘onion (12)’\\
<uu>:  \>  /uː/  \>   \textit{muuro} /muːro/ [muːro] ‘fire (3)’\\
<v>:  \>  /v/  \>   \textit{muvubo} /muvubo/ [muvuβo]  ‘bellows (3)’\\
<w>:  \>  /w/  \>   \textit{muwendo} /muwendo/ [muweːndo] ‘amount (3)’\\
<y>:  \>  /j/  \>   \textit{yaaya} /jaːja/ [jaːja] ‘babysitter (1)’\\
<z>:  \>  /z/  \>   \textit{nzala} /nzala/ [nzala] ‘hunger (9)’
 \end{tabbing}
\z

Long vowels are represented by doubling the respective vowel grapheme, i.e. with <aa>, <ee>, <ii>, <oo>, <uu>.
In a number of context, vowels are always phonetically long and these cases of predictable vowel lengthening are indicated by the respective single grapheme. 
This happens in two major contexts. 
First, when a vowel follows a combination of a consonant and a glide word-internally, it is always lengthened, in the orthography, the respective vowel is represented by a single vowel grapheme, as in~(\ref{ex-orth-longvowels1}), see the discussion of glide formation and compensatory vowel lengthening in Section~\ref{sec-phonology-glide-formation}. 
Second, all vowels before a nasal cluster (i.e.\,a cluster of a nasal and an obstruent) are realised as long vowels, as in (\ref{ex-orth-longvowels2}), but represented with a single vowel grapheme. 

\ea \label{ex-orth-longvowels}
\begin{xlist}	
	\ex Consonant + glide\\ \textit{mwana} [mwaːna] ‘child (1)’ \\
	 \textit{byalo} [bjaːlo] ‘villages (8)' \label{ex-orth-longvowels1}
	\ex Nasal cluster\\  \textit{nsonga} [nsoːŋga] ‘reason (9)'\\
	 \textit{macunda} [macuːnda] ‘yoghurt (6)'
	\label{ex-orth-longvowels2}
\end{xlist}	
\z

\subsection{Double consonant graphemes} \label{sec-orthogr-double}

Double consonant graphemes are written in two situations in Ru\-ruu\-li\hyp{}Lu\-nya\-la. 
First, the phoneme /b/ is represented by <bb> orthographically to distinguish it from the phoneme /β/ written as <b>. 
Double consonant graphemes  <nn> and <mm> are written when the first person subject prefix  \textit{n-} attaches to a verb root beginning with a nasal consonant, as in~(\ref{ex-ortho-geminate}). 
In the case of the digraph <ny>, the prefixed form is represented by <nny>, as in~(\ref{ex-ortho-geminate-ny}).

\ea \label{ex-ortho-geminate}
\begin{xlist}	
    \ex \label{ex-ortho-geminate-a}
    \glll mmaite\\
    	    n-maite\\
	\textsc{1sg.sbj}-know:\textsc{pfv}\\
    \glt  ‘I know’

   \ex  \label{ex-ortho-geminate-b}
    \glll annenere\\
    	    a-n-nen-ire\\
	\textsc{1sbj}-\textsc{1sg.obj}-bite-\textsc{pfv}\\
    \glt  ‘he has bitten me'
    
    \ex  \label{ex-ortho-geminate-ny}
    \glll nnyaatire\\
        n-nyaat-ire\\
	\textsc{1sg.sbj}-roll-\textsc{pfv}\\
    \glt  ‘I have rolled'
    
\end{xlist}	
\z


\subsection{Graphemes <g> vs.\,<j> and <k> vs.\,<c>} \label{sec-orthogr-gjkc}
In many contexts, we write <g> or <j> when they represent /g/ and /ɟ/ respectively and are realised as [g] or [ɟ], following the conventions outlined in~(\ref{tab-graphemes}).  
However, the consonant /g/ is palatalised and realised as [ɟ] in certain contexts before the front vowels /i/ and /e/ or by the glide /j/ (see Section \ref{sec-phonology-palatisalition} on palatalisation). 
Specifically, this affects the agreement prefix  \textit{gi-} of the noun class 4, as in~(\ref{ex-ortho-g1}), as well as the agreement prefix  \textit{gi-} of class 9 in the varieties which have this prefix variant. 
Despite the palatalisation, we write <g> in these cases. 

Similarly, in many contexts, we write <k> or <c> when they represent /k/ and /c/ respectively and are realised as [k] or [c], following the conventions outlined in~(\ref{tab-graphemes}). 
However, the phonemes /k/ is palatalised in some contexts  before the high front vowel /i/ or the glide /j/ and is realised as [c]. 
This happens systematically with the class 7 prefix /ki/ both on the respective nouns and on various agreement targets, as in~(\ref{ex-ortho-k1}) where depending on the context the prefix is either spelled as <ki> or as <ky>.
The palatalisation is also possible but less common in some other contexts, as e.g.\,within verbs, as in (\ref{ex-ortho-k2}).

\ea \label{ex-ortho-gk}
\begin{xlist}	
    \ex \label{ex-ortho-g1}
    \gllll  Emisaale gini gisai.\\
    [emisaːle ɟini ɟisai]\\
        	    e-misaale gi-ni gi-sai\\
	\textsc{aug}-4.tree 4-\textsc{prox} 4-good\\
    \glt  ‘These trees are good.’

   \ex  \label{ex-ortho-k1}
    \gllll Kampala nikyo kibuga kya Uganda ekikulu.\\
[kaːmpala nico ciβuga ca ugaːnda ecikulu]\\
     Kampala  ni-kyo  kibuga  kya  Uganda  e-ki-kulu\\
   1.Kampala \textsc{cop}-7 7.city 7.\textsc{gen} 1.Uganda  \textsc{aug}-7-principal\\
    \glt  ‘Kampala is Uganda's capital city.'

\ex	\label{ex-ortho-k2}
\gllll okwikiriza\\
[okwiːkiriza]/[okwiːciriza]\\
	o-ku-kiriz-a\\
	\textsc{aug}-\textsc{inf}-believe-\textsc{fv}\\
	\glt  ‘to believe'
\end{xlist}	
\z


\subsection{Graphemes <r> and <l>} \label{sec-orthography-lr}

As discussed in Section~\ref{sec-phonology-consonants}, /l/ and /r/ are two distinct phonemes in the speech of some speakers and are written as <l> and <r> respectively, as illustrated in~(\ref{ex-orth-rl}).

\ea \label{ex-orth-rl}
\begin{xlist}
\ex  <l> represents /l/:  \textit{muta\textbf{l}a} `area (3)',  \textit{nkoo\textbf{l}e} `cow pea (1)',  \textit{kikoo\textbf{l}a} `leaf (7)',  \textit{wuu\textbf{l}a} `to beat'
\ex <r> represents /r/:  \textit{buju\textbf{r}a} `drizzle (14)',  \textit{so\textbf{r}a} `to harvest',  \textit{wuu\textbf{r}a} `to hear'
\end{xlist}
\z

The phoneme /l/ is pronounced as [r] before the front vowels /e/ and /i/ and this alternation is reflected in the orthography, as the examples in~(\ref{ex-orth-rtol}) illustrate.

\ea \label{ex-orth-rtol}
\begin{xlist}
\ex  \textit{ocapu\textbf{l}a} [ocapula] `you speak fast' vs.\\ \textit{ocapwi\textbf{r}e} [ocapwiːre] `you have spoken fast'
\ex  \textit{Mu\textbf{l}i} [muli]  \textit{bantu babbi}. `You are bad people.' vs.\\ 	
 \textit{Ekyo nimwo ki\textbf{r}i} [ciri]. `That's how it is.'
\end{xlist}
\z

Though both major varieties Ruruuli and Lunyala distinguish /l/ and /r/ in certain phonological context and minimal pairs exist in both varieties, in a number of lexical items in Lunyala the realisation with [l] corresponds to the realisation with [r] in Ruruuli. 
For these lexical items, two spelling variants are included into the dictionary and cross-referenced accordingly.


\subsection{Loan words} \label{sec-orthography-loans}
The orthography of loan words reflects their pronunciation, as in~(\ref{ex-orth-loans}).

\ea \label{ex-orth-loans}
 \textit{mupolotesitanti} `Protestant (1),  \textit{vanisi} `varnish (1)',  \textit{kampuni} `company (9)',   \textit{vesiti} `vest (9),  \textit{videyo} `video (9)',  \textit{nekileesi} `necklace (9)',  \textit{vanila} `vanilla (1a)'
\z

\noindent Furthermore, the orthography of place names that have been indigenised reflects their pronunciation by Ruruuli-Lunyala native speakers, as in~(\ref{ex-orth-loans-proper1}). 
Similarly, personal names which have been indigenised are written to reflect their pronounciation, as in~(\ref{ex-orth-loans-proper2}).

\ea \label{ex-orth-loans-proper}
\begin{xlist}
\ex  \textit{Yerusaalemi} `Jerusalem (9)',  \textit{Galiraaya} `Galilee (1)' \label{ex-orth-loans-proper1}
\ex  \textit{Kalooli} `Charles (1)',  \textit{Pawulo} `Paul (1)',  \textit{Meere} `Mary (1)'\label{ex-orth-loans-proper2}
\end{xlist}
\z

A number of phonological segments not attested as phonemes in Ruuli are occasionally preserved in loanwords. 
Specifically, the grapheme <h> appears in a handful of loanwords to represent the phonological segment /h/, e.g.\, \textit{mahadi}  ‘disagreement’ (a borrowing from Runyoro) and  \textit{swahabba} [swahaba] `friend' (from Arabic). 
As in the speech of some speakers, it undergoes phonological adaptation and is either realised as [w] or elided, alternative orthographic representations are added to the dictionary and cross-references, e.g.\,both  \textit{mahadi} and  \textit{maadi} ‘disagreement’.  
On rare occasions, this segment is retained in the speech of all speakers, as in  \textit{swahabba} [swahaba] `friend' and only this form is entered in the dictionary. 
The word  \textit{Allah} is conventionally written with <h> but realised as [alaː].

Two further phonological segments occur only in loanwords. 
The phonological segment /ʃ/ is attested in several loanword and is represented as <sh> in the orthography, as e.g.\,in  \textit{gaalimooshi} `train' and  \textit{shariya} `sharia'. 
The phonological segment /tʃ/  is attested in several loanword and is represented as <ch> in the orthography, as e.g.\, \textit{muchwezi} `demigods who occupied the Bunyoro kingdom'.

\subsection{Capitalisation}\label{sec-orthogrpaphy-capitalisation}
The first word of every sentence is capitalised. 
Furthermore, capitalisation applies to proper nouns, such as names of people, places, and clans, as well as to days of the week and names of months.
Two cases are to be distinguished: 
With names of people and places, the first letter of the stem is always capitalised. 
This leads to the situation that some sentences start with two capital letters. 
This happens when the first word is a proper name of a person or a name of a place and has an augment, which is also capitalised, as e.g.\,on  \textit{OKato} in~(\ref{ex-orth-capitalisationA}). Sentence internally only the first letter of the stem is capitalised, as in  \textit{oKato} in~(\ref{ex-orth-capitalisationB}).

\ea 
\begin{xlist}
\ex \label{ex-orth-capitalisationA}
	\glll OKato akyali e Kampala.\\
	  o-Kato a-kya-li e Kampala\\
		\textsc{aug}-1.Kato \textsc{1sg.sbj}-\textsc{pers}-\textsc{cop} 23.\textsc{loc} 1.Kampala\\
\glt ‘Kato is still in Kampala.’

\ex \label{ex-orth-capitalisationB}
	\glll  OIsingoma ayanana oKato.\\
	o-Isingoma a-yanan-a o-Kato\\
		\textsc{aug}-1.Isingoma \textsc{1sbj}-resemble-\textsc{fv} \textsc{aug}-1.Kato\\
\glt ‘Isingoma resembles Kato.’ 
\end{xlist}
\z

Other nouns listed above (e.g.\,the names of languages and ethnic groups, the names of the days of the week) have only one letter capitalised. 
Sentence-medially, this is the first letter of the stem, which can be preceded by an augment, which is not capitalised. 
However, when such a proper noun occurs at the beginning of a sentence and has an augment, then the augment is capitalised and the first letter of the stem is in lowercase, compare the word for Baganda in the first lines of the two examples in~(\ref{ex-orth-capitalisationC}).

\ea \label{ex-orth-capitalisationC}
\begin{xlist}
\ex 
	\glll Eirai abakali aBaganda tibalyanga ntaama.\\
	  e-irai a-bakali a-Baganda ti-ba-li-a-nga ntaama\\
		 \textsc{aug}-in\_the\_past \textsc{aug}-2.woman \textsc{aug}-2.Baganda \textsc{neg}-\textsc{2sbj}-eat-\textsc{fv}-\textsc{hab} 9.mutton\\
\glt ‘In the past, Baganda women used not to eat mutton.’ 

\ex
	\glll Abaganda bangiriza muno oKanca.\\
	  a-Baganda ba-angiriz-a muno o-Kanca\\
		\textsc{aug}-2.Baganda \textsc{2sbj}-praise-\textsc{fv} much \textsc{aug}-1.god\\
\glt ‘The Baganda praise God a lot.'
\end{xlist}
\z

\subsection{Word division} \label{sec-orthography-worddivision}

In addition to all nouns, adjectives, adverbs, and verbs, the following word classes and grammatical markers are written disjunctively: locatives (Section~\ref{sec-morh-locative}), conjunctions, genitive markers (Section~\ref{sec-NP-genitive}), object relative pronoun (Section~\ref{sec-NP-relative}), demonstrative and possessive pronouns (Sections~\ref{sec-morph-demonstrpro} and~\ref{sec-morph-posspro} respectively), as well as question words (Section~\ref{sec-interrogatives-content}). 
Some examples are provided in~(\ref{ex-orth-worddivision}).

\newpage
\ea \label{ex-orth-worddivision} 
\begin{xlist}	
\ex Conjunctions and locatives \label{ex-orth-worddivision-conjunctions}\\
	\glll	Izukira \textbf{ni} twairukiire \textbf{oku} Lango.\\
	izukir-a ni tu-a-iruk-iire o-ku Lango\\
	remember-\textsc{fv} 	when  1\textsc{pl.sbj}-\textsc{pst}-run-\textsc{appl}:\textsc{pfv} \textsc{aug}-17.\textsc{loc} Lango\\
	\trans `Remember when we ran to Lango.'
   \ex Genitive and demonstrative pronouns\\
    \glll abakali		\textbf{ba}		biro	\textbf{bini}\\
    a-bakali		ba		biro	bi-ni\\
   \textsc{aug}-2.woman	2.\textsc{gen}		8.time	8-\textsc{prox}\\
    \glt ‘today’s women’

\ex Relative pronouns\\
	\glll Tulina onyonyi \textbf{gwe} bayeta okisyo.\\
 		tu-lin-a o-nyonyi gwe ba-et-a o-kisyo\\
		1\textsc{pl.sbj}-have-\textsc{fv} \textsc{aug}-1.bird 1.\textsc{rel} \textsc{2sbj}-call-\textsc{fv} \textsc{aug}-1.weaver.bird\\
	\trans `We have a bird which they call weaver bird.'
\end{xlist}	
\z

All reduplicated forms represent a single word and are thus written as one orthographic word, as in~(\ref{ex-orth-reduplocation}).

\ea \label{ex-orth-reduplocation}
\begin{xlist}	
	\ex Verbs:  \textit{kubbaakubba} ‘to beat repeatedly’,  \textit{sekaaseka} ‘to laugh repeatedly’, \mbox{ \textit{somaasoma}} ‘to read repeatedly’ 
	\ex Adverbs:  \textit{bbugubbugu} ‘luminously',  \textit{biralebirale} `carelessly'
	\ex Nouns:  \textit{bukiikikiiki} ‘supremacy (14)’,  \textit{kagembegembe} `scapula bone (12)'
\end{xlist}	
\z

All compounds are written as one orthographic word, as in~(\ref{ex-orth-compounds}).

\ea \label{ex-orth-compounds}
\begin{xlist}	
	\ex  \textit{mukubbyabigogo} `lenient person, merciful person' (a compound of the verb  \textit{kubba} `to hit, beat' and the noun  \textit{bigogo} `banana fibre (8)')
	\ex  \textit{wamukwano} ‘friend (1)’,  \textit{wambalaasi} `horseman (1)' (compounds with the genitive particle of the noun class 1  \textit{wa})
\end{xlist}	
\z

The standard negation prefix  \textit{ti-} is written in one word with the respective verb, as in~(\ref{ex-orth-negation-1}), whereas the negative copula  \textit{ti} used for non-verbal predication is written as a separate orthographic word, as in~(\ref{ex-orth-negation-2}) and (\ref{ex-orth-negation-3}) (see Section~\ref{sec-negation}).

\ea \label{ex-orth-negation}
\begin{xlist}	
\ex \label{ex-orth-negation-1}
	\glll Tinkwaba.\\
	 ti-n-ku-ab-a\\
		\textsc{neg}-\textsc{1sg.sbj}-\textsc{prog}-go-\textsc{fv}\\
	\glt ‘I am not going.’ 	

\ex \label{ex-orth-negation-2}
	\glll Omukali ti musomesya.\\
	 o-mukali ti musomesya\\
		\textsc{aug}-1.woman \textsc{neg}.\textsc{cop} 1.teacher\\
	\glt ‘The woman is not a teacher.’ 
	
\ex \label{ex-orth-negation-3}
	\glll  Omukali ti mukooto.\\
	o-mukali ti mu-kooto\\
		\textsc{aug}-1.woman \textsc{neg}.\textsc{cop} 1-big\\
	\glt ‘The woman is not big.’
\end{xlist}
\z


\subsection{Apostrophe}\label{sec-orthography-apostrophe}

The apostrophe is used to mark the elision of a vowel and respectively of a letter in the following contexts:
(a) when the clausal conjunction  \textit{ni} `and',  \textit{ni} `when' or  \textit{nga} `while' precedes the vowel, as in~(\ref{ex-ortho-apostrophe}); 
(b) when  \textit{ni} is used as the comitative/instrumental preposition  (`with') or as the conjunction `and' to link noun phrases precedes the vowel.

\ea \label{ex-ortho-apostrophe}
\begin{xlist}	
    \ex \label{ex-ortho-apostrophe-a}
    \glll	Onte anywire amaizi n' akuuka.\\
        o-nte a-nyw-ire a-maizi ni a-kuuk-a\\
	\textsc{aug}-1.cow \textsc{1sbj}-drink-\textsc{pfv} \textsc{aug}-6.water \textsc{conj} \textsc{1sbj}-be.satisfied-\textsc{fv}\\
    \glt  ‘The cow had drunk water and is satisfied.’
	  
\ex  \label{ex-ortho-apostrophe-b}
    \glll	Oteesimba ku iteewo ng' akulya.\\
    	   o-ti-esimb-a ku iteewo nga a-ku-li-a\\
	2\textsc{sg.sbj}-\textsc{neg}-stand-\textsc{fv} 17.\textsc{loc} 1.your\_father \textsc{conj} \textsc{1sbj}-\textsc{prog}-eat-\textsc{fv}\\
    \glt  ‘Don’t stand next to your father while he is eating.'
    
\end{xlist}	
\z
