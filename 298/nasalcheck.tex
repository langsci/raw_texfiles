Latex \textbackslash{}'-notation\\
\noindent
-- ĩ,-- ũ,-- ẽ,-- ɔ̃,-- ɛ̃,-- ã\\
--\'{ĩ},-- \'{ũ},-- \'{ẽ},-- \'{ɔ̃},-- \'{ɛ̃},-- \'{ã}\\
--\`{ĩ},-- \`{ũ},-- \`{ẽ},-- \`{ɔ̃},-- \`{ɛ̃},-- \`{ã}\\
--\^{ĩ},-- \^{ũ},-- \^{ẽ},-- \^{ɔ̃},-- \^{ɛ̃},-- \^{ã}\\
--\v{ĩ},-- \v{ũ},-- \v{ẽ},-- \v{ɔ̃},-- \v{ɛ̃},-- \v{ã}\\
%
Unicode\\
--ĩ́,--ṹ,--ẽ́,--ɔ̃́,--ɛ̃́,--ã́\\
--ĩ̀,--ũ̀,--ẽ̀,--ɔ̃̀,--ɛ̃̀,--ã̀\\
--ĩ̂,--ũ̂,--ẽ̂,--ɔ̃̂,--ɛ̃̂,--ã̂\\
--ĩ̌,--ũ̌,--ẽ̌,--ɔ̃̌,--ɛ̃\\
%
\itshape
Italics\\
Latex \textbackslash{}'-notation\\
-- ĩ,-- ũ,-- ẽ,-- ɔ̃,-- ɛ̃,-- ã\\
--\'{ĩ},-- \'{ũ},-- \'{ẽ},-- \'{ɔ̃},-- \'{ɛ̃},-- \'{ã}\\
--\`{ĩ},-- \`{ũ},-- \`{ẽ},-- \`{ɔ̃},-- \`{ɛ̃},-- \`{ã}\\
--\^{ĩ},-- \^{ũ},-- \^{ẽ},-- \^{ɔ̃},-- \^{ɛ̃},-- \^{ã}\\
--\v{ĩ},-- \v{ũ},-- \v{ẽ},-- \v{ɔ̃},-- \v{ɛ̃},-- \v{ã}\\
%

Das hierdrüber ist nur als Referenz/zum Vergleich

Das hierdrunter ist wichtig. Alle Diakritika sollten geeigneten vertikalen Abstand haben und an der Achse ausgerichtet sein. 

Unicode\\
\noindent
--ĩ́,--ṹ,--ẽ́,--ɔ̃́,--ɛ̃́,--ã́\\
--ĩ̀,--ũ̀,--ẽ̀,--ɔ̃̀,--ɛ̃̀,--ã̀\\
--ĩ̂,--ũ̂,--ẽ̂,--ɔ̃̂,--ɛ̃̂,--ã̂\\
--ĩ̌,--ũ̌,--ẽ̌,--ɔ̃̌,--ɛ̃̌,--ã̌\\

\normalfont
--õ̀,--Ṽ̀
\itshape 
--õ̀,--Ṽ̀


\eabox{
\begin{tikzpicture}[every node/.style={inner sep = 0mm, text height=1.5ex}]
  \node(w1b1)[                  ]{b};
  \node(w1a1)[right=0mm of w1b1]{a};
  \node(w1bat)[right=2mm of w1a1]{ba-t};
  \node(w1i1)[right=0mm of w1bat]{i};
  %
  \node(to)[ right=10mm of w1i1]{$\to$\vphantom{b}};
  %
  \node(w2b1)[right=10mm of to ]{b};
  \node(w2a1)[right=0mm of w2b1]{a};
  \node(w2b2)[right=2mm of w2a1]{b};
  \node(w2a2)[right=0mm of w2b2]{a};
  \node(w2t1)[right=0mm of w2a2]{-t};
  \node(w2i1)[right=0mm of w2t1]{i};
  %
  \node(H1)[below = 8mm of w1a1]{H};
  \node(H2)[below = 8mm of w1i1]{H};
  \node(H3)[below = 8mm of w2a1]{H};
  \node(H4)[below = 8mm of w2i1]{H};
  %
  \draw(H1)--(w1a1);
  \draw(H2)--(w1i1);
  \draw(H3)--(w2a1);
  \draw(H4)--(w2i1);
  \draw[dashed](H3.north)--(w2a2.south);
\end{tikzpicture}
}

\clearpage

\ea{\label{embed1}
  \glll mɛ̀ɛ̀ bɛ̀ [mɛ́ gyámbɔ́ bédéwɔ̀]\textsubscript{PRES}\\
        mɛ̀ɛ̀ bɛ̀ mɛ-H gyámbɔ-H H-be-déwɔ̀ \\
        1\textsc{sg}.{\FUT} be 1\textsc{sg}-\textsc{prs} cook-{\R} {\OBJ}.{\LINK}-be8-food\\}\jambox*{[{\FUT} - \textsc{prs}]}
    \trans `I will be cooking food.'
\z


\ea{\label{embed1}
  \glll mɛ̀ɛ̀\\
        mɛ̀ɛ̀ bɛ̀\\
        be8-food\\}\jambox{[{\FUT} - \textsc{prs}]}
    \trans `I will be cooking food.'
\z

As a minimal pair, \REF{embed2} shows that a change of the tense-mood category in the second constituent entails a change in the relation between newly adopted time perspective and the situation. While the {\itshape bɛ̀} constituent still anchors the time perspective in the \textsc{future}, the situation of cooking will have been completed in the \textsc{remote past}.

\ea{\label{embed2}
  \glll mɛ̀ɛ̀ bɛ̀ [mɛ́ɛ̀ gyámbɔ́ bédéwɔ̀]\textsubscript{{\PST}2}  \\
        mɛ̀ɛ̀ bɛ̀ mɛ́ɛ̀ gyámbɔ-H H-be-déwɔ̀ \\
        1\textsc{sg}.{\FUT} be 1\textsc{sg}.{\PST}2 cook-{\R} {\OBJ}.{\LINK}-be8-food\\ \jambox{[{\FUT} - {\PST}2]}
    \trans `I will have cooked food.'}
\z
