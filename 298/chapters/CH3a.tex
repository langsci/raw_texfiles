\chapter{Parts of speech}
\label{sec:POS}


In this chapter, I describe the parts of speech in Gyeli, also referred to as word classes. The presentation of Gyeli's parts of speech system relies on a grammatical rather than semantic classification into categories. Following \citet[1-2]{schachter2007}, I consider grammatical properties such as ``the word’s distribution, its range of syntactic functions, and the morphological or syntactic categories for which it is specifiable'' as determining criteria for parts of speech classification. 

I generally distinguish lexical and grammatical word classes as well as open and closed classes.\footnote{Gyeli words maximally contain three segmental morphemes with the possibility to host additional tonal morphemes. This is discussed in detail in \chapref{sec:WordForm}. The restriction on word length is, however, not morphological in nature, but phonological, as outlined in \sectref{sec:SyllDist}, generally permitting only up to three syllables in a word.}
Gyeli has only two open word classes, namely the lexical classes of nouns and verbs. Given their limited number of members, adjectives and adverbs are closed classes in Gyeli, unlike many other languages in which these are open classes. The semantic functions that they carry in languages with large adjective and adverb classes are taken over by nouns. In addition to these typical lexical word classes, Gyeli also has a lexical, closed class of ideophones. 

The frequency of lexical word classes' occurrence in the Gyeli text corpus is displayed in \tabref{Tab:POSnolex}. Lexical words constitute 46.9\% of the words in the corpus.\footnote{As described in \sectref{sec:Data}, the corpus comprises 3304 words in total. For the distribution of word class frequencies, only 3133 words were taken into account, omitting e.g.\ code-switching and repetitions.} Out of these lexical words, 87.8\% constitute open class words, namely nouns and verbs. The closed lexical word classes with the most tokens are adverbs, followed by ideophones and finally adjectives.


\begin{table}
\begin{tabular}{ll rr}
 \lsptoprule
  \multicolumn{2}{l}{Word class} &  \multicolumn{2}{l}{Frequency}  \\ %0 in {\SG}: 51,  in {\PL}: 21
 \midrule
  {\bfseries Open} &  & {\bfseries 1289} &  {\bfseries 87.8\%} \\
  \midrule
  & Nouns  & 630  & 48.9\% \\
 &   Verbs  &  659  & 51.1\%  \\ % really 659 before 568
 \midrule
  {\bfseries Closed} &  & {\bfseries 179} & {\bfseries 12.2\%} \\
  \midrule
 & Adjectives &  9 & 5\%   \\
  & Adverbs   & 150 & 83.8\%   \\ % 2x 3/6, 2x 3/0
 & Ideophones    &  20  & 11.2\% \\
 \midrule
Total & & 1468 & 100\% \\
 \lspbottomrule
\end{tabular}
\caption{Frequency of lexical word classes (46.9\% of tokens in the corpus)}
\label{Tab:POSnolex}
\end{table}


In comparison, grammatical words constitute more than half of the corpus with 53.1\%. Their various subcategories are summarized in \tabref{Tab:POSnogram}.  Following \citet{schachter2007} with slight modifications,\footnote{Differences between \citet{schachter2007} and my parts of speech classification concern the subclasses of major categories. While \citet[35]{schachter2007} only subsume role markers, quantifiers, classifiers, and articles under noun adjuncts, I treat every grammatical word class that can appear in the noun phrase as an element of the noun phrase.} I distinguish pronouns, other pro-forms, elements of the noun phrase, elements of the verb phrase, adpositions, conjunctions, and other minor word classes in Gyeli, each of which has some subclasses. Elements of the verb phrase constitute the most frequent grammatical word category with 33.3\%. Within this category, the subject-tense-aspect-mood-polarity (\textsc{stamp}) marker is the most common with 430 occurrences (77.5\%).

\iffalse

%??? table too large for page
\begin{table}
\small
\begin{tabular}{ll|ll}
 \lsptoprule
  \multicolumn{2}{l}{Word class} &  \multicolumn{2}{l}{Frequency}  \\ %0 in {\SG}: 51,  in {\PL}: 21
 \midrule
 {\bfseries Pronouns}  &  & {\bfseries 272} {\bfseries 15.3\%} \\ % plus possessives 58
  & Subject pronouns  & 73 & 4.1\% \\
  & Non-subject pronouns  & 122 & 6.8\% \\
  & Interrogative pronouns  &  11 & .6\% \\
  & Possessor pronouns  & 58 & 3.2\% \\
  & Reflexive pronoun {\itshape mɛ́dɛ́}  & 8 & .4\% \\
 {\bfseries Other pro-forms} &   & {\bfseries 506} {\bfseries 28.4\%} \\
  & \textsc{stamp} marker  & 430 & 24.1\% \\
  & Interrogative pro-forms  & 32  & 1.8\% \\
  & Pro-adverb & 33 & 1.8\% \\
  & Pro-clause &  5 & .3\% \\
  & Pro-sentence & 6 & .3\% \\
 {\bfseries Noun adjuncts} &  & {\bfseries 312} {\bfseries 17.5\%} \\
 & Limiting modifiers  &  15 & .8\% \\
 & Demonstratives  & 53  & 3\% \\
 & Anaphoric marker (free)  & 2 & .1\% \\ % 11 anaphoric markers are bound
 & Agreeing quantifiers   & 36 & 2\% \\
 & Invariable quantifiers   & 6 & .3\% \\
 & Attributive/genitive markers   & 114  & 6.4\% \\
 & Plural markers  & 11 & .6\% \\
 & Adpositions   & 75 & 4.2\% \\
 {\bfseries Verb adjuncts} &  & {\bfseries 206} {\bfseries 11.6\%} \\
 & Auxiliaries   &  176 & 9.9\% \\
 & Verbal particles   & 30 & 1.7\% \\
 {\bfseries Conjunctions} &  & {\bfseries 284} {\bfseries 15.9\%} \\
 & Coordinators   & 60 & 3.4\% \\
 & Complementizer  & 109 & 6.1\% \\
& Comitative marker  & 62  & 3.5\% \\ % nee, is 58 und 41 {\CONJ}
& Adverbializers  & 14 & .8\% \\ % plus ka if
  & Discourse structuring {\itshape yɔ́ɔ̀}  & 39 & 2.2\% \\
 {\bfseries Others} &  & {\bfseries 200} {\bfseries 11.2\%} \\
 %& Similative marker  & 4 & .2\% \\
 & Copulas  & 70 & 3.9\% \\
% & Infinitives  & 37 & 2\% \\
 & Sentential modifiers  & 57 & 3.2\% \\
 & Interjections  & 31 & 1.7\% \\
 & Exclamations  & 42 & 2.3\% \\
 \midrule
  Total & & 1780 & 100\% \\
 \lspbottomrule
\end{tabular}
\caption{Frequency of grammatical word classes}
\label{Tab:POSnogram}
\end{table}

\fi


\begin{table}
\small
{%
\begin{tabular}{ll rr}
 \lsptoprule
  \multicolumn{2}{l}{Word class} &  \multicolumn{2}{l}{Frequency}  \\ %0 in {\SG}: 51,  in {\PL}: 21
 \midrule
 {\bfseries Pronouns}  &  & {\bfseries 240} & {\bfseries 14.4\%} \\
 \midrule
  & Subject pronouns  &   61 & 25.4\% \\ %before 73
  & Non-subject pronouns  &   103 & 42.9\% \\ %before 122
  & Interrogative pronouns  &    10 & 4.2\% \\
  & Possessor pronouns  &   59 & 24.6\% \\
  & Reflexive pronoun {\itshape mɛ́dɛ́}  &   7 & 2.9\% \\
 \midrule
  \multicolumn{2}{l}{\bfseries Other pro-forms}   & {\bfseries 63} &  {\bfseries 3.8\%} \\
 \midrule
  & Interrogative pro-forms  &   19  & 30.2\% \\ %before 32
  & Pro-adverb &   33 & 52.4\% \\
  & Pro-clause &    5 & 7.9\% \\
  & Pro-sentence &   6 & 9.5\% \\
 \midrule
 \multicolumn{2}{l}{\bfseries Elements of the noun phrase} & {\bfseries 233} & {\bfseries 14\%} \\
 \midrule
 & modifiers with agreement prefix  &    54 & 23.2\% \\
 & modifiers with plural agreement only  &   5  & 2.1\% \\
 & modifiers with agreeing free morpheme  &   167 & 71.7\% \\
 & prenominal invariable modifiers  &   0 &  0\%  \\
 & postnominal invariable modifiers   &   7 & 3\% \\
 \midrule
 \multicolumn{2}{l}{\bfseries Elements of verbal complex}  & {\bfseries 555} &  {\bfseries 33.3\%} \\
 \midrule
  & \textsc{stamp} marker  &   430 & 77.5\% \\
 & Auxiliaries   &   75  & 13.5\% \\ % only counted true, semi go with verbs % before 176
 & Verbal particles   &   50 & 9\% \\
 \midrule
 {\bfseries Adpositions} &  & {\bfseries 156} &  {\bfseries 9.4\%} \\
 \midrule
 & Prepositions  &   120  & 76.9\% \\
 & Postpositions   &   36 & 23.1\% \\
 \midrule
 {\bfseries Conjunctions} &  & {\bfseries 180} &  {\bfseries 10.8\%} \\
 \midrule
 & Coordinators   &   56 & 31.1\% \\
 & Subordinators   &   124 & 68.9\% \\
 \midrule
 \multicolumn{2}{l}{\bfseries Other minor classes}  & {\bfseries 238} &  {\bfseries 14.3\%} \\
 \midrule
 & Copulas  &   55 & 23.1\% \\
 & Identificational marker  &    13 & 5.5\% \\
  & Discourse structuring {\itshape yɔ́ɔ̀}  &   39 & 16.4\% \\
 & Question markers  &    1 & .4\% \\
% & Infinitives  & 37 & 2\% \\
 & Sentential modifiers  &   57 & 23.9\% \\
 & Extrasentential modifiers  &    73 & 30.7\% \\
 \midrule
  Total & & \textbf{1665} & \textbf{100\%} \\ %116 less than before
 \lspbottomrule
\end{tabular}
}
\caption{Frequency of grammatical word classes (53.1\% of tokens in the corpus)}
\label{Tab:POSnogram}
\end{table}

%& Similative marker  ADD {\TO} {\pOS} prepositions CHAPTER {\AND} verbs with prep

%Pronouns and other pro-forms constitute over 45\% of grammatical words in the corpus. Among them, the subject-tense-aspect-mood-polarity (\textsc{stamp}) marker, a clitic encoding subject agreement and grammatical information within the clause, such as tense, aspect, and negation, is the most common with 430 occurrences in the corpus. In comparison, noun adjuncts, i.e.\ grammatical parts of speech that can occur in the noun phrase, are comparatively rare. Non of them attain more than 6.4\% (attributive/genitive markers); the total of noun adjuncts amounts to 17.5\% of all word classes in the corpus. Verb adjuncts, namely auxiliaries and verbal particles, are less frequent than noun adjuncts, which might be reflected in the general noun/verb ratio with nouns being more frequent than verbs. Even conjunctions such as coordinators, complementizers, comitative markers, adverbializers, and discourse elements, are slightly more frequent than verb adjuncts.


With regard to open versus closed word classes, the majority of the word tokens in the corpus belong to the closed classes in Gyeli. All grammatical parts of speech  presented in \tabref{Tab:POSnogram}
are closed classes.\footnote{Parts of speech with zero occurrences are attested from elicitations, but are not represented in the corpus.} In addition, the lexical classes of adjectives, adverbs, and ideophones belong to the closed word classes, as explained above. Thus, closed classes constitute 58.9\% (1844 in total numbers) of the 3133 word corpus.
The relative dominance of closed word classes in Gyeli is remarkable since it correlates with a morphological type of language that is closer to the analytic end of the analytic--synthetic scale. As \citet[23]{schachter2007} point out,
\begin{quote} closed word classes tend to play a more prominent role in analytic languages than they do in synthetic languages. This is because much of the semantic and syntactic work done by the members of closed word classes in analytic languages is done instead by affixes in synthetic languages.
\end{quote} 


I will describe each part of speech in the remainder of this chapter, providing defining properties for each category. I start with the open word classes of nouns and verbs, giving information on selected subclasses, for instance the mass/count distinction in nouns. I then proceed with the other lexical classes of adjectives, adverbs, and ideophones before discussing grammatical classes.







	
	

\section{Nouns}
\label{sec:N}


There has been much discussion in the literature as to what a {\itshape noun} is, a linguistic term that is often used intuitively. \citet[10]{rijkhoff2002} maintains that ``there is still no general consensus among typologists on what constitutes a noun''. There is not even a unanimous agreement as to whether every language has a noun category. \citet{gil2013} claims, for instance, that Riau Indonesian does not have a noun (nor a verb) word class. \citet[12]{rijkhoff2002} distinguishes between (i) languages without a major word class of nouns, (ii) languages where nouns cannot be distinguished from other word classes, and (iii) those languages that do have a distinct noun word class. \citet[5]{schachter2007}, on the other hand, hold that ``[t]he distinction between nouns and verbs is one of the few apparently universal parts-of-speech distinctions''. They further explain that alleged examples of languages which would fall in category (i) or (ii) according to Rijkhoff had been based on incomplete data and therefore cannot be considered as counter-examples against this universal word class distinction. In any case, scholars seem to agree that at least most languages of the world have nouns as a distinct word class \citep[720]{tamm2006}. 

According to \citet[708]{evans2000}, linguists usually define nouns by three different types of criteria, namely semantic, morphological, and syntactic.
In terms of semantics, a common definition is given by \citet[5]{schachter2007} who consider nouns a ``class of words in which occur the names of most persons, places, and things''. Similar definitions are provided by other authors, for example by \citet[720]{tamm2006} and \citet[710]{evans2000}. All these scholars emphasize, however, that this is a traditional definition of convenience, but that membership of a word in a certain part of speech has to be established on other grounds. There may be nouns that refer to other entities than persons, places or things, while, on the other hand, there may be persons, places or things that denoted by some other word class than nouns. 

Another way of viewing nouns is to distinguish them from other open word classes such as verbs, adverbs, and adjectives on the basis of different morphosyntactic properties  (see,
%for instance,
e.\,g. \citealt{bhat2000} and \citealt{baker2003}). The advantage of this approach is that it emphasizes the specific structures within a parts-of-speech system of a given language rather than over-generalizing across languages. Nouns may be inflected for categories such as number, case, possession, and definiteness \citep[722]{tamm2006}. They may trigger agreement of these categories as heads of a noun phrase. Syntactically, they may take a certain position within a noun phrase that serves as an argument or adjunct, while dependent word classes are arranged in specific ways around them.

As \citet[733]{lehmann2000}  put it concisely, ``[l]ike any other grammatical category, the word class ``noun'' has no universal status {\itshape a priori}; rather, it is a language-specific category''. I will discuss noun properties in Gyeli in detail in the following section. This will help to distinguish nouns from other parts of speech as well as to establish subcategories of nouns that share some nominal features, but not all of them.


\subsection{Noun properties}
\label{sec:Nprop}

I define Gyeli nouns by their structure, function, and distribution in a phrase, distinguishing them from other word classes. As is typical for Bantu languages, Gyeli has an elaborate noun classification system distinguishing nine agreement classes (\sectref{sec:AGR}) which form six major genders (\sectref{sec:genders}). The agreement classes are labeled by digits from 1 through 9, while genders are marked by pairings of agreement classes, for instance gender 1/2, which pairs agreement classes 1 and 2.  The single agreement classes are also specified for number: agreement classes labeled with odd numbers encode singular and pair with even numbered classes that typically express plurality. 

Agreement classes are established on the basis of agreement patterns on dependent elements. Nouns inherently belong to a gender and trigger agreement on their agreement targets. Agreement targets and their agreement forms in Gyeli are listed in \tabref{Tab:AGRtargets}. They include the various pronominal paradigms, the subject-tense-aspect-mood-polarity (\textsc{stamp}) marker and \textsc{stamp} copula as verbal indexing as well as some elements of the noun phrase (\sectref{sec:NAdjuncts}), namely demonstratives, attributive and anaphoric markers, nominal modifiers\footnote{There are five nominal modifiers in Gyeli, which encompass a variety of semantic/functional classes and which show diverse agreement prefix patterns. They are individually listed in \tabref{Tab:DEICMOD}. In \tabref{Tab:AGRtargets} I represent them as three groups: modifiers with a stem-initial consonant ``\textsc{mod}(-C)'', modifiers with a stem-initial vowel ``\textsc{mod}(-V)'', and those that only show agreement in the plural ``{\NUM}, {\GEN}''.} distinguished by consonant-initial and vowel-initial stems, and the plural agreement only in some numerals and the genitive marker. Agreement targets are sorted by their agreement strategy in terms of free morphemes or agreement prefixes in \tabref{Tab:AGRcl}. As for pronominal forms, only non-speech act participants (third person) agree in gender. In contrast, speech act participants are only distinguished in terms of number. The full pronominal paradigms, including speech act participants, is given in \tabref{Tab:Pros}.

Agreement class affiliation is transparently marked on some nouns in some agreement classes by a noun class prefix (\sectref{sec:NC}). Noun class prefixes are, however, not a consistent diagnostic for agreement class affiliation.  As the gender and agreement system of nouns is a phenomenon that affects the noun phrase and indexing at large,  I discuss this in detail in \sectref{sec:Gender}.

\begin{table}
\begin{tabularx}{\textwidth}{l llllllllX}
 \lsptoprule
{\AGR} class   &  1 & 2 & 3 & 4 & 5 & 6 & 7 & 8 & 9 \\
\midrule
\multicolumn{10}{l}{\bfseries Pronouns} \\
 {\SBJ} &   nyɛ̀ & bá & wú & mí & lí & má & yí & bé & nyì\\
  {\OBJ} &   nyɛ̂ & b-ɔ̂ & w-ɔ̂ & my-ɔ̂ & l-ɔ̂ & m-ɔ̂ & y-ɔ̂ & by-ɔ̂ & ny-ɔ̂ \\
  {\POSS} &   w- & b- & w- & mí- & l- & m- & y- & bí- & ny- \\
\midrule
\multicolumn{10}{l}{\bfseries Verbal index} \\
  {\STAMP} &  a/nyɛ/nu & ba & wu & mi & le & ma & yi & bi & nyi \\
  {\COP} &   àà/nùù & báà & wúù & míì & léè & máà & yíì & béè & nyíì \\
\midrule
\multicolumn{10}{l}{\bfseries Nominal modifiers} \\
 {\DEM} &   nû & bâ & wɔ̂ & mî & lê & mâ & yî & bî & nyî \\
 {\ATT} &  wà & bá & wá & mí & lé & má & yá & bí & nyá \\
   {\ANA} &   nú- & bá- & wɔ́- & mí- & lé- & má- & yí- & bí- & nyí- \\
\textsc{mod}(-C) &   m- & bà- & m/$\emptyset$- & mì- & lè- & mà- & $\emptyset$- & bì- & m/$\emptyset$- \\   
 \textsc{mod}(-V) &  w/n- & b- & w- & my- & l- & m-  & y- & by- & ny- \\
  {\NUM}, {\GEN} &   - & bá- & - & mí- & - & má- & - & bí- & -\\
  \lspbottomrule
\end{tabularx}
\caption{Parts of speech controlled by the noun with agreement forms}
\label{Tab:AGRtargets}
\end{table}




Structurally, nouns consist minimally of a stem and, depending on the noun type, can take noun class prefixes as well as similative and object-linking H tone prefixes, as outlined in \sectref{sec:Prefix}. This sets them apart from verbs which cannot take prefixes. While the agreement targets of nouns also consist of a stem plus prefix, these agreement targets can only take one prefix and that prefix generally differs in its form from noun class prefixes.

On the clause level, most nouns in Gyeli serve as subjects, objects, and adjuncts, as discussed in detail in \sectref{sec:verbalC}, as well as copular complements, as outlined in \sectref{sec:COP}. Nominalized past participles are an exception to this and can only occur as nominal predicates in copula constructions. All nouns can generally occur as bare nouns in their positions.

On the phrase level, nouns function as the head of the construction where they appear in initial position, followed by both agreeing and invariable modifiers, as outlined in \chapref{sec:NP}. In more complex noun phrases such as attributive constructions, the first constituent is always a noun, followed by an attributive or genitive marker and then containing another word, (e.g.\ a noun or verb{\textemdash}see \sectref{sec:CONC} for more information on attributive constructions).
With respect to their morphosyntactic behavior, nouns have a grammatical gender and trigger agreement on their agreement targets (see \sectref{sec:Gender}).\footnote{I view agreement phenomena as a major reason to posit the noun as the lexical head of the phrase rather than assuming a (covert) functional head. The noun as the agreement trigger determines the morphological shape of all agreement targets, including demonstratives that could serve as potential determiner heads.}


Phonologically, nouns allow syllabic and tonal patterns that are disallowed in verbs. For instance, noun root onsets may be complex with clusters of up to three consonants, while this pattern is not found in verbs. Also, diphthongs can be found in monosyllabic noun stems and rarely in the first and second syllables of disyllabic nouns. In contrast, diphthongs are always restricted to monosyllabic stems in verbs. For more information, see \sectref{sec:SyllDist}.  
Tonologically, nouns show a greater variety of patterns, allowing, for instance, H tones on second and third TBUs. Verbs, however, have underlyingly toneless TBUs in second and third syllables which surface as L tones in isolation, as explained in \sectref{sec:Tinventory}.


\subsection{Noun types}
\label{sec:Ntyp}

Gyeli nouns do not constitute a unified class. Instead, they have further subclasses which show different morphosyntactic behavior. This is nothing unusual from a typological perspective; as \citet[8]{schachter2007} point out:
\begin{quote} In most languages some grammatical distinction is made between common nouns, which are used to refer to any member of a class of persons, etc.\ (e.g.\ girl, city, novel), and proper names, which are used to refer to specific persons, etc.\ (e.g.\ Mary, Boston, Ivanhoe). \end{quote}

\noindent Gyeli has three types of nouns: common nouns, proper names, and nominalized past participles. I discuss them one by one in the following sections.




\subsubsection{Common nouns}
\label{sec:commonN}

Common nouns differ from other noun types  in their morphophonological structure as well as their morphosyntactic behavior.
Structurally, common nouns in Gyeli consist minimally of a nominal stem with up to three prefixes maximally added, the first of which is tonal, as shown in the template in \REF{Nstruc}. The different prefix types are described in \sectref{sec:Prefix}.\footnote{Further information as well as an explanation of terminological distinctions of ``noun class'', ``agreement class'', and ``gender'' are provided in \sectref{sec:Gender}.}

\ea  \label{Nstruc}  object-linking H tone -- noun class -- similative marker -- stem
\z

Common nouns can thus take a larger variety and number of prefixes compared with other noun types: proper names can only take a similative prefix, as described in \sectref{sec:properN} and nominalized past participles can only take a nasal noun class prefix, as described in \sectref{sec:NounPart}.

Another difference between common and other nouns is the potential of the former for number inflection. While most common nouns (with the exception of uncountable nouns) have a singular and plural counterpart, as reflected by their pairing of different agreement and noun classes, proper names and nominalized past participles do not inflect for number.

On a phrasal level, common nouns and proper names differ as well. In nominal possessive constructions, common possessor nouns  require an attributive marker, as discussed in \sectref{sec:CONC}. In contrast, proper names take a distinct genitive marker instead, as described in \sectref{sec:GEN}. Nominalized past participles do not occur in possessive constructions.


In summary, a set of tests helps to reliably identify whether a word is a common noun or not. A Gyeli common noun can:

\begin{enumerate}
\itshapeem serve as the subject of a clause
\itshapeem serve as  the  first constituent of a noun + noun construction
\itshapeem be modified by an agreeing demonstrative or possessive pronoun
\itshapeem possibly make a number distinction (even though not all nouns do so) 
\end{enumerate}

\noindent I discuss the number distinction in more detail in \sectref{sec:mass}.






\subsubsection{Proper names}
\largerpage[-2]
\label{sec:properN}


Proper names appear to be often viewed as one category and refer to names of people and places.  In Gyeli, however, proper names of persons and proper names of places form two distinct subcategories of one noun type that I broadly call ``proper names''.  While the two subcategories share some features in which they differ from common nouns, they also differ in a range of aspects. \tabref{Tab:PropName} lists the features that distinguish all proper names from common nouns as well as those in which person and place names differ from one another.


\begin{table}
\begin{tabular}{lcc}
 \lsptoprule
Feature		& 					Person names	& Place names \\
 \midrule
No noun class marker		& 			\checkmark		& 		\checkmark	\\
No plural formation			& 			\checkmark		&		\checkmark	\\
No object-linking H tone		& 			\checkmark		& 		\checkmark	\\
Restriction to a few agreement classes & 		\checkmark		& 		\checkmark	\\\tablevspace
Similative prefix 			& 			\checkmark		& 			\\
Vocative suffix				& 			\checkmark		& 			\\
Special genitive marking		& 			\checkmark		& 			\\
 \lspbottomrule
\end{tabular}
\caption{Features of proper names}
\label{Tab:PropName}
\end{table}

 

In contrast to common nouns, proper names of persons and places never take noun class prefixes nor do they have singular/plural pairings. Names of people can, however, take the associative plural ({\AP}) marker {\itshape bà} which precedes the proper name, as in {\itshape bà Àdà}, referring to Ada and his family or relatives or, depending on the context, to people that share character traits with Ada (people like Ada). The associative plural marker {\itshape bà} is not restricted to proper names, but is also used with common nouns and pronouns, as discussed in \sectref{sec:AP}. As proper names do not take noun class prefixes, they do not provide any TBU to take an object-linking H tone, as discussed in \sectref{sec:OBJTone}.

All proper names trigger agreement just like common nouns. In comparison to common nouns, they are very restricted in the agreement classes to which they are affiliated. 
All proper names of persons are a subcategory of class 1. In contrast, all proper names of places such as settlements, villages, towns, rivers, and countries are generally in class 7, with the exception of {\itshape kàmɛ̀rún} `Cameroon', which is also in class 1. Since many of the place names are derived from common nouns,\footnote{The source noun of place names does not necessarily have to come from Gyeli, but could come from another language in the area. Still, the original meaning is recognized and allows for other agreement classes than class 7. Also, even though there are some lexical differences, cognates across languages of the area are often recognizable to speakers and are found in the same gender.}  place names can also agree in gender with the noun they are derived from. For instance,  the village name {\itshape Ngòló} is derived from the Bulu word {\itshape nkôl} `hill'.\footnote{The Bulu name for the village is  {\itshape Nko'olong}.}  % (alexandre 1955: 217)
Since the cognate {\itshape nkùlɛ́ }`hill' in Gyeli belongs to gender 3/4, the village name can trigger agreement patterns both in class 7 and class 3.

Person names feature a range of characteristics that place names do not exhibit. Names of persons productively take the similative prefix {\itshape ná} in the derivation of female names, as discussed in \sectref{sec:SIM}. In contrast, I did not find any place name with this prefix. 
Person names can further take the vocative suffix -{\itshape o}, as discussed in \sectref{sec:VOCSuff}. 

Finally, person and place names differ in their marking of noun + noun genitive constructions when the possessor is a proper name. While all examples in \REF{GenAtt} are structurally identical, person names take a special genitive marker (\sectref{sec:GEN}), as shown in \REF{GenAtt1}. In contrast, place names \REF{GenAtt2} pattern with common nouns \REF{GenAtt3} in that they take an attributive marker (\sectref{sec:ATT}).


\ea \label{GenAtt}
  \ea \label{GenAtt1} person name\\
\gll  j-ínɔ̀ {\bfseries ngá} Námpùndì \\
	le5-name {\GEN} $\emptyset$1.{\PN} \\
 \trans `the name of [the woman] Nampoundi'
\ex\label{GenAtt2} place name\\
\gll j-ínɔ̀ {\bfseries lé} Ngòló \\
    le5-name 5:{\ATT} $\emptyset$7.{\PN}	\\
 \trans `the name of [the village] Ngolo'
\ex\label{GenAtt3} common noun\\
\gll  j-ínɔ̀ {\bfseries lé} síngì \\
	le5-name 5:{\ATT} $\emptyset$7.cat \\
 \trans `the name of the cat'
\z
\z




\newpage

\subsubsection{Ethnographic note on naming strategies}

The Bagyeli have bipartite names, consisting of a vernacular name that is followed by a Christian French name.\footnote{The sample of proper names comprises 111 female and male names and covers all proper names from three Gyeli villages, namely Ngolo, Bomnapenda, and Bibira. It also includes some of the names from yet other villages such as Lebdjom (in the Basaa speaking area) and Ebobissé (within Kribi town).} Taking a Christian name seems to imitate the naming strategy of the Bantu farmers since Christianity does not play a big role in most Gyeli villages. Unless a Gyeli village is in very close contact and on good terms with their farming neighbors, the Bagyeli tend not to go to church and I do not know of any Gyeli village that has their own church at the time of writing. Since the Christian religion is very strong among the Bantu farmers, however, claiming to be Christian in front of outsiders and having a Christian name seem to serve at reducing stigmatization and creating common ground between the Bagyeli and Bantu farmers. Also, the Bagyeli who attend school are more likely to use their Christian name, at least officially, since it is required for enrollment.  In practical terms, however, I have met a few Bagyeli who had forgotten their Christian name. This is not implausible given that the Bagyeli do not call each other by their Christian, but by their vernacular name, and that there is often no official documentation such as birth certificates or ID cards that would remind people of their names. 

The vernacular name is either considered typical Gyeli or a name that is found in other languages of the area as well. If a name occurs in other languages as well, it is most often shared with the Kwasio dialects Mabi and Ngumba, even if the person was born in, for instance, the Bulu contact region. If a name is shared by other languages than Mabi and Ngumba, such as Basaa, Bulu, or Fang, it is almost certainly predictable that the person comes from that specific contact region.

Many of the vernacular names have a (derived) meaning, often from the plant world or animal kingdom. Also, many of them are not gender-specific, but can be used for men and women alike. For others, female names can be derived from some male names. The derivations of a female from a male name are numerous and seem largely unpredictable. Differences between a male and a female form of the same name encompass tone differences as in {\itshape Mimbe} (male: {\itshape Mìmbɛ̂}, female: {\itshape Mímbɛ̂}), different prefixation ({\itshape Mgbâ} (M) > {\itshape Mímgbâ} (F) and {\itshape Sàmɛ̀} > {\itshape Màsámɛ̀} (F)), as well as denasalization of a final vowel ({\itshape Mbɔ̀} (F) > {\itshape Mbɔ̃̀} (M)). The most productive derivation strategy is through the similative prefix {\itshape Na}- as in {\itshape Nanze} with its male counterpart {\itshape Nze} or {\itshape Nandtoungou}, which is derived from {\itshape Ntoungou}.
\tabref{Tab:PN} provides examples of vernacular names as found amongst my consultants and Bagyeli from other Gyeli villages. The table\footnote{A blank cell in the table means that no certain information is available. In contrast, a hyphen (in the Meaning column) means that speakers state that there is no associated meaning with a name.} specifies whether a name is used for men and/or women,\footnote{The superscripted \textsuperscript{D} after the gender means that the name has a counterpart in the opposite sex: Mandzoué (F) > Mandzong (M), Mba (M) >  Mimba (F),  Mímbɛ̂ (F) > Mìmbɛ̂ (M), Nanze (F) > Nze (M), Nandtoungou (F) > Toungou (M), Tsimbo (F) > Batsimbo (M).} its potential use in other languages of the area, and its meaning (if known).

The orthography of names\footnote{The orthography is provided by different Mabi speakers since the Gyeli speakers are mostly illiterate.} is a mix between Bantu and French notation strategies which, in some parts, seem to lack a strict convention. For instance, the sound /u/ can be represented by either the French style {\textlangle}ou{\textrangle}  or the Bantu notation {\textlangle}u{\textrangle}. A word-final /e/, as in {\textlangle}Mamende{\textrangle} or {\textlangle}Mabale{\textrangle}, can either be written with plain {\textlangle}e{\textrangle} or with the French style {\textlangle}é{\textrangle}; accents in local orthography do not mark tone. Other versions seem to be admissible as well, for example varying between {\textlangle}Mabale{\textrangle}, {\textlangle}Mabalé{\textrangle}, {\textlangle}Mabali{\textrangle}, and potentially {\textlangle}Mabally{\textrangle}. This variation can be explained both by idiosyncratic preferences as well as dialectal variation in pronunciation.

\begin{table} 
\fittable{%
\begin{tabular}{l lll}
\lsptoprule
Name  & Gender & Languages & Meaning	\\
 \midrule
Ada & M, F & Gyeli, Kwasio, Fang, Bulu & -- 				 \\
Bibanga & F & Gyeli, Kwasio, Fang & -- 			 \\
Bikanda & M, F & & 			\\
Biyang & M & Gyeli, Kwasio & remedy			 \\
Bouolpuma & M & Gyeli & rotten breadfruit 			\\
Bwedila%/Bouedjila
 & M & Gyeli, Kwasio 	& -- 			 \\
Kimpile & F & & 			\\
Luonga & F & Gyeli, Kwasio & group		 \\
Mabalé & M & Gyeli, Kwasio & 			 \\
Mambi & M & Gyeli & behavior 			\\
Mandzoué & F\textsuperscript{D} & & 			\\
Manligui & F & & 			\\
Mba & M\textsuperscript{D} & Gyeli & rank 				 \\
Mbiambo & F & Gyeli & plenty 			 \\
Mimbanji%/Mimbanguie 
& M & Gyeli & arbalest, crossbow 	 \\
Mímbê & F\textsuperscript{D} & Gyeli, Basaa, Bulu & -- 			 \\
Minlar & M & Gyeli & union 		 \\
Nalingui & F & Gyeli, Kwasio & -- 	\\
Nanze & F\textsuperscript{D} & Gyeli, Kwasio, Bulu, Basaa &	panther		\\
Nandtoungou & F\textsuperscript{D} & Gyeli, Kwasio & --		\\
Nashuong & F & Gyeli & young palm heart?	\\
Ngolo & F & Gyeli, Kwasio & --				\\
Ngo Minsem & F & Gyeli, Basaa & daughter of Minsem		\\
Nguiamba & M & &					\\
Ngusa & M & Gyeli, Basaa & -- 		\\
Nziwu & M & Gyeli, Kwasio & Great antelope			\\
Sedyua%/Segua 
& M & Gyeli, Kwasio & derived from civet?	\\
Tsimbo & F\textsuperscript{D} & Gyeli, Kwasio &	outcast	\\
 \lspbottomrule
\end{tabular}}
\caption{Examples of Gyeli proper names (in local orthography)}
\label{Tab:PN}
\end{table}



In addition to the vernacular and Christian name, many of my consultants, both men and women, have nicknames by which they are consistently called in everyday life. They acquire their nicknames either through their parents or peers or even sometimes come up with a nickname on their own. Usually, nicknames refer to something that a person has achieved or say something about the person's character. Nicknames also come from Western languages (French, English). Examples of nicknames used in Ngolo include {\itshape Bataillon} or {\itshape Délégué}. Also outsiders might receive a nickname; the project's cameraman was thus called {\itshape Freeboy}, presumably due to his nonchalant attitude towards kneeling in the mud while filming. There seems to be a tendency to pick nicknames originating from other languages, as is particularly obvious with Western words. Local languages also provide nicknames, for instance {\itshape ə̀və́lə̀ tíd} `red animal' from Bulu, which was given to a woman for her bright color of skin.



\subsubsection{Nominalized past participles}
\label{sec:NounPart}

Nominalized past participles are defective nouns that are the most deviant noun type.\footnote{Their category label does not imply that there are non-nominalized participles.} All nouns of this category are derived from verbs and function like a past participle, as illustrated in \REF{NounPart1}. More information on the derivation process is provided in \sectref{sec:NOMPart}.

\ea \label{NounPart1}
\glll yíì nkɛ̀lá\\
yíì n-kɛ̀l-a \\
7.ID N-hang-{\NOM} \\
\trans `It is hung up [lit. a hung-up person/thing].'
\z

Unlike full nouns, nominalized past participles never allow a plural form. Thus, while the nominal predicate in \REF{NounPart2a} takes the plural noun class marker {\itshape ba}-, agreeing in number with the subject, this is not the case for the nominalized past participle in \REF{NounPart2b}.

\ea \label{NounPart2}
  \ea \label{NounPart2a}
\glll Àdà nà Màmbì báà {\bfseries bà}ngɛ̀lɛ́nɛ̀ \\
 Àdà nà Màmbì báà ba-ngɛ̀lɛ́nɛ̀ \\
$\emptyset$1.{\PN} {\COM} $\emptyset$1.{\PN} 2.{\COP} ba2-teacher \\
\trans `Ada and Mambi are teachers.'
\ex\label{NounPart2b} 
\glll Àdà nà Màmbì báà {\bfseries mbánâ} \\
 Àdà nà Màmbì báà m-bán-a \\
$\emptyset$1.{\PN} {\COM} $\emptyset$1.{\PN} 2.{\COP} N-marry-{\NOM} \\
\trans `Ada and Mambi are married [lit. are married ones].'
\z
\z


The occurrence of nominalized past participles is restricted to the predicate position of a \textsc{stamp} copula construction (\sectref{sec:SCOP}), as shown in \REF{NounPart1} and \REF{NounPart2}. Consequently, they do not serve as an argument or adjunct, unlike common nouns and proper names. Given their distributional restriction, they never occur in a position where they would trigger agreement, for instance through the addition of agreement targets in the predicate {\NP}. Likewise, speakers would not replace the nominalized past participle with a pronoun that could indicate its affiliation with an agreement class.

Another hypothesis would be to consider these forms as verbs, given their verbal stem and translation. Despite significant differences from common nouns and proper names,  I do not adopt this analysis, but instead classify nominalized past participles as a defective noun type. Evidence for this comes from their prefixation and tonal behavior, and their distribution in sentences which distinguishes them from verbs.   Morphologically, verbs do not take prefixes, but only suffixes. The nominalized past participle, however, consistently takes a nasal prefix. Verbs only have tonal specifications for the first syllable while the potential second and third syllables are underlyingly toneless and thus surface as L in isolation, as explained in \sectref{sec:Tonology}. In contrast, nominalized past participles never surface L on the last syllables, but either H or HL.  Also in terms of their distribution in sentences, nominalized past participle forms cannot be verbs since verbs follow the subject-tense-aspect-mood-polarity (\textsc{stamp}) marker, as described in \sectref{sec:SCOP}. These participles cannot combine with the \textsc{stamp} marker. They only occur in \textsc{stamp} copula constructions (\sectref{sec:COP}). There are several predication types for copula constructions, including nominal and adjectival copulas, but never verbs. \REF{COPNom} contrasts a nominalized past participle with a passive construction in \REF{COPNomb}, as the translation of the nominalized past participle construction might suggest a passive reading.

\ea \label{COPNom}
  \ea \label{COPNoma}
\glll ndáwɔ̀ nyíì mbúyâ (nà vìyɔ́) \\
	ndáwɔ̀ nyíì m-búy-a (nà vìyɔ́) \\
	9$\emptyset$.house 9.{\COP} N-destroy-{\NOM} {\COM} 8$\emptyset$.fire \\
 \trans `The house is destroyed (by fire).'
\ex\label{COPNomb}
\glll ndáwɔ̀ nyí búyá (nà vìyɔ́)\\
	ndáwɔ̀ nyi-H búy-a-H nà vìyɔ́\\
	9$\emptyset$.house 9-\textsc{prs} destroy-{\PASS}-{\R} {\COM} 8$\emptyset$.fire \\
 \trans `The house is being destroyed by fire.'
\z
\z

The nominalized past participle and the passive construction both allow for an instrumental oblique. The form of the \textsc{stamp} copula in \REF{COPNoma} and the \textsc{stamp} marker in \REF{COPNomb} are, however, distinct, as is the participle form with its nasal and its tonal pattern in which it differs from the verbal form in the passive.

While  the passive and the nominalized past participle are two distinct categories, both categories are, however, linked semantically and formally. In terms of semantics, their subjects are the undergoer of an action while the agent would appear in an adjunct or not at all. This is true for both categories, but since the nominalized past participle is more about the result, the agent is mentioned very rarely.

Formally, both categories take a suffix -{\itshape a}.  There are two possibilities to analyze -{\itshape a} with respect to the different categories. Either, one could posit that it is the same suffix which just takes different tonal patterns in different categories. Or one could assume two different suffixes -{\itshape a}, which each come with their own tonal patterns for the passive and the nominalized past participle. I choose the second option, as reflected  in the glosses. The reason for this is not only the different tone patterns associated with the different suffixes, but also a (synchronically) insufficient link between the two categories. Thus, glossing both suffixes -{\itshape a} as passive (and assuming that nominalization is primarily encoded through the nasal prefix in the nominalized past participle) presupposes a derivation chain with passivization as a necessary step.  This assumption is, however, not justified since many verbs with a nominalized past participle form lack a passive form: only 105 (27\%) verbs take a passive form, but 325 (86\%) have a nominalized past participle form.

 


\subsection{Nouns and countability}
\label{sec:mass}

%criteria in absence of articles, no many/much distinction, the same quantifier is used for countable and uncountable nouns (\sectref{sec:many})
% combination with numerals
% sensitive to quantifier {\itshape mànjìmɔ̀} `whole, entire' \sectref{sec:mandjimo}

Gyeli has a ``mass/count distinction'' like many languages in the world. Formally, one can distinguish countable nouns, those that occur both in a singular and a plural form, from non-countable nouns, which do not show a singular/plural distinction. Countable nouns typically describe discrete individual entities such as humans, animals, plants, tools and the like. 

Non-countable nouns are most frequently and regularly found in the transnumeral gender 6. (More information on the gender and agreement system is provided in \sectref{sec:Gender}.) Semantically, all liquids fall into this class, as exemplified in \REF{LMN}.

\ea \label{LMN} Liquid mass nouns
  \ea  ma-jíwɔ́ `water'
\ex ma-vúdɔ́ `oil'
\ex ma-tàngò `palm wine'
\ex ma-vínó `pus'
\ex ma-nzálɛ̀ `urine'
\ex ma-dyúmù `sperm' 
\z
\z

\noindent In addition, deverbal event nouns of gender 6, as in \REF{DEN}, are uncountable. More information on their derivation process is provided in \sectref{sec:NOM6}.

\ea \label{DEN} Deverbal event nouns
  \ea  ma-nyû `drink (n.)' <  nyùlɛ `drink (v.)'
\ex ma-bwã̂sà `thoughts' < bwã̂sa `think' 
\ex ma-bwàlɛ̀ `birth' <  bwàlɛ `be born'
\ex ma-sâ `game (playing) < sâ `do'
\ex ma-tálá `beginning' <  tálɛ `begin'
\ex ma-dìlá `funeral' <  dìlɛ `bury'
\z
\z

There are other non-countable nouns with only a plural form in other agreement classes, but they seem to be less frequent. They mostly belong to class 8 and comprise entities that usually occur in groups, for instance {\itshape bè-sìngì} `spirits'. They also include deverbal nouns such as {\itshape bè-déwɔ̀} `food', which is derived from {\itshape dè} `eat'.

Then there are nouns that only have a singular form. Most often, they are abstract nouns of class 7, as illustrated in \REF{AN}.

\ea \label{AN} Abstract nouns
  \ea  dú `lie'
\ex sɔ̀mɔ̀nɛ̀ `complaint'
\ex ngɔ̀ngɔ̀lɛ́ `sadness, compassion'
\ex pɔ́nɛ̀ `truth'
\ex sɔ́nɛ̀ `shame'
\ex mɛ̀vâ `pride'
\z
\z

There are a few other singular nouns without a plural form in other classes. Semantically, they describe mass entities which have a rather unspecified shape and lack clear-cut boundaries such as {\itshape pfùdɛ́} `mold' (cl. 9) or {\itshape dùwɔ́} `sky' (cl. 5). {\itshape bíwɔ̀} `bad luck' (cl. 3) is another example of an abstract noun. 
Also a few nouns in agreement class 8 lack a plural form. This is remarkable since class 8 is generally a plural class. As explained in \sectref{sec:NC}, however, there are also singular nouns that trigger class 8 agreement, namely those that lack the CV- noun class prefix {\itshape be}-.  Examples of singular-only class 8 nouns include {\itshape vísɔ́} `sun' and {\itshape vìyɔ́} `fire'.  More examples of uncountable nouns are given in \sectref{sec:MinGen} on inquorate genders.

Finally, there are nouns which display mixed characteristics of both non-count\-able and countable nouns. They have a singular and a plural form, and semantically designate granular aggregates such as {\itshape nsɛ́/mì-nsɛ́} `sand' or {\itshape ndísì/mì-ndísì} `rice'. In their singular form, they behave like other non-countable nouns, for instance transnumeral liquids. This becomes especially obvious when modified by some invariable quantifiers (\sectref{sec:mandjimo}) and some nominal quantifiers (\sectref{sec:NomQUANT}). If used in the plural form, these  nouns get a reading of `different types of' or `different units of'. In this usage, they grammatically behave more like countable nouns.









\section{Verbs}
\label{sec:V}



Nouns and verbs constitute the two major word classes in possibly all languages in the world, as \citet[408]{viberg2006} points out. There is, however, still a need to consider what verbs are and how they are distinguished from nouns.  \citet[9]{schachter2007} provide a general, semantically based definition, stating that
\begin{quote}
{\itshape Verb} is the name given to the parts-of-speech class in which occur most of the words that express actions, processes, and the like.
\end{quote}

\noindent Other properties that the authors highlight include, for instance, that verbs foreground temporal relations as well as their function as predicates. After all, characteristics of verbs (as any other word class) are language specific and therefore, it makes sense to distinguish them based on a given language's properties. In Gyeli, nouns and verbs are distinct in many ways. As shown in \chapref{sec:Phon}, they differ on phonological grounds, for example in their distribution of phonemes and tones, nouns allowing a larger degree of freedom while verbs have more restrictions on the occurrence of consonants, vowels, and tones. On a morphological level, nouns take prefixes which Gyeli verbs do not. In contrast, verbs take extension suffixes which is not the case for nouns. In terms of syntactic function, verbs serve canonically as predicates while nouns (or noun phrases) constitute arguments of a given predicate. These various formal differences show clearly that nouns and verbs in Gyeli belong to different word classes.

In the following, I will first describe the structure of the verb.  I then discuss different verb types, including main verbs and auxiliary verbs. 




\subsection{Verb structure}
\label{sec:StructVerb}

The Gyeli verb consists of a lexical root that can take a valence-changing suffix and a tense-mood marking tonal morpheme, as shown in \tabref{Tab:Gyeliverb}. 


\begin{table}
\begin{tabular}{llll}
 \lsptoprule
Slot &   Radical & Prefinal &  Final \\
\midrule
Function &  lexical root & valence change & tense-mood \\
Tone pattern  & H or L & toneless & -H \\
   \lspbottomrule
\end{tabular}
\caption{The Gyeli verb structure\label{Tab:Gyeliverb}}
\end{table}


\tabref{Tab:Gyeliverb} indicates the ``slot'' in which the root and the suffixes occur and is based on the segmental morphological Bantu verb schema by \citet[184]{gueldemann2003}. I extend this schema to also accommodate tonal morphemes. In contrast to the lexical root and the valence changing suffix, which are always segmentally expressed, the final tense-mood marking morpheme is exclusively tonal. The absence or presence of an H tone that attaches to the right of the verb stem encodes past tenses and the realis mood (\sectref{sec:SimpPred}). Lexical roots are specified for either an H or an L tone, while valence changing suffixes are underlyingly toneless. 

While \posscitet{gueldemann2003} Bantu verb schema has eight slots, four before the root and three after the root, Gyeli has a more reduced verbal structure. For instance, subject concord and preverbal tense-aspect-mood information are not encoded on the verb, but by a preverbal subject-tense-aspect-mood-polarity (\textsc{stamp}) clitic (\sectref{sec:SCOP}) and/or complex predicates with auxiliaries (\sectref{sec:CompPred}).

I follow the Bantuist tradition (e.g.\ \citealt{guthrie71}, \citealt{hyman93}, and  \citealt{schadeberg2003}) in my terminological distinction between  {\itshape radical} and {\itshape stem}. The radical, also called {\itshape root}, is the ``irreducible core'' (\citealt[14]{guthrie71}) of the verb that cannot be parsed into further morphemes. In Gyeli, its phonological structure is typically C(C)VC-, but there are exceptions in surface forms pertaining to an additional vowel in some disyllabic underived verbs (\sectref{sec:StemFinalV}) and the deletion of the root-final consonant in monosyllabic verb forms (\sectref{sec:RFCV}). 

The root in Gyeli can function as an independent word without any further bound morphemes attached, as exemplified in \REF{Vroot1} for monosyllabic verb roots. All monosyllabic verbs consist of a root only. Under derivation, a root-final consonant (or variants thereof) will surface, as described in \sectref{sec:RFCV}. This root-final consonant is deleted in monosyllabic roots in order to adhere to an open syllable structure.

\ea \label{Vroot1} Monosyllabic roots
  \ea  dè `eat'
\ex kwê `fall'
\ex bvúɔ̀ `break (v.t.)'
\z
\z

\noindent  Also some disyllabic verb roots satisfy the criterion of an irreducible core, as in \REF{Vroot2}.


\ea \label{Vroot2} Disyllabic roots
  \ea  bámɔ `scold'
\ex púndi `polish'
\ex gyàga `buy'
\z
\z

The root can take an extension or expansion derivation suffix that brings about a valence change. A list of all verbs in the database and their extension morphemes is given in  \appref{sec:AppendixI}.
The root and the potential suffix constitute the stem.\footnote{Traditionally, the stem additionally includes the {\itshape final vowel} that encodes tense-aspect-mood information in more agglutinative Bantu languages. In these languages, Bantuists use the term {\itshape base} to designate the root and potential derivation suffixes without the final vowel. In Gyeli, however, there is no final vowel. Therefore, this distinction is not necessary.}
There are also disyllabic verbs that consist of a root plus extension suffix, as shown in \REF{Vroot3}. Derivation with extension and expansion suffixes is described in \sectref{sec:VDeriv}.

\ea \label{Vroot3} Disyllabic stems
  \ea   bèn-a `be refused' (passive extension -{\itshape a})
\ex\label{Vroot3b} jì-bɔ `close sth.' (-{\itshape bɔ} expansion)
\ex\label{Vroot3c} vú-lɛ `get rid of sth.' (-{\itshape lɛ} expansion)
\z
\z

Thus, whether a disyllabic verb consists of a root only, as in \REF{Vroot2}, or constitutes a stem with a root plus suffix, as in \REF{Vroot3}, depends on the synchronic function of the second syllable.   In synchronic disyllabic verb roots, the vowel of the second syllable is part of the lexeme since its shape is not predictable on morphophonological or morphosyntactic grounds. In contrast, in a disyllabic stem, the second syllable functions as a valence changing suffix. A root vs.\ stem contrast can be found even with the same lexeme, as for instance with the root {\itshape bédɔ} `mount (v.t.)' whose passive form {\itshape béd-a} `be mounted' is analyzed as a stem.
A more detailed discussion on the status of the final vowel as part of the root is given in \sectref{sec:StemFinalV}.

The number of transparent derivational suffixes a root can take is restricted to one.\footnote{As discussed in \sectref{sec:VDeriv}, two categories, e.g.\ applicative and passive, can be merged into one morpheme through vowel change of the applicative suffix in trisyllabic verbs.} Derivational extensions can, however, come as mono- or disyllabic suffixes, allowing a maximum of three syllables in a stem, as shown in \REF{Vroot4}.

\ea \label{Vroot4} Trisyllabic stems
  \ea   gyámb-ɛlɛ `cook for sb.' (applicative extension -{\itshape ɛlɛ})
\ex lɛ̀b-ala `follow each other' (reciprocal extension -{\itshape ala})
\ex dyɛ́g-ɔwɔ `get in a leaning position' (positional extension -{\itshape ɔwɔ})
\z
\z

In the following, I will discuss the shape of the verb root in more detail, focusing on two issues. First, I explore the status of the Gyeli stem-final vowel, arguing that it does not occupy the ``final'' slot of \posscitet{gueldemann2003} morphological Bantu verb structure. I then describe root-final consonants and their variants.




\subsubsection{Stem-final vowel} 
\label{sec:StemFinalV}
\largerpage

Alhough the Gyeli verb structure is significantly different from \posscitet{gueldemann2003} morphological verb schema, one might wonder whether Gyeli does have a vowel in the ``final'' slot, which is typically related to tense-aspect-mood.
Due to a canonical CV syllable structure, Gyeli verbs always end in a vowel, but they are by no means comparable to the ``final vowel'' in the ``final'' slot found in eastern and southern Bantu languages where the final vowel has a grammatical function. In contrast, Gyeli root and stem final vowels are lexically specified.  As discussed in \sectref{sec:CardVowels}, vowel quality is restricted by the stem's syllable length. In monosyllabic verbs, any of the seven vowels, except for /o/, can occur in final position, while disyllabic verbs only allow five vowels in this position, /i/, /o/, /ɛ/, /ɔ/, /a/. Trisyllabic verb stems only allow /ɛ/, /a/, and /ɔ/ as a final vowel.

Another argument for not considering Gyeli stem-final vowels as occupying the final slot of \posscitet{gueldemann2003} Bantu verb structure comes from verb extensions. When Bantu languages such as Swahili add an extension morpheme in the prefinal slot, the final vowel is not necessarily affected by this. The Swahili stem {\itshape chek-a} `laugh', for instance, keeps the final vowel {\itshape -a} even if the stem is extended by a causative morpheme {\itshape -Ish-}:\footnote{The capital {\itshape I} denotes a front vowel that is subject to vowel harmony.} {\itshape chek-esh-a} `make laugh'. Extension morphemes in Gyeli, however, come with their own final vowels and override a disyllabic root-final vowel  as in {\itshape jílɔ} `be satisfied' $\rightarrow$ {\itshape jíl-ɛsɛ} `make satisfied'.

While all final vowels in verbs are lexically specified, they differ with regards to their morpheme affiliation. There are three types of verb-final vowels. First, a verb-final vowel is the nucleus of the verb root in monosyllabic verb forms. It is tonally specified and does not usually change in derived forms. The root vowel ends up in the final position because the final root consonant is deleted, as illustrated in \REF{Vroot5}. The deleted root-final consonants in parentheses only surface with derived forms of the verb, as with the passive forms in \REF{Vroot5}. (More information on root-final consonant deletion is provided in the next section.)

\ea \label{Vroot5}
  \ea  dy{\bfseries à}(y) `sing' < dyày-a `be sung'
\ex kw{\bfseries à}(g) `grind' < kwàg-a `be ground'
\ex nd{\bfseries à}(ng) `cross' < ndàng-a `be crossed'
\z
\z

\largerpage
Second, in disyllabic verb roots, the final consonant is followed by a lexicalized (underlyingly toneless) vowel. This vowel is synchronically part of the root since its quality is not predictable and does not have any grammatical function. In derived forms, this vowel is deleted, as shown in \REF{Vroot6a}. The fact that these additional root vowels are not specified for tone, a property they share with verb extension and expansion suffixes, suggests that diachronically they were derivation suffixes as well.

\ea \label{Vroot6a}
  \ea  fùl{\bfseries ɔ} `descend' < fùl-a `be descended'
\ex dyɔ̀d{\bfseries ɛ} `deceive' < dyɔ̀d-a `be deceived'
\ex gyáng{\bfseries a} `work' < gyáng-ɛsɛ `make sb.\ work'
\z
\z

\noindent Final vowels of monosyllabic verb forms with a diphthong or long vowel as nucleus are treated the same way. As shown in \REF{Vroot6b}, the second vowel of the diphthong gets deleted in derived forms.

\ea \label{Vroot6b}
  \ea  bvú{\bfseries ɔ̀} `break (v.t.)' < bvú.g-ɛ `make break'
\ex dyù{\bfseries ù} `kill' < dyù.w-a.la `kill each other'
\ex ní{\bfseries ɛ̀} `be beautiful' < ní.ng-ɛ.sɛ `make beautiful'
\z
\z

\noindent Historically, these verbs were likely disyllabic, as the examples in \REF{Vroot6a}. This would have involved a process in which first the root final consonant got deleted and then the vowel of the second syllable was merged with the first syllable's nucleus. Synchronically, the second vowel of the diphthong is clearly part of the root vowel since it is specified for tone.

The third type of stem-final vowel is specified through the derivation suffix a root can take, as shown in \REF{Vroot7}.

\ea \label{Vroot7}
  \ea  dyúw-ɛl{\bfseries ɛ} `listen to'
\ex ntɛ́g-al{\bfseries a} `bother each other'
\ex pwàs-ɔw{\bfseries ɔ} `stretch oneself out'
\z
\z

\noindent The segments of derivation suffixes do not change in different tense-aspect-mood categories, but their tonal patterns do (\sectref{sec:GramTM}).






\subsubsection{Suppletive root vowels}
\label{sec:SRV}

Gyeli has a few verbs which change their root vowel in (some) derived forms. I view these as lexically specified exceptions since they do not follow any predictable pattern and are generally rare. All suppletive root vowel forms are given in \tabref{Tab:SupplRV}.


\begin{table}
\begin{tabularx}{\textwidth}{l@{~}X llX l@{~}}
\lsptoprule
\multicolumn{2}{c}{Underived form}  & Reciprocal         & Passive & Causative  & Variants \\
 \midrule
lùà 	& `curse' &  lɔ̀g-ala &  lɔ̀g-a & lɔ̀g-ɛsɛ	&  		ua/ɔ \\
lṹã̀		& `whistle' &  lɔ́ng-ala &  lɔ́ng-a & 	&  	ua/ɔ \\
túà		& `move places' & tɔ́g-ala		&  & tɔ́g-ɛsɛ &  ua/ɔ \\ 		
bwà 		& `become big' &  bɔ̀g-ala &  & 		& 			wa/ɔ	 \\
bwádɔ 	&  `wear' &  bɔ́d-ala &   & bɔ́d-ɛsɛ 	&  		wa/ɔ	\\
bwɛ̀dɔwɔ	& `be tasty' &   & 		& bɔ́d-ɛsɛ	& 		we/ɔ  \\
% \mydashline
\tablevspace
b{\bfseries wè}		& `catch' & b{\bfseries è}y-ala & 	b{\bfseries ù}l-ɛ	& 		&   we/u  \\
k{\bfseries wê} 		& `fall' 	&  k{\bfseries wé}y-ala & 		&  k{\bfseries ù}-ɛsɛ  & 	 we/u \\
lâ		& `harvest' & léy-ala		& léy-a &  &  a/e \\ 
lága		& `contaminate' & lég-ala	 &  & lég-ɛsɛ  &  a/e \\ 
bô 		& `lie down'                 &          & búg-a 	       & 	& 		 o/u \\
yíɛ̀ 		& `dodge' 	& yé-ala &  &   & 		iɛ/e \\
d{\bfseries è}		& `eat' &  d{\bfseries í}y-ala & d{\bfseries í}b-a	& 	d{\bfseries í}l-ɛsɛ	&    e/i  \\
 \lspbottomrule
\end{tabularx}
\caption{Root-final consonant variants (monosyllabic verbs)}
\label{Tab:SupplRV}
\end{table} 

Ten out of the thirteen suppletive root vowels are regular in the sense that all derived forms have the same suppletive vowel. For instance, {\itshape lùà} `curse' takes {\itshape ɔ̀} as root vowel in its reciprocal, passive, and causative forms. Also, the suppletive vowels retain the same tonal pattern as in the underived form, namely H for underived verbs which have an HL pattern and L for L underived verbs. There are a few more irregular cases, however, which have different suppletive vowels for different derived forms and/or tonal changes on the suppletive vowel.  For example, {\itshape bwè} `catch' retains /e/ in the reciprocal form {\itshape bèyala}, but loses the glide /w/, while it has a suppletive vowel /u/ in the passive form {\itshape bùlɛ}. All root vowels remain L. In contrast, {\itshape kwê} `fall' has a regular reciprocal form {\itshape kwéyala}, both in terms of the vowel and its tone, but an irregular causative form {\itshape kùɛsɛ} with both a suppletive vowel and a tonal change from H to L. Finally, {\itshape dè} `eat' has the same suppletive vowel /i/ for all derived forms, but all derived forms have an H instead of an L tone.

Most verbs with suppletive root vowels have monosyllabic stems containing the diphthong /ua/ or the glide /w/, which is changed to /ɔ/ in derived forms. The verb of the underived form is, however, not predictive of a necessary vowel change in derived forms since verbs generally keep their glides and vowels in derived forms. \REF{SRV1} gives an opposition between a regular and an irregular form.

\ea \label{SRV1}
  \ea  bwà `give birth' → bwàl-ɛsɛ ({\CAUS})
\ex bwà `become big' → b{\bfseries ɔ̀}g-ala ({\RECIP})
\z
\z

\noindent Other suppletive forms, for instance from /a/ to /e/ in {\itshape lâ} `harvest' or /e/ to /i/ in {\itshape dè} `eat' seem even more exceptional.



\subsubsection{Root-final consonant variants}
\label{sec:RFCV}

Generally all verb roots (with a few exceptions) have a final consonant, which is lexically specified and only surfaces when a vowel-initial derivation suffix attaches. In monosyllabic stems \REF{Vroot1} and with derivation suffixes that are consonant-initial such as -{\itshape lɛ} or -{\itshape bɔ} in \REF{Vroot3}, the root-final consonant is deleted. In turn, when deriving a monosyllabic verb, the question is which root-final consonant it will have.

As shown in \tabref{Tab:EpenC}, the majority of monosyllabic stems have the same root-final consonants in all their derived forms.\footnote{This is based on 86 monosyllabic verb stems. As discussed in \sectref{sec:SyllV}, there are 88 monosyllabic verb stems in my database. Yet, not all of them undergo derivation. {\itshape dɔ̀} `negotiate' and {\itshape kɛ̀}  `go' do not have any derived forms and therefore the underlying root-final consonant never surfaces.} The types of consonant that can consistently appear root finally are limited to seven: /ŋg/, /g/, and /y/ are the most frequent ones while /l/, /s/, /n/, and /w/ are rare. There are two exceptions to this general pattern. First, eleven monosyllabic verb stems have different root-final consonants with different verb extensions, and second, there is one verb which consistently takes no root-final consonants in any of its forms.


\begin{table}
\small
\begin{tabular}{lrr lll}
\lsptoprule
Root ending & \multicolumn{2}{l}{Frequency}  & \multicolumn{3}{l}{Example} \\
\midrule
{Consonant} & 	{69} & {85.2\%} &  &  &   \\
\midrule
 ~~ /nɡ/ & 	23 & {26.7\%} & sã̂ `vomit' & → &  sá{\bfseries ng}ala `vomit together' \\
~~  /g/   &  22 & {25.6\%} & dvùɔ̀ `hurt' & → & dvù{\bfseries g}ɛsɛ `make hurt' \\
 ~~ /y/   & 17 & {19.8\%} & bà `smoke' & → & bà{\bfseries y}aga `smoke (by itself)' \\
~~  /l/    & 3  & {3.5\%} & vɔ̂ `be calm' & → & vɔ́{\bfseries l}ɛsɛ `make calm' \\ 
~~  /s/  & 2  & {2.3\%} & sɔ́ɔ̀ `continue' & → & sɔ́{\bfseries s}ɛlɛ `continue with sth.' \\ 
~~  /n/  & 1  & {1.2\%} & nyɛ̂ `see' & → & nyɛ́{\bfseries n}ala `see one another' \\
~~   /w/  &	1  & {1.2\%} & dyû `kill' & → & dyú{\bfseries w}ala `kill one another' \\
   \midrule
{Variable}   & {11} & {13.6\%} & see \tabref{Tab:RFCVmono} &  &  \\
\midrule
{No consonant} & 	{1} & {1.2\%}   & dyâ `lie down' & → & dyáala `lie down together' \\
 \lspbottomrule
\end{tabular}
\caption{Root-final consonants in the derivation of monosyllabic verbs}
\label{Tab:EpenC}
\end{table} 

The diversity of root-final consonants surfacing in derived verb forms likely has a historical explanation. Some monosyllabic verb stems may originate from a diachronic extension that got reduced and merged with the monosyllabic root. In the process, the onset consonant of the second syllable \textendash the historical extension suffix \textendash got lost in monosyllabic forms and the suffix vowel got merged with the root vowel.  This reduction is synchronically reflected in monosyllabic verb stems with diphthongs and long vowels, as discussed in \sectref{sec:Diph} and \sectref{sec:VLength}. The original consonants still surface in some derived forms. This scenario would explain why only a limited number of consonants can now serve as root-final consonants: they are related to a limited number of suffixes, some of which do not exist anymore.

The quality of the root-final consonant that will surface in the derivation of monosyllabic verbs is not (entirely) predictable on  phonological grounds, as the oppositions in \REF{RFC0} to \REF{RFC2} show.

\ea \label{RFC0}
  \ea  bwà `give birth'  → bwà{\bfseries l}-ɛsɛ `make give birth'
\ex bwà `become big' → bɔ̀{\bfseries g}-ala `become big together'
\z
\ex \label{RFC1}
  \ea  bâ `marry'  → bá{\bfseries n}-ala `marry each other'
\ex bà `smoke' → bà{\bfseries y}-ala `smoke together'
\z
\ex\label{RFC2}
  \ea  nyâ `suckle'  → nyá{\bfseries ng}-ɛsɛ `breast-feed'
\ex nyàà `defecate' → nyà{\bfseries g}-ɛsɛ `make defecate'
\z
\z

\noindent There are, however, some tendencies that allow us to predict the underlying root-final consonant based on the phonological shape of the monosyllabic verb stem.  
Monosyllabic stems ending in nasal vowels, for instance, almost exclusively have /ŋg/ as root-final consonant, as exemplified in \REF{epenng}. This ties in with the scenario of a historical extension suffix that has been lost: /ŋg/ may have been the onset of the suffix that was lost, while nasality survived on the root vowel.

\eabox{ \label{epenng}
\begin{tabular}{lllll}
lã̂ & `pass' & → & là{\bfseries ng}ɛlɛ & `let pass, spend time' \\
kẽ̀ & `shave' & → & kè{\bfseries ng}ala & `shave one another' \\
sã̂ & `vomit' & → & sá{\bfseries ng}ɛsɛ & `make vomit' \\
dyũ̂ & `be hot' & → & dyú{\bfseries ng}ɛlɛ & `heat sth.' \\
\end{tabular}
}

Another tendency is found with monosyllabic verb stems containing a diphthong. Their final root consonant is almost exclusively /g/, as shown in \REF{epeng}, with a few exceptions concerning the diphthong /iɛ/, which sometimes may also take /y/ as in {\itshape tsíyala} `cut each other', derived from {\itshape tsíɛ̀} `cut'.


\eabox{\label{epeng}
\begin{tabular}{lllll}
dvùɔ̀ & `hurt (v.i.)' & → & dvù{\bfseries g}ala & `hurt one another' \\
lùà & `curse' & → & lɔ̀{\bfseries g}a & `be cursed' \\
tɔ̀à & `boil (v.i.)' & → & tɔ̀{\bfseries g}ala & `boil together' \\
líɛ̀ & `cede, let' & → & lí{\bfseries g}ala & `let to one another' \\
\end{tabular}
}

% \noindent % hyphenation issues
All other root-final consonants seem not to be predictable on phonological grounds.


There are two exceptions to the general pattern described so far. First, in a few cases, the same underived monosyllabic verb stem has different root-final consonants with different extension morphemes.  \tabref{Tab:RFCVmono} gives an exhaustive list of all final root consonant variants for monosyllabic verbs that occur in the database. While there are usually only two variants for the same lexical root, {\itshape dè} `eat' shows that there can be even three variants.\footnote{The passive form of {\itshape dyɔ̀} `laugh' is derived from the applicative form {\itshape dyɔ̀l-ɛsɛ}, which affects not only the final vowel, but changes both vowels /ɛ/ of the extension to /a/.} 

\begin{table}
\fittable{\begin{tabular}{ll lll ll}
\lsptoprule
\multicolumn{2}{c}{Underived form}  & Reciprocal         & Passive & Causative & Applicative & Variants \\
 \midrule
bâ 			 & `marry' &  bá{\bfseries n}-ala & 	       & bá{\bfseries l}-ese	& 		& n/l \\		
bwè 			& `catch' &  bè{\bfseries y}-ala & bù{\bfseries l}-ɛ & 		& 		& 		y/l \\
vû 			&  `leave' &  vú{\bfseries y}-ala &  vú{\bfseries m}-a & 	&  		& 	y/m \\
sĩ́ĩ̀ 			& `approach' & sí{\bfseries ng}-ala & 		& 		&  sí{\bfseries s}-ɛlɛ & ng/s \\
níyɛ			& `be beautiful' & ní{\bfseries ndy}-ala  & 		& ní{\bfseries ng}-ɛsɛ	& 		& ng/ndy  \\
vè'è 			& `try on clothes' &  vè{\bfseries g}-ala & 	& 		&  vè{\bfseries '}ɛlɛ  & 	g/'  \\
dyà			& `sing'    & dyà-ala		& dyà{\bfseries y}-a & 	& 		& 		y/none \\
kwê 			& `fall' 	&  kwé{\bfseries y}-ala & 		&  kù-ɛsɛ  & 		& 	y/none \\
dã̂ 			& `draw water' &  dà{\bfseries ng}-ala &  dã̀-ã̀la & 	&  dã̀-ã̀lɛ	& 	ng/none \\
dyɔ̀			& `laugh'		& dyɔ̀-ala		& dyɔ̀{\bfseries l}as-a & dyɔ̀{\bfseries l}-ɛsɛ &  & l/none \\ 
dè 			& `eat' 	& dì{\bfseries y}-ala & dí{\bfseries b}-a &  dí{\bfseries l}-ɛsɛ & 	& 	y/b/l \\
 \lspbottomrule
\end{tabular}}
\caption{Root-final consonant variants (monosyllabic verbs)}
\label{Tab:RFCVmono}
\end{table} 

Root-final consonant variants likely occur for the same reason that root-final consonants take different shapes generally. Gyeli probably had more derivation suffixes diachronically and possibly allowed more suffixes than the synchronic limit of three syllables. Different final root consonants may reflect remnants of former extension suffixes or diachronic stacking of derivation suffixes. For instance, /l/ could be related to the expansion suffix -{\itshape lɛ}, as discussed in \sectref{sec:DiaEx}. /s/ in {\itshape sís-ɛlɛ} `approach sb.' could be related to the causative suffix -{\itshape ɛsɛ}. 

Other variant forms may rather reflect an ongoing reduction of segmental material, as in {\itshape vè'è} `try on clothes', which has retained a probably older final consonant /g/ in its reciprocal form {\itshape vèg-ala} that got reduced to a glottal stop in the monosyllabic and applicative forms. The next step on the continuum of segmental reduction is the complete loss of the final root consonant.

Final root consonant variants also occur with disyllabic verb roots, but they are less frequent. \tabref{Tab:RFCVbi} shows all their occurrences found in the database.

\begin{table}
\fittable{\begin{tabular}{ll llll l}
  \lsptoprule
\multicolumn{2}{c}{Underived form}  & Reciprocal         & Passive & Causative & Applicative & Variants \\
 \midrule
dyú{\bfseries w}ɔ & `hear' &  dyù{\bfseries w}-ala & 	&   dyú{\bfseries g}-ɛsɛ & dyú{\bfseries w}-ɛlɛ & w/g \\
lí{\bfseries y}ɛ & `leave' &  lí{\bfseries g}-ala & 	& 	& 	& y/g \\
vò{\bfseries w}a & `wake up' &  vò{\bfseries w}-ala	& 	&  vò{\bfseries l}-ɛsɛ 	& 	& w/l \\
tì{\bfseries n}ɔ &  `harvest tubers' &  tì{\bfseries n}-ala &  tì{\bfseries l}-ɛ & 	& 	& n/l \\
bí{\bfseries y}ɔ &  `hit' &  bí{\bfseries n}-ala &  bí{\bfseries l}-a &  bí{\bfseries l}-ese &  bí{\bfseries y}-ɛlɛ & y/n/l \\
 \lspbottomrule
\end{tabular}}
\caption{Root-final consonant variants (disyllabic verbs)}
\label{Tab:RFCVbi}
\end{table} 

%Comparatively, disyllabic verb roots allow for a larger variety of root consonants, including frequently also /m/, /d/, /mb/, /nd/, and /ŋg/.

The second exception concerns the lack of a root-final consonant in which case adjacent vowels are allowed. Only one verb is known that has a derived form with a zero final root consonant, but no variant consonant in another derived form: {\itshape bvû} `think' whose reciprocal form is {\itshape bvúala}.
In all the other cases of zero root-final consonants, there is another consonant variant in another derived form. The variants of zero-consonant and root-final consonant in derived verb forms are listed in \tabref{Tab:RFCVmono}. Other variants of zero-consonants do not show in derived verbs, but in the nominalized past participle ({\NP}P) forms, which are discussed in \sectref{sec:NOMPart}. All instances  of variants showing up only in the nominalized past participle are given in \tabref{Tab:zeroRFC}.


\begin{table}
\small
\begin{tabularx}{\textwidth}{lX lX l}
\lsptoprule
\multicolumn{2}{c}{Underived form}  & Reciprocal         & {\NP}P  & Variants \\
 \midrule
dyâ			& `lie down'    &  dyá-ala      	& ndyá{\bfseries y}-â     & none/y \\
sâ			& `do'              &  sá-ala      	& nsá{\bfseries y}-â     & none/y \\
yíɛ̀			& `avoid'         &  yé-ala      	& nyé{\bfseries y}-â     & none/y \\
kã̂ 			& `wrap' 	       &  kã́-ala              & nkã́{\bfseries l}-â	     &   none/l \\
láà			& `tell'		& lá-ala		& nlá{\bfseries w}-â        & none/w \\ 
 \lspbottomrule
\end{tabularx}
\caption{Zero root-final consonant variants in nominalized past participles}
\label{Tab:zeroRFC}
\end{table} 

As described in \sectref{sec:SyllV}, there is some variation in the production of vowel sequences in verb stems. While synchronically vowel sequences are found in verb stems, these have  alternate forms with a glottal stop, as illustrated by the two variants in \REF{RFCglottal}.

\ea \label{RFCglottal}
  \ea  múɛlɛ `nibble'  → mú-ala ({\RECIP}) → mú-ɛsɛ ({\CAUS})
\ex mú{\bfseries '}ɛlɛ `nibble'  → mú{\bfseries '}-ala ({\RECIP}) → mú{\bfseries '}-ɛsɛ ({\CAUS})
\z
\z

\noindent The exact distribution of one variant in comparison to the other is not known. There is variation across speakers as well as within the same speaker. This tendency, however, seems to align with the loss of segmental material posited for other verb forms.

 






\subsection{Verb types}
\label{sec:StructTypes}

I distinguish three different verb types in Gyeli, based on their morphosyntactic behavior: main verbs, auxiliary verbs, and light verbs. I define and describe each of these and their potential subtypes in turn.


\subsubsection{Main verbs}
\label{sec:MainVerbs}

I view the main verb as the lexical verb in a phrase which, according to \citet[796]{anderson2011a}, contributes lexical content to an expression.  The main verb in Gyeli always serves as the semantic head of a clause, but is only the syntactic, finite head in simplex predicate constructions. In complex predicate constructions, the syntactic head is an auxiliary or semi-auxiliary (\sectref{sec:AUX}), while the main verb appears in its non-finite form.  In contrast to true auxiliaries, main verbs can occur on their own in a simplex predicate construction. 


In the simple predicate construction in \REF{simPred}, the main verb {\itshape gyɛ́sɔ́} `look for' is the syntactic and semantic head of the clause.

\ea \label{simPred}
  \glll  mùdã̂ á {\bfseries gyɛ́sɔ́} bédéwɔ̀ \\
        m-ùdã̂ a-H gyɛ́sɔ-H H-be-déwɔ̀  \\
           \textsc{n}1-woman 1-\textsc{prs} look.for-{\R} {\OBJ}.{\LINK}-be8-food  \\
    \trans `The woman looks for food.'
\z

\noindent As the syntactic head, the main verb is inflected for its tense-mood category, as described in \sectref{sec:GramTM}. In this case, {\itshape gyɛ́sɔ́} `look for' is a finite form, carrying a realis-marking H tone.

In contrast, in a complex predicate construction, the main verb is the semantic head of the clause.  An auxiliary or light verb serves as the syntactic head, as exemplified in \REF{simPredb} with the negative subjunctive auxiliary verb {\itshape dúù}. In this example, the auxiliary is the finite verb encoding the tense-mood category it belongs to. The main verb takes its non-finite form, namely with an underlyingly toneless final vowel, as described in \sectref{sec:toneless}.

\ea \label{simPredb}
  \glll  mùdã̂ á {\bfseries dúù} {\bfseries gyɛ́sɔ̀} bédéwɔ̀ \\
        m-ùdã̂ a-H dúù gyɛ́sɔ H-be-déwɔ̀  \\
           \textsc{n}1-woman 1-\textsc{prs} {\NEG}.{\SBJV} look.for {\OBJ}.{\LINK}-be8-food  \\
    \trans `The woman must not look for food.'
\z

%\subsubsection{Infinitives}

The non-finite form in \REF{simPredb} is, at the same time, the {\bfseries infinitive} form. Infinitive forms in Gyeli do not receive any special morphological or tonal marking, but are identical to their citation form. As shown in \sectref{sec:Tinventory}, second and third syllables are underlyingly toneless, surfacing with an L tone. Infinitive forms are found in complex predicates \REF{simPredb} as well as two types of subordinate clauses. First, they occur in subordinate infinitival clauses (\sectref{sec:InfSub}), as in \REF{INFPred1}.

\ea \label{INFPred1}
  \glll    [{\bfseries pámɔ̀} tísɔ̀nì]\textsubscript{{\INF}} á súmɛ́lɛ́ bùdì \\
            {\db}pámɔ tísɔ̀nì a-H súmɛlɛ-H b-ùdì \\
             {\db}arrive $\emptyset$7.town 1-\textsc{prs} greet-{\R} ba2-people   \\
    \trans `Having arrived in town, he greets the people.'
\z

\noindent These subordinate infinitival clauses can also be negated with the negative auxiliary {\itshape tí}, as in \REF{INFPred2}.

\ea \label{INFPred2}
  \glll  à múà nà bábɛ̀ [tí {\bfseries wúmbɛ̀} wɛ̀] \\
          a múà nà bábɛ̀ {\db}tí wúmbɛ wɛ̀   \\
         1 be {\COM} $\emptyset$7.illness {\db}{\NEG} want-{\R} die \\
    \trans `He was sick, not wanting to die.'
\z


\noindent And second, the main verb of certain attributive clauses with the complementizer {\itshape nâ} appears in its infinitival form, as shown in \REF{INFPred3} and explained in more detail in \sectref{sec:COMPINF}.

\ea \label{INFPred3}
  \glll mùdã̂ à lɔ́ sìsɛ̀lɛ̀ nɔ́nɛ́gá [nâ nyɛ̂ nà {\bfseries kɔ́sɛ̀}] \\
m-ùdã̂ a lɔ́ sìs-ɛlɛ n-ɔ́nɛ́gá {\db}nâ nyɛ̂ nà kɔ́sɛ \\
\textsc{n}1-woman 1.{\PST} {\RETRO} scare-{\AP}PL 1-other {\db}{\COMP} 1 {\COM} cough     \\
  \trans `The woman scared the other by her coughing.'
\z


Infinitives are also found in non-verbal clauses where the infinitive is linked with the \textsc{stamp} copula {\itshape yíì} of agreement class 7 to its predicate, as shown in \REF{INFcop}. This construction is further described in \sectref{sec:COP}

\ea \label{INFcop}
  \glll  jíwɔ̀ yíì bíwɔ̀ \\
        jíwɔ yíì bíwɔ̀ \\
           steal 7.{\COP} bad  \\
    \trans `To steal is bad.'
\z

\noindent Verbal clauses are discussed in \sectref{sec:verbalC} and complex predicates are explained in more detail in \sectref{sec:CompPred}.

In contrast to other types of verbs, lexical verbs take a range of different valencies (intransitive, transitive, ditransitive), as illustrated in \REF{Val}.

\ea \label{Val}
  \ea{\label{Vala}
   \glll   Màmbì à {\bfseries kɛ́} \\
          Màmbì a kɛ̀-H \\
            $\emptyset$1.{\PN} 1.{\PST}1 fall-{\PST}\\}\jambox*{(intransitive)}
    \trans `Mambi fell.'
\ex{\label{Valb}
\glll  Màmbì à {\bfseries bɛ́} lé \\
       Màmbì a bɛ̀-H lé \\
           $\emptyset$1.{\PN} 1.pst plant-{\R}  $\emptyset$7.tree\\}\jambox*{(transitive)}
    \trans `Mambi planted a tree.'
\ex{\label{Valc}
\glll  Màmbì à {\bfseries vɛ́} Bìyã́ màntúà\\
       Màmbì a vɛ̀-H  Bìyã́ ma-ntúà \\
           $\emptyset$1.{\PN} 1.{\PST} give-{\R}  $\emptyset$1.{\PN} ma6-mango \\}\jambox*{(ditransitive)}
    \trans `Mambi gave Biyang mangoes.'
 \z
\z

The valency of a verb is lexically specified, but can also be changed through verb extensions, which are explained in \sectref{sec:VDeriv}. Valency change and verb extensions also relate to different voices a main verb can express, such as active, middle voice, and passive voice. Examples of each are shown in \REF{Vale}.

\ea \label{Vale}
  \ea  \label{Valea}
   \glll  Màmbì à {\bfseries vìdɛ́} màtúà  \\
          Màmbì a vìdɛ-H màtúà \\
          $\emptyset$1.{\PN} 1.{\PST} turn-{\R} $\emptyset$1.car   \\
    \trans `Mambi turned the car.'
\ex\label{Valeb}
\glll  màtúà à {\bfseries vìdɛ́gá} \\
      màtúà a vìd-ɛga-H  \\
       $\emptyset$1.car 1.{\PST} turn-{\AUTOCAUS}-{\PST}     \\
    \trans `The car turned (around).'
\ex\label{Valec}
\glll  màtúà à {\bfseries vìdá} (nà Màmbì) \\
      màtúà a vìd-a-H nà Màmbì \\
        $\emptyset$1.car 1.{\PST} turn-{\PASS}-{\PST} {\COM} $\emptyset$1.{\PN}    \\
    \trans `The car was turned (by Mambi).'
 \z
\z

\subsubsection{Special cases of main verbs}

There are two subtypes of main verbs, namely main verbs that require a preposition with their object and main verbs that require a cognate object.
Main verbs requiring a preposition with their object argument are generally rare with only 14 verbs (3.7\%) of the 377 verbs in the database. In most cases, the comitative preposition {\itshape nà} is required. All twelve cases are listed in \tabref{Tab:Verbsna}.

\begin{table}
\begin{tabular}{ll}
 \lsptoprule
báàla nà &	 `repeat sth.' \\
bága nà & 	`stop sth.' \\
bísi nà & 	`pay attention to' \\
bvúda nà & 	`quarrel' \\
gyíka nà & 	`resemble sb./sth.' \\
kàmbɔ nà	 & `defend sth.' \\
làdo nà & 	`meet sb.' \\
náàta nà	&  `stick to' \\
njì nà & 	`bring, come for' \\
tã́ã̀la nà & 	`judge sb.' \\
túwanɛ nà	 & `meet with sb.' \\
vúba nà & 	`hug sb.' \\
 \lspbottomrule
\end{tabular}
\caption{Main verbs requiring the comitative {\itshape nà}}
\label{Tab:Verbsna}
\end{table}

The other preposition that links an argument is the directional {\itshape bà}. It occurs only in two verbs of very similar meaning in the database, namely {\itshape sĩ́ĩ̀ bà} `approach sth.' and {\itshape sísɔ bà} `approach sth.'
Obviously, the prepositions {\itshape nà} and {\itshape bà} occur more frequently in the text corpus, but they are usually found in adjunct noun phrases.

Gyeli has a few verbs that take a cognate object as argument, as in \REF{Vcognate} where the verb is marked in bold.

\ea \label{Vcognate}
  \ea  {\bfseries gyá} gyà `sing (a song)'
\ex {\bfseries sá} sálɛ́ `work (a work)'
\ex {\bfseries kɛ́} kɛ̀ndɛ̀ `walk (a walk)'
\z
\z

All these verbs can also take a different lexeme as an object, as for instance, in \REF{Cogna}. They cannot appear without an object, as \REF{Cognb} shows.

\ea \label{Cogn}
  \ea  \label{Cogna}
   \glll  mɛ́ kɛ́ tísɔ̀nì  \\
          mɛ-H kɛ̀-H tísɔ̀nì \\
         1\textsc{sg}-\textsc{prs} go-{\R} $\emptyset$7.town    \\
    \trans `I go to town.'
\ex[*]{\label{Cognb}
\glll  mɛ́ kɛ̀ \\
      mɛ-H kɛ̀  \\
      1\textsc{sg}-\textsc{prs} go    \\
    \trans `I walk.'}
 \z
\z

At the same time, the cognate objects can also appear with other verbs, as shown in \REF{CognO}.

\ea \label{CognO}
  \glll yɔ́ɔ̀ bá {\bfseries téé} {\bfseries kɛ̀ndɛ̀} \\
        yɔ́ɔ̀ ba-H téè-H kɛ̀ndɛ̀ \\
         so 2-\textsc{prs} start.walking-{\R} $\emptyset$7.walk   \\
    \trans `So they go on a walk.'
\z



\subsubsection{Auxiliaries and semi-auxiliaries}
\label{sec:AUX}

A set of verbs in Gyeli occur as the finite verbal element in a complex predicate construction without (fully) contributing to its lexical content. (Complex predicate constructions are discussed in \sectref{sec:CompPred}.) I call these verbs ``auxiliaries'', which I subdivide into true auxiliaries and semi-auxiliaries.  They both precede the lexical verb. \REF{G114} illustrates the contrast between a complex predicate with a semi-auxiliary (the modal {\itshape yánɛ} `must') in \REF{G114a} and its simplex predicate counterpart in \REF{G114b}. In the complex predicate construction, the semi-auxiliary {\itshape yánɛ} is inflected for tense-mood (see \sectref{sec:GramTM}), while the lexical verb {\itshape dyâ} `lie down' appears in its non-finite form. In the simplex predicate construction, the lexical verb receives the tense-mood marking H tone.

\ea \label{G114}
  \ea \label{G114a}
  \glll  mɛ̀ {\bfseries yánɛ́} {\bfseries dyâ} vâ kùgúù dẽ̀ màfú mábáà \\
     mɛ yánɛ-H dyâ vâ kùgúù dẽ̀ ma-fú má-báà \\
          1\textsc{sg} must-{\R} lie.down here $\emptyset$7.evening today ma6-day 6-two  \\
    \trans `I had to sleep here in the evening two days ago [from today].'
\ex\label{G114b} 
  \glll  mɛ̀ {\bfseries dyá} vâ kùgúù dẽ̀ màfú mábáà \\
     mɛ  dyâ-H vâ kùgúù dẽ̀ ma-fú má-báà \\
          1\textsc{sg}.{\PST} lie.down-{\R} here $\emptyset$7.evening today ma6-day 6-two  \\
    \trans `I slept here in the evening two days ago [from today].'
\z
\z

True auxiliaries and semi-auxiliaries can be distinguished along two parameters, as shown in \tabref{Tab:AUX}: (i) full conjugation potential across different tense-mood categories vs.\ restrictions thereof and (ii) full lexical meaning vs.\ no lexical meaning. True auxiliaries are restricted in the tense-mood category they can appear in, as detailed in \sectref{sec:ComplAUX}, as well as in the verbal predicate type they occur in: true auxiliaries can never appear on their own in a simple predicate construction, but require the addition of a lexical verb. Semi-auxiliaries, in contrast, have full lexical meaning and the potential to serve as the finite element in a simple predicate construction. They have full conjugation potential across all tense-mood categories in both simple and complex predicate constructions.


\begin{table}
\begin{tabularx}{\textwidth}{l QQ}
\lsptoprule
			& 	Full inflection 	& Inflectional restrictions \\
 \midrule
No lexical  & 	& {\bfseries True auxiliaries} \\
meaning		  & 	& {\itshape nzíí} {\PROG}.\textsc{prs},
		  {\itshape nzí} {\PROG}.{\PST},
		  {\itshape nzɛ́ɛ́} {\PROG}.{\SUBORD},
		  {\itshape lɔ̀} {\RETRO},
		  {\itshape sàlɛ́} {\NEG}.{\PST},
		  {\itshape pálɛ́} {\NEG}.{\PST},
		  {\itshape kálɛ̀} {\NEG}.{\FUT},
		  {\itshape tí} {\NEG}.{\IMP}  \\
				\midrule
Lexical 	 & {\bfseries Semi-auxiliaries} & \\
meaning                 & 	{\itshape kɛ̀} `go', {\itshape lã̀} `pass'
			{\itshape njì} `come', {\itshape lígɛ} `stay',
			{\itshape sílɛ} `finish', {\itshape pã̂} `do first',
			{\itshape táalɛ} `begin', {\itshape bàga nà} `stop'
                        {\itshape lèmbɔ} `know', {\itshape kwàlɛ} `like',
			{\itshape wúmbɛ} `want', ({\itshape yánɛ} `must')    &   {\itshape bwàá} `have',
                                                                                {\itshape múà} `be almost',
                                                                                {\itshape dúù} `must not' \\
\lspbottomrule
\end{tabularx}
\caption{Auxiliary types}
\label{Tab:AUX}
\end{table}


Since tense-mood categories are only marked tonally, but true auxiliaries are restricted to specific categories, it cannot be proven that they take tonal inflection instead of having a fixed tonal pattern, as there are no contrastive pairs. There are several reasons, however, to classify true auxiliaries as finite verbal elements. First, their tonal patterns coincide with the tonal patterns of their respective tense-mood category. Second, they occur in the same position as semi-auxiliaries that clearly inflect for tense-mood tonal marking. Third, they are followed by a non-finite lexical verb.

Semi-auxiliaries and true auxiliaries can be thought of as distributed towards opposite ends of a grammaticalization scale. Semi-auxiliaries are closest to main verbs while true auxiliaries are highly grammaticalized. 
While most (semi-) auxiliaries fall neatly in either one of the auxiliary types, there are nevertheless some exceptions which behave slightly differently, reflecting their different stages on the grammaticalization path. This is the case for {\itshape dúù} `must not', which is restricted to present and subjunctive clauses and cannot appear as the finite verb in a simple predicate but, unlike true auxiliaries, it has a lexical meaning.\footnote{Lexical meaning is based on speaker intuition. Speakers are entirely consistent in ascribing the meaning {\itshape ne pas devoir}  `must not' to {\itshape dúù}, and identify the word as the counterpart of {\itshape yánɛ} `must'. In contrast, speakers find it very difficult to describe what true auxiliaries mean.} The same is true for {\itshape bwàá} `have' with its restriction to the two past categories, and {\itshape múà} `be almost' with its restriction to the future.
Another outlier within the semi-auxiliaries is the deontic modal {\itshape yánɛ} `must', which is the only one that cannot appear in a simple predicate construction. In this respect, it patterns with true auxiliaries, but has a lexical meaning like semi-auxiliaries. Since it has no tense-mood category restrictions, I classify it as a semi-auxiliary.

Both true auxiliaries and semi-auxiliaries encode elements of various functional domains, i.e.\ there is no one-to-one mapping from their form to one specific function.  True auxiliaries comprise some aspect markers and all negation auxiliaries. Semi-auxiliaries also encode some aspect markers as well as modality and motion/posture verbs.




%Light verbs: Language-internal criteria (distributional differences) to distinguish light verbs from main and auxiliary verbs.
%light verbs contribute to a semantically complex but syntactically monoclausal predication and [...] form a syntactically distinct class. \citet[64]{butt2010}

%go, come, finish, give a scream

%\begin{quote}Even though the light verbs clearly do not have the same predicational content as their full/main verb counterparts, they are always exactly form identical to a full verb and inflect exactly like that full verb. This characteristic sets light verbs apart from auxiliaries in terms of historical change, as auxiliaries may be form identical to a full verb at the initial stages of reanalysis from verb to auxiliary, but then quickly tend to develop away from the original form of the full verb. \citet[53]{butt2010} \end{quote}












\section{Adjectives}
\label{sec:QUAL}


Gyeli has a small set of adjectives, as listed in \tabref{Tab:QUAL}.\footnote{There is one other nominal modifier that semantically expresses a quality, which is {\itshape nyá} `big'. As it differs structurally from the adjectives presented in this section, I discuss it in \sectref{sec:nya}.} They constitute a closed class in Gyeli and denote properties of the noun such as value, dimension, and color.

\begin{table}
\begin{tabular}{lll}
 \lsptoprule
% \multicolumn{2}{l}{quality} \\
 {quality} & mpà &  `good' \\
 & bíwɔ̀ &  `bad' \\
 & díyɛ̀  & `expensive' \\
 & pówàlà & `calm' \\
 & nátĩ̂ & `straight' \\
 \midrule
% \multicolumn{2}{l}{size} \\
 {size}  & píyɔ̀ &  `small' \\
 & nɛ́nɛ̀ &  `big' \\
 \midrule
% \multicolumn{2}{l}{color} \\
 {color} & námbàmbàlà & `white' \\
 & návyûvyû & `black' \\
 & nábèbè & `red' \\
 & nápfûpfû & `darkened color' \\
 & náyɛ̂yɛ̂ & `lightened color' \\
 \lspbottomrule
\end{tabular}
\caption{Adjectives}
\label{Tab:QUAL}
\end{table}

Morphosyntactically, adjectives can be clearly delimited from other parts of speech such as nouns and verbs. Adjectives do not exhibit any verbal qualities such as combining with a \textsc{stamp} marker or an aspect marker. They can also be clearly distinguished from nouns as they do not exhibit (most) typical nominal behavior. First, they do not take a singular and/or plural form. Second, they do not have the possibility of being modified by other elements of a noun phrase such as demonstratives or possessor pronouns. They can, however, serve as the head of an attributive construction, as further explained below.

This word class in Gyeli meets the broad criteria for adjectives given in the typological literature (which often mixes semantic and morphosyntactic criteria), for instance, following \citet[16]{bhat94} in terms of ``(i) their belonging, prototypically, to the semantic class of properties, and (ii) their having modification (of a noun) as the primary (categorial) function''. 
\citet{dixon2004}, who postulates that every language has a class of adjectives which is distinct from nouns and verbs, adds to this list predicative use of adjectives, for example as a copula complement. 

Besides these broad criteria, however, adjectives form a vastly diverse class cross-linguistically, as for instance pointed out by \citet{segerer2008} for adjectives in African languages. Gyeli adjectives are unusual from a Bantu perspective in that they do not take any agreement prefixes, but are invariable in their form, both in attributive and predicative use.

In attributive use, adjectives modify nouns in two different default constructions, as shown in \REF{ADJtemp}. Either the adjective directly follows the head noun or it appears as the second constituent in an attributive construction where the attributive marker agrees with the head noun.

\ea \label{ADJtemp}
    \ea {}[{\N} {\ADJ}]
    \ex {}[{\N} {\ATT} {\ADJ}]
    \z
\z

\noindent Examples of both construction types are given in \REF{qualia} and \REF{qualib}, respectively.

\ea \label{qualia}
  \ea \label{qualia1}
 \gll  nkɔ́lɔ̀ mpà  \\
          $\emptyset$3.watch good  \\
    \trans `a/the good watch'
\ex \label{qualia2}
  \gll   nkɔ́lɔ̀ nɛ́nɛ̀ \\
          $\emptyset$3.watch big \\
    \trans `a/the big watch'
\ex \label{qualia3}
  \gll    nkɔ́lɔ̀ nábèbè \\
          $\emptyset$3.watch red \\
    \trans `a/the red watch'
\z
\ex \label{qualib}
  \ea \label{qualib1}
 \gll  nkɔ́lɔ̀ wá mpà  \\
          $\emptyset$3.watch 3:{\ATT} good  \\
    \trans `good watch'
\ex \label{qualib2}
  \gll   nkɔ́lɔ̀ wá nɛ́nɛ̀ \\
          $\emptyset$3.watch 3:{\ATT} big \\
    \trans `big watch'
\ex \label{qualib3}
  \gll    nkɔ́lɔ̀ wá nábèbè \\
          $\emptyset$3.watch 3:{\ATT} red \\
    \trans `red watch'
\z
\z

\noindent Constructions that either take or optionally omit the attributive marker are discussed in \sectref{sec:CONC}.

The order of adjective and noun can also be reversed, as a  more marked form. The adjective can either precede the noun directly or can appear as the head of an attributive construction in which case the attributive marker takes the default agreement form of class 7. Choices between construction forms
usually entail a change in meaning, as shown in \REF{qualif1}.

\ea \label{qualif1}
  \ea \label{qualif1a}
 \gll  sɔ́ wà nɛ́nɛ̀  \\
          $\emptyset$1.friend 1:{\ATT} big  \\
    \trans `big friend'
\ex \label{qualif1b}
  \gll   sɔ́ nɛ́nɛ̀\\
          $\emptyset$1.friend big \\
    \trans `important friend'
\ex \label{qualif1c}
  \gll    nɛ́nɛ̀ yá sɔ́ \\
          big 7:{\ATT} friend \\
    \trans `big size of the friend'
\z
\z

It is difficult to detect the exact meaning contrast present. It depends on the lexical semantics of the adjective and noun in question and the construction they stand in. Another example of meaning contrast across different construction types is given in \REF{qualif2}. While the use of the attributive marker is optional in both constructions, it is preferred in \REF{qualif2a} and dispreferred in \REF{qualif2b}

\ea \label{qualif2}
  \ea \label{qualif2a}
 \gll  m-wánɔ̀ (wà) bíwɔ̀  \\
          \textsc{n}1-child 1:{\ATT} bad  \\
    \trans `bad child [bad character traits]'
\ex \label{qualif2b}
  \gll   bíwɔ̀ (yá) m-wánɔ̀\\
          bad 7:{\ATT} \textsc{n}1-child \\
    \trans `ugly child'
\z
\z

There are also examples where a switch of constituents does not seem to change the meaning as speakers state that both mean exactly the same, as in \REF{qualif3} and \REF{qualif4}, although in these cases both constituents are clearly nouns, which have a plural form and which can be modified by demonstratives and possessor pronouns.

\ea \label{qualif3}
  \ea \label{qualif3a}
 \gll  nkwɛ̌ (wá) nkpámá  \\
          $\emptyset$3.basket 3:{\ATT} $\emptyset$3.newness  \\
    \trans `new basket'
\ex \label{qualif3b}
  \gll   nkpámá (wá) nkwɛ̌\\
          $\emptyset$3.newness 3:{\ATT} $\emptyset$3.basket \\
    \trans `new basket'
\z
\z

\ea \label{qualif4}
  \ea \label{qualif4a}
 \gll  m-ùdì (wà) nkángɛ̀\\
          \textsc{n}1-person 1:{\ATT} $\emptyset$3.courage  \\
    \trans `courageous person'
\ex \label{qualif4b}
  \gll   nkángɛ̀ (wá) m-ùdì\\
          $\emptyset$3.courage 3:{\ATT} \textsc{n}1-person \\
    \trans `courageous person'
\z
\z


In predicative use, the adjective serves as the copula complement as shown in \REF{qualif5}.

\ea \label{qualif5}
\gll m-àmbɔ̀ máà {\bfseries mpà} \\
	ma6-thing 6:COP good\\
    \trans `Things are good.'
\z


\noindent The adjective clearly shows no agreement morphology, although this would be expected with all plural classes. The same is true for an adjectival complement in a negative non-verbal construction, as in \REF{qualif6}.

\ea \label{qualif6}
\glll  mìnsáyá mí bèyá sâ mí bɛ́lɛ́ {\bfseries mpà}\\
      mi-nsáyá mí bèya-H sâ mi-H bɛ́-lɛ́ mpà\\
	mi4-thing 4:{\ATT} 2\textsc{pl}-\textsc{prs} do 4-\textsc{prs} be-{\NEG} good  \\
    \trans `The things that you do are not good.'
\z


Adjectives can be used as parameters of comparison in comparison constructions, as described in \chapref{sec:CCx}.  They are, however, not marked morphologically in these constructions. Finally, they can also be used adverbially to modify a verb, as discussed in \sectref{sec:G3ADV}.




Some special remarks are in order for color adjectives. As shown in \tabref{Tab:QUAL}, all color term adjectives (and the quality adjective {\itshape nátĩ̂} `straight') have in common that they start with the similative marker {\itshape ná-}, as described in \sectref{sec:NOMSIM}. There is evidence that,  historically, color terms in at least some related languages of this area were verbs. These verbs used for color descriptions then developed into other parts of speech. For instance, in Bulu the basic color terms are synchronically nouns: {\itshape évìndì} `black', {\itshape évèlè} `red', and {\itshape éfùmùlù} `white'.\footnote{\citet{bates04} gives the verbal color forms for Bulu as follows: {\itshape vé} ‘be/get red’, {\itshape vìn} ‘be/get black’, and {\itshape fùm} ‘be white’ without mentioning any nominal color forms. \citet[44]{alexandre55} explains that these verbs can take a causative suffix {\itshape vìn} `be black' $\rightarrow$ {\itshape vìn-ì} `make black'. These causative verbs were then nominalized and assigned to noun class 5 with the prefix {\itshape é-}. \citet[68]{alexandre55} states that this class usually hosts deverbal nouns derived from stative verbs.} In Gyeli, it is likely that such color verbs were grammaticalized, together with the {\itshape ná} similative marker, into a synchronic uninflected element of the noun phrase.

Another argument that color adjectives are grammaticalized verbs including a similative marker comes from the atypical terms {\itshape nápfûpfû} `darkened color' and {\itshape náyɛ̂yɛ} `lightened color', which describe a change of color as opposed to a specific hue. When asked for the meaning of these atypical colors, speakers give a verbal explanation, namely that a more prototypical color such as `black', `white', or `red' has changed by either having become darker ({\itshape nápfûpfû}) or lighter, being `bleached out' ({\itshape náyɛ̂yɛ}). In contrast, other colors are referred to by French adjectives in explanations.

According to traditional color theories, these two special color terms are unusual in that they do not fit into basic color words that have been investigated cross-linguistically (see, for instance, \citealt{berlin69}). 
Nevertheless, I classify {\itshape nápfûpfû} `darkened color' and {\itshape náyɛ̂yɛ} `lightened color' as color terms since they only show up in discourse when talking about colors and they were systematically used by speakers in the color booklet task \citep{majid2007a}.\footnote{Gyeli has more color terms than the adjectives listed in \tabref{Tab:QUAL}. Other color terms include, for instance, {\itshape mpùlɛ́} `yellow', which is derived from the name of a tree with yellow bark ({\itshape Enantia chlorantha}), or {\itshape màká} `green', which is a noun also means `leaves'. Those other color terms are, however, recently acquired and differ in their morphosyntactic status in that they are nouns rather than adjectives, as further explained in \citet{grimm2014}.}







\section{Adverbs}
\label{sec:ADV}

%GIVE SHORT VERSION OF {\ADV}ERB SECTION {\AND} THINK ABOUT HOW {\TO} {\DIST}RIBUTE {\INF}O IN SENSIBLE WAY

%(add lii)

%ná still, again, anymore: change from sentential modifier to adverb??? 




Adverbs, along with nouns, verbs, and adjectives, constitute an open part-of-speech class. According to \citet[20]{schachter2007}, adverbs may have various subclasses, such as directional adverbs (`down'), degree adverbs (`extremely'), manner adverbs (`quickly'), time adverbs (`today'), or sentence adverbs (`unfortunately'). These subclasses show that adverbs do not necessarily modify verbs, but may also modify adjectives or other adverbs or even whole sentences. \citet[20]{schachter2007} thus provide a broad definition of adverbs as elements which ``function as modifiers of constituents other than nouns''. 

In general, the class of adverbs in Gyeli is rather restricted in diversity, just as in many other Bantu languages. Thus, in the Gyeli text corpus, as described in \sectref{sec:Data},
 fewer than 20 different adverbs occurred. One reason for this is that, according to  \citet[126]{creissels2008}, in many African languages, ``the possibility of deriving manner adverbs from other categories or to use adjectives as verb modifiers, is very limited''.  This is also true for Gyeli where the meaning of typical English manner adverbs is instead expressed by ideophones, as will be discussed in \sectref{sec:IDEO}, or by nouns in complement position, as in \REF{maNCOMP}.


\ea \label{maNCOMP}
  \glll     màlɛ́ndí máà vɛ̀ɛ̀ kwè mípìndí  \\
	ma-lɛ́ndí máà vɛ̀ɛ̀ kwè H-mi-pìndí \\
              ma6-palm.nut 6.{\DEM}.{\PROX} only fall {\OBJ}.{\LINK}-mi4-unripeness   \\
    \trans `These palm nuts only fall when they are unripe.'
\z

\noindent Despite this restricted diversity, Gyeli adverbs occur pervasively in all types of text genres (dialogues, folktales, autobiographic narratives). Almost a quarter of all intonation phrases in the Gyeli text corpus (123 (23\%) of 540 intonation phrases) include an adverb.

Gyeli adverbs are invariable and do not receive any specific morphological marking, e.g.\ through suffixes, like the English -{\itshape ly} or French -{\itshape ment}. Subclasses of adverbs can be distinguished through several morphosyntactic properties and/or a combination of them. I will consider the following three subclasses as described by their most salient characteristics:

\begin{description}
\item[Group 1:] adverbs optionally combining with {\LOC} preposition {\itshape ɛ́} 
\item[Group 2:] adverbs that can occur in noun + attributive constructions 
\item[Group 3:] adverbial lexemes that can act as nominal modifiers in {\NP}s 
%\item[Group 4:]adverbial lexemes that occur as nouns in {\NP}s 
%\item[Group 5:]adverb that does not exhibit any of the above mentioned properties
\end{description}
%\caption{Main characteristics of adverb groups}
%\label{Tab:ADV1}
%\end{table}

Subclassification of adverbs in the literature is typically done on a semantic basis, such as manner, temporal or locative adverbs. The choice of semantic categories may, however, be arbitrary and may not match the morphosyntactic categories of a language. In Gyeli, morphosyntactic classes map onto semantic categories, as shown in \tabref{Tab:ADV2}.  Group 1 consists entirely of deictic adverbs which include locative and manner deictics.  Group 2 hosts temporal adverbs and group 3 contains manner adverbs.%, and group 4 locative and directional adverbs. Group 5 only has one member, namely an anaphoric adverb.

Nevertheless, the defining criteria for adverbial subclasses in Gyeli are four morphosyntactic properties as listed in the column names of \tabref{Tab:ADV2}: (i) the potential combination with the locative {\itshape ɛ́}, (ii) use of a lexeme as both adverb modifying a verb and adjective/quantifier modifying a noun, (iii) occurrence in noun + attributive marker construction, and (iv) occurrence in phrase-final position only. The last column also provides information on the derivational source of the adverbs. Yet, since this is not a morphosyntactic property, it does not determine adverbial classification.


\begin{table}
\begin{tabularx}{\textwidth}{l C@{\hspace{.5\tabcolsep}}c@{\hspace{.5\tabcolsep}}C@{\hspace{.5\tabcolsep}}C@{\hspace{.5\tabcolsep}}C@{\hspace{.5\tabcolsep}}C}
  \lsptoprule
Group & Semantic core& {\LOC} {\itshape ɛ́} & \ADJ{\slash}{\QUANT} & {\ATT} constr. & only final position   & derivational source    \\
           \midrule
       1  &      deictic  &      x           & --          &      (x)          &  --         & underived    \\
       2a  &      temporal &   --          & --         &     x            & --          & underived   \\
       2b  &      temporal &   --          & --         &     x            & --          & denominal  \\
       3  &      manner &    --           &  x          &      --         & x           &  \ADJ/{\QUANT}\\
       \lspbottomrule
\end{tabularx}
\caption{Criteria for adverb classification}
\label{Tab:ADV2}
\end{table}

The distinctive characteristic of group 1 adverbs is their potential combination with the locative preposition {\itshape ɛ́} which no other adverbial subclass allows for. Also, some (but not all) group 1 adverbs can be used in noun + attributive marker constructions. This property is defining for group 2 adverbs. Group 3 adverbs are the only ones to be restricted to phrase-final position only while all other adverbs can also occur at the beginning of a phrase. Lexemes occurring in group 3 can also be used as adjectives or quantifiers to modify nouns. 




\subsection{Group 1 adverbs: Deictic} 
\label{sec:G1ADV}

Adverbs of group 1 are all deictic in nature, including both locative and manner deixis. They are the most frequent ones occurring in natural text out of all adverb types. 
Deictic adverbs, as any deictic elements, are often accompanied by gestures or assume  common knowledge of the specific place under discussion. \tabref{Tab:DeicAdv} provides a summary of deictic adverbs in Gyeli as well as their numeric frequency in the Gyeli text corpus.\footnote{Obviously, this is a very limited corpus, but it shows some tendencies as to which adverb gets used more frequently.} The deictic elements represented in the table mostly function as adverbs, namely when they occur with verbs, but as the last column shows, almost all of them may also occur in the nominal domain modifying nouns.  \sectref{sec:LOCe} provides more information on the locative {\itshape ɛ́}.

\begin{table}
\begin{tabular}{llrr}
\lsptoprule
Deictic element & Gloss & \multicolumn{2}{c}{Frequency} \\\cmidrule(lr){3-4}
                        &            &        with verb  & with noun \\
                        \midrule
(ɛ́) vâ & `here' & 41 & 2 \\
(ɛ́) pɛ̀ & `over there' & 21 & 0 \\
(ɛ́) wû & `there' & 12 & 3 \\ 
% wɛ̂ & `there' \\ put into noun deictic, maybe with tè
(ɛ́) tè & `there' & 8 & 13 \\
%(ɛ́) mpù  & `like this' &  14   & 0    \\
%bà & `to, {\itshape chez}' & 2 \\
\lspbottomrule
\end{tabular}
\caption{Deictic adverbs}
\label{Tab:DeicAdv}
\end{table} 



\subsubsection*{Formal commonalities}

I view deictic adverbs as a category, based on formal similarity and their potential co-occurrence with the locative marker {\itshape ɛ́}, which distinguishes them from other adverb subclasses. 
All deictic adverbs are monosyllabic. They do not seem to be derived from another part of speech, in contrast to, for instance, group 3 adverbs. Some of them may, however, also be used to modify nouns rather than verbs, namely as the second constituent in noun + attributive marker constructions, as discussed in \sectref{sec:CONC}.  The distribution of deictic adverbs as modifying verbs as opposed to nouns is illustrated in \tabref{Tab:DeicAdv} under ``Frequency''. \REF{DeicNom} gives an example of a deictic element as nominal modifier while the examples in the remainder of this section show deictic adverbs modifying verbs.


\ea \label{DeicNom}
  \glll mɛ̀gà mɛ́ɛ̀ dyúwɔ́ nzã́ã̀ [dúwɔ̀ lé {\bfseries tè}] \\
        mɛ-gà mɛ́ɛ̀ dyúwɔ-H nzã́ã̀ {\db}d-úwɔ̀ lé tè \\
         1\textsc{sg}-{\CONTR} 1\textsc{sg}.{\PST}2 feel-{\R} $\emptyset$7.appetite {\db}le5-day 5:{\ATT} there \\
    \trans `As for me, I had a craving [for meat] that day.'
\z

\noindent Contrasting deictics as verbal versus nominal modifiers, there is a tendency that the more frequently a (locative) deictic element occurs as verbal modifier, the less frequently it is found as a nominal modifier. This is the case, for instance, with {\itshape vâ} `here'. Within the Gyeli text corpus, {\itshape vâ} is found 41 times as a verbal modifier, but only twice as a nominal modifier.  Conversely, the less frequently a deictic adverb modifies verbs, the more often it occurs as a nominal modifier as with {\itshape tè} `there', which occurs only 8 times with verbs, but 13 times with nouns. 

%
\iffalse
The manner deictic {\itshape (ɛ́) mpù} never occurs as nominal modifier.  It generally serves to introduce gestures and ideophones, as for instance in \REF{ADVmpu1}. In this example, the first occurrence of {\itshape mpù} frames the ideophone while the second refers to a gesture that may actually be made or not, i.e.\ the gesture is most likely implied, but not necessarily made.


\ea \label{ADVmpu1}
  \glll yɔ́ɔ̀ Nzàmbí njí {\bfseries mpù} bã̂ã̂ã̂ã̂ njì dígɛ̀ {\bfseries mpù} \\
       yɔ́ɔ̀ Nzàmbí njî-H mpù bã̂ã̂ã̂ã̂ njì dígɛ̀ mpù \\
        so $\emptyset$1.{\PN}.\textsc{prs} come-{\R} like.this {\IDEO}:walking.far come look like.this \\
    \trans `So Nzambi comes like this [depiction of walking a long distance], comes looking like this.'
\z

\noindent   Also, {\itshape mpù} is used in comparison constructions as in \REF{ADVmpu2}. In these cases, {\itshape mpù} is translated as `like, than' rather than `like this'.

\ea \label{ADVmpu2}
  \glll Màmá à ndáà gyà ntɛ̀ {\bfseries mpù} Màmbì \\
       Màmá a ndáà gyà ntɛ̀ mpù Màmbì.  \\
        $\emptyset$1.{\PN}  1.have also $\emptyset$7.length $\emptyset$3.size like $\emptyset$1.{\PN} \\
    \trans `Mama is as tall as Mambi.'
\z
\fi %

\subsubsection*{Phrase position}
A further distinctive morphosyntactic property in adverbial subclasses is the phrase position in which adverbs can occur. As a default position, all adverb classes occur phrase finally. This is also true for group 1 adverbs, as shown in \REF{ADV1finala} and \REF{ADV1finalb}.

\ea \label{ADV1finala}
  \glll    mɛ́ bvú nâ nkwálá wúù tfùndɛ́ mɛ̀ {\bfseries vâ} \\
           mɛ-H bvû-H nâ nkwálá wúù tfùndɛ-H mɛ̀ vâ \\
              1\textsc{sg}-\textsc{prs} think-{\R} {\COMP} $\emptyset$3.machete 3.{\PST}2 miss-{\R} 1\textsc{sg} here \\
    \trans `I think that the machete missed [injured] me here.'
\ex \label{ADV1finalb}
  \glll mɛ́ pã́ ná kɛ̀ dígɛ̀ mùdì wà nû {\bfseries ɛ́} {\bfseries pɛ́} \\
        mɛ-H pã̂-H ná kɛ̀ dígɛ m-ùdì wà nû ɛ́ pɛ́ \\
        1\textsc{sg}.\textsc{prs} try-{\R} again go see \textsc{n}1-person 1:{\ATT} 1.{\DEM}.{\PROX} {\LOC} there \\
    \trans `I try again and go to see that person there.'
\z


In contrast to group 3, group 1 adverbs also pervasively appear in phrase-initial positions, as in \REF{ADV1initiala} and \REF{ADV1initialb}. This position is clearly correlated with information structure, moving the deictic adverb into a focus position.\footnote{See \sectref{sec:IS} on information structure for a more detailed discussion.} While also group 2 (temporal) adverbs can occur in this initial focus position, deictic adverbs are significantly more frequently focused in the Gyeli text corpus.

\ea \label{ADV1initiala}
  \glll  {\bfseries ɛ́} {\bfseries vâ} mɛ̀ dyùwɔ́ nâ {\bfseries ɛ́} {\bfseries vâ} yíì sílɛ̀ njì búlɛ̀\\
         ɛ́ vâ mɛ dyùwɔ-H nâ ɛ́ vâ yíì sílɛ̀ njì búlɛ\\
         {\LOC} here 1\textsc{sg}.{\PST}1 hear-{\R} {\COMP} {\LOC} here 7.{\FUT} finish.{\FUT} come destroy\\
    \trans `I heard that it [the road] will all come here to be destroyed [the plants].'
\ex \label{ADV1initialb}
  \glll  {\bfseries ɛ́} {\bfseries pɛ́-ɛ́} mɛ̀ɛ̀ lwɔ̃̂ nyà ndáwɔ̀ \\
        ɛ́ pɛ́-ɛ́ mɛ̀ɛ̀ lwɔ̃̂ nyà ndáwɔ̀ \\
          {\LOC} there-{\DIST} 1\textsc{sg}.{\FUT} build real $\emptyset$9.house\\
    \trans `Over there, I will build a real house.'
\z



\noindent If a deictic adverb occurs in the initial focus position, it is often repeated again at the end of the phrase in its default position, as shown in \REF{ADV1botha} and \REF{ADV1bothb}.


\ea \label{ADV1botha}
  \glll {\bfseries ɛ́} {\bfseries pɛ̀} bà sílɛ́ bî lwɔ̃̂ mándáwɔ̀ {\bfseries ɛ́} {\bfseries pɛ̀}\\
        ɛ́ pɛ̀ ba sílɛ-H bî lwɔ̃̂ H-ma-ndáwɔ̀ ɛ́ pɛ̀\\
        {\LOC} there 2.{\PST}1 finish-{\R} 1\textsc{pl}.{\OBJ} build {\OBJ}.{\LINK}-ma6-house {\LOC} there\\
    \trans `There, they have finished to build us houses.'
\ex \label{ADV1bothb}
  \glll {\bfseries ɛ́} {\bfseries wû} bèyá lwɔ̃́ kwádɔ́ yã̂ {\bfseries ɛ́} {\bfseries wû}\\
         ɛ́ wû bèyá lwɔ̃̂-H kwádɔ́ y-ã̂ ɛ́ wû\\
         {\LOC} there 2\textsc{pl}[Kwasio] build-{\R} $\emptyset$7.village 7-{\POSS}.1\textsc{sg} {\LOC} there\\
    \trans `Over there, you (pl.) build my village over there.'
\z


The use of the locative {\itshape ɛ́} is more frequent when the adverb occurs phrase initially while post-verbal and  phrase-final occurrences allow for a higher degree of optionality as to whether the locative is used or not. 
The higher degree of locative {\itshape ɛ́} omission when the deictic adverb occurs phrase finally might be phonologically conditioned. Phrase finally, the locative {\itshape ɛ́} usually follows a vowel either from a preceding verb or noun and may undergo deletion in fast speech. When asked, speakers state that the use of the locative {\itshape ɛ́} is possible in both phrase-initial and phrase-final positions. It is less clear at this point whether the co-occurrence of the locative {\itshape ɛ́} with a deictic adverb is generally optional, comparable to the optional use or omission of the attributive marker as discussed in \sectref{sec:CONOM} or whether the locative {\itshape ɛ́} is always underlyingly present with deictic adverbs and its omission in the surface form is purely phonological. 

\subsubsection*{Distinctions within the locative deictic system}
Gyeli uses a range of deictic elements to refer to places or locations in varying distance to the speaker. Since most of these elements would be translated as `there' in English, the system merits a more thorough explanation. In general, distances in Gyeli are relative rather than absolute in that `here', for instance,  can denote a place within the hand-reach of the speaker, but could also talk about a whole village. On the other hand, `over there' can then be a distant place or, in other cases,  a place even within  the village, depending on the discourse topic. 

Semantically, the clearest distinction is between {\itshape vâ} `here', which refers to the relative immediate surroundings of the speaker, and {\itshape pɛ̀} `over there', which denotes the place furthest away. In French, {\itshape pɛ̀} gets translated as {\itshape là-bas}. {\itshape wû} and {\itshape tè} would both be translated as `there', or {\itshape là} in French, which makes it more difficult to grasp their semantic distinctions. Differences in their morphosyntactic behavior can help to disentangle their meaning contrast.

In the default case, it seems that {\itshape wû} denotes a medial distance between {\itshape vâ} `here' and {\itshape pɛ̀} `over there' and occurs mainly in the verbal domain. In contrast, {\itshape tè} is mostly used with nouns rather than with verbs where {\itshape tè} seems to be related more to specificity and/or anaphora than to actual location. In that sense, {\itshape tè} may be less part of the distance-related deictic system, as \REF{Advte1} illustrates. In this example, {\itshape tè} is more existential than about distance.


\ea \label{Advte1}
  \glll     bã̂ yɔ́ɔ̀ yíì {\bfseries tè} \\
	bã̂ y-ɔ́ɔ̀ yíì tè \\
              $\emptyset$7.word 7-2\textsc{sg}.{\POSS}  7.ID there  \\
    \trans `You are understood [lit. your word is there].'
\z

\noindent Also in \REF{Advte2}, the use of {\itshape tè} is not primarily locative, but more anaphoric to the circumstances of earning only 250 Cameroon Francs.

\ea \label{Advte2}
  \glll ká bá kɛ́ wɛ̂ vɛ̀ bé-bwúyà  bébáà nà mà-wú mátánɛ̀ wɛ́ sá {\bfseries tè} ná  \\
        ká ba-H kɛ̀-H wɛ̂ vɛ̀ H-be-bwúyà  bé-báà nà ma-wú má-tánɛ̀ wɛ-H sâ-H tè ná  \\
        if 2-\textsc{prs} go-{\R} 2\textsc{sg}.{\OBJ} give {\OBJ}.{\LINK}-be8-hundred 8-two {\COM} ma6-ten 6-five 2\textsc{sg}-\textsc{prs} do-{\R} there how  \\
    \trans `If they go to give you 250 (Francs), how do you manage there? [because it's very little money]'
\z

In other cases, however, as in \REF{Advte3}, {\itshape tè} is place-denoting just like the other deictic adverbs. Speakers state that, in this example, {\itshape tè} can also be replaced by {\itshape pɛ̀} or {\itshape wû} in both instances.

\ea \label{Advte3}
  \glll     {\bfseries tè} mɛ̀ɛ̀ jíbì kɛ̀ lwɔ̃̂ {\bfseries tè} \\
	tè mɛ̀ɛ̀ jíbi kɛ̀ lwɔ̃̂ tè \\
             there 1\textsc{sg}.{\FUT} start go build there  \\
    \trans `There, I will first go to build there.'
\z

Further, distance cannot be the only distinctive criterion within the locative deictic system: an increased sense of distance can be added phonologically by lengthening the final vowel of the adverb and an H tone, as shown in \REF{pEE} and in \REF{ADV1initialb} above.


\ea \label{pEE}
  \glll     lèkfúdɛ̀ à nzí bíyɔ̀ nlô {\bfseries pɛ́ɛ́}  \\
	le-kfúdɛ̀ a nzí bíyɔ nlô pɛ́-ɛ́ \\
              le5-idiot 1 {\PROG}.{\PST} hit $\emptyset$3.head over.there-{\DIST}  \\
    \trans `The idiot was hitting his head far over there.'
\z

\noindent This way of expressing further distance by vowel lengthening and H tones is possible with both {\itshape pɛ̀} and {\itshape wû}. An example for the latter is given in \REF{wUU}. In contrast, this does not seem to be possible with {\itshape tè}, which indicates again that {\itshape tè} behaves differently from the other more purely locative deictic elements.\footnote{{\itshape vâ} `here' also does not allow for final vowel lengthening and an H tone, but that is clearly a semantic restriction since it denotes a place that is close to the speaker.}

\ea \label{wUU}
  \glll  báà tfùbɔ̀ báà tfùbɔ̀ mpàgó wá nùmbà {\bfseries wúú}\\
         báà tfùbɔ̀ báà tfùbɔ̀ mpàgó wá nùmbà wú-ú\\
          2.{\FUT} pierce 2.{\FUT} pierce $\emptyset$3.road 3:{\ATT} $\emptyset$1.logger there-{\DIST} \\
    \trans `They will cut, they will cut. The road of the loggers there.'
\z

Another difference between {\itshape wû} and {\itshape tè} concerns the combination with a vocative morpheme -{\itshape o} which, at the same time, can further take an H tone to indicate distance between the speaker and the addressee. This vocative morpheme can be used with {\itshape wû}, as shown in \REF{wuo}, but not with {\itshape tè} nor any other deictic element.

\ea \label{wuo}
  \glll mùdì kí tàtɔ̀ {\bfseries wúó} \\
        m-ùdì kí tàtɔ wú-o-H \\
        \textsc{n}1-person {\NEG} scream there-{\VOC}-{\DIST} \\
    \trans `Nobody scream over there!'
\z

In summary, it seems that {\itshape vâ} `here', {\itshape wû} `there' and {\itshape pɛ̀} `over there' form the core locative deictic system while {\itshape tè} `there' takes over other functions (specificity, anaphora) as a default, but can also act as a deictic element within the locative system. The different properties of the various locative deictics as discussed above are summarized in \tabref{Tab:LOCdeic}.


\begin{table}
\fittable{%
\begin{tabular}{ll clcc}
  \lsptoprule
Deictic & Gloss & {\LOC} {\itshape ɛ́} & mostly modifying & {\DIST} marking &  Vocative -{\itshape o}  \\  \midrule
vâ       & `here'  &        x         &     verbal              &         --           &      -- \\
wû      & `there' &        x         &      verbal             &           x           &       x  \\
pɛ̀       & `over~there'  &  x      &      verbal            &            x           &      -- \\
tè        & `there'  &       x         &      nominal          &           --          &     -- \\
\lspbottomrule
\end{tabular}}
\caption{Morphosyntactic properties of locative deictics}
\label{Tab:LOCdeic}
\end{table} 

\subsection{Group 2 adverbs: Temporal}
\label{sec:G2ADV}

Adverbs of group 2 have four members which are all temporal and listed in \tabref{Tab:ADVGroup2}. While group 2 adverbs form a unitary morphosyntactic category, they differ in their derivational source. While {\itshape tɛ́ɛ̀} `now' and {\itshape dẽ̂} `today' seem to be underived lexemes, the other two adverbs in the group are clearly derived from nouns: {\itshape nàkùgúù} `yesterday' is derived from {\itshape kùgúù} `evening' and {\itshape nàmɛ́nɔ́} `tomorrow' from {\itshape mɛ́nɔ́} `morning'. The {\itshape nà}- prefix in these adverbs is a derivational similative marker, as described in \sectref{sec:SIM}.


\begin{table}
\begin{tabular}{ll l}
\lsptoprule
Adverb & Gloss & Derivational source \\  \midrule
tɛ́ɛ̀ & `now' & underived \\
dẽ̂ & `today'  & underived\\
nàkùgúù & `yesterday' & denominal \\
nàmɛ́nɔ́ & `tomorrow' & denominal \\
\lspbottomrule
\end{tabular}
\caption{Group 2 adverbs}
\label{Tab:ADVGroup2}
\end{table} 


The defining property of group 2 temporal adverbs is that they can all also occur in nominal modification as second constituent in a noun + attributive marker construction, as in \REF{G2nom}.

\ea \label{G2nom}
  \ea  \label{G2nom1}
  \glll   bèdéwɔ̀ bé {\bfseries dẽ̂} \\
	be-déwɔ̀ bé dẽ̂  \\
           be8-food 8:{\ATT} today    \\
    \trans `food of today'
\ex\label{G2nom2}
 \glll  nlã̂ wá {\bfseries nàkùgúù} \\
	nlã̂ wá nàkùgúù \\
         $\emptyset$3.story 3:{\ATT} yesterday  \\
    \trans `yesterday's story'
\z
\z

\noindent While some group 1 adverbs exhibit the same property, deictic adverbs also combine with the locative {\itshape ɛ́}, unlike group 2 temporal adverbs.


All group 2 adverbs occur phrase finally as a default position. Examples are given in \REF{ADV2a} through \REF{ADV2c}.

\ea \label{ADV2a}
  \glll wɛ́ làwɔ́ {\bfseries tɛ́ɛ̀}  \\
        wɛ-H làwɔ-H tɛ́ɛ̀  \\
         2\textsc{sg}-\textsc{prs} talk-{\R} now   \\
    \trans `You speak now.'
\ex \label{ADV2b}
  \glll  nyɛ̀ náà à múà wɛ̂ bíyɔ̀ {\bfseries dẽ̂} \\
        nyɛ náà à múà wɛ̀ bíyɔ̀ dẽ̂\\
           1 {\COMP} 1 {\PROSP} 2\textsc{sg}.{\OBJ} hit today \\
    \trans `He [says] that he is about to beat you today.'
\ex \label{ADV2c}
  \glll mɛ̀ nzí kɛ̀ jí {\bfseries nàkùgúù} \\
       mɛ nzí kɛ̀ jí nàkùgúù \\
      1\textsc{sg} {\PROG}.{\PST} go $\emptyset$7.forest yesterday \\
    \trans `I was going to the forest yesterday.'
\z

 
They can all also occur phrase initially, as shown in \REF{ADV2d}. In these cases, they are in focus, as discussed for group 1 adverbs and in \sectref{sec:IS} on information structure. In \REF{ADV2d}, the narrator stresses that the mice will only eat the skulls the next day, as contrastive focus to the possibility that they might eat them right away.


\ea \label{ADV2d}
  \glll àà {\bfseries nàmɛ́nɔ́} bwáà dè {\bfseries nàmɛ́nɔ́} \\
       àà nàmɛ́nɔ́ bwáà dè nàmɛ́nɔ́ \\
       {\EXCL} tomorrow 2\textsc{pl}.{\FUT} eat tomorrow  \\
    \trans `Ah, tomorrow you will eat, tomorrow.'
\z

\noindent In comparison to group 1 adverbs, which occur frequently in this focus position, group 2 adverbs are rarely found in this position in natural text.



%kɔ́ɔ̀ only, just % 3.23
 %kɔ́ɔ̀ m-ùdì m-vúdũ̂ `seulement une personne'
% kɔ́ɔ̀ b-ùdì bà-vúdû `toujours les même personnes'

%m-ùdì kɔ́ɔ̀ m-vúdũ̂ `seulement une personne'

%nâ more, still, again %3.53

%where does "really" belong to?


\subsection{Group 3 adverbs: Manner} 
\label{sec:G3ADV}

Group 3 adverbs are defined by the double affiliation of their lexemes to the part of speech of adjectives (\sectref{sec:QUAL}) or nominal modifiers (\sectref{sec:MODAgrPre}). Semantically, they map onto manner adverbs.
Manner adverbs are rare in Gyeli, both in terms of number and occurrence. \tabref{Tab:ManAdv} gives an exhaustive list of all manner adverbs found in the Gyeli text corpus as well as those stemming from questionnaire elicitation. Each of these manner adverbs occurs only a couple of times in the corpus, thus their natural frequency seems to be generally low. Gyeli seems rather to have a preference to express the manner of an action or event by ideophones, as will be discussed in \sectref{sec:IDEO}.

\begin{table}
\begin{tabular}{lll}
\lsptoprule
Manner adverb & Gloss & Affiliation to other {\pOS} \\
\midrule
mpà & 	good & invariable adjective \\
bíwɔ̀   & 	bad & invariable adjective \\
fí   & 	        different  & deictic modifier  (→ short form of -fúsì) \\
bvùbvù    &       a lot  & invariable quantifier \\ 
\lspbottomrule
\end{tabular}
\caption{Manner adverbs and their affiliated parts-of-speech}
\label{Tab:ManAdv}
\end{table} 

%All of these manner adverbs are also found as nominal modifiers where they differ though in their behavior, as shown in \sectref{sec:MOD} and \ref{sec:INV}. Most of them such as {\itshape mpà} `good', {\itshape bíwɔ̀} `bad', and {\itshape bvùbvù} `much' are invariable also in noun phrases. Only -{\itshape fí} `different', the short form of -{\itshape fúsì} used as a deictic modifier, agrees with its head noun. In the verbal domain, however, all of them are invariable.

\hspace*{-3.5pt}In terms of their position, manner adverbs exclusively occur (intonation) phrase finally. Thus, the adverb may follow the verb if there is no object, as demonstrated in \REF{Advmpa} and \REF{Advfi}.

\ea \label{Advmpa}
  \glll     wɛ̀ nzíí bàlɛ̀ {\bfseries mpà} \\
	wɛ nzíì-H bàlɛ mpà \\
              2\textsc{sg} {\PROG}.\textsc{prs}-{\R} keep good   \\
    \trans `You are remembering well [lit. your are keeping the words well].'
\ex\label{Advfi}
  \glll     wɛ́ ná báàla nà nyɛ́ {\bfseries fí} nà wɛ́ ndyándyá ná sálɛ́ ɛ́ pɛ̂  \\
	wɛ-H ná báàla-H nà nyɛ̂-H fī́ nà wɛ-H ndyándya-H ná sálɛ-H ɛ́ pɛ̂ \\
              2\textsc{sg}-\textsc{prs} again repeat-{\R} {\COM} see-{\R} different {\COM} 2\textsc{sg}-\textsc{prs} work-{\R} again $\emptyset$7.work {\LOC} there   \\
    \trans `You repeat [it] again and try something else and you work there again.'
\z

\noindent If the clause has an object, the manner adverb will follow the object instead of the verb, as shown in \REF{Advbiwo} and \REF{Advbvu}.

\ea \label{Advbiwo}
  \glll   á sìmbɔ́ màtúà {\bfseries bíwɔ̀}   \\
	a-H sìmbɔ-H màtúà bíwɔ̀ \\
              1-\textsc{prs} drive-{\R} $\emptyset$1.car bad   \\
    \trans `He drives the car poorly.'
\ex \label{Advbvu}
  \glll     mɛ̀ɛ́ jí-lɛ́ wɛ̂ {\bfseries bvùbvù} \\
	mɛ̀ɛ́ jí-lɛ́ wɛ̂ bvùbvù \\
              1\textsc{sg}.\textsc{prs}.{\NEG} ask-{\NEG} 2\textsc{sg} much   \\
    \trans `I don't ask you [for] much.'
\z

\noindent In contrast to adverb groups 1 and 2, manner adverbs cannot be used in a phrase-initial focus position.

\subsection{Discussion: Multiple adverbs} 
\label{sec:MultiADV}

While only one adverb can appear phrase initially, multiple adverbs can occur phrase finally. 
There seem to be some ordering principles among multiple phrase-final adverbs slot, with some adverbs seem closer to the center of the phrase than others. Since multiple adverbs do not occur very frequently in natural speech, it is not possible at this point to give a full account of adverb order in multiple adverb constructions. The present examples, however, suggest that group 1 adverbs are closest to the center, i.e.\ verb and following object, as shown in \REF{ADVM1} and \REF{ADVM2}.


\ea \label{ADVM1}
  \gll pílì bèyá lɔ́ njì {\bfseries ɛ̀} {\bfseries vá} {\bfseries tɛ́ɛ̀} {\bfseries dé}  \\
         when 2\textsc{pl} {\RETRO} come {\LOC} here now today  \\
    \trans `When you just arrived here now today,...'
\z

\ea \label{ADVM2}
  \glll  mɛ̀ nzí dyá {\bfseries vâ} {\bfseries kùgúù} [{\bfseries dẽ̂} màfú mábáà] \\
        mɛ̀ nzí dyá vâ kùgúù {\db}dẽ̂ ma-fú má-báà. \\
          1\textsc{sg} {\PROG}.{\PST}1 lie.down here $\emptyset$7.evening {\db}today ma6-day 6-two  \\
    \trans `I was here in the evening two days ago.'
\z

\noindent Other generalizations as to whether any of the other adverb subclasses are closer to the center or the periphery of the clause require more investigation. This is most likely also correlated with information structure factors.























\section{Ideophones}
\label{sec:IDEO}


Ideophones are widely attested in the literature on African languages (see, for instance, \citet{doke1935}, who coined the term, \citet{westermann07} on Ewe, \citet{dumestre98} on Bambara, \citet{alexandre66} on Bulu, or  \citet{newman2001} on Hausa) and also found in Gyeli.  In defining the term {\itshape ideophone}, I refer to \citet[25]{dingemanse2011} who views ideophones as ``marked words that depict sensory imagery'', a definition that deserves some further explanation. First, according him, ideophones are often marked by phonological peculiarities and/or stand out from other words by means of ``special word forms, expressive morphology, relative syntactic independence and foregrounded prosody'' (p. 26). Second, the fact that ideophones are words implies that they are ``conventionalized minimal free forms with specifiable meanings''. Gyeli speakers use ideophones in a conventionalized way able to describe the meaning of single ideophones consistently.\footnote{Ideophones that are identical or similar in their form and meaning seem to be consistently used in the languages of the area either through genealogical affiliation or language contact. In any case, they are easily recognized and understood by speakers of neighboring languages such as Mabi and Bulu.} Third, \citet[27]{dingemanse2011} makes the point that ideophones rather {\itshape depict} than describe their referents. This is similarly explained by \citet[280]{guldemann2008} who notes that ``Metaphorically, one can characterize ideophones as a performance or a gesture in disguise of a word''.
Finally, Dingemanse restricts ideophones to a semantic domain depicting sensory imagery which he views as ``perceptual knowledge that derives from sensory perception of the environment and the body'' (p. 28).
He argues that this semantic-functional definition makes sense for cross-linguistic comparison while grammatical-structural features of ideophones have to be considered language specifically.  

Gyeli ideophones\footnote{There are 19 occurrences of ideophones in the corpus, comprising 16 different ideophones.} modify verbs in some cases, namely when they behave like adverbs. Even when they are syntactically more independent or occur in  complement clauses, they depict the way an event happens.
Generally, Gyeli ideophones structurally stand out from other words in terms of their phonological shape and their syntactic integration into a phrase.

\subsection{Phonological shape of ideophones} 
\label{sec:IDEOphon}

Ideophones in Gyeli are phonologically  marked by various means, including reduplication or a repetitive character, final vowel lengthening, and special syllable structure such as closed syllables or syllables consisting of a consonant only. These three properties usually do not all occur in the same ideophone, but are partially mutually exclusive. For instance, final vowel lengthening excludes the possibility of a closed syllable. Also, reduplication does not usually  occur with final vowel lengthening while closed syllable ideophones may also be reduplicated.  Ideophones are also more specified for the use of alveolar versus postalveolar fricatives and affricates, allowing for less variation. For that reason, I exceptionally represent ideophones with IPA notation in this section. 

\subsubsection*{Reduplication/repetitive character}

Many Gyeli ideophones involve reduplication or repetition, where a word is minimally reduplicated. In most cases, however, the word gets repeated multiple times, i.e.\ more than twice, usually three to five or six times, depending on the ideophone and the dramatic effect aimed at in the discourse. 
For all repetitive ideophones it holds that the number of repeated syllables is not necessarily conventionalized. Each ideophone seems to have a preference for the number of repetitions as represented in the following examples, but the number is not fixed.

Repetitive ideophones can be divided into those that have the same tone on each repeated syllable and those that change their tonal melody across repeated syllables. In \REF{IDEOredup1}, for instance, the ideophones involve repeated monosyllabic words each carrying the same tone.


\eabox{\label{IDEOredup1}\begin{tabularx}{\textwidth - \widthof{(70)~~~~}}{@{}l@{ }Q@{}}
ʃyɛ̂ ʃyɛ̂ & `depiction of sneaking' \\
tʃɔ̀p tʃɔ̀p tʃɔ̀p & `depiction of dripping sound or sound  of walking in mud'  \\
mtʃà mtʃà mtʃà & `depiction of picky eating (only taking certain items off a plate)' \\
kɛ́ kɛ́ kɛ́ kɛ́ kɛ́ & `depiction of placing objects in a row' \\
tsùk tsùk tsùk tsùk & `depiction of noise that mice make' \\
\end{tabularx}}

In contrast, the ideophones in \REF{IDEOredup2} show an alternating tonal pattern with repeated monosyllabic words alternating between H and L tones. One could argue that two syllables, an H plus an L, actually  constitute one unit that gets repeated rather than the single syllable. The fact that these ideophones are often used with an uneven number of syllables, however, indicates that also for tonally alternating ideophones the repeated unit is usually the monosyllabic word.


\eabox{\label{IDEOredup2}\begin{tabularx}{\textwidth - \widthof{(70)~~~~}}{@{}l@{ }Q@{}}
gbĩ́  gbĩ̀ gbĩ́  gbĩ̀  gbĩ́& `depiction of small objects moving in space (e.g.\ bacteria roaming in a body)'\\
wùù wúú wùù wúú & `depiction of sound of bees'\\
\end{tabularx}}

There are a few instances, however, where the word is disyllabic and again, it is the word that gets reduplicated, as shown in \REF{IDEOredup3}. In contrast to monosyllabic ideophone words, disyllabic ones are only subject to reduplication, but usually do not get repeated more than twice.


\eabox{\label{IDEOredup3}\begin{tabular}{@{}ll@{}}
kpúdùm kpúdùm & `depiction of drumming' \\ 
kpàdà kpàdà & `depiction of drumming on bamboo pipes' \\
mátʃà màtʃà & `depiction of eating in little bits'  \\
\end{tabular}}


Semantically, ideophones that involve reduplication or repetition often depict iterative events, for example repeated motion such as drumming or dripping water or recurring sounds such as noise of mice.

\subsubsection*{Final lengthening}

A large group of Gyeli ideophones systematically employs final vowel lengthening, as shown in \REF{IDEOlength}. The extreme length, often until the speaker needs to take another breath, is marked by four vowels (instead of two for phonological long vowels).  All of these lengthened ideophones occur as monosyllabic words only.


\eabox{\label{IDEOlength}\begin{tabular}{@{}ll@{}}
ndɛ̃́ɛ̃́ɛ̃́ɛ̃́ & `depiction of staring' \\
wɔ́ɔ́ɔ́ɔ́ɔ́ & `depiction of moving by foot or motorbike'  \\
bã̂ã̂ã̂ã̂ & `depiction of walking a long distance fast' \\
wùùùù & `depiction of pouring liquids or granulars'  \\
pfáááá & `depiction of flinging a long object or slinging' \\
tèèèè & `depiction of waiting' \\
\end{tabular}}

In comparison to iterative, repetitive ideophones, this group depicts events that either persist in time, for instance staring or waiting, or depict distances, as it is the case with flinging an object (into some distance) or moving into the distance.

As mentioned above, this group of ideophones that receives its special marking in the sense of \posscitet{dingemanse2011} definition by vowel lengthening usually does not combine with reduplication. There are a few exceptions, however. For instance, {\itshape wùùùù} `depiction of pouring liquids or granulars' was found to be used in a reduplicated form, depicting the situation when the main character in the Nzambi story (see \appref{sec:Nzambi}) repeatedly pours fuel onto a house. 


\subsubsection*{Special syllable structure} Some ideophones in Gyeli are further phonologically marked by a closed final syllable structure. As such, ideophones form an exception to a general rule of open syllables in the language (\sectref{sec:Syllable}).
Closed syllables in ideophones frequently end in /m/, but also voiceless obstruents such as /f/ or /k/. Most of them are monosyllabic, as in \REF{IDEOsyll1}.


\eabox{\label{IDEOsyll1}\begin{tabular}{@{}ll@{}}
wɔ̀m & `depiction of (sudden) silence' \\
ùf & `depiction of sound when something catches fire suddenly'  \\
gbìm & `depiction of putting or falling down of a person or object' \\
bààm & `depiction of closing or finishing something'  \\ 
\end{tabular}}

\noindent There are also disyllabic ideophones whose second syllable is closed, ending in the nasal /m/, as shown in \REF{IDEOsyll2}.

\eabox{\label{IDEOsyll2}\begin{tabular}{@{}ll@{}}
pfùtùm & `depiction of sound when jumping into water' \\
pùdùm & `depiction of falling into mud or throwing stone into water'  \\
ntɔ̀ndɔ̀m & `depiction of monkeys jumping in trees' \\
\end{tabular}}


Most of these closed syllable ideophones occur without reduplication. In these cases, they typically depict some sort of suddenness  (sudden silence, suddenly catching fire) or an endpoint of an event (falling, closing, hitting water). There are, however, also a few examples of closed syllable ideophones which involve reduplication such as {\itshape wùf wùf} `depiction of walking mice'.

The other unusual syllable type found in ideophones is that of a consonantal nucleus. Examples are given in \REF{IDEOsyll3}. The voiceless bilabial in {\itshape p p p p} `depiction of smoking pipe' is produced with an ingressive airstream, imitating the inhaling when smoking.


\eabox{\label{IDEOsyll3}\begin{tabular}{@{}ll@{}}
ḿ m̀ m̀ m̀ ḿ & `depiction of someone mumbling to himself' \\
p p p p & `depiction of smoking pipe'  \\
\end{tabular}}


\subsection{Morphosyntactic properties of ideophones}
\label{sec:IDEOsyn}

In terms of word class, ideophones have been assigned to different parts of speech in the literature, depending on the language. \citet[173]{dwyer2003} provide examples from different African languages where ideophones are categorized, for instance, as verbs, adjectives, interjectionals, special classes, but most commonly as adverbs. They further specify that ideophones
\begin{quote}
often differ syntactically from the rest of the grammar. 1) usually occur either before or after a sentence; 2) often don't fit into any of the standard categories for parts of speech.
(p. 174)
\end{quote}
These generalizations also apply in Gyeli. Gyeli ideophones constitute a word class on their own as characterized by their syntactic independence, i.e.\ outside of the syntactic phrase. Possible positions where ideophones are found are (i) at the end of an intonation phrase, (ii) independently, i.e.\ outside of an intonation phrase, and (iii) as complements in complement clauses.

\subsubsection*{Ideophones at the end of intonation phrases}
Ideophones in Gyeli frequently occur at the end of an intonation phrase as in \REF{IDEOadv1} and \REF{IDEOadv2}. In these cases, ideophones are similar to adverbs in their position and their function, namely depicting the manner in which an action or event happens.


\ea \label{IDEOadv1}
  \glll yɔ́ɔ̀ mùdã̂ dígɛ́ mísì {\bfseries ndẽ́ẽ́ẽ́ẽ́} \\
       yɔ́ɔ̀ m-ùdã̂ dígɛ-H m-ísì ndẽ́ẽ́ẽ́ \\
        so \textsc{n}1-woman watch-{\R} ma6-eye {\IDEO}:staring \\
    \trans `So the woman looks with her eyes [depiction of staring].'
\z


\ea \label{IDEOadv2}
  \glll  bá kɛ́ ndáà nà tɛ́lɛ́ mákùndù má kùrã̂  {\bfseries kɛ́}-{\bfseries kɛ́}-{\bfseries kɛ́}-{\bfseries kɛ́}-{\bfseries kɛ́}\\
         ba-H kɛ-H ndáà nà tɛ́lɛ-H H-ma-kùndù má kùrã̂  kɛ́-kɛ́-kɛ́-kɛ́-kɛ́ \\
        2-\textsc{prs} go-{\R} also {\COM} put-{\R} {\OBJ}.{\LINK}-ma6-clay.house 6:{\ATT} $\emptyset$7.electricity {\IDEO}:repeated.placement \\
    \trans `They also go and put up electricity poles for clay houses [depiction of putting the electricity poles along the road].'
\z

In contrast to adverbs, ideophones also occur in constructions with the deictic element {\itshape mpù} `like this', as shown in \REF{IDEOadv3}.


\ea \label{IDEOadv3}
  \glll yɔ́ɔ̀ Nzàmbí njí mpù {\bfseries bã̂ã̂ã̂ã̂} njì dígɛ̀ mpù \\
        yɔ́ɔ̀ Nzàmbí nji-H mpù bã̂ã̂ã̂ã̂ njì dígɛ̀ mpù \\
        so $\emptyset$1.{\PN} come-{\R} like.this {\IDEO}:walking.far come look like.this \\
    \trans `So Nzambi comes like this [depiction of walking a long distance], comes looking like this.'
\z


\subsubsection*{Ideophones as {\itshape nâ} complements} Similarly, the same sort of signaling happens when ideophones are used as complements in {\itshape nâ} clauses, as illustrated in \REF{IDEOna}.

%[??? PUT MPU {\AND} NA CONSTRUCTIONS IN ONE PARAGRAPH AS DEICTIC FRAMING? IS SYNTACTIC {\ANA}LYSIS AS {\COMPL}EMENT CORRECT AT ALL?]


\ea \label{IDEOna}
  \glll Nzàmbí, màbɔ́ɔ̀ nkwɛ́ɛ̀ dé nâ {\bfseries vɔ́sì} \\
    Nzàmbí ma-bɔ́ɔ̀ nkwɛ́ɛ̀ dé nâ vɔ́sì \\
          $\emptyset$1.{\PN} ma6-breadfruit $\emptyset$3.basket {\LOC} {\COMP} {\IDEO}:pouring \\
    \trans `Nzambi pours the breadfruit into the basket.'
\z

\noindent This type of construction is parallel to reported speech, as discussed in \citet{guldemann2008}. For more information on Gyeli complement constructions and reported speech, see \sectref{sec:CompC}.

\subsubsection*{Syntactic independence of ideophones}
Gyeli ideophones occur independently from an intonation phrase, rather forming an intonation phrase on their own. In this, they differ from adverbs which cannot occur as independent intonation phrases. In \REF{IDEOfree1}, the ideophone occurs before the intonation phrase it refers to in the discourse. The ideophone is separated from the following intonation by a short pause.


\ea \label{IDEOfree1}
  \glll  {\bfseries gbĩ́-gbĩ̀-gbĩ́-gbĩ̀-gbĩ́}   à múà nà bábɛ̀ tí wúmbɛ́ wɛ̀ \\
            gbĩ́-gbĩ̀-gbĩ́-gbĩ̀-gbĩ́  a múà nà bábɛ̀ tí wúmbɛ-H wɛ̀ \\
         {\IDEO}:roaming 1\textsc{sg}.{\PST}1 {\PROSP} {\COM} $\emptyset$7.illness {\NEG} want-{\R} die\\
    \trans `[depiction of disease roaming in his body] He was about to be sick, not wanting to die.'
\z

\noindent Intonationally independent ideophones can also follow the intonation phrase they are semantically linked to in the discourse, as shown in \REF{IDEOfree2}.

\ea \label{IDEOfree2}
  \glll wɛ́ dyúwɔ́ mpù bàmìntùlɛ̀ bɔ́gá bá tsígɛ̀ {\bfseries tsùk-tsùk-tsùk}\\
        wɛ-H dyúwɔ-H mpù ba-mìntùlɛ̀ bɔ́-gá ba-H tsígɛ̀ tsùk-tsùk-tsùk\\
        2\textsc{sg}-\textsc{prs} hear-{\R} like.this ba2-mouse 2-other 2-\textsc{prs} take.off {\IDEO}:rustling\\
    \trans `You hear like this the other mice take off [depiction of noise made by mice].'
\z

\noindent In addition to intonational breaks, the end of an intonation phrase can be indicated by the tonal melody. In \REF{IDEOfree2}, it is the L tone on {\itshape tsígɛ̀} `take off', which shows the end on the intonation phrase. If the ideophone was part of the same intonation phrase, the final tone on {\itshape tsígɛ̀} would be H.


\section{Pronouns}
\label{sec:PRO}

Gyeli has different types of pronouns, i.e.\ grammatical free morphemes that can replace a noun phrase. The different pronominal paradigms arise from the pronouns' differing syntactic functions and distributions. I distinguish subject pronouns from non-subject pronouns. The latter are used in object and adjunct function. For the reader's convenience, I gloss them simply as {\OBJ}.  Gyeli has further interrogative pronouns, possessor pronouns, and a reflexive pronoun {\itshape mɛ́dɛ́} `self' that follows subject and non-subject pronouns. \tabref{Tab:Pros} illustrates all pronoun paradigms and, for comparison of forms, the verbal {\STAMP} marker (\sectref{sec:SCOP}). Most paradigms can be subdivided into speech act participants (1\textsc{sg}, 1\textsc{pl}, 2\textsc{sg}, and 2\textsc{pl}), which are not marked for gender agreement, and non-speech act participants (third person), which are marked for one of the nine agreement classes.  

As described in detail in \sectref{sec:POSS}, possessor pronouns reference the possessor by their pronominal root. The pronominal root is the same for all non-speech act participants, as indicated by 3\textsc{sg} and 3\textsc{pl} in \tabref{Tab:Pros}. The possessee is referenced by an agreement prefix, which is listed for each agreement class as well. 
Some paradigms are specified for tones and marked as such, for instance subject and non-subject pronouns. In contrast, {\STAMP} markers and possessor pronouns have different tonal patterns, depending on the tense/aspect/mood/polarity category they encode or the possessee agreement class. 

\begin{table}
\begin{tabular}{lll}
 \lsptoprule
Paradigm & Singular & Plural \\
\midrule
Subject pronouns &  1\textsc{sg} {\itshape mɛ̀} & 1\textsc{pl} {\itshape bí}  \\
 &   2\textsc{sg} {\itshape wɛ̀} & 2\textsc{pl} {\itshape bé} \\
&   cl.1 {\itshape nyɛ̀} & cl.2 {\itshape bá}  \\
 &  cl.3 {\itshape wú} & cl.4  {\itshape mí}   \\
&  cl.5 {\itshape lí} & cl.6 {\itshape má} \\
&   cl.7 {\itshape yí} & cl.8 {\itshape bé} \\
&   cl.9 {\itshape nyì} &  \\
  \midrule
{\STAMP} markers &  1\textsc{sg} {\itshape mɛ} &   1\textsc{pl} {\itshape ya} \\
   &   2\textsc{sg} {\itshape wɛ} &  1\textsc{pl} {\itshape bwa} \\
   & cl.1 {\itshape a/nyɛ/nu} &   cl.2 {\itshape ba} \\
   & cl.3 {\itshape wu} &  cl.4  {\itshape mi} \\
   & cl.5 {\itshape le} &  cl.6 {\itshape ma} \\
   & cl.7 {\itshape yi} &  cl.8 {\itshape be} \\
   & cl.9 {\itshape nyi} & \\
\midrule   
Non-subject pronouns ({\OBJ}) &   1\textsc{sg} {\itshape mɛ̂} & 1\textsc{pl} {\itshape bî} \\
 &  2\textsc{sg} {\itshape wɛ̂} & 2\textsc{pl} {\itshape bê} \\
 & cl.1 {\itshape nyɛ̂} &  cl.2 {\itshape b-ɔ̂} \\
 & cl.3 {\itshape w-ɔ̂} &  cl.4  {\itshape my-ɔ̂} \\
 & cl.5 {\itshape l-ɔ̂} & cl.5 {\itshape m-ɔ̂} \\
 & cl.7 {\itshape y-ɔ̂} & cl.8 {\itshape by-ɔ̂} \\
 & cl.9 {\itshape ny-ɔ̂}  & \\
\midrule 
Possessor pronouns &     1\textsc{sg} -{\itshape ã} &  1\textsc{pl} -{\itshape isi/usi} \\
 &  2\textsc{sg} -{\itshape ɔ} &   2\textsc{pl} -{\itshape inɛ/unɛ}\\
 &  cl.1 {\itshape w-ɛ} & cl.2 {\itshape b-awɔ}  \\
 &  cl.3 {\itshape w}- &   cl.4 {\itshape mi}-  \\
 &  cl.5 {\itshape l}- & cl.6 {\itshape m}-  \\
 & cl.7 {\itshape y}- &  cl.8 {\itshape bi}- \\
 &  cl.9 {\itshape ny}- &  \\
\midrule 
Interrogative pronouns & {\itshape nzá} `who' & {\itshape bànzá} `who' \\
 & {\itshape gyí} `what'  & \\
\midrule
Reflexive pronoun {\itshape mɛ́dɛ́} `self' & & \\
 \lspbottomrule
\end{tabular}
\caption{Pronoun paradigms}
\label{Tab:Pros}
\end{table}

Generally, agreement class 2 pronouns are also used for impersonal reference. For instance, active clauses with the impersonal {\itshape ba} pronoun are preferred over passive constructions (\sectref{sec:PASS}). This pronoun can also be used in impersonal relative clauses, expressing `who' in the subordinate clause even if the referent of the main clause is expressed by a different agreement/person class (\sectref{sec:Relativeclauses}).




\subsection{Subject pronouns}
\label{sec:SBJPRO}

Subject pronouns are rarely used in Gyeli, with only 17 occurrences in the corpus, since subject noun phrases are mostly expressed by a noun or entirely dropped, leaving only the \textsc{stamp} marker (\sectref{sec:SCOP}) as portmanteau morpheme that expresses subject agreement on the predicate. Subject pronouns are used for subject focus of, mostly, speech act participants. Non-speech act participants are focused through other information structure strategies (\sectref{sec:IS}).



\tabref{Tab:SBJpro} provides the subject pronoun forms for both speech and non-speech act participants. All subject pronouns are specified for tone (unlike the \textsc{stamp} markers, which take their tonal marking from the tense-mood category they encode). Most persons have an H tone pronoun, with the exceptions of the first and second person singular and the pronouns of agreement classes 1 and 9.

\begin{table}
\begin{tabular}{lll}
 \lsptoprule
 & Singular & Plural \\
\midrule
Speech act participants & 1\textsc{sg} {\itshape mɛ̀} & 1\textsc{pl} {\itshape bí} \\
 & 2\textsc{sg} {\itshape wɛ̀} & 2\textsc{pl} {\itshape bé} \\
 \midrule
Non-speech act participants (3\textsuperscript{rd} person) & cl.1 {\itshape nyɛ̀} & cl.2 {\itshape bá} \\
 & cl.3 {\itshape wú} & cl.4  {\itshape mí} \\
& cl.5 {\itshape lí} & cl.6 {\itshape má} \\
 & cl.7 {\itshape yí} & cl.8 {\itshape bé} \\
&  cl.9 {\itshape nyì} & \\
 \lspbottomrule
\end{tabular}
\caption{Subject pronouns}
\label{Tab:SBJpro}
\end{table}

While many subject pronouns are segmentally identical to the \textsc{stamp} markers of their person/class (see \tabref{Tab:Pros} for comparison), there are a few exceptions which clearly show that subject pronouns form a distinct paradigm.  These exceptions include the first and second person plural,  and the pronoun of agreement class 1. To indicate this distinction in the glosses, I mark subject pronouns with  `{\SBJ}', while the \textsc{stamp} marker is only marked for its agreement class/person, as in \REF{SBJpro1} where subject pronoun (in bold) and \textsc{stamp} marker differ in their form.

\ea \label{SBJpro1}
  \glll  dɔ̃̀ {\bfseries bí} yá táálɛ́ bê yàlànɛ̀ àà \\
        dɔ̃̀ bí ya-H táálɛ-H bê yàlanɛ àà \\
       so[French] 1\textsc{pl}.{\SBJ}  1\textsc{pl}-\textsc{prs} begin-{\R} 2\textsc{pl} respond[Bulu] {\EXCL}   \\
    \trans `So we start to respond to you, mhm.'
\z

Other subject pronouns are segmentally identical to their \textsc{stamp} marker and might only differ tonally, depending on the tense-mood category, as in \REF{SBJpro2}.

\ea \label{SBJpro2}
  \glll ah mbúmbù {\bfseries wɛ̀} wɛ́ tɛ́lɛ́ núndɛ̀ \\
      ah mbúmbù wɛ wɛ-H tɛ́lɛ-H nú-ndɛ̀ \\
        {\EXCL} $\emptyset$1.namesake 2\textsc{sg}.{\SBJ} 2\textsc{sg}-\textsc{prs} stand-{\R} 1-{\ANA} \\
    \trans `Ah namesake, is it you who is standing there?'
\z

\noindent Thus, although the agreement class 2 subject pronoun {\itshape bá} is segmentally identical to its \textsc{stamp} marker, the two forms differ due to the \textsc{future} marking on the \textsc{stamp} marker in \REF{SBJpro3}.

\ea \label{SBJpro3}
  \glll {\bfseries bá} báà bù mpàgó \\
       bá báà bù mpàgó \\
        2.{\SBJ} 2.{\FUT} break $\emptyset$3.road  \\
    \trans `THEY will build a road.'
\z

The subject pronoun always occurs in subject position and always precedes the \textsc{stamp} marker. If the subject is preceded by a fronted object, as for instance an interrogative pronoun in \REF{SBJpro4}, the object pronoun will precede both the subject pronoun and \textsc{stamp} marker.

\ea \label{SBJpro4}
  \glll     gyí bí yá tfúgà yá tfúgá nà gyí\\
             gyí bí ya-H tfúga ya-H tfúga-H nà gyí\\
              what 1\textsc{pl}.{\SBJ} 1\textsc{pl}-\textsc{prs} suffer 1\textsc{pl}-\textsc{prs} suffer-{\R} {\COM} what \\
    \trans `What do we suffer, we suffer from what?'
\z

There are certain words that can enter between the subject pronoun and the \textsc{stamp} marker. These are, for instance, the contrastive marker -{\itshape gà}  (\sectref{sec:CONTRS}) that attaches to the subject pronoun, as in \REF{SBJpro5}.

\ea \label{SBJpro5}
  \glll  yɔ́ɔ̀ Nzàmbí {\bfseries nyɛ̀gà} à kɛ̃́ɛ̃̀ dígɛ̀ mísì \\
         yɔ́ɔ̀ Nzàmbí nyɛ-gà a kɛ̃́ɛ̃̀ dígɛ m-ísì \\
           so $\emptyset$1.{\PN} 1.{\SBJ}-{\CONTR} 1.{\PST}1 go.{\COMPL} watch ma6-eye  \\
    \trans `So this Nzambi has gone and was thinking very hard [lit. he watched with his eyes].'
\z

\noindent Other nominal modifiers, such as {\itshape bɔ́ɔ̀} `other' in \REF{SBJpro6} or {\itshape ndáà}  `also' in \REF{SBJpro7} occur between the subject pronoun and the \textsc{stamp} marker.

\ea \label{SBJpro6}
  \glll   {\bfseries bí} {\bfseries bɔ́ɔ̀} yá bígɛ́ mpá'à wá vɛ́ \\
           bí b-ɔ́ɔ̀ ya-H bígɛ-H mpá'à wá vɛ́ \\
           1\textsc{pl}.{\SBJ} 2-other 1\textsc{pl}-\textsc{prs} develop-{\R} $\emptyset$3.side 3:{\ATT} which  \\
    \trans `How will we others [in contrast to other Gyeli villages] make progress?'
\ex \label{SBJpro7}
  \glll  ɛ̀sɛ́ béé ndáà bèyá làwɔ́ fàlà \\
       ɛ̀sɛ́ béé ndáà bèya-H làwɔ-H fàlà \\
        is.it[French] 2\textsc{pl}.{\SBJ} also 2\textsc{pl}[Kwasio]-\textsc{prs} speak-{\R} $\emptyset$1.French  \\
    \trans `Isn't it, you (pl.), you also speak French.'
\z






\subsection{Non-subject pronouns}
\label{sec:OBJPRO}


Gyeli has a paradigm of non-subject pronouns which are used for object and oblique noun phrases. They are glossed as ``{\OBJ}''. They are significantly more frequent in the corpus than subject pronouns, counting 99 occurrences.

As shown in \tabref{Tab:ProOBJ}, the non-subject index forms for 1\textsc{sg}, 1\textsc{pl}, 2\textsc{sg}, 2\textsc{pl}, as well as cl. 1 are segmentally identical to their subject counterparts. All the other non-subject pronouns, namely agreement classes 2 through 9, differ structurally in that they have a non-subject pronoun root -{\itshape ɔ̂} that takes an agreement prefix. All non-subject pronouns are specified for an HL tone, which is a distinctive feature when compared to subject pronouns.

\begin{table}
\begin{tabular}{lll}
 \lsptoprule
 & Singular & Plural \\
\midrule
Speech act participants & 1\textsc{sg} {\itshape mɛ̂} & 1\textsc{pl} {\itshape bî/bíyɛ̀} \\
 & 2\textsc{sg} {\itshape wɛ̂} & 2\textsc{pl} {\itshape bê} \\
 \midrule
Non-speech act participants & cl.1 {\itshape nyɛ̂} & cl.2 {\itshape b-ɔ̂} \\
 & cl.3 {\itshape w-ɔ̂} & cl.4  {\itshape my-ɔ̂} \\
& cl.5 {\itshape l-ɔ̂} & cl.6 {\itshape m-ɔ̂} \\
 & cl.7 {\itshape y-ɔ̂} & cl.8 {\itshape by-ɔ̂} \\
&  cl.9 {\itshape ny-ɔ̂} & \\
 \lspbottomrule
\end{tabular}
\caption{Non-subject pronouns}
\label{Tab:ProOBJ}
\end{table}

 Non-subject pronouns that serve as objects occur in all object positions discussed in \sectref{sec:verbalC} and \sectref{sec:IS}.  The basic position is after the verb, as in \REF{NSBJ1x} and \REF{NSBJ3}.

\ea \label{NSBJ1x}
  \glll  bwáá lã́ {\bfseries bɔ̂}  \\
         bwáa-H lã-H b-ɔ̂ \\
            2\textsc{pl}-\textsc{prs} tell-{\R} 2-{\OBJ}     \\
    \trans `You tell them!'
\ex \label{NSBJ3}
  \glll byɔ̂ bé vɛ́ {\bfseries bíì} màpè'è \\
        byɔ̂ be-H vɛ̀-H bíì ma-pè'è \\
        8.{\OBJ} 8-\textsc{prs} give-{\R} 1\textsc{pl}.{\OBJ} ma6-wisdom  \\
    \trans `They give us wisdom.'
\z

Non-subject pronouns serving as objects can also be dislocated to the left edge of the clause, as in \REF{NSBJ1}.  In this marked position \REF{NSBJ3} as well as in the in-situ focus position in \REF{NSBJ1}, the pronoun is optionally lengthened for emphasis.

\ea \label{NSBJ1}
  \glll  {\bfseries yɔ́ɔ̀} mɛ́ wúmbɛ́ wû \\
        yɔ́ɔ̀ mɛ-H wúmbɛ-H wû \\
         7.{\OBJ} 1\textsc{sg}-\textsc{prs} want-{\R} there \\
    \trans `That is what I want there.'
\z

The first person plural often occurs with the special form {\itshape bíyɛ̀} in the corpus, as in \REF{NSBJ2}. This seems even more emphatic than the lengthened form {\itshape bíì}. The data is not sufficient, however, to pinpoint the exact distribution and functional difference between the two emphatic forms. The first person plural is the only person category that has such a suppletive emphatic form.

\ea \label{NSBJ2}
  \glll bvúlɛ̀ bá ntɛ́gɛ́lɛ́ ndáà bíyɛ̀ \\
       bvúlɛ̀ ba-H ntɛ́gɛlɛ-H ndáà bíyɛ̀ \\
         ba2.Bulu 2-\textsc{prs} bother-{\R} also 1\textsc{pl}.{\OBJ}   \\
    \trans `The Bulu bother us, too.'
\z

\noindent Non-subject pronouns also occur in obliques, as in \REF{NSBJ2x}.

\ea \label{NSBJ2x}
  \glll á nyùlɛ́nyúlɛ́  kɔ̀fí nà yɔ̂ \\
       a-H nyùlɛ-nyulɛ-H  kɔ̀fí nà y-ɔ̂  \\
         1-\textsc{prs} drink-{\HAB}-{\R} $\emptyset$7.coffee {\COM} 7-{\OBJ}   \\
    \trans `He usually drinks coffee with it [sugar]'
\z

Finally, non-subject pronouns are used as an information structure strategy after nominal subjects to mark subject focus, as in \REF{beEMPH2x}. 

\ea \label{beEMPH2x}
  \glll   ngùndyá tè {\bfseries nyɔ̂} bɛ́ nyî \\
          ngùndyá tè nyɔ̂ bɛ̀-H nyî \\
           $\emptyset$9.raffia there 9.{\OBJ} be-{\R} 9.{\DEM}.{\PROX}  \\
    \trans `The raffia there, IT is that.'
\z

Just like subject pronouns, they can take the contrastive marker -{\itshape gà} to indicate switch-reference or mark in-situ focus, as shown in \sectref{sec:CONTRS}. 


\subsection{Interrogative pronouns}
\label{sec:INTERRPRO}

In addition to subject and non-subject pronouns, Gyeli also has two interrogative pronouns: {\itshape nzá} `who' for human referents and {\itshape gyí} `what' for non-human and inanimate referents.\footnote{Although many animals are grammatically classified within the same ``animate'' gender 1/2 as human referents, all animals are referred to with the non-personal interrogative pronoun {\itshape gyí}.}
These interrogative pronouns replace a nominal {\NP}, which is shown in \REF{nza} and \REF{gyi}, respectively. In \REF{nza}, the interrogative replaces the subject {\NP} {\itshape m-ùdũ̂} `man' while, in \REF{gyi}, the interrogative {\itshape gyí} replaces the object {\NP} {\itshape má-jíwɔ́} `water'. In that sense, they behave like personal pronouns. Both interrogatives are used in all noun phrase environments, namely as subjects, objects, and obliques.

\ea \label{nza}
  \ea  \label{nza1}
  \glll     [mùdũ̂] à nyɛ́ mùdã̂ \\
              {\db}m-ùdũ̂ a nyɛ̂-H m-ùdã̂\\
    {\db}\textsc{n}1-man 1.{\PST}1 see-{\R} \textsc{n}1-woman \\
    \trans `The/a man saw the/a woman.'
\ex\label{nza2}
 \glll    {\bfseries nzá}  à nyɛ́ mùdã̂  \\
           nzá  a nyɛ̂-H m-ùdã̂ \\
          who 1.{\PST}1 see-{\R} \textsc{n}1-woman \\
    \trans `Who saw the/a woman?'
\z
\ex \label{gyi}
  \ea  \label{gyi1}
  \glll     mùdũ̂ á nyùlɛ́ [májíwɔ́] \\
             m-ùdũ̂ a-H nyùlɛ-H {\db}H-ma-jíwɔ́ \\
              \textsc{n}1-man 1-\textsc{prs} drink-{\R} {\db}{\OBJ}.{\LINK}-ma6-water \\
    \trans `The/a man drinks water.'
\ex\label{gyi2}
 \glll     {\bfseries gyí} mùdũ̂ á nyùlɛ̀  \\
           gyí m-ùdũ̂ a-H nyùlɛ \\
             what \textsc{n}1-man 1-\textsc{prs} drink \\
    \trans `What does the man drink?'
\z
\z

Interrogative pronouns in oblique phrases are shown with the comitative marker {\itshape nà} in \REF{nanzá} and \REF{nagyí}.

\ea \label{nanzá}
  \ea  \label{nanzá1}
  \glll    mùdũ̂ à kɛ́ màkítì [nà Àdà] \\
           m-ùdũ̂ a kɛ̀-H m-àkítì {\db}nà Àdà \\
              \textsc{n}1-man 1.{\PST}1 go-{\R} ma6-market {\db}{\COM} $\emptyset$1.{\PN}  \\
    \trans `The/a man went to the market with Ada.'
\ex\label{nanzá2}
 \glll      {\bfseries nà} {\bfseries nzá} mùdũ̂ à kɛ́ màkítì  \\
       nà nzá m-ùdũ̂ a kɛ̀-H m-àkítì  \\
           {\COM} who \textsc{n}1-man 1.{\PST}1 go-{\R} ma6-market\\
    \trans `With whom did the man go to the market?'
\z
\ex \label{nagyí}
  \ea  \label{nagyí1}
  \glll    mùdũ̂ à kɛ́ màkítì [nà tṹũ̀]\\
                 m-ùdũ̂ a kɛ̀-H m-àkítì {\db}nà  tṹũ̀ \\
              \textsc{n}1-man 1.{\PST}1 go-{\R} ma6-market {\db}{\COM} $\emptyset$7.axe \\
    \trans `The/a man went to the market with an axe.'
\ex\label{nagyí2}
 \glll    {\bfseries nà} {\bfseries gyí} mùdũ̂ à kɛ́ màkítì \\
         nà gyí m-ùdũ̂ a kɛ̀-H m-àkítì  \\
             {\COM} what \textsc{n}1-man 1.{\PST}1 go-{\R} ma6-market \\
    \trans `With what did the man go to the market?'
\z
\z


{\itshape nà nzá} `with whom' is interesting in that {\itshape nzá} seems to take a plural marker if the expected answer is more than one person, as shown in \REF{banzá}. Since the prefix {\itshape bà}- comes with an L tone, it seems to behave like either a noun class or agreement prefix. Since {\itshape nzá} only occurs with humans, the prefix is invariably class 2 {\itshape bà}-, therefore it is difficult to test whether the prefix belongs to a noun or a modifier.

\ea \label{banzá}
  \ea  \label{banzá1}
  \glll    mùdũ̂ à kɛ́ màkítì [nà Àdà nà Màmbì] \\
           m-ùdũ̂ a kɛ̀-H m-àkítì {\db}nà Àdà nà Màmbì \\
              \textsc{n}1-man 1.{\PST}1 go-{\R} ma6-market {\db}{\COM} $\emptyset$1.{\PN} {\COM} $\emptyset$1.{\PN} \\
    \trans `The/a man went to the market with Ada and Mambi.'
\ex\label{banzá2}
 \glll      {\bfseries nà} {\bfseries bànzá} mùdũ̂ à kɛ́ màkítì \\
          nà bà-nzá m-ùdũ̂ a kɛ̀-H m-àkítì \\
           {\COM} 2-who ba1-man 1.{\PST}1 go-{\R} ma6-market\\
    \trans `With whom did the man go to the market?'
\z
\z


\subsection{Possessor pronouns}
\label{sec:POSS}

\noindent Possessor pronouns in Gyeli consist of a root indicating the possessor and a prefix that agrees with the possessee, as shown in \REF{POSS}.

\ea \label{POSS}
  \ea  \label{POSS1}
  \gll     m-ùdì w-ɔ̂ \\
                \textsc{n}1-man 1-{\POSS}.2\textsc{sg} \\
    \trans `your ({\SG}) man'
\ex\label{POSS2}
 \gll     mì-nkwɛ́ my-áwɔ́ \\
                mi4-basket 4-{\POSS}.3\textsc{pl} \\
    \trans `their baskets'
\z
\z

\subsubsection*{Possessor roots}  \tabref{Tab:SegPoss} shows the possessor roots. While most possessor roots are used for all agreement classes, there are both segmental and tonal changes depending on the phonological shape of agreement prefixes and the agreement class affiliation respectively.

\begin{table}
\begin{tabular}{lll}
 \lsptoprule
 & Singular & Plural \\
  \midrule
 1 & \itshape{-ã} & \itshape{-isi} (\itshape{-usi}) \\
 2 & \itshape{-ɔ} & \itshape{-inɛ} (\itshape{-unɛ}) \\
 3 & \itshape{-ɛ} & \itshape{-awɔ} \\
  \lspbottomrule
\end{tabular}
\caption{Basic possessor roots}
\label{Tab:SegPoss}
\end{table}

 \noindent  Some possessor roots are influenced in their segmental form by the shape of the possessee agreement prefix. The first and second person plural are subject to variation if the possessee belongs to class 1 or 3. Then, the first high front vowel used in all other agreement classes turns into a high back vowel as an assimilation to the agreement prefix {\itshape w-} in class 1 and 3. The contrast between the two root shapes is illustrated in \REF{w}.

\ea \label{w}
  \ea  \label{w1}
  \gll     gyà y-{\bfseries í}sí \\
                7.music 7-{\POSS}.1\textsc{pl} \\
    \trans `our music'
\ex\label{w2}
 \gll     m-wánɔ̀ w-{\bfseries ù}sí \\
                \textsc{n}1-child 1-{\POSS}.1\textsc{pl} \\
    \trans `our child'
\z
\z

The agreement class that the possessor root takes also determines the tonal pattern of the root. The tonal pattern of the first and second person singular are the same in every agreement class, as shown in \tabref{Tab:TonePoss}. The vast majority of agreement classes take an H tone in the third person singular and an HH pattern for the plural possessor roots. Classes 1 and 9, however, are different: the third person singular has a falling HL tone and the plural persons are LH.

\begin{table}
\begin{tabular}{l ll ll}
 \lsptoprule
 & \multicolumn{2}{c}{Basic tonal pattern} & \multicolumn{2}{c}{Exceptions: cl.~1 and 9} \\
 \cmidrule(lr){2-3}\cmidrule(lr){4-5}
 Person & Singular & Plural & Singular & Plural \\
  \midrule
 1 & -ã̂ & -ísí (-úsí) & -ã̂ & -{\bfseries ìsí} (-{\bfseries ùsí}) \\
 2 & -ɔ̂ & -ínɛ́ (-únɛ́) & -ɔ̂ & -{\bfseries ìnɛ́} (-{\bfseries ùnɛ́}) \\
 3 & -ɛ́ & -áwɔ́ & -{\bfseries ɛ̂} & -{\bfseries àwɔ́} \\
  \lspbottomrule
\end{tabular}
\caption{Tonal patterns of possessor pronouns}
\label{Tab:TonePoss}
\end{table}

In natural text, as opposed to elicitation, third person singular possessor pronouns are often lengthened, as shown in \REF{POSSlength}.

\begin{exe} 
\ex\label{POSSlength} 
  \glll  èé lûngà yá sã́ {\bfseries wɛ́ɛ̀} yɔ́ɔ̀ yíì \\
          èé lûngà yá sã́ w-ɛ̂ y-ɔ́ɔ̀ yíì \\
           EXCL $\emptyset$7.grave 7:ATT $\emptyset$1.father 1-POSS.3\textsc{sg} 7-{\OBJ} 7.COP\\
    \trans `Right, his father's grave is over there.'
\end{exe}

\subsubsection*{Possessee agreement prefixes} Possessor pronouns index the possessee by means of an agreement prefix.  \tabref{Tab:PossPre} lists the prefixes for the various agreement classes.

\begin{table}
\begin{tabularx}{.5\textwidth}{Xl}
 \lsptoprule
\textsc{agr} class & {\AGR} prefix \\
  \midrule
 1 & w-  \\
 2 & b- \\
 3 & w- \\
4 & mí- \\
5 & l- \\
6 & m- \\
7 & y- \\
8 & bí- \\
9 & ny- \\
  \lspbottomrule
\end{tabularx}
\caption{Possessee agreement prefixes}
\label{Tab:PossPre}
\end{table}

\noindent Prefixes of classes 4 and 8 ending in a high front vowel are assimilated to the pronominal root. If the root starts with a high front vowel /i/ as for the first and second person plural ({\itshape -ísí} and {\itshape -ínɛ́}), the vowel of the prefix is deleted:

\eabox{\label{POSS41}\begin{tabular}{@{}llllll@{}}
\multicolumn{6}{@{}l@{}}{class 4:} \\
 {\itshape mi-} & + & {\itshape -ísí} & $\rightarrow$ &  {\itshape mísí} & `our' \\
 {\itshape mi-} & + & {\itshape -ínɛ́} & $\rightarrow$ & {\itshape mínɛ́} & `your ({\PL})' \\
\end{tabular}
}

\eabox{\label{POSS81}\begin{tabular}{@{}llllll@{}}
\multicolumn{6}{@{}l@{}}{class 8:}  \\
 {\itshape bi-} & + & {\itshape -ísí} & $\rightarrow$ &  {\itshape bísí} & `our' \\
 {\itshape bi-} & + & {\itshape -ínɛ́} & $\rightarrow$ & {\itshape bínɛ́} & `your ({\PL})' \\
\end{tabular}}

\noindent For the other roots starting in different vowels, the prefix vowel is assimilated and becomes a glide:

\eabox{\label{POSS42}\begin{tabular}{@{}llllll@{}}
\multicolumn{6}{@{}l@{}}{class 4:} \\
 {\itshape mi-} & + & {\itshape -ã̂} & $\rightarrow$ &  {\itshape myã̂} & `my' \\
 {\itshape mi-} & + & {\itshape -ɔ̂} & $\rightarrow$ & {\itshape myɔ̂} & `your ({\SG})' \\
 {\itshape mi-} & + & {\itshape -ɛ́} & $\rightarrow$ &  {\itshape myɛ́} & `his/her' \\
 {\itshape mi-} & + & {\itshape -áwɔ́} & $\rightarrow$ & {\itshape myáwɔ́} & `their' \\
\end{tabular}}

\eabox{\label{POSS82}\begin{tabular}{@{}llllll@{}}
\multicolumn{6}{@{}l@{}}{class 8:}  \\
 {\itshape bi-} & + & {\itshape -ã̂} & $\rightarrow$ &  {\itshape byã̂} & `my' \\
 {\itshape bi-} & + & {\itshape -ɔ̂} & $\rightarrow$ & {\itshape byɔ̂} & `your ({\SG})' \\
 {\itshape bi-} & + & {\itshape -ɛ́} & $\rightarrow$ &  {\itshape byɛ́} & `his/her' \\
 {\itshape bi-} & + & {\itshape -áwɔ́} & $\rightarrow$ & {\itshape byáwɔ́} & `their' \\
\end{tabular}}

\noindent I assume that possessee agreement prefixes of agreement classes 2 through 8 are tonally specified with an H tone, even if their vowel is deleted in front of the vowel-initial possessor stem, while those for agreement classes 1 and 9 have an associated L tone. This explains the tonal differences for the third person singular and the first and second person plural. 


\subsection{Reflexive pronoun {\itshape mɛ́dɛ́}}
\label{sec:REFL}

The reflexive pronoun {\itshape mɛ́dɛ́} `self' is used both as a reflexive and an emphatic function. With the reflexive function, the reflexive pronoun is restricted to the object and adjunct positions. 

In object noun phrases, {\itshape mɛ́dɛ́} `self' directly follows the object pronoun, indicating identity between the subject and the object, as in \REF{mede} for all animate person categories.\footnote{The other non-speech act participant categories, namely agreement classes 3 through 9, all adhere to the same pattern.}

\ea \label{mede}
  \ea \label{medea}
  \glll  mɛ́ nyɛ́ {\bfseries mɛ̂} {\bfseries mɛ́dɛ́} \\
          mɛ-H nyɛ̂-H mɛ̂ mɛ́dɛ́   \\
          1\textsc{sg}-\textsc{prs} see-{\R} 1\textsc{sg}.{\OBJ} self   \\
    \trans `I see myself.'
\ex\label{medeb}
  \glll  wɛ́ nyɛ́ {\bfseries wɛ̂} {\bfseries mɛ́dɛ́} \\
          wɛ-H nyɛ̂-H wɛ̂ mɛ́dɛ́   \\
          2\textsc{sg}-\textsc{prs} see-{\R} 2\textsc{sg}.{\OBJ} self   \\
    \trans `You (sg.) see yourself.'
\ex\label{medec}
  \glll  á nyɛ́ {\bfseries nyɛ̂} {\bfseries mɛ́dɛ́} \\
          a-H nyɛ̂-H nyɛ̂ mɛ́dɛ́   \\
          1-\textsc{prs} see-{\R} 1\textsc{sg}.{\OBJ} self   \\
    \trans `S/he sees her/himself.'
\ex\label{meded}
  \glll  yá nyɛ́ {\bfseries bî} {\bfseries mɛ́dɛ́} \\
          ya-H nyɛ̂-H bî mɛ́dɛ́   \\
          1\textsc{pl}-\textsc{prs} see-{\R} 1\textsc{pl}.{\OBJ} self   \\
    \trans `We see ourselves.'
\ex\label{medee}
  \glll  bwá nyɛ́ {\bfseries bê} {\bfseries mɛ́dɛ́} \\
          bwa-H nyɛ̂-H bê mɛ́dɛ́   \\
          2\textsc{pl}-\textsc{prs} see-{\R} 2\textsc{pl}.{\OBJ} self   \\
    \trans `You (pl.) see yourselves.'
\ex\label{medef}
  \glll  bá nyɛ́ {\bfseries bɔ̂} {\bfseries mɛ́dɛ́} \\
          ba-H nyɛ̂-H b-ɔ̂ mɛ́dɛ́   \\
          2-\textsc{prs} see-{\R} 2-{\OBJ} self   \\
    \trans `They see themselves.'
\z
\z


The reflexive pronoun can appear in subject position, as in \REF{medeax}. This construction, however, is pragmatically more marked, as subjects are typically topics (\sectref{sec:IS}) and as such less marked. With the reflexive pronoun in subject position, the lines between reflexive and emphatic function become more blurred. 

 \ea \label{medeax}
  \glll  mɛ̀ {\bfseries mɛ́dɛ́} mɛ́  nyɛ́ mɛ̂ \\
          mɛ mɛ́dɛ́ mɛ-H nyɛ̂-H mɛ̂   \\
        1\textsc{sg} self 1\textsc{sg}-\textsc{prs} see-{\R} 1\textsc{sg}.{\OBJ} self   \\
    \trans `ME, I see myself.'
\z   


It is also grammatical to drop the reflexive pronoun altogether and only use the object pronoun, as in \REF{medesans}. For the first and second person singular and plural, it is inferred that the subject and object are coreferential. For third persons, however, the use of the object pronoun alone would lead to the interpretation that subject and object are not coreferential. Therefore, {\itshape mɛ́dɛ́} `self' must be used in these environments. The use of the reflexive pronoun is also preferred over the object pronoun alone with the first and second person, probably for the parallel structure with the third person reflexive marking. 

 \ea \label{medesans}
  \glll  mɛ̀ mɛ́  nyɛ́ mɛ̂ \\
          mɛ mɛ-H nyɛ̂-H mɛ̂   \\
        1\textsc{sg} 1\textsc{sg}-\textsc{prs} see-{\R} 1\textsc{sg}.{\OBJ} self   \\
    \trans `I see myself.'
\z
    
Reflexive pronouns are also used in adjunct position, as shown in \REF{refladj}.   

 \ea \label{refladj}
  \glll mɛ̀ nzí sâ yî púù yá mɛ̂ mɛ́dɛ́ \\
        mɛ  nzí sâ yî púù yá mɛ̂ mɛ́dɛ́ \\
        1\textsc{sg} {\PROG}.{\PST} do 7.{\DEM}.{\PROX} $\emptyset$7.reason 7:{\ATT} 1\textsc{sg} self \\
    \trans `I was doing this for myself.'
\z    

With an emphatic function, the reflexive pronoun can be used in all kinds of noun phrases: subject, object, and adjunct.  Typically, {\itshape mɛ́dɛ́} `self' follows a pronoun, as with the subject pronoun in \REF{mede1} and in the adjunct in \REF{mede2}.

\ea \label{mede1}
  \glll bímbú lɛ́ mámbòngò mâ wɛ̀ {\bfseries mɛ́dɛ́} dígɛ̂ {\bfseries mɛ́dɛ́} \\
         bímbú lɛ́ ma-mbòngò mâ wɛ̀ mɛ́dɛ́ dígɛ̂ mɛ́dɛ́ \\
       $\emptyset$5.amount 5:{\ATT} ma6-plant 6.{\DEM}.{\PROX} 2\textsc{sg}.{\SBJ} self look.{\IMP} self   \\
    \trans `The amount of these plants, yourself, look yourself,'
\z

\ea \label{mede2}
  \glll   àà ndáwɔ̀ dé tù nyɛ̀ {\bfseries mɛ́dɛ́} támé   \\
           àà ndáwɔ̀ dé tù nyɛ̀ mɛ́dɛ́ támé      \\
          1.{\COP} $\emptyset$9.house {\LOC} inside 1. {\SBJ} self alone \\
    \trans `He is in his house all by himself.'
\z

Unlike with its reflexive function, the reflexive pronoun can also occur after other parts of speech than pronouns when used emphatically. In \REF{mede3}, for instance, it occurs after the finite verb form, referring to the subject. Given that other words, such as the finite verb form in this example, can enter between the subject and reflexive pronoun, I analyze {\itshape mɛ́dɛ́} `self' as a free morpheme.

\ea \label{mede3}
  \glll  à múà {\bfseries mɛ́dɛ́} nyá mùdì \\
          a múà mɛ́dɛ́ nyá m-ùdì   \\
         1 be.almost self real \textsc{n}1-person    \\
    \trans `He was himself a real (old) man.'
\z


{\itshape mɛ́dɛ́} `self' also follows nouns (instead of pronouns), as in \REF{mede4} where it follows the left-dislocated object noun.


\ea \label{mede4}
  \glll sá {\bfseries mɛ́dɛ́} mɛ̀ nzí sâ yî \\
        sá mɛ́dɛ́ mɛ nzí sâ yî \\
        $\emptyset$7.thing self 1\textsc{sg} {\PROG}.{\PST} do 7.{\DEM}.{\PROX} \\
    \trans `The thing itself, I was doing this.'
\z





















\section{Other pro-forms}
\label{sec:proform}

Other pro-forms substitute other elements than nouns on the phrasal, clausal, or sentential level. In this section, I describe interrogative pro-forms, pro-adverbs, the pro-clausal tag question marker {\itshape ngáà}, and pro-sentence forms.



\subsection{Interrogative pro-forms}
\label{sec:INTERR}

I treat interrogative pro-forms separately from interrogative pronouns {\itshape nzá} `who' and {\itshape gyí} `what' (\sectref{sec:INTERRPRO}) which clearly replace a noun phrase. In contrast, interrogative pro-forms can replace a range of word classes or phrases. For instance, {\itshape líní} `when' might stand instead of an adverb {\itshape tɛ̂} `now' or a complex oblique noun phrase {\itshape mbvû lã̂} `last year'. 

Interrogative pro-forms differ in their structural complexity. Simple forms only include the interrogative word. Complex forms require the interrogative form to occur in a special construction, either with the locative preposition {\itshape ɛ́} or in a noun + noun attributive construction.




\subsubsection{Simple interrogative pro-forms}
\label{sec:INTERRS}

Simple interrogative pro-forms are used in questions to replace either a noun phrase or a temporal adverb. They occur independently as free morphemes. Gyeli has three pro-forms, as listed in \REF{simpinterr},  that occur in simple interrogative constructions.

\ea\label{simpinterr}
  \ea  líní `when'
    \ex vɛ́ `where'
    \ex ná `how'
    \z
\z

The interrogative pro-form {\itshape líní} `when' exclusively occurs in simple constructions, no matter if it occurs at the beginning or the end of the question phrase, as shown in \REF{lini}.

\ea \label{lini}
  \ea  \label{lini1}
  \glll    mùdũ̂ à kɛ́ màkítì [nàkùgúù] \\
      m-ùdũ̂ a kɛ̀-H ma-kítì {\db}nà-kùgúù\\
              \textsc{n}1-man 1.{\PST}1 go-{\R} ma6-market {\db}{\SIM}-$\emptyset$7.evening \\
   \trans `The/a man went to the market yesterday.'
\ex\label{lini2}
 \glll    {\bfseries líní} mùdũ̂ à kɛ́ màkítì  \\
           líní m-ùdũ̂ a kɛ̀-H ma-kítì\\
             when \textsc{n}1-man 1.{\PST}1 go-{\R} ma6-market \\
    \trans `When did the man go to the market?'
\ex\label{lini3}
 \glll     mùdũ̂ à kɛ́ màkítì {\bfseries líní} \\
            m-ùdũ̂ a kɛ̀-H ma-kítì líní\\
              \textsc{n}1-man 1.{\PST}1 go-{\R} ma6-market when \\
    \trans `When did the man go to the market?'
\z
\z

 The main use of {\itshape líní} `when' is in temporal adverbial clauses (\sectref{sec:ADVfull}) to express simultaneity. In fact, its use as an interrogative pro-form is rare, even if possible, as shown in \REF{lini}. When a question asks for a temporal adjunct in the answer, speakers prefer to use complex interrogatives, which can be translated as `what day' and `what time', as discussed in the next section.

In contrast to {\itshape líní} `when', the other two interrogative pro-forms {\itshape vɛ́} `where' and {\itshape ná} `how' only appear in simple constructions if they are used in-situ at the end of the phrase, as illustrated in \REF{wherevE} and \REF{howna}.

\ea \label{wherevE}
  \glll ɛ́ ná mwánɔ̀ nùù {\bfseries vɛ́} \\
       ɛ́ ná m-wánɔ̀ nùù vɛ́ \\
       {\LOC} how \textsc{n}1-child 1.{\COP} where  \\
    \trans `What! Where is the child?'
\z


\ea \label{howna}
  \glll kó mbúmbù nyɛ̀ nzí lèmbò dyùù bɔ̂ fàmíì bá bùdì {\bfseries ná} \\
       kó mbúmbù nyɛ nzí lèmbo dyùù b-ɔ̂ fàmíì bá b-ùdì ná \\
       {\EXCL} $\emptyset$1.namesake 1.{\SBJ} {\PROG}.{\PST} know kill 2-{\OBJ} $\emptyset$1.family 2:{\ATT} ba2-person how \\
    \trans `Oh namesake, how could he kill them, the family of people?'
\z

\noindent If they are used phrase initially, however, they obligatorily occur in a complex construction with the preposition {\itshape ɛ́}, as discussed in the following.


\subsubsection{Complex interrogative pro-forms}
\label{sec:INTERRC}

Complex interrogative words can be complex in different ways. They can be formed with (i) the locative preposition {\itshape ɛ́} (\sectref{sec:PREP}) or  (ii) a noun + noun attributive construction (\sectref{sec:CONC}). 


Gyeli has two interrogative pro-forms that are constructed with the locative preposition {\itshape ɛ́} preceding the interrogative form: {\itshape ɛ́ ná} `how'  and {\itshape ɛ́ vɛ́} `where'. Examples of both  interrogatives that require a temporal and a manner adjunct in the answer are given in \REF{eve} and \REF{ena}, respectively. Ungrammatical examples stem from elicited grammaticality judgments.

\ea \label{eve}
  \ea  \label{eve1}
  \glll mùdũ̂ à kɛ́ [màkítì] \\
    m-ùdũ̂ a kɛ̀-H {\db}ma-kítì \\
              \textsc{n}1-man 1.{\PST}1 go-{\R} {\db}ma6-market  \\
    \trans `The/a man went to the market.'
\ex\label{eve2}
 \glll    {\bfseries ɛ́} {\bfseries vɛ́} m-ùdũ̂ à kɛ́  \\
            ɛ́ vɛ́ m-ùdũ̂ a kɛ̀-H \\
             {\LOC} where \textsc{n}1-man 1.{\PST}1 go-{\PST} \\
   \trans `Where did the man go?'
\ex[*]{\label{eve3}
 \glll    {\bfseries vɛ́} m-ùdũ̂ à kɛ́  \\
        vɛ́ m-ùdũ̂ a kɛ̀-H \\
        where \textsc{n}1-man 1.{\PST}1 go-{\PST} \\
   \trans `Where did the man go?'}
\z
\z

\ea \label{ena}
  \ea  \label{ena1}
  \glll mùdũ̂ à kɛ́ màkítì [nà màtúà] \\
    m-ùdũ̂ a kɛ̀-H ma-kítì {\db}nà màtúà \\
             \textsc{n}1-man 1.{\PST}1 go-{\R} ma6-market {\db}{\COM} $\emptyset$1.car\\
   \trans `The/a man went to the market by car.'
\ex\label{ena2}
 \glll {\bfseries ɛ́} {\bfseries ná} mùdũ̂ à kɛ́ màkítì  \\
        ɛ́ ná m-ùdũ̂ a kɛ̀-H ma-kítì \\
             {\LOC} how \textsc{n}1-man 1.\textsc{prs} go-{\R} ma6-market \\
    \trans `How did the man go to the market?'
\ex[*]{\label{ena3}
 \glll {\bfseries ná} mùdũ̂ à kɛ́ màkítì  \\
        {\db}ná m-ùdũ̂ a kɛ̀-H ma-kítì \\
            {\db}how \textsc{n}1-man 1.\textsc{prs} go-{\R} ma6-market \\
    \trans `How did the man go to the market?'}
\z
\z

\noindent The complex form {\itshape ɛ́ ná} `how' is also used as a greeting in \REF{enahow}

\ea \label{enahow}
  \glll mbúmbù {\bfseries ɛ́} {\bfseries ná} \\
        mbúmbù ɛ́ ná \\
        $\emptyset$1.namesake {\LOC} how  \\
    \trans `Namesake, how is it?'
\z



The second option for complex interrogatives are interrogative pro-forms such as {\itshape vɛ́} `which' and {\itshape níyɛ̀} `how many', which occur as the second constituent in an attributive construction with a noun and an attributive marker, as in \REF{intercomp1} and explained in detail in \sectref{sec:NIntPro}.


\ea \label{intercomp1}
  \ea \label{intercomp1a}
 \glll    lèfû lé vɛ́   \\
            le-fû lé vɛ́   \\
             le5-day 5:{\ATT} which \\
   \trans `Which day?'
\ex\label{intercomp1b}
 \glll    màfû má níyɛ̀  \\
            ma-fû má níyɛ̀  \\
             ma6-day 6:{\ATT} how.may  \\
   \trans `How many days?'
\z
\z


\noindent Besides asking for nominal entities or their quantities in the answer, these interrogatives systematically combine with temporal nouns such as {\itshape dúwɔ̀} `day' or {\itshape wùlà} `time, hour' in order to form temporal interrogative constructions.










\subsection{Pro-adverbs {\itshape mpù} and {\itshape ndɛ̀náà }}
\label{sec:ProADV}

The pro-adverbs {\itshape mpù} and {\itshape ndɛ̀náà} generally refer to the manner of an event and are translated with `like this'. The semantic difference between the two pro-forms is not clear. They seem to have a very similar distribution in the corpus and speakers state that they can be used interchangeably. However, {\itshape mpù} is significantly more frequent in the corpus with 24 occurrences in comparison to six occurrences of {\itshape ndɛ̀náà}.

Both pro-adverbs signal a non-verbal gesture or part of the communication that is happening simultaneously to speech time. In \REF{likethis1}, the speaker is communicating the number of his children by showing two fingers; {\itshape mpù} is signaling this non-verbal gesture.

\ea \label{likethis1}
  \glll  bwánɔ̀ {\bfseries mpù} [gesture showing 2]\\
          b-wánɔ̀ mpù \\
         ba2-child like.this   \\
    \trans `that many children [gesture showing 2].'
\z

\noindent Similarly, in \REF{likethis2}, {\itshape ndɛ̀náà} indicates that the greeting is ongoing between the speech act participants.

\ea \label{likethis2}
  \glll     mɛ́ sùmɛ́lɛ́ bê {\bfseries ndɛ̀náà} \\
            mɛ-H sùmɛlɛ-H bê ndɛ̀náà \\
              1\textsc{sg}-\textsc{prs} greet-{\R} 2\textsc{pl}.{\OBJ} like.that   \\
    \trans `I greet you like this.'
\z

\noindent {\itshape mpù} often introduces the use of ideophones, as in \REF{likethis3}.

\ea \label{likethis3}
  \glll yɔ́ɔ̀ Nzàmbí njí {\bfseries mpù} bã̂ã̂ã̂ã̂ njì dígɛ̀ mpù \\
        yɔ́ɔ̀ Nzàmbí njî-H mpù bã̂ã̂ã̂ã̂ njì dígɛ mpù \\
        so $\emptyset$1.{\PN} come-{\R} like.this {\IDEO}:walking far come look like.this \\
    \trans `So Nzambi comes like this [depiction of walking a long distance], comes looking like this.'
\z

\noindent The deictic reference of pro-adverbs can also be anaphoric rather than signaling an ongoing or immediately following non-verbal communicative event. This is the case in \REF{likethis4}, for instance, where {\itshape ndɛ̀náà} summarizes the situation that the speaker has elaborated previously.


\ea \label{likethis4}
  \glll  bon pílì yí báàlá nà bɛ̀ {\bfseries ndɛ̀náà} ndɛ̀náà ndáà ná \\
        bon pílì yi-H báàla-H nà bɛ̀ ndɛ̀náà ndɛ̀náà ndáà ná \\
          good[French] when 7-\textsc{prs} repeat-{\R} {\COM} be like.that like.that also still  \\
    \trans `So, when it continues and is still like this and like that.'
\z

As \REF{likethis4} and \REF{likethis5} show, {\itshape mpù} and {\itshape ndɛ̀náà} `like this' can both occur directly after the finite verb, as expected for an adverb. While {\itshape mpù} is often followed by an object, this is not the case for {\itshape ndɛ̀náà} in the corpus. Speakers state, however, that it would be perfectly grammatical.

\ea \label{likethis5}
  \glll wɛ́ dyúwɔ́ {\bfseries mpù} bàmìntùlɛ̀ bɔ́gá bá tsígɛ̀ tsùk tsùk tsùk \\
         wɛ-H dyúwɔ-H mpù ba-mìntùlɛ̀ bɔ́-gá ba-H tsígɛ tsùk-tsùk-tsùk \\
        2\textsc{sg}-\textsc{prs} hear-{\R} like.this ba2-mouse 2-other 2-\textsc{prs} take.off {\IDEO}:rustling \\
    \trans `You hear like this the other mice take off [depiction of noise of mice].'
\z

{\itshape Mpù}, unlike {\itshape ndɛ̀náà}, is often preceded by the preposition {\itshape ɛ́}, as in \REF{likethis6}.

\ea \label{likethis6}
  \glll yɔ́ɔ̀ Nzàmbí dígɛ́ mísì {\bfseries ɛ́} {\bfseries mpù} \\
      yɔ́ɔ̀ Nzàmbí dígɛ-H m-ísì ɛ́ mpù \\
       so $\emptyset$1.{\PN} look-{\R} ma6-eye {\LOC} like.this  \\
    \trans `So Nzambi looks with the eyes like this.'
\z

\noindent Neither the specific function of {\itshape ɛ́} in combination with {\itshape mpù} nor its distribution are clear, however. 




\subsection{Pro-clausal {\itshape ngáà}}
\label{sec:ProClause}

The pro-clausal tag question particle {\itshape ngáà} is used to verify the truth value of a clause in leading polar questions (\sectref{sec:Questions}), as in \REF{Qtag1}. It is extra-clausal as evidenced by a phonetic break that separates {\itshape ngáà} from the main clause and its ability to occur by itself, for instance as a response to an interlocutor's statement. A {\itshape ngáà} response by itself expresses either surprise, a truth verification (`is that right?', `really?'), or agreement (`isn't that right!', `really!').

\ea \label{Qtag1}
  \glll {\bfseries ngáà} wɛ́ nyɛ́ mpù \\
       ngáà wɛ-H nyɛ̂-H mpù \\
       Q(tag) 2\textsc{sg}-\textsc{prs} see-{\R} like.this  \\
    \trans `Right, you see that?'
\z

\noindent {\itshape ngáà} appears both at the beginning of the question, as in \REF{Qtag1}, or at the end of it, as in \REF{Qtag2}.


\ea \label{Qtag2}
  \glll wɛ́ nyɛ́ mpù {\bfseries ngáà} \\
       wɛ-H nyɛ̂-H mpù ngáà \\
        2\textsc{sg}-\textsc{prs} see-{\R} like.this Q(tag) \\
    \trans `You see that, don't you?'
\z

The pro-clausal particle is used in both affirmative and negated questions. An example of the latter is given in \REF{Qtag3}.

\ea \label{Qtag3}
  \glll wɛ̀ɛ́ nyɛ́lɛ́ mpù {\bfseries ngáà} \\
       wɛ̀ɛ́ nyɛ̂-lɛ mpù ngáà \\
        2\textsc{sg}.{\NEG}.\textsc{prs} see-{\NEG} like.this Q(tag) \\
    \trans `You don't see that, do you?'
\z


Pro-clausal {\itshape ngáà} is also used independently on its own as a response to a statement, expressing surprise or verifying the truth value of the statement, comparable to English  `really?' or `is that true?'.

In affirmative questions, Gyeli also uses French loanwords or code-switching.\footnote{The status of these French words in Gyeli is not clear at the moment.} In \REF{Qtag4}, {\itshape ɛ̀sɛ́} taken from French {\itshape est-ce} `is it' is used as tag question marker. There seems to be a preference to use it phrase initially.

\ea \label{Qtag4}
  \glll  {\bfseries ɛ̀sɛ́} béé ndáà bèyá làwɔ́ fàlà \\
       ɛ̀sɛ́ béé ndáà bèya-H làwɔ-H fàlà \\
        is.it[French] 2\textsc{pl}.{\SBJ} also 2\textsc{pl}[Kwasio]-\textsc{prs} speak-{\R} $\emptyset$1.French  \\
    \trans `Isn't it, you (pl.) also, you speak French.'
\z

In contrast, {\itshape nɔ́ɔ̀} from French {\itshape non} `no' is used phrase finally with the same function, as in \REF{Qtag5}.

\ea \label{Qtag5}
  \glll  béé ndáà bèyá làwɔ́ fàlà {\bfseries nɔ́ɔ̀}\\
       béé ndáà bèya-H làwɔ-H fàlà nɔ́ɔ̀\\
        is.it[French] 2\textsc{pl}.{\SBJ} also 2\textsc{pl}[Kwasio]-\textsc{prs} speak-{\R} $\emptyset$1.French no[French]\\
    \trans `You (pl.) also, you speak French, isn't it?'
\z




\subsection{Pro-sentence forms}
\label{sec:ProSent}

Pro-sentence forms replace an entire sentence. They are typically answers to to polar questions (\sectref{sec:Questions}), making a statement about its truth value. They can, however, also occur as response to a statement that the speaker agrees or disagrees with. Gyeli has several pro-sentence forms for each agreement and disagreement signal. \REF{yes} provides a list of pro-forms that signal agreement. These different pro-forms seem to correlate with pragmatic and semantic differences. {\itshape ɛ́ɛ̀} seems to be the regular way to say `yes', while {\itshape ɛ̀hɛ́ɛ́} is used more emphatically to signal strong agreement. The exact use of the other pro-forms is less well understood.

\ea \label{yes}
  \ea \label{yes1} ɛ́ɛ̀ `yes'
\ex\label{yes2} áà `yes' 
\ex\label{yes3} èè `yes' 
\ex\label{yes4} ɛ́ɛ́ `yes' 
\ex\label{yes5} m̀ḿḿ `yes'
\ex\label{yes6} ɛ̀hɛ́ɛ́  `yes'
\z
\z

When asked for the translation of `yes', speakers would answer with \REF{yes1}. In natural speech as in the corpus, however, a range of other agreement signaling pro-forms are used. They all have in common that they only consist of a long vowel or nasal. The tonal melody and vowel length is crucial in distinguishing agreement from disagreement, as the segmentally similar but tonally different pairs in, for instance, \REF{yes4} and \REF{no2} show. Agreement signals have long segments with either a  falling  or L or H tone, as in  \REF{yes1} through \REF{yes4}. \REF{yes5} and \REF{yes6}, which are tonally identical, are used for emphatic agreement, as in English `exactly!'.
Also {\itshape yà}, or its emphatic form {\itshape yáà}, has been observed in the corpus. These forms are likely loanwords from German.\footnote{Some German loanwords from colonial times (until 1918) are still widespread in the area, for instance also in Mabi. These include, for instance, {\itshape dunkel} `dark' and {\itshape Dummkopf} `idiot', although Cameroonians are not always sure about their meaning.}

There are fewer pro-forms for disagreement than for agreement. The default form is {\itshape tɔ̀sâ} in \REF{no1}, which is derived from the negative polarity item {\itshape tɔ̀} (\sectref{sec:InvQUANT1}) and the noun {\itshape sâ} `thing'.

\ea \label{no}
  \ea \label{no1} tɔ̀sâ `no'
\ex\label{no2} ɛ́'ɛ̂ `no'
\ex\label{no3} ḿ'm̂ `no'
\z
\z

The other two forms in \REF{no2} and \REF{no3} are identical in their tonal pattern. They also differ from agreement forms in their relative brevity. Disagreement forms are never lengthened, but rather short. In \REF{no2} and \REF{no3}, the medial glottal stop reinforces the impression of short segments.












\begin{table}[b]%long distance to avoid split footnote
\begin{tabularx}{\textwidth}{r@{\qquad\qquad}XlXll}
 \lsptoprule
\textsc{agr}  & 	-vúdũ̂ 	& 	-fúsì        &  	-ɛ́sɛ̀	& -ɔ́(nɛ́)gá	  & numerals  \\
class & 	`one' 	& 	`different'  &  `all'	       & `other'	  & `2' through `5'  \\
 \midrule
1 & 	m- 		& 	m-	& 	w-		&      n- 	&  	 \\
2 & 	bà- 		& 	bà-	& 	b-		&      b- 	& bá-	 \\
3 & 	m- 		& 	$\emptyset$- 	& w-		& 	w-  	& 	 \\
4 & 	mì- 		& 	mì-	& 	my-		& 	my- 	& mí- 	\\
5 & 	lè- 		& 	lè-	& 	l-		&      l- 	&  	\\
6 & 	mà-  		& 	mà-	& 	m-		&      m- 	 & má-	\\
7 & 	$\emptyset$- 	& $\emptyset$- & y- &     y- 	 &  	\\
8 & 	bì- 		& 	bì-	& 	by-		&     by- 	& bí- 	 \\
9 & 	m- 		& $\emptyset$-& ny-	&     ny- 	 &  	\\
 \lspbottomrule
\end{tabularx}
\caption{Agreement prefixes of nominal modifiers}
\label{Tab:DEICMOD}
\end{table}




\section{Elements of the nominal phrase}
\label{sec:NAdjuncts}

In this section, I describe all the elements that occur in a noun phrase, apart from the noun, which has been discussed in \sectref{sec:N}. As a basic classification criterion, I distinguish nominal modifiers that agree with the head noun and those that do not, i.e.\ which are invariable.  
	
	%(tests: 1. multiple stacking of same category not allowed so Dem and Poss not same category
		  % 2. modification with intensifier bvubvu for adjectives, not determiners)
	

Agreeing elements in the Gyeli noun phrase differ in the form of agreement encoding. For some parts of speech, agreement is achieved through a prefix. This is the case for all elements discussed in \sectref{sec:MODAgrPre} and \sectref{sec:MODAgrPL}. Other elements, such  as demonstratives in \sectref{sec:DEM} and attributive markers in \sectref{sec:ATT}, show agreement through an unbound agreeing morpheme that differs across different agreement classes.

Invariable modifiers, i.e.\ elements that do not agree with the head noun, differ in their position relative to the noun. Some invariable modifiers precede the head noun (\sectref{sec:InvQUANT1}), some occur post-nominally (\sectref{sec:InvQUANT2}). The structure of the noun phrase and its various types are presented in \chapref{sec:NP} as well as the gender and agreement system.






\subsection{Modifiers with agreement prefix}
\label{sec:MODAgrPre}


Gyeli has five patterns of agreement prefixes, as shown in \tabref{Tab:DEICMOD}. Agreement prefixes attach to a variety of agreement targets, including numerals and some quantifiers.\footnote{These nominal modifiers could be argued to constitute adjectives on the basis of their agreement prefixes. Adjectives are, however, usually taken to be `lexical' (or content) words, according to \citet[121]{rijkhoff2002}, and describe properties such as ``size, weight, color, age, and value''. In Gyeli, they do not take agreement prefixes, as described in \sectref{sec:QUAL}. At the same time, these modifiers do not pattern with nouns either. There are, however, some nouns that function as quantifiers, as described in \sectref{sec:NomQUANT}.} All modifiers in \tabref{Tab:DEICMOD} follow the head noun.


While agreement prefixes of specific agreement classes are often similar in their shape, there are differences that define distinct agreement patterns. The agreement patterns for -{\itshape vúdũ̂} `one' and -{\itshape fúsì} `different' are only distinguished in agreement classes 3 and 9. The agreement patterns for -{\itshape ɛ́sɛ̀} `all' and -{\itshape ɔ́(nɛ́)gá} `other' differ in their agreement class 1 prefix. For semantic reasons, agreement prefixes for plural numerals only ever allow plural agreement prefixes. They are different from other agreement patterns in that they are the only ones to take an H tone prefix; agreement prefixes of all other patterns that have a tone bearing unit always have an L tone.

Some differences can be explained on a phonological basis, namely vowel deletion or assimilation in the prefix if the following stem starts with a vowel. This is, for instance the case with class 2 {\itshape bà}- before consonants in comparison to class 2 {\itshape b}- before vowels. Differences in prefix shape that are conditioned by phonological rules are not taken as evidence for different agreement patterns. In the following, I present each prefix agreement pattern and the lexical stems that take it.



\subsubsection{{\itshape -vúdũ̂} `one, same'} 
\label{sec:one}

{\itshape -vúdũ̂} can denote both the cardinal numeral `1' and the deictic modifier meaning `same'. It is distinct from the agreement pattern of the other agreeing numerals `2' through `5' in the L tone on CV- prefixes.

As the cardinal numeral `1', {\itshape -vúdũ̂} logically only occurs with singular entities it modifies. If it is used in order to express identity of entities, however, {\itshape -vúdũ̂} also takes an agreement prefix for plural classes, as shown in \tabref{Tab:same}.

\begin{table}
\begin{tabular}{ll ll}
 \lsptoprule
cl. 1 & mùdì & {\bfseries m}-vúdũ̂  & `one/same person' \\
cl. 2 & bùdì & {\bfseries bà}-vúdũ̂ & `same people' \\
cl. 3 & nkɛ̌ & {\bfseries m}-vúdũ̂́ & `one/same basket' \\
cl. 4 & mi-nkwɛ̌ & {\bfseries mì}-vúdũ̂ & `same baskets' \\
cl. 5 & le-dùndà & {\bfseries lè}-vúdũ̂ & `one/same sparrow' \\
cl. 6 & ma-dùndà & {\bfseries mà}-vúdũ̂ &  `same sparrows' \\
cl. 7 & síngì & {\bfseries $\emptyset$}-vúdũ̂ & `one/same cat' \\
cl. 8 & be-síngì & {\bfseries bè}-vúdũ̂ & `same cats' \\
cl. 9 & ndáwɔ̀ & {\bfseries m}-vúdũ̂ & `one/same house' \\
 \lspbottomrule
\end{tabular}
\caption{{\AGR}-{\itshape vúdũ̂} `one/same' in various agreement classes}
\label{Tab:same}
\end{table}


\subsubsection{{\itshape -fúsì} `different'} 
-{\itshape fúsì} `different' follows the noun it modifies just as the other modifiers that show agreement through a prefix.  Examples for {\itshape fúsì} `different' in different agreement classes are provided in \tabref{Tab:different}.

\begin{table}
\begin{tabularx}{\textwidth}{XXXl}
 \lsptoprule
cl. 1 & mùdì & {\bfseries m}-fúsì  & `a different person' \\
cl. 2 & bùdì & {\bfseries bà}-fúsì & `different people' \\
cl. 3 & nkɛ̌ & {\bfseries $\emptyset$}-fúsì & `a different basket' \\
cl. 4 & mi-nkwɛ̌ & {\bfseries mì}-fúsì & `different baskets' \\
cl. 5 & le-dùndà & {\bfseries lè}-fúsì & `a different sparrow' \\
cl. 6 & ma-dùndà & {\bfseries mà}-fúsì &  `different sparrows' \\
cl. 7 & síngì & {\bfseries $\emptyset$}-fúsì & `a different cat' \\
cl. 8 & be-síngì & {\bfseries bè}-fúsì & `different cats' \\
cl. 9 & ndáwɔ̀ & {\bfseries $\emptyset$}-fúsì & `a different house' \\
 \lspbottomrule
\end{tabularx}
\caption{{\AGR}-{\itshape fúsì} `different' in various agreement classes}
\label{Tab:different}
\end{table}







\subsubsection{{\itshape -ɛ́sɛ̀} `all'} 
\label{sec:ModAll}

%Non-numeral quantifiers can semantically be distinguished as intersective, universal, or proportionality quantifiers in the nominal domain (D- quantifiers). 
% Finally, {\itshape proportionality} quantifiers such as `most', `half of', or `many of' relate a given quantity to a set of entities. \citet[398]{zerbian2008} propose that Bantu languages generally use complex morphosyntactic constructions to express these. 

%Investigating Gyeli quantifiers show, however, that just like in many other languages these semantic distinctions do not map onto distinct construction types. Rather, Gyeli distinguishes four types of quantifiers which do not correspond with the semantic groupings presented above. Gyeli quantifiers can either be expressed as a nominal genitive construction parallel to the English expression `a multitude of x'. This construction type is the most frequent one and discussed in  \sectref{sec:NomQUANT}. Then,  Also, there are invariable quantifiers (see \sectref{sec:InvQUANT}) that either precede or follow the noun, but they do not agree with it. Further, the `all' quantifier {\itshape -ɛ́sɛ̀} is a modifier and agrees with its head noun as shown in this section. Finally, reduplication is a means of expressing `each' in a distributive sense as decribed in  \sectref{sec:RedQUANT}.

The universal quantifier {\itshape ɛ́sɛ̀}  `all' agrees with the head noun through an agreement prefix.  Universal quantifiers express totality and contain items such as `all' and `every' \citep[394]{zerbian2008}. \tabref{Tab:All} provides examples of the quantifier for all agreement classes showing the agreement prefix in bold. The agreement prefix for `all' is the same as the possessee agreement of possessor roots. As most other modifiers, `all' follows the head noun.

\begin{table}
\begin{tabularx}{\textwidth}{XXXl}
 \lsptoprule
cl. 1 & mùdì & {\bfseries w}-ɛ́sɛ̀ & `all (the parts of) the person' \\
cl. 2 & bùdì & {\bfseries b}-ɛ́sɛ̀ & `all people' \\
cl. 3 & nkwɛ̌ & {\bfseries w}-ɛ́sɛ̀ & `all (the parts of) the basket' \\
cl. 4 & mi-nkwɛ̌ & {\bfseries my}-ɛ́sɛ̀ & `all baskets' \\
cl. 5 & le-dùndá & {\bfseries l}-ɛ́sɛ̀ & `all (the parts of) the sparrow' \\
cl. 6 & ma-dùndà & {\bfseries m}-ɛ́sɛ̀ & `all sparrows' \\
cl. 7 & síngì & {\bfseries y}-ɛ́sɛ̀ & `all (the parts of) the cat' \\
cl. 8 & be-síngì & {\bfseries by}-ɛ́sɛ̀ & `all cats' \\
cl. 9 & ndáwɔ̀ & {\bfseries ny}-ɛ́sɛ̀ & `all the house' \\
 \lspbottomrule
\end{tabularx}
\caption{{\AGR}-{\itshape ɛ́sɛ̀} `all' in various agreement classes}
\label{Tab:All}
\end{table}

In Gyeli, {\itshape ɛ́sɛ̀} `all' is typically used with plural nouns.  Also singular forms can, however, be modified by -{\itshape ɛ́sɛ̀} `all'  in a specific context, which is also shown in \tabref{Tab:All}. This special context requires a situation where a typical singular entity consists of or is cut up into several parts. Taking the example of a cat, {\itshape síngì yɛ́sɛ̀} `all the cat' would mean that a cat is cut up into different parts, but then all the parts are used, which is different from meaning `the whole cat' (\sectref{sec:mandjimo}), as shown in \REF{allcat}.

\ea \label{allcat}
  \ea \label{allcat1}
 \gll  síngì y-ɛ́sɛ̀  \\
          $\emptyset$7.cat 7-all  \\
    \trans `all (the parts of) the cat'
\ex \label{allcat2}
  \gll    síngì mànjìmɔ̀ \\
              $\emptyset$7.cat whole \\
    \trans `the whole cat (in its entirety)'
\z
\z















\largerpage
\subsubsection{{\itshape -ɔ́(nɛ́)gá} `(an)other'}
\label{sec:other}

The full form `other' in careful speech is {\itshape -ɔ́nɛ́gá}. 
In fast speech, however, a shortened form {\AGR}-{\itshape ɔ́gá} is used where {\itshape nɛ́} is omitted. The option to omit {\itshape nɛ́} is indicated by the brackets in \tabref{Tab:Other}.

\begin{table}
\begin{tabularx}{\textwidth}{XXXl}
 \lsptoprule
cl. 1 & mùdì & {\bfseries n}-ɔ́(nɛ́)gá & `another person' \\
cl. 2 & bùdì & {\bfseries b}-ɔ́(nɛ́)gá & `other people' \\
cl. 3 & nkɛ̌ & {\bfseries w}-ɔ́(nɛ́)gá & `another basket' \\
cl. 4 & mi-nkwɛ̌ & {\bfseries my}-ɔ́(nɛ́)gá & `other baskets' \\
cl. 5 & le-dùndà & {\bfseries l}-ɔ́(nɛ́)gá & `another sparrow' \\
cl. 6 & ma-dùndà & {\bfseries m}-ɔ́(nɛ́)gá & `other sparrows' \\
cl. 7 & síngì & {\bfseries y}-ɔ́(nɛ́)gá & `another cat' \\
cl. 8 & be-síngì & {\bfseries by}-ɔ́(nɛ́)gá & `other cats' \\
cl. 9 & ndáwɔ̀ & {\bfseries ny}-ɔ́(nɛ́)gá & `another house' \\
 \lspbottomrule
\end{tabularx}
\caption{{\AGR}-{\itshape ɔ́ (nɛ́) gá} `other' in various agreement classes}
\label{Tab:Other}
\end{table}






\subsubsection{Anaphoric marker {\itshape ndɛ̀}}
\label{sec:ANAfree}

The anaphoric marker {\itshape ndɛ̀} signals reference to an entity that has been mentioned before in the discourse. It occurs in two variants: (i) with an agreement prefix and (ii) as the stem only without an agreement prefix.  The variant with agreement prefix is more frequent in the text corpus with almost six times more agreeing than free stem forms. A natural text example of {\itshape ndɛ́} with an agreement prefix is given in \REF{41t}.

\ea \label{41t}
  \glll bèdéwò {\bfseries bíndɛ̀} byɔ̀ mɛ́ lɔ́ njì lɛ́bɛ̀lɛ̀ bédéwò bà wɛ̀\\
        be-déwò bí-ndɛ̀ byɔ̂ mɛ-H lɔ́ njì lɛ́bɛlɛ H-be-déwò bà wɛ̀ \\
           be8-food 8-{\ANA} 8.{\OBJ} 1-\textsc{prs} {\RETRO} come  follow be8-food {\AP} 2\textsc{sg}.{\OBJ}  \\
    \trans `That (aforementioned) food, I have come to look for the food at your place.'
\z


Anaphoric markers have their own set of agreement prefixes, as summarized in \tabref{Tab:ANA}, which occur with no other part of speech. 

\begin{table}
\begin{tabularx}{.5\textwidth}{Xl}
 \lsptoprule
\textsc{agr} class & Prefix form \\
  \midrule
 1 & nú-  \\
 2 & bá- \\
 3 & wɔ́- \\
4 & mí- \\
5 & lé- \\
6 & má- \\
7 & yí- \\
8 & bí- \\
9 & nyí- \\
  \lspbottomrule
\end{tabularx}
\caption{Agreement prefixes of the anaphoric marker {\itshape ndɛ́}} 
\label{Tab:ANA}
\end{table}

\noindent % hyphenation
I view these agreement prefixes as grammaticalized from demonstratives (\sectref{sec:DEM}). First, the prefixes are segmentally identical to the proximal demonstrative paradigm involving a plain vowel (as opposed to the long vowels of the distal paradigm). The tonal pattern differs, however, since the prefix that attaches to {\itshape ndɛ̀} has an H tone rather than a falling tone as in the proximal paradigm. 

Second, demonstratives and the anaphoric marker are functionally and semantically related. They both serve to pick out referents from a set of entities. The anaphoric marker can be understood as a specification of general demonstratives in that it points the addressee to a referent that is not spatially distant, but that has come up in the discourse before. This specification seems, however, optional since both demonstratives in anaphoric contexts and anaphoric markers can appear independently of each other.



Another possibility would be to analyze the CV morph as an attributive marker. As shown in \sectref{sec:ATT}, many of the attributive markers across different agreement classes have  a CV shape with a plain vowel and an H tone. Most attributive markers link a noun to a second constituent that could be another noun or another part of speech, such as an adjective or interrogative pronoun, as discussed in \sectref{sec:CONC}. Thus, this analysis would also make sense syntactically. Arguments against this explanation, however, concern the form of some attributive markers and their distribution. First, the attributive marker forms of agreement classes 1, 3, 7, and 9 differ from the CV shape element found with {\itshape ndɛ̀}. For instance, in agreement class 1, the attributive marker is {\itshape wà}, while {\itshape ndɛ̀} would be preceded by {\itshape nú}-; in agreement class 7, the attributive marker is {\itshape yá}, but {\itshape ndɛ̀} is preceded by {\itshape yí}-. Second, there are examples where {\itshape ndɛ̀} plus its preceding CV morph follow a true attributive, as shown in \REF{ANAnoATT}. This makes it clear that the morph cannot be an attributive marker.

\ea \label{ANAnoATT}
  \glll mùdì {\bfseries wà} {\bfseries nú}ndɛ́ dígɛ́ mísì. \\
       m-ùdì wà nú-ndɛ̀ dígɛ-H m-ísì \\
        \textsc{n}1-person 1:{\ATT} 1-{\ANA} look-{\R} ma6-eye \\
    \trans `That (aforementioned) person thinks very hard [lit. looks with his eyes].'
\z



Unlike other nominal modifiers that always agree with their head noun, the anaphoric marker can also appear with its stem only.  When following an identificational marker, {\itshape ndɛ̀} occurs without an agreement prefix, as shown in \REF{ANA41t}, which was uttered at the end of a story.


\ea \label{ANA41t}
  \glll kàndá wɛ́ {\bfseries ndɛ̀} \\
        kàndá wɛ́ ndɛ̀ \\
        $\emptyset$7.proverb ID {\ANA}  \\
    \trans `The story is this.'
\z

\noindent The anaphoric marker {\itshape ndɛ́} also appears as a bare stem after nouns, as in \REF{ANA42t}, which is a response to a question about the chief.

\ea \label{ANA42t}
  \glll  àà kfúmá {\bfseries ndɛ̀} wà Nlúnzɔ̀ \\
        àà kfúmá ndɛ̀ wà Nlúnzɔ̀ \\
        ECXL $\emptyset$1.chief {\ANA}  1:{\ATT} $\emptyset$1.{\PN}  \\
    \trans `Ah, that chief from Nlunzo!'
\z







\subsubsection{Agreeing plural numerals} 
\label{sec:ModNUM}

Numerals may, depending on the language, form various numeral series such as enumeratives, cardinal, ordinal, or distributive numerals. In Gyeli, only a few cardinal numerals agree with the noun, namely -{\itshape vúdũ̂}  `1' (\sectref{sec:one}) and the numerals from `2' through `5', which have a different agreement pattern and are discussed in this section.
Generally, cardinal numerals are used attributively with nouns when counting items.\footnote{Gyeli numerals do not belong to one uniform category. There are monomorphemic (simple) and polymorphemic (complex) numerals.  Even simple numerals do not belong to one category in terms of parts of speech, but can be classified into three types: (i) agreeing modifiers -{\itshape vúdũ̂} `1' (\sectref{sec:one}) and numerals from `2' through `5' (this section), (ii) invariable modifiers (\sectref{sec:InvNUM}), and (iii) nouns (\sectref{sec:NomQUANT}). Complex numerals constitute either a coordination construction or a noun + modifier {\NP} or a combination of the two.}


The (cardinal) numerals -{\itshape báà} `2', -{\itshape láálɛ̀} `3', -{\itshape nã̂} `4', and -{\itshape tánɛ̀} `5' agree with their head noun. As enumeratives, i.e.\ in general counting without referring to a specific entity, the class 8 prefix {\itshape bí}- is used. The agreement prefixes of agreeing numerals and some examples are listed in \tabref{tab:NumPre}.\footnote{Since all the numerals that take agreement markers are inherently plural, singular class prefixes are never used.}

\begin{table}
\begin{tabular}{ll ll}
 \lsptoprule
\textsc{agr} class & {\AGR} prefix & Example & Gloss \\
  \midrule
2 & {\bfseries bá-} & {\itshape b-ùdì bá-báà} & `two people' \\
4 & {\bfseries mí-} & {\itshape mi-nkwɛ̂ mí-báà} & `two baskets' \\
6 & {\bfseries má-} & {\itshape ma-kí má-báà} & `two eggs' \\
8 & {\bfseries bé-} & {\itshape be-síngì bé-báà} & `two cats' \\
  \lspbottomrule
\end{tabular}
\caption{Agreement prefixes of modifying numerals}
\label{tab:NumPre}
\end{table}

All agreement prefixes on the agreeing numerals come with an H tone, in contrast to noun class prefixes and agreement prefixes of other modifiers (see \sectref{sec:MODAgrPre}).

One could argue that these agreement prefixes should not be analyzed as such, but may rather constitute attributive markers (\sectref{sec:ATT}) which have the same shape and tone as these prefixes. This is unlikely, however, because enumeratives always require a default prefix even though they are not modifying any noun. It is thus more likely to assume that numerals take a default prefix rather than a default attributive marker in a headless construction. Further, also the genitive marker takes H tone prefixes (\sectref{sec:GEN}).

The cardinal numerals from `2' through `5'  invariably follow the head noun, as shown in \REF{ModNum1}.

\ea \label {ModNum1}
    \ea  \gll b-ùdã̂ {\bfseries bá}-báà\\
    ba2-woman 2-two \\
    \glt  `two women'
    \ex \gll
    b-ùdã̂ {\bfseries bá}-láálɛ̀ \\
    ba2-woman 2-three \\
    \glt  `three women'

    %\vspace{1pc}

    \ex
    \gll    b-ùdã̂ {\bfseries bá}-nã̂\\
    ba2-woman 2-four \\
    \glt `four women'
     \ex
     \gll    b-ùdã̂ {\bfseries bá}-tánɛ̀ \\
    ba2-woman 2-five \\
    \glt  `five women'
    \z
\z

The same noun phrase structure is used in the formation of complex numerals that involve an underlying arithmetic operation of multiplication. In this case, the agreeing numeral will follow a nominal base numeral, as shown in \REF{ModNum2}, to form multiples of the base.

\ea \label {ModNum2}
\ea  \glll màwúmɔ̀ mábáà\\
ma-wúmɔ̀ má-báà\\
ma6-ten 6-two \\
\glt `twenty [10 x 2]'
 \ex \glll màwúmɔ̀ máláálɛ̀ \\
ma-wúmɔ̀ má-láálɛ̀ \\
   ma6-ten 6-three \\
\glt `thirty [10 x 3]'

%\vspace{1pc}

\ex
\glll bìbwúyà bínã̂ \\
bi-bwúyà {\bfseries bí}-nã̂\\ 
bi8-hundred 2-four \\
\glt `four hundred [100 x 4]'
\ex
\glll bàtɔ́dyínì bátánɛ̀ \\
ba-tɔ́dyínì {\bfseries bá}-tánɛ̀ \\
   ba2-thousand 2-five \\
\glt `five thousand [1000 x 5]'
\z
\z


Agreeing numerals `2' through `5' can never modify singular nouns for semantic reasons. They therefore lack any singular counterparts. I still distinguish them from modifiers discussed in the next section since those modifiers do occur with singular forms.
















\subsection{Modifiers with plural agreement only}
\label{sec:MODAgrPL}
 

There are two modifiers in Gyeli, the genitive marker {\itshape ngá} and {\itshape nyá} ‘big’,
which never take an agreement prefix for singular agreement classes, but require them for plural classes. Based on this characteristic, I classify them as a special subtype of modifiers. They differ, however, in many other properties. First, the genitive marker {\itshape ngá} only occurs in noun + noun constructions (\sectref{sec:CONC}), following the head noun it modifies. In contrast, {\itshape nyá} `big' precedes the head noun and is, together with the invariable negative polarity item {\itshape tɔ̀}, the only element that can precede the head noun. The genitive marker {\itshape ngá} and {\itshape nyá} `big' also differ in the tonal pattern of their agreement prefixes: {\itshape ngá} takes an H tone CV prefix, while agreement prefixes of {\itshape nyá} are underlyingly toneless. 





\subsubsection{Genitive marker {\itshape ngá}}
\label{sec:GEN}

Gyeli has a split genitive/attributive system, using different sets of associativity markers depending on the status of the head noun. In Bantu studies, these markers are also called associative or connective markers \citep{velde2013}. 
The genitive marker {\itshape ngá} is used instead of an attributive marker (\sectref{sec:ATT}) if the second constituent in a noun + noun construction is a proper name, as illustrated in \REF{house}. This highlights the special status of proper names in contrast to common nouns (\sectref{sec:properN}).

\ea \label{house}
  \ea \label{house1}
 \gll    ndáwɔ̀ {\bfseries ngá} Àdà \\
                $\emptyset$9.house {\GEN} $\emptyset$1.{\PN}\\
    \trans `Ada's house'
\ex \label{house2}
  \gll     ndáwɔ̀ {\bfseries nyà} m-bvùlè \\
                $\emptyset$9.house 9:{\ATT} \textsc{n}1-Bulu \\
    \trans `the Bulu\footnote{Bulu describes a neighboring ethnic group to the Bagyeli as well as their language which is classified as Bantu A74.} man's house'
\z
\z

\noindent Further, the genitive marker is used in the interrogative pronoun constructions such as {\itshape pú'ù ngá nzá} `for whom' when the answer could potentially be a proper name.  In question words where something else than a proper name is expected as an answer, as in {\itshape pú'ù yá gyí} `for what', the attributive is used.

The genitive marker only takes an agreement marker if the preceding possessee noun occurs in the plural. If {\itshape ndáwɔ̀} `house' in \REF{house} was in its plural form, the example would change as in \REF{housex} with a plural marker on {\itshape ngá}.

\ea \label{housex}
  \ea \label{housex1}
 \gll    mà-ndáwɔ̀ {\bfseries má-ngá} Àdà \\
                ma6-$\emptyset$9.house 6-{\GEN} $\emptyset$1.{\PN}\\
    \trans `Ada's houses'
\ex \label{housex2}
  \gll     mà-ndáwɔ̀ {\bfseries má} m-bvùlè \\
                ma6-house 6:{\ATT} \textsc{n}1-Bulu \\
    \trans `the Bulu man's houses'
\z
\z


\noindent If it is singular, however, the genitive marker takes a default form {\itshape ngá}. \tabref{Tab:nga} shows the agreement pattern of genitive markers with the non-agreeing singular forms in the left and the agreeing plural forms in the right column.

\begin{table}
\begin{tabular}{ll ll}
 \lsptoprule
\multicolumn{2}{l}{Singular classes} & \multicolumn{2}{l}{Plural classes}\\\cmidrule(lr){1-2}\cmidrule(lr){3-4}
cl. 1 & ngá &  cl. 2 & bá-ngá \\
cl. 3 & ngá &  cl. 4 & mí-ngá \\
cl. 5 & ngá &  cl. 6 & má-ngá \\
cl. 7 & ngá &  cl. 8 & bé-ngá \\
cl. 9 & ngá &        &     \\
 \lspbottomrule
\end{tabular}
\caption{Agreement marking of genitive markers}
\label{Tab:nga}
\end{table}

The agreement prefix, although it seems to be identical with the attributive marker, belongs prosodically to the genitive word {\itshape ngá}. In contrast, following speakers' intuitions, the attributive marker is prosodically an independent word. I therefore do not view agreeing plural forms of the genitive linker  as constructions containing both attributive and genitive markers. Instead, the H tone agreement prefixes are parallel to those used with agreeing plural numerals.

There is another logical possibility to explain the H tone on the agreement prefix, namely leftwards high tone spreading from the {\itshape -ngá} root. I rule this possibility out for two reasons.
First, high tone spreading from the right to the left does occur in Gyeli, but it seems to be restricted to the verbal domain (as with underlyingly toneless verb extension morphemes, which are discussed in \sectref{sec:HTSr}). Therefore, it seems unlikely that the H tone from the {\itshape -ngá} root would spread leftwards onto the prefix.

Second, contrasting cases of L tone CV- agreement prefixes that occur with other modifiers, such as {\itshape -vúdû} `same, one' and {\itshape -fúsì} `different', suggest that the CV- agreement prefixes for the genitive marker (and numerals from `2' through `5') are indeed specified for an H tone. The other modifiers also start with an H tone stem, but they still have CV- agreement prefixes that surface with an L tone. There could be a rule that H tone spreading is restricted to a certain class of agreement targets, but given these two arguments, it seems unlikely. The ultimate proof against H tone spreading, namely checking what happens with the CV- prefixes if the stem starts with an L tone, is not testable because all modifier roots  that take an H tone CV- agreement prefix ({\itshape -ngá} and the numerals `2' through `5') start with an H or HL mora, but never with an L.



\subsubsection{{\itshape nyá} `big'}
\label{sec:nya}


{\itshape -nyá} meaning `big', `important', `luxurious',  `beautiful'  could qualify as an adjective since it denotes a property of a noun. The semantic difference between {\itshape nyá} `big' and the adjective {\itshape nɛ́nɛ̀} `big' is that the second typically refers to size as in \REF{nya1}. {\itshape nyá}, however, talks more about the value as demonstrated in \REF{nya2}. In Cameroonian French, value is often translated with size so that a {\itshape grand panier} `big basket' could, besides referring to the size, also talk about its value.

\ea \label{nya}
  \ea \label{nya1}
 \gll  nkwě wá nɛ́nɛ̀\\
          $\emptyset$3.basket 3:{\ATT} big \\
    \trans `a/the big basket'
\ex \label{nya2}
  \gll   nyá nkwě \\
             big $\emptyset$3.basket \\
    \trans `a/the important/beautiful/luxurious basket'
\z
\z



Agreement of {\itshape nyá}  is only marked if the head noun comes in a plural form. If the head noun is singular, {\itshape nyá} is invariable as shown in \tabref{Tab:nya}. This behavior is similar to the genitive marker discussed in \sectref{sec:GEN}.

\begin{table}
\begin{tabular}{ll ll}
 \lsptoprule
\multicolumn{4}{l}{Singular classes} \\
 \midrule
cl. 1 & nyá & m-ùdì & `important person'  \\
cl. 3 & nyá & nkwě  & `great basket'        \\
cl. 5 & nyá & le-dùndá & `big sparrow'      \\
cl. 7 & nyá & lé       & `great tree'            \\
cl. 8b & nyá & bwálɛ̀ & `beautiful canoe'  \\
cl. 9 & nyá &   ndáwɔ̀ & `luxurious house'   \\
 \midrule
\multicolumn{4}{l}{Plural classes} \\
 \midrule
cl. 2 & ba-nyá & b-ùdì & `important people' \\
cl. 4 & mi-nyá & mì-nkwě & `great baskets'  \\
cl. 6 & ma-nyá & mà-dùndá & `big sparrows' \\
cl. 8a & be-nyá & bè-lé & `great trees' \\
 \lspbottomrule
\end{tabular}
\caption{{\itshape nyá} in various agreement classes}
\label{Tab:nya}
\end{table}

\noindent Another particularity is the syntactic position of {\itshape nyá}, preceding the noun where\-as basically all other modifiers follow the noun.


\subsection{Modifiers with agreeing free morpheme}
\label{sec:MODAgrForm}

There are two nominal modifiers in Gyeli which do not express agreement with the head noun through a prefix that attaches to a root that is consistent across different agreement classes, but that have free agreeing morphemes which differ across agreement classes. This is the case for demonstratives and for the attributive marker.
		
		

\subsubsection{Demonstratives}
\label{sec:DEM}



Gyeli has two sets of demonstratives distinguishing different degrees of distance between the speaker and the object or person he or she is talking about. One set of demonstratives, the proximal demonstratives, refers to objects or persons close to the speaker. Distal demonstratives are employed when the object or person in question is further away from the speaker (but not necessarily close to the addressee). 

Proximal and distal demonstratives are formally distinguished by different tonal patterns and vowel lengthening of the distal pronouns. \tabref{Tab:dem} contrasts the two sets of demonstratives. While proximal demonstratives end in a simple vowel with a falling HL tonal pattern, distal demonstratives all have a lengthened vowel with an H tone.

\begin{table}
\begin{tabular}{lll}
 \lsptoprule
 & proximal & distal \\
  \midrule
 1 & \itshape{nû} & \itshape{núú} \\
 2 & \itshape{bâ} & \itshape{báá} \\
 3 & \itshape{wɔ̂} & \itshape{wɔ́ɔ́} \\
4 & \itshape{mî} & \itshape{míí} \\
5 & \itshape{lɛ̂} & \itshape{lɛ́ɛ́} \\
6 & \itshape{mâ} & \itshape{máá} \\
7 & \itshape{yî} & \itshape{yíí} \\
8 & \itshape{bî} & \itshape{bíí} \\
9 & \itshape{nyî} & \itshape{nyíí} \\
  \lspbottomrule
\end{tabular}
\caption{Gyeli demonstratives}
\label{Tab:dem}
\end{table}

\noindent Both proximal and distal demonstratives follow the noun they modify in a noun phrase as shown in \REF{DEM}.

\ea \label{DEM}
  \ea  \label{DEM1}
  \gll     m-ùdì nû \\
                \textsc{n}1-man 1.{\DEM}.{\PROX} \\
    \trans `this man'
\ex\label{DEM2}
 \gll     m-ùdì núú \\
                \textsc{n}1-man 1.{\DEM}.{\DIST} \\
    \trans `that man'
\z
\z

These demonstratives are also used as presentational or identificational markers in non-verbal predicates of the pattern `This is a house.' Such constructions are discussed in \sectref{sec:nonverbalC}.





\subsubsection{Attributive markers}
\label{sec:ATT}

Attributive markers constitute another class of function words that agree with their head noun. In Bantu studies, they are also called genitive, connective, or associative markers \citep{velde2013}. Gyeli has a split system with a ``genitive'' paradigm marking possessors that are expressed by proper names in the second constituent (\sectref{sec:GEN}) and an ``attributive'' paradigm marking all other nominal associativity constructions. 


Attributive markers serve as a linking element between a noun and typically another noun, as shown in \REF{CONM}. They also link a noun to an adjective, verb, interrogative, or numeral, as described in \sectref{sec:CONC}. 

\ea \label{CONM}
  \ea  \label{CONM1}
  \gll     síngì {\bfseries yá} jìí \\
                $\emptyset$7.cat 7:{\ATT} $\emptyset$7.forest \\
    \trans `wildcat [lit. cat of the forest]'
\ex\label{CONM2}
 \gll    lè-lɔ̂ {\bfseries lé} síngì \\
                le5-ear 5:{\ATT} $\emptyset$7.cat \\
    \trans `the cat's ear'
\z
\z

Attributive markers are also used in relative clauses, as exemplified in \REF{RELPro} and discussed in detail in \sectref{sec:Relativeclauses}.

\ea \label{RELPro}
  \ea  \label{RELPro1}
  \gll     síngì yá yí kwè \\
                $\emptyset$7.cat 7:{\ATT} 7.\textsc{prs} fall \\
    \trans `the cat that falls'
\ex\label{RELPro2}
 \gll     síngì yá mɛ́ nyɛ̂ \\
                $\emptyset$7.cat 7:{\ATT} 1\textsc{sg}.\textsc{prs} see \\
    \trans `the cat that I see'
\z
\z

\citet{meeussen67}, and later \citet[219]{velde2013}, posit that the canonical form for Bantu attributives is {\AGR}-{\itshape a}, a root -{\itshape a} which is preceded by an agreement prefix. Many Gyeli attributives follow this canonical form. Exceptions to this tendency are found, however, in classes 4, 5, and 8 which come with high and mid vowel roots rather than with -{\itshape a}, as shown in \tabref{Tab:AGRCON}. For this reason, I do not segment attributive markers in glosses, but generally use the colon ``:{\ATT}''.  Attributive markers in Gyeli typically have an H tone, except for those of classes 1 and 9, which both come with an L tone.

\begin{table}
\begin{tabular}{ll}
 \lsptoprule
\textsc{agr} class & {\ATT} marker \\
  \midrule
 1 & wà  \\
 2 & bá \\
 3 & wá \\
4 & mí \\
5 & lé \\
6 & má \\
7 & yá \\
8 & bé \\
9 & nyà \\
  \lspbottomrule
\end{tabular}
\caption{Attributives in the different agreement classes}
\label{Tab:AGRCON}
\end{table}














\subsection{Prenominal invariable modifiers}
\label{sec:InvQUANT1}

Elements that can occur prenominally in Gyeli are restricted in number and distribution. In simple noun phrases, only {\itshape nyá} `big' (\sectref{sec:nya}) and {\itshape tɔ̀} `any' (\sectref{sec:to}) can precede the noun. {\itshape nyá} `big' agrees with the head noun only if the noun is plural, otherwise it is invariable; {\itshape tɔ̀} `any' is always invariable. Other prenominal elements precede second constituents in noun + noun constructions, serving as connectors. They also differ in their agreement behavior ranging from agreeing elements such as the attributive marker (\sectref{sec:ATT}) to those that only agree with plural nouns as the genitive marker (\sectref{sec:GEN}) and the invariable similative marker (\sectref{sec:SIMword}). I discuss the two invariable prenominal elements in the following, namely the negative polarity item {\itshape tɔ̀} and the similative marker {\itshape ná}.


\subsubsection{Negative polarity item {\itshape tɔ̀} `any'}
\label{sec:to}

The negative polarity item {\itshape tɔ̀} `any' does not agree with the head noun, as shown in \REF{to}. {\itshape tɔ̀} `any' never agrees, no matter if it precedes a singular or plural noun. In that, it differs from the genitive marker {\itshape ngà} (\sectref{sec:GEN}), which  agrees with plural nouns.

\ea \label{to}
  \ea  \label{to1}
  \gll    mɛ̀ɛ́ nyɛ́-lɛ́ [{\bfseries tɔ̀} m-ùdì] \\
              1\textsc{sg}.\textsc{prs}.{\NEG} see-{\NEG} {\db}{\bfseries any} \textsc{n}1-person \\
    \trans `I don't see anyone.'
\ex \label{to2}
  \gll    mɛ̀ɛ́ nyɛ́-lɛ́ [{\bfseries tɔ̀} b-ùdì] \\
              1\textsc{sg}.\textsc{prs}.{\NEG} see-{\NEG} {\db}{\bfseries any} ba2-person \\
    \trans `I don't see any people.'
\z
\z

The use of {\itshape tɔ̀} in negated sentences is grammatically not obligatory, as shown in \REF{to3}, where the same sentence as in \REF{to1} occurs without {\itshape tɔ̀} `any'. Semantically, however, there is a difference in that no person at all is seen in \REF{to1}, while \REF{to3} negates a specific, known person.

\ea \label{to3}
 \gll  mɛ̀ɛ́ nyɛ́-lɛ́ m-ùdì  \\
          1\textsc{sg}.\textsc{prs}.{\NEG} see-{\NEG} \textsc{n}1-person  \\
    \trans `I don't see the person.'
\z


%***also as negative coordinator***


\subsubsection{Similative marker {\itshape ná}}
\label{sec:SIMword}

The similative marker {\itshape ná} occurs both as a free morpheme and a prefix. The free morpheme {\itshape ná} functions as a predicative marker that is restricted to naming constructions, linking the noun {\itshape jínɔ̀} `name' with a proper name, as shown in \REF{SIMword}. It is distinct from other copula forms discussed in \sectref{sec:SIM} and is labelled as a ``similative'' marker for its segmental identity with the similative prefix, which is more productive and more obviously denotes similarity (\sectref{sec:SIM}). 


\ea \label{SIMword}
  \glll èè mɛ̀ jínɔ̀ {\bfseries ná} Màmà \\
         èè mɛ̀ j-ínɔ̀ ná Màmà \\
       yes 1\textsc{sg} le5-name {\SIM} $\emptyset$1.{\PN}  \\
    \trans `Yes, my name is Mama.'
\z

The free similative marker is invariable, even if the noun + noun construction has plural constituents. As illustrated in \REF{SIMword2}, number has to be identical in the first and second constituent, but the connecting similative marker {\itshape ná} does not change.

\ea \label{SIMword2}
  \glll bà mínɔ̀ {\bfseries ná} Màmà nà Màmbì \\
         bà m-ínɔ̀ ná Màmà nà Màmbì \\
        2 ma6-name {\SIM} $\emptyset$1.{\PN}  {\CONJ} $\emptyset$1.{\PN} \\
    \trans `Their names are Mama and Mambi.'
\z






\subsection{Postnominal invariable modifiers}
\label{sec:InvQUANT2}

Most modifiers in Gyeli occur after the noun. This is also true for most non-agreeing modifiers, such as invariable numerals and some quantifiers.

\subsubsection{Invariable numerals} 
\label{sec:InvNUM}


Gyeli has monomorphemic cardinal numerals which do not agree with the noun, as shown in \REF{ModNumx2}. As such, they might be thought of belonging to the same category of adjectives (\sectref{sec:QUAL}). In contrast to adjectives, however, they never occur in a construction involving an attributive marker.

\ea\label {ModNumx2}
    \ea
    \gll b-ùdã̂ ntùɔ́\\
    ba2-woman six \\
\glt `six women'
    \ex
    \gll
    b-ùdã̂ mpúɛ̀rɛ́ \\
    ba2-woman seven \\
\glt `seven women'

    %\vspace{1pc}

    \ex
    \gll  b-ùdã̂ lɔ̀mbì\\
    ba2-woman eight \\
\glt `eight women'
    \ex
    \gll b-ùdã̂ rèbvùá \\
    ba2-woman nine \\
\glt `nine women'
    \z
\z


\subsubsection{Quantifier {\itshape bvùbvù} `many, much'}
\label{sec:many}

{\itshape bvùbvù} `many, much'  is a  quantifier that does not agree with the head noun.\footnote{Under a formal-semantic concept, \citet[388]{zerbian2008} define `many' as an {\itshape intersective} quantifier, belonging to those ``quantifiers whose truth conditions can be given in terms of the intersection of the noun meaning and the predicate meaning."  Other intersective quantifiers are, for instance, `several', `few', `a certain/other', `some' or `no'. The authors state that most intersective quantifiers in Bantu languages agree with their head noun. This is not true for Gyeli, which has a range of non-agreeing quantifiers (or uses attributive constructions (\sectref{sec:NomQUANT}) in order to express quantifiers such as `many' or `few').}
It is not sensitive to a mass/count distinction and occurs both with countable and uncountable nouns alike, as shown in \REF{Invbvu1} and \REF{Invbvu2}.


\ea \label{Invbvu}
  \ea \label{Invbvu1}
 \gll  b-ùdì  bvùbvù \\
          ba2-people many  \\
    \trans `many people'
\ex \label{Invbvu2}
  \gll    mà-jíwɔ́ bvùbvù \\
              ma6-water much \\
    \trans `much water'
\z
\z

This quantifier has a nominal counterpart in agreement class 9 which can be used in a noun + noun attributive construction (\sectref{sec:NomQUANT}). The nominal quantifier has a different tone pattern, as shown in \ref{Invbvu1x}. 

\ea \label{Invbvu1x}
 \gll   bvúbvù nyà b-ùdì \\
        $\emptyset$9.multitude 9:{\ATT} ba2-people  \\
    \trans `many people'
\z    

{\itshape bvúbvù nyà} seems to be the more marked form which occurs less frequently than the invariable modifier. Possible meaning differences are subtle; speakers claim that both mean the same and can be used in the same contexts.





\subsubsection{Quantifier {\itshape mànjìmɔ̀} `whole, entire'}
\label{sec:mandjimo}

{\itshape mànjìmɔ̀} `whole, entire' is another invariable quantifier that follows the head noun, as in \REF{whole}. Despite the similarity to the nominal modifier {\itshape njìmɔ̀ wá} `a certain' and something that looks like a class 6 prefix, {\itshape mànjìmɔ̀} is not a noun since it lacks noun properties such as the possibility to be modified by, for instance, demonstratives or possessive pronouns, or entering a noun + noun attributive construction as the head.


\ea \label{whole}
  \ea  \label{whole1}
  \gll     púsí mànjìmɔ̀\\
                $\emptyset$7.bottle whole \\
    \trans `the whole bottle'
\ex\label{whole2}
 \gll     ndáwɔ̀ mànjìmɔ  \\
             $\emptyset$9.house whole  \\
    \trans `the entire house'
\ex\label{whole3}
 \gll   bè-síngì mànjìmɔ  \\
             be8-cat whole   \\
    \trans `the entire cats'
\z
\z

{\itshape mànjìmɔ̀} is sensitive to a mass/count distinction in that it does not appear with uncountable nouns, neither liquids nor granular aggregates, as shown in \REF{wholeno}. Using {\itshape mànjìmɔ̀} with mass nouns requires a specification of the physical entity, for instance a bottle as in \REF{wholeno3}.

\ea \label{wholeno}
  \ea[*]{\label{wholeno1}
  \gll   mà-tàngɔ̀ mànjìmɔ̀\\
        ma6-palm.wine whole \\
    \trans `the whole palm wine'}
\ex[*]{\label{wholeno2}
 \gll  ndísì mànjìmɔ  \\
    $\emptyset$3.rice whole  \\
    \trans `the entire rice'}
\ex \label{wholeno3}
  \gll     púsí (yá) má-vúdɔ̀ mànjìmɔ̀\\
                $\emptyset$7.bottle {\db}7:{\ATT} ma6-oil whole \\
    \trans `a whole bottle of oil'
\z
\z

\noindent In contrast to the singular form of granular aggregate mass nouns, which cannot occur with {\itshape mànjìmɔ̀}, their plural counterpart allows for its use as in \REF{massPL}. In this case, however, it is understood that the noun comes in packaged entities, for instance in sachets or bags, or that different types of the noun are involved.

\ea \label{massPL}
  \gll     mì-ndísì mànjìmɔ̀ \\
                mi4-rice whole \\
    \trans `the whole rice [all of its types or packages]'
\z

























\section{Elements of the verbal complex}
\label{sec:VerbAd}

In this section, I describe the elements that occur in a verbal predicate other than the verb, which has been outlined in \sectref{sec:V}. These elements include the subject-tense-aspect-mood-polarity (\textsc{stamp}) marker and two verbal particles that follow the inflected verb form.



\subsection[The subject-tense-aspect-mood-polarity marker]{The subject-tense-aspect-mood-polarity marker}
\label{sec:SCOP}

The subject-tense-aspect-mood-polarity (\textsc{stamp}) marker, following terminology coined by Anderson (\citeyear{anderson2011b}, \citeyear{anderson2015}), is a clitic directly preceding the inflected verb form. As a portmanteau morpheme, it encodes subject agreement as well as tense, mood, aspect, and negation. \tabref{Tab:SCOPAGR1} shows the basic segmental shape of the \textsc{stamp} marker in the different agreement classes, omitting the tonal pattern which changes across tense-mood, aspect, and negation categories. There are three different forms of the \textsc{stamp} marker for agreement class 1. {\itshape a} is the basic, unmarked form. {\itshape nyɛ} seems to be an instance of interference from Kwasio, but as Gyeli speakers use it so regularly, they mostly view it as part of their language. {\itshape nu} seems to be related to the demonstrative form and may be used as a more marked form for reference tracking. 


\begin{table}
\small
\begin{tabular}{ ll ll p{.6cm}llllllll}
 \lsptoprule
 1\textsc{sg} & 2\textsc{sg} & 1\textsc{pl} & 2\textsc{pl} & 1 & 2 & 3 & 4 & 5 & 6 & 7 & 8 & 9 \\
 \midrule
 mɛ  & wɛ & ya & bwa & a{\slash}nyɛ{\slash}nu & ba & wu & mi & le & ma & yi & be & nyi  \\
 \lspbottomrule
\end{tabular}

\caption{Segmental forms of the \textsc{stamp} marker in different {\AGR} classes}
\label{Tab:SCOPAGR1}
\end{table}

The tonal pattern and sometimes vowel length of the \textsc{stamp} marker change across different tense-mood categories, as shown in \tabref{Tab:SCOPAGR}, which lists the \textsc{stamp} markers' form and surface tones for all agreement classes in all tense-mood (\textsc{TM}) categories (\textsc{cat}). In combination with specific tonal patterns of the verb, the \textsc{stamp} marker tones instantiate basic tense-mood distinctions, as discussed in \sectref{sec:GramTM}.


\begin{table}
\fittable{%
\begin{tabular}{l ll ll p{.6cm}llllllll}
 \lsptoprule
\textsc{tm cat} & 1\textsc{sg} & 2\textsc{sg} & 1\textsc{pl} & 2\textsc{pl} & 1 & 2 & 3 & 4 & 5 & 6 & 7 & 8 & 9 \\
 \midrule
\textsc{prs} &  mɛ́   & wɛ́ & yá & bwá(á) & á/ nyɛ́/ nú & bá & wú & mí & lé & má & yí & bé & nyí  \\
\textsc{inch} &   mɛ̀ɛ́ & wɛ̀ɛ́ & yàá & bwàá & àá & bàá & wùú & mìí & lèé & màá & yìí & bèé & nyìí  \\ 
\textsc{fut} & {\bfseries mɛ̀ɛ̀} &  {\bfseries wɛ̀ɛ̀} & yáà & bwáà & {\bfseries àà/ nyɛ̀ɛ̀/ nùù} & báà & wúù & míì & léè & máà & yíì & béè & nyíì  \\
% \midrule
\textsc{pst1} & mɛ & wɛ & ya & bwa(a) & a/ nyɛ/ nu & ba & wu & mi & le & ma & yi & be & nyi  \\
\textsc{pst2} & mɛ́ɛ̀ & wɛ́ɛ̀ & yáà & bwáà & áà/ nyɛ́ɛ̀/ núù & báà & wúù & míì & léè & máà & yíì & béè & nyíì \\
% \midrule
\textsc{imp} & --  & --   & yá & -- & -- & -- & -- & -- & -- & -- & -- & -- & --  \\ 
\textsc{sbjv} & mɛ́   & wɛ́ & yá & bwá(á) & á/ nyɛ́/ nú & bá & wú & mí & lé & má & yí & bé & nyí  \\
 \lspbottomrule
\end{tabular}}
\caption{Patterns of \textsc{stamp} markers in different {\AGR} classes and TM categories}
\label{Tab:SCOPAGR}
\end{table}

The \textsc{fut} category has an exceptional tonal pattern for certain agreement classes, which are marked in bold. The vowel of the second person plural is either pronounced with a long or a short vowel if the tone is not a contour tone, i.e.\ if it is either H or L. 


Class 1 has {\itshape a} as a basic form and an alternate form {\itshape nyɛ}.\footnote{This form could originate from Kwasio.} At the same time, {\itshape nyɛ} is identical with the non-subject pronoun of agreement class 1. Both forms are equally used and speakers state that both are part of the Gyeli language, although the {\itshape a} form is more frequently found in texts.  Also, agreement class 1 has a third alternate form, namely {\itshape nu} which is identical with the class 1 demonstrative. It can, however, also be used as a \textsc{stamp} marker with the specific tonal pattern for each tense-mood category. In this, the class 1 \textsc{stamp} marker is exceptional because demonstratives of other agreement classes cannot function as a \textsc{stamp} marker.

\subsubsection*{Toneless \textsc{past 1} category} I suggest that, underlyingly, the L surface form of the \textsc{pst1} category is tonally not specified and only surfaces phonetically as L. This is comparable to other grammatical morphemes such as noun class prefixes or verbal derivation morphemes as discussed in \sectref{sec:toneless}. I view this phonetically L form as a tonally underspecified default form because it does not only occur in the \textsc{past 1} category, but also serves as general default form in other tense-mood categories when these are combined with true auxiliaries encoding aspect (\sectref{sec:ComplAUX}). It further provides the basic form from which the \textsc{present} category is derived with an H tone. Consequently, in the glossing of examples, the surface L \textsc{stamp} markers are represented as being toneless in the underlying line. \textsc{prs} \textsc{stamp} forms are underlyingly represented as toneless \textsc{stamp} markers which receive an H tone, characterizing this category.


\subsubsection*{Tone pattern in the \textsc{future} category}
As shown in \tabref{Tab:SCOPAGR}, the general pattern for the \textsc{future} is a long vowel with an HL tonal melody. While in other tense-mood categories the tonal and vowel length pattern is the same for each agreement class, in the \textsc{future}, the first and second person singular as well as the class 1 \textsc{stamp} marker deviate from this pattern, having a long vowel with an L tonal melody, as in \REF{FUTexept}.

\ea \label{FUTexept}
  \ea  mɛ̀ɛ̀ dè `I will eat'
\ex wɛ̀ɛ̀ dè `you will eat'
\ex àà/nyɛ̀ɛ̀ dè `s/he will eat'
\z
\z




The \textsc{stamp} marker precedes the finite verb, but is not part of the verb as it can, in fast speech, be assimilated or even omitted in certain tense-mood categories. I outline both cases in turn.

\subsubsection*{\textsc{stamp} marker assimilation}
Depending on the morphophonological shape of the \textsc{stamp} marker, this clitic can undergo assimilation with preceding vocalic material in fast speech. This applies mainly to the agreement class 1 \textsc{stamp} marker whose segmental material consists of the vowel {\itshape a}. Given that it is not preceded by a consonant, unlike the \textsc{stamp} markers of all other agreement classes, it can assimilate with the final vowel of a preceding verb or noun.

An example of \textsc{stamp} marker assimilation with both preceding verbs and nouns is provided in \REF{SCOPAs1}. In the first instance, the \textsc{stamp} marker assimilates to the verb {\itshape njì} `come' of the preceding phrase. Thus, \textsc{stamp} marker assimilation in fast speech is not restricted to in-phrase assimilation, but can also cross phrase boundaries.

\ea \label{SCOPAs1}
  \glll  à nj{\bfseries â} dyùmɔ́ bùd{\bfseries àà} dyùmɔ́ bùd{\bfseries àà} dyùmɔ́ bùd{\bfseries àà} dyùmɔ́ bùdì  \\
           a nji-H a dyùmɔ-H b-ùdì a dyùmɔ-H  b-ùdì a dyùmɔ-H b-ùdì   \\
          1.{\PST}1 come-{\R} 1.{\PST}1 heal-{\R} ba2-person 1.{\PST}1 heal-{\R} ba2-person 1.{\PST}1 heal-{\R} ba2-person  \\
    \trans `He came, he was healing people.'
\z

\noindent In the other assimilation instances in \REF{SCOPAs1}, the \textsc{stamp} marker assimilates to the nominal object {\itshape bùdì} `people', also of the previous phrase. In both cases, the final vowel of the noun is elided while the vowel of the \textsc{stamp} marker surfaces. At the same time, the tone of the omitted vowel survives, as seen with the contour tone on [njí + à]  $\rightarrow$ /njâ/. In the second instance, while vowel quality is assimilated to the \textsc{stamp} marker, both tone and vowel length survive, surfacing in a long vowel: [bùdì + à] $\rightarrow$ /bùdàà/.

\subsubsection*{\textsc{stamp} marker assimilation with proper names}
As seen in the previous example, in \textsc{stamp} marker assimilation it is usually the preceding vocalic material of a noun or verb that is deleted. This is different for \textsc{stamp} marker assimilation with proper names. Proper names do not change their vowel quality, but assimilate tonally to the class 1 \textsc{stamp} marker whose vocalic material is being elided, as shown in \REF{SCOPAs2}.

\ea \label{SCOPAs2}
  \ea \label{SCOPAs2a}
\glll Màmbì á kwè   $\rightarrow$ /Màmb{\bfseries í} kwè/ \\ 
      Màmbì a-H kwè \\
	$\emptyset$1.{\PN} 1-\textsc{prs} fall\\
	\trans `Mambi falls.'
\ex \label{SCOPAs2b}
  \glll Màmbì àá kwè  $\rightarrow$ /Màmb{\bfseries ìí} kwè/ \\
        Màmbì àá kwè \\
          $\emptyset$1.{\PN} 1.{\INCH} fall    \\
    \trans `Mambi is at the beginning of falling.'
\z
\z

Tonal changes on the proper name do not depend on tonal or phonological patterns of the name, but are controlled by the noun's feature of being a proper name (\sectref{sec:properN}).  The fact that proper names receive special morphosyntactic treatment in Gyeli is also seen in the split genitive system (\sectref{sec:GEN}). 

If the proper name's final tone and the \textsc{stamp} marker's tone are identical, there is no tonal or vocalic surface change, but the \textsc{stamp} marker simply is elided, as shown in \REF{SCOPAs3a} for  the proper name {\itshape Màmbì} ending in an L tone and a following L \textsc{stamp} marker and, in \REF{SCOPAs3b}, the proper name {\itshape Bìyã́} ending in an H tone in combination with a \textsc{prs} H tone \textsc{stamp} marker.


\ea \label{SCOPAs3}
  \ea  \label{SCOPAs3a}
   \glll   Màmbì à kwé   $\rightarrow$ /Màmbì kwé/ \\
          Màmbì a kwè-H \\
            $\emptyset$1.{\PN} 1.{\PST}1 fall-{\PST}  \\
    \trans `Mambi fell.'
\ex\label{SCOPAs3b}
\glll  Bìyã́ á sàgà $\rightarrow$ /Bìyã́ sàgà/ \\
       Bìyã́ a-H sàga \\
           $\emptyset$1.{\PN} 1-\textsc{prs} frighten    \\
    \trans `Biyang is frightened.'
 \z
\z

\noindent These cases are thus rather instances of \textsc{stamp} marker omission than \textsc{stamp} marker assimilation, which leads to the next section on \textsc{stamp} marker omission.



\subsubsection*{\textsc{stamp} marker omission}
Under certain circumstances, the \textsc{stamp} marker can be elided rather than assimilated. \textsc{stamp} marker omission requires some conditions. First, the clause has to be either in the \textsc{present} or the \textsc{recent past}, as in \REF{SCOPOMtense},\footnote{In this example, the class 1 \textsc{stamp} marker takes the alternate shape of the demonstrative rather than the default shape {\itshape a}. The shape of the class 1 \textsc{stamp} marker does not, however, influence the possibility of its omission.} while the other tense-mood categories (\sectref{sec:GramTM}) exclude \textsc{stamp} marker omission.  The parentheses indicate that the use of the \textsc{stamp} marker is optional while a lack of parentheses indicates that the \textsc{stamp} marker has to be used obligatorily.

\ea \label{SCOPOMtense}
  \ea  \label{SCOPOMtense1}
  \glll  kálɛ́ (nú) kwè $\rightarrow$ /kálɛ́ kwè/ \\
         kálɛ́ nu-H kwè \\
              $\emptyset$1.sister 1-\textsc{prs} fall \\
    \trans `The sister falls.'
\ex\label{SCOPOMtense2}
  \glll  kálɛ́ (nù) kwé $\rightarrow$ /kálɛ́ kwé/ \\
          kálɛ́ nu kwé  \\
              $\emptyset$1.sister 1.{\PST}1 fall \\
    \trans `The sister fell (recently).'
\ex \label{SCOPOMtense3}
  \glll  kálɛ́ núù kwé $\rightarrow$ */kálɛ́ kwé/ \\
       kálɛ́ núù kwè-H  \\
              $\emptyset$1.sister 1.{\PST}2 fall-{\PST} \\
    \trans `The sister fell (a long time ago).'
\ex\label{SCOPOMtense4}
  \glll  kálɛ́ nùù kwè $\rightarrow$ */kálɛ́ kwè/ \\
          kálɛ́ nùù kwè \\
              $\emptyset$1.sister 1.{\FUT} fall \\
    \trans `The sister will fall.'
\ex\label{SCOPOMtense5}
  \glll  kálɛ́ nùú kwè $\rightarrow$ */kálɛ́ kwè/ \\
        kálɛ́ nùú kwè \\
              $\emptyset$1.sister 1.{\INCH} fall \\
    \trans `The sister starts to fall.'
\z
\z

Second, a nominal subject has to surface, excluding, however, all nouns with a CV noun class prefix, as in \REF{NCOM1}, and plural subject nominals, as in \REF{NCOM2}.\footnote{Potential \textsc{stamp} marker omission was checked for a range of nouns, controlling for different tonal and phonological patterns, noun class affiliation, number, animacy, and different verbs. For simplicity, I only contrast two nouns in their singular and plural form, both belonging to gender 5/6. Most nouns in this gender have a CV prefix in both class 5 and class 6, but preceding a vowel-initial stem, the prefix only consists of a consonant, providing a good testing ground for different phonological environments.} (This is parallel to the potential omission of the attributive marker discussed in \sectref{sec:CONOM}, which has similar conditioning factors.)


%\noindent The \textsc{inch} category is somewhere in between. Mostly, omission is not allowed, as for example in \REF{INCHOM1}, but there are some exceptions as in \REF{INCHOM2}.

%\begin{exe}
%\ex\label{INCHOM}
%\begin{xlist}
%\ex \label{INCHOM1}
 % \gll  dàá kwè\\
 %             5.crab.{\INCH} fall \\
  %  \trans `The crab started to fall.'
%\ex\label{INCHOM2}
 %\gll  sɔ̀ yìí kwè \\
 %        7.saw 7.{\INCH} fall \\
  %  \trans `The saw started to fall.'
%\z
%\z



\ea \label{NCOM1}
  \ea  \label{NCOM1a}
   \glll   lèndzólɛ̀ lé kwè   $\rightarrow$ */lèndzólɛ̀ kwè/ \\
          le-ndzólɛ̀ le-H kwè   \\
            le5-tear 5-\textsc{prs} fall  \\
    \trans `The tear falls.'
\ex\label{NCOM1b}
\glll   màndzólɛ̀ má kwè   $\rightarrow$ */màndzólɛ̀ kwè/ \\
          ma-ndzólɛ̀ ma-H kwè   \\
            ma6-tear 6-\textsc{prs} fall  \\
    \trans `The tears fall.' 
\z
\z

In \REF{NCOM1}, both examples are excluded from \textsc{stamp} marker omission, based on the CV noun class prefix. In contrast, in \REF{NCOM2} with consonantal noun class prefixes,  only the plural noun in \REF{NCOM2b} does not allow \textsc{stamp} marker omission, but its singular counterpart in \REF{NCOM2a} does allow it.

\ea \label{NCOM2}
  \ea  \label{NCOM2a}
   \glll    jáwɛ̀ (lé) kwè  $\rightarrow$ / jáwɛ̀ kwè/ \\
             j-áwɛ̀ le-H kwè  \\
            le5-goliath.frog 5-\textsc{prs} fall  \\
    \trans `The goliath frog falls.'
\ex\label{NCOM2b}
\glll   máwɛ̀ má kwè   $\rightarrow$ */máwɛ̀ kwè/ \\
          m-áwɛ̀ ma-H kwè   \\
            ma6-goliath.frog 6-\textsc{prs} fall  \\
    \trans `The goliath frogs fall.' 
\z
\z

\noindent At the same time, these two examples also illustrate that animacy does not play a role, neither does general noun class affiliation since both examples belong to gender 5/6.

The \textsc{stamp} marker can also be elided with more complex noun phrases such as noun + possessive constructions, as in \REF{OmNPoss}. The tense reading comes from the \textsc{absolute completive} marker {\itshape mɔ̀} (\sectref{sec:mo}), which is restricted to the \textsc{recent past}.

\ea \label{OmNPoss}
  \glll nyɛ̀ nâ [sɔ́ wɔ́ɔ̀]\textsubscript{NP} nɔ̀ɔ́ mɔ̀ mwánɔ̀ \\
        nyɛ nâ {\db}sɔ́ w-ɔ́ɔ̀ nɔ̀ɔ̀-H mɔ̀ m-wánɔ̀ \\
      1 {\COMP} {\db}$\emptyset$1.friend 1-{\POSS}.2\textsc{sg} take-{\R} {\PRF} 1-child   \\
    \trans `She [says] `Your friend has taken the child.''
\z

There are also examples in the corpus showing that noun + noun attributive constructions may occur without a \textsc{stamp} marker, as in \REF{OmNN}. The tense reading of this utterance is ambiguous. As \textsc{stamp} marker omission only occurs in \textsc{present} and \textsc{recent past}, this narrows possible interpretations down. In \REF{OmNN}, formal marking allows both tense interpretations. Through common ground, however, it is clear that it has to be the \textsc{present} since all participants know that the road has not been built yet.

\ea \label{OmNN}
  \glll  mɛ́ dyúwɔ́ nâ [mpàgó wá pɔ́dɛ̀]\textsubscript{\NP} lã́ vâ \\
        mɛ-H dyúwɔ-H nâ {\db}mpàgó wá pɔ́dɛ̀ lã̀-H vâ \\
            1\textsc{sg}-\textsc{prs} hear-{\R} {\COMP} {\db}$\emptyset$3.street 3:{\ATT} $\emptyset$1.port pass-{\R} here \\
    \trans `I hear that the road to the port passes [will pass] here.'
\z

Third, the \textsc{stamp} marker can also be elided when the subject noun phrase is expressed by a pronoun, as in \REF{OmInterr} with the interrogative pronoun {\itshape nzá} `who'. The tense reading in this example comes from the shape of the \textsc{progressive} auxiliary, which has a different form for the \textsc{past} (\sectref{sec:PROG}).


\ea \label{OmInterr}
  \glll nzá nzíí mɛ̂ nyɛ̂ \\
    nzá nzíí mɛ̂ nyɛ̂ \\
         who {\PROG}.\textsc{prs} 1\textsc{sg}.{\OBJ} see\\
    \trans `Who is seeing me?'
\z



In a quotative index, which signals reported discourse, both the nominal subject and the \textsc{stamp} marker can be elided, as shown in \REF{SOm}, where a \textsc{stamp} marker would normally precede {\itshape kì} `say'.


\ea \label{SOm}
  \glll à kɛ̃́ɛ̃̀ nyî pɛ̀ dyúwɔ̀ à dígɛ́ɛ̀ à díg-â dígɛ́ɛ̀ kì nâ nzá nyɛ́ mɛ̂ \\
        a kɛ̃́ɛ̃̀ nyî pɛ̀ dyúwɔ̀ a dígɛ́ɛ̀ a dígɛ́ɛ̀ a dígɛ́ɛ̀ kì nâ nzá nyɛ̂-H mɛ̂ \\
       1.{\PST}1 go.{\PRF} enter there on.top 1.{\PST}1 watch.{\PRF} 1.{\PST}1 watch.{\PRF} 1.{\PST}1 watch.{\PRF} say {\COMP} who see-{\R} 1\textsc{sg}.{\OBJ}  \\
    \trans `He went inside there on top and watched and watched and watched. [He] says: ``Who sees me?''.'
\z

\noindent Following \citet[105]{guldemann2008}, not encoding the speaker in a quotative index is permissible in some languages since the speaker ``is normally the central character in a given discourse context'' so that ``such a participant tends to assume the minimum force of reference, and in some languages this is zero expressed''.

%The Gyeli text corpus shows clearly that the most common case involves the occurrence of the SCOP rather than its omission. \tabref{Tab:SCOPdistr} summarizes the variation in the presence of the SCOP. It shows that in 412 instances, out of a total of 472,\footnote{Identificational markers agreeing with the subject and constituting the predicate at the same time were not counted since they do not classify as SCOPs. Also special constructions of quotative indexes where the SCOP is present, but no verb, were not taken into consideration. A third case that was excluded concerns imperatives since imperatives never take a SCOP and thus do not show any variation.}
% the SCOP is present. This number includes both cases where the SCOP is the subject of the clause and cases where the SCOP is preceded by more noun phrase material such as nouns, noun + possessive constructions, and so forth.



%\begin{table}
%\centering
%\begin{tabular}{l|rr}
% \midrule
%SCOP present         & 412 & 87.3\% \\
%Assimilation             & 6     & 1.3\%    \\
%Omission with {\NP}    & 47   & 9.9\% \\
%Subject ellipsis       &  7      & 1.5\% \\
%(Imperatives)            & 18     & 3.7\%      \\
% \midrule
%Total                    & 472   \\
% \midrule
%\end{tabular}
%\caption{Variation of SCOP presence in the text corpus}
%\label{Tab:SCOPdistr}
%\end{table}

%While SCOP assimilation as discussed in \sectref{sec:SCOPAs} and complete subject ellipsis are rather rare, the omission of the SCOP when a noun phrase is present occurs in 9.9\% of the cases. Imperatives are listed in the table as well because their category is characterized by the absence of a SCOP.










\subsection{Verbal particles}
\label{sec:VParticle}

There are two other elements that appear in the Gyeli verbal complex, namely particles that follow the finite verb form. This includes the \textsc{absolute completive} marker {\itshape mɔ̀} and the verbal plural marker {\itshape nga}.

\subsubsection{\textsc{Absolute completive} {\itshape mɔ̀}}
\label{sec:mo}

The \textsc{absolute completive} marker, glossed as {\COMPL}, is a clitic that attaches to the inflected verb form. It has two variants, namely a postverbal particle {\itshape mɔ̀}, as in \REF{absmo1a}, and a long nasalized vowel with a falling HL tone \REF{absmo1b}. The latter is said to be more typical Gyeli, but {\itshape mɔ̀} is also productively used.\footnote{It can be excluded that {\itshape mɔ̀} is a loan form from Mabi since the cognate form in Mabi is {\itshape mà}.}


\ea \label{absmo1}
\ea\label{absmo1a}
  \glll    bà kwɛ̀lɔ́ mɔ̀ yɔ̂  \\
           ba kwɛ̀lɔ-H mɔ̀  y-ɔ̂ \\
             2.{\PST}1  cut-{\R} {\COMPL} 7-{\OBJ}  \\
    \trans `They have (already) cut it.'
\ex\label{absmo1b}
  \glll    bà kwɛ̀lɔ̃́ɔ̃̀ yɔ̂ \\
          ba kwɛ̀lɔ̃́ɔ̃̀ y-ɔ̂   \\
             2.{\PST}1 cut:{\COMPL}.{\R} 7-{\OBJ}    \\
    \trans `They have (already) cut it.'
\z
\z

\noindent The variant with the final lengthened and nasalized vowel is the contracted form of {\itshape mɔ̀}. The segmental nasal has been deleted, but nasality survived on the lengthened vowel. Also, the tonal pattern of the realis-marking H plus the L tone {\itshape mɔ̀} is maintained.

While there are some verbs, as in \REF{absmo1}, which can take both the {\itshape mɔ̀} form and the contracted form, other verbs can only take one or the other. {\itshape lámbɔ} `trap', for instance, can only take the contracted form as in \REF{absmo2a}, while the non-contracted form in \REF{absmo2b} is judged as ungrammatical. It seems to be lexically determined whether a verb takes one or the other or both forms.

\ea \label{absmo2}
\ea\label{absmo2a}
  \glll    mɛ̀ lámbɔ̃́ɔ̃̀ kù  \\
           mɛ lámbɔ̃́ɔ̃̀ kù  \\
             1\textsc{sg}  trap.{\R}.{\COMPL} $\emptyset$1.rat  \\
    \trans `I have (already) trapped the rat.'
\ex[*]{\label{absmo2b}
  \glll   mɛ̀ lámbɔ́ mɔ̀ kù \\
          mɛ̀ lámbɔ-H mɔ̀ kù \\
        1\textsc{sg} trap-{\R} {\COMPL} $\emptyset$1.rat    \\
    \trans `I have (already) trapped the rat.'}
\z
\z

\noindent There is a tendency for semi-auxiliaries, such as {\itshape kɛ̀} `go' and {\itshape sílɛ} `finish', to only occur with  the contracted \textsc{absolute completive} form, while {\itshape dyúwɔ} `hear' has only been observed to occur with the full form {\itshape mɔ̀}.

I consider {\itshape mɔ̀} a free morpheme rather than a verbal suffix since  tonal inflection of past tense and/or realis mood (\sectref{sec:SimpPred}) through the grammatical H tone happens on the preceding verb. If {\itshape mɔ̀} was a suffix, it would be the suffix (and the preceding toneless verbal derivation morphemes) that would take the grammatical H tone in non-final position. This, however, is not the case, as \REF{absmo1c} shows.

\ea \label{absmo1c}
  \glll    mɛ̀ lùng{\bfseries á} {\bfseries mɔ̀} bvùbvù\\
           mɛ lùnga-H mɔ̀  bvùbvù \\
             1\textsc{sg}  grow-{\R} {\COMPL}  lots \\
    \trans `I have (already) grown lots.'
\z

There is no other element that can occur between the verb and the verbal particle.  With verbs that require the comitative marker {\itshape nà} (\sectref{sec:MainVerbs}), for instance, {\itshape nà} follows the verbal particle, as in \REF{absmo7}.

\ea \label{absmo7}
  \glll     nà wɛ̀  làdɔ́ {\bfseries mɔ̀} nà ngã̀ \\
          nà wɛ  làdtɔ-H mɔ̀ nà ngã̀ \\
           Q 2\textsc{sg}.{\PST}1  meet-{\R} {\COMPL} {\COM} $\emptyset$1.healer  \\
    \trans `Have you ever/already met the healer?'
\z









\subsubsection{Verbal plural particle {\itshape (n)ga}}
\label{sec:nga}

The verbal plural particle {\itshape ga} and its variant {\itshape nga} pluralize addressees in an \textsc{imperative} construction (\sectref{sec:imp}). The particle occurs with the first and second person plural. Just like the \textsc{absolute completive} marker {\itshape mɔ̀}, the verbal plural particle directly follows the finite verb. The two particles never co-occur since they are restricted to different tense-mood categories. \REF{gaimp1} shows examples of the second person plural with the variant {\itshape ga}; \REF{ngaimp1} includes examples with the variant {\itshape nga}.

\ea \label{gaimp1}
  \ea  dê {\bfseries gà} `eat (2{\PL})!'
\ex kɛ̂ {\bfseries gà} `go (2{\PL})!'
\z
\z


\ea \label{ngaimp1}
  \ea \label{ngaimp1a} lã̂ {\bfseries ngà} `count (2{\PL})!'
\ex\label{ngaimp1b} gyàgâ {\bfseries ngà} `buy (2{\PL})!'
\ex sílɛ̂ {\bfseries ngà} `finish (2{\PL})!'
\z
\z

\largerpage
The first person plural, which also involves the use of the verbal plural particle, has the same structure as the second person plural, just with the addition of the first person plural \textsc{stamp} marker {\itshape yá}, as shown in \REF{gaimp1y} and \REF{ngaimp1y}. 

\ea \label{gaimp1y}
  \ea  yá dê {\bfseries gà} `let's eat!'
\ex yá kɛ̂ {\bfseries gà} `let's go!'
\z
\z

\ea \label{ngaimp1y}
  \ea \label{ngaimp1x} yá lã̂ {\bfseries ngà} `let's count!'
\ex\label{ngaimp1z} yá gyàgâ {\bfseries ngà} `let's buy!'
\ex yá sílɛ̂ {\bfseries ngà} `let's finish!'
\z
\z


In terms of the  distribution of {\itshape ga} versus {\itshape nga}, it seems that {\itshape ga} is the default case that is used with most verbs. {\itshape nga}, in contrast, appears definitely when a monosyllabic verb ends in a nasal vowel as it is the case with {\itshape lã̂}  `read, count', as in \REF{ngaimp1x}.  Nasal vowels are, however, not the only factor that triggers the plural particle to surface with a nasal since {\itshape nga} is also found with di- and trisyllabic verbs which do not end in a nasal vowel, as shown in \REF{ngaimp1z}.



There also seems to be a certain degree of free variation since both {\itshape ga} and {\itshape nga} can occur with the same verb form, as in \REF{IMPga}. During elicitations, speakers noted that both versions are equally good. 

\ea \label{IMPga}
  \ea \label{IMPgaa} dê {\bfseries gà} `eat (2{\PL})'
\ex dê {\bfseries ngá} `eat (2{\PL})'
\z
\z

{\itshape ga/nga} always follows the finite verb, as can be seen in the contrast between the positive and the negated cohortative forms (\sectref{sec:NEGti}) in \REF{IMPga2}. In both cases, the verbal particle pluralizes the subject.  In \REF{IMPga2a}, the finite verb is {\itshape gyàgâ} `buy' in a simple predicate. In contrast, \REF{IMPga2b} shows a complex predicate where the finite verb is the negation auxiliary {\itshape tí}. The verbal plural particle follows the auxiliary, preceding the lexical verb.

\ea \label{IMPga2}
  \ea \label{IMPga2a}
  \glll  yá {\bfseries gyàgâ} {\bfseries ngá} bèkálàdè \\
        yá gyàgâ nga bèkálàdè \\
           1\textsc{pl}-\textsc{prs} buy.{\IMP} {\PL} be8-book  \\
    \trans `Let's buy books!'
\ex\label{IMPga2b}
  \glll  yá {\bfseries tí} {\bfseries ngá} gyàgà békálàdè \\
        ya-H tí nga gyàga H-be-kálàdè \\
           1\textsc{pl}-\textsc{prs} {\NEG}.{\R} {\PL} buy {\OBJ}.LINk-be8-book \\
    \trans `Let's not buy books!'
\z
\z

I consider {\itshape ga/nga} as a particle rather than a suffix that attaches to the finite verb of an \textsc{imperative} construction. If the particle was a suffix, one would expect it to take the HL tone that is characteristic of the \textsc{imperative} category. Instead, the particle is underlyingly toneless, behaving like toneless CV- noun class prefixes. Phrase finally, {\itshape ga/nga} surfaces as L, as shown in \REF{gaimp1} and \REF{ngaimp1}. If a nominal object follows, however,  {\itshape nga} ``steals'' the object-linking H tone (\sectref{sec:HLinker}), which would otherwise surface on the noun class prefix in \REF{IMPga2a}. In that case, {\itshape be-kálàdè} surfaces with an L tone on the prefix.
The same is true when the particle is followed by {\itshape wámíyɛ̀} `quickly', as in \REF{ngapos1}.

\ea \label{ngapos1}
  \glll tí {\bfseries ngá} dè wámíyɛ̀ \\
       tí nga-H dè wámíyɛ̀ \\
        {\NEG} {\PL}-{\OBJ}.{\LINK} eat fast \\
    \trans `Don't (2{\PL}) eat fast.'
\z

\noindent If {\itshape nga} precedes a non-finite verb in a complex predicate, however, no H tone attaches, as shown in \REF{ngapos2}.

\ea \label{ngapos2}
  \glll sílɛ̂ {\bfseries ngà} nyî vâ \\
       sílɛ̂ nga nyî vâ \\
        finish.{\IMP} {\PL} enter here \\
    \trans `Enter (2{\PL}) here [one after the other until everybody is in the house].'
\z

\noindent The H tone on {\itshape nga} in \REF{IMPga2b} is therefore not from the object-linking H tone, but originates from the H tone on the preceding auxiliary {\itshape tí}. The object-linking H tone in this case attaches to the prefix of the nominal object.



%***tone pattern: underlyingly toneless, steals {\OBJ}.{\LINK} H tone from following object:
%The tonal difference in \REF{ti1} might seem to be explicable as either high tone spreading from the preceding {\itshape tí}, or lack thereof in \REF{ti1b} with {\itshape dê}, or as a phenomenon tied to the phrase-final or non-phrase final position of the plural marker. \REF{ti2} shows, however, that this is not the case













\section{Adpositions} 
\label{sec:Adpos}

Following \citet{hagege2010}, adpositions mark the relationship between a predicate, sentence, or non-predicative noun and an element that is governed by the adposition. This governed element is often a noun phrase, but may also include other word classes in Gyeli, as I will show below for the locative preposition {\itshape ɛ́} that combines with certain adverbs. 
Gyeli has both prepositions (\sectref{sec:PREP}) and postpositions (\sectref{sec:POST}). 

In Gyeli, I formally distinguish adpositions from elements of the noun phrase (\sectref{sec:NAdjuncts}) such as the genitive marker (\sectref{sec:GEN}) and the attributive marker (\sectref{sec:ATT}), based on agreement behavior and distributional differences. While adpositions are non-inflecting words, genitive and attributive markers agree with their head noun. As genitive and attributive markers modify nouns, they can appear with any noun phrase (subject, object, adjunct). In contrast, adpositions are almost exclusively restricted to oblique phrases (with the exception of the associative plural marker {\itshape bà} discussed in \sectref{sec:AP}).\footnote{The defining criteria to distinguish arguments and adjuncts include word order and tonal behavior (\sectref{sec:GR}).}
  






\subsection{Prepositions} 
\label{sec:PREP}

Gyeli has a limited set of prepositions, including only one locative preposition {\itshape ɛ́}. Also the comitative marker {\itshape nà}, {\itshape tí} `without', and the associative marker {\itshape bà} fall into this category.


\subsubsection{Locative marker {\itshape ɛ́}}
\label{sec:LOCe}

The preposition {\itshape ɛ́}\footnote{The corresponding preposition in Mabi is {\itshape ɔ́}.} is most frequently used to accompany a locative adverb as discussed in \sectref{sec:G1ADV} and listed in \REF{eADV}.

\ea \label{eADV}
  \ea  ɛ́ vâ `here'
\ex ɛ́ wû `there (medial)'
%\ex ɛ́ wɛ̂ `there (MID)'
\ex ɛ́ pɛ̀ `there (distal)'
\ex ɛ́ bà `to, at'
\z
\z

\noindent Further, the preposition {\itshape ɛ́} can precede a noun in a locative context as in \REF{eN}.

\ea \label{eN}
  \ea  ɛ́ tísɔ̀nì `in town'
\ex ɛ́ nkɔ̀lɛ́ `on the line'
\z
\z

Semantically, {\itshape ɛ́} is used as a locative preposition when the described location is about any spatial relation except containment. Spatial containment relations are expressed by the postposition {\itshape dé} as discussed in \sectref{sec:LOCde}.

It is possible that {\itshape ɛ́} is also used as a directional preposition, as in \REF{totown}, which shows two lexical options of saying `I go to town', differing in the noun used for the landmark. Due to phonological assimilation, however, it is not possible to clearly prove the presence of the locative marker in this environment since the preposition is identical with the final vowel of the verb, in which case the locative preposition would be deleted in its surface form.

\ea \label{totown}
  \ea \label{totown1}
  \glll    mɛ́ kɛ́ mã̂\\
          mɛ́ kɛ̀-H-ɛ́? m-ã̂\\
              1\textsc{sg}.\textsc{prs} go-{\R}-{\LOC}? ma6.sea \\
    \trans `I go to town.'\footnote{From the perspective of the village Ngolo, the town Kribi is located towards the sea line. Therefore, speakers most frequently refer to the direction of the sea when they talk about the town.}
\ex\label{totown2}
  \glll    mɛ́ kɛ́ tísɔ̀nì\\
         mɛ́ kɛ̀-H-ɛ́ tísɔ̀nì\\
              1\textsc{sg}.\textsc{prs} go-{\R}-{\LOC}? $\emptyset$7.town \\
    \trans `I go to town.'
\z
\z

In a case such as in \REF{efields}, it is thus not clear if the H tone on the noun class prefix comes from an underlying locative marker {\itshape ɛ́} or if the noun is treated as an object receiving an object-linking H tone (see \sectref{sec:GR}).

\ea \label{efields}
  \glll    mɛ́ kɛ́ mánkɛ̃̂\\
         mɛ́ kɛ̀-H-ɛ́ ma-nkɛ̃̂\\
              1\textsc{sg}.\textsc{prs} go-{\R}-{\LOC}? ma6-field \\
    \trans `I go to the fields.'
\z

  %[{\EXT}END]

%Finally, the locative preposition {\itshape ɛ́} is also used as part of complex interrogative constructions as in {\itshape ɛ́ ná} `how' and {\itshape ɛ́ vɛ́} `where' which are discussed in detail in \sectref{sec:INTERRC}.



\subsubsection{Comitative marker {\itshape nà}}
\label{sec:COM}

In keeping with Bantu terminology, I call the  comitative preposition {\itshape nà} a marker. This preposition broadly expresses association between nominal entities or a referent and a predicate.  As such, it is often translated as `with', as in \REF{COM1}.

\ea \label{COM1}
  \glll mùdã̂ kɛ̂ {\bfseries nà} nyɛ̀ mánkɛ̃̂ \\
       m-ùdã̂ kɛ̀-H nà nyɛ̀ H-ma-nkɛ̃̂ \\
         n1-woman go-{\R} {\COM} 1 {\OBJ}.{\LINK}-ma6-field  \\
    \trans `Woman, go with her to the fields.'
\z

The comitative marker is used in conjunction with the verb {\itshape bɛ̀} `be' to form {\itshape bɛ̀ nà} `be with' > `have'  to express possession, as in \REF{COM2}.

\ea \label{COM2}
  \glll mɛ̀ kí bɛ̀ {\bfseries nà} tsídí \\
       mɛ kí bɛ̀ nà tsídí \\
       1\textsc{sg}.{\PST}1 {\NEG}[Kwasio] be {\COM} $\emptyset$1.meat  \\
    \trans `I didn't have any meat.'
\z

{\itshape nà} is used in an instrumental sense, as in \REF{COM3a}.

\ea \label{COM3a}
  \glll mɛ̀ vùlɔ́ pɛ́mbɔ́ {\bfseries nà} ntfúmò \\
       mɛ vùlɔ-H pɛ́mbɔ́ nà ntfúmò  \\
       1\textsc{sg}.{\PST}1 cut-{\R} $\emptyset$7.bread {\COM} $\emptyset$3.knife  \\
    \trans `I cut the bread with a knife.'
\z

\noindent Extended uses of the instrumental sense are given in \REF{COM3} through \REF{COM5}.

\ea \label{COM3}
  \glll nyɛ̀gà váà nyɛ̀gá tsíyɛ́ sâ {\bfseries nà} màlɛ́ndí \\
         nyɛ-gà váà nyɛ-gá tsíyɛ́ sâ nà ma-lɛ́ndí \\
          1.{\SBJ}-{\CONTR} here 1.{\SBJ}-{\CONTR} live-{\R} only {\COM} 6-palm.tree \\
    \trans `Him here, he lives only from palm trees.'
\z

\ea \label{COM4}
  \glll  mɛ̀ múà wɛ̀ {\bfseries nà} nzà \\
        mɛ múà wɛ̀ nà nzà \\
          1\textsc{sg} be.almost die {\COM} $\emptyset$9.hunger  \\
    \trans `I'm about to die from hunger.'
\z

\ea \label{COM5}
  \glll  à múà á kɛ́ jìí dé tù {\bfseries nà} ndzǐ gyâ    \\
          a múà a-H kɛ̀-H jìí dé tù nà ndzǐ gyâ      \\
       1.{\PST}1 be.almost 1-\textsc{prs} go-{\R} $\emptyset$7.forest {\LOC} inside {\COM} $\emptyset$9.path $\emptyset$7.length \\
    \trans `He was about to go into the forest, on the long path'
\z

With some verbs, the use of {\itshape nà} is lexicalized (\sectref{sec:MainVerbs}), as in \REF{COM6}, where the combination of {\itshape njì} `come' and the comitative yields the meaning `bring'.

\ea \label{COM6}
  \glll   ɛ́ pɛ̀ nâ á njíyɛ̀ mɛ̂ {\bfseries nà} yɔ̂ \\
         ɛ́ pɛ̀ nâ a-H njíyɛ mɛ̀ nà y-ɔ̂ \\
         {\LOC} there {\COMP} 1-\textsc{prs} come.{\SBJV} 1\textsc{sg}.{\OBJ} {\COM} 7-{\OBJ}  \\
    \trans `so that she bring me that [food]'
\z


The comitative also coordinates noun phrases, as in \REF{COM7}, where {\itshape nà} links a subject pronoun and a noun + possessive pronoun construction.

\ea \label{COM7}
  \glll  bá {\bfseries nà} mùdã̂ wɛ̂ \\
         bá nà m-ùdã̂ w-ɛ̂\\
           2.{\SBJ} {\COM} 1-woman 1-{\POSS}.3\textsc{sg}  \\
    \trans `they and his wife'
\z

Finally, the comitative marker {\itshape nà} is frequently used in temporal or spatial adjuncts using the non-finite form {\itshape pámɔ} `arrive'.\footnote{Since {\itshape pámɔ} is uninflected and does not carry any person marking, and it seems to be used as a fixed expression, I consider {\itshape nà} as a comitative rather than a verb phrase coordinating conjunction.}

\ea \label{COM8}
  \glll èhè báà bù mpàgó {\bfseries nà} pámò pɛ̀ Kyíɛ̀ngɛ̀ \\
        èhè báà bù mpàgó nà pámo pɛ̀ Kyíɛ̀ngɛ̀ \\
        {\EXCL} 2.{\FUT} break $\emptyset$3.road {\COM} arrive over.there $\emptyset$7.{\PN}  \\
    \trans `Yes, they will build a road up to Kienge [river and name for the town Kribi].'
\z







\subsubsection {{\itshape tí} `without'}

The preposition {\itshape tí} `without' is the negative counterpart to the comitative {\itshape nà}. It is used, for example, in \REF{withouttown}.

\ea \label{withouttown}
  \glll    mɛ́ kɛ́ tísɔ̀nì tí wɛ̂\\
         mɛ́ kɛ̀-H tísɔ̀nì tí wɛ̂ \\
              1\textsc{sg}.\textsc{prs} go-{\R} $\emptyset$7.town without 2\textsc{sg}.{\OBJ} \\
    \trans `I go to town without you.'
\z


The use of {\itshape tí} as a preposition is a derived function from its primary status as a negative auxiliary (\sectref{sec:NEGti}).


\largerpage[1]
\subsubsection{Associative plural marker {\itshape bà}}
\label{sec:AP}

The associative plural marker {\itshape bà}\footnote{\citet{creissels2016} provides an in-depth discussion of the associative plural marker in Tswana (Bantu S31) from a historical and typological perspective.} is a preposition that marks not only the relationship between a governed element to a predicate or sentence, as is the case for the other prepositions described above, but also to a non-predicative noun. {\itshape bà} is segmentally identical with the agreement class 2 subject pronoun and denotes a group of related people when used with a noun, as in \REF{AP1} and \REF{AP2}. The relationship that is marked in these instances is that between the governed nominal and an abstract group of referents that is not identical with the governed nominal, but associated to it. 

\ea \label{AP1}
  \glll  mɛ́ kɛ́ jìyɔ̀ vé yá {\bfseries bà} fàmí wã̂ \\
          mɛ-H kɛ̀-H jìyɔ vé ya-H bà fàmí w-ã̂ \\
            1\textsc{sg}-\textsc{prs} go-{\R} stay where 1\textsc{pl}-\textsc{prs} {\AP} $\emptyset$1.family 1-{\POSS}.1\textsc{sg} \\
    \trans `Where will I live, we with my family?'
\z

\noindent \REF{AP2} is similar to \REF{COM7}, but differs in that no comitative marker is used. The tonal pattern of {\itshape ba} also differs: as the associative plural, {\itshape bà} has an L tone, as a subject pronoun, it has an H tone.

\ea \label{AP2}
  \glll bà mùdã̂ wà nû \\
       bà m-ùdã̂ wà nû \\
          {\AP} \textsc{n}1-woman 1:{\ATT} 1.{\DEM}.{\PROX}  \\
    \trans `the people/family of this woman'
\z

{\itshape bà} is also used in the same way as the other prepositions described above, {\pagebreak}linking the governed element to a predicate or sentence. In these cases, the as\-sociative plural marker {\itshape bà} often precedes a non-subject pronoun and expresses directionality towards human entities, as in \REF{AP3}.\footnote{This is similar to the French use of {\itshape chez} `to' that is used for human goals.}

\ea \label{AP3}
  \glll mɛ́ lɔ́ njì bàgyɛ̃̂ {\bfseries bà} wɛ̂ \\
       mɛ-H lɔ́ njì ba-gyɛ̃̂ bà wɛ̂ \\
       1\textsc{sg}-\textsc{prs} {\RETRO} come ba2-stranger {\AP} 2\textsc{sg}.{\OBJ}  \\
    \trans `I just came as a guest to you.'
\z


Other directionals that  typically require a preposition  in English, such as `go up', `go down', or `go around', are expressed by verbs in Gyeli, as illustrated in \REF{directionals}. Therefore, they do not include further adpositions.

\ea \label{directionals}
  \ea \label{directionals1}
  \gll    mɛ́ bédégá nkùlɛ́\\
              1\textsc{sg}.\textsc{prs} ascend $\emptyset$3.hill \\
    \trans `I go up the hill.'
\ex\label{directionals2}
  \gll    mɛ́ sìlégá nkùlɛ́\\
              1\textsc{sg}.\textsc{prs} descend $\emptyset$3.hill \\
    \trans `I go down the hill.'
\ex\label{directionals3}
  \gll    mɛ́ kɛ́ vyàmbɛ̀lɛ̀ nkùlɛ́\\
              1\textsc{sg}.\textsc{prs} go surround $\emptyset$3.hill \\
    \trans `I go around the hill.'
\z
\z



%benefactive (*** BEN bà 1.27)


%***for mpá'à 1.152





\subsection{Postpositions} 
\label{sec:POST}

Gyeli has a few postpositions which mostly express location. I distinguish three groups. The first and most frequent category includes {\itshape dé} `in/on' and {\itshape tù} `inside' which can co-occur. The second category comprises simple locative postpositions that cannot combine with any other postposition and that are clearly derived from location nouns. The third group consists of only one temporal postposition {\itshape wɛ̂}, which cannot combine with other adpositions either, but which differs from group two postpositions in that it is not derived from nouns.




\subsubsection[Combinable postpositions {\itshape dé} and {\itshape tù}]{Combinable postpositions {\itshape dé} `in/on' and {\itshape tù} `inside'}
\label{sec:LOCde}

The locative postpositions {\itshape dé} `in/on' and {\itshape tù} `inside' generally encode a spatial relation of \textsc{containment}. Most commonly, both postpositions co-occur where {\itshape dé} directly follows the noun and {\itshape tù} follows {\itshape dé}, as shown in \REF{deN}.\footnote{It is possible that {\itshape dè} was diachronically a preposition to {\itshape tù} `inside', which may have been a noun originally.}

\ea \label{deN}
  \ea  \gll ndáwɔ̀ dé tù  \\
      $\emptyset$9.house {\LOC} inside \\
\trans`in the house'
\ex  \gll mìnkĩ́ dé tù  \\
 $\emptyset$1.pot {\LOC} inside \\
\trans `in the pot'
\z
\z

Examples of the co-occurrence of both postpositions from natural text are provided in \REF{detu1} and \REF{detu2}.

\ea \label{detu1}
  \glll  bɔ́nɛ́gá báà ná jìí {\bfseries dé} {\bfseries tù} \\
        b-ɔ́nɛ́gá báà ná jìí dé tù \\
          2-other 2.{\COP} still $\emptyset$7.forest {\LOC} inside  \\
    \trans `The others are still in the forest.'
\z

\ea \label{detu2}
  \glll   àà ndáwɔ̀ {\bfseries dé} {\bfseries tù} nyɛ̀ mɛ́dɛ́ támé   \\
           àà ndáwɔ̀ dé tù nyɛ mɛ́dɛ́ támé      \\
          1.{\COP} $\emptyset$9.house {\LOC} inside 1 self alone \\
    \trans `He is in the house all by himself.'
\z

Both postpositions can, however, occur without the other one while maintaining their meaning of spatial \textsc{containment}, as in \REF{deNde} and \REF{deNtu}. The exact semantic difference between constructions with both postpositions, only {\itshape dé}, or only {\itshape tù} is not clear at this point and likely requires a systematic study of postposition combinations with a large set of different nouns as spatial reference points.  Generally, it seems, however, that the component of \textsc{containment} is stronger with {\itshape tù} `inside'.



\ea \label{deNde}
  \ea  ndáwɔ̀ dé \\ `in the house'
\ex mìnkĩ́ dé \\ `in the pot'
\z
\z

\ea \label{deNtu}
  \ea  ndáwɔ̀ tù \\ `inside the house'
\ex mìnkĩ́ tù  \\ `inside the pot'
\z
\z

%\vspace{.3cm}

In contrast to {\itshape tù} `inside', {\itshape dé} can also describe spatial relations of \textsc{contact} as in \REF{decontact}.

\ea \label{decontact}
  \gll   nsɔ̃́ wúù wɛ̀ nyúlɛ̀ dé \\
           $\emptyset$3.worm 3.\textsc{cop} 2\textsc{s}   $\emptyset$9.body \textsc{loc}  \\
    \trans `The worm is on your body.'
\z

\noindent I therefore gloss {\itshape dé} more generally as \textsc{loc} while {\itshape tù} has the more specific meaning `inside'. {\itshape dé} as a locative postposition is not only formally but also semantically distinct from the locative preposition {\itshape ɛ́}, which I also gloss as {\LOC}, but which lacks the connotation of \textsc{containment}. Cases of {\itshape dé} as encoding \textsc{contact} rather than \textsc{containment} may have some semantic similarity with the locative preposition {\itshape ɛ́} in \sectref{sec:LOCe}, although {\itshape ɛ́} seems to mark close proximity rather than contact.

Examples of the locative postposition {\itshape dé} only that come from natural text are given in \REF{deonly1} through \REF{deonly3}.

\ea \label{deonly1}
  \glll mbúmbù lèbvúú léè nlémò {\bfseries dé} \\
         mbúmbù le-bvúú léè nlémò dé \\
        N1.namesake le5-anger 5.{\COP} $\emptyset$3.heart {\LOC}  \\
    \trans `The namesake is angry [lit. has anger in his heart].'
\z

\noindent As \REF{deonly2} shows, {\itshape dé} can also be used to indicate directionality rather than location.

\ea \label{deonly2}
  \glll Nzàmbí màbɔ́ɔ̀ nkwɛ́ɛ̀ {\bfseries dé} nâ vɔ́sì \\
        Nzàmbí ma-bɔ́ɔ̀ nkwɛ́ɛ̀ dé nâ vɔ́sì \\
          $\emptyset$1.{\PN} ma6-breadfruit $\emptyset$3.basket {\LOC} {\COMP} {\IDEO}:pouring\\
    \trans `Nzambi pours the breadfruit into the basket.'
\z

\noindent The same is true for figurative directionality with the verb {\itshape vìdɛga dé} `turn into' in \REF{deonly3}.

\ea \label{deonly3}
  \glll mìntángánɛ́ mí múà vìdɛ̀gà {\bfseries dé} \\
       mi-ntángánɛ́ mi-H múà vìdɛga dé \\
       mi4-white.person 4-\textsc{prs} be.almost turn {\LOC}  \\
    \trans `They are about to turn into white people.'
\z


Examples of the sole use of {\itshape tù} `inside' as postposition in natural text is less frequent, but attested as in \REF{tuonly}.

\ea \label{tuonly}
  \glll bùdì bà sílɛ̃́ɛ̃̀ mɛ̂ wɛ̀ ndáwɔ̀ {\bfseries tù} vâ \\
        b-ùdì ba sílɛ̃́ɛ̃̀ mɛ̂ wɛ̀ ndáwɔ̀ tù vâ \\
       ba2-person 2.{\PST}1 finish.{\COMPL} 1\textsc{sg}.{\OBJ} die $\emptyset$9.house inside here  \\
    \trans `The people have all died here inside the house.'
\z




\subsubsection{Simple locative postpositions}
\label{sec:LOCgen}

Some of the locative nouns described in \sectref{sec:NomLOC} can also be used as locative postpositions. They behave like the postposition {\itshape dé} as explained in  \sectref{sec:LOCde}, but differ in their degree of grammaticalization.  In contrast to the locative postposition {\itshape dè}, these other locative postpositions are clearly used as nouns and as such their meaning is obvious. \REF{Post} lists the various nouns that can be also used as postpositions. In contrast to attributive constructions involving two nouns (\sectref{sec:CONC}), the locative postpositions juxtapose the two nouns without the attributive marker. 

\ea \label{Post}
  \ea  ndáwɔ̀ {\bfseries dyúwɔ̀} `on top/over the house' <  dyúwɔ̀ `top'
\ex ndáwɔ̀ {\bfseries sí} `under the house' <  sí `ground'
\ex ndáwɔ̀ {\bfseries písɛ̀} `behind the basket' <  písɛ̀ `back'
\ex ndáwɔ̀ {\bfseries sɔ̀} `in front of the house' <  sɔ́ `front' 
\ex ma-ndáwɔ̀ {\bfseries tɛ́mɔ́} `between the houses' < tɛ́mɔ́ `middle'
\z
\z





\subsubsection{Temporal postposition {\itshape wɛ̂ }}
\label{sec:POSTwe}

Gyeli has one temporal postposition {\itshape wɛ̂}, which follows time denoting nouns as in \REF{postwe}.

\ea \label{postwe}
  \ea  mɛ́nɔ́ wɛ̂  `in the morning'
\ex dùwɔ̀ wɛ̂  `in the day'
\ex kùgúù wɛ̂  `in the evening'
\ex bvùlɛ́ wɛ̂  `at night'
\z
\z


\noindent A natural text example is given in \REF{temppost}.

\ea \label{temppost}
  \glll   yá sàgà mɛ́nɔ́ {\bfseries wɛ̂} nyɛ́ɛ̀ mápà má njìbù má bwámɔ́ ndáwɔ̀ dé tù \\
          ya-H sàga mɛ́nɔ́ wɛ̂ nyɛ́ɛ̀ H-ma-pà má njìbù ma-H bwámɔ-H ndáwɔ̀ dé tù     \\
        1\textsc{pl}-\textsc{prs} be.surprised 7$\emptyset$.morning in see.{\SBJV} {\OBJ}.{\LINK}-ma6-paw 6:{\ATT} $\emptyset$1. antelope 6-\textsc{prs} come.out-{\R} $\emptyset$9.house {\LOC} inside  \\
    \trans `We are surprised in the morning to see traces of an antelope which come out of the house.'
\z





\section{Conjunctions}
\label{sec:CONJ}

Conjunctions are used in complex clauses
%(\chapref{sec:CC}) %%more detailed cross-ref follows. Drop to handle hyphenation
and link phrases and clauses, resulting in coordination (\sectref{sec:Coord}) or subordination (\sectref{sec:Sub}). Conjunctions that link the same type of clause or phrase are referred to as ``coordinators''. Subordinating conjunctions include complementizers and adverbializers.

\subsection{Coordinators}
\label{sec:COORD}

Gyeli has three coordinators, as shown in \REF{Coordi}.

\ea \label{Coordi}
  \ea \label{Coordia} nà `and' (\sectref{sec:Conjunction})
\ex\label{Coordib} kânà/nânà `or' (\sectref{sec:Disjunction})
\ex\label{Coordic} ndí `but' (\sectref{sec:AdverseCoord})
\z
\z

\noindent More details and examples are given in the respective sections.




\subsection{Subordinators}
\label{sec:COMP}

The most frequent subordinator in Gyeli is the complementizer {\itshape nâ} that links a complement clause to the main clause, as discussed in \sectref{sec:CompC}. The subordinator {\itshape ká} `if' introduces conditional clauses, which are more free with respect to their position before or after the main clause, as discussed in \sectref{sec:Cond}.


\section{Minor word classes}
\label{sec:Others}

This last section includes all minor parts of speech, ranging from connectors in non-verbal sentences{\textemdash}copulas and the identificational marker {\itshape wɛ́}{\textemdash}to question marker, and extrasentential elements.

\subsection{Copulas}
\label{sec:Copulas}

A copula links two elements, namely the subject and the predicate, in a non-verbal clause (\sectref{sec:COP}). In Gyeli, the copula agrees with the head noun. The agreeing copula is formally identical to the \textsc{stamp} marker (\sectref{sec:SCOP}) and takes a long vowel with an HL default tonal pattern. Exceptional person categories, including the first and second person singular and the agreement class 1 copula, have a long vowel with an L tone, as shown in \tabref{Tab:COPx}.

\begin{table}
\begin{tabular}{lll}
 \lsptoprule
 & Singular & Plural \\
\midrule
Speech act participants & 1\textsc{sg} {\itshape mɛ̀ɛ̀} & 1\textsc{pl} {\itshape yáà} \\
 & 2\textsc{sg} {\itshape wɛ̀ɛ̀} & 2\textsc{pl} {\itshape bwáà} \\
 \tablevspace
Non-speech act participants & cl.1 {\itshape àà/nùù} & cl.2 {\itshape báà} \\
(3\textsuperscript{rd} person) & cl.3 {\itshape wúù} & cl.4  {\itshape míì} \\
& cl.5 {\itshape léè} & cl.6 {\itshape máà} \\
 & cl.7 {\itshape yíì} & cl.8 {\itshape béè} \\
&  cl.9 {\itshape nyíì} & \\
 \lspbottomrule
\end{tabular}
\caption{Copula forms across agreement classes}
\label{Tab:COPx}
\end{table}




\subsection{Identificational marker {\itshape wɛ́}}
\label{sec:IDwe}

Another element used in non-verbal clauses is the identificational marker {\itshape wɛ́}, which links a subject to a demonstrative or deictic adverb. Unlike the copula, however, a non-verbal clause with {\itshape wɛ́} can also just consist of a subject and the identificational marker. Both constructions are described with examples in \sectref{sec:ID}.


\subsection{Question markers}
\label{sec:Qna}

Gyeli has two question markers: {\itshape nà} and {\itshape nànâ}. The first generally signals a question, while the second is emphatic and is thus pragmatically marked. Examples and a more detailed discussion are provided in \sectref{sec:Questions}.


\subsection{Sentential modifiers}
\label{sec:SentModPOS}

Sentential modifiers include {\itshape ná} `again, still', {\itshape lìí} `not yet', and {\itshape ndáà} `also'. They are distinguished from adverbs (\sectref{sec:ADV}) in that sentential modifiers usually occur immediately after the finite verb form, which is not possible for adverbs in complex predicate constructions, as adverbs rather appear at both the left and the right edge of a sentence. \sectref{sec:SentMod} gives more information about the function of sentential modifiers within a clause.


\subsection{Extrasentential elements}
\label{sec:Extrasent}

The Gyeli corpus contains a number of extrasentential elements. I roughly distinguish interjections from exclamations. Interjections are words that do not relate to the rest of the sentence in a grammatical way. They are, however, lexical words. Exclamations, in contrast, are not considered as lexical words, but rather sounds that convey attitudes and emotions.

\subsubsection{Interjections}
\label{sec:Interjections}

Most (recognizable) interjections used in Gyeli are  loanwords from French.\footnote{It is possible that I classify some local interjections with exclamations since their meaning is generally hard to describe for speakers and the difference between a lexical word and an emotion encoding sound is possibly not always very clear.} Interjections have a discourse structuring function and often appear at the beginning of an intonation phrase, as in \REF{INTERJ1}.

\ea \label{INTERJ1}
  \glll {\bfseries dɔ̃̀} sí nyã̂ nyíì búùlɛ̀ yá Ngɔ̀lɔ́ \\
        dɔ̃̀ sí ny-ã̂ nyíì búùlɛ̀ yá Ngɔ̀lɔ́ \\
        so[French] $\emptyset$9.ground 9-{\POSS}.1\textsc{sg} 9:{\COP} $\emptyset$7.old.camp 7:{\ATT} $\emptyset$3.{\PN}  \\
    \trans `So [French: {\itshape donc}], my land is the old settlement of Ngolo.'
\z

Pragmatically, interjections are also used to reinforce common ground, as in \REF{INTERJ2} where the speaker acknowledges that he and the addressee are on the same page.

\ea \label{INTERJ2}
  \glll  {\bfseries voilà} wɛ̀ɛ̀ njǐ nà njǐ wɛ̀ɛ̀ njǐ nà njǐ\\
        voilà wɛ̀ɛ̀ njǐ nà njǐ wɛ̀ɛ̀ njǐ nà njǐ\\
          ok[French] 2\textsc{sg}.{\COP} $\emptyset$9.path {\CONJ} $\emptyset$9.path 2\textsc{sg}.{\COP} $\emptyset$9.path {\CONJ} $\emptyset$9.path\\
    \trans `Exactly, you are on the right track.'
\z

Even though the Bagyeli of Ngolo report that their French is, if at all, very limited, they are all able to use these French interjections, as well as {\itshape allez} `come on'  and {\itshape alors} `so, then'.

\subsubsection{Exclamations}
\label{sec:EXCL}

Exclamations reveal the speaker's attitude and emotion towards a situation, usually encoding agreement, disagreement, surprise, or getting the addressee's attention. All exclamations can be manipulated in terms of their length. A longer sound (and often increased volume) correlates with higher emotional intensity.

A widely used exclamation in the area (not only in Gyeli) is {\itshape  ɛ́ɛ́kɛ̀}, which signals general surprise about either a positive or negative event. In \REF{eeke1}, {\itshape ɛ́ɛ́kɛ̀} is a reaction to a character in a story who wants to eat a child. The exclamation refers potentially to both the narrator's attitude and the reaction of the woman in the story whose child will be eaten.

\ea \label{eeke1}
  \glll  {\bfseries ɛ́ɛ́kɛ̀} mùdã̂ à gyɛ̃́ɛ̃̀ \\
        ɛ́ɛ́kɛ̀ m-ùdã̂ a gyɛ̃́ɛ̃̀ \\
          {\EXCL} \textsc{n}1-woman 1.{\PST}1 cry.{\COMPL} \\
    \trans `Oh, the woman cried.'
\z


Exclamations are also frequently used in reported discourse, as in \REF{eeke}.

\ea \label{eeke}
  \glll  yɔ́ɔ̀ bá kí nâ {\bfseries ɛ́ɛ́kɛ̀} \\
            yɔ́ɔ̀ ba-H ki-H nâ ɛ́ɛ́kɛ̀ \\
         so 2-\textsc{prs} say-{\R} {\COMP} {\EXCL} \\
    \trans `So they say that [exclamation of surprise]!'
\z

Another frequent exclamation is {\itshape áá} or {\itshape áà} or {\itshape àà}. The tonal pattern seems to depend, at least partially,  on the distance between speaker and addressee, with an H tone indicating distance and an L tone proximity. {\itshape áá} has been observed to occur often to introduce a question, as in \REF{aa1} and \REF{aa2}. It seems comparable to the English exclamation `oh!' expressing surprise or desire.

\ea \label{aa1}
  \glll  {\bfseries áá} gyí wɛ́ lɔ́ njì gyɛ́sɔ̀ \\
        áá gyí wɛ-H lɔ́ njì gyɛ́sɔ \\
           {\EXCL} what 2\textsc{sg}-\textsc{prs} {\RETRO} come look.for  \\
    \trans `Ah, what have you just come to look for?'
\z

\ea \label{aa2}
  \glll     {\bfseries áá} bíì màndáwɔ̀ má zì, yáà mɔ̂ fúàlà bwɛ̂ lèwùlà lé vɛ́\\
          áá bíì ma-ndáwɔ̀ má zì yáà m-ɔ́ fúala bwɛ̂ le-wùlà lé vɛ́\\
              {\EXCL} 1\textsc{pl}.{\OBJ} ma6-house 6:{\ATT} $\emptyset$7.tin[Bulu] 1\textsc{pl}.{\FUT} 6-{\OBJ} end receive le5-hour 5:{\ATT} which \\
    \trans `Ah, us, tin houses, when will we receive them?'
\z

{\itshape àà} is also used in addressing someone and getting the addressee's attention.

\ea \label{8aa3}
  \glll   mɛ̀ bìyɛ́ làwɔ̀ nâ {\bfseries àà} bwánɔ̀ bã̂ \\
           mɛ bìyɛ-H làwɔ nâ àà b-wánɔ̀ b-ã̂ \\
          1\textsc{sg} in.vain? speak {\COMP} {\EXCL} ba2-child 2-{\POSS}.1\textsc{sg} \\
    \trans `I say in vain: ``ah, my children. . .''.'
\z

\noindent The H tone on {áá} in \REF{aa4} indicates that mother and father are far away from the speaker.

\ea \label{aa4}
  \glll áá nyáò áá táò \\
       áá nyá-ò áá tá-ò \\
       {\EXCL} \textsc{n}1-mother-{\VOC} {\EXCL} \textsc{n}1-father-{\VOC}  \\
    \trans `Oh mother, oh father!'
\z

A similar function of attention seeking and address is found with {\itshape ɔ́ɔ́ɔ́} in \REF{ooo} and {\itshape ɛ́} in \REF{eee}, comparable to English `hey!'.

\ea \label{ooo}
  \glll  nyɛ̀ nâ {\bfseries ɔ́ɔ́ɔ́} mùdã̂ \\
          nyɛ nâ ɔ́ɔ́ɔ́ m-ùdã̂ \\
            1 {\COMP} {\EXCL} \textsc{n}1-woman  \\
    \trans `He [said]: ``Oh, wife!''.'
\z

\ea \label{eee}
  \glll {\bfseries ɛ́} mwánɔ̀ wã̂ dyúwɔ̀ \\
       ɛ́ m-wánɔ̀ w-ã̂ dyúwɔ̀ \\
        {\EXCL} \textsc{n}1-child 1-{\POSS}.1\textsc{sg} on \\
    \trans `Hey, about my child!'
\z

Exclamations with a clear negative connotation are {\itshape yééé} as a sound of disappreciation and {\itshape kɛ́ɛ́ɛ́} (with varying length). The latter expresses outrage and strong disapproval, as in \REF{keee1} where the speaker expresses his indignation after learning that his child had been eaten by his friend.

\ea \label{keee1}
  \glll nyɛ̀ nâ {\bfseries kɛ́ɛ́ɛ́ɛ́} \\
        nyɛ nâ kɛ́ɛ́ɛ́ɛ́ \\
       1 {\COMP} {\EXCL}  \\
    \trans `He [says]: ``What!''.'
\z

This exclamation can also be used less strongly in a pejorative way, as in \REF{keee2}. Here, {\itshape kɛ́ɛ́ɛ́} shows the belittling attitude of the speaker towards his children.

\ea \label{keee2}
  \glll  nyɛ̀ nâ {\bfseries kɛ́ɛ́ɛ́} bwánɔ̀ bã̂ mɛ̀ sílɛ̃́ɛ̃̀ bɔ̂ dyùù \\
        nyɛ nâ kɛ́ɛ̀ b-wánɔ̀ b-ã̊ mɛ sílɛ̃ɛ̃̀ b-ɔ̂ dyùù \\
        1 {\COMP} {\EXCL} ba2-child 2-{\POSS}.1\textsc{sg} 1\textsc{sg} finish.{\COMPL} 2-{\OBJ} kill \\
    \trans `He [says]: ``Ha, my children, I have already killed them all''.'
\z



%wɛ́ɛ̀


