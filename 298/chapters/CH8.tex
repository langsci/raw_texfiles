\chapter{Complex clauses}
\label{sec:CC}

Complex clauses are those which are comprised of more than one clause, following the standard notion of complex clauses, including coordination and subordination, as given, for instance, by \citet{wegener2012}. A complex clause is coordinated when the two (or more) clauses it is comprised of are equal in their status. Usually, coordination involves the combination of two (or more) independent clauses. In contrast to coordination, in subordination, clauses are combined that are not symmetrical in their status. They are formed by combining a superordinate clause, i.e.\ a clause that can occur independently, with a dependent clause, i.e.\  a clause that cannot occur on its own. In this chapter, I present different types of coordination and subordination. I finally discuss the special case of reported discourse, which I do not view as a type of subordination, but rather as being organized at a higher discourse level.


\section{Coordination}
\label{sec:Coord}

\citet[1]{haspelmath2007} defines coordination as:
``syntactic constructions in which two or more units of the same type are combined into a larger unit and still have the same semantic relations with other surrounding elements''. 
He points out that these units can either be words (e.g.\ verbs), phrases (e.g.\ noun phrases), subordinate clauses, or full sentences. In terms of terminology, Haspelmath calls the units that are combined ``coordinands'', while the element that links the coordinands is called ``coordinator''.

Gyeli uses a range of coordinators which broadly map onto different coordination relations as distinguished by Haspelmath:
\begin{enumerate}
\itshapeem combination (conjunction)
\begin{itemize}
\itshapeem conjunction {\itshape nà} `and'
\itshapeem asyndetic (covert) coordination
\end{itemize}
\itshapeem alternative (disjunction) {\itshape nânà/kânà} `or'
\itshapeem contrast (adversative coordination) {\itshape ndí} `but'
\end{enumerate}

The most frequent coordinator in the corpus is {\itshape nà} for conjunction\footnote{Although the conjunction {\itshape nà} and the comitative marker {\itshape nà} are form-identical, I distinguish them on the basis of their distribution. Conjunctions coordinate verb phrases, while the comitative marker coordinates noun phrases (\sectref{sec:NPCoord}).} with 42 occurrences, followed by {\itshape ndí} with 9 instances. Both covert coordination and disjunction are rather rare in the corpus, and there are only a couple of examples of each. Nevertheless, corpus examples have been supplemented with elicitations. I discuss each of these coordination strategies in turn.


\subsection{Conjunction with {\itshape nà} `and'}
\label{sec:Conjunction}

Conjoining two clauses with the conjunction {\itshape nà} is the most frequent coordination strategy in the Gyeli corpus. {\itshape nà} usually appears between two clauses in one utterance, but can also occur at the beginning of an utterance, linking the clause to the previous text, as in \REF{Coord}.
{\itshape nà} is never found sentence finally.


\ea\label{Coord}
  \glll   {\bfseries nà} pándɛ̀ vâ bùdì báà bɛ̀ \\
          nà pándɛ̀ vâ b-ùdì báà bɛ \\
      {\CONJ} arrive here ba2-person 2.{\DEM}.{\PROX} be.there\\
    \trans `[He is going into the forest on the long path.] And having arrived here, these people are there.'
\z



There are structural differences among conjoined clauses relating to the overt expression or elision of subjects and objects. In the following, I will first discuss subject expression and elision before turning to objects. Other differences that are explained as well in the following examples pertain to general symmetry and asymmetry of the two coordinands in terms of clause type, word order, and aspect marking.

\subsubsection*{Subject expression in both coordinands}
Two clauses can be conjoined with {\itshape nà} in cases where both coordinands display overt subjects. This is true for both same and different subjects. 
Subjects are always overtly expressed in both coordinands if they are not identical. In \REF{Coord1}, for example, a lexical noun phrase serves as subject, while the second clause only marks subject agreement on the \textsc{stamp} copula. The two coordinands are asymmetrical in terms of their clause type.  The first coordinand represents an intransitive verbal clause, while the second constitutes a non-verbal copula construction.


\ea\label{Coord1}
  \glll bon mpɔ̀ngɔ̀ sílɛ̃́ɛ̃̀ {\bfseries nà} béè bànáyɛ̂yɛ̂  \\
        bon mpɔ̀ngɔ̀ sílɛ̃́ɛ̃̀ nà béè ba-náyɛ̂yɛ̂  \\
      OK[French] $\emptyset$7.generation finish.{\COMPL} {\CONJ} 2\textsc{pl}.{\COP} 2-bleached.out  \\
    \trans `OK, the generation has been wiped out and you are bleached out [white].'
\z

\REF{Coord2} also has different subjects in the two coordinands.  At the same time, it is noteworthy that both have the same aspect marker which cannot be elided in the second constituent.


\ea\label{Coord2}
  \glll     yá lɔ́ fúàlà {\bfseries nà} mɛ̀ lɔ́ làwɔ̀ \\
           ya-H lɔ́ fúala nà mɛ lɔ́ làwɔ \\
              1\textsc{pl}-\textsc{prs} {\RETRO} end {\CONJ} 1\textsc{sg} {\RETRO} talk \\
    \trans `We just finished and I just spoke.'
\z


Although the subject of the second coordinand can be elided if it is identical with the subject of the first coordinand,  there are circumstances in which speakers prefer overt subject expression in the second clause over elision. This is, for instance, the case when both coordinands are relatively complex, as in \REF{Coord3}.


\ea\label{Coord3}
  \glll mɛ́ lámbó Nzàmbí wà nû {\bfseries nà} mɛ́ wúmbɛ́ lèmbò ɛ́ mpù à bùdɛ́ mɛ̂  \\
       mɛ-H lámbo-H Nzàmbí wà nû nà mɛ-H wúmbɛ-H lèmbo ɛ́ mpù a bùdɛ-H mɛ̂  \\
        1\textsc{sg}-\textsc{prs} trap-{\R} $\emptyset$1.{\PN} 1:{\ATT} 1.{\DEM}.{\PROX} {\CONJ} 1\textsc{sg}-\textsc{prs} want-{\R} know {\LOC} like.this 1 have-{\R} 1\textsc{sg}.{\OBJ}  \\
    \trans `I trap this Nzambi and I want to know like this how he takes me (what he thinks of this story).'
\z

\noindent Overt expression of the same subject is also preferred when the two coordinands differ in their aspect marking, as shown in \REF{Coord4}.


\ea\label{Coord4} 
  \glll  dɔ̃̀ bɛ̀yá lɔ́ kɛ̀ {\bfseries nà} bɛ̀yà nzíí pándɛ̀ \\
          dɔ̃̀ bɛ̀ya-H lɔ́ kɛ̀ nà bɛ̀ya nzíí pándɛ \\
          so[French] 2\textsc{pl}-\textsc{prs} {\RETRO} go {\CONJ} 2\textsc{pl} {\PROG}.\textsc{prs} arrive \\
    \trans `So, you just came and you are arriving.'
\z

Another instance where the subject of the first coordinand is resumed in the second is when the two clauses differ with respect to their information structure. In \REF{Coord5}, the first coordinand has a left-dislocated object, while the second appears in basic word order.


\ea\label{Coord5} 
  \glll   bèkúmbɛ́ báà njì nà byɔ̂ {\bfseries nà} báà njì lwɔ̃̂ mándáwɔ̀\\
          be-kúmbɛ́ báà njì nà by-ɔ̂ nà báà njì lwɔ̃̂ H-ma-ndáwɔ̀\\
           be8-roof  2.{\FUT} come {\COM} 8-{\OBJ} {\CONJ} 2.{\FUT} come build {\OBJ}.{\LINK}-ma6-house\\
    \trans `They will bring roofs and they will come and build houses.'
\z

\subsubsection*{Subject elision in second coordinand}
Subjects of second coordinands can be elided under identity with the first coordinand. The subject of the first coordinand, however, cannot be elided. Elision, where possible, is generally preferred over overt expression and occurs twice as often in the corpus as overt subject expression. An example of subject elision in the second coordinand is given in \REF{Coord6}.


\ea\label{Coord6}
  \glll   vɛ̀ɛ̀ mùdì nyɛ̀ jã́ã̀sà {\bfseries nà} kɛ́ jìí dé tù nà ndzǐ pámò dẽ̂\\
           vɛ̀ɛ̀ m-ùdì nyɛ jã́ã̀sà nà kɛ̀-H jìí dé tù nà ndzǐ pámò dẽ̂ \\
           only \textsc{n}1-person 1 disappear {\CONJ} go-{\R} $\emptyset$7.forest {\LOC} inside {\COM} $\emptyset$9.path arrive today \\
    \trans `Suddenly the person disappears and goes in the forest on the path till today.'
\z

A very common conjunction type is represented in \REF{Coord7a}, which encodes a chain of events. First, the agent has gone and then stuffed the top of the roof with straw. The occurrence of the coordinator {\itshape nà} clearly distinguishes the sentence in \REF{Coord7a} from the one in \REF{Coord7b}, where no coordinator is present.


\ea\label{Coord7}
\ea\label{Coord7a}
  \glll áà sílɛ́ kɛ̀ {\bfseries nà} dvù{\bfseries wɔ́} dyúwɔ̀ \\
     áà sílɛ-H kɛ̀ nà dvùwɔ-H dyúwɔ \\
        1.{\PST}2 finish-{\R} go {\CONJ} stuff-{\R} $\emptyset$7.top \\
    \trans `He has gone and stuffed the top [with straw].'
\ex\label{Coord7b}
  \glll áà sílɛ́ kɛ̀ dvù{\bfseries wɔ̀} dyúwɔ̀ \\
     áà sílɛ-H kɛ̀ dvùwɔ dyúwɔ \\
        1.{\PST}2 finish-{\R} go stuff $\emptyset$7.top \\
    \trans `He has gone to stuff the top [with straw].'
\z
\z

\noindent \REF{Coord7b} represents an instance of a complex auxiliary construction. As such, the verb {\itshape dvùwɔ} occurs in its infinitival form, i.e.\ with a final L tone. In contrast, under coordination as in \REF{Coord7a}, the verb is tonally inflected for tense and mood and thus occurs with an H tone.


Finally, conjunction constructions can have multiple coordinands, as \REF{Coord8} shows. This complex example contains both coordinands with elided subjects and overt subject expression.


\ea\label{Coord8} 
  \glll wɛ́ ná báàlá {\bfseries nà} nyɛ́ fí {\bfseries nà} wɛ́ ndyándyá ná sálɛ́ ɛ́ pɛ̀ {\bfseries nà} wɛ́ kòlá ná mɔ̀nɛ́ nû\\
      wɛ-H ná báàla-H nà nyɛ̂-H fí nà wɛ-H ndyándya-H ná sálɛ́ ɛ́ pɛ̀ nà wɛ-H kòla-H ná mɔ̀nɛ́ nû\\
         2\textsc{sg}-\textsc{prs} again repeat-{\R} {\CONJ} see-{\R} different {\CONJ} 2\textsc{sg}-\textsc{prs} work-{\R} again $\emptyset$7.work {\LOC} over.there {\CONJ} 2\textsc{sg}-\textsc{prs} add-{\R} again $\emptyset$1.money 1.{\DEM}.{\PROX} \\
    \trans `You repeat again and try something else [find other work] and you work there again and you add this money [the same amount of 250 Francs] again.'
\z

\subsubsection*{Object elision} In contrast to subjects, objects can be elided under identity in both the first and the second coordinand. \REF{Coordobj1} provides an example where the identical subject and object are expressed in both coordinands. In \REF{Coordobj2}, the object is elided in the first coordinand, while it is elided in the second coordinand in \REF{Coordobj3}. At the same time, it is possible to also elide the identical subject in the second coordinand, as indicated by the parentheses.


\ea\label{Coordobj} 
\ea\label{Coordobj1}
\glll mɛ́ sɛ́lɔ́ béntɔ̀gɔ̀ nà mɛ́ vúlɔ́ béntɔ̀gɔ̀\\
 mɛ-H sɛ́lɔ-H H-be-ntɔ̀gɔ̀ nà mɛ-H vúlɔ-H H-be-ntɔ̀gɔ̀\\
1\textsc{sg}-\textsc{prs} peel-{\R} {\OBJ}.{\LINK}-be8-sweet.potato {\CONJ} 1\textsc{sg}-\textsc{prs} cut-{\R} {\OBJ}.{\LINK}-be8-sweet.potato \\
\trans `I peel sweet potatoes and I cut sweet potatoes.'
\ex\label{Coordobj2}
\glll mɛ́ sɛ́lɔ́ nà (mɛ́) vúlɔ́ béntɔ̀gɔ̀\\
 mɛ-H sɛ́lɔ-H nà mɛ-H vúlɔ-H H-be-ntɔ̀gɔ̀\\
1\textsc{sg}-\textsc{prs} peel-{\R} {\CONJ} 1\textsc{sg}-\textsc{prs} cut-{\R} {\OBJ}.{\LINK}-be8-sweet.potato \\
\trans `I peel and (I) cut sweet potatoes.'
\ex\label{Coordobj3}
\glll mɛ́ sɛ́lɔ́ béntɔ̀gɔ̀ nà (mɛ́) vúlɔ̀\\
 mɛ-H sɛ́lɔ-H H-be-ntɔ̀gɔ̀ nà mɛ-H vúlɔ\\
1\textsc{sg}-\textsc{prs} peel-{\R} {\OBJ}.{\LINK}-be8-sweet.potato {\CONJ} 1\textsc{sg}-\textsc{prs} cut \\
\trans `I peel sweet potatoes and (I) cut [them].'
\z
\z

In addition to the overt expression of a nominal object and its elision, there is a third option, namely to express an object pronominally, as shown in \REF{Coordobjb}. In \REF{Coordobj4}, the natural interpretation is that the objects of the coordinated clauses are identical. If, however, the first coordinand has a pronominal object, while the second has a nominal object, as in \REF{Coordobj5}, the two objects are likely not identical, but the pronoun would refer to an antecedent from previous discourse.



\ea\label{Coordobjb} 
\ea\label{Coordobj4}
\glll mɛ́ sɛ́lɔ́ béntɔ̀gɔ̀\textsubscript{i} nà (mɛ́) vúlɔ́ byɔ̂\textsubscript{i} \\
mɛ-H sɛ́lɔ-H H-be-ntɔ̀gɔ̀ nà mɛ-H vúlɔ-H byɔ̂ \\
1\textsc{sg}-\textsc{prs} peel-{\R} {\OBJ}.{\LINK}-be8-sweet.potato {\CONJ} 1\textsc{sg}-\textsc{prs} cut-{\R} 8.{\OBJ} \\
\trans `I peel sweet potatoes and (I) cut them.'
\ex\label{Coordobj5}
\glll mɛ́ sɛ́lɔ́ byɔ̂\textsubscript{i} nà (mɛ́) vúlɔ́  béntɔ̀gɔ̀\textsubscript{j} \\
mɛ-H sɛ́lɔ-H  byɔ̂  nà mɛ-H vúlɔ-H H-be-ntɔ̀gɔ̀ \\
 1\textsc{sg}-\textsc{prs} peel-{\R} 8.{\OBJ} {\CONJ} 1\textsc{sg}-\textsc{prs} cut-{\R} {\OBJ}.{\LINK}-be8-sweet.potato  \\
\trans `I peel them and (I) cut sweet potatoes.'
\z
\z


\subsubsection*{{\itshape nà} in non-clausal coordination}
The conjunction {\itshape nà} is not only used in clausal coordination, but also in coordination of, for instance, noun phrases, as shown in \REF{Coord9}.



\ea\label{Coord9}
  \glll  {\bfseries nà} mìmbàngá {\bfseries nà} màsá {\bfseries nà} bègyí {\bfseries nà} bègyí byɛ́sɛ̀ béè sílɛ̀ ntàmànɛ̀\\
       nà mi-mbàngá nà ma-sá nà be-gyí nà be-gyí by-ɛ́sɛ̀ béè sílɛ ntàmanɛ\\
           {\CONJ} mi4-coconut.tree {\CONJ} ma6-African.plum {\CONJ} be8-what {\CONJ} be8-what 8-all 8.{\FUT} finish ruin\\
    \trans `And the coconut trees and the African plum trees and so on and so forth, they will all be ruined.'
\z

\noindent Also, this coordinator can conjoin two oblique phrases, as in \REF{Coord10}.\footnote{Note that {\itshape pámò} `arrive' is consistently used in a preposition-like function of `till'.}


\ea\label{Coord10} S V X\textsubscript{1} `and' X\textsubscript{2}\\
  \glll àá bámálá tɔ́bá mpfùmɔ̀ {\bfseries nà} pámò mɛ́nɔ́\\
       àá bámala-H tɔ́bá mpfùmɔ̀ nà pámo mɛ́nɔ́\\
       1.{\INCH} scold-{\R} since  $\emptyset$3.midnight {\CONJ} arrive $\emptyset$7.morning \\
    \trans `He starts to scold [now] at midnight and [it] will last until the morning.'
\z

\noindent Coordination of verbs sharing the same object has not been observed in the corpus.


\subsection{Covert coordination}
A minor strategy to conjoin clauses is asyndetic coordination, i.e.\ coordination without any overt coordinator. This is also called ``covert coordination''. In Gyeli, covert coordination seems to be quite restricted and involves two clauses with different verbal predicates, the second of which is ditransitive. The second clause then not only shares the first clause's subject, but also its object, both of which are elided in the second clause, as shown in \REF{coCo1} and \REF{coCo2}.\footnote{Instances of such covert coordination constructions where the second clause has a transitive verb which it shares with the first clause have not been observed. Future research will have to show whether such constructions are possible.}

 
\ea\label{coCo1} S V\textsubscript{1} O\textsubscript{1} [`and'] V\textsubscript{2} O\textsubscript{2}\\
  \glll [yɔ́ɔ̀ mùdã̂ tɔ́kɛ́ mwánɔ̀] [kàlànɛ̀ nyɛ̂] \\
       {\db}yɔ́ɔ̀ m-ùdã̂ tɔ́kɛ-H m-wánɔ̀ {\db}kàlanɛ nyɛ̂ \\
      {\db}so \textsc{n}1-woman collect-{\R} \textsc{n}1-child {\db}hand.over 1.{\OBJ}\\
    \trans `So the woman picks up the child [and] hands [it] over to him.'
\z


\ea\label{coCo2} S V\textsubscript{1} O\textsubscript{1} [`and'] V\textsubscript{2} O\textsubscript{2}\\
  \glll  [yɔ́ɔ̀ mɛ́ tɔ́kɛ́ mɔ̀nɛ́ wɛ̂] [vɛ̀ nyɛ̂] \\
         {\db}yɔ́ɔ̀ mɛ-H tɔ́kɛ-H mɔ̀nɛ́ w-ɛ̂ {\db}vɛ̀ nyɛ̂ \\
         {\db}so 1\textsc{sg}-\textsc{prs} collect-{\R} $\emptyset$1.money 1-{\POSS}.3\textsc{sg} {\db}give 1.{\OBJ}   \\
    \trans `So I collect her money [and] give [it to] her,'
\z

I analyze these constructions as instances of covert coordination rather than complex predicate constructions for two reasons. First, the verb of the first clause is not a typical auxiliary verb. As explained in \sectref{sec:CompPred}, auxiliaries generally belong to one of three verb classes, namely aspectual verbs, deictic motion verbs, and modal verbs. {\itshape tɔ́kɛ} `collect' clearly does not fit into any of these categories and has not been observed in any other instances to occur as auxiliary in complex predicate constructions. Second, while complex predicates often describe one event expressed by the final main verb, clauses with covert coordination clearly encode a sequence of events. Thus, in \REF{coCo1}, the woman first picks up her child and then hands it over to another person.


%Combination relations have mostly to do with the temporal axis [Longacre 1985] and express sequences that are either sequential, simultaneous or atemporal. Coordinations in Gyeli that are temporally simultaneous are always coordinated with the comitative {\itshape nà}, as shown in \REF{Combina}.

%\begin{exe}
%\ex\label{Combina} Simultaneous
 % \glll     á gyímbɔ́ {\bfseries nà} sá mákwásì \\
 %             a-H gyímbɔ-H nà sâ-H H-ma-kwásì \\
 %              1-\textsc{prs} dance-{\R} {\COM} do-{\R} {\OBJ}.{\LINK}-ma6-clapping \\
 %   \trans `He is dancing and clapping (his hands).'
%\end {exe}


%There seem, however, to be restrictions on the use of this marker since it can not link clauses with different subjects as in \REF{CombiCna1}. Instead, either no linker is used \REF{CombiCna2}, or a temporal adverb such as {\itshape yɛ́rɛ́} 'then' is preferred \REF{CombiCna3}.

%\begin{exe}
%\ex\label{CombiCna}
%\begin{xlist}
%\ex[*]{\label{CombiCna1}
%  \gll   yíì múà mà-sílá má má-vɔ̀dá nà b-ùdì b-ɛ́sɛ̀ bá níyɛ̀ bé-sálɛ̀  \\
%             7 {\ASP}? ma6-end 6:{\ATT} ma6-vacation {\COM} ba2-people 2-all 2.\textsc{prs} return be8-work  \\
%    \trans 'It's the end of the holidays and all people return to work.'}
%\ex \label{CombiCna2}
%  \gll  yíì múà mà-sílá má má-vɔ̀dá. b-ùdì b-ɛ́sɛ̀ bá níyɛ̀ bé-sálɛ̀  \\
%              7 {\ASP}? ma6-end 6:{\ATT} ma6-vacation {\COM} ba2-people 2-all 2.\textsc{prs} return be8-work \\
%    \trans `It's the end of the holidays. All people return to work.'
%\ex \label{CombiCna3}
%  \gll  yíì múà mà-sílá má má-vɔ̀dá, yɛ́rɛ́ b-ùdì b-ɛ́sɛ̀ bá níyɛ̀ bé-sálɛ̀  \\
%              7 {\ASP}? ma6-end 6:{\ATT} ma6-vacation {\COM} ba2-people 2-all 2.\textsc{prs} return be8-work \\
 %   \trans `It's the end of the holidays,  then all people return to work.'
%\end{xlist}
%\end{exe}

%\noindent - temporally sequential events are preferably expressed by temporal adverbs in a subordinate clause, but the comitative is also possible in a coordination 

%\begin{exe}
%\ex\label{Combisequ} Sequential
%\begin{xlist}
%\ex \label{Combisequ1}
%  \glll     mɛ̀ jí mbɛ̂ {\bfseries kɔ́ɔ̀/yɛ́rɛ́} {\bfseries kɛ̀} \\
%              mɛ jî-H mbɛ̂ kɔ́ɔ̀/yɛ́rɛ́ kɛ̀ \\
%               1\textsc{sg}.{\PST}1 open-{\PST}1 $\emptyset$3.door just/then go \\
%    \trans `I opened the door, just/then leaving.'
%\ex \label{Combisequ2}
%  \glll     mɛ̀ jí mbɛ̂ {\bfseries nà} {\bfseries kɛ̀} \\
%              mɛ jî-H mbɛ̂ nà kɛ̀ \\
%               1\textsc{sg}.{\PST}1 open-{\PST}1 $\emptyset$3.door {\COM} go \\
%    \trans `I opened the door and left.'
%\end{xlist}
%\end {exe}

%\noindent - atemporal events are usually not coordinated with the comitative {\itshape nà}, but simply juxtaposed in linkless clauses \\
%- sentences from the questionnaire were modified into either oppositions or sequences ('Doctors are rich and lawyers marry pretty girls.' $\rightarrow$ 'Doctors are rich, but/then lawyers marry pretty girls.') \\

%\begin{exe}
%\ex\label{Atemp} Atemporal
%  \glll     bàmpɛ̂ bá dé bátídí (?nà) bàkfúbɔ̀ bá dé mákà \\
%              ba-mpɛ̂ ba-H dè-H H-ba-tídí (?nà) ba-kfúbɔ̀ ba-H dè-H H-ma-ka \\
%               ba2-dog 2-\textsc{prs} eat-{\R} {\OBJ}.{\LINK}-ba2-meat (?{\COM}) ba2-chicken 2-\textsc{prs} eat-{\R} {\OBJ}.{\LINK}-ma6-leaf \\
%    \trans `Dogs eat meat (?and) chicken eat leaves.'
%\end {exe}




\subsection{Disjunction with {\itshape kânà/nânà} `or'}
\label{sec:Disjunction}

Disjunction, also called ``alternative coordination'', can be expressed with one of the coordinators {\itshape kânà} and {\itshape nânà} `or'. Disjunction is rather rare in the corpus, where only the variant {\itshape kânà} appears, but speakers state that it can always be replaced with {\itshape nânà}. Just like the conjunction coordinator {\itshape nà}, {\itshape kânà}/{\itshape nânà} can appear in between clauses and sentence initially, as in \REF{or}. Here, Nzambi explains that his friend told him to kill people in order to help them get white skin. He then concludes in a new sentence `Or I also broke the interdiction', as an alternative judgement of his deeds.


\ea\label{or} 
  \glll {\bfseries kánâ} mɛ̀ kɔ̀bɛ́ ndáà tsì mɛ̀ɛ́ lémbólɛ́\\
        kánâ mɛ kɔ̀bɛ-H ndáà tsì mɛ̀ɛ́ lémbo-lɛ\\
        or 1\textsc{sg}.{\PST}1 break-{\R} also $\emptyset$7.interdiction 1\textsc{sg}.\textsc{prs}.{\NEG} know-{\NEG} \\
    \trans `[You were telling me to do so.] Or I also broke the interdiction, I don't know.'
\z

\REF{or1} represents an example where the disjunctive coordinator appears between two clauses. Again, it shows that both coordinators {\itshape nânà} and {\itshape kânà} can be used as `or'. In contrast to conjunction, in disjunction, there seems to be a general preference to express the (same) subject overtly in both coordinands. Thus, {\itshape wɛ́} `you' is repeated also in the second clause.


\ea\label{or1}
  \glll     wɛ́ njí nà bî {\bfseries nânà/kânà} wɛ́ lígɛ̀ \\
             wɛ-H njî-H nà bî nânà/kânà  wɛ-H lígɛ \\
               2\textsc{sg}-\textsc{prs} come-{\R} {\COM} 1\textsc{pl}.{\OBJ} or 2\textsc{sg}-\textsc{prs} stay \\
    \trans `Do you come with us or do you stay?'
\z

{\itshape kânà} can also be used in both of the coordinands, expressing `either...or'. This is shown in \REF{or2}. In this construction, the coordinator in the second clause can be abbreviated to {\itshape kâ}.


\ea \label{or2}
  \glll    {\bfseries kânà} àà njì nà byɔ̂ {\bfseries kâ(nà)} àà lúmɛ̀lɛ̀ \\
             kânà àà njì nà by-ɔ̂ kâ(nà) àà lúmɛlɛ \\
               or 1.{\FUT} come {\COM} 8-{\OBJ} or 1.{\FUT} send\\
    \trans `Either he will bring them [books] or he will send [them].'
\z

\noindent \REF{or2} also shows that the second coordinand elides its object which it shares with the first clause. Elision of shared objects is also a feature of covert coordination, as shown in \REF{coCo1}.

Finally, \REF{or3} represents a case where the first and the second coordinand are asymmetrical in that the second coordinand consists only of a negated substitute {\itshape m̀m̂} `no' of the first clause. The speaker makes a suggestion in the first coordinand, but then changes his mind and suggests the opposite.


\ea\label{or3}
  \glll mùdã̂ kɛ́ nà nyɛ̀ mánkɛ̃̂ {\bfseries kánâ} m̀m̂ \\
       m-ùdã̂ kɛ̀-H nà nyɛ̀ H-ma-nkɛ̃̂ kánâ m̀m̂ \\
         \textsc{n}1-woman go-{\R} {\COM} 1 {\OBJ}.{\LINK}-ma6-field or no  \\
    \trans `The woman [his wife] shall go with him to the field or not.'
\z





\subsection{Adversative coordination with {\itshape ndí} `but'}
\label{sec:AdverseCoord}

Adversative coordination is expressed by {\itshape ndí} `but' in Gyeli.
\citet{haspelmath2007} distinguishes different subtypes of contrast, depending on the origin of conflict. Thus, the adversative coordinator can be (i) ``oppositive'', as in \REF{but1}, (ii) ``corrective'', as in \REF{but2}, or (iii) ``counterexpectative'', as in \REF{but3}.\footnote{Examples of these different adversative subtypes stem from \citet{mauri2008}.} Gyeli does not make any lexical distinction between these subtypes, but expresses all of them with the same adversative coordinator {\itshape ndí} `but'.


\ea\label{but1} Oppositive\\
  \glll     mɛ̀ gyàgá békùndá {\bfseries ndí} Àdà à gyàgá tsílɛ̀ yá sɔ́tì \\
             mɛ gyàga-H H-be-kùndá ndí Àdà a gyàga-H tsílɛ̀ yá sɔ́tì \\
               1\textsc{sg}.{\PST}1 buy-{\PST}1 {\OBJ}.{\LINK}-be8-shoe but $\emptyset$1.{\PN} 1.{\PST}1 buy-{\PST}1 $\emptyset$7.smallness 7:{\ATT} $\emptyset$1.trousers \\
    \trans `I bought shoes whereas Ada bought shorts.'
\z

\newpage
\ea\label{but2} Corrective\\
  \glll     á sàlɛ́ bédtɔ̀ nkòlɛ́ mpfùndɔ̀ {\bfseries ndí} à nzí kɛ̀ nà kɛ̀ tsídɛ́ɛ̀ \\
              a-H sàlɛ́ bédtɔ̀ nkòlɛ́ mpfùndɔ̀ ndí a nzî-H kɛ̀ nà kɛ̀ tsídɛ́ɛ̀ \\
               1-{\NEG} {\PST}.{\NEG} ascend $\emptyset$3.hill fast but 1.{\PST}1 {\PROG}-{\R} go {\COM} $\emptyset$7.walk slow \\
    \trans `He didn't run up the hill, but went slowly.'
\z


\ea \label{but3} Counterexpectative\\
  \glll     Àdà á dyà ntɛ́ bvùbvù {\bfseries ndí} àá lálɛ́ basket \\
              Àdà a-H dyà ntɛ́ bvùbvù ndí àá lá-lɛ́ basket \\
               $\emptyset$1.{\PN} 1-\textsc{prs} $\emptyset$7.tallness $\emptyset$3.size much but 1.\textsc{prs}.{\NEG} play-{\NEG} basketball \\
    \trans `Ada is very tall, but he doesn't play basketball.'
\z

\noindent Just like other coordinators, {\itshape ndí} `but' occupies the initial position within a clause, as shown by the double occurrence of {\itshape ndí} in \REF{but4}.


\ea\label{but4}
  \glll {\bfseries ndí} mɛ̀ɛ́ sálɛ́ wɛ̂ bvùbvù {\bfseries ndí} vɛ̀dáà mɛ́ dyúwɔ́ nâ wɛ́ɛ̀ dé mwánɔ̀ nɔ́ɔ̀ \\
        ndí mɛ̀ɛ́ sâ-lɛ́ wɛ̂ bvùbvù ndí vɛ̀dáà mɛ-H dyúwɔ-H nâ wɛ́ɛ̀ dè-H m-wánɔ̀, nɔ́ɔ̀ \\
       but 1\textsc{sg}.\textsc{prs}.{\NEG} do-{\NEG} 2\textsc{sg}.{\OBJ} much but but[Bulu] 1\textsc{sg}-\textsc{prs} understand-{\R} {\COMP} 2.{\PST}2 eat-{\R} \textsc{n}1-child no  \\
    \trans `But I don't do you wrong, but I understand that you have eaten [my] child, haven't you?'
\z


In contrast to other coordinators, {\itshape ndí} is the only one that is prone to code-switching, which systematically happens both to Bulu and French. In \REF{but5}, the Bulu coordinator {\itshape vɛ̀dáà} `but' is used instead of {\itshape ndí}. In other cases, {\itshape ndí} and {\itshape vɛ̀dáà} are both used, the Gyeli variant preceding the Bulu one, as shown in \REF{but4}.



\ea\label{but5}
  \glll      yí ntɛ́gɛ̀lɛ̀ {\bfseries vɛ̀dáà} mɛ́ sùmbɛ́lɛ́ bê \\
           yi-H ntɛ́gɛlɛ vɛ̀dáà mɛ-H sùmbɛlɛ-H bê \\
              7-\textsc{prs} disturb but[Bulu] 1\textsc{sg}-\textsc{prs} greet[Kwasio]-{\R} 2\textsc{pl}.{\OBJ}  \\
    \trans `That disturbs, but I greet you.'
\z

\noindent Also, {\itshape ndí} is often substituted by the French form {\itshape mais} ({\itshape mɛ́} in Gyeli) `but', as in \REF{but6}.


\ea\label{but6}
  \glll ká wɛ́ sílɛ́ kɛ̀ sâ sálɛ́ {\bfseries mɛ́} pílì wɛ́ kɛ́ nâ  wɛ́ kɛ́ jíì mònɛ́ wɔ̂ á làwɔ́ wɛ̂ nyùmbò\\
        ká wɛ-H sílɛ-H kɛ̀ sâ sálɛ́ mɛ́ pílì wɛ-H kɛ̀-H nâ  wɛ-H kɛ̀-{\R} jíì mònɛ́ w-ɔ̂ a-H làwɔ-H wɛ̂ nyùmbò\\
         if 2\textsc{sg}-\textsc{prs} finish-{\R} go do work.7 but[French] when 2\textsc{sg}-\textsc{prs} go-{\R} {\COMP} 2\textsc{sg}-\textsc{prs} go-{\R} ask $\emptyset$1.money 1-{\POSS}.2\textsc{sg} 1-\textsc{prs} tell-{\R} 2\textsc{sg} $\emptyset$3.mouth   \\
    \trans `If you go do all the work[for a Bulu person]. . . but when you [later] go and ask for your money, he [the Bulu person] frowns at you.'
\z





\section{Subordination}
\label{sec:Sub}

As described by \citet[46-48]{haspelmath2007}, coordination and subordination generally differ in two main respects. First, while coordination can be used for both phrases and clauses, subordination only applies to clauses. Second, in contrast to coordination, clauses in subordination are not symmetrical. 
I take a traditional view on subordination, as summarized in \citet[15]{cristofaro2003},\footnote{Although \citet{cristofaro2003} proposes a different approach to subordination, her summary of the traditional view is very helpful.} which is defined by morphosyntactic criteria of syntactic embedding and structural dependency. 

In syntactic embedding, the subordinate clause functions as a constituent of another clause and combines with a specific element of the main clause. In Gyeli, relative clauses (\sectref{sec:Relativeclauses}) are embedded in verbal or non-verbal clauses, modifying a noun. In contrast, complement clauses (\sectref{sec:CompC}) serve as arguments of a predicate, combining with verbs. Adverbial clauses (\sectref{sec:ADVC}) are defined by their structural dependency on the main clause. Gyeli has several subtypes of adverbial clauses which all have in common that they cannot be used independently of the main clause. Some of them are also inflectionally reduced.













\subsection{Relative clauses}
\label{sec:Relativeclauses}

Relative clauses are embedded clauses which combine with a noun phrase constituent of a matrix clause. \citet[206]{andrews2007} offers the following functional definition:
``A relative clause (RC) is a subordinate clause which delimits the reference of an NP by specifying the role of the referent of that NP in the situation described by the RC''. 
In Gyeli, relative clauses follow a nominal head. They generally have the same syntactic structure as simple main clauses:

\begin{center}
NP [({\ATT}) S V O (X)]\textsubscript{REL}
\end{center} 

\noindent There are differences, however, in terms of the expression, elision, or cross-ref\-er\-encing of the nominal head in the relative clause, depending on its function within the relative clause, as discussed below.  Relative clauses may be introduced by an attributive marker, which in many cases is optional. 

Gyeli relative clauses are usually externally headed. I only found one example of a headless relative clause, as shown in \REF{REL3}. In this construction, the relative clause appears as the copula complement in a non-verbal predicate construction, following the \textsc{stamp} copula. The subject of the main clause serves as the object complement of the relative clause and is cross-referenced by a resumptive pronoun at the end of the relative clause. There is, however, no expression of a head.


\ea\label{REL3} 
  \glll  lèbvúú  lé tè lɔ́ɔ̀ [yá bùdɛ́ lɛ̂]\textsubscript{{\REL}} \\
         le-bvúú  lé tè lɔ́ɔ̀ {\db}ya-H bùdɛ-H lɛ̂ \\
         le5-anger 5:{\ATT} there 5.{\COP} {\db}1\textsc{pl}-\textsc{prs} have-{\R} 5.{\OBJ}    \\
    \trans `We have this anger. [lit. The anger there it is that which we have].'
\z

\noindent Other free relative clauses, as discussed in \sectref{sec:RELtype}, usually occur with a default head that takes different shapes depending on whether the head denotes a human or not.


I explore relative clauses in Gyeli in various directions. In \sectref{sec:NPheads}, I investigate the range of syntactic functions of noun phrases in the matrix clause that can serve as the head of a relative clause. I treat cleft constructions as a special subtype of relative clauses in \sectref{sec:cleft}. I then describe clause linkage of relative clauses in \sectref{sec:RELlink}. In \sectref{sec:RELsyn}, I show the different syntactic roles that the nominal head of a relative clause can take within the relative clause. I provide examples of different types of relative clauses such as restrictive, non-restrictive, and free relative clauses in \sectref{sec:RELtype} and finally give a few examples of complex relative clauses in \sectref{sec:ComplREL}. Data on relative clauses stem both from the Gyeli corpus and from answers to the relative clause questionnaire by \citet{downing2010}.


\subsubsection{Nominal heads and the main clause}
\label{sec:NPheads}

Noun phrase types that can be modified by a relative clause in Gyeli include all available noun phrases in a verbal clause, namely subject, object, and oblique noun phrases, as illustrated in \REF{REL1} through \REF{REL6}.
In \REF{REL1}, the relative clause modifies the subject noun phrase of a verbal main clause.


\ea\label{REL1}
 \glll  bwánɔ̀-békúmbé [{\bfseries bé} bà njí nà byɔ̂]\textsubscript{{\REL}} bé tɛ́lɛ́ màbé\\
         b-wánɔ̀-be-kúmbé {\db}bé ba njì-H nà by-ɔ̂ be-H tɛ́lɛ-H mà-bé\\
          ba2-child-be8-tin {\db}8:{\ATT} 2.{\PST}1 come-{\R} {\COM} 8-{\OBJ} 8-\textsc{prs} stand-{\R} here-8\\
    \trans `The few tin roofs that they brought stand here.'
\z

Relative clauses can modify object noun phrases. In \REF{REL4a}, the first object of a double object construction is followed by a relative clause.


\ea\label{REL4a} 
  \glll  vɛ̂ mwánɔ̀ wɔ́ɔ̀ [{\bfseries wà} wɛ̀ bùdɛ́ nû]\textsubscript{{\REL}} mwánɔ̀-sâ yá dè \\
         vɛ̂ m-wánɔ̀ w-ɔ́ɔ̀ {\db}wà wɛ bùdɛ-H nû m-wánɔ̀-sâ yá dè \\
          give.{\IMP} \textsc{n}1-child 1-{\POSS}.2\textsc{sg} {\db}1:{\ATT} 2\textsc{sg} have-{\R} 1:{\DEM}.{\PROX} \textsc{n}1-child-$\emptyset$7.thing 7:{\ATT} eat \\
    \trans `Give your child that you have here a little to eat.'
\z

\noindent The relative clause can also modify the second object in a double object construction, as in \REF{REL4b}.


\ea\label{REL4b} 
  \glll  vɛ̂ mɛ̂ sâ mwánɔ̀ wɔ́ɔ̀ [{\bfseries wà} wɛ̀ bùdɛ́ nû]\textsubscript{{\REL}} \\
         vɛ̂ mɛ̂ sâ m-wánɔ̀ w-ɔ́ɔ̀ {\db}wà wɛ bùdɛ-H nû \\
          give.{\IMP} 1\textsc{sg}.{\OBJ} only \textsc{n}1-child 1-{\POSS}.2\textsc{sg} {\db}1:{\ATT} 2\textsc{sg} have-{\R} 1:{\DEM}.{\PROX} \\
    \trans `Give me only your child that you have here.'
\z


Further, left-dislocated object noun phrases can be modified by a relative clause, as shown in \REF{REL5}.


\ea\label{REL5}
  \glll nyɛ̀ nâ yáà mɛ́ láà nâ sá [wɛ́ sá nɔ́gá mùdì]\textsubscript{{\REL}} àà yɔ̂ wɛ̂ nyê \\
      nyɛ nâ yáà mɛ-H láà nâ sá {\db}wɛ-H sâ-H nɔ́-gá m-ùdì àà y-ɔ̂ wɛ̂ nyê\\
        1 {\COMP} yes 1\textsc{sg}-\textsc{prs} say {\COMP} $\emptyset$7.thing {\db}2\textsc{sg}-\textsc{prs} do-{\R} 1-other \textsc{n}1-person 1.{\FUT} 7-{\OBJ} 2\textsc{sg} return\\
    \trans `He: Yes, I say, the thing that you do to another person, he will return [it] to you.'
\z

\noindent Finally, relative clauses may modify oblique noun phrases, as illustrated with the locative oblique in \REF{REL6}.


\ea\label{REL6}
  \glll à làdó nà sɔ́ ɛ́ ndáwɔ̀ dé tù [{\bfseries nyà} sã́ wɛ̂ à lwɔ̃̂]\textsubscript{{\REL}} \\
      a làdo-H nà sɔ́ ɛ́ ndáwɔ̀ dé tù {\db}nyà sã́ w-ɛ̂ a lwɔ̃̂\\
      1.{\PST}1 meet-{\R} {\COM} $\emptyset$1.friend {\LOC} $\emptyset$9.house {\LOC} inside {\db}9:{\ATT} $\emptyset$1.father 1-{\POSS}.3\textsc{sg} 1.{\PST}1 build  \\
    \trans `He met with a friend in the house that his father built.'
\z

Relative clauses further appear in noun phrases of non-verbal clauses. They can appear both with the main clause's subject, as in \REF{REL2} and with noun phrases in complement position, as in \REF{REL2b}.


\ea\label{REL2} 
  \glll  bã̀ [{\bfseries yá} bwánɔ̀ bá lɔ́ làwɔ̀]\textsubscript{{\REL}} yíì tè \\
         bã̀ {\db}yá b-wánɔ̀ ba-H lɔ́ làwɔ yíì tè \\
         $\emptyset$7.word {\db}7:{\ATT} ba2-child 2-\textsc{prs} {\RETRO} speak 7.{\COP} there \\
    \trans `The word that the children just said is there. [it is true]'
\z



\ea\label{REL2b} 
  \glll  bàngyɛ́'ɛ̀lɛ̀ báà bùdì [bá gyíkɛ́sɛ́ bwánɔ̀]\textsubscript{{\REL}} \\
         ba-ngyɛ́'ɛ̀lɛ̀ báà b-ùdì {\db}ba-H gyíkɛsɛ-H bwánɔ̀ \\
         ba2-teacher 2.{\COP} ba2-person {\db}2-\textsc{prs} teach-{\R} ba2-child \\
    \trans `Teachers are people who teach children.'
\z

\noindent A special type of non-verbal clause that embeds a relative clause is the so-called cleft construction, which I discuss in the following section.









\subsubsection{Cleft constructions}
\label{sec:cleft}

Cleft constructions describe a type of non-verbal matrix clause in which the relative clause is embedded. Gyeli has two cleft constructions, involving either (i) a \textsc{stamp} copula or (ii) the identificational marker {\itshape wɛ́}.   Both constructions have in common that they are pragmatically motivated as an information structure strategy expressing focus (\sectref{sec:IS}).

Cleft constructions with a \textsc{stamp} copula are characterized by the default \textsc{stamp} copula of agreement class 7 {\itshape yíì} `it is' (\sectref{sec:COP}), followed by a (pro-)nominal predicate which serves as the head of the relative clause:
\begin{center}
yíì \textsc{np} [...]\textsubscript{{\REL}}
\end{center}
As shown in \REF{cleft1}, the class 7 \textsc{stamp} copula is also used when the following predicate appears in a plural class. In terms of information structure, the subject is in focus, as an answer to the question `Who eats mangoes?'.


\ea\label{cleft1}
  \glll {\bfseries yíì} {\bfseries bwánɔ̀} [bá dé mántúà]\textsubscript{{\REL}} \\
        yíì b-wánɔ̀ {\db}ba-H dè-H H-ma-ntúà  \\
          7.{\COP} ba2-child {\db}2-\textsc{prs} eat-{\R} {\OBJ}.{\LINK}-ma6-mango  \\
    \trans `It's the children who eat mangoes.'
\z

Also with cleft constructions, the use of the attributive marker is optional, as indicated by the parentheses in \REF{TREL3}. Since the attributive marker and the following \textsc{stamp} marker are identical in their shape, the omission of the attributive marker is preferred.


\ea\label{TREL3} 
  \glll yíì bwánɔ̀ bùdã̂ [({\bfseries bá}) bá sá másâ ɛ́ jíwɔ́]\textsubscript{{\REL}} \\
         yí b-wánɔ̀ b-ùdã̂ {\db}(bá) ba-H sâ-H H-ma-sâ ɛ́ jíwɔ́ \\
         7.{\COP} ba2-child ba2-woman (2:{\ATT}) 2-\textsc{prs} do-{\R} {\OBJ}.{\LINK}-ma6-game {\LOC} $\emptyset$7.river  \\
    \trans `It's the girls who are playing by the river.'
\z

\noindent While cleft constructions are mostly used to express subject focus, as in \REF{TREL3}, the nominal predicate can also serve as the object of the relative clause, as in \REF{TREL4}.


\ea\label{TREL4} 
  \glll yíì bwánɔ̀ bùdã̂ [wɛ̀ nzí nyɛ̂]\textsubscript{{\REL}} \\
         yíì b-wánɔ̀ b-ùdã̂ {\db}wɛ̀ nzí nyɛ̂ \\
         7.{\COP} ba2-child ba2-woman {\db}2\textsc{sg} {\PROG}.{\PST} see  \\
    \trans `It's the girls that you saw.'
\z

\noindent \REF{TREL5} provides an example of a double object construction, where the indirect object of the relative clause is encoded by the external head of the relative clause.


\ea\label{TREL5} 
  \glll yíì bwánɔ̀ bùdã̂ [{\bfseries bá} àà lúmɛ̀lɛ̀ bèkúlà]\textsubscript{{\REL}} \\
         yíì b-wánɔ̀ b-ùdã̂ {\db}bá àà lúmɛlɛ be-kúlà \\
         7.{\COP} ba2-child ba2-woman {\db}2:{\ATT} 1.{\FUT} send be8-present \\
    \trans `It's the girls that she will send presents to.'
\z

Under negation, the \textsc{stamp} copula is replaced by the verbal copula {\itshape bɛ̀} `be', as expected and discussed in \sectref{sec:COPbe}. Thus, in \REF{cleft2}, the negated correction of the statement `That woman ate the mangoes' is expressed by the negated verbal copula {\itshape bɛ́lɛ́} for `it is not X', while for the affirmative cleft, the \textsc{stamp} copula is used again.


\ea\label{cleft2}
  \glll tɔ̀sâ {\bfseries yí} {\bfseries bɛ́lɛ́} {\bfseries mùdã̂} {\bfseries núndɛ̀} {\bfseries yíì} {\bfseries mɛ̂} [mɛ̀ nzí dè mántúà]\textsubscript{{\REL}}\\
        tɔ̀sâ yí bɛ̀-lɛ m-ùdã̂ nú-ndɛ̀ yíì mɛ̂ {\db}mɛ nzí dè H-ma-ntúà\\
       no 7.\textsc{prs} be-{\NEG} \textsc{n}1-2 woman 1-{\ANA} 7.{\COP} 1\textsc{sg}.{\OBJ} {\db}1\textsc{sg} {\PROG}.{\PST} eat {\OBJ}.{\LINK}-ma6-mango \\
    \trans `[That woman ate the mangoes{\textemdash}] No, it is not that woman, it is me who ate the mangoes.'
\z

\largerpage
The second cleft type uses the identificational marker {\itshape wɛ́}, following a subject pronoun which serves as the head of the relative clause:
\begin{center}
{\PRO} {\ID} [...]\textsubscript{{\REL}}
\end{center}
This construction is used if the subject in focus consists of a complex lexical noun phrase, as in \REF{cleft3}. One might think of it as a resumptive cleft or an afterthought focus marking. As in the previous examples, omission of the attributive marker is preferred (but its use is grammatical).


\ea\label{cleft3} 
  \glll ntɛ́mbɔ̀ wà mùdã̂ wã̂ nyɛ̀ {\bfseries wɛ́} [bùdɛ́ mwánɔ̀ wà mùdã̂ mvúdũ̂]\textsubscript{{\REL}} \\
       ntɛ́mbɔ̀ wà m-ùdã̂ w-ã̂ nyɛ wɛ́ {\db}bùdɛ-H m-wánɔ̀ wà m-ùdã̂ m-vúdũ̂ \\
        $\emptyset$1.younger.sibling 1:{\ATT} \textsc{n}1-woman 1-{\POSS}.1\textsc{sg} 1 ID {\db}have-{\R} \textsc{n}1-child 1:{\ATT} \textsc{n}1-woman 1-one    \\
    \trans `My wife's younger sister, it is her who has one girl.'
\z


Both cleft types, with the \textsc{stamp} copula and identificational marker {\itshape wɛ́}, can appear in combination as a double cleft construction, as shown in \REF{cleft4}.  In these double clefts, first the \textsc{stamp} copula cleft type is used and then the identificational cleft with {\itshape wɛ́}. These constructions seem to be more marked than simple clefts and thus seem to emphasize the subject focus even more.


\ea\label{cleft4}  
\ea\label{cleft4b}
  \glll tɔ̀sâ [{\bfseries yíì} ntɛ̀mbɔ́ wɛ̂] [nyɛ̂ {\bfseries wɛ́}] [nzí dè mántúà]\textsubscript{{\REL}}\\
         tɔ̀sâ {\db}yíì ntɛ̀mbɔ́ w-ɛ̂ {\db}nyɛ̂ wɛ́ {\db}nzí dè H-ma-ntúà\\
       no {\db}7.{\COP} $\emptyset$1.sibling 1-{\POSS}.3\textsc{sg} {\db}1.{\OBJ} {\ID} {\db}{\PROG}.{\PST} eat ma6-mango \\
    \trans `[The woman ate the mangoes, didn't she?{\textemdash}] No, it is her sister who ate the mangoes.'
\ex \label{cleft4c}
  \glll tɔ̀sâ [{\bfseries yíì} síngì] [yɔ̂ {\bfseries wɛ́}] [nzí dè]\textsubscript{{\REL}}\\
        tɔ̀sâ {\db}yíì síngì {\db}y-ɔ̂ wɛ́ {\db}nzí dè\\
       no {\db}7.{\COP} $\emptyset$7.monkey {\db}7-{\OBJ} ID {\db}{\PROG}.{\PST} eat  \\
    \trans `[The woman ate the mangoes, didn't she?{\textemdash}] No, it is the monkey who ate [them].'
\z
\z





%with complementizer:

%no cleft, but rather sentence focus?
%\begin{exe} 
%\ex\label{cleft2}
%  \glll   yíì nâ báà bvúbvù. \\
%         yíì nâ báà bvúbvù \\
%            7.ID {\COMP} 2.ID many \\
%    \trans `It is that they are many.'
%\end{exe}


%\begin{exe} 
%\ex\label{C127} 
%  \glll  báà nâ bìsɔ́mɔ̀nɛ̀ bìsɔ́mɔ̀nɛ̀ bé nyì. \\
%        báà nâ bi-sɔ́mɔ̀nɛ̀ bi-sɔ́mɔ̀nɛ̀ be-H nyì \\
%         2.{\COP} {\COMP} be8-complaint be8-complaint 8-\textsc{prs} enter   \\
%    \trans `it is them that complaints over complaints start.'
%\end{exe}





%\subsubsection{Associative Focus}
%\label{sec:AssFOC}





\subsubsection{Linkage of relative clauses}
\label{sec:RELlink} 

Gyeli does not have a distinct marker of relative clauses such as, for instance, relative pronouns. Instead, an attributive marker ({\ATT}) can be used to indicate the embedding relation between subordinate clause and modified noun phrase. This attributive marker, which agrees in gender with the head noun, is also used in noun + noun constructions, as discussed in \sectref{sec:ATT}. In most cases, however, the use of the attributive marker is optional so that a relative clause is often not marked by a dedicated morpheme. The circumstances under which speakers omit the attributive marker in contrast to using it are not clear. In the corpus, about half of the relative clauses appear with an attributive marker, as in \REF{REL4b}, and about half without, as in \REF{REL5}. Few generalizations can be made at this point as to what conditions the marker's appearance or omission. Both appearance and omission occur with attributive markers of all agreement classes, singular and plural. Further, attributive markers and their omission are found with all subject, object, and oblique noun phrases that are  modified. Finally, the role that the head noun plays in the relative clause does not seem to be decisive for the appearance or omission of the attributive marker, since examples of both variants are found for cases where the head of the relative clause is the subject or any type of object of the relative clause, as I will show below.  The only criterion that seems to favor attributive marker deletion is when the attributive marker and the following \textsc{stamp} marker are identical in shape, as for instance in \REF{SREL1}.

All relative clauses are marked prosodically in that they are treated as distinct intonation units. As such, verb-final relative clauses do not take a realis-marking H tone in the realis moods as they would within an intonation phrase. In \REF{SREL1x}, the verb {\itshape sâ}  `do' surfaces with the underlying verb tone and does not take the realis-marking H tone that it would take if the verb was not at the boundary of an intonation phrase. 

\ea\label{SREL1x}
  \glll sá [wɛ́ {\bfseries sâ}]\textsubscript{{\REL}} yí bɛ́lɛ́ mpà\\
  sá {\db}wɛ-H sâ yi-H bɛ̀-lɛ mpà\\
  $\emptyset$7.thing {\db}2\textsc{sg}-{\PRS} do 7-{\PRS} be-{\NEG} good\\ 
  \trans `The thing you do is not good.'
\z

Also, a pause indicates the end of a relative clause.




\subsubsection{Nominal heads and the relative clause}
\label{sec:RELsyn}

Relative clauses can further be distinguished based on the syntactic function of the head noun within the relative clause. The head noun can serve, for instance, as the subject of the relative clause, but also as an object or an oblique.

In \REF{SREL1}, the nominal head noun {\itshape bwánɔ̀-bùdã̂̂} `girls' serves as the subject of the relative clause. In these constructions, the nominal head of the matrix clause is cross-referenced by the \textsc{stamp} marker indicating subject agreement. The relative clause follows the basic word order of S V. In the absence of an attributive marker, prosody is the only means to indicate the relative clause which otherwise would not be distinguishable from a basic clause followed by another basic clause.


\ea\label{SREL1}
  \glll  bwánɔ̀-bùdã̂ [bá lìmbɔ́ dyúà]\textsubscript{{\REL}} bá sá másâ ɛ́ nsá'à wá jíwɔ́ \\
         b-wánɔ̀-b-ùdã̂ {\db}ba-H lìmbɔ-H dyúà ba-H sâ-H H-ma-sâ ɛ́ nsá'à wá jíwɔ́ \\
         ba2-child-ba2-woman {\db}2-\textsc{prs} know-{\R} swim 1-\textsc{prs} do-{\R} {\OBJ}.{\LINK}-ma6-game {\LOC} $\emptyset$3.shore 3:{\ATT} $\emptyset$7.river  \\
    \trans `The girls who know how to swim are playing at the riverbanks.'
\z

The head of the relative clause can also take the function of an object of the relative clause, as in \REF{SREL2} and \REF{SREL3}. In both examples, the head noun serves as the object for the main clause as well as for the relative clause with a structure of NP\textsubscript{O} [S V \_\textsubscript{O} (X)]. The object is only expressed in the main clause, but not in the relative clause where it is neither repeated nor cross-referenced. 


\ea\label{SREL2} 
  \glll bá dyúwɔ́ lɛ́kɛ́lɛ̀ [{\bfseries lé} wɛ́ làwɔ̀]\textsubscript{{\REL}} \\
        ba-H dyúwɔ-H H-lɛ-kɛ́lɛ̀ {\db}lé wɛ-H làwɔ \\
         2-\textsc{prs} understand {\OBJ}.{\LINK}-le5-language {\db}5:{\ATT} 2\textsc{sg}-\textsc{prs} speak \\
    \trans `They understand the language that you speak.'
\z

\noindent In contrast to \REF{SREL2}, \REF{SREL3} appears without the attributive marker, but the argument structure is identical. Both examples are grammatical either way, with or without the attributive marker.


\ea\label{SREL3} 
  \glll  bí bɔ́gà yá wúmbɛ́ ndáà pã́ã̀ nyɛ̂ sâ [bá gyíbɔ́ ngyùlɛ̀ wá kùrã̂]\textsubscript{{\REL}} \\
         bí bɔ́-gà ya-H wúmbɛ-H ndáà pã́ã̀ nyɛ̂ sâ {\db}ba-H gyíbɔ-H ngyùlɛ̀ wá kùrã̂ \\
          1\textsc{pl}.{\SBJ} 2-other 1\textsc{pl}-\textsc{prs} want-{\R} also start see $\emptyset$7.thing {\db}2-\textsc{prs} call-{\R} $\emptyset$3.light 3:{\ATT} $\emptyset$7.electricity[French]  \\
    \trans `We others, we also want to first see the thing they call the light of electricity.'
\z

Double object constructions within the relative clause function similarly. The nominal head outside of the relative clause can function both as the direct and the indirect object of the relative clause, as shown in \REF{SREL4} and \REF{SREL5}, respectively. The underlying structures for both examples can be represented as NP\textsubscript{DO} [S V IO \_\textsubscript{DO}]  for  \REF{SREL4} and NP\textsubscript{IO} [S V \_\textsubscript{IO} DO] for \REF{SREL5}.
Since, however, the order of two objects is relatively free, as described in \sectref{sec:SVOO}, it is theoretically ambiguous which of the two objects corresponds to the nominal head outside of the relative clause and which role the object has that appears in the relative clause. It seems that (pragmatic) context and animacy effects determine the interpretation of patient and recipient roles.


\ea \label{SREL4} 
  \glll kálàdɛ̀ [yá Àdà nzí vɛ̀ mɛ̂]\textsubscript{{\REL}} yíì mpâ \\
         kálàdɛ̀ {\db}yá Àdà nzí vɛ̀ mɛ̂ yíì mpâ \\
        $\emptyset$7.book {\db}7:{\ATT} $\emptyset$1.{\PN} {\PROG}.{\PST} give 1\textsc{sg}.{\OBJ} 7.{\COP} good \\
    \trans `The book that Ada gave me is nice.'
\z


\ea \label{SREL5} 
  \glll mwánɔ̀-mùdã̂ [mɛ̀ nzí vɛ̀ kálàdɛ̀]\textsubscript{{\REL}} áà mpâ \\
         m-wánɔ̀-m-ùdã̂ {\db}mɛ nzí vɛ̀ kálàdɛ̀ áà mpâ \\
         \textsc{n}1-child \textsc{n}1-woman {\db}1\textsc{sg} {\PROG}-{\PST}1 give $\emptyset$7.book 1.{\COP} good \\
    \trans `The girl to whom I gave the book is nice.'
\z


If the nominal head of a relative clause corresponds to an oblique within the relative clause, it has to be marked by a resumptive pronoun following the comitative marker {\itshape nà}, as in \REF{SREL6}.


\ea \label{SREL6} 
  \glll ntfúmɔ̀ [yá tsíyɛ́ pɛ́mbɔ́ {\bfseries nà} {\bfseries wɔ̂}]\textsubscript{{\REL}} wú vúlɔ́lɛ́ ná \\
         ntfúmɔ̀ {\db}ya-H tsíyɛ-H pɛ́mbɔ́ nà w-ɔ̂ wu-H vúlɔ-lɛ ná \\
         $\emptyset$3.knife {\db}1\textsc{pl}-\textsc{prs} cut-{\R} $\emptyset$7.bread {\COM} 3-{\OBJ} 3-\textsc{prs} slice-{\NEG} anymore\\
    \trans `The knife we cut bread with does not slice anymore.'
\z

\noindent The same resumptive pronoun is used in constructions where the relative clause has a verb requiring a preposition, such as {\itshape ládo nà} `meet with' in \REF{SREL7}. In these cases, however, the object and its preposition appear in the object position after the verb, followed by potentially other oblique noun phrases.


\ea \label{SREL7} 
  \glll sɔ́ [mɛ̀ ládó {\bfseries nà} {\bfseries nyɛ̂} mbvû lã̀]\textsubscript{{\REL}} àà pándɛ̀ njì dígɛ̀ bî nàmɛ́nɔ́ \\
         sɔ́ {\db}mɛ ládo-H nà nyɛ̂ mbvû lã̀ àà pándɛ njì dígɛ bî nàmɛ́nɔ́ \\
         $\emptyset$1.friend {\db}1\textsc{sg}.{\PST}1 meet-{\R} {\COM} 1.{\OBJ} $\emptyset$3.year pass 1.{\FUT} arrive come watch 1\textsc{pl}.{\OBJ} tomorrow \\
    \trans `The friend I met last year will come to see us tomorrow.'
\z

Finally, also possessors can be relativized, as shown in \REF{SREL8}, where there is a gap for the possessor. 


\ea\label{SREL8} 
  \glll sɔ́ [mɛ̀ nzí kɔ̀lɛ̀ másínì]\textsubscript{{\REL}} áà wɛ́\\
         sɔ́ {\db}mɛ nzí kɔ̀lɛ másínì áà wɛ-H \\
         $\emptyset$1.friend {\db}1\textsc{sg}.{\PST} {\PROG}.{\PST}.{\R} borrow $\emptyset$1.bike 1.{\PST}2 die-{\PST} \\
    \trans `The friend whose bike I borrowed died.'
\z




\subsubsection{Types of relative clauses}
\label{sec:RELtype}


The relative clauses discussed so far were ``restrictive'' relative clauses, i.e.\ the relative clause limits the referent(s) of the head to a subset of entities.
There are, however, other types of relative clauses, such as non-restrictive, cleft, and free relative clauses. As I will show, these have the same structure as restrictive relative clauses.

Non-restrictive relative clauses do not limit the referent to a subset, but add information to a known participant or entity. This is the case in \REF{TREL1}, where the head of the non-restrictive relative clause serves as its subject. This structure is the same as its restrictive counterpart in \REF{SREL1}.


\ea\label{TREL1}
  \glll   Àdà [á lìmbɔ́ mbásâ]\textsubscript{{\REL}} àà sɔ́ wã́ã̀ \\
         Àdà {\db}a-H lìmbɔ-H mbásâ àà sɔ́ w-ã́ã̀ \\
         $\emptyset$1.{\PN} {\db}1-\textsc{prs} know-{\R} $\emptyset$7.hunt 1.{\COP} $\emptyset$1.friend 1-{\POSS}.1\textsc{sg}  \\
    \trans `Ada, who knows how to hunt, is my friend.'
\z

The same is true for non-restrictive relative clauses whose head has the object role in the relative clause, as in \REF{TREL2}.


\ea\label{TREL2}
  \glll míyù wã́ã̀ [wɛ̀ nzí nyɛ̂ ndáwɔ̀]\textsubscript{{\REL}} àà ngyɛ́'ɛ̀lɛ̀ \\
          míyù w-ã́ã̀ {\db}wɛ nzí nyɛ̂ ndtáwɔ̀ àà ngyɛ́'ɛ̀lɛ̀ \\
         $\emptyset$1.sibling 1-{\POSS}.1\textsc{sg} {\db}2\textsc{sg}.{\PST}1 {\PROG}.{\PST}1 see $\emptyset$9.house 1.{\COP} \textsc{n}1-teacher\\
    \trans `My brother, who you saw at the house, is a teacher.'
\z



The third type of relative clause that \citet{downing2010} elicit in their questionnaire is free relative clauses. According to \citet{mcarthur2005}, in these constructions, the
``relative word in the nominal relative clause has no antecedent, since the antecedent is fused with the relative''. In English, {\itshape I hate what you like.} is an example of a free relative clause. In Gyeli, free relatives with a human referent are either expressed by the generic noun {\itshape mùdì} `person' or by the interrogative pronoun {\itshape nzá} `who', as shown in \REF{FREL1}. In this example, the free relative serves as the subject of the relative clause.


\ea\label{FREL1}
  \glll mɛ́ nyɛ́ {\bfseries mùdì/nzá} [nzí njì pá'à wã́ã̀]\textsubscript{{\REL}}\\
         mɛ-H nyɛ̂-H m-ùdì/nzá {\db}nzî-H njì pá'à w-ã́ã̀ \\
         1\textsc{sg}-\textsc{prs} see-{\R} \textsc{n}1-person/who {\db}{\PROG}-{\PST}1 come $\emptyset$3.side 3-{\POSS}.1\textsc{sg}   \\
    \trans `I see the person/who passed by me.'
\z

\REF{FREL2} gives an example of a free relative clause where the head is the object of the relative clause. If the generic noun {\itshape mùdì} `person' is chosen to express the free relative, the attributive marker {\itshape wà} of agreement class 1 can be used. In contrast, if the interrogative pronoun {\itshape nzá} is used, the use of the attributive marker is excluded.


\ea\label{FREL2}
  \glll mɛ̀ lã́ bɔ̀ mùdì [{\bfseries wà} Àdà kwàlɛ̀]\textsubscript{{\REL}} \\
         mɛ lã̂-H b-ɔ̂ m-ùdì {\db}wà Àdà kwàlɛ̀ \\
         1\textsc{sg}.{\PST}1 tell 2-{\OBJ} \textsc{n}1-person {\db}1:{\ATT} $\emptyset$1.{\PN} like   \\
    \trans `I told them who Ada likes.'
\z

If the referent of a free relative clause is inanimate, the generic noun {\itshape sâ} `thing' is used or the interrogative pronoun {\itshape gyí} `what', as \REF{FREL1a} demonstrates. In this example, a resumptive pronoun has to appear in the relative clause. Whether {\itshape sâ} `thing' or the interrogative pronoun {\itshape gyí} `what' is used, the resumptive pronoun will be of agreement class 7 in both cases.


\ea\label{FREL1a}
  \glll mɛ́ nyɛ́ {\bfseries sâ/gyí} [bá njí nà yɔ̂]\textsubscript{{\REL}}\\
         mɛ-H nyɛ̂-H sâ/gyí {\db}ba-H njì-H {\COM} y-ɔ̂\\
         1\textsc{sg}-\textsc{prs} see-{\R} $\emptyset$7.thing/what {\db}2-\textsc{prs} come-{\R} {\COM} 7-{\OBJ}  \\
    \trans `I see the thing/what they bring.'
\z

Free relatives can also be formed with an interrogative pronoun where the interrogative serves as an object of the relative clause. This is the case in \REF{FREL3} where {\itshape nzá} `who' serves as the indirect object of the clause.


\ea\label{FREL3} 
  \glll mɛ́ lìmbɔ́ nzá [àà líbɛ̀lɛ̀ béyìgà]\textsubscript{{\REL}} \\
          mɛ-H lìmbɔ-H nzá {\db}àà líbɛlɛ H-be-yìgà \\
         1\textsc{sg}-\textsc{prs} know-{\R} who {\db}1.{\FUT} show {\OBJ}.{\LINK}-be8-picture  \\
    \trans `I know who she will show the pictures to.'
\z







\subsubsection{Complex relative clauses}
\label{sec:ComplREL}

Relative clauses can be complex in various respects. They can either involve relative clause internal coordination or complementation. \REF{CREL1} shows an instance of asyndetic coordination within the relative clause. The head of both coordinands is the same, namely {\itshape lé} `tree'. It serves as an object in both coordinands.


\ea\label{CREL1} 
  \glll lé [{\bfseries yá} wɛ́ nyɛ̂]\textsubscript{REL} [bá gyíbɔ́ ngàlɛ́]\textsubscript{{\REL}} yíì \\
        lé {\db}yá wɛ-H nyɛ̂ {\db}ba-H gyíbɔ-H ngàlɛ́ yíì \\
       $\emptyset$7.tree {\db}7:{\ATT} 2\textsc{sg}-\textsc{prs} see {\db}2-\textsc{prs} call-{\R} $\emptyset$7.tree.species 7.{\COP}   \\
    \trans `The tree that you see and that they call `ngàlɛ́' is that.'
\z

\noindent Relative clauses can also be coordinated overtly with the conjunction {\itshape nà}, as shown in \REF{CREL2}.


\ea\label{CREL2} 
  \glll bwánɔ̀ [bà sílɛ̃́ɛ̃̀ lã̂ békálàdɛ̀ {\bfseries nà} bà sílɛ̃́ɛ̃̀ dyíkɛ̀sɛ̀]\textsubscript{{\REL}} bá kùgá nà kɛ̀ ndáwɔ̀\\
        b-wánɔ̀ {\db}ba sílɛ̃́ɛ̃̀ lã̂ H-be-kálàdɛ̀ nà ba sílɛ̃́ɛ̃̀ dyíkɛsɛ ba-H kùga-H nà kɛ̀ ndáwɔ̀\\
       ba2-child {\db}2.{\PST}1 finish.{\COMPL} read {\OBJ}.{\LINK}-be8-book {\CONJ} 2.{\PST}1 finish.{\COMPL} study 2-\textsc{prs} can-{\R} {\COM} go $\emptyset$9.house\\
    \trans `The children who have finished reading their books and who have finished studying can go home.'
\z

Finally, there are examples of relative clauses in which the head has a a role in an embedded complement clause, as in \REF{CREL3}.


\ea\label{CREL3} 
  \glll mùdì [mɛ́ bvúálá [{\bfseries nâ} à nzí làwɔ̀]\textsubscript{{\COMP}}]\textsubscript{{\REL}} à nzí láà dó \\
        m-ùdì {\db}mɛ-H bvúala-H {\db}nâ à nzí làwɔ à nzí láà dó \\
       \textsc{n}1-person {\db}1\textsc{sg}-\textsc{prs} think-{\R} {\db}{\COMP} 1 {\PROG}.{\PST} talk 1 {\PROG}.{\PST} tell $\emptyset$7.lie  \\
    \trans `The person that I think she spoke with was lying.'
\z













\subsection{Complement clauses and purpose clauses}
\label{sec:Compna}

The complementizer {\itshape nâ} in Gyeli marks both complement clauses and purpose clauses. There is some structural overlap between both construction types pertaining to the use of the complementizer and a dependent clause that is marked as such by the use of the subjunctive. There are, however, some differences which are reflected by a different tonal behavior with respect to the occurrence or absence of the realis-marking H tone. The complementizer further introduces reported speech and inflectionally reduced dependent clauses where the verb occurs in its non-finite form. I discuss these different constructions in turn. There is another instance where the complementizer {\itshape nâ} is used, namely in combination with an adverb as a subordinator in adverbial clauses, as discussed in \sectref{sec:AComp}.  

\subsubsection{Complement clauses}
\label{sec:CompC}

Complement clauses serve as arguments of a predicate, following \citet[52]{noonan2007}, who defines complement clauses as follows:
``By complementation, we mean the syntactic situation that arises when a notional sentence or predication is an argument of a predicate''. 
In Gyeli, clausal complementation most often occurs with verbs of perception (`hear', `see'), consciousness (`know', `remember', `think'), intention (`want', `like'), and attitude/emotion (`hate', `be happy'). 
Both obligatory arguments, as in \REF{COMP1}, and optional arguments, as in \REF{COMP}, are expressed by complement clauses. Complement clauses form one intonation unit with the main clause, as indicated by the realis-marking H tone on the verb {\itshape wúmbɛ} `want' in \REF{COMP1} and {\itshape sìsɔ} `be happy' in \REF{COMP}. In this, they differ from purpose clauses with the complementizer {\itshape nâ}, as discussed in the next section.


\ea\label{COMP1}
  \glll  mɛ́ wúmbɛ́ [{\bfseries nâ} á gyámbɔ́ɔ̀ bèdéwɔ̀]\textsubscript{{\COMP}} \\
         mɛ-H wúmbɛ-H {\db}nâ a-H gyaḿbɔ́ɔ̀ be-déwɔ̀ \\
         1\textsc{sg}-\textsc{prs} want-{\R} {\db}{\COMP} 1-\textsc{prs} cook.{\SBJV} be8-food \\
    \trans `I want her/him to cook food.'
\ex\label{COMP}
  \glll mɛ́  sìsɔ́ [{\bfseries nâ} mɛ̀ {\bfseries nzɛ́ɛ́} nyɛ̂ mándáwɔ̀] \\
         mɛ-H sìsɔ-H {\db}nâ mɛ nzɛ́ɛ́ nyɛ̂ H-ma-ndáwɔ̀ \\
         1\textsc{sg}-\textsc{prs} be.happy-{\R} {\db}{\COMP} 1\textsc{sg} {\PROG}.{\SUB} see {\OBJ}.{\LINK}-ma6-houses \\
    \trans `I'm happy that I'm seeing the houses.'
\z

In addition to being introduced by the complementizer {\itshape nâ}, Gyeli also marks the dependent clause in these constructions by using the subjunctive form when expressing intentions or orders, as in \REF{COMP1} (\sectref{sec:opt}), and the subordinate form of the progressive marker in \REF{COMP} (\sectref{sec:PROG}).


Also verbs of consciousness serve as predicates that take complement clauses. This is the case, for instance, with {\itshape lèmbo} `know', as shown in \REF{COMP2} and \REF{COMP3}.


\ea\label{COMP2}
  \glll   á lèmbó [{\bfseries nâ} bùdì báà bá múà búɛ̀lɛ̀ {\bfseries nâ} bá dyúù nyɛ̂]\textsubscript{{\COMP}}\\
          a-H lèmbo-H {\db}nâ b-ùdì báà ba-H múà búɛlɛ̀ nâ ba-H dyúù nyɛ̂\\
1-\textsc{prs} know-{\R} {\db}{\COMP} ba2-person 2.{\DEM}.{\PROX} 2-\textsc{prs} {\PROSP} fish {\COMP} 2-\textsc{prs} kill.{\SBJV} 1.{\OBJ}\\
    \trans `He knows that these people are about to fish (look for him) in order to kill him.'
\ex\label{COMP3} 
  \glll ndí wɛ́ lèmbó [{\bfseries nâ} mbvúndá nyíì bvúdà nà mbvúndá]\textsubscript{{\COMP}} \\
        ndí wɛ-H lèmbo-H {\db}nâ mbvúndá nyíì bvúda nà mbvúndá \\
         but 2\textsc{sg}-\textsc{prs} know-{\R} {\db}{\COMP} $\emptyset$9.trouble 9.{\FUT} fight {\COM} $\emptyset$9.trouble \\
    \trans `But you know that violence will create more violence.'
\z


\noindent The same is true for {\itshape bvû} `think', as in \REF{COMP4}.


\ea\label{COMP4}
  \glll    mɛ́ bvú [{\bfseries nâ} nkwálá wúù tfùndɛ́ mɛ̂ vâ]\textsubscript{{\COMP}} \\
           mɛ-H bvû-H {\db}nâ nkwálá wúù tfùndɛ-H mɛ̂ vâ \\
              1\textsc{sg}-\textsc{prs} think-{\R} {\db}{\COMP} $\emptyset$3.machete 3.{\PST}2 miss-{\R} 1\textsc{sg}.{\OBJ} here \\
    \trans `I think that the machete had missed [injured] me here.'
\z

\noindent Also verbs of perception can function as predicates taking complement clauses. An example is given in \REF{COMP5}.



\ea\label{COMP5} 
  \glll  mɛ́ dyúwɔ́ [{\bfseries nâ} mpàgó wá pɔ́dɛ̀ lã́ vâ]\textsubscript{{\COMP}}\\
        mɛ-H dyúwɔ-H {\db}nâ mpàgó wá pɔ́dɛ̀ lã̀-H vâ\\
            1\textsc{sg}-\textsc{prs} hear-{\R} {\db}{\COMP} $\emptyset$3.street 3:{\ATT} $\emptyset$1.port pass-{\R} here\\
    \trans `I hear that the road to the port passes [will pass] here.'
\z

\REF{kuga} shows that complement clauses are also used with stative verbs such as {\itshape kùga} `be enough'.


\ea\label{kuga}
  \glll    ká yí nyí mɛ̂ mbɔ̀ mpángì yí {\bfseries kùgá} nâ nyíì wɛ̂ mbɔ̀ \\
           ká yi-H nyî-H mɛ̂ m-bɔ̀ mpángì yi-H kùga-H nâ nyíì wɛ̂ m-bɔ̀ \\
             when 7-\textsc{prs} enter-{\R} 1\textsc{sg} \textsc{n}3-arm $\emptyset$7.bamboo 7.\textsc{prs} be.enough-{\R} {\COMP} enter.{\SBJV} 2\textsc{sg} \textsc{n}3-arm \\
    \trans `When it goes into my arm . . . the bamboo can sting your arm.'
\z

While complement clauses typically occur in verbal predicates, they can also be used in the 
complementation of non-verbal predicates, as in \REF{BComp2}.
In this example, the main clause expresses a prohibition, while the dependent clause specifies what the prohibition is about. The dependent clause complements the nominal predicate of the non-verbal clause.


\ea\label{BComp2} 
  \glll  yíì mpíndá [{\bfseries nâ} mɛ́ déè]\textsubscript{COMP} \\
         yíì mpíndá {\db}nâ mɛ-H déè \\
         7.{\COP}  $\emptyset$9.prohibition {\db}{\COMP} 1\textsc{sg}-\textsc{prs} eat.{\SBJV} \\
    \trans `It is forbidden that I eat.'
\z

The complement clause can even serve as the predicate itself in a non-verbal clause, as shown in \REF{nacompyi}.


\ea\label{nacompyi}
  \glll   yíì nâ báà bvùbvù \\
         yíì nâ báà bvùbvù \\
            7.{\COP} {\COMP} 2.{\COP} many \\
    \trans `It is that they are many.'
\z

Traditionally, quotes in reported discourse are viewed as a subtype of clausal complementation. As I will show in  \sectref{sec:RD}, however, reported discourse constructions are formally not the same.




\subsubsection{Purpose clauses with {\itshape nâ}}
\label{sec:Purposena}

Purpose clauses are dependent clauses that are introduced by the complementizer {\itshape nâ} and generally express purpose or intention, as illustrated  in \REF{purpna1}. Unlike complement clauses, however, the dependent clause does not function as an argument of the main clause.  Another difference to complement clauses is that the main clause is treated as an intonation phrase unit, while with complement clauses, the dependent clause is also part of that unit. This can be seen in the tonal behavior with respect to the realis-marking H tone. In \REF{purpna1}, the verb {\itshape gyámbɔ} `cook' in the main clause surfaces with a final L tone. In contrast,  a complement clause would license the realis-marking H tone to surface, as shown in \REF{COMP1} above.


\ea\label{purpna1}
  \glll mɛ́ gyám{\bfseries bɔ̀} [nâ wɛ́ déè]\textsubscript{{\COMP}}\\
        mɛ-H gyámbɔ {\db}nâ wɛ-H déè\\
      1\textsc{sg}-\textsc{prs} cook {\db}{\COMP} 2\textsc{sg}-\textsc{prs} eat.{\SBJV} \\
    \trans `I cook so that you eat.'
\z

Another example of a purpose clause is given in \REF{BComp4}. In this instance, too, the subjunctive is used.


\ea\label{BComp4}
  \glll á lúndɛ́lɛ́ bɔ̂ lèkàá lé ndáwɔ̀ nyî [{\bfseries nâ} bɛ́ɛ̀ vyâ]\textsubscript{{\COMP}} \\
       a-H lúndɛlɛ-H b-ɔ̂ le-kàá lé ndáwɔ̀ nyî {\db}nâ bɛ́ɛ̀ vyâ \\
       1-\textsc{prs} fill-{\R} 2-{\OBJ} le5-kind 5:{\ATT} $\emptyset$9.house 9.{\DEM}.{\PROX} {\db}{\COMP} be.{\SBJV} full \\
    \trans `He fills them in this kind of house so that it [the house] be full.'
\z

In contrast,  \REF{BComp3} appears with a \textsc{present} tense-mood marking in the {\itshape nâ} clause, although also a subjunctive marking is equally possible.


\ea\label{BComp3}
  \glll ɔ̀ múà gyɛ́sɔ̀ [{\bfseries nâ} wɛ́ kɛ̀]\textsubscript{{\COMP}} \\
       ɔ múà gyɛ́sɔ {\db}nâ wɛ-H kɛ̀ \\
         2\textsc{sg}[Kwasio] {\RETRO} search {\db}{\COMP} 2\textsc{sg}-\textsc{prs} go   \\
    \trans `You are about to want to leave.'
\z

Purpose clauses with {\itshape nâ} not only modify main clauses, but also other dependent clauses, as for instance adverbial subordinate clauses in \REF{BComp5}. In this example, the adverbial clause precedes the main clause and so does the complementizer clause, which modifies the adverbial clause.



\ea\label{BComp5}
  \glll [pílì wɛ́ kɛ̀ [{\bfseries nâ}  wɛ́ kɛ́ jíì mònɛ́ wɔ̂]\textsubscript{{\COMP}}]\textsubscript{{\ADV}} á làwɔ́ wɛ̂ nyùmbò\\
        {\db}pílì wɛ-H kɛ̀ {\db}nâ  wɛ-H kɛ̀-{\R} jíì mònɛ́ w-ɔ̂ a-H làwɔ-H wɛ̂ nyùmbò\\
        {\db}when 2\textsc{sg}-\textsc{prs} go {\db}{\COMP} 2\textsc{sg}-\textsc{prs} go-{\R} ask $\emptyset$1.money 1-{\POSS}.2\textsc{sg} 1-\textsc{prs} tell-{\R} 2\textsc{sg} $\emptyset$3.mouth   \\
    \trans `Whenever you go ask [a Bulu person] for your money, he frowns at you.'
\z


%without main clause?

%\begin{exe} 
%\ex\label{BComp6} 
 % \glll {\bfseries nâ} wɛ̀ sílɛ̃́ɛ̃̀ nyàà dyùù mpɔ̀ngɔ̀ yá bùdì! \\
 %       nâ wɛ sílɛ̃́ɛ̃̀ nyàà dyùù mpɔ̀ngɔ̀ yá b-ùdì \\
 %        {\COMP} 2\textsc{sg}.{\PST}1 finish.{\COMPL} shit kill $\emptyset$7.generation 7:{\ATT} ba2-person \\
 %   \trans `That you have completely killed a generation of people!'
%\end{exe}












\subsubsection{Reported discourse and other depictions}
\label{sec:RD}

The complementizer {\itshape nâ} introduces {\itshape depictions}\footnote{A more detailed discussion on the concept of depiction in contrast to description is given in \citet{clark90}, \citet{guldemann2008}, and \citet{dingemanse2015}. \citet{soulaimani2018}, for instance, investigates in particular the role of gesture and voice patterns in reported discourse.} such as reported discourse, ideophones, and gestures that contribute content to the speech event, for instance as embodied reenactments.
As we shall see below, these depiction constructions differ from both complement and purpose clause uses. This is in line with \posscitet{spronck2019} claim that reported speech forms a dedicated syntactic domain. 

In the following, I will mostly concentrate on reported discourse, since this is most pervasive in the text corpus, and then conclude this section with examples of ideophones and gestures that are introduced by {\itshape nâ}.
In terms of the terminology related to reported discourse, I follow \citet[6]{guldemann2008}:

\begin{quote}
Reported discourse is the representation of a spoken or mental text from which the reporter distances him-/herself by indicating that it is produced by a source of consciousness in a pragmatic and deictic setting that is different from that of the immediate discourse.
\end{quote}

Structurally, \citet{guldemann2008} distinguishes the quote, i.e.\ the reported spoken or mental text, from the quotative index ({\QI}), which serves to introduce the quote. Thus, in \REF{RS}, the unit marked as ``{\QI}'' introduces the reported text which, in turn, is marked by ``{\RD}''.

\ea\label{RS}
  \glll  [yɔ́ɔ̀ bá kí {\bfseries nâ}]\textsubscript{{\QI}} [ɛ́ɛ́kɛ̀ mwánɔ̀ wɛ́ɛ̀ mùdã̂ wà nù à bwã́ã̀]\textsubscript{{\RD}}\\
         {\db}yɔ́ɔ̀ ba-H kì-H nâ {\db}ɛ́ɛ́kɛ̀ m-wánɔ̀ w-ɛ̂ m-ùdã̂ wà nù a bwã́ã̀ \\
         {\db}so 2-\textsc{prs} say-{\R} {\COMP} {\db}{\EXCL} \textsc{n}1-child 1-{\POSS}.3\textsc{sg} \textsc{n}1-woman 1:{\ATT} 1:DEM 1.{\PST}1 give.birth.{\PRF} \\
    \trans `So they say: ``Oh, his child who is the wife of that one, has already given birth".'
\z

The structures both of the quotative index and of the quote differ from typical matrix and subordinate clauses. As for the {\QI}, the complementizer {\itshape nâ} belongs prosodically to the {\QI} and not to the quote, which is indicated by a pause after the complementizer.\footnote{This phenomenon has also been noted, for instance, in Hausa, as \citet[236]{guldemann2008} points out.} In some cases, the complementizer also undergoes salient lengthening, in addition to the following pause, as shown in \REF{RS1}.\footnote{In this example, the speaker has switched to Bulu and is reminded by the interpreter to speak in Gyeli. He then repeats what he has said by quoting his own speech. His quote is emphasized by the lengthened complementizer.} This does not happen in purpose clauses where {\itshape nâ} rather belongs to the dependent clause, also prosodically.\footnote{Concerning the relationship between complement clauses and instances of reported discourse, there might be a continuum, since also complement clauses with `say' or `think' verbs in the main clause may constitute examples of reported discourse as representations of spoken or mental text.}



\ea\label{RS1} 
  \glll  [mɛ́ làwɔ́ {\bfseries náà}]\textsubscript{{\QI}} [màndáwɔ̀ má zì má kùgáà mɛ̂ vâ]\textsubscript{{\RD}} \\
         {\db}mɛ-H làwɔ-H nâ {\db}ma-ndáwɔ̀ má zì ma-H kùgáà mɛ̂ vâ \\
            {\db}1\textsc{sg}-\textsc{prs} talk-{\R} {\COMP} {\db}ma6-house 6:{\ATT} $\emptyset$7.tin 6-\textsc{prs} be.enough.{\SBJV} 1\textsc{sg}.{\OBJ} here \\
    \trans `[Speak Gyeli!{\textemdash}] I say that there should be enough tin (roofed) houses here for me.'
\z

Most {\QI}s in Gyeli are bipartite, containing a verbal predicate, usually a say-verb, and the complementizer {\itshape nâ}. This is the case in \REF{RS} with the say-verb {\itshape kì} `say', which is the most common and frequent predicate in a {\QI}, and in \REF{RS1} with {\itshape làwɔ} `talk'. Another element that can appear in the {\QI} is the verbal copula {\itshape bùdɛ́} `have', as shown in \REF{RS2}.


\ea\label{RS2} 
  \glll  [mɛ́ mɛ̀ bùdɛ́ nâ]\textsubscript{{\QI}} [ɛ́ pɛ̀ ɛ́ wû bèyá lwɔ̃́ kwádɔ́ yã̂ ɛ́ wû]\textsubscript{{\RD}} \\
       {\db}mɛ́ mɛ bùdɛ-H nâ ɛ́ pɛ̀ {\db}ɛ́ wû bèya-H lwɔ̃̂-H kwádɔ́ y-ã̂ ɛ́ wû\\
         {\db}but[French] 1\textsc{sg} have-{\R} {\COMP} {\db}{\LOC} over.there {\LOC} there 2\textsc{pl}[Kwasio]-\textsc{prs} build-{\R} $\emptyset$7.village 7-{\POSS}.1\textsc{sg} {\LOC} there\\
    \trans `But I say that over there, there you (pl.) build my village over there.'
\z

When {\itshape bùdɛ́} is used in a {\QI}, it generally seems to imply a wish, request, order, or some sort of intention expression, as also shown in \REF{RS3}.


\ea\label{RS3}
  \glll [bvúlɛ̀ bà bùdɛ́ nâ]\textsubscript{{\QI}} [ká wɛ̀ ngyɛ̀lì wɛ̀ bùdɛ́ tsídí wɔ̂]\textsubscript{{\RD}} bá sɛ̀ngɛ́ nyɛ̂ sí \\
        {\db}bvúlɛ̀ ba bùdɛ-H nâ {\db}ká wɛ n-gyɛ̀lì wɛ bùdɛ-H tsídí w-ɔ̂ ba-H sɛ̀ngɛ-H nyɛ̂ sí \\
          {\db}ba2.Bulu 2 have-{\R} {\COMP} {\db}if 2\textsc{sg} \textsc{n}1-Gyeli 2\textsc{sg} have-{\R} $\emptyset$1.animal 1-{\POSS}.2\textsc{sg} 2-\textsc{prs} lower-{\R} 1.{\OBJ} down   \\
    \trans `The Bulu say that if you, Gyeli, you have your animal, they lower it [its price].'
\z

QIs in Gyeli can also occur without any predicate,  which distinguishes them from matrix clauses of complement clauses. Minimally, they contain speak\-er reference in the form of a subject pronoun and the complementizer {\itshape nâ}, as demonstrated in \REF{RS4}.


\ea\label{RS4}
  \glll  [nyɛ̀ nâ]\textsubscript{{\QI}} [ooh mùdã̂ bàmbɛ́ kɛ̀ jíì mbɔ́mbɔ̀ mwánɔ̀ sá yí dè]\textsubscript{{\RD}} \\
         {\db}nyɛ̀ nâ {\db}ooh m-ùdã̂ bàmbɛ́ kɛ̀ jíì mbɔ́mbɔ̀ m-wánɔ̀ sá yí dè \\
        {\db}1.{\SBJ} {\COMP} {\db}{\EXCL} \textsc{n}1-woman sorry go ask.{\IMP} $\emptyset$1.namesake \textsc{n}1-child $\emptyset$7.thing 7.{\DEM} eat  \\
    \trans `He: `Oh, wife, excuse me, go and ask the namesake [the other Nzambi] for a little to eat.''
\z

\hspace*{-3.5pt}Non-clausal {\QI}s, as in \REF{RS4}, provide another argument against analyzing repor\-ted discourse as typical clausal complementation. These non-clausal {\QI}s, which occur pervasively in the corpus, do not possess any predicate that could require a complement clause.\footnote{\citet[226-233]{guldemann2008} lists other arguments against a sentential complementation analysis for direct reported discourse. For instance, often the {\QI} does not have to be expressed at all in direct reported discourse. Also, if the {\QI} includes a predicate, the predicate does not  necessarily have a quote-oriented valency.} Instead of analyzing the {\QI} as the matrix clause of the quote that serves as a complement, it seems more consistent to view
 the {\QI} being the tag to the quote on a higher structural level than sentential units, as \citet[231]{guldemann2008} explains.

While the arguments that Güldemann puts forth apply to direct reported discourse, I also extend them to indirect reported discourse for there is no structural difference in marking direct and indirect speech in Gyeli. Differences only concern ``quote-internal referential adjustments'' (p. 234) such as pronominal marking and the use of exclamations, which are restricted to direct reported discourse. In the corpus, most instances of reported discourse are direct.
There are, however, also examples of indirect speech, as in \REF{IRS}.


\ea\label{IRS}
  \glll [mùdì wà sɔ̀ndyɛ́ à nzí kí nâ]\textsubscript{{\QI}} [ká mɛ̀ nyɛ́ àksìdɛ̃̂]\textsubscript{{\RD}}\\
        {\db}m-ùdì wà sɔ̀ndyɛ́ a nzî-H kì-H nâ {\db}ká mɛ nyɛ̂-H àksìdɛ̃̂ \\
         {\db}\textsc{n}1-person 1:{\ATT} $\emptyset$1.police 1.{\PST} {\PROG}-{\R} say {\COMP} {\db}if 1\textsc{sg}.{\PST} see-{\R} $\emptyset$1.accident[French] \\
    \trans `The police officer asked whether I saw that accident.'
\z

Also the quote displays characteristics that are not usually associated with subordinate clauses, which has been noted for other languages as well, for instance by \citet{spronck2017}. Quotes can be significantly longer or shorter than usual subordinate clauses. They can actually comprise several sentences (see, for instance \REF{50} through \REF{n53} in \appref{sec:Nzambi}).
On the other hand, they can consist of only an exclamation, as in \REF{RS6}.


\ea\label{RS6}
  \glll  [yɔ́ɔ̀ bá kí nâ]\textsubscript{{\QI}} [ɛ́ɛ́kɛ̀]\textsubscript{{\RD}} \\
            {\db}yɔ́ɔ̀ ba-H ki-H nâ {\db}ɛ́ɛ́kɛ̀ \\
         {\db}so 2-\textsc{prs} say-{\R} {\COMP} {\db}{\EXCL} \\
    \trans `So they say: [{\EXCL} of surprise]!'
\z



\noindent \REF{RS6} illustrates neatly how quotes may depict rather than describe speech events.


\subsubsubsection*{{\bfseries Ideophones}} These complementizer constructions also extend to the depiction of non-speech events in the form of ideophones (\sectref{sec:IDEO}). Just like with reported speech, the complementizer {\itshape nâ} can introduce an ideophone, as in \REF{IDEOna1} and \REF{IDEOna2}.\footnote{For a dynamic and dramatic effect in the narration, the verb in \REF{IDEOna2} is not expressed, but the action is clear from the ideophone.}


\ea\label{IDEOna1}
  \glll nâ wɔ̀m, mùdì núú jí nâ wɔ̀m\\
       nâ wɔ̀m m-ùdì núú jî-H nâ wɔ̀m \\
        {\COMP} {\IDEO} \textsc{n}1-person 1.{\DEM}.{\DIST} stay-{\R} {\COMP} {\IDEO} \\
   \trans `[I request that] there be silence, that person should stay silent.'
\ex\label{IDEOna2}
  \glll Nzàmbí màbɔ́ɔ̀ nkwɛ́ɛ̀ dé nâ vɔ́sì \\
        Nzàmbí ma-bɔ́ɔ̀ nkwɛ́ɛ̀ dé nâ vɔ́sì \\
          $\emptyset$1.{\PN} ma6-breadfruit $\emptyset$3.basket {\LOC} {\COMP} {\IDEO}:pouring\\
    \trans `Nzambi pours the breadfruit into the basket.'
\z

In contrast to reported discourse, however, the complementizer is not part of a {\QI} in such constructions, but can either occur without a matrix clause at the beginning of a sentence, as in \REF{IDEOna1}, or at the end of the phrase in a typical adjunct position describing manner, as in \REF{IDEOna2}.



\subsubsubsection*{{\bfseries Gestures}} Parallel to the depiction of manner in non-speech events with ideophones, the complementizer is also used in non-sound depictions of gestures or bodily reenactments, as in \REF{Gest}.


\ea\label{Gest}
  \glll ká á dígɛ́ nâ [gesture] á nyɛ́ mbúmbù wɛ́ɛ̀ á pámò \\
  ká a-H dígɛ-H nâ [gesture] a-H nyɛ̂-H mbúmbù w-ɛ́ɛ̀ a-H pámo\\
       when 1-\textsc{prs} look-{\R} {\COMP} [gesture]  1-\textsc{prs} see-{\R} $\emptyset$1.namesake 1-{\POSS}.3\textsc{sg} 1-\textsc{prs} arrive \\
    \trans `When he looks like [imitation of manner of looking], he sees his namesake who arrives.'
\z




\subsubsection{Complementizer + infinitive constructions}
\label{sec:COMPINF}

The complementizer {\itshape nâ} is also used in subordination of inflectionally reduced clauses which are similar to infinitival adverbial constructions without subordinator (\sectref{sec:InfSub}). The difference is, however, that subordination is marked by the complementizer {\itshape nâ} (and not ``linkless'') and that the subject of the subordinate clause is overtly marked. If the subject of the main clause and the subject of the subordinate clause are coreferential,  as in \REF{COMPINF1}, the subject is still marked by a pronoun.


\ea\label{COMPINF1}
  \glll mùdã̂ à lɔ́ sìsɛ̀lɛ̀ nɔ́nɛ́gá [nâ nyɛ̂ nà kɔ́sɛ̀]\\
m-ùdã̂ a lɔ́ sìs-ɛlɛ n-ɔ́nɛ́gá {\db}nâ nyɛ̂ nà kɔ́sɛ \\
\textsc{n}1-woman 1.{\PST} {\RETRO} scare-{\AP}PL 1-other {\db}{\COMP} 1.{\SBJ} {\COM} cough \\
  \trans `The woman scared the other by her coughing.'
\z

In contrast, subjects in infinitival adverbial constructions are zero-expressed. Their subject referent is retrieved from the context and very often coreferential with the subject of the main clause. In complementizer + infinitive constructions, however, the subjects of the main and the dependent clause are clearly marked when they differ in their reference, as in \REF{COMPINF2}.



\ea\label{COMPINF2}
  \glll 
bèlɛ́ɛ́ bè lɔ́ kwè nâ mùdã̂ nà tsíndɔ̀ mùdũ̂ \\
be-lɛ́ɛ́ be lɔ́ kwè nâ m-ùdã̂ nà tsíndɔ m-ùdũ̂ \\
be8-glass 8.{\PST} {\RETRO} fall {\COMP} \textsc{n}1-woman {\COM} push \textsc{n}1-man \\
\trans `The glasses fell, the woman having pushed the man.'
\z

Generally, these constructions encode complex causal chains.











\subsection{Adverbial clauses}
\label{sec:ADVC}

Adverbial clauses function as modifiers of verb phrases or entire clauses \citep{thompson2007}.
I distinguish four types of adverbial clauses in Gyeli, as shown in \tabref{Tab:ADVC}. This distinction is based on the inflectional status of the verb, the type of clause linkage devices \citep{hetterle2015}, and other subordinate markers, such as special aspect forms.

\begin{table}

\begin{tabularx}{\textwidth}{X lll}
 \lsptoprule
Clause type	&	Adverbial & Gloss & Function\\
 \midrule
\multirow{5}{*}{{Full adverbial clause}}  &  {\itshape líní} & `when' & temporal \\
				& {\itshape pílì/pílɔ̀} & `when' & temporal \\
				& {\itshape tɔ̀}      & `even, although' & concessive \\
				& {\itshape púù yá} & `because' & causal \\
				& {\itshape yɔ̃́ɔ̃̀} & `time' & temporal \\
				& {\itshape ká}     & `if'           & conditional \\  \midrule
Adverbial +  &  {\itshape lí nâ} & `when' & simultaneity \\ 
complementizer clause				& {\itshape sɔ́ɔ̀ nâ} & `before' & anteriority \\
				& {\itshape púù nâ} & `because' & causal \\  \midrule
\multirow{3}{*}{Adverbial infinitival clause}   & $\emptyset$   &   & anteriority,  \\
                                                 &    &   & simultaneity, \\
                                                &    &   & sequential \\  \midrule
{{\itshape nzɛ́ɛ́} subordination} &  $\emptyset$ &  & simultaneity\\
 \lspbottomrule
\end{tabularx}
\caption{Adverbial clause types}
\label{Tab:ADVC}
\end{table}

First, full adverbial clauses have fully inflected verb forms and contain minimally a subject argument and a verb. They are linked to the main clause by an adverbial or by a nominal construction that acts like an adverb. I discuss most full adverbial clause constructions in \sectref{sec:ADVfull}. Conditional clauses are a type of full adverbial clause. As I discuss them at length, paying special attention to irrealis-marking, I describe these constructions separately in \sectref{sec:Cond}. The second type of adverbial clauses (\sectref{sec:AComp}) uses a combined clause linkage device including an adverbial and the complementizer {\itshape nâ}. The third type of adverbial clause (\sectref{sec:InfSub}) is special in that it has no clause linkage device and the dependent clause is reduced: it lacks subject expression and the verb appears in its non-finite form. Finally, subordination can be encoded by the special progressive form {\itshape nzɛ́ɛ́}, which is exclusively used in dependent clauses, as discussed in \sectref{sec:SUBnzee}.


\subsubsection{Full adverbial clauses}
\label{sec:ADVfull}

Gyeli uses a range of adverbializers to introduce full subordinate clauses, including temporal, concessive, causal, and conditional clauses. These adverbializers differ in their grammatical characteristics, ranging from adverbs to nominals, but all of them function as a subordinator in adverbial clauses.
There are three variants for temporal adverbializers, namely {\itshape líní} and {\itshape pílì} or {\itshape pílɔ̀}. {\itshape pílì} occurs most frequently in the corpus, while {\itshape pílɔ̀} and {\itshape líní} may be loanwords from neighboring languages, since they are also used in, for instance, Mabi. When asked, speakers state, however, that they are also Gyeli words.


\subsubsubsection*{{\bfseries Temporal {\itshape líní} `when'}}
The adverb {\itshape líní} `when' is a temporal adverb that only showed up in elicitation, but not in the corpus. \REF{lini1x} gives an example of a preposed adverbial clause with {\itshape líní}.



\ea\label{lini1x}
  \glll   [{\bfseries líní} á sílɛ́ dè mántúà]\textsubscript{{\ADV}} à tí ná dyúwɔ̀ nzà \\
{\db}líní a-H sílɛ-H dè H-ma-ntúà, a tí ná dyúwɔ nzà  \\
{\db}when 1-\textsc{prs} finish-{\R} eat {\OBJ}.{\LINK}-ma6-mango 1 {\NEG} anymore feel $\emptyset$9.hunger\\
    \trans `When he has eaten mangoes, he does not feel hungry anymore.'
\z

\noindent \REF{lini2x} provides an example of a postposed adverbial clause with {\itshape líní}. Both sentences express temporal sequences, the event of the adverbial clause happening before the event of the main clause.


\ea\label{lini2x}
  \glll     á súmɛ́lɛ́ bùdì [{\bfseries líní} á pámɔ́ tísɔ̀nì]\textsubscript{{\ADV}} \\
            a-H súmɛlɛ-H b-ùdì {\db}líní a-H pámɔ-H tísɔ̀nì  \\
            1-\textsc{prs} greet-{\R} ba2-person {\db}when 1-\textsc{prs} arrive-{\R} $\emptyset$7.town \\
    \trans `He greets the people after having arrived in town.'
\z





\subsubsubsection*{{\bfseries{Temporal {\itshape pílì/pílɔ̀} `when'}}}
%status of pílì not entirely clear. It could also have nominal status and is then followed by a relative clause that consistently lacks an attributive marker. Further complication: in terms of its meaning, it sometimes seems to mean 'sometimes, at times' and also precedes a phrase. In those cases, it is not clear whether pílì introduces an adverbial subordinate clause or rather is a left-dislocated oblique. Probably, it has a status in-between the two and is more grammaticalized than other nouns, comparable to sí 'ground' which in certain contexts also functions as the directional adverb 'down'.

%decide to treat it like adverbial since it is parallel in structure to other adverbials such as {\itshape ká} `if' in that it can be preposed or postposed to the main clause. Further, it never occurs with attributive marker, if it is the head of a relative clause and speakers consistently translate {\itshape pílì} as {\itshape quand} `when' in French and seems to be restricted to own intonation phrase, other than obliques. Further, other bare nominal obliques have not been observed to systematically function as the head of relative clauses.

The temporal adverb {\itshape pílì} is the most frequently used temporal adverb  in the corpus, introducing a dependent clause. (In elicitation, also {\itshape pílɔ̀} was sometimes used.) Adverbial phrases with {\itshape pílì} can either precede or follow the main clause. In \REF{pili1}, it precedes the main clause.


\ea\label{pili1}
  \glll    [{\bfseries pílì} mɛ́ làwɔ́ mpù]\textsubscript{{\ADV}} mɛ̀ɛ́ válɛ́ làwɔ̀ \\ 
          {\db}pílì mɛ-H làwɔ-H mpù mɛ̀ɛ́ vá-lɛ́ làwɔ \\
           {\db}when 1\textsc{sg}-\textsc{prs} speak-{\R} like.this 1\textsc{sg}.\textsc{prs}.{\NEG} tolerate-{\NEG} speak  \\
    \trans `When I speak like this, I'm not lying [lit. I don't tolerate to talk].'
\z

Also in \REF{pili2}, the adverbial clause is preposed to the main clause. In this example, the dependent clause includes a non-verbal predicate with the verbal copula {\itshape múà} and a nominal locative predicate.


\ea\label{pili2}
  \glll  [{\bfseries pílì} yí múà ndáwɔ̀ nyà mànyɔ̀ ndɛ̀náà]\textsubscript{{\ADV}} á kí náà à múà njì bvúdà nà wɛ̂\\
      {\db}pílì yí múà ndáwɔ̀ nyà ma-nyɔ̀ ndɛ̀náà a-H kì-H nâ a múà njì bvúda nà wɛ̂ \\
         {\db}when 7 be $\emptyset$9.house 9:{\ATT} ma6-drink like.this 1-\textsc{prs} say-{\R} {\COMP} 1 {\PROSP} come quarrel {\COM} 2\textsc{sg}.{\OBJ}  \\
    \trans `When it is in a bar like this, he says that he is about to come quarrel with you.'
\z

\noindent Adverbial clauses with {\itshape pílì} can also be postposed, as shown, for instance, in \REF{pili3}.


\ea\label{pili3} 
  \glll báà bù mpàgó [{\bfseries pílì} pɔ̀dɛ̀ àà lã̀]\textsubscript{{\ADV}} \\
      báà bù mpàgó {\db}pílì pɔ̀dɛ̀ àà lã̀ \\
        3.{\FUT} break $\emptyset$3.road {\db}when $\emptyset$1.port 1.{\FUT} pass \\
    \trans `They will build a road when the port passes.'
\z

\REF{pili4} provides a more complex example of a postposed adverbial clause. Here, the adverbial clause follows the basic word order S V O, while the object is expressed by a complement clause.


\ea\label{pili4} 
  \glll  wɛ́ yànɛ́ ná gyàgà ndísì [{\bfseries pílì} wɛ́ lèmbó [nâ bùdì bá ndáwɔ̀ bvùbvù]\textsubscript{{\COMP}}]\textsubscript{{\ADV}} \\
     wɛ-H yànɛ-H ná gyàga ndísì {\db}pílì wɛ-H lèmbo-H nâ b-ùdì bá ndáwɔ̀ bvùbvù \\
        2\textsc{sg}-\textsc{prs} must-H again buy $\emptyset$3.rice {\db}when 2\textsc{sg}-\textsc{prs} know-{\R} {\COMP} ba2-person 2:{\ATT} $\emptyset$9.house many \\
    \trans `You must again buy rice, when you know that there are many people at home.'
\z




\subsubsubsection*{{\bfseries Concessive {\itshape tɔ̀} `even, although'}}
Another adverbial subordinator used to introduce dependent clauses is the concessive {\itshape tɔ̀} `even, although', which also appears in nominal modification, expressing `any', as described in \sectref{sec:InvQUANT1}. Again, adverbial clauses introduced by {\itshape tɔ̀} can both precede and follow the main clause, as shown in \REF{Conc1} and \REF{Conc2}, respectively.


\ea\label{Conc1}
  \glll [{\bfseries tɔ̀} wɛ̀ɛ́ kwálɛ́lɛ́ nyɛ̂]\textsubscript{{\ADV}} wɛ́ yànɛ́ nyɛ̂ bégyɛ́mɔ̀ \\
         {\db}tɔ̀ wɛ̀ɛ́ kwálɛ-lɛ nyɛ̂ wɛ-H yànɛ-H nyɛ̂ H-be-gyɛ́mɔ̀ \\
        {\db}even 2\textsc{sg}.\textsc{prs}.{\NEG} like-{\NEG} 1.{\OBJ} 2\textsc{sg}-\textsc{prs} must-{\R} see {\OBJ}.{\LINK}-be8-good.manner \\
    \trans `Even if you don't like him, you must still be polite [lit. see good manners].'
\z



\ea\label{Conc2}
  \glll à bwámɔ́ jî [{\bfseries tɔ̀} mpù á sàlɛ́ sílɛ́ sùkúlì]\textsubscript{{\ADV}} \\
         a bwámɔ-H jî {\db}tɔ̀ mpù á sàlɛ́ sílɛ-H sùkúlì \\
        1.{\PST} receive-{\PST}1 $\emptyset$7.position {\db}even like.this 1.{\PST}.{\NEG} {\NEG}.{\PST} finish-{\R} $\emptyset$7.school \\
    \trans `He got the job although he didn't finish school.'
\z



\subsubsubsection*{{\bfseries Causal {\itshape púù yá} `because'}}
{\itshape púù yá} marks the causal relation relation between the main clause and the dependent clause it introduces. Strictly speaking, it is not an adverb but a noun with an attributive marker, literally meaning `reason of'. The dependent clause that follows {\itshape púù yá} is then the second constituent of the nominal attributive construction. In contrast to other adverbial clauses, {\itshape púù yá} clauses have only been observed to follow main clauses, as illustrated in \REF{Causal1}.


\ea\label{Causal1}
  \glll yà nzí gyâ jìí [{\bfseries púù} {\bfseries yá} lévídó lè múà jî] \\
        ya nzî-H gyâ jìí {\db}púù yá le-vídó le múà jî \\
         1\textsc{pl}.{\PST} {\PROG}-{\R} sleep $\emptyset$7.forest {\db}$\emptyset$7.reason 7:{\ATT} le5-darkness 5.{\PST} {\PROSP} $\emptyset$7.forest \\
    \trans `We slept in the forest because it was about to get dark in the forest.'
\z

In the corpus, {\itshape púù yá} is not used to introduce subordinate clauses, but only in oblique phrases, as discussed in \sectref{sec:OBL}. Data for subordinate clauses stem from elicitation. In the corpus, the expression of causal relations between main and dependent clauses is subject to code-switching to Bulu, as shown in \REF{Causal2}.


\ea\label{Causal2}
  \glll   tè mɛ̀ɛ̀ jíbì kɛ̀ lwɔ̃̂ tè [{\bfseries àmú} vâ mɛ̀ɛ́ bɛ́lɛ́ nà sí ɛ́ vâ] \\
          tè mɛ̀ɛ̀ jíbì kɛ̀ lwɔ̃̂ tè {\db}àmú vâ mɛ̀ɛ́ bɛ́-lɛ́ nà sí ɛ́ vâ \\
           there 1\textsc{sg}.{\FUT} first go build there {\db}because[Bulu] here 1\textsc{sg}.\textsc{prs}.{\NEG} be-{\NEG} {\COM} $\emptyset$
9.ground {\LOC} here \\
    \trans `There, I will first go construct there because here I don't have any land.'
\z

\subsubsubsection*{{\bfseries Temporal relative clauses}}
Also the bare noun {\itshape yɔ̃́ɔ̃̀} `time' is used adverbially as a subordinator of adverbial clauses, as in \REF{TempREL1}.


\ea\label{TempREL1}
  \glll yíì mpà [{\bfseries yɔ̃́ɔ̃̀} wɛ́ kã́ yɔ̂ dúmbɔ́]\textsubscript{{\REL}} \\
       yíì mpà {\db}yɔ̃́ɔ̃̀ wɛ-H kã̂-H y-ɔ̂ dúmbɔ́ \\
         7.{\COP} good {\db}$\emptyset$7.time 2\textsc{sg}-\textsc{prs} wrap-{\R} 7-{\OBJ} $\emptyset$7.package\\
    \trans `It is good when you wrap it in a (leaf) package.'
\z












\subsubsection{Conditional clauses with {\itshape ká} `if'}
\label{sec:Cond}

The subordinator {\itshape ká} `if' introduces conditional clauses, comparable to {\itshape if}-clauses in English.
{\itshape ká} has been observed to also function as a temporal rather than a conditional marker, as shown in \REF{ka1}.


\ea\label{ka1}
  \glll [{\bfseries ká} á dígɛ́ nâ [gesture]] á nyɛ́ mbúmbù wɛ́ɛ̀ á pámò \\
        {\db}ká a-H dígɛ-H nâ [gesture] a-H nyɛ̂-H mbúmbù w-ɛ̂ a-H pámo \\
       {\db}when 1-\textsc{prs} look-{\R} {\COMP} [gesture]  1-\textsc{prs} see-{\R} $\emptyset$1.namesake 1-{\POSS}.3\textsc{sg} 1-\textsc{prs} arrive \\
    \trans `When he looks like [gesture], he sees his namesake who arrives.'
\z

\noindent The remainder of this section is, however, dedicated to {\itshape ká} as a conditional marker, which seems to be its primary function in terms of frequency.

In all instances in the corpus, the {\itshape ká}-clause is preposed to the main clause.  Examples of preposed conditional clauses are given in \REF{ka2} through \REF{ka4}. The sentences in \REF{ka2} and \REF{ka3} show that the basic word order in the dependent clause is maintained.


\ea\label{ka2}
  \glll [{\bfseries ká} wɛ́ wúmbɛ́ jímbɛ̀lɛ̀ lébímbú]\textsubscript{{\COND}} déè pɛ́mbɔ́ mwánɔ̀ sâ\\
         {\db}ká wɛ-H wúmbɛ-H jímbɛlɛ H-le-bímbú déè pɛ́mbɔ́ m-wánɔ̀ sâ\\
        {\db}if 2\textsc{sg}-\textsc{prs} want-{\R} lose {\OBJ}.{\LINK}-le5-weight eat.{\SBJV} $\emptyset$7.bread \textsc{n}1-child $\emptyset$7.thing\\
    \trans `If you want to lose weight, you should eat less bread.'
\z

\noindent The same is true for negated conditional clauses, as in \REF{ka3}.


\ea\label{ka3}
  \glll [{\bfseries ká} wɛ̀ɛ́ wúmbɛ́lɛ́ ndáà]\textsubscript{{\COND}} mɛ́ nɔ̀ɔ́ nkwɛ̂ wá mábɔ́'ɔ̀\\
        {\db}ká wɛ̀ɛ́ wúmbɛ-lɛ́ ndáà mɛ-H nɔ̀ɔ̀-H nkwɛ̂ wá H-ma-bɔ́'ɔ̀\\
         {\db}if 2\textsc{sg}.\textsc{prs}.{\NEG} want-{\NEG} also 1\textsc{sg}-\textsc{prs} take-{\R} $\emptyset$3.basket 3:{\ATT} {\OBJ}.{\LINK}-ma6-breadfruit\\
    \trans `If you don't want [this] either, I take the basket with the breadfruit.'
\z

Conditional clauses can, however, also take a special word order in terms of focus strategies, as it is the case in \REF{ka4}. In this example, the object pronoun is fronted and occurs between the modal auxiliary and the main verb so that the main verb is in focus position.


\ea\label{ka4}
  \glll [{\bfseries ká} kɛ̃́ɛ̃́sɔ́ yí wúmbɛ́ wɛ̂ dyɔ̀dɛ̀]\textsubscript{{\COND}} wɛ́ kílɔ̀wɔ̀\\
        {\db}ká kɛ̃́ɛ̃́sɔ́ yi-H wúmbɛ-H wɛ̂ dyɔ̀dɛ wɛ-H kílɔwɔ\\
         {\db}if $\emptyset$7.peer 7-\textsc{prs} want-{\R} 2\textsc{sg}.{\OBJ} deceive 2\textsc{sg}-\textsc{prs} be.vigilant\\
    \trans `If somebody wants to deceive you, you are vigilant'
\z

\noindent From elicitation, it is known that conditional {\itshape ká} clauses can also be postposed to the main clause, as shown in \REF{ka5}.


\ea\label{ka5}
  \glll mɛ̀ɛ̀ njì nàmɛ́nɔ́ [{\bfseries ká} Àdà á wúmbɛ́ nâ mɛ́ pándɛ́ɛ̀]\textsubscript{{\COND}}\\
       mɛ̀ɛ̀ njì nàmɛ́nɔ́ {\db}ká Àdà a-H wúmbɛ-H nâ mɛ-H pándɛ́ɛ̀\\
       1\textsc{sg}.{\FUT} come tomorrow {\db}if $\emptyset$1.{\PN} 1-\textsc{prs} want-{\R} {\COMP} 1-\textsc{prs} arrive.{\SBJV}\\
    \trans `I will come tomorrow if Ada wants me to come.'
\z



Conditional clauses can usually express different degrees of realis or irrealis, making a statement about the likelihood whether the event in the main clause will really happen. In English, this is achieved by the use of different tenses. In Gyeli also, different tense-mood categories can be used in conditional clauses, as shown in \REF{ka6} through \REF{ka9}. Generally, the same tense-mood category is used in the conditional clause as is also used in the main clause. Thus, in \REF{ka6}, the main clause appears in the \textsc{present} and so does the conditional clause. When the \textsc{present} tense-mood category is used, the conditional has a high realis degree, i.e.\ the event of the main clause is very likely to happen. In such instances, where the reading is generic, {\itshape ká} may also be replaced by {\itshape pílì} `when'.


\ea\label{ka6}
  \glll [{\bfseries ká} mɛ́ bwé nkwànò]\textsubscript{{\COND}} mɛ́ dè \\
       {\db}ká mɛ-H bwè-H nkwànò mɛ-H dè \\
       {\db}if 1\textsc{sg}-\textsc{prs} obtain-{\R} $\emptyset$3.honey 1\textsc{sg}-\textsc{prs} eat \\ 
 \trans `If I get honey, I eat [it].'
\z

In order to mark irrealis conditions, other tense-mood categories are used.
The most salient strategy to mark a conditional clause as irrealis, however, is the use of the irrealis marker {\itshape kɔ̀}. In \REF{ka7}, for instance, the main and conditional clause appear in the \textsc{future}, which is inherently an irrealis category (\sectref{sec:GramTM}). The speaker can then choose to use the irrealis marker {\itshape kɔ̀} in order to express that it is rather unlikely that he will find honey. If {\itshape kɔ̀} is not used, the speaker indicates that it is more likely to find honey in the future.


\ea\label{ka7}
  \glll [{\bfseries ká} mɛ̀ɛ̀ bwé nkwànò]\textsubscript{{\COND}} (kɔ̀) mɛ̀ɛ̀ dè \\
       {\db}ká mɛ̀ɛ̀ bwè-H nkwànò {\db}kɔ̀ mɛ̀ɛ̀ dè\\
       {\db}if 1\textsc{sg}.{\FUT} obtain-{\R} $\emptyset$3.honey {\db}{\IRR} 1\textsc{sg}.{\FUT} eat \\
 \trans `If I obtain honey, I will eat [it].'
\z

The same choice is given for conditionals in the \textsc{recent past}, as \REF{ka8} shows. Parentheses around {\itshape kɔ̀} indicate its optionality. Again, when the irrealis marker is used, it emphasizes the likelihood that the event of the main clause will not happen. In contrast to the \textsc{present} use in \REF{ka6}, the \textsc{recent past} seems to indicate a lower likelihood of finding honey.


\ea\label{ka8}
  \glll [{\bfseries ká} mɛ̀ bwé nkwànò]\textsubscript{{\COND}} (kɔ̀) mɛ̀ dé \\
      {\db}ká mɛ bwè-H nkwànò {\db}kɔ̀ mɛ dè-H \\
       {\db}if 1\textsc{sg}.{\PST}1 obtain-{\R} $\emptyset$3.honey {\db}{\IRR} 1\textsc{sg}.{\PST}1 eat-{\PST} \\
 \trans `If I obtained honey, I would eat [it].'
\z

The only circumstances where {\itshape kɔ̀} is systematically used is the clear irrealis context, which is further expressed by the \textsc{remote past}. This is shown in \REF{ka9}. Here, the speaker talks about an event that clearly did not happen.


\ea\label{ka9}
  \glll [{\bfseries ká} mɛ́ɛ̀ bwé nkwànò]\textsubscript{{\COND}} kɔ̀ mɛ́ɛ̀ dé \\
       {\db}ká mɛ́ɛ̀ bwè-H nkwànò kɔ̀ mɛ́ɛ̀ dè-H \\
       {\db}if 1\textsc{sg}.{\PST}2 obtain-{\R} $\emptyset$3.honey {\IRR} 1\textsc{sg}.{\PST}2 eat-{\PST} \\
 \trans `If I had obtained honey, I would have eaten [it].'
\z

In the corpus, conditional clauses only appear with \textsc{present} marking, while data on other tense-mood categories in conditional clauses stem from elicitation. 










\subsubsection{Adverbials + complementizer constructions}
\label{sec:AComp}

In contrast to true complement clauses (\sectref{sec:CompC}), dependent clauses that are introduced by an adverbial subordinator in combination with {\itshape nâ} behave more like other adverbial dependent clauses in two respects. First, they constitute an intonation phrase on their own and second, they can both precede and follow the main clause. Some of the adverbials used in combination with {\itshape nâ} are also used to introduce full adverbial clauses (\sectref{sec:ADVfull}), such as {\itshape líní} `when' vs.\ {\itshape lí nâ} `when'. The semantic differences seem subtle; speakers state that both forms can be used interchangeably.


There are two temporal adverbials in Gyeli which combine with the complementizer {\itshape nâ}, namely {\itshape lí} `when' and {\itshape sɔ́ɔ̀} `before'. This is most likely not an exhaustive list and other adverbializers might be possible in this construction type as well.

\REF{TEMP1} gives an example of a postposed adverbial + complementizer clause, using the adverbial {\itshape lí} `when'. Semantically, the sentence expresses simultaneity, the event of the main clause happening at the same time as the event of the dependent clause.


\ea\label{TEMP1}
  \glll mɛ̀ nzí nɔ̂ fɔ́tɔ̀ [{\bfseries lí} {\bfseries nâ} Àdà à nzí bɛ̀ à nzɛ́ɛ́ dè mántúà]  \\
        mɛ nzî-H nɔ̂ fɔ́tɔ̀ {\db}lí nâ Àdà a nzî-H bɛ̀ a nzɛ́ɛ́ dè H-ma-ntúà  \\
         1\textsc{sg}.{\PST} {\PROG}.{\PST}1 take $\emptyset$1.photo {\db}when {\COMP} $\emptyset$1.{\PN} 1.{\PST} {\PROG}.{\PST}1 be 1 {\PROG} eat ma6-mango \\
    \trans `I was taking photos, while Ada was eating mangoes.'
\z

In contrast, the dependent clause in \REF{TEMP2} precedes the main clause it modifies. In this example, the adverbial subordinator {\itshape sɔ́ɔ̀} `before' is used, expressing anteriority. Thus, the event of the main clause happens before the event of the subordinate clause.


\ea\label{TEMP2}
  \glll [{\bfseries sɔ́ɔ̀} {\bfseries nâ} á pámó tísɔ̀nì] á súmɛ́lɛ́ bùdì\\
        {\db}sɔ́ɔ̀ nâ a-H pámo-H tísɔ̀nì a-H súmɛlɛ-H b-ùdì\\
         {\db}before {\COMP} 1-\textsc{prs} arrive-{\R} $\emptyset$7.town 1-\textsc{prs} greet-{\R} ba2-person\\
    \trans `Before he arrives in town, he greets the people.'
\z


The subordinator {\itshape púù nâ} `reason that' expresses purpose in the dependent clause it introduces and is a variant of the noun plus attributive construction {\itshape púù yá}, which is discussed in \sectref{sec:ADVfull}.  An example is provided in \REF{puuna}.


\ea\label{puuna} 
  \glll  yá pándɛ́ nà síngìlìtì [{\bfseries púù} {\bfseries nâ} wɛ́ bwádɔ́ɔ̀ nyɛ̂ púù màbwálɛ́]\\
         ya-H pándɛ-H nà síngìlìtì {\db}púù nâ wɛ-H bwádɔ́ɔ̀ nyɛ̂ púù ma-bwálɛ́\\
         1\textsc{pl}-\textsc{prs} arrive-{\R} {\COM} $\emptyset$1.shirt {\db}$\emptyset$7.reason {\COMP} 2\textsc{sg}-\textsc{prs} wear.{\SBJV} 1.{\OBJ} $\emptyset$7.reason ma6-birth\\
    \trans `We bring the shirt so that you wear it for [your] birthday.'
\z

\noindent Semantically, there seems to be a difference in that {\itshape púù yá} has a causal reading in the sense of `because', whereas {\itshape púù nâ}  expresses purpose, translated as `so that'.





\subsubsection[Infinitival adverbial clauses]{Infinitival adverbial clauses without subordinator}
\label{sec:InfSub}

Gyeli has one type of adverbial clause that lacks a dedicated clause linker \citep[109]{hetterle2015}. Instead of an overt morphosyntactic subordinator, the subordination relation is expressed by an infinitival verb and the lack of any subject agreement and tense, aspect, mood marking. The subject is identified with a salient discourse antecedent which often coincides with the subject of the main clause, but not necessarily, as seen in \REF{INFpre3} and \REF{INFpre4}. The tense-mood interpretation is similar to that of past and present gerunds (except that there is neither dedicated gerund nor tense marking), encoding the wide range of temporal relations to the main clause of anteriority, simultaneity, and posteriority.
Infinitival clauses without subordinators are also marked prosodically as a clausal unit by a pause between the dependent and the main clause.


 Infinitival clauses can both be preposed and postposed to the main clause, as I show in the following. Infinitival clauses can further have the verb in their initial position or the infinitival verb can be preceded by another element such as the negation marker {\itshape tí} or a sequential marker.

\subsubsubsection*{{\bfseries Preposed infinitival clauses}}
Preposed infinitival clauses, as in \REF{INFpre1} through \REF{INFpre6}, often express temporal sequences, the event of the infinitival clause being anterior to the event of the main clause. Thus, in \REF{INFpre1}, the event of arriving in town is completed at the time of greeting people.\footnote{In my translation into English, I choose the gerund -{\itshape ing} form, since it allows the omission of the subject in the subordinate clause. I do not imply, however, that there are any other parallels between the English translation and the Gyeli structure. Speakers translate these constructions with a past participle form in French, for example for \REF{INFpre1} as {\itshape Arrivé en ville, il salue les gens}.}



\ea\label{INFpre1}
  \glll    [pámɔ̀ tísɔ̀nì]\textsubscript{SUB} á súmɛ́lɛ́ bùdì\\
            {\db}pámɔ tísɔ̀nì a-H súmɛlɛ-H b-ùdì \\
             {\db}arrive $\emptyset$7.town 1-\textsc{prs} greet-{\R} ba2-people\\
    \trans `Having arrived in town, he greets the people.'
\z

\REF{INFpre1} and \REF{INFpre2} are both instances where the implied subject of the infinitival clause is coreferential with the subject of the main clause. In \REF{INFpre1}, it is the same person who arrives in town and then greets the people. In \REF{INFpre2}, the person first eats mangoes and then, as a result, does not feel hungry anymore. The subject interpretation for the infinitival clause has to be, however, clear from the context. In the right context, it is also possible that the subject of the infinitival clause in \REF{INFpre1} is interpreted as non-coreferential to the one in the main clause, for instance when the speaker talks about his own arrival in town, but about a different person greeting the people (a similar case is presented below in \REF{INFpre4} where the implied agent of the subordinate clause and the subject of the main clause are not coreferential). In \REF{INFpre2}, the coreferential reading is reinforced due to the causality chain: because the person ate the mangoes, he is not hungry anymore.


\ea\label{INFpre2}
  \glll   [sílɛ dè mántúà]\textsubscript{{\SUB}} à tí ná dyúwɔ̀ nzà\\
           {\db}sílɛ dè H-ma-ntúà a tí ná dyúwɔ nzà\\
              {\db}finish eat {\OBJ}.{\LINK}-ma6-mango 1 {\NEG} anymore feel $\emptyset$9.hunger\\
    \trans `Having finished eating mangoes, he does not feel hunger anymore.'
\z


In other cases, it is not quite clear whether the subject of the main and the infinitival clause are coreferential. In \REF{INFpre3}, for instance, the narrator talks about a healer who has turned into an  antelope and has vanished into the forest, while the people of his village are following him with the intention of killing him. The infinitival clause in \REF{INFpre3} allows both interpretations of either the healer having arrived `here', i.e.\ in the forest, or the people of his village.


\ea\label{INFpre3}
  \glll   [nà pándɛ̀ vâ]\textsubscript{{\SUB}} bùdì báà bɛ̀ \\
          {\db}nà pándɛ̀ vâ b-ùdì báà bɛ \\
      {\db}{\COM} arrive here ba2-person 2.{\DEM}.{\PROX} be.there   \\
    \trans `And having arrived here, these people are there.'
\z



%express anteriority

%\begin{exe}
%\ex\label{ago}
%\ex\label{ago1}
 % \glll    [mìmbvú mílálɛ̀ lã̀]\textsubscript{INF} mɛ̀ bɛ́ mɛ̀ bɛ̃́ɛ̃̀ sùkúlì. \\
 %          mi-mbvú mí-lálɛ̀ lã̀ mɛ bɛ̀-H mɛ̀ bɛ̃́ɛ̃̀ sùkúlì \\
 %            mi4-year 4-three pass 1\textsc{sg}.{\PST}1 be-{\R} 1\textsc{sg} be.{\COMPL} $\emptyset$7.school   \\
 %   \trans `Three years ago I was in school.'
%\end{exe}

In other instances, the subject of the main clause and the implied subject of the infinitival clause are clearly different. \REF{INFpre4} is uttered by the same narrator in the same story. The context here is that the people of the village look for the healer in his hut and discover that he is not there. Thus, the infinitival clause has the people of the village as its implied subject, while the main clause's subject is {\itshape mùdì} `person'.


\ea\label{INFpre4}
  \glll  [kɛ̀ dígɛ̀ mpù]\textsubscript{{\SUB}} mùdì nú bɛ́lɛ́  \\
          {\db}kɛ̀ dígɛ mpù m-ùdì nú bɛ́-lɛ́ \\
          {\db}go look like.this \textsc{n}1-person 1.{\DEM}.{\DIST} be-{\NEG}\\
    \trans `Going looking like this, nobody is there.'
\z


While the main clause can have most of the tense-mood categories that are allowed in a main clause, excluding \textsc{imperatives}, past categories and the \textsc{future} as well as the \textsc{present} are most commonly found in the corpus. There are, however, also examples of the \textsc{inchoative} in the main clause, as shown in \REF{INFpre5}.


\ea\label{INFpre5}
  \glll  [ndɛ̀náà pámò lébũ̂]\textsubscript{{\SUB}} àá gyì\\
         {\db}ndɛ̀náà pámo H-le-bũ̂ àá gyì\\
        {\db}like.this arrive {\OBJ}.{\LINK}-le5-river.bank 1.{\INCH} cry \\
    \trans `Having arrived like this [without the child] at the river bank, she starts to cry.'
\z

\noindent While most preposed infinitival clauses seem to express temporal sequences, they may also express purpose, as in \REF{INFpre6}.


\ea\label{INFpre6}
  \glll  [dɔ̃̀ pɛ̀ tsíyɛ̀ pɔ́nɛ́ lèkɛ́lɛ̀]\textsubscript{{\SUB}} bvúlɛ̀ bá ntɛ́gɛ́lɛ́ bíì ɛ́ vâ \\
         {\db}dɔ̃̀ pɛ̀ tsíyɛ pɔ́nɛ́ le-kɛ́lɛ̀ bvúlɛ̀ ba-H ntɛ́gɛlɛ-H bíì ɛ́ vâ \\
         {\db}so[French] there cut $\emptyset$7.truth le5-word ba2.Bulu 2-\textsc{prs} bother-{\R} 1\textsc{pl}.{\OBJ} {\LOC} here   \\
    \trans `So, to say the truth, the Bulu bother us here.'
\z



\subsubsubsection*{{\bfseries Postposed infinitival clauses}}
Infinitival clauses can also follow the main clause, as shown in \REF{INFpost1} through \REF{INFpost5}. Postposed infinitival clauses seem to express purpose or manner rather than temporal sequences as with preposed clauses. In \REF{INFpost1} and \REF{INFpost2}, the infinitival clause modifies the main clause which is comprised of a non-verbal predicate. In both instances, the implied subject of the infinitival clause is coreferential with the subject of the main clause. Also, both express purpose, comparable to English {\itshape in order to} sentences.

%***
%\begin{exe} 
%\ex\label{OmN}
%  \glll àá gyì àá gyì [dyúmò njì nyɛ̂ nɔ̀ɔ̀] \\
%       àá gyì, àá gyì, dyúmò njì nyɛ̂ nɔ̀ɔ̀ \\
 %      1.{\INCH} cry 1.{\INCH} cry $\emptyset$1.spouse come 1.{\OBJ} take  \\
 %   \trans `She's at the beginning of crying, she's at the beginning of crying, the husband comes to fetch her.'
%\end{exe}


\ea\label{INFpost1}
  \glll     wɛ̀ nà ngvùlɛ̀ [kɛ̀ sɔ́lɛ̀gà wû]\textsubscript{{\SUB}} nà njí kù ɛ́ sì\\
            wɛ nà ngvùlɛ̀ {\db}kɛ̀ sɔ́lɛga wû nà njì-H kù ɛ́ sì\\
           2\textsc{sg} {\COM} $\emptyset$9.strength {\db}go fall there {\COM} come-{\R} fall[Kwasio] {\LOC} $\emptyset$9.ground\\
    \trans `You are strong [to go and climb a raffia palm tree], tumbling and falling to the ground. [The speaker talks about the strenuous work of climbing a tree to collect raffia leaves for roofs.]'
\z

\REF{INFpost2} also shows that infinitival clauses can be subject to non-basic word order. While in the basic word order, the object follows the verb, in \REF{INFpost2}, an object pronoun is fronted, as discussed in \sectref{sec:OBJfront} on information structure.\footnote{This example is also noteworthy because the fronted object pronoun usually occurs between the auxiliary verb {\itshape sílɛ} `finish' and the main verb {\itshape lwɔ̃̂} `build'. In this example, however, it occurs before the auxiliary.}


\ea\label{INFpost2}
  \glll bá nà ngvùlɛ̀ [bíyɛ̀ sílɛ̀ lwɔ̃̂ mándáwɔ̀]\textsubscript{{\SUB}}\\
        bá nà ngvùlɛ̀ {\db}bíyɛ̀ sílɛ lwɔ̃̂ H-ma-ndáwɔ̀ \\
        2 {\COM}  $\emptyset$9.strength {\db}1\textsc{pl}.{\OBJ} finish build {\OBJ}.{\LINK}-ma6-house \\
    \trans `They have the strength to build us all houses.'
\z

While preposed infinitival clauses directly precede the main clause, postposed infinitival clauses can constitute one of several subordinate clauses following the main clause. In these multiple subordinate constructions, the infinitival dependent clause usually modifies the clause it follows. In some cases, however, the zero-expressed subject referent can be ambiguous, as in \REF{INFpost3}. This example consists of a main clause, followed by an adverbial subordinate clause and an infinitival clause. The two subordinate clauses are juxtaposed. The subject of the infinitival clause could be coreferential with either the subject of the main clause or that of the infinitival clause.


\ea\label{INFpost3} S V O [{\ADV}] [{\INF}]\\
  \glll báà bù mpàgó [pílì pɔ́dɛ̀ àà vâ]\textsubscript{{\ADV}} [njì tsíyɛ̀ vâ]\textsubscript{{\SUB}}\\
      báà bù mpàgó {\db}pílì pɔ́dɛ̀ àà vâ {\db}njì tsíyɛ̀ vâ\\
         2.{\FUT} break $\emptyset$3.road {\db}when $\emptyset$1.port 1.{\COP} here {\db}come cut here\\
    \trans `They will build a road when the port is here, coming cross-cutting here.'
\z

\REF{INFpost4} is also comprised of a main clause, followed by two subordinate clauses, namely a complement and an infinitival clause. In this case, however, the infinitival clause picks its referent from the complement rather than the main clause.


\ea\label{INFpost4} S V [[{\COMP}] [{\INF}]]\\
  \glll bɔ́nɛ́gá bá lɔ́ sílɛ̀ làwɔ̀ [nâ bvúlɛ̀ bá ntɛ́gɛ́lɛ́ bágyɛ̀lì]\textsubscript{{\COMP}} [kɛ̀ nà kwàlɛ̀ bùdã̂ kɛ̀ nà kwàlɛ̀ bùdã̂ bá bá-gyɛ̀lì]\textsubscript{{\SUB}} \\
      bɔ́-nɛ́gá ba-H lɔ́ sílɛ làwɔ {\db}nâ bvúlɛ̀ ba-H ntɛ́gɛlɛ-H H-ba-gyɛ̀lì kɛ̀ nà kwàlɛ b-ùdã̂ {\db}kɛ̀ nà kwàlɛ b-ùdã̂ bá ba-gyɛ̀lì \\
        2-other 2-\textsc{prs} {\RETRO}  finish speak {\db}{\COMP} ba2.Bulu 2-\textsc{prs} bother-{\R} {\OBJ}.{\LINK}-ba2-Gyeli {\db}go {\COM} love ba2-woman go {\COM} love ba2-woman  2:{\ATT} ba2-Gyeli\\
    \trans `The others have just said that the Bulu bother the Bagyeli, coming and loving the women, coming and loving the women of the Bagyeli.'
\z

Finally, noun phrase constituents of an infinitival clause can also serve as the head of another embedded clause, as shown in \REF{INFpost5}. In this example, the main clause is followed by an infinitival clause, a relative clause and then another infinitival clause. The subject referent of the first infinitival clause is coreferential with the subject of the main clause. The object noun phrase of the first infinitival clause serves as subject head to the following relative clause. The second infinitival clause takes the subject of the relative clause as implied subject which, ultimately, is the object of the first infinitival clause.


\ea\label{INFpost5} S V X [[{\INF}1] [{\REL}] [{\INF}2]]\\
  \glll   yá sàgà mɛ́nɔ́ wɛ̂ [nyɛ̂ mápà má njìbù]\textsubscript{{\SUB}} [má bwámɔ́ ndáwɔ̀ dé tù]\textsubscript{{\REL}} [kɛ̀ dɛ́ndì]\textsubscript{{\INF}} \\
          ya-H sàga mɛ́nɔ́ wɛ̂ {\db}nyɛ̂ H-ma-pà má njìbù {\db}ma-H bwámɔ-H ndáwɔ̀ dé tù {\db}kɛ̀ d-ɛ́ndì \\
        1\textsc{pl}-\textsc{prs} be.surprised $\emptyset$7.morning in {\db}see {\OBJ}.{\LINK}-ma6-paw 6:{\ATT} $\emptyset$1.antelope {\db}6-\textsc{prs} come.out-{\R} $\emptyset$9.house {\LOC} inside {\db}go le5-courtyard \\
    \trans `We are surprised in the morning to see paws of an antelope which come out of the house, going into the courtyard.'
\z




The non-finite verb in infinitival subordinate clauses can be preceded by either a negation marker {\itshape tí} or sentential modifiers, as I show in the following.

\subsubsubsection*{{\bfseries Infinitival subordinate clauses with {\itshape tí} negation}}
The negation marker {\itshape tí} can precede the non-finite verb of an infinitival subordinate clause, as in \REF{InfAdv1} and \REF{InfAdv2}.



\ea\label{InfAdv1} 
  \glll  à múà nà bábɛ̀ [{\bfseries tí} wúmbɛ̀ wɛ̀]\textsubscript{{\SUB}} \\
          a múà nà bábɛ̀ {\db}tí wúmbɛ wɛ̀   \\
         1 be {\COM} $\emptyset$7.illness {\db}{\NEG} want-{\R} die \\
    \trans `He was sick, not wanting to die.'
\z

\noindent The main clause in \REF{InfAdv1} is comprised of a verbal copula construction and modified by the infinitival subordinate clause. Semantically, the events of the main and the subordinate clause happen simultaneously: the person is sick and, at the same time, does not want to die.

%As with asyndetic infinitival clauses, the subject of the dependent clause is not explicit, but a matter of interpretation whether the subject of the dependent clause is coreferential with the subject of the main clause or not. While, in \REF{InfAdv1}, the implied subject of the dependent clause is coreferential with the one of the main clause, this is not the case in \REF{InfAdv2}. Here, the subject of the main clause is the healer who roams the forest in the shape of an  antelope while the implied subject of the dependent clause is the people of the village, while the healer is the object referent of the dependent clause (`without seeing him [the healer])'.


\ea\label{InfAdv2}
  \glll    nà kɛ́ jìí dé tù nà ndzǐ pámò dẽ̂ [{\bfseries tí} nyɛ̂ nyɛ̂]\textsubscript{{\SUB}} \\
          nà kɛ-H jìí dé tù nà ndzǐ pámò dẽ̂ {\db}tí nyɛ̂ nyɛ̂ \\
         {\COM} kɛ̀-{\R} $\emptyset$7.forest {\LOC} inside {\COM} $\emptyset$9.path arrive today {\db}{\NEG} see 1.{\OBJ} \\
    \trans `And (he) goes in the forest on the path till today, without seeing him.'
\z


\subsubsubsection*{{\bfseries Sequential marker {\itshape vɛ̀ɛ̀}}}
{\itshape vɛ̀ɛ̀} and {\itshape kɔ́ɔ̀} are both used as sentential modifiers, as described in \sectref{sec:SentMod}. They can also appear in an infinitival subordinate clause where they directly precede the verb, as in \REF{InfAdv3}.



\ea\label{InfAdv3} 
  \glll  à nɔ̀ɔ́ brìkɛ̂ [{\bfseries vɛ̀ɛ̀} bɛ́dɛ̀ ndáwɔ̀]\textsubscript{{\SUB}} \\
         a nɔ̀ɔ̀-H brìkɛ̂ {\db}vɛ̀ɛ̀ bɛ́dɛ ndáwɔ̀ \\
           1.{\PST}1 take-{\R} $\emptyset$1.lighter[French] {\db}{\SEQU} light $\emptyset$9.house \\
    \trans `He took the lighter, just lighting the house.'
\z

\noindent The sentential modifier in \REF{InfAdv3} can be omitted without making the sentence ungrammatical. It changes, however, the meaning of the sentence. Without it, the infinitival dependent clause would express purpose `He took the lighter in order to light the house.' The intended meaning with the  sentential modifier is sequential: the person first takes the lighter and then sets the house on fire.

A special case is presented in \REF{InfAdv4} where the infinitival clause has an overt subject. The verb {\itshape kwè} `fall'  still appears in its infinitival form, lacking the realis-marking H tone. Since infinitival dependent clauses are very rare in the corpus, it is not possible at this point to establish what conditions the overt marking of subjects in this clause type.


\ea\label{InfAdv4} 
  \glll  má dvúmɔ́lɛ́ mbvú mbì mbvû [màlɛ́ndí máà {\bfseries vɛ̀ɛ̀} kwè mípìndí]\textsubscript{{\SUB}} \\
        ma-H dvúmɔ́-lɛ́ mbvú mbì mbvû {\db}ma-lɛ́ndí máà vɛ̀ɛ̀ kwè H-mi-pìndí \\
           6-\textsc{prs} produce-{\NEG}  $\emptyset$3.year like[Kwasio] $\emptyset$3.year {\db}ma6-palm.tree 6.{\DEM}.{\PROX} only fall {\OBJ}.{\LINK}-mi4-unripeness\\
    \trans `They don't produce [fruit] every year, these palm trees from which only unripe [fruit] fall.'
\z

\subsubsubsection*{{\bfseries Sequential marker {\itshape kɔ́ɔ̀}}}
The sequential marker {\itshape kɔ́ɔ̀} seems to have exactly the same function as {\itshape vɛ̀ɛ̀} when introducing a dependent clause. While both sentential modifiers are compared in \sectref{sec:SentMod}, their potential distributional and semantic differences are even less clear as clause-introducing devices. It rather seems that they are freely interchangeable in this function. An example of {\itshape kɔ́ɔ̀} introducing an  infinitival subordinate clause is given in \REF{InfAdv5}.



\ea\label{InfAdv5} 
  \glll  à jí mbɛ̂ [{\bfseries kɔ́ɔ̀} gyíbɔ̀ bwánɔ̀]\textsubscript{{\SUB}} \\
         a jì-H mbɛ̂ {\db}kɔ́ɔ̀ gyíbɔ bwánɔ̀ \\
           1.{\PST}1 open-{\R} $\emptyset$3.door {\db}{\SEQU} call ba2-child \\
    \trans `She opened the door, just calling the children.'
\z

As with {\itshape vɛ̀ɛ̀}, omitting the sentential modifier in \REF{InfAdv5} gives a purpose reading of `She opens the door in order to call the children.' In contrast, {\itshape kɔ́ɔ̀} gives a sequential interpretation.


\subsubsection{Subordination with progressive marker {\itshape nzɛ́ɛ́}}
\label{sec:SUBnzee}

Subordination can also be encoded by the subordinate form of the progressive marker, {\itshape nzɛ́ɛ́}, which, in main clauses, takes different forms (\sectref{sec:PROG}). In \REF{frame1}, the subordinate clause expresses simultaneity. Without the subordinate form of the aspect marker, the second clause would formally be identical to a main clause and could appear on its own.


\ea\label{frame1}
  \glll  á gyímbɔ̀ [à {\bfseries nzɛ́ɛ́} sâ mákwásì] \\
        a-H gyímbɔ {\db}a nzɛ́ɛ́ sâ H-ma-kwásì \\
            1-\textsc{prs} dance {\db}1 {\PROG}.{\SUB} do {\OBJ}.{\LINK}-ma6-clapping \\
    \trans `He dances while clapping.'
\z
















