\part{Introduction}
\chapter{Introduction}

This essay provides an extended introduction to the scripture translations of \ia{Biraban}Biraban and \ia{Threlkeld, L.E.}Lancelot Threlkeld, which began around 1825 and continued until Threlkeld's death in 1859. It analyses Threlkeld’s linguistic field work in Raiatea prior to coming to New South Wales. It places the translations he undertook in the context of \is{Linguistics!Missionary linguistics}Australian missionary linguistics and the rapid advance of the settler frontier, for which he was a key eyewitness. It analyses the motivation and collaboration between \ia{Biraban}Biraban and \ia{Threlkeld, L.E.}Threlkeld in the light of discoveries of new manuscripts, including that of the Gospel of St Matthew, as well as Threlkeld’s personal diary, neither of which have previously been analysed. The review includes a linguistic and ethnographic analysis of the complete corpus of \ia{Biraban}Biraban and \ia{Threlkeld, L.E.}Threlkeld’s collaboration. It includes a complete list of the Threlkeld manuscripts and the many printed editions, including those available online. For historical purpose, this volume includes high definition scans of the unique edition of The Gospel by St Luke \citep{fraser_gospel_1891}, presented by \ia{Fraser, John}Fraser to the British and Foreign Bible Society, together with GMS 83, \ia{Threlkeld, L.E.}Threlkeld’s original manuscript, illuminated by Annie Layard, and gifted to Auckland library by Sir George Grey.

\section{The Threlkeld legacy}

\ia{Threlkeld, L.E.}Lancelot Threlkeld (1788--1859) was responsible for the creation of a remarkable corpus of linguistic material during the years he worked as a missionary in and around the modern Australian cities of Newcastle and Lake Macquarie in the Hunter River region of New South Wales. The sources for his life have now been extensively investigated by researchers interested not only in his work analysing and recording \il{Hunter River Lake Macquarie language (HRLM)}Hunter River Lake Macquarie language (HRLM), but also in his more conventional missionary commitment to the conversion, civilisation and uplift of the Aboriginal people and his \is{Humanitarianism}humanitarian intervention on their behalf during \is{Black War}the “Black War” which afflicted the colonial settlement in the 1830s. \ia{Threlkeld, L.E.}Threlkeld was never a conventional member of colonial society: his commitment to the Aboriginal people, religious ardour, unconventional lifestyle, and curious obsessions -- not just with linguistics but with postal reform, \is{Anti-Catholicism}anti-Catholicism, and \is{Colonisation!French colonisation}French colonisation of the Pacific -- set him apart.

There have now been a series of studies of \ia{Threlkeld, L.E.}Threlkeld’s biography and role as a missionary and linguistic reformer. His grammar and scripture translations -- the first into any Aboriginal language -- achieved iconic status very early. A fine edition of his grammar was created for the New South Wales display at the Great Exhibition in London’s Crystal Palace in 1851. Over forty years later, the Maitland schoolmaster \ia{Fraser, John}John Fraser was responsible for publishing or re-publishing much of \ia{Threlkeld, L.E.}Threlkeld’s linguistic output for exhibition at the World’s Columbian Exhibition, also known as the World Fair, held in Chicago in 1893. In recent times, there is a major biographical study by \ia{Gunson, Niel}Niel Gunson, who also edited for publication the majority of \ia{Threlkeld, L.E.}Threlkeld’s correspondence, reports to the \is{London Missionary Society}London Missionary Society and the New South Wales government, and his published observations on traditional Aboriginal beliefs, but not his linguistic output. Subsequently, \citet[11--13]{reynolds_whispering_1998} has described his \is{Humanitarianism}humanitarian efforts; \ia{Johnston, Anna}Anna Johnston has analysed the characteristics of his missionary writing; \ia{Roberts, David A.}Roberts has described his linguistic mission, and \ia{Carey, Hilary M.}Carey has investigated the course of his partnership with \ia{Biraban}Biraban and the creation of what she refers to as the “Colonial Bible” -- an artefact which was representative of the colonial and post-colonial forces which fostered the creation of \ia{Biraban}Biraban and \ia{Threlkeld, L.E.}Threlkeld’s linguistic corpus.

There is a large and problematic literature relating to \ia{Threlkeld, L.E.}Threlkeld’s legacy, with opinions varying from the outright hostile views of colleagues and contemporaries to the almost equally exasperated views of his most recent biographer, Anna \citet[2]{johnston_paper_2011}, who declares him to be “opinionated, self-regarding, litigious, and pious”. In recent scholarship, he has attracted most interest from those who see him as part of the imperial networks generated by settler colonialism, a one-man generator of paper and controversy forever at war with those unfortunate enough to be burdened with his line management \citep{lambert_colonial_2006}. According to Lester (\cite{lester_colonization_2014}; \cite{lester_british_2002,lester_colonial_2002,lester_obtaining_2002}), even his ostensible advocacy for Aboriginal people has been disputed as mendacious virtue signalling, another link in the chain of \is{Humanitarianism}anti-humanitarian settler discourse which provided a threadbare cover for genocide. \citet{mitchell_good_2011} and \citet{curthoys_taking_2018} provide important discussions of \ia{Threlkeld, L.E.}Threlkeld’s \is{Humanitarianism}humanitarian mission and work as a court interpreter in the context of the eruption of settler violence in colonial New South Wales in the 1840s.

While not disputing \ia{Threlkeld, L.E.}Threlkeld’s undoubted capacity to annoy, these critical interpretations have been made without a full assessment of \ia{Threlkeld, L.E.}Threlkeld’s most significant legacy, namely his translation of two of the four gospels, \pagebreak{}St Luke and St Mark,\footnote{For a list of \ia{Threlkeld, L.E.}Threlkeld’s publications, see \citet[374--376]{gunson_australian_1974b}. One manuscript copy of the Gospel of St Luke is now in State Library of New South Wales (SLNSW) MS A1325; for the copy illuminated by Annie Layard for Sir George Grey, see Auckland Libraries, GMS 83 \citep{carey_secret_2011} and the present digital edition (\hyperref[sec:Section_3]{Section 3}).} and his incomplete translation of the Gospel of St Matthew (1837) as well as number of minor works.\footnote{“Incomplete MS by \ia{Threlkeld, L.E.}Rev. L.E. Threlkeld of St Matthew’s Gospel in the \il{Hunter River Lake Macquarie language (HRLM)}\il{Hunter River Lake Macquarie language (HRLM)!Awabakal}Awabakal language 1837”, SLNSW MS DCL238193. This manuscript was not known to \ia{Wafer, James}Wafer and \ia{Carey, Hilary M.}Carey in their earlier study \citep[124, n. 16]{wafer_waiting_2011}.} Jeremy \citet{steele_awabakal_2024} has recently placed a morpheme by morpheme analysis of \ia{Biraban}Biraban and \ia{Threlkeld, L.E.}Threlkeld’s scripture translations online. This enables further linguistic study of their contribution to \is{Linguistics!Missionary linguistics}missionary linguistics in colonial Australia.

\section{Language in colonial Australia}

Australia was the last continent to be \is{Colonisation!European colonisation}colonised from Europe, though its conquest was in many respects completed more rapidly and with more attendant \is{Cultural destruction}cultural destruction than those of Asia, Africa or the Americas. Nevertheless, the collection of word lists and examples of the ways of speech of Aboriginal peoples was a feature of early European voyages from the time of William Dampier in 1688 \citep[2]{mcgregor_missionary_2008} and the first settlement in 1788 \citep{troy_sydney_1992}. Recognition of the importance of these remnants has been relatively slow. In \citeyear{dixon_languages_1980}, \citeauthor{dixon_languages_1980} asserted that \is{Linguistics!Aboriginal linguistics}Aboriginal linguistics did not exist prior to the 1940s -- and that all previous efforts by colonial amateurs and especially missionaries had little if any value \citep{dixon_languages_1980}. The revival of interest in \is{Linguistics!Missionary linguistics}missionary linguistics has been encouraged by the work of \is{London Missionary Society}the Society for the History of Linguistics in the Pacific, which first met in 2008 in association with the Australian Historical Association and the Australian Linguistics Society.\footnote{The seventh conference was postponed due to the coronavirus epidemic. Continuing research is supported by the Research Unit for Indigenous Languages at the University of Melbourne and the ARC Centre of Excellence for the Dynamics of Language, \url{https://www.dynamicsoflanguage.edu.au/news-and-media/media-releases/article/?id=shlp6-conference-adelaide-13-14-dec-2018} (Accessed 31 July 2024).}
\citet{mcgregor_notitle_2008} has edited the first book-length account of the history of \is{Linguistics!Aboriginal linguistics}Aboriginal linguistics, which includes appropriate attention to \is{Linguistics!Colonial linguistics}colonial and \is{Linguistics!Missionary linguistics}missionary linguistics as well as academic practice. \citet[9]{mcgregor_missionary_2008} identifies three phases in the history of research on Australian languages: a pre-phonemic (or pre-scientific phase) from 1788 to 1929, dominated by word lists from settlers and travellers, with a few more extended works by \is{Linguistics!Missionary linguistics}missionary linguists; an intermediate phase from 1930 to 1959, with the first trained investigators including \ia{Elkin, A.P.}A.P. Elkin, \ia{Laves, Gerhardt}Gerhardt Laves, and \ia{Capell, Arthur}Arthur Capell; and the modern phase, from the 1960s to the present, dominated by academic linguists. \ia{Threlkeld, L.E.}Threlkeld can justly be seen as the major linguist of the first, pre-scientific phase of the study of Aboriginal languages.

Understanding of the \il{Hunter River Lake Macquarie language (HRLM)}Hunter River Lake Macquarie language, and of \ia{Threlkeld, L.E.}Threlkeld and his world, has been enhanced by the gradual emergence of \ia{Threlkeld, L.E.}Threlkeld’s papers and their deposit in the State Library of New South Wales in a series of bequests, as well as by the digitisation of manuscripts in Auckland, London, Sydney and Canberra.\footnote{See \hyperref[Appendix_A]{Appendix A} for a list of \ia{Threlkeld, L.E.}Threlkeld’s published works and manuscripts and links to digitised versions, where these are now available.} Fascination with the original \il{Hunter River Lake Macquarie language (HRLM)}language of the Hunter and Lake Macquarie peoples was sustained by antiquarians such as the late Percy Haslam, while the foundation for scholarly investigation started in the 1960s with Arthur \citeauthor{capell_aboriginal_1970} (\citeyear{capell_aboriginal_1970, capell_new_1966, capell_studies_1966, capell_linguistic_1963}) and continues under sponsorship of Muurrbay, who commissioned \citeauthor{lissarrague_salvage_2006}’s \textit{Salvage grammar} (\citeyear{lissarrague_salvage_2006}) as well as Wafer and Lissarrague’s guide to Hunter-Hastings languages (\citeyear{wafer_handbook_2008}).

\section{Previous studies}

\ia{Threlkeld, L.E.}Threlkeld had a remarkable life, which has now been the subject of a series of biographies by John \citet[xii--xv]{fraser_australian_1892}, Ben W. \citet{champion_lancelot_1939}, Niel \citeauthor{gunson_australian_1974a} (\citeyear{gunson_australian_1974a,gunson_australian_1974b, gunson_threlkeld_1967}), Anna \citet{johnston_paper_2011, johnston_blister_2006} and local historian Peter \citet{murray_mission_2018}. Despite its curious agenda, which aimed to demonstrate the biblical antecedents of the Australian Aboriginal people \citep{carey_babylon_2020}, \ia{Fraser, John}John Fraser’s edition of \ia{Threlkeld, L.E.}Threlkeld’s unpublished \il{Hunter River Lake Macquarie language (HRLM)}HRLM Gospel of St Luke and other texts paid tribute to \ia{Threlkeld, L.E.}Threlkeld and \ia{Biraban}Biraban’s linguistic achievement. As \citet{gunson_british_1994} argued, \ia{Fraser, John}Fraser and his circle were key figures in early \is{Colonial ethnography}colonial ethnography of the Pacific, though hampered by old-fashioned biblical preoccupations. \citeauthor{champion_lancelot_1939}’s historical biography (\citeyear[280]{champion_lancelot_1939}) had the advantage of drawing on private papers then in the hands of the Arndell family, “who willingly made both this and other valuable family records available for research purposes”. Some of these papers, including the genealogical papers collected by James Threlkeld in 1748 \citep{threlkeld_family_genius_1767} and \ia{Threlkeld, L.E.}Threlkeld’s unpublished translations of the Gospel of Mark (\citeyear{threlkeld_gospel_1837}) and his unfinished Matthew (\citeyear{threlkeld_gospel_1837-1}), are now in the State Library of New South Wales. Others, such as different volumes of \ia{Threlkeld, L.E.}Threlkeld’s Private Journal, were consulted by \ia{Champion, Ben W.}Champion at the Mission to Seamen, but are no longer in the public domain. There is a digitised version of a single volume of \citeauthor{threlkeld_journal_1828-1846}’s Journal for \citeyear{threlkeld_journal_1828-1846}, though the original has disappeared; more papers were held by Camden College, the theological college established in 1864 for the training of clergy for the \is{Congregationalism}Congregational Union in Sydney \citep{gunson_threlkeld_1967}.

\ia{Champion, Ben W.}Champion provides a genealogical account of \ia{Threlkeld, L.E.}Threlkeld’s life and work in New South Wales. His verdicts are informed by an intimate knowledge of the locations of the mission and of those with memories of the sites. However, he provides little in the way of footnotes and makes no reference to \ia{Threlkeld, L.E.}Threlkeld’s missionary work in the \is{South Seas}South Seas. He also sides with the \ia{Marsden, Samuel}Rev. Samuel Marsden in his judgement on \ia{Threlkeld, L.E.}Threlkeld’s character, his theatricality, self-justification, and rapid changes of mood. \citet[327]{champion_lancelot_1939} criticised \ia{Threlkeld, L.E.}Threlkeld for his heavy expenditure on the construction of a mission house and an access road as well as his reluctance to leave Newcastle, where he lingered for sixteen months (actually until finally moving his family to his new residence of “Bahtahbah” in September 1826. Even his linguistic work, perhaps the single most significant achievement of his life, is seen in a negative light: “From the moment when \ia{Threlkeld, L.E.}Threlkeld received his ‘Instructions’ from the deputation, his sole object, his ruling passion, was to master the aboriginal language. All other matters were to be made subservient to this great task” \citep[317]{champion_lancelot_1939}. In the overall critique of \ia{Threlkeld, L.E.}Threlkeld, \ia{Champion, Ben W.}Champion was influenced by the missionary’s remarkably injudicious account of his mission to Raiatea and the conduct of the \is{London Missionary Society}London Missionary Society \citep{threlkeld_statement_1828}. Intended, according to \ia{Threlkeld, L.E.}Threlkeld, only to circulate privately, this heavy-handed and one-sided interpretation of his disputes with the \is{London Missionary Society}London Missionary Society was annotated by the two members of the \is{London Missionary Society}LMS delegation, the \ia{Tyerman, Daniel}Rev. Daniel Tyerman (d. 1828) and the Sheffield businessman \ia{Bennet, George}George Bennet (d. 1841). Having already been forced to deal with complex hostilities between \ia{Threlkeld, L.E.}Threlkeld and the older missionaries in the \is{South Seas}South Seas, \ia{Bennet, George}Bennet was astonished at the continuing rancour revealed by \ia{Threlkeld, L.E.}Threlkeld’s account. While not immune to \ia{Threlkeld, L.E.}Threlkeld’s linguistic gifts, \ia{Champion, Ben W.}Champion took the view of the establishment -- both the LMS delegation and Directors and their colonial agent the \ia{Marsden, Samuel}Rev. Samuel Marsden -- and endorsed \ia{Threlkeld, L.E.}Threlkeld as a perennially dissatisfied troublemaker.

Niel \citeauthor{gunson_australian_1974a}’s two volume biography (\citeyear{gunson_australian_1974a,gunson_australian_1974b}), includes most of the family history narrated by \ia{Champion, Ben W.}Champion, and new editions of \ia{Threlkeld, L.E.}Threlkeld’s correspondence with the \is{London Missionary Society}LMS, ethnographic accounts of the Aboriginal people, his lengthy reports on the progress of the first and second Lake Macquarie Missions and accounts by the Quaker travellers, \ia{Backhouse, James}James Backhouse and George Washington Walker, who visited in April 1836. \ia{Gunson, Niel}Gunson provides a much richer and more balanced account of \ia{Threlkeld, L.E.}Threlkeld’s missionary and intellectual achievements, placing it within the theological context of his \is{Calvinism}Calvinistic Dissent and teasing apart the complexities of his battles with religious and secular authority. It is \ia{Gunson, Niel}Gunson who recognised the significance of \ia{Threlkeld, L.E.}Threlkeld’s ethnographic and linguistic work, which remains the most detailed and insightful account of any of the original people of the southern east coast of Australia. He also provides a full account of his work in the \is{South Seas}South Seas, which was of longer duration and arguably had greater impact than his mission to the Aboriginal people of Lake Macquarie. While increasingly difficult to obtain, it remains the standard account. Anna \citet{johnston_paper_2011} examined the literary context for \ia{Threlkeld, L.E.}Threlkeld’s disputes with authority, showing particular insight into the tangled network of correspondence between colony and metropole. Local historian Peter \citet{murray_mission_2018} has made excellent use of previous published sources and biographies to provide a dense local narrative informed by knowledge of the Lake Macquarie and Newcastle area.

Historians have also been equally intrigued by his relationship with his main informant, \ia{Biraban}Biraban or \ia{M’Gill [McGill], John [Johnny]| see {Biraban}}John M’Gill. \citet{carey_lancelot_2004} placed \ia{Threlkeld, L.E.}Threlkeld and \ia{Biraban}Biraban’s partnership in the wider context of the practice of \is{Linguistics!Missionary linguistics}missionary linguistics, the creation of a colonial Bible in Australia \citep{carey_lancelot_2010} and the importance of linguistic texts about the passing of the frontier \citep{carey_death_2009}. Others have been chiefly impressed that the feat of scripture translation into Aboriginal languages could occur at all. Hence Roland \citet[151--152]{boer_last_2008} refers to \ia{Biraban}Biraban as “a native assistant” who supported \ia{Threlkeld, L.E.}Threlkeld’s “bravura act of grammatical analysis and translation”, though no more remarkable than the feats of other \is{Linguistics!Missionary linguistics}missionary linguists nearby in the Pacific and in more distant mission fields from the Arctic to Africa; others are at pains to stress \ia{Biraban}Biraban’s subject status. For \citet[46]{van_toorn_writing_2006}, \ia{Biraban}Biraban’s authorial message was “refracted and translated”, though occasionally visible in his reported dreams and conversations. Other readings of the linguistic partnership between \ia{Biraban}Biraban and \ia{Threlkeld, L.E.}Threlkeld have been provided by \citet{roberts_language_2008} and \citet{keary_christianity_2009}.

Further understanding of the relationship between \ia{Biraban}Biraban and \ia{Threlkeld, L.E.}Threlkeld will be enhanced by a a full reading and interpretation of their most extensive and enduring legacy, namely the corpus of translations into \il{Hunter River Lake Macquarie language (HRLM)}HRLM for which \ia{Biraban}Biraban was the principal and, in some cases, the only informant. The most important breakthrough for the \is{Language revitalisation}revival of the \il{Hunter River Lake Macquarie language (HRLM)}HRLM language was Amanda \citeauthor{lissarrague_salvage_2006}’s \textit{Salvage grammar}, published in \citeyear{lissarrague_salvage_2006} with the support of the Wonnarua Nation Aboriginal Corporation and the NSW Department of Aboriginal Affairs. This provided a more or less stable orthography, an appropriate naming for the \il{Hunter River Lake Macquarie language (HRLM)}Hunter River Lake Macquarie language (HRLM) in place \il{Hunter River Lake Macquarie language (HRLM)!Awabakal}of “Awabakal”, which lacks historical or linguistic currency, and a close study of the grammatical features and lexical data that can be extracted from \ia{Threlkeld, L.E.}Threlkeld’s published work.

But the greater part of \ia{Threlkeld, L.E.}Threlkeld’s linguistic legacy is contained in the manuscripts of his translations, which have never been subjected to the same kind of analysis. Indeed, \citeauthor{lissarrague_salvage_2006} concludes her \textit{Salvage Grammar} (\citeyear[107]{lissarrague_salvage_2006}) by observing that this work did not include any of the “long texts” created by \ia{Threlkeld, L.E.}Threlkeld and \ia{Biraban}Biraban, including the Gospel of St Luke (1831) and the Gospel of St Mark (1837), and that this was not accidental. She suggests that because of the nature of the texts, further analysis was unlikely to reveal more about the worldviews of the people of the Hunter River and Lake Macquarie region. This is a pessimistic evaluation of this material. Alternatively, as \citet{rademaker_why_2016} suggests, the way forward lies in closer and more effective collaborations between professional linguists and historians for the purpose of providing high quality editions of the full corpus of translations produced by \ia{Threlkeld, L.E.}Threlkeld, \ia{Biraban}Biraban and other, nameless informants. In an earlier study, Wafer and \ia{Carey, Hilary M.}Carey (\citeyear{wafer_waiting_2011}) suggested these lengthy texts provide a unique record of a language which is no longer spoken as a first language, as well as of the transformation and \is{Transculturation}transculturation of its speakers.\footnote{I thank James Wafer for providing guidance on the linguistic discussion in this essay, as well as the transliteration of \il{Hunter River Lake Macquarie language (HRLM)}HRLM words into a modern orthography.}

The account which follows focusses on \ia{Biraban}Biraban and \ia{Threlkeld, L.E.}Threlkeld’s biography and their linguistic partnership. For the dispute with the \is{London Missionary Society}LMS and colonial authorities, it is necessary to refer to the historians already mentioned.

\section{Biraban}

To a considerable degree, current understanding of the \il{Hunter River Lake Macquarie language (HRLM)}HRLM language is the product of the working partnership between \ia{Biraban}Biraban (c. 1800--1845), or Johnny M’Gill, known before 1828 as We-pohng \citep[31, n. 39]{gunson_australian_1974a}, and the missionary \ia{Threlkeld, L.E.}Lancelot Threlkeld. \ia{Biraban}Biraban was brought up in the soldiers’ barracks in Sydney and had learnt excellent \il{English}English before meeting and forming his close association with the missionary. It was under the name of “Barabahn, or McGil, Chief of the Tribe at Bartabah, on Lake Macquarie” that \ia{Biraban}McGill received an engraved breastplate, “a Reward for his assistance in reducing the Native Tongue to a written Language” (\textit{\citefield{noauthor_notitle_1803}{shorttitle}}, 12 January 1830: 2). While it is doubtful that he was ever recognised as a “king” or “chief” by his own people, he was crucial to knowledge exchange between Aboriginal people in the Hunter River and Lake Macquarie region and the first generation of European settlers.

\ia{Biraban}Biraban (who was not named) may have met \ia{Threlkeld, L.E.}Threlkeld in Newcastle during the latter’s first visit to reconnoitre the Lake Macquarie site of the mission in 1825. \ia{Threlkeld, L.E.}Threlkeld’s \is{London Missionary Society}LMS Journal notes that, on Wednesday 11 April 1825, about 40 natives assembled around his Newcastle house and, after cooking kangaroo, they performed a dance, “which was on account of our arrival among them” \citep[fol. 4]{threlkeld_journal_1824}. \ia{Biraban}Biraban may have been the “trusty native who speaks good \il{English}English” who went with \ia{Threlkeld, L.E.}Threlkeld to Lake Macquarie on 21 April. By 9 June, \ia{Threlkeld, L.E.}Threlkeld seems to have cemented the relationship and travelled back to the Lake, where “Mac’gill” is named along with “Dismal” for “felling trees to make room for the erection of our house and prepare for planting some Indian corn. The natives appear anxious for our settling out there” \citep[fol. 5]{threlkeld_journal_1824}. M’Gill and Dismal remained there at least a month as part of the little mission establishment, doing manual work alongside a European convict servant. \ia{Threlkeld, L.E.}Threlkeld became increasingly dependent on \ia{Biraban}Biraban for his progress in the language, as well as for clearing land for his house and other work, and in return provided him and others with tools and provisions. When the mission moved from Newcastle to Lake Macquarie, \ia{Threlkeld, L.E.}Threlkeld notes in his Report for 1827 that his attempts to keep them at the mission, persuade them to build huts or remain for schooling were useless: “but for employing them at a heavy expense, not one would have remained at this Station a week” \citep[96]{gunson_australian_1974a}.

\ia{Biraban}Biraban was always his own man, without permanent ties to the mission, but his name continues to appear in \ia{Threlkeld, L.E.}Threlkeld’s “Returns of the Black Natives belonging to Lake Macquarie and Newcastle” from 1828 until 1840 \citep[360--370]{gunson_australian_1974b}. In the earliest return for 21 May 1828, “M’Gill” is listed with the Aboriginal name “We-pohng” and another man, Jemmy Jackass, whose Aboriginal name was “We-rah-kah-tah” is said to be “King of the District”. At this date Threlkeld listed 64 named individuals, 24 men, 26 women and 14 children. In 1833, when blankets were issued at Lake Macquarie, “M’Gill”, whose native name is not given, is called “chief” of the Lake Macquarie people. There is also a “Little M’Gill” listed among the children. In 1835, only Young McGill, whose native name is Ninnoai and whose probable age is 16, makes an appearance. McGill, whose native name is now given for the first time as \ia{Biraban}Birabān, is present, with his probable age given as 40 (born c. 1795?). He is still there in 1836, and in 1838 both Old and Young McGill are present. At this date, Old McGill’s estimated age is given as 30 (implying a date of birth c. 1805?), and Young McGill, whose native name has now also changed to Birabān, are both present. In 1840, McGill Senior (Birabān) is said to be 38 and Little McGill (Birabān) is said to be 20. They do not appear again.

While interested in philosophical questions, or possibly just very tolerant of \ia{Threlkeld, L.E.}Threlkeld’s repeated attempts to engage him in Christian conversation, \ia{Biraban}Biraban never converted to Christianity or learnt to read. In his Report dated 8 October 1828, \ia{Threlkeld, L.E.}Threlkeld referred to his discussions with McGill: “Our conversations vary and arise from enquiries into their customs and habits. Easy sentences, passages from scripture, and information on Christian subjects are attempted” \citep[98]{gunson_australian_1974a}. \ia{Threlkeld, L.E.}Threlkeld was delighted when McGill reported a conversation that he had continued with other blacks about “Jehovah” and asked for a picture of Jehovah that would make his meaning plainer. 

\ia{Biraban}Biraban’s relationship with \ia{Threlkeld, L.E.}Threlkeld appears to have deepened over time. He not only agreed to accompany him as a court interpreter but protected both him and his family against threats of violence from both their convict servants and hostile blacks. On one occasion, McGill “very coolly requested the loan of a gun” to shoot another Aboriginal Australian who had threatened the missionary. On another, he offered to “smash the brains” of one of \ia{Threlkeld, L.E.}Threlkeld’s convict servants who threatened his son, Joseph, then 11 years old (\ia{Threlkeld, L.E.}Threlkeld, Circular Report, 8 October 1828) \citep[99]{gunson_australian_1974a}. However, his affection for Threlkeld’s family and willingness to work for him for wages should not be confused with commitment to \ia{Threlkeld, L.E.}Threlkeld’s mission.

\ia{Biraban}Biraban’s descent into alcoholism was a subject of bitter regret to \ia{Threlkeld, L.E.}Threlkeld, who seems to have regarded this as a personal betrayal. In his 6\textsuperscript{th} Annual Report for 1836, he claimed that “the elder M’Gill… seldom visits me, he displays his knowledge at Newcastle Town, where drink has attraction far more strong than my study possesses at the Lake” \citep[133]{gunson_australian_1974a}. One important impact of this lament was that \ia{Threlkeld, L.E.}Threlkeld exaggerated the loss of native speakers who could benefit from his translations, even while recognising that the language to the north was similar and apparently mutually intelligible to speakers of \il{Hunter River Lake Macquarie language (HRLM)}HRLM. \ia{Threlkeld, L.E.}Threlkeld is likely to have been responsible for the obituary in the \textit{Sydney Morning Herald} noting the death of \ia{Biraban}M’Gill (Biraban) in Newcastle on 14 April 1846 (\textit{\citefield{noauthor_notitle_1846}{shorttitle}}, 1 May 1846). This refers to him as “an aboriginal native well known a few years back at the Supreme Court as assistant interpreter”, as well as “a living witness” against the claims of the “French Phrenologists” that the Australian Aboriginal people were physically and intellectually deficient.\footnote{Note that this is a later date than that provided by Niel \citet{gunson_biraban_1966} in his biography of \ia{Biraban}Biraban for the \textit{Australian Dictionary of Biography}.} While there is no reason to doubt \ia{Threlkeld, L.E.}Threlkeld’s respect and affection, \ia{Biraban}Biraban was also critical for validating his linguistic mission. In his guise as “McGil, Chief of the Tribe at Bartabah”, to give him the title on his breastplate, \ia{Threlkeld, L.E.}Threlkeld may have hoped that \ia{Biraban}Biraban could play the role of Pomare II (1782--1821), who authenticated \ia{Nott, Henry}Henry Nott’s translation of a \il{Tahitian}Tahitian Bible.\footnote{Unlike \ia{Biraban}Biraban, Pomare II was baptised (on 16 May 1819). \ia{Nott, Henry}Nott’s translation of part of Samuel, Book I is in the \href{https://archiveshub.jisc.ac.uk/search/archives/a4dfd14c-de1b-38de-9c4a-e9746d28242f?component=eb29e95e-031f-3e54-8c32-131797a03820}{University of Manchester Library GB 133 Eng MS 401}. \ia{Nott, Henry}Nott subsequently supported the printing of the Bible.} \citet[81]{gunson_pomare_1969} suggests that Pomare’s “will to power” and his conversion reflected a conviction that the “God of the British” was more powerful than the traditional gods and that there was political advantage in subverting the old religion, honouring the missionaries and propitiating them with gifts. In contrast, there is every reason to think that \ia{Biraban}Biraban, having unrivalled knowledge of British law, religion and society, saw nothing which persuaded him to trade them for his traditional beliefs.

\section{Threlkeld}

\subsection{Biography, education and conversion}

Like many who entered missionary service with the \is{London Missionary Society}London Missionary Society, \ia{Threlkeld, L.E.}Threlkeld had modest origins. According to a family genealogy \citep{threlkeld_family_genius_1767}, \ia{Threlkeld, L.E.}Threlkeld was named after an ancestor who allied himself to the Lancastrian cause during the \is{Wars of the Roses}Wars of the Roses \citep[15]{gunson_australian_1974a}. Another ancestor, Caleb Threlkeld (1676--1728), was a dissenting clergyman and physician, who published the first scientific account of the flora of Ireland. By the time \ia{Threlkeld, L.E.}Threlkeld was born in London on 20 October 1788, the family were neither prosperous nor genteel but moved in the ranks of artisans and small tradesmen. This is important because these layers of English society were critical to the explosive growth of the missionary movement and its seedbed in the evangelical and Wesleyan \is{Religious revival}religious revival. Threlkeld’s father, a turner and brush maker, proposed that he “sail with a relation to the East Indies”, then train as an apothecary before \ia{Threlkeld, L.E.}Threlkeld tried his luck on the stage. A significant moment in \ia{Threlkeld, L.E.}Threlkeld’s life was his conversion, probably around the time of his marriage, to a more committed Christian life. He married Martha Goss (1789--1824) in 1808, when they were both still teenagers, and together they came under the influence of the Rev. Cradock Glascott (1743--1831), Vicar of Martha’s hometown parish of Hatherleigh, in Devon, from 1781 until his death. An early follower of John Wesley, Glascott sided with the \is{Calvinism}Calvinist party in the controversy over \is{Arminianism}Arminianism and for a time served as a minister with the Countess of Huntingdon’s Connexion, the Welsh \is{Calvinism}Calvinist sect founded by Selina Hastings in 1783 \citep{harding_countess_2003}. The Threlkelds absorbed both Glascott’s \is{Calvinism}Calvinistic theology, preferring the strict interpretation of salvation by God alone, as well as his evangelical warmth. Initially, \ia{Threlkeld, L.E.}Threlkeld itinerated in rural Devon before convincing Martha, who was reluctant to leave England, that he should propose himself as a candidate to the \is{London Missionary Society}London Missionary Society.

\enlargethispage{\baselineskip}

The \is{London Missionary Society}London Missionary Society (1795), one of the great “voluntary” societies which emerged in the wake of the Protestant \is{Religious revival}\is{Evangelical revival| see {Religious revival}}evangelical revival, was formed following a meeting in London in November 1794 “to spread the knowledge of Christ among heathen and other unenlightened nations” \citep[10]{soas_guide_2017}.\footnote{For the early history of \is{London Missionary Society}the Society to 1895, see \citet{lovett_history_1899}; \citet{goodall_history_1954} continues to 1945, and \citet{thorogood_gales_1994} until the more recent era.} Unlike the venerable “chartered” societies of the established churches, including the Society for Promoting Christian Knowledge (1698), the Society for the Propagation of the Gospel (1701) and the Society in Scotland for Promoting Christian Knowledge (1709), its finances depended on external fundraising rather than direct grants from government. The Society was non-denominational and from 1796 it embraced the “fundamental principle” that “its design is not to send \is{Presbyterianism}Presbyterianism, Independency, \is{Episcopalianism}Episcopacy, or any other form of Church Order and Government (about which there may be difference of opinion among serious persons), but the glorious Gospel of the blessed God, to the heathen” \citep[2]{goodall_history_1954}. Of the major British societies, the \is{London Missionary Society}LMS was the least imperialist \citep[5]{darch_missionary_2009}, reflecting its support base among the lower middle and artisan classes, the stronghold of liberal English and Welsh nonconformity. The \is{London Missionary Society}LMS was governed by a Board of Directors made up of its constituent churches and managed through its annual general meeting held, from 1831 to 1891 at Exeter Hall. While overtly democratic, this was not extended to the missionaries employed under its aegis who were expected to follow the strict formal “Instructions” provided to every missionary prior to taking up a missionary appointment. In addition, the \is{London Missionary Society}LMS missionaries to the \is{South Seas}South Seas were placed under the watchful eye of the Senior Colonial Chaplain, the NSW-based \ia{Marsden, Samuel}Rev. Samuel Marsden, who approved expenditure, liaised with government and offered hospitality and advice to missionaries coming to and from the Islands. The \is{London Missionary Society}LMS was the least wealthy of the major societies, and this is a significant factor in the financial quagmire which enveloped \ia{Threlkeld, L.E.}Threlkeld’s Australian mission.

\subsection{South Seas, 1816}

\ia{Threlkeld, L.E.}Threlkeld’s subsequent work at Lake Macquarie can only really be understood in the light of his first missionary appointment with the \is{London Missionary Society}LMS to the \is{South Seas}South Seas, where he was one of three “apostles” to Raiatea in the “Society Islands” (now French Polynesia) \citep[106]{lovett_history_1899}. \ia{Threlkeld, L.E.}Threlkeld was initially unhappy that he was bound for the Pacific rather than to his first preference, which was to Africa. He sailed for the \is{South Seas}South Seas on 23 January 1816 but was detained for over a year in Rio de Janeiro on account of the illness of his wife and their first child. Once in the Islands, he committed himself to the new mission and its focus on the Word, which meant translation, education, preaching and printing the gospel in the \il{Tahitian}Tahitian language. While \ia{Threlkeld, L.E.}Threlkeld was delayed in Rio, his fellow missionary \ia{Ellis, William}William Ellis (1794--1872) brought the first printing press to the Islands. Within a year \ia{Ellis, William}Ellis had printed thousands of copies of a spelling book, a \il{Tahitian}Tahitian catechism with scripture extracts and the Gospel of St Luke, translated by \ia{Nott, Henry}Henry Nott (1774--1844) under the direct supervision of the military conqueror of Tahiti, Pomare II \citep[215]{lovett_history_1899}.

\ia{Threlkeld, L.E.}Lancelot and Martha Threlkeld were collected from Rio by the same ship taking a group of four missionaries, including the \ia{Williams, John}Rev. John Williams (1796--1839), to the \is{London Missionary Society}LMS mission in Tahiti. They arrived in Hobart on 21 March and Sydney on 12 May 1817 where they were welcomed by Governor Lachlan Macquarie and \ia{Marsden, Samuel}Samuel Marsden and given pastoral work in the colony. On 11 September 1818 the Threlkelds finally arrived in Huahine. After years of failure, the \is{London Missionary Society}LMS mission to the \is{South Seas}South Seas was bearing a harvest of souls and these were eventful times. Following the military defeat of their rivals, the Pomare dynasty had made strategic use of the \is{London Missionary Society}LMS missionaries to secure their conquest and broker advantageous relations with Christian powers. \is{Mass conversion}Mass conversion to Christianity, which began under Pomare II (c. 1782--1821), was proclaimed in regular reports published in\is{London Missionary Society} the Society’s \textit{Missionary Chronicle} as a triumph of the gospel. From 1821 to 1829, the \is{London Missionary Society}LMS appointed the \ia{Tyerman, Daniel}Rev. Daniel Tyerman and a businessman, \ia{Bennet, George}George Bennet, as a delegation “for the \textit{furtherance} of the Gospel and for the promotion of \textit{civilization} among the natives” \citep[207]{sibree_register_1923}. \ia{Threlkeld, L.E.}Threlkeld’s mission in Huahine and subsequently Raiatea was therefore conducted under the watchful eye of auditors neither of whom proved sympathetic to his perennial difficulties with complying with orders.\footnote{\ia{Bennet, George}Bennet and \ia{Tyerman, Daniel}Tyerman’s reports on their travels to the \is{South Seas}South Seas, New Zealand, Sydney, Java, Singapore, Macao, Canton, Malacca and Penang, India, Mauritius and the Cape of Good Hope were published in the Transactions of the \is{London Missionary Society}Missionary Society and edited for publication by James \citet{montgomery_journal_1831}. \ia{Tyerman, Daniel}Tyerman died suddenly on 30 July 1829, which meant \ia{Bennet, George}Bennet’s hostile view of \ia{Threlkeld, L.E.}Threlkeld prevailed within \is{London Missionary Society}the Society. \ia{Bennet, George}Bennet died in London on 13 November 1841.}

Writing to \is{London Missionary Society}the Society from Parramatta on 20 March 1824 \citep[box 2/ folder 3]{lms_australia_1798-1968}, the missionary and settler \ia{Elder, John Rawson}John Elder (1722--1836) warned that the extraordinary success in Tahiti could be a danger to other missionaries, who might be disappointed at the slow progress of their own enterprise, especially where there was no plentiful harvest (Matthew 9:35), but rather the “day of small things” (Zechariah 4:10) -- not to be despised but a challenge to the spirit: “While the account published about Otaheite may have done good”, he suggested, “like a charm, in inducing many to support Mission, it may have done harm in discouraging other missionaries who not meeting with the same appearances, may be apt to think they had had no success at all” \citep[box 2/ folder 3]{lms_australia_1798-1968}. He also warned that the success of the mission owed more to the Revolution that drove the missionaries to flee to Port Jackson in 1808 than to the Gospel. This would be relevant to the \is{London Missionary Society}LMS mission to New South Wales, which soon pitched from crisis to crisis under \ia{Threlkeld, L.E.}Threlkeld’s fitful command.

Missionary success in the \is{South Seas}South Seas mitigated the disruptive impact of epidemic disease, endemic warfare between rival chiefs and rapid transition to cash cropping and commerce in western goods including cloth, iron tools and alcohol. From Sydney on his way to Tahiti, \ia{Williams, John}Williams wrote on 2 September 1817, “We long to reach Tahiti. We hear that the word of the Lord is prospering wonderfully there. Our enemies, and even infidels, say that nothing but a miracle could have wrought such a change” \citep[33]{prout_memoirs_1843}. With direct experience of the fraught state of the mission, including the repeated need for the missionaries to retreat to Sydney, the \ia{Marsden, Samuel}Rev. Samuel Marsden was more cautious, advising the \is{London Missionary Society}LMS party of the realities including the need for strict financial control at all times \citep[19]{gunson_australian_1974a}. Even \ia{Marsden, Samuel}Marsden was shocked, however, when in June 1819 the \is{London Missionary Society}LMS cancelled the stipend allocated to the missionaries by the Sydney committee -- ensuring they were entirely dependent on their patrons in Tahiti. This increased pressure on the missionaries to commercialise their work in the islands and hastened the transition from subsistence agriculture to cash crops and integration into the wider Pacific trade network.

\ia{Williams, John}Williams and \ia{Threlkeld, L.E.}Threlkeld represented a new generation of missionaries and a change from the thrifty, faith-led audacity of their predecessors. Both were older, better educated, and more knowledgeable in the ways of the world. Before coming to the mission, \ia{Threlkeld, L.E.}Threlkeld had had a modest career in business and on the stage, as well as receiving medical training in surgery and midwifery and \is{British system| see {Lancasterian system}}\is{Lancasterian system}the “British” or \is{Lancasterian system}Lancasterian system of education prior to his departure \citep[16--17]{gunson_australian_1974a}. He had been taught preaching, biblical exegesis and \il{English}English grammar by the Rev. Matthew Wilks, all the linguistic training he ever received prior to his adventures in Australian and the \is{South Seas}South Seas. \ia{Williams, John}Williams studied at the celebrated Dissenting academy conducted by Dr David Bogue (1750--1825) at Gosport, near Plymouth. With this polish, both \ia{Williams, John}Williams and \ia{Threlkeld, L.E.}Threlkeld were positioned between older and younger missionaries in Tahiti. The new arrivals were disturbed by the extent to which their colleagues were controlled by Pomare I, as well as by the impoverished state of the missionaries, whose children were running wild and naked with those of the Tahitians. After a series of tempestuous meetings, \ia{Williams, John}Williams and \ia{Threlkeld, L.E.}Threlkeld took up the invitation of the high chief, Tamatoa I, to move to Raiatea \citep[243]{lovett_history_1899}, where they created a more democratic polity for the rising Christian community. As \citet[26--27]{garrett_live_1982} explains, they established auxiliary societies to recruit native teachers who were largely responsible for the effective conversion of the Islands. To support the mission, they devised payments in commodities including pigs, coconut oil, arrowroot and cotton and, controversially, acquired a number of small ships -- in which the missionaries held shares -- which undertook mixed commercial and \is{Proselytism}proselytising duties between the islands. \ia{Threlkeld, L.E.}Threlkeld introduced radical changes intended to limit the absolute power of Tamatoa, including a new law code \citep[112]{williams_narrative_1839} with trial by jury \citep[248]{lovett_history_1899} and \is{Congregationalism}congregational management of the church.

In his memoir of \ia{Williams, John}Williams, Ebenezer \citet[42]{prout_memoirs_1843} likened the passage of \ia{Williams, John}Williams and \ia{Threlkeld, L.E.}Threlkeld to Raiatea in September 1818 to that of Paul and Silas to Macedonia, the site of the first Christian mission to Europe as well as the missionary call, “Come over into Macedonia, and help us” (Acts 16:9 KJV). As soon as the \ia{Williams, John}Williams and \ia{Threlkeld, L.E.}Threlkeld families arrived, a great feast was prepared for them, consisting of five large hogs for \ia{Williams, John}Williams and his wife, and the same for the \ia{Threlkeld, L.E.}Threlkelds. In addition, they were provided with rolls of cloth and crates of yams, taro, cocoa nuts, plantain and bananas \citep[43]{prout_memoirs_1843}. On the spiritual front, \ia{Threlkeld, L.E.}Threlkeld and his fellow missionaries demanded the Islanders “utterly abolish” their idols (Isaiah 2:13). On Rarotonga, \citet[98--99]{williams_narrative_1839} describes how these were given up to be exhibited in the chapel or to be sent to England.\footnote{An image of two missionaries and their wives seated and receiving the toppled gods, together with the text “And the idols he shall utterly abolish” (Isaiah 2:18) appears on the front cover of \citeauthor{williams_narrative_1839}’ (\citeyear{williams_narrative_1839}) account of the Tahitian mission.} In return the people received books, including \il{Tahitian}Tahitian catechisms, spelling books and scripture translations, which \ia{Williams, John}Williams argued were essential to the progress of literacy and conversion: “I think it a circumstance of very rare occurrence that a religious impression is produced upon the minds of a people, except by addressing them in their mother tongue” \citep[104]{williams_narrative_1839}. In July 1822, \ia{Threlkeld, L.E.}Threlkeld and \ia{Williams, John}Williams despatched the forfeited gods to the \is{London Missionary Society}LMS Museum in Blomfield Street in London, first opened in 1814 \citep{hooper_embodying_2007}.\footnote{See also \ia{Threlkeld, L.E.}Threlkeld’s letter to Judge Burton, 17 November 1838 \citep[274]{gunson_australian_1974b}: “It would not disappoint your expectations were you to call at the Missionary Museum, Mission House, Bloomfield Street, Finsbury Square, London, where all the Gods are deposited, and not one of them dare stir for the life of him!”} They were transferred to the British Museum in 1891 \citep[308, n. 47]{gunson_australian_1974b}.

Besides toppling idols and halting the practices of human sacrifice, polygamy and cannibalism, the missionaries shifted the economy toward trade goods that would support the growing missionary enterprise. In 1815, \ia{Marsden, Samuel}Marsden cautiously gave his approval to the missionaries taking items for barter with the natives, such as axes, hammers and hoes \citep{marsden_letter_1818}. On 24 March 1823, \ia{Williams, John}Williams wrote to \ia{Marsden, Samuel}Marsden from Raiatea to celebrate the growing display of goods featured at annual missionary meetings: “You may rejoice Dear Sir in the accomplishment of the object to which your exhortations have been unremitted, \& on which your desires have been constant placed, viz -- The permanent introduction of the art of making sugar in these Islands -- to which may be added -- The Knowledge of Tobacco” \citep[box 2/ folder 2]{lms_australia_1798-1968}. On 20 May 1825, the \is{London Missionary Society}LMS delegation used their valedictory letter to the missionaries to embark on a cotton factory which would save \is{London Missionary Society}the Society the cost of clothing the missionaries \citep[299]{tyerman_extracts_1827}, noting that the Lord “loveth a cheerful giver” and the need to guard against covetousness. This was the ideal \is{London Missionary Society}LMS mission: frugal, economically independent and tied to the world economy in desirable plantation-grown tropical commodities. Meanwhile conversion continued through the work of native teachers.

Also critical to the mission was the presence and participation of missionary wives. In Island society, wives were as highly valued, at least in hogs, as their husbands, and it was a substantial blow when Martha Threlkeld died in Raiatea on 7 March 1824, after a short illness. In a letter to the \is{London Missionary Society}LMS Directors, \ia{Williams, John}Williams reported that on hearing the news, King Tamatoa, the chiefs and native deacons came to comfort \ia{Threlkeld, L.E.}Threlkeld, staying with him all night prior to the burial \citep[139]{prout_memoirs_1843}. Almost immediately, \ia{Threlkeld, L.E.}Threlkeld decided to return to England -- partly to find a new missionary helpmeet, without whom he felt unable to continue the Raiatea mission. Alternatively, according to the former \is{London Missionary Society}LMS missionary \ia{Hayward, James}James Hayward, writing from Sydney on 2 July 1824 \citep[box 2/ folder 2]{lms_australia_1798-1968}, \ia{Threlkeld, L.E.}Threlkeld mainly wished to return to London so as to justify his controversial decision to change the date of the Sabbath on Raiatea.\footnote{\ia{Hayward, James}Hayward supported \ia{Threlkeld, L.E.}Threlkeld in his bitter dispute with J.D. Lang, leaving his position as elder in Lang’s Scots Church and helping to establish a \is{Congregationalism}Congregational Church in Sydney \citep[325]{gunson_australian_1974b}.} The \is{London Missionary Society}LMS delegation, \ia{Tyerman, Daniel}Tyerman and \ia{Bennet, George}Bennet, agreed to \ia{Threlkeld, L.E.}Threlkeld’s departure, no doubt regarding this as an opportunity to ease tensions between rival missionary parties. Leaving his three younger daughters with missionary friends, \ia{Threlkeld, L.E.}Threlkeld left for Sydney with his son Joseph Thomas. He never returned to Tahiti. His friend \ia{Williams, John}John Williams, martyred on the beach of Erromango in Vanuata in 1839 and subsequently cannibalised, became, with David Livingstone, one of the most celebrated of all \is{Congregationalism}Congregationalist missionaries.

\subsection{Sydney and Caddie, 1824}

\ia{Threlkeld, L.E.}Threlkeld and his son Joseph arrived in Sydney on 19 August 1824 and, as on his previous arrival, the missionary was allocated preaching duties along the circuit of \is{Non-conformism}Non-conformist chapels of the colony. The oldest of these was Ebenezer Chapel (1808--1809) on Cattai Creek near Windsor, a bastion for \ia{Threlkeld, L.E.}Threlkeld’s \is{Calvinism}Calvinistic Methodism, though the site was later captured by mainstream \is{Presbyterianism}Scottish Presbyterians. Thomas Arndell (1753--1821), who had been an assistant to Surgeon General John White of the \is{First Fleet}First Fleet, was among those who contributed to the cost of the chapel \citep{fletcher_arndell_1966}, built just cross the river from Arndell’s still surviving 1821 cottage and historic grain silos \citep[217]{boon_hawkesbury_2017}. In 1807, Arndell married Elizabeth Burley alias Dalton (1766--1843), a former convict indicted for stealing fourteen linen handkerchiefs and transported for seven years on the “Lady Penrhyn” in 1787. On the voyage she conceived a child following a liaison with a seaman \citep{gillen_founders_1989}. The \ia{Marsden, Samuel}Rev. Samuel Marsden signed an affidavit that he married the Arndells in St Matthew’s \is{Anglicanism}Anglican Church in Windsor; the marriage prospered, and Arndell raised Elizabeth’s children as his own. Although resigned to marriage in the established church, Elizabeth had been baptised in Swallow Street Scotch Church in Westminster, London, and seems to have transmitted her Scotch \is{Calvinism}Calvinist religious preferences to her family. While \citet{fletcher_arndell_1966} refers to Arndell as an \is{Anglicanism}Anglican, it seems likely the family favoured Ebenezer Chapel at Cattai, which attracted a series of Independent, \is{Calvinism}Calvinist preachers, including \is{London Missionary Society}LMS missionaries passing to and from the \is{South Seas}South Seas \citep[21]{gunson_australian_1974a}.

\ia{Threlkeld, L.E.}Threlkeld had few social pretensions and had no problem courting a currency lass when invited to visit the Arndell family at their farm, “Caddie Park” (now Cattai National Park), on the Hawkesbury River. He soon proposed to Sarah Arndell (1796--1853), then 28 and eighteen years his junior, exciting her with the prospect of a missionary partnership. \ia{Marsden, Samuel}Marsden was again the celebrant, marrying them in St John’s Parramatta on 20 October 1824 (\textit{\citefield{noauthor_notitle_1803}{shorttitle}}, 21 October 1824: 3). In his Public Journal, \ia{Threlkeld, L.E.}Threlkeld asked God to bless their union, “making us helpmeets to each other in our Missionary employment” \citep[21]{gunson_australian_1974a}. Sarah helped raise Threlkeld’s four surviving children by Martha Goss, as well as five more who arrived between 1825 and 1834. In the colony, \ia{Threlkeld, L.E.}Threlkeld put the health of his family first, and it is testimony to his commitment that, with the exception of his first-born, all nine of the children were living when he died in 1859.\footnote{Dates of births, deaths and marriages, with sources, see \ia{Threlkeld, L.E.}Rev. Lancelot Edward Threlkeld, 1788--1859, Australian Royalty, <\url{https://australianroyalty.net.au/}> (Accessed 1 July 2019).} In tragic contrast, of the ten children of \ia{Williams, John}John and Mary Williams, only three survived their peripatetic missionary travels. The Arndells and Threlkelds retained their \is{Congregationalism}Congregationalist values, as reflected in the handsome memorial to Lancelot and Sarah in the Independent section of Sydney’s Rookwood Cemetery (see \hyperref[Appendix_B]{Appendix B}).

It was possibly in 1824, while dividing his time between Sydney and Caddie, that \ia{Threlkeld, L.E.}Threlkeld began the first of his thorough researches into Aboriginal languages. Before his arrival at Newcastle in 1825, he had already collected the “Speci\-mens of the language of the Aborigines of New South Wales to the northward of Sydney” \citep{threlkeld_specimens_1824}, based on elicitation sessions with an unnamed speaker (or speakers). Perhaps he made contact with them while staying on the Hawkesbury. This eleven page manuscript has about 460 words and phrases and 204 short questions and sentences \citep[104--106]{threlkeld_threlkeld_1822-1862}, and is accompanied by a single page headed “Songs of the Natives of New South Wales to the North of Sydney” \citep[107]{threlkeld_threlkeld_1822-1862}. \citet[23]{capell_aboriginal_1970} identified the handwriting as \ia{Threlkeld, L.E.}Threlkeld’s, “agreeing with the writing of the unpublished Gospel of St. Mark”, but there are still significant mysteries about this manuscript. It is undated, and the only indication of where the language elicitations took place is the vague reference in the title: “to the northwards of Sydney”. There is a general consensus among linguists that the language of this collection is \il{Hunter River Lake Macquarie language (HRLM)}HRLM (\cite[168--174]{wafer_handbook_2008}, \cite{lambert_colonial_2006}), but there are some small phonological and lexical differences from the dialect recorded at Lake Macquarie.\footnote{For further discussion of this manuscript see Karskens and McKenna (\citeyear[102]{karskens_nah_2019}, and forthcoming).}

This is just one of several language manuscripts included among the unpublished \ia{Threlkeld, L.E.}Threlkeld papers in the SLNSW (see \hyperref[Appendix_A]{Appendix A}). The authorship of the other three has never been positively established. One is headed “Port Macquarie” and includes 196 words and a few short phrases and sentences \citep[104--106]{threlkeld_threlkeld_1822-1862}. It is in a different hand and has been attributed to \ia{Threlkeld, L.E.}Threlkeld’s daughter \citep[23]{capell_aboriginal_1970}. As well, there are two collections from northern Australia: “Native Language Port Essington Australia” \citep[119--120]{threlkeld_threlkeld_1822-1862} and “Native Language at Port Raffles New Holland” \citep[121--122]{threlkeld_threlkeld_1822-1862}.\footnote{Port Essington was a short-lived colony on the Coburg Peninsula, in what is now the Northern Territory. It operated from 1824 to 1849. Port Raffles, a short distance to the east, was even more short-lived (1827--1829).} While it is possible to speculate on the origin of these, there is no certainty about how they came into \ia{Threlkeld, L.E.}Threlkeld’s hands, and he seems to have found no further use for them. They are not included in either of his comparisons of the linguistic features of a number of Aboriginal languages published in 1839 and 1850.\footnote{For the first comparison, published in his “Annual Report” on the Mission for 1839, see \citet[161]{gunson_australian_1974a}. For another comparison, based on a different set of data, see \citet[70--71]{threlkeld_key_1850}.}

In Sydney, the \is{London Missionary Society}LMS delegation found much to admire, reporting in a series of letters: “This colony is, we doubt not, destined by Divine Providence to be a great nation, and is in very flourishing circumstances” \citep[290]{tyerman_extracts_1827}. At the same time, they were also deeply concerned at the conditions of the Aboriginal people and secured the agreement of the Governor for a new mission which would for the first time be conducted in the \il{Hunter River Lake Macquarie language (HRLM)}Aboriginal language. \ia{Threlkeld, L.E.}Threlkeld, now re-married and planning to return to the \is{South Seas}South Seas, agreed to “devote himself to this great work” and that this could be done without any injury to the mission in the \is{South Seas}South Seas. \ia{Tyerman, Daniel}Tyerman and \ia{Bennet, George}Bennet were also pleased that there was a better feeling toward the Aboriginal people and their potential for Christian uplift. Previously, there had been agreement that there was no prospect of instructing the Aboriginal people in their own language:

\begin{quote}
     When we came to this colony, all with whom we conversed agreed with us that is was a very desirable thing to give christian instruction to the natives, towards which we were informed nothing had been done. But they were also persuaded that the object could never be accomplished, except indeed that a few might perhaps be taught sufficient \il{English}English to understand something of religion. There was an almost perfect unity of opinion in the colony (we know not one exception) that it would be impracticable to obtain the language, and even if it could be obtained, that it would be found too poor to be of any use in conveying moral and religious ideas. \citep[294]{tyerman_extracts_1827}
\end{quote}

Now not only the \is{London Missionary Society}LMS, but also the Wesleyans and the \is{Anglicanism}Anglican \is{Church Missionary Society}CMS were contemplating a mission of this type.\footnote{For the Wesleyan mission and the scandal about its claims for rapid progress in the language, see \citet{roberts_beong_2009}. The \is{Church Missionary Society}CMS mission to Wellington Valley also attempted a linguistic mission, though with less success than \ia{Threlkeld, L.E.}Threlkeld. See \citet{carey_lancelot_2004}.} After considering possible sites for a mission, rejecting first Moreton Bay and then Bathurst and Wellington Valley, the delegation settled on the location of Reid’s Mistake, south of Newcastle, which they had viewed with \ia{Threlkeld, L.E.}Threlkeld, informing the \is{London Missionary Society}LMS in their letter of 8 February 1825: “The natives here are numerous, more accessible than those about Sydney, etc. and less immoral. They expressed themselves glad to hear that a person was coming to reside among them, who would teach them good things” \citep[294]{tyerman_extracts_1827}. Following his reconnoitre of the site, \ia{Threlkeld, L.E.}Threlkeld received detailed Instructions from \ia{Tyerman, Daniel}Daniel Tyerman and \ia{Bennet, George}George Bennet, dated from Sydney, 24 February 1825 \citep[18--22]{threlkeld_threlkeld_1822-1862}. \ia{Threlkeld, L.E.}Threlkeld was advised to “take a house at the Town of Newcastle for a short period”, which he proceeded to do, and then to “erect a suitable residence … on some appropriate spot with the limits of the land which has been given”. His first task, however, was to learn the language:

\begin{quote}
     As a knowledge of the language of the Natives must be regarded as essential to the success of your Mission, you will deem it your duty, while the house is in progress, as well as after you have taken up your abode in it, to be using your best efforts to acquire it; while it will greatly facilitate the progress of your work, to make yourself familiar with their customs, superstitions, and habits. By a knowledge of these, you will see what the principal difficulties opposing your success are, while an intimate acquaintance with their language will enable you to communicate that information, respecting the Gospel of Jesus, which will be best adapted to remove the obstacles, and to ensure success. \citep[18]{threlkeld_threlkeld_1822-1862}
\end{quote}

Besides learning the language, \ia{Threlkeld, L.E.}Threlkeld was instructed to seek to “abandon their debasing habits, and to imitate those of civilized society”. He was to set up Schools, “for the instruction of all” and to seek to heal both their mental and their physical woes. As his highest duty, he was urged to master the language and to preach to the Aboriginal people in their own language.

\begin{quote}
     The preaching of the Gospel being the great means which is ordained of God, to effect the conversion of sinners, and to promote their sanctification and meetness for glory, and to break down the vile superstitions which degrade the heathen world, your immediate solicitude will be, to acquire such a knowledge of the language of the people among whom you are to dwell as to be qualified, as soon as possible, to preach to them, in their own tongue, the wonderful works of God. \citep[19]{threlkeld_threlkeld_1822-1862}
\end{quote}

\ia{Threlkeld, L.E.}Threlkeld lost no time in carrying out these instructions. Only months after his arrival in Newcastle, he sent a manuscript of his “Orthography and Orthoepy” to Governor Brisbane \citep{threlkeld_orthography_1825} and another copy with his letter to the \is{London Missionary Society}LMS dated Newcastle, 10 October 1825 \citep[box 2/ folder 3]{threlkeld_orthography_1825}.

\section{“Bahtahbah” LMS mission to NSW, 1825--1829}

Initially, \ia{Threlkeld, L.E.}Threlkeld intended to return to Raiatea with his bride; instead, he accepted an offer to begin a mission to the Aboriginal people of New South Wales. The choice was not without complications, and Governor Brisbane originally proposed that a fully-funded government mission should be undertaken at Moreton Bay (now Brisbane), or possibly Wellington Valley, where a controversial mission had been conducted by the Wesleyans \citep{roberts_beong_2009}. According to \citeauthor{threlkeld_statement_1828}’s (\citeyear{threlkeld_statement_1828}) highly contested account of the mission,{\enlargethispage{\baselineskip}\footnote{For the vigorous rebuttal of Threlkeld’s claims (\citeyear{threlkeld_statement_1828}), see the copy annotated by \ia{Bennet, George}George Bennet in the National Library of Australia (Call number mc N 1475). One of the milder objections is to the note on the cover, requesting that “Persons who may accidentally obtain a perusal, will abstain from publishing its contents”, to which \ia{Bennet, George}Bennet’s response is: “A ridiculous pretence! Everyone will instantly see that this note is intended to insure its further diffusion”.} both sites were rejected in favour of a new \is{London Missionary Society}LMS mission to Lake Macquarie, near Newcastle, on a government land grant of 10,000 acres held “in trust for the Aborigines of Lake Macquarie”, but without a stipend or assigned servants. The wording here is important, because the land was neither a grant to the \is{London Missionary Society}missionary society nor to its missionaries but rather a trust for the Aboriginal people. In his Public Journal for 26 January 1825, \ia{Threlkeld, L.E.}Threlkeld notes his “many misgivings of heart” resulting from his continued financial dependence on the distant directors of \is{London Missionary Society}the Society \citep[85]{gunson_australian_1974a}. \ia{Threlkeld, L.E.}Threlkeld’s personal understanding of the subsequent saga can be traced through his extensive public and private journals. There are valuable and complementary interpretations of the resultant tangled history of the \is{London Missionary Society}LMS and the Lake Macquarie Mission by \citeauthor{gunson_australian_1974a} (\citeyear{gunson_australian_1974a, gunson_australian_1974b}) and \citet{johnston_blister_2006}, though what follows is based on a fresh interpretation of the sources, including a number which have come to light in recent times.

From May 1825 until September 1826, \ia{Threlkeld, L.E.}Threlkeld lived in Newcastle, where the Commandant allowed him the use of the Government Cottage. Here he focussed on learning the language and corresponded with his \is{Humanitarianism}humanitarian friends, including the Attorney General Saxe Bannister and Lieutenant Richard Sadleir, on behalf of Aboriginal people \citep[94]{gunson_australian_1974a}. Sadlier was a “pious Protestant” who had charge of the boys’ Orphan School in Liverpool and was visited by the Quaker travellers \ia{Backhouse, James}James Backhouse and George Washington Walker \citep[5]{backhouse_extracts_1838}.

Progress in the new mission was slow and full of frustration. In Raiatea, \ia{Threlkeld, L.E.}Threlkeld had been accustomed to living in close proximity to a large population of native people with their own villages and strict hierarchical social order. He was never alone but supported by a team of like-minded missionaries who shared his spiritual, linguistic, and commercial objectives to teach, civilise and convert the people. There had been rapid progress in learning the language which was the medium for teaching and preaching and the work of translation was well advanced. But in Sydney, Newcastle and Lake Macquarie, \ia{Threlkeld, L.E.}Threlkeld was isolated and financially dependent, costs were high and there were problems of security. On 7 May 1825, the day he said goodbye to the \is{London Missionary Society}LMS delegation and sailed from Sydney to the former penal colony of Newcastle, \ia{Threlkeld, L.E.}Threlkeld noted in his Journal that the “greatest danger is from robbers. Newcastle having but just immerged from being a penal settlement the most choice rogues are of course here” \citep[fol. 3v]{threlkeld_journal_1824}. The handwritten annotation in the \is{London Missionary Society}LMS was more concerned at the high costs: “Estimate for building the cottage £325 !” and “Appears to have determined to employ the natives in the erection of a house for the mission” \citep[fos. 5--5v]{threlkeld_journal_1824}. Neither prospect accorded with the delegation’s instructions to observe the most stringent economy, secure financial independence and embrace opportunities to support the mission from cheerful giving.

In 1826, \ia{Threlkeld, L.E.}Threlkeld moved his family to the newly constructed mission house at “Bahtahbah”, overlooking Belmont Bay and centrally located on the grant of land which covered the entire northern peninsula of Lake Macquarie. Access to Newcastle was provided by a track, constructed at considerable financial angst to \is{London Missionary Society}the Society, but a source of constant concern to \ia{Threlkeld, L.E.}Threlkeld because of the danger from ruffians, often former convicts, who frequented the site. His 1827 Report stressed the difficulties of obtaining a sufficient number of Aboriginal people to stay close to the mission, given the many rival attractions provided by Newcastle and even distant Port Stephens, including alcohol, prostitution and wheat flour rather than corn meal \citep[Circular]{threlkeld_specimens_1827}.

\subsection{Orthography and Orthoepy (1825, 1827)}

One of \ia{Threlkeld, L.E.}Threlkeld’s earliest challenges was to establish the most appropriate way of spelling an Australian language. This was seen clearly by him as his most significant initial task, but it was one in which he was not entirely successful. At first, he compromised with his own best intuition and \il{English}used “English” (ie, non-phonetic) spelling. The introduction to the \citeyear{threlkeld_orthography_1825} “Orthography and Orthoepy” states:

\begin{quote}
     The connection existing betwixt the British and the Aborigines is the reason for adopting the \il{English}English alphabet as a foundation for the native tongue. The tables show the fixed sounds of the letters and syllables agreeably to the \il{English}English examples … Time only can decide on its practicability. (\ia{Threlkeld, L.E.}Threlkeld to \is{London Missionary Society}LMS, Newcastle, 10 October 1825) \citep[box 2/ folder 3]{lms_australia_1818-1831}
\end{quote}

All missionaries were expected to begin by “fixing” the language -- so that a consistent set of rules might be followed by subsequent missionaries, and that converts might be able to recognise the same words wherever they occurred.\footnote{The orthography for \il{Tahitian}Tahitian was fixed from about 1822 \citep[73, n. 30]{gunson_australian_1974a}.} This created very considerable difficulties in societies where there were many subtle distinctions between languages separated by small distances and spoken by only small number of people. For pragmatic and political reasons, missionaries also attempted to secure a single literary language with ascendancy over other varieties.\footnote{For the hegemonic implications of this practice see \citet{landau_realm_1995}, and for \is{Linguistics!Missionary linguistics}missionary linguistics in Australia, \citet{carey_lancelot_2004}.}

\ia{Threlkeld, L.E.}Threlkeld also had to contend with \ia{Marsden, Samuel}Marsden, who opposed both missions to the Aboriginal people and attempts to learn their languages. Even in New Zealand, where \ia{Marsden, Samuel}Marsden enthusiastically backed the mission, \ia{Marsden, Samuel}Marsden objected to experts such as the \ia{Lee, S.}Rev. S. Lee, Professor of Oriental Languages at the University of Cambridge, who had published a \il{Maori}Maori grammar for the \is{Church Missionary Society}Church Missionary Society in \citeyear{lee_grammar_1820} \citep{lee_grammar_1820}. After spending an afternoon struggling with \ia{Kendall, Thomas}Kendall and \ia{Lee, S.}Lee’s grammar, he was ready to abandon the attempt, noting: “as the New Zealanders were so quick in learning our language and could pronounce the vowels so well according to our custom, I thought it would be advisable to retain the \il{English}English pronunciation of the vowels, as this would greatly facilitate the acquirement of the language” \citep[374]{elder_letters_1932}. Such views were anathema to advocates of the linguistic mission, which placed a premium on preaching and teaching in the language of the natives. After 1830, \ia{Threlkeld, L.E.}Threlkeld was instructed to change the spelling system to that recommended by the committee charged with management of his second mission at Ebenezer (T.C. Harrington to \ia{Threlkeld, L.E.}Threlkeld, 13 May 1830) \citep[252--253]{gunson_australian_1974b}.

\subsection{Specimens of the Language (published 1827)}

After deciding on a system of spelling, \ia{Threlkeld, L.E.}Threlkeld’s next task was to accumulate vocabulary and sentences to build up a model of the language. To the \is{London Missionary Society}LMS he announced his intention of continuing to progress his work with the language and of delivering new instalments as each section was completed. He was as good as his word and, by September of \citeyear{threlkeld_orthography_1825}, had completed, in manuscript, “Specimens of the Language of the Aborigines”; this was published with his Report for \citeyear{threlkeld_specimens_1827} two years later. Dismayed with the many challenges to the mission, \citet{threlkeld_specimens_1827} lamented that the “Specimens” were “all that could be obtained for encouragement at this time”.

“Specimens” was a relatively simple production, and \ia{Threlkeld, L.E.}Threlkeld was right about its limitations. Nevertheless, it reflects his commitment to absorbing the language in the field rather than trying to force it to comply with European grammars. \citeauthor{threlkeld_specimens_1827}’s Preface (\citeyear[iii]{threlkeld_specimens_1827}) explains that he was not attempting to make a “speculative arrangement of grammar”, but simply to choose, out of some fifteen hundred examples, the “most satisfactory”. At this stage, \ia{Threlkeld, L.E.}Threlkeld did not consider that he was ready to begin the work of conversion but was more concerned to demonstrate the capacity of the language. The Preface makes reference to “my Black teacher”, which suggests that some of the content may have been obtained from \ia{Biraban}Biraban. Until he could master the language, more spiritual progress would have to wait: “To attempt instruction before I can argue with them as men, would be injurious, because Christianity does not make its votaries mere machines, but teaches them how to give an answer to every one that asketh” \citep[iv]{threlkeld_specimens_1827}. In accordance with this dictum, the model sentences all concern secular matters, grouped into four sections: pronouns, interrogative sentences (What? What is this? Why does she cry there? On account of the corpse she is crying), imperative sentences (stop, remain, be still, let it be, do not strike her) and a final page with “specimens of the different tenses of the verb”. The places named include Newcastle (\textit{Mulubinbah}) and Sydney (though not Lake Macquarie) and an unidentified location called \textit{Pahmi}, but not scriptural placenames. Similarly, the individuals named in the sentences were living people, including \textit{Berahbahn} (Eagle Hawk), \textit{Bumburukahn}, Patty, Dismal and Bun, as well as Europeans, \ia{Threlkeld, L.E.}Threlkeld and Mr Brooks, but there were no scriptural names. That was for the future.

\section{“Ebenezer” government mission, 1831--1841}

Except for his tentative progress in acquiring the language, \ia{Threlkeld, L.E.}Threlkeld’s first mission was a more or less unqualified disaster. Following his repeated failure to stay within moderate financial constraints or to recognise the authority of their agent, \ia{Marsden, Samuel}Samuel Marsden, the \is{London Missionary Society}LMS withdrew its financial support in 1828. \ia{Threlkeld, L.E.}Threlkeld dated the termination of his association with \is{London Missionary Society}the Society from 20 October 1829 \citep[114]{gunson_australian_1974a}. From this time until 1841, \ia{Threlkeld, L.E.}Threlkeld’s mission was funded by the colonial government, with the support of the established Church of England, on \ia{Threlkeld, L.E.}Threlkeld’s own land grant. In 1831, the Colonial Secretary advised that the government would allow him to retain his four convict servants, with allowance for their clothing and rations and an additional salary of £150 \citep[115]{gunson_australian_1974a}. At last, he had some financial security and relative independence.

The second mission began on 29 December 1831, the day \ia{Threlkeld, L.E.}Threlkeld moved from the old mission site on the east side of Lake Macquarie to the new site on the west side \citep[115]{gunson_australian_1974a}. \ia{Threlkeld, L.E.}Threlkeld named it “Ebenezer”, invoking the Ebenezer stone of 1 Samuel 7:12, a verse made popular through Robert Robinson’s 1758 hymn “Come, Thou Fount of Every Blessing”.\footnote{The verse continues: “Here I raise my Ebenezer/ Hither by Thy help I’ve come/ And I hope by thy good pleasure/ Safely to arrive at home.”} It was a popular name for mission stations, including the Moravian mission near Lake Hindmarsh in Victoria and the Rhenish Society mission in the Cape Colony, which was visited by \ia{Backhouse, James}Backhouse and Walker \citep[59, 71]{backhouse_extracts_1840}. It might also have recalled, for the Threlkelds, the Ebenezer Chapel on Cattai Creek, where they first met.

From the beginning, the new mission re-affirmed an emphasis on the linguistic project but broadened its remit to the needs of the established church. The mission was given more sympathetic support, which included a committee made up of Archdeacon W.G. Broughton (1788--1853), Assistant Colonial Secretary T.C. Harington (1798--1863) and the Rev. William Cowper (1778--1858), meeting in the office of the short-lived Church and School Corporation.\footnote{The charter of the Church and School Corporation, which provided generous support to the Church of England in the colony, was abolished in 1833. On his arrival in the colony, Bishop Broughton was forced to preside over a much reduced provision for the Church, which affected both missions to the Aboriginal people and education \citep[57--58]{kaye_conflict_2009}.} One of the committee’s first acts was to direct \ia{Threlkeld, L.E.}Threlkeld to adopt a new spelling system, modelled substantially on that already in use by \is{Church Missionary Society}CMS missionaries in New Zealand and the Pacific -- the very model so opposed by \ia{Marsden, Samuel}Marsden (T.C. Harington to \ia{Threlkeld, L.E.}Threlkeld, 13 May 1830) \citep[252--253]{gunson_australian_1974b}. \ia{Threlkeld, L.E.}Threlkeld happily complied. The Committee’s oversight had other consequences. In his First Report to Archdeacon Broughton, \ia{Threlkeld, L.E.}Threlkeld agreed to suspend work on scripture to complete a new commission for a selection of prayers from the \is{Anglicanism}Anglican liturgy and “an history from the old testament for the use of the Aborigines” (\ia{Threlkeld, L.E.}Threlkeld to Broughton, 9 January 1832) \citep[115]{gunson_australian_1974a}. The translation of the prayers was completed by 1834 \citep{threlkeld_selection_1834}, however the Old Testament readings do not appear to have survived. The Ebenezer mission was therefore the site for the production of all of Threlkeld’s major grammatical studies as well as his substantial achievements in translating scripture.

\subsection{An Australian Grammar (1834)}

In relatively quick succession, using the new spelling system, \ia{Threlkeld, L.E.}Threlkeld published \textit{An Australian Grammar} (\citeyear{threlkeld_australian_1834}) and \textit{An Australian Spelling Book} (\citeyear{threlkeld_australian_1836}). In the \textit{Grammar}, \ia{Threlkeld, L.E.}Threlkeld fulsomely thanked his new patron, Archdeacon Brougton, for “these first fruits of labour under your auspices”, and \is{London Missionary Society}the Society for Promoting Christian Knowledge for subsidising the costs of publication. He also explained why he had abandoned \il{English}the “English” orthography in favour of the “Polynesian”, with some modifications \citep[vii]{threlkeld_australian_1834}. \ia{Threlkeld, L.E.}Threlkeld alerted readers to the differences between languages across the continent, but also suggested that all would prove related in some way. He also included a section \is{Barbarisms}on “barbarisms” -- words that in his opinion had been introduced by sailors and, “in the use of which both black and whites labour under the mistaken idea, that each one is conversing in the others language” \citep[xi]{threlkeld_australian_1834}.\footnote{The words include \textit{boojery} ‘good’, \textit{bail} ‘no’, \textit{boge} ‘bathe’, \textit{boomering} ‘a weapon’, \textit{jin} ‘a wife’, \textit{kangaroo} ‘an animal’, etc.} As with \textit{Specimens}, the first two parts of the Grammar are secular and pragmatic. Part III is different and includes ethnographic details about spirit beings, including \textit{Ko-in}, \textit{Tip-pa-kal},\footnote{Written \textit{Tip-pa-hal} in the \textit{Grammar}. This is undoubtedly a misprint.} \textit{Pór-ráng}, names of sacred places, such as \textit{Pór-ro-bung} (name of a Mystic Ring), and common places, such as \textit{Bo-un} (on Wallis’s Plains), \textit{But-ta-ba}, (name of a hill on the margin of the lake), \textit{Mu-lu-bin-ba} (the site of Newcastle). There is also an extensive vocabulary naming birds, plants, animals, objects, parts of the body and verbs. The linguistic examples are more systematic than in the \textit{Specimens} and arranged to demonstrate the different uses of the verb. For the first time, the selections indicate that \citet[121]{threlkeld_australian_1834} was attempting to preach and teach in the newly acquired language: “Who made the sun? Jehovah did”, or: “He did good, some were blind, he made them to see”, but these are considerably outnumbered by more prosaic matters, or by \citeauthor{threlkeld_australian_1834}’s (\citeyear[128]{threlkeld_australian_1834}) no doubt frequent appeal to \ia{Biraban}Biraban to continue the interminable lessons: “Speak to me in the black’s language”, he might say, or “Stay, stay, that I may have some conversation”.

\ia{Threlkeld, L.E.}Threlkeld was meticulous in informing the highest levels of colonial government of the progress of both his linguistic efforts and the mission. He sent a copy of the \textit{Spelling Book} to Governor Richard Bourke, who had succeeded Ralph Darling in December 1831 and would remain until 1837.\footnote{For acknowledgement, see G.K. Holden to \ia{Threlkeld, L.E.}Threlkeld, 6 June 1836 , H. Watson Parker to \ia{Threlkeld, L.E.}Threlkeld, 20 May 1840 \citep[39--42]{threlkeld_threlkeld_1822-1862}.} He forwarded copies of the mission reports to Sir John Franklin in Van Diemen’s Land in 1839, and to Sir George and Lady Gipps in 1840. Simultaneously, he continued to work on his translations of the Gospel of Luke, completed in 1831, Mark, completed in 1837, and the unfinished Matthew (begun in 1837).

\subsection{Australian Spelling Book (1836)}

The \textit{Spelling Book} is quite different to \ia{Threlkeld, L.E.}Threlkeld’s earlier publications because it was intended to be used not by other missionaries or linguists but by native speakers of \il{Hunter River Lake Macquarie language (HRLM)}HRLM, in order to teach them how to read. For this reason, letters, words and sentences (apart from the guide to pronunciation) are listed without translation. In his report from Lake Macquarie in 1827, \ia{Threlkeld, L.E.}Threlkeld indicates that he had a small class of eight Aboriginal children and had been trying to teach them “the Alphabet in their own language” \citep[96]{gunson_australian_1974a}, but that he was unable to keep them with him because of the competing attractions of Newcastle. The \textit{Spelling Book} provided the means to teach literacy, though it is doubtful if any children ever learnt to read in this way. As \citet[126]{threlkeld_journal_1828-1846} reported in his diary, \ia{Biraban}Biraban began to learn the vowels in 1831, but otherwise made no progress in literacy in his own language.

Like other primers or introductory readers, the \textit{Spelling Book} begins with consonants, vowels and numbers (sections 1--3) before progressing to words (4--5) and short phrases (7). There are then a series of lessons, beginning with short sentences from scripture and progressing to lengthier passages. While it seems unusual today for an introduction to reading to be taken exclusively from the Bible, this was the normal expectation of elementary reading material for all children. It is also an indication of the progress that \ia{Threlkeld, L.E.}Threlkeld was making in the language, and in his ultimate objective of translating the entire Bible. Selections come from all parts of the Bible, the Old and New Testaments, the Prophets and the Acts of the Apostles. It also suggests which texts \ia{Threlkeld, L.E.}Threlkeld found potentially the most useful in introducing Christian concepts to the subjects of the mission.

As a primer, or first reader, the \textit{Spelling Book} had its limitations. The nine sections or \textit{winta}, follow a progression which is based primarily on an outline of the Christian catechism, rather than a selection based on easy introduction to the challenges of reading and writing. Hence, the first section, titled \textit{Eloi} (‘God’), includes texts on the love and creative power of the one God, and the second outlines his might and mystery: for God “all things are possible” (Matthew 1:26). The third section is headed \textit{Pirriwul} (\textit{piriwal} ‘chief, king’)\footnote{\il{Hunter River Lake Macquarie language (HRLM)}HRLM words in italics are \ia{Threlkeld, L.E.}Threlkeld’s original forms; those in brackets are taken from the word lists in \citet[113--166]{lissarrague_salvage_2006}, using the latter’s phonemic orthography, slightly adapted. For example, Lissarrague’s \textit{piR[i]wal} has been simplified, above, to \textit{piriwal}. Such adaptations have been necessary for contemporary pedagogic purposes, as reflected in \ia{Edgar-Jones, Sharon}Edgar-Jones and \ia{Burgman, Albert}Burgman’s \il{Wanarruwa| see {Hunter River Lake Macquarie language (HRLM)}}\textit{Wanarruwa Beginner’s Guide} (\citeyear{edgar-jones_wanarruwa_2019}). I thank \ia{Wafer, James}James Wafer for providing the transliteration into modern orthography of the examples in these footnotes.} and concerns Jesus Christ, “who is Lord of all” (Acts 10:36), with further texts in the next section on the need to call on God, who “is nigh unto all them that call upon him” (Psalm 145:18); a “good shepherd” (Psalm 23:1) who is “full of compassion and merciful” (James 5:11). Winta 5, entitled \textit{Ngoro} (\textit{nguru} ‘three, third’) concerns the Trinity, which must have been rather confusing for neophyte readers; and winta 6, on the Holy Spirit (\textit{Marai Yirriyirri}), includes the warning against “him that blasphemeth against the Holy Spirit” (Luke 12:10). The first engaging narrative which might appeal to children comes in winta 7, which translates the nativity story and the coming of the Messiah from Luke 2:9. Winta 8 has stern texts on the Atonement and Judgment to come, “for if God spared not the angels” (2 Peter 2:4); but the final lesson, winta 9, is taken from Matthew 6:9, and it is the Lord’s Prayer.

While the restlessness of his young mission charges meant that little if any of this carefully structured pedagogic program is likely to have hit home, the lessons did have another purpose. By preparing these translations into \il{Hunter River Lake Macquarie language (HRLM)}HRLM on topics connected with central Christian themes, \ia{Threlkeld, L.E.}Threlkeld was obliged to make necessary decisions about which \il{Hunter River Lake Macquarie language (HRLM)}HRLM words to use for notions such as God, Lord, angel, heaven and hell, spirit and soul. In some cases, he was forced to retreat and adopt loan words, such as \textit{Eloi} or \textit{Jehova} for God, \textit{Tartarus} for hell, \textit{Angelo} for angel, or \textit{Shépu} for sheep. For other concepts, he found a suitable \il{Hunter River Lake Macquarie language (HRLM)}HRLM word, including \textit{moroko} for heaven (\textit{muruku} ‘heaven, sky’), \textit{Ngolomulli kan} for Saviour,\footnote{In the current orthography this would be \textit{ngulumalikan}, which could be interpreted as ‘one who perfects’. \ia{Threlkeld, L.E.}Threlkeld appears never to have provided a gloss of the verb \textit{nguluma}.} \textit{Marai Yirriyirri} for Holy Spirit (\textit{yiririri} ‘sacred’, \textit{maray} ‘soul, spirit’) and \textit{Pirriwul} for Lord (\textit{piriwal} ‘chief, king’). These were all necessary for the grand project still to come, the translation of whole books of scripture.

\subsection{Selection of Prayers (1834)}

The removal of the support of the \is{London Missionary Society}LMS and its replacement with the colonial government required a shift in the focus of the mission. \ia{Threlkeld, L.E.}Threlkeld placed a renewed and intensified emphasis on the linguistic mission, but the priorities of the established church saw a lesser emphasis on \textit{scriptura sola} (‘scripture alone’) as the mode of instruction and preaching. The authorities of the established church wanted there to be translations of prayers and the \is{Anglicanism}Anglican liturgy, as well as passages from the historical books of the Old Testament which were welcome by Christians of all persuasions. \ia{Threlkeld, L.E.}Threlkeld duly complied.

In his First Report to Archdeacon Broughton on 9 January 1832, \ia{Threlkeld, L.E.}Threlkeld advised that he had set aside the final revision of St Luke, “until a manual of selected prayers from the Liturgy, and an history from the old testament for the use of the Aborigines shall be completed” \citep[115]{gunson_australian_1974a}. Broughton also wished that the linguistic work pioneered by \ia{Threlkeld, L.E.}Threlkeld should be made available to the missionaries of the \is{Anglicanism}Anglican \is{Church Missionary Society}Church Missionary Society (CMS) at Wellington Valley. In a letter dated 6 August 1832, the archdeacon requested “such parts of your grammatical Introduction to the Aboriginal Language as you may have completed” to be given to William Watson and Johann Christian Sebastian Handt. \ia{Threlkeld, L.E.}Threlkeld obliged, but when the sheets were returned, he was no doubt irritated to find that four sheets had been retained at Wellington Valley \citep[116]{gunson_australian_1974a}. That \ia{Threlkeld, L.E.}Threlkeld could be difficult is all too evident; however, he was generous and supportive with his work on the language, and consistently ensured that copies were distributed as widely as possible, not only to patrons in the church and colonial government but also to public libraries and learned associations.\footnote{Among the \ia{Threlkeld, L.E.}Threlkeld Papers at the SLNSW (\citeyear{threlkeld_threlkeld_1822-1862}), there are acknowledgements for these.} This was despite the fact that conflict over who deserved intellectual credit for the work of translation and the high prestige of being the first “apostle” to translate the gospel into a new language marred the Wellington Valley mission, as it did the work of translation elsewhere.

The Selection of Prayers for the Morning and Evening from the Service of the Church of England was completed in manuscript by 1834 (SLNSW A 1446). Like the scripture selections at the end of the \textit{Spelling Book}, this compilation consists of continuous prose, with the original from the \textit{Book of Common Prayer} provided at the rear of the manuscript. It is not clear if \ia{Threlkeld, L.E.}Threlkeld or any other minister ever used these prayers in a liturgical setting. Broughton was a conservative high churchman and so, unsurprisingly, the texts he selected for translation show little in the way of innovation, or concession to his presumed Aboriginal hearers. They open with a text from 1 John 1 on the need for confession.\footnote{1 John 1: “If we say that we have no sin we deceive ourselves and the truth is not in us: but if we confess our sins God is faithful and just to forgive us our sins and to cleave us from all unrighteousness.”} This is followed by the General Confession from the Book of Common Prayer of 1662, then morning and evening prayers, with the usual conclusion.\footnote{“The peace of God, which passeth all understanding, keep your hearts and minds in the knowledge and love of God, and of his Son Jesus Christ our Lord; and the blessing of God Almighty, the Father, the Son, and the Holy Ghost, be amongst you, and remain with you always. \textit{Amen}.”}

\subsection{Old Testament stories (before 1834)}

Also at Broughton’s suggestion, \ia{Threlkeld, L.E.}Threlkeld completed translations of a number of Old Testament stories, intended as reading lessons. Although this work is currently untraceable, \citet{threlkeld_annual_1836} provided a relatively detailed description of its contents in his 6\textsuperscript{th} Annual Report for Broughton, a copy of which he forwarded to the \is{London Missionary Society}LMS.\footnote{Also published by \citet[115]{gunson_australian_1974a}.} In giving an account of his method of teaching, he notes that sections of the Old Testament, “had been translated also to form reading lessons” as well as “means of instruction in divine truth”. They included “The Creation of the World”, “The Creation of Man and Woman”, “Institutions of Marriage”, “The Fall of Man”, “Of the Deluge”, “The Confusion of Tongues at Babel”, “Abraham interceding for Sodom and Gomorrah” and the “Destruction of Sodom and Gomorrah”. \ia{Threlkeld, L.E.}Threlkeld expressed some frustration that the message of these powerful scriptural narratives was not being heeded with sufficient attention: “Billy Blue, when riding out with my son one day in the bush, was asked what he thought of the account which was read to him? He replied, that he thought it was all gammon that master had told him about the Creation, for who was there who saw God create man!” \citep[134]{gunson_australian_1974a}.

\section{Scripture translation}

The final phase of \ia{Threlkeld, L.E.}Threlkeld’s linguistic project, the end to which all his previous studies had been directed, was the translation of scripture. There were a number of reasons why missionaries focused with such determination on the translation of the gospel. Scripture was fundamental to the work of \is{Proselytism}proselytisation, persuading individuals of the good news that was necessary to salvation and conversion to Christianity. In the British Empire, the major British missionary societies worked with the British and Foreign Bible Society (BFBS) to ensure that translations of scripture, into all the languages necessary for the global missionary movement, were available for the work of teaching and conversion to Christianity. The Bible Society recommended that translators not work alone, but accept the support of a committee of experts. Indeed, without such support it was unlikely that their work would receive endorsement or publication.\footnote{According to \citet[398]{bliss_encyclopaedia_1891}: “The general rule of the British and Foreign Bible Society on this subject is as follows: ‘That whenever it is practicable to obtain a board of competent persons to translate or revise a version of the Scriptures, it is undesirable to accept for publication the work of a single translator or reviser’.”} Equally important, translators were instructed to choose a central language when commencing their work to avoid the waste of Christian money \citep[406]{bliss_encyclopaedia_1891}.

\enlargethispage{\baselineskip}

Given the high status attached to this activity, it is surprising how few translations into any Australian indigenous language were printed prior to the late twentieth century. The \textit{Encyclopedia of Missions} (\citeyear[573]{bliss_encyclopaedia_1891}) has a lengthy article on scripture translation and an Appendix which lists Bible versions by language and geographic area. Of 269 translations listed, Australia is represented by a single word: “nothing” \citep[573]{bliss_encyclopaedia_1891}. This was not entirely accurate. In 1966, the Bible Society prepared a catalogue of all its holdings in the Oceanic languages of Australia, New Zealand, that part of Australia and New Zealand and the adjacent islands administered by Australia, and all the main island groups of the North and South Pacific \citep{dance_oceanic_1963}. Of the 712 printed scripture translations into the languages of this vast area, there were four from Australia. The earliest of these was \ia{Threlkeld, L.E.}Lancelot Threlkeld’s 1857 revised translation of the Gospel of St Luke, which was extensively edited by \ia{Fraser, John}John Fraser. \citet[xi]{fraser_australian_1892} states that “this translation remained in manuscript and had disappeared” yet he had recently discovered a copy in the Public Library of Auckland which was now published for the first time. \ia{Fraser, John}Fraser also had his edition of \ia{Threlkeld, L.E.}Threlkeld’s \textit{Luke} separately printed for private circulation. A copy of the offprint edition was donated to the Bible Society by \ia{Fraser, John}Fraser himself and is Chapter Two of this volume. In a letter dated 28 May 1892, which accompanies the copy in the Bible Society’s collection, he wrote:

\begin{quote}
     From the title page herein enclosed you will perceive that our N.S. Wales Government has just published a volume on the Australian Languages. Part of that volume is the Gospel by St Luke in an Australian Dialect. As this part of the volume is quite unique and is likely to continue to be so, I have got a few copies of it printed separately, and herewith send a copy for your Society’s Library, if you will do me the honour of accepting it.\footnote{\ia{Fraser, John}Fraser to Secretary, British and Foreign Bible Society, 28 May 1892. Cambridge UL, Bible Society’s Collection.}
\end{quote}

\ia{Fraser, John}Fraser’s courtesy is also probably responsible for the copy of the same offprint in the collection of the \is{London Missionary Society}London Missionary Society.

A modified scan of \ia{Fraser, John}Fraser’s edition of \ia{Threlkeld, L.E.}Threlkeld was the basis for the new edition of the \il{Hunter River Lake Macquarie language (HRLM)!Awabakal}Awabakal Gospel of St Luke published by the Bible Society in \citeyear{threlkeld_euangelion_1997}. Unfortunately, at two removes from \ia{Threlkeld, L.E.}Threlkeld’s original manuscripts, this introduced new errors, despite the best intentions of Christians from the Newcastle region who collaborated in its production.\footnote{Not all accents from \ia{Fraser, John}Fraser are reproduced, although they are important for understanding what \ia{Threlkeld, L.E.}Threlkeld intended; and a note assigns the fourth (and final) manuscript version to the collection of “Sir James Grey” \citep[7]{threlkeld_euangelion_1997}, rather than Sir George Grey, the former governor whose patronage of \ia{Threlkeld, L.E.}Threlkeld and enthusiasm for ethnographic research was significant in the development of scientific collections in the colonies of South Australia, New Zealand and South Africa -- all of which benefitted from his rule.} The 1997 edition has a significance beyond that of the original, however, as a gesture in the contemporary movement for reconciliation between Indigenous Australians and modern churches. The new edition bears a statement, signed by church leaders from across Newcastle: “This … Gospel of Luke was presented to the indigenous communities of the greater Newcastle area on Sunday, 14 December 1997 at Cullen Park, Belmont, as a symbol of our desire to be reconciled and to walk together now and in the future.”\footnote{See the Introduction \citep[3]{threlkeld_euangelion_1997}.} Below are the signatures of church leaders, listed alphabetically by their surnames, representing the Assemblies of God Churches, Churches of Christ, \is{Anglicanism}Anglican Diocese of Newcastle, Catholic Diocese of Maitland-Newcastle, Greek Orthodox of Newcastle, Seventh-day Adventist Church, \is{Presbyterianism}Presbyterian Church of Australia, Lutheran Church of Australia, Baptist Union of NSW, Uniting Church in Australia, and the Salvation Army. Similar moves have been made by churches throughout Australia, but this one was especially powerful, not least because it created a bond linking the modern churches with the first major missionary work.

As the longest extant written work in any of the Aboriginal languages of south-eastern Australia and the first translation of any book of scripture, \ia{Biraban}Biraban and \ia{Threlkeld, L.E.}Threlkeld’s translation of the Gospel of St Luke has iconic significance for Indigenous communities in Newcastle, Lake Macquarie and the Hunter River region, for historians of \is{Linguistics!Missionary linguistics}missionary linguistics and for all Australian Christians. It is unique, and likely to grow in status as other \is{Language revitalisation}language revitalisation projects continue.\footnote{It is difficult to know if the translated Gospel of St Luke would have been capable of being read by a native speaker. \citet[2]{gunson_australian_1974a} estimated its accuracy at “about fifty per cent”, though he does not explain how he arrived at this figure.} The next section traces the development of translation as it progressed from manuscript to print.

\subsection{Gospel of St Luke (1831, 1832, 1857)}

Working closely with \ia{Biraban}Biraban, \ia{Threlkeld, L.E.}Threlkeld’s first efforts to translate the Gospel of Luke were presented in a Circular Report to the \is{London Missionary Society}LMS on 8 October 1828 \citep[100--101]{gunson_australian_1974a}. Using basically the \textit{Specimens} orthography of 1827, he provided a translation of Luke 7:11--12, which gives an account of Jesus coming to Nain and raising a young man from the dead. This passage is provided with an interlinear gloss in which at least some of the morphemes are treated as distinct components. \ia{Threlkeld, L.E.}Threlkeld’s comprehension of the linguistic structure of the language was advancing, and he was clearly reluctant to accept that the translation was acceptable if he could not parse every word. In a letter to Archdeacon Broughton on 18 July 1829, he was pleased that: “I am now as far as the 8\textsuperscript{th} of St Luke’s Gospel, which gospel I hope to render into their language this year” \citep[106]{gunson_australian_1974a}. Unfortunately, he seems to have finally worn out \ia{Biraban}Biraban’s patience with the work of translation and he lamented that he was often without any natives at the mission, “especially the one who assists in the language” \citep[106]{gunson_australian_1974a}.

\enlargethispage{\baselineskip}

By October, Threlkeld was able to report to \ia{Marsden, Samuel}Marsden (Threlkeld to \ia{Marsden, Samuel}Marsden, 26 October 1829) \citep[106]{gunson_australian_1974a}, his nemesis, that he was engaged “in a rough translation of St Luke” and had completed fourteen chapters, after which he would be equipped to teach the blacks in their own language. The Gospel of Luke was a weapon in the missionary’s campaign to continue with translation work despite \ia{Marsden, Samuel}Marsden’s scepticism and the withdrawal of funding: “I do not perceive it my duty to abandon the long neglected Blacks of this country possessing the knowledge I have already attained” \citep[107]{gunson_australian_1974a}. He made the same claim to the Directors of the \is{London Missionary Society}London Missionary Society in a letter written the same day: “I am engaged of instructing the Aborigines of N.S. Wales, in this vicinity, in the truths of the Gospel in their own tongue”, though only claiming to have completed as far as the 11\textsuperscript{th} chapter rather than thirteen (\ia{Threlkeld, L.E.}Threlkeld to Messrs Hankey and Orme, 29 October 1829) \citep[107]{gunson_australian_1974a}.\footnote{In a letter to Governor Ralph Darling on 26 October 1829 the number of chapters was also given as 14 \citep[107]{gunson_australian_1974a}.}

It was possibly something of a relief when Archdeacon Broughton suggested to \ia{Threlkeld, L.E.}Threlkeld that he put aside the translation of Luke and undertake some minor translating projects instead (\ia{Threlkeld, L.E.}Threlkeld to Broughton, 9 January 1832) \citep[115]{gunson_australian_1974a}. By the time of his Second Report to Broughton on 21 January 1833, he could observe much better progress: “The Gospel of St Luke … requires only some slight alteration in doubling consonants \&c in order to accord with the rules which are found to arise in the natural construction of the language” (\ia{Threlkeld, L.E.}Threlkeld to Broughton, 21 January 1833) \citep[117]{gunson_australian_1974a}. Since the Aboriginal people were continuing to avoid the mission, it might also be suspected that progress depended to some extent on the contribution of Threlkeld’s son, Joseph, “who speaks fluently the native tongue” \citep[117]{gunson_australian_1974a}. The following year, in his Report dated 28 December 1833, \ia{Threlkeld, L.E.}Threlkeld claimed the work was all but complete, simply awaiting final correction \citep[119]{gunson_australian_1974a}. The following year he advised the Colonial Secretary that he had been using the texts to preach to “small parties of the natives, sometimes in the open air, sometimes in my barn” (\ia{Threlkeld, L.E.}Threlkeld to Alexander M’cleay, 7 November 1834) \citep[120]{gunson_australian_1974a}. By the next year, \ia{Threlkeld, L.E.}Threlkeld listed the Gospel of St Luke as among the main “subjects” of his missionary activity, but without further annotation (\ia{Threlkeld, L.E.}Threlkeld 5\textsuperscript{th} Report, 2 December 1835) \citep[123]{gunson_australian_1974a}.\footnote{The other “subjects” are accompanied by such comments as “under revisal”, “In manuscript” and “In progress”.}

The choice of Luke’s gospel is an interesting one from the point of view of \ia{Threlkeld, L.E.}Threlkeld’s mission. Of the three synoptic gospels (Matthew, Mark and Luke),\footnote{The three Synoptic Gospels of Matthew, Mark and Luke follow a similar narrative frame which differs from the fourth Gospel of John. For a solo translator, such as \ia{Threlkeld, L.E.}Threlkeld, it was considerably less work to translate the synoptics than John.} it provides the longest account of the birth and childhood of Christ. It also has the only versions of a number of important parables and miracles, including the Good Samaritan (Luke 10:25--37), the Rich Fool (12:13--21), the Prodigal Son (15:11--32), Lazarus and the Rich Man (16:19--31), and the Pharisee and the Tax Collector (18:9--14). The theme of these parables is that wealthy elites face a challenge in meeting the strict ethical conditions of Christian teaching, with its obligations to the poor and those excluded from society. \ia{Threlkeld, L.E.}Threlkeld lived his life in accordance with these virtues, repeatedly demanding justice for those at the margins of society, such as the Aboriginal people, or those oppressed by such examples of an unfeeling and mercenary bureaucracy as his superiors in the \is{London Missionary Society}London Missionary Society or the colonial establishment in New South Wales.

Luke’s gospel also provided a series of useful stories which could be incorporated into elementary instruction in literacy or preaching in the native language. The key method for either spontaneous preaching or a sermon within a formal religious service was to select a text and then explain its significance to the people. When preaching to the American Indians, David Brainerd, one of \ia{Threlkeld, L.E.}Threlkeld’s heroes, frequently chose texts from Luke \citep[300, 317, 328, 330]{edwards_life_1826}, including Luke 13:24--28: “Divine truths fell with weight and power upon the audience, and seemed to reach the hearts of man” \citep[330]{edwards_life_1826}. Without a reliable translation of scripture, it was not possible to preach with authority. Beyond its direct importance for \is{Proselytism}proselytism, extracts from the Bible formed the basis for most primary instruction for the poor. Instruction in the Bible was not just for native students in mission schools. Both the major systems of primary education, the \is{Lancasterian system}Lancasterian system of the Quaker Joseph Lancaster (1778--1838), favoured by \is{Non-conformism}Non-conformists such as \ia{Threlkeld, L.E.}Threlkeld, and the \is{Bell system| see {Madras system}}Bell or \is{Madras system}Madras system of the \is{Episcopalianism}Scottish Episcopalian Andrew Bell (1753--1832), used scripture extracts as the foundation for elementary reading and writing.

\subsection{Gospel of St Luke: The manuscripts}

There are two known manuscripts of \ia{Threlkeld, L.E.}Threlkeld’s translation of Luke’s gospel: one in the State Library of New South Wales (MLMSS A1325), dated 1831; and a second version of 1857, made at the request of Sir George Grey, which is now in the Grey Collection at the Auckland Central City Library (GMS 83). Also in Auckland, there is a Lexicon for the Gospel of St Luke (GMS 82), which was never finished but sent posthumously to Sir George Grey following \ia{Threlkeld, L.E.}Threlkeld’s death. All three manuscripts were written in \ia{Threlkeld, L.E.}Threlkeld’s neat, copperplate handwriting. Though prepared “for the press”, they were not published in his lifetime, but appeared in print, for the first time, in 1892, in editions prepared by the Maitland schoolmaster \ia{Fraser, John}John Fraser.

For the stages in the preparation of this major translation project, we are largely reliant on \ia{Threlkeld, L.E.}Threlkeld’s meticulous dating in the original manuscripts, and on the account given in his \textit{Australian Reminiscences}, published in \textit{Christian Keepsake} in 1835. At that time, \ia{Threlkeld, L.E.}Threlkeld claimed he had completed three revisals of the text, and that it only awaited the completion of the grammar and orthography before it could be submitted to the press \citep[42]{gunson_australian_1974a}. According to his annotation to the copy in the State Library of New South Wales (A1325), he began a second revision on 10 January 1831 and finished it eight months later, in August of the same year. The final revision was undertaken at the request of Sir George Grey and completed on Saturday, 7 November 1857 \citep[72, n. 4]{gunson_australian_1974a}.\footnote{If \ia{Threlkeld, L.E.}Threlkeld had completed three revisions by 1835, and the 1831 MS held by SLNSW (A1325) is the one he refers to as his “second” revision, there must have been a third produced between those two dates. This would make the 1857 version in the Grey collection the \textit{fourth} revision. The lack of manuscripts for the “first” and “third” revisions suggests either that they no longer exist (or are perhaps still to be located), or else that the numerical sequencing is not intended to be taken too literally. Allowance may need to be made for partially corrected drafts and similar interim provisions. I thank \ia{Wafer, Jim}Jim Wafer for this observation.}

\subsection{Gospel of Mark (1837)}

Having completed Luke’s gospel, \ia{Threlkeld, L.E.}Threlkeld wrote, in his Report of 1836 for Bish\-op Broughton, that he had begun translating Mark, “after which I propose Matthew and John, which with Luke already accomplished will complete the Evangelists”. Significantly, he no longer suggested that the revisions would be made with the assistance of \ia{Biraban}Biraban, but rather of his son, “from the superior knowledge he has acquired of the \il{Hunter River Lake Macquarie language (HRLM)}Aboriginal language” \citep[134]{gunson_australian_1974a}.

\ia{Threlkeld, L.E.}Threlkeld completed his \textit{Evangelion Mark-úmba} (‘Gospel of Mark’)\footnote{Full title: \textit{Unni ta Evangelion Ngiakai Yitirrir Tóttóng-pittul-mulli-ka-ne Jesu-úmba Krist-ko-ba Upatoara ta Mark-úmba} (literally, ‘This is the Gospel called the Joyous News of Jesus Christ written by Mark’).} in \citeyear{threlkeld_gospel_1837}. It was a substantial project, which took him 283 numbered pages to complete. \ia{Threlkeld, L.E.}Threlkeld evidently worked on the translation for six months -- though there are many signs that he did not consider it adequate. The date on the top of the first page is Tuesday, December 13, 1836. The final date, at the end of the sixteenth chapter, is 23 June 1837. \ia{Threlkeld, L.E.}Threlkeld’s Mark was therefore begun about five years after his translation of Luke, the first complete version of which was produced in 1831. It appears that \ia{Threlkeld, L.E.}Threlkeld was unsatisfied with the Mark translation, and there are many indications in the sole surviving manuscript that it was still in a relatively incomplete state. These include numerous underscored words, with marginal annotations indicating alternative readings or concepts for which he was unable to find an appropriate term in \il{Hunter River Lake Macquarie language (HRLM)}HRLM. The alternative readings often give the impression that \ia{Threlkeld, L.E.}Threlkeld was reserving a set of queries to put to \ia{Biraban}Biraban, his principal informant. Not all the queries are answered. Perhaps \ia{Threlkeld, L.E.}Threlkeld was unable to make an appointment with \ia{Biraban}Biraban or another native speaker. Some of these unanswered queries relate to problematic words or concepts.

There is an indication of the limitations of the translation in some challenging passages, for example, Mark 15:16, where three words are simply adopted from the KJV, which has: “And the soldiers led him away into the hall, called Praetorium”. \ia{Threlkeld, L.E.}Threlkeld seems not to have been able to find a suitable translation for “soldier” or “hall” or “Praetorium”, so they all appear in the verse. The next verse has “purple” and “Crown”, also undigested, though in this case both terms are underscored and marked with a query, so it is clear that \ia{Threlkeld, L.E.}Threlkeld was aware that it would be better to find some other way to translate them. While the difficulties of finding or creating words for concepts or objects utterly beyond the experience of an Indigenous hunter-gatherer people cannot be minimised, perhaps some effort might have been made to adapt local ideas. Is there no food or fragrant substance, for example, that might have been substituted for “myrrh” or “wine” (both Mark in 15:23)? In the text, \ia{Threlkeld, L.E.}Threlkeld often places alternative readings within angle brackets. Thus, even in this imperfect version of his final completed translation, \ia{Threlkeld, L.E.}Threlkeld reveals himself to be a painstaking linguist -- one who, aware of his deficiencies, attempted to set targets for improvement by consultation, probably with a native speaker.

In a facing comment to Mark 3:4, \ia{Threlkeld, L.E.}Threlkeld wrote, “In Luke it is muroko tin to [p. 40].” This internal evidence supports the surmise, indicated also by the dates on the manuscripts, that, before he turned to Mark, \ia{Threlkeld, L.E.}Threlkeld had revised Luke to a point where he could treat it as a reliable standard. The choice of Luke as a first translation project is noteworthy, because Mark, the shortest of the gospels, is generally the earliest to be translated in missionary endeavours. Luke, on the other hand, the longest gospel and the one, it is generally believed, written by the same writer as Acts, has the most challenging theology and narrative.

\subsection{Gospel of Matthew (1834--)}

In 1834, \citeauthor{threlkeld_australian_1834} noted in his \textit{Australian Grammar} that he had “just commenced” his work on Matthew, and also that he was continuing his instruction of “two native youths in writing and reading their own language”. In addition, he was providing reading lessons selected from the Old Testament, presumably from his translations of Old Testament stories (now missing) and from the structured lessons in his \textit{Australian Spelling Book}.

The manuscript of the Matthew translation is actually dated “July 10\textsuperscript{th} 1837”, and the title page indicates that the work (or, at least, this version of it) was begun at Kurri Kurri. This is the only surviving draft, sketched out as far as Matthew 4:19, with the same kinds of queries and annotations as found in the Mark manuscript. The lack of other versions suggests that \ia{Threlkeld, L.E.}Threlkeld never completed this last major translation project. He did, however, finish and publish an analytical work on \il{Hunter River Lake Macquarie language (HRLM)}HRLM in \citeyear{threlkeld_key_1850}, after the mission had closed and he had relocated to Sydney. This is called \textit{A Key to the Structure of the \il{Hunter River Lake Macquarie language (HRLM)}Aboriginal Language}.

\section{Linguistic note on the Hunter River Lake Macquarie language (\il{Hunter River Lake Macquarie language (HRLM)}HRLM)}

It seems unlikely that the \il{Hunter River Lake Macquarie language (HRLM)}Hunter River Lake Macquarie language had a name, other than that associating it with the region in which it was spoken. In the title of his translation of the Gospel of St Luke, \ia{Threlkeld, L.E.}Threlkeld calls it the language of “the Aborigines, located in the vicinity of Hunter’s River, Lake Macquarie \& tc., New South Wales”. The “\& tc.” hints that the language boundaries were not fixed with any certainty, at least as far as \ia{Threlkeld, L.E.}Threlkeld was able to discern. Given the immense disruption created by \is{Colonisation!European colonisation}European colonisation, leading to population collapse together with the near \is{Cultural destruction}destruction of the Aboriginal language and culture in this region, this is hardly surprising. The same regional identification is followed in the title of \ia{Threlkeld, L.E.}Threlkeld’s \il{Hunter River Lake Macquarie language (HRLM)}HRLM grammar (\citeyear{threlkeld_australian_1834}) and spelling book (\citeyear{threlkeld_australian_1836}). For his edition, the Scottish-born schoolteacher, John \citet{fraser_australian_1892} named it “An Australian language as spoken by the \il{Hunter River Lake Macquarie language (HRLM)!Awabakal}Awabakal people the people of Awaba or Lake Macquarie (near Newcastle, New South Wales)”. \il{Hunter River Lake Macquarie language (HRLM)}HRLM is the term adopted by \citeauthor{lissarrague_salvage_2006} for her \textit{Salvage Grammar} (\citeyear{lissarrague_salvage_2006}) and has the advantage of being concise and reflecting the naming practice of those with access to native speakers. It is used throughout this volume.

There were at least three distinct dialects of \il{Hunter River Lake Macquarie language (HRLM)}HRLM for which a large number of names have been recorded, with widely varied spelling. Where \il{Hunter River Lake Macquarie language (HRLM)}HRLM was once spoken, contemporary Aboriginal communities identify themselves by these dialect names, \il{Hunter River Lake Macquarie language (HRLM)}including \il{Hunter River Lake Macquarie language (HRLM)!Awabakal}“Awabakal” for that spoken at Lake Macquarie and by \ia{Biraban}Biraban, \il{Hunter River Lake Macquarie language (HRLM)!Kurringgai}Kurringgai for the southern dialect, \il{Hunter River Lake Macquarie language (HRLM)!Wonnarua}Wonnarua and \il{Hunter River Lake Macquarie language (HRLM)!Worimi}Worimi for the Hunter River dialect, and possibly, \il{Hunter River Lake Macquarie language (HRLM)!Geawegal}Geawegal for that on the upper reaches of the Hunter River (see \hyperref[sec:Map_2]{Map 2 in Appendix C}). Note that these dialect names and regions differ from the named local land councils recognised by the New South Wales Aboriginal Land Council.

The territory covered by speakers of the dialects of \il{Hunter River Lake Macquarie language (HRLM)}HRLM was extensive and appears to have stretched along the Pacific coast from the Hunter River in the north, across the Central Coast, and ending somewhere north of Sydney. In a report to the NSW Legislative Council in 1838, \ia{Threlkeld, L.E.}Threlkeld suggested that the language extended inland along the Hunter River for about sixty miles. This is a region of approximately 22,048 square kilometres (8,500 square miles) and today includes some of the most productive land in Australia, incorporating the Hunter Valley wineries and coal fields, and one of the world’s most important centres for thoroughbred horse breeding and training. For size comparison, the ancestral land of the speakers of \il{Hunter River Lake Macquarie language (HRLM)}HRLM was more than 25 per cent larger than the traditional boundaries of Yorkshire, which at 15,000 square kilometres (6,000 square miles) was England’s largest county. There is some uncertainty about where the language boundaries of \il{Hunter River Lake Macquarie language (HRLM)}HRLM and its various dialects should be placed, but other languages were spoken both around the modern city of Sydney and beyond the Hunter River and there are likely to have been zones where \is{Bilingualism}bilingualism and mutual intelligibility was common. \citet[12--14]{lissarrague_salvage_2006} discusses the sources and evidence for the extent and distribution of the language, referring to previous linguistic studies by \is{Linguistics!Colonial linguistics}colonial linguists and missionaries such as \citet{hale_languages_1846}, \citet{fraser_australian_1892} and Müller. Given the status of \il{Hunter River Lake Macquarie language (HRLM)}HRLM as one of the best recorded languages on the east coast of the Australian continent, now the home to approximately 350,000 people on the Central Coast, and 680,000 in the Hunter region, including the city of Newcastle, it is surprising how few specialist studies there have been of the original language. It remains a region where many Indigenous Australians choose to live. According to the Australian Bureau of Statistics, the 2021 Census enumerated over 23,000 Aboriginal and Torres Strait Islander people in the Hunter Valley (excluding Newcastle); for the Newcastle and Lake Macquarie region the number was just under 20,000, while for the NSW Central Coast the number was 17,000. Of this 60,000, it is unknown how many trace their ancestors to the Hunter River Lake Macquarie area, but Aboriginal and Torres Strait people make up a large and culturally engaged community. There are active efforts at \is{Language revitalisation}language revival.

\il{Hunter River Lake Macquarie language (HRLM)}HRLM belongs to the Pama-Nyungan family of Australian languages, along with more than 300 of the more than 400 languages of Australia identified by Ethnologue, the online archive of the Summer Institute of Linguistics. Of the 400, 224 have survived with living speakers while 190, including \il{Hunter River Lake Macquarie language (HRLM)}HRLM, are \is{Extinct languages}extinct. \ia{Threlkeld, L.E.}Threlkeld found it challenging to learn HRLM when compared with his previous experience acquiring \il{Raiatean}Raiatean. However, he does not ascribe this to any innate complexity of the language which in some ways provides an inviting opportunity for \is{Language revival| see {Language revitalisation}}\is{Language revitalisation}language revival. \il{Hunter River Lake Macquarie language (HRLM)}HRLM has a relative straightforward phonology for \il{English}English speakers, including two rhotics and just three vowels. There is a useful beginner’s guide to learning Wanarruwa (ie, \il{Hunter River Lake Macquarie language (HRLM)}HRLM) by \ia{Edgar-Jones, Sharon}Sharon Edgar-Jones, \ia{Burgman, Albert}Albert Burgman and \ia{Wafer, Jim}Jim Wafer (\citeyear{edgar-jones_wanarruwa_2019}). This follows the orthography and grammatical guidelines summarised by \citet{lissarrague_salvage_2006}, who is also the source for the precis of the language which follows.

In terms of grammar, \il{Hunter River Lake Macquarie language (HRLM)}HRLM is a suffixing language, with variations in the meaning of core morphemes placed at the end of words. It has both independent pronouns, as in \il{English}English, as well as markers for person (first, second and dual) and case. It is highly \is{Inflection}inflected, with nouns and adjectives forming nine cases: absolutive, ergative, instrumental, perlative, locative, allative 1, allative 2, ablative, causal, dative and genitive. Mysteriously, \citet[10]{threlkeld_australian_1834} recognised seven \is{Declension| see {Inflection}}\is{Inflection}declensions, which do not exist, and just seven cases, all of them encountered in traditional grammars of \il{Greek}Greek and \il{Latin}Latin: genitive, dative, acquisitive, vocative, and ablative.

\citet[28--74]{threlkeld_selection_1834} had great difficulty determining the structure of the verb in \il{Hunter River Lake Macquarie language (HRLM)}HRLM, possibly because of the challenge of identifying verb roots, a common feature of other Aboriginal languages. He claimed to identify fifteen “kinds” of verbs, possibly because he was looking to discover separate categories of active and passive verbs, which occur in \il{Latin}Latin and \il{Greek}Greek, but not in \il{Hunter River Lake Macquarie language (HRLM)}HRLM: 1. active transitive; 2. active intransitive; 3. active transitive reciprocal; 4. continuative; 5. causative by permission, or preventive with a negative;  6. causative by personal agency, or, those which denote the exertion of personal energy to produce the effect upon the object; 7. causative by instrumental agency; 8. effective, or, those which denote an immediate effect produced by the agent on the object; 9. neuter, or, those which describe the quality, state, or existence of a thing; 10. double, or those that denote an increase in the state, or, quality, or energy; 11. privative, or those which denote the absence of some property; 12. imminent, or those which denote a readiness to be or to do; 13. inceptive, or those which describe the state as actually beginning to exist; 14. iterative, or, those which denote a repetition of the state or action; 15. spontaneous, or, those which denote an act of the agent’s own accord. There are no \is{Conjugation| see {Inflection}}\is{Inflection}conjugations in \il{Hunter River Lake Macquarie language (HRLM)}HRLM (unlike \il{Latin}Latin which has four), so this curious list may be the result of \ia{Threlkeld, L.E.}Threlkeld’s determination to press the matter in his grammatical conversations with \ia{Biraban}Biraban. \ia{Threlkeld, L.E.}Threlkeld goes on to identify six tenses, all familiar from traditional grammars: 1. present; 2. preter-perfect 3. perfect past aorist; 4. pluperfect; 5. future definite, and 6. future aorist. He also identified three moods: imperative, permissive and subjunctive. 

To deal with the variety of verbs in \il{Hunter River Lake Macquarie language (HRLM)}HRLM, \citet{lissarrague_salvage_2006} describes eleven tenses: present (habitual), present (concurrent), future (near), future (general), past (general), past (distant), past (recent), imperative, hortative, irrealist, hypothetical, and apprehensional, two aspects (continuous, iterative), and two voices (reflexive, reciprocal). There are also a number of derivational suffixes, which \ia{Threlkeld, L.E.}Threlkeld found particularly confusing, for intransitivity, verbalising and causation. Then there are “clitics”, or unstressed bound \is{Pseudo-words}pseudo-words, which are common in \il{English}English, but were not clearly identified by \ia{Threlkeld, L.E.}Threlkeld either.

\citet[120--125]{wafer_waiting_2011} noted some of the problematic aspects of \is{Linguistics!Missionary linguistics}missionary linguistics, especially, as is the case with the \ia{Threlkeld, L.E.}Threlkeld/\ia{Biraban}Biraban translations, where the only evidence of an \is{Extinct languages}extinct language are scripture translations, grammars and word lists created with the intention of evangelisation. In the absence of native speakers, or speakers of closely related \is{Cognate languages}cognate languages, it is highly challenging to determine what reliance can be placed on translations as evidence of the living languages for which they have become, effectively, the mausoleum. We asked whether, despite the numerous obstacles, it was possible to extract additional linguistic information from these sources. On the basis of one grammatical example, we concluded that \ia{Threlkeld, L.E.}Threlkeld appears to have had more intuition about the functioning of some aspects of \il{Hunter River Lake Macquarie language (HRLM)}HRLM than might be supposed from his formal works of grammar, such as the \textit{Australian Grammar} (\citeyear{threlkeld_australian_1834}).

This might be demonstrated from the use of the clitic \textit{=pa}. \citet[76]{threlkeld_australian_1834} defined \textit{Ba} as an adverb of time meaning ‘when, at the time that’, and that it must always be followed by \textit{Ngai-ya} meaning ‘then, at that time governed by the following particle’. However, this is a very limited explanation of its full range of meaning and uses. According to \citet[9]{anderson_what_2005}, simple clitics are “unaccented variants of free morphemes, which may be phonologically reduced and subordinated to a neighbouring word.”  Using examples from the \ia{Lissarrague, Amanda}Lissarrague database of \il{Hunter River Lake Macquarie language (HRLM)}HRLM texts, which include Threlkeld’s \textit{Specimens of the Language} (\citeyear{threlkeld_specimens_1824}, \citeyear{threlkeld_specimens_1827}), \textit{Grammar} (\citeyear{threlkeld_australian_1834}), and \textit{Key} (\citeyear{threlkeld_key_1850}), but not the Gospel of St Luke or other scripture translations, \ia{Wafer, James}Wafer suggested that it is possible to identify \ia{Threlkeld, L.E.}Threlkeld’s mistakes in the use of \textit{=pa}, especially in his own compositions rather than elicitations from \ia{Biraban}Biraban. Despite some slips, \ia{Threlkeld, L.E.}Threlkeld’s knowledge of the language was “surprisingly good” and reflected the kinds of error expected of any second language learner \citep[132]{carey_lancelot_2010}. He was able to identify and deploy all the major use of this clitic \textit{=pa}, despite his inability to define these uses in his conventional grammar. Deeper analysis will require the attention of a trained linguist, using modern editions of \il{Hunter River Lake Macquarie language (HRLM)}HRLM scripture.

Beyond its linguistic repertoire, the \il{Hunter River Lake Macquarie language (HRLM)}HRLM gospels are a rich resource for investigating translation strategies deployed in the colonial era for “fixing” a previously unwritten language. They demonstrate the extent to which missionary translators resorted to neologisms, archaisms and borrowing to expand unwritten languages for scripture. As noted above, for his \textit{Spelling Book} (\citeyear{threlkeld_australian_1836}), \ia{Threlkeld, L.E.}Threlkeld had already made key decisions about which words to use for terms and concepts for which there was no \il{Hunter River Lake Macquarie language (HRLM)}HRLM equivalent and did his best to avoid simply importing the relevant \il{Hebrew}Hebrew, \il{Greek}Greek, \il{Latin}Latin or \il{English}English word. This was in line with the policy of the British and Foreign Bible Society, which asserted that all languages were fit vehicles for scripture, including those which were in unwritten form. This was a marked departure from pre-modern scripture translation, where scripture was invariably translated into the prestige written language of the civil power, including \il{Latin}Latin and \il{Greek}Greek in the ancient and medieval world, and \il{Spanish}Spanish, \il{German}German and \il{English}English in the age of European empires. 

The following table, using examples from the \il{Hunter River Lake Macquarie language (HRLM)}HRLM Gospel of St Luke suggests some of the solutions which \ia{Threlkeld, L.E.}Threlkeld and \ia{Biraban}Biraban found to these problems (see \hyperref[tab:1.1]{Table 1.1}). All are taken from \ia{Threlkeld, L.E.}Threlkeld’s unfinished “Lexicon to the Gospel according to Saint Luke”, sent in \citeyear{threlkeld_aboriginal_1859} to Sir George Grey by Threlkeld’s son following his father’s death and now in the Grey Collection of Auckland City Library \citep{threlkeld_aboriginal_1859}. \ia{Threlkeld, L.E.}Threlkeld provided this explanation of the use of the Lexicon, originally intended to support his two grammatical studies, which were exhibited in London at the Royal Exhibition of 1851:

\begin{quote}
     This Lexicon will only refer to words used in the Gospel of the Apostle Saint Luke, but for the exemplification of those tenses and cases as may not be used therein reference must be made to the Australian Grammar, and to the Key to the Structure of the \il{Hunter River Lake Macquarie language (HRLM)}Aboriginal Language.
\end{quote}

As he had done for the manuscript of the Gospel of St Luke, \citet[fol. 5]{threlkeld_aboriginal_1859} inserted a portrait of \ia{Biraban}Biraban in the front of the Lexicon, “[a]s a tribute of respect to the departed worth of M’Gill, the intelligent Aborigines, whose valuable assistance enabled me to overcome very many difficulties in the Language much sooner than otherwise would have been accomplished”. As the following examples show, \ia{Threlkeld, L.E.}Threlkeld favoured borrowing from \il{Greek}Greek and \il{Hebrew}Hebrew for the terms used to translate religious concepts or beings (Beelzebub, Eloi, Jehovah). He also transliterated letters which challenged native speakers, including all sibilants and vowels other than a, o and i. For alien plants, food, and natural phenomena, he retained the \il{English}English, as he did for numbers. He left scriptural names largely unchanged, except where they included letters which were not part of \il{Hunter River Lake Macquarie language (HRLM)}HRLM. The large number of borrowings from \il{Greek}Greek, as well as \il{English}English, suggest \ia{Threlkeld, L.E.}Threlkeld wished to avoid the appearance of over reliance on \il{English}English words. For new \il{Hunter River Lake Macquarie language (HRLM)}HRLM readers, who may have known \il{English}English, it would involve considerable additional memorisation. For evil characters, he drew on traditional \il{Hunter River Lake Macquarie language (HRLM)}HRLM words for traditional healers and sorcerers (\textit{Ka-ra-kul}). 

Like other missionary translators, \ia{Threlkeld, L.E.}Threlkeld was sensitive to the risk of using indigenous terms for traditional spiritual concepts at odds with Christian beliefs and theology. For example, he explains that \textit{Ma-mu-ya} (‘a ghost’) should not be used to translate the Holy Ghost, because this word referred to the ghost of a dead person, not a living spirit being, whether of God or Man. It was less challenging to find words to translate moral concepts, which would appear to cross cultural lines. Hence, \textit{Man-ki-ye} (‘one who habitually takes’), \textit{Man-ki-yi-kara} (‘do not steal’), and even \textit{Man-ki-ye-nukung-ka} (‘a rapist’), along with \textit{Ma-ta-ye} (‘a glutton’). The proper term for a living spirit \textit{Ma-rai}, is used for various terms, including that for the soul. There are words for feeling and emotion, ranging from angry (\textit{Buk-ka-kay-ke}), to meditative (\textit{Min-ki}), the word for a fellow feeling, or a feeling of sympathy, compassion or penitence. Finally, from the word for being ready (\textit{Mi-ring-il}) comes a term for one who takes charge, a Saviour (\textit{Mi-ro-mul-li-kan}).

\enlargethispage{\baselineskip}

\begin{table}
     \label{tab:1.1}
     \caption{Selected terms and definitions from \citeauthor{threlkeld_aboriginal_1859}’s Lexicon of St Luke (\citeyear{threlkeld_aboriginal_1859}).}
\begin{tabular}{p{3.6cm} p{8cm}}
     \lsptoprule
     Lexicon of St Luke & Gloss\\
     \midrule
     \textit{Abel} & English\il{English}, Abel\\
     \textit{Abel-umba} & Of, belonging to Abel\\
     \textit{Alphai} & From the \il{Greek}Greek Alpheus\\
     \textit{Basileo} & From the \il{Greek}Greek. Kingdom\\
     \textit{Beelzebub} & From the \il{English}English, Beelzebub\\
     \textit{Bing-ai!} & An affectionate address to a Brother. Brother!\\
     \textit{Buk-ka-kay-ke} & To be in an angry, wrathful, savage state, an avenger\\
     \textit{Bum-bum} & A reduplication. Kiss, Kissing\\
     \textit{Bum-bung-ngul-li-ko} & For to take a kiss by force\\
     \textit{Bun-kil-li-ngel} & The place of smiting. The threshing floor. The pugilistic ring. The field of battle.\\
\end{tabular}
\end{table}

\newpage

\begin{longtable}{p{3.6cm} p{8cm}}
     \textit{Calf} & English\il{English}, Calf\\
     \textit{Centurion} & English\il{English}, Centurion\\
     \textit{David} & Da-bid. From the \il{Greek}Greek. David. The Aborigines do not pronounce either V, or F, generally substituting B for V as above, and P for F as Pish for Fish.\\
     \textit{De-bil-de-bil} & A reduplication. Insensitive. A term used for some evil being of whom the Aborigines, are exceedingly afraid.\\
     \textit{Elia} & English\il{English} from Elias\\
     \textit{El-o-i} & From the \il{Hebrew}Hebrew Eloim, God.\\
     \textit{Evangelion} & From the \il{Greek}Greek, Evangelion. The Gospel\\
     \textit{Gentile} & English\il{English}, Gentile\\
     \textit{Grammatece} & From the \il{Greek}Greek Scribes\\
     \textit{Jehova} & English\il{English}, Jehovah\\
     \textit{I-e-ro, I-e-ron} & From the \il{Greek}Greek Temple\\
     \textit{Jesu, from the \il{English}English}\\
     \textit{Jesou, from the \il{Greek}Greek} & Jesus\\
     \textit{Jew, \il{English}English} & Jew\\
     \textit{Kai-bung} & Light, of any kind, Lamp, Candle\\
     \textit{Kai-bung-ngel} & The place of the light, as the Candlestick, Lamp\\
     \textit{Kaisara}\\
     \textit{Kaisa} & From the \il{Greek}Greek, Caesar\\
     \textit{Kaisa-um-ba-ta} & It is that which belongs to Caesar. That which is Caesar’s\\
     \textit{Ka-ra-kul} & One who cures by charms; A sorcerer, a Doctor; a Physician; One who belongs to the Faculty of charmers.\\
     \textit{Ka-u-mul-li-ngel} & The place where the gathering together is caused. The place of assembly; the council chamber; the parliament house.\\
     \textit{Kau-wul-lo ko-na-ra} & A great multitude; a mob, or an agent. Did, does, or will do, according to the tense of the verb\\
     \textit{Ka-va-na} & From the \il{English}English Governor\\
     \textit{Lepro} & From the \il{English}English Leprosy\\
     \textit{Ma-bo-ngun} & A widow\\
     \textit{Mai-ya} & A maker; a serpent; the serpent genus\\
     \textit{Mal-ma} & Lightning\\
     \textit{Ma-mu-ya} & A Ghost, the spirit of a departed person, not the spirit of a living person which is \textit{Marai}, which see [below]. It would be highly improper to say \textit{Mamuya} \textit{yirriyirri}, literally the Holy Ghost, to convey our idea of the Holy Ghost because the term would mean to an Aborigine the Holy Ghost of some dead person, whereas \textit{Marai} \textit{yarriyurri} means the Holy Spirit of some living being, whether of God or Man.\\
     \textit{Man-ki-ye} & One who habitually takes; an habitual taker; a thief.\\
     \textit{Man-ki-yi-kara} & Prohibitory, do not steal; do not take; do not receive.\\
     \textit{Man-ki-ye-nu-king-ka} & A taker of woman; a woman stealer; an adulterer\\
     \textit{Ma-rai} & Spirit; Soul, of a living being, not a Ghost. \\
     \textit{Ma-rai-kan} & One who is a spirit; being spirit; possessed of a spirit; having a spirit.\\
     \textit{Ma-rai mu-run-ba} & Spirits belonging to you; your spirits; your souls\\
     \textit{Ma-ta-ye} & One habitually given to greediness\\
     \textit{Min-ki} & A fellow feeling; a something within a person, a sensation; the feeling of sympathy, sorrow, compassion, penitence, patience, repentance, pondering.\\
     \textit{Min-ki-kan} & One who sympathizes; feels; repents etc. A Penitent; being penitent\\
     \textit{Mi-ring-il} & Ready; prepared to remove; to go on a journey; to battle\\
     \textit{Mi-ro-mul-li-kan} & One who takes charge of; to take care of; to watch over; to keep; to save from harm; A Saviour.\\
     \lspbottomrule
\end{longtable}

Further study of the terms devised for scripture translation will potentially broaden knowledge about the emotional register of \il{Hunter River Lake Macquarie language (HRLM)}HRLM, the range of conversations it facilitated, and its capacity to cross the cultural void between coloniser and colonised.

\section{Conclusion}

Towards the end of his time at Lake Macquarie, \ia{Threlkeld, L.E.}Threlkeld seems to have been overwhelmed by pessimism about the fate of the language and its people. The mission was closed at the end of 1841, and he had moved to Sydney by the end of 1843. But even as late as 1837, \ia{Biraban}Biraban (McGill) and other speakers of the language had not entirely abandoned the mission. From \ia{Threlkeld, L.E.}Threlkeld’s lengthy 7\textsuperscript{th} Report, dated 30 December 1837 and addressed to the Colonial Secretary, it appears that \ia{Biraban}McGill had once again returned to the mission, not to be evangelised but to work: “We have now \ia{Biraban}M’Gill and his tribe employed at a job of Burning off for which 6 of them receive daily rations of Tea, sugar, tobacco, Flour and beef, with the promise of Clothes when the work is completed” \citep[140]{gunson_australian_1974a}. With this Report, \ia{Threlkeld, L.E.}Threlkeld included a printed copy of the 15\textsuperscript{th} Chapter of the Gospel According to St Luke.

Work on translation of the other gospels had not entirely ceased. \ia{Threlkeld, L.E.}Threlkeld’s 8\textsuperscript{th} Report (1838) listed the Gospel of St Mark as completed and in manuscript, which indicates that the work of translation was at least under contemplation. Unfortunately, neither Bishop Broughton in Sydney, nor the SPCK in London supported further financial subvention to the project. In a letter of December 1834 to the SPCK, Broughton wrote that he did recommend publication of the first of \ia{Threlkeld, L.E.}Threlkeld’s gospel translations:

\begin{quote}
     Although the translation of St Luke’s Gospel into the \il{Hunter River Lake Macquarie language (HRLM)}aboriginal language afforded a most gratifying proof of the industry and ingenuity of Mr \ia{Threlkeld, L.E.}Threlkeld and I had every reason to be in my own mind satisfied of the accuracy of the version … I could not but be sensible of the impossibility of subjecting it to such a scrutiny and test, as would be necessary to afford general assurance that the work conveyed an accurate representation of the original. \citep[172, n. 41]{gunson_australian_1974a}
\end{quote}

To his contemporaries, it seemed there was no scholarly or evangelistic reason for issuing \ia{Biraban}Biraban and \ia{Threlkeld, L.E.}Threlkeld’s various scripture translations in print. Today the situation is rather different. In the Hunter Valley, Newcastle and Lake Macquarie region, there is an active \is{Language revitalisation}language revitalisation movement underway. This has generated teaching materials including a \textit{Beginners’ Guide} to the language \citep{edgar-jones_wanarruwa_2019}. The present historical introduction is intended to support this movement. Hopefully, this will lead to the publication of a modern edition or editions of the Gospel of St Luke, Gospel of St Mark, and Gospel of St Matthew. There are a number of reasons to anticipate this future scholarship. Modern editions will expand the known vocabulary of the \il{Hunter River Lake Macquarie language (HRLM)}HRLM language and make possible a more nuanced understanding of its structure. \citet{steele_awabakal_2024} has provided the basis for ongoing research of this kind.

Perhaps the friendly working relationship between \ia{Biraban}Biraban and his missionary can serve to inspire co-operation and mutual respect among these diverse parties and future collaborations between Indigenous Australians, historians, ethnographers and historians of \is{Linguistics!Missionary linguistics}missionary linguistics.
