\chapter*{Appendix A: Threlkeld manuscripts and early printed editions}
\addcontentsline{toc}{chapter}{Appendix A: Threlkeld manuscripts and early printed editions}
\label{Appendix_A}
\markboth{Appendix A: Threlkeld manuscripts and early printed editions}{}

\ia{Threlkeld, L.E.}Threlkeld was an early adopter of the printing press for missionary purposes and was quick to print copies of important documents, including his Instructions from the \is{London Missionary Society}LMS, Reports on the Mission and, later, his own complaint about his treatment. He also prepared high quality copies, in his own neat handwriting, of his linguistic manuscripts, and distributed copies of them to patrons as well as to the \is{London Missionary Society}LMS. He was less successful in securing a publisher for his scripture translations, though first the SPCK and later the NSW government underwrote the publication of his \textit{Spelling Book} and the \textit{Key} to the grammar. Manuscript copies of his correspondence, journals and linguistic studies are held by a number of repositories, including the London School of Oriental and African Studies (for the \is{London Missionary Society}LMS), the State Library of New South Wales and the National Library of Australia. The SLSNSW manuscripts have been microfilmed and, more recently, digitised. Manuscripts held in UK repositories were microfilmed as part of the Australian Joint Copying Project, and have now been digitised. The University of Newcastle NSW has also been active in digitising works held in other collections.

This list includes, first, repositories, with shelf marks for original manuscripts, identifiers for AJCP and SLNSW microfilms and hyperlinks for digitised manu\-scripts and printed works.

\section*{Manuscript repositories}
\addcontentsline{toc}{section}{Manuscript repositories}

Auckland, Central City Library, Grey manuscripts

\begin{enumerate}
    \item \href{https://kura.aucklandlibraries.govt.nz/digital/collection/manuscripts/id/13511/rec/2}{GMS 82.} An Aboriginal and English Lexicon to the Gospel according to Saint Luke, in which the Letters, Words and Phrases occurring in this Gos\-pel are distinctly explained. By L.E. Threlkeld. Sydney New South Wales, 1859.
    \item \href{https://kura.aucklandlibraries.govt.nz/digital/collection/manuscripts/id/13367/rec/1}{GMS 83.} Evangelion unni ta Jesu-um-ba Christ-ko-ba Upatoara Louka-\linebreak{}umba = Gospel according to Saint Luke. Translated into the Language of the Aborigines, located in the vicinity of the Hunter’s River, Lake Macquarie, New South Wales, in the year 1831, and further revised by the translator, L.E. Threlkeld, Minister, 1857. Illuminated by Annie Layard.
    \item GL T15.5--7. Letters to Sir George Grey from L.E. Threlkeld.
    \item NZMS 575 Letter to Sir George Grey concerning Threlkeld’s translation of St Luke and his \textit{Australian Spelling Book}.
\end{enumerate}

London, School of Oriental and African Studies

\begin{enumerate}
    \item London, School of Oriental and African Studies. London Missionary Society AJCP, M1--116; M698--670. “The Orthography and Orthoepy of a dialect of the Aborigines of New South Wales, Part 1, by L.E. Threlkeld, Missionary from the London Missionary Society, Newcastle September 1825”, LMS, Australian Incoming, box 2/ folder 2.
    \item London Missionary Society, “South Sea Mission” Papers, c. 1800--1915. One volume of miscellaneous letters and papers compiled at an unknown date. Includes translation of letter to Threlkeld from Queen Pomare ML MSS A381: CY 877.
\end{enumerate}

Sydney, State Library of New South Wales

\begin{enumerate}
    \item A 382 Threlkeld Papers, 1822--1862. A382, Microfilm: CY 820 (frames 777--909); digitised \href{https://transcripts.sl.nsw.gov.au/document/382-reverend-lancelot-edward-threlkeld-papers-1822-1862}{SLNSW} (original).
    \item A 1325 [Threlkeld] Gospel of St Luke, 1831. A 1325, Microfilm: CY 3110 (fr. 1--186).
    \item A 1446 Broughton, W. G. and Threlkeld, L. E. A selection of prayers for the Aborigines of Australia, 1834. A1446, Microfilm: CY 2214 (fr. 76--91); digitised \href{https://transcripts.sl.nsw.gov.au/document/1446-selection-prayers-aborigines-1834-translated-northumberland-dialect-l-e-threlkeld}{SLNSW} (original); University of Newcastle (\href{https://downloads.newcastle.edu.au/library/cultural%20collections/pdf/1834-prayers.pdf}{pdf of AJCP microfilm}).
    \item At 15/ folder 2. Papers relating to Threlkeld and the British and Foreign Bible Society, 26 July 1831. Includes draft letters from the BFBS to Threlkeld and draft letter to Rev. S. Marsden relating to Threlkeld’s Translation of St. Luke’s Gospel.
    \item DCL238193. “Incomplete MS by Rev. L.E. Threlkeld of St Matthew’s Gospel in the Awabakal language 1837”, SLNSW MS DCL238193.
    \item MLMSS 2111/1--2 Threlkeld Papers, 1817--71, with The Gospel of St. Mark translated into the language of the Lake Macquarie Aboriginal peoples, 1837. ML MSS. 2111/1--2, Microfilm: CY 341 (fr. 723--935); digitised \href{https://transcripts.sl.nsw.gov.au/document/series-02-gospel-st-mark-translated-language-lake-macquarie-aborigines-1837}{SLNSW} (original ms of Gospel of St Mark).
    \item MLMSS 3729 Threlkeld diary and loose papers, 1857--59. ML MSS.3729, Microfilm: CY 854 (fr. 544--692).
    \item MSMSS 4464 Threlkeld Family Papers, 1808--1860, 1902, 1980. Partial transcription.
    \item MLMSS 7527 Threlkeld Letters from Thomas Arndell and other papers, 1824--40, 1902. Acquired in February 2005. Letters written by Thomas Arndell to Rev. L.E. Threlkeld, 1824--40. Subjects include farming, prices of stock and produce, family matters and bushrangers. Includes Threlkeld’s marriage certificate, 1824. CY 4494.
    \item MLMSS 9441 Threlkeld Papers, 1834--1838.
    \item Microfilm MAV/FM4/6217 Threlkeld Journals, c. December 1828--c. February 1846. Microfilm: Journal 1828--1846 (frames 1--173), Loose letters (frames 174--209), Digitised University of Newcastle (\href{https://hunterlivinghistories.com/2008/02/20/lost-threlkeld-manuscript-online/}{pdf of SLNSW microfilm}). \linebreak{}[Original MS in private hands.]
    \item Microfilm MAV/FM4/1626--1627 Sir Thomas Brisbane Papers, 1812--1837. Includes: “Miscellaneous papers”, The Orthography and Orthoepy of a dialect of the Aborigine of New South Wales by L.E. Threlkeld, part 1, Newcastle, 1825, Manuscript. [Original in National Library of Australia.]
\end{enumerate}

Canberra, National Library of Australia

\begin{enumerate}
    \item MS 4036 Papers of Sir Thomas Brisbane, L.E. Threlkeld “Orthography and Orthoepy of a dialect of the Aborigines of New South Wales, Part 1., Sep\-tember 1825”. \href{http://nla.gov.au/nla.obj-233244754/findingaid#nla-obj-426047906}{Guide to the Papers of Sir Thomas Brisbane}. [Another ms copy was sent to the LMS.]
\end{enumerate}

\newpage

\section*{Printed editions (with links to digitised versions)}
\addcontentsline{toc}{section}{Printed editions (with links to digitised versions)}

Trove, National Library of Australia

\begin{enumerate}
    \item Threlkeld 1827. \href{https://nla.gov.au/nla.obj-1711000/view}{Specimens of a dialect, of the Aborigines of New South Wales: being the first attempt to form their speech into a written Language} (Sydney: “Monitor Office” by Arthur Hill) Call number: FC F1147.
    \item Threlkeld 1828. \href{http://nla.gov.au/nla.obj-41049134/view?partId=nla.obj-41049142}{A Statement Chiefly Relation to the Formation and Abandonment of a Mission to the Aborigines of New South Wales} (Sydney: R.Howe) Call number: mc N 1475. [This copy includes the annotations by LMS Deputation’s George Bennet.]
    \item Threlkeld 1834. \href{https://nla.gov.au/nla.obj-115132947/view?partId=nla.obj-484920396}{An Australian Grammar: Comprehending the Principles and Natural Rules of the Language, as Spoken by the Aborigines in the Vicinity of Hunter’s River, Lake Macquarie, etc. New South Wales} (Sydney: pr. Stephens and Stokes) Call number FC F1858.
    \item Threlkeld 1836. \href{https://nla.gov.au/nla.obj-52763467/view?partId=nla.obj-88016030}{An Australian Spelling Book in the Language as Spoken by the Aborigines in the Vicinity of the Hunter’s River, Lake Macquarie, New South Wales} (Sydney: [s.n.]) Call number: FRM F2192.
\end{enumerate}

\noindent University of Newcastle Library, Cultural Collections

\begin{enumerate}
    \item Fraser 1892. \href{https://downloads.newcastle.edu.au/library/cultural%20collections/pdf/al1892whole.pdf}{An Australian Language as spoken by the Awabakal the people of Awaba or Lake Macquarie (near Newcastle, New South Wales) being an account of their Language, Traditions, and Customs: by L. E. Threlkeld. Re-arranged, condensed, and edited, with an Appendix, by John Fraser.} Sydney: pr. Charles Potter, 1892.
\end{enumerate}

\section*{Websites and digitisation projects (manuscripts and printed editions)}
\addcontentsline{toc}{section}{Websites and digitisation projects (manuscripts and printed editions)}

\noindent \href{https://downloads.newcastle.edu.au/library/cultural%20collections/awaba/index.html}{Awaba, University of Newcastle NSW}

\smallskip

\noindent Curated collection of historical, ethnographic and linguistic sources with interpretive essays created with the support of Indigenous communities and the Wollotuka School of Aboriginal Studies, University of Newcastle NSW. Includes first transcript of Threlkeld’s Gospel of Mark.

\bigskip \bigskip

\noindent \href{https://hunterlivinghistories.com/dreaming/}{Hunter Living History, University of Newcastle NSW}

\smallskip

\noindent Provides access to digitised manuscripts, printed words, newspapers and other documents relevant to the Hunter Region of NSW arranged chronologically. Includes links to Threlkeld’s published and manuscript works originally made to support \href{https://downloads.newcastle.edu.au/library/cultural%20collections/awaba/index.html}{Awaba}.

\bigskip

\noindent \href{https://transcripts.sl.nsw.gov.au/section/rediscovering-indigenous-languages}{Rediscovering Indigenous Languages, State Library of New South Wales}

\smallskip

\noindent This site aims to transcribe and digitise historic Aboriginal word lists, records and other documents held in the State Library. It includes three Threlkeld manu\-scripts:

\begin{enumerate}
    \item \href{https://transcripts.sl.nsw.gov.au/document/382-reverend-lancelot-edward-threlkeld-papers-1822-1862}{A 382: Reverend Lancelot Edward Threlkeld papers,}
    \item \href{https://transcripts.sl.nsw.gov.au/document/1446-selection-prayers-aborigines-1834-translated-northumberland-dialect-l-e-threlkeld}{A 1446: Selection of prayers for Aborigines, 1834;}
    \item MLMSS 2111/ folder 2 \href{https://transcripts.sl.nsw.gov.au/document/series-02-gospel-st-mark-translated-language-lake-macquarie-aborigines-1837}{Series 02: The Gospel of St Mark, translated into the language of Lake Macquarie Aborigines, 1837.}
\end{enumerate}