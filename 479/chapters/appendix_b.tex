\chapter*{Appendix B: Threlkeld’s grave in Sydney’s Rookwood Cemetery}
\addcontentsline{toc}{chapter}{Appendix B: Threlkeld’s grave in Sydney’s Rookwood Cemetery}
\addcontentsline{toc}{section}{Threlkeld’s grave in Sydney’s Rookwood Cemetery}
\label{Appendix_B}
\markboth{Appendix B: Threlkeld’s grave in Sydney’s Rookwood Cemetery}{}

Photo: Hilary Carey, 26 March 2011

\bigskip

\includegraphics[scale=0.24]{figures/Threlkelds_Grave.png}

\bigskip

\enlargethispage{\baselineskip}

The epitaph reads:

\begin{quote}
    In Memory of the \ia{Threlkeld, L.E.}Reverend Lancelot Edward Threlkeld M.E.S., born in the city of London October 20\textsuperscript{th} 1788, ordained in 1815. He laboured at Tahiti and Raiatea, a coadjutor of the \ia{Williams, John}Reverend John Williams in connection with the \is{London Missionary Society}London Missionary Society until 1824 when he undertook a mission to the Aborigines at Lake Macquarie, of whose language he wrote a grammar and a lexicon. Appointed Chaplain of Seamen at this port in 1845, he became Minister of the Mariners Church erected chiefly through his exertions, where, on the last day of his life he twice preached the gospel from 22\textsuperscript{nd} chapter of St Luke 15\textsuperscript{th} and 16\textsuperscript{th} verses and 1\textsuperscript{st} chapter of Romans 16\textsuperscript{th} verse and retiring from the pulpit to his chamber slept in Christ Sunday 10\textsuperscript{th} October 1859, aged 71 years.
\end{quote}

\begin{quote}
    \textit{O that without a lingering groan \\
    I may the welcome word receive \\
    My body with my charge lay down \\
    And cease at once to work and live} \\
\end{quote}

The dedication chosen by Threlkeld’s family for his grave indicates their wish to show that his life reflected the highest standards of duty expected of an evangelical Christian. The same verse was chosen by \ia{Crowther, Jonathan}Jonathan Crowther to illustrate the exemplary character of the death of John Wesley, who also lived to old age and died, “without a lingering groan”: “His death”, wrote \citet[207]{crowther_portraiture_1815}, “was an admirable close of so laborious and useful a life.” The ideal was, for Wesley, a life of constant motion, work and usefulness. The verse came from Hymn 45 of John and Charles Wesley’s \textit{Methodist Hymn Book}. The third verse, “O that without a lingering groan”, was based on Numbers xx. 28 and was often given up by Wesley himself at the conclusion of Society meetings, to remind members to frequently reflect on the likely event of their own death, for which they should be ready at any moment \citep[39]{stevenson_methodist_1870}.