\chapter{Diskussion}\label{K7}

%------------------------------------------------------------------------

Das vorangegangene Kapitel war der Analyse der verschiedenen Studiendaten gewidmet. Nach der Betrachtung der subjektiven Angaben im Rahmen der Umfragen und der Eye-Tracking-Daten werden im Folgenden nun die Ergebnisse der einzelnen Studienteile kontrastiert und in Bezug sowohl zu bestehenden Forschungsergebnissen als auch zu der aufbereiteten Theorie gesetzt. Die Diskussion folgt dabei der bereits bestehenden linearen Struktur des vorangegangenen Analyseteils und verweist innerhalb der jeweiligen Unterkapitel (Abschnitt~\ref{K7:subsec:fix-count}–\ref{K7:para:sacdur}, S.\,\pageref{K7:subsec:fix-count}\,ff.) auf vergleichbare Studien aus dem aktuellen Forschungsumfeld. Entlang dieses Weges werden auch immer wieder die zu Beginn dieser Arbeit präsentierten theoretischen Grundlagen aufgegriffen. Das Kapitel endet mit einem Überblick über weiterführende Denkanstöße, praktische Schlussfolgerungen sowie der kritischen Auseinandersetzung mit dem Studienaufbau (Abschnitt~\ref{K7:sec:BlindeFlecke}, S.\,\pageref{K7:sec:BlindeFlecke}). 

%------------------------------------------------------------------------

\section{Merkmale der CvK an Skype und dem Translator}
\label{K7:sec:Skype-CMC}

%------------------------------------------------------------------------

Im vorausgehenden Kapitel wurde die Chatkommunikation über Skype jeweils einmal mit und einmal ohne Vermittlung durch den Skype Translator analysiert. Die Studienteilnehmer{\textperiodcentered}innen beider Teile stammen aus einer Zielgruppe, die nativ mit den Vorzügen dieser Kommunikationstechnologie\is{Kommunikation!-stechnologie} aufgewachsen ist. Das Durchschnittsalter der Teilnehmer{\textperiodcentered}innen in den drei Erhebungen liegt bei etwa 24 Jahren und somit in einer Generation, die von der Sozialforschung als \emph{Generation Y} eingegrenzt wird. Diese Personengruppe umfasst in etwa die Jahrgänge 1985--1999 und zeichnet sich u.\,a. durch einen intuitiven Umgang mit und zugleich intensiven bis exzessiven Konsum von Medien (in diesem Fall sind besonders die im Verlauf dieser Arbeit thematisierten Kommunikationsmedien wie Skype, WhatsApp u.\,ä.\ gemeint) aus.

Weiterhin wird dieser Generation nachgesagt, mehr Zeit und Geld in Bildung zu investieren als alle anderen. Interpretiert man hierbei die Fremdsprachenkenntnisse als möglichen Indikator, so zeichnen sich die Personen in den Erhebungen nicht nur formell über die offiziellen Abschlüsse als gebildet aus. Damit einher gehen auch ein oder mehrere Auslandsaufenthalte von durchschnittlich etwa 12 Monaten, die häufig im Rahmen eines Studiums oder bereits nach dem Schulabschluss absolviert werden. Die Mehrheit gibt Deutsch als Muttersprache an -- ein Kriterium, das für die Rekrutierung der {Proband{\textperiodcentered}innen} zwingend erfüllt werden musste --, jedoch beherrschen die meisten Personen mindestens eine, wenn nicht sogar mehrere weitere Fremdsprachen. Dies ist zunächst das Englische, aber auch die in der deutschen Schullandschaft weit verbreiteten Französisch und Spanisch sind häufig vertreten. 

In allen Umfrageergebnissen fällt auf, dass Skype nicht mehr die erste Wahl für die Kommunikation über das Internet ist. Die überwiegende Mehrheit der Teilnehmer{\textperiodcentered}innen an der Online-Umfrage (144 Personen) nutzt Skype demnach lediglich weniger als monatlich. Die gleichen Verhältnisse sind in den Eingangsfragebögen beider Studienteile wiederzufinden. Über die vergangenen Jahre hat die Anwendung harte Konkurrenz zu jeder Funktion aus dem Portfolio erhalten. Hiervon zeugen sowohl die Angaben in der Online-Umfrage als auch die der Proband{\textperiodcentered}innen in beiden Versuchsaufbauten. Auch wenn Skype lange Zeit eine Ikone für die Videokommunikation\is{Kommunikation!Video-} über das Internet darstellte, sind es heute Dienste wie WhatsApp\is{WhatsApp}, Facetime\is{Apple Facetime} oder der Facebook Messenger\is{Facebook Messenger}, die größere Präsenz auf der mentalen Landkarte der Endnutzer{\textperiodcentered}innen aufweisen.

Zugleich scheint es, als wäre dem Dienst nicht gänzlich der Rang abgelaufen worden. Davon zeugt die nicht unbeträchtliche Anzahl an Umfrageteilnehmer{\textperiodcentered}innen an der Online-Erhebung, die besonders für Videochats\is{Chat!Video-} auf Skype zurückgreifen. Als starker Kontrast wirken allerdings die Angaben der Studienteilnehmer{\textperiodcentered}innen in beiden Studienvarianten, bei denen Skype sowohl generell als auch in puncto am häufigsten genutzte Anwendung hinter WhatsApp, Apple Facetime und Telegram zurückfällt.

Sowohl die Auswertung des deutschlandweit verteilten Online-Fragebogens als auch die Antworten der Versuchsteilnehmer{\textperiodcentered}innen zeugt davon, dass der Skype Translator bislang keinerlei Reichweite besitzt. Die Funktion ist keiner Person bekannt gewesen. Auch die Angaben des Abschlussfragebogens deuten darauf hin, dass dem Skype Translator an sich wenig Aufmerksamkeit zuteil wurde, auch wenn die Proband{\textperiodcentered}innen sich in Ermanglung von Katalanischkenntnissen auf die MÜ-Ausgabe verlassen mussten. Zugleich fielen den Personen grundlegende syntaktische, semantische und pragmatische Fehler auf. Die Chatsituation hat dies offensichtlich nicht sonderlich stark beeinflusst. 

Weiterhin ist auf Grundlage der Durchsicht der Bildschirmaufzeichnungen festzustellen, dass die Versuchspersonen zu keinem Zeitpunkt willentlich innerhalb des Chats zurückscrollten, um ältere Nachrichten noch einmal erneut zu lesen. Die Aufmerksamkeit war die gesamte Chatsituation über immer auf die jüngsten Nachrichten gerichtet. Die Informationsintegration verläuft daher offenbar strikt linear, selbst wenn ein MÜ-System in die Kommunikation involviert ist.

\begin{sloppypar}
Auf Seite der Entwickler drängt sich somit allerdings die Frage auf, welchen Vorteil große, multinationale Konzerne wie Microsoft\is{Microsoft} oder Google\is{Google} von dem Angebot haben. Wenn die Software einerseits nicht (mehr) Marktführer ist und zugleich die eingeschaltete Funktion offenbar kaum bewusst von den Nutzer{\textperiodcentered}innen wahrgenommen wird, stehen sich an dieser Stelle Zugewinn für die Unternehmen und Aufwand auf den ersten Blick in einem Missverhältnis gegenüber. Als naheliegender Erklärungsansatz könnte vermutet werden, dass die Echt\-zeit-Über\-setz\-ung\is{Echtzeit-Übersetzung} als Funktion eines der Softwareprodukte lediglich ein Nebenerzeugnis eines anderen Unternehmensbereiches ist. Microsoft und Google sind qua Unternehmensausrichtung auf natürliche Sprachdaten angewiesen und generieren durch ihre Produkte zugleich Unmengen an Daten. Nicht zuletzt die Betriebssystemsparte (Windows, Android uvm.) bedarf der Lokalisierung. Somit kann die Aufbereitung der Sprachdaten in Form des Skype Translators\is{Skype!Skype Translator} womöglich als Entgegenkommen der Konzerne auf die Endnutzer{\textperiodcentered}innen angesehen werden. Zumal die omnipräsente Verfügbarkeit von übersetzten Inhalten bislang noch immer ein Wunschtraum für Unternehmen und Privatpersonen darstellt und das Angebot eines Echtzeit-Übersetzungssystems\is{Echtzeit-Übersetzung!-ssystem} somit gute Werbung darstellt.
\end{sloppypar}

Hieraus ergibt sich dann jedoch eine weitere Frage: Welchen Nutzen bringt das so breit gefächerte Sprachenportfolio? Das bloße Vorhandensein der Sprachdaten bei den Konzernen vereinfacht sicherlich die Entwicklung solcher Technologie auch für \glqq kleinere\grqq{} Sprachen. Das wiederum stärkt einerseits das Bewusstsein für diese Sprachen. Zugleich sind häufig dennoch zu wenige Daten vorhanden, als dass von Beginn an qualitativ zumindest akzeptable Resultate beobachtet werden können. 

In diesem Zusammenhang macht ein Beitrag im Handelsblatt aus dem Jahre 2020 deutlich, welchen Stellenwert kleinere Sprachen für die Modellentwickler{\textperiodcentered}innen bei den großen Technologiekonzernen haben. So zitiert \citet[]{holzki_digitale_2020} den Chefentwickler von Google, \glqq dass es viel einfacher ist, eine elfte Sprache zu lernen als die ersten zehn\grqq{}. Da Google gleichzeitig das Ziel ausweist, \glqq Modelle trainieren [zu] können, die das Erlernen so generalisieren, dass sie nur kleine Datenmengen brauchen\grqq{} (ebd.), ergibt sich hieraus das Bild eines Konzerns, der vermeintlich kleinere Sprachen als gute Sparringspartner für die Entwicklung von NLP-Systemen ansieht, die nur mit wenigen Ressourcen viel leisten und gute Ergebnisse liefern. Die Möglichkeit, Inhalte für kleinere Sprachen im digitalen Raum verfügbar zu machen, scheint demnach nicht zwangsläufig ein erklärtes Ziel. Vielmehr geht es darum, die Modellentwicklung bei der Datenverarbeitung von natürlicher Sprache soweit zu verallgemeinern, dass mit möglichst wenig Aufwand möglichst viele Sprachen und Sprachdaten verarbeitet werden können. 

Mit Blick auf die Konzeption sowohl der Online-Umfrage\is{Umfrage!online} als auch des inhaltlich gleichen Eingangsfragebogens\is{Fragebogen!Eingangs-} der beiden Eye-Tracking-Studien ist als kritikwürdig festzuhalten, dass einzelne, im Rückblick als durchaus aufschlussreich erscheinende Fragen nicht aufgegriffen wurden. So hätte es sich im Rückblick als hilfreich erwiesen, das genauere Nutzungsverhalten von Chatanwendungen\is{Chat!-anwendung}, die eine Alternative zu Skype\is{Skype!Alternativen zu} darstellen, zu erheben. Hiermit wäre es möglich gewesen, eine genauere Aussage über die Gründe, weshalb diese Alternativen Skype gegenüber bevorzugt werden, und generell über die Usability\is{Usability} (s. Abschnitt \ref{K3:sec:Usability}, S.\,\pageref{K3:sec:Usability}) zu treffen. Auch eine detaillierter Erhebung der Nutzung der einzelnen Funktionen von Skype hätte in diesem Zusammenhang sicherlich dazu beigetragen, das Nutzungsverhalten der Anwendung genauer zu umschreiben.

Nicht zuletzt muss sich jedoch auch stets vergegenwärtigt werden, dass innerhalb des Angebots an Kommunikationsanwendungen Skype die einzige ist, die bislang eine kostenfreie maschinelle Übersetzung in Echtzeit für die Chatkommunikation anbietet.

%------------------------------------------------------------------------

\section{Visuelle Inspektion}
\label{K7:sec:visuelle-inspektion}

%------------------------------------------------------------------------

Die visuelle Inspektion\is{Inspektion!visuelle} der Eye-Tracking-Daten wurde exemplarisch an den Bildschirmfotos jeweils einer Versuchsperson für das katalanisch-deutsche und einer Person für das monolinguale Setting durchgeführt. Hierzu wurde eine Visualisierung der Indikatoren \emph{Fixationen}, \emph{Sakkaden} und \emph{Blinzler} verwendet. Einen vergleichbaren Ansatz verfolgt \citet[224\psq]{bergstrom_9_2014} bei der strukturellen Untersuchung von Social-Media-Webseiten. Auch hier liefert die visuelle Inspektion des \emph{gaze path}\is{gaze path} sowie der Fixationen wertvolle Hinweise zur grundlegenden Wahrnehmung des Designs sowie zur Orientierung der Nutzer{\textperiodcentered}innen der jeweiligen Seite und kann darüber hinaus zur Bewertung der \emph{Usability} verwendet werden. \citet[167\psq]{bergstrom_chapter_2014} hingegen orientiert sich bei der Untersuchung der Informationsverteilung bei Webinhalten an Heatmaps und nutzt diese zur Identifizierung relevanter Bereiche auf dem Bildschirm.

Generell lässt sich deshalb mit Blick auf die in dieser Arbeit durchgeführte Untersuchung feststellen, dass die Konzentration der erfassten Indikatoren zwischen den beiden Versuchsaufbauten schwankt. Die Wolke, die die Fixationen\is{Fixation}, die Sakkaden\is{Sakkade} und Blinzler\is{Blinzler} jeweils im Bereich der Eingabemaske bilden, wirkt im Falle des katalanisch-deutschen Versuchsaufbaus dichter als im monolingualen Setting. Das Auftreten der drei Indikatoren im zweisprachigen Versuch ist besonders im Bereich der eingehenden bzw. maschinell übersetzten Nachrichten auszumachen.

\begin{sloppypar}
Diese Beobachtungen können mit höchster Vorsicht und nur vorläufig mit Blick auf die weitere Diskussion auf folgenden Erklärungsansätzen basieren: Zum einen verlagern die Proband{\textperiodcentered}innen im einsprachigen Versuchssetting ihre Aufmersamkeit auf die eingehenden Nachrichten, da beide Nachrichtenarten in der Muttersprache abgefasst sind. Eine erneute Betrachtung der eigenen Beiträge zwecks Überprüfung des Inhalts ist also nicht bzw. nur in Ausnahmefällen nötig. Eine sprachenbedingte Konzentration der Aufmerksamkeit findet nicht hier, sondern nur im zweisprachigen Versuch statt, wo zudem im linken Bereich des Chats gleich drei Nachrichtenarten (2\,$\times$\,MÜ, 1\,$\times$\,katalanisches Original) eingehen. Das entspricht den Beobachtungen von \citet[169]{bergstrom_chapter_2014}, wonach Inhalte die größte Aufmerksamkeit der/des Leser{\textperiodcentered}in an der Stelle gewinnen, auf die sich die Person am meisten fokussiert. Ferner deuten die Durchschnittswerte der AOI-Größe im monolingualen Versuch darauf hin, dass die Chatpartner{\textperiodcentered}innen womöglich längere Nachrichten am Stück geschrieben haben. Auch das sorgt in der Folge für eine Konzentration auf den in Frage stehenden Bereich.
\end{sloppypar}

Zum anderen kann die Konzentration in beiden Settings auf den Bereich oberhalb der Eingabemaske dahingehend gedeutet werden, dass die Versuchspersonen selten über das untere Drittel nach oben hinausspringen, um ältere Nachrichten noch einmal zu lesen. Dieses Verhalten kommt zwar gelegentlich vor, das Hauptaugenmerk liegt jedoch eindeutig auf den neusten Beiträgen. Grundsätzlich jedoch verbringen die Studienteilnehmer{\textperiodcentered}innen einen substantiellen Teil mit der Betrachtung der linksbündigen, eingehenden Nachrichten, so etwa der MÜ-Ausgabe. \citet[229]{bergstrom_9_2014} hebt hervor, dass ein zweispaltiger Aufbau in eben diesem Fall eine strukturierende Wirkung haben soll. Ursprünglich in Bezug auf soziale Medien gedacht, gilt die Beobachtung, wonach ein zweispaltiges Design mit der Präsentation von Inhalten in kleinen Blöcken (analog zu den Chatbeiträgen) als Hilfsmittel zur Informationsverarbeitung beiträgt \citep[229\psq]{bergstrom_9_2014}. Verstärkt wird diese Konzentration sicherlich auch durch die Einblendung des Hinweises, dass die Person gegenüber gerade schreibt. Diese Nachricht wird unmittelbar zwischen Eingabemaske und neu eingehenden Nachrichten, ebenfalls linksbündig, angezeigt.


%------------------------------------------------------------------------

\section{Fixatorische Augenbewegung}
\label{K7:sec:fixations}

%------------------------------------------------------------------------

%------------------------------------------------------------------------

\subsection{Fixationsanzahl}
\label{K7:subsec:fix-count}

%------------------------------------------------------------------------

\is{Fixation!-sanzahl|(}

In absoluten Zahlen fallen in beiden Versuchssettings Parallelen auf: Die eigenen Beiträge erhalten jeweils deutlich weniger Fixationen, als auf die eingehenden Nachrichten in deutscher Sprache verfallen. Die durchschnittliche Fixationsanzahl im monolingualen Versuchsaufbau ist bei beiden Nachrichtenkategorien jedoch höher als im Setting Katalanisch-Deutsch. Eine mögliche Erklärung für diese Werte ist, dass die Versuchspersonen im Setting Katalanisch-Deutsch ihre Aufmerksamkeit auf mehrere Reize verteilen müssen. Dort sind vier verschiedene Nachrichtenarten zu erfassen, im einsprachigen Setting nur zwei. Daher lesen die Personen die vier Nachrichtenarten unterschiedlich, zumal nur zwei von diesen in ihrer Muttersprache abgefasst sind bzw. zwei von einem MÜ-System stammen. Andererseits besteht die Möglichkeit, dass die Studienteilnehmer{\textperiodcentered}innen im katalanisch-deutschen Versuch mehrere kürzere Nachrichten versenden, da sie im Vergleich zum einsprachigen Setting -- ob bewusst oder unbewusst -- weniger komplexe Nachrichten versenden, um die Kommunikation aufrechtzuerhalten. 

Beide Versuchssettings weisen unter Beachtung der Fixationsanzahl signifikante Merkmale auf. Die Ergebnisse von \citeauthor{jakobsen_reading_2017} zum übersetzenden und verstehenden Lesen deuten in diese Richtung. Das übersetzende Lesen erhält im Schnitt mehr Fixationen als das reine auf Verständnis ausgerichtete Lesen \citep[36]{jakobsen_reading_2017}. Eingehende bzw. Fremdbeiträge erhalten mehr Fixationen als eigene bzw. ausgehende Nachrichten. Wird die Thema-Rhema-Pro\-gres\-sion\is{Thema-Rhema-Progression} missachtet, beispielsweise durch die Betrachtung von AOI mit höherer Ordnungszahl (progressive erste Fixation), so erhalten die einzelnen Beiträge weniger Fixationen. Da dieses Phänomen statistisch sowohl bei beiden Datensätze als auch bei den einzelnen Nachrichtenkategorien auftritt, kann vermutet werden, dass eine lineare Thema-Rhema-Progression innerhalb des Chats als wichtige Einflussgröße auf die Fixationsanzahl, und dadurch ausgedrückt die Lesetiefe, wirkt.

Auch wenn bei der statistischen Untersuchung mittels nicht"=parametrischer Tests\is{Statistik!Testverfahren!nicht-parametrisches} neun von zehn Vergleichspaare in Bezug auf die Fixationsanzahl pro AOI im Setting Katalanisch-Deutsch signifikant unterschiedlich sind, fällt besonders das eine unauffällige Vergleichspaar \emph{CatO-GerO} auf. Auf Grundlage dieses Testergebnisses ist deshalb anzunehmen, dass sich die zentrale Tendenz der Fixationsanzahl zwischen den beiden Kategorien nicht signifikant unterscheidet. Wenn nun weiterhin angenommen wird, dass die Versuchspersonen die eigenen verfassten Beiträge bereits kennen und deshalb weniger Fixationen auf die Kategorie \emph{GerO} entfallen, so bedeutet dies im Umkehrschluss für die Kategorie \emph{CatO}, dass das katalanische Original nur oberflächlich gelesen wird. Diese Möglichkeit wird durch Abgleich mit den Testergebnissen im einsprachigen Versuchsaufbau unterstrichen, in dem sich alle Kategoriepaare -- also die eigenen und die Fremdbeiträge -- in puncto Fixationsanzahl signifikant unterscheiden, was wiederum auf eine unterschiedliche Lesetiefe hindeutet.

Das Verhältnis der Fixationsanzahl nach Reiz in der Mutter- und Fremdsprache deckt sich mit den Erkenntnissen von \citet[70]{jakobsen_chapter_2017}, der in Bezug auf vier verschiedene Arten des Lesens feststellt, dass das Lesen von einem bestehenden Zieltext kognitiv fordernder ist als die Erfassung eines Ausgangstextes. Während in der Untersuchung von \citeauthor{jakobsen_chapter_2017} die Gründe für dieses Verhältnis in der Vertrautheit des durch die/den Übersetzer{\textperiodcentered}in produzierten Zieltext bzw. der Fremdheit des Ausgangstexts gesucht werden, gelten für die Studie in dieser Arbeit vertauschte Rollen: Der Ausgangstext ist der produzierte, vertraute Text, wohingegen die Ausgabe der maschinellen Übersetzung sowie die Beiträge des Gegenübers die unbekannten Texte sind.

In beiden Versuchen ist zudem eine positive Korrelation von Größe der AOI\is{Area of Interest!Größe des} und Anzahl an Fixationen festzustellen, die in die jeweiligen Bereiche fallen. Die naheliegende Erklärung hierfür ist, dass mit zunehmender Länge des zu lesenden Textes bzw. der damit einhergehenden Vergrößerung der Textbox mehr Fixationen getätigt werden müssen, um die gesamte Nachricht zu erfassen. Das gilt in beiden Versuchsanordnungen ungeachtet der beteiligten Sprachen.
\is{Fixation!-sanzahl|)}



%------------------------------------------------------------------------

\subsection{Dauer der ersten Fixation}
\label{K7:subsec:ffdur}

%------------------------------------------------------------------------

\is{Fixation!Dauer der ersten|(}
Die Dauer der ersten Fixation wird als Indikator für den anfänglichen Verarbeitungsprozess eines Wortes angesehen wird, während dem auf das mentale Lexikon zugegriffen wird. Es fällt auf, dass eingehende Nachrichten in beiden Versuchssettings durchschnittlich länger fixiert werden als eigene Beiträge. Die Ausnahme stellt dabei der Durchschnittwert der MÜ-Ausgabe ins Deutsche dar, der geringer ist als der des deutschen Originals (\emph{GerO}: 296,46, \emph{GerMT}: 278,75) und somit nahe bei dem Wert des eigenen Beitrags im Setting Deutsch-Deutsch (\emph{A}: 273,56) liegt. Die im Schnitt längste Dauer der ersten Fixation wird mit 304\,ms bei den katalanischsprachigen Originalnachrichten festgestellt. Generell sind alle Durchschnittswerte vergleichbar mit denen von \citet[382]{holmqvist_eye_2011}.

In beiden Versuchsaufbauten besteht eine hohe Standardabweichung der Werte. Nach \citet[7]{dormann_parametrische_2013} deutet die hohe Standardabweichung auf eine hohe Streuung der Werte um den Median hin, da generell davon ausgegangen werden kann, dass unter Annahme einer Normalverteilung des Datensatzes \glqq etwa 68\% der Datenwerte $\pm$ 1 Standardabweichung und etwa 95\% $\pm$ 2 Standardabweichungen um den Mittelwert\grqq{} \citep[7]{dormann_parametrische_2013} herumliegen. Da die Dauer der ersten Fixation allerdings mehreren nur begrenzt kontrollierbaren, subjektiven Faktoren wie Müdigkeit, Konzentration, Lichtverhältnisse oder auch Koffeinkonsum unterliegt, ist diese Abweichung nicht weiter verwunderlich (\citealt[376]{rayner_eye_1998}, \citealt[][378\psq]{holmqvist_eye_2011}).

Die inferenzstatistische\is{Inspektion!statistische} Betrachtung der einzelnen AOI-Kategorien deutet weder im Setting Deutsch-Deutsch noch im Setting Katalanisch-Deutsch auf signifikante Unterschiede bei der Dauer der ersten Fixation hin. Ausgehend von den Erkenntnissen von \citet{jakobsen_chapter_2017} könnte die durchschnittliche Dauer der ersten Fixation pro AOI-Kategorie mit der Vertrautheit und der Wahrnehmung durch die Proband{\textperiodcentered}innen erklärt werden. Ergeben sich keine statistisch signifikanten Unterschiede bei der Dauer der ersten Fixation der einzelnen AOI-Kategorien, ist anzunehmen, dass weder bei der Betrachtung der MÜ-Ausgaben ins Katalanische oder Deutsche noch bei dem katalanischen Original das mentale Lexikon, sprich: der kognitive Verarbeitungsprozess, in vergleichbarem Maße angeregt wird. \citet[8\psq]{doherty_eye_2010} berichten in ihrer Untersucht zur MÜ-Evaluierung\is{maschinelle Übersetzung!Evaluation} mittels Eye-Tracking von einem ähnlichen Effekt: Vermeintlich leichter zu verarbeitende, von menschlichen Revisoren als gut eingestufte Sätze der MÜ-Ausgabe werden von den Proband{\textperiodcentered}innen teilweise länger fixiert als Testsätze, die als schlecht markiert sind. Außerdem wurde in der referenzierten Studie ein Gewöhnungsprozess beobachtet: Mit Ausschluss der ersten fünf Sätze verschiebt sich die durchschnittliche Fixationsdauer. Gute Test-Sätze werden unter dieser Bedingung durchschnittlich kürzer fixiert als schlechte, auch wenn der Effekt minimal ist. Da im Gegensatz zu diesen ähnlichen Lesestudien die AOI auf Nachrichtenebene und nicht auf Wortebene festgelegt wurden, stellt sich hier die Frage nach der Interpretierbarkeit der Werte.

Die Dauer der ersten Fixation bezieht sich auf das gesamte AOI, das die ganze Nachricht umschließt, und nicht auf ein einzelnes Wort. Die erste Fixation erfolgt am linken Zeilenanfang jedes Beitrages gemäß der konventionellen Leserichtung des Katalanischen und Deutschen von links nach rechts. Es liegt so die Vermutung nahe, dass die Dauer der ersten Fixation mit einer anfänglichen Verarbeitung des ersten Wortes im fokalen Sichtbereich der Teilnehmer{\textperiodcentered}innen in Verbindung steht und nicht mit dem gesamten Textblock.

Auch die statistische Betrachtung\is{Inspektion!statistische} der progressiven ersten Fixation\is{Fixation!progressive erste} in Bezug auf die Dauer der ersten Fixation ergibt signifikante Unterschiede bei zwei untersuchten Datensätzen: der Gesamtdatensatz im Setting Katalanisch-Deutsch sowie die MÜ ins Deutsche. Hierzu ergeben sich zwei Erklärungen: Zum einen ist es möglich, dass entweder die Versuchsperson oder ihr Gegenüber mehrere aufeinanderfolgende, zusammenhängende Nachrichten gesendet hat. Da diese vom Skype Translator\is{Skype!Skype Translator} getrennt voneinander maschinell übersetzt werden, erhält jede gesendete Nachricht automatisch ihre jeweilige maschinelle Übersetzung\is{maschinelle Übersetzung}. Das wiederum bietet der Versuchsperson die Möglichkeit, mit der jeweils jüngsten Nachricht zu beginnen, die jedoch zugleich die höchste aktuelle Ordnungszahl besitzt. Wenn die Person sich nun vom unteren Bildschirmrand nach oben, zu den älteren eingegangenen Nachrichten hocharbeitet, kann dies die Dauer der ersten Fixation beeinträchtigen. Zum anderen folgen in der Annotationsreihenfolge\is{Annotationsreihenfolge} aller AOI auf die MÜ ins Deutsche wieder deutsche Originalbeiträge.

Die Studie von \citet[433]{inhoff_parafoveal_1986} deutet zudem darauf hin, dass die Fenstergröße mit der Fixationsdauer korreliert. Allerdings ergibt sich ein entsprechendes Ergebnis nur im katalanisch-deutschen, nicht jedoch im deutsch-deutschen Versuchsaufbau. Eine mögliche Erklärung ist, dass die Versuchspersonen durch die vier unterschiedlichen Nachrichtenarten und deren räumlicher Anordnung auf dem Bildschirm stärker kognitiv gefordert werden, als die Personen, die an der einsprachigen Studie mit nur zwei Nachrichtenarten teilgenommen haben. Wie auch schon bei der Fixationsanzahl angemerkt, ist weiterhin zu untersuchen, ob auch ein sprachenabhängiges Kommunikationsverhalten als Faktor für den Umfang der Nachrichten in Erwägung zu ziehen ist. 

% Schließlich ist zu bedenken, dass die Proband{\textperiodcentered}innen die Chatbeiträge\is{Chat!-beitrag} in ihrer Muttersprache erfassen. Der Zugriff auf das mentale Lexikon sollte sowohl bei Fremd- als auch Eigenbeiträgen ohne größere Verarbeitungsprozesse ablaufen. Zugleich steht dieser Befund im Gegensatz zu der Untersuchung von \citet{bisson_processing_2014}, die signifikante Unterschiede bei der Fixationsdauer von fremd- und muttersprachlichen Untertiteln feststellen.

\is{Fixation!Dauer der ersten|)}

%------------------------------------------------------------------------

\subsection{Regressionen}
\label{K7:subsec:regressions}

%------------------------------------------------------------------------


Die Analyse der Regressionen erfolgte unterteilt in zwei Arten: Regressionen in ein und solchen aus einem AOI.\is{Regression!eingehende|(}\is{Regression!ausgehende|(} In absoluten Zahlen gehen auf den fremden Beiträgen in beiden Settings jeweils mehr Regressionen ein als auf den eigenen Nachrichten der Versuchspersonen. Die Mittelwerte der Kategorien \emph{GerO}, \emph{CatO} und \emph{B} liegen allesamt über 1. Somit wird auf jeden Beitrag dieser Kategorien jeweils im Schnitt mindestens einmal zurückgekehrt. Die beiden Nachrichtenarten der MÜ ins Katalanische sowie ins Deutsche weisen einen Schnitt von 0,65 bzw. 0,67, die eigenen Nachrichten \emph{A} im einsprachigen Setting von 0,76 auf. In diese AOI wird im Schnitt demnach weniger als einmal zurückgekehrt.

Umgekehrt gehen von beiden MÜ-Ausgaben mehr Regressionen aus als von den von einem Menschen verfassten Nachrichten. Ähnlich gestaltet sich das Verhältnis für das deutsch-deutsche Setting, auch hier werden mehr Regressionen aus den Fremdbeiträgen heraus getätigt. Auffällig ist dabei, dass die Mittelwerte für die Kategorien \emph{GerO} und \emph{CatO} mit jeweils 0,15 und 0,18 deutlich unter dem Schnitt von \emph{A} und \emph{B} mit 0,75 bzw. 0,89 liegt. Aus allen aufgezählten Beitragskategorien wird folglich im Schnitt weniger als eine Regression ausgeführt. Die Eingabemaske weist in beiden Versuchsaufbauten und in Bezug auf beide Arten von Regressionen vergleichsweise hohe Werte auf. Dies kann als Indiz für das Leseverhalten der Proband{\textperiodcentered}innen genommen werden, die zur Eingabemaske zurückkehren, um eine neue eigene Nachricht zu verfassen, oder dort verweilen, bis eine neue Nachricht auf dem Bildschirm eingeht. 

Die Regressionen stellen zudem den einzigen Indikator dar, der inferenzstatistisch mit komplexen Modellen\is{Statistik!Testverfahren!Regressionsmodell}\is{Regressionsmodell!binominales logistisches} untersucht wurde. Da die Betrachtung der abhängigen Variablen in diesem Fall nur zwei Zustände möglich waren (Regression: ja/nein), wurde mit logistischen Regressionsmodellen gearbeitet, die nicht notwendigerweise eine Normalverteilung der Daten voraussetzen. Im Versuchsaufbau Katalanisch-Deutsch war eine Modellierung sowohl für die eingehenden als auch die ausgehenden Regressionen möglich. Aufgrund nicht hinreichender Modellwerte sind die Ergebnisse jedoch nur begrenzt interpretierbar. Deshalb wurden die Ergebnisse der Modellierung mit nicht"=parametrischen Tests erneut überprüft. Im deutsch-deutschen Setting war eine Modellierung nur für eingehende Regressionen möglich, scheiterte allerdings komplett für die abhängige Variable der ausgehenden Regressionen. Die verwendeten vier unabhängigen Variablen wurden für beide Regressionsarten jeweils mit nicht"=parametrischen Tests\is{Statistik!Testverfahren!nicht-parametrisches} auf Plausibilität\is{Plausibilität} überprüft.

Es ist festzustellen, dass die eingehenden Regressionen in beiden Studienteilen von der AOI-Kategorie abhängen. Genauer noch deuten die Ergebnisse der Regressionsmodelle für die eingehenden Regressionen darauf hin, dass besonders die Beiträge des Gegenübers eingehende Regressionen provozieren. Im Setting Katalanisch-Deutsch kann dies als Indiz genommen werden, dass die Proband{\textperiodcentered}innen die MÜ-Ausgabe ins Deutsche mit der katalanischen Originalnachricht vergleichen und sich dabei möglicherweise an markanten Wörtern oder Zahlen orientieren. Im einsprachigen Studienteil ist es denkbar, dass die Versuchspersonen noch einmal auf die Nachricht des Gegenübers zuückkehren, um zu überprüfen, ob die eigene Nachricht auch an alle wichtigen Informationen anknüpft. 

Bezieht man nun die chronologische Reihenfolge der AOI mit ein, so ergibt sich ein vergleichendes Leseverhalten: Die Wahrscheinlichkeitswerte des Modells für die eingehenden Regressionen deuten primär auf Augenbewegungen hin, die zu einem zuvor bereits betrachteten, mit einer niedrigeren Ordnungszahl versehenen, AOI zurückspringen. Wenn also eine erhöhte Wahrscheinlichkeit besteht, dass Regressionen in das deutsche oder katalanische Original getätigt werden, so erfolgen diese Rücksprünge in der Regel von dem jeweilig darauffolgenden AOI mit nächsthöherer Ordnungszahl, also den deutschsprachigen \emph{GerMT} im Falle des katalanischen Originals bzw. der katalanischen MÜ (\emph{CatMT}) im Falle des deutschen Originals. Der auffallend hohe Wahrscheinlichkeitswert der Regressionen in das AOI der Eingabemaske ist demnach schlichtweg der zwangsläufigen Nutzung zum Verfassen von Nachrichten geschuldet.

Während die Regressionen in das katalanischen Originals mutmaßlich von der MÜ ins Deutsche zurück einfallen, ist es verwunderlich, wieso vom katalanischen Original Regressionen in die vorausgegangene und zum deutschen Original gehörende MÜ ins Katalanische fallen (sollten). Ein erster Erklärungsansatz könnte in der kontinuierlichen Nummerierung der AOI liegen und somit lediglich technisch bedingt sein. 

Mit Blick auf die Wahrscheinlichkeitswerte des Modells für die ausgehenden Regressionen deuten die Ergebnisse im Falle der MÜ ins Deutsche darauf hin, dass die Proband{\textperiodcentered}innen zurückspringen zu vorausgehenden Nachrichten (der Kategorie \emph{CatO}), um diese womöglich erneut zu betrachten. Bei der gesunkenen Wahrscheinlichkeit im Falle des deutschen und katalanischen Originals liegt der Erklärungsansatz nahe, dass die Originalbeiträge schlichtweg keine notwendige Vergleichsmöglichkeit zum Rücksprung bieten. Dem deutschen Original geht innerhalb der Sequenz der AOI-Annotation die MÜ-Ausgabe ins Deutsche, dem katalanischen Original die MÜ ins Katalanische voraus. Außerdem wurde das deutsche Original von den Testpersonen selbst verfasst und beinhaltet somit keinerlei neue Information.

Für Lesestudien auf Wortebene heben \citeauthor{inhoff_regressions_2019} die Bedeutung von Regressionen für das Textverständnis hervor und gehen dabei auf verschiedene Erklärungsansätze ein, weshalb es zu größeren Regressionen kommt. Wenn das gerade betrachtete Element nicht im semantischen oder grammatikalischen Einklang mit dem zuvor erfassten linguistischen Kontext steht, so wird mit großer Wahrscheinlichkeit eine Regression ausgeführt \citep[3]{inhoff_regressions_2019}. Die Regression wird folglich genutzt, um eine linguistische Fehlinterpretation zu korrigieren. Weiterhin ist es möglich, dass Regressionen nicht zwingend zielgerichtet stattfinden, sondern lediglich unbewusst dazu verwendet werden, um mehr Verarbeitungszeit für die aufgenommenen Reize zu gewinnen \citep[3\psq]{inhoff_regressions_2019}. Als dritte Hypothese verweisen \citeauthor{inhoff_regressions_2019} auf die Mögilchkeit, dass Regressionen dem besseren Zugriff auf das Arbeitsgedächtnis dienen \citep[4]{inhoff_regressions_2019}. Überträgt man diese Erklärungsansätze von der Wortebene auf ganze Chatbeiträge\is{Chat!-beitrag}, so lassen sich ebenfalls drei Annahmen treffen.

Einerseits können die Studienergebnisse dahingehend interpretiert werden, dass die Proband{\textperiodcentered}innen besonders bei der MÜ häufiger Regressionen ausführen, weil die MÜ fehlerbehaftet ist und sie im Sinne der ersten Hypothese von \citeauthor{inhoff_regressions_2019} eben diese linguistischen Fehlinterpretationen ausgleichen. In diesem Falle müsste untersucht werden, ob die Regressionen unmittelbar an fehlerbehafteten Stellen der Chatbeiträge stattfinden.

Andererseits könnten die Regressionen aus der und in die MÜ-Ausgabe jedoch auch dafür sprechen, dass die Versuchspersonen die Beiträge zunächst komplett gelesen haben und dann, gemäß der zweiten und dritten Annahme, durch die Regressionen die Zeit zur kognitiven Verarbeitung ausweiten. In jedem Fall kann angenommen werden, dass die Regressionen an bestimmten Punkten innerhalb des Chatbeitrags\is{Chat!-beitrag} geschehen. Entweder werden die Regressionen genutzt, um die fehlerhafte MÜ auszugleichen, somit wäre das Leseverhalten sprunghaft und punktuell, oder die regressiven Augenbewegungen dienen der tiefergehenden Verarbeitung des Gelesenen und spiegeln somit ein eher vergleichendes, umfassendes Leseverhalten wider. Es zeichnet sich in jedem Fall ab, dass die MÜ-Ausgabe anders referenziert wird als die Originalbeiträge.

\is{Regression!eingehende|)}\is{Regression!ausgehende|)}


%------------------------------------------------------------------------

\subsection{Dauer des ersten Durchlaufs}
\label{K7:subsec:IAFRD}

%------------------------------------------------------------------------

\is{Durchlauf!Dauer des ersten|(}
Die Dauer des ersten Durchlaufs unterscheidet sich in beiden Versuchsaufbauten jeweils nach AOI-Kategorie. Im monolingualen Setting sind es die eingehenden Nachrichten des Gegenübers, die in absoluten Zahlen eine längere Dauer des ersten Durchlaufs aufweisen. Im katalanisch-deutschen Versuchsaufbau fallen in absoluten Zahlen vor allem die beiden Arten der MÜ auf. Die durchschnittliche Dauer des ersten Durchlaufs durch die original deutschsprachigen Beiträge liegt in beiden Versuchen bei etwa 600\,ms. Für den ersten Durchlauf durch die MÜ ins Deutsche benötigen die Proband{\textperiodcentered}innen hingegen im Schnitt 816\,ms. Noch auffälliger ist dann der Mittelwert bei den eingehenden, deutschsprachigen Nachrichten im monolingualen Versuch, der 1.266\,ms beträgt. Die hohe Standardabweichung in beiden Versuchsaufbauten zeugt davon, dass eine hohe intersubjektive Varianz besteht. So ist es möglich, dass die zu verarbeitenden Textboxen im monolingualen Setting generell größer sind als die in der vom Skype Translator\is{Skype!Skype Translator} vermittelten Kommunikationssituation\is{Kommunikation!-ssituation}.

Diese Vermutung drängt sich auch mit Blick auf die Durchschnittswerte der AOI-Größe\is{Area of Interest!Größe des} in Pixel auf (s. \tabref{K6:tab:CatDe:mean-sd-iaarea}, S.\,\pageref{K6:tab:CatDe:mean-sd-iaarea} und \ref{K6:tab:DeDe:mean-sd-iaarea}, S.\,\pageref{K6:tab:DeDe:mean-sd-iaarea}). Während alle AOI-Kategorien in beiden Settings einen Mittelwert um 28.000\,px aufweisen, stellen die eingehenden Nachrichten des Gegenübers im einsprachigen Setting mit durchschnittlich 40.000\,px einen auffällig hohen Wert dar. Somit kann vermutet werden, dass die Gegenüber im Setting Deutsch-Deutsch vergleichsweise umfangreiche Nachrichten am Stück verfasst haben, während die Beiträge im ka\-ta\-la\-nisch-deutschen Versuch hingegen kürzer gefasst waren.

Unter Anwendung statistischer Tests sind es alle Vergleichspaare mit Beteiligung der MÜ ins Deutsche, die signfikante Unterschiede in der zentralen Tendenz der Durchlaufdauer aufweisen. Wenn davon auszugehen ist, dass neue Informationen in Form von eingehenden Nachrichten die längere Durchlaufdauer hervorrufen, lässt sich dies auch auf das Setting Katalanisch-Deutsch, und konkret auf die MÜ ins Deutsche, anwenden. Es handelt sich in jedem Fall um neue Informationen in der Muttersprache der Versuchsperson. Weiterhin kann davon ausgegangen werden, dass die MÜ bislang in irgendeiner Form fehlerbehaftet ist. Die Versuchspersonen sind deshalb veranlasst, langsamer zu lesen, um die Schwachstellen auszugleichen. \citet[388]{holmqvist_eye_2011} beschreibt dies mit Schwierigkeiten von Versuchspersonen, allgemeine oder auch lexikalische Informationen aus den präsentierten Reizen zu extrahieren.

So erklärt sich auch die signifikant unterschiedliche Dauer der Betrachtung der MÜ ins Deutsche im Vergleich mit allen anderen AOI-Kategorien. Die Testergebnisse der übrigen Vergleichspaare sind dahingehend zu interpretieren, dass keine signifikanten Unterschiede in der zentralen Tendenz der Dauer des ersten Durchlaufs bestehen. Die Betrachtungsdauer der katalanischen MÜ und Originalbeiträge unterscheidet sich demnach nicht wesentlich von der der deutschen Originalbeiträge. Ausgehend vom deutschen Original, das den Versuchspersonen gegenüber als inhaltlich bereits bekannt vorausgesetzt werden kann, ergibt sich für die MÜ ins Katalanische und das katalanische Original die Schlussfolgerung, dass diese beiden Beitragsarten mit vergleichsweise geringerer Tiefe gelesen und verarbeitet werden. Die nicht auffälligen Testergebnisse in diesem Fall deuten im Umkehrschluss darauf hin, dass die Versuchspersonen nicht gänzlich über diese Nachrichten hinweggehen, da die Dauer ansonsten ebenfalls signifikant auffällig wäre.

Bemerkenswert ist zudem das Ergebnis der Signifikanztests, die die Dauer des ersten Durchlaufs unter Beachtung der progressiven ersten Fixation\is{Fixation!progressive erste} untersuchen. Im katalanisch-deutschen Setting ergaben sich weder auf den gesamten Datensatz bezogen noch unterteilt nach den einzelnen AOI-Kategorien signifikante Unterschiede. Es scheint für die Dauer des ersten Durchlaufs nicht von Belang, ob vorab ein AOI mit höherer Ordnungszahl betreten wurde und somit bereits eine Vorwegnahme an noch aktuelleren Informationen im Chat stattgefunden hat. In der einsprachigen Variante sind die Werte jedoch sowohl in Bezug auf den Gesamtdatensatz als auch auf die ausgehenden, eigenen Nachrichten signifikant.

Die Größe der AOI\is{Area of Interest!Größe des} korreliert in beiden Studienvarianten signifikant mit der Dauer des ersten Durchlaufs. Dabei kann generell davon ausgegangen werden, dass eine größere Textbox mehr Informationen beinhaltet und somit auch eine längere Lesezeit benötigt. Das kann zunächst ungeachtet der beteiligten Sprachen angenommen werden.

Eine Diskussion der Werte der Eingabemaske erscheint nicht sinnvoll. Da die Eingabemaske als statisches AOI\is{Area of Interest!statisches} annotiert wurde, ergibt sich der Mittelwert aus der Gesamtsumme der Dauer des ersten Durchlaufs, geteilt durch die Anzahl an Versuchsteilnehmer{\textperiodcentered}innen. Da die Gesamtsumme sich jedoch tatsächlich nur aus den Zeiten zusammensetzt, die zwischen dem erstmaligen Fixieren\is{Fixation!erstmalige} dieses AOI und dem erstmaligen Verlassen stehen, ist die Aussagekraft begrenzt .
\is{Durchlauf!Dauer des ersten|)}

%------------------------------------------------------------------------

\subsection{Gesamtverweildauer}
\label{K7:subsec:dwelltime}

%------------------------------------------------------------------------

\is{Verweildauer!Gesamt-|(}
\begin{sloppypar}
Die durchschnittliche Gesamtverweildauer für die eingehenden Nachrichten (\emph{B}) im monolingualen Setting hebt sich mit etwa 8.900\,ms deutlich von den Werten der anderen AOI-Kategorien ab. Während die eigenen Beiträge (\emph{A}) im einsprachigen Versuch durchschnittlich etwa 3.000\,ms fixiert werden, sind es im katalanisch-deutschen Versuchsaufbau 2.200\,ms (\emph{GerO}). Die anderen Beitragsarten werden jeweils etwa 2.100\,ms (\emph{CatO}), 3.600\,ms (\emph{CatMT}) und 3.900\,ms (\emph{GerMT}) fixiert. Auffällig ist dabei, dass beide MÜ-Ausgaben jeweils etwa 1.000\,ms länger als die Originalbeiträge betrachtet werden.

Die unterschiedlichen Durchschnittswerte der Beitragsarten im einsprachigen Versuch sind insofern auffällig, als nur sieben Versuchspersonen beteiligt waren. Eine Erklärung dieser Werte ist daher, dass die Gegenüber in der Studienvariante deutlich längere Nachrichten verfasst haben, die die Proband{\textperiodcentered}innen lesen mussten. Die Werte der Eingabemaske wirken im Falle der einsprachigen Studie nur dann plausibel, wenn angenommen wird, dass der Bereich der Eingabemaske zugleich auch immer als Ruheort für die Augen dient, während die Chat"=Partner{\textperiodcentered}innen\is{Chat!-partner{\textperiodcentered}in} ihre Nachrichten verfassen. Die Werte der Eingabemaske im Setting unter Beteiligung des Skype Translators wurden bereits im Zuge der Analyse problematisiert. Hier scheint ein Messfehler oder eine ungenaue Erfassung vorzuliegen.

Mit Blick auf die statistische Untersuchung fallen im Setting Katalanisch-Deutsch die Vergleichspaare auf, bei denen die MÜ-Ausgabe ins Deutsche beteiligt ist. Die Verweildauer der MÜ ins Deutsche unterscheidet sich signifikant von der der katalanischen MÜ-Ausgabe, des katalanischen Originals sowie des deutschen Originals. Gemessen an den absoluten Zahlen liegt somit die Vermutung nahe, dass die MÜ ins Deutsche eine längere Verarbeitungszeit bei den Proband{\textperiodcentered}innen hervorruft. Als mögliche Erklärung sind hier sprachliche Fehler in der Ausgabe denkbar. Auch ist es, mit Blick auf die anderen Indikatoren denkbar, dass die Versuchspersonen häufiger zu der MÜ ins Deutsche zurückkehren, um den Inhalt erneut zu lesen. Umgekehrt erweisen sich die gepaarten Vergleiche von Eingabe, den beiden katalanischen Nachrichtenarten und dem deutschen Original als statistisch nicht auffällig. Dies kann erneut mit dem vermuteten Messfehler auf dem AOI der Eingabemaske in Verbindung gebracht werden. Im Setting Deutsch-Deutsch fallen die Unterschiede in der zentralen Tendenz zwischen allen Vergleichspaaren statistisch signifikant aus. Die Verweildauer in den AOI der eigenen Nachrichten unterscheidet sich also von der in den AOI des Gegenübers sowie in der Eingabemaske. Allerdings sind, wie oben angemerkt, die Werte der statisch annotierten Eingabemaske nur bedingt plausibel.
\end{sloppypar}

Die Erkenntnisse von \citet[]{jakobsen_chapter_2017} bieten hierzu einen Erklärungsansatz. Bei den eigenen Beiträgen handelt es sich um bekannte, vertraute Texte, die für die Proband{\textperiodcentered}innen leichter zu erfassen sind. Bei den Beiträgen des Gegenübers handelt es sich hingegen um unbekannte Texte, die zunächst verarbeitet werden müssen. Die längere Verweildauer auf dem AOI der Eingabemaske ist demnach auf die Erkenntnisse von \citet[]{jakobsen_chapter_2017} zurückzuführen, nachdem die Lesewahrnehmung und die Textproduktion gleichermaßen kognitiven Aufwand erfordern. 

Durchgeführte Korrelationstest zeigen außerdem, dass in beiden Settings die Größe des AOI\is{Area of Interest!Größe des} und die Gesamtverweildauer in einem positiven Verhältnis zueinander stehen. Je größer das AOI, sprich: die Textbox im Chat, ist, desto länger ist die Gesamtverweildauer. Eine Erklärung hierzu ist in der begrenzten Distanz von 7--9 Zeichen pro Sakkade zu finden, die die Augen nur springen können und der Lesefortschritt somit begrenzt ist.

\is{Verweildauer!Gesamt-|)}

%------------------------------------------------------------------------

\subsection{Regressive Durchlaufdauer}
\label{K7:subsec:iaregpd}

%------------------------------------------------------------------------

Die regressive Durchlaufdauer drückt die Dauer aller Fixationen aus, die zwischen dem erstmaligen Betreten eines AOI und dem Verlassen hin zu einem AOI mit höhrer Ordnungszahl gemessen wird. Es wird auch die Fixationsdauer hinzuaddiert, die nach dem erstmaligen Betreten des AOI in vorausgehenden AOI mit niedrigerer Ordnungszahl erfasst wird. Die Ergebnisse können deshalb zur Untersuchung des Leseverhaltens genutzt werden. Je kürzer die regressive Durchlaufdauer ist, desto früher haben die Versuchspersonen das jeweilige AOI in Richtung eines AOI mit höherer Ordnungszahl verlassen und somit möglicherweise nicht alle Informationen aus der Chatnachricht und dem angebotenen Kontext in Form der vorausgehenden Nachrichten aufgenommen. Im Setting Deutsch-Deutsch ist die regressive Durchlaufdauer der Chatbeiträge der Versuchspersonen sowohl im Schnitt als auch absolut länger als auf denen des Gegenübers. Im Gegensatz dazu ist die Betrachtungsdauer der eigenen Beiträge und die des katalanischen Originals im Setting Katalanisch-Deutsch sowohl absolut als auch im Schnitt kürzer als die anderen beiden Arten von Textbeiträgen. Auf Grundlage dieser Werte kann angenommen werden, dass die Originalnachrichten (\emph{GerO} und \emph{CatO}) im Setting Katalanisch-Deutsch sowie die Nachrichten des Gegenübers (\emph{B}) im einsprachigen Versuchsaufbau wesentlich früher in Richtung eines anderen AOI verlassen wurden als jeweils im Falle der MÜ-Ausgabe (\emph{CatMT} und \emph{GerMT}) und der eigenen Beiträge (\emph{A}). Auch bei dieser Untersuchung sind die Werte der statisch annotierten Eingabemaske nur bedingt zu interpretieren, da die regressive Durchlaufdauer das erstmalige Betreten bis zum erstmaligen Verlassen des AOI erfasst. 

\begin{sloppypar}
In beiden Versuchsaufbauten waren die Daten nicht normal verteilt und wiesen zudem eine starke inter-individuelle Varianz auf. Das ist für naturalistisch orientierte Untersuchungen dieser Art nicht ungewöhnlich. Die unterschiedlichen Ergebnisse der Signifikanztests in Bezug auf die einzelnen AOI-Kategorien sind hingegen beachtenswert. Im einsprachigen Versuchsaufbau finden sich keine signifikanten Unterschiede bei der regressiven Durchlaufdauer zwischen den beiden Beitragsarten. Im Gegensatz dazu weisen bis auf zwei Ausnahmen (\emph{CatMT-Eingabe} und \emph{CatO-GerO}) alle möglichen Vergleichspaare im ka\-ta\-la\-nisch-deutschen Versuch signifikante Unterschiede in der zentralen Tendenz auf. Auch hat die progressive erste Fixation\is{Fixation!progressive erste} in beiden Versuchen einen Einfluss auf die regressive Durchlaufdauer. Die AOI, die von einem anderen AOI mit höherer Ordnungszahl betreten wurden, unterscheiden sich in der regressiven Durchlaufdauer signifikant von jenen, die chronologisch betreten wurden.
\end{sloppypar}

Bei der Beteiligung der MÜ im Falle des Skype Translators unterscheidet sich die Lesedauer nicht nur zwischen Beiträgen in den beteiligten Sprachen, sondern auch zwischen Nachrichten, die zwar in der gleichen Sprache abgefasst, jedoch entweder von einem Menschen oder einer Maschine generiert wurden. Demnach können die niedrigen absoluten und durchschnittlichen Werte im katalanisch-deutschen Setting auf ein sprunghaftes Leseverhalten zurückgeführt werden. Die MÜ-Ausgabe der jeweiligen Textbeiträge stellt einen neuen Reiz auf dem Bildschirm dar. 

Die Proband{\textperiodcentered}innen, die so zunächst ihre eigenen Nachrichten (\emph{GerO}) lesen, springen zu der MÜ ins Katalanische, um diese mit dem Original zu vergleichen. Das gleiche Muster kann auch den Werten des katalanischen Originals zugrunde gelegt werden. Die Versuchspersonen sind im Begriff, die katalanischsprachige Nachricht zu lesen, wenn dann plötzlich die maschinelle Übersetzung auf dem Bildschirm eingeblendet wird. Entweder erneut aus dem Antrieb heraus, nun einen Vergleich zur Verfügung stehen zu haben, der in diesem Fall sogar in der Muttersprache abgefasst ist, oder auf Grundlage der Eye-Mind-Hypothese von \citet{just_theory_1980} springen sie in die MÜ ins Deutsche. Umgekehrt wiederum verbleiben die Personen offensichtlich lange auf der Ausgabe der MÜ. Hierzu kann angenommen werden, dass die Nachricht zu\"nachst vollständig gelesen wird, bevor sich die Versuchspersonen entscheiden, ob sie zum nächsten AOI springen oder noch einmal zum entsprechenden Original zwecks Vergleich zurückkehren.

Die Annahme wird durch einen Vergleich der Werte der regressiven Durchlaufdauer mit denen der selektiven regressiven Durchlaufdauer sowie der Gesamtverweildauer bestärkt, deren jeweilige Werte deutlich niedriger sind. Die absoluten und durchschnittlichen Werte der beiden MÜ-Kategorien sind dort beispielsweise etwa nur halb so groß. Daraus lässt sich schließen, dass die zeitliche Differenz, um die die regressive Durchlaufdauer größer ist, in vorausgehenden AOI mit niedrigerer Ordnungszahl verbracht wurde. Es zeigt sich, dass die Versuchspersonen zunächst die ins deutsche maschinell übersetzte Nachricht erstmalig fixieren, wodurch die Messung der regressiven Durchlaufdauer beginnt. Danach springen sie zu Vergleichszwecken zu dem katalanischsprachigen Orignal zurück, wobei die Messung der regressiven Durchlaufdauer weiterläuft, jedoch nicht die der Gesamtverweildauer oder der selektiven regressiven Durchlaufdauer.

%------------------------------------------------------------------------

\subsection{Pupillengröße}
\label{K7:subsec:pupilsize}

%------------------------------------------------------------------------

\is{Pupillengröße|(}
Die Pupillengröße auf den Fremdbeiträgen mit Ausnahme des katalanischen Originals liegt in beiden Settings jeweils über dem Schnitt. Besonders auffällig ist dabei die durchschnittliche Pupillengröße bei Betrachtung der maschinell übersetzten Nachrichten ins Katalanische. Diese liegt 18 Einheiten über dem Schnitt. Doch auch die Pupillengröße bei Betrachtung der MÜ ins Deutsche liegt über dem Gesamtschnitt. Auf Grundlage der Erkenntnisse von \citet[]{hess_pupil_1964, beatty_task-evoked_1982} können die Ergebnisse dahingehend interpretiert werden, dass die Proband{\textperiodcentered}innen bei der Verarbeitung der einzelnen Chatbeiträge\is{Chat-!beitrag} unterschiedlich stark kognitiv ausgelastet sind. \citet[17\psq]{krejtz_eye_2018} kommen in ihrer Studie zu dem Ergebnis, dass die Pupillengröße von dem Schwierigkeitsgrad der gestellten Aufgabe abhängt. Dabei stellen sich sowohl Unterschiede zwischen den Versuchspersonen als auch innerhalb jeder Versuchsteilnahme heraus. \citet[317\psq]{iqbal_towards_2005} stellen hierzu in einer anderen Studie fest, dass sich die Pupille bei kognitiv anspruchsvollen Aufgaben weitet. Proband{\textperiodcentered}innen, die ein Dokument bearbeiten, weisen eine höhere Auslastung während der Sprachverarbeitung als während des Leseverstehens auf.
Ein vergleichbares Bild zeichnen die Daten dieser Arbeit sowohl für das einsprachige Setting als auch für die Untersuchung mit Skype Translator ab.   

\begin{sloppypar}
Die statistische Untersuchung der Pupillengröße ergibt zunächst, dass die Werte signifikant mit der Größe des AOI korrelieren. Dies gilt in beiden Versuchssettings. In Hinblick auf die einzelnen Kategorien ist in beiden Versuchsanordnungen weiterhin die Eingabemaske hervorzuheben. Die Pupillengröße ist bei beiden Versuchen im Falle der Eingabemaske am größten. Ein Grund für die durch diese Werte ausgedrückte kognitive Auslastung könnte einerseits das Verfassen von neuen Nachrichten sein, während die Proband{\textperiodcentered}innen die Eingabemaske fixieren. Andererseits drücken diese Werte möglicherweise das Warten auf neue eingehende Nachrichten des Gegenübers aus. Die Werte der eigenen Beiträge sind hingegen jeweils am geringsten. So scheint es, als stelle die eigene Nachricht in beiden Settings die geringsten kognitiven Anforderungen an die Versuchspersonen. 
\end{sloppypar}

Weiterhin deuten die Testergebnisse in beiden Versuchsanordnungen darauf hin, dass inter-individuelle Variablität besteht, da die durchgeführten Kruskal-Wallis-Tests jeweils signifikante Unterschiede zwischen den Proband{\textperiodcentered}innen ergaben. Die Ergebnisse in Bezug auf die beteiligte Sprache werden auch durch die Studie von \citet[608\psq]{hyona_pupil_1995} unterstützt, in der sich signifikante Unterschiede bei der Pupillengröße zwischen zu lesenden Wörtern in der Fremd- und Muttersprache ergaben. Die Verarbeitung in der Fremdsprache führt zu einer stärkeren Pupillenweitung als in der Muttersprache. Die Betrachtung der einzelnen AOI ergibt im Setting Deutsch-Deutsch keine signifikanten Unterschiede zwischen den einzelnen Beitragsarten, sondern nur im Vergleich mit der Eingabemaske. Im katalanisch-deutschen Setting ergeben sich hingegen bei 8 von 10 möglichen Vergleichspaaren signifikante Unterschiede. Zwei Vergleichspaare unter Beteiligung der MÜ ins Deutsche (jeweils \emph{CatMT-GerMT} und \emph{CatO-GerMT}) sind unauffällig. Aus zweierlei Gründen ist das beachtenswert. Erstens wird die MÜ-Ausgabe ins Deutsche (\emph{GerMT}) auf Grundlage des katalanischen Originals (\emph{CatO}) erzeugt. Zweitens steht dieses Testergebnis somit in einem Widerspruch zu den Erkenntnissen von \citeauthor{hyona_pupil_1995}, wonach die Betrachtung von Wörtern in der Fremdsprache kognitiv fordernder sind als in der Muttersprache \citep[610]{hyona_pupil_1995}.

Offen bleibt noch die Einordnung der progressiven ersten Fixation\is{Fixation!progressive erste}. In beiden Versuchsanordnungen ergeben sich in Bezug auf den gesamten Datensatz und auf die jeweiligen AOI signifikante Unterschiede in der Pupillengröße zwischen AOI, die aus einem anderen AOI mit höhrerer Ordnungszahl heraus betreten wurden, und jenen, die in chronologischer Reihenfolge besucht wurden. Es liegt die Vermutung nahe, dass dies der Verarbeitung von neuen Informationen im Rahmen der Thema-Rhema-Progression\is{Thema-Rhema-Progression} geschuldet ist. Möglicherweise ist diese Auffälligkeit mit dem Konzeptionsprozess von Nachrichten zu erklären. Während die Proband{\textperiodcentered}innen die Eingabemaske betrachten, verfassen sie die neuen Nachrichten und sind demnach kognitiv anders ausgelastet als bei einer Leseaufgabe wie der Betrachtung der eingehenden und ausgehenden Nachrichten.

Nichtsdestotrotz bleibt die Aussagekraft der Unterschiede bei der Pupillengröße begrenzt. Wie auch \citet[10]{doherty_eye_2010} abwägen, sind einerseits zwar Veränderungen der Durchschnittswerte erkennbar, andererseits kann dies auch auf die kleine Fontgröße, einen Gewöhnungsprozess oder schlichtweg darauf zurückzuführen sein, dass der Verarbeitungsprozess der MÜ-Ausgabe nicht wesentlich kognitiv fordernder ist. Auch \citeauthor{krejtz_eye_2018} weisen auf genau dieses Problem bei der Untersuchung der Pupillengröße hin. Die alleinige Analyse der Pupillengröße besitzt nur eine limitierte Aussagekraft, da die Erfassung vielen Ungenauigkeiten und Einflussfaktoren unterworfen ist \citep[19]{krejtz_eye_2018}. Die in dem Artikel als Alternative vorgeschlagene Untersuchung der Microsakkaden wurde im Rahmen dieser Arbeit allerdings nicht durchgeführt.


\is{Pupillengröße|)}

%------------------------------------------------------------------------

\section{Sakkadische Augenbewegungen}
\label{K7:sec:sac-movements}

%------------------------------------------------------------------------

%%%%%%%%%%%%%%%%%%%%%%%%%%%%%
% Sakkadenanzahl
% Absolut / Mittelwerte
%%%%%%%%%%%%%%%%%%%%%%%%%%%%%

\subsection{Sakkadenanzahl}\label{K7:para:Sakkadenanzahl}
Das sakkadische Blickverhalten der Proband{\textperiodcentered}innen wurde unter Verwendung von drei Indikatoren untersucht: die Sakkadenamplitude und die \mbox{-dauer} sowie die absolute Sakkadenanzahl. Genauer noch wurde die Anzahl an Sakkaden unterteilt nach Richtung und AOI, sodass auch hierüber Rückschlüsse auf das Leseverhalten möglich sind. Bereits an dieser Stelle ist auf die unterschiedliche Anzahl an Sakkaden pro AOI zu verweisen, die sich innerhalb der Untersuchung der Sakkadendauer und -amplitude ergibt. Die divergierenden Zahlen sind mutmaßlich auf die vergleichsweise grobe Annotation mit dynamischen AOI zurückzuführen, die die genaue Erfassung aller Sakkaden an ihrem Ansatz- und Absatzpunkt nicht durchweg mit voller Präzision zulassen.  

Generell ist in beiden Versuchsanordnungen festzustellen, dass etwa drei bis vier Mal so viele Sakkaden auf die eingehenden, bzw. maschinell in die deutsche Sprache übersetzten Nachrichten (\emph{GerO}: 9 und \emph{A}: 9,76) verfallen wie auf die eigenen Beiträge. Die Mittelwerte der deutschen Originalbeiträge im zweisprachigen Versuch sowie deren MÜ ins Katalanische sind hingegen kleiner als der Mittelwert der eigenen Nachrichten im einsprachigen Setting, auch wenn der Unterschied nur 0,76 beträgt. Die durchschnittliche Sakkadenanzahl innerhalb der MÜ-Beiträge ins Katalanische (\emph{CatMT}: 8:36) ist allgemein am niedrigsten. Die durchschnittliche Sakkadenanzahl der katalanischen Originalbeiträge (\emph{CatO}: 11,98) im zweisprachigen Versuchsaufbau sowie deren MÜ ins Deutsche (\emph{GerMT}: 13,23) liegt über der der beiden Beitragsarten im einsprachigen Setting (\emph{A}: 9,76 und \emph{B}: 10,27). Gemessen an der Sakkadenanzahl bedarf die MÜ ins Deutsche offenbar eine intensivere Betrachtung -- und damit einhergehende Verarbeitung -- als von deutschen Muttersprachler{\textperiodcentered}innen verfasste Nachrichten auf Deutsch. 


%%%%%%%%%%%%%%%%%%%%%%%%%%%%%
% Sakkadenrichtung
%%%%%%%%%%%%%%%%%%%%%%%%%%%%%

Mit Blick auf die Sakkadenrichtung dominieren die nach rechts gerichteten Sprünge der Augen ungeachtet der AOI-Kategorie. Das entspricht der zu erwartenden Leserichtung in den beiden beteiligten Sprachen. Auch die Bewegungen nach links haben einen untereinander vergleichbar großen Anteil an der Gesamtzahl, der allerdings über dem in anderen Studien beobachteten Wert liegt. \citet[6]{buscher_attentive_2012} beispielsweise gehen von einem üblichen Anteil an Regressionen bei Leseaufgaben von 10--15\,\% aus\footnote{Hierbei handelt es sich um sakkadische Regressionen. Sie werden auf Grundlage des Ansatz- und Absatzpunktes der jeweiligen Sakkade (bzw. des Winkels den diese beiden Punkte relativ zur horizontalen Ebene auf dem Bildschirm beschreiben) identifiziert und stehen damit immer in Verbindung mit der Leserichtung. Daher dürfen diese Regressionen auch nicht mit dem Indikator verwechselt werden, der im Methodikteil beschrieben und in der Analyse verwendet wird. Bei jenen Regressionen handelt es sich um ein proprietäres Maß der Software von \citeauthor{sr_research_ltd_eyelink_2019} auf Grundlage der Ordnungszahl der fixierten AOI.}. Der Anteil an Sakkaden, die im zweisprachigen Setting innerhalb der katalanischen Originalbeiträge nach rechts getätigt werden, ist zugleich auffällig. Während der Anteil der nach rechts gerichteten Sakkaden in den AOI-Kategorien \emph{CatMT}, \emph{GerO} sowie \emph{GerMT} zwischen 58\,\% und 70\,\% liegt, macht er bei der Eingabemaske und beim katalanischsprachigen Original nur 43\,\% bzw. 51\,\% aus. Möglicherweise erklärt sich dieser geringere Anteil damit, dass die Versuchspersonen in beiden Fällen eher kursorisch springen als intensiv lesen. Diese Möglichkeit wird auch durch die Durchschnittswerte der Sakkadenamplitude und -dauer gestützt. \citet[124]{biedert_reading_2012} unterscheiden in ihrer Studie ähnlich zwischen Sakkaden, die tiefgehendes und solchen, die überfliegendes Leseverhalten (\emph{skimming}) ausdrücken. In der Studie wird die Unterscheidung allerdings auf Grundlage der über eine Sakkade hinweg zurückgelegten Zeichen getroffen, wie sie auch \citet[379\psq]{rayner_eye_1998} als Maß anwendet.

\begin{sloppypar}
Wenn ein vergleichendes Leseverhalten der Versuchsteilnehmer{\textperiodcentered}innen zugrunde gelegt wird, kann angenommen werden, dass die Personen innerhalb desselben katalanischsprachigen AOI zurückspringen, um markante Stellen zu identifizieren. Wenn das AOI über mehrere Zeilen geht oder mehrere Chatbeiträge derselben Person aufeinander folgen, erklärt sich so der höhere Anteil an nach oben gerichteten Sakkaden. Einschränkend stellen jedoch \citet[129]{biedert_robust_2012} in ihrer Studie fest, dass die Fenstergröße der präsentierten Leseaufgabe einen starken Einfluss auf die Genauigkeit der Sakkaden hat. So führe ein zu kleines Fenster dazu, dass die Zeilenabstände eines Beispieltextes zu nah aneinanderstehen, wodurch der Eye-Tracker nicht mehr präzise in der Lage sei, die Sakkaden den einzelnen Zeilen zuzuordnen. Umgekehrt führe ein zu großes Fenster dazu, dass die Sakkaden von zu viel \emph{Noise}, also Ungenauigkeit, begleitet würden. Eine ähnliche Abwägung wurde auch mit Bezug auf die Nachrichtengröße bei Skype im Methodikteil dieser Arbeit problematisiert und ist bei der Bewertung der Befunde an dieser Stelle im Hinterkopf zu behalten. 

Außerdem ist es möglich, dass die Versuchsteilnehmer{\textperiodcentered}innen zwischen den eigenen und fremden Beiträgen wechseln. In diesem Fall deutet der Anteil an nach links gerichteten Sakkaden auf ein abgleichendes Leseverhalten zwischen den beiden katalanischen Nachrichtenarten und den eigenen Nachrichten hin. 

Eine andere Erklärung ist der Redaktionsprozess einer Nachricht an sich, der vermehrte Korrekturen und Abwandlungen der Nachricht notwendig macht. Hierbei ist die Reihenfolge zu beachten, in der die Nachrichten auf dem Bildschirm eingeblendet werden. So ist es denkbar, dass der hohe Anteil nach oben gerichteter Sakkaden in den AOI des katalanischen Originals durch ein abgleichendes Leseverhalten der Versuchspersonen mit den jeweils vorausgehenden eigenen Nachrichten und deren MÜ ins Katalanische hervorgerufen wird. Das ist allerdings noch tiefgreifender zu überprüfen. Bislang mutet es seltsam an, dass vergleichsweise viele nach oben gerichtete Sakkaden von dem katalanischen Original ausgehen.

%%%%%%%%%%%%%%%%%%%%%%%%%%%%%
% Inferenzstatistik
% Sakkadenamplitude
%%%%%%%%%%%%%%%%%%%%%%%%%%%%%

\subsection{Sakkadenamplitude}\label{K7:para:sacamp}
Da keine Normalverteilung vorliegt, war eine Betrachtung der Sakkadenamplitude in beiden Settings unter inferenzstatistischen Gesichtspunkten nur mit nicht"=parametrischen Tests möglich. Beide Datensätze weisen jeweils die gleiche durchschnittliche Amplitude von 1,84 vor. Auch die durchschnittliche Amplitude der jeweils eigenen Beiträge ist vergleichbar (\emph{GerO}: 1,94 und \emph{A}: 1,95). Obwohl sich die Amplitude der beiden Beitragskategorien im Setting Deutsch-Deutsch um 0,14 in absoluten Zahlen unterscheidet, ergeben sich keine statistischen Auffälligkeiten aus dem Vergleich der beiden AOI-Kategorien (\emph{A-B}). Die zentrale Tendenz der Amplitude im katalanisch-deutschen Versuchsaufbau hingegen unterscheidet sich bei 8 von 10 Vergleichspaaren signifikant. Dieses Ergebnis kann dahingehend interpretiert werden, dass eine unterschiedliche Lesetiefe zwischen den einzelnen AOI-Kategorien vorliegt. Statistisch unauffällig sind nur die Paare \emph{CatMT-Eingabe} und \emph{CatO-GerO}. 
\end{sloppypar}

Die höhere Amplitude kann als Indiz für eine geringere Lesetiefe genommen werden, da sie die zurückgelegte Strecke einer Sakkade zwischen An- und Absatz abbildet. Je größer also der Amplitudenwert ist, desto länger sind die einzelnen Sakkaden, die innerhalb der AOI erfasst wurden. Eine größere Distanz zwischen Ansatzpunkt und Absatz stellt also die Spanne dar, über die das Auge springt und in der möglicherweise mehrere Zeichen liegen. Das gilt offenbar für alle Vergleichspaare mit Ausnahme der oben genannten zwei Paare \emph{CatMT-Eingabe} und \emph{CatO-GerO}. Die Produktionsaktivität wurde hingegen in dieser Studie nicht erfasst, stellt möglicherweise jedoch einen Einflussfaktor dar. In anderen Studien wurde der Faktor der Produktion in der Fremd- vs. in der Muttersprache in das Studiendesign integriert. Das ist hier nicht der Fall. Die Proband{\textperiodcentered}innen müssen nicht selbst auf einer oder aus einer Fremdsprache produzieren.

%%%%%%%%%%%%%%%%%%%%%%%%%%%%%
% Absolute Werte / Durchschnitt
% Sakkadendauer
%%%%%%%%%%%%%%%%%%%%%%%%%%%%%

\subsection{Sakkadendauer}\label{K7:para:sacdur}
Wie auch die Amplitude drückt die Sakkadendauer die Lesetiefe aus. Höhere Werte bedeuten längere Sakkaden, was auf eine geringere Lesetiefe hindeutet. In Hinblick auf die Sakkadendauer wurden nur die Durchschnittswerte erfasst. Eine Darstellung der aufsummierten Werte scheint für diesen Indikator nicht sinnvoll. Die hohe Standardabweichung zeugt nebenher von einer hohen Varianz innerhalb der beiden Datensätze. Dies kann mehrere Gründe haben: Individuelle Varianz der Proband{\textperiodcentered}innen, hohe Fehlerraten, bislang noch nicht beachtete Variablen und falsch-positive Beobachtungen. Auch die visuelle Untersuchung, exemplarisch anhand von \figref{K6:fig:sacmove-TN24} (s. S.\,\pageref{K6:fig:sacmove-TN24}), deutet darauf hin, dass die meisten Sakkaden in das untere Drittel des Bildschirms -- und dort in die linke Hälfte -- fallen. Im Versuchssetting Katalanisch-Deutsch werden dort die eingehenden Nachrichten des Gegenübers sowie alle MÜ-Ausgaben angezeigt. Nur wenige Sakkaden gehen über diesen Bereich hinaus.

In beiden Versuchsaufbauten liegt die Sakkadendauer innerhalb der Eingabemaske über dem Durchschnitt. Die Werte innerhalb der Beiträge des Gegenübers im monolingualen Setting sowie innerhalb der MÜ ins Deutsche im Versuchsaufbau Katalanisch-Deutsch liegen hingegen unter dem Gesamtdurchschnitt. Die Durchschnittsdauer der Sakkaden innerhalb der eigenen Nachrichten ist im monolingualen Setting größer als der Durchschnittswert, im Versuchsaufbau Ka\-ta\-la\-nisch-Deutsch liegt sie unter dem Schnitt. Im Falle der eigenen Beiträge im Setting Deutsch-Deutsch sowie der Eingabemaske in beiden Versuchen ist dies möglicherweise mit der Vertrautheit der Nachrichteninhalte zu erklären, die zudem auch noch in der Muttersprache der Versuchspersonen abgefasst sind. Die niedrigeren Werte innerhalb der eigenen Beiträge sowie der MÜ ins Deutsche im Setting Katalanisch-Deutsch deuten darauf hin, dass die Lesetiefe in diesen beiden Fällen hoch ist. Das wiederum kann in Verbindung mit der Leserichtung und der Sakkadenamplitude als vergleichendes Leseverhalten der Proband{\textperiodcentered}innen jeweils zwischen Original und maschineller Übersetzung gewertet werden. Die Personen wechseln zwischen \emph{GerO} und \emph{CatMT} bzw. zwischen \emph{CatO} und \emph{GerMT}, um die beiden Paare zu vergleichen und so markante Stellen zu identifizieren.


%%%%%%%%%%%%%%%%%%%%%%%%%%%%%
% Inferenzstatistik
% Sakkadendauer
%%%%%%%%%%%%%%%%%%%%%%%%%%%%%

Zur statistischen Untersuchung der Sakkadendauer wurden ebenfalls nicht"=parametrische Tests angewendet, da keine Normalverteilung bei beiden Datensätzen vorlag. In beiden Versuchsaufbauten unterscheidet sich die Sakkadendauer innerhalb der ausgehenden, eigenen Nachrichten und der eingehenden, fremden Beiträgen signifikant. Im Setting Katalanisch-Deutsch gilt dies für die deutschen Originalbeiträge, die bedeutsame Unterschiede im Vergleich zur MÜ-Ausgabe jeweils ins Deutsche und Katalanische aufweisen. Im einsprachigen Versuchsaufbau unterscheidet sich die Sakkadendauer im Vergleichspaar \emph{A-B} ebenfalls signifikant. Im Gegensatz dazu ergeben sich für die MÜ-Ausgabe ins Deutsche signifikante Unterschiede zu den Werten des katalanischen Originals und der Eingabemaske. Auch im anderen Versuchsaufbau weist der Vergleich der Sakkadendauer zwischen Fremdbeiträgen und Eingabemaske auf signifikante Unterschiede hin. Mit Rückgriff auf die Feststellung von \citet[322]{holmqvist_eye_2011}, wonach die Sakkadendauer mit dem Schwierigkeitsgrad der präsentierten Aufgabe steigt, kann angenommen werden, dass die Betrachtung der einzelnen Beitragsarten eine unterschiedlich schwierige Aufgabe für die Proband{\textperiodcentered}innen darstellt. Das wiederum bietet insbesondere eine Erklärung für das statistisch unauffällige Paar \emph{CatMT-GerMT}, wonach die Betrachtung der MÜ-Ausgabe ungeachtet der Sprache für die Personen gleichermaßen kognitiv fordernd ist.

Das Vergleichspaar \emph{A-Eingabe} im einsprachigen Setting bleibt ebenso statistisch unauffällig wie auch das Paar \emph{GerO-Eingabe} im anderen Versuch. Dies lässt sich mit der Vertrautheit der Proband{\textperiodcentered}innen mit dem erfassten Informationen erklären. Die Nachricht wird in der Eingabemaske verfasst und ist deckungsgleich mit der Anzeige, die kurz darauf als ausgehende Nachrichten der AOI-Kategorie \emph{A} bzw. \emph{GerO} erfasst wird. Die Schlussfolgerung liegt somit nahe, dass die unterschiedlichen Beitragsarten eine unterschiedliche Lesetiefe, ausgedrückt durch die Sakkadendauer und -amplitude, im monolingualen Chat hervorrufen. Unter Bezugnahme auf die absoluten Zahlen der Amplitude und Sakkadendauer kann weiterhin angenommen werden, dass die Lesetiefe der MÜ ins Deutsche (\emph{GerMT}) bzw. der deutschsprachigen Fremdbeiträge (\emph{B}) weitaus höher ist und eine stärkere kognitive Auslastung hervorruft als das Lesen der deutschen Originalbeiträge.

Wie in den Tabellen\,\ref{K6:tab:CatDe:sacamp} (S.\,\pageref{K6:tab:CatDe:sacamp}) und \ref{K6:tab:CatDe:sacdur} (S.\,\pageref{K6:tab:CatDe:sacdur}) erkenntlich wird, sind sowohl die durchschnittliche Sakkadenamplitude\is{Sakkade!-namplitude} als auch die durchschnittliche Sakkadendauer\is{Sakkade!-ndauer} vergleichbar mit früheren Studien \citep[]{rayner_eye_1998, gangl_lexical_2018, nikolova_binocular_2018}.

Ausgehend von der bestehenden Literatur \citep[]{beatty_task-evoked_1982} ist einschränkend festzustellen, dass die alleinige Betrachtung der sakkadischen Augenbewegung nur begrenzt umfassende Erkenntnisse im Rahmen dieser Studie liefert. Zwar ist eine generelle Untersuchung des sakkadischen Blickverhaltens dahingehend sinnvoll, als es einen Teil der wissenschaftlichen Kartierung des Skype Translators darstellt. Es bestehen offenkundige Unterschiede im sakkadischen Blickverhalten der Proband{\textperiodcentered}innen zwischen den beiden Versuchssettings. Innerhalb der beiden Settings zeichnen die ein- und ausgehenden Nachrichten jeweils markante sakkadische Augenbewegungen ab. Jedoch wird ebenso sichtbar, dass die Betrachtung der Sakkaden allein nicht fruchtbar ist, sondern in jedem Fall die Notwendigkeit einer kombinierten Untersuchung mit den fixatorischen Augenbewegungen besteht. Die alleinige Betrachtung der Sakkaden ist schlichtweg in diesem Rahmen, besonders aufgrund der im Vergleich zu anderen Studien vergleichsweise kleinen Schriftgröße, zu ungenau.

%------------------------------------------------------------------------

\section{Blinde Flecken}
\label{K7:sec:BlindeFlecke}

%------------------------------------------------------------------------

% And where do we go from here?

An dieser Stelle soll noch auf verschiedene denkbare Ausweitungen bzw. Vertiefungen des Forschungsthemas eingegangen werden. 

\subsection{Controlled World}\label{K7:para:controlled-world}

Der Bereich der kontrollierten Umwelt (\emph{controlled world}) lässt sich gut mit den Angaben in den Ausgangsfragebögen verknüpfen. Die Studienteilnehmer{\textperiodcentered}innen wurden dort gefragt, unter welchen Umständen sie eine Technologie wie den Skype Translator noch einmal nutzen würden. Die Rückmeldungen waren zwiespältig: Mehrere Proband{\textperiodcentered}innen befürworteten den Einsatz des Skype Translators im Rahmen der Arbeit oder für berufliche Kontakte. Diese Meinung zielt also unmittelbar auf Bereiche ab, die im Bereich der Qualitäts- und Risikobewertung von Sprachdienstleistern möglicherweise als Hoch-Risiko-Texte qualifiziert werden. Als Gegenpol hierzu lehnt ein anderer Teil der Versuchspersonen genau dieses Einsatzfeld ab und grenzt die Nutzbarkeit des Skype Translators auf Gespräche mit Freund{\textperiodcentered}innen und Bekannt{\textperiodcentered}innen ein. Zu beiden Standpunkten wird Forschung betrieben. So untersucht \citet{beiswenger_analyse_2017} etwa den Einsatz von Chat-Kommunikation am Arbeitsplatz. Die Charakteristika von Chat-Kommunikation in einem informellen Umfeld werden hingegen beispielsweise von \citet{beiswenger_zu_2017} oder \citet{fiser_whatsapp_2017} betrachtet.

Implizit werfen diese Aussagen also die Frage auf, wie stark eine Kommunikationssituation kontrolliert werden kann -- durch und mit Technologien wie dem Skype Translator.  

\subsection{Visual World Paradigm}\label{K7:para:vwp}
Die Erkenntnisse aus der Studie lassen in Bezug auf das Visual World Paradigm\footnote{S.a. \url{https://likan.info/en/paradigm/language/visual-world-paradigm/}, letzter Aufruf am \datum{}.} noch Lücken offen. Auf Grundlage der statistischen und der visuellen Inspektion der Daten bleibt bisweilen ungeklärt, ob die Proband{\textperiodcentered}innen willentlich und bewusst zwischen Original und maschineller Übersetzung wechseln, oder ob die Betrachtung der fremdsprachlichen Textboxen lediglich dem Reflex folgt, auf neue, eingehende Reize zu reagieren. Im ersten Fall könnten daher zwei Annahmen aufgestellt werden: Entweder suchen die Proband{\textperiodcentered}innen lediglich nach vergleichbaren Anhaltspunkten wie Eigennamen, Zahlen oder Wörtern, die dieselbe Herkunft und Zeichenform in beiden Sprachen aufweisen, und springen deshalb punktuell zwischen Original und MÜ, oder die Versuchspersonen vergleichen die beiden Einblendungen akribisch und versuchen tatsächlich den gesamten Chatbeitrag abzugleichen.

\subsection{Sichtbarkeit der Übersetzer{\textperiodcentered}innen}\label{K7:para:sichtbarkeit-uebersetzer}
\begin{sloppypar}
Auch wenn es nur informell und zuweilen höchst interpretativ zwischen den Antwortzeilen der Fragebögen durchschimmert, wirkt es so, als spiele die maschinelle Übersetzung in der Wahrnehmung der Proband{\textperiodcentered}innen allenfalls eine untergeordnete Rolle. Der/die Übersetzer{\textperiodcentered}in bleibt in dieser Technologie also weitestgehend unsichtbar und tritt nur dann ins Bewusstsein der Versuchsteilnehmer{\textperiodcentered}innen, wenn einer der Gesprächsabschnitte unzulänglich übersetzt wird. Der Status Quo jedoch scheint die Annahme zu sein, dass die Kommunikation gelingt und daher keine weitere Aufmerksamkeit auf dieses System fallen muss. Offenbar sind sich die Proband{\textperiodcentered}innen zwar bewusst, dass sie an einer von einer Maschine vermittelten Kommunikation teilnehmen (s. Beispiele\,\ref{K6:exp:situation:roboter} und \ref{K6:exp:situation:maschine}, S.\,\pageref{K6:exp:situation:roboter}). Dass es sich explizit um ein maschinelles Übersetzungssystem handelt, wird ihnen jedoch erst auf Rückfrage zur Qualität klar (s. Beispiele \ref{K6:exp:wahrnehmung:verwirrung} sowie \ref{K6:exp:wahrnehmung:verwirrung2}, S.\,\pageref{K6:exp:wahrnehmung:verwirrung}). Damit würde die Auseinandersetzung mit dem Skype Translator in die Richtung der bereits von \citet[]{venuti_translators_1986} problematisierten Sichtbarkeit der Übersetzer{\textperiodcentered}innen gehen. 
\end{sloppypar}

Über diesen Punkt sollte auch noch einmal eine Rückanbindung an die Verwendung des Katalanischen als Studiensprache (s. u. Abschnitt~\ref{K2:subsec:Kat-DigRaum}, S.\,\pageref{K2:subsec:Kat-DigRaum}) stattfinden. So könnte die geringe Wahrnehmung des MÜ-Systems auch der relativen Nähe der beteiligten Sprachen geschuldet sein. Wie bereits mehrfach als Denkanstoß angeboten, verfügen das Katalanische und das Deutsche über dasselbe Zeichensystem. Besonders Versuchspersonen, die eine andere romanische Sprache beherrschen, fühlen sich deshalb möglicherweise auch mit dem Katalanischen vertraut, wodurch im Bewusstsein die Tatsache abgeschwächt wird, dass eine MÜ in der Kommunikation zwischengeschaltet ist. Mit zunehmender Distanz zwischen den beteiligten Sprachen -- etwa bei einer Studienwiederholung mit Beteiligung des Russischen, Griechischen oder Chinesischen (s.\,u.) -- könnte daher vielleicht auch die Aufmerksamkeit für die maschinelle Übersetzung stärker werden.

\subsection{Pilotierung}\label{K7:para:pilotierung}
Diese Studie ist nicht als Erweiterung einer bestehenden Forschungslinie gedacht. Vielmehr ist es so, dass diese Studie eine erste Pilotierung eines bislang wenig betrachteten, aber immer mehr in den Vordergrund drängenden Feldes darstellt. Um die maschinell übersetzte, computervermittelte Kommunikation eingehender zu untersuchen, bieten sich mehrere Studien zur Erweiterung dieses Bereichs an: 

Über die Betrachtung des Skype Translators unter Beteiligung von Sprachen desselben Zeichensystems hinaus von Interesse und sicherlich eine zu schließende Forschungslücke ist die Untersuchung des Kommunikationsverhaltens von Personen, denen eines der beteiligten Schriftsystemen gänzlich unbekannt ist (s.\,o.). Eine auf die multi-subjektive und womöglich auch kollaborative Interaktion in Chat-Gesprächen ausgelegte Studie böte außerdem einen gänzlich anderen Blick auf internationale, synchrone CvK.

\begin{sloppypar}
Eine weitere -- bislang nur vorläufige -- Beobachtung ist, dass die Proband{\textperiodcentered}\linebreak[3] innen im katalanisch-deutschen Versuch kürzere Nachrichten schreiben als die Personen im monolingualen Setting. Das kann daran liegen, dass die Personen unbewusst auf die MÜ bzw. den fremdsprachigen Gegenüber Rücksicht nehmen, während sie sich im Falle des ein- und zugleich muttersprachlichen Versuchs der Aussagen sicherer sind. Einschränkend ist hierbei der Einsatz mehrerer Gesprächspartner{\textperiodcentered}innen in beiden Versuchsaufbauten zu beachten. Das individuelle Schreibverhalten der verschiedenen Personen ist ohne Zweifel ein Faktor, der bei der Untersuchung der Länge der einzelnen Chatbeiträge miteinbezogen werden muss.
\end{sloppypar}

Damit einher geht die Betrachtung des nunmehr umstrukturierten Layouts des Skype Translators. Da nun nicht mehr jeder einzelne Beitrag mit der einhergehenden MÜ direkt unterhalb der Nachricht angezeigt wird, sondern alle Nachrichten automatisch maschinell übersetzt in der jeweiligen Systemsprache des verwendeten Endgerätes dargestellt werden, besteht auch hier Raum für mögliche Hypothesen. Ein Vorteil dieses Layouts für naturalistisch ausgerichtete Studien mit Eye-Tracking"=Einsatz ist sicherlich die bessere Nachvollziehbarkeit der Augenbewegungen. Jeder beteiligten Person werden nicht mehr vier verschiedene Beitragsarten (bei zwei Personen jeweils Original und MÜ in den beteiligten Sprachen) angezeigt, sondern lediglich zwei. Und diese sind darüber hinaus auch im üblichen zweispaltigen Chat-Aufbau eindeutig voneinander getrennt. Erst mit Klick auf die jeweilige Nachricht wird dann das Original bzw. die MÜ an selbiger Stelle angezeigt.

Aus korpuslinguistischer Sicht sind zudem die im Rahmen dieses Projektes entstandenen natürlichen Sprachdaten aufzubereiten und zu untersuchen. Neu an diesem kleinen Korpus ist die Tatsache, dass es sich nicht nur um ein bilinguales Korpus aus deutschen und katalanischen Textchat-Beiträgen handelt, sondern dass darüber hinaus auch noch deren jeweilige maschinelle Übersetzung zur Verfügung steht. In diesem Zusammenhang können auch die Datensätze des Key-Loggings zur Untersuchung hinzugezogen werden. 

%     \begin{quote}
%     \textsc{C. Sinner via Slack zum 27. Col{\textperiodcentered}loqui Germanocatalà} \emph{Wäre vielleicht interessant, das mit der Frage danach zu verknüpfen, wie manche die Texte überhaupt verstehen. Ich denke da an diese Geschichte "in die Schnelligkeit bin ich der erste. In die Richtigkeit nicht so"}
%     \end{quote}

Die Studie sollte weiterhin in einzelnen Bereichen detaillierter wiederholt werden. So stellt \citet[]{obrien_eye_2009} in ihrer Untersuchung fest, dass es zu einem stetigen Gewöhnungsprozess bei der Verarbeitung der präsentierten Reize kommt. Daher schlossen sie die ersten Sekunden der Reizexponierung aus der Analyse aus, woraufhin sich auch die Resultate änderten. Eine erneute Durchführung der hier beschrieben Studie könnte genau diese Prozesse erforschen: Wie verhalten sich die Werte der Durchlaufzeit, der Verweildauer oder die Fixationsanzahl, wenn jede Nachrichtenkategorie auch zeitlich unterteilt wird? Angenommen, man betrachtete sowohl den Chatverlauf als Ganzen als auch die einzelnen Nachrichten zeitlich unterteilt nach Beginn, Mitte und Ende. Die Ergebnisse erlaubten sicherlich Aufschluss über Gewöhnungsprozesse im Umgang mit der MÜ sowie mit der Technologie an sich. 

Weiterhin wäre es wünschenswert, die Studie auch mit Personen durchzuführen, die keinem derart global vernetztem Umfeld wie das der Universität entstammen. Die hier rekrutierten Versuchspersonen sind allesamt Studierende in einem Alter und aus einer Generation, für die es weitaus selbstverständlicher ist, mit verschiedenen Sprachen im Alltag und im Beruf konfrontiert zu sein, als es noch vor wenigen Jahren der Fall war. Die Annahmen über die Problemlösungsstrategien der Teilnehmer{\textperiodcentered}innen sind also ganz andere als bei Personen, die nicht derart häufig und tief mit Fremdsprachen in Kontakt stehen. Die Rekrutierung von Personen, die nicht zur Generation Y gehören, ist daher womöglich als ertragreich zu bewerten. Es wäre demnach spannend zu beobachten, wie diese Zielgruppe mit derartiger Technologie umgeht. Gleiches gilt auch für Personen, die einen formell niedrigeren Bildungsgrad aufweisen als auf akademischem Niveau.

Ebenso bleibt ein soziologisches Spannungsfeld bestehen, auf dem in den nächsten Jahren sicherlich vielseitige und tiefgreifende Diskussionen geführt sowie stetige Neuerungen bekanntgegeben werden. Bislang stellt die Nutzeridentität ein mittel- bis schwerwiegendes Problem für die Maschine dar. Nicht nur der Umgang mit Eigennamen ist damit gemeint, sondern auch die Referenzierung über Personalpronomen und damit die Zuschreibung des eigenen Geschlechts. In den gesammelten Sprachdaten dieser Arbeit kam es beispielsweise zur Übersetzung des katalanischen Eigennamens \emph{Alba}. Die MÜ hatte diese Einheit nicht als solche erkannt, sondern das gemeinsprachliche Lexem identifiziert, dem im Deutschen „der Morgengrauen“ entspricht. Gleichermaßen problematisiert \citet{roser_warum_2018} den Umgang mit der Geschlechtsidentität sowohl zwischen Nutzer{\textperiodcentered}innen als auch zwischen Nutzer{\textperiodcentered}in und Maschine. Konkret weist der Autor darauf hin, dass die Maschinen bislang eher weiblich geprägt sind, was sich bereits in der von Werk eingestellten Stimme und dem Anredenamen wie \emph{Siri}, \emph{Alexa} oder \emph{Cortana} widerspiegelt.

%------------------------------------------------------------------------

\section{Zusammenfassung}
\label{K7:sec:zusammenfassung}

%------------------------------------------------------------------------

Die triangulierten Ergebnisse sowohl der Online-Umfrage als auch den beiden Versuchsaufbauten der Eye-Tracking-Studie zeichnen das Bild einer Sprachtechnologie, die in einem bemerkenswerten Zwiespalt steht. Einerseits hinkt Skype, sowohl was den Funktionsumfang als auch die Nutzungsintensität angeht, offenbar geläufigeren Diensten hinterher. Zugleich führt die Marktmacht Microsofts\is{Microsoft} jedoch dazu, dass der Dienst nicht ganz abgehängt wird. Dafür ist die Nutzung der Software besonders für Unternehmen nach wie vor zu groß. Aber auch der Kontakt zwischen Privatpersonen in verschiedenen Ländern scheint ein Bereich zu sein, der bevorzugt die Kommunikation über Skype verlangt. In diesem Kontext scheint es beinahe verwunderlich, dass ausgerechnet Skype im europäischen Kontext der bislang einzige Dienst ist, der eine maschinelle Übersetzung für die Chat-Kommunikation anbietet.

Die folgenden wesentliche Erkenntnisse lassen sich aus den Studien entnehmen: Einerseits ist die maschinelle Übersetzung ein starker Einflussfaktor bei der Kommunikation zwischen verschiedensprachigen Personen, die der Sprache des Gegenübers nicht mächtig sind. Die Vergleichsstudien belegen, dass das Vorhandensein der MÜ-Ausgabe einen Einfluss auf das Leseverhalten und die Wahrnehmung der Situation haben. 

Aus berufspraktischer Sicht weist \citet[214\psq]{beiswenger_analyse_2017} darauf hin, dass die CvK in vielen beruflichen Situation heutzutage einen gewöhnlichen Bestandteil darstellt. Auch wenn es ungewöhnlich anmuten mag, nutzen selbst Kolleg{\textperiodcentered}\linebreak[3]innen, die nur wenige Zentimeter voneinander entfernt sitzen, vermehrt CvK, um sich nicht aus dem Arbeitsfluss zu reißen und wenden sich erst dann persönlich zu, wenn eine grundlegende Klärung des Anliegens erfolgt ist. Mit dieser Perspektive bietet sich auch für den Skype Translator eine Chance. So könnte die Technologie folglich zur Kontaktaufnahme und anfänglichen Erörterung von Problemen zwischen Personen mit einem gewissen Anliegen verwendet werden (z.\,B.\ tatsächlich zur Vorbereitung auf einen Auslandsaufenthalt). Aufbauend auf diesem Erstkontakt kann dann festgelegt werden, wie die weitere Begleitung oder gar Beratung auszusehen hat. Die Kommunikation von Angesicht zu Angesicht ist dabei jedoch unerlässlich, sodass der Technologie eher die Rolle einer Ergänzung als denn einer Ersetzung zukommt.
