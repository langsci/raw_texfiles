\chapter{Arbeitshypothesen}\label{K4} % Change X to a consecutive number; for referencing this chapter elsewhere, use \ref{ChapterX}
\begin{sloppypar}
Das folgende Kapitel stellt die Arbeitshypothesen ausgehend von der Forschungsfrage dar und erklärt sie eingehend.
\end{sloppypar}

Während die Untersuchung der Nutzbarkeit von maschinell übersetzten Texten besonders im Umfeld des Posteditings über eine lange Forschungstradition verfügt, ist die Betrachtung im Spannungsfeld von Mensch und Maschine noch weitestgehend Neuland. Zwar gibt es immer wieder Versuche, maschinelle übersetzte Inhalte in den Wahrnehmungsbereich der Endnutzer{\textperiodcentered}innen von Kommunikationsmedien zu schieben, jedoch sind diese spärlich vertreten und lassen bislang eine spürbare Aufmerksamkeit vermissen. So bieten alle großen Social-Media-Plattformen seit mehreren Jahren bereits maschinell übersetzte Inhalte in anderen als der Systemsprache des verwendeten Endgeräts. Eine intensive Auseinandersetzung mit dem Mehrwert, den eine solche Erweiterung der Interaktions- und Kommunikationsmöglichkeiten bietet, hat bislang aus übersetzungswissenschaftlicher Sicht nur begrenzt, und dort eher im professionellen Rahmen, stattgefunden, sodass sich für die nachfolgende Analyse die nachstehenden Arbeitshypothesen definieren lassen. 

Da die computervermittelte Kommunikation zwischen Menschen eine zutiefst subjektive Komponente beinhaltet, werden über die kognitiven Prozesse hinaus auch Hypothesen auf Grundlage der zur Verfügung stehenden Fragebögen getroffen. Deshalb stellt sich hier die Frage, wie und ob sich die Nutzer{\textperiodcentered}innen des Skype Translators überhaupt bewusst sind, dass ein MÜ-System der Kommunikation zwischengeschaltet ist. Da die verwendete Kohorte aus gegenwärtig eingeschriebenen Studierenden besteht, die darüber hinaus allesamt aus einer Generation stammen, die ohnehin permanent vom medialen und technologischen Wandel begleitet ist, kann angenommen werden, dass die Kohorte generell affin mit derartiger Software umgeht. Konkreter bedeutet dies, dass die Proband{\textperiodcentered}innen eine Gesprächssituation über die gestellte Aufgabendauer aufrechterhalten können. Dies geschieht trotz bzw. wegen der Implikationen, die sich aus den im Theorieteil präsentierten Kommunikationsmodellen (Abschnitt \ref{K2:subsec:modelle-theorien-cvk}, S.\,\pageref{K2:subsec:modelle-theorien-cvk}) sowie der Charakterisierung als Nähe- und Distanzkommunikation in Bezug auf die Chatsituation ergeben. Sollten Verständigungsprobleme eintreten, werden diese von ihnen vermutlich wahrgenommen, aber auch ausgeglichen.  

Die Verbindung von Fragebögen und Eye-Tracking-Studie bietet Raum für weitere Annahmen. Die Kommunikationssituation wird auf dem Niveau der gestellten Aufgabe erfolgreich bewältigt. Eine fehlerfreie, publikationsreife Qualität der Kommunikation ist für die Teilnehmer{\textperiodcentered}innen nebensächlich. Der Erfolg der Kommunikation liegt vielmehr in dem flüssigen, reibungslosen Austausch zwischen den beteiligten Personen. 

Der monolinguale Versuchsaufbau dient als Referenzpunkt für die Betrachtung der Eye-Tracking-Daten. Hier interagieren zwei Personen über Skype, allerdings in derselben Sprache und ohne sonstige technologische Mittelung. Es ist daher davon auszugehen, dass sich Elemente der Chatkommunikation, wie sie im Theorieteil dargestellt wurden, auch hier aufzeigen lassen. Dazu gehört im einsprachigen Chat sicherlich besonders die Fokussierung auf die Beiträge des Gegenübers, da diese neue bzw. bislang unbekannte Informationen darstellen, die die Versuchspersonen noch verarbeiten müssen. Der Skype Translator hingegen erweitert diese Kommunikation um die entsprechende Ausgabe der MÜ in beide Sprachen. Die Versuchsteilnehmer{\textperiodcentered}innen im Setting Katalanisch-Deutsch werden deshalb vermutlich ihre Aufmerksamkeit auf alle vier verschiedenen Beitragsarten richten und das Kommunikations- sowie Leseverhalten dementsprechend anpassen. Mehr noch werden die Studienteilnehmer{\textperiodcentered}innen durch eben diese informative Erweiterung ein hohes Maß an Struktur innerhalb des Chats erkennen lassen, dass sich insbesondere an den aktuellsten Nachrichten orientiert.   

Weiterhin ergeben sich mehrere Annahmen aus dem Aufeinandertreffen von Mensch und Maschine. Die erste ist, dass sich die Elemente der Chatkommunikation, wie sie im theoretischen Teil aufbereitet wurden, auch in dieser um die MÜ-Ausgabe erweiterten Konstellation wiederfinden lassen. Deshalb kann zweitens erwartet werden, dass die unterschiedlichen Beitragsarten einen Einfluss auf das Kommunikationsverhalten haben. Dies lässt sich sowohl ausgehend von den absoluten Werten der Eye-Tracking-Daten als auch auf Grundlage der inferenzstatistischen Untersuchung erkennen. 

Das Leseverhalten ist sowohl markant für die jeweiligen Originalbeiträge als auch deren MÜ-Ausgabe und für die jeweilige Sprache, in der sie verfasst sind. Es stellt sich also die Frage, wie diese vier verschiedenen Beitragsarten konkret wahrgenommen werden. Von besonderem Interesse ist dabei die Wahrnehmung der MÜ-Ausgabe, da zu vermuten ist, dass diese trotz aller Fortschritte in den vergangenen Jahren noch immer in irgendeiner Weise fehlerbehaftet ist und sich auch im Leseverhalten der Studienteilnehmer{\textperiodcentered}innen widerspiegelt.

Da bewusst nur Personen ohne Vorkenntnisse im Katalanischen ausgewählt wurden, kann angenommen werden, dass die Proband{\textperiodcentered}innen ein Leseverhalten an den Tag legen, das sich zwischen Original und MÜ-Ausgabe unterscheidet und zugleich jedoch aufeinander bezieht. Mit Rückbezug auf den Umgang mit Wissen (Abschnitt~\ref{K2:sec:Wissensrepräsentation}, S.\,\pageref{K2:sec:Wissensrepräsentation}), besonders die Tatsache, dass Lesen und Schreiben auch immer Produktion, Integration und Konstruktion von Wissen seien, werden sich dementsprechend eindeutige Muster in der Betrachtung der einzelnen Indikatoren finden lassen. In diesem Zusammenhang steht die Vermutung, dass katalanischsprachige Originalbeiträge ebenfalls von den Proband{\textperiodcentered}innen in den Strukturierungsprozess der Kommunikation integriert werden.
