\author{Felix Hoberg}
\title{Informationsintegration in mehrsprachigen Textchats}
\subtitle{Der Skype Translator im Sprachenpaar Katalanisch-Deutsch}
\renewcommand{\lsSeries}{tmnlp}
\renewcommand{\lsSeriesNumber}{17}
\ISBNdigital{978-3-96110-337-9}
\ISBNhardcover{978-3-98554-033-4}
\BookDOI{10.5281/zenodo.5902971}
\typesetter{Felix Hoberg, Felix Kopecky}
\proofreader{Andreas Hölzl}
\renewcommand{\lsID}{340}
\BackBody{Die vorliegende Arbeit widmet sich der Informationsintegration in maschinell \"ubersetzten, mehrsprachigen Textchats am Beispiel des Skype Translators im Sprachenpaar Katalanisch-Deutsch. Der Untersuchung von Textchats dieser Konfiguration wurde sich bislang nur wenig zugewendet. Deshalb wird der zun\"achst grundlegend explorativ ausgerichteten Forschungsfrage nachgegangen, wie Personen eine maschinell \"ubersetzte Textchat-Kommunikation wahrnehmen, wenn sie nicht der Sprache des Gegen\"ubers mächtig sind. Damit einher geht auch die Untersuchung der Informationsextraktion und -verarbeitung zwischen Nachrichten, die in der eigenen Sprache verfasst wurden, und der Ausgabe der Maschinellen \"Ubersetzung.

Zur Erfassung des Nutzungsverhalten im Umgang mit Skype und dem Skype Translator wurde mit einer deutschlandweit an Studierende gesendeten Online-Umfrage gearbeitet. In einer zweiteiligen, naturalistisch orientierten Pilotstudie unter Einsatz des Eye-Trackers wurde das Kommunikationsverhalten von Studierenden mit deutscher Muttersprache einerseits in maschinell vom Skype Translator \"ubersetzten Chats mit katalanischen Muttersprachler{\textperiodcentered}innen und andererseits, als Referenz, in monolingualen, rein deutschsprachigen Chats ohne Skype Translator untersucht. Bei den Teilnehmer{\textperiodcentered}innen an diesen Studien handelt es sich um zwei unabhängige Gruppen. Beide wurden ebenfalls mit Frageb\"ogen zum Nutzungsverhalten und zu den Eindr\"ucken des Skype Translators erfasst.

Das sicher \"uberraschendste Ergebnis der Studie ist, dass die Versuchspersonen einen substanziellen Teil der Chatkommunikation auf der M\"U-Ausgabe in beiden beteiligten Sprachen verbringen. Die Untersuchung der Sakkaden und Regressionen deutet auf einen sprunghaften Wechsel zwischen Originalnachricht und M\"U hin. Der Schwerpunkt der Aufmerksamkeit liegt dabei konsequent auf den neusten Nachrichten. Es ist daher anzunehmen, dass die Versuchspersonen die M\"U-Ausgabe aktiv in die Kommunikation miteinbeziehen und wesentliche Informationen zwischen Original und M\"U abzugleichen versuchen.}
\lsCoverTitleSizes{40pt}{14mm}
