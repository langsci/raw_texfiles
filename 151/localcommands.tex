%add all your local new commands to this file

\newcommand{\form}[1]{\mbox{\emph{#1}}}
\newcommand{\uf}[1]{\mbox{/#1/}}

% borrowed from expex and converted from plan tex to latex
\newcommand{\judge}[1]{{\upshape #1\hspace{0.1em}}}
\newcommand{\ljudge}[1]{\makebox[0pt][r]{\judge{#1}}}

\newcommand\tikzmark[1]{\tikz[remember picture, baseline=(#1.base)] \node[anchor=base,inner sep=0pt, outer sep=0pt] (#1) {#1};} % for adding decorations, arrows, lines, etc. to text
\newcommand\tikzmarknamed[2]{\tikz[remember picture, baseline=(#1.base)] \node[anchor=base,inner sep=0pt, outer sep=0pt] (#1) {#2};} % for adding decorations, arrows, lines, etc. to text
\newcommand\tikzmarkfullnamed[2]{\tikz[remember picture, baseline=(#1.base)] \node[anchor=base,inner sep=0pt, outer sep=0pt] (#1) {\vphantom{X}#2};} % for adding decorations, arrows, lines, etc. to text; this one works best for decorations above a line of text because it adds in the heigh of a capital letter and takes two arguments - one for the node name and one for the printed text

\newcommand{\sub}[1]{$_{\text{#1}}$} % for non-math subscripts
\newcommand{\subit}[1]{\sub{\textit{#1}}} % for italics non-math subscripts
\newcommand{\1}{\rlap{$'$}\xspace} % for the prime in X' (the \rlap command allows the prime to be ignored for horizontal spacing in trees, and the \xspace command allows you to use this in normal text without adding \ afterwards). This isn't crucial, but it helps the formatting to look a little better.

% Aissen:
\newcommand\tikzmarkfull[1]{\tikz[remember picture, baseline=(#1.base)] \node[anchor=base,inner sep=0pt, outer sep=0pt] (#1) {\vphantom{X}#1};} % for adding decorations, arrows, lines, etc. to text; this one works best for decorations above a line of text because it adds in the heigh of a capital letter and takes one argument that serves as the name and the printed text
\newcommand{\bridgeover}[2]{% for a line that creates a bridge over text, connecting two nodes
	\begin{tikzpicture}[remember picture,overlay]
	\draw[thick,shorten >=3pt,shorten <=3pt] (#1.north) |- +(0ex,2.5ex) -| (#2.north);
	\end{tikzpicture}
}
\newcommand{\bridgeoverht}[3]{% for a line that creates a bridge over text, connecting two nodes and specifing the height of the bridge
	\begin{tikzpicture}[remember picture,overlay]
	\draw[thick,shorten >=3pt,shorten <=3pt] (#2.north) |- +(0ex,#1) -| (#3.north);
	\end{tikzpicture}
}
\newcommand{\bridgeoverex}{\vspace*{3ex}} % place before an example that has a \bridgeover so that there is enough vertical space for it

% Chung:
\newcommand{\lefttabular}[1]{\begin{tabular}{p{0.5in}}#1\end{tabular}}

% Kaisse:
\newcommand{\mgmorph}[1]{|(#1)| {#1}}
\newcommand{\mgone}[2][$\times$]{\node at (#2.base) [above=2ex] (1#2) {\vphantom{X}#1};}
\newcommand{\mgtwo}[2][$\times$]{\mgone{#2} \node at (#2.base) [above=4.5ex] (2#2) {\vphantom{X}#1};}
\newcommand{\mgthree}[2][$\times$]{\mgtwo{#2} \node at (#2.base) [above=7ex] (3#2) {\vphantom{X}#1};}
\newcommand{\mgftl}[1]{\node at (1#1) [left] {(};}
\newcommand{\mgftr}[1]{\node at (1#1) [right] {)};}
\newcommand{\mgfoot}[2]{\mgftl{#1}\mgftr{#2}}
\newcommand{\mgldelim}[2]{\node at (#2.west) [left,inner sep = 0pt, outer sep = 0pt] {#1};}
\newcommand{\mgrdelim}[2]{\node at (#2.east) [right,inner sep = 0pt, outer sep = 0pt] {#1};}

\newcommand{\squish}{\hspace*{-3pt}}

% Kavitskaya:
\newcommand{\assoc}[2]{\draw (#1.south) -- (#2.north);}
% \newcolumntype{L}{>{\raggedright\arraybackslash}X}

% Lepic & Padden:
\newcommand{\fitpic}[1]{\resizebox{\hsize}{!}{\includegraphics{#1}}} % from http://tex.stackexchange.com/a/148965/42880
\newcommand{\signpic}[1]{\includegraphics[width=1.36in]{#1}}
% \newcolumntype{C}{>{\centering\arraybackslash}X}

% Spencer:

\newcommand{\textex}[1]{\textit{#1}\xspace}
\newcommand{\lxm}[1]{\textsc{#1}\xspace}

% Thrainsson:

\renewcommand{\textasciitilde}{\char`~} % for use with TTF/OTF fonts (see comments on http://tex.stackexchange.com/a/317/42880)
\newcommand{\tikzarrow}[2]{% for an arrow connecting two nodes
\begin{tikzpicture}[remember picture,overlay]
\draw[thick,shorten >=3pt,shorten <=3pt,->,>=stealth] (#1) -- (#2);
\end{tikzpicture}
}

\newlength{\padding}
\setlength{\padding}{0.5em}
\newcommand{\lesspadding}{\hspace*{-\padding}}
\newcommand{\feat}[1]{\lesspadding#1\lesspadding}

% Hammond

\usepackage[]{graphicx}\usepackage[]{xcolor}
%% maxwidth is the original width if it is less than linewidth
%% otherwise use linewidth (to make sure the graphics do not exceed the margin)
\makeatletter
\def\maxwidth{ %
  \ifdim\Gin@nat@width>\linewidth
    \linewidth
  \else
    \Gin@nat@width
  \fi
}
\makeatother

\definecolor{fgcolor}{rgb}{0.345, 0.345, 0.345}
\newcommand{\hlnum}[1]{\textcolor[rgb]{0.686,0.059,0.569}{#1}}%
\newcommand{\hlstr}[1]{\textcolor[rgb]{0.192,0.494,0.8}{#1}}%
\newcommand{\hlcom}[1]{\textcolor[rgb]{0.678,0.584,0.686}{\textit{#1}}}%
\newcommand{\hlopt}[1]{\textcolor[rgb]{0,0,0}{#1}}%
\newcommand{\hlstd}[1]{\textcolor[rgb]{0.345,0.345,0.345}{#1}}%
\newcommand{\hlkwa}[1]{\textcolor[rgb]{0.161,0.373,0.58}{\textbf{#1}}}%
\newcommand{\hlkwb}[1]{\textcolor[rgb]{0.69,0.353,0.396}{#1}}%
\newcommand{\hlkwc}[1]{\textcolor[rgb]{0.333,0.667,0.333}{#1}}%
\newcommand{\hlkwd}[1]{\textcolor[rgb]{0.737,0.353,0.396}{\textbf{#1}}}%
\let\hlipl\hlkwb

\usepackage{framed}
\makeatletter
\newenvironment{kframe}{%
 \def\at@end@of@kframe{}%
 \ifinner\ifhmode%
  \def\at@end@of@kframe{\end{minipage}}%
  \begin{minipage}{\columnwidth}%
 \fi\fi%
 \def\FrameCommand##1{\hskip\@totalleftmargin \hskip-\fboxsep
 \colorbox{shadecolor}{##1}\hskip-\fboxsep
     % There is no \\@totalrightmargin, so:
     \hskip-\linewidth \hskip-\@totalleftmargin \hskip\columnwidth}%
 \MakeFramed {\advance\hsize-\width
   \@totalleftmargin\z@ \linewidth\hsize
   \@setminipage}}%
 {\par\unskip\endMakeFramed%
 \at@end@of@kframe}
\makeatother

\definecolor{shadecolor}{rgb}{.97, .97, .97}
\definecolor{messagecolor}{rgb}{0, 0, 0}
\definecolor{warningcolor}{rgb}{1, 0, 1}
\definecolor{errorcolor}{rgb}{1, 0, 0}
\newenvironment{knitrout}{}{} % an empty environment to be redefined in TeX

\usepackage{alltt}

%revised version started: 12/17/16

%NEEDS: allbib.bib - already added to the master bibliography file.
%cited references only: bibexport -o mhTMP.bib main1-blx.aux
%PLUS sramh-img*, sramh.tex

%added stuff
\newcommand{\add}[1]{\textcolor{blue}{#1}}
%deleted stuff
\newcommand{\del}[1]{\textcolor{red}{(removed: #1)}}
%uncomment these to turn off colors
\renewcommand{\add}[1]{#1}
\renewcommand{\del}[1]{}

%shortcuts
\newcommand{\w}{\ili{Welsh}}
\newcommand{\e}{\ili{English}}
\newcommand{\io}{Input Optimization}




 \newcommand{\hand}{\ding{43}}
\newcommand{\rot}[1]{\begin{rotate}{90}#1\end{rotate}} %shortcut for angled text%  
\newcommand{\rotcon}[1]{\raisebox{-5ex}{\hspace*{1ex}\rot{\hspace*{1ex}#1}}}

%\usepackage{langsci-optional}
\usepackage{langsci-gb4e}
\usepackage{langsci-lgr}

\usepackage{listings}
\lstset{basicstyle=\ttfamily,tabsize=2,breaklines=true}

%added by author
% \usepackage{tipa}
\usepackage{multirow}
\graphicspath{{figures/}}
\usepackage{langsci-branding}

\usepackage{arydshln}
\usepackage{rotating}

%
\newcommand{\sent}{\enumsentence}
\newcommand{\sents}{\eenumsentence}
\let\citeasnoun\citet

\renewcommand{\lsCoverTitleFont}[1]{\sffamily\addfontfeatures{Scale=MatchUppercase}\fontsize{44pt}{16mm}\selectfont #1}
  
\newcommand{\tworow}[1]{\multirow{2}{*}{#1}}



\newenvironment{indentquote}{\list{}{\rightmargin0pt\leftmargin7mm} 
\item\relax}
{\endlist}
