\documentclass[output=paper,
modfonts
]{LSP/langsci}


%\usepackage{langsci-optional}
\usepackage{langsci-gb4e}
\usepackage{langsci-lgr}

\usepackage{listings}
\lstset{basicstyle=\ttfamily,tabsize=2,breaklines=true}

%added by author
% \usepackage{tipa}
\usepackage{multirow}
\graphicspath{{figures/}}
\usepackage{langsci-branding}


%
\newcommand{\sent}{\enumsentence}
\newcommand{\sents}{\eenumsentence}
\let\citeasnoun\citet

\renewcommand{\lsCoverTitleFont}[1]{\sffamily\addfontfeatures{Scale=MatchUppercase}\fontsize{44pt}{16mm}\selectfont #1}
  

\title{Where, if anywhere, are parameters? A critical historical overview of parametric theory}
\shorttitlerunninghead{Where, if anywhere, are parameters?}
\author{%
Frederick J. Newmeyer\affiliation{University of Washington, University of British Columbia, and Simon Fraser University}
}

\ChapterDOI{10.5281/zenodo.495465}
% \sectionDOI{} %will be filled in at production
% \epigram{}

\abstract{Since the late 1970s, crosslinguistic variation has generally been handled by means of UG-specified parameters. On the positive side, thinking of variation in terms of parameterized principles unleashed an unprecedented amount of work in comparative syntax, leading to the discovery of heretofore unknown morphosyntactic phenomena and crosslinguistic generalizations pertaining to them. On the negative side, however, both macroparameters and microparameters have proven themselves to be empirically inadequate and conceptually nonminimalist. Alternatives to parameters are grounded approaches, epigenetic approaches, and reductionist approaches, the last two of which seem both empirically and conceptually quite promising.
}

%\usepackage[hyperfootnotes=false]{hyperref}

\begin{document}
\maketitle

\section{Introduction}

The existence of crosslinguistic \isi{variation} has always been problematic
for syntacticians. If there is a
\isi{universal grammar}, one might ask, then why aren't all languages exactly
the same? In the earliest work in \isi{generative syntax}, characterizing the
space in which languages could differ, whether at the surface or at a
deep level, was not a priority. At the time, surface differences between
languages and \isi{dialects} were generally attributed to language-particular
rules or filters.

In the late 1970s, however, a strategy was developed that allowed the
simultaneous development of a rich theory of \isi{Universal Grammar} (UG)
along with a detailed account of the limits of crosslinguistic
\isi{morphosyntactic} variation. In this view, syntactic \isi{complexity} results
from the interaction of grammatical subsystems, each characterizable in
terms of its own set of general principles. The central goal of
syntactic theory now became to identify such systems and to characterize
the degree to which they might vary (be ``parameterized'') from language
to language. \citet{chomsky1995} describes succinctly how such \isi{variation}
might be accounted for in what, by the early 1980s, was called the
``principles-and-parameters'' (P\&P) approach.

\begin{quote}
Within the P\&P approach the problems of \isi{typology} and language \isi{variation}
arise in somewhat different form than before. Language differences and
\isi{typology} should be reducible to choice of values of \isi{parameters}. A major
research problem is to determine just what these options are, and in
what components of language they are to be found. \citep[6]{chomsky1995}
\end{quote}

The first mention of \isi{parameters}, I believe, was in \citet{chomsky1976}:

\begin{quote}
Even if conditions are language- or rule-particular, there are limits to
the possible diversity of grammar. Thus, such conditions can be regarded
as \isi{parameters} that have to be fixed (for the language, or for the
particular rules, in the worst case), in language learning \ldots{} It
has often been supposed that conditions on applications of rules must be
quite general, even universal, to be significant, but that need not be
the case if establishing a ``parameteric'' condition permits us to reduce
substantially the class of possible rules. \citep[315]{chomsky1976}
\end{quote}

An interesting question is why Chomsky at this point would propose
\isi{parameters}, since there is nothing in his 1976 paper that suggests that
they need to be incorporated into the theory. A possible answer is that
in the same year an MIT dissertation appeared \citep{kim1976} that showed
that \ili{Korean} obeys a form of the Tensed-S-Condition,\is{tense} even though \ili{Korean}
does not distinguish formally between tensed and non-tensed clauses.
That fact might have planted the seed for the idea of parameterized
principles. At around the same time, an ``external'' inspiration for
\isi{parameters} was provided by the work of Jacques Monod and François Jacob
\citep{monod1972,jacob1977}. Their idea was that slight differences in
timing and arrangement of regulatory mechanisms that activate genes
could result in enormous differences.  \citet[28]{berwick2011n}
has claimed that ``Jacob's model in turn provided part of the
inspiration for the \isi{Principles and Parameters (P\&P)} approach to
language \ldots{}''

Whatever the direct inspiration for parameterized principles might have
been, their adoption triggered an unprecedented explosion of work in
comparative syntax. One unquestionably positive consequence of the P\&P
approach to linguistic theory was to spur investigation of a wide
variety of languages, particularly those with structures markedly
different from some of the more familiar Western ones. The explanation
for this is straightforward. In earlier transformational grammar
(oversimplifying somewhat), one worked on the grammar of \ili{English}, the
grammar of \ili{Thai}, the grammar of Cherokee, and so on, and attempted to
extract universal properties of grammars from the principles one found
in common among these constructed grammars. But now the essential unity
of all grammars, within the limits of \isi{parametric variation}, was taken as
a starting point. One could not even begin to address the grammar
of some language without asking the question of how principles of \isi{Case},\is{case}
\isi{binding}, bounding, and so on are parameterized in that language. And
that in turn demanded that one have a rough feel for the degree of
parameterization possible for the principle. As Chomsky noted, to
delimit the domain of core grammar, we ``rely heavily on grammar-internal
considerations and comparative evidence, that is, on the possibilities
for constructing a reasonable theory of UG and considering its
\isi{explanatory power} in a variety of language types, with an eye open to
the eventual possibility of adducing evidence of other kinds'' \citep[9]{chomsky1981}.

The core idea of the P\&P approach is that both the principles of UG and
the possible parameter settings are part of our genetic endowment:\is{genetics}

\begin{quote}
{[}W{]}hat we ``know innately'' are the principles of the various
subsystems of S\textsubscript{0} {[}= the initial state of the language
faculty  --  FJN{]} and the manner of their interaction, and the
\isi{parameters} associated with these principles. What we learn are the
values of these \isi{parameters} and the elements of the periphery (along with
the lexicon,\is{lexicon} to which similar considerations apply). The language that
we then know is a system of principles with \isi{parameters} fixed, along with
a periphery of marked exceptions.\is{exception} \citep[150--151]{chomsky1986}
\end{quote}

The original idea was that there are a small number of \isi{parameters} and
small number of settings. This idea allowed two birds to be killed with
one stone. Parametric theory could explain the rapidity of \isi{acquisition},
given the poor input,\is{input} and explain the crosslinguistic distribution of
grammatical elements. As Norbert Hornstein noted:

\begin{quote}
The second reason in favor of parameter setting models has been their
ability to provide (at least in principle) an answer to Plato's Problem
{[}the fact that we know so much about language based on so little
direct evidence  --  FJN{]}. The idea is that construing language
\isi{acquisition} as parameter setting eases the problem faced by the child,
for setting parameter values is easier than learning the myriad possible
rules of one's native language. In other words, the PLD {[}= Primary
Linguistic Data  --  FJN{]} can be mined for parameter values more easily
than it can be for rules. \citep[165]{hornstein2009}
\end{quote}

The need to base a theory of \isi{parametric variation} on the investigation
of a wide variety of languages resulted in what Bernard Comrie, always a
major critic of the generative approach, referred to approvingly as ``one
of the most interesting recent developments in linguistic \isi{typology} {\ldots}
the entry of \isi{generative syntax} into the field'' \citep[458]{comrie1988}.
Comparative studies of the distribution of null-subjects, \isi{binding}
domains, configurationality, and so on became routine by the 1980s and
provided a generative interpretation of the kind of crosslinguistic
typological studies that were initiated by the work of Joseph Greenberg.
In this regard, it is instructive to observe Chomsky's changing
rhetorical evaluation of Greenbergian typological work. His first
reference to Greenberg was somewhat dismissive, noting that ``Insofar as
attention is restricted to surface structures, the most that can be
expected is the discovery of statistical tendencies, such as those
presented by \citet{greenberg1963n}'' \citep[118]{chomsky1965}. In 1981, Chomsky
offered what was perhaps his first favorable reference to this line of
research:

\begin{quote}
Universals of the sort explored by Joseph Greenberg and others have
obvious relevance to determining just which properties of the \isi{lexicon}
have to be learned in this manner in particular grammars  --  and to put
it in other terms just how much has to be learned as grammar develops in
the course of \isi{language acquisition}. \citep[95]{chomsky1981}
\end{quote}

By 1982 he was writing that ``Greenbergian universals {\ldots} are ultimately
going to be very rich. {\ldots} They have all the difficulties that people
know, they are ``surfacy,'' they are statistical, and so on and so
forth, but nevertheless they are very suggestive'' \citep[111]{chomsky1982}.
And in 1986, they are ``important, {\ldots} yielding many generalizations that
require explanation {\ldots}'' \citep[21]{chomsky1986}.

In this paper, I do not question the fertility of the research program
that was launched by the P\&P approach. What I do is to provide a
critical review of the various approaches that have been taken to
\isi{parameters} since the late 1970s, discussing their \emph{conceptual}
strengths and weaknesses. Given space limitations, my overview will in
places be unavoidably somewhat superficial. The paper is organized as
follows. Sections~\ref{sec:newmeyer:2} through~\ref{sec:newmeyer:5} outline various approaches that have been
taken with respect to \isi{parameters}: UG-principle-based, microparametric,
macroparametric, and interface-based, respectively. Some of the 
major conceptual and empirical problems with the classical
view of \isi{parameters} are outlined in \sectref{sec:newmeyer:6}, and \sectref{sec:newmeyer:7} discusses alternatives to the classical
approach. \sectref{sec:newmeyer:8} is a brief conclusion.

\section{Parameterized UG principles}\label{sec:newmeyer:2}

All of the subsystems of principles in the Government-Binding Theory
were assumed to be parameterized. Consider a few concrete examples:

\begin{exe}
\ex Examples of parameterized UG principles:
\begin{xlist}
\ex  BINDING \citep{lasnik1991}. Principle C is parameterized to allow for
sentences of the form \emph{John\textsubscript{i} thinks that
John\textsubscript{i} is smart} in languages like \ili{Thai} and \ili{Vietnamese}.

\ex GOVERNMENT \citep{manzini1987}. The notion ``Governing Category''
is defined differently in different languages.

\ex BOUNDING \citep{rizzi1982}. In \ili{English}, NP and S are bounding nodes for
Subjacency, in \ili{Italian} NP and S'.

\ex X-BAR \citep{stowell1981}. In \ili{English}, heads precede their complements; in
Japanese heads follow their complements.

\ex CASE and THETA-THEORY \citep{travis1989}. Some languages assign Case\is{case}
and/or Theta-roles to the left, some to the right.
\end{xlist}
\end{exe}

Fewer and fewer parameterized UG principles have been proposed in recent
years for the simple reason that there are fewer and fewer widely
accepted UG principles. The thrust of the \isi{Minimalist Program} (MP) has
been to reduce the narrow syntactic component and to reinterpret broad
universal principles as economy effects of efficient computation.
Economy principles are generally assumed not to be parameterized:

\begin{quote}
There is simply no way for principles of efficient computation to be
parameterized {[}\ldots{]}, it strikes me as implausible to entertain
the possibility that a principle like ``Shortest Move''\is{movement} could be active in
some languages, but not in others. Put differently, {[}\ldots{]} there
can be no \isi{parameters} within the statements of the general principles
that shape natural language syntax. \citep[210]{boeckx2011}
\end{quote}

On the same page Boeckx proposes the ``Strong Uniformity Thesis'':
Principles of narrow syntax are not subject to parameterization; nor are
they affected by lexical \isi{parameters}.

It should be noted that the very idea of looking for principles of UG
has fallen into disrepute in recent years. For example, Chomsky has
attributed to them what can only be described as negative qualities:

\begin{quote}
{[}T{]}ake the LCA (Linear Correspondence Axiom) [\citealt{kayne1994}]. If
that theory is true, then the phrase structure is just more complicated.
Suppose you find out that government is really an operative property.
Then the theory is more complicated. If ECP really works, well, too bad;
language is more like the spine {[}i.e., poorly designed  --  FJN{]} than
like a snowflake {[}i.e., optimally designed{]}. \citep[136]{chomsky2002}
\end{quote}

So if in theory there are very few UG principles and no \isi{parameters}
associated with them, then the question is where and how to capture
systematic \isi{crosslinguistic variation}.\is{variation} Given the organization of grammars
in a P\&P-type model, the simplest assumption to make is that one group
of languages contains a particular feature\is{features} attraction mechanism that
another group lacks, thus allowing the presence or absence of this
mechanism to divide languages into typological classes. Some early
examples can be illustrated by whether or not a feature\is{features} setting
determines whether V moves\is{movement} to I in a particular language \citep{emonds1978,pollock1989}, whether V moves to C (to derive V2 order)\is{word order} \citep{den1977a}, and whether N incorporates into V \citep{baker1988}.

A major debate within parametric theory has centered on the host of the
attracting feature.\is{features} In ``microparametric'' approaches, the locus of
\isi{variation} lies in individual functional heads. ``Macroparametric''
approaches are not so restricted. They will be discussed in \sectref{sec:newmeyer:3} and \sectref{sec:newmeyer:4}
respectively.

\section{The Borer-Chomsky Conjecture and microparametric approaches}\label{sec:newmeyer:3}

Hagit Borer, in \emph{Parametric Syntax} \citep{borer1984}, made two
proposals, which she may or may not have regarded as variants of each
other. One is that \isi{parameters} are restricted to the idiosyncratic
properties of lexical items, the other that they are restricted to the
inflectional\is{inflection} system. Borer wrote:

\begin{quote}
\ldots{}interlanguage \isi{variation} would be restricted to the idiosyncratic
properties of lexical items. These idiosyncrasies, which are clearly
learned, would then interact with general principles of UG in a
particular way. \citep[2--3]{borer1984}
\end{quote}

By way of example, she discussed a rule that inserts a \isi{preposition} in
Lebanese \ili{Arabic}  --  a rule that does not exist in \ili{Hebrew}:

\ea
ø −−−−−−\textgreater{} \emph{la} / {[}\textsubscript{PP} \ldots{}
NP{]}
\z

Along the same lines, \citet{manzini1987} pointed to
language-particular anaphors that have to be associated with \isi{parameters}:
\emph{cakicasin} and \emph{caki} in \ili{Korean}; \emph{sig} and \emph{hann}
in \ili{Icelandic}.

Now every language has thousands of lexical items, but nobody ever
entertained the possibility that every lexical item might be a locus for
\isi{parametric variation}. Borer's proposal that only inflectional\is{inflection} elements
provide the locus for \isi{parametric variation} was designed to forestall
this possibility. In the same book she wrote:

\begin{quote}
It is worth concluding this chapter by reiterating the conceptual
advantage that reduced all interlanguage \isi{variation} to the properties of
the inflectional\is{inflection} system. The inventory of inflectional\is{inflection} rules and of
grammatical formatives in any given language is idiosyncratic and
learned on the basis of \isi{input} data. \citep[29]{borer1984}
\end{quote}

The restriction of \isi{parameters} to the inflectional\is{inflection} system is a somewhat
different proposal than their restriction to lexical items. After all,
not all lexical items are part of the inflectional\is{inflection} system and not all
inflections\is{inflection} are lexical. However, Borer took ``inflectional''\is{inflection} in a pretty
broad sense, namely, to encompass Case\is{case} and agreement relations,
theta-role assignment, and so on. She recognized an immediate problem
here: Inflection-based\is{inflection} \isi{parameters} could not handle some of the best
known cases of \isi{crosslinguistic variation}\is{variation} such as differences in
head-order and extraction possibilities.

In any event, the hypothesis that the locus of \isi{parametric variation} is
restricted to exclude major lexical categories came to be known as the
``Borer-Chomsky Conjecture''.

\citet{borer1984} appeared before the distinction between lexical and
\isi{functional categories} had been elaborated. Once this distinction had
become well accepted, it seemed natural to associate \isi{parameters} with
functional heads, rather than with inflectional\is{inflection} items. This idea was
first proposed as The Functional Parameterization Hypothesis (FPH) in
\citet{fukui1988}. In this view, only functional elements in the \isi{lexicon}
(that is, elements such as Complementizer, Agreement, Tense,\is{tense} etc.) are
subject to \isi{parametric variation}.\footnote{Fukui himself exempted
  ordering restrictions from this hypothesis.}

It is important to stress that FPH is not a simple extension of the idea
that \isi{parameters} are inflection-located.\is{inflection} There have been countless
\isi{functional categories} proposed that have nothing to do with inflection,\is{inflection}
no matter how broadly this concept is interpreted. So adverbs, topic,
focus,\is{focus} and so on are typically thought to be housed in functional
categories, even though they are not in many languages ``inflectional''.\is{inflection}

Associating \isi{parameters} with functional heads has been claimed to have
both methodological and theoretical advantages. Methodologically, it
allows ``experiments'' to be constructed comparing two closely-related
variants, thereby pinpointing the possible degree of variation. The
ideal situation then would be to compare speech varieties that differ
from each other only in terms of (most ideally) one or, failing that,
only a few variables. Richard Kayne remarks:

\begin{quote}
If it were possible to experiment on languages, a syntactician would
construct an experiment of the following type: take a language, alter a
single one of its observable syntactic properties, examine the result
and see what, if any, other property has changed as a consequence. If
some property has changed, conclude that it and the property that was
altered are linked to one another by some abstract parameter. Although
such experiments cannot be performed, I think that by examining pairs
(and larger sets) of ever more closely related languages, one can begin
to approximate the results of such an experiment. To the extent that one
can find languages that are syntactically extremely similar to one
another, yet clearly distinguishable and readily examinable, one can
hope to reach a point such that the number of observable differences is
so small that one can virtually see one property covarying with another. \citep[5--6]{kayne2000}
\end{quote}

In other words, in Kayne's view, this ``micro\isi{parametric variation}'' (as he
called it) is the best testing ground for the hypothesis that syntactic
\isi{variation} can be reduced to a finite set of \isi{parameters}.

Along more theoretical lines, it has been claimed that
functional-category-situated micro\isi{parameters} impose a strong limit on
what can vary, crosslinguistic differences now being reduced to
differences in features,\is{features} thereby restricting learning to the \isi{lexicon}
\citep{kayne2000,roberts2010,thornton2013}.\footnote{As an anonymous reviewer points out, this claim is
  highly dependent on the nature of the \isi{features} and the role that they
  play in the system.} Indeed, Chomsky has often asserted that
micro\isi{parameters} are necessary in order to solve Plato's Problem:

\begin{quote}
Apart from lexicon,\is{lexicon} {[}the set of possible human languages{]} is a
finite set, surprisingly; in fact, a one-membered set if \isi{parameters} are
in fact reducible to lexical properties {[}associated with functional
categories  --  FJN{]} \ldots{} How else could Plato's problem be
resolved? \citep[26]{chomsky1991}
\end{quote}

\section{Macroparameters}\label{sec:newmeyer:4}

Not all minimalists have embraced the Borer-Chomsky Conjecture and
consequent turn to micro\isi{parameters}. Mark Baker, in particular, while not
denying that there are micro-level points of \isi{variation} between
languages, has defended what he calls ``macroparameters'' \citep{baker1996},
that is, parametric differences that cannot be localized in simple
differences in attracting \isi{features} of individual functional heads. He
gives as examples, among others, the Head Directionality Parameter (i.e.\
VO vs.\ OV), where \isi{functional categories} play no obvious role, the
Polysynthesis Parameter, which in his account refers to the lexical
category ``Verb'', and an agreement parameter \citep{baker2008} distinguishing
\ili{Niger-Congo} languages from \ili{Indo-European} languages, which, in opposition
to a strong interpretation of the Borer-Chomsky Conjecture, applies to
the full range of \isi{functional categories}. Another example of a
\isi{macroparameter} is the compounding parameter of \citet[328]{snyder2001}, which
divides languages into those that allow formation of endocentric
compounds\is{compound} during the syntactic \isi{derivation} and those that do not. The
NP/DP \isi{macroparameter} of \citet{boskovic2011} distinguishes ``NP
languages'', which lack articles, permit left-branch extraction and
scrambling,\is{scrambling} but disallow NEG-raising, from ``DP languages'', which can
have articles, disallow left-branch extraction and scrambling,\is{scrambling} but allow
NEG-raising. And \citet{huang2007} points to many features that distinguish
\ili{Chinese}-type languages from \ili{English}-type languages, including a
generalized classifier system, no \isi{plural} morphology,\is{morphology} extensive use of
light verbs, no agreement, tense,\is{tense} or \isi{case} morphology,\is{morphology} no overt
\emph{wh}-movement,\is{movement} and radical pro-drop.

Baker and other advocates of macro\isi{parameters} share the conviction long
held by advocates of holistic \isi{typology} that languages can be partitioned
into macro-\isi{scope} broad classes, typically (or, at least, ideally) where
the setting of one feature\is{features} entails a cascade of shared typological
properties. As Baker puts it, ``the macroparametric view is that there
are at least a few simple (not composite) \isi{parameters} that define
typologically distinct sorts of languages'' \citep[355]{baker2008}.

\section{Parameters as being stated at the interfaces}\label{sec:newmeyer:5}

Under the perspective that \isi{parameters} are stated at the \isi{interfaces},
lexical items are subject to a process of generalized late insertion of
semantic, formal, and morphophonological \isi{features} after the syntax,
which is where all \isi{variation} would take place. Or, as another
possibility, the parametric differences would derive from the way in
which such \isi{features} are interpreted by the \isi{interfaces} or by processes
that manipulate the \isi{features} on the path from spell out to the
\isi{interfaces}. There has been some debate as to whether there is parametric
variation at the Conceptual-Intentional (C-I) interface. Angel Gallego
remarks:

\begin{quote}
\ldots{} it would be odd for semantic \isi{features} to be a source of
variation,\is{variation} which leaves us with formal and morphophonological \isi{features}
as more likely suspects. \ldots{} Considered together, the observations
by \citet{chomsky2001b} and \citet{kayne2005,kayne2008} appear to place \isi{variation} in
the morphophonological manifestation of closed classes (i.e. functional
categories, which contain unvalued features).''\is{features} \citep[543--544]{gallego2011}
\end{quote}

However, for \citet{ramchand2008} the narrow syntax provides a
``basic skeleton'' to C-I, but languages vary in terms of how much their
lexical items explicitly encode about the reference of variables like T,
Asp, and D.

\section{Conceptual and empirical problems with parameters}\label{sec:newmeyer:6}

Before moving on to nonparametric approaches to variation, it would be
useful to highlight some of the main problems with the classic view of
\isi{parameters} as being innately-provided grammatical constructs (for an
earlier discussion, see \citealt{newmeyer2005}).

\subsection{No \isi{macroparameter} has come close to working}

The promise of \isi{parameters} in general and macro\isi{parameters} in particular
is that from the interaction of a small number of simply-stated
\isi{parameters}, the vast complexity of human language morphosyntax might be
derived. As Martin Haspelmath put it:

\begin{quote}
According to the principles and \isi{parameters} vision, it should be possible
at some point to describe the syntax of a language by simply specifying
the settings of all syntactic \isi{parameters} of \isi{Universal Grammar}. We would
no longer have any need for thick books with titles like \emph{The
Syntax of Haida} (cf. \citeauthor{enrico2003} 2003's 1300-page work), and instead we
would have a simple two-column table with the \isi{parameters} in the first
column and the positive or negative settings in the second column.
\citep[80]{haspelmath2008}
\end{quote}

Needless to say, nothing remotely like that has been achieved. The
problem is that ``few of the implicational statements at the heart of the
traditional Principles-and-Parameters approach have stood the test of
time'' \citep[216]{boeckx2011}. The clustering effects are simply not very
robust. The two most-studied macro\isi{parameters}, I believe, are the Null
Subject (\isi{Pro-drop}) and the Subjacency \isi{parameters}, neither of which is
much evoked in recent work. As for the former: ``History has not been kind to
to the \isi{Pro-drop} Parameter as originally stated'' \citep[352]{baker2008}. And
Luigi Rizzi notes that ``In retrospect, {[}subjacency effects{]} turned
out to be a rather peripheral kind of variation. Judgments are complex,
graded, affected by many factors, difficult to compare across languages,
and in fact this kind of \isi{variation} is not easily amenable to the general
format of \isi{parameters} \ldots{}'' \citep[16]{rizzi2014}.

\subsection{``Microparameter'' is just another word for ``language-particular
rule''}\label{sec:newmeyer:6.2}

Let's say that we observe two \ili{Italian} \isi{dialects}, one with a
\emph{do}-support-like structure and one without. We could posit a
microparametric difference between the dialects, perhaps hypothesizing
that one contains an attracting feature\is{features} that leads to \emph{do}-support
and one that does not. But how would such an hypothesis differ in
substance from saying that one dialect has a rule of \emph{do}-support
that the other one lacks? Indeed, Norbert Hornstein has stressed that
``microparameter'' is just another words for ``rule'':\footnote{See \citet[22--27]{rizzi2014} for a defense of the idea that micro\isi{parameters} are not
  merely rules under a different name and \citet{boeckx2014} for a reply to
  Rizzi.}

\begin{quote}
Last of all, if \isi{parameters} are stated in the \isi{lexicon} (the current view),
then parametric differences reduce to whether a given language contains
a certain lexical item or not. As the \isi{lexicon} is quite open ended, even
concerning functional items as a glance at current cartographic work
makes clear, the range of \isi{variation} between grammars/languages is also
open ended. In this regard it is not different from a rule-based
approach in that both countenance the possibility that there is no bound
on the possible differences between languages. \citep[165]{hornstein2009}
\end{quote}

Michal Starke has made a similar observation:

\begin{quote}
Thirty years ago, if some element moved in one language but not in
another, a \isi{movement} rule would be added to one language but not to the
other. Today, a feature\is{features} ``I want to move'' (``EPP', ``strength', etc.) is
added to the elements of one language but not of the other. In both
cases (and in all attempts between them), \isi{variation} is expressed by
stipulating it. Instead of a theory, we have brute force markers.
\citep[140]{starke2014}
\end{quote}

\subsection{There would have to be hundreds, if not thousands, of parameters}

Tying \isi{parameters} to \isi{functional categories} was a strong conjecture in the
1980s, since there were so few generally recognized functional
categories at the time. There were so few, in fact, that it was easy to
believe that only a small number of \isi{parameters} would be needed. \citet[112]{pinker1994}, for example, speculated that there are just ``a few mental
switches'. \citet[259]{lightfoot1999} suggested that there are about 30 to 40
\isi{parameters}. For \citet[16]{adger2003n}, ``There are only a few \isi{parameters}'.
\citet{roberts2005} increased the presumed total to between 50
and 100. \citet[734]{fodor2001} was certainly correct when she observed that
``it is standardly assumed that there are fewer \isi{parameters} than there are
possible rules in a rule-based framework; otherwise, it would be less
obvious that the amount of learning to be done is reduced in a
parametric framework'. At this point in time, many hundreds of
\isi{parameters} have been proposed. \citet{gianollo2008} propose 47 \isi{parameters} for \isi{DP} alone on the basis of 24 languages,
only five of which are non-\ili{Indo-European}, and in total representing only
3 families. \citet{longobardi2011} up the total to 63 binary
\isi{parameters} in DP. As Cedric Boeckx has stressed: ``It is not at all clear
that the exponential growth of \isi{parameters} that syntacticians are willing
to entertain is so much better a situation for the learner than a model
without \isi{parameters} at all'' \citep[157]{boeckx2014}.

One way to circumvent this problem would be to posit nonparametric
differences among languages, thereby maintaining the possibility of a
small number of \isi{parameters}. Let us examine this idea now.

\subsection{Nonparametric differences among languages undercut the entire
parametric program}

Are all \isi{morphosyntactic} differences among languages due to differences
in parameter setting? Generally that has been assumed not to be the
case. Charles Yang was expressing mainstream opinion when he wrote that
``\ldots{} it seems highly unlikely that all possibilities of language
\isi{variation} are innately specified \ldots{}' \citep[191]{yang2011n}. From the
beginning of the parameteric program it has been assumed that some
\isi{features} are extraparametric. Outside of (parametrically relevant) core
grammar are:

\begin{quote}
\ldots{} borrowings, historical residues, inventions, and so on, which
we can hardly expect to  --  and indeed would not want to  --  incorporate
within a principled theory of UG. \ldots{} How do we delimit the domain
of core grammar as distinct from marked periphery? \ldots{} {[}We{]}
rely heavily on grammar-internal considerations and comparative
evidence, that is, on the possibilities for constructing a reasonable
theory of UG and considering its explanatory potential in a variety of
language types \ldots{} \citep[8--9]{chomsky1981}
\end{quote}

In other words, some language-particular \isi{features} are products of
extraparametric language-particular rules. Consider, for example, the
treatment of \ili{Hixkaryana} in \citet{baker2001}, based on an earlier proposal in
\citet{kayne1994}. This language for the most part manifests OVS word order:\is{word order}

\ea 
\langinfo{Hixkaryana}{}{\citealt{derbyshire1985}}\\
 \gll Kanawa yano toto\\  
canoe took person\\
\glt `The man took the canoe.' 
\z

One's first thought might be that what is needed is a parameter allowing
for OVS order.\is{word order} But in fact Baker rejects the idea that a special \isi{word
order} parameter is involved here. Rather, he argues that \ili{Hixkaryana} is
(parametrically) SOV and allows the \isi{fronting} of VP by a \isi{movement} rule:

\ea S{[}OV{]} → {[}OV{]}S
\z

In other words, in this account \isi{word order} is determined \emph{both} by
a parameter and a language-specific rule.

It is quite implausible that every syntactic difference between
languages and \isi{dialects} results from a difference in parameter settings.
Consider the fact that there are several dozen systematic
\isi{morphosyntactic} differences between the Norfolk dialect and standard
\ili{British English}\il{Norfolk English} \citep{trudgill2003}, most of which appear to be analytically
independent. If each were to be handled by a difference in parameter
setting, then, extrapolating to all of the syntactic distinctions in the
world's languages, there would have to be thousands  --  if not millions
 --  of \isi{parameters}. That is obviously an unacceptable conclusion from an
evolutionary\is{evolution} standpoint, given that the set of \isi{parameters} and their
possible settings is, by hypothesis, innate. Furthermore, many processes
that can hardly be described as ``marginal'' have been assumed to apply in
PF syntax (where the standard view, I believe, is that \isi{parameters} are
not at work), including \isi{extraposition} and \isi{scrambling} \citep{chomsky1995};
object shift \citep{holmberg1999,erteschik-shir2005}; head movements\is{movement}
\citep{boeckx2001}; the \isi{movement} deriving V2 order \citep{chomsky2001b}; linearization (i.e. VO vs. OV)\is{word order} \citep{chomsky1995,takano1996,fukui1998,uriagereka1999}; and even \emph{Wh}-movement\is{movement}
\citep{erteschik-shir2005}.

I think that it is fair to say that, after 35 years of investigation,
nobody has a clear idea about which syntactic differences should be
considered parametric and which should not be.\footnote{But see \citet{smith2009n} for an interesting discussion of criteria for
  distinguishing parametric and nonparametric differences.} But one
thing seems clear: If learners need to learn rules anyway, very little
is gained by positing \isi{parameters}.

\subsection{Parametric theory is arguably inherently unminimalist}

There are a number of ways that the assumptions of the \isi{Minimalist Program}\is{Minimalist Program} have entailed
a rethinking of \isi{parameters} and the division of labor among the various
components for the handling of \isi{variation}. In one well-known formulation,
``FLN {[}= the faculty of language in the narrow sense  --  FJN{]}
comprises only the core computational mechanisms of \isi{recursion} as they
appear in narrow syntax and the mapping to the \isi{interfaces}' \citep[1573]{hauser2002}. Where might \isi{parameters} fit into such a
scenario? In one view, ``\ldots{} if minimalists are right, there cannot
be any parameterized principles, and the notion of \isi{parametric variation}
must be rethought.' \citep[206]{boeckx2011}. That is, given that the main
thrust of the minimalist program is the reduction to the greatest extent
possible of the elements of UG, there would seem to be no place for
innately-specified \isi{parameters}.

Despite the above, a great deal of work within the general envelope of
the MP is still devoted to fleshing out \isi{parameters}, whether micro or
macro. For example, \citet[202--203]{yang2011n} writes that ``a finite space of
\isi{parameters} or constraints\is{constraint} is still our best bet on the logical problem
of \isi{language acquisition}'. Note also that in many approaches, ``the
mapping to the \isi{interfaces}' encompasses a wide variety of operations. To
give one example, ``UG makes available a set \textbf{F} of \isi{features}
(linguistic properties) and operations C\emph{\textsubscript{HL}}
\ldots{} that access F to generate expressions'' \citep[100]{chomsky2000}. In
addition to \isi{features} and the relevant operations on them, minimalists
have attributed to the narrow syntax principles governing agreement,
labelling, transfer, probes, goals, \isi{deletion}, and economy principles
such as Last Resort, \isi{Relativized Minimality} (or Minimize Chain Links),
and \isi{Anti-Locality}. None of these fall out from \isi{recursion} \emph{per se}, but
rather represent conditions that underlie it or that need to be imposed
on it. To that we can add the entire set of mechanisms pertaining to
phases, including what nodes count for phasehood and the various
conditions that need to be imposed on their functioning, like the \isi{Phase
Impenetrability Condition}. And then there is the categorial inventory
(lexical and functional), as well as the formal \isi{features} they manifest.
The question, still unresolved, is whether any of these principles,
conditions, and substantive universals could be parameterized, in
violation of the Strong Uniformity Thesis, but not of weaker proposals.
If so, that would seem to allow for \isi{parametric variation} to be
manifested in the journey towards the \isi{interfaces}.

\section{Alternatives to the classic Principles-and-Parameters model}\label{sec:newmeyer:7}

\largerpage[-3]
\citet{chomsky2005} refers to ``the three factors in language design', namely,
genetic\is{genetics} endowment, experience, and principles not specific to the
faculty of language. The last-named ``third factor explanations', which
include principles of data analysis and efficient computation, among
other things, provide a potential alternative to the nonminimalist
poliferation of \isi{parameters} and their settings.\footnote{In what follows,
  I consider classical \isi{functional explanations} of grammatical structure
  to be of the third factor type. It is not clear whether Chomsky shares
  that view.}

The following subsections discuss alternatives to the classic P\&P
model, all appealing to one degree or another to third factor
explanations. They are grounded approaches (\sectref{sec:newmeyer:7.1}), \isi{epigenetic} (or
emergentist) approaches (\sectref{sec:newmeyer:7.2}), and reductionist approaches (\sectref{sec:newmeyer:7.3}).

\subsection{Grounded approaches}\label{sec:newmeyer:7.1}
A grounded approach is one in which some principle of UG is grounded in
 --  that is, ultimately derived from  --  some third factor principle.
Along these lines, a long tradition points to a particular constraint,\is{constraint}
often an island constraint, and posits that it is a grammaticalized
processing principle. One of the first publications to argue for
\isi{grounded constraints} was \citet{fodor1978}, where two \isi{island constraints} are
posited, one of which is the Nested Dependency Constraint (NDC):\is{constraint}

\ea The Nested Dependency Constraint: If there are two or more
filler-gap dependencies in the same sentence, their scopes may not
intersect if either disjoint or nested dependencies are compatible with
the well-formedness conditions of the language.
\z

As Fodor noted, the processing-based origins of this constraint seem
quite straightforward.

Another example is the \isi{Final Over Final Constraint}\is{constraint} (FOFC), proposed
originally in \citet{holmberg2000}:

\ea \isi{Final Over Final Constraint}: If α is a head-initial phrase and β is
a phrase immediately dominating α, then β must be head-initial. If α is
a head-final phrase, and β is a phrase immediately dominating α, then β
can be head-initial or head-final.
\z

As one consequence of the FOFC, there are COMP-TP languages that are
verb-final, but there are no TP-COMP languages that are verb-initial.
Holmberg and his colleagues interpret this \isi{constraint} as the following
UG principle:

\ea \label{ex:newmeyer:7} A theoretical reinterpretation of the FOFC: If a phase-head PH has
an EPP-feature,\is{features} then all the heads in its complement domain from which
it is nondistinct in categorial \isi{features} must have an EPP-feature. \citep[13]{biberauer2008}
\z 

\citet{walkden2009} points out that FOFC effects are accounted for by the
processing theory developed in \citet{hawkins2004} and hence suggests that
\REF{ex:newmeyer:7} is a good example of a grounded UG principle.\footnote{\citet{mobbs2014}
  builds practically all of Hawkins's parsing theory into UG.}

Note that neither Fodor nor Walkden have reduced the number of UG
constraints;\is{constraint} they have merely attributed the origins of these
constraints to what in Chomsky's account would be deemeed a third
factor. Naturally, the question arises as to whether these principles
would need to be parameterized. The answer is ``apparently so', since the
NDC does not govern \ili{Swedish} grammar \citep[75]{engdahl1985} and the FOFC is
not at work in \ili{Chinese} \citep{chan2013}. In other words, grounded approaches,
whatever intrinsic interest they might possess, do not prima facie
reduce the number of UG principles and \isi{parameters}.

\subsection{Epigenetic approaches}\label{sec:newmeyer:7.2}

Let us turn now to ``\isi{epigenetic}' or ``emergentist'' approaches to
variation,\is{variation} where \isi{parameters} are not provided by an innate UG. Rather,
parametric effects arise in the course of the \isi{acquisition} process
through the interaction of certain third factor learning biases and
experience. UG creates the space for \isi{parametric variation} by leaving
certain \isi{features} underspecified. There are several proposals along these
lines, among which are 
\citet{gianollo2008,boeckx2011}; and \citet{biberauer2014}
(preceded by many papers by the same four authors).
For reasons of space, I focus exclusively on \citet{biberauer2014}. In their
way of looking at things, the child is conservative in the \isi{complexity} of
the formal \isi{features} that it assumes are needed (what they call ``feature
economy') and liberal in its preference for particular \isi{features} to
extend beyond the \isi{input} (what they call ``input\is{input} generalization'' and which
is a form of the superset bias). The idea is that these principles drive
\isi{acquisition} and thus render \isi{parameters} unnecessary, while deriving the
same effects. Consider first their \isi{word order} hierarchy, represented in
Figure 1:



\begin{figure}[ht]
\begin{forest}
[a. Is head-final present? 
 [No: \textbf{head-initial}] 
 [{b. Yes: Present on all heads?}
 [Yes: \textbf{head-final}]
 [{No: present on $[+\hbox{V}]$ heads? }
 [Yes: \textbf{head-final in the clause only}] 
 [d. No: present on {\ldots}]]]
 ]\end{forest}
%\includegraphics[width=4.60000in,height=2.48889in]{media/image1.jpeg}
\centering\caption{The Word Order Hierarchy of \citet[110]{biberauer2014}}
\end{figure}

To illustrate with a made up example, let's say that a language is
consistently head-initial except in \isi{NP}, where the noun follows its
complements. However, there is a definable class of nouns in this
language that do precede their complements and a few nouns in this
language behave idiosyncratically in terms of the positioning of their
specifiers and complements (much like the \ili{English} word \emph{enough},
which is one of the few degree modifiers that follows the adjective). In
their theory, the child will go through the following stages of
\isi{acquisition}, zeroing in step-by-step on the adult grammar. First it will
assume that all phrases are head-initial, even noun phrases. Second, it
will assume that all NPs are head-final. Third, it will learn the
systematic class of exceptions\is{exception} to the latter generalization, and
finally, it will learn the purely idiosyncratic exceptions.\is{exception}

The other hierarchies proposed in Biberauer et al. are more complex and
depend on many assumptions about the feature\is{features} content of particular
categories. Consider for example their null argument\is{arguments} hierarchy and the
questions posed by the child in determining the status of such \isi{arguments}
in its grammar (\figref{fig:newmeyer:2}).

\begin{figure}[ht]
\begin{forest}
 [a. Are uφ-features present\\ on probes?
 [No:\\ \textbf{Radical pro-drop}]
 [Yes:\\ Are uφ-features present \\on all probes?
 [Yes:\\ \textbf{Pronominal arguments}]
 [No:\\ Are uφ-features fully specified\\ on some probes?
 [No: \\\textbf{Non-pro drop}]
 [Yes:\\ Are uφ-features fully specified\\ on T?
 [Yes: \\\textbf{Consistent null subject}]
 [No:\\ {\ldots}]]]
 ]]
 \end{forest}

%\includegraphics[width=4.54444in,height=4.20000in]{media/image2.jpeg}
\centering\caption{The Null Argument Hierarchy of \citet[112]{biberauer2014}}
\label{fig:newmeyer:2}
\end{figure}

There are many issues that one might raise about this \isi{acquisition}
scenario, the most important of which being whether children proceed
from general to the particular, correcting themselves as they go, and
gradually zero in on the correct grammar. Indeed, many acquisitionists
argue for precisely the reverse set of steps, in which children learn
idiosyncratic items before broad generalizations. Theresa Biberauer
herself acknowledges (p.c., November 21, 2014) that these steps can be
overridden in certain circumstances. For example, the early stages might
correspond to a pre-production stage or possibly acquirers\is{acquisition|(} pass through
certain stages very quickly if counterevidence to an earlier hypothesis
is readily available. And finally, as they note, \isi{frequency} effects can
distort the smooth transition through the hierarchies.

One might also ask how much the child has to know \emph{in advance} of
the progression through the hierarchy. For example, given their
scenario, the child has to know to ask ``Are unmarked phi-features\is{features} fully
specified on some probes?'' That implies a lot of grammatical knowledge.
Where, one might ask, does this knowledge come from and how does the
child match this knowledge with the input?\is{input}\footnote{An anonymous referee
  asks: ``To be honest, I don't see how the approach sketched here is
  different from parameter setting. Perhaps it's my own ignorance of
  Biberauer et al.'s proposal, but if the hierarchy of questions that
  the learner must address is innate, how does this differ from
  \isi{parameters} that are innately specified?'' As I understand their
  proposal, the hierarchy of questions falls out from general learning
  principles, though I am hazy on the details of precisely how.}

Despite these unresolved questions, the Biberauer et al. approach
presents a view that preserves the major insights of parametric theory
without positing UG-based \isi{parameters}. As such, it needs to be taken very
seriously.\is{acquisition|)}

\subsection{Reductionist approaches and the need for language-particular rules}\label{sec:newmeyer:7.3}

Reductionist approaches differ from \isi{epigenetic} ones by reducing still
further the role played by an innate UG in determining crosslinguistic
variation.\is{variation} For example, returning to the FOFC, \citet{trotzke2013} provide evidence that the best motivated account is to
remove it entirely from the grammar, since, in their view, it can be
explained in its entirety by systematic properties of performance
systems. They also deconstruct the Head-Complement parameter in a
similar fashion:

\begin{quote}
{[}T{]}he physics of speech, that is, the nature of the articulatory and
perceptual apparatus requires one of the two logical orders, since
pronouncing or perceiving the head and the complement simultaneously is
impossible. Thus, the head-complement parameter, according to this
approach, is a third-factor effect. \citep[4]{trotzke2013}
\end{quote}

Which option is chosen, of course, has to be built into the grammar of
individual languages, presumably via its statement as a
language-particular rule.\is{parameter}

To take another example, \citet{kayne1994} provided an elaborate UG-based
parametric explanation of why rightward \isi{movement} is so restricted in
language after language. But \citet{ackema2002} argue that the
apparent ungrammaticality of certain ``right-displaced'' syntactic
structures should not be accounted for by syntax proper (that is, by the
theory of competence), but rather by the theory of performance. In a
nutshell, such structures are difficult to process. A necessary
consequence of their approach is that it is necessary to appeal to
language-particular rules to account for the fact that languages differ
from each other in the degree to which displacement to the right is
permitted.

The microparametric approach to \isi{variation} is well designed to capture
the fact that even closely related speech varieties can vary from each
other in many details. The question is why one would want to appeal to
\isi{microparameters} when the traditional term ``rule'' seems totally
appropriate (see \sectref{sec:newmeyer:6.2}). The resistance to the idea of reviving the idea
of (language-particular) rules is unsettling to some, perhaps because
the idea of ``rules'' brings back the ghosts of pre-generative
structuralism, where it was believed by some that ``languages could
differ from each other without limit'' \citep[96]{joos1957}, and the spectre
of early transformational grammar, where grammars were essentially long
lists of rules. But to call a language-particular statement a ``rule'' is
not to imply that \emph{anything} can be a rule. Possible rules are
still constrained by UG. That of course raises the question of what is
in UG. An obvious candidate is the \isi{Merge} operation or something
analogous, but surely there must be a lot more than that. For example,
it is hard to see how the broad architecture of the grammar could be
learned inductively. Consider the fact that syntactic operations have no
access to the segmental phonology:\is{phonology} There is no language in which
displacement  --  Internal \isi{Merge}, if you will  --  targets only those
elements with front vowels. It seems probable that this state of affairs
derives from UG.

However, if the general thrust of the work of John A. Hawkins is correct
(see \citealt{hawkins1994,hawkins2004,hawkins2014}), the major constraints\is{constraint} on the nature of
rules derive from the exigencies of language processing. No language has
a rule that lowers a filler exactly two clauses deep, leaving a gap in
\isi{initial position}. Such a rule, while theoretically possible, is so
improbable (for processing reasons) that it will never occur. Norbert
Hornstein's approach to \isi{variation}, succinctly stated in the following
passage, also stresses that it is not necessary to appeal to UG to
explain why certain logically possible properties of grammars do not
occur:

\begin{quote}
There is no upper bound on the ways that languages might differ though
there are still some things that grammars cannot do. A possible \isi{analogy}
for this conception of grammar is the variety of geometrical figures
that can be drawn using a straight edge and compass. There is no upper
bound on the number of possible different figures. However, there are
many figures that cannot be drawn (e.g. there will be no triangles with
20 degree angles). Similarly, languages may contain arbitrarily many
different kinds of rules depending on the PLD {[} = primary linguistic
data  --  FJN{]} they are trying to fit. However, none will involve
\isi{binding} relations in which antecedents are c-commanded by their
anaphoric dependents or where questions are formed by lowering a
\emph{Wh}-element to a lower CP. Note that this view is not incompatible
with languages differing from one another in various ways. \citep[167]{hornstein2009}
\end{quote}

In my view, the idea that a grammar is composed of language-particular
rules constrained by both UG principles and third factor principles is
an appealing vision that stands to inform research on \isi{crosslinguistic
variation}\is{variation} in the years to come.

\section{Conclusion}\label{sec:newmeyer:8}
Since the late 1970s, \isi{crosslinguistic variation}\is{variation} has generally been
handled by means of UG-specified \isi{parameters}. On the positive side,
thinking of \isi{variation} in terms of parameterized principles unleashed an
unprecedented amount of work in comparative syntax, leading to the
discovery of heretofore unknown \isi{morphosyntactic} phenomena and
crosslinguistic generalizations pertaining to them. On the negative
side, however, both macro\isi{parameters} and micro\isi{parameters} have proven
themselves to be empirically inadequate and conceptually nonminimalist.
Alternatives to \isi{parameters} are grounded approaches, \isi{epigenetic}
approaches, and reductionist approaches, the last two of which seem both
empirically and conceptually quite promising.

\section*{Acknowledgements}
Earlier versions of this paper were
  presented at the Division of Labor Conference in Tübingen in January
  2015 and at a University of Illinois Colloquium in February 2015. I
  would like to thank Laurence Horn, Robert Borsley, and an anonymous
  referee for their comments on the pre-final manuscript.

\printbibliography[heading=subbibliography,notkeyword=this]

% \todos



\end{document}

