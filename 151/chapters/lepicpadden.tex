\documentclass[output=paper,
modfonts
]{LSP/langsci}


%\usepackage{langsci-optional}
\usepackage{langsci-gb4e}
\usepackage{langsci-lgr}

\usepackage{listings}
\lstset{basicstyle=\ttfamily,tabsize=2,breaklines=true}

%added by author
% \usepackage{tipa}
\usepackage{multirow}
\graphicspath{{figures/}}
\usepackage{langsci-branding}

%
\newcommand{\sent}{\enumsentence}
\newcommand{\sents}{\eenumsentence}
\let\citeasnoun\citet

\renewcommand{\lsCoverTitleFont}[1]{\sffamily\addfontfeatures{Scale=MatchUppercase}\fontsize{44pt}{16mm}\selectfont #1}
  

%\graphicspath{{./figures/lepicpadden/}}
\ChapterDOI{10.5281/zenodo.495463}
\title{A-morphous iconicity}

\author{%
Ryan Lepic\affiliation{University of California, San Diego}\lastand
Carol Padden\affiliation{University of California, San Diego}
}

% \sectionDOI{} %will be filled in at production
% \epigram{}

\abstract{
A-morphous Morphology is a morpheme-less theory of word-internal structure \citep{anderson1992}. Under this approach, derivational patterns are analyzed using Word Formation (redundancy) Rules. By specifying systematic relations among the words of a language, Word Formation Rules generally describe, rather than derive, the structure of complex words. Here, on the basis of data from American Sign Language, we present a complementary view of lexical iconicity. We suggest that in the discussion of iconicity and of morphological structure alike, a distinction can be made between those signs whose internal structure has been eroded away, and those signs whose motivated internal structure is analyzable as part of a systematic pattern.
}

\begin{document}
\maketitle

\section{Introduction}

Stephen R. Anderson's \textit{A-morphous Morphology} characterizes \isi{morphology} as a form of linguistic knowledge \citep[181]{anderson1992}. This characterization is a response to resilient misconceptions about word \isi{structure} and the \isi{lexicon}: \isi{morphology} is traditionally thought to primarily involve an inventory of minimally meaningful forms and the general mechanisms through which these meaningful forms are combined to make complex words. This procedural view of \isi{morphology} is in turn often justified by reference to de Saussure's definition of the linguistic sign. However, \citet{anderson1985,anderson1992,Andersoninpress} has shown that, in contrast to the ``exaggeratedly minimal'' \citep[326]{anderson1992} analysis of complex words as composed incrementally from independently meaningful pieces, the \isi{Saussurean sign} is a holistic, conventional relation ``between a possibly complex form and its possibly complex meaning'' \citep[193]{anderson1992}.\is{complexity}

  Treating \isi{morphology} as the knowledge that speakers have about holistic relationships between complex word forms and their complex meanings leads to a quite different conceptualization of the \isi{lexicon}. Rather than merely a list of minimally meaningful forms, a speaker's lexical knowledge must also comprise systematic relations between and among the whole words of their language. Anderson proposes that these systematic relationships can be formalized using what are referred to as \textit{Word Formation (redundancy) Rules} (after \citealt{Jackendoff1975}). As a formal representation of patterns of similarity and difference among related words, Word Formation Rules are ``only superficially'' a process by which new words are actively created or procedurally derived; their primary job is to codify systematic correspondences between words as an \isi{aspect} of any speaker's linguistic knowledge \citep[186]{anderson1992}. This perspective is motivated by the treatment of \isi{syntax} as the knowledge that speakers have about how words are organized into sentences and of \isi{phonology} as the knowledge that speakers have about how sounds are organized into words in linguistic theory.

  In this chapter, we demonstrate that Anderson's ``a-morphous'' view of morphological \isi{structure} provides a template for the study of iconic motivation in sign language \isi{structure}, as well. Characterizing \textit{morphology}\is{morphology} as the knowledge that speakers have about the relationships between word forms and their meanings leads to a quite different conceptualization of \textit{iconicity}, the perception of a motivated link between word forms and their meanings.

  Iconicity has traditionally posed a challenge to the field of sign language \isi{linguistics}. Because the linguistic sign relation is commonly characterized as an \textit{arbitrary} pairing of a word form and its meaning, the obvious links between sign forms and their meanings originally presented an obstacle to the recognition of sign languages as \isi{natural} human languages. A way around this obstacle was to argue that lexical signs are essentially arbitrary, despite their apparent iconicity, and moreover that signs can be shown to consist of smaller meaningless formative units. This view casts iconicity aside as etymological residue that is irrelevant for the understanding of recurring structural patterns in sign languages.

  Our claim is that, like \isi{morphology}, iconicity is an \isi{aspect} of linguistic knowledge. Our perspective follows Anderson's \citeyearpar{anderson1992} key observation about the nature of synchronically analyzable morphological \isi{structure}: While responsible for the formation and analysis of \textit{new} words, \isi{derivational} \isi{morphology} is not typically actively engaged in the \isi{derivation} of \textit{established} words from smaller meaningful components. Instead, the perception of transparent word-internal morphological \isi{structure} is a reflection of the knowledge that speakers have about the relationships between whole words and their analogous constituent parts. Here, we demonstrate that this approach can also account for several morphological patterns in American Sign Language (ASL) in which a \textit{motivated,} \textit{iconic link between meaning and form} serves as the organizing principle.

  Our analysis of sign-internal \isi{structure} in ASL builds from Anderson's treatment of Word Formation Rules as formal representations of patterns of similarity and difference among related whole words. When we consider whole words to be derivationally related to one another, partial relations among related words can be captured with a general rule, without expecting that whole words should exhibit incremental, semantically \isi{compositional} morphological\is{morphology} \isi{structure} (see also \citealt{ackerman2016}, \citealt{Aronoff1976}, \citealt{Bochner1993},  \citealt{Hay2005}, \citealt{Aronoff2007}, \citealt{Blevins2016}, \citealt{Andersoninpress}). Under this view, whole words are the primary unit of morphological organization. Related whole words with transparent, analyzable internal \isi{structure} participate in morphological patterns that can be described by a structural rule. However, individual words can become quite reduced and \isi{opaque} over time, such that they eventually lose their synchronic morphological connection to other words in the language. Taken seriously, Anderson's view leads to the conclusion that analyzable word-internal \isi{structure} is most often a gradient reflection of \isi{etymological history}, and only infrequently the derived output of a synchronic operation.

\section{The erosion of \isi{transparency} in lexical signs}\label{sec:2}

The field of sign language \isi{linguistics} has been compelled to demonstrate that, despite their apparent semantic and gestural \isi{transparency}, signs are arbitrary linguistic symbols that are analyzable into smaller formal units (see \citealt{Stokoe1960},  \citealt{Klima1979}, and \citealt{Supalla1986} for examples). Underlying this work is the assumption that conventional linguistic symbols are, by definition, inherently arbitrary. Accordingly, if they are truly linguistic in nature, lexical signs should also be arbitrary symbols, even if they were once iconically motivated (however, see \citealt{Wilcox1995}, \citealt{Taub2001}, \citealt{Perniss2010}, and \citealt{Emmorey2014} for critical reviews of this assumption). This perspective leads to the conclusion that iconicity is a secondary, etymological feature of individual signs, and ultimately erodes over time. For example, \citet[718]{Frishberg1975} compares old (ca.\ 1918) and modern (ca.\ 1965) versions of several ASL signs, and argues that over time, ``in general, signs have become less transparent, pantomimic, and iconic; they have become more arbitrary, conventionalized, and symbolic.''

  Accordingly, when comparing old and modern versions of signs like \textsc{cow} and \textsc{horse}, which are both articulated at the signer's head, and are motivated by an image of the animal's horns and ears, respectively, we can appreciate that the older forms are more faithful to their original motivating image, while the newer forms have lost some of their original iconicity: The older form of \textsc{cow} is signed with two hands, one for each of the paired horns to be represented Figure 1a, while the newer, more typical form is signed with only one hand Figure 1b. This \isi{change} over time has an articulatory motivation. It requires less effort to move one hand than it does to move both hands, and, because the second hand is configured identically to the dominant\is{dominance} hand in these cases, the absence of the second hand does not hinder recognition of the target sign. As a result, the involvement of the second, non-dominant hand has been deleted from these signs in the course of history (\citealt{Battison1974}, \citealt{Frishberg1975}).

\begin{figure}
	\begin{tabular}{lcc}
	& \signpic{figure_1ai} & \signpic{figure_1aii} \\
	a. & (beginning of sign) & (end of sign) \\
	& \signpic{figure_1bi} & \signpic{figure_1bii} \\
	b. & (beginning of sign) & (end of sign) \\	
	\end{tabular}	
	\caption{The ASL sign \textsc{cow} signed (a) with two hands and (b) with one hand.}
	\label{fig:1}
\end{figure}

  As another example, \citet[437--438]{napoli2014n} demonstrate that synchronically, in casual signing, the form of the iconic sign \textsc{hour} is often altered to make the sign less difficult to articulate, which can result in the formation of a less iconic sign. In the citation form, the iconic sign \textsc{hour} is articulated with a dominant\is{dominance} index finger tracing a full circle around the palm of the non-dominant hand. The iconic motivation for this sign is the \isi{movement} of a minute hand around the face of a clock, with the non-dominant hand representing the face of a clock, and the dominant\is{dominance} hand representing the angle and \isi{movement} of the minute hand. In the citation form, the wrist serves as a hinge for the circular \isi{movement} of the dominant\is{dominance} hand (Figure 2a). However, in the more casual form of \textsc{hour}, the locus of the dominant\is{dominance} hand's \isi{movement} is transferred away from the wrist to the elbow and shoulder (Figure 2b). This \isi{change} partially disrupts the iconic image of a clock hand tracing a journey around the clock face, as it is the whole hand, rather than the extended finger alone, which traces a circular \isi{movement}. This \isi{change} obscures the iconic representation of a clock's minute hand in the sign \textsc{hour}, and again, this \isi{change} is favored for an articulatory reason, as it avoids ``a physiologically awkward \isi{movement}'' (\citealt[438]{napoli2014n}). The logical conclusion, based on examples like \textsc{cow} and \textsc{hour}, is that in time, true, systematic processes work to erode the coincidental, iconic origins of any sign.

\begin{figure}
\begin{tabularx}{\linewidth}{lCCC}
& \signpic{figure_2ai} & \signpic{figure_2aii} & \signpic{figure_2aiii} \\
a. & (beginning of sign) & (middle of sign) & (end of sign) \\
& \signpic{figure_2bi} & \signpic{figure_2bii} & \signpic{figure_2biii} \\
b. & (beginning of sign) & (middle of sign) & (end of sign) \\
\end{tabularx}	
\caption{The ASL sign \textsc{hour} signed (a) with the locus of rotation at the wrist and (b) with the locus of rotation at the elbow.}
\label{fig:2}
\end{figure}


  We contend that these discussions about erosion of iconicity require more nuance. The cases cited above are indeed instances of signs reducing in ways that partially obscure their original motivating visual image. In these examples, the iconic motivation for a single sign is overcome by articulatory considerations. We assume that the primary \isi{constraint} on this phonetic reduction is that the overall form of the sign itself should nevertheless remain recognizable as ``the same sign'': Processes of phonetic reduction can erode the forms of signs only once they have been registered as conventional lexical items with conventional forms and agreed-upon meanings, to begin with. However, we note that even in the face of phonetic reduction, the reduced versions of the signs \textsc{cow} or \textsc{hour} actually remain quite faithful to their iconic motivations. Both signs still transparently represent the horn of an animal and the face and hand of a clock, respectively. In these cases, at least, the shift is not from ``wholly iconic sign'' to ``wholly arbitrary sign,'' but rather from ``more transparent conventional sign'' to ``less transparent conventional sign.''

  Here it is important to note that this sort of gradient phonetic erosion also affects \textit{morphological} \isi{transparency} in conventional signs. Like spoken words\footnote{A anonymous reviewer rightly comments that this erosion is likely also modulated by \isi{frequency}. In \ili{English}, the classic example \textit{cupboard} has undergone \isi{assimilation} and reduction that obscures its connection to its original constituent words, while other (newer/less frequent) words like \textit{clipboard} retain their original \isi{compound} pronunciation (see \citealt{Zipf1935}, \citealt{Bybee2001}). The same reviewer eloquently notes that written alphabetical systems have also developed through this type of ``creeping \isi{opacity},'' in which the written symbols became streamlined and less connected to their ``original causal denotata.''} and like iconic signs, morphologically complex signs, once registered as conventional pairings of meaning and form, may begin to drift in ways that obscure their original etymology. An example discussed by \citet[707]{Frishberg1975} is the ASL sign \textsc{home}. The conventional sign \textsc{home} derives etymologically from the composition of the signs \textsc{eat} and \textsc{sleep} (this combination can be glossed as \textsc{eat+sleep)}. These signs were almost certainly selected to represent the concept `home' because a ``home'' is ``where one eats and sleeps.'' However, as a function of its lexical entrenchment as a conventional sign, \textsc{eat+sleep} has drifted both in form and in meaning. It has been reanalyzed as a semantically holistic sign meaning `home', and has reduced in form so as to mask its former transparent relationship to its original constituent signs. As a result of this drift over time, the sign \textsc{home} no longer bears an overt morphological relationship to its former constituent signs \textsc{eat} and \textsc{sleep} in modern ASL.

  A related, synchronic example is the sign \textsc{student}, which is etymologically derived from the composition of the signs \textsc{learn} (whose form is iconically motivated by the image of moving an object into the mind) and \textsc{person} (whose form is iconically motivated by the silhouette of a human figure). While the citation form for \textsc{student} still retains much of its analyzable internal \isi{structure} as a composite of \textsc{learn+person} (Figure 3a), in casual signing, \textsc{student} is typically reduced to the point that its analyzable morphological \isi{structure} is no longer identifiable (Figure 3b).

\begin{figure}
	\begin{tabularx}{\linewidth}{lCCC}
		& \signpic{figure_3ai} & \signpic{figure_3aii} & \signpic{figure_3aiii} \\
		a. & (beginning of sign) & (middle of sign) & (end of sign) \\
		& \signpic{figure_3bi} & \signpic{figure_3bii} & \signpic{figure_3biii} \\
		b. & (beginning of sign) & (middle of sign) & (end of sign) \\
	\end{tabularx}	
	\caption{The ASL sign \textsc{student} (a) in a fuller, more transparent form (``\textsc{learn+person}'') and (b) in a reduced, more \isi{opaque} form (``\textsc{student}'').}
	\label{fig:3}
\end{figure}

Similar reduction can also be observed in the casual forms of the related signs \textsc{interpreter} and \textsc{teacher}. Like the sign \textsc{student}, these signs are morphologically complex, and they can be analyzed as previously derived from \textsc{interpret+person} and \textsc{teach+ person}. These signs participate in a \isi{productive} \isi{derivational} pattern in ASL involving the addition of \textsc{person} as an ``agentive suffix.''\is{suffixation} However, as frequently occurring signs, \textsc{interpreter}, \textsc{student}, and \textsc{teacher} have all drifted in ways that render their morphological \isi{structure} increasingly \isi{opaque} in casual signing. We discuss some implications of this erosion (and possible \isi{reanalysis}) of transparent morphological \isi{structure} in \sectref{sec:3}.%\todo{check teach + person hyphenation}

  As in (spoken and signed) \isi{morphology} (see \citealt{Bybee2006b}), the gradual loss of iconicity is not an across-the-board phenomenon. Iconicity can also persist within signs when it becomes \textit{morphologized}, or made systematic as a learned, language-internal pattern (see \citealt[337]{anderson1992}). The loss of iconicity is therefore not as inevitable as is commonly believed. An example of a sign which might be considered to have lost its iconicity (an analysis that has been debunked by \citealt[153]{Wilcox1995}, \citealt[228]{Taub2001}, and \citealt[123]{Wilcox2004}) is \textsc{very-slow}. The sign \textsc{slow} is articulated with the dominant\is{dominance} hand sliding over the back of the non-dominant hand in a single \isi{movement} (Figure 4a). The slow \isi{movement} of the hand can be considered iconically motivated, as the friction resulting from the contact between the two hands causes the sign to be articulated somewhat slowly. In the derived sign \textsc{very-slow}, however, the \isi{movement} pattern has changed: \textsc{very-slow} is articulated with a short initial hold, followed by a quick, larger burst of \isi{movement} (Figure 4b). We will demonstrate that this \isi{change} in \isi{movement} is characteristic of an ``intensive'' \isi{derivational} pattern in ASL, but these facts originally led \citet[30]{Klima1979}, for example, to conclude that the iconicity of \textsc{slow} has been ``overridden and submerged'' in the formation of the sign \textsc{very-slow}, as it is signed with a very \textit{fast} \isi{movement}.

\begin{figure}
	\begin{tabular}{lcc}
		& \signpic{figure_4ai} & \signpic{figure_4aii} \\
		a. & (beginning of sign) & (end of sign) \\
		& \signpic{figure_4bi} & \signpic{figure_4bii} \\
		b. & (beginning of sign) & (end of sign) \\	
	\end{tabular}	
\caption{The forms of the ASL signs (a) \textsc{slow} and (b) \textsc{very-slow} differ primarily in their speed and size: \textsc{very-slow} is signed with a faster, larger \isi{movement}.}
\label{fig:4}
\end{figure}
  
  While the \isi{movement} of the sign \textsc{very-slow} is indeed quite fast, the process that derives the intensive version of \textsc{slow} by changing its original form to incorporate a quick burst of \isi{movement} is at once iconically motivated \textit{and} systemically motivated: it is also at work in the formation of a number of other ASL signs. These signs include predicate adjectives like \textsc{very-clever, very-expensive,} and \textsc{very-stubborn}, and they all have in common that they derive the intensive form of a sign by increasing the intensity of its conventional \isi{movement}. Accordingly, it is not the case that \textsc{very-slow} has lost its iconicity. Instead, \textsc{very-slow} has taken on a different type of iconic motivation, one that happens to be at odds with the idea that the only way to represent ``incredible slowness'' is to use a very slow \isi{movement} \citep[229]{Taub2001}. In this intensive \isi{derivational} pattern in ASL, the intensity of a sign's \isi{movement} is increased, thereby iconically signaling an increase in the intensity of the sign's meaning \citep[153]{Wilcox1995}.

  This systematic iconic correspondence between whole words is precisely the kind of relationship that can be described using a \isi{Word Formation Rule}. The fact that aspects of the pattern happen to be iconically motivated in no way precludes this rule from having been taken up and made systematic in ASL. Indeed, this discussion of the loss of iconicity in individual signs, and the preservation of iconicity when it is relevant for language-internal \isi{structure}, is entirely compatible with Anderson's a-morphous view of \isi{morphology}, in which conventional words are Saussurean signs, regardless of whether they contain transparent, analyzable \isi{structure}. Though they are often referred to as ``lexical entries,'' Saussurean signs cannot be considered entries on a structureless list. Instead, a language user's morphological knowledge also encompasses their knowledge of the relationships between the established words of their language.\footnote{An anonymous reviewer comments that \citet{anderson1992} presents a realizational theory of \isi{morphology}, in which an inflected word's semantic content precedes and determines its phonological\is{phonology} form. This is in opposition to concatenative theories in which a word's form determines its content. Our a-morphous analysis of iconicity in ASL word formation is meant to be consistent with a realizational theory of \isi{inflection}. We do not discuss ASL \isi{inflectional} \isi{morphology} here because there are several competing perspectives as to what should even count as \isi{morphosyntactic} \isi{inflection} (namely agreement) in ASL. Reviewing these perspectives takes us beyond the \isi{scope} of this chapter, but the reader is referred to \citet{LilloMartin2011}, \citet{Wilbur2013}, and \citet{Wilcox2016} for a sense of these different perspectives.}

Word Formation Rules are descriptions of the phonological,\is{phonology}  syntactic, and semantic differences and correspondences between two or more morphologically related forms. For example, the rule that describes the relationship between pairs of \ili{English} words like \textit{breath} and \textit{breathe}, \textit{loss} and \textit{lose}, and \textit{grief} and \textit{grieve} (minimally) specifies a \isi{change} in word-final voicing, a \isi{change} in syntactic category, and concomitant changes in meaning. However, as in the analysis of non-concatenative \isi{morphology} in spoken languages, the formal representation of phonological\is{phonology}  changes in sign language \isi{morphology} has the potential to obfuscate more than to clarify. This problem is also compounded by the fact that there is no dominant conventional system for describing sign forms on \isi{analogy} to the International Phonetic Alphabet for spoken language research.

  In order to discuss Word Formation Rules in ASL, we require a representational system that will allow us to recognize that signs are holistic pairings of complex form and complex meaning.\is{complexity} The convention of labeling ASL signs with \ili{English} metalinguistic glosses is, by itself, inadequate for this task. Labeling signs with \ili{English} glosses illustrates that they have conventional, holistic meanings, and implies that they similarly have conventional, agreed-upon forms. In order to facilitate an analysis of the iconic \isi{structure} within ASL signs, we will adopt Taub's \citeyearpar{Taub2001} convention of listing the aspects of form in a sign with their corresponding aspects of meaning. As Meir and colleagues (\citealt[874]{Meir2010}; \citealt[316]{Meir2013}) have demonstrated, such \textit{iconic mapping} diagrams make it clear that iconicity is neither a deterministic nor a \isi{compositional} property of signs: a sign may be a conventional pairing of form and meaning and \textit{also} exhibit transparent and motivated aspects of \isi{structure}. Crucially, the perception of iconicity arises as a consequence of the fact that signs are conventional pairings of a potentially complex form and potentially complex meaning, and not from a \isi{compositional} analysis of the sign's parts.\footnote{A reviewer notes, and it has been pointed out previously (e.g.\ \citealt{Fernald2000}), that there are also similarities between ASL \isi{morphology} and iconic, sound-symbolic elements in spoken language such as \isi{phon(a)esthemes} and \isi{ideophones}. Phonaesthemes are recurring pairings of meaning and form occurring in words that cannot otherwise be analyzed as exhibiting \isi{compositional} morphological \isi{structure} (\citealt[49]{anderson1992}, \citealt{Bergen2004}). Ideophones are depictive, sound-symbolic words that appear in a variety of languages \citep{Dingemanse2012}. We expect that an ``a-morphous'' analysis of iconicity and \isi{morphology} should also extend to these classes of words, but leave the details of this project for future work.} The meaning of the whole facilitates the (re)analysis of its parts, rather than the other way around.

As an illustration of an iconic \isi{mapping} in ASL, consider the sign \textsc{slow}, already described impressionistically above (and pictured in Figure 4a). This sign has a conventional form and meaning, and aspects of its form can be analyzed as transparently motivated by its meaning. These correspondences can be represented through an explicit pairing of aspects of the sign's form with aspects of its meaning, as in \tabref{lptab:1}.

\begin{table}
\begin{tabular}{ll}
\lsptoprule
Form & Meaning\\
\midrule
non-dominant hand & a stationary object\\

back of the non-dominant hand & a surface that creates friction\\

dominant hand & an object in motion\\

palm of the dominant hand & a surface that creates friction\\

contacting \isi{movement} & contact between two surfaces\\

dragging \isi{movement} & a \isi{movement} slowed by friction\\
\lspbottomrule
\end{tabular}
\caption{Aspects of the iconic \isi{mapping} for \textsc{slow}.}
\label{lptab:1}
\end{table}

  This representation illustrates that the conventional sign \textsc{slow} exhibits analyzable internal \isi{structure}: formational aspects of this sign can be linked to aspects of the visual and kinesthetic images that provide the sign's iconic motivation. For example, in this case, it is possible to assign an iconic \isi{aspect} of meaning to each of the two hands, as well as the manner in which the dominant\is{dominance} hand contacts the non-dominant hand.

  The benefit of the representation in \tabref{lptab:1} is that it allows us to discuss the relationship between the form and meaning of the whole sign as well as the relationship between the whole and its parts, including how these aspects of \isi{structure} may \isi{change} from sign to sign. For example, as discussed above, the \isi{Word Formation Rule} for the ``intensive'' pattern alters the conventional \isi{mapping} for \textsc{slow} by changing the character of the base sign's \isi{movement}: the sign \textsc{very-slow} is formed with a \isi{movement} pattern that is superimposed onto the form of the original sign \textsc{slow}, keeping the overall trajectory of the \isi{movement} but adding a brief initial hold followed by a quicker and larger burst of motion. The resulting sign \textsc{very-slow} (pictured in Figure 4) can be represented as in \tabref{lptab:2}. In this representation, the aspects of form and meaning that have been changed by the intensive \isi{Word Formation Rule} are emphasized in bold.

\begin{table}
\caption{Aspects of the iconic \isi{mapping} for \textsc{very-slow}.}
\label{lptab:2}

\begin{tabular}{ll}
\lsptoprule
Form & Meaning\\
\midrule
non-dominant hand & a stationary object\\

back of the non-dominant hand & a surface that creates friction\\

dominant hand & an object in motion\\

flat palm of the dominant hand & a surface that creates friction\\

contacting \isi{movement} & contact between two surfaces\\

\textbf{brief initial hold} & \textbf{buildup of pressure}\\

\textbf{quick, large movement} & \textbf{release of built-up pressure}\\
\lspbottomrule
\end{tabular}
\end{table}

  Following Anderson's \citeyearpar[186]{anderson1992} formulation of the –\textit{able} \isi{Word Formation Rule}, for example, we can think of this relationship between pairs of signs \textsc{slow} and \textsc{very-slow} in the following way:\textsc{} The form of the intensive \isi{Word Formation Rule} specifies that the intensive form of a sign is made by changing its \isi{movement} pattern. However, rather than a true ``\isi{derivational}'' rule, the \isi{Word Formation Rule} is regarded as a description of the systematic differences between the signs represented in \tabref{lptab:1} and \tabref{lptab:2}.

\section{The (re)analysis of lexical iconicity}\label{sec:3}
Iconic mappings provide a way to represent the relationship between the form and meaning of a conventional complex sign. They also provide a way to specify how aspects of a sign's analyzable internal \isi{structure} can be reanalyzed by speakers based on the meaning of the complex sign that they appear in. In this section, we explore this tradeoff between form and meaning. We begin with the ASL sign \textsc{time,} as an illustrative example of how a sign's relationship to its original iconic motivation can become obscured, and even how the form of the conventional sign can subsequently be reanalyzed by signers. We suggest that such \isi{reanalysis} can only happen in a system where the sign relation between form and meaning takes precedence over the \isi{compositional} \isi{structure} that originally contributed to the sign's creation.\footnote{ Of course, signers can, and often do, create new signs, as well. We analyze these new signs as repurposing the patterns and elements that recur among established signs. A popular (in both senses) article from 2015, for example, discusses some ASL candidates for internet slang like \textit{selfie} and \textit{photobomb}. These potential signs make creative new use of old sign parts, though we hesitate to analyze them as semantically ``\isi{compositional}'' in the traditional sense. The article is accessible online (\url{http://www.hopesandfears.com/hopes/now/internet/168477-internet-american-sign-language}), as is some additional commentary from an ASL news ``vlog'' (\url{https://www.youtube.com/watch?v=wI8o8zgEK88}).} 

The ASL sign \textsc{time} is formed with the crooked index finger of the dominant\is{dominance} hand tapping the back of the non-dominant wrist (\figref{fig:lepic:figure5}). ASL signers and non-signers alike readily recognize the similarities between this sign and the act of tapping the face of a wristwatch, for example as part of a \isi{gesture} of impatience. The sign \textsc{time} can therefore be considered to have a transparent iconic motivation, stemming from the cultural association of reading a wristwatch with the telling of time. For signers, this analysis of \textsc{time}'s iconic motivation is also reinforced by the fact that the ASL sign \textsc{wristwatch} is indeed articulated in the same location, at the back of the non-dominant wrist.

\begin{figure}
\signpic{figure_5}
\caption{The ASL sign \textsc{time}.}
\label{fig:lepic:figure5}
\end{figure}

  However, this etymological description of the ASL sign \textsc{time} is in fact a folk \isi{reanalysis}. As \citet[177]{Shaw2010} explain, ``the origin of \textsc{time} was identified long before the advent of the wristwatch in 1904.'' They demonstrate that as early as 1785, the \ili{French} Sign Language sign \textsc{time} was recorded in a form similar to that of ASL, its daughter language, with the crooked index finger repeatedly contacting the back of the non-dominant hand. The image motivating the form of this historical sign is the design and function of an early mechanical clock that uses a hammer to strike a bell at the stroke of an hour. Historical texts documenting Old \ili{French} Sign Language describe this sign's form as showing ``the hammer which taps the bell'' and using the index finger to ``ring the hour on the back of the hand which is in the guise of a bell'' (\citealt{Ferrand1785} and \citealt{Lambert1865}, respectively, as cited by \citealt[177--178]{Shaw2010}). Following Taub's \citeyearpar{Taub2001} conventions for analyzing iconic mappings, we can represent aspects of the \isi{mapping} between the phonological and semantic elements of this historical sign \textsc{time} as in \tabref{tab:3lp}:

\begin{table}
\caption{Aspects of the historical iconic \isi{mapping} for \textsc{time}.}
\label{tab:3lp}
\begin{tabular}{ll}
\lsptoprule
Form & Meaning\\
\midrule
non-dominant hand & the bell of a clock\\

back of the non-dominant hand & surface of the bell\\

dominant hand & a figure which rings the clock\\

crooked index finger & the hammer which strikes the bell\\

contacting \isi{movement} & the hammer striking the bell\\

repeated \isi{movement} & a repeated action\\
\lspbottomrule
\end{tabular}
\end{table}

By the time the wristwatch became popular in the early 1900s, the sign \textsc{time} had been in use for well over a century. Having already been established as a conventional pairing of form and meaning, it was, presumably, no longer primarily analyzed as deriving its meaning from its constituent parts. Parallel to the examples of \textsc{cow} and \textsc{hour} mentioned above, the formational aspects of the sign \textsc{time} began to drift slightly, such that the index finger moved ``a few centimeters from the back of the hand to the back of the wrist'' \citep[178]{Shaw2010}. The sign \textsc{time} was also no longer concretely linked to the image of a particular time-telling device. Accordingly, to the extent that they were associated with any meaning at all, the parts of the sign \textsc{time} must have derived their meanings by association with the meaningful whole sign. The sign's existing internal \isi{structure} was thus open to \isi{reanalysis} as motivated by the image of a wristwatch, as is represented in \tabref{tab:4lp}. Here we see that the aspects of form are the same across both \tabref{tab:3lp} and \tabref{tab:4lp}, however the mapped \textit{meanings} differ between the historical and modern versions of the sign \textsc{time.} 

\begin{table}
\caption{Aspects of the modern iconic \isi{mapping} for \textsc{time}.}
\label{tab:4lp}
\begin{tabular}{ll}
\lsptoprule
Form & Meaning\\
\midrule
non-dominant hand & \textbf{a human hand}\\

back of the wrist & \textbf{the location of a wristwatch}\\

dominant hand & \textbf{a human hand}\\

crooked index finger & \textbf{a human finger}\\

contacting \isi{movement} & \textbf{a human finger contacting a wristwatch}\\

repeated \isi{movement} & \textbf{a repeated action}\\
\lspbottomrule
\end{tabular}
\end{table}

We re-emphasize that this iconic \isi{reanalysis} could only happen because the holistic relation between \textsc{time}'s form and meaning takes precedence over the aspects of \isi{structure} that originally contributed to its creation. The sign \textsc{time} provides a very nice example, but it is not an exceptional case: all conventional signs in ASL are by definition registered as learned pairings of form and meaning, and many sign forms also remain open to iconic interpretation and \isi{reanalysis}. Of course, conventional signs can serve as the \isi{input} for \isi{productive} \isi{derivational} \textit{morphological} processes as well. As a result, the motivating factors of language internal systematicity (\isi{morphology}) and of analyzable visual imagery (iconicity) are inextricably interlinked as aspects of lexical motivation in ASL.

  Another relevant example, a somewhat uncommon sign which we refer to here as \textsc{hash-things-out}, is ultimately a reduced derivative of the ASL verb \textsc{debate.} As we will show, the sign \textsc{debate} is both iconically and derivationally related to a number of other ASL signs that conventionally connote `argumentation', including \textsc{argue}, \textsc{oppose}, \textsc{struggle,} \textsc{discuss}, and \textsc{discuss-in-depth}. These signs are all morphologically related in ASL, though their corresponding \ili{English} translations are not. Rather than getting bogged down in a discussion of the nuances of meaning between the \ili{English} meta-language glosses, we will focus primarily on the relationship between form and meaning among these morphologically-related iconic signs. We begin with the sign \textsc{argue}, which is formed with the index fingers of both hands pointing toward one another and simultaneously moving up and down several times (\figref{fig:6}).

\begin{figure}
	\begin{tabular}{cc}
		\signpic{figure_6i} & \signpic{figure_6ii} \\
		(beginning of sign) & (end of sign) \\
	\end{tabular}	
\caption{The ASL sign \textsc{argue}.}
\label{fig:6}
\end{figure}

  The iconic motivation for the sign \textsc{argue} is the visual image of two people engaged in heated conversation, with each hand representing a participant in the argument, and with the orientation of the two hands towards one another representing that each participant's communicative efforts are directed toward the other (see \citealt{Lepic2016} regarding use of the two hands to represent paired referents in lexical signs). This sign's form also seems be motivated by the rhythmic properties of the beat gestures that often accompany continuous speech, and by the form of the finger-shaking \isi{gesture} that often accompanies ``scolding'' or ``telling somebody off.'' The association between form and meaning in the conventional sign \textsc{argue} can thus be represented as in \tabref{tab:5lp}.
  
\begin{table}
\caption{Aspects of the iconic \isi{mapping} for \textsc{argue}.}
\label{tab:5lp}
\begin{tabular}{ll}
\lsptoprule
Form & Meaning\\
\midrule
dominant hand & one side of an argument\\

non-dominant hand & the other side of an argument\\

orientation of hands toward each other & two sides communicating with each other\\

index finger handshape & the direction of attention\\

coordinated \isi{movement} of the hands & communicative interaction between sides\\

repeated \isi{movement} & an on-going process\\
\lspbottomrule
\end{tabular}
\end{table}

The signs \textsc{oppose} (\figref{fig:7}) and \textsc{struggle} (\figref{fig:8})\textsc{} are formed similarly to the sign \textsc{argue}, with two index fingers pointed toward one another, however the \isi{movement} patterns for these signs are different. While \textsc{argue} is articulated with repeated up-and-down movements, \textsc{oppose} is signed with the hands pulling away from one another in a single motion, and \textsc{struggle} is signed with both hands repeatedly moving back-and-forth together along the imagined line they form.

\begin{figure}
	\begin{tabular}{cc}
		\signpic{figure_7i} & \signpic{figure_7ii} \\
		(beginning of sign) & (end of sign) \\
	\end{tabular}	
	\caption{The ASL sign \textsc{oppose}.}
	\label{fig:7}
\end{figure}

\begin{figure}
	\begin{tabular}{cc}
		\signpic{figure_8i} & \signpic{figure_8ii} \\
		(beginning of sign) & (end of sign) \\
	\end{tabular}	
	\caption{The ASL sign \textsc{struggle}.}
	\label{fig:8}
\end{figure}

In the sign \textsc{oppose}, the \isi{movement} of the hands away from one another can be analyzed as motivated by an image of two participants in an argument giving up and retreating from one another. In the sign \textsc{struggle}, the \isi{movement} of the hands together can be analyzed as motivated by an image of two opposing forces retreating and advancing together in turn. These associations between form and meaning can be represented as in Tables 6 and 7, respectively. Note that the first several aspects of the iconic \isi{mapping}, such as the use and relative orientation of the two hands, are shared between the signs \textsc{argue}, \textsc{oppose}, and \textsc{struggle}: the aspects that differ between these signs are again marked in bold. Here, again, the benefit of the iconic \isi{mapping} notation is that it makes recurring configurations of form and meaning explicit among related and conventional iconic signs.

\begin{table}
\caption{Aspects of the iconic \isi{mapping} for \textsc{oppose}.}
\label{tab:6lp}
\resizebox{\linewidth}{!}{
\begin{tabular}{ll}
\lsptoprule
Form & Meaning\\
\midrule
dominant hand & one side of an argument\\

non-dominant hand & the other side of an argument\\

orientation of hands toward each other & two sides communicating with each other\\

index finger handshape & the direction of attention\\

\textbf{\isi{movement} of hands away from each other} & \textbf{retreating to opposite sides of an argument}\\

\textbf{single movement} & \textbf{a single event}\\
\lspbottomrule
\end{tabular}}
\end{table}

\begin{table}
\caption{Aspects of the iconic \isi{mapping} for \textsc{struggle}.}
\label{tab:7lp}
\resizebox{\linewidth}{!}{
\begin{tabular}{ll}
\lsptoprule
Form & Meaning\\
\midrule
dominant hand & one side of an argument\\

non-dominant hand & the other side of an argument\\

orientation of hands toward each other & two sides communicating with each other\\

index finger handshape & the direction of attention\\

\textbf{\isi{movement} along the same plane} & \textbf{advancing and falling back in an argument}\\

\textbf{repeated movement} & \textbf{an on-going process}\\
\lspbottomrule
\end{tabular}}
\end{table}

  Turning now to the related signs \textsc{discuss}, \textsc{discuss-in-depth}, and \textsc{debate}, we see that these signs similarly use the index finger of the dominant\is{dominance} hand to represent one side of an argument, however, in each of these signs, the ``opposing side'' is represented quite differently. In the sign \textsc{discuss}, the ``other side'' is actually not represented at all: This sign\textsc{} is conventionally formed with the index finger of the dominant\is{dominance} hand repeatedly striking the flat palm of the non-dominant hand (\figref{fig:9}). The form of the sign \textsc{discuss} is also partially motivated by the visual image of a list of written topics under discussion; in this sign, the non-dominant hand represents the message itself, serving as the primary target of communicative effort and as the place of articulation for the dominant\is{dominance} hand. Note that the flat palm of the non-dominant hand similarly represents a surface for written material in signs like \textsc{jot-down} (\figref{fig:10}), \textsc{learn} (first two segments of \figref{fig:3}a above), and \textsc{write}. We do not provide an in-depth analysis of these ``written-upon surface'' signs here, but see \citet[118]{Frishberg1973} and \citet[75]{Aronoff2003} for additional discussion. The association of form and meaning in the sign \textsc{discuss} can be represented as in \tabref{tab:8lp}, which again exhibits several aspects of \isi{structure} that have been seen already in the iconic mappings for \textsc{argue, oppose}, and \textsc{struggle}.
  
\begin{figure}
	\begin{tabular}{cc}
		\signpic{figure_9i} & \signpic{figure_9ii} \\
		(beginning of sign) & (end of sign) \\
	\end{tabular}	
	\caption{The ASL sign \textsc{discuss}.}
	\label{fig:9}
\end{figure}

\begin{figure}
	\begin{tabular}{cc}
		\signpic{figure_10i} & \signpic{figure_10ii} \\
		(beginning of sign) & (end of sign) \\
	\end{tabular}	
	\caption{The ASL sign \textsc{jot-down}.}
	\label{fig:10}
\end{figure}

\begin{table}
\caption{Aspects of the iconic \isi{mapping} for \textsc{discuss}.}
\label{tab:8lp}
\begin{tabular}{ll}
\lsptoprule
Form & Meaning\\
\midrule
dominant hand & one side of an argument\\

index finger handshape & the direction of attention\\

\textbf{non-dominant hand} & \textbf{topics under discussion}\\

\textbf{flat palm handshape} & \textbf{a written surface}\\

repeated \isi{movement} & an on-going process\\
\lspbottomrule
\end{tabular}
\end{table}

The sign \textsc{discuss-in-depth} is in turn formed similarly to the sign \textsc{discuss}, with the index finger contacting the flat palm of the non-dominant hand. However, rather than remaining in a single, fixed location, the hands move together between two locations, signed at first in front of the signer's body, and then away from the body to represent a second interlocutor (\figref{fig:11}). The \isi{mapping} for this sign is represented in \tabref{tab:9lp}. Like the signs \textsc{oppose} and \textsc{struggle}, this \isi{movement} between two locations represents the contributions of two participants to the discussion. However, unlike the sign \textsc{oppose}, here there is not an implicit contrast between ``sides of an argument.'' Instead, the addition of another's perspective to the discussion is collaborative, and the discussion takes on greater depth as a result.\largerpage

\begin{figure}
	\begin{tabular}{cc}
		\signpic{figure_11i} & \signpic{figure_11ii} \\
		(beginning of sign) & (end of sign) \\
	\end{tabular}	
	\caption{The ASL sign \textsc{discuss-in-depth}.}
	\label{fig:11}
\end{figure}

\begin{table}
\caption{Aspects of the iconic \isi{mapping} for \textsc{discuss-in-depth}.}
\label{tab:9lp}
\begin{tabularx}{\textwidth}{lX}
\lsptoprule
Form & Meaning\\
\midrule
dominant hand & one side of an argument\\

index finger handshape & the direction of attention\\

non-dominant hand & topics under discussion\\

flat palm handshape & a written surface\\

\textbf{\isi{movement} along the same plane} & \textbf{two sides communicating with each other}\\

repeated \isi{movement} & an on-going process\\
\lspbottomrule
\end{tabularx}
\end{table}

\largerpage[-2]
When we move to consider the related sign \textsc{debate}, we again find opposition between two sides, which are mapped onto each of the two hands. The sign \textsc{debate} is formed similarly to the signs \textsc{discuss} and \textsc{discuss-in-depth}, with the index finger of the dominant\is{dominance} hand repeatedly striking the flat palm of the non-dominant hand. However, \textsc{debate} also exhibits what is known as ``dominance reversal''\is{dominance} (\citealt{Frishberg1985};  \citealt{Padden1987}): in the formation of this sign, the index finger of the dominant\is{dominance} hand first strikes the non-dominant hand, then the hands switch roles and configurations, and the index finger of the non-dominant hand strikes the flat palm of the dominant\is{dominance} hand, with this reversal being articulated several times in succession (\figref{fig:12}).\footnote{The direction of the \isi{movement} in the sign \textsc{debate} is also changed; the hands move right-and-left instead of forward-and-back as in the previous examples. We suspect that this is because a side-to-side \isi{movement} is easier to articulate while also reversing the \isi{dominance} of the hands, though this changed direction of \isi{movement} may well have a semantic motivation (or lend itself to \isi{reanalysis} based on a semantic motivation), as well.} The iconic image motivating the form of the sign \textsc{debate}, then, is that one side discusses its case, then the other side discusses its own case, and these discussions continue back and forth. The iconic \isi{mapping} for this sign is given in \tabref{tab:10lp}.

\begin{figure}
	\begin{tabular}{cc}
		\signpic{figure_12i} & \signpic{figure_12ii} \\
		(beginning of sign) & (end of sign) \\
	\end{tabular}	
	\caption{The ASL sign \textsc{debate}.}
	\label{fig:12}
\end{figure}

\begin{table}
\caption{Aspects of the iconic \isi{mapping} for \textsc{debate}.}
\label{tab:10lp}
\begin{tabular}{ll}
\lsptoprule
Form & Meaning\\
\midrule
dominant hand & one side of an argument\\

index finger handshape & the direction of attention\\

non-dominant hand & topics under discussion\\

flat palm handshape & a written surface\\

\textbf{reversal of dominance} & \textbf{yielding the floor to another perspective}\\

repeated \isi{movement} & an on-going process\\
\lspbottomrule
\end{tabular}
\end{table}

Coming finally to the sign \textsc{hash-things-out}, this sign's form is quite similar to the sign \textsc{debate}, the two signs differing primarily in that \textsc{hash-things-out} has a faster and smaller series of movements. The form of the sign \textsc{hash-things-out} has undergone some \isi{restructuring} that partially obscures the iconic role of the flat palm as representing a written surface, and of the \isi{alternation} between two distinct points of view. Similarly, the sign's meaning is ``softened,'' still denoting a discussion or negotiation, but with less emphasis on the the number and alignment of the participants in the discussion. The sign \textsc{hash-things-out} is articulated with the dominant\is{dominance} index finger of one hand briefly striking the flat palm of the other hand, with this motion alternating between hands multiple times in quick succession (\figref{fig:13}). This sign is not as amenable to analysis in terms of its iconic \isi{structure} as the preceding signs in this section, as it has undergone some degree of phonetic erosion: though we can identify, through comparison to the related signs \textsc{debate} and \textsc{discuss}, that the contacting motion between the dominant\is{dominance} and non-dominant hands is not an arbitrary coincidence, the simplest account for this sign is that it is a phonetically reduced and semantically idiosyncratic derivative of the sign \textsc{debate}. The sign \textsc{hash-things-out} has drifted both in meaning and in form from the sign \textsc{debate}, yielding a new, related sign.

\begin{figure}
	\begin{tabular}{cc}
		\signpic{figure_13i} & \signpic{figure_13ii} \\
		(beginning of sign) & (end of sign) \\
	\end{tabular}	
	\caption{The ASL sign \textsc{hash-things-out}.}
	\label{fig:13}
\end{figure}

  The point of this extended discussion is to demonstrate that there is no clear delineation between \isi{morphology} and iconicity in these examples. As with morphological (re)analysis, the assessment of an iconic motivation necessarily follows from the primary association of a potentially complex\is{complexity} form with a potentially complex meaning in a holistic sign. Each of the signs discussed in this section can be described both in terms of the relationship between the sign's form and its motivating visual image, and of the sign's conventional form and meaning relative to other conventional ASL signs. Similar to the discussion of the sign \textsc{very-slow}, here we see that aspects of iconic representation can also become systematic across groups of signs, and codified as a morphological pattern.

  Importantly, an a-morphous theory of iconicity recognizes that lexical signs are the primary unit of morphological organization, and an analysis of the relationship between meaning and form necessarily proceeds from there. Aspects of form such as the flat hand or the extended index finger may come to be associated with aspects of meaning by virtue of their systematic re-use across formationally and semantically related signs. However, as in spoken language \isi{morphology}, it is the identifiable parts that may gain their meanings by association with their complex\is{complexity} wholes, rather than the other way around.

\section{Iconicity in word formation}

In this section, we discuss two additional patterns that are iconic and systematic in ASL. Both patterns relate to the distinction between morphologically-related pairs of nouns and verbs. These patterns are referred to in the literature as the ``noun-verb pair'' pattern and the ``handling-instrument'' pattern, respectively.

  A quite widely-discussed morphological\is{morphology} pattern in ASL concerns pairs of related verbs and nouns that differ only in their conventional \isi{movement} pattern. Following Supalla and Newport's \citeyearpar[100--102]{Supalla1978} original formulation, these verbs and nouns are related pairs of signs such that ``the verb expresses the activity performed with or on the object named by the noun.'' Because they associate verbs and nouns through a non-concate\-native phonological operation, these pairs of signs have been compared to verb-noun pairs in \ili{English} that differ in terms of syllabic \isi{stress} (\textit{recórd/récord}) or vowel quality (\textit{bleed/ blood}), for example. The classic examples are the ASL verb \textsc{sit} and the noun \textsc{chair}: both signs are formed with the index and middle fingers extended and held together on each hand (a configuration typically referred to as the ``U handshape''), and in the articulation of both signs, the hands have the same orientation and overall \isi{movement}, with the dominant\is{dominance} hand moving to contact the top of the non-dominant hand. However, the \isi{movement} pattern differs between the two signs: \textsc{sit} is signed with the dominant\is{dominance} hand moving to rest on the top of the non-dominant hand (\figref{fig:14}a), while \textsc{chair} is signed with a shorter, repeated \isi{movement} (\figref{fig:14}b).
  
\begin{figure}
	\begin{tabular}{lcc}
		& \signpic{figure_14ai} & \signpic{figure_14aii} \\
		a. & (beginning of sign) & (end of sign) \\
		& \signpic{figure_14bi} & \signpic{figure_14bii} \\
		b. & (beginning of sign) & (end of sign) \\	
	\end{tabular}	
	\caption{The ASL sign \textsc{sit} (a) is articulated with a longer, single \isi{movement}, and \textsc{chair} (b) is articulated with a shorter, repeated \isi{movement}.}
	\label{fig:14}
\end{figure}

  Across noun-verb pairs, nouns are signed with repeated, restrained movements, while verbs are signed with longer, continuous movements: Supalla and Newport identify a number of sub-patterns within this broader generalization, and across all pairs that they identify, the nouns are all articulated with a constrained, repeated \isi{movement}. However, several different \isi{movement} patterns are observed among the verbs. These sub-patterns include a single unidirectional \isi{movement} (as in \textsc{fly-airplane} and \textsc{airplane);} a repeated unidirectional \isi{movement} (as in \textsc{sweep} and \textsc{broom}); and a repeated bidirectional \isi{movement} (as in \textsc{erase-chalkboard} and \textsc{eraser}).

  Relevant for our purposes here is the fact that the forms of these signs remain quite iconic in synchronic ASL; not only does the configuration and orientation of the hand iconically profile aspects of the referent object, as we discuss below, but, similar to the discussion of the sign \textsc{very-slow} in \sectref{sec:2}, the contrasting \isi{movement} patterns themselves have an underlying iconic motivation (and see \citealt{Wilcox2004} for additional discussion of this point). Regarding the multiple verbal \isi{movement} sub-patterns found among noun-verb pairs, Supalla and Newport note that ``in general, \textit{single} \isi{movement} in the sign corresponds to single, punctual or perfective action. \textit{Repeated} \isi{movement}, in contrast, refers to durative or iterative activity which is made of punctual actions'' \citep[103--104]{Supalla1978}. This description suggests that the perception of iconicity has not diminished from these signs. These verbs can be analyzed as transparently representing motion with motion, with, for example, the single phonological motion of the sign \textsc{sit} motivated by the motion of a human body settling on a flat surface.

  The \isi{movement} pattern for nouns can similarly be analyzed as motivated by meaning (see also \citealt[131--132]{Wilcox2004}). In ASL, repeated forms can represent repeated actions or, more abstractly, a general activity or the instrument canonically associated with an action. A \isi{derivational} process discussed by  \citet[343]{Padden1987}, for example, is the ``activity noun'' rule, in which small repeated movements derive the noun \textsc{acting} from the verb \textsc{act,} and the noun \textsc{swimming} from the verb \textsc{swim}. This is also consistent with the cross-linguistic use of reduplicated forms to derive nouns from verbs (e.g., \citealt{Nivens1993}, \citealt{Kouwenberg2001}, \citealt{Adelaar2005}): \citet[984]{Kouwenberg2015}, for example, provide \textit{kriep-kriep} `scrapings' as a noun derived through \isi{reduplication} of the verb \textit{kriep} `to scrape' in Jamacian, and \textit{doro-doro} `sieve' as a noun derived through \isi{reduplication} of the verb \textit{doro} `to sift' in Sranan. In noun-verb pairs in ASL, as well, the repetition of the verb's phonological \isi{movement} is used to denote the instrument associated with the action by de-emphasizing the action inherent to the verb.

  To make the relationship between related verbs and nouns concrete, in Tables 11 and 12 we provide the iconic \isi{mapping} for the signs \textsc{sit} and \textsc{chair}, respectively. These iconic mappings are identical except for their phonological movements and the corresponding aspects of meaning. These differences in \isi{movement} mark this pair of signs as participating in the ``noun-verb'' pattern in ASL.


\begin{table}
\caption{Aspects of the iconic \isi{mapping} for \textsc{sit}.}
\label{tab:11lp}
\begin{tabular}{ll}
\lsptoprule
Form & Meaning\\
\midrule
non-dominant hand & a surface to be sat on\\

dominant hand & an object in motion\\

U-handshape & paired human legs\\

contacting \isi{movement} & human figure settles on the surface\\

\textbf{single, continuous movement} & \textbf{a single, perfective action}\\
\lspbottomrule
\end{tabular}
\end{table}

\begin{table}
\caption{Aspects of the iconic \isi{mapping} for \textsc{chair}.}
\label{tab:12lp}
\begin{tabular}{ll}
\lsptoprule
Form & Meaning\\
\midrule
non-dominant hand & a surface to be sat on\\

dominant hand & an object in motion\\

U-handshape & paired human legs\\

contacting \isi{movement} & human figure settles on the surface\\

\textbf{repeated, constrained movement} & \textbf{an object that is acted on}\\
\lspbottomrule
\end{tabular}
\end{table}

  In our recent work, we have discussed another pattern that distinguishes a subset of related verbs and nouns in ASL: the ``handling and instrument'' pattern \citep{Padden2015}. Unlike the noun-verb pairs described above, which are distinguished from one another based on properties of their \isi{movement}, handling and instrument signs are distinguished from one another based primarily on their phonological\is{phonology}  handshapes. As an example, in ASL, the concept `toothbrush' can be represented by either of two related forms, both of which involve a constrained, repeated \isi{movement} near the mouth: the ``handling'' form has the hand configured in a variant of the fist handshape, shaped as though grasping an imagined toothbrush. The corresponding ``instrument'' form additionally has the index finger extended, representing the shape of the toothbrush, itself (see \citealt[82]{Padden2015}).

  Another pair of signs fitting this pattern are two variant forms for the concept `nail polish': Both `nail polish' forms are articulated with the fingers of the dominant \is{dominance}hand repeatedly brushing the fingers of the non-dominant hand. The handling form is signed with the dominant\is{dominance} hand in what is known as the ``F handshape'', with the index finger contacting the thumb as though grasping a small, thin brush (\figref{fig:15}a), and the instrument form is signed with the ``U handshape'', with index and middle finger extended and closed, representing the bristles of a small brush (\figref{fig:15}b).
  
\begin{figure}
	\begin{tabular}{lcc}
		& \signpic{figure_15ai} & \signpic{figure_15aii} \\
		a. & (beginning of sign) & (end of sign) \\
		& \signpic{figure_15bi} & \signpic{figure_15bii} \\
		b. & (beginning of sign) & (end of sign) \\	
	\end{tabular}	
	\caption{Two signs meaning `nail polish'. The handling form (a) is signed with a ``grasping'' handshape, while the instrument form (b) is signed with a ``brushing'' handshape.}
	\label{fig:15}
\end{figure}

  In ASL, handling and instrument forms can both also function as verbs, for example `to brush one's teeth' or `to paint one's nails', when signed with the appropriate longer \isi{movement} pattern. However, analyzing elicited sentences in a vignette description task \citep{Padden2015}, we found previously that ASL signers are more likely to employ handling and instrument forms as verbs and nouns, respectively. Unlike the \isi{movement} contrast that distinguishes noun-verb pairs, the association of handling forms with verbs and instrument forms with nouns is a statistically reliable preference, rather than a categorical rule. As an example of this preferred pattern, consider the sentence in (\ref{ex:1}). In this signed sentence, the signer used an instrument form to name the object, and a handling form to name the action associated with that object:

\ea
    \label{ex:1}
\textsc{nail-polish, woman paint-nails}\\
`A woman paints her nails with nail polish.'
\z

\largerpage
In this example, the topicalized sign \textsc{nail-polish} is identifiable as a noun because of the \isi{semantics} of the sentence, as well as its participation in the movement-based noun-verb pattern described above: This sentence was uttered as a description of a short vignette in which a woman painted her nails, and in this sentence, the sign \textsc{nail-polish} is articulated with the short, restrained \isi{movement} that is characteristic of derived nouns. In contrast, the sign \textsc{paint-nails} is identifiable as a verb: it is articulated with several longer, unidirectional movements to represent `brushing' as an on-going process. In this sentence, the handshapes for the noun \textsc{nail-polish} and the verb \textsc{paint-nails} also differ: \textsc{nail-polish} is formed with the index and middle finger together representing the brush used to apply nail polish, while \textsc{paint-nails} is formed with the fingers configured as though handling the brush as a small object. These aspects of form and meaning for the signs \textsc{nail-polish} and \textsc{paint-nails} can be represented as in Tables 13 and 14, below.

\begin{table}
\caption{Aspects of the iconic \isi{mapping} for \textsc{nail-polish}.}
\label{tab:13lp}
\begin{tabular}{ll}
\lsptoprule
Form & Meaning\\
\midrule
dominant hand & the hand of an \isi{agent}\\

\textbf{U-handshape} & \textbf{the bristles of a small brush}\\

non-dominant hand & a human hand\\

repeated, constrained \isi{movement} & an object that is manipulated\\
\lspbottomrule
\end{tabular}
\end{table}

\begin{table}
\caption{Aspects of the iconic \isi{mapping} for \textsc{paint-nails}.}
\label{tab:14lp}
\begin{tabular}{ll}
\lsptoprule
Form & Meaning\\
\midrule
dominant hand & the hand of an \isi{agent}\\

\textbf{F-handshape} & \textbf{grasping a small object}\\

non-dominant hand & a human hand\\

repeated, unidirectional \isi{movement} & the repeated action of a human \isi{agent}\\
\lspbottomrule
\end{tabular}
\end{table}

  The preferential pairing of instrument forms with nouns and handling forms with verbs can be analyzed as motivated by the fact that in a handling form, the phonological\is{phonology}  \isi{structure} profiles the action performed by a human \isi{agent} (see also \citealt{Brentari2012} and \citealt{Hwang2016}). The phonological \isi{structure} of the instrument form additionally profiles the shape of the object used to perform the action.

  In these ``noun-verb pair'' and ``handling and instrument'' examples, it is possible to associate an \isi{aspect} of form (such as a handshape or \isi{movement} pattern) with a syntactic category (such as noun or verb) and/or an \isi{aspect} of meaning (such as agency or duration). But these consistent form-meaning pairings are identifiable only in comparison to other signs: There is no recurring ``handling \isi{affix}'' or ``derived noun \isi{affix}'' to mark these patterns. Instead, both noun-verb pairs and handling-instrument signs are distinctive patterns recognizable only through their iconicity. They are paradigmatic\is{paradigm} relationships that are identifiable on the basis of their iconic motivations.

\section{Conclusion}

In this chapter, we have taken inspiration from Anderson's a-morphous theory of \isi{morphology}, which views Saussurean signs as holistic pairings of potentially complex form and potentially complex meaning.\is{complexity} From this perspective, rather than complex words deriving their meanings from the meanings of their parts, it is instead the parts of a complex word that may derive their meanings from the whole words that they appear in. We have demonstrated that this ``a-morphous'' view provides a fresh perspective on sign-internal motivation, regardless of whether this motivation can be considered morphological\is{morphology} or iconic. In sign languages, the perception of iconicity or morphological \isi{complexity} arises from speaker (re)analysis of the relation between form and meaning among the words of a language. Any whole word may drift in form and meaning in a way that obscures its original, analyzable internal \isi{structure}. However, signs also often receive \isi{analogical} support from other, related signs. As the field of sign language \isi{linguistics} continues to recognize the relationship between iconicity and \isi{morphology} as related aspects of motivation in linguistic \isi{structure}, Anderson's insights will continue to provide a useful framework for analysis for some time to come.

%\section*{Abbreviations}
\section*{Acknowledgements}
We wish to thank the editors and contributors to this volume, particularly Claire Bowern, for this opportunity to honor Steve Anderson. Ryan Bennett and two anonymous reviewers provided helpful feedback on this work. Thank you also to Mark Aronoff, Lynn Hou, Irit Meir, Hope Morgan, Corrine Occhino, Wendy Sandler, Amira Silver-Swartz, and Tessa Verhoef for additional discussion and comments.

{\sloppy
\printbibliography[heading=subbibliography,notkeyword=this]
}
% \todos


\end{document}
