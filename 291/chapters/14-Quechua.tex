\documentclass[output=paper]{langscibook}
\ChapterDOI{10.5281/zenodo.13208566}
\author{Gladys Camacho Rios\affiliation{State University of New York at Buffalo} and Adam J. R. Tallman\affiliation{Friedrich-Schiller-Universität Jena}}
\title{Word structure and constituency in Uma Piwra South Bolivian Quechua}
\abstract{This chapter provides a description of word structure in Uma Piwra South Bolivian Quechua, a variety of Quechua spoken by monolinguals in rural Bolivia. We show that the language displays a mostly fixed ordering of suffixes and clitics. Cases where variable ordering does occur appear to not be associated with a scope difference. We apply the word/phrase constituency tests to Uma Piwra South Bolivian Quechua (SBQ) and show that the traditional notion of the word has some general support when constituency tests are aggregated.}

\IfFileExists{../localcommands.tex}{%hack to check whether this is being compiled as part of a collection or standalone
   \usepackage{langsci-optional}
\usepackage{langsci-gb4e}
\usepackage{langsci-lgr}

\usepackage{listings}
\lstset{basicstyle=\ttfamily,tabsize=2,breaklines=true}

%added by author
% \usepackage{tipa}
\usepackage{multirow}
\graphicspath{{figures/}}
\usepackage{langsci-branding}

   
\newcommand{\sent}{\enumsentence}
\newcommand{\sents}{\eenumsentence}
\let\citeasnoun\citet

\renewcommand{\lsCoverTitleFont}[1]{\sffamily\addfontfeatures{Scale=MatchUppercase}\fontsize{44pt}{16mm}\selectfont #1}
  
   %% hyphenation points for line breaks
%% Normally, automatic hyphenation in LaTeX is very good
%% If a word is mis-hyphenated, add it to this file
%%
%% add information to TeX file before \begin{document} with:
%% %% hyphenation points for line breaks
%% Normally, automatic hyphenation in LaTeX is very good
%% If a word is mis-hyphenated, add it to this file
%%
%% add information to TeX file before \begin{document} with:
%% %% hyphenation points for line breaks
%% Normally, automatic hyphenation in LaTeX is very good
%% If a word is mis-hyphenated, add it to this file
%%
%% add information to TeX file before \begin{document} with:
%% \include{localhyphenation}
\hyphenation{
affri-ca-te
affri-ca-tes
an-no-tated
com-ple-ments
com-po-si-tio-na-li-ty
non-com-po-si-tio-na-li-ty
Gon-zá-lez
out-side
Ri-chárd
se-man-tics
STREU-SLE
Tie-de-mann
}
\hyphenation{
affri-ca-te
affri-ca-tes
an-no-tated
com-ple-ments
com-po-si-tio-na-li-ty
non-com-po-si-tio-na-li-ty
Gon-zá-lez
out-side
Ri-chárd
se-man-tics
STREU-SLE
Tie-de-mann
}
\hyphenation{
affri-ca-te
affri-ca-tes
an-no-tated
com-ple-ments
com-po-si-tio-na-li-ty
non-com-po-si-tio-na-li-ty
Gon-zá-lez
out-side
Ri-chárd
se-man-tics
STREU-SLE
Tie-de-mann
}
    \bibliography{../localbibliography}
    \togglepaper[14]
}{}


\begin{document}
\maketitle


\section{Introduction} % (fold)
\label{sbq:sec:introduction}

This chapter provides a description of word and constituency structure in South Bolivian Quechua (SBQ) as spoken by monolingual elders in the rural town of Uma Piwra. The analysis is based on 50 hours of naturalistic speech and native speaker judgements. Data were gathered in a monolingual context by one of the coauthors who is a native speaker of South Bolivian Quechua. The first section describes the planar structure of the verb complex in SBQ, discussing cases where the analysis diverges from previous work on the language. The second section focuses on morphosyntactic domains and the third section describes phonological domains. The concluding section summarizes the results in relation to the word bisection thesis and contextualizes them in terms of discussions around the distinction between lexicalist versus syntactic approaches to Quechua morphology \citep{weber1983relationship, muysken:1981}.

% section introduction (end)

\section{The language and its speakers} % (fold)
\label{sbq:sec:language}

South Bolivian Quechua (SBQ) has around 1,610,000 speakers. It is spoken in the urban and rural areas or local towns of the departments of Cochabamba, Chuquisaca, Oruro and Potosí \citep{plazamartinez:2009}. Urban areas are Spanish-dominant and most people in these areas only speak Spanish. Most Quechua-Spanish bilinguals living in urban contexts learned Quechua from their monolingual parents who have migrated from a rural area in their childhood or youth. Rural area towns or local towns are located at a further distance from the urban areas. People in these towns are Quechua-dominant, and in most cases monolingual. In the town of Uma Piwra, there are eight families and just 17 people in total. This town is inhabited almost entirely by elders because of out-of-town migration by the youth. People use Quechua to communicate in their everyday life. People in this town mostly do agricultural work such as the cultivation of different legumes.

Data for this chapter were collected with ten monolingual speakers in Uma Piwra. All of them are above 70 years old. The analysis is supported with 50 hours of spontaneous conversations collected in daily conversations. One of the authors collected the data in audio and video formats. The analysis is also supported with 60 hours of elicitation. 

\largerpage[-2]
\section{Verbal planar structure} % (fold)
\label{sbq:sec:planarstructures}

The verbal planar structure is provided in \tabref{sbq:tab:verbalplanar}. The orthographic verbal word, practiced by literate Quechua speakers, runs over the \ref{vpos:vcore}-\ref{vpos:SAsuf} span of the verbal planar structure.

A distinction between affixes (marked with ``-'') and clitics (marked with ``='') is represented orthographically for expositional reasons. We follow most of the literature on Quechuan languages in this regard, but attach no theoretical significance to the distinction. In the verbal domain, the definition of an affix is a morph which is bound and occurs between the verb core of position \ref{vpos:vcore} and its inflectional suffix of position \ref{vpos:SAsuf}. Clitics are any bound morpheme that does not follow this definition. Affixes and clitics are bound in the sense of not being able to stand alone as an elliptical utterance (i.e. they cannot be free forms). Roots in the nominal domain are free forms. Notice that the clitics are represented twice in the verb structure on each side of the verb. This reflects the fact that SBQ clitics can occur before or after the verb. In the planar fractal method we fix the core in position.\footnote{To avoid confusion we point out here that representing the clitics twice does not reflect a view by which clitics start out in a base position and move either before or after a stationary verb. The verb core is fixed to be stationary in the planar structure in order to code constituency test results consistently in relation to edges.}

\largerpage[-2]

\begin{longtable}{Slll}
\caption{Verbal planar structure of South Bolivian Quechua} 
\label{sbq:tab:verbalplanar} 
\endfirsthead 
\endhead
\lsptoprule
	\multicolumn{1}{l}{Pos.}    & Type  & Elements  & Forms \\ \midrule
\label{vpos:XP1}     & zone  & NP, PP, S & \textit{} \\
\label{vpos:ña1}     & slot  & already   & \textit{=ña}                  \\
\label{vpos:raq1}     & slot  & already   & \textit{=raq}            \\
\label{vpos:puni1}   & slot  & certainly     & \textit{=puni} \\
\label{vpos:taq1} & slot & adverbial, coord.         & \textit{=taq}   \\
\label{vpos:sina1}    & slot  & uncertainty   & \textit{=sina}     \\
\label{vpos:pis1}    & slot  & also   & \textit{=pis}     \\
\label{vpos:clitics} & slot & top., neg., dub., assert. & \textit{=chu, =chus, =chá, =pis}   \\
\label{vpos:ri} & slot & interrogative & \textit{=ri, , =rí}   \\
\label{vpos:qa} & slot & topic marker  & \textit{=qa}   \\
\label{vpos:XP2}     & zone          & NP, PP, S               & \textit{}               \\
\label{vpos:vcore}       & slot  &  \textbf{verb core}   & \textit{}      \\
\label{vpos:derivsuf1}   & slot  & derivational suffixes and clusters &   \\
\label{vpos:ri.suffix}   & slot  & inceptive     & \textit{-ri}               \\
\label{vpos:recipsuf} & slot     & reciprocal        & \textit{-na}              \\
\label{vpos:asssuf}  & zone  & adverbial, assistive  & \textit{-rqu,-rpa,-ysi} \\
\label{vpos:affect}  & slot  & affective     & \textit{-ri}      \\
\label{vpos:chi}     & slot  & causative, passive     & \textit{-chi, -chiku}     \\
\label{vpos:advsuf} & slot  & reflexive, motion, adverbial  & 
\textit{-ku,-mu,-pu, -kampu,-kapu,-mpu}    \\
\label{vpos:naya} & slot  & proximal future  & \textit{-naya}    \\
\label{vpos:prog}    & slot  & progressive       & \textit{-sha}    \\
\label{vpos:lla1}    & slot  & limitative        & \textit{-lla}                  \\
\label{vpos:1.obj}   & slot  & \First{}.\textsc{obj}    & \textit{-wa}     \\
\label{vpos:when}    & slot  & when              & \textit{-qti}                  \\
\label{vpos:tame}    & zone  & TAME, second person       & \textit{-rqa,-sqa,-na}, \textit{-su,-sa}       \\ 
\label{vpos:SAsuf}  & slot     & A/S person    &    \textit{-ni}, \textit{-yki},\textit{-nki},\textit{-n}    \\ 
\label{vpos:3.obj}  & slot     & future:\Third{}.\textsc{obj}    &    \textit{-q(a)}   \\ 
\label{vpos:SAsuf2}  & slot     & person    &    \textit{-n} \\ 
\label{vpos:plural} & slot & plural & \textit{-chik}, \textit{-yku}, \textit{-ku} \\
\label{vpos:lla2}    & slot  & limitative    & \textit{-lla}     \\ 
\label{vpos:XP3}     & zone  & NP, PP, S &      \\
\label{vpos:ña2}     & slot  & already   & \textit{=ña}                  \\
\label{vpos:raq2}     & slot  & already   & \textit{=raq}            \\
\label{vpos:puni2}   & slot  & certainly     & \textit{=puni} \\
\label{vpos:taq2} & slot & adverbial, coord.         & \textit{=taq}   \\
\label{vpos:sina2}    & slot  & uncertainty   & \textit{=sina}     \\
\label{vpos:pis2}    & slot  & also   & \textit{=pis}     \\
\label{vpos:clitics2} & slot & top., neg., dub., assert. & \textit{=chu, =chus, =chá}   \\
\label{vpos:ri2} & slot & interrogative & \textit{=ri, =rí}   \\
\label{vpos:qa2} & slot & topic marker  & \textit{=qa}   \\
\label{vpos:aux}    & slot  & regretative   &   \textit{karqa, kaq} \\
\label{vpos:XP4}    & zone  &   NP, PP, S   &   \\
\lspbottomrule
\end{longtable}


%\begin{table}
%\caption{Nominal planar structure of South Bolivian Quechua}
%\label{tab:nominalplanar} 
%\begin{tabular}{Tlll}
% \lsptoprule
% 	\multicolumn{1}{l}{Pos.}    & Type  & Elements  & Forms \\ \midrule
% \label{npos:dem1}    & slot      & demonstrative & \textit{kay}, \textit{chay}, \textit{jaqay} \\
% \label{npos:adj}    & slot      & adjective, (noun) & \textit{} \\
% \label{npos:ncore}    & slot & \textbf{noun core}   &      \\
% \label{npos:dimin}    & slot & diminutive    & \textit{-ito,-ita,-situ,-sita} \\
% \label{npos:plursuf}    & slot   & plural    & \textit{-s(ni), -kuna}             \\
% \label{npos:lla4}    & slot     & limitative        & \textit{=lla}     \\
% \label{npos:posssuf}    & slot   & possessives    & \textit{-y, -yki, -n, -nchik, -yku, -nku}          \\
% \label{npos:jina}    & slot   & similar    & \textit{jina}       \\
% \label{npos:lla3}    & slot     & limitative        & \textit{=lla}     \\
% \label{npos:casesuf}    & slot  & accusative, locative / by means of    & \textit{-ta,-pi}  \\
% \label{npos:pis}    & slot      & also      & \textit{-pis}              \\
% \lspbottomrule
% \end{tabular}
% \end{table}

\largerpage[1.5]
The structure provided above requires some commentary in light of previous descriptions of SBQ morphology. Some previous analyses described SBQ as displaying a type of ``layered'' structure in its verb with a some degree of variable suffix ordering whereby suffix order necessarily covaries with a difference in scope \citep{muysken:1981,vanderkerke:1996}.\footnote{ \citet[209]{adelaar2004} state: ``The order in which suffixes occur in a verb form is essentially fixed, although more than one option may be available in some parts of the suffix inventory", but do not elaborate on which suffixes they are referring to.} However, data gathered from spontaneous speech and rechecked with native speakers suggest that these analyses assigned flexible syntax-like ordering to suffixes in SBQ where there is none. First, variable ordering of affixes is much more limited in SBQ than is implied by \citet{muysken:1981} and \citet{vanderkerke:1996}. For instance, the suffix \textit{-chi} `causative' always precedes the suffix \textit{-pu} `benefactive' as in (\ref{ex:chipu}). The suffix \textit{-ri} `nicely' always precedes \textit{-chi} `causative' as in (\ref{ex:richi}). This is in contrast to previous literature which suggests that the causative morpheme can be variably ordered with most of the suffixes of the language.

\ea \label{ex:chipu}{
	\glll {} yaku -ta chura -chi -pu -n -ku ni -n -ku \\
	        v: \ref{vpos:XP2} - \ref{vpos:vcore} -\ref{vpos:chi} -\ref{vpos:advsuf} -\ref{vpos:SAsuf} -\ref{vpos:plural}  \ref{vpos:vcore} -\ref{vpos:SAsuf} -\ref{vpos:plural}  \\
		    {} water -\Acc{} install -\Caus{} -\Ben{} -3 -\Pl{} say -3 -\Pl{} \\
	\glt `People say they (the governors) made (the water installers) to install water on (Juan de la Cruz's) behalf.' \\ Sp. `Dicen que se lo hicieron instalar agua (para Juan de la Cruz).' \hfill }
\z

\ea \label{ex:richi}{
    \glll 	{} caballiyu -ri -spa apa -ri -chi -ku -n eh \\
		    v: \ref{vpos:vcore} -\ref{vpos:ri} -\ref{vpos:tame} \ref{vpos:vcore} -\ref{vpos:ri} -\ref{vpos:chi} -\ref{vpos:advsuf} -\ref{vpos:SAsuf} \\
			{} ride -nicely -\Gerund{} carry -nicely -\Caus{} -\Refl{} -3.\Sg{} eh \\
	\glt `It (the fox) makes it (the ship) to carry it (the fox) while nicely riding on it (the ship)'. \\ Sp. `(El zorro) Se hace llevar cabalgando (sobre la oveja) pues.' \hfill }
\z


Variable ordering occurs in two areas of the verb complex in SBQ. The suffixes \textit{-rqu} `nimbly, affective' and  \textit{-rpa} `suddenly' variably order with respect to \textit{-ysi} `assistive'. The verbal planar structure captures this variable ordering by having a zone for position \ref{vpos:asssuf} containing these three elements. Examples illustrating the variable ordering of morphemes in this position are provided in (\ref{ex:rpaysi2}) and (\ref{ex:ysirpa}).

\ea \label{ex:rpaysi2}{
	\glll {} kuntan t'iqpa \textbf{-rpa} \textbf{-ysi} -lla -sa -yki ni -spa 	\\
	    	v: \ref{vpos:XP2} \ref{vpos:vcore} -\ref{vpos:asssuf} -\ref{vpos:asssuf} -\ref{vpos:lla1} -\ref{vpos:tame} -\ref{vpos:SAsuf} \ref{vpos:vcore} -\ref{vpos:tame} \\
		    {} soon peel \textbf{-suddenly} \textbf{-\Assist{}} -only -\Second{}.\textsc{obj} -\First{}.\Sg{} say -\Gerund{} \\
	\glt `I will soon help you to peel out (the dry corn) saying...'\\ Sp. `Enseguida te ayudaré no más a pelar (el maíz) diciendo...' \hfill }  
\z

\ea \label{ex:ysirpa}{
	\glll {} Q'ala uña pili -situ -sni -y -ta uqu \textbf{-ysi} \textbf{-rpa} -nku \\
        v: \ref{vpos:XP1} [\ref{vpos:XP2} - - - - -] \ref{vpos:vcore} -\ref{vpos:asssuf} -\ref{vpos:asssuf} -\ref{vpos:SAsuf} \\
		{} entirely little duck -\Aff{} -\Pl{} -1\Sg{}.\Poss{} -\Acc{} devour \textbf{-\Assist{}} \textbf{-suddenly}  -3\Pl{}	\\
	\glt `They (wild birds) will help to devour my little duck's (food).' \\ Sp. `Los pájaros le ayudan a terminar de comer la comida de mis patitos.' \hfill }
\z

There is also variable ordering in position \ref{vpos:tame} of the verb complex. The second person and future markers \textit{-su} and \textit{-sa} display variable ordering with the marker \textit{-rqa} and \textit{-sqa}. These variable orderings are not associated with a difference in scope, contradicting previous literature, a point which we elaborate on in \sectref{sbq:sec:non-permutability} on non-permutability domains.

%confusions based on productivity.
Secondly, previous analyses overgeneralize and imply or state that all SBQ suffixes were fully productive in the sense that they could combine with all verb roots. Some suffixes display this property, but others appear to be ``lexicalized'' in the sense that they should be treated as listemes with the verb roots they can combine with. Suffixes in position \ref{vpos:derivsuf1} in the verb complex are not productive as they are restricted to specific verb roots. We will refer to such suffixes as ``lexicalized'' in what follows.

Not recognizing the distinction between lexicalized suffixes and other suffixes leads to a different descriptive claim concerning which suffixes in SBQ can variably order. Without taking into account the relative productivity of formatives in relation to their syntagmatic distribution, we could conclude that many suffixes freely permute with one another in the SBQ verb complex. For instance, many of the examples with variable ordering presented in previous literature are provided with the causative \textit{-chi}. However, this can be plausibly attributed to an analysis that posits a ``lexicalized'' (unproductive, root-suffix listeme) \textit{-chi} of position \ref{vpos:derivsuf1} and a productive \textit{-chi} of position \ref{vpos:chi}.\footnote{Note again that the planar structure tells us nothing about the relative productivity of the elements which occur in its positions. If we want to capture degrees of productivity we establish a constituency variable that refers to spans of structure where productivity breaks down according to some cross-linguistically applicable definition of productivity.} Evidence for this analysis comes from the fact that the doubling of \textit{-chi} `causative' can only be achieved on a handful of roots and in such cases the first \textit{-chi} is always directly right-adjacent to the verb root. The examples in (\ref{ex:chinotproductive}) illustrate the basic point that \textit{-chi} `causative' doubling is not productive in the language. Furthermore, in many cases where a surface repetition of \textit{-chi} is found the root cannot occur independently from the causative suffix.

\ea \label{ex:chinotproductive}
        \ea[]{
    \gll wan-chi-chi (*\textit{wan} cannot occur as a root independent of \textit{-chi}) \\
    die?-\Caus{}-\Caus{} \\
    \glt `make someone kill' \\
    } 
    \ex[]{
    \gll riku-chi-chi \\
    see-\Caus{}-\Caus{} \\
    \glt `(make) someone show' \\
    }
    \z
\z

Surface repetition of \textit{-chi} `causative' occurs when a root with a lexicalized formative \textit{-chi} `causative' of position \ref{vpos:derivsuf1} is combined with the productive \textit{-chi} of position \ref{vpos:chi} as illustrated in (\ref{ex:quchichi}). Note that a lexicalized \textit{-chi} can never be interrupted by another suffix.

\ea \label{ex:quchichi}
	\glll {} ataque -s -wan qu -chi -chi -ku -nku wakin =qa \\
        v: \ref{vpos:XP2} - - \ref{vpos:vcore} -\ref{vpos:derivsuf1} -\ref{vpos:chi} -\ref{vpos:advsuf} -\ref{vpos:SAsuf} \ref{vpos:XP3} \ref{vpos:clitics2} \\
		{} ataque -\Pl{} -\Com{} give -\Caus{} -\Refl{} -3:\Pl{} other =\Top{}	 \\
	\glt `Others (cry so much) that they cause an attack on their own bodies.' \\ Sp. `Otros (lloran tanto) hasta que el llanto excesivo les cause ataque a si mismos.' 
\z


In our view, apparent examples of variable ordering surface because verbs such as \textit{qu} `give' and \textit{riku} `see' can appear with \textit{-chi} in two positions. However, this variable ordering is marginal. Most verbs only allow \textit{-chi} `causative' to appear in position \ref{vpos:chi}, where the morpheme occurs productively. Note also that \textit{qu-} is not strictly a \textit{morpheme} in the sense that it is a form with an identifiable meaning independent of its context. The root \textit{qu-} only acquires meaning through combination with the formative \textit{-chi}, and the unproductive \textit{-chi} in this context must be listed with this verb root.

Note that the morpheme \textit{-chi} `causative' cannot be variably ordered with reflexive, motion or adverbial suffixes of position \ref{vpos:advsuf}. This point is illustrated with the ungrammatical sentences in (\ref{ex:kuchi}) and (\ref{ex:quchichi2}).

\ea \label{ex:kuchi}
	\glll {}  t'anta -ta urqhu \textbf{-chi} \textbf{-ku} -ni / *t'anta-ta urqhu -\textbf{ku} \textbf{-chi} -ni \\
        v: \ref{vpos:XP1} - \ref{vpos:vcore} -\ref{vpos:chi} -\ref{vpos:advsuf} -\ref{vpos:SAsuf} / \ref{vpos:XP1}- \ref{vpos:vcore} -\ref{vpos:advsuf} -\ref{vpos:chi} -\ref{vpos:SAsuf}  \\
		{} bread -\Acc{} take.out -\Caus{} -\Refl{} -\First{}:\Sg{} / bread-\Acc{} take.out -\Refl{} -\Caus{} -1:\Sg{} 	\\
	\glt `I made someone take out the bread (out of the oven) by himself.' \\ Sp. `Hice sacar pan para mi.' \hfill 
\z

\ea \label{ex:quchichi2}
	\glll {} waka -ta mi -chi \textbf{-chi} \textbf{-mu} -n  / *waka -ta mi -chi \textbf{-mu} \textbf{-chi} -n \\
        v: \ref{vpos:XP2} - \ref{vpos:vcore} -\ref{vpos:derivsuf1} -\ref{vpos:chi} -\ref{vpos:advsuf} \ref{vpos:SAsuf} \\
		{} cow -\Acc{} graze -\Caus{} -\Caus{} -go\&do -\Third{}:\Sg{} / cow -\Acc{} graze -\Caus{} -go\&do -\Caus{} -3:\Sg{}	\\
	\glt `She made someone go and graze the cow.' \\ Sp. `Ella hizo patear la vaca (con alguien más).' \hfill 
\z

Furthermore, based on the data available to us from Uma Piwra some of the apparent examples of variable affix ordering presented in \citet[636]{muysken:1986affixorder} are based on glossings for which we cannot find clear evidential support. For instance, there are (at least) two \textit{-na} suffixes. The position \ref{vpos:recipsuf} \textit{-na} is a reciprocal and the position \ref{vpos:tame} \textit{-na} is a modal obligatative suffix that must co-occur with the auxiliary \textit{karqa}. 

Muysken's translation of some examples with \textit{-na} suggest that there is variable ordering between \textit{-chi} `causative' and \textit{-na} `reciprocal'. Evidence for the variable ordering disappears when we provide what we consider to be more accurate translations for the relevant sentences coupled with empirical motivated glosses of the suffixes in question.\footnote{We hasten to add here that Muysken was probably describing a different dialect of SBQ and that the translations he provided were more appropriate for the dialect he described. We do not know enough about variation in Bolivian Quechua at this point to know for sure. This caveat should be applied to all cases where we present an analysis which diverges from Muysken's.} We provide the relevant example below with Muysken's translation beside our own. When \textit{-chi} `causative' occurs before \textit{-na}, the latter formative provides a modal meaning, whereas when \textit{-chi} `causative' occurs after \textit{-na} the formative has the reciprocal meaning. Contrary to what is implied by \citet[634]{muysken:1986affixorder}, we have not been able to corroborate the claim that a post \textit{-chi} `causative' \textit{-na} is grammatical without the modal auxiliary \textit{karqa} `regretative'.

\ea \label{ex:china}
	\glll {} riku -chi -na -nku karqa / *riku-chi-na-nku \\
	    	v: \ref{vpos:vcore} -\ref{vpos:chi} -\ref{vpos:tame} -\ref{vpos:SAsuf} \ref{vpos:aux} \\
		    {} see -\Caus{} -\Oblig{} -3.\Pl{} \Regret{}  \\
	\glt `What a shame that they seem to have shown (it) to (someone)'. \\ Sp. `Era que le muestren (algo) a (él,ella o a ellos, ellas).' \\   
 \citet[636]{muysken:1986affixorder}: *`they make each other see (something).' (based on \textit{-na} ‘reciprocal')
\z

\ea \label{ex:nachi}
	\glll {} riku -na -chi -nku 	\\
	    	v: \ref{vpos:vcore} -\ref{vpos:recipsuf} -\ref{vpos:chi} -\ref{vpos:SAsuf}  \\
		    {} see -\Recp{} -\Caus{} -3.\Pl{} \\
	\glt `They make (them) see each other.' \hfill \citep[44]{muysken1989:dialectvariationquechua}
\z

According to the analysis adopted here, SBQ has a number of ``lexicalized'' suffix clusters (see \citealt[208--209]{adelaar2004} for preliminary commentary on this phenomenon in Quechuan languages). A lexicalized suffix cluster refers to a string of formatives that always occur adjacent to one another, but where the meaning of the whole is not clearly discernible from its parts. Historically, the formatives were likely different morphemes. However, synchronically, breaking apart such clusters in one's morphemic analysis results in an unmotivated complexification of the structure of the verb complex (see \citealt{Reuse1994} for similar phenomena in Siberian Yupik Eskimo).

Lexicalized suffix clusters do not occupy individual positions but rather replace entire spans of positions. Some of the better understood suffix clusters are listed in \tabref{tab:suffixclusters}. These suffixes are analyzed as replacing spans based on distributional facts. A \ref{vpos:derivsuf1}-\ref{vpos:naya} span replacing suffix such as \textit{-yamu} `do on purpose' cannot co-occur with any morphemes from positions \ref{vpos:derivsuf1} to \ref{vpos:naya}.

\begin{table}[htp]
\caption{Suffix clusters Uma Piwra South Bolivian Quechua}
\label{tab:suffixclusters}
\begin{tabular}{lll}
\lsptoprule
Form     & Meaning                             & Replaces span \\ \midrule
\textit{-yamu}    & `on purpose'                        & \ref{vpos:derivsuf1}-\ref{vpos:naya}         \\
\textit{-yapu}    & `completely and irreversible'       & \ref{vpos:derivsuf1}-\ref{vpos:naya}         \\
\textit{-rqamu}   & `rapidily and diligently'           & \ref{vpos:ri.suffix}-\ref{vpos:naya}         \\
\textit{-rqapu}   & `do diligently on someone's behalf' & \ref{vpos:derivsuf1}-\ref{vpos:naya}         \\
\textit{-kamu}   & `do to st.possessed and go'            & \ref{vpos:advsuf}-\ref{vpos:naya}         \\
\textit{-kampu}   & `do and move for safety'            & \ref{vpos:advsuf}-\ref{vpos:naya}         \\
\textit{-kapu}   & `entirely'            & \ref{vpos:advsuf}-\ref{vpos:naya}         \\
\textit{-rqakamu} & 'do diligently'                     & \ref{vpos:derivsuf1}-\ref{vpos:naya}         \\
\textit{-yarpa}   & `without waiting, thinking'         & \ref{vpos:derivsuf1}-\ref{vpos:affect}         \\
\textit{-rpaya}   & `to do V with force on plural P'    & \ref{vpos:derivsuf1}-\ref{vpos:affect}        \\
\lspbottomrule
\end{tabular}
\end{table}

There are no suffix-suffix clusters that can co-occur with the root-adjacent suffix \textit{-chi} `causative'. It is for this reason that the lexicalized \textit{-chi} `causative' is in position \ref{vpos:derivsuf1} with the suffix-suffix clusters.

With regard to lexicalized suffix-suffix combinations, Muysken considers and dismisses the possibility that certain \textit{-ku}-suffix combinations might be lexicalized and better treated as units \citep[298]{muysken:1981}. Muysken's discussion is partially relevant to the analysis presented here because we do analyze some candidate \textit{-ku-}suffix combinations as lexicalized clusters of formatives (particularly those in position \ref{vpos:advsuf}). For Muysken, our morphemes \textit{-kamu} `do and go', \textit{-kampu} `do and go' and \textit{-kapu} `entirely' might be analyzed as underlyingly \textit{-ku-mu}, \textit{-ku-mpu} and \textit{-ku-pu}, respectively, subject to a vowel dissimilation rule (/a/ \rightarrow /u/ prior to the phoneme /u/). Muysken considers the lexicalization analysis ``highly implausible" because the formative \textit{-ku} reoccurs with different suffixes.

In his lexicalist analysis of Quechua word structure, \citet[297--298]{muysken:1981} argues against treating certain formative combinations as lexicalized combinations. With respect to root-suffix lexicalized forms, Muysken argues that there is no independent evidence for such an analysis, except for a single case of root contraction (\textit{wañu-chi} `die-\Caus{}' \rightarrow \textit{wañu-chi}). 

Regarding Muysken's assumption that strings of \textit{-ka}C(C)u are, in fact, \textit{-ku}-suffixes, where \textit{-ku} is subject to a dissimilation rule, Muysken notes that \textit{-ku} reoccurs with several other suffixes. We have struggled to understand the appeal of Muysken's argument for three reasons. First, the formative \textit{-ku} does \textit{not} occur in such cases without the stipulation of the dissimilation rule, which has no motivation anywhere else in the grammar, rendering the argument circular. Secondly, the instances of \textit{-ka}/\textit{-ku} in such combination do not have the same meaning as \textit{-ku} in other contexts. The suffix \textit{-ku} of position \ref{vpos:advsuf} adds a reflexive meaning, and the suffix \textit{-ku} of position \ref{vpos:plural} adds a plural meaning. Neither of these meanings is easily discernible with the \textit{-ka} in the context of \textit{-kamu} `do and go (volitional A/S)', \textit{-kampu} `do and go' and \textit{-kapu} `entirely'. Finally, it is unclear why  the reoccurring presence of a formative should be surprising if Quechua is subject to processes of grammaticalization and lexicalization as any other language (see \citealt{Reuse1994}).

As to Muysken's dismissal of root-\textit{chi} lexicalizations, he does not mention the distributional difference between lexicalized \textit{-chi} and productive \textit{-chi}, nor the fact that the roots to which the candidate lexicalized form combines often do not have independent meanings. Furthermore, Muysken admits that his analysis overgenerates as it predicts that one should always be able to double (or repeat \textit{ad infinitum)} \textit{-chi} in a verb complex. He does not account for why \textit{-chi} cannot repeat consistently, but shrugs off the problem with a promissory statement that some unknown semantic theory will be able to capture the distributional facts in the future: ``The overgeneration strategy followed here will have to find support when a more precise theory of semantic interpretation for causatives is sketched; we will return to it then" \citep[297]{muysken:1981}. He never sketches such a theory. Muysken is, however, correct that lexicalized \textit{-chi} and productive \textit{-chi} impart the same (or highly comparable) meanings and display the same form. With regards to \textit{-chi}, therefore, we will assume that \textit{-chi} could be treated as one morpheme or two and fracture relevant constituency variables accordingly (see \sectref{sbq:sec:non-permutability}).

Positions \ref{vpos:1.obj} through \ref{vpos:plural} are filled out by obligatory inflectional markers that mark number and person of the subject and/or object and a tense distinction. The most basic tense distinction is future versus non-future.  The presentation below is limited to describing the patterns that we find and discussing the relative position of the formatives in positions of the verbal template dedicated to `inflection' and to describe deviations from biuniqueness along the \ref{vpos:1.obj}-\ref{vpos:plural} span. Our main goal is to show how the inflectional paradigm is fit out in the verbal planar structure in Uma Piwra Quechua, not (necessarily) to provide a novel analysis of person/number inflection.

The intransitive verb paradigm for past/present and future tense is provided below in \tabref{tab:intransitive}.

\begin{table}
\caption{Intransitive suffixes in SBQ / Transitive suffixes with third person P arguments}
\label{tab:intransitive}
\begin{tabular}{
>{\columncolor[HTML]{FFFFFF}}l 
>{\columncolor[HTML]{FFFFFF}}l 
>{\columncolor[HTML]{FFFFFF}}l} \lsptoprule
    S(/A>P) & Past/Present & Future     \\ \midrule
    \First{}\Sg{}(\rightarrow3)               & \textit{-ni}          & \textit{-sa-q}      \\
    \Second{}\Sg{}(\rightarrow3)              & \textit{-nki}         & \textit{-nki}       \\
    \Third{}\Sg{}(\rightarrow3)               & \textit{-n}           & \textit{-n-qa}      \\
    \First{}\Pl{}.incl(\rightarrow3)          & \textit{-n-chik}      & \textit{-su-n-chik} \\
    \First{}\Pl{}.excl(\rightarrow3)          & \textit{-yku}         & \textit{-sa-yku}    \\
    \Second{}\Pl{}(\rightarrow3)              & \textit{-nki-chik}    & \textit{-nki-chik}  \\
    \Third{}\Pl{}(\rightarrow3)               & \textit{-n-ku}        & \textit{-n-qa-n-ku} \\
\lspbottomrule
\end{tabular}
\end{table}

For transitive clauses with a third person object, the verbal suffixes are the same as they are for the intransitive clauses provided in \tabref{tab:intransitive}. The paradigm shows that future marking in SBQ is marked by a few different formatives depending on the person; \textit{-sa} and \textit{-su} in position \ref{vpos:tame}, and \textit{-q} and \textit{-qa} of position \ref{vpos:3.obj} could be regarded as future markers in some sense.

The first person object marker is \textit{-wa}, which fits out position \ref{vpos:1.obj} of the verb complex. When the first person object is singular, there are no additional complications in the paradigm as illustrated in \tabref{tab:transitive:A1sg}.

\begin{table}[htp]
\caption{Transitive suffixes, first person singular P}
\label{tab:transitive:A1sg}
\begin{tabular}{lll} \lsptoprule
    {\color[HTML]{333333}}
    S(/A>P) & Past/Pres.   & Future                        \\ \midrule
    \First{}\Sg{}\rightarrow\First{}\Sg{}           & \textit{-ku-ni}       & \textit{-ku-saq}      \\
    \Second{}\Sg{}\rightarrow\First{}\Sg{}          & \textit{-wa-nki}      & \textit{-wa-nki}       \\
    \Third{}\Sg{}\rightarrow\First{}\Sg{}           & \textit{-wa-n}        & \textit{-wa-n-qa}      \\
    \First{}\Pl{}.incl\rightarrow\First{}\Sg{}      & \textit{-}            & \textit{-}             \\
    \First{}\Pl{}.excl\rightarrow\First{}\Sg{}      & \textit{-}            & \textit{-}             \\
    \Second{}\Pl{}\rightarrow\First{}\Sg{}          & \textit{-wa-nki-chik} & \textit{-wa-nki-chik}  \\
    \Third{}\Pl{}\rightarrow\First{}\Sg{}           & \textit{-wa-n-ku}     & \textit{-wa-n-qa-n-ku}  \\ 
\lspbottomrule
\end{tabular}
\end{table}

A few complications arise when we consider the first person plural objects. When the first person object is plural, the morpheme \textit{-nchik} occurs for the inclusive and \textit{-yku} appears for the exclusive. Note that \textit{-nchik}$\tilde{x}$\textit{-chik} modifies the first person inclusive and the second person \textit{subject} in paradigm \ref{tab:intransitive}. The morpheme \textit{-nku}$\tilde{x}$\textit{-yku} marks the first person exclusive and third person subject in \tabref{tab:intransitive}. Thus, the plural markers are not coded directly for specific grammatical relations but appear to obey some type of person hierarchy regarding what participant they modify \citep[]{cerronpalomino1987, lakamperwunderlich1998}.

\begin{table}[htp]
\caption{Transitive suffixes in SBQ with P as plural first person object}
\label{tab:transitivesuffixes:1objpl}
\begin{tabular}{@{}lllll@{}}
\lsptoprule
         & Past/Pres.             & Future                 & Past/Pres.  & Future      \\
         & \rightarrow\First{}\Pl{}.\textsc{incl} & \rightarrow\First{}\Pl{}.\textsc{incl} & \rightarrow\First{}\Pl{}.\textsc{excl}    & \rightarrow\First{}\Pl{}.\textsc{excl}    \\ \midrule
    \First{}\Sg{}          & \textit{-}              & \textit{-}                   & \textit{-}           & \textit{-}           \\
    \Second{}\Sg{}      & \textit{-}              & \textit{-}                   & \textit{-wa-yku}     & \textit{-wa-sa-yku} \\
    \Third{}\Sg{}       & \textit{-wa-n-chik}     & \textit{-wa-su-n-chik}       & \textit{-wa-yku}     & \textit{-wa-sa-yku} \\
    \First{}\Pl{}.\textsc{incl}  & \textit{-ku-n-chik}     & \textit{-ku-su-n-chik}       & \textit{-}           & \textit{-}           \\
    \First{}\Pl{}.\textsc{excl}  & \textit{-}              & \textit{-}                   & \textit{-ku-sa-yku}  & \textit{-ku-sa-yku} \\
    \Second{}\Pl{}      & \textit{-}              & \textit{-}                   & \textit{-wa-yku}     & \textit{-wa-sa-yku} \\
    \Third{}\Pl{}       & \textit{-wa-n-chik}     & \textit{-wa-su-n-chik}       & \textit{-wa-yku}     & \textit{-wa-sa-yku} \\
\lspbottomrule
\end{tabular}
\end{table}

We have seen throughout that the morphs \textit{-sa}, \textit{-su} and \textit{-q(a)} reoccur throughout the future paradigm. The suffix \textit{-sa} occurs in future forms when the subject is first person. The suffix \textit{-qa} occurs when there is third person subject in the future.  The suffix \textit{-su} surfaces in place of \textit{-sa} when a first person inclusive is either the subject or the object; \textit{-su} blocks \textit{-sa} in position \ref{vpos:tame}. The generalization works until we consider the paradigm where the second person is the object provided in \tabref{tab:transitive:A2}. On the surface it appears that \textit{-su} codes 3 \rightarrow 2, even when the verb is present.

\begin{table}[htp]
\caption{Transitive suffixes in SBQ}
\label{tab:transitive:A2}
\begin{tabular}{lllll}
\lsptoprule
    & Past/Pres.     & Future    & Past/Pres.        & Future            \\    
    & \rightarrow\Second{}\Sg{}        & \rightarrow\Second{}\Sg{}       & \rightarrow\Second{}\Pl{}  & \rightarrow\Second{}\Pl{} \\ \midrule
    \First{}\Sg{}       & \textit{-yki}     &   \textit{-sa-yki}    & \textit{-yki-chik}         & \textit{-sa-yki-chik}      \\
    \Second{}\Sg{}      & \textit{-ku-nki}  & \textit{(-ku-nki)}    & \textit{-ku-nki-chik}      & \textit{-ku-nki-chik}      \\
    \Third{}\Sg{}       & \textit{-su-nki}  & \textit{-su-nki}      & \textit{-su-nki-chik}      & \textit{-su-nki-chik}      \\
    \First{}\Pl{}.incl  & \textit{-}        & \textit{-}            & \textit{-}                 & \textit{-}                 \\
    \First{}\Pl{}.excl  & \textit{-yku}     & \textit{-sa-yku}      & \textit{-yku}              & \textit{-y-sa-ku}           \\
    \Second{}\Pl{}      & \textit{-}        & \textit{-}            & \textit{-ku-nki-chik}      & \textit{(-ku-nki-chik)}    \\
    \Third{}\Pl{}       & \textit{-su-n-ku} & \textit{-su-n-qa-nku} & \textit{-su-nki-chik}      & \textit{-su-n-qa-chik}      \\
\lspbottomrule
\end{tabular}
\end{table}

\largerpage
The rule for accounting for the distribution of \textit{-su} is as follows:

\ea 
    \ea \textit{-su} falls into position \ref{vpos:tame} if 3 \rightarrow 2;
    \ex \textit{-su} falls into position \ref{vpos:tame} if  2+1 (incl.) is a participant and the predicate is future.
    \z
\z 

The rule for accounting for the distribution of \textit{-sa} is as follows:

\ea 
    \ea  \textit{-sa} falls into position \ref{vpos:tame} if 1 or 1+3 (excl.) and the predicate is future;
    \ex The presence of \textit{-su} (in position \ref{vpos:tame}) blocks \textit{-sa} from occurring.
    \z 
\z 

The rule for accounting for the distribution of \textit{-qa}\sim\textit{-q} is as follows:

\ea 
    \ea \textit{-qa} fills position \ref{vpos:3.obj} if 3\textsubscript{i}  \rightarrow 3\textsubscript{j} and the predicate is future;
    \ex \textit{-q} fills positions \ref{vpos:SAsuf} and/or \ref{vpos:3.obj} if \First{}\Sg{} is subject and the predicate is future.
    \z
\z 

We can add that \textit{-q} in position \ref{vpos:3.obj} also occurs as part of an imperfective auxiliary verb construction in combination with \textit{-ka} in position \ref{vpos:aux} (see \sectref{sec:pitchaccentdomain} for some preliminary discussion).

The rules for accounting for the distribution of morphemes in positions \ref{vpos:SAsuf} and \ref{vpos:SAsuf2} are as follows:

\ea 
    \ea \textit{-ni} fills out position \ref{vpos:SAsuf} when 1 \rightarrow 3 and the predicate is non-future;
    \ex \textit{-nki} fills out position \ref{vpos:SAsuf} when 2 is subject;
    \ex \textit{-yki} fills out position \ref{vpos:SAsuf} when \First{}\Sg{} \rightarrow 2;
    \ex \textit{-n}...\textit{-n} fill out position \ref{vpos:SAsuf} and \ref{vpos:SAsuf2},
    respectively, elsewhere (when \ref{vpos:SAsuf2} is not filled out);
    \ex \textit{-n-n} is realized as \textit{-n}.
    \z 
\z

Note that \textit{-sa-q} could also be reanalyzed as a single suffix that occurs when the subject is first person singular and the predicate is future. Furthermore, there is ambiguity in terms of which position \textit{-q} is supposed to fit out in our analysis. It could fit out position \ref{vpos:SAsuf} or \ref{vpos:3.obj}. The person/number suffixes \textit{-ni} \textit{-nki}, \textit{-n}, and \textit{-yki} can also be broken down into smaller parts as long as we are willing to admit even more complexity into the realization rules \citep{myler2017}. Such analytic issues will not concern us in this chapter as they do not affect the application of constituency variables as far as we know.

\largerpage
We can now briefly consider elements outside of the traditional Quechua word. As it is well known, noun phrases can be variably ordered with the verb across varieties of Quechua. If no clitics occur between the verb and the NP, we assume that this NP fits out position \ref{vpos:XP4}. The NPs never interrupt the \ref{vpos:vcore}-\ref{vpos:plural} span.

\ea \label{ex:VP}{
	\glll {} apa -mu -nqa runtu -s -ta eh \\
        v: \ref{vpos:vcore} -\ref{vpos:advsuf} -\ref{vpos:SAsuf} \ref{vpos:XP4} - -  \\
		{} bring -\Dir{} -3\Sg{}.\Fut{} egg -\Pl{} -\Acc{} eh \\
	\glt `She will bring the eggs here ah (certainly).' \\ Sp. `Ella traerá los huevos aquí.'
	\hfill }
\z

Examples of A-V and and P-V order are provided in (\ref{ex:AV}) and (\ref{ex:PV}), respectively. If there are no clitics before the NP in this position, we assume it fills out position \ref{vpos:XP2}. 

\ea \label{ex:AV}{
	\glll {} papasu -yki -pis tarpu -kamu -n ari ¿i? \\
        v: \ref{vpos:XP2} - - \ref{vpos:vcore} -\ref{vpos:advsuf} -\ref{vpos:SAsuf} - - \\
		{} father -2\Sg{}.\Poss{} -also plant -own.choice -3\Sg{} eh right   \\
	\glt `Your father also goes and plants potatoes ah, right?' \\ Sp. `Tu padre también va y siembra ¿no ve?'
	\hfill } 
\z

\ea \label{ex:PV}{
	\glll {} papa -ta alla -kampu -n-ku \\
        v: \ref{vpos:XP2} - \ref{vpos:vcore} -\ref{vpos:advsuf} \ref{vpos:SAsuf}-\ref{vpos:plural} \\
		{} potato -\Acc{} harvest -go\&do -3\Pl{}  \\
	\glt `They went to harvest their potatoes (the direction involves away from the speaker and for safety reasons).' \\ Sp. `Fueron a cavar su papa (por seguridad).' 
	\hfill } 
\z

An NP is represented in position \ref{vpos:XP1} if it occurs before the clitics and before the verb.

\ea {
    \glll {} alqu manka -ta =chus lluqchi -yamu -n ima =chá \\
    v: \ref{vpos:XP1} - - \ref{vpos:clitics} \ref{vpos:vcore} -\ref{vpos:advsuf} -\ref{vpos:SAsuf} - - \\
    {} dog pot -\Acc{} =\Dub{} touch -go\&do -3\Sg{} what =\Dub{} \\
    \glt `Perhaps it went and touched the dog's pot or I don't know what.' \\ Sp. `No sé si fue a tocar la olla del perro o no sé que.'}
\z 

Another example is provided in (\ref{ex:APV}) with the fronted P-NP \textit{-pi} `what'. Notice that while the clitics occur in a fixed order with respect to each other, they do not necessarily all have to be adjacent, i.e. they do not have to `cluster'. The example in (\ref{ex:APV}) illustrates this point, with the clitics \textit{=taq} `and' and \textit{=ri} `interrogative', which occur in a fixed order in relation to each other, though they do not occur adjacently.

\ea \label{ex:APV}{
	\glll {} pi =taq sam -ita -ta -pis wayk'u -pu -n =ri Ay llakiy  \\
        v: \ref{vpos:XP1} =\ref{vpos:taq1} \ref{vpos:XP2} - - - \ref{vpos:vcore} -\ref{vpos:advsuf} -\ref{vpos:SAsuf} =\ref{vpos:ri2} - -  \\
	    {} who =\Conj{} food -\Dim{} -\Acc{} -also cook -\Ben{} -3\Sg{} =\Inter{} ay sad   \\
	\glt `Then who cooks him food? How sad!' \\ Sp. `Y quien se lo cocina comidita, ay que triste!'
	\hfill }
\z

SBQ displays a constraint on the distribution of XPs in the verb complex. Position \ref{vpos:XP3} \textit{cannot} be fit out if the auxiliary position \ref{vpos:aux} is filled. This fact somewhat complicates the interpretation of non-interruptability domains, as we will see in \sectref{sbq:sec:non-interruptability} below.


\section{Morphosyntactic domains} % (fold)
\label{sbq:sec:morphosyntacticdomains}

Four types of morphosyntactic constituency variables are applied to SBQ according to the classification in \citet{tallmancoincidence:2020}: (i) non-permutability; (ii) non-interruptability; (iii) ciscategorial selection; (iv) subspan repetition. 
We first consider a variable, \textsc{free occurrence}, which is usually taken to be a morphosyntactic test \citep{haspelmathword:2011}, but which is, in fact, indeterminate because it is interpreted with respect to a notion of \textit{boundedness} which is ambiguous between a morphosyntactic and a phonological interpretation. After this we move to the four more straightforwardly morphosyntactic tests.

\subsection{Free occurrence (\ref{vpos:vcore}-\ref{vpos:SAsuf}, \ref{vpos:vcore}-\ref{vpos:qa2})}
\label{sbq:sec:freeoccurrence}

% \subsubsection{Verb complex (\ref{vpos:vcore}-\ref{vpos:SAsuf}, \ref{vpos:vcore}-\ref{vpos:qa2})}
% \label{sbq:sec:freeoccurrence:verb}

In the verb complex, all elements are optional except for position \ref{vpos:1.obj}-\ref{vpos:plural} suffixes, which obligatorily mark the person/number of the subject. Position \ref{vpos:SAsuf} must be filled regardless of person. One cannot remove the suffix of this position, but all other elements can be dropped as illustrated in the examples in (\ref{ex:freeoccurrence1}) and (\ref{ex:freeoccurrence2}).

\ea \label{ex:freeoccurrence1}
	\glll {} (nuqa	=puni)	kuti -yu -chi -kampu -sha -rqa *(-ni) \\
        v: \ref{vpos:XP1} =\ref{vpos:puni1} \ref{vpos:vcore} -\ref{vpos:derivsuf1} -\ref{vpos:chi} -\ref{vpos:advsuf} -\ref{vpos:prog} -\ref{vpos:tame} (-\ref{vpos:SAsuf})  \\
		{} 1.\Sg{} =\Emph{} return -\Cmpl{} -\Caus{} -safely -\Prog{} -\Pst{}.\Rep{} -1.\Sg{} \\
	\glt `I was causing (the sheep) to safely return (back home).' \\ Sp. `Yo estaba haciendo que (la oveja) regresa a casa con seguridad.' \hfill 
\z

\ea \label{ex:freeoccurrence2}
	\glll {} (mana) (chilvi-situ-s-ta) (qayna) wisq'a -yu -rqa *(-ni) (=chu) \\
            v: \ref{vpos:XP1} \ref{vpos:XP1} \ref{vpos:clitics} \ref{vpos:vcore} -\ref{vpos:advsuf} -\ref{vpos:tame} -\ref{vpos:SAsuf} =\ref{vpos:clitics2}  \\
            {} (\Neg{}) (chick-\Dim{}-\Poss{}-\Acc{}) (yesterday) lock -\Cmpl{} -\Pst{}.\Rep{} *(-1.\Sg{}) (=\Neg{}) \\
	\glt `(Yesterday), I did (not) lock up (my little chicks)/them.' \\ Sp. `Ayer, no encerré mis pollitas.'   
\z

The \textsc{minimal free occurrence} domain is the smallest span that can be fit out by a single free form and is a complete utterance overlapping the verb core. For the verb complex this test identifies the \ref{vpos:vcore}-\ref{vpos:SAsuf} span. The \textsc{maximal free occurrence} domain identifies a span that covers the largest string that can be fit out by a single free form which can occur as complete utterance. In order to determine this domain, we should consider what elements outside of the traditional word are \textit{bound} in the sense that they cannot stand as a free utterance.

All of the morphemes from positions \ref{vpos:ña2} through \ref{vpos:ri2} are bound in the sense that they cannot be minimal free forms. Verb forms can appear without any overt NPs and the clitics up to position \ref{vpos:ri2}. The clitics cannot occur preverbally (in positions \ref{vpos:ña1} through \ref{vpos:qa}) and position \ref{vpos:XP2} would necessarily be fitted out by a free form if it occurred. 

The example in (\ref{ex:freeoccurrence3}) shows a cluster of clitics occurring after the verb complex. The verb form in (\ref{ex:freeoccurrence3}) is a single free form.

\ea \label{ex:freeoccurrence3}
	\glll {}  ... ranti -ni \textbf{=ña} \textbf{=chu} eh  \\
    v: ... \ref{vpos:vcore} -\ref{vpos:SAsuf} =\ref{vpos:ña2} =\ref{vpos:clitics2} -     \\
     {} ... buy -1\Sg{} \textbf{=already} \textbf{=\Neg{}} eh  \\
	\glt `(I used to buy lemon very often) now I no longer buy it.' \\ Sp. `(Acostumbraba comprar limón) Ahora, ya no compro pues.' \hfill 
\z

% =chu   Negative or interrogative 
% Here is an example where  =chu is negative. 
% limun-tá 		anchata=taq 	ranti-q		ka-ni 		á,  
% lemon-AC.TOP	a lot=CONJ	buy-?		be-1SG	ah
% kunán 		má 	ranti-ni=ña=chu 		á
% now.TOP	NEG	buy-1SG=already=NEGA	ah
% ‘I used to buy lemon very often, now I no longer buy it'
%Gladys: An example of just the verb with clitics from position 30 (=qa, =chu, =chus, =cha)

The domain extends to position \ref{vpos:qa2}. This is illustrated in the following reported speech construction in (\ref{ex:freeoccurrence4}). 

\ea \label{ex:freeoccurrence4}
	\glll {} (tipi -ysi -n sapa dia) ... ni -wa -rqa \textbf{=qa}   \\
    v: (\ref{vpos:vcore} -\ref{vpos:asssuf} -\ref{vpos:SAsuf} \ref{vpos:XP4} -) ... \ref{vpos:vcore} -\ref{vpos:1.obj} -\ref{vpos:SAsuf} =\ref{vpos:qa2} \\
    {} peal.corn -\Assist{} -3\Sg{} every day ... say -1\Sg{}:P -3\Sg{}:\Pst{} \textbf{=\Top{}}	\\
	\glt `“She (Teofila) helps  peel the corn's dried skin”, s/he told me.' \\ Sp. `Me dijo que cada día le ayuda a pelar maíz' 
\z

The \textsc{maximal free occurrence} domain therefore identifies a \ref{vpos:vcore}-\ref{vpos:qa2} span for the verb complex.

% =(qa) elided form and replaced with and accent 
% In  the  next example  =qa is elided in limuntá(=qa) “lemon-ACC=TOP” and  also in  kunán(=qa) ‘now=TOP'

% =chu   Negative or interrogative 
% Here is an example where  =chu is negative. 
% limun-tá 		anchata=taq 	ranti-q		ka-ni 		á,  
% lemon-AC.TOP	a lot=CONJ	buy-?		be-1SG	ah
% kunán 		má 	ranti-ni=ña=chu 		á
% now.TOP	NEG	buy-1SG=already=NEGA	ah
% ‘I used to buy lemon very often, now I no longer buy it'
% Here is an example where =chu is interrogative. 
% tuku-rpa-sunman			ratu=chu?
% finish-quickly-1PL.INCL.COND	fast=INT?
% Do you think we can finish quickly?  (peeling dry corn)
% =chus doubt 
% imayna=chus	sillp'anchu,	má 	nuqa 	yacha-ni=chu
% how=DUB	sillp'anchu,	NEG	1SG	know-1SG=NEG
% ‘I have no much idea what sillp'anchu looks like, I don't have 
% =chá possibility 
% kunán 		don 	Santiago-s	mikhun-ita-ta=chá	
% now.TOP	mr	Santiago-PL	foot-DIM-ACC=DUB
% misk'í		mikhu-yu-sha-n
% sweet		eat-PFT-PROG-3SG
% ‘Now the  Santiagos, must be enjoying a delicious meal'

% \subsubsection{Noun complex (\ref{npos:ncore}-\ref{npos:ncore}, \ref{npos:ncore}-\ref{npos:pis})}
% \label{sbq:sec:freeoccurrence:noun}

% Noun roots can be complete utterances. This is illustrated in \ref{ex:nounfree}, where a speaker responds to an interview question with a noun root.

% \ea \label{ex:nounfreeform}
%     \ea 
%     \gll Ima -ta chay -pi puqu -chi -nku \\
%     what -\Acc{} \Dem{} -\Loc{} produce -\Caus{} -3\Pl{} \\    
%     \glt `What do they produce there? Qué producen ahí?'
%     \ex 
%     \glll {} papa \\
%     n: \ref{npos:ncore} \\
%     {} potato \\
%     \glt `potatoes. papas'
%     \z
% \z

% Thus, the \textsc{minimal free occurrence domain} for the noun complex is the \ref{npos:ncore}-\ref{npos:ncore} span.

% Adjectives in SBQ occur before the noun. Adjectives are also free forms in SBQ as illustrated in \ref{ex:adjfreeform}. 

% \ea \label{ex:adjfreeform}
%     \ea 
%     \gll Mayqín maqa -sqa \\
%     which.one hit -3\Sg{}:\Pst{} \\  
%     \glt `Which (cat) hit it (kittie)?'
%     \ex 
%     \glll {} yana \\
%     n: \ref{npos:adj} \\
%     {} black \\
%     \glt `black.'
%     \z
% \z

%All other morphemes of the noun complex from position \ref{npos:dimin} to \ref{npos:pis} are bound. The \textsc{maximal free occurrence domain} identifies a \ref{npos:ncore}-\ref{npos:pis} span in the noun complex.


% (COMMENT) No encuentro un adjetivo con forma libre en el corpus, pero se que puede ser posible. Que hago?

%Gladys:Give a plural noun with a case marker N+s+case & N+jina+case
% nuqa 	oveja-s-ta		kunán		wata-mu-saq
% 1SG	sheep-PL-ACC	now.TOP	tie-GO&DO-1SG.FUT
% ‘I will now go and tie  the sheep'



%(COMMENT) There is no examples  with N+jina+case :( there are examples N+case+jina
% uqa-ta=jina		alla-rpa-nchik 			chay-ta=qa
% uqa-ACC=like	harvest-quickly-1PL.INCL	DEM-ACC=TOP
% ‘(peanuts) we harvest similar to uqa (andean tuber)'

% jusi-wan            ruthu-yku  cut.off-1PL.INCL    instrumento-INST

% ajinamanta
% like.this 

% jina q'usni-chi-mu-sha-nku, carga-sha-nku

% jina-ni
% make-1SG

% jina-n
% matar-3SG

% na-
% qu-
% jina-

\subsection{Non-interruptability (\ref{vpos:vcore}-\ref{vpos:prog}, \ref{vpos:vcore}-\ref{vpos:SAsuf}, \ref{vpos:vcore}-\ref{vpos:aux})}
\label{sbq:sec:non-interruptability}

\textsc{Non-interruptability domains} are spans of structure that cannot be interrupted by free forms, combinations of free forms, or elements that need to be represented in non-adjacent positions in the planar structure. We can identify a few non-interruptability domains in SBQ.

% \subsubsection{Verb complex (\ref{vpos:vcore}-\ref{vpos:prog}, \ref{vpos:vcore}-\ref{vpos:SAsuf}, \ref{vpos:vcore}-\ref{vpos:aux})}

The span \ref{vpos:vcore}-\ref{vpos:SAsuf} of the verb complex cannot be interrupted by a free form as in \textit{qayna} `yesterday' or \textit{mana} `negative' as in (\ref{ex:non-interruptableqayna}) and (\ref{ex:non-interruptablemana}), respectively.

\ea \label{ex:non-interruptableqayna}{
	\glll {} *chilvi -situ -s -ta  wisq'a -yu qayna -rqa -ni (=chu) \\
        v: \ref{vpos:XP2} - - - \ref{vpos:vcore} -\ref{vpos:derivsuf1} -\ref{vpos:tame} - -\ref{vpos:SAsuf} (=\ref{vpos:clitics2})  \\
        {} chick -\Dim{} -\Poss{} -\Acc{} lock -\Cmpl{} yesterday -\Pst{}.\Rep{} -1.\Sg{} (=\Neg{}) \\
	\glt `(Yesterday), I did (not) lock up (my little chicks)/them' \\ Sp. `Ayer, no encerré mis pollitas.'
	}
\z

\ea \label{ex:non-interruptablemana}{
	\glll {} *chilvi -situ -s -ta wisq'a -yu mana -rqa -ni (=chu) \\
            v: \ref{vpos:XP2} - - - \ref{vpos:vcore} -\ref{vpos:derivsuf1} - -\ref{vpos:tame} -\ref{vpos:SAsuf} =\ref{vpos:clitics2}  \\
            {} chick -\Dim{} -\Poss{} -\Acc{}  lock -\Cmpl{} \Neg{} -\Pst{}:\Rep{} -1.\Sg{} (=\Neg{}) \\
	\glt `(Yesterday), I did not lock up (my little chicks)/them.' \\ Sp. `Ayer, no encerré mis pollitas.'
    }
\z

Nor can the structure be interrupted by ``promiscuous'' (elements placed in more than one non-adjacent position) forms such as \textit{=chu} `negative' as illustrated in (\ref{ex:non-interruptablechu}).

\ea \label{ex:non-interruptablechu}{
	\glll {} *chilvi -situ -s -ta  wisq'a -yu =chu -rqa -ni \\
            v: \ref{vpos:XP2} - - - \ref{vpos:vcore} -\ref{vpos:derivsuf1} - -\ref{vpos:tame} -\ref{vpos:SAsuf}  \\
            {} chick -\Dim{} -\Poss{} -\Acc{} lock -\Cmpl{} =\Neg{} -\Pst{}:\Rep{} -1.\Sg{} \\
	\glt `(Yesterday), I did not lock up (my little chicks)/them' \\ Sp. `Ayer, no encerré mis pollitas.'
	}
\z

The \textsc{non-interruptability by free form domain} identifies a \ref{vpos:vcore}-\ref{vpos:SAsuf} span. There are no free forms that can interrupt this span. 

The non-interruptablity variable can be fractured such that one version refers to non-interruptability by free forms \citep{haspelmathword:2011} and another refers to bound but structurally promiscuous elements such as \textit{=lla} `limitative' \citep{liebercompounds:2017}. The \ref{vpos:vcore}-\ref{vpos:SAsuf} span can be interrupted by \textit{=lla} which can occur outside this domain as well. An example of \textit{-lla} `limitative' occurring in position \ref{vpos:lla1} before the person-number marker of position \ref{vpos:SAsuf} is provided in (\ref{ex:non-interruptablella1}).

\ea \label{ex:non-interruptablella1}{
	\glll {} wallpa -s -pis jaqay wayq'u -s -pi -pis wacha -mu -lla -n-ku  \\
            v: \ref{vpos:XP2} - - \ref{vpos:XP2} - - - - \ref{vpos:vcore} -\ref{vpos:advsuf} -\ref{vpos:lla1} -\ref{vpos:SAsuf}-\ref{vpos:plural} \\
            {} chicken -\Pl{} -also that river -\Pl{} -\Loc{} -also lay.egg -\Mot{} -\Limit{} -3-3.\Pl{}  \\
	\glt ‘The chickens as well, including those there go and lay eggs, no more or less.' \\ Sp. 'Las gallinas tambien, incluso allá en los rios van y ponen huevos no más.' \hfill 
	}
\z

The same morpheme can occur on the other side of the person-number concord suffixes as well in position \ref{vpos:lla2}. An example is provided in (\ref{ex:non-interruptablella2}). 

\ea \label{ex:non-interruptablella2}{
	\glll {} qhamá chay -man sat'i -yu -ku -n -lla =ña =taq  \\
            v: \ref{vpos:vcore} \ref{vpos:XP2} - \ref{vpos:vcore} -\ref{vpos:derivsuf1} -\ref{vpos:advsuf} -\ref{vpos:SAsuf} -\ref{vpos:lla2} =\ref{vpos:ña2} {} \\
%            n: - \ref{npos:ncore} \ref{npos:casesuf} -\ref{npos:casesuf} - - - - - - - \\
            {} look that -\All{} get.into -\Cmpl{} -\Refl{} -3\Sg{} =\Limit{} =already =and \\
	\glt `Look he went back to putting himself there (into the mud) no more or less.' \\ Sp. `Mirá y se volvió a meter ahí (ej. barro, un lugar etc).' \hfill 
	}
\z

The right edge of the \textsc{-lla non-interruptability domain} is \ref{vpos:when}. The left edge is \ref{vpos:vcore}. Evidence for this is provided in in (\ref{ex:llaV}).

\ea \label{ex:llaV}{
    \glll {} q'ipi -jina -lla t'aka -yarpa -ku -n \\
    v: \ref{vpos:XP2} - - \ref{vpos:vcore} -\ref{vpos:derivsuf1} -\ref{vpos:advsuf} -\ref{vpos:SAsuf} \\
%    n: \ref{npos:ncore} -\ref{npos:jina} -\ref{npos:lla3} \\
    {} lump -like -\Limit{} fall.down -suddenly -\Refl{} -3\Sg{} \\
    \glt `Suddenly, it spilt as if it was from a lump (from a car towards the ground)' \\ Sp. `De pronto el se derramó como si fuera un bulto (del auto hacia el suelo).' }
\z 
% \ea \label{ex:llaV}
%     \gllll {} qankuna ujin -ita -ta =taq mikhu -nki-chik kay -pi nuqayku allqu =jina \textbf{=lla} \textbf{mikhu} -ku -yku ari \\
%     v: \ref{vpos:XP1} \ref{vpos:XP1} - - \ref{vpos:taq1} \ref{vpos:vcore} -\ref{vpos:SAsuf}-\ref{vpos:plural} \ref{vpos:XP1} - \ref{vpos:XP1} \ref{vpos:XP2} - - \ref{vpos:vcore} -\ref{vpos:advsuf} -\ref{vpos:SAsuf} - \\
%     n: \ref{npos:ncore} \ref{npos:ncore} -\ref{npos:dimin} -\ref{npos:casesuf} - - - \ref{npos:ncore} -\ref{npos:casesuf} \ref{npos:ncore} \ref{npos:ncore} =\ref{npos:jina} =\ref{npos:lla3} - - - \\
%     {} 2\Pl{} different -\Dim{} -\Acc{} =\Conj{} eat -2-\Pl{} \Dem{} -\Loc{} 1:3\Pl{} dog =like \textbf{=\Limit{}} \textbf{eat} -\Refl{} -1+3:\Pl{} - \\
%     \glt `You eat different type of food, we here eat like dogs. COMMENT I think we need to delete this example UPQ speakers will not be happy with this.How about the following example I will think of another example. Also I think example 30 is not correct. I made a mistake it should be sat'i-yu-ku-lla-n=ña=taq COMMENT '
% \z

The \textsc{-lla non-interruptability domain}, thus, identifies a \ref{vpos:vcore}-\ref{vpos:prog} span. In SBQ, when an auxiliary fills out position \ref{vpos:aux}, no XP can occur in position \ref{vpos:XP3}. In auxiliary verb constructions, we can thus identify \textsc{a non-interruptability by complex of free forms domain} that identifies the span \ref{vpos:vcore}-\ref{vpos:aux}. 

%We might add a span that gets an NP

% \subsubsection{Noun complex (\ref{npos:ncore}-\ref{npos:pis}, \ref{npos:dem1}-\ref{npos:pis})}

% The noun complex cannot be interrupted by any free elements, with the exception of adjectives. The morpheme \textit{mana} `negative' is a free form and it cannot interrupt a noun complex for example, illustrated in \ref{ex:non-interruptablemanaNP}.

% \ea \label{ex:nonterruptabilitymanaNP}
%     \glll {} qhari (*mana) -s (*mana) -ta \\
%     n: \ref{npos:ncore} - \ref{npos:dimin} - -\ref{npos:casesuf}  \\
%     {} man (\Neg{}) -\Dim{} (\Neg{}) -\Acc{} \\
%     \glt Intended: `The men did not / Not the men ...'
% \z 

% There are two ways of interpreting non-interruptability in the noun complex - one considers non-interruptability by free forms and another considers non-interruptability by combinations of free forms. The \textsc{simplex free form non-interruptability domain} is \ref{npos:ncore}-\ref{npos:pis}. The \textsc{complex free form non-interruptability domain} identifies a \ref{npos:dem1}-\ref{npos:pis} span. 

% As, with the verb complex, the morpheme \textit{-lla} can interrupt spans of the noun complex. The morpheme \textit{-lla} `limitative' can occur in position \ref{npos:lla3} or position \ref{npos:lla4} (less frequently). The latter is illustrated in \ref{ex:-llainNP}.

% \ea \label{ex:-llainNP}
%     \glll {} solo kay wallpa -sni -lla -y má kusa=chu eh \\
%     n: - \ref{npos:dem1} \ref{npos:ncore} -\ref{npos:plursuf} -\ref{npos:lla4} -\ref{npos:posssuf} - - -  \\
%     only \Dem{} chicken-\Pl{}-\Limit{}-1\Sg{}:\Poss{} \Neg{} good=\Neg{} eh \\
%     \glt `These chickens of mine aren't good.' / Sp. `Estas mis gallinas no mas no son buenas pues.'
% \z 

% The \textsc{-lla non-interruptability domain} in the noun complex identified a \ref{npos:ncore}-\ref{npos:plursuf}. 

% \ea 
%     \glll {} \\
%     n: \\
%     \glt 
% \z 

% (2) chay phich-itu-sni-lla-n lluqsi-n á
% DEM phich-DIM-PL-LIM-3SG.POSS go.out-3SG á
% "Esos sus phichus (parte dura del pelo de la oveja) no más salen pues"


%Adam: It's unclear using this methodology whether =lla should be regarded as promiscuous in the NP.

%Gladys: Un ejemplo con =lla en el NP - y que muestra donde se occurre lla en el NP, es despues del jina=lla *lla-jina, lla-ta *ta=lla


%(COMMENT) jina occurs as a base form, it can stand for a verb root. also jinallata occurs as a free form as follows:
%mainly jina = alike
% (b) Example. 
% jina-lla-ta
% like.that-LIM-ACC
% without anything

%(COMMENT) =jina occurs singly with nouns. 
% =jina
% wawa=jina   suya-ku-nchik
% kid=like    wait-REF-1PL.INCL
% 'We wait as we were kids'
%(COMMENT)
% There are so mane examples of. ajina 'like that/that way' this AJINA can be followed by -ta ACC, negative =CHU,  Interrogative =CHU, accusative + certainly -ta=puni, dubitative =chá. I Will summary: 
% ajinata
% ajinachu
% ajinachu?
% ajinatapuni
% ajinachá

%Gladys: Investigar relacion entre el orden de sufijos de caso y lla, poner ejemplos

% (c)  =jina=ta=puni Example 
% chay=jina=ta=puni 		apura-ku-nki 
% DEM=like=ACC=certainly	hurry-REF-3PL

% campo=man, 	campo=man” 
% town=ALL, 	town=ALL

% ni-wa-n chay 		warmiwawa-y-pis,
% say-1OBJ<3SG	daughter-1SG-also	
% ‘It's  unbelievable  how  you hurry  up  to go back to  the town, to the  town' tells me my daughter. 

%We need to make the same argument about the noun complex as well. 

\subsection{Non-permutability (\ref{vpos:vcore}-\ref{vpos:vcore}, \ref{vpos:vcore}-\ref{vpos:recipsuf})}

\label{sbq:sec:non-permutability}

\textsc{Non-permutability domains} are spans of structure where elements cannot permute. In this section we defend the fixedness of order in the verb complex in more detail, as well as describing the domains of non-permutability in Quechua. 

% \subsubsection{Verb complex (\ref{vpos:vcore}-\ref{vpos:vcore}, \ref{vpos:vcore}-\ref{vpos:recipsuf})}

At least 11 clitics have been in identified in Uma Piwra SBQ. The ``clitics'' are listed in \tabref{tab:clitics} with their glosses and the positions that they can occur in. All of the positions for clitics are slots.\footnote{Note that \citet{myler2017} argues that object markers are clitics in (all?) Quechuan languages. Here we wish to emphasize (again) that the planar-fractal method assumes no distinction between affix and clitic. We only refer to `clitics' for expositional reasons, to make our discussion more easily readable to Quechuanists and because it is easy to discuss clitics as a class because they cluster together in a fixed templatic order.}

\begin{table}[htp]
\caption{Clitics in Uma Piwra South Bolivian Quechua}
\label{tab:clitics}
\begin{tabular}{lll}
\lsptoprule
Formative & Meaning           & Position \\ \midrule
\textit{=ña }      & already           & \ref{vpos:ña1}, \ref{vpos:ña2}   \\
\textit{=raq}     & conjunction       & \ref{vpos:raq1}, \ref{vpos:raq2}   \\
\textit{=puni}     & always, certainly & \ref{vpos:puni1}, \ref{vpos:puni2}   \\
\textit{=taq}     & coordinator & \ref{vpos:taq1}, \ref{vpos:taq2}   \\
\textit{=sina}     & apparently        & \ref{vpos:sina1}, \ref{vpos:sina2}   \\
\textit{=pis}      & also              & \ref{vpos:pis1}, \ref{vpos:pis2}   \\
\textit{=ri}       & doubt             & \ref{vpos:clitics}, \ref{vpos:clitics2}    \\
\textit{=chu}      & negative          & \ref{vpos:clitics}, \ref{vpos:clitics2}    \\
\textit{=chus}     & or                & \ref{vpos:clitics}, \ref{vpos:clitics2}    \\
\textit{=ri}       & although          & \ref{vpos:clitics}, \ref{vpos:clitics2}    \\
\textit{=rí}       & interrogative     & \ref{vpos:ri}, \ref{vpos:ri2}   \\
\textit{=qa}       & topic             & \ref{vpos:qa}, \ref{vpos:qa2}   \\
\lspbottomrule    
\end{tabular}
\end{table}

The clitic \textit{=ña} `already' occurs before \textit{=raq} `conjunction' as in (\ref{ex:ñaraq}). A minimally contrastive sentence where the order of these elements is reversed is ungrammatical. The clitic \textit{=raq} `conjunction' must precede the clitic \textit{=puni} as in (\ref{ex:raqpuni}), a minimally contrastive sentence with the order reversed is ungrammatical. 

\ea \label{ex:ñaraq}{
    \glll {} jap'i -n \textbf{=ña} \textbf{=puni} eh sesenta -yuq -mán jaqay -man jap'i -nku eh / ... *=puni=ña ...  \\
    v: \ref{vpos:vcore} -\ref{vpos:SAsuf} =\ref{vpos:ña2} =\ref{vpos:puni2} - - - - - - \ref{vpos:vcore} -\ref{vpos:SAsuf} - \\
    {} get.paid -3\Sg{} \textbf{=already} \textbf{=certainly} eh sixty-\Gen{} -\Abl{} \Dem{} -\All{} get.paid -3\Pl{} eh   \\
    \glt `For sure he received his payment/bonus, those who are 70 years and up receive the payment.' \\ Sp. `Claro que recibe su pago (bono), de los 70 años en adelante reciben el pago.' }
\z

\ea \label{ex:raqpuni}{
    \glll {} mana kuti -mu -nqa \textbf{=raq} \textbf{=puni} eh / (*=puni=raq)  \\
    v: \ref{vpos:XP2} \ref{vpos:vcore} -\ref{vpos:advsuf} -\ref{vpos:SAsuf} =\ref{vpos:raq2} =\ref{vpos:puni2} - \\
    {} \Neg{} return -\Dir{} -3\Sg{}:\Fut{} \textbf{still} \textbf{=certainly} - \\
    \glt `No, clearly he will still come back.' \\  Sp. No, claro que aún va a regresar.' }
\z 


% (3)    mana, kutimunqaraqpuni á
% mana,     kuti-mu-nqa=raq=puni             á
% NEG,     return-DIR-3SG.FUT=CONJ=certainly    ah
% ‘no, claro que aún va a regresar'

% (4)    *mana, kutimunqaraqpuni á
% mana,     *kuti-mu-nqa=puni=raq             á
% NEG,     return-DIR-3SG.FUT=certainly=CONJ    ah

The clitic \textit{=puni} `certainly' always occurs before \textit{=taq} `conjunct' as in \REF{ex:punitaq}. The reverse order (\textit{*=taq=puni}) is ungrammatical.

\ea \label{ex:punitaq}{
    \glll {} ch'aki -pu -sha -n \textbf{=puni} \textbf{=taq} kay jina qanqa -pi =qa í?   \\
    v: \ref{vpos:vcore} -\ref{vpos:advsuf} -\ref{vpos:prog} -\ref{vpos:SAsuf} =\ref{vpos:puni2} =\ref{vpos:taq2} \ref{vpos:XP3} - \ref{vpos:XP3} - \ref{vpos:qa2} -  \\
    {} dry -\Cmpl{} -\Prog{} -3 \textbf{=certainly} \textbf{still} \Dem{} like heat -\Loc{} =\Top{} right \\
    \glt `And clearly it is drying in so much heat.' \\  Sp. `Y claro que se está secando pues en tanto calor.' }
\z 

The clitic \textit{=taq} `coordinator` always occurs before \textit{=sina} `dubitative' as in (\ref{ex:taqsina}). The reverse order is ungrammatical.

\ea \label{ex:taqsina}{
    \glll {} para -lla -nqa \textbf{=taq} \textbf{=sina} manchay t'irkura -ri -mu -sha -n ni -sha -rqa =qa (*=sina=taq) \\
    v: \ref{vpos:vcore} -\ref{vpos:lla1} -\ref{vpos:SAsuf} =\ref{vpos:taq2} =\ref{vpos:sina2} \ref{vpos:XP2} \ref{vpos:vcore} -\ref{vpos:ri.suffix} -\ref{vpos:advsuf} -\ref{vpos:prog} -\ref{vpos:SAsuf} \ref{vpos:vcore} -\ref{vpos:prog} -\ref{vpos:SAsuf} =\ref{vpos:qa2}  \\
    {} rain -\Limit{} -3\Sg{}:\Fut{} \textbf{=\Conj{}} \textbf{=\Dub{}} extremely sprout.rain -\Aff{} -\Dir{} -\Prog{} -3\Sg{} say -\Prog{} -3\Sg{}:\Pst{} =\Top{}   \\
    \glt `I think it is going to rain again, it is said that the rain is being made.' \\ Sp. `Creo que va a llover otra vez, dice que se estaba armando la lluvia.' }
\z 

The clitic \textit{=pis} `also' always occurs before \textit{=chu} `negative' as in (\ref{ex:pischu}). The reverse order is ungrammatical.

\ea \label{ex:pischu}{
    \glll {} ni uyari -sha -lla -n =pis =chu puñu -rpa -n =ña
    (*=chu=pis) \\
    v: \ref{vpos:XP3} \ref{vpos:vcore} -\ref{vpos:prog} -\ref{vpos:lla1} -\ref{vpos:SAsuf} =\ref{vpos:pis2} =\ref{vpos:clitics2} \ref{vpos:vcore} -\ref{vpos:asssuf} -\ref{vpos:SAsuf} =\ref{vpos:ña2} \\
   {} \Neg{} listen -\Prog{} -\Limit{} -3\Sg{} =also =\Neg{} sleep -suddenly -3\Sg{} =already  \\
    \glt `She is not even listening.' \\ Sp. `Ni siquiera está escuchando, (ya se quedó dormida.)' }
\z 

The topicalizer \textit{=qa} always appears after all of the other clitics as in (\ref{ex:ñaqa}) and (\ref{ex:punisinaqa}).

\ea \label{ex:ñaqa}{
    \glll {} ay qunqa -pu -sqa -ni \textbf{=ña} \textbf{=qa} \\
    v: - \ref{vpos:vcore} -\ref{vpos:advsuf} -\ref{vpos:tame} -\ref{vpos:SAsuf} =\ref{vpos:ña2} =\ref{vpos:qa2} \\
    {} - forget -\Cmpl{} -\Pst{} -1\Sg{} \textbf{=already} \textbf{=\Top{}} \\
    \glt `Ay, I have already forgotten.' \\ Sp. `Ay, ya me había olvidado.' } 
\z 

\ea \label{ex:punisinaqa}{
    \glll {} ri -sha -yku pasiu -man jamu -ri -sha -lla -n \textbf{=puni} \textbf{=sina} \textbf{=qa}  \\
    v: \ref{vpos:vcore} -\ref{vpos:prog} -\ref{vpos:SAsuf} \ref{vpos:vcore} - \ref{vpos:vcore} -\ref{vpos:ri.suffix} -\ref{vpos:prog} -\ref{vpos:lla1} -\ref{vpos:SAsuf} =\ref{vpos:puni2} =\ref{vpos:sina2} =\ref{vpos:qa2}  \\
    {} go -\Prog{} -1:2\Pl{} walk.around -\All{} come -\Aff{} -\Prog{} -\Limit{} -3\Sg{} \textbf{=certainly} \textbf{=\Dub{}} \textbf{=\Top{}}   \\
    \glt  `We are passing through. I think he is still coming, right?' \\ Sp. `Estamos yendo de paseo. Creo que sigue viniendo ¿no es cierto?' }  \\
\z 

The clitics cannot fill out positions \ref{vpos:ña1} through  \ref{vpos:qa2} unless there is an element in position \ref{vpos:XP1}. The suffix \textit{-ri} `inceptive' always occurs before \textit{-na} `reciprocal' as in (\ref{ex:rina}). The reverse order is ungrammatical as in (\ref{ex:nari}); a clitic cannot occur in first position before an NP.

\ea \label{ex:rina}{
    \glll {} aysa -ri -na -ku -sun eh nuqa -lla -nchik =taq \\
    v: \ref{vpos:vcore} -\ref{vpos:ri.suffix} -\ref{vpos:recipsuf} -\ref{vpos:advsuf} -\ref{vpos:SAsuf} - \ref{vpos:XP3} - - =\ref{vpos:taq2} \\
    {} pull -\Incept{} -\Recp{} -\Refl{} 1:2\Pl{} - hand -\Limit{} -1:2\Pl{} =\Conj{} \\
    \glt `We ourselves will take each other's hands one by one.' \\ Sp. `Nosotras mismas nos tomaremos de la mano uno al otro pues.' }
\z 

\ea \label{ex:nari}{
    \glll {} *aysa -na -ri -ku -sun eh nuqa -lla -nchik =taq \\
    v: \ref{vpos:vcore} -\ref{vpos:recipsuf} -\ref{vpos:ri.suffix} -\ref{vpos:advsuf} -\ref{vpos:SAsuf} - \ref{vpos:XP3} - - \ref{vpos:taq2} \\
%    n: - - - - - - - \ref{npos:ncore} -\ref{npos:lla3} -\ref{npos:posssuf} - \\
    {} pull -\Recp{} -\Incept{} -\Refl{} 1:2\Pl{} - hand -\Limit{} -1:2\Pl{} =\Conj{} \\
    \glt Intended: `We ourselves will take each other's hands one by one.' \\ Sp. `Nosotras mismas nos tomaremos de la mano uno al otro pues.' }
\z 

Position \ref{vpos:asssuf} is a zone. This means that the morphemes that can fit out this domain can be variably ordered. The morphemes \textit{-ysi} `assistive' and \textit{-rpa} `quickly' can be variably ordered for instance. Both of the orders are attested in the corpus as in (\ref{ex:ysirpa2}) and (\ref{ex:rpaysi}). The variable ordering is illustrated in (\ref{ex:ysirpa2}) and (\ref{ex:rpaysi}) below.

\ea \label{ex:ysirpa2}{
    \glll {} chay caraju -s q'ala uña pili -situ -s -ni -y -ta uqu \textbf{-ysi} \textbf{-rpa} -nku eh \\
    v: \ref{vpos:XP1} - - - - \ref{vpos:XP2} - - - - - \ref{vpos:vcore} -\ref{vpos:asssuf} -\ref{vpos:asssuf} -\ref{vpos:SAsuf} - \\
    {} \Dem{} idiot -\Pl{} entirely little duck -\Dim{} -\Pl{} -E -1\Sg{}\Poss{} -\Acc{} devour \textbf{-\Assist{}} \textbf{-quickly} -3\Pl{} - \\
    \glt `Those carajus (birds) help finishing my little ducks' food up …' \\ Sp. 'Esos (pájaros) le ayudan a terminar de comer la comida de mis patitos …' }
\z 

\ea \label{ex:rpaysi}{
    \glll {} jajaja kuntan t'iqpa \textbf{-rpa} \textbf{-ysi} -lla -sqa -yki ni -spa \\
    v: - \ref{vpos:XP2} \ref{vpos:vcore} -\ref{vpos:asssuf} -\ref{vpos:asssuf} -\ref{vpos:lla1} -\ref{vpos:tame} -\ref{vpos:SAsuf} \ref{vpos:vcore} -\ref{vpos:tame} \\
    {} - soon peel \textbf{-quickly} \textbf{-\Assist{}} -\Limit{} -? -2\Sg{}<1\Sg{} say -\Gerund{}   \\
    \glt `Hahaha,  I will quickly  help you peeling (the corn) soon.' \\ Sp. `Jajaja, enseguida voy a ayudarte a pelar el maíz.' }
\z 

In position \ref{vpos:asssuf}, \textit{-ysi} `assistive' and \textit{-rqu} `nimbly' can variably order as is illustrated by comparing (\ref{ex:ysirqu}) and (\ref{ex:rquysi}). The variable ordering between these morphemes is not very prominent in the corpus, but it is certainly possible.

\ea \label{ex:ysirqu}{
    \glll {} libri -ta jamu -rqu -nku q'alata mikhu \textbf{-ysi} \textbf{-rqu} -nku ni -n   \\
    v: \ref{vpos:XP2} - \ref{vpos:vcore} -\ref{vpos:asssuf} -\ref{vpos:SAsuf} \ref{vpos:XP2} \ref{vpos:vcore} -\ref{vpos:asssuf} -\ref{vpos:asssuf} -\ref{vpos:SAsuf} \ref{vpos:vcore} -\ref{vpos:SAsuf} \\
    {} extremely -\Acc{} come -nimbly -3\Pl{} completely eat \textbf{-\Assist{}} \textbf{-nimbly} -3\Pl{} say -3\Sg{} \\
    \glt `Nimbly they come, they help to finish eating their food completely.' \\ Sp. `Agilmente vienen, dice que le ayudan a comer su comida completamente.'
    }
\z 

\largerpage
\ea \label{ex:rquysi}{
    \glll {} jaqay chimpa -pi ujchhika sar -ita -ta ruthu \textbf{-rqu} \textbf{-ysi} -mu -wa -rqa \\
    v: \ref{vpos:XP1} - - - \ref{vpos:XP2} - - \ref{vpos:vcore} -\ref{vpos:asssuf} -\ref{vpos:asssuf} -\ref{vpos:advsuf} -\ref{vpos:1.obj} -\ref{vpos:SAsuf} \\
    {} \Dem{} front -\Loc{} a.little corn -\Dim{} -\Acc{} cut \textbf{-nimbly} \textbf{-\Assist{}} -go\&do -1:P -3\Sg{}:\Pst{} \\
    \glt `There in front, he helped me cut the maize a little.' \\ Sp. `Allá, al frente, me ayudó a cortar un poquito de maíz.'
    }
\z

There is no evidence that the variable ordering of morphemes in position \ref{vpos:asssuf} corresponds to a difference in scope. Both the possible orderings are ambiguous with respect to the scope. The following restrictions are also present. The assistive suffix \textit{-ysi} cannot co-occur with the affective \textit{-ri} ‘affective' from the following slot after the zone where \textit{-ysi} `assistive' occurs. Only \textit{-rqu} ‘nimbly' and \textit{-rpa} ‘suddenly' can occur with \textit{-ri}. The suffixes \textit{-rqu} `nimbly' and \textit{-rpa} `suddenly' from the zone in the verb template can occur with \textit{-ri} from the next slot. Example (\ref{ex:rquri}) shows -\textit{rqu} preceding \textit{-ri} `affective' and (\ref{ex:rpari}) shows \textit{-rpa} `nimbly' preceding \textit{-ri} `affective'. The suffix \textit{-ri} `affective' cannot precede any of the suffixes from the zone. The reverse order is ungrammatical.

\ea \label{ex:rquri}{
    \glll {} chay -ta tumpá qunqa \textbf{-rqu} \textbf{-ri} -ni / ... *-ri-rqu ... \\
    v: \ref{vpos:XP1} - \ref{vpos:XP1} \ref{vpos:vcore} -\ref{vpos:asssuf} -\ref{vpos:ri.suffix} -\ref{vpos:SAsuf}   \\
    {} \Dem{} -\Acc{} a.little forget -nimbly -\Aff{} -1\Sg{} / ... -\Aff{}-nimbly ... \\
    \glt `I forgot that story a little.' \\ Sp. `Esa (historia) me olvidé un poco.' }
\z 

\ea \label{ex:rpari}{
    \glll {} kay paloma pasa -sha -nman i ajná pasa \textbf{-rpa} \textbf{-ri} -n / ... *-rpa-ri ... \\
    v: \ref{vpos:XP2} - \ref{vpos:vcore} -\ref{vpos:prog} -\ref{vpos:SAsuf} - \ref{vpos:XP2} \ref{vpos:vcore} -\ref{vpos:asssuf} -\ref{vpos:ri.suffix} -\ref{vpos:SAsuf}  \\
    {} \Dem{} pigeon pass -\Prog{} -3\Sg{}:\Cond{} - like:\Acc{} pass \textbf{-suddenly} \textbf{-\Aff{}} -3\Sg{} / ... -\Aff{}-suddenly ... \\
    \glt `Just like the pigeon that would be passing, passing flying.' \\ Sp. `Así como la paloma que estaría pasando (volando), asi mismo pasan (volando).'}
\z 

The affective \textit{-ri} precedes the causative \textit{-chi} as in (\ref{ex:richi2}) but the causative \textit{-chi} cannot precede the affective \textit{-ri}.

\ea \label{ex:richi2}{
    \glll {} ajin -ita -n -ta llami \textbf{-ri} \textbf{-chi} -wa -rqa / ... *-chi-ri ... \\
    v: \ref{vpos:XP2} - - - \ref{vpos:vcore} -\ref{vpos:ri.suffix} -\ref{vpos:chi} -\ref{vpos:1.obj} -\ref{vpos:SAsuf} / ... \ref{vpos:chi}-\ref{vpos:ri.suffix} ... \\
    {} \Dem{} -\Dim{} -3\Sg{}:\Poss{} -\Acc{} try \textbf{-\Aff{}} \textbf{-\Caus{}} -1:P -3\Sg{}:\Pst{} / ...  \\
    \glt `She had me try it (a chunk of huminta), small like this.' \\ 
    Sp. `Me hizo probar (un trozo de huminta) asi pequeño.' }
\z 

The causative \textit{-chi} precedes the reflexive \textit{-ku}, the associated motion simplex and complex suffixes \textit{-mu}, \textit{-kampu}, \textit{-kamu}, and also the benefactive suffix \textit{-pu} and the completive \textit{-kapu}, \textit{-mpu}. Reversing the order between \textit{-chi} and any of these morphemes is not accepted. The ordering relations are illustrated in (\ref{ex:chiku}) through (\ref{ex:chikamu}) below. 

\ea \label{ex:chiku}{
   \glll{} tukuy ima -wan atipa \textbf{-chi} \textbf{-ku} -n jamp'atu -wan waqta atipa -chi -ku -lla -n =taq / ... *-ku-chi ... \\
   v: \ref{vpos:XP2} - - \ref{vpos:vcore} -\ref{vpos:chi} -\ref{vpos:advsuf} \ref{vpos:XP2} - - \ref{vpos:vcore} -\ref{vpos:chi} -\ref{vpos:advsuf} -\ref{vpos:lla1} -\ref{vpos:SAsuf} =\ref{vpos:taq2} \\
   {} all what -\Com{} beat \textbf{-\Caus{}} \textbf{-\Refl{}} -3\Sg{} frog -\Com{} another beat -\Caus{} -\Refl{} -\Limit{} -3\Sg{} =\Conj{} / ... -\Refl{}-\Caus{} ... \\
   \glt `It makes everyone gain/win with everything, with the toad as well, he makes him win/gain.' \\ Sp. `Se hace ganar con todos, con el sapo también se hace ganar (el zorro).' }
\z 

\ea \label{ex:chimu}{
    \glll {} pay ri -nqa eh punchu ruwa \textbf{-chi} \textbf{-mu} -nqa eh /  ... *-mu-chi ... \\
    v: \ref{vpos:XP2} \ref{vpos:vcore} -\ref{vpos:SAsuf} - \ref{vpos:XP2} \ref{vpos:vcore} -\ref{vpos:chi} -\ref{vpos:advsuf} -\ref{vpos:SAsuf} -  \\
    {} 3\Sg{} go -3\Sg{}:\Fut{} eh poncho knit \textbf{-\Caus{}} \textbf{-go\&do} -3\Sg{}:\Fut{} eh / ... -\Mot{}-\Caus{} ... \\
    \glt `She is going to go, she will go and make a poncho...' \\ Sp. `Ella va a ir pues, ella irá a hacer tener poncho...' }
\z

\ea \label{ex:chikampu}{
    \glll {} nuqa =puni kuti -yu -ri \textbf{-chi} \textbf{-kampu} -sha -rqa -ni / ... *-chi-kampu ... \\
    v: \ref{vpos:XP1} =\ref{vpos:puni1} \ref{vpos:vcore} -\ref{vpos:derivsuf1} \ref{vpos:ri.suffix} -\ref{vpos:chi} -\ref{vpos:advsuf} -\ref{vpos:prog} -\ref{vpos:tame} -\ref{vpos:SAsuf} \\
    {} 1\Sg{} =certainly come -\Cmpl{} -\Aff{} \textbf{-\Caus{}} \textbf{-\Dir{}} -\Prog{} -\Pst{} -1\Sg{} / ... -\Caus{}-\Mot{} ... \\
    \glt `I always decided to bring my sheep on the way back.' \\ Sp. `Yo siempre decidí traer de regreso (mis ovejas).' }
\z

\ea \label{ex:chikamu}{
    \glll {} Zapatú -y sira \textbf{-chi} \textbf{-kamu} -saq / ... *-kamu-chi ... \\
    v: \ref{vpos:XP2} - \ref{vpos:vcore} -\ref{vpos:chi} -\ref{vpos:advsuf} -\ref{vpos:SAsuf}  \\
    {} shoe -1\Sg{}:\Poss{} sew \textbf{-\Caus{}} \textbf{-go\&do} -1\Sg{}:\Fut{} / ... -go\&do-\Caus{} ...  \\
    \glt `I will go and repair my shoe.' \\ Sp. `Iré y haré costurar mi zapato.' }
\z 

The suffix \textit{-naya} `about to' occurs in position \ref{vpos:naya}. Out of all the suffixes that fit out slot \ref{vpos:advsuf}, only \textit{-kamu} `go and do' can be followed by \textit{-naya} `proximal future' as we can see in example (\ref{ex:kamunaya}). The other suffixes from position \ref{vpos:advsuf} cannot combine with \textit{-naya}: *\textit{-ku-naya}, *\textit{-kampu-naya}, *\textit{-kapu-naya}, *\textit{-mpu-naya} and so on. 

\ea \label{ex:kamunaya}{
   \glll {} t'aka -ri -kamu -naya -sha -lla -n =ña =taq \\
   v: \ref{vpos:vcore} -\ref{vpos:ri.suffix} -\ref{vpos:advsuf} -\ref{vpos:naya} -\ref{vpos:prog} -\ref{vpos:lla1} -\ref{vpos:SAsuf} =\ref{vpos:ña2} =\ref{vpos:taq2} \\
   {} spill -\Incept{} -\Dir{} -about.to -\Prog{} -again -3\Sg{} =already =\Conj{} \\
   \glt `You see, they (the rain drops) are starting to fall (lit. spill) again.' \\  Sp. `Ves? (Las gotas de lluvia) está empezando a caer de nuevo.'}
\z 

The morpheme \textit{-sha} `progressive' always precedes \textit{-wa} `first person singular' as in (\ref{ex:shawa}). There can be no morpheme intervening between these two and the reverse order is ungrammatical.

\ea \label{ex:shawa}{
    \glll{} kay -pi graba \textbf{-sha} \textbf{-wa}-n eh / ... *-wa-sha ... \\
    v: \ref{vpos:XP2} - \ref{vpos:vcore} -\ref{vpos:prog} -\ref{vpos:1.obj} - \\
    {} \Dem{} -\Loc{} record \textbf{-\Prog{}} \textbf{-1\Sg{}:P} -3\Sg{} - / ... -1\Sg{}:P-\Prog{} ...  \\
    \glt `She is recording me here.' \\ Sp. `Me está grabando aquí pues.' }
\z

 The morpheme \textit{-lla} `limitative' will always follow the morpheme \textit{-qti} `when' and the reverse order is not grammatical. This is also illustrated with (\ref{ex:shaqtilla}).

\ea \label{ex:shaqtilla}{
    \glll{} t'ika \textbf{-sha} \textbf{-qti} -n jampi -na eh t'ika \textbf{-sha} \textbf{-qti} \textbf{-lla} -n =raq / ... *-qti-sha ... / ... *-lla-qti ... \\
    v: \ref{vpos:vcore} -\ref{vpos:prog} -\ref{vpos:when} -\ref{vpos:SAsuf} \ref{vpos:vcore} -\ref{vpos:tame} \ref{vpos:vcore} -\ref{vpos:prog} -\ref{vpos:tame} -\ref{vpos:lla1} -\ref{vpos:SAsuf} =\ref{vpos:raq2} \\
    {} bloom -\Prog{} -when -3\Sg{} fumigate -\Oblig{} eh bloom -\Prog{} -when -\Limit{} -3\Sg{} still / ... -when-\Prog{} / ... -\Limit{}-when ... \\
    \glt `One has to fumigate when they are blooming, when it just starts to bloom.' \\ Sp. `Hay que fumigar cuando está floreciendo, cuando a penas está floreciendo.' }
\z 

The second person markers \textit{-su}\sim\textit{-sa} can variably order with the past tense marker as in (\ref{ex:surqa}) and (\ref{ex:rqasu}). The rest of the suffixes occur in a fixed order.

\ea \label{ex:surqa}{
    \glll {} pí ni \textbf{-su} \textbf{-rqa} \\
    v: \ref{vpos:XP1} \ref{vpos:vcore} -\ref{vpos:tame} \\
    {} who say \textbf{-2:P} \textbf{-\Pst{}} \\
    \glt `Who told you?' \\ Sp. `quien te ha dicho?' }
\z 

\ea \label{ex:rqasu}{
    \glll {} qayna tarde ri -sha -lla -n =taq  ni -sha -rqa -su -nki =qa \\
    v: \ref{vpos:XP2} - \ref{vpos:vcore} -\ref{vpos:prog} -\ref{vpos:lla1} -\ref{vpos:SAsuf} =\ref{vpos:taq2} \ref{vpos:vcore} -\ref{vpos:prog} -\ref{vpos:tame} -\ref{vpos:tame} -\ref{vpos:SAsuf} -\ref{vpos:qa2}  \\
    {} yesterday afternoon go -\Prog{} -\Limit{} -3 =\Conj{} say -\Prog{} \textbf{-\Pst{}} \textbf{-2:P} -2 \\
    \glt `He told you yesterday in the evening that he is going again.' \\ Sp. `Te ha dicho que ayer por la tarde nuevamente estabas yendo.' }
\z 

\textsc{Non-permutability domains} can be fractured into two types depending on whether the lexical \textit{-chi} `causative' of position \ref{vpos:derivsuf1} and the \textit{-chi} of position \ref{vpos:chi} are treated as a single morpheme or not. We could regard these morphs as instances of the same morpheme on the grounds that they have the same form and function (see \sectref{sbq:sec:planarstructures}). On the other hand, they could be treated as distinct morphemes on the grounds that only position \ref{vpos:chi} can be filled out consistently. The \textsc{non-permu\-ta\-bil\-i\-ty domain (\textit{-chi} = \textit{-chi})} identifies a span overlapping the verb core where elements cannot be variably ordered on the interpretation that there is a single \textit{-chi} morpheme. This domain identifies a \ref{vpos:vcore}-\ref{vpos:vcore} span. The \textsc{non-permutability domain (\textit{-chi} $\neq$ \textit{-chi})} overlapping the verb core according to the view that we should understand lexical \textit{-chi} and productive \textit{-chi} as distinct elements. This domain identifies a \ref{vpos:vcore}-\ref{vpos:recipsuf} span. After position \ref{vpos:recipsuf}, morphemes can variably order in position \ref{vpos:asssuf}.

\subsection{Ciscategorial selection (\ref{vpos:vcore}-\ref{vpos:prog}, \ref{vpos:vcore}-\ref{vpos:tame}, \ref{vpos:ña1}-\ref{vpos:aux})}
\largerpage

A \textsc{ciscategorial selection domain} is a span of structure that refers to elements which are selectively restrictive such that they can only combine with a single part of speech category. 

Ciscategorial selection can be fractured into at least two types. The \textsc{minimal ciscategorial domain} identifies a span overlapping the verb core containing positions whose elements can only combine with the verb. This domain covers the \ref{vpos:vcore}-\ref{vpos:tame} span. It covers the verbal complex until we arrive at position \ref{vpos:SAsuf}. After this position there are a number of number/person agreement markers which also appear in the nominal paradigm.

The \textsc{maximal ciscategorial selection domain} has to be fractured according to whether we consider clitics to be verb ciscategorial or not (the strict interpretation). A lax ciscategorial selection domain assumes that a morpheme is ciscategorial if it does not combine with other part of speech classes semantically. A strict ciscategorial selection domain assumes that an element is ciscategorial \textit{if} it only combines with overt verbs.

Clitics are different from the morphemes that fit out positions \ref{vpos:vcore} to \ref{vpos:plural} of the verb complex. They do not require an overt verb to surface in an utterance – they can combine with any construction where a predicate is involved. In SBQ, predicates can be expressed without an overt verb. Clitics can occur in such non-verbal predicate constructions as in (\ref{ex:NVPpuni})–(\ref{ex:NVPsinaqa2}). 

\ea \label{ex:NVPpuni}
    \gll Cochabamba -pí chay -lla \textbf{=puni} \\
    Cochabamba -\Loc{}:\Top{} \Dem{} -\Limit{} \textbf{=always} \\
    \glt `In Cochabamba things are always the same.' \\ Sp. `En Cochabamba siempre es lo mismo.'
\z

\ea \label{ex:NVPsinaqa1}
    \gll Abran -paq -pi \textbf{=sina} \textbf{=qa} í \\
    Abrah -\Gen{} -\Loc{} \textbf{=\Dub{}} \textbf{=\Top{}} right.\Inter{} \\
    \glt `I think it was by the house of Abran.' \\ Sp. `Creo que donde Abran (la casa de Abran), cierto?'  
\z 

\ea \label{ex:NVPpunicha}
    \gll chay ch'ul -itu -n \textbf{=puni} \textbf{=chá} mana na -n \\
    \Dem{} hat -\Dim{} -3\Sg{}:\Poss{} \textbf{=certainly} \textbf{=\Dub{}} \Neg{} \Dem{} -3\Sg{}.\Poss{} \\
    \glt `This should always be his hat, not his other thing (lit. not his this).' \\ Sp. `Eso debe ser su gorrita siempre, no su este.'
\z 

\ea \label{ex:NVPsinaqa2}
    \gll k'ullu -lla \textbf{=sina} \textbf{=qa} \\
    trunk -\Limit{} \textbf{=DUB} \textbf{=TOP} \\ 
    \glt `I think it's just a trunk.' \\ Sp. `Creo que solo es un tronco.' \\ 
\z 

The \textsc{maximal strict ciscategorial domain} identifies the largest span overlapping the verb core that contains ciscategorial verbal elements. This domain identifies a \ref{vpos:vcore}-\ref{vpos:SAsuf} span. This domain excludes clitics because they can appear without a verb as in (\ref{ex:Npuni}).

\ea \label{ex:Npuni}
    \gll kay -lla -pi =puni \\
    \Dem{} -\Limit{} -\Loc{} =certainly \\
    \glt `Clearly, he is just there.' \\ Sp. `Obvio, siempre está aquí.'
\z 

The domain also does not include person-number agreement markers because some of these also occur as possessives on the verb; these are not ciscategorial selective. The lax ciscategorial domain assumes that for an element to be verbal ciscategorial, it need only not combine with nouns or adjectives. A verb ciscategorial element can combine appear without a verb as long as it modifies the predicate. All clitics are verbal ciscategorial according to the lax definition.

The \textsc{maximal lax ciscategorial domain} identifies a \ref{vpos:ña1}-\ref{vpos:aux} span. This domain would include all of the clitics and the auxiliary as well. 


\subsection{Subspan repetition (\ref{vpos:XP1}-\ref{vpos:plural}, \ref{vpos:vcore}-\ref{vpos:plural}, \ref{vpos:XP1}-\ref{vpos:XP4}, \ref{vpos:vcore}-\ref{vpos:chi})}

Subspan repetition variables are constituency tests based on constructions that involve some type of structural repetition or recursion (reduplication, coordination, subordination, serialization etc.). The structural properties of recursive structures in Uma Piwra SBQ are still not completely understood. Below we review some preliminary findings.

SBQ has two strategies for combining clauses. One strategy involves juxtaposition of clauses. Another strategy involves combining a matrix clause with a nominalized clause. There are a few different types of nominalizers, however, as far as we have been able to discern they all select for the same span of structure in the verb complex. Future research might discern that each clausal nominalization should be treated as a separate clause-linkage strategy. Subspan repetition tests are fractured into at least types. The minimal domain refers to the span of structure wherein none of the elements can display widescope. The maximal domain refers to the span of structure where positions \textit{can} be filled out in each repeated span (see Introduction to this Volume (Section 7) for a formalization).

\subsubsection{Clause (asyndetic) juxtaposition}
\largerpage
For the clause juxtaposition strategy the maximal domain simply identifies the entire verb complex (\ref{vpos:XP1}-\ref{vpos:XP4}). The minimal domain is more challenging to determine. NPs and PPs can always be `shared' across juxtaposed clauses, and thus the minimal domain would exclude positions \ref{vpos:XP1}, \ref{vpos:XP2}, \ref{vpos:XP3}, and \ref{vpos:XP4}. Our preliminary evidence suggests that clitics do not scope over multiple juxtaposed clauses. For instance, \textit{=chu} adds interrogative or negative semantics. In the examples  below it displays local scope modifying only the clause it appears in (see \citealt{bickel:capturing} for discussion from a typological perspective).

\ea \label{ex:cliticnoscope1}
    \glll{} apa -mu -wa -rqa   qan =chu apa -chi -mu -wa -rqa -nki \\
    v: \ref{vpos:vcore} -\ref{vpos:advsuf} -\ref{vpos:1.obj} -\ref{vpos:tame} \ref{vpos:XP1} =\ref{vpos:clitics} \ref{vpos:vcore} -\ref{vpos:chi} -\ref{vpos:derivsuf1} -\ref{vpos:1.obj} -\ref{vpos:tame} -\ref{vpos:SAsuf} \\
    {} take -go\&do -1.P -\Pst{} 2\Sg{} =\Inter{} take -\Caus{} -go\&do -1.P -\Pst{} -2\Sg{}>1\Sg{}  \\
    \glt `Yes, she brought it (the chicha drink) did you sent it to me?.' \\ Sp. `Si, ha traido (la chicha), tu me has enviado?' 
\z 

\ea \label{ex:cliticnoscope2}{
	\glll {} limun-tá anchata =taq ranti -q ka -ni eh kunán má ... ranti -ni \textbf{=ña} \textbf{=chu} eh  \\
    v: \ref{vpos:XP1}- \ref{vpos:taq1} \ref{vpos:vcore} -\ref{vpos:3.obj} \ref{vpos:vcore} -\ref{vpos:SAsuf} -\ref{vpos:SAsuf} \ref{vpos:XP2} \ref{vpos:XP2} \ref{vpos:vcore} -\ref{vpos:SAsuf} =\ref{vpos:ña2} =\ref{vpos:clitics2}      \\
    {} lemon-\Acc{}:\Top{} a.lot =\Conj{} buy-\Nmlz{} be-1\Sg{} eh now \Neg{} ... buy -1\Sg{} =still =\Neg{} eh  \\
	\glt `I used to buy lemon very often, now I no longer buy it (lemon).' (*`I never bought lemon very often, now I do not either).' \hfill } 
\z

We conjecture preliminarily that clitics do not display multiclausal scope over juxtaposed or linked clauses. However, we have not yet systematically tested the claim with all clitics. Examples from naturalistic speech are hard to interpret without speaker commentary (e.g. Does \textit{=puni} `certainly' scope over one clause or two?). 

None of the suffixes in the verb complex scope over juxtaposed clauses. For instance, if two juxtaposed clauses are future, future marking appears twice as in \textit{-sa} in (\ref{ex:muscope1}). Notice also that the subject marker \textit{-q} is repeated in both clauses. 

\ea \label{ex:muscope1}
   \glll {} jap'i -sa -q kunán mikhu -mu -sa -q ni -sqa   \\
   v: \ref{vpos:vcore} -\ref{vpos:tame} -\ref{vpos:SAsuf} \ref{vpos:XP3} \ref{vpos:vcore} -\ref{vpos:tame} -\ref{vpos:SAsuf} -\ref{vpos:vcore} -\ref{vpos:tame} \\
   {} hunt \Fut{} -1\Sg{} now:\Top{} eat -go\&do -\Fut{} -1\Sg{} say -\Gerund{}  \\
   \glt `I will go and hunt and then I will go and eat.' \\  Sp. ‘Le voy a cazar y le voy a comer, dijo (el zorro).'
\z 

Another illustrative example is provided in (\ref{ex:muscope2}). The causative suffix \textit{-chi} must appear in both clauses as well and cannot scope over both.

\ea \label{ex:muscope2}
    \glll {} wasi -ta ruwa -chi \textbf{-kamu} -saq pirqa -chi -saq \\
    v: \ref{vpos:XP2} - \ref{vpos:vcore} -\ref{vpos:chi} -\ref{vpos:advsuf} -\ref{vpos:SAsuf} \ref{vpos:vcore} -\ref{vpos:chi} -\ref{vpos:SAsuf} \\
    {} house -\Acc{} build -\Caus{} \textbf{-go\&do} -1\Sg{}:\Fut{} build.wall -\Caus{} -1\Sg{}:\Fut{}  \\
    \glt `I will go and I will build a house and I will go (somewhere else) to build a wall.' \\ Sp. `Ire y haré construir una casa, iré y haré elevar la pared.' \\ 
    `I went to build a house and build a wall.' \\ Sp. ‘Iré y hare construir una casa, hare elevar la pared'
\z 

The only suffixes that display multiclausal scope in Quechua appear to be the associated motion markers (\textit{-mu}, \textit{-kamu}, \textit{-kampu}). Wide scope over two clauses can be seen with the suffix \textit{-mu} `go and do' in (\ref{ex:muscope1}) and with the suffix \textit{-kamu} `go and do'. These facts require us to fracture the minimal domain into a stricter and laxer interpretation. On the strict interpretation, the minimal domain spans from \ref{vpos:vcore}-\ref{vpos:chi}. On the lax interpretation, the minimal domain spans from \ref{vpos:vcore}-\ref{vpos:plural}.

\subsubsection{Clause combination with clausal nominalization}

Thus far, five nominalizers have been documented: \textit{-na} `obligatative', \textit{-q} `habitual', \textit{-sqa} `preterite', \textit{-spa} `gerund', and \textit{-ytawan} `prior event'. Clausal nominalizations can have all the positions of the verb complex filled out except the clitics, to our knowledge.

The main structural difference between main clauses and nominalized clauses with respect to the planar-structure is that nominalized clauses cannot take clitics, are obligatorily verb final and do not take (most) of the inflectional suffixes from positions \ref{vpos:1.obj}-\ref{vpos:plural}. The maximal domain of the nominalized clause is therefore \ref{vpos:XP1}-\ref{vpos:lla1}. The lax minimal domain is \ref{vpos:vcore}-\ref{vpos:lla1}.

Apart from the inflectional suffixes and the associated  motion suffixes, none of the other suffixes are shared across conjuncts in a nominalized clause. For illustration consider (\ref{ex:ytawan1}). In the sentence below, the inceptive \textit{-ri} can only modify the main verb \textit{kuti} `turn over'. The inceptive meaning does not carry over to the verb \textit{qu} `give' (fig. `kill').

\ea \label{ex:ytawan1}
    \glll{} ujta kuti \textbf{-ri} \textbf{-ytawan} qu -ni caraju pharaq pharaq ni -rpa -chi -ni eh \\ 
    v: \ref{vpos:XP2} \ref{vpos:vcore} -\ref{vpos:ri.suffix} - \ref{vpos:vcore} -\ref{vpos:SAsuf} - - - \ref{vpos:vcore} -\ref{vpos:asssuf} -\ref{vpos:chi} -\ref{vpos:SAsuf} - \\
    {} suddenly turn.over \textbf{-\Incept{}} \textbf{-\Nmlz{}:\Prior{}} give -1\Sg{}:\Pst{} dammit pharaq pharaq say -suddenly -\Caus{} -1\Sg{}:\Pst{}  \\
    \glt `As I turned quickly I gave it to him (killed him), and at this moment he made him say ``pharaq pharaq".' \\ Sp. `De pronto me di la vuelta y le di (arrojé con piedra a la perdiz, maté), y le hice decir “pharaq, pharaq” (revolcar en el piso aleteando antes de morir).'
\z

The causative only has scope over the verb complex it combinations with as illustrated in (\ref{ex:spa1}).

\ea \label{ex:spa1}
    \glll{} jak'u -chi -kamu -spa na -yku wakin -tá vende -ku -yku \\
    v: \ref{vpos:vcore} -\ref{vpos:chi} -\ref{vpos:advsuf} -? -\ref{vpos:SAsuf} \ref{vpos:XP2} - \ref{vpos:vcore} -\ref{vpos:advsuf} -\ref{vpos:SAsuf} \\
    {} grind -\Caus{} -go\&do -\Gerund{} ? -3\Pl{} the.rest -\Acc{}:\Top{} sell -\Refl{} -3\Pl{} \\
    \glt `Making them crush the rest, we sold it.' (*we made them sell it) \\ Sp. `Lo demás (el resto del trigo), haciendo moler na-mos, nos vendemos.'
\z 
    
The morpheme \textit{-rpa} `quickly' does not display wide scope over clauses and must be repeated as well as in (\ref{ex:rparpa}).

\ea \label{ex:rparpa}
    \glll {} má ñapis carnaval jina pasa \textbf{-rpa} \textbf{-lla} -n ñapis chamu -n pasa \textbf{-rpa} \textbf{-lla }-n =ña =taq eh \\  
    v: \ref{vpos:XP2} \ref{vpos:XP2} \ref{vpos:XP2} \ref{vpos:XP2} \ref{vpos:vcore} -\ref{vpos:asssuf} -\ref{vpos:lla1} -\ref{vpos:SAsuf} \ref{vpos:XP2} \ref{vpos:vcore} -\ref{vpos:asssuf} -\ref{vpos:lla1} \ref{vpos:SAsuf} =\ref{vpos:ña2} \ref{vpos:taq2} - \\
    {} \Neg{} suddenly carnival like go.away -suddenly -\Limit{} -3\Sg{} suddenly arrive -3\Sg{} go.away -suddenly -\Limit{} 3\Sg{} -again \\
    \glt `No, quickly as with the carnival it goes rapidly, quickly we arrived and went.' \\ Sp. `No, de pronto asi como el carnaval se va rápidamente no más, quickly it arrived and quickly it went away.'
\z 

The only morphemes that consistently scope over nominalized clauses are position \ref{vpos:SAsuf} morphemes. Examples above and (\ref{ex:nchika}) below illustrate this.
  
\ea \label{ex:nchika}
    \glll{} khana -spa suysu -spa ruwa -nchika eh \\
    v: \ref{vpos:vcore} - \ref{vpos:vcore} -? \ref{vpos:vcore} -\ref{vpos:SAsuf} - \\
    {} burn -\Gerund{} sift -\Gerund{} make -1:2\Pl{} eh \\ 
    \glt  `Burning (sit'ikira) exists, we made legia from that.'  \\ Sp. `Quemando (sit'ikira), cerniendo hacemos (legia) de eso.' 
\z 

Just as with the juxtaposed clauses, the suffix \textit{-mu} can display wide scope in nominalized clause combinations as in (\ref{ex:muscope3}).

\ea \label{ex:muscope3}
    \glll{} sapa viernes Anzaldo -pi chicharron -ta mikhu -ri (\textbf{-mu}) -ytawan patan -pi aqhá toma -ri \textbf{-mu} -ni eh \\
    v: \ref{vpos:XP2} - \ref{vpos:XP2} - \ref{vpos:XP2} - \ref{vpos:vcore} -\ref{vpos:ri.suffix} -\ref{vpos:advsuf} - \ref{vpos:XP2} - \ref{vpos:XP2} \ref{vpos:vcore} -\ref{vpos:ri.suffix} -\ref{vpos:advsuf} -\ref{vpos:SAsuf} \\
    {} every friday Anzaldo -\Loc{} Chicharron -\Acc{} eat -\Aff{} (\textbf{-go\&do}) -\Acc{} on.top -\Loc{} chicha:\Acc{} drink -\Aff{} \textbf{-go\&do} -1\Sg{}  \\
    \glt `Every Friday in Anzaldo, I go eat Chicharron then I drink chicha on top of it.'
\z

We tentatively conjecture that the strict minimal domain for such clauses could therefore be \ref{vpos:vcore}-\ref{vpos:chi}. This result could change in light of more research on the scopal properties of more of the suffixes in the SBQ verb complex, which are not always easy to discern from naturalistic speech alone. The results of this section are rather speculative at this point because the possibility that different types of clause linkage constructions display different scope/gapping/conjunction facts with respect to all the suffixes of the verb complex has not been explored systematically. 

\section{Phonological domains} % (fold)
\label{sbq:sec:phonologicaldomains}

We have been able to identify and describe three phonological domains in SBQ: (i) the pitch accent domain; (ii) a suffix deletion domain; (iii) a vowel lowering domain. SBQ is typically described with a vowel lowering rule whereby /u/ becomes /a/ word internally at some morpheme boundaries \citep[]{muysken:1981}. As discussed in \sectref{sbq:sec:planarstructures}, we do not think that this analysis is correct, at least for the data available to us. Secondly, as far as we have been able to discern, SBQ's intonational phrasing consists of either L\% or H\% marked on the last syllable of an utterance. However, we have not yet investigated the topic in detail, and thus our proposal for intonational phrasing remains tentative.

\largerpage[2]
\subsection{Stress/pitch accent (\ref{vpos:vcore}-\ref{vpos:plural}, \ref{vpos:vcore}-\ref{vpos:aux})}
\label{sec:pitchaccentdomain}

In this section, we provide a brief description of pitch accent in SBQ. Describing SBQ pitch patters requires no reference to lexical tone categories. A bitonal L+H* pitch accent appears to always map to the stressed syllable (see \citealt{pierrehumbertbeckman:1986, gussenhovenbruce:1999} for terminology/notation). Intonational phrases are marked by a final L\% or H\%, and usually by an initial \%H. Intonation level high tones are marked by ``\uparrow'' and intonation level low tones are marked by ``\downarrow''. An accute accent ``´'' marks the L+H* bitonal unit in the phonetic transcription. Providing a description of SBQ intonation must be left for future research at this point. It is not yet clear whether stress is realized by any acoustic correlates apart from the f0 of the pitch accent. We use the ``´'' accent to mark the presence of the pitch accent.

What we and much of the literature refer to as ``clitics" fall within the pitch accent domain of SBQ, which is illustrated in (\ref{ex:stress1}) (see \citealt{bills1971introdution, herrerosanchez:gramaticaquechua:1978, adelaar1977:tarmaquechua, cerronpalominoquechua:1994} for similar descriptions). The stressed syllables occur on the penultimate syllable in a \ref{vpos:vcore}-\ref{vpos:clitics2} span in the verb complex. This is illustrated in (\ref{ex:stress1}). In the following example, there is also a final L\% utterance level tone. In the examples below, `stress' is marked with an accute accent.

\ea \label{ex:stress1}{
	[má.na.dʒi.khu.rin.pu.ní.tʃu \downarrow] \\
	\glll {} mana rikhuri -n =puní =chu \\
            v: \ref{vpos:XP2} \ref{vpos:vcore} -\ref{vpos:SAsuf} =\ref{vpos:puni2} =\ref{vpos:clitics2}  \\
            {} \Neg{} appear -\Third{} =certainly =\Neg{} \\
	\glt `It certainly does not appear.' \\  Sp. `Ciertamente no aparece.'
	}
\z

The pitch accent does not fall on the penultimate position when there is a clitic which bears stress as in \textit{=chá} in the example in (\ref{ex:stress2}).


\begin{figure}
    \centering
    \includegraphics[height=.45\textheight]{figures/manarikuniripunichu.png}
    \caption{Pitch track of the sentence \textit{mana rikhurinpunichu}}
    \label{fig:manarikhurinpunichu}
\end{figure}


\ea \label{ex:stress2}{
	\glll {} mikhu -n -ku =chá imana -n -kú =chus \\
            v: \ref{vpos:vcore} -\ref{vpos:SAsuf} -\ref{vpos:plural} =\ref{vpos:clitics2} \ref{vpos:vcore} -\ref{vpos:SAsuf} -\ref{vpos:plural} =\ref{vpos:clitics2} \\
            {} eat -3 -\Pl{} =\Dub{} not.know.what -3 -\Pl{} =\Dub{}  \\
	\glt `I don't know what they did, perhaps they eat it.' \\ Sp. `No estoy segura si comen o no. No sé que hacen.' 
	}
\z

Copula constructions in Quechua are formed with the morpheme \textit{ka-} `be'. Copulas receive stress like any other verb.

\ea{}
    $[$se.ɲóɾ .má. ʎuq.si.mun.qá.tʃu. pé.ɾo. mí.sa. kán.qa. nín \downarrow$]$ \\
    \glll {} señor má lluqsi -mu -n -qa =chu pero misa \textbf{ka} \textbf{-n} \textbf{-qa} ni -n \\
    v: \ref{vpos:XP1} - \ref{vpos:vcore} -\ref{vpos:advsuf} -\ref{vpos:SAsuf} -\ref{vpos:3.obj} =\ref{vpos:clitics2} - \ref{vpos:XP1} \ref{vpos:vcore} -\ref{vpos:SAsuf} -\ref{vpos:3.obj} \ref{vpos:vcore} -\ref{vpos:SAsuf}     \\
    {} saint \Neg{} go.out -\Dir{} -3 -\Fut{}:3 =\Neg{} but mass \textbf{be/\Aux{}} -3 -\Fut{}:3 say -3  \\
    \glt `It is said that the saint won't leave (from the church), but there will be a festival.' \\ Sp. `Dice que el Santo no va a salir (de la iglesia) pero va a haber fiesta.'
\z 

Auxiliary verb constructions combine a main verb with \textit{ka-} or \textit{tiya-}. Preliminary research suggests that the auxiliary verb falls within the stress domain of the verb. In other words, the auxiliary verb behaves like the sentence level ``clitics'' in SBQ with respect to pitch accent assignment. An illustrative example is provided in (\ref{ex:auxiliary1}) with the verb complex \textit{purimuq kayku} `going'. The auxiliary \textit{-q ... ka-}  appears to encode an imperfective aspect, judging by the Spanish translation, but the precise meaning of the auxiliary construction requires more research still.\footnote{One can also observe that the positioning of the stress-attracted pitch accent is potentially complicated by the presence of sentence-level particles such as \textit{á} `yes, surely' as in \textit{riqpuni kani a} `Well, obviously I went'. Here the pitch accent moves back onto the main verb \textit{ri-q-puni}, perhaps conditioned by the presence of a pitch accent on the particle \textit{á}. Again, the precise interaction between intonational phrasing and pitch accent assignment requires more research in SBQ.}


\ea \label{ex:auxiliary1}
    [\uparrow ɾiʁ.pú.ni ga.ni eh \uparrow tʃaɾaŋguitu guitaʁúp nuʁaj puɾimuq káik \downarrow] \\
    \glll{} ri -q =puni ka-ni eh charangu -itu guitarra -pi nuqaykú puri -mu -q ka-yku    \\
    v: \ref{vpos:vcore} -\ref{vpos:3.obj} =\ref{vpos:puni2} \ref{vpos:aux} - \ref{vpos:XP2} - \ref{vpos:XP2} - \ref{vpos:XP2} \ref{vpos:vcore} -\ref{vpos:advsuf} -\ref{vpos:3.obj} \ref{vpos:aux}        \\
    {} go -\Ipfv{} =certainly \Aux{}-1\Sg{} yes charango -\Dim{} guitar -\Loc{} 1\Pl{}:\Excl{}:\Top{} walk -go\&do -\Ipfv{} \Aux{}-1\Pl{}:\Excl{} \\
    \glt `Well I obviously went, right, and we went singing guitar.' \\ Sp. `Claro que iba pues, charanguito, nosotros caminábamos (cantando) en guitarra.'  
\z 

\begin{figure}
    \centering
    \includegraphics[height=.45\textheight]{figures/purimaqkayku.png}
    \caption{Pitch track of the sentence \textit{ri-q=puni ka-ni eh charangu-itu guitarra-pi nuqaykú puri-mu-q ka-yku} provided in (\ref{ex:auxiliary1})}
    \label{fig:purimaqkayku}
\end{figure}


In the verb complex, the pitch accent domain is \ref{vpos:vcore} to \ref{vpos:plural} when position \ref{vpos:XP3} is filled out with a noun phrase or adverb. If position \ref{vpos:XP3} is not filled out, then the domain extends further right to position \ref{vpos:aux}. The \textsc{minimal pitch accent domain} is thus \ref{vpos:vcore}-\ref{vpos:plural} and the \textsc{maximal pitch accent domain} is \ref{vpos:vcore}-\ref{vpos:aux}.

\subsection{Final syllable/suffix deletion (\ref{vpos:XP1}-\ref{vpos:plural})}

\label{sec:suffixdeletion}

SBQ has a suffix deletion rule. The rule applies to \textit{-ta} `accusative', \textit{-spa} `gerundive', \textit{mana} `negative', and \textit{-qa} `topicalizer'. The rule applies after stress assignment such that the underlying presence of \textit{-ta} or \textit{=qa} can be observed from the position of stress one syllable to the left of where one would expect it even where all the segmental material is deleted. An illustrative example where \textit{-ta} `accusative' is deleted is provided in (\ref{ex:tadeletion1}) and (\ref{ex:tadeletion2}). The final stress occurs on \textit{papa} `potato', because the accusative suffix \textit{-ta} deletes after stress is assigned. 

\ea \label{ex:tadeletion1}
    [nuʁa \textbf{pa.ˈpa} wajk'ujuni hamunʁa kunan nispa] \\
    \glll{} nuqa papa\sout{-ta} wayk'u -yu -ni jamu -n -qa kunan ni -spa \\ 
    v: \ref{vpos:XP2} \ref{vpos:XP2} \ref{vpos:vcore} -\ref{vpos:advsuf} -\ref{vpos:SAsuf} \ref{vpos:vcore} -\ref{vpos:SAsuf} -\ref{vpos:3.obj}  \ref{vpos:XP3} \ref{vpos:vcore} - \\
    {} 1\Sg{} father:\Acc{} cook -\Cmpl{} -1 come -3 -\Fut{}:3 now say -\Gerund{} \\
    \glt `I cooked potatoes, thinking ``she's coming today".' \\ Sp. `Yo cociné papa, pensando ``va a venir hoy".'
\z 

\ea \label{ex:tadeletion2}
    [ʎan.ˈt'á a.pa.na kaχ bu.ʐu.pi\downarrow] \\
    \glll {} llant'á \sout{-ta} apa -na kaq burru -pi \\
    v: \ref{vpos:XP2} \ref{vpos:vcore} -\ref{vpos:tame} \ref{vpos:aux} - \ref{vpos:XP4} -   \\
    {} firewood:\Acc{} take -\Imp{} \Aux{}-\Hab{} donkey -\Loc{} \\
    \glt `One had to take the lumber on the donkey.' \\ Sp. `Se tenía que llevar llena en burro.'
\z 

The gerund \textit{-spa} is reduced to \textit{-s} in some contexts. When this occurs the stress also occurs in the vowel of the last syllable as in (\ref{ex:spa}).

\ea \label{ex:spa}
    [kaj kaldeɾaman tʃu.ˈɾás tʃajatʃísun aɾi \downarrow] \\
    \glll {} kay kaldera-man churá -s\sout{pa} chaya -chi -sun ari \\
    v: \ref{vpos:XP2} - \ref{vpos:vcore} ? cook -\ref{vpos:chi} -\ref{vpos:SAsuf} - \\
    {} \Dem{} boiler-to put -\Gerund{} cook -\Caus{} -\Fut{}:1\Pl{} eh \\
    \glt  `We will cook, after placing the boiler (somewhere).'	\\ Sp. `Pues haremos cocer luego de haber colocado a la caldera.'
\z 

In the following example \textit{mana} deletes its final syllable in (\ref{ex:mana}).

\largerpage
\ea \label{ex:mana}
    [six.tim.bri ki.ʎa.pi tʃaj.man.ˈta ma.na ˈma wa.ʁan.tʃu tʃaj.man.ta \downarrow] \\ 
    \glll {} sijtimbri killa -pi Chaymantá mana ... \textbf{má} waqa -n =chu 		chaymanta  \\
    v: \ref{vpos:XP1} \ref{vpos:XP1} - \ref{vpos:XP1}:\ref{vpos:qa} \ref{vpos:XP2} ... \ref{vpos:XP2} \ref{vpos:vcore} -\ref{vpos:SAsuf} =\ref{vpos:clitics} \ref{vpos:XP4} \\
    {} September month -\Loc{} then:\Top{} \Neg{} ... \textbf{\Neg{}} cry -3\Sg{} =\Neg{} then \\
    \glt `In the month of   September, not after. After that (the fox) doesn't cry.' \\ Sp. `En el mes de septiembre. Después no. Después ya no llora (el zorro).'
\z 

Note that in the example above a post-verbal and utterance final \textit{mana} does not delete its final syllable in (\ref{ex:mana}). The deletion rule cannot apply when a the suffix occurs on a verb or noun which is utterance-final. This is illustrated with the examples below (\ref{vpos:spa.nodelete}) and (\ref{ex:qa.nodelete}).

\ea \label{vpos:spa.nodelete}
    [qha.wa.mu.ni.pú.ni rís.pa a \uparrow] \\
    \glll {} qhawa -mu -ni =puni ri \textbf{-spa} eh  (*...-s...) \\
    v: \ref{vpos:vcore} -\ref{vpos:advsuf} -\ref{vpos:SAsuf} =\ref{vpos:puni2} \ref{vpos:XP3} \\
    {} watch -go\&do -1\Sg{} =certainly go -\Gerund{} - - \\
    \glt `Clearly I am going to look when I go.' \\ Sp. `Claro que voy a mirar cuando voy pues.' 
\z 

\ea \label{ex:qa.nodelete}
    [tú.ta en.té.ɾo t'om.pu.tʃín.tʃik tʃáj.ta q'i.ta.tá.qa áɾi \downarrow] \\
    \glll {} tuta entero t'impu -chi -n -chik chay-ta q'ita-ta \textbf{=qa} ari  \\
    v: \ref{vpos:XP2} - \ref{vpos:vcore} -\ref{vpos:chi} -\ref{vpos:SAsuf} -\ref{vpos:plural} \ref{vpos:XP3} \ref{vpos:XP3} =\ref{vpos:qa2} \\
    {} night all boil -\Caus{} -3 -1\Pl:\Incl{} \Dem{}-\Acc{} arrope-\Acc{} =\Top{} eh  \\
    \glt   `We boil all night, the maiz arrope.' \\ Sp. `Hacemos hervir toda la noche, el arrope de maíz ah.'
\z 

The span of structure identified by the \textsc{suffix deletion domain} is \ref{vpos:XP1}-\ref{vpos:plural}. This is because the rule applies before verbs and to elements in positions \ref{vpos:XP1} and \ref{vpos:XP2}.  


\subsection{Vowel lowering (\ref{vpos:vcore}-\ref{vpos:derivsuf1}, \ref{vpos:vcore}-\ref{vpos:plural})}
\label{sec:highvowelassimilation}

In SBQ the phonemes /u/ and /i/ have allophonic variants [u]\sim[o] and [i]\sim[e], respectively. The low variants [o] and [e] occur adjacent to /q/.

\ea \label{ex:oe} 
    \ea \textit{q'uñi} [q'o.ɲi] `hot'
    \ex \textit{uquy} [o.ʁoj] `to devour'
    \ex \textit{t'iqi} [t'e.ʁe] `full'
    \ex \textit{uqa} [o.ʁa] `kind of potato'
\z 
\z 

The high variants [u] and [i] occur elsewhere. Some illustrative examples are provided in (\ref{ex:ui}).

\ea \label{ex:ui}
    \ea 
    $[$ta.ki.ni$]$ \
    \gll taki-ni \\
    sing-1\Sg{} \\
    \glt `I sing.' 
    \ex
    $[$tu.suŋ.ki$]$ \
    \gll tusu-nki \\
    sing-2\Sg{} \\
    \glt `You sing.'
    \ex
    $[$ham.piŋ.ku$]$ \
    \gll jampi-nku \\
    cure-3\Pl{} \\
    \glt `They cure.'
    \ex 
    $[$ta.puŋ.ki$]$ \
    \gll tapu-nki \\
    ask-2\Sg{} \\
    \glt `You ask.'
    \z 
\z

Note that the vowel lowering to [o] and [e] occurs across morpheme junctures as well. Illustrative examples are provided in (\ref{ex:oe2}).

\ea \label{ex:oe2}
    \ea
    [ta.ker.ʁa.ni] \
    \gll taki-rqa-ni \\
    sing-\Pst{}-1\Sg{} \\
    \glt `I sang.'

    \ex 
    [tu.sor.ʁaŋ.ki] \
    \gll tusurqanki  \\
    dance-\Pst{}-2\Sg{} \\
    \glt `You danced.'
    
    \ex 
    [ham.per.ʁaŋ.ku] \
    \gll jampi-rqa-nku \\
    cure-\Pst{}-3\Pl{} \\
    \glt `They cured.' \\
    \ex 
    [waj.k'or.ʁaj.ku] \
    \gll wayk'u-rqa-yku \\
    cook-\Pst{}-1\Pl{}:\Excl{} \\
    \glt `We (excl.) cooked.'
    
    \ex
    [ta.por.ʁaŋ.ki] \
    \gll tapu-rqa-nki \\
    ask-\Pst{}-2\Sg{} \\
    \glt `You asked.'
\z 
\z 

The vowel lowering rule appears to not apply across all junctures, however. For instance in (\ref{ex:nolowering1}) we find that the rule does not apply across a \ref{vpos:XP2}-\ref{vpos:vcore} juncture in the verb complex. 

\ea \label{ex:nolowering1}
    [kaj tu.tu.mas.pi ʁo.tʃi.sun.tʃik a \downarrow]   \\ 
    \glll {} kay tutuma-s-pi qu -chi -su -n -chik eh \\
    v: \ref{vpos:XP2} - \ref{vpos:vcore} -\ref{vpos:chi} -\ref{vpos:tame} -\ref{vpos:SAsuf} -\ref{vpos:plural} - \\
    {} \Dem{} tutuma-\Pl{}-\Loc{} drink -\Caus{} -\Fut{} -n -1\Pl{} -    \\
    \glt `In these tutumas we are going to drink.' \\ Sp. `En estas tutumas vamos a beber pues.' 
\z 

There is no evidence for the lowering rule across most of the boundaries of the verb complex, because most of these boundaries contain no suffixes with uvular consonants. The rule does not apply across a \ref{vpos:clitics2}-\ref{vpos:XP4} juncture, however.

The \textsc{minimal vowel lowering domain} in the verb complex spans \ref{vpos:vcore}-\ref{vpos:derivsuf1}. There is evidence that the rule applies inside the verb root and across the boundary between the verb core and the derivational suffixes/clusters. The \textsc{maximal vowel lowering domain} in the verb complex identifies a \ref{vpos:vcore}-\ref{vpos:clitics2}. 

\section{Summary and discussion} % (fold)
\largerpage[-1]%longdistance
\label{sbq:sec:summary}

The pooled results of constituency tests in the verb complex are displayed in \figref{fig:sbq.testspooled}.

\begin{figure}
    \centering
    \includegraphics[height=.45\textheight]{figures/quechua_layerspooled_20230924.pdf}
    \caption{Domains of Uma Piwra South Bolivian Quechua}
    \label{fig:sbq.testspooled}
\end{figure}

Judging by highest convergences, one candidate word constituent emerges from the results. The first identifies the \ref{vpos:vcore}-\ref{vpos:plural} span (layer 8). This domain corresponds to the traditional word. It corresponds to the domain of (maximal) vowel lowering, (minimal) pitch accent, and (minimal) asyndetic clause conjunction. The first two domains are phonological and the third one is indeterminate. Thus, the planar-fractal method seems to reveal that the traditional word in Quechua is closer to a phonological one. 

If the traditional word is a phonological domain, it is less clear what the morphosyntactic word corresponds to in SBQ. Based on convergences, it could be either the domain from \ref{vpos:vcore}-\ref{vpos:prog}, or from \ref{vpos:vcore}-\ref{vpos:SAsuf}. Whatever the case, insofar as we assume the tests identify constituent structure, the `word' of SBQ appears to bear some internal structure, even while, as we have argued in this chapter, claims concerning ``layering'' of suffixes are somewhat exaggerated.

The convergence plot could also be interpreted as suggesting a shift from word to phrase structure in the shift from layer 8 to layer 9. There are a few properties of the distribution of the domains that suggest such a shift. First, the layers from 1--8 cluster relatively close together in terms of the spans which they occupy compared to domains above layer 8. When one ascends from layer to layer, they are not strongly separated from each other in terms of span size until we arrive at the shift from layer 8 to layer 9. Secondly, after layer 9 most of the layers are `maximal' interpretations of other tests. The results suggest that maximal domains might be better associated with phrase-level domains. Nevertheless the relationship is not perfect (consider \textsc{maximal ciscategorial selection strict}). Also domains associated with conjunction occur at layer 8 and above. These properties show a general separation between morphology and syntax in Uma Piwra Quechua.

We think that the results might bring some clarity to an apparent disagreement in the Quechua literature regarding the lexical vs. syntactic structure of the `word' (\citealt{muysken:1981} vs. \citealt{weber1983relationship}).

\citet{muysken:1981} argues that in Quechua there is a general separation of morphology from syntax. He formulates his claims in terms of different properties associated with word formation rules (WFRs) versus prase structure rules (PSRs). Muysken, focusing on Tarma Quechua of Peru, makes the following arguments about WFRs: (i) WFRs can only create structures with up to three branches (compared to phrase structure rules which are ``unconstrained"); (ii) WFRs are subject to a ``unitary base hypothesis", meaning all affixes will be ciscategorial; (iii) WFRs cannot impose co-occurrence constraints between elements that are not in the same cycle; (iii) WFRs create opaque allomorphy, but PSRs do not.\footnote{A fourth point Muysken makes is about a ``subjacency condition", which imposes constraints on WFRs such that they cannot refer to elements across different cycles. We found this claim somewhat hard to decipher, but as far as we have been able to discern it also holds of Uma Piwra word-internal elements. However, it is not clear to us whether the claim might not also hold over word-external relations, if one attempts to define a cycle (or ``phase") at this level of grammar. We leave this question to future research.}

It is somewhat difficult to assess Muysken's claims about branching now, because there are models of syntax available where phrase structure rules are constrained in ways that he states they could not be (e.g. models which impose binary branching). But the basic claim might hold if we assume that the variable ordering of clitics and NPs around the verbal word implies a flatter constituency structure. The ``unitary base hypothesis" translates to the \textsc{ciscategorial selection domain}. There appears to be only rough support for this claim in Uma Piwra SBQ. The morpheme \textit{-lla} intervenes in the verbal word, and at least plural suffixes could be analzyzed as transcategorial, modifying a subject or a verb or a possessor of a noun. The third point made by Muysken is partially a matter of analysis. The opaque \textit{-ku}-based allomorphy might disappear once we admit complex suffixes, as we suggested in this chapter. Based on the properties he attributes to word formation rules required to describe Quechua, Muysken argues that ``...it is both accurate and helpful to postulate the [morphology-syntax] dichotomy" \citep[279]{muysken:1981}. 

Weber argues against Muysken based on data from Huallaga Quechua:

\begin{quote}
    ... Quechua provides considerable evidence that morphology and syntax must be closely integrated, and that strictly separating them makes capturing certain regularities of the language--if not impossible-- at least very difficult I take the position that morphology and syntax are not distinct components and that they should be treated as a single domain called \textit{morpho-syntax}. \citep[162]{weber1983relationship}

\end{quote}
The main arguments that Weber provides are cases where a suffix scopes over a word-external element or cases where co-reference appears to require the construction of word formation rules (e.g. causativization, passivization) which are interlaced with the construction of phrases.\footnote{Weber \citeyear[176]{weber1993binding} also argues that ditropic clitics (``wrong-way cliticization") provide evidence against the morphology-syntax distinction. This argument only seems to follow if one does not admit that phonological words or phrases can be misaligned with morphosyntactic words or phrases.} Weber seems to assume throughout his argument that any issues of semantic scope should be reflected in base-generated phrase structure rules. It is not clear if this assumption is adopted by Muysken, as he focuses more on distributional and phonological differences between elements inside words compared to those outside. We suspect, therefore, that the theoretical issue is somewhat obscured because the empirical facts are mediated by phrase structure rules coupled with unstated assumptions about what these phrase structure rules are supposed to account for, represent or predict.

Putting aside the details of the formal proposal, in a general sense, both Muysken and Weber are correct. Quechua suffixes are deeply intertwined with syntax accomplishing expressive feats often reserved for ``syntax" in less polysynthetic languages. On the other hand, dichotomous structure emerges when we aggregate over surface constituency diagnostics, suggesting a quantal morphology-to-syntax-like shift in organization from morph to utterance. Perhaps a better research strategy is to focus on the empirical details which could motivate the dichotomous structure in language after language first, before attempting to enshrine these in universal formal principles based on data from a couple of languages. We might be able to model modularity in morphosyntax,  while also recognizing its typological plasticity. 

\section*{Acknowledgements}
This chapter would not be possible without the support of the monolingual Quechua speakers from Uma Piwra rural town in Southern Bolivia. The monolingual speakers of this town kindly shared the way they use Quechua in their everyday-life to me. One topic of our conversations in every data life during the fieldwork was how honored I was to be able to hear the way my grandparents spoke and use Quechua. They showed also how proud they were of the way they speak the language.
I am also thankful to the funding agency that supported economically my fieldwork NSF DLI-DDRI grant from the National Science Foundation BCS-2035185 awarded to the University of Texas at Austin.
I am also thankful to the department of linguistics and the support of faculty there.  Likewise, UT supports their students economically to students to successfully continue with their research. Throughout my fieldwork I received many economic resources such as Carlotta Smith, Joel Sherzer, and thematic continuing grants. 
Finally, the robust natural corpus would not be possible to achieve transcribing without the help of the new pioneering Quechua scholars who transcribed the data in ELAN. 

\printglossary

\printbibliography[heading=subbibliography,notkeyword=this]

\end{document}
















