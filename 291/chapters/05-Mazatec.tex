\documentclass[output=paper]{langscibook}
\ChapterDOI{10.5281/zenodo.13208548}
\author{Shun Nakamoto\affiliation{Universidad Nacional Autónoma de México}}
\title{Constituency in Ayautla Mazatec}
\abstract{This study reports the result of 30 constituency diagnostics applied to Ayautla Mazatec, a Popolocan (Otomanguean) language from Oaxaca, Mexico, following the methods laid out by \citet{tallman20beyond,tallman2021constituency}. This language shows a high convergence rate of morphosyntactic diagnostics, while phonological diagnostics rarely converge.}

\IfFileExists{../localcommands.tex}{%hack to check whether this is being compiled as part of a collection or standalone
   \usepackage{langsci-optional}
\usepackage{langsci-gb4e}
\usepackage{langsci-lgr}

\usepackage{listings}
\lstset{basicstyle=\ttfamily,tabsize=2,breaklines=true}

%added by author
% \usepackage{tipa}
\usepackage{multirow}
\graphicspath{{figures/}}
\usepackage{langsci-branding}

   
\newcommand{\sent}{\enumsentence}
\newcommand{\sents}{\eenumsentence}
\let\citeasnoun\citet

\renewcommand{\lsCoverTitleFont}[1]{\sffamily\addfontfeatures{Scale=MatchUppercase}\fontsize{44pt}{16mm}\selectfont #1}
  
   %% hyphenation points for line breaks
%% Normally, automatic hyphenation in LaTeX is very good
%% If a word is mis-hyphenated, add it to this file
%%
%% add information to TeX file before \begin{document} with:
%% %% hyphenation points for line breaks
%% Normally, automatic hyphenation in LaTeX is very good
%% If a word is mis-hyphenated, add it to this file
%%
%% add information to TeX file before \begin{document} with:
%% %% hyphenation points for line breaks
%% Normally, automatic hyphenation in LaTeX is very good
%% If a word is mis-hyphenated, add it to this file
%%
%% add information to TeX file before \begin{document} with:
%% \include{localhyphenation}
\hyphenation{
affri-ca-te
affri-ca-tes
an-no-tated
com-ple-ments
com-po-si-tio-na-li-ty
non-com-po-si-tio-na-li-ty
Gon-zá-lez
out-side
Ri-chárd
se-man-tics
STREU-SLE
Tie-de-mann
}
\hyphenation{
affri-ca-te
affri-ca-tes
an-no-tated
com-ple-ments
com-po-si-tio-na-li-ty
non-com-po-si-tio-na-li-ty
Gon-zá-lez
out-side
Ri-chárd
se-man-tics
STREU-SLE
Tie-de-mann
}
\hyphenation{
affri-ca-te
affri-ca-tes
an-no-tated
com-ple-ments
com-po-si-tio-na-li-ty
non-com-po-si-tio-na-li-ty
Gon-zá-lez
out-side
Ri-chárd
se-man-tics
STREU-SLE
Tie-de-mann
}
    \bibliography{../localbibliography}
    \bibliography{../collection_tmp}
    \togglepaper[5]
}{}

\glsresetall

\begin{document}
\maketitle

\section{Introduction}
%\subsection{About this study}\label{sec-in-ov}
This study aims at providing a comprehensive list of phonological and morphosyntactic diagnostics which define syntagmatic domains in the Ayautla Mazatec sentence with a verbal predicate, %in order to critically assess the notion of wordhood in this language,
following the methods laid out by \citet{tallman20beyond,tallman2021constituency}.
\S\ref{sec-in-lg} introduces the language background, and \S\ref{sec-in-planar} lays out the methodology, before examining the planar structure in Ayautla Mazatec (\S\ref{sec-ayaplanar}) and the constituency diagnostics (\S\ref{sec-diag}). \S\ref{sec-concl} concludes this study.

\subsection{Language background}\label{sec-in-lg}
Ayautla Mazatec (ISO 639-3: vmy; Glottocode: ayau1235) is a Lowland variety of Mazatec spoken mainly in San Bartolomé Ayautla in the northernmost part of the state of Oaxaca, Mexico, by approximately 3,000 individuals of all ages.
Mazatec, along with Ngiwa (Chocholtec-Popoloca) and moribund Ixcatec, belongs to the Popolocan branch of Otomanguean linguistic family  \citep{fdm51,gud58,hamp58,hamp60}.
San Martín Duraznos Mixtec \parencitetv{chapters/06-Mixtec},
Teotitlán del Valle Zapotec \parencitetv{chapters/07-Zapotec} and
Zenzontepec Chatino \parencitetv{chapters/08-Chatino} are its distant relatives. Despite the genetic distance, Ayautla Mazatec shows striking similarities in its planar structures and, to some extent, the results of constituency diagnostics, with Teotitlán del Valle Zapotec and Zenzontepec Chatino.
The data for this study come from my own fieldwork since 2017. Recorded textual examples from unpublished sources are indicated by a unique identifier. For the sake of illustration, however, many examples were elicited.

Phonologically, Ayautla Mazatec is a heavily tonal language—/1/ being the lowest tone and /4/ the highest—, with a boundary-sensitive phonological process of tone sandhi caused by floating /4/, transcribed as /(4)/ (\citealt[171--196]{nakamoto20}; see also \S\ref{sec:d:sandhi1}-\ref{sec:d:psandhi} on sandhi-related constituency diagnostics).
Morphologically, inflectional exponence of person/number and aspect/mood involves extensive weak or strong suppletion of the verb roots and suppletive allomorphy of the prefixes derived from verb roots \citep[267--319]{nakamoto20}.
Morphosyntactically, Ayautla Mazatec is a strongly head-marking language (cf.~\citealt{nichols86}) with split intransitivity, and consistently shows the typological correlates of VO languages (cf.~\citealt{dryer07}). A root-level distinction between verbs, nouns, adjectives and positionals — mostly used as dependent components of compound verbs (cf.~\citealt{cowancowan47}; \citealt[343]{kalstrometal95})—is well motivated \citep[27--63]{nakamoto20}. As other Popolocan languages, Ayautla Mazatec has a relatively small number of verb roots, approximately 150 (\citealt[267--268]{nakamoto20}; see also \citealt[161]{pikek48}; \citealt[328]{pikee67}; \citealt{austinetal95}). This paucity of verb roots is compensated by an extensive use of verb compounding and derivational prefixes that originated in verb roots.


\subsection{Planar structure, constituency diagnostics and wordhood}\label{sec-in-planar}
In order to explicitly represent the linear nature of speech and to avoid \textit{a priori} definitions of word, clitic or any other syntagmatic units of analysis, I adopt a radically flat representation of clause structure (planar structure) advocated by \citeauthor{tallman20beyond} (\citeyear{tallman20beyond,tallman2021constituency}, this volume).\footnote{For expositional reasons, however, I follow the segmentation and boundary signs used in \citet{nakamoto20}, where a plus sign (+) represents the compound boundary; a hyphen (-) the affix boundary; and an equal (=) the boundary of proclitics, second position clitics, enclitics and focus marker.

Spaces are written between orthographic words, which treat proclitics (with or without second position clitics) as separate words.

See \S\ref{sec-ayaplanar} on proclitics, second position clitics and enclitics in Ayautla Mazatec.} 

A planar structure is made up of \textbf{positions}, which may be a \textbf{slot} filled with an \textbf{element}—i.e.~a morpheme, a compound stem, a phrase or a group of morphemes/stems/phrases in a paradigmatic relationship—, or a \textbf{zone}, where more than one element can occur without restrictions on their ordering.

 %Connectors (position \ref{conj}) and attitudinal particles (\ref{att}) are also bound morphemes, 
%In this study, clitics in are arbitrarily defined as bound morphemes outside the positions \ref{prog}-\ref{com} in verbal predicates, in addition to . Therefore, clitics include various sets of morphemes each set with its own distributional properties.

Once the planar structure is established, language-specific \textbf{constituency tests} are mapped onto it around the \textbf{core} (root or compounded/derived stem). A constituency test is a phonological and/or morphosyntactic phenomenon which defines a syntagmatic domain, such as the domain of stress assignment, the domain of free occurrence, and so on.
If a given test is ambiguous and delimits different spans according to interpretation, \textbf{test fracturing} is applied following \citet{tallman2021constituency}. For example, if the positive evidence and the negative evidence of a phenomenon define different domains, they are treated as two \textbf{constituency diagnostics}.
Every constituency diagnostic then is well-defined, in the sense that it has a beginning position and an ending position.

According to this methodology, wordhood is not defined in accordance with some preestablished criteria (see \citealt{haspelmathword:2011} for a critique on the universality of common wordhood criteria). Instead, wordhood can be understood as one of such domains on which two or more constituency diagnostics \textbf{converge}.
%See \citet{tallman20beyond,tallman2021constituency}, (this volume) for more details.


\section{Ayautla Mazatec planar structures}\label{sec-ayaplanar}
In this section, first I present the planar structure of sentences with a verbal predicate (\S\ref{sec-ayaplanar:vpred}). Then I explore internally complex positions in the verbal planar structure, i.e.~adverbs (\S\ref{sec-ext1}) and noun complexes (\S\ref{sec-ext2}), followed by a clarification about extra-clausal operations of topicalization and afterthoughts  (\S\ref{sec-extraclausal}).

Note that subordinate clauses, including relative clauses, complement clauses and adverbial clauses, will be treated in \S\ref{sec:d:subspan} as repeated subspans of the planar structure.

\subsection{Verbal predicate}\label{sec-ayaplanar:vpred}
\tabref{tab:verbtemplate} shows the planar structure of the Ayautla Mazatec sentence with a verbal predicate, which is built around the verb root (position \ref{v}). Non-verbal predicates are outside the scope of this study (but see \citealt[55--57, 252]{nakamoto20}).
%The template does not represent all components of a sentence. The abridgements and omissions will be discussed in the following subsections (\S\ref{sec-ext}, \S\ref{sec-extraclausal}).% Concretely, each of the adverb (positions \ref{adv1}, \ref{np1}, \ref{adv2}, \ref{np2}) and the noun phrase (\ref{np1}, \ref{np2}) has internal structure. 


\begin{table}
     \caption{Verbal planar structure of Ayautla Mazatec}
     \label{tab:verbtemplate}
  \begin{tabularx}{\textwidth}{Srlll}
 \lsptoprule
\multicolumn{1}{r}{Pos.}     & Type  & Elements  & Forms \\ \midrule      
 \label{conj} & slot          & connector          & \textit{ha\1, ʔba\1, ʔa\2sa\1, tu\1\ssn ka\2}, etc. \\
 \label{adv1}& zone          & sentence adv., topicalization              & \textit{} \\
 \label{focw} & slot          & polar \Q{}, `even', `only' & \textit{tu\1, su\2ba\4, na\2se\3\ff, ʔa\2}, etc.\\
 \label{np1}& zone          & noun complex, adverb               & \textit{} \\
 \label{foc1}& slot          & focus             & \textit{=\3\ff} \\
\label{ant} & slot          & anterior/posterior& \textit{he\2\ff=, khe\2\4=(…=\4hĩ\4)} \\
\label{alm} & slot          & `almost'             & \textit{me\2he\4=} \\
\label{bi2}& slot          & negation + ant/post      & \textit{bi\4=, bi\4=…(\ssn te\1=), ni\1=} \\
\label{tmp1}& slot          & temporal sequence         & \textit{=hba\4, =hba\4ni\2\3, =hba\4ra\2, =ra\2} \\
\label{ev3a}& slot          & modality, evidentiality          & \textit{=ʃu\3\ff, =hĩ\4, =ru\1}\\
\label{lit} & slot          & `a little'              & \textit{=\1tɕi\4} \\
\label{in1} & slot          & past habitual, `always' & \textit{=ʔĩ\3\ff, =\ssn tshɛ\4} \\
\label{nde1}& slot          &  negation + ant/post 2& \textit{\ssn te\1=} \\
\label{pron}& slot          & adverb, pronoun (P/R)               & \textit{ʔã\2, hi\2\3, k\lab i\2, hĩ\1, ɲã\3\ff, hũ\4}, etc. \\
\label{prog}& slot          & progressive               & \textit{ti\2\ff-, te\2-}, etc. \\
\label{aspm}& slot          & aspect/mode               & \textit{b-, t-, k\lab\1-}, etc. \\
\label{mot} & slot          & associated motion         & \textit{hi\2-, \ssn te\2\ff-, e\1-, hɛ\2ʔɛ\2-}, etc. \\
\label{voi} & slot          & causative, inchoative     & \textit{tsi\2\ff-, a\2-}, etc. \\
\label{v}& slot          & verb root(s)                   & \\ 
\label{com}& slot          & comitative                 & \textit{-ko\1\3} \\
\label{foc2}& slot          & focus             & \textit{=\3\ff} \\
\label{tmp2}& slot          & temporal sequence         & \textit{=hba\4, =hba\4ni\2\3, =hba\4ra\2, =ra\2} \\
\label{ev3b}& slot          & modality, evidentiality                   & \textit{=ʃu\3\ff, =hĩ\4, =ru\1} \\
\label{lit2}& slot          & `a little'              & \textit{=\1tɕi\4} \\
\label{in2} & slot          & past habitual, `always' & \textit{=ʔĩ\3\ff, =\ssn tshɛ\4} \\
\label{maz:enc1}& zone         & adverbial/quantifier clitics        & \textit{=h\ssn ku\2\3, =je\2he\2, =\ssn ka\2ɲi\3\ff}, etc. \\ 
\label{maz:enc2}& slot          & modal clitics        & \textit{=\4hĩ\4, =ɲi\2\3, =ni\4ɲi\2\3} \\
\label{maz:enc3}& slot  & pronominal clitics                & \textit{=a\2 $\sim$ =a\1, =i, =i\1, =a\3\ff, =u\4}, etc. \\
\label{np2} & zone          & noun complex, adverb                 & \textit{ } \\
\label{att} & slot          & attitudinal particles    & \textit{ja\2ʔa\2, je\2he\2, tsa\2kɛ\2\4, \ssn te\1}, etc.         \\
\label{aft} & zone          & afterthoughts, vocative words     & \textit{ } \\
 \lspbottomrule
\end{tabularx}
 \end{table}

Some constructions involve two positions. The existence of a second formative is indicated by three dots (…) after the first formative. The position for the latter is indicated in the template as ``XX 2'', such as ``negation + ant/post 2'' of the position \ref{nde1}.

Morphemes that occur in certain positions are arbitrarily called ``clitics'', for expositional reasons. Each class of clitics is defined as a group of bound morphemes with distributional properties in common. Morphemes in positions \ref{ant}-\ref{bi2} and \ref{nde1} are called proclitics. A group of morphemes occur in position \ref{tmp1}–\ref{in1} if any of \ref{np1} or \ref{ant}-\ref{bi2} is occupied; if not, they occupy \ref{tmp2}–\ref{in2}; these elements can be considered as Wackernagel or second position clitics (cf.~\citealt{2p20}). Focus marker (position \ref{foc1} or \ref{foc2}) is also segmented as a clitic. Morphemes in positions \ref{maz:enc1}-\ref{maz:enc3} are called enclitics. Inside the positions for clitics around the verb root, i.e.~positions \ref{prog}-\ref{voi} and \ref{com}, are called affixes. Together with the clitics, ``connectors'' (position \ref{conj}) and ``attitudinal particles'' (position \ref{att}) are bound morphemes too. However, given that they do not interact distributionally with the rest of the sentence, this study will pay little attention to them. 


\subsection{Adverbs}\label{sec-ext1}
The adverb has the following template (\tabref{tab:advtemplate}). If there are ``adverb 1'' and  ``adverb 2'' at the same time, they constitute only one free form.
Positions \ref{a:tmpa}-\ref{a:lita}, \ref{a:tmpb}-\ref{a:litb} or \ref{a:tmpc}-\ref{a:litc} are occupied if there are no other elements in the positions \ref{np1}–\ref{bi2} of the verbal template. Note that temporal, modality/evidentiality and `a little' clitics have a scope over the whole verbal predicate and not on the adverb.

\begin{table}
     \caption{Planar structure for adverbs}
     \label{tab:advtemplate} 
\begin{tabularx}{\textwidth}{UXll} \lsptoprule
\multicolumn{1}{r}{Pos.}    & Type  & Elements  & Forms \\ \midrule
\label{a:neg} & slot          & negation    & \textit{bi\4=, ni\1=} \\
\label{a:tmpa} & slot          & temporal sequence         & \textit{=hba\4, =hba\4ni\2\3, =hba\4ra\2, =ra\2} \\
\label{a:eva} & slot          & modality, evidentiality          & \textit{=ʃu\3\ff, =hĩ\4, =ru\1}\\
\label{a:lita} & slot          & `a little'              & \textit{=\1tɕi\4} \\
 \label{a:adv1}& slot          & adverb  & \textit{ʔba\1…(=\ssn te\1), ʔba\1…(k\lab ã\1\3)}, etc. \\
\label{a:tmpb} & slot          & temporal sequence         & \textit{=hba\4, =hba\4ni\2\3, =hba\4ra\2, =ra\2} \\
\label{a:evb} & slot          & modality, evidentiality          & \textit{=ʃu\3\ff, =hĩ\4, =ru\1}\\
\label{a:litb} & slot          & `a little'              & \textit{=\1tɕi\4} \\
 \label{a:advb} & slot          &  adverb 2& \textit{=\ssn te\1, k\lab ã\1\3}, etc. \\
\label{a:advm} & slot          & adverb type marker                        & \textit{=tsa\2, =ni\1, =\1}\\
\label{a:tmpc} & slot          & temporal sequence         & \textit{=hba\4, =hba\4ni\2\3, =hba\4ra\2, =ra\2} \\
\label{a:evc} & slot          & modality, evidentiality          & \textit{=ʃu\3\ff, =hĩ\4, =ru\1}\\
\label{a:litc} & slot          & `a little'              & \textit{=\1tɕi\4} \\
\lspbottomrule
 \end{tabularx}
 \end{table}

\newpage
\subsection{Noun complex}\label{sec-ext2}
The noun complex has the following template (\tabref{tab:nountemplate}). Positions \ref{n:tmpa}-\ref{n:lita}, \ref{n:tmpb}-\ref{n:litb} or \ref{n:tmpc}-\ref{n:litc} of this template are occupied if there are no other elements in the positions \ref{np1}–\ref{bi2} of the verbal template. Absolute state marker \textit{=\1} occurs at the end of a noun complex (i) without a possessor, (ii) without a demonstrative, (iii) not in vocative function, and (iv) without a floating tone /4/ immediately before it \citep[241--250]{nakamoto20}.%\footnote{Absolute state marker \textit{=\1} in Ayautla is cognate of `(syntactic) (down)glide' in Huautla Mazatec described in \citet[95--106]{pikek48}. However, as I will discuss in \S\ref{sec:d:subspan}, its syntactic distribution is different from that of Huautla.}

\begin{table}
     \caption{Planar structure for noun complexes}
     \label{tab:nountemplate}
\begin{tabularx}{\textwidth}{TXll}
\lsptoprule   
\multicolumn{1}{r}{Pos.}   & Type  & Elements  & Forms \\ \midrule
%\label{n:conj} & slot & conjunction & \textit{a\2sa\1, }
\label{more} & slot          & `more', `also'           & \textit{\ssn ki\2sa\1, ko\1\3} \\
\label{n:tmpa}& slot          & temporal sequence         & \textit{=hba\4, =hba\4ni\2\3, =hba\4ra\2, =ra\2} \\
\label{n:eva}& slot          & modality, evidentiality          & \textit{=ʃu\3\ff, =hĩ\4, =ru\1}\\
\label{n:lita} & slot          & `a little'              & \textit{=\1tɕi\4} \\
\label{add}  & slot          & additive, `entire'   & \textit{\ssn ki\2-, \ssn ka\2-} \\
\label{qua}& slot          & numeral, quantifier   & \textit{h\ssn ku\2\3, \ssn khĩ\4}, etc. \\
\label{n:tmpb}& slot          & temporal sequence         & \textit{=hba\4, =hba\4ni\2\3, =hba\4ra\2, =ra\2} \\
\label{n:evb}& slot          & modality, evidentiality          & \textit{=ʃu\3\ff, =hĩ\4, =ru\1}\\
\label{n:litb} & slot          & `a little'              & \textit{=\1tɕi\4} \\
\label{n}& slot          & noun root(s)    & \textit{} \\
 \label{adj}& slot          & adjective     & \textit{} \\
 \label{poss}& slot          & possesssor  & \textit{=na\1, =\1ri\2, =re\1, =ni\1, =na\3\ff{}, =\1nu\4} \\
  \label{dem}& slot          & demonstrative     & \textit{=bi\1, =bju\1}\\
 \label{rel}& slot          &  relative clause&  \\
\label{abs}& slot          & absolute state marker & \textit{=\1}                  \\
\label{n:tmpc}& slot          & temporal sequence         & \textit{=hba\4, =hba\4ni\2\3, =hba\4ra\2, =ra\2} \\
\label{n:evc}& slot          & modality, evidentiality          & \textit{=ʃu\3\ff, =hĩ\4, =ru\1}\\
\label{n:litc} & slot          & `a little'              & \textit{=\1tɕi\4} \\
 \lspbottomrule
 \end{tabularx}
 \end{table}


\subsection{On extra-clausal operations}\label{sec-extraclausal}
Within the verbal planar structure, I do not break down topicalization or left-dislocation (position \ref{adv1}) and afterthoughts or right-dislocation (position \ref{aft}) in the planar structure, because they show no distributional interactions with the rest of the sentence. Specifically, the modality/evidentiality clitic used in the main clause appears duplicated in a topicalized constituent—as illustrated by the first instance of reported information \textit{=ʃu\3\ff} in (\ref{ex:extraclausal})—or an afterthought, without altering the definition of the second position of the verbal predicate clause which begins with \textit{su\2ba\4}.\footnote{In interlinearized examples, the first line corresponds to the surface (or post-sandhi) form, the second line to the underlying (or pre-sandhi) form, the third to the verbal planar structure and the fourth and the fifth, if necessary, to the nominal and adverbial planar structure. The lines for planar structures are indicated with `v:', `n:' and `adv:', respectively. The last two lines correspond to morpheme-by-morpheme gloss and free translation with a unique identifier for textual examples.}

\ea \label{ex:extraclausal}
\textit{\textbf{ˈh\ssn ku\2\3ʃu\3	ja\2ni\4ˈtɕa\2\1} su\2ˈba\4	ˈɕi\1re\1\3ʃu\3  ˈk\lab{}ã\2\4} \\
\gllll
{} h\ssn{}ku\2\3	\textbf{=ʃu\3\ff{}}	ja\3\ff+ni\2tɕa\2	=\1	su\2ba\4	ɕi\1 	=re\1	=\3\ff{}		\textbf{=ʃu\3\ff{}}	k\lab{}-		ã\2\4 \\
v: \ref{adv1}[	{}			{}			{}]	\ref{focw}	\ref{np1}[ {}]		\ref{foc1}	\ref{ev3a}	\ref{aspm}	\ref{v}\\
n: \ref{qua}		\ref{n:evb}	\ref{n}				\ref{abs} -		\ref{n}	\ref{poss}	- 		-		 	- 			-\\
{} \textsubscript{\Top}[one			{=\Rep}		tree+ocote			{=\Abst}] only	[piece	{=\Poss}3] {=\Foc}	{=\Rep}		\Pfv- \Pfv:become\\
\glt `an ocote tree reportedly fell apart into mere pieces.' (\citealt[139]{sanchez20}, English by SN)
\z


\section{Constituency diagnostics}\label{sec-diag}
In this section I will describe the following 14 tests and 30 constituency diagnostics I have so far identified for Ayautla Mazatec. Tests 1-6 are treated generally as morphosyntactic tests, while tests 8-14 are phonological tests. Pause (test 7) is sometimes considered a morphosyntactic test \citep[cf.][]{haspelmathword:2011} and sometimes a phonological test \citep[cf.][]{dixonaikhenvald02,gerdts14}. The syntagmatic spans identified by the diagnostic appear in parentheses. If test fracturing applies, the smaller span appears first.

\begin{enumerate}
    \item Free occurrence, or minimum free form (\ref{v}-\ref{v}, \ref{prog}-\ref{maz:enc3})
    \item Deviation from biuniqueness (\ref{prog}-\ref{v}, \ref{prog}-\ref{maz:enc3})
    \item Ciscategorial selection (\ref{prog}-\ref{v}, \ref{prog}-\ref{maz:enc3})
    \item Non-interruptability (\ref{prog}-\ref{maz:enc3}, \ref{ant}-\ref{maz:enc3})
    \item Fixed order or non-permutability (\ref{prog}-\ref{com}, \ref{nde1}-\ref{foc2})
    \item Subspan repetition (\ref{prog}-\ref{v}, \ref{prog}-\ref{v}, \ref{focw}-\ref{com}, \ref{ant}-\ref{maz:enc3}, \ref{focw}-\ref{np2}, \ref{adv1}-\ref{np2})
    \item Pauses and fillers (\ref{prog}-\ref{maz:enc3})
    \item Stress assignment (\ref{v}-\ref{com}, \ref{prog}-\ref{com})
    \item *ɛ.j constraint (\ref{v}-\ref{v}, \ref{nde1}-\ref{in2})
    \item *3.(2)4 constraint (\ref{v}-\ref{v}, \ref{prog}-\ref{foc2})
    \item Syllable-internal segmental interactions (\ref{aspm}-\ref{v}, \ref{aspm}-\ref{maz:enc3})
    \item Disyllabic sandhi-blocking tone sequences (\ref{prog}-\ref{v}, \ref{prog}-\ref{foc2})
    \item Obligatory sandhi (\ref{prog}-\ref{maz:enc3})
    \item Possible sandhi (\ref{prog}-\ref{maz:enc3}, \ref{adv1}-\ref{aft})
\end{enumerate}

In the following subsections, I will describe each one of these constituency tests.

\subsection{Free occurrence (\ref{v}-\ref{v}, \ref{prog}-\ref{maz:enc3})}\label{sec:d:free}
\is{free occurrence} %
Being a minimum free form, that is, possibly constituting an utterance (and not two), has been an oft-cited criterion of wordhood \citep[cf.][39]{haspelmathword:2011}. The minimal and the maximal extension of a free form in Ayautla Mazatec verbal predicates delimit different spans, thus providing two constituency diagnostics.

On the one hand, the \textsc{minimal minimum free form} consists only of the verb root (position \ref{v}) if the verb is a non-derived stative verb in third person form, such as (\ref{ex:minmff}).

\ea \label{ex:minmff} \textit{ja\2\st{}ʔa\2\3}\\
\glll {} ja\2ʔa\2\3\\
v: \ref{v}\\
{} carry:3\\
\glt `he has, holds, carries.'
\z

On the other hand, \textsc{maximal minimum free form} includes the maximal range of elements which can occur as a single free-standing form, which spans from \ref{prog} to \ref{maz:enc3}, as illustrated in (\ref{ex:maxmff}).

\ea \label{ex:maxmff} \textit{te\4\st{}hbja\2\3\1}\\
\glll {} te\2- b- hi\2\3 =a\1\\
v: \ref{prog} \ref{aspm} \ref{v} \ref{maz:enc3}\\
{} \Prog:1- \Hab- go:1 =1\Sg\\
\glt `I am going.'
\z

Before position \ref{prog} for progressive prefix, position \ref{pron} is occupied by adverbs and independent pronouns which can be free-standing forms. 
After position \ref{maz:enc3} for pronominal enclitics, position \ref{np2} for noun complexes and/or adverb also consists of one or more free standing forms. Therefore, the maximal definition of this test is \ref{prog}-\ref{maz:enc3}.

\subsection{Deviation from biuniqueness (\ref{prog}-\ref{v}, \ref{prog}-\ref{maz:enc3})}\label{sec:d:deviation}
\is{deviation from biuniqueness} %
Deviation from biuniqueness (or one-to-one correspondence between form and function) has often been referred to as a characteristic of words but not phrases \citep[cf.][54]{haspelmathword:2011}. Ayautla Mazatec shows many cases of non-automatic allomorphy, i.e.~many-to-one correspondences between form and function \citep[cf.][132]{pikek48}.

Specifically, progressive (position \ref{prog}; \citealt[50--52]{nakamoto20}), aspect/mode (position \ref{aspm}; \citealt[39--50, 288-306]{nakamoto20}), associated motion (position \ref{mot}; \citealt[52--53]{nakamoto20}), voice (position \ref{voi}; \citealt[29--30]{nakamoto20}), verb root (position \ref{v}; \citealt[270--288]{nakamoto20}) and bound pronouns (position \ref{maz:enc3}; \citealt[236--239]{nakamoto20}) show allomorphy not conditioned by phonology. Some allomorphies in these positions are illustrated in the following pair of examples. In (\ref{ex:devbiuniq}), the progressive \textit{ti\2\ff{}-} $\sim$ \textit{te\2-}, the andative \textit{hi\4-} $\sim$ \textit{ʔi\2-}, the causative \textit{tsi\2\ff{}k-} $\sim$ \textit{ni\2k-} and the verb root \textit{i\2se\3\ff} $\sim$ \textit{i\2se\4} show different allomorphs conditioned by the agent person, in addition to the habitual \textit{b-} $\sim$ \textit{m-} conditioned phonologically as well as lexically.\footnote{The allomorphs listed above are not at all exhaustive. See corresponding sections in \citet{nakamoto20} cited above.}

\ea \label{ex:devbiuniq}
 \ea \textit{ti\2hbi\4tsi\2ki\2se\2\st{}thẽ\4\1}\\
 \glll {} ti\2\ff{}- b- hi\4- tsi\2\ff{}k- i\2se\3\ff{}+thẽ\1\\
 v: \ref{prog} \ref{aspm} \ref{mot} \ref{voi} \ref{v}\\
 {} \Prog:3- \Hab- \Andt:3- \Caus:3- rise:3\\
 \glt `he is going there to raise (something)'

 \newpage
 %formatting
\ex \textit{te\2ʔmi\2ni\2ki\2se\4\st{}thẽ\1\3}\\
 \glll {} te\2- m- ʔi\2- ni\2k- i\2se\4+thẽ\1 =i\3\\
 v: \ref{prog} \ref{aspm} \ref{mot} \ref{voi} \ref{v} \ref{maz:enc3}\\
 {} \Prog:2- \Hab- \Andt:2- \Caus:2- rise:2 =2\Sg\\
 \glt `you are going there to raise (something)'
 \z
\z

Given the discontinuity of positions which show deviation from biuniqueness, this test can be fractured into two constituency diagnostics. The minimal interpretation of this test includes the positions \ref{prog}-\ref{v}, where all positions show deviation from biuniqueness (\textbf{minimal deviation from biuniqueness}). The maximal interpretation of this test includes the positions \ref{prog}-\ref{maz:enc3}, which covers all positions outside which  deviation from biuniqueness is known not to be observed (\textbf{maximal deviation from biuniqueness}).

\subsection{Ciscategorial selection (\ref{prog}-\ref{v}, \ref{prog}-\ref{maz:enc3})}\label{sec:d:select}
\is{ciscategorial selection} %
Whether a given morpheme occurs exclusively with certain lexical categories or not has been a major criterion for distinguishing clitics from affixes \citep[cf.][45]{haspelmathword:2011}. In Ayautla Mazatec verbal predicates, progressive (position \ref{prog}), aspect/mode (position \ref{aspm}), associated motion (position \ref{mot}), voice (position \ref{voi}), verb roots (position \ref{v})—root-level distinction of lexical categories is clear in this language—, %comitative (position  \ref{com}),\footnote{Comitative \textit{ko\1\3} may occur as preposition before a noun phrase.}
adverbial/quantifier clitic (position \ref{maz:enc1}) and bound pronouns (position  \ref{maz:enc3}) are limited to verbal predicates.\footnote{Although a similar set of bound pronouns is used to indicate the possessor of some bodypart and kinship terms, it has unpredictable differences between the one used for predicates the one used for possession. I excluded comitative (position \ref{com}) from this list, because it can be used as preposition in noun complexes.} For example, in (\ref{ex:ciscat:2}), independent pronouns \textit{k\lab{}i\2} `\Pronom{}3' and \textit{ʔã\2} `\Pronom{}1\Sg' as well as comitative \textit{ko\1\3} are used in non-verbal predicates.

\ea \label{ex:ciscat:2}
\textit{ma\2\st{}sẽ\2 ta\1\2 bi\4 \ssn te\1 \textbf{ko\1\3} \textbf{\st{}k\lab{}i\2} he\2\ff{} tu\1 \textbf{\st{}ʔã\2}}\\
\gll {} ma\2sẽ\2 ta\1\2 bi\4= \ssn te\1 ko\1\3= k\lab{}i\2 he\2= tu\1 ʔã\2\\
%n: - - - - - \ref{n} \\
{} half but \Neg= anymore with= \Pronom{}3 already= only \Pronom{}1\Sg\\
\glt `some (lit. half) but not with them anymore, now it's only me' (180624-002 08:56)
\z

As with the previous test, ciscategorial selection can be interpreted in different manners. On the one hand, all elements in positions \ref{prog}-\ref{v} occur exclusively in verbal predicates (\textbf{minimal ciscategorial selection}). On the other hand, no position outside \ref{prog}-\ref{maz:enc3} shows word-class selectivity (\textbf{maximal ciscategorial selection}).

\subsection{Non-interruptability (\ref{prog}-\ref{maz:enc3}, \ref{ant}-\ref{maz:enc3})}\label{sec:d:interrupt}
\is{interruptibility} %
Another common definition of word is that of the uninterruptible string of morphemes \citep[cf.][44]{haspelmathword:2011}. However, following previous critiques \citep[cf.][]{haspelmathword:2011,tallman2021constituency}, I provide two constituency diagnostics, namely, \textsc{non-interrup\-tible by a single free form} and \textsc{non-interruptible by complex free form(s)} or \textsc{more than one free form}, which give different results in Ayautla Mazatec.

The maximum span of non-interruptible elements by a single free form begins at the progressive (position \ref{prog}). Immediately before it (position \ref{pron}), a non-focused adverb (\ref{ex:prona}) or an emphatic pronoun in P/R function (\ref{ex:pronb})—both of which are free forms—can occur.


\ea 
 \ea \label{ex:prona} \textit{bi\4ʃu\3 ʔba\1 tsɛ̃\2\st{}ʔɛ̃\3 k\lab{}e\1\st{}ʔe\4ri\2}\\
 \gllll {} bi\4= =ʃu\3\ff{} ʔba\1 tsɛ̃\2ʔɛ̃\3\ff{} k\lab{}- e\1ʔe\4 =ri\2\\
v: \ref{bi2} \ref{ev3a} \ref{pron}[ {}] \ref{aspm}- \ref{v} =\ref{maz:enc3}\\
adv: - - \ref{a:adv1} \ref{a:advb} - - -\\
{} \Neg{}= =\Rep{} [so do:3] \Pot- \Pot:beat =3/2\Sg\\
\glt `he won't beat you like that.' 
 \ex \label{ex:pronb} \textit{bi\4ʃu\3 \st{}hi\2\3 k\lab{}e\1\st{}ʔe\4ri\2}\\
 \glll {} bi\4= =ʃu\3\ff{} hi\2\3 k\lab{}- e\1ʔe\4 =ri\2\\
v: \ref{bi2} \ref{ev3a} \ref{pron} \ref{aspm}- \ref{v} =\ref{maz:enc3}\\
%adv: - - \ref{a:adv1} \ref{a:adv2} - - -\\
{} \Neg{}= =\Rep{} \Pronom{}2\Sg{} \Pot- \Pot:beat =3/2\Sg\\
\glt `he won't beat you.' 
 \z
\z

If position \ref{pron} is filled with two free forms, or complex free form, like (\ref{ex:pronc:1}, \ref{ex:pronc:2}), the result is ungrammatical.
\ea 
\ea \label{ex:pronc:1}
*bi\4ʃu\3 ʔba\1 tsɛ̃\2\st{}ʔɛ̃\3 \st{}hi\2\3 k\lab{}e\1\st{}ʔe\4ri\2\\
\gllll {} bi\4= =ʃu\3\ff{} ʔba\1 tsɛ̃\2\st{}ʔɛ̃\3\ff{} hi\2\3 k\lab{}- e\1ʔe\4 =ri\2\\
v: \ref{bi2} \ref{ev3a} \ref{pron}[ -] \ref{pron} \ref{aspm} \ref{v} \ref{maz:enc3}\\
adv: - - \ref{a:adv1} \ref{a:advb} - - - -\\
{} \Neg= =\Rep{} [so do:3] \Pronom{}2\Sg{} \Pot- \Pot:beat =3/2\Sg\\
\glt intended: `he won't beat you like that.'
\ex \label{ex:pronc:2}
*bi\4ʃu\3 \st{}hi\2\3 ʔba\1 tsɛ̃\2\st{}ʔɛ̃\3 k\lab{}e\1\st{}ʔe\4ri\2\\
\gllll {} bi\4= =ʃu\3\ff{} hi\2\3 ʔba\1 tsɛ̃\2\st{}ʔɛ̃\3\ff{}  k\lab{}- e\1ʔe\4 =ri\2\\
v: \ref{bi2} \ref{ev3a} \ref{pron} \ref{pron}[ -] \ref{aspm} \ref{v} \ref{maz:enc3}\\
adv: - - - \ref{a:adv1} \ref{a:advb} - - -\\
{} \Neg= =\Rep{} \Pronom{}2\Sg{} [so do:3] \Pot- \Pot:beat =3/2\Sg\\
\glt intended: `he won't beat you like that.'
\z
\z

On the other hand, the maximum span non-interruptible by more than one free form covers positions \ref{ant}-\ref{maz:enc3}, delimited by two zones which may have noun complexes and adverbs (positions \ref{np1}, \ref{np2}), possibly filled with complex free forms.% Although a focus marker (position \ref{foc1}) occurs after a focused noun phrase or an adverb (position \ref{np1})

In sum, non-interruptability by a free form spans from position \ref{prog} to position \ref{maz:enc3}, while non-interruptability by more than one free forms covers positions \ref{ant}-\ref{maz:enc3}.

\subsection{Fixed order or non-permutability (\ref{prog}-\ref{com}, \ref{nde1}-\ref{foc2})}
Fixed order or non-permutability of morphemes has typically been considered characteristic within a word but not a phrase (cf. \citealt[19--20]{dixonaikhenvald02}). However, the ambiguity of this test is notorious, since strict ordering of syntactic elements and variable ordering of affixes have also been reported.
In this study, following the critique by \citet[\S5.4]{tallman2021constituency}, I divide the non-permutability test in two versions. One is \textsc{strict non-permutability}, which entails a span of positions whose elements always occur in a fixed order. The other is \textsc{non-permutability without scopal difference}, which, in addition to the previous one, includes positions with variably ordered elements, where this variable ordering corresponds to a difference in scope.

Strict non-permutability defines positions \ref{prog}-\ref{com} as its span, since morphemes in these positions \ref{prog}-\ref{com} in this order.
Just outside progressive (position \ref{prog}), there is a position for adverbs and independent pronouns (position \ref{pron}). Adverbs in this position have scope over the predicate, while adverbs in position \ref{adv1} have scope over the whole sentence. For that reason, some adverbs cannot occur in one of the two positions. For example, \textit{ʔba\1 tsɛ̃\2ʔɛ̃\3} `that way' can occur in position \ref{pron} (\ref{ex:pronpr:1}) but cannot in position \ref{adv1} (\ref{ex:pronpr:2}).

\ea 
\ea \label{ex:pronpr:1}
\textit{bi\4 \textbf{ʔba\1 tsɛ̃\2\st{}ʔɛ̃\3} e\2\st{}hɲu\4na\1}\\
\glll {} bi\4= ʔba\1~tsɛ̃\2ʔɛ̃\3\ff{} e\2hɲu\4 =na\1 \\
v: \ref{bi2} \ref{pron}  \ref{aspm}:\ref{v} \ref{maz:enc3}\\
{} \Neg= that.way \Pfv:deceive:3 =3/1\Sg\\
\glt `he didn't deceived me that way.'
\ex \label{ex:pronpr:2}
*\textbf{ʔba\1 tsɛ̃\2\st{}ʔɛ̃\3} bi\4 e\2\st{}hɲu\4na\1\\
\glll {} ʔba\1~tsɛ̃\2ʔɛ̃\3\ff{} bi\4= e\2hɲu\4 =na\1 \\
v: \ref{adv1} \ref{bi2}  \ref{aspm}:\ref{v} \ref{maz:enc3}\\
{} that.way  \Neg= \Pfv:deceive:3 =3/1\Sg\\
\glt intended: `that way he didn't deceived me.'
\z
\z

%Independent pronouns in this position can only be in patient- or recipient-like function, while the same pronouns can be an intransitive subject or an agent-like argument in transitive verbs when it is found in positions \ref{np1} or \ref{np2}. Therefore, in the following examples, while a patient-like argument can be expressed by an independent pronoun in position \ref{pron} (\ref{ex:pronpr:1}) or position \ref{np2} (\ref{ex:pronpr:2}), an agent-like argument cannot be expressed by an independent pronoun in position \ref{np2} (\ref{ex:pronpra:1}), but only in position \ref{np2} (\ref{ex:pronpra:1}).

Just before the preverbal adverb or independent pronoun (position \ref{pron}) is for the second part of the bipartite proclitic \textit{bi\4\ldots\ssn{}te\1} `not\ldots{}anymore' (position \ref{nde1}), which is exclusive to this position.\footnote{Probably the second morpheme of this construction \textit{\ssn{}te\1} comes from the same morpheme used as a sentence-final attitudinal particle \textit{\ssn{}te\1} `thus'. However, I consider this construction undecomposable, therefore I do not consider \textit{\ssn{}te\1} as a permutable morpheme.} Therefore, \ref{nde1} is included in the less strict definition of non-permutability.

After the comitative (position \ref{com}), the focus marker (position \ref{foc2}) can also occur in another place (position \ref{foc1}). This difference too corresponds to different focalized constituents. Therefore, \ref{foc2} is included in the broader interpretation of non-permutability.

Outside these positions (\ref{nde1}-\ref{foc2}) are Wackernagel clitics (positions \ref{tmp1}-\ref{in1} and \ref{tmp2}-\ref{in2}), which occur in different positions within the clause, but do not involve differences in scope. Therefore, all these positions involve permutable elements.

In sum, strict non-permutability defines \ref{prog}-\ref{com} as its span, while non-permuta\-bil\-ity without scopal difference defines \ref{nde1}-\ref{foc2}.

\subsection{Subspan repetition (\ref{prog}-\ref{v}, \ref{prog}-\ref{v}, \ref{focw}-\ref{com}, \ref{ant}-\ref{maz:enc3}, \ref{focw}-\ref{np2}, \ref{adv1}-\ref{np2})}\label{sec:d:subspan}
\is{total reduplication} %
\is{coordination} %
\is{subspan repetition} %
Some constructions specify a span of positions which can be repeated and thus can be employed as constituency tests of \textsc{subspan repetition} \citep[cf.][]{tallman2021constituency}. So far I have identified total reduplication, verbal parallelism (both at \S\ref{sec:d:subspan1}), grammatical nominalization with absolute state marker \textit{=\1} (\S\ref{sec:d:subspan2}), and coordination (\S\ref{sec:d:subspan3}) as distinct subspan repetition constructions.

%See also \S\ref{sec:d:psandhi} on possible sandhi, which partially overlaps with this test.


%, hence subspan repetition test in Ayautla Mazatec also serves as \textbf{possible sandhi} test.

%The smallest subspan repetition construction in Ayautla Mazatec is formation of \textbf{abstract nouns} with \textit{k\lab{}ha\1-}.

%\ea \label{ex:lexnmlz1}
%\textit{k\lab{}ha\1be\2hnu\4\1}\\
%\glll {} k\lab{}ha\1- be\2hnu\4 =\1
%\z

\subsubsection{Total reduplication (\ref{prog}-\ref{v}) and verbal parallelism (\ref{prog}-\ref{v})}\label{sec:d:subspan1} 
The smallest subspan repetition constructions in Ayautla Mazatec are  \textsc{total reduplication} and \textsc{verbal parallelism}, which specify positions \ref{prog}-\ref{v}.

Total reduplication in Ayautla Mazatec, illustrated in (\ref{ex:tot}) below, repeats a subspan of the verbal predicate, regardless of phonological conditions such as the syllable structure and the number of syllables. This process indicates the exhaustivity of the action expressed by the verb and therefore it is limited to dynamic verbs.
In (\ref{ex:tota}), position \ref{pron} is excluded from the repeated subspan. In (\ref{ex:totb}), position \ref{com} is excluded from the repeated subspan.

\ea \label{ex:tot}
\ea \label{ex:tota} \textit{he\2 \st{}k\lab{}i\2 bo\2ʔo\2\st{}ja\4 bo\2ʔo\2\st{}ja\4ʔĩ\3}\\  %je kui bo'oyá bo'oyá'īn
\glll {} he\2\ff= k\lab{}i\2 b- o\2ʔo\2+ja\4 b- o\2ʔo\2+ja\4 =ʔĩ\3\ff\\
v: \ref{ant} \ref{pron} \textsubscript{redup1}[\ref{aspm} \ref{v}] \textsubscript{redup2}[\ref{aspm} \ref{v}] \ref{in2}\\
{} already= \Pronom{}3 \textsubscript{redup1}[\Hab- hit+\Pos:inside] \textsubscript{redup2}[\Hab- hit+\Pos:inside] =\Pst.\Hab\\
\glt `he already used to beat and beat him.'\\
\ex\label{ex:totb} \textit{pa\2\3\st{}la\1 te\2khe\2\st{}ʔ\ssn{}ki\3 te\2khe\2ʔ\ssn{}ki\2\st{}ko\4ʔĩ\3ɲa\3\2}\\  % pala tekje'ngī tekje'ngikó'īnñạ̄
\glll {} pa\2\3la\1 te\2- khe\2+ʔ\ssn{}ki\3 te\2- khe\2+ʔ\ssn{}ki\3\ff{} -ko\1\3 =ʔĩ\3\ff{} =ɲi\3\ff{} =a\2\\
v: \ref{np1} \textsubscript{redup1}[\ref{prog} \ref{aspm}:\ref{v}] \textsubscript{redup2}[\ref{prog} \ref{aspm}:\ref{v}] \ref{com} \ref{in2} \ref{maz:enc2} \ref{maz:enc3}\\
{} spade \textsubscript{redup1}[\Prog- \Hab:pull:1+dig] \textsubscript{redup2}[\Prog- \Hab:pull:1+dig] -\Com{} =\Pst.\Hab{} =\Asr{} =1\Sg{}\\
\glt `I used to be digging and digging with a spade.'
\z
\z

Verbal parallelism construction also repeats from progressive (position \ref{prog}) to verb root (position \ref{v}). However, unlike total reduplication, each repeated subspan has a different positional root, which is part of the position \ref{v} for verb roots. This construction expresses the distributivity of an action, therefore is only available for dynamic verbs.
In (\ref{ex:parallel}), subspan \ref{prog}-\ref{v} is repeated each followed by +\textit{tsha\3\ff} `sideways' and +\textit{ni\2ɲa\2} `quadrupedal'. Note that combinations of the positional roots show considerable flexibility, reflecting each speaker's expressivity.

\ea \label{ex:parallel}
 \textit{\st{}h\ssn{}ku\2\3 ku\1\ssn{}tu\1tɕu\1\st{}tsĩ\3 he\2 ti\2thu\4\st{}tsha\3 ti\2thu\4ni\2ɲa\2\st{}ko\1\3}\\  %jngu kùndùchùtsīn je titjútsja titjúniñakǒ
\gllll {} h\ssn{}ku\2\3 ku\1\ssn{}tu\1tɕu\1tsĩ\3 he\2\ff=  ti\2\ff- thu\4+tsha\3\ff{} ti\2\ff- thu\4+ni\2ɲa\2 -ko\1\3\\
v: \ref{np1}[ ] \ref{ant} \textsubscript{vpar1}[\ref{prog} \ref{aspm}:\ref{v}] \textsubscript{vpar2}[\ref{prog} \ref{aspm}:\ref{v}] \ref{com}\\
n: \ref{qua} \ref{n} - - - - - -\\
{} one bottle already= \textsubscript{vpar1}[\Prog- \Hab:come.out+\Pos:sideways] \textsubscript{vpar2}[\Prog- \Hab:come.out+\Pos:quadrupedal] -\Com\\
\glt `he is already staggering with a bottle.'
\z

\subsubsection{Nominalization (\ref{focw}-\ref{com}, \ref{focw}-\ref{np2})}\label{sec:d:subspan2} 
Grammatical \textsc{nominalization} in Ayautla Mazatec\footnote{Abstract noun formation by \textit{k\lab{}ha\1}- also targets some verb forms in habitual (or neutral) aspect with corresponding segmental prefix. However, given its limited productivity, I do not discuss here the abstract noun formation.} targets predicates and derives noun- or adverb-like constituents which mean events, participants or circumstantial situations. Syntactically, nominalization may function as arguments or adjuncts in positions \ref{np1} or \ref{np2}, in addition as the optative form in a main clause by insubordination, which is in complementary distribution to imperative \citep[cf.][248, 47-50]{nakamoto20}.

Within the planar structure, nominalization is indicated at two positions: (i) the complementizer/adverbial subordinator \textit{\ssn ka\2 $\sim$ \tlg\2} or the relativizer \textit{\xy i\2 $\sim$ \tlg\2} at the beginning, and (ii) absolute state marker \textit{=\1} (cf. \S\ref{sec-ext2}) at the end.\footnote{Nominalization with \textit{\ssn ka\2} and absolute state \textit{=\1} is not exclusive to verbs; numerals can be nominalized too \citep[329--330]{nakamoto20}.} 
In the example (\ref{ex:nmlzinicial}) below, nominalization \textit{\ssn ka\2 tu\1 t\xy a\2\st\ssn tu\4\1 ki\2tsi\2ka\2\st{}ʔbi\3 \st{}tõ\2\4\1} `that only Antonio distributed money' begins, except for the nominalizer (subordinator here) itself, at position \ref{focw} for focus introducer \textit{tu\1} `only'.

\ea \label{ex:nmlzinicial}
 \textit{tsa\2\st{}be\2\4ʃu\3 \ssn tsʔja\3\2 \ssn ka\2 \textbf{tu\1} t\xy a\2\st\ssn tu\4\1 ki\2tsi\2ka\2\st{}ʔbi\3 \st{}tõ\2\4\1}\\
\gllll {} tsa\2- be\2\4 =ʃu\3\ff{} \ssn tsʔe\3\ff{} =a\2 \ssn ka\2 tu\1 t\xy a\3\ff{}+\ssn tu\4 =\1 ki\2- tsi\2k- a\2ʔbi\3\ff{} tõ\2\4 =\1\\
v: \ref{aspm} \ref{v} \ref{ev3b} \ref{np2}[- -] \ref{np2}[\ref{conj} \ref{focw} \ref{np1}[- -] \ref{aspm} \ref{voi} \ref{v} \ref{np2}[- -]]\\
n: - - - \ref{n} \ref{poss} - - \ref{n} \ref{abs} - - - \ref{n} \ref{abs}\\
{} \Pfv- see:3 \Rep{} [brother:\Sap{} =2\Sg{}] \Sub{} only [Antonio =\Abst{}] \Pfv- \Caus- be.distributed [money =\Abst]\\
\glt `my brother saw that only Antonio distributed money.' 
\z

%tsabexu nts'ia nga tu chandu kitsika'bi ton

Due to the distributional ambiguity of absolute state marker, however, nominalization test is fractured into minimal and maximal interpretations, corresponding to positions \ref{focw}-\ref{com} and \ref{focw}-\ref{np2}, respectively.

\textsc{Minimal nominalization} ends at the final position where the absolute state marker is unambiguously observed at the clause  (and not the noun complex) level. For example, in (\ref{ex:nmlz1}), the nominalization \textit{\ssn ka\2 tõ\2\4\1ʃu\2 ki\4ski\2ne\2} `that it ate money' ends with a verb root (position \ref{v}) followed by an absolute state marker \textit{=\1}. Similarly, in (\ref{ex:nmlz2}), the nominalization \textit{\ssn ka\2 ka\2hbi\2ko\1\3} `when he took it' ends with a comitative \textit{-ko\1\3} (position \ref{com}) followed by an absolute state marker \textit{=\1}. The syllable with the absolute state marker in question is emphasized in boldface. %In (\ref{ex:nmlz1}), the noun phrase headed by the nominalization \textit{\ssn{}ka\2 he\2 thĩ\3\ldots} `that already exists\ldots' has a relativizer \textit{=\tlg\2} and a relative clause \textit{khi\2ja\2ni\2hi\4} `is buried', in addition to an absolute state marker \textit{=\1} at the end. In the following examples, the syllable with the absolute state marker in question is emphasized in boldface.


\ea \label{ex:nmlz1}
 \textit{ta\1\2 \st{}k\lab i\2ʃu\2 \st{}k\lab ha\4\1 \ssn ka\2 \st{}tõ\2\4\1ʃu\2 ki\4ski\2\textbf{\st{}ne\2\1}}\\
\gllll {} ta\1\2 k\lab{}i\2 =ʃu\3\ff{} k\lab ha\1 \ssn ka\2 tõ\2\4 =\1 =ʃu\3\ff{} ki\2s- ki\2ne\2 \textbf{=\1}\\
v: \ref{conj} - - - -[\ref{conj} \ref{np1}[- -] \ref{ev3a} \ref{aspm} \ref{v}] -  \\
n: - \ref{n} - \ref{n} \ref{n}(\Nmlz)[- \ref{n} \ref{abs} - - -] \ref{abs}\\
{} but \Pronom{}3 =\Rep{} matter [\Sub{} [money =\Abst{}] =\Rep{} \Pfv- eat:3] \textbf{=\Abst}\\
\glt `but the matter is that it [the donkey] ate money.'  (180816-002 00:55)
% %formatting
\ex \label{ex:nmlz2}
 \textit{ha\1 ka\2\st{}ʔbja\2\3\1 \ssn ka\2 ka\2hbi\2\textbf{\st{}ko\1\3\1}}\\
\gllll {} ha\1 ka\2- ʔbe\2\3 =a\1 \ssn ka\2 ka\2b- hi\2 -ko\1\3 \textbf{=\1}\\
v: \ref{conj} \ref{aspm} \ref{v} \ref{maz:enc3} \ref{np2}[\ref{conj} \ref{aspm} \ref{v} \ref{com} -]\\
n: - - - - \ref{n}(\Nmlz)[- - - -] \ref{abs}\\
{} well \Pst- see:1 =1\Sg{} [[\Sub{} \Pst- go:3 -\Com{}]  \textbf{=\Abst}]\\
\glt `I saw (him) when he took it.'  \citep[249]{nakamoto20}
\z


In contrast, absolute state marker \textit{=\1} does not occur in the subsequent positions until the post-verbal noun complex (position \ref{np2}). On the one hand, focus marker \textit{=\3\ff} (position \ref{foc2}) has a floating tone /4/, which blocks the occurrence of absolute state marker \citep[248--250]{nakamoto20}. On the other hand, absolute state marker \textit{=\1} does not cooccur with second position clitics and enclitics (positions \ref{tmp2}-\ref{maz:enc3}), as illustrated by (\ref{ex:nmlz5}). The syllable on which the absolute state marker would occur is indicated in boldface.

\ea \label{ex:nmlz5}
 \textit{ni\2\st{}ʃthĩ\2\3\ \ssn ka\2 k\lab he\1\ssn ti\2\textbf{\st{}bɛ\4}}\\
\gllll {} ni\2ʃthĩ\2\3 \ssn ka\2 k\lab-  he\1\ssn ti\2ba\4 =i\\
v: - \Nmlz{}[\ref{conj} \ref{aspm} \ref{v} \ref{maz:enc3}]\\
n: \ref{n} \ref{rel}[- - - -]\\
{} day [\Sub{} \Pot- come =2\Sg]\\
\glt `the day you will come.' \citep[245]{nakamoto20}
\z

This cooccurrence restriction between the absolute state marker \textit{=\1} and the clitics (positions \ref{tmp2}-\ref{maz:enc2}) plausibly has a historical explanation. Given the broader distribution of the absolute state marker \textit{=\1} in nouns, it is safe to attribute its origin to the nominal morphosyntax. Within noun complexes, however, this morpheme does not occur if the noun is possessed, while the possessor is indicated by the nearly identical dependent pronouns used in verbal predicates. I suggest that this parallelism between the nominal template and the verbal template plays a role in blocking the absolute state marker after clitics in the verbal planar structure.

\textsc{Maximal nominalization} includes the post-verbal noun phrase (position \ref{np2}), which is the last position where the absolute state marker \textit{=\1} is found. However, it is indeterminate as to whether the absolute state marker in this position is due to the nominalization, the noun phrase, or both. For example, in (\ref{ex:nmlz4}), the end of the nominalization \textit{\ssn ka\2 he\2 \ssn ti\2ba\4 \xy i\2h\ssn ku\2\3\1} `that the other already came back' coincides with the end of the noun phrase \textit{\xy i\2h\ssn ku\2\3\1} `the other' within it. Note that the final absolute state marker \textit{=\1} is inside the inner bracket if it occurs at the noun phrase level, and outside the inner bracket if it occurs at the clause level.

\ea \label{ex:nmlz4}
 \textit{ʔba\1 ka\2ʔ\ssn ta\2 bi\4ʃu\3 tsa\2\st{}be\2\4 \ssn ka\2 he\2 \ssn ti\2\st{}ba\4 \xy i\2\textbf{\st{}h\ssn ku\2\3\1}}\\
\gllll {} ʔba\1 ka\2ʔ\ssn ta\2 bi\4= =ʃu\3\ff{} tsa\2- be\2\4 \ssn ka\2 he\2\ff= \ssn ti\2ba\4 \xy i\2h\ssn ku\2\3 \textbf{=\1}\\
v: \ref{conj} \ref{np1} \ref{bi2} \ref{ev3a} \ref{aspm} \ref{v} \ref{np2}[\ref{conj} \ref{ant} \ref{aspm}:\ref{v} \ref{np2}[- -]]\\
n: - - - - - - \ref{n}[- - - \ref{n} \ref{abs}]\\
{} and even \Neg= =\Rep{} \Pfv- know:3 [\Sub{} already= \Pfv:come:3 [the.other \textbf{=\Abst}]]\\
\glt `and he didn't even notice when the other already came back.'  (\citealt[132]{sanchez20}, English by SN)
\z

Absolute state marker \textit{=\1} does not occur after the attitudinal particles (position \ref{att}), such as \textit{ ja\2ʔa\2} `well' in example (\ref{ex:nmlz3}). I suggest that this is because the attitudinal particle (position \ref{att}) is found outside the nominalization.

\ea \label{ex:nmlz3}
 \textit{\ssn ka\1tʔa\2 tu\1 ti\2ma\4\ssn ka\2tsa\4\st\ssn ka\tlg\3\2 tsu\2\st{}ʔba\2ʔĩ\2 ja\4\textbf{ʔa\2}}\\
\glll {} \ssn ka\1tʔa\2 tu\1 ti\2\ff- m- a\2\ssn ka\2tsa\4\ssn ka\2 =\3\ff{} =\tlg\2 tsu\2ʔba\2 =ʔĩ\3\ff{} ja\2ʔa\2\\
v: \ref{conj} \ref{focw} \ref{prog} \ref{aspm} \ref{v} \ref{foc2} \ref{np2}[\ref{conj} \ref{aspm}:\ref{v} \ref{in2}] \ref{att}\\
{} because only \Prog- \Hab- run:3 =\Foc{} [\Sub{} \Hab:wander:3 =\Pst.\Hab{}] well\\
\glt `because he only ran when he used to go around.' (180629-002 1:02)
\z

In sum, minimal nominalization includes positions \ref{focw}-\ref{com} and maximal nominalization \ref{focw}-\ref{np2}.


\subsubsection{Coordination (\ref{ant}-\ref{maz:enc3}, \ref{adv1}-\ref{np2})}\label{sec:d:subspan3} 
Coordination is also fractured into minimal and maximal interpretation. \textsc{Minimal coordination} includes all positions which cannot be elided, or if elided, the semantic scope changes. In (\ref{ex:coord}), \textit{ʃthe\3} `garbage' is the only omitted or optional element in the second coordinated constituent. Thus, positions \ref{ant}-\ref{maz:enc3} correspond to the minimal interpretation of coordination test.


\ea \label{ex:coord}
 \textit{tsa\2\st{}ʔbja\2\3\1 \ssn{}ka\2 he\2 ki\4ka\2\st{}kɛ\3 \st{}ʃthe\3 ʔba\1 he\2 ki\4ka\2te\2\st{}tɕe\2}\\
 \glll {} tsa\2- ʔbe\2\3 =a\1 \ssn{}ka\2 he\2\ff= ki\2k- a\2ka\3\ff{} =i ʃthe\3\ff{} ʔba\1 he\2\ff= ki\2k- a\2te\2tɕa\2 =i\\
 v: \ref{aspm} \ref{v} \ref{maz:enc3} \ref{conj} [\ref{ant} \ref{aspm} \ref{v} \ref{maz:enc3} \ref{np2}] \ref{conj} [\ref{ant} \ref{aspm} \ref{v} \ref{maz:enc3}]\\
{} \Pfv- know =1\Sg{} \Sub{} [already= \Pfv- burn:2 =2\Sg{} garbage] and [already= \Pfv- sweep:2 =2\Sg]\\
\glt `I know that you already burned the garbage and swept it.'
\z

\textsc{Maximal coordination}, on the other hand, includes an entire sentence except for the conjunction itself (position \ref{conj}), the attitudinal particles (position \ref{att}) and afterthoughts (position \ref{aft}), i.e.~positions \ref{adv1}-\ref{np2}.

\subsubsection{Summary of subspan repetition}
In sum, both total reduplication and verbal parallelism define positions \ref{prog}-\ref{v} as their repeated subspans; nominalization specifies positions \ref{focw}-\ref{com} minimally and \ref{focw}-\ref{np2} maximally; and coordination test covers positions \ref{ant}-\ref{maz:enc3} minimally, and \ref{adv1}-\ref{np2} maximally.

\subsection{Pauses and fillers (\ref{prog}-\ref{maz:enc3})}\label{sec:d:pause}
\is{pause} %
\is{filler} %
\textsc{Pausability}, or possibility of having a pause, is defined here as the contiguous positions containing the verb core not interrupted by any pausable juncture.

Although this test is difficult to elicit—a speaker of a language may divide a string of speech and pronounce syllable by syllable, even if it has extralinguistic function, such as the clarification of pronunciation—, this test has been used as a constituency diagnostic \citep[cf.][609]{gerdts14}.

In this study, I identify pausable junctures from the transcribed instances of filler \textit{hu\1ni\2} `er' in naturally occurred speech. Specifically, I have observed that pauses with a filler \textit{hu\1ni\2} may occur after a conjunction (position \ref{conj}, example \ref{ex:p1}), a proclitic such as negation \textit{bi\4=} (position \ref{bi2}, example \ref{ex:p2}), a second position clitic in pre-predicate position such as \textit{=hba\4ni\2\3} `at once' (position \ref{tmp1}, example \ref{ex:p1}) or reported information \textit{=ʃu\3\ff} (position \ref{ev3a}, example \ref{ex:p3}), an independent pronoun in patient-like or recipient-like function (position \ref{pron}, example \ref{ex:p4}), or between a person/number enclitic (position \ref{maz:enc3}) and a following noun phrase (position \ref{np2}), as in (\ref{ex:p5}). However, in a sample of 13 short texts with 129 transcribed instances of \textit{hu\1ni\2} `er', none intrudes on the positions \ref{prog}-\ref{maz:enc3}.

\ea 
\ea \label{ex:p1} \textit{ʔba\1, \textbf{hu\1ni\2}, tu\1 khja\2\st{}ʔa\4\3ʃu\3 \st{}h\ssn{}ku\2\3hba\4ni\2\3, \textbf{hu\1ni\2}, ki\4tsi\2\st{}\ssn{}ka\4. \st{}h\ssn{}ku\2hba\4ni\2\3ʃu\3 ha\2\st{}ne\4}\\
 \glll {} ʔba\1, hu\1ni\2, tu\1 khja\2ʔa\4 =\3\ff{} =ʃu\3\ff{} h\ssn{}ku\2\3 =hba\4ni\2\3 hu\1ni\2 ki\4- tsi\2\ssn{}ka\4 h\ssn{}ku\2 =hba\4ni\2\3 =ʃu\3\ff{} ha\2ne\4 \\
 v: \ref{conj} - \ref{focw} \ref{np1} \ref{foc1} \ref{ev3a} \ref{np1} \ref{tmp1} - \ref{aspm} \ref{v} \ref{np1} \ref{tmp1} \ref{ev3a} \ref{aspm}:\ref{v}\\
{} and \Fill{} only when =\Foc{} =\Rep{} one =at.once \Fill{} \Pfv- burst one =at.once =\Rep{} \Pfv:sound\\
\glt `and, er, suddenly one (thunderclap) at once, er, bursted, one roared at once.' (\citealt[138]{sanchez20}, English by SN)


\ex \label{ex:p2} \textit{bi\4, \textbf{hu\1ni\2}, bi\4 \st{}tsha\2\1nu\4\2 \ssn{}tsu\1\st{}ʔba\3\2}\\
 \gllll {} bi\4= hu\1ni\2 bi\4= tsha\2 =\1nu\4\2 \ssn{}tsu\1ʔba\3\ff{} =a\2\\
 v: \ref{bi2} - \ref{bi2} \ref{aspm}:\ref{v} \ref{maz:enc3} \ref{np2}[ ]\\
 n: - - - - - \ref{n} \ref{poss}\\
{} \Neg= \Fill{} \Neg= \Hab:give:1 =1\Sg/2\Pl{} [mouth =1\Sg]\\
\glt `I don't, er, I don't give you my words (lit. my mouth).' (180624-002 15:44)

\ex \label{ex:p3} \textit{ʔba\1 he\2ʃu\3, \textbf{hu\1ni\2}, ʔba\1 he\2ʃu\3 kjo\1 tse\2kʔe\4\st{}\ssn{}tu\2\ssn{}ka\2ni\3{} \ssn{}te\1} \\ %, ʔe\2\ssn{}the\4ʃu\2 na\4xyi\1na\2\ssn{}ta\lg{}\3\1\\
\glll {} ʔba\1 he\2\ff= =ʃu\3\ff{} hu\1ni\2 ʔba\1 he\2\ff= =ʃu\3\ff{} kjo\1 tse\2k- ʔe\4\ssn{}tu\2 =\ssn{}ka\2ɲi\3\ff{} \ssn{}te\1\\ %, ʔe\2\ssn{}the\4ʃu\2 na\4xyi\1na\2\ssn{}ta\lg{}\3\1\\
v: \ref{conj} \ref{ant} \ref{ev3a} - \ref{conj} \ref{ant} \ref{ev3a} \ref{pron} \ref{aspm} \ref{v} \ref{maz:enc1} \ref{att}\\
{} and already= =\Rep{} \Fill{} and already= =\Rep{} there \Pfv- sit:\Pl{} =again thus\\
\glt `and already, er, and they already established themselves there again.' (180811-001-e2 04:08)

\ex \label{ex:p4} \textit{ha\1 \st{}k\lab{}i\2ru\1 nɛ\1ʔɛ\2ni\1\st{}sti\2\3na\1 ʔba\1 ni\1\st{}ma\1\3 thi\2\st{}ʔmi\4re\1 tsa\2 \st{}k\lab{}i\2, \textbf{hu\1ni\2}, ki\2sʔe\2\st{}ne\tlg{}\4\1 pre\2si\2den\2\3\st{}te\1}\\
 \gllll {} ha\1 k\lab{}i\2 =ru\1 nɛ\1ʔɛ\2+ni\1sti\2\3 =na\1 ʔba\1 ni\1ma\1 =\3\ff{} thi\2- m- ʔĩ\4 =re\1 tsa\2 k\lab{}i\2 hu\1ni\2 ki\2- s- ʔe\2ne\4 =\tlg{}\1 pre\2si\2den\2\3te\1\\
 v: \ref{conj} \ref{np1}[ ] \ref{np1}[ ] \ref{np1} \ref{np1} \ref{foc1} \ref{prog} \ref{aspm} \ref{v} \ref{maz:enc3} \ref{conj} \ref{pron} - \ref{aspm} \ref{voi} \ref{v} \ref{maz:enc3} \ref{np2}\\
 n: - \ref{n} \ref{n:evc} \ref{n} \ref{poss} - - - - - - - - - - - - - - -\\
 %adv: - - - - - \ref{a:adv1} \ref{adv1}
{} well [\Pronom{}3 =\Assm{}] [man+child =\Poss1\Sg{}] like.that much =\Foc{} \Prog- \Hab- be.told =3/3 if \Pronom{}3 \Fill{} \Pfv- \Impers- impose =3/3 president\\
\glt `well, I assume that they are telling it to my husband if they had given him, er, the \textit{cargo} of president.' (180630-001 16:32)

\ex \label{ex:p5} \textit{…ʔba\1 \ssn{}ka\2, \ssn{}ka\2 si\1\st{}khĩ\2re\1, \textbf{hu\1ni\2}, \xy{}i\2 he\2 he\2\st{}sun\4\1}\\
 \gllll {} ʔba\1 \ssn{}ka\2 \ssn{}ka\2 si\1+khĩ\2{} =re\1 hu\1ni\2 \xy{}i\2 he\2\ff= he\2sun\4 =\1\\
 v: \ref{conj} \ref{conj} \ref{conj} \ref{aspm}:\ref{v} \ref{maz:enc3} - \ref{np2}[\ref{conj} \ref{ant} \ref{aspm}:\ref{v} -]\\
 n: - - - - - - \ref{n} - - \ref{abs}\\
{} and \Sub{} \Sub{} \Pot:make:3+far =3/3 \Fill{} [\Rel{} already= \Pfv:die:\Pl{} =\Abst{}]\\
\glt `... and (he said) that, that it would keep him away from, er, those who already died.' (180816-002 04:43)

\z
\z

In addition, during my participant observation as an Ayautla Mazatec learner, I have noticed that many Ayautla Mazatec speakers find difficult to follow my utterances if I have any interruption in positions \ref{prog}-\ref{maz:enc3}. %\footnote{Another factor which may be relevant here is intonational catathesis. However, it is outside the scope of this study.}
Therefore, I infer that a pause in these positions yields infelicitous utterances which require additional task of processing.
Hence, the impossibility of having a pause defines a domain which consists of positions \ref{prog}-\ref{maz:enc3}.

\subsection{Stress assignment (\ref{v}-\ref{com}, \ref{prog}-\ref{com})}\label{sec:d:stress}
\is{stress} %
Stress in Ayautla Mazatec is phonetically semi-long with an increased intensity, and is obligatory, culminative and predictably assigned on the right edge (or the end) of the stress domain. In order to determine the left edge (or the beginning) of stress domain, however, test fracturing is applied. According to the positive evidence, all stressable positions from the verb root onward are included (\textsc{minimal stress assignment}). According to the negative evidence, all unstressed positions from the stressed syllable backward until the first unstressed position are included (\textsc{maximal stress assignment}).

\textsc{Minimal stress assignment} includes positions \ref{v}-\ref{com}, which are verb roots or contiguous to verb roots and have the possibility to bear stress. In (\ref{ex:str1}), the final syllable of position \ref{v} is stressed. In (\ref{ex:str2}), position \ref{com} is stressed instead of the final syllable of position \ref{v}. Elements after  position \ref{com} do not shift the stress, as partially illustrated in (\ref{ex:str3}).

\ea
\ea \label{ex:str1} \textit{ba\2\st{}sẽ\4}\\
\glll {} b- a\2sẽ\4\\
v: \ref{aspm} \ref{v}\\
{} \Hab- stand:3\\
\glt `he stands.'
\ex \label{ex:str2} \textit{ba\2se\2\st{}ko\1\3}\\
\glll {} b- a\2sẽ\4 -ko\1\3\\
v: \ref{aspm} \ref{v} \ref{com}\\
{} \Hab- stand:3 -\Com\\
\glt `he helps (lit.~stands with)'
\ex \label{ex:str3} \textit{\ssn{}ku\1 ha\1 ba\2se\2\st{}ko\1\3hĩ\4ni\2\3re\1 je\2he\2}\\
\glll {} \ssn{}ku\1 ha\1 b- a\2sẽ\4 -ko\1\3 =hĩ\4 =ɲi\2\3 =re\1 je\2he\2\\
v: \ref{conj} \ref{conj} \ref{aspm} \ref{v} \ref{com} \ref{ev3b} \ref{maz:enc2} \ref{maz:enc3} \ref{att}\\
{} you.know well \Hab- stand:3 -\Com{} =\Infr{} =\Asr{} =3/3 anyway\\
\glt `well, you know, he should help them anyway.' (181118-002 38:55)

\z
\z

Maximal stress assignment covers positions \ref{prog}-\ref{com}, where the stress is found only in \ref{v}-\ref{com}, i.e.~the domain of minimal stress assignment. Outside this domain, independent pronouns (position \ref{pron}) %and adverbial/quantifier enclitics (position \ref{maz:enc1})
have their own stress, as illustrated in (\ref{ex:str4}).% and (\ref{ex:str5}), respectively.

\ea \label{ex:str4} \textit{bi\4 \st{}k\lab{}i\2 ni\2\st{}k\lab{}ɛ\1\3}\\
\glll {} bi\4= k\lab{}i\2 ni\2 -ko\1\3 =i\\
v: \ref{bi2} \ref{pron} \ref{aspm}:\ref{v} \ref{com} =\ref{maz:enc3}\\
{} \Neg= \Pronom{}3 \Hab:do:2 -\Com{} =2\Sg\\
\glt `don't touch it.'
\z

Aside from the phonetic correlates of duration and intensity, the stressed syllable has several phonotactic traits as its phonological correlates. Specifically, the stressed syllable tends to have more phonological contrasts. When a lexical root is found in unstressed syllables by suffixation or compounding, it tends to undergo denasalization  \citep[110--111]{nakamoto20}, deaspiration \citep[111--113]{nakamoto20}, monosyllabification of disyllabic roots \citep[113--114]{nakamoto20} and tone neutralization \citep[154--161]{nakamoto20}.
\is{phonotactics} %
In this study, however, I do not treat these phonotactic traits as separate constituency tests. These neutralization processes are morphophonological in nature and the same phonotactic traits in stressed syllables may be found outside the minimal stress assignment domain, thus they cannot provide well-defined constituency diagnostics. 

In summary, the domains of stress assignment can be positively defined as positions \ref{v}-\ref{com} and negatively as positions \ref{prog}-\ref{com}.

\subsection{*ɛ.j constraint (\ref{v}-\ref{v}, \ref{nde1}-\ref{in2})}\label{sec:d:Ej}
\is{phonotactics} %
*ɛ.j, or constraint against a sequence of /ɛ/ and /j/ at the syllable boundary, is remedied by alternating (or dissimilating) underlying /ɛ/ to /a/ when such a sequence occurs as a result of morpheme concatenation, i.e.~ɛ > a / \_ j. It is the only segmental constraint across the syllable boundary I have so far identified in Ayautla Mazatec \citep[97--98]{nakamoto20}. \textsc{Minimal *ɛ.j} defines the position \ref{v} as a domain within which this alternation takes place (\ref{ex:Ej1}).

\ea \label{ex:Ej1} \textit{ʔba\2\textbf{\ssn{}tha\4}\st{}ja\2}\\
\glll {} b- ʔa\2\ssn{}thɛ\4+ja\2\\
v: \ref{aspm} \ref{v}\\
{} \Hab- change:3+\Pos:inside\\
\glt `it (state, situation) changes.'
\z

\textsc{Maximal *ɛ.j} can be defined by skipping over the junctures where this alternation cannot take place until one finds its initial and final positions. For example, between the third and the fourth syllables of (\ref{ex:Ej2a}), or between the sixth and the seventh syllables of (\ref{ex:Ej2b}). This domain includes positions \ref{nde1}-\ref{in2}. Note that no morpheme ends with /ɛ/ or begins with /j/ between positions \ref{nde1}-\ref{prog} or \ref{com}-\ref{in2}.

\ea \label{ex:Ej2}
\ea \label{ex:Ej2a} \textit{\ssn{}kɛ\2\st{}ʔɛ\1\textbf{\ssn{}tshɛ\4} ja\2\st{}khã\4}\\
\glll {} \ssn{}kɛ\2ʔɛ\1 =\ssn{}tshɛ\4 j- a\2khã\4\\
v: \ref{np1} \ref{in1} \ref{aspm} \ref{v}\\
{} here =always \Pfv- break\\
\glt `he always broke it here.'
\ex\label{ex:Ej2b} \textit{ki\2tsi\2t\xy{}i\2kũ\2tɛ̃\2\textbf{ʔɛ̃\2\3}je\2he\2na\1}\\
\glll {} ki\2- tsi\2+t\xy{}i\2kũ\2+tɛ̃\2ʔɛ̃\2\3 =je\2he\2 =na\1\\
v: \ref{aspm} \ref{v} \ref{maz:enc1} \ref{maz:enc3}\\
{} \Pfv- do:3+sacred+word(?) =all.\Inan{} =3/1\Sg\\
\glt `he blessed all of them for me.'
\z
\z 

Therefore, *ɛ.j defines \ref{v} and \ref{nde1}-\ref{in2} as its minimal and maximal interpretations.

\subsection{*3.(2)4 constraint (\ref{v}-\ref{v}, \ref{prog}-\ref{foc2})}\label{sec:d:32434}
\citet[154--161]{nakamoto20} described that lexical tones except /1/ and /2/ tend to neutralize in pretonic syllables. Among such instances, neutralization of /3/ before /24/ and /4/ takes place obligatorily within certain domain, i.e.~\textbf{*3.(2)4}. According to positive evidence, \textsc{minimal *3.(2)4} is obligatorily found in compound verbs (position \ref{v}). For example, in (\ref{ex:32434:1}), the underlying /3/ neutralized obligatorily with /2/ before /4/. The syllable which undergoes neutralization is emphasized in boldface.

\ea \label{ex:32434:1} \textit{ba\2\textbf{ne\2}\st{}sũ\4}\\
\glll {} b- a\2ne\3\ff{}+sũ\4\\
v: \ref{aspm} \ref{v}\\
{} \Hab- wash:3+\Pos:above\\
\glt `he washes (the surface of).'
\z

\textsc{Maximal *3.(2)4} can be established in positions \ref{prog}-\ref{foc2} where negative evidence of *3.24 and *3.4 is available. In (\ref{ex:32434:2}), the underlying sequence of /3/ in position \ref{pron} followed by /4/ does not undergo neutralization. In (\ref{ex:32434:3}), the underlying sequence of /3/ followed by /4/ in position \ref{tmp2} does not undergo neutralization.

\ea \label{ex:32434:2} \textit{bi\4 \textbf{\st{}ɲa\3} ja\4\st{}tʔa\2na\3}\\
\glll {} bi\4= ɲa\3\ff{} j- a\4+tʔa\2 =na\3\ff\\
v: \ref{bi2} \ref{pron} \ref{aspm} \ref{v} \ref{maz:enc3}\\
{} \Neg= \Pronom{}1\Incl{} \Pfv- lay:3+\Pos:stuck =3\Incl\\
\glt `he didn't registered us.'
\ex \label{ex:32434:3} \textit{ba\2\textbf{\st{}ne\3}hba\4ni\2\3}\\
\glll {} b- a\2ne\3\ff{} =hba\4ni\2\3\\
v: \ref{aspm} \ref{v} =\ref{tmp2}\\
{} \Hab- wash:3 =at.once\\
\glt `he washes at once.'
\z

Therefore, *3.(2)4 constraint defines position \ref{v}-\ref{v} as its minimal interpretation and positions \ref{prog}-\ref{foc2} as its maximal interpretation.

\subsection{Syllable-internal segmental interactions (\ref{aspm}-\ref{v}, \ref{aspm}-\ref{maz:enc3})}\label{sec:d:syl}
\is{phonotactics} %
\is{syllabification} %
Given that every free form in Ayautla Mazatec begins with a consonant and ends with a vowel \citep[cf.][83--85]{nakamoto20}, the existence of \textsc{syllable-internal segmental interactions} suggests some grade of fusion between morphemes.\footnote{The situation is different with tonal morphemes, because the stem or the host to such tonal morphemes always has its own tone(s) and thus is pronounceable. Therefore, the fusion between the stem or host and the tonal morpheme is a phenomenon local to each juncture.}
Specifically, aspect/mode (position \ref{aspm}), associated motion (position \ref{mot}), voice (position \ref{voi}) and verb root (position \ref{v}) have consonant-initial morphemes, while associated motion (position \ref{mot}), voice (position \ref{voi}), verb roots (position \ref{v}) and pronominal enclitics (position \ref{maz:enc3}) include vowel-initial morphemes. Example (\ref{ex:syll1}) illustrates some syllable-internal segmental interactions around the verb root: habitual prefix is syllabified with andative; causative prefix is syllabified with verb root; and verb root is syllabified with pronominal enclitic.

\ea\label{ex:syll1} \textit{hbi\2tsi\2kʔo\4\st{}ja\2\3\1}\\
\glll {} b- hi\2- tsi\2\ff{}k- ʔo\2\3+ja\2\3 =a\1\\
v: \ref{aspm} \ref{mot} \ref{voi} \ref{v} \ref{maz:enc3}\\
{} \Hab- \Andt:1- \Caus- go.out:1+\Pos:inside =1\Sg\\
\glt `I put out, switch off.'
\z

\textsc{Minimal syllable-internal segmental interactions} can thus be defined as \ref{aspm}-\ref{v}, the span in which all elements are known to show syllable-internal segmental interactions, while \textsc{maximal syllable-internal segmental interactions} covers \ref{aspm}-\ref{maz:enc3}, outside which syllable-internal segmental interactions are not found.


\subsection{Disyllabic sandhi-blocking tone sequences (\ref{prog}-\ref{v}, \ref{prog}-\ref{foc2})}\label{sec:d:sandhi1}
\is{tone sandhi} %
Tone sandhi in Ayautla Mazatec is a phonological process which consists of a progressive association of floating /4/ across syllables (= tone bearing units), /4/ being the highest tone and /1/ the lowest. As a result of tone sandhi, the syllables with underlying /1/, /13/, /2/ or /23/ are generally substituted by /4/. However, the applicability, the obligatoriness, and the output of tone sandhi are subject to tonal and prosodic conditions of the syllable receiving the floating /4/ \citep[171--196]{nakamoto20}.

Among several phonological and non-phonological conditions which block the application of tone sandhi \citep[184--191]{nakamoto20}, \textsc{disyllabic sandhi-blocking tone sequences} constitute one of the tonal and prosodic conditions (the other being `1possible sandhi'', see \S\ref{sec:d:psandhi}). If the syllable which receives the floating /4/ is the first syllable of a /1.24/, /1.4/, /2.24/ or /2.4/ sequence within the positions \ref{prog}-\ref{v}, the application of sandhi is blocked.\footnote{Similar blocking conditions have been reported for other Mazatec varieties with tone sandhi, such as Soyaltepec \citep[63--64]{pikee56} and Chiquihuitlán \citep{nakamoto18}.} For example, in (\ref{ex:sbseqa1}) and (\ref{ex:sbseqa2}), the second syllable of the example is part of the underlying /2.4/ sequence and tone sandhi fails to apply, while in (\ref{ex:sbseqa3}), tone sandhi does apply to the second syllable which is not part of a /2.4/ sequence.

\ea \label{ex:sbseq}
\ea  \label{ex:sbseqa1} \textit{he\2 ti\2tsi\4\st{}the\2}\\
\glll {} he\2\ff= ti\2\ff- tsi\4- the\2\\
v: \ref{ant} \ref{prog} \ref{aspm}:\ref{voi} \ref{v}\\
{} already= \Prog:3- \Hab:\Caus:3- cough\\
\glt `he is already clearing his throat.'
\ex \label{ex:sbseqa2} *he\2 ti\4tsi\4\st{}the\2\\
\ex \label{ex:sbseqa3} \textit{he\2 ti\4ma\2\st{}hɲu\4}\\
\glll {} he\2\ff= ti\2\ff- m- a\2-hɲu\4\\
v: \ref{ant} \ref{prog} \ref{aspm} \ref{v}\\
{} already= \Prog:3- \Hab- \Inch-night\\
\glt `it is already getting dark.'
\z
\z

In contrast, such underlying sequences fail to block tone sandhi if one of the two syllables is found outside the positions \ref{prog}-\ref{foc2}. For example, the second and the third syllables in (\ref{ex:sbseqb1}), in positions \ref{pron} and \ref{aspm}, have an underlying /1.4/ sequence, but it does not block tone sandhi. The same is true in (\ref{ex:sbseqb2}), where the underlying /1.4/ sequence in positions \ref{v} and \ref{tmp2} cannot block the application of tone sandhi.

\ea \label{ex:sbseqb}
\ea \label{ex:sbseqb1} \textit{bi\4ʔĩ\2 \st{}hĩ\4\1 tsu\4\st{}ja\2ni\2\3}\\
\glll {} bi\4= =ʔĩ\3\ff{} hĩ\1 tsu\4+ja\2 =ɲi\2\3\\
v: \ref{bi2} \ref{in1} \ref{pron} \ref{aspm}:\ref{v} \ref{maz:enc2}\\
{} \Neg= =\Pst.\Hab{} \Pronom1\Incl{} \Hab:say:3+\Pos:inside =\Asr\\
\glt `he did not used to explain it to us.'

\ex \label{ex:sbseqb2} \textit{skhe\2\st\xy{}i\4\1hba\4ni\2\3}\\
\glll {} s- khe\3\ff+\xy{}i\1 =hba\4ni\2\3\\
v: \ref{aspm} \ref{v} \ref{tmp2}\\
{} \Pot- pull+piece(?) =at.once\\
\glt `he blows his nose at once.'
\z
\z

\noindent
Therefore, minimal disyllabic sandhi-blocking tone sequences spans \ref{prog}-\ref{v}, while maximal disyllabic sandhi-blocking tone sequences covers \ref{prog}-\ref{foc2}.

%\subsection{Complex sandhi blocking}
\subsection{Obligatory sandhi (\ref{prog}-\ref{maz:enc3})}\label{sec:d:sandhi2}
\is{tone sandhi} %
Tone sandhi in Ayautla Mazatec, or progressive association of a floating /4/, is obligatory within the positions \ref{prog}-\ref{maz:enc3} and is optional outside this domain.
\textsc{Obligatory sandhi} is illustrated in (\ref{ex:obl}). Sandhi from progressive (position \ref{prog}) to habitual and inchoative (positions \ref{aspm} and \ref{voi}) as well as sandhi from verb root (position \ref{v}) to pronominal clitic (\ref{maz:enc3}) are obligatory. If sandhi does not apply in any of these positions, the result is ungrammatical (\ref{ex:obl2}, \ref{ex:obl3}, \ref{ex:obl4}), where the syllables with underapplication of sandhi are highlighted.

\ea \label{ex:obl}
\ea \label{ex:obl1} \textit{ti\2ma\4ni\2\st{}hɲa\2na\4\2}\\
\glll {} ti\2\ff- m- a\2- ni\2hɲa\3\ff{} =na\1\\
v: \ref{prog} \ref{aspm} \ref{voi} \ref{v} \ref{maz:enc3}\\
{} \Prog:3- \Hab- \Inch- be.sleepy =3/1\Sg\\
\glt `I'm getting sleepy.'

\ex \label{ex:obl2} *ti\2\textbf{ma\2}ni\2\st{}hɲa\2na\4\2 \\
\ex \label{ex:obl3} *ti\2ma\4ni\2\st{}hɲa\3\textbf{na\1}\\
\ex \label{ex:obl4} *ti\2\textbf{ma\2}ni\2\st{}hɲa\3\textbf{na\1}\\
\z
\z

However, tone sandhi from position \ref{pron} or to position \ref{np2} is optional, as illustrated in the following examples. In (\ref{ex:opts1}), tone sandhi from the second morpheme (position \ref{pron}) may or may not apply; if applied, the underlying tone /3/ alternates with /2/ \citep[cf.][142--143]{nakamoto20}. We can observe the same in (\ref{ex:opts2}) with the first floating /4/ associated with the third morpheme (position \ref{ev3b}). So far I have not been able to identify what else conditions the application of sandhi outside the domain of obligatory sandhi in Ayautla Mazatec.

\ea 
\ea \label{ex:opts1} \textit{bi\4 \st{}ɲa\3 \textbf{su\1}\st{}ba\1na\3}\\
$\sim$ \textit{bi\4 \st{}ɲa\2 \textbf{su\4}\st{}ba\1na\3}\\
\glll {} bi\4= ɲa\3\ff{} su\1ba\1 =na\3\ff\\
v: \ref{bi2} \ref{pron} \ref{aspm}:\ref{v} \ref{maz:enc3}\\
{} \Neg= \Pronom1\Incl{} \Pot:catch =3/1\Incl\\
\glt `he won't catch us.'

\ex \label{ex:opts2} \textit{ja\2te\2\st{}ɲa\2\3ʃu\3 \textbf{thju\1}na\2\st{}ɲa\2re\4\2}\\
$\sim$ \textit{ja\2te\2\st{}ɲa\2\3ʃu\2 \textbf{thju\4}na\2\st{}ɲa\2re\4\2}\\
\gllll {} j- a\2te\2ɲa\2\3 =ʃu\3\ff{} thju\1na\3ɲa\3\ff{} =re\1\\
v: \ref{aspm} \ref{v} \ref{ev3b} \ref{np2}[ ]\\
n: - - - \ref{n} \ref{poss}\\
{} \Pfv-  sell:3 =\Rep{} [dog =\Poss3]\\
\glt `(reportedly) he sold his dog.' \citep[179]{nakamoto20}
\z
\z

Therefore, the obligatoriness of sandhi defines \ref{prog}-\ref{maz:enc3} as its domain.


\subsection{Possible sandhi (\ref{prog}-\ref{maz:enc3}, \ref{adv1}-\ref{aft})} \label{sec:d:psandhi}
In addition to the disyllabic sandhi-blocking tone sequences (\S\ref{sec:d:sandhi1}), the other prosodic condition which impedes the application of tone sandhi is that of \textsc{possible sandhi} domain.
\citet[176--177]{nakamoto20} reported that tone sandhi in Ayautla Mazatec is blocked across the boundary of coordination and verbal parallelism construction. For example, in (\ref{ex:psdh1}), tone sandhi is blocked between asyndetically coordinated verbs, i.e.~\textit{k\lab{}ʔi\3\ff} `he will drink' and \textit{ski\1\ssn{}ta\2\st{}ja\4} `he will cry', as well as \textit{se\3\ff} `he will sing' and \textit{ste\1} `he will dance'; in (\ref{ex:psdh2}), tone sandhi is blocked between two verb forms in a verbal parallelism construction (cf.~\S\ref{sec:d:subspan1}), i.e.~\textit{thu\2tsha\3\ff} and \textit{thu\2ni\2ɲa\2}. In both cases, the rest of phonological conditions (tone and stress) for tone sandhi to be realized are satisfied \citep[cf.][180--184]{nakamoto20}; the syllable which would receive a floating tone /4/ is highlighted.

\ea \label{ex:psdh}
\ea \label{ex:psdh1} \textit{\st{}k\lab{}ʔi\3, \textbf{ski\1}\ssn{}ta\2\st{}ja\4, \st{}se\3, \textbf{\st{}ste\1}, k\lab{}ha\1hɲa\2\st{}hbe\4}\\
\glll {} k\lab{}ʔi\3\ff{} ski\1\ssn{}ta\2ja\4 se\3\ff{} ste\1 k\lab{}ha\1hɲa\2+hbe\4\\
v: [\ref{aspm}:\ref{v}] [\ref{aspm}:\ref{v}] [\ref{aspm}:\ref{v}] [\ref{aspm}:\ref{v}] [\ref{aspm}:\ref{v}]\\
{} [\Pot:drink] [\Pot:cry] [\Pot:sing] [\Pot:dance] [\Pot:lie.down+\Pos:asleep]\\
\glt `he will drink, cry, sing, dance and sleep.' \citep[176]{nakamoto20}\\
\ex\label{ex:psdh2} \textit{thu\2\st{}tsha\3 \textbf{thu\2}ni\2\st{}ɲa\2}\\
\glll {} thu\2+tsha\3\ff{} thu\2+ni\2ɲa\2\\
v: \textsubscript{vpar1}[\ref{aspm}:\ref{v}] \textsubscript{vpar2}[\ref{aspm}:\ref{v}]\\
{} \textsubscript{vpar1}[\Hab:come.out+\Pos:sideways] \textsubscript{vpar2}[\Hab:come.out+\Pos:quadrupedal]\\
\glt `he staggers.' \citep[177]{nakamoto20}
\z
\z

This fact can be interpreted as follows: tone sandhi cannot extend over two planar structures. However, Ayautla Mazatec has additional prosodic restrictions as to its application: within the verbal planar structure, even though the rest of phonological conditions (i.e.~tone and stress) are met, connectors (position \ref{conj}), focus introducers (polar question, `even', `only'; position \ref{focw}) and adverbs (positions \ref{np1}, \ref{pron}, \ref{np2}) do not undergo sandhi. For example, in (\ref{ex:psdh3}), even though \textit{tu\1} `only' satisfies the rest of tonal and stress-related conditions, it never undergoes tone sandhi.

\ea \label{ex:psdh3} \textit{he\2 ti\2ma\4\st{}ba\2ʔĩ\2na\4\2, nɛ\1ʔɛ\3, \st{}ɲa\4 \textbf{\st{}tu\1} k\lab{}hɛ\2\st{}ʔɛ\1ni\2\3}\\
\glll {} he\2\ff{}= ti\2\ff- m- a\4-ba\2 =ʔĩ\3\ff{} =na\1 nɛ\1ʔɛ\3\ff{}  ɲa\4\ff{} tu\1 k\lab{}hɛ\2ʔɛ\1 =ni\2\3\\
v: \ref{ant} \ref{prog} \ref{aspm} \ref{v} \ref{in2} \ref{maz:enc3} \ref{aft} \ref{np1} \ref{np1} \ref{aspm}:\ref{v} \ref{maz:enc2}\\
{} already= \Prog- \Hab{}- \Inch{}-sad =\Pst.\Hab{} =3/1\Sg{} sir where only be.from =\Asr{}\\
\glt `I was already worrying, sir, where the hell do you come from?' (\citealt[141]{sanchez20}, English by SN)\\
\z

Since the positions without restrictions on applying tone sandhi are discontinuous, the possible sandhi test is fractured into maximal and minimal interpretations: \textsc{minimal possible sandhi} spans positions \ref{prog}-\ref{maz:enc3}—where no position includes morphemes which block tone sandhi not by underlying tone or stress—, while \textsc{maximal possible sandhi} spans positions \ref{adv1}-\ref{aft} where all positions without such prosodic restrictions are included.


\section{Summary and discussions}\label{sec-concl}
\figref{fig:conv_pooled} summarizes the constituency diagnostics in Ayautla Mazatec described in this study, sorted by domain size. Looking at the domains, the span \ref{prog}-\ref{maz:enc3} (``Layer 9'' in the Figure) has the highest number of convergences of 7, followed by \ref{prog}-\ref{v} (``4'') with 5 convergences, followed by \ref{v} (``1'') with 3 convergences, and \ref{prog}-\ref{com} (``5''), \ref{prog}-\ref{foc2} (``6'') as well as \ref{ant}-\ref{maz:enc3} (``11'') with 2 convergences.

Looking at the individual positions, progressive (position \ref{prog}) is the initial position for 16 diagnostics. As for the final position, the bound pronouns (position \ref{maz:enc3}) has the highest convergence rate with 10 diagnostics, followed by the verb root (position \ref{v}) with 9 diagnostics.

\begin{figure}
    \caption{Convergence of domains}
    \label{fig:conv_pooled}
    \includegraphics[width=\textwidth]{figures/mazatec_pooled.png}
\end{figure}


\hspace*{-1.1pt}A closer look at individual diagnostics reveals that the convergences are mainly observed among morphosyntactic diagnostics (see also figures \ref{fig:conv_ms} and \ref{fig:conv_phon} below for convergence of morphosyntactic and phonological domains, respectively). For example, out of the 5 diagnostics which converge at the span \ref{prog}-\ref{v}, 4 are morphosyntactic diagnostics while only one of them is phonological; in the same vein, 4 out of the 7 diagnostics at the span \ref{prog}-\ref{maz:enc3} are morphosyntactic diagnostics while two are phonological and the other concerns the pause. Therefore, at least based on morphosyntactic criteria, we can recognize two strong candidates for words or phrases. Contrasting these candidates with my working concept of affix-clitic distinction, the span \ref{prog}-\ref{v} differs from it in excluding the comitative (position \ref{com}).\footnote{I included comitative in my working concept of word, because it is inside the stress assignment domain (see \S\ref{sec:d:stress}).}

\begin{figure}
    \caption{Convergence of morphosyntactic domains}
    \label{fig:conv_ms}
    \includegraphics[width=\textwidth]{figures/mazatec_ms_plot.png}
\end{figure}

\begin{figure}
    \caption{Convergence of phonological domains}
    \label{fig:conv_phon}
    \includegraphics[width=\textwidth]{figures/mazatec_phon_plot.png}
\end{figure}

However, phonological diagnostics (cf.~\figref{fig:conv_phon}) tend not to converge with other phonological diagnostics in Ayautla Mazatec.  Out of 13 diagnostics, we have convergences of at most two phonological constituency diagnostics, namely, minimal *ɛ.j constraint and that of minimal *3.(2)4 constraints at position \ref{v}; maximal *3.(2)4 constraint and maximal sandhi-blocking tone sequences at positions \ref{prog}-\ref{foc2}; and obligatory sandhi and minimal possible sandhi at positions \ref{prog}-\ref{maz:enc3}. It is partly due to my analytical decision of not treating the correlates of stress as separate constituency diagnostics (see \S\ref{sec:d:stress}). It is also partly due to the fact that some positions do not have adequate phonological context for a given diagnostic. For example, the focus marker \textit{=\3\ff} (positions \ref{foc1}, \ref{foc2}) consists of tones and cannot provide any positive evidence for segmental processes; comitative \textit{-ko\1\3} is the only morpheme in position \ref{com} and is not amenable to some tonal tests.\footnote{A reviewer suggested ignoring positions for tonal morphemes, which would correspond to focus marker (positions \ref{foc1} and \ref{foc2}), some of the aspect/mode markers (position \ref{aspm}) and some of the person/number markers (position \ref{maz:enc3}), as well as absolute state marker in noun complexes (position \ref{abs}). However, tones in Ayautla Mazatec are basically concatenated in linear order, such as /1/ + /3/ + /1/ >{} /131/ and /3/ + /1/ + /3/ >{} /313/ \citep[cf.][196--207]{nakamoto20}. Therefore, it is important to represent the linear order of these morphemes.}

Diachronically, these variations of prosodic constituents in one or two positions, namely, between positions \ref{prog} and \ref{aspm} and between positions \ref{v} and \ref{com}, seems to be accounted for by recent grammaticalizations.
Both progressive (position \ref{prog}) and comitative (position \ref{com}) have their etymologies identifiable outside the verb complex, namely, posture verbs and preposition `with', respectively.
A possible interpretation is to see from aspect/mood (position \ref{aspm}) to verb root(s) (position \ref{v}) as a historically stable and more established constituent, and to see the prosodically indeterminate status of progressive (position \ref{prog}) and comitative (position \ref{com}) as a result of their recent grammaticalization on the way to cohere with the inner constituent.

Synchronically, however, Ayautla Mazatec situation supports the position of \citet[704]{Schiering2010} who state that `prosodic domains are conceived of as language-particular, intrinsic and highly specific properties of individual phonological rules or constraints'. Therefore, as same as in Tibeto-Burman language Limbu studied in \citet{Schiering2010}, we cannot accommodate Ayautla Mazatec prosodic constituents into some allegedly universal hierarchy of phonological stem, word or phrase.

In conclusion, Ayautla Mazatec verbal predicates show six domains where two or more constituency diagnostics converge. The convergences are mainly observed among morphosyntactic diagnostics, and phonological domains tend not to converge in this language. This result supports the non-universality of prosodic domains suggested by \citet{Schiering2010} and, due to the lack of a strong candidate for a phonological word, is against the word bisection thesis advocated by \citet{dixonaikhenvald02}.


\section*{Acknowledgements}
Ayautla Mazatec speakers, especially Josefina Sánchez; Hiroto Uchihara, Adam J.~R.~Tallman, Sandra Auderset, Naonori Nagaya, and an anonymous reviewer for their comments; Sandra Auderset for her help in plotting constituency diagnostics; errors and shortcomings are exclusively mine.

\printglossary

\printbibliography[heading=subbibliography,notkeyword=this]

\end{document}
