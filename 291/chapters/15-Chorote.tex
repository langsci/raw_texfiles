\documentclass[output=paper]{langscibook}
\ChapterDOI{10.5281/zenodo.13208568}
\author{Javier J. Carol\orcid{}\affiliation{University of Buenos Aires}}
\title{Wordhood in Chorote (Mataguayan)}
\abstract{This chapter discusses the results of applying constituency tests to Chorote (Iyojwa’aja’ or “Ribereño” variety, ISO code: crt), a Mataguayan language spoken in Argentina and Paraguay. The outcomes are interesting for a number of reasons.\par Firstly, Chorote has the highest number of positions of all the languages surveyed in this volume: 46 positions. There are several reasons for this: the high number of categories expressed (lexical arguments/adjuncts, personal clitics, agreement and TAME markers, adverbs, applicatives), the strict ordering of elements such as adverbs or applicatives, which forces one to assign a distinct position to each of them, and the fact that many morphemes can appear in multiple positions in the template.\par  Secondly, no clear wordhood candidate emerges from the application of constituency diagnostics. Only one subspan reaches the convergece of three diagnostics, and four or perhaps five more spans the convergence of two diagnostics each. Chorote, thus, provides another example of a language where language-internal evidence suggests no quantal shift from word (morphotactic) to phrasal (syntactic) structure, but rather a smooth cline, morphology-like to syntax-like combination.\par  Thirdly, Chorote makes apparent some methodological problems in the application of the diagnostics based on ciscategorial selection. A general issue is that it is difficult to define a cross-linguistically valid notion of transcategoriality given the presence of mixed constructions, i.e. those that cannot be neatly categorized as verbal or nonverbal predication. The predicate in Chorote can be syntactically headed not only by verbs but also by Ns, NPs, DPs, pronouns, and negation, which take then most of the usually ``verbal" markers. Furthermore, NPs and DPs can take some of the ``verbal" TAME markers even when they function as arguments. All these facts pose questions regarding how ciscategoriality should be defined as a comparative concept, since it is not clear whether it should be defined with respect to verbs or to predicates in general.}
\IfFileExists{../localcommands.tex}{
  \addbibresource{../localbibliography.bib}
  \usepackage{langsci-optional}
\usepackage{langsci-gb4e}
\usepackage{langsci-lgr}

\usepackage{listings}
\lstset{basicstyle=\ttfamily,tabsize=2,breaklines=true}

%added by author
% \usepackage{tipa}
\usepackage{multirow}
\graphicspath{{figures/}}
\usepackage{langsci-branding}

  
\newcommand{\sent}{\enumsentence}
\newcommand{\sents}{\eenumsentence}
\let\citeasnoun\citet

\renewcommand{\lsCoverTitleFont}[1]{\sffamily\addfontfeatures{Scale=MatchUppercase}\fontsize{44pt}{16mm}\selectfont #1}
   
  %% hyphenation points for line breaks
%% Normally, automatic hyphenation in LaTeX is very good
%% If a word is mis-hyphenated, add it to this file
%%
%% add information to TeX file before \begin{document} with:
%% %% hyphenation points for line breaks
%% Normally, automatic hyphenation in LaTeX is very good
%% If a word is mis-hyphenated, add it to this file
%%
%% add information to TeX file before \begin{document} with:
%% %% hyphenation points for line breaks
%% Normally, automatic hyphenation in LaTeX is very good
%% If a word is mis-hyphenated, add it to this file
%%
%% add information to TeX file before \begin{document} with:
%% \include{localhyphenation}
\hyphenation{
affri-ca-te
affri-ca-tes
an-no-tated
com-ple-ments
com-po-si-tio-na-li-ty
non-com-po-si-tio-na-li-ty
Gon-zá-lez
out-side
Ri-chárd
se-man-tics
STREU-SLE
Tie-de-mann
}
\hyphenation{
affri-ca-te
affri-ca-tes
an-no-tated
com-ple-ments
com-po-si-tio-na-li-ty
non-com-po-si-tio-na-li-ty
Gon-zá-lez
out-side
Ri-chárd
se-man-tics
STREU-SLE
Tie-de-mann
}
\hyphenation{
affri-ca-te
affri-ca-tes
an-no-tated
com-ple-ments
com-po-si-tio-na-li-ty
non-com-po-si-tio-na-li-ty
Gon-zá-lez
out-side
Ri-chárd
se-man-tics
STREU-SLE
Tie-de-mann
} 
  \togglepaper[15]%%chapternumber
}{}

\begin{document}
\maketitle 
%\shorttitlerunninghead{}%%use this for an abridged title in the page headers

\section{Introduction}
\largerpage
This chapter describes and discusses the application of a set of constituency tests to the verb complex of Chorote, a Mataguayan language spoken in the Argentine and Paraguayan Chaco. The data come from extensive original fieldwork (over 120 days in yearly trips from 2005 to 2011) and text corpora, namely the collection of 33 short stories compiled in \citet{DraysonGomez2000}, a New Testament translation into \citet{NewTestamentTranslationintoChorote1997}, cited by the corresponding verses, and a few other scattered materials.

Chorote is interesting for typological issues in constituency for a number of reasons. First, it has more positions than any other language surveyed this volume: \ref{chovpos:heavyDP46} positions. The number of categories expressed (lexical arguments or adjuncts, pronominal clitics, agreement and TAME markers, adverbs, and a number of applicatives) is relatively high. Furthermore, many of them display a strict order with respect to each other, which forces one to assign a distinct position to each of them. But this alone does not suffice to explain the number of positions in the template. Many morphemes can occur in many different positions in the template. Some TAME markers, for instance, may occur bound not only to the verb (or non-verbal predicate) but also to other word-classes or phrases. Moreover, there is a critical distributional distinction between DPs headed by demonstratives, and NPs (or DPs headed by possessives). The former appear to the right of the predicate that selects them — verb, noun or adposition/applicative, and the latter to the left. This results in a  duplication of many positions dedicated to verbal arguments, and also those of the applicatives/adpositions (these are a single lexical class that attach to the verb (``applicatives'') or to the nominal (``adpositions'')).

Secondly, no clear wordhood candidate emerges from the application of the constituency tests used in this chapter. The diagnostics discussed here are both morphosyntactic (free occurrence, non-interruption, non-permutability, subspan repetition, deviation from biuniqueness) and phonological (accent, \textit{yod}-insertion, palatalization). All but one are fractured into two or more subtests. All in all, the highest convergence level is two diagnostic subtests per layer, which is reached by five or perhaps six subpans. There is significant convergence in the left edge of the domain containing the predicate: 14 subtests converge on the position of the verbal prefixes and a further six subtests on the verbal stem or predicate. However, there is much less convergence on the right edge. The highest convergence is again three subtests, reached by three or maybe four positions. But if we do not count subtests of the same diagnostic, convergence is reduced to two diagnostics in two of those positions. Chorote thus provides another example of a language where language-internal evidence suggests no quantal shift from word (morphotactic) to phrasal (syntactic) structure, but rather a relatively smooth cline from morphology-like to syntax-like combination \citep{Tallman2021}.

Thirdly, Chorote reveals some methodological problems in the application of the diagnostics based on ciscategorial selection. A general issue is that it is difficult to define a cross-linguistically valid notion of transcategoriality given the presence of mixed constructions, i.e. those that cannot be neatly categorized as verbal or nonverbal predication.\footnote{I would like to thank Adam  Tallman for pointing this issue out to me.} The ciscategoriality diagnostic for verbs can be formulated as follows: if a morpheme can only co-occur with verbal bases but not with other word classes, then it belongs to the verb word, i.e. it is ciscategorial to it. But the predicate in Chorote can be syntactically headed, not only by verbs, but also by nouns (N), noun phrases (NP), determiner phrases (DP), pronouns, and negation, which then take most of the usually ``verbal'' markers. Furthermore, NPs and DPs can take some of the ``verbal'' TAME markers even when the former function as arguments. All these facts pose questions regarding how ciscategoriality should be defined as a typological variable (or perhaps as a comparative concept), since it is not clear whether it should be defined with respect to verbs or with respect to predicates in general.

The rest of this chapter is organized as follows: §\ref{cho:sec:languageandspeakers} gives information on the language and its speakers; §\ref{bkm:Ref88847287} presents the predicate planar structure, §\ref{bkm:Ref82962135} presents morphosyntactic wordhood diagnostics except for those related to ciscategoriality, §\ref{bkm:Ref82962099} presents phonological diagnostics, §\ref{bkm:Ref88847156} presents diagnostics based on ciscategorial selection and discusses some implications of the Chorote facts, and §\ref{bkm:Ref88847346} concludes.

\largerpage
\section{The language and its speakers}
\label{cho:sec:languageandspeakers}
Chorote is a Mataguayan language spoken by about 2000 people in Argentina (Salta Province) and about 500 in Paraguay (Boquerón county), where it is known as Manjúi. The language is fairly vital, and children learn it in both countries, although much less in communities surrounding the cities (Tartagal in Argentina, Mariscal Estigarribia in Paraguay).\footnote{The 2010 census in Argentina gives the number of 2270 people who recognize themselves as Chorote \citep{INDEC2012}. The \textit{Encuesta Complementaria de Pueblos Indígenas} [Complementary Poll on Indigenous Peoples] conducted in 2004-2005 documented 1700 Chorote speakers under a total of 1768 participants \citep{INDEC2005}. The 2012 census in Paraguay gives the number of 582 Manjúi \citep{DGEEC2012}. Based on my own personal field experience, I would say it is reasonable to assume that most or almost all those who recognize themselves as Manjúi speak the language and that probably a majority of them are monolingual.} The family is often called ``Matakoan'' in English-speaking literature, according to the convention of naming the family by adding \textit{-an} to the name of the better-known language. But since \textit{Matako} is considered pejorative, the Spanish-speaking literature prefers \textit{mataguayo} (Mataguayan). This chapter focuses on the riverside or \textit{Iyojwa'aja'} variety (ISO code: crt), spoken in Argentina in communities by the Pilcomayo river and surrounding the city of Tartagal.

The language has a complex phonology. It has six phonological vowels; /a e i o u a*/. The phoneme /a*/ (or rather /ɑ/) is realized as /e/ after a palatal(ized) phone. In the same environment, /e/ is realized as /i/ and merges with phonological /a/ elsewhere. When no palatal(ized) phone precedes it, stressed /i u/ vowels lower to mid vowels, but do not merge with phonological /e, o/, which in turn are open in such environment. The surface contrast is thus roughly [e o] versus [ɛ ɔ]. In the notation used here, lowered phonological /i u/ are transcribed <\textit{ẹ ọ}>; the practical spelling used in the communities, including in educational and religious texts does not distinguish them from phonological /e o/, using <\textit{e o}> in either case.

Plain consonants are /p t kʲ k hw hl l s h m n w j Ɂ/ (<\textit{p t ky k jw jl l s j n w y '}>). There is a series of laryngealized consonants /p' t' kʲ' k' ts' Ɂm Ɂn Ɂl Ɂw Ɂj/ whose phonological status is debatable, at least for non{}-stop sounds (\citealt{Carol2014fon}, \citealt{GutierezNercesian2021}). I transcribe the laryngealized consonants as <C'> for stops and as <'C> for sonorants. A widespread process of progressive palatalization creates palatalized allophones for almost any consonant, including laryngealized ones. They can mostly be regarded as a sequence of C plus a palatal glide, which I transcribe as <y>. Palatalized phones are accordingly transcribed as <Cy> (although recall that velars <\textit{ky, k'y}> have phonemic status; they can have in turn palatalized counterparts, which are transcribed <\textit{sy, ts'y}>). To avoid confusion, from now on I will use the orthographic conventions even in phonological representations, e.g. /y/ represents a palatal glide, except for the fact that I will use /h/ and /hC/ to correspond to orthographic \textit{j} and \textit{jC} respectively.

A sequence of a consonant and a glottal stop produces a laryngealized consonant, e.g. \textit{y+'ut} \MVRightarrow{} \textit{'yut} `(s)he puts in'; a sonorant plus a voiceless laryngeal (transcribed as <\textit{j}>) gives <\textit{jC}> e.g. \textit{in+jetik} \MVRightarrow{} \textit{ijnetik} `someone's head'. In these cases, the corresponding glosses are separated by a colon, e.g. 3:put for \textit{'yut}. A voiceless laryngeal is lost after an obstruent. Before a pause, any sonorant is laryngealized: for vowels, a final <'> is added, and for consonants the stop is added before the consonant, e.g. \textit{jlam} `(s)he/it' \MVRightarrow{} \textit{jla'm} /\_\#\#; sometimes an ``echo vowel'' is also inserted, e.g. \textit{jla'am}.

Syllable structure is CV(C). To avoid onsetless sylables, /y/ is inserted between a suffix/enclitic that begins with a vowel and a base that ends in a vowel, and a glottal stop  <'> is inserted elsewhere. This will be described in more detail in §\ref{bkm:Ref89781621}. Onset position can only be filled by one plain or laryngealized consonant (or by clusters of C and glottal stop if these are not considered single phonemes), including their palatalized allophones. Neither palatalized or phonemically laryngealized consonants nor /w, h/ occur in coda position. However, laryngealized consonants can occur in final position because of the phrase-level process mentioned above that laryngalizes any sonorant before pause. The language displays stress marking, realized as an increase of intensity in the stressed syllable. The stress falls in the first or second syllable of the stem and is thus not fully predictable. In the practical spelling used here, an acute accent is used to mark any stress that does not fall on the first syllable of the stem, and any orthographic word has no more than one stress. Raising and neutralization processes often occur in unstressed vowels.

The practical spelling adopted here follows the one used in the speech communities and introduced by missionaries, but differs in the following respects: 1) the orthographic word can only have one accent, so I write \textit{i{}-}\textbf{\textit{ni}} \textit{'}\textbf{\textit{we}}\textit{nis} `they see each other' or \textit{ta-}\textbf{\textit{ka}} \textbf{\textit{le}}\textit{ja'n} `they wash (antipassive)' (stressed syllables in boldface) instead of \textit{ini'wenis}, \textit{takaleja'n} as in the missionaries' spelling; 2) I transcribe the palatalized consonants as \textit{Cy} instead of \textit{Ci} before vowels, e.g. \textit{kya} instead of \textit{kia} `that (moving away or disappeared)', 3) I use graphical accents when they don't fall on the first syllable of the stem (missionaries' spelling uses no accents), and 4) I distinguish \textit{ẹ}, \textit{ọ} from \textit{e}, \textit{o}.

The language has an active-inactive alignment, and phrases can be both head-initial and head-final, depending on the ``weight'' of the complement - complements headed by a demonstrative are ``heavy'' and follow the phrase head, while other complements are ``light'' and precede it. This will be dealt with in detail in §\ref{bkm:Ref73306870}.

In glosses, no distinction is made between clitics and affixes, and morphemes are always separated by ``-'', i.e. ``='' is not used. The main reason is that the distinction between affix and clitic is mainly based on the notion of ``word'', which is what this chapter seeks to scrutinize. As will be seen, it is not easy to determine what a word is in Chorote. Furthermore, the distinction often relies on assumptions related to cis-/transcategoriality (affixes have categorial preferences, clitics do not) which are complicated to apply in Chorote compared to other languages.

\section{The predicate planar structure} 
\label{bkm:Ref88847287}
The predicate planar structure of Iyojwa'aja' Chorote has 46 positions (see \tabref{tab:chor:key:1}).

\begin{longtable}{Slp{9cm}}
    \caption{The predicate planar structure}
    \label{tab:chor:key:1}
    \centering
    \endfirsthead 
    \endhead
    \lsptoprule
    \multicolumn{1}{l}{\bfseries Position} & {\bfseries Type} & {\bfseries Elements}\\ \midrule
    \label{chovpos:conjunct1}   & Zone & Conjunction or conjunctive locution; interrogative \textit{ma}(?)\\
    \label{chovpos:top2} & Zone & Topic: DP \{A, S\}, Adverb, Adverbial clause\\
    \label{chovpos:foc3}  & Zone & Focalized adverb or AdvP\\
    \label{chovpos:complementizer4} & Slot & Complementizers \textit{ti}, \textit{ka} \\
    \label{chovpos:lightDP5}     & Zone & N', N, light DP, Pronoun/Adverb\\
    \label{chovpos:lightDP6}     & Zone & N', N, light DP, Pronoun/Adverb\\
    \label{chovpos:mirativereportative7}     & Slot & Mirative \textit{-p'an}; reportative \textit{-jen} \\
    \label{chovpos:prospectiveja8}     & Slot & Prospective \textit{ja}\\
    \label{chovpos:incompletive9}     & Slot & Incompletive \textit{-ta(j)}\\
    \label{chovpos:negationje10}     & Slot & Negation \textit{je}\\
    \label{chovpos:indirectevidentialmirative11}     & Slot & Indirect evidential \textit{-t'i}; mirative \textit{-p'an}\\
    \label{chovpos:interrogativemi12}     & Slot & Interrogative \textit{-mi}\\
    \label{chovpos:kyak13}     & Slot & Demonstrative \textit{kyak} `that (way)'\\
    \label{chovpos:activenonactive14}     & Slot & Cross referencing active/nonactive markers\\
    \label{chovpos:rflxantipassive15}     & Slot & Reflexive/reciprocal \textit{ni(n)}; antipassive \textit{ka}\\
    \label{chovpos:predbase16}     & \textbf{Slot} & \textbf{Predicate} \textbf{base}\\
    \label{chovpos:ptcp17}     & Slot & Participle \textit{{}-k}\\
    \label{chovpos:antipassvblz18}     & Slot & Causative; antipassive; verbalizer\\
    \label{chovpos:concord19}     & Slot & Concord 1\\
    \label{chovpos:concord20}     & Slot & Concord 2\\
    \label{chovpos:perdurative21}     & Slot & Perdurative \textit{-jli}\\
    \label{chovpos:momentary22}     & Slot & Momentary \textit{-a}\\
    \label{chovpos:irrealis23}     & Slot & Irrealis \textit{-a}\\
    \label{chovpos:reportativejen24}     & Slot & Reportative \textit{-jen}\\
    \label{chovpos:indirectevidential25}     & Slot & Indirect evidential \textit{-t'i}\\
    \label{chovpos:mirative26}     & Slot & Mirative \textit{-p'an}\\
    \label{chovpos:incompletive27}     & Slot & Incompletive \textit{-ta(j)}\\
    \label{chovpos:interrogativemi28}     & Slot & Interrogative \textit{-mi}\\
    \label{chovpos:concord29}     & Slot & Concord (3\textsc{pl}, A/S) \textit{-is}\\
    \label{chovpos:applorientation30}     & Slot & Applicative: orientation \textit{-a(j)}\\
    \label{chovpos:applinstrumental31}     & Slot & Applicative: instrumental \textit{-e(j)}\\
    \label{chovpos:distaley32}     & Slot & Applicative: distal \textit{-ey}\\
    \label{chovpos:appllocative33}     & Slot & Applicative: locatives \textit{-jiy}, \textit{-jam}\\
    \label{chovpos:applipcntloc34}     & Slot & Applicative: punctual locative \textit{-'e}, distributive/comitative \textit{-k'i}, and possibly others\\
    \label{chovpos:oblique35}     & Slot & Oblique marker (only realized if position \ref{chovpos:appl36} is filled)\\
    \label{chovpos:appl36}     & Slot & Applicative\\
    \label{chovpos:pluractionaldownwards37}     & Slot & Pluractional/downwards \textit{-jen}\\
    \label{chovpos:pluractionaliterative38}     & Slot & Pluractional/iterative \textit{-'ni(j)}\\
    \label{chovpos:remotepastperfect39}     & Slot & Remote past \textit{-pe(j)}; perfect -\textit{('V…)je(j)}\\
    \label{chovpos:tempaspdisc40}     & Zone & Temporal, aspectual and discourse particles\\
    \label{chovpos:locatives41}     & Slot & Locatives\\
    \label{chovpos:lightDP42}     & Zone & N, N', light DP\\
    \label{chovpos:lightDP43}     & Zone & N, N', light DP\\
    \label{chovpos:adpositionsappl44}     & Slot & Adpositions-applicatives\\
    \label{chovpos:heavyDP45}     & Zone & Heavy DP (A, S, O, Obl, Possessor of \label{chovpos:lightDP46})\\
    \label{chovpos:heavyDP46}     & Zone & Heavy DP (A, S, O, Obl); S\\
\lspbottomrule
\end{longtable}


The analysis whereby Chorote has 46 positions is conservative for two reasons. First, at least some of the particles in position \ref{chovpos:tempaspdisc40} might be rigidly ordered with respect to one another, which might require that we split this position into many. These particles include \textit{pet} `please' (among other meanings), \textit{'ne} `then', -\textit{na'a} `later', among many others. Some appear always in the same order in texts, though no exhaustive elicitation work has been conducted to test whether this order can be altered. 

Secondly, texts produced by elderly speakers occasionally show a few morphemes in positions not recorded in \tabref{tab:chor:key:1}; this will be shown in §\ref{sec:promiscuouselements}. My main consultant, >60 years old when the fieldwork was conducted (2007-2011), accepted these forms in elicitation, though he did not produce them spontaneously, nor did other consultants represented in my corpus. Therefore, I decided to build a table with a ``standard'' version of the language and exclude these cases.

On the other hand, the number could be reduced if we assume some internal constituent. In effect, one could assume an initial host position (e.g. a complementizer head) and a clitic cluster bound to it. Under such an analysis, the number of positions that hosts fronted elements (which is especially high in the Chorote varieties discussed in §\ref{sec:promiscuouselements}), as well as the many positions for some bound morphemes, could each be reduced to one. The positions could also be reduced if some morphemes (e.g. adverbs) proved to modify others, so that they would constitute a zone, rather than distinct positions. 

The rest of this section is organized as follows. §\ref{bkm:Ref99649617} describes the orthographic word in Chorote, §\ref{sec:promiscuouselements} discusses the fact that some elements can occur in more than one position, and §\ref{bkm:Ref73306870} explains the distribution of NPs and DPs in Chorote.

\subsection{The orthographic word}
\label{bkm:Ref99649617}
This section is concerned with the orthographic ``word" in Chorote, which largely corresponds to the practice of writing spaces by missionary linguists. Such a notion offers an idea of what the traditional notion of word is like in the language. In broad terms, it corresponds to a stress domain. Thus, the orthographic word containing the verb core usually has the unstressed personal prefixes of position 14 as its left edge. But when the unstressed prospective marker \textit{ja} of position 8 surfaces, it usually becomes the left edge and, more rarely, the complementizer \textit{ka} of position 4 (most usually written in a separate word). 

However, when the stressed reflexive or antipassive markers of position 15 surfaces, the correlation between the orthographic word and the stress domain is broken. Drayson and the authors of religious texts include it in the orthographic word altogether with the verb core, so that the word has two accents - in position 15 and in position 16, e.g. \textit{i-ni-`we'en} (14-15-16) `(S)he sees himself/herself'. Nevertheless, some native speakers split this into two orthographic words, with only one stress each: \textit{i-ni `we'en} (14-15 16).

As for the right edge, the remote past marker of positon 39 can be the right edge or appear in a different orthographic word, which is consistent with its facultative stress. Something similar occurs with the markers of position 40, some of which are stressed, while others are not.

But when the oblique second person markers of position 35 surface, the correlation with the stress domain is again broken. This morpheme bears (secondary) stress, despite which it is written altogether in word with the verb in most texts. (Although I have not tested it, it is possible that some speakers would write it as a different word). 

In the two cases mentioned above in which, in the missionaries' writing, orthographic word does not correspond to a stress domain, some sort of `morphological word' criterion seems to be involved. Namely, the stressed morphemes of position 15 form a morphological domain with the verb core of position 16 according to all the tests that will be discussed below. Similarly, the marker of position 35 forms a domain, at least, with the material immediately to its left, if not with the verb core too.

\subsection{``Promiscuous'' elements}
\label{sec:promiscuouselements}
Some TAME markers may appear bound not only to the predicate but also to other hosts that are often defined as phonological domains, and some of them even occur as free particles: see the incompletive \textit{-ta(j)/-tye(j)} bound to the verb in (\ref{bkm:Ref87657836}) and to the prospective pre-verbal particle in (\ref{bkm:Ref87657851}); the indirect evidential \textit{-t'i(y)} bound to the verb in (\ref{bkm:Ref87657861}) and to negation (also pre-verbal) in (\ref{bkm:Ref87657869}); the interrogative -\textit{mi} bound to the verb in (\ref{bkm:Ref87657864}) and to negation in (\ref{bkm:Ref87657869}), and as a sentence-initial particle in (\ref{bkm:Ref87657872});\footnote{I tentatively assign this particle the same position as the conjunctions, because they do not introduce a topic or a focus, and there is evidence they precede adverbial clauses. Since many of the neighbouring positions do not seem to occur in interrogatives, the precise relative position of \textit{ma} is difficult to determine.} and the mirative -\textit{p'an} bound to the verb in (\ref{bkm:Ref87657876}), to negation in (\ref{bkm:Ref87657878}) and to an initial DP/NP in (\ref{bkm:Ref87657881}).\footnote{Cases where morphemes are basically unsegmentable but occur over a whole span are glossed with an x:y notation, where x is the left-most position occupied by the morph and y is the right-most element. An example is \textit{'yijén-} `\textsc{third person}+know' in (\ref{bkm:Ref87657861}), which covers positions 14 through 16. Where a position involves more than one morpheme, as is often the case in zones, the position is enclosed in brackets and only the first one receives a number. An example is \textit{kya si'yús} 'the fish' in (\ref{bkm:Ref87657861}).}

% \footnote{1, 2, 3: first, second and third person; \textsc{al.poss:} alienable possession; \textsc{aps:} antipassive; \textsc{comp:} complementizer; \textsc{dem:} demonstrative; \textsc{epth}: epenthetic vowel; \textsc{evid:} evidential; \textsc{incomp:} incompletive; \textsc{indef.poss:} indefinite possessor; \textsc{int:} interrogative; \textsc{irr:} irrealis\textsc{;} \textsc{jen}\textsc{:} pluractional\textsc{/}downwards; \textsc{mir:} mirative\textsc{;} \textsc{mom:} momentary\textsc{;} \textsc{near.fut:} near future\textsc{;neg:} negation;  \textsc{P:} adposition/applicative: \textsc{P}\textsc{\textsubscript{dist}}\textsc{:} distal\textsc{;}  \textsc{P}\textsc{\textsubscript{distr}}\textsc{:} distributive\textsc{;} \textsc{P}\textsc{\textsubscript{inst}}\textsc{:} instrumental\textsc{;} \textsc{P}\textsc{\textsubscript{loc}}\textsc{:} locative\textsc{;} \textsc{P}\textsc{\textsubscript{or}}\textsc{:} orientation\textsc{;} \textsc{P}\textsc{\textsubscript{punct}}\textsc{:} punctual\textsc{;} \textsc{pcpl:} participle; \textsc{pej}\textsc{:} remote past, every time; \textsc{perf:} perfect\textsc{;} \textsc{plact:} pluractional\textsc{;} \textsc{poss:} possessive\textsc{;} \textsc{prosp:} prospective\textsc{;} \textsc{vbz:} verbalizer\textsc{;}  \textit{v}: light verb; pl, \textsc{pl:} plural.}

\ea Incompletive \textit{-ta(j)/-tye(j)} \\
    \ea\label{bkm:Ref87657836}post-verbal\\ {
    \glll {} Y- ik- \textbf{tye} \\
    v: \ref{chovpos:activenonactive14} \ref{chovpos:predbase16} \textbf{\ref{chovpos:incompletive27}} \\ 
    {} \Third- leave -\Incomp{}\\
    \glt `(S)he was leaving/about to leave (but didn't).'
    }
    \ex\label{bkm:Ref87657851}After prospective { \\ 
    \glll {} Jo- \textbf{ta} y- ik\\ 
    v: \ref{chovpos:prospectiveja8} \textbf{\ref{chovpos:incompletive9}} \ref{chovpos:activenonactive14} \ref{chovpos:predbase16}\\ 
    {} \Prosp{}- \Incomp{} \First.\Irr{}- leave\\
    \glt `I would leave', `I intended to leave'.
    }
    \z
\z 

\ea  Indirect evidential \textit{-t'i(y)} and interrogative \textit{-mi/ma}\\
    \ea\label{bkm:Ref87657861}post-verbal evidential {\\ 
    \glll {} 'Yijén- \textbf{t'iy}{}- i\\ 
   v: \ref{chovpos:activenonactive14}:\ref{chovpos:predbase16} \textbf{\ref{chovpos:indirectevidential25}} \ref{chovpos:applinstrumental31} \\ 
    {} \Third{}:know- \textbf{\Evid{}}- \Ap.\Ins{}\\ 
    \glt `(S)he must probably know.'
    }
    \ex\label{bkm:Ref87657864} post-verbal interrogative\\ { 
    \glll {} Ji- woy- e- t'i- tye- \textbf{mi} [ka Ø{}- 'wen- a syunye i'nyó'…]\\
    v: \ref{chovpos:activenonactive14} \ref{chovpos:predbase16} \ref{chovpos:concord20} \ref{chovpos:indirectevidential25} \ref{chovpos:incompletive27} \textbf{\ref{chovpos:interrogativemi28}} \ref{chovpos:heavyDP46} - - - - -\\ 
    {} \Second{}- \Lv{}- \Second\Pl{}- \Evid{}- \Incomp{}- \textbf{\Inter{}} \Comp{}- \Second.\Irr{}- see- \Second\Pl{} \Dem{} man\\ 
    \glt `Did you intend to see that man?' (Mt 11:8)
    }
    \ex\label{bkm:Ref87657869}Evidential and interrogative after negation \\ 
    \glll {} ¿Je- \textbf{t'i}- \textbf{mi}  'naján- ay- i..? \\
   v: \ref{chovpos:negationje10} \textbf {\ref{chovpos:indirectevidentialmirative11}} \textbf {\ref{chovpos:interrogativemi12}} \ref{chovpos:activenonactive14}:\ref{chovpos:predbase16} \ref{chovpos:concord20} \ref{chovpos:applinstrumental31}\\ 
   {} \Neg{}- \textbf{\Evid{}}- \textbf{\Inter{}} \Second{}:know- \Second\Pl{}- \Ap.\Ins{}\\
    \glt `Don't you (pl.) know.'
     
    \ex\label{bkm:Ref87657872} Sentence-initial interrogative\\ {
    \glll {} ¿\textbf{Ma} y- am- 'nijne? \\ 
    v: \textbf{\ref{chovpos:conjunct1}} \ref{chovpos:activenonactive14} \ref{chovpos:predbase16} \ref{chovpos:pluractionaliterative38}:\ref{chovpos:remotepastperfect39}\\ 
    \ {} Inter{} \Third{}- go.away- \Plact:\Prf{}\\ 
    \glt `Are they already gone?'
    }
    \z
\z 


\ea\label{bkm:Ref87658388} Mirative \textit{-p'an}\\ 
    \ea\label{bkm:Ref87657876} post-verbal\\ { 
    \glll {} ¡A- sẹnyan- \textbf{p'an}{}- taj [kya si'yús]!\\ 
    v: \ref{chovpos:activenonactive14} \ref{chovpos:predbase16} \textbf{\ref{chovpos:mirative26}} \ref{chovpos:incompletive27} \ref{chovpos:heavyDP45} - \\ 
    {} \First{}- roast- \Mir{}- \Incomp{} \Dem{} fish\\ 
    \glt `I have roasted the fish in vain!' (E.g. because you are not hungry)
    }
    \ex\label{bkm:Ref84109704}\label{bkm:Ref87657878} After negation\\ {
    \glll {} Je \textbf{p'an} Ø{}- 'ẹs- i- ji [ka i'nyát i'nyó Ø{}- wujw- a- ja'm]\\
    v: \ref{chovpos:negationje10} \textbf{\ref{chovpos:indirectevidentialmirative11}} \ref{chovpos:activenonactive14} \ref{chovpos:predbase16} 
    \ref{chovpos:applinstrumental31}/\ref{chovpos:distaley32}
    \ref{chovpos:distaley32}/\ref{chovpos:appllocative33} \ref{chovpos:heavyDP46} - - - - - -\\ 
    {} \Neg{} \Mir{} \Third{}- be.good- \Ap.\Ins{} \Ap.\Loc{} \Comp{} water man \Third{}- be.big- \Irr{}- \Ap.\Loc{}\\ 
    \glt `It is not good if a man gets full of water.' (\citealt[116]{DraysonGomez2000})
    }
    \ex\label{bkm:Ref87657881} After initial NP/DP\\ {
    \glll {} [As- taj] \textbf{p'an} je Ø{}- wujw- a- k'i, [ti a- 'wen- k'i ja- pọ i'nyó'…]\\ 
    v: \ref{chovpos:lightDP5} - \ref{chovpos:mirativereportative7} \textbf{\ref{chovpos:negationje10}} \ref{chovpos:activenonactive14} \ref{chovpos:predbase16} \ref{chovpos:momentary22} \ref{chovpos:applipcntloc34} \ref{chovpos:heavyDP46} - - - - - -\\ 
    {} \Second\Pl{}.\Poss{}- sight \Mir{} \Neg{} \Third{}- big- \Mom{}-  \Ap.\Distr{} \Comp{} \First{}- see -\Ap.\Distr{} \Dem{}- \Pl{} man\\\\ 
    \glt `It turns out that you (pl.) do not remember when I saw those men.' (Lit. `your sight is not large enough to)' (Mt 16:9)
    }
    \z 
\z 

Many, if not all of these morphemes, have the appearance of second position clitics, with differences regarding how far they can move to the left, the mirative \textit{-p'an} and the reportative \textit{-jen} reaching the leftmost positions. Therefore, the high number of positions might be somewhat deceiving. If the approach mentioned above turned out to be correct, each morpheme of the clitic cluster would have its own position in the planar fractal method, but the different positions of each morpheme when fronted could be reduced to one: bound to the initial host position. The mirative \textit{-p'an}, for instance, occupies a second position in both (\ref{bkm:Ref84109704}) and (\ref{bkm:Ref87657881}) but, since in (\ref{bkm:Ref84109704}) negation occupies the first position and in (\ref{bkm:Ref87657881}) a DP precedes negation, \textit{-p'an} appears to the right and to the left of negation respectively, in two apparently different positions. If the host positions, namely negation in (\ref{bkm:Ref84109704}) and the DP in (\ref{bkm:Ref87657881}), were the same, so would be that of \textit{-p'an}. We leave more thorough consideration of this possibility for future research.

Moreover, as mentioned above, some texts by elderly speakers occasionally contain morphemes occurring in positions not recorded in \tabref{tab:chor:key:1} as in \REF{chor:ex:key:1}.

\ea\label{chor:ex:key:1}Elements in positions not recorded in \tabref{tab:chor:key:1} \\
    \ea  Discontinuous first person plural inactive with mirative \textit{p'an} in between\\ {
    {} \textbf{kas} \textbf{p'an} \textbf{ts}'ẹlya- k'i'\\  
    \glll {} kas p'an s- 'ila- k'ye \\ 
    v: Ø \textbf{\ref{chovpos:mirativereportative7}} \textbf{\ref{chovpos:activenonactive14}} \ref{chovpos:predbase16} \ref{chovpos:applipcntloc34}\\ 
    {} \First\Pl{} \Mir{} \First\Pl{}- be.eager- \Ap{}.\Distr{}\\
    \glt `We wanted to keep (eating)!' (\citealt{DraysonGomez2000}: 40)
    }
    \ex  Discontinuous first person plural inactive with negation \textit{je} in between\\ {
    \glll {} \textbf{kas-} \textbf{é} \textbf{s}-  ọjme'n\\
    v: Ø \textbf{\ref{chovpos:negationje10} \ref{chovpos:activenonactive14}} \ref{chovpos:predbase16} \ref{chovpos:pluractionaliterative38}\\ 
    {} \First\Pl{}- \Neg{} \First\Pl{}- be.drunk:\Jen{}\\ 
    \glt `We are not drunk.' \citep{Gerzenstein1983}
    }
    \ex  Indirect evidential \textit{-t'i} attached to reflexive/reciprocal \\ { 
    \glll {} i- ni- \textbf{t'i} 'wen- k- in- a- ja'ajme [ja Santiago]\\
    v: \ref{chovpos:activenonactive14} \ref{chovpos:rflxantipassive15} \textbf{{}Ø} \ref{chovpos:predbase16} \ref{chovpos:ptcp17} \ref{chovpos:antipassvblz18} \ref{chovpos:momentary22} \ref{chovpos:appllocative33}:\ref{chovpos:remotepastperfect39} \ref{chovpos:heavyDP45} - \\ 
    {} \Third{}- \Refl{}- \textbf{\Evid{}} see- \Pp{}- \Vblz{}- \Mom{}- \Ap.\Loc{}:\Prf{} \Dem{} Santiago\\
    \glt `He appeared in a vision (lit. `made himself seen') to Santiago.' (1 Cor 15: 7)
    }
    \z 
\z 

The properties of the first person plural inactive \textit{kas-} might have a metrical explanation, at least historically. Capable of forming a closed syllable, unlike the personal prefixes of the form CV, \textit{kas-} could have constituted a minimal foot (and therefore a minimal word). This has been argued for in the sister Nivaclé language by \citet[178--179]{Gutierrez2015}, where a cognate of \textit{kas-} exists, although in the nominal domain. Thus, if a morpheme needs to be not smaller than a foot in order to appear left-dislocated and/or to be a host, \textit{kas\-}, or more precisely its first segment, is the only verbal person marker that meets this condition. (In the ``standard'' variety the first person plural inactive occupies position \ref{chovpos:activenonactive14} only, and there is no reason to assume two segments.)

\subsection{Distribution of DPs/NPs}
\label{bkm:Ref73306870}
Another factor that multiplies positions in Chorote is the fact that complements may appear to the right or to the left of their heads. Complements headed by a demonstrative are heavy and surface to the right, while other complements are light and surface to the left. Examples (\ref{bkm:Ref82955341}) and (\ref{bkm:Ref77082664}) illustrate this in the nominal domain. Here the possessor appears to the right in (\ref{bkm:Ref82955341}) and to the left in (\ref{bkm:Ref77082664}). 

\ea\label{bkm:Ref82955254} Heavy and light complements in the nominal domain\\ 
    \ea\label{bkm:Ref82955341} Heavy complement \\ {
    \gll jl- as na Juan\\
    \Third\Poss{}- son \Dem{} Juan\\
    \glt `Juan's son.'
    }
    \ex\label{bkm:Ref77082664} Light complement\\ {
    \gll Juan jl- as\\
    Juan \Third\Poss{}- son\\ 
    \glt `Juan's son'
    } 
    \z
\z 


The phenomenon in the verbal domain is shown in (\ref{bkm:Ref76053713}), where \textit{(puwa) alẹnas} `the dogs' is the subject of a transitive verb (A). But in the first sentence it is a heavy DP and appears post-verbally, in position \ref{chovpos:heavyDP45}, whereas in the second sentence it is light (=no demonstrative) and appears pre-verbally, in position \ref{chovpos:lightDP5}. (Light NPs/DPs are licensed to the right when followed by an irrealis nominal marker, however; see (\ref{bkm:Ref84106076}), (\ref{bkm:Ref84107083}) and (\ref{bkm:Ref90043129}) below).


\ea\label{bkm:Ref76053713}Heavy and light complements in the verbal domain \\ 
\glll {} i- 'wi'in \textbf[{pu-} \textbf{wa~} \textbf{alẹna-} \textbf{s]\textsubscript{(A)}}. \textbf{[Alẹna-} \textbf{s]\textsubscript{(A)}} i- jyan- a- 'ni [ja- pa 'najáte]\textsubscript{(O)}\\
v: \ref{chovpos:activenonactive14} \ref{chovpos:predbase16} \textbf{\ref{chovpos:heavyDP45}} - - -   \textbf{\ref{chovpos:lightDP5}} - \ref{chovpos:activenonactive14} \ref{chovpos:predbase16} \ref{chovpos:momentary22} \ref{chovpos:pluractionaliterative38} \ref{chovpos:heavyDP45} - - \\ 
{} \Third{}- find \Dem{}- \Pl{} dog- \Pl{} dog- \Pl{} \Third{}- chase- \Mom{}- \Plact{} \F{}- \Dem{} rabbit\\
\glt {} `…the dogs find [some rabbit]. The dogs then chase the rabbit.' \citep[48]{DraysonGomez2000}
\z 


The same conditioning on the position applies when the NPs/DPs are complements of adpositions/applicatives, but in this case it affects the position of the heads as well. In effect, Chorote has morphemes that may encliticize to their complements (``adpositions'') or to the verb (``applicatives''), depending on whether they take a light complement or not, respectively. It is argued elsewhere (\citealt{Carol2011b}, \citealt{CarolSalanova2012}) that they are simply grammatical adpositions that may occur superficially bound to the verb in some cases. They are glossed here indistinctly as ``P''.

In (\ref{bkm:Ref80638892}) there is no light complement, but an optional heavy complement \textit{(na Mosik)}; hence, P encliticizes to the verb. (\ref{bkm:Ref80638899}) and (\ref{bkm:Ref80638902}) exemplify Ps with light complements. These may be NPs, as in (\ref{bkm:Ref80638899}), or oblique pronouns, as in (\ref{bkm:Ref80638902}), which have different positions in the template. The complex element \textit{oblique pronoun}+\textit{adposition}, in turn, attaches to the verb. Therefore, the same applicative/adposition (the distal in this case) may appear in three different positions: \ref{chovpos:distaley32}, \ref{chovpos:appl36} and \ref{chovpos:adpositionsappl44}.


\ea\label{ex:chor:key:2}Heavy and light complements with Ps \\ 
    \ea\label{bkm:Ref80638892}P bound to V; optional heavy complement \\ {
    \glll {} Ø- Tajl- ej- \textbf{e} wek ([na Mosik])\\ 
    v: \ref{chovpos:activenonactive14} \ref{chovpos:predbase16} \ref{chovpos:applinstrumental31} \textbf{\ref{chovpos:distaley32}} \ref{chovpos:tempaspdisc40} (\ref{chovpos:heavyDP45} -)\\
    {} \Third{}- come.out- \Ap.\Ins{}- \Ap.\Dist{} finally \Dem{} Mosik\\ 
    \glt {} `(S)he finally obtains it (from Mosik).' Lit. `comes out with it from Mosik.'
    }
    \ex\label{bkm:Ref80638899}P bound to a light nominal complement \\ {
    \glll {} Ø- Tajl- e tewk- \textbf{i}\\ 
    v: \ref{chovpos:activenonactive14} \ref{chovpos:predbase16} \ref{chovpos:applinstrumental31}  \ref{chovpos:lightDP43} \textbf{\ref{chovpos:adpositionsappl44}}\\ 
    {} \Third{}- come.out- \Ap.\Ins{} river- \Ap.\Dist{}\\
    \glt `(S)he finally obtains it from the river.'
    }
    \ex\label{bkm:Ref80638902} P bound to a pronoun\\ {
    \glll {} Ø- Tajl- a 'a- \textbf{i} \textbf{wek}\\ 
    v: \ref{chovpos:activenonactive14} \ref{chovpos:predbase16} \ref{chovpos:applinstrumental31} \ref{chovpos:oblique35} \textbf{\ref{chovpos:appl36}} \ref{chovpos:tempaspdisc40}\\ 
   {}  \Third{}- come.out \Ap.\Ins{}- \Second{}- \Ap.\Dist{}-   finally \\
    \glt`(S)he finally obtains it from you.'
    }
    \z 
\z 

\section{Morphosyntactic diagnostics}
\largerpage
\label{bkm:Ref82962135}
This section discusses the morphosyntactic diagnostics, except for those related to ciscategorial selection, which will be discussed in more detail in §\ref{bkm:Ref88847156}.
\subsection{Free occurrence (\ref{chovpos:predbase16}{}-\ref{chovpos:predbase16}; \ref{chovpos:complementizer4}{}-\ref{chovpos:tempaspdisc40})}
\label{bkm:Ref90253511}

This abstract type identifies the minimal free form, i.e. a complete utterance that is a single free form. The test can be fractured into \textit{minimal} and \textit{maximal}. The \textit{minimal} subtype identifies the smallest possible minimal free form that contains the predicate head. This corresponds just to position \ref{chovpos:predbase16}, i.e. the verb root or non-verbal predicate, which stands alone as an utterance in imperatives, e.g. \textit{kasit} `stand up'. 

The \textit{maximal} subtype identifies the span defined by the largest minimal free form that contains the predicate head, i.e. the largest possible span containing a predicate head (typically a verb) plus the more distant bound elements to the right and to the left, such that no other free form intervenes. This defines the subspan \ref{chovpos:complementizer4}{}-\ref{chovpos:tempaspdisc40}. I have no examples of this subspan in main clauses in spontaneous speech, but (\ref{bkm:Ref90130576}) shows an example in an embedded clause. The left edge of this span is occupied in main clauses by the complementizers \textit{ka} and, more rarely, \textit{ti}.\footnote{The complementizer \textit{ti} introduces temporal completives and others selecting \textit{realis}. Examples of \textit{ti} in main clauses are not as clear as those of \textit{ka}; one of them is the second \textit{ti} of (\ref{bkm:Ref90321059}). The interrogative \textit{ma} does appear in main clauses, but its inclusion in this position is only tentative.} \textit{Ka} selects for the irrealis mode on the predicate and behaves as a phonological proclitic. It heads some complement, conditional, temporal and in general future-oriented embedded clauses, as well as some main clauses including optative, hortative, and prohibitive. An example of \textit{ka} on the left edge in a main clause is given in (\ref{bkm:Ref90130557}). 

\ea  Free occurrence \textit{maximal}
    \ea\label{bkm:Ref90130576} Subspan \ref{chovpos:complementizer4}{}-\ref{chovpos:tempaspdisc40} in embedded clause\\ {
    \glll {} \textbf{\textit{ka}} \textit{Ø-} \textit{'nes-} \textit{ta-} \textbf{\textit{na'a}} \\
    v: \textbf{\ref{chovpos:complementizer4}} \ref{chovpos:activenonactive14} \ref{chovpos:predbase16} \ref{chovpos:incompletive27} \textbf{\ref{chovpos:tempaspdisc40}}\\ 
    {} \textbf{\Comp{}} \Third.\Irr{}- arrive- \Incomp{}- \textbf{later}\\
    \glt`When (s)he/it arrives.'
    }
    \ex\label{bkm:Ref90130557} Left edge in main clause\\ {
    \glll {} \textbf{Ka} y- ik\\
    v: \textbf{\ref{chovpos:complementizer4}} \ref{chovpos:activenonactive14} \ref{chovpos:predbase16}\\ 
    {} \textbf{\Comp{}} \First.\Irr{}- go.away\\\\ 
    \glt `I'm leaving, may I go'.
    }
    \z 
\z 

Positions \ref{chovpos:conjunct1} through \ref{chovpos:top2} include phrases and other elements that can occur as free forms and are thus excluded from the span. Notice that there are free forms between \ref{chovpos:complementizer4} and \ref{chovpos:predbase16} but, since they are not obligatory, they are irrelevant for this diagnostic.

The right edge is more problematic. In \ref{bkm:Ref90130576} it is represented by the adverb \textit{na'a} `later' in position \ref{chovpos:tempaspdisc40}. Example (\ref{bkm:Ref90131499}) below is another example of \textit{na'a} as right edge but in a main clause; \textit{pet} `please' also belongs in position \ref{chovpos:tempaspdisc40}.

\ea\label{bkm:Ref90131499}Free occurrence (maximal): right edge (\citealt[70]{DraysonGomez2000}) \\
\glll {} Jo- kyu- \textbf{nye'e} \textbf{pet} ts'ijí [i- 'wit-] e\\
   v: \ref{chovpos:predbase16} \ref{chovpos:tempaspdisc40} \textbf{\ref{chovpos:tempaspdisc40}} \ref{chovpos:tempaspdisc40} \ref{chovpos:locatives41} \ref{chovpos:lightDP43} -   \ref{chovpos:adpositionsappl44}\\ 
{} go- a.while- \textbf{later} \textbf{please} there \First\Poss{}- place- \Ap \\
\glt `Go to my place (i.e. my house) there.'
\z 
 This adverb does not occur as a free form in my material. But at least some of the other adverbs of position \ref{chovpos:tempaspdisc40} are free forms, e.g. \textit{t'e},\footnote{This has the same underlying form as the evidential \textit{-t'i} of position 23, namely /t'ey(h)/. The difference in the surface forms correspond to regular differences between stressed and unstressed vowels.} which can function as an answer to questions with the meaning `who knows'. For other elements in the same position I have no conclusive evidence regarding their bounded character; some can appear fronted in position \ref{chovpos:top2} (e.g. \textit{ta'a} `immediately', `already') and have been spontaneously uttered alone at least in metalinguistic uses, unlike applicatives and unstressed TAME markers, but perhaps the latter holds as well for the remote past \textit{pe(j)} of position \ref{chovpos:remotepastperfect39}; notice that both positions \ref{chovpos:remotepastperfect39} and \ref{chovpos:tempaspdisc40} host optionally stressed elements. As for material to the right of position \ref{chovpos:tempaspdisc40}, there is no doubt it must be excluded from the span. The locative adverbs of \ref{chovpos:locatives41} \textit{ts'ijí} `there' or \textit{'nijí} `here' are clearly free forms. Further to the right the only bound elements are the adpositions in position \ref{chovpos:adpositionsappl44}, but they occur bound to nouns (in position \ref{chovpos:lightDP43}), which in turn are free forms; the adpositions can also occur bound to verbs functioning as applicatives, but they are then analyzed as occurring in a different position, see §\ref{bkm:Ref73306870}.

\subsection{Non-interruption (\ref{chovpos:activenonactive14}{}-\ref{chovpos:remotepastperfect39}/\ref{chovpos:pluractionaliterative38}; \ref{chovpos:mirativereportative7}{}-\ref{chovpos:locatives41}; \ref{chovpos:activenonactive14}{}-\ref{chovpos:momentary22})}

The diagnostic of non-interruption identifies the span of positions that includes the predicate head and cannot be interrupted by some interrupting element \citep{Tallman2021}. The diagnostic is fractured according to how the interrupting element is defined. The \textit{single free interruptor} subtest defines the interruptor as any free form and is the most straightforward version of the diagnostic. This gives the span \ref{chovpos:activenonactive14}{}-\ref{chovpos:remotepastperfect39} or maybe \ref{chovpos:activenonactive14}{}-\ref{chovpos:pluractionaliterative38}, see (\ref{bkm:Ref90038202}).

\ea\label{bkm:Ref90038202} Non-interruption - \textit{single free interruptor} \\
    \ea  Span \ref{chovpos:activenonactive14}{}-\ref{chovpos:remotepastperfect39}\\ {
    \glll {} \textbf{Y}{}- am \textbf{pej}\\
    v: \textbf{\ref{chovpos:activenonactive14}} \ref{chovpos:predbase16} \textbf{\ref{chovpos:remotepastperfect39}} \\ 
    {} \textbf{\Third}- go.away \textbf{\Rem.\Pst{}} \\ 
    \glt `(S)he/it left.'
    }
    \ex  Span \ref{chovpos:activenonactive14}{}-\ref{chovpos:pluractionaliterative38}\\ {
    \glll {} \textbf{Y}{}- am- \textbf{'ni}\\
    v: \textbf{\ref{chovpos:activenonactive14}} \ref{chovpos:predbase16} \textbf{\ref{chovpos:pluractionaliterative38}}\\ 
    {} \textbf{\Third}- go.away- \textbf{\Plact{}} \\
    \glt `They left', `(s)he/it left repeatedly'
    }
    \z 
\z 

Regarding the left boundaries, position \ref{chovpos:kyak13} hosts a demonstrative pronoun \textit{kyak} (and less frequently other pronouns) that indicates distancing from the speaker (either through the speaker's or through the subject's movement), but which functions also as a locative or a manner adverb `this/that way'. This is the first interruptor found to the left of the predicate head. The rest of the demonstratives in this paradigm occur most usually before negation in positions \ref{chovpos:lightDP5} and \ref{chovpos:lightDP6}, but \textit{kyak} is documented between the negation of position \ref{chovpos:negationje10} (and presumably its enclitics of \ref{chovpos:indirectevidentialmirative11} and \ref{chovpos:interrogativemi12}) and the verbal prefixes of position \ref{chovpos:activenonactive14} when the verb is \textit{-wo}, a light verb meaning `do', `become, be' (among many other meanings) and its derivatives. (The expressions with \textit{-wo} `do, become, be' might be somewhat lexicalized, but since the verb can bear TAME morphemes and inflect for person, I still consider it a verbal head in position \ref{chovpos:predbase16} and not an auxiliary verb)

\ea\label{ex:chor:key:3}Single interruptor \textit{kyak} on the left\\ 
    \ea
    \glll {} Je \textbf{kyak} i- yo- ø\\
    v: \ref{chovpos:negationje10} \textbf{\ref{chovpos:kyak13}} \ref{chovpos:activenonactive14} \ref{chovpos:predbase16} \ref{chovpos:distaley32} \\ 
    {} \Neg{} \Dem{} \Third{}- do- \Ap.\Dist{}\\
    \glt `It is not like that, it is not the same'.
    \ex 
    \glll {} Jlam'ne je \textbf{kyak} ji- won- ay- i [na- pọ as- 'lejwa- s] [ka Ø- en- a'yi ni syuni ti- jnajyi]\\ 
    v: \ref{chovpos:conjunct1} \ref{chovpos:negationje10} \textbf{\ref{chovpos:kyak13}} \ref{chovpos:activenonactive14} \ref{chovpos:predbase16} \ref{chovpos:concord20} \ref{chovpos:distaley32} \ref{chovpos:heavyDP45} - - - - \ref{chovpos:heavyDP46} - - - - - - - \\
     {} but \Neg{} \Dem{} \Second{}- do- \Second\Pl{}- \Ap{} \Dem{}{}- \Pl{} \Second\Pl{}.\textsc{poss}{}- fellow- \Pl{} \Comp{} 2.\Irr{}- put- \Second\Pl{}:\Ap{} \Dem{} \Dem{} \Third{}- be.straight:\Ap.\Loc{}\\
    \glt `But you neglect justice with your fellows' (Lit `you don't do the same to your fellows when you make justice').' (Lc 11:42)
    \z 
\z 

To the right of position \ref{chovpos:predbase16} the closest free forms are some of the adverbs of \ref{chovpos:tempaspdisc40}, or perhaps the remote past of \ref{chovpos:remotepastperfect39} if considered a free form - it can be uttered alone at least in metalinguistic uses, unlike applicatives and monosyllabic adpositions, and also fronted in other varieties of Chorote. Neither can these or any other free forms intervene between positions \ref{chovpos:kyak13} and \ref{chovpos:tempaspdisc40} (or \ref{chovpos:remotepastperfect39}), so the span defined by this diagnostic subtype is \ref{chovpos:activenonactive14}{}-\ref{chovpos:remotepastperfect39} (or -\ref{chovpos:pluractionaliterative38}).

The interruptor can also be defined as a construct that contains more than one free form, i.e. a \textit{multiple free interruptor}. The interruptors of positions \ref{chovpos:kyak13} and \ref{chovpos:tempaspdisc40} (or \ref{chovpos:remotepastperfect39}) seen above are single forms, so they do not count in this version of the diagnostic; the same holds for negation (position \ref{chovpos:negationje10}), which is also a free form. The most typical multiple free interruptors are NPs \citep{Tallman2021}. Recall that even light NPs may be multiple interruptors, since they can consist of a possessive construction with two Ns, like the one shown in (\ref{bkm:Ref77082664}). In fact, these light NPs are the interruptors that function as the boundaries for this diagnostic. On the left side, light NPs appear in positions \ref{chovpos:lightDP5} and \ref{chovpos:lightDP6}; although usually only one of these positions is filled, (\ref{bkm:Ref84096642}) illustrates the need to postulate two distinct positions for light pre-verbal NPs. 

\ea\label{bkm:Ref84096642}Multiple interruptor to the left. Two NPs as interruptors (Rom 4: 13) \\ 
\glll {} \textbf{[Si-} \textbf{nya]} \textbf{[ji-} \textbf{'lij]} i- wijnam [pa Abraham] \\
v: \ref{chovpos:lightDP5} - \ref{chovpos:lightDP6} - \ref{chovpos:activenonactive14} \ref{chovpos:predbase16}:\ref{chovpos:appllocative33} \ref{chovpos:heavyDP45} - \\
{} \textbf{\First\Pl.\Poss{}}- \textbf{father} \Third\Poss{}- \textbf{speech} \Third{}- give:\Ap.\Loc{} \Dem{} Abraham\\ 
\glt `God (lit. `our father') gave his word to Abraham.'
\z 

To the right of the predicate head the first complex interruptor is the NP of position \ref{chovpos:lightDP42} exemplified in (\ref{bkm:Refexample13}) and (\ref{bkm:Ref84106076}). (\ref{bkm:Ref84106076}) shows a construction with a complex light NP to the right of the verb, which is licensed by the irrealis nominal marker \textit{-a} that follows the NP. The construction will be explained in more detail later in this section. (The NP means literally `son of boy', but \textit{jlas} is here a diminutive: `young/little boy'.) In the following position \ref{chovpos:lightDP43} there is also an NP (complement of the adposition of position \ref{chovpos:adpositionsappl44}) that can be complex. The example in (\ref{bkm:Refexample13}) illustrates the need of postulating two different zones for positions \ref{chovpos:lightDP42} and \ref{chovpos:lightDP43}: 

\ea \label{bkm:Refexample13} Positions \ref{chovpos:lightDP42} and \ref{chovpos:lightDP43} (\citealt{DraysonGomez2000}: 114) \\
\glll {} Ø- Laj [i'nyát -a] [s- ate jl{}- as-] i' \\
    v: \ref{chovpos:activenonactive14} \ref{chovpos:predbase16} \ref{chovpos:lightDP42} - \ref{chovpos:lightDP43} - - - \ref{chovpos:adpositionsappl44} \\
    {} \Third{}- not\_exist water -\Irr{} \First\Pl{}.\textsc{poss}{} -pitcher \Third\Poss{}- son -\Ap.\Loc{} \\ 
\glt `There was no water in our (little) pitchers.'
\z

\ea\label{bkm:Ref84106076} Multiple interruptor to the right \\
\glll {} a- wo \textbf{[jwemik} \textbf{jl-} \textbf{as-} \textbf{a']}.\\ 
v: \ref{chovpos:activenonactive14} \ref{chovpos:predbase16} \textbf{\ref{chovpos:lightDP42}} - - - \\ 
{} \First{}- \Lv{} \textbf{boy} \textbf{\Third\Poss{}}- \textbf{son}- \textbf{\Irr{}} \\
\glt `I was a young boy.' (\citealt[122]{DraysonGomez2000})
\z 

In sum, the multiple interruptors closest to the predicate head are in positions \ref{chovpos:lightDP6} on the left and \ref{chovpos:lightDP42} on the right. Thus, the span defined by this diagnostic is in principle \ref{chovpos:mirativereportative7}{}-\ref{chovpos:locatives41}. An example including both \ref{chovpos:mirativereportative7} and \ref{chovpos:mirativereportative7}{}-\ref{chovpos:locatives41} is lacking; (\ref{bkm:Ref84107083}) shows the span \ref{chovpos:mirativereportative7}{}-\ref{chovpos:pluractionaliterative38}, and \ref{chovpos:locatives41} was shown in (\ref{bkm:Ref90131499}). 

\ea\label{bkm:Ref84107083} Multiple free interruptor: left edge in position \ref{chovpos:mirativereportative7} \\
\glll {} Kyak- \textbf{p'an} i- yo- ø{}- pi [pa i'nyó] [ti i- yo- ø ka i- wo aye'wu- ye']\\ 
v: \ref {chovpos:lightDP5}- \textbf{\ref{chovpos:mirativereportative7}} \ref{chovpos:activenonactive14} \ref{chovpos:predbase16} \ref{chovpos:distaley32} \ref{chovpos:pluractionaliterative38} \ref{chovpos:heavyDP45} - \ref{chovpos:heavyDP46} - - - - - - - - \\ 
{} \Dem{}- \textbf{\Mir{}} \Third{}- do- \Ap.\Dist{}- \Rem.\Pst{} \Dem{} man \Comp{} \Third.\Irr{}- \Lv{}- \Ap.\Dist{} \Comp{} \Third{}- do shaman -\Irr{} \\
\glt `That is what a man used to do when he wanted to become a shaman.' (\citealt[134]{DraysonGomez2000})
\z 

Note however that position \ref{chovpos:mirativereportative7} is occupied by the mirative and (less frequently) by the reportative. If these morphemes turned out to be second position clitics as discussed in §\ref{sec:promiscuouselements}, such that the pronoun \textit{kyak} here occupied a clause-initial position rather than position \ref{chovpos:lightDP5}, then the left boundary should be the first positively fixed element that follows the interruptor NP. This would give us the prospective particle \textit{ja} of position \ref{chovpos:prospectiveja8}.

Finally, the interruptor can also be defined as a \textit{non-fixed} element. This subtype defines the subspan \ref{chovpos:activenonactive14}{}-\ref{chovpos:momentary22}, exemplified in (\ref{bkm:Ref90047260}). In this example, the relevant part is in an embedded clause introduced by the complementizer \textit{ti}. Therefore, the positions are given for the embedded clause only. In the main clause, enclosed in brackets, \textit{sek yi'i} `there is [the fact]' can be translated as `then' and selects most usually for a verb with the momentary morpheme in position \ref{chovpos:momentary22}.

\ea\label{bkm:Ref90047260}Non-fixed interruptor: subspan \ref{chovpos:activenonactive14}{}-\ref{chovpos:momentary22} \\
\glll {} [Sek y- i -'i] ti \textbf{y}{}- am- \textbf{a'}.\\
v: [- - - -] \ref{chovpos:complementizer4} \textbf{\ref{chovpos:activenonactive14}} \ref{chovpos:predbase16} \textbf{\ref{chovpos:momentary22}} \\
{} [\Dem{} \Third{}- be -\Ap.\Punct{}] \Comp{} \textbf{\Third{}}- go.away- \textbf{\Mom{}} \\ 
\glt`Then (lit. there it is) (s)he left.'
\z 

In this subtype, the left boundary is still the demonstrative pronoun \textit{kyak} that occurs in position \ref{chovpos:kyak13} among others, i.e. the same as in the \textit{single free interruptor} version, but the right boundary is now the irrealis marker \textit{-a} of \ref{chovpos:irrealis23}. This morpheme occurs bound to non-verbal predicates and certain `adjective-like' verbs (Class V in \tabref{tab:chor:key:2} below) indicating irrealis mood, as the embedded clause in \REF{bkm:Ref84109704}.; another example is shown \REF{bkm:Ref84285967}. In the remaining classes of verbs, irrealis is realized by means of a special set of personal prefixes in position \ref{chovpos:activenonactive14}. The suffix \textit{-a} appears furthermore following nouns or light NPs in a predicative construction, like the one shown in (\ref{bkm:Ref84106076}). Another example of this can be seen in (\ref{bkm:Ref84107083}), whose relevant part is repeated and adapted in \REF{bkm:Ref84285943}.\footnote{The multiple meanings of this construction, consisting of a light verb \textit{-wo/-yo} `do, become, be' and an N(P) followed by the irrealis, depend on the N(P) involved: `become', `look for', `build', `use' or even `be', among others \citep[909--910]{Carol2015}. In these constructions the existence of the referent of the (N)P is not asserted, e.g. with negated existential verbs, existential verbs under conditional clauses, constructions with the meaning `looking for' etc., or the referent comes into existence by the event itself, e.g. with verbs meaning `build', `become' and others; the meaning `be' of (\ref{bkm:Ref84106076}) is the only exception.}

\ea\label{bkm:Ref90043129}\textit{Irrealis} suffix/enclitic\\ 
    \ea\label{bkm:Ref84285967} On the predicate head \\ {
    \glll {} Ka Ø{}- wujw- \textbf{a'}\\ 
    v: \ref{chovpos:complementizer4} \ref{chovpos:activenonactive14} \ref{chovpos:predbase16} \textbf{\ref{chovpos:irrealis23}}\\ 
    {} \Comp{} \Third{}- be.big- \textbf{\Irr{}}\\
    \glt `If it were big.'
    }
    \ex\label{bkm:Ref84285943}On the noun in constructions with light verb \\ {
    \glll {} i- yo [aye'wu- \textbf{ye']} \\ 
    v: \ref{chovpos:activenonactive14} \ref{chovpos:predbase16} \ref{chovpos:lightDP43} - \\ 
    {} \Third{}- \Lv{} shaman- \textbf{\Irr{}} \\
    \glt `(He) becomes a shaman.'
    }
    \z 
\z 

Thus, the span of interruptability by a non-fixed element is \ref{chovpos:activenonactive14}{}-\ref{chovpos:momentary22}. An important issue, however, is that the momentary morpheme \textit{-a} of position \ref{chovpos:momentary22} has the same form as the irrealis suffix/enclitic, and they never co-occur. Attempts to elicit both together consistently failed, even in contexts where both are selected for.\footnote{These contexts are (a) the sequence \textit{sekyi'i ti} `there is (the fact) that…', approximately equivalent to `then', which selects for the momentary morpheme on the verb and (b) the complementizer \textit{ka}, which selects for irrealis. When (a) is in the prospective form, it includes \textit{ka} instead of \textit{ti}: \textit{sek jane'yi ka…}, so that both contexts co-occur. In that case, when the verb marks irrealis with the irrealis set of personal prefixes, both irrealis and momentary co-occur. But when the predicate is non-verbal or a Class V verb, such that irrealis need to be indicated with \textit{-a}, only one \textit{-a} surfaces.} Clearly, they are two different abstract morphemes, since they can cooccur when irrealis is marked by a special set of personal prefixes, and that is the reason why two different positions have been assigned; besides, only the irrealis is non-fixed. But the choice in the linear order between them, in the absence of empirical evidence, has a purely theoretical motivation: aspectual morphemes are usually expected to be more internal than modal ones.

\subsection{Non-permutability (\ref{chovpos:activenonactive14}{}-\ref{chovpos:irrealis23})}

This diagnostic identifies a span whose elements cannot be variably ordered with respect to each other. In Chorote this defines the span \ref{chovpos:activenonactive14}{}-\ref{chovpos:momentary22}, already exemplified in (\ref{bkm:Ref90047260}).

We have already discussed the pronoun \textit{kyak}, which can appear both to the right of negation, in position \ref{chovpos:kyak13}, or to the left, in positions \ref{chovpos:lightDP5} or \ref{chovpos:lightDP6}. Thus, \textit{kyak} should be excluded from the span, since its position is interchangeable with that of negation.

As for the right edge of the span, the irrealis morpheme in position \ref{chovpos:irrealis23}, exemplified in (\ref{bkm:Ref84285967}), only occurs in that position in the verbal template and is therefore not interchangeable with any other element. Thus, I consider it the right edge in this diagnostic. It is true that it can occur to the right of the reportative (position \ref{chovpos:reportativejen24}), as in (\ref{bkm:Ref84285943}), but in that case it belongs in the nominal template and is not relevant for the diagnostic. The reportative, in turn, can occur in a different position besides position \ref{chovpos:reportativejen24}, so it must be excluded from the span; see (\ref{bkm:Ref89009542}).

\ea\label{bkm:Ref89009542} Reportative occurring in position \textbf{\ref{chovpos:mirativereportative7}} \\ 
\glll {} Sek- \textbf{jin} y- i- 'i- pe [syupi i'nyó' ji- kyo t'isyé\textsuperscript{(}'\textsuperscript{)}n y- i'lya- je'].\\
v: \ref{chovpos:lightDP5} \textbf{\ref{chovpos:mirativereportative7}} \ref{chovpos:activenonactive14} \ref{chovpos:predbase16} \ref{chovpos:applipcntloc34} \ref{chovpos:remotepastperfect39} \ref{chovpos:heavyDP45} - - - - - - - \\ 
{} there \textbf{\Rep{}} \Third{}- be- \Ap.\Punct{} -\Rem.\Pst{} \Dem{} man \Third\Poss{}- hand \Third\Poss{}:flesh \Third{}- be.dry- \Ap.\Loc{}\\
\glt `There was (hearsay) a man whose hand was dry.' (Mc 3:1) 
\z 

The diagnostic as described above corresponds to \textit{strict} non-permutability. A possible fracture discussed in the introduction of this book and in \citet{Tallman2021} considers scope: \textit{flexible} non-permutability admits inside the span elements whose order can be altered if that entails changes in scope. However, I have not considered this version of the diagnostic for Chorote due to lack of reliable data on scope changes when order is altered. In §\ref{sec:promiscuouselements} it was shown that the TAME markers of positions \ref{chovpos:reportativejen24} through \ref{chovpos:interrogativemi28} can be fronted to positions \ref{chovpos:conjunct1}, \ref{chovpos:mirativereportative7}{}-\ref{chovpos:incompletive9}, and \ref{chovpos:indirectevidentialmirative11}{}-\ref{chovpos:interrogativemi12}. In some cases, these morphemes occur bound to nominals and have nominal scope, as discussed in \citet{Carol2015}, and thus do not occupy a position in the verbal template; but in others, as those of §\ref{sec:promiscuouselements}, it is not clear to me whether the variation in position affects scope in the verbal template. 

Despite that, it is worth considering what happens to the right of those TAME markers since, were it confirmed that their variable ordering is sensitive to scope, we would have results for the diagnostic of flexible permutability. 

After the TAME markers comes third person plural marker \textit{-is}, which only occurs in position \ref{chovpos:concord29}, and thereafter the string of Ps functioning as applicatives, which are rigidly ordered in positions \ref{chovpos:applorientation30} through \ref{chovpos:applipcntloc34} if no complement (NP or pronoun) surfaces. But if an oblique pronominal complement surfaces in position \ref{chovpos:oblique35}, then the P which selects the pronoun as its complement occurs to the right of the latter in position \ref{chovpos:appl36}, which may alter its relative order regarding other Ps.\footnote{Recall that I assume that the Ps always belong to the verbal template, but they can surface as applicatives or adpositions depending on the ``weight'' of the NP. If an alternative analysis were adopted, according to which the Ps surfacing as adpositions belonged to a distinct template other than the verbal template, there would be no permutability of Ps.} In \ref{bkm:Ref84594918} the P called here ``orientation'' appears to the left of the instrumental, but in \ref{bkm:Ref84601457} the former takes a first person pronominal complement and surfaces thus to the right of the instrumental.\footnote{The basic allomorphs are \textit{-a(j)} for ``orientation'' and \textit{-e(j)} for the instrumental. After a vowel, epenthetic /y/ is inserted and included as part of the Ps. This /y/ in turn raises front vowels, as explained in \sectref{cho:sec:languageandspeakers}, thus the Ps result in \textit{-yej}, \textit{{}-yij}, respectively. The P ``orientation'' takes here a suppletive form when the pronominal complement surfaces.} This does not relate to semantic scope; hence, position \ref{chovpos:applorientation30} is outside the span of a flexible non-permutability (under the assumption that all TAME markers are potentially able to display scopal variation).

\ea\label{ex:chor:key:4} Permutability of Ps\\ 
    \ea\label{bkm:Ref84594918} Orientation before instrumental in regular order\\ {
    \glll {} i- nyu- \textbf{yej-} \textbf{e}\\
    v: \ref{chovpos:activenonactive14} \ref{chovpos:predbase16} \textbf{\ref{chovpos:applorientation30}} \textbf{\ref{chovpos:applinstrumental31}}\\ 
    {} \Third{}- pass- \textbf{\Ap.\Or{}}- \textbf{\Ap.\Ins{}}\\
    \glt `(S)he helps him/her.'
    }
    \ex\label{bkm:Ref84601457} Instrumental to the right when it takes a non-null complement\\ {
    \glll {} i- nyu- \textbf{yij}{}- k'i- \textbf{'m}\\
    v: \ref{chovpos:activenonactive14} \ref{chovpos:predbase16} \textbf{\ref{chovpos:applinstrumental31}} \ref{chovpos:oblique35} \textbf{\ref{chovpos:appl36}}\\
    {} \Third{}- pass- \textbf{\Ap.\Ins{}}- \First\Sg{}- \textbf{\Ap.\Or{}}\\
    \glt `(S)he helps me.'
    }
    \z 
\z 

\subsection{Subspan repetition (\ref{chovpos:prospectiveja8}{}-\ref{chovpos:pluractionaliterative38}; \ref{chovpos:prospectiveja8}{}-\ref{chovpos:remotepastperfect39})}

For this diagnostic, I consider repetition of a subspan of contiguous positions in a construction that is functionally equivalent to `and' coordination. In fact, the construction in question probably involves subordination: it is \textit{jla'yi ti/ka}, where \textit{jla'yi} means literally `his/her/its fellow' and \textit{ti}/\textit{ka} are the complementizers that select for \textit{realis} and irrealis predicate, respectively.\footnote{Coordination of NPs also involves \textit{jla'yi} (or the feminine \textit{jla'yiki'}), but with determiners instead of complementizers. For `or' coordinating VPs or clauses, Chorote uses \textit{ni'ne}, with the same complementizers; \textit{ni'ne} could be translated in isolation as `perhaps, maybe'. The behavior of \textit{ni'ne ti/ka} is apparently similar to that of \textit{jla'yi ti}, but less examples were found.} If we consider elements that \textit{must} be repeated in order to be interpreted, i.e. the \textit{minimal} subtype of this diagnostic, the subspan is \ref{chovpos:prospectiveja8}{}-\ref{chovpos:pluractionaliterative38}. If we consider the elements that \textit{can} be repeated, i.e. the \textit{maximal} version, the span is \ref{chovpos:prospectiveja8}{}-\ref{chovpos:remotepastperfect39}. In (\ref{bkm:Ref84601460}) we can see evidence for the left edge: failing to repeat the prospective marker \textit{ja} gives an ungrammatical result. The elements to the left of \textit{ja} cannot be repeated.

\ea\label{bkm:Ref84601460}Prospective \textit{ja} repeated under coordination \\ 
\glll {} \textbf{Ja}{}- 'yit- aj- a- 'a [na si- 'lij] jla'yi ti \textbf{ja}{}- y- amti- jyen- a /*y- amti- jyen- a \\ 
v: \textbf{\ref{chovpos:prospectiveja8}} \ref{chovpos:activenonactive14}:\ref{chovpos:predbase16} \ref{chovpos:concord19} \ref{chovpos:applinstrumental31} \ref{chovpos:applipcntloc34} \ref{chovpos:heavyDP45} - - \ref{chovpos:conjunct1} \ref{chovpos:complementizer4} \textbf{\ref{chovpos:prospectiveja8}} \ref{chovpos:activenonactive14} \ref{chovpos:predbase16} \ref{chovpos:antipassvblz18} \ref{chovpos:concord19} /*\ref{chovpos:activenonactive14} \ref{chovpos:predbase16} \ref{chovpos:antipassvblz18} \ref{chovpos:concord19} \\ 
{} \textbf{\Prosp{}}- \First:stab?- \First\Pl{}- \Ap.\Ins{}- \Ap.\Punct{} \Dem{} \First\Pl.\Poss{}- language and \Comp{} \textbf{\Prosp{}}- \First{}- speak- \Caus{}- \First\Pl{}/ *\First{}- speak- \Caus{}-  \First\Pl{} \\ 
\glt `Let us write our language and read it.' (From an educational book, \citealt{Drayson1999})
\z 

In (\ref{bkm:Ref84630584}) we can see the repetition of the remote past \textit{pe(j)}, but not in (\ref{bkm:Ref84630586}), even though the remote past \textit{is} interpreted in the second conjunct; both examples come from the same text and presumably the same speaker. This can be taken as evidence that repetition of \textit{pe(j)} is optional. 

\newpage
\ea\label{ex:chor:key:5} Remote past \textit{pe(j)} under coordination\\
    \ea\label{bkm:Ref84630584} Repeated \\ {
    \glll {} [Naka ni ø- paj- k'i ti a- pe'ya- k, si- tyet- e i- 'wi'in,] a- 'yen- a- 'nij- \textbf{pe} jla'yi ti a- kyes- a- 'a- \textbf{pe} \\
    v: [- - - - - - - - - - - - - -] \ref{chovpos:activenonactive14} \ref{chovpos:predbase16} \ref{chovpos:concord19} \ref{chovpos:pluractionaliterative38} \ref{chovpos:remotepastperfect39} \ref{chovpos:conjunct1} \ref{chovpos:complementizer4} \ref{chovpos:activenonactive14} \ref{chovpos:predbase16}- \ref{chovpos:concord19}- \ref{chovpos:applipcntloc34}- \ref{chovpos:remotepastperfect39} \\ 
    {} [\Dem{} \Dem{} \Third{}- be.ancient- \Ap.\Distr{} \Comp{} \First{}- hear- \First\Pl{} \First\Pl.\Poss{}- eye- \Pl{} \Third{}- see] \First{}- look- \First\Pl{}- \Plact{}- \textbf{\Rem.\Pst{}} and \Comp{} \First{}- touch- \First\Pl{}- \Ap.\Punct{}- \textbf{\Rem.\Pst{}}\\ 
    \glt `[That what was at the beginning (lit. `long ago'), what our eyes saw,] what we observed and touched…' (1 John 1:1)
    }
    \ex\label{bkm:Ref84630586} Not repeated\\ {
    \glll {} […a- wo- k- i s- amtiky- e- 'as- e naka syunye] a- 'wen- a- \textbf{pe} jla'yi ti a- pe'ya- k.\\
    v: […- - - - - - - - - - -] \ref{chovpos:activenonactive14} \ref{chovpos:predbase16} \ref{chovpos:concord19} \ref{chovpos:remotepastperfect39} \ref{chovpos:conjunct1} \ref{chovpos:complementizer4} \ref{chovpos:activenonactive14} \ref{chovpos:predbase16} \ref{chovpos:concord19}\\
    {} [\First{}- \Lv{}- \First\Pl{}- \Ap.\Dist{} \First\Pl.\Poss{}- speech- \Irr{}- \Second\Pl{}- \Ap.\Dist{} \Dem{} \Dem{}] \First{}- see- \First\Pl{}- \textbf{\Rem.\Pst{}} and \Comp{} \First{}- hear- \First\Pl{} \\
    \glt `[We tell you what] we saw and heard.' (1 John 1:3) }
    \z 
\z 
 
Moreover, this is consistent with the general behavior of \textit{pe(j)}, which is ``optional'' in the sense that it does not need to follow the predicate when the information it provides can be recovered e.g. from previous discourse. The pluractional\textit{'ni(j)} is \textit{not} repeated in the second member of the coordination of (\ref{bkm:Ref84630584}) and thus not interpreted (\textit{'ni(j)} in combination with \textit{'yen} `watch, look at' gives literally `watch repeatedly', i.e. `observe'.) This is a clear indication of its difference with respect to \textit{pe(j)}.

\subsection{Deviation from biuniqueness (\ref{chovpos:activenonactive14}{}-\ref{chovpos:antipassvblz18}; \ref{chovpos:predbase16}{}-\ref{chovpos:antipassvblz18}; \ref{chovpos:activenonactive14}{}-\ref{chovpos:concord29})}

This type identifies deviations from the biuniquess relation between form and meaning, which might be used as indication of word boundaries. The most common case in Chorote is more than one form corresponding to a single meaning. In this regard, the personal prefixes of position \ref{chovpos:activenonactive14} are the left edge. There are five different sets for the third person. They are predictable in some cases — transitive verbs always select for Class I \textit{i-/y-}, antipassive verbs for Class III \textit{t-/tV-,} and there is some correlation between semantic classes and prefixes, but the latter are not fully predictable in intransitive non-antipassive verbs (\citealt{Carol2013,Carol2014}). To the left of the prefixes of \ref{chovpos:activenonactive14} no such deviation is ever found.

\begin{table}[htp]
    \centering
    \caption{Third person verbal prefixes}
    \label{tab:chor:key:2}
    \begin{tabular}{llll}
    \lsptoprule
                & Realis   & Irrealis   & Goes with... \\ 
                &  /\_C (\_\textit{k}) \textit{{\textasciitilde}} \_V &  /\_C \textit{{\textasciitilde}} \_V  & \\ \midrule
         Class I & \textit{i-} (\textit{ya-}) \textit{{\textasciitilde} y-} & \textit{in-} & transitive, active and inactive\\
        Class II & \textit{Ø-} & \textit{in-} & active and some inactive\\
        Class III & \textit{ti-} (\textit{ta-})\textit{{}-} \textit{{\textasciitilde} t-} & \textit{in- {\textasciitilde} int-} & active and a few inactive\\
        Class IV & \textit{in- {\textasciitilde} n-} & \textit{in-} & inactive\\
        Class V & \textit{Ø-} & \textit{Ø-…-a} & inactive\\
\lspbottomrule
    \end{tabular}
\end{table}
 
The rightmost position where such deviation can be found is \ref{chovpos:antipassvblz18}, which hosts antipassive and causative suffixes. Leaving aside the indirect causative \textit{-jan/-yin}, whose allomorphy is limited and predictable, the antipassive and especially the direct causative suffixes display a strong allomorphy which cannot be predicted on phonological or semantic grounds, see (\ref{bkm:Ref88868113}){}-(\ref{bkm:Ref88868117}); for simplicity, with causatives only bases ending in vowel are shown. Sometimes the same verbal base is acceptable with two distinct allomorphs, as can be seen with \textit{-po-yi} `be full of' in (\ref{bkm:Ref89006803}) and (\ref{bkm:Ref89006805}) (again, the epenthetic \textit{y} inserted between vowels is analyzed as part of the suffix, so in (\ref{bkm:Ref89006805}) \textit{-it} becomes \textit{-yit}).

\ea\label{bkm:Ref88868113}Allomorphs of the antipassive suffix\\
    \ea  Regular antipassive with \textit{{}-jan}, verb \textit{-lej} `wash'\\ {
    \glll {} ta- ka le- \textbf{ja'n}\\ 
    v: \ref{chovpos:activenonactive14} \ref{chovpos:rflxantipassive15} \ref{chovpos:predbase16} \textbf{\ref{chovpos:antipassvblz18}}\\
    {} \Third{}- \Antip{} wash- \textbf{\Antip{}}\\
    \glt`(S)he does the washing'
    }
    \ex  Irregular antipassive with \textit{{}-n}, verb \textit{-jlu} `send'\footnote{Here the demoted object is reintroduced by the instrumental applicative.}\\ {
    \glll {} ta- ka jlọ- \textbf{n}{}- i\\
    v: \ref{chovpos:activenonactive14} \ref{chovpos:rflxantipassive15} \ref{chovpos:predbase16} \textbf{\ref{chovpos:antipassvblz18}} \ref{chovpos:applinstrumental31}\\
    {} \Third{}- \Antip{} send- \textbf{\Antip{}}- \Ap.\Ins{}\\
    \glt `(S)he sends.'
    }
    \ex\label{bkm:Ref89007237} Irregular antipassive with \textit{{}-ki}, verb \textit{-lan} `kill'\\ {
    \glll {} ta{}- ka{}- lan{}- \textbf{ki'}\protect\footnotemark\\ 
    v: \ref{chovpos:activenonactive14} \ref{chovpos:rflxantipassive15} \ref{chovpos:predbase16} \textbf{\ref{chovpos:antipassvblz18}}\\ 
    {} \Third{}- \Antip{}- kill- \textbf{\Antip{}}\\ 
    \glt `(S)he kills.'
    }
    \ex  Irregular antipassive with no suffix, verb \textit{-sinyan} `grill'\\{
    \glll ta- ka sẹnya'n\\ 
    v: \ref{chovpos:activenonactive14} \ref{chovpos:rflxantipassive15} \ref{chovpos:predbase16}\\
    \Third{}- \Antip{} grill\\
    \glt `(S)he makes a barbecue.'
    }
    \z
\z 

\footnotetext{An alternative analysis, perhaps historically more accurate, consists in splitting \textit{{}-ki} into \textit{-k}, the participle of position 17, and \textit{-i(y)}, a verbalizer in position 18 that creates denominal verbs. In any case, the right edge of the domain under discussion would still be position 18.}

\ea\label{bkm:Ref88868117}Allomorphs of the direct causative suffix\\
    \ea  Suffix -\textit{jat}, verb \textit{-nu} `pass by'\\ {
    \glll {} i- nyu- \textbf{jwat}\\
    v: \ref{chovpos:activenonactive14} \ref{chovpos:predbase16} \textbf{\ref{chovpos:antipassvblz18}} \\ 
    {} \Third{}- pass\_by- \textbf{\Caus{}}\\
    \glt `(S)he makes [someone] pass by.' 
    }
    \ex\label{bkm:Ref89006803} Suffix -\textit{nit}, verb \textit{-po-yi} `be full of'\\ {
    \glll {} i- pyo- \textbf{nit}{}- i\\ 
    v: \ref{chovpos:activenonactive14} \ref{chovpos:predbase16} \textbf{\ref{chovpos:antipassvblz18}} \ref{chovpos:applinstrumental31}\\ 
    {} \Third{}- be.full- \textbf{\Caus{}}- \Ap.\Ins{}\\
    \glt `(S)he fills with it.'
    }
    \ex\label{bkm:Ref89006805} Suffix -\textit{it}, verb \textit{-po-yi} `be full of'\\ {
    \glll {} i- pyo- \textbf{yit}{}- i\\
    v: \ref{chovpos:activenonactive14} \ref{chovpos:predbase16} \textbf{\ref{chovpos:antipassvblz18}} \ref{chovpos:applinstrumental31}\\ 
    {} \Third{}- be.full- \textbf{\Caus{}}- \Ap.\Ins{}\\
    \glt `(S)he fills with.'
    }
    \ex  Suffix -\textit{jVnit}, verb \textit{-pu} `exist'\\ {
    \glll {} i- pyu- \textbf{junit}\\
    v: \ref{chovpos:activenonactive14} \ref{chovpos:predbase16} \textbf{\ref{chovpos:antipassvblz18}}\\
    {} \Third{}- exist- \textbf{\Caus{}}\\
    \glt `(S)he creates, brings into existence, imports.'
    }
    \ex  Suffix -\textit{t}, verb \textit{-'uy} `enter, get in(to)'\\ {
    \glll {} 'yu- \textbf{t}\\ 
    v: \ref{chovpos:activenonactive14}:\ref{chovpos:predbase16} \textbf{\ref{chovpos:antipassvblz18}}\\ 
    {} 3:be.full- \textbf{\Caus{}}\\
    \glt `(S)he puts (it) in (e.g. a pocket).'
    }
    \z 
\z 


No deviations of this kind occur beyond position \ref{chovpos:antipassvblz18}. In position \ref{chovpos:concord19} there appear the concord suffixes of 1\textsc{pl}. Although this morpheme has at least three allomorphs, their distribution is phonologically conditioned: \textit{-Vk} after \textit{-j}, \textit{-k} after V and \textit{-a(j)} elsewhere, e.g. \textit{alej-ek} `we wash', \textit{awo-k} `we fish' and \textit{a'wen-a(j)} `we see', respectively.

The exponence of irrealis could also be seen as a case of deviation from biuniqueness, since it is realized both through a suffix \textit{-a} and a distinct set of personal prefixes. However, it is of a different kind, because the different forms appear in different positions - \ref{chovpos:activenonactive14} and \ref{chovpos:irrealis23}. Besides, the occurrence of one or the other exponent is fully predictable on categorial grounds (see (\ref{bkm:Ref90043129}) and text above). In sum, the diagnostic analyzed up to now is inapplicable to irrealis exponence, which is better treated as a distinct morphological category rather than as a question of allomorphy.

The previous account describes a language-specific fracture of the diagnostic, called \textit{inflectional class} in \tabref{tab:chor:key:4} (§\ref{bkm:Ref88847346}) because its left boundary includes prefixes that define the inflectional class of the verb. An alternative fracture is possible, where only ``derivational'' morphology is included; this is referred to as \textit{fossilization} in \tabref{tab:chor:key:4}. In this case, the right edge is still position \ref{chovpos:antipassvblz18}, but the inflectional personal prefixes of position \ref{chovpos:activenonactive14} would fall outside the span. The antipassive morpheme of position \ref{chovpos:rflxantipassive15} too, but for a different reason: even if considered derivational, its allomorphy seems to be predictable. The basic allomorph is \textit{ka}, as shown in (\ref{bkm:Ref88868113}); if the base begins with a glottal stop followed by a stressed vowel, an epenthetic \textit{n} is inserted, which fuses with the glottal in \textit{'n}, cf. \textit{taka}\textbf{\textit{'n}}\textit{eyasan} `teach (intransitive)' < \textit{ta-ka}\textbf{\textit{n-}}\textit{'éya*san}, cf. \textit{'yiyas} `(s)he teaches [someone]' < \textit{y-'éya*san}, while with bases beginning with a vowel its form is \textit{k-} e.g. \textit{ta-}\textbf{\textit{k-}}\textit{ámtijnye'n} (\ref{chovpos:activenonactive14} \ref{chovpos:rflxantipassive15} \ref{chovpos:predbase16}:\ref{chovpos:antipassvblz18}) `(s)he talks', cf. the transitive \textit{y-amti-'ni} (\ref{chovpos:activenonactive14} \ref{chovpos:predbase16} \ref{chovpos:pluractionaliterative38}) `(s)he talks [about someone]'. Thus, the left edge is the predicate head of position \ref{chovpos:predbase16}, and the span is defined as \ref{chovpos:predbase16}{}-\ref{chovpos:antipassvblz18}.

\largerpage
\hspace*{-2.1pt}There is a third fracture which considers \textit{extended exponence}. This phenomenon is seen in the personal prefixes of position \ref{chovpos:activenonactive14} on the left, and on the verbal third person plural marker \textit{-is} in position \ref{chovpos:concord29} on the right, both of which are exponents of third person; \textit{-is} cross{}-references third person subject of transitives and also intransitives with an oblique introduced by an adposition or applicative, see (\ref{bkm:Ref89007017}). Notice there are other cases of extended exponence between these edges: the antipassive morpheme \textit{ka} in position \ref{chovpos:rflxantipassive15} may determine changes in the root, as in \ref{bkm:Ref89007237}, where the root becomes deaccented, as well as the verbal plural markers of positions \ref{chovpos:concord19} and \ref{chovpos:concord20}, which are specific to first and second person, respectively, and thus show extended exponence of these features.

\ea\label{bkm:Ref89007017}Extended exponence - third person: subspan \ref{chovpos:activenonactive14}{}-\ref{chovpos:concord29}\\
\glll {} \textbf{y}{}- am- \textbf{is}{}- i (ja- pa jlọsye)\\ 
v: \textbf{\ref{chovpos:activenonactive14}} \ref{chovpos:predbase16} \textbf{\ref{chovpos:concord29}} \ref{chovpos:applinstrumental31} (\ref{chovpos:heavyDP45} \ref{chovpos:heavyDP45} \ref{chovpos:heavyDP45})\\ 
{} \textbf{\Third{}}- go.away- \textbf{\Third\Pl{}}- \Ap.\Ins{} f- \Dem{} girl  \\
\glt `They take (the girl) away.' 
\z 

\largerpage[2]
\section{Phonological diagnostics}
\label{bkm:Ref82962099}
\subsection{Accent (\ref{chovpos:rflxantipassive15}{}-\ref{chovpos:predbase16}; \ref{chovpos:predbase16}{}-\ref{chovpos:applipcntloc34}; \ref{chovpos:complementizer4}{}-\ref{chovpos:tempaspdisc40})}

Three subtypes are considered under this rubric. \textit{Accent minimal-minimal} is defined as the minimal span containing the positions where the accent can appear in utterances with only one accent. In such cases, the accent falls almost always on the verbal root or non-verbal predicate of position \ref{chovpos:predbase16}, but in fossilized, irrregular antipassives, it falls on the antipassive morpheme in \ref{chovpos:rflxantipassive15}, as in (\ref{bkm:Ref89007237}). Thus the span is \ref{chovpos:rflxantipassive15}{}-\ref{chovpos:predbase16}. 

The \textit{minimal-maximal} subtype considers the maximal span where no position other than the predicate head, that is, position \ref{chovpos:predbase16}, can bear stress. Since position \ref{chovpos:rflxantipassive15} can bear the stress alongside with position \ref{chovpos:predbase16} in regular and some irregular antipassives, as well as in in reflexives, (cf. in \textit{i-ní 'wé'en} (\ref{chovpos:activenonactive14}{}-\ref{chovpos:rflxantipassive15} \ref{chovpos:predbase16}) `(s)he sees himself/herself') then it should be excluded from the span and the left edge should be placed in position \ref{chovpos:predbase16}. As for the right edge, the oblique second person pronoun \textit{-a} of position \ref{chovpos:oblique35} bears secondary stress and, optionally, another main stress, as in (\ref{bkm:Ref86922217}).\footnote{This oblique pronoun is usually written together with the previous element in Chorote texts. However, according to the convention adopted in this chapter — no more than one stress per orthographic word — it is written separately (see \sectref{bkm:Ref99649617}). The second person plural oblique marker \textit{-(')as} could arguably be segmented as \textit{-(')a-s}, where \textit{-s} would be the plural also found in nouns after a vowel. For the initial glottal stop, see below in this section.} Thus the \textit{minimal-maximal} span is \ref{chovpos:predbase16}{}-\ref{chovpos:applipcntloc34}.

\ea\label{bkm:Ref86922217}Accent \textit{minimal-maximal}: stressed pronoun of position \ref{chovpos:oblique35} as right boundary\\
\glll {} si{}- tyánt'ya \textbf{á}{}- j\\ 
v: \ref{chovpos:activenonactive14} \ref{chovpos:predbase16} \textbf{\ref{chovpos:oblique35}} \ref{chovpos:appl36}\\
{} \First\Sg{}- know \textbf{\Second{}}- \Ap.\Ins{}\\ 
\glt `I know you (sg.)'
\z 

The \textit{maximal-maximal} subtype includes the longest possible span with only one accent. Since positions \ref{chovpos:rflxantipassive15} and \ref{chovpos:oblique35} are not necessarily present in a word, they do not define a boundary in this subtype. The \textit{accent maximal-maximal} subspan is \ref{chovpos:complementizer4}{}-\ref{chovpos:tempaspdisc40}, already exemplified in (\ref{bkm:Ref90130576}), repeated here as (\ref{bkm:Ref85988661}); see also (\ref{bkm:Ref90130557}) and (\ref{bkm:Ref90131499}) for examples of the edges in spans occurring in main clauses.

\ea\label{bkm:Ref85988661}Accent \textit{maximal-maximal}: subspan \ref{chovpos:complementizer4}{}-\ref{chovpos:tempaspdisc40}\\
\glll {} \textbf{\textit{ka}} \textit{Ø{}-} \textit{'nes{}-} \textit{ta{}-} \textbf{\textit{na'a}} \\
v: \textbf{\ref{chovpos:complementizer4}} \ref{chovpos:activenonactive14} \ref{chovpos:predbase16} \ref{chovpos:incompletive27} \textbf{\ref{chovpos:tempaspdisc40}} \\ 
{} \textbf{\Comp{}} \Third.\Irr{}- arrive- \Incomp{}- \textbf{later} \\
\glt `When (s)he/it arrive.'
\z 


Notice that other particles in position \ref{chovpos:tempaspdisc40} are stressed, e.g. \textit{tá'a} `immediately; already', \textit{pet} `please' and others, as well as the locative adverbs of position \ref{chovpos:remotepastperfect39} \textit{ts'ijí}, \textit{'nijí} and others, as well as all the elements occurring to their right.

None of the fractures proposed for this test gives position \ref{chovpos:activenonactive14} as a left limit, which is the position with the highest number of convergences for the lef edge. This is because position 15 is stressed. Otherwise, the left edge for the minimal-maximal subtype would be position 14. In fact, position 14 \textit{is} a left edge in the minimal-maximal subtype, but of a domain that excludes the verb core of position 16 and includes only the verbal prefix of position 14 and the reflexive or the antipassive of position 15. In the following section we will see another potential diagnostic which points to position 14 as a left edge.

\subsection{Another potential diagnostic related to accent (\ref{chovpos:activenonactive14}{}-)}

There is another potential phonological diagnostic which has not been included in \tabref{tab:chor:key:3}. For other languages of the family, namely Wichí and Nivaclé, an iambic rhythmic type has been proposed (\citealt{Nercesian2014}, \citealt{Gutierrez2015}, \citealt{Gutierrez2016}). This can be clearly seen in the following alternation in 'Weenhayek Wichí, where long vowels (written as geminates) regularly correspond to stressed vowels in Chorote: \textit{qasiit} `stand (up)' (imperative) vs. \textit{ta-qaasit} `(s)he/it stands' \citep{Claesson2016}. In \citet{NikulinCarolForthcoming} it is argued that this may have been the default stress pattern in Proto-Mataguayan, activated when no underlying accent is present in a three-mora window at the left margin of the "word" (i.e. of the stress domain). If applied to Chorote, and if the iambic structure should be aligned with the left edge of the verb word, this would suggest a span in which position \ref{chovpos:activenonactive14} constitutes the left edge, since it is the leftmost possible initial syllable of a iamb containing the verbal root; see (\ref{bkm:Ref85989483}), where stress is indicated with an acute accent for the sake of clarity:

\ea\label{bkm:Ref85989483} Iambic structure of the verb\\ 
\glll {} ta- kásit\\
    v: \ref{chovpos:activenonactive14} \ref{chovpos:predbase16}\\ 
    {} \Third{}- stand\\ 
\glt `(S)he stands (up).'
\z 

In Chorote, a default left-aligned iambic type would also explain why most non-possessed nouns bear the stress in the second syllable of the stem, e.g. \textit{ajwén\-ta} `chicken', but possessed nouns in the first one when the stem begins with a consonant and the possessive prefix has the form (C)V-, e.g. \textit{i-pyúsi'} `my beard': in either case, the iambic structure is preserved. There are still some cases of alternation in adpositions that take possessives to indicate the complement: \textit{kyajwẹ́} `under, in the lower part of' versus \textit{ji-kwájẹ} `under it, in its lower part'. This would also explain the rare cases where the stress does not fall in the first vowel of a verbal stem, e.g. \textit{'najwél} `(s)he is shy/ashamed', explained as /n+'ahwél/ (3-be.shy), where the prefix is C-, rather than (C)V, and thus does not add a syllable.\footnote{Under this hypothesis, cases like \textit{y-imi'n} [yími'n] (3-love) `(s)he loves', where the iambic structure is not preserved, are explained by assuming an underlying long vowel in the first syllable of the stem in the proto-language. This is not mere speculation, but what actually happens in present-day 'Weenhayek Wichí, cf. \textit{ya-}\textbf{\textit{huu}}\textit{min} (3-love) `(s)he loves'. In sum, the accent in the proto-language (whatever its nature was) would have fallen in the second \textit{mora}.}

Nevertheless, there is no clear evidence that this iambic pattern is still syncronically productive in Chorote. In verbal stems, the position of the stress is fixed: contrast the 'Weenhayek example above with Chorote \textit{kásit} `stand up' (imperative), \textit{ta-kásit} `(s)he/it stands (up)'. In any case, it is an indication of a metrical criterion for defining the left edge of the phonological word which might have been productive for a long time in the (pre)history of the language.

\subsection{Insertion of /y/ between vowels (\ref{chovpos:predbase16}{}-\ref{chovpos:distaley32}, \ref{chovpos:predbase16}{}-\ref{chovpos:applipcntloc34}, \ref{chovpos:predbase16}{}-\ref{chovpos:adpositionsappl44})}
\label{bkm:Ref89781621}
When two vowels are in hiatus across a base-suffix (or -enclitic) border, epenthetic /y/ is inserted. With only one documented exception,\footnote{The exception occurs between the light verb \textit{-wo} `do, become, be' and the distal \textit{-ey}, where no /y/ is inserted; instead, the allomorph \textit{-y} of the distal is selected, resulting in the form \textit{-wo-y}. An example of this is shown in (\ref{bkm:Ref86938272}).} this applies consistently to any suffix/enclitic element of the predicate up to the distal applicative \textit{-e(y)} of position \ref{chovpos:distaley32}, see \ref{bkm:Ref86008749}; recall that morphemes that can surface both as applicatives or adpositions, like the distal, are glossed here as `P' in either case. The /y/ is considered to be part of the suffix/enclitic in the template, and is not segmented as a different morph elsewhere in this chapter unless explicitely indicated. In positions \ref{chovpos:appllocative33} and \ref{chovpos:applipcntloc34} the Ps begin with a consonant, e.g. \textit{{}-'e} or \textit{{}-k'i}, so there is no context for insertion; however, as these elements belong to the same class as the distal, it seems reasonable to consider some version in which the right edge extends to position \ref{chovpos:applipcntloc34}. To the right of \ref{chovpos:applipcntloc34}, the only case of /y/ insertion occurs between the P of position \ref{chovpos:adpositionsappl44} (acting there as postposition) and its host, the (N)P of position \ref{chovpos:lightDP43}, see \ref{bkm:Ref86008747}.. We take thus that the right edge is position \ref{chovpos:distaley32} in the \textit{minimal-minimal} version, \ref{chovpos:applipcntloc34} in the \textit{maximal-minimal} version, and \ref{chovpos:adpositionsappl44} in the \textit{maximal-maximal} version of the /y/ insertion diagnostic. In the latter case I assume that the NP is part of the verbal domain; see §\ref{bkm:Ref73306870} on the distribution of applicatives/adpositions and NPs. Since /y/ insertion does not occur between a prefix (or proclitic) and a base, the left boundary has to be position \ref{chovpos:predbase16}.

\ea  Insertion of epenthetic /y/\\ 
    \ea\label{bkm:Ref86008749} \textit{Minimal-minimal} - between the applicative of position \ref{chovpos:distaley32} and its host: span \ref{chovpos:activenonactive14}{}-\ref{chovpos:distaley32}\\ {
    \glll {} \textup{/a}\textup{{}-} \textup{ho}\textup{{}-} \textup{ey/}~ \MVRightarrow{}~ o{}- jo{}- \textbf{y}{}- i\\ 
    v: \ref{chovpos:activenonactive14} \ref{chovpos:predbase16} \ref{chovpos:distaley32} \MVRightarrow{} \textbf{\ref{chovpos:activenonactive14}} \ref{chovpos:predbase16} \textbf{\ref{chovpos:distaley32}} \ref{chovpos:distaley32}\\ 
    {} \First{}- go- \Ap.\Dist{} \textsc{\MVRightarrow{}} \First{}- go- \Epen{}- \Ap.\Dist{} \\ 
    \glt`I went there.'
    }
    \ex\label{bkm:Ref86008747} \textit{Maximal-maximal} - between the postposition of \ref{chovpos:adpositionsappl44} and its host: span \ref{chovpos:activenonactive14}{}-\ref{chovpos:adpositionsappl44}\\ {
    \glll {} o- jo [']Iwit'osi- \textbf{y}{}- i /*o- jo- \textbf{y}{}- iwit'osi- \textbf{y}{}- i\\ 
    v: \ref{chovpos:activenonactive14} \ref{chovpos:predbase16} \ref{chovpos:lightDP43} \textbf{\ref{chovpos:adpositionsappl44}} \ref{chovpos:adpositionsappl44} \\
    {} \First{}- go Tartagal- \Epen{}- \Ap.\Dist{} /\First{}- go- \Epen{}- Tartagal- \Epen{}- \Ap.\Dist{}\\ 
    \glt `I went to Tartagal.'
    }
    \z 
\z 

Now let us address some analytical issues that deserve consideration. To the right of position \ref{chovpos:adpositionsappl44} all elements begin with a consonant, so what remains to be considered is only what happens between positions \ref{chovpos:applipcntloc34} and \ref{chovpos:adpositionsappl44}. 

Before an underlying initial vowel, Chorote regularly inserts a glottal stop whenever no /y/ is inserted. This can be seen before the initial vowel of \textit{Iwit'osi} `(the city of) Tartagal' in (\ref{bkm:Ref86008747})\footnote{Actually, in normal/fast speech the first vowel of the N is assimilated to the preceding vowel across the glottal stop, so \textit{Iwit'osi} `Tartagal' becomes ['owit'osi]. Vowel assimilation across laryngeals is a regular process in Chorote.}. We do not expect /y/ insertion there, since \textit{Iwit'osi} is not a suffix or enclitic. We also find that a glottal stop, rather than /y/, is inserted at the beginning of a P in position \ref{chovpos:adpositionsappl44} when it is a polysyllabic adposition, see (\ref{bkm:Ref86079447}) (the relevant inserted glottal stop is added between brackets, since it is not written in Chorote spelling). Contrast this with the insertion of /y/ before position \ref{chovpos:adpositionsappl44} when the P is monosyllabic, as in (\ref{bkm:Ref86008747}).

\ea\label{bkm:Ref86079447}Glottal stop insertion before polysyllabic P in position \ref{chovpos:adpositionsappl44}\\
    \ea  Following an N(P)\\ {
    \glll {} y- i a'lénta [']\textbf{apé'e}\\ 
    v: \ref{chovpos:activenonactive14} \ref{chovpos:predbase16} \ref{chovpos:lightDP43} \textbf{\ref{chovpos:adpositionsappl44}} \\ 
    {} \Third{}- be horse \textbf{\Ap.over}\\
    \glt `(S)he is on the horse.'
    }
    \ex  Not following an N(P)\\ {
    \glll {} i{}- jyo [']\textbf{apé'e} [jlaják tikíjnaki']\\
    v: \ref{chovpos:activenonactive14} \ref{chovpos:predbase16} \textbf{\ref{chovpos:adpositionsappl44}} \ref{chovpos:heavyDP45} - \\ 
    {} \Third{}- go \textbf{\Ap.over} \Dem{} mountain\\ 
    \glt `(S)he climbed that mountain.'
    }
    \z 
\z 

I take this, together with the fact that polysyllabic adpositions can bear stress (unlike monosyllabic ones), as an indication that they make up a stress projecting domain, so they are not phonologically bound elements.\footnote{As one of the editors points out, a question for future work arises here - whether adjacent stress domains form a larger prosodic domain. Perhaps some stressed syllables are stronger than others, forming a larger prosodic domain. If so, adpositions might be candidates for `weaker' syllables.} Thus, /y/ insertion occurs \textit{inside} this stress domain, i.e. between bound elements, and glottal stop insertion \textit{between} these domains.

There are still two places between positions \ref{chovpos:applipcntloc34} and \ref{chovpos:locatives41} where \textit{prima facie} a morpheme-initial vowel may occur: in position \ref{chovpos:oblique35}, with the second person oblique pronoun \textit{-a} (singular), \textit{-as} (plural) seen in (\ref{bkm:Ref86922217}) and repeated here as (\ref{bkm:Ref86924174}), and position \ref{chovpos:remotepastperfect39}, with the perfect \textit{{}-Vje(j)/-V…je(j)}, seen in (\ref{bkm:Ref86924177}). In both cases a glottal stop occurs. 

\ea\label{ex:chor:key:29} Inserted or underlying glottal stop?\\
    \ea\label{bkm:Ref86924174} Before oblique second person marker\\ {
    \glll {} si- tyant'ya \textbf{[']a}{}- (j)\\ 
        v: \ref{chovpos:activenonactive14} \ref{chovpos:predbase16} \ref{chovpos:oblique35} \ref{chovpos:appl36} \\ 
        {} \First{}- know \Second{}- \Ap.\Ins{} \\
    \glt `I know you (sg.).'
    }
    \ex\label{bkm:Ref86924177} Before the perfect marker \\ {
    \glll {} y- am- a- \textbf{'aja}\\
        v: \ref{chovpos:activenonactive14} \ref{chovpos:predbase16} \ref{chovpos:momentary22} \ref{chovpos:remotepastperfect39} \\ 
        {} \Third{}- go.away- \Mom{}- \textbf{\Prf{}}\\
    \glt `(S)he/it left again.'
    }
    \z 
\z 


It is difficult to determine whether the initial glottal stop in (\ref{ex:chor:key:29}) is inserted or underlying. In \REF{bkm:Ref86924174} there is some evidence to consider it underlying, i.e. /'a*/, not /a*/. The evidence for treating the glottal stop here as underlying is as follows; (i) the glottal stop here triggers glottalization of a preceding
/s/ into /ts'/, like an underlying glottal stop, and unlike an epenthetic glottal stop: e.g. \textit{'ẹ}\textbf{\textit{s}} + \textbf{\textit{'}}\textit{asé(j)} \MVRightarrow{} \textit{ẹs}\textbf{\textit{ts'}}\textit{yase} `it is good for you (pl.)' (with further palatalization due to the previous vowel, see below), and (ii) the glottal stop here labializes after a rounded vowel, like an underlying glottal stop, and unlike an epenthetic glottal stop: i.e. /'/ \MVRightarrow{} /'w/, e.g. \textit{ijy}\textbf{\textit{o}} + \textbf{\textit{'}}\textit{asé(j)} \MVRightarrow{} \textit{ijy}\textbf{\textit{o}'w}\textit{asé(j)} `(s)he goes to you'. These two processes are not documented in my material with epenthetic glottal stops, but they are with the locative P \textit{{}-'e} of position \ref{chovpos:appllocative33}, whose initial glottal stop is underlying beyond any doubt , e.g. \textit{yiyi}\textbf{\textit{s}} + \textbf{\textit{'}}\textit{e} \MVRightarrow{} \textit{yiyits'i'} `(they) are in…'; \textit{'y}\textbf{\textit{u}} + \textbf{\textit{'}}\textit{e} \MVRightarrow{} \textit{'yu}\textbf{\textit{'w}}\textit{e'} `it fits in…'.\footnote{The underlying character of the glottal stop in \textit{-'e} can be seen in the contrast with the distal and other Ps with initial subjacent vowel, which expectedly take /y/, while \textit{-'e} does not: \textit{yi+ -}\textbf{\textit{'e}} \textrm{\MVRightarrow{}} \textbf{\textit{yi'i'}} `(s)he/it is at (a precise place)' vs. \textit{yi} + \textbf{\textit{-ey}} \textrm{\MVRightarrow{}} \textbf{\textit{yiyi}} `(s)he/it arrives (at a distant place)'.} 

Alternatively, one could consider that the morphemes of \REF{ex:chor:key:29} belong to a distinct accent projecting domain, like polysyllabic adpositions, and hence glottal stop instead of /y/ is inserted there. There is some historical evidence for this.\footnote{The Wichí cognates for this second person marker show a long vowel \citep{Claesson2016}, i.e. a bimoraic morpheme, like polysyllabic adpositions.} In sum, even if there is no conclusive evidence for glottal stop insertion between positions \ref{chovpos:applipcntloc34} and \ref{chovpos:locatives41}, there is no evidence at all for /y/ insertion, so the right boundaries for the three versions of this diagnostic hold as determined above, namely as positions \ref{chovpos:distaley32}, \ref{chovpos:applipcntloc34}, and \ref{chovpos:adpositionsappl44}.

As for the left edge, since /y/ insertion applies to suffixes only, the left edge can only be the predicate head itself; (\ref{bkm:Ref87643628}) shows that /y/ insertion does not apply in prefix-base boundary (/y/ is segmented there as distinct morpheme for the sake of clarity). Note that in what follows the first line contains a surface form, and the second line an underlying form to which the  processes under discussion apply.

There is no evidence of /y/ being inserted between two positions to the left of the predicate (although it is inserted \textit{inside} positions, e.g when the position contains an N(P), such as in positions \ref{chovpos:conjunct1} through \ref{chovpos:foc3}). Therefore, in any version of this diagnostic the left edge is position \ref{chovpos:predbase16}.

\ea\label{bkm:Ref87643628} No /y/ insertion without suffixation/encliticization\\
    \ea  Before the personal prefixes\\ {
    \gllll {} ø{}- ẹmi'n / *a- \textbf{y}{}- imi'n\\ 
    {} /a- imin/ \\
    v: \ref{chovpos:activenonactive14} \ref{chovpos:predbase16} \\ 
   {} \First{}- love / *\First{}- \Epen{}- love\\
    \glt `I love it/him/her.'
    }
    \ex\label{bkm:Ref89180056} Before the reflexive-reciprocal base\\ {
    \gllll {} a{}- nín{}- ẹmi'n / *a{}- ni \textbf{y}{}- ími'n \\ 
    {} /a{}- ni imin/ \\ 
    v: \ref{chovpos:activenonactive14} \ref{chovpos:rflxantipassive15} \ref{chovpos:predbase16} \\ 
    {} \First{}- \Refl{} love / *\First{}- \Refl{}- \Epen{}- love\\
    \glt `I love myself.'
    }
    \z 
\z 

This diagnostic has two serious limitations: (i) it cannot be used to define the left edge of a word, since it applies to suffixes/enclitics only, and (ii) it cannot be applied to any of the positions where only items beginning with a consonant exist. 

However, an alternative formulation might be interesting. In effect, notice that in (\ref{bkm:Ref89180056}), instead of the glottal stop, an epenthetic \textit{n} is inserted between the final vowel of the reflexive/reciprocal of position \ref{chovpos:rflxantipassive15} and the initial one of the verb. Therefore, if the diagnostic were formulated as ``non glottal stop insertion'', rather than ``/y/ insertion'', position \ref{chovpos:rflxantipassive15} should be added to the span as its left edge.

\subsection{Palatalization (\ref{chovpos:activenonactive14}{}-\ref{chovpos:predbase16}; \ref{chovpos:activenonactive14}{}-\ref{chovpos:tempaspdisc40}; \ref{chovpos:activenonactive14}{}-\ref{chovpos:antipassvblz18}/\ref{chovpos:indirectevidential25}; \ref{chovpos:activenonactive14}{}-\ref{chovpos:tempaspdisc40}; \ref{chovpos:activenonactive14}{}-\ref{chovpos:predbase16}; \ref{chovpos:activenonactive14}{}-\ref{chovpos:heavyDP46})}

Underlying /i, y/ palatalize all consonants, while epenthetic [i] and underlying /u/ palatalize only coronals. The former is referred to as ``first palatalization'' and the latter as ``second palatalization''. Thus, for instance, prefixes of the form \textit{i-} (possessive and irrealis active first person, \textit{realis} active third person) palatalize any consonant because \textit{i} is underlying there (first palatalization), but prefixes of the form \textit{Ci} (nominal and verbal) palatalize coronals only, because the \textit{i} in such cases is not underlying but derived (second palatalization) \citep{Carol2014}.

Palatalization usually means \textit{C} \MVRightarrow{} \textit{Cy}, but also /w/ \MVRightarrow{} /y/ before rounded vowels, /ky/ \MVRightarrow{} /sy/, and /k'y/ \MVRightarrow{}/ts'y/ (notice that /k\textsuperscript{(}'\textsuperscript{)}/ and /k\textsuperscript{(}'\textsuperscript{)}y/ are distinct phonemes; surface \textit{k\textsuperscript{(}}\textit{'\textsuperscript{)}}\textit{i} reflects subjacent /ky\textsuperscript{(}'\textsuperscript{)}i/). Neither the context nor the process itself are always transparent for two reasons: (i) /y/ is regularly dropped in coda after triggering palatalization, and (ii) \textit{Cy} causes raising of a following \textit{e} into \textit{i}, among other vowel changes, and thus /Cye/ appears superficially as \textit{Ci}; see (\ref{bkm:Ref86938272}). 

\ea\label{bkm:Ref86938272}Palatalization of /w/, deletion of /y/ in coda and /e/ \MVRightarrow{} /i/ after palatal \\
\gllll {}  i- yo- ø pi \\ 
    {} /i{}- wo{}- y peh/ \\ 
    v: \ref{chovpos:activenonactive14} \ref{chovpos:predbase16} \ref{chovpos:distaley32} \ref{chovpos:remotepastperfect39} \\
    {} \Third{}- \Lv{}- \Ap.\Dist{} \Rem.\Pst{}\\ 
\glt `(S)he said/wanted/did it.'
\z 

A number of domains can be identified based on palatalization phenomena in Iyojwa'aja' Chorote. On the one hand, diagnostics can be subdivided into \textit{minim}\textit{al} and \textit{maximal}: a set of \textit{minimal} subtypes that define contiguous subspans of positions that trigger and/or undergo palatalization whenever the relevant context exists, and a set of \textit{maximal} ones that define the largest possible span where \textit{all} the occupied positions trigger or undergo palatalization (in other words, a span outside of which no palatalization is known to occur inside the verbal template). On the other hand, the diagnostics can be classified according to the target and environment of palatalization. We can consider A) the ``first'' palatalization as a whole, B) the ``first'' palatalization excluding that of /k\textsuperscript{(}'\textsuperscript{)}y/,which is somewhat exceptional, and C) palatalization of coronals only, regardless of whether they are affected by the first or the second palatalization rule. In all, six different domains arise; see \tabref{tab:chor:key:3} for a summary of these tests.

\begin{table}[htp]
\centering
\caption{Palatalization diagnostics}
\label{tab:chor:key:3}
\begin{tabular}{llll}
\lsptoprule
\textbf{Subtype} & \textbf{Specific} \textbf{fracture} & \textbf{Left} \textbf{edge} & \textbf{Right} \textbf{edge}\\ \midrule
Minimal-A & With k\textsuperscript{(}'\textsuperscript{)}y & \ref{chovpos:activenonactive14} & \ref{chovpos:predbase16}\\
Maximal-A & With k\textsuperscript{(}'\textsuperscript{)}y & \ref{chovpos:activenonactive14} & \ref{chovpos:tempaspdisc40}\\
Minimal-B & Without k\textsuperscript{(}'\textsuperscript{)}y & \ref{chovpos:activenonactive14} & \ref{chovpos:antipassvblz18}/\ref{chovpos:indirectevidential25}\\
Maximal-B & Without k\textsuperscript{(}'\textsuperscript{)}y & \ref{chovpos:activenonactive14} & \ref{chovpos:tempaspdisc40}\\
Minimal-C & Coronals only & \ref{chovpos:activenonactive14} & \ref{chovpos:predbase16}\\
Maximal-C & Coronals only & \ref{chovpos:activenonactive14} & \ref{chovpos:heavyDP46}\\
\lspbottomrule
\end{tabular}
\end{table}

\newpage
The need for postulating the subtype B is based on the absence of palatalization of \textit{-k} in positions \ref{chovpos:ptcp17} and \ref{chovpos:concord19}, which is unexpected, because no other diagnostic places boundaries there. The absence of palatalization in position \ref{chovpos:ptcp17} is especially surprising if we consider that the position corresponds to `derivational' morphology. Moreover, the \textit{j}{}-initial causatives in position \ref{chovpos:antipassvblz18} do show palatalization: /i-limi-\textbf{hat}/ \MVRightarrow{} \textit{i-limi-}\textbf{\textit{jyet}} (3-be.white-\textsc{caus}) `make white'. These data suggest that it is reasonable to set aside the palatalization of /k\textsuperscript{(}'\textsuperscript{)}y/ as a special case, thus justifying the B subtype of the diagnostic.

For any version of this diagnostic the left edge is position \ref{chovpos:activenonactive14}, which hosts prefixes (or proclitics) that trigger palatalization (see e.g. (\ref{bkm:Ref86938272})) but do not undergo palatalization of any kind, e.g. \textit{t-amti'} `(s)he speaks' is never realized  as *\textit{ty-amti'}, not even when preceded by /i, y/. In the \textit{minimal} subtypes, the contiguous subspan is thus necessarily interrupted in position \ref{chovpos:activenonactive14}. 

It is true that there is no direct evidence that palatalization cannot operate between positions \ref{chovpos:kyak13} and \ref{chovpos:activenonactive14}, because the pronouns of position \ref{chovpos:kyak13} never end in /i/ or /y/; hence, they provide no context for palatalization. But neither other NPs nor any other material ending in /i/ or /y/ trigger palatalization of the prefixes of position \ref{chovpos:activenonactive14}, so there is no reason to suppose that an eventual pronoun in position 13 ending in /i, y/ would. Furthermore, since position \ref{chovpos:kyak13} is occupied by a tonic pronoun which is arguably an NP and can occupy other positions as well, I find no reason to suspect that the relationship between positions \ref{chovpos:kyak13} and \ref{chovpos:activenonactive14} should be any different from that between position \ref{chovpos:activenonactive14} and any other position to its left.

As for the \textit{maximal} subtypes, position \ref{chovpos:activenonactive14} is also the left edge for verbal non-imperative predicates, where position \ref{chovpos:activenonactive14} is obligatory. In reflexive-reciprocal and antipassive verbs in imperative mode position \ref{chovpos:activenonactive14} is empty but position \ref{chovpos:rflxantipassive15} is occupied by the reflexive/reciprocal \textit{ni} and the antipassive \textit{ka}.\footnote{Of course, it could be argued that the position of the personal prefixes is not actually empty in imperatives, but occupied by an abstract morpheme with zero exponence. Some indirect support for this is found in the Manjúi variety, where an optional \textit{a-} second person prefix for imperative exists. In any case, this implies no changes for the diagnostics.} If these morphemes were shown to be palatalized by some element to the left of position \ref{chovpos:activenonactive14}, the position of that element would be considered to be the left edge. However, there is no way to prove this, since \textit{ka} and \textit{ni} can never be target of palatalization. The regular antipassive marker \textit{ka} does not palatalize because /k/ (unlike /ky/) never palatalizes in the variety under consideration,\footnote{In the Montaraz varieties of Argentina and Paraguay (known as Iyo'awújwa' and Manjúi respectively), in contrast, /k/ palatalizes, thus the antipassive becomes \textit{kya} after prefixes of the form \textit{i-} (\citealt{Gerzenstein1983}; \citealt{Carol2018}). However, even in these varieties \textit{ka} is not palatalized by elements to the left of position 14.} while the reflexive/reciprocal morpheme \textit{ni} should be analyzed as underlying  /yne/ or /yni/, and thus  palatalization cannot be applied.\footnote{We can infer this from the following: \textit{ni} is stressed, and stressed \textit{i} only surfaces after a palatal(ized) phone (underlying /i/ is otherwise \textit{ẹ}, a closed mid or very open high vowel). Thus, the previous /n/ must be palatalized, which in turn supposes a previous /y/ which regularly falls in coda.


In the Montaraz varieties the reflexive/reciprocal is \textit{wet} and can be palatalized by a personal prefix of the form /i-/ in \textit{wit} or \textit{yit}, depending on the variety. However, it has not been documented that elements to the left of position 14 are able to palatalize \textit{wet}.} In non-verbal predicates, where position \ref{chovpos:rflxantipassive15} is also empty, the left edge could be pushed further to the left if the non-verbal predicate of position \ref{chovpos:predbase16} could undergo palatalization. But this is not the case, hence position \ref{chovpos:activenonactive14} remains as the left edge.

What remains to be considered are the right edges. Let us address the A{}-set first. The \textit{minimal{}-A} palatalization is limited by the participle /-ky/ of position \ref{chovpos:ptcp17}, which does not undergo palatalization, as can be seen in (\ref{cho:ex:lackofpalatalization}), so the span is \ref{chovpos:activenonactive14}{}-\ref{chovpos:predbase16}. I found no clear examples of a context of palatalization for the participle in the verbal domain, hence (\ref{cho:ex:lackofpalatalization}) comes from the nominal domain; recall also that palatalized phones neutralize with plain ones in coda position, so only examples with onsets are shown.\footnote{That there is a context for palatalization in \REF{cho:ex:lackofpalatalization} can be seen in the contrast with \textit{t-amti-}\textbf{\textit{ts'i}}\textit{{}-ji'n} `they speak', with the same root, where \textit{-ts'i} is the palatalized allomorph of the distributive \textit{-k'i} of position 34.} (In (\ref{cho:ex:lackofpalatalization}) /Cye/ \MVRightarrow{} \textit{Ci}; if we had /k/ instead of /ky/ we would expect *\textit{amtike}.)

\ea\label{cho:ex:lackofpalatalization}Lack of palatalization in the participle -\textit{ky} of position \ref{chovpos:ptcp17}.\\ 
\gll y- amti- \textbf{ky}{}- e'\\
\First\Sg.\Poss{}- speak- \textbf{\Pp{}}- \Irr{}\\
\glt `my speech (irrealis).'
\z 


The \textit{maximal{}-A} version reaches the TAM particles of position \ref{chovpos:tempaspdisc40}, which undergo first palatalization: \textit{pet} `please'\footnote{The translation `please' (\textit{por favor}) was suggested to me with imperatives. In other cases, however, it is very difficult to find an equivalence. It seems to indicate a benefit for some participant.} \MVRightarrow{} \textit{pit}, \textit{kyu} `a while' \MVRightarrow{} \textit{syu}, -\textit{na'a} `later' \MVRightarrow{} \textit{{}-nye'e}, etc. The subspan is thus \ref{chovpos:activenonactive14}{}-\ref{chovpos:tempaspdisc40}, see examples of the right edge in (\ref{bkm:Ref90244498}).

\ea\label{bkm:Ref90244498}Palatalization \textit{Maximal-A}: right edge in position \ref{chovpos:tempaspdisc40}\\
    \ea   {
    \gllll  {} jwel- i \textbf{syu'}.\\
    {} /hwel{}- ey kyu/\\
    v: \ref{chovpos:predbase16} \ref{chovpos:distaley32} \textbf{\ref{chovpos:tempaspdisc40}}\\ 
   {} tell- \Ap.\Dist{} a.while\\
    \glt`Tell him/her.'
    }
    \ex   {
    \gllll {} kyak iyo- Ø \textbf{pit}\\
    {} /i{}- wo{}- y pet/\\ 
    v: \ref{chovpos:indirectevidentialmirative11} \ref{chovpos:activenonactive14} \ref{chovpos:predbase16} \ref{chovpos:distaley32} \textbf{\ref{chovpos:tempaspdisc40}}\\ 
   {} \Dem{} \Third{}- \Lv{}- \Ap.\Dist{} ??\\
    \glt `This is the way it is.'
    }
    \z
\z 

The locatives \textit{ts'ijí}, \textit{'nijí} of position \ref{chovpos:locatives41} do not provide a target for palatalization. The NPs/DPs to the right of position \ref{chovpos:locatives41} do, but first palatalization is not documented there. First palatalization does occur in the postpositions in position \ref{chovpos:adpositionsappl44}, but is always triggered by their complement in position \ref{chovpos:lightDP43} which in turn does not palatalize, so it cannot define a maximal subspan.

Subtype B ignores palatalization of /k\textsuperscript{(}'\textsuperscript{)}/, which fails to occur not only in position \ref{chovpos:ptcp17} but also in positions \ref{chovpos:concord19} and \ref{chovpos:oblique35}, see (\ref{bkm:Ref90223831}).\footnote{This suggests considering that at least the palatalization of /k\textsuperscript{(}'\textsuperscript{)}y/ might be a `lexical' process, i.e. one that allows exceptions, so that the lack of palatalization in a certain position does not necessarily place it outside the word, if this is a valid wordhood diagnostic. The lack of palatalization of /k\textsuperscript{(}'\textsuperscript{)}y/ in position \ref{chovpos:oblique35}, however, is not as surprising as that of positions \ref{chovpos:ptcp17} and \ref{chovpos:concord19}, since this position is outside the boundaries determined by several diagnostics, and also /ts'/ fails to palatalize in position \ref{chovpos:oblique35}.}

 That \textit{-k'i} is in a palatalization context in \ref{bkm:Ref90246066}. can be seen in the contrast with \textit{'yen-a-}\textbf{\textit{jyi'n}} (look-2pl-\textit{JEN}) `watch', where \textit{-jen\#} \MVRightarrow{} \textit{{}-jyi'n} after the second person plural marker which is underlyingly /ay/; for \ref{bkm:Ref90244880}. it can be seen in the contrast with \textit{ti jna-}\textbf{\textit{jyi'}} (3-be.straight-P\textsc{\textsubscript{Ploc}}) `it goes straight', where \textit{-ji\#} \MVRightarrow{} \textit{{}-jyi'} after the root which is underlyingly /hnay/. Notice that if palatalization took place in \ref{bkm:Ref90244880}. we would see its traces in the regular \textit{e} \MVRightarrow{} \textit{i} after a palatal and for \ref{bkm:Ref90246072}.

\ea\label{bkm:Ref90223831}Lack of palatalization of \textit{k\textsuperscript{(}}\textit{'\textsuperscript{)}}\textit{y}\\
    \ea\label{bkm:Ref90246072} Concord first person plural: position \ref{chovpos:concord19}\\ {
    \glll {} ø{}- amti- \textbf{k-} i  (*- \textbf{s}{}- i)\\ 
    v: \ref{chovpos:activenonactive14} \ref{chovpos:predbase16} \textbf{\ref{chovpos:concord19}} \ref{chovpos:distaley32}\\ 
    {} \First{}- speak- \textbf{\First\Pl{}}- \Ap.\Dist{}\\ 
    \glt `we talke(d) about (it)'
    }
    \ex\label{bkm:Ref90246066}Oblique marker first person singular: position \ref{chovpos:oblique35} \\ {
    \gllll {} wen- a- \textbf{k'i-} 'm    (*-\textbf{ts'i}{}-'m)\\ 
    {} /wen- ay- k'V- m/ \\ 
    v: \ref{chovpos:predbase16} \ref{chovpos:concord20} \textbf{\ref{chovpos:oblique35}} \ref{chovpos:appl36}\\ 
    {} give- \Second\Pl{}- \First\Sg{}- \Ap.\Loc{}\\
    \glt `give (it to) me'
    }
    \ex\label{bkm:Ref90244880}Oblique marker first person plural: position \ref{chovpos:oblique35}\\ {
    \gllll {} ti{}- jna- \textbf{ts'e-} 'm     (*-\textbf{ts}'i- m)\\ 
     {} /t- hnay- \textbf{ts'e}{}- m/\\ 
    v: \ref{chovpos:activenonactive14} \ref{chovpos:predbase16} \textbf{\ref{chovpos:oblique35}} \ref{chovpos:appl36}\\ 
    {} \Third{}- be.right- \textbf{\First\Pl{}}- \Ap.\Loc{}\\
    \glt `it is our job, it is proper for us'
    }
    \z 
\z 


The \textit{minimal-B} subtype has a right edge in position \ref{chovpos:antipassvblz18}. The causatives \textit{-jan, -jat} of position \ref{chovpos:antipassvblz18} palatalize to \textit{-jyen, -jyet}, as exemplified above. In positions \ref{chovpos:concord19} through \ref{chovpos:indirectevidential25} there is no way to know whether palatalization applies, either because the relevant morphemes begin in a vowel or for other reasons.\footnote{The morphemes in positions \ref{chovpos:concord20} through \ref{chovpos:irrealis23} begin in a vowel or in \textit{Ci}, and thus contain no target for palatalization. The evidential of \ref{chovpos:indirectevidential25} is underlyingly /-t'ey/, but since it is unstressed, it becomes \textit{-t'i(y)}, so again a target is lacking. And the reportative of \ref{chovpos:reportativejen24} is scarcely documented in my material in this position, and I am not able to confirm or discard palatalization.} A positive instance of absence of palatalization is the mirative -\textit{p'an} of position \ref{chovpos:mirative26}, which never palatalizes into *\textit{pyan} or *\textit{pyen}; contrast this with (\ref{bkm:Ref90247058}) with \textit{'wanjli-jen} \MVRightarrow{} \textit{'wanjli-}\textbf{\textit{jyi'n}} `(s)he rests', where the same root triggers palatalization of \textit{-jen} (position \ref{chovpos:pluractionaldownwards37}).

\ea\label{bkm:Ref90247058}No palatalization in position \ref{chovpos:mirative26} \\ 
\glll {} ø{}- 'wanjli- \textbf{p'an}{}- e\\ 
    v: \ref{chovpos:activenonactive14} \ref{chovpos:predbase16} \textbf{\ref{chovpos:mirative26}} \ref{chovpos:applinstrumental31}\\ 
    {} \Third{}- remain- \textbf{\Mir{}}- \Ap.\Ins{}\\ 
\glt `also..!', `even…!' (Spanish \textit{incluso}, \textit{hasta})
\z 

The \textit{maximal-B} subtype defines again the span \ref{chovpos:activenonactive14}{}-\ref{chovpos:tempaspdisc40}, like the \textit{maximal-A}, see examples in (\ref{bkm:Ref90244498}). 

The subtype C considers palatalization of coronals only, which can be triggered by underlying but also by some derived /i/, and also by /u/. 

The \textit{minimal}{}-C version gives just the span \ref{chovpos:activenonactive14}{}-\ref{chovpos:predbase16}. A positive boundary is the oblique first person plural \textit{{}-ts'e} of position \ref{chovpos:oblique35}, which fails to palatalize, as shown in \ref{bkm:Ref90244880}. But between \ref{chovpos:predbase16} and \ref{chovpos:oblique35} there is no target for palatalization, or any other way to verify whether palatalization would apply. Most morphemes do not begin in a coronal; others do (positions \ref{chovpos:perdurative21} and \ref{chovpos:indirectevidential25}) but they have the form \textit{Ci}, and although the concord marker of position \ref{chovpos:concord29} \textit{-is} shows palatalization in \textit{-isy} before a vowel, the trigger is morpheme-internal, and thus it should not define an edge. 

Finally, the \textit{maximal}{}-C version gives the subspan \ref{chovpos:activenonactive14}{}-\ref{chovpos:heavyDP46}. Palatalization affects the initial phonemes of the DPs in positions \ref{chovpos:heavyDP45}{}-\ref{chovpos:heavyDP46} when they begin in a coronal: the demonstratives of the form \textit{Ca} appear as \textit{Ci} (where C=coronal), and the demonstratives \textit{jlaja}, \textit{jla'a} as \textit{jlyaCa} or \textit{jliCa}, see (\ref{bkm:Ref90247745}).\footnote{The basic allomorphs are \textit{-a(j)} for `orientation' and \textit{-e(j)} for the instrumental. After a vowel, epenthetic /y/ is inserted and included as part of the Ps. This /y/ in turn raises front vowels, as explained in \sectref{cho:sec:languageandspeakers}, thus the Ps result in \textit{-yej}, \textit{{}-yij}, respectively. The P `orientation' takes here a suppletive form when the pronominal complement surfaces.} Although the examples available are for palatalization in position \ref{chovpos:heavyDP45} only, it seems reasonable to extend the edge to the following position \ref{chovpos:heavyDP46}, since it hosts elements of the same class, i.e. heavy DPs.

\ea\label{bkm:Ref90247745}Palatalization \textit{Maximal-C}: right edge in the DPs of \ref{chovpos:heavyDP45}{}-\ref{chovpos:heavyDP46}\\
\glll {} y- i- 'i [\textbf{jlyaja} Orán]\\
    v: \ref{chovpos:activenonactive14} \ref{chovpos:predbase16} \ref{chovpos:applipcntloc34} \textbf{\textup{\ref{chovpos:heavyDP45}}} - \\ 
    {} \Third{}- be- \Ap.\Punct{} \textbf{\Dem{}:f} Orán\\
\glt `[It] is in Orán.' (\citealt[100]{DraysonGomez2000})
\z 


\section{Ciscategoriality revised}
\label{bkm:Ref88847156}
A domain of ciscategoriality is defined in \citet{Tallman2021} as the span of structure wherein all elements are ciscategorial. If applied to wordhood diagnostics, it means that if, e.g., a morpheme can only attach to verbs but not to other word classes, then it belongs to the verb word. In other words, only ciscategorially selected elements belong to the word. Chorote is interesting in this regard because it allows not only the verb to head the predicate in position \ref{chovpos:predbase16}, but also other word classes — Ns/NPs, pronouns and even negation, which then take most of the usually `verbal' markers. Furthermore, NPs and DPs can take some of the `verbal' TAME markers even when they function as arguments. These two facts pose questions regarding how ciscategoriality should be defined as a comparative concept, since it is not clear whether it should be defined with respect to verbs or to predicates in general. Furthermore, Chorote is also interesting regarding typology of transcategoriality \citep{Robert2003} because it does not display transcategoriality evenly throughout its grammar. Thus, data from Chorote reveal that cis/transcategoriality is a matter of degree.

With respect to wordhood diagnostics, I suggest that two versions of ciscategorial selection diagnostics should be considered: a strict one, specific to verbs, which includes elements that can only be selected by verbs, and a lax one, which considers every element that can only be selected by the predicate head, no matter whether it is verbal or non-verbal, but not by the same categories in non-predicative functions. Importantly, notice that the \textit{lax} subtype does not really define a word class, but rather a set of elements in predicate function.

Distinguishing between verbal and non-verbal predicates, in turn, forces one to address the question of what constitutes a verb in Chorote. The section is thus organized as follows: §\ref{bkm:Ref87562721} proposes a definition of verb in Chorote; and §\ref{bkm:Ref90249662} applies the diagnostics.

\subsection{Defining verb in Chorote}
\label{bkm:Ref87562721}
In previous work (\citealt{Carol2013, Carol2014}) verbs in Chorote were defined as the words that take the personal prefixes of position \ref{chovpos:activenonactive14}. This of course makes ciscategoriality diagnostics circular, if those prefixes are used to define the left edge of a span on the basis that only verbs can combine with them, as will be seen below. But this definition is also problematic for a different reason. Class V verbs (see \tabref{tab:chor:key:2}) take the personal prefixes, but differ from typical verbs in the form of the irrealis, where they take the suffix/enclitic \textit{-a} of position \ref{chovpos:irrealis23}, like nominal predicates, and not the irrealis set of personal prefixes; see (\ref{bkm:Ref87562993}). Furthermore, their third person prefix is always zero, and the plural of any person is expressed through a suffix \textit{-(i)s} identical in form to the most common plural suffix of nouns. I underscore that this \textit{-(i)s}, unlike the \textit{-is} of position \ref{chovpos:concord29}, is not used just with third person, but with \textit{any} person, and comes inmmediately after the stem and \textit{before} the TAME morphemes, all of which brings this Class V closer to the nominal domain. Provisionally, I assign this plural \textit{-(i)s} the same position as the predicate head, i.e. \ref{chovpos:predbase16}, just like the nominal plural in nominal predicates.

\ea\label{bkm:Ref87562993}Exponence of irrealis in different kinds of predicates \\ 
    \ea  Typical verb\\ {
    \glll {} Ja \textbf{n}{}- ek \\ 
    v: \ref{chovpos:prospectiveja8} \textbf{\ref{chovpos:activenonactive14}} \ref{chovpos:predbase16}\\ 
    {} \Prosp{} \textbf{\Third.\Irr{}}- go.away\\ 
    \glt `(S)he/it will leave.'
    }
    \ex\label{bkm:Ref87668503}Nominal predicate \\ {
    \glll  {} Ja anéchiyas- as- \textbf{a'}\\ 
    v: \ref{chovpos:prospectiveja8} \ref{chovpos:predbase16} \ref{chovpos:predbase16} \textbf{\ref{chovpos:irrealis23}}\\ 
    {} \Prosp{} chief- \Pl{}- \textbf{\Irr{}}\\ 
    \glt \textsc{`}He will be chief.'
    }
    \ex  Class V verb\\ {
    \glll {} Ja Ø- 'esy- \textbf{e'}\\ 
    v: \ref{chovpos:prospectiveja8} \ref{chovpos:activenonactive14} \ref{chovpos:predbase16} \textbf{\ref{chovpos:irrealis23}}\\ 
    {} \Prosp{} \Third{}- be.good- \textbf{\Irr{}}\\
    \glt `(S)he/it will be good.'
    }
    \ex  Plural Class V verb\\ {
    \glll {} Ja Ø- is- \textbf{ísy-}\textbf{e'}\\
    v: \ref{chovpos:prospectiveja8} \ref{chovpos:activenonactive14} \ref{chovpos:predbase16} \textbf{\ref{chovpos:predbase16}} \textbf{\ref{chovpos:irrealis23}}\\ 
    {} \textbf{\Prosp{}} \Third{}- be.good- \textbf{\Pl{}- \Irr{}}\\
    \glt `They are good.'
    }
    \z
\z 
Being something between nouns and verbs regarding their morphosyntax, and considering the notions they express (`big', `nice', `white', etc.) it looks attractive to label Class V verbs as adjectives, as \citet{Drayson2009} does in his dictionary. However, I have preferred to label them ``verbs'' for two reasons. Firstly, they take the same person indices as verbs in first and second person. Secondly, there is no evidence that these candidate adjectives in attributive function are structurally different from relativized clauses with a (typical) verbal predicate. Even though they appear superficially juxtaposed to nouns, as adjectives do in European languages, the same goes for verbs. The analysis of these is as free relative clauses with a null relative pronoun \citep{Carol2014}, which is the most usual strategy when the relative pronoun is the subject; see (\ref{bkm:Ref87667507}) (otherwise a demonstrative usually surfaces as an explicit relative). Thus, there is no syntactic evidence to assign (\ref{bkm:Ref87667507}) a syntax different from that of (\ref{bkm:Ref87667510}).

\ea\label{ex:chor:key:38} Class V verbs and typical verbs in attributive function\\ 
    \ea\label{bkm:Ref87667510} Class V verb \\ {
    \glll {} Si'yús \textbf{Ø{}-} \textbf{wuj} in{}- ka{}- je'.\\ 
    v: \ref{chovpos:lightDP5} \textbf{\ref{chovpos:lightDP5}} \textbf{\ref{chovpos:lightDP5}} \ref{chovpos:activenonactive14} \ref{chovpos:predbase16} \ref{chovpos:appllocative33}\\ 
    {} fish \Third{}- \textbf{be.big} \Third{}- have\_joy{}- \textbf{\Ap.\Loc{}}\\
    \glt `The big fish is tasty.' (Lit. `contains joy inside')
    }
    \ex\label{bkm:Ref87667507} Typical verb\\ {
    \glll {} Pi i'nyó' \textbf{'yijén-} \textbf{e} i- jlyut- i'.\\ 
    v: \ref{chovpos:top2} \ref{chovpos:top2} \textbf{\ref{chovpos:top2}:\ref{chovpos:top2}} \textbf{\ref{chovpos:top2}} \ref{chovpos:activenonactive14} \ref{chovpos:predbase16} \ref{chovpos:appllocative33}\\ 
    {} \Dem{} person \textbf{\Third:be.wise}- \textbf{\Ap.\Ins{}}  \Third{}- rub- \Ap.\Loc{}\\ 
    \glt `The man who knows (how to make fire) drills (a piece of wood with stick).' (\citealt[62]{DraysonGomez2000}) 
    }
    \z 
\z 

The next question is whether Class V verbs can head a DP/NP in argument function, i.e. assume the typical syntactic function of nouns. Class V verbs can head a DP/NP in argument function, but so can typical verbs, and in the same way. Besides the zero relative shown in \REF{ex:chor:key:38}, demonstratives can also function as relative pronouns, so that a verb preceded by a demonstrative can be `nominalized' in this way; (\ref{bkm:Ref87667811}) shows a lexicalized case. In sum, an NP/DP can be both Dem+N or Dem+V, so here there is no reason to assign Class V verbs a different, more nominal status than that of the other verbs.

\ea\label{bkm:Ref90302758}\label{bkm:Ref87667811} Typical verb heading a DP\\ 
\gll Jana ta- kelisye'n\\ 
\Dem{} \Third{}- sing\\
\glt `radio/tape recorder' Lit. `the one that sings'
\z 

Furthermore, when a DP/NP headed by a typical verb takes part in a construction that requires nominal irrealis (see example (\ref{bkm:Ref84285943}) and the text that precedes it), the nominal irrealis morpheme \textit{-a} surfaces, as in any DP/NP headed by a noun, see (\ref{bkm:Ref89643749}). Therefore, there is again no reason to assign Class V verbs in argumental function a distinct, non-verbal status.

\ea\label{bkm:Ref89643749} DP headed by a verb with the nominal irrealis morpheme\\
\glll {} Ø{}- Laj [ya{}- ka ta{}- kelisyen{}- \textbf{a}']\\
v: \ref{chovpos:activenonactive14} \ref{chovpos:predbase16} \ref{chovpos:lightDP43} - - - - \\ 
{} \Third{}- not\_exist \First\Sg.\Poss{}- \Ali.\Poss{} \Third{}- sing- \textbf{\Irr{}}\\
\glt `I have no radio/tape recorder.' Lit. `There is no radio/tape recorder of mine.'
\z 

\subsection{Diagnostics based on ciscategoriality}
\label{bkm:Ref90249662}
The strict and lax versions of ciscategoriality proposed above can combine with the known \textit{minimal}{}-\textit{maximal} distinction - a subspan of contiguous positions that satisfy the requirements, or the longest possible subspan which only includes elements that satisfy the requirements, respectively. Or, in other words, a minimal subspan which only includes ciscategorial elements, and a maximal subspan outside of which all elements are transcategorial. In sum, we obtain four diagnostics.

\subsubsection{Strict ciscategoriality (\ref{chovpos:activenonactive14}{}-\ref{chovpos:concord20}; \ref{chovpos:activenonactive14}{}-\ref{chovpos:pluractionaldownwards37})}

The \textit{strict} (i.e. specific to verbs) \textit{minimal} version has the person prefixes of position \ref{chovpos:activenonactive14} and the concord morphemes of position \ref{chovpos:concord20} as its edges, see (\ref{bkm:Ref89691417}). Neither occur in non-verbal predicates, where person/number is indicated through oblique morphemes (and a postposition bound to them), see (\ref{bkm:Ref89694538}).

\ea\label{bkm:Ref89691417} Ciscategoriality - strict minimal: span \ref{chovpos:activenonactive14}{}-\ref{chovpos:concord20} \\ 
\glll {} \textbf{ji}{}- 'wen- \textbf{a'}\\
    v: \textbf{\ref{chovpos:activenonactive14}-} \ref{chovpos:predbase16}-      \textbf{\ref{chovpos:concord20}}\\ 
    {} \Second{}- see- \Second\Pl{}\\
\glt `You (pl.) see it/him/her.'
\z 

\ea\label{bkm:Ref89694538}Person marking in nominal predication \\
    \glll {} I- lis     \textbf{as}{}- e'm \\
    v: \ref{chovpos:predbase16} \ref{chovpos:predbase16} \textbf{\ref{chovpos:oblique35}} \ref{chovpos:appl36} \\
    {} \First\Sg.\Poss{}- sons  \Second\Pl{}- \Ap.\Loc{} \\
\glt `You are my sons.'
\z 


The left edge is \ref{chovpos:activenonactive14} only if we consider that Class V verbs are real verbs. Otherwise, the edge should be placed in position \ref{chovpos:rflxantipassive15} - only transitives can take the reflexive/reciprocal and antipassive morphemes, and Class V are not among them.

The positions to the right of \ref{chovpos:concord20} can co-occur with non-verbal predicates. The perdurative of \ref{chovpos:perdurative21} is scarcely documented with non-verbs in my material, but co-occurs with a noun in \textit{jlọma-jli'} (day-\textsc{perd}) `during the day', as well as in \textit{'wena-jli-yi} (different\_thing-\textsc{perd-P}\textsc{\textsubscript{inst}}) `but' (Spanish `sino'; \citealt{DraysonGomez2000}: \textsc{'wenajliyi}), lexicalized as a conjunction. The momentary, in position \ref{chovpos:momentary22}, can also combine with other word classes; see (\ref{bkm:Ref87668301}) (also \ref{bkm:Ref89689070}a-b below); in \ref{ex:chor:key:43a}a-b (as well as in (\ref{bkm:Ref89689070}), I assume negation occupies the position of the predicate head, i.e. position \ref{chovpos:predbase16} .\footnote{That there is a context for palatalization in \ref{cho:ex:lackofpalatalization} can be seen in the contrast with \textit{t-amti-}\textbf{\textit{ts'i}}\textit{{}-ji'n} `they speak', with the same root, where \textit{-ts'i} is the palatalized allomorph of the distributive \textit{-k'i} of position 34.} Also the irrealis and the other TAME morphemes that follow combine with non-verbs, as was seen in (\ref{bkm:Ref84285943}) and (\ref{bkm:Ref87668503}), and will be seen below in (\ref{bkm:Ref89689070}). 

\ea\label{bkm:Ref87668301} Momentary with non-verbs \\ 
    \ea\label{ex:chor:key:43a}  {
    \glll {} [A: E{}- jetik Ø{}- a'tye{}- je'?] B: Je- \textbf{ye} 'ne'\\ 
        v: \ref{chovpos:predbase16} \textbf{\ref{chovpos:momentary22}} \ref{chovpos:tempaspdisc40}\\ 
        {} [A: \Second.\Poss{}- head-  \Third{}- hurt- \Ap.\Loc{}] B: \Neg{}- \textbf{\Mom{}} now \\
    \glt [A: `Do you have a headache?'] B: `Not anymore.'
    }
    \ex  {
    \glll {} [Syupa] ti jlọma{}- \textbf{ye}{}- t'i{}- jyi… \\
        v: [-] \ref{chovpos:complementizer4} \ref{chovpos:predbase16} \textbf{\ref{chovpos:momentary22}} \ref{chovpos:indirectevidential25} \ref{chovpos:remotepastperfect39}\\ 
        {} [\Dem{}] \Comp{} \textup{day}- \textbf{\Mom{}}- \Evid{}- \Prf{}\\ 
    \glt `[Then] the next day…' Lit. `when it was day again…'
    }
    \z 
\z 

The \textit{strict maximal} subtype has the same left edge as the \textit{minimal} one. The right edge is position \ref{chovpos:pluractionaldownwards37}, occupied by the polysemous \textit{-jen}, which combines even (though rarely) with Class V verbs, as in (\ref{bkm:Ref89686002}), where it functions as a plural marker. What makes Class V verbs different is that they pattern with nouns in many respects, as shown in (\ref{ex:chor:key:38}). Nevertheless, they pattern with the other verbs here and not with nouns, on which  \textit{-jen} is not documented.


\ea\label{bkm:Ref89686002} Ciscategoriality - strict maximal: subspan \ref{chovpos:activenonactive14}{}-\ref{chovpos:pluractionaldownwards37}\\
\gllll {} kas- 'wasajne'n\\ 
    {} /\textbf{kas}{}- 'wasan- \textbf{jen}/ \\
    v: \textbf{\ref{chovpos:activenonactive14}} \ref{chovpos:predbase16} \textbf{\ref{chovpos:pluractionaldownwards37}}\\ 
    {} \textbf{\First\Pl{}}- be.alive- \textbf{\Jen{}}\\
\glt `We are alive.'
\z 

Any material to the left of position \ref{chovpos:activenonactive14} can combine with other word classes, even the prospective of position \ref{chovpos:prospectiveja8}, which combines with nominal predicates, see (\ref{bkm:Ref87668503}). As for material to the right of \ref{chovpos:pluractionaldownwards37}, position \ref{chovpos:pluractionaliterative38} is occupied by the pluractional -\textit{'ni(j)}, which usually attaches to verbs, even from Class V, but which can be seen attached to negation in (\ref{bkm:Ref89689070}).

\ea\label{bkm:Ref89689070} Pluractional \textit{'ni(j)} of position \ref{chovpos:pluractionaldownwards37} with non-verbs\\
\glll {} 'Yina je{}- ye{}- \textbf{'ni} wata'a [ka Ø{}- tojw{}- a{}- k'i']. \\
    v: \ref{chovpos:conjunct1} \ref{chovpos:predbase16} \ref{chovpos:momentary22} \textbf{\ref{chovpos:pluractionaliterative38}} \ref{chovpos:tempaspdisc40} \ref{chovpos:heavyDP46} - - - - \\ 
    {} I\_mean(?) \Neg{}- \Mom{}- \textbf{\Plact{}} so\_much \Comp{} \Third{}- be.distant- \Irr{}- \Ap.\Distr{}\\
\glt `I mean, it is not so distant [as the previous place].' (\citealt[94]{DraysonGomez2000})
\z 

\subsubsection{Lax ciscategoriality (\ref{chovpos:activenonactive14}{}-\ref{chovpos:momentary22}; \ref{chovpos:prospectiveja8}{}-\ref{chovpos:tempaspdisc40})}

The \textit{lax} subtype is similar to the strict one, but replacing ``verb'' by ``main predicate of the clause'', whether verbal or not. Recall that, as stated above, this cannot define a word class, but an element that displays a predicate function. As a consequence, if an element, e.g. the oblique markers in position \ref{chovpos:oblique35} that host applicatives, can combine with a non-verbal category, e.g. nouns, only when the noun is in predicate function, this does not mean that the oblique markers belong to the noun class - if they can only combine when the noun heads a predicate, then they belong to predicates, not to nouns themselves.

The \textit{lax-minimal} subtype also has the person prefixes in position \ref{chovpos:activenonactive14} as its left edge. By definition, when the predicate is non-verbal, this position, as well as position \ref{chovpos:rflxantipassive15}, is simply empty. The right edge is the momentary morpheme in position \ref{chovpos:momentary22}, which only occurs bound to the predicate head, verbal or not, as in (\ref{bkm:Ref87668301}). The span is thus \ref{chovpos:activenonactive14}{}-\ref{chovpos:momentary22}; an example of this was provided in (\ref{bkm:Ref90047260}).\footnote{The translation `please' (\textit{por favor}) was suggested to me with imperatives. In other cases, however, it is very difficult to find an equivalence. It seems to indicate a benefit for some participant.} In turn, the irrealis of position \ref{chovpos:irrealis23} attaches to nouns in argument function when the existence of the entity denoted by the N(P) is not asserted, as in (\ref{bkm:Ref84285943}) and (\ref{bkm:Ref89643749}), and is thus excluded from the span identified by this diagnostic. 

The \textit{lax-maximal} version of the diagnostic has the prospective morpheme of position \ref{chovpos:prospectiveja8} as its left edge; see examples in (\ref{bkm:Ref87562993}). The right edge is position \ref{chovpos:tempaspdisc40}, which contains some adverbs only documented attached to the predicate head. In sum, the subspan is \ref{chovpos:prospectiveja8}{}-\ref{chovpos:tempaspdisc40}. An example of the \textit{lax-maximal} subspan is (\ref{bkm:Ref87669106}).

\ea\label{bkm:Ref87669106} Ciscategoriality - lax maximal: subspan \ref{chovpos:prospectiveja8}\ref{chovpos:tempaspdisc40}. \\
\glll {} \textbf{Ja}- kas- 'wasan- a- jan{}- \textbf{na'a} \\ 
    v: \textbf{\ref{chovpos:prospectiveja8}} \ref{chovpos:activenonactive14} \ref{chovpos:predbase16} \ref{chovpos:irrealis23} \ref{chovpos:pluractionaldownwards37} \textbf{\ref{chovpos:tempaspdisc40}} \\ 
    {} \textbf{\Prosp{}}- \First\Pl{}- be.alive- \Irr{}- \Jen{}- \textbf{later} \\
\glt `We will be alive, we will survive'.
\z 

The morphemes \textit{ja} and \textit{-na'a} at the edges of (\ref{bkm:Ref87669106}) are only documented bound to the predicate head, although other adverbs in position \ref{chovpos:tempaspdisc40} are free, e.g. fronted to position \ref{chovpos:foc3}; see §\ref{bkm:Ref90253511}. The complementizers of position \ref{chovpos:complementizer4} usually co-occur with predicate heads of any class, but not always - in cases like (\ref{bkm:Ref89691723}) \textit{sa'am} `we' can hardly be considered a predicate head; in this example, the first \textit{ti} in position \ref{chovpos:top2} introduces the topic \textit{sa'am} `we', the second \textit{ti} in the usual position \ref{chovpos:complementizer4} apparently heads the main clause, and the last \textit{ti} in position \ref{chovpos:heavyDP46} heads an adverbial clause. Thus, position \ref{chovpos:complementizer4} is excluded from this subspan.


\ea\label{bkm:Ref90321059} \label{bkm:Ref89691723} Complementizer \textit{ti} not introducing a clause \\
\glll {} Jlampet \textbf{ti} sa'am ti a- wa- k- i [siuni- wa jloma- s] [ti Ø- 'nes- a- t'i- pi ni Si- nya' jl- amt- is].\\
v: \ref{chovpos:conjunct1} \textbf{\ref{chovpos:top2}} \ref{chovpos:top2} \ref{chovpos:complementizer4} \ref{chovpos:activenonactive14} \ref{chovpos:predbase16} \ref{chovpos:concord19} \ref{chovpos:distaley32} \ref{chovpos:heavyDP45} - - - \ref{chovpos:heavyDP46} - - - - - - - - - - -\\
{} but \textbf{\Comp{}} \First\Pl{} \Comp{} \First{}- be- \First\Pl{}- \Ap.\Dist{}-  \Dem{}- \Pl{} day- \Pl{} \Comp{} \Third{}- arrive- \Mom{}- \Evid{}- \Rem.\Pst{} \First\Pl.\Poss{}- father \Third\Poss{}- word- \Pl{}\\
\glt `But we were already there those days when the Gospel arrived.' (\citealt[106]{DraysonGomez2000})
\z 

\section{Conclusions}\label{bkm:Ref88847346}

From the application of constituency diagnostics to Chorote using the methodology advocated in \citet{Tallman2021} there does not emerge an obvious wordhood candidate. As can be seen in Tables 4 and 5, which summarize the results of applying morphosyntactic and phonological diagnostics, respectively. As stated in the introduction, the maximum of diagnostic subtypes for a complete span is two, for the subspans \ref{chovpos:complementizer4}{}-\ref{chovpos:tempaspdisc40}, \ref{chovpos:activenonactive14}{}-\ref{chovpos:predbase16}, \ref{chovpos:activenonactive14}{}-\ref{chovpos:momentary22}, \ref{chovpos:activenonactive14}{}-\ref{chovpos:tempaspdisc40}, \ref{chovpos:predbase16}{}-\ref{chovpos:applipcntloc34}, and maybe \ref{chovpos:activenonactive14}{}-\ref{chovpos:antipassvblz18}. If each edge is taken separately, the left edge shows more convergence, with 14 diagnostic subtypes converging on the personal prefixes in position \ref{chovpos:activenonactive14}, and six on the predicate head in position \ref{chovpos:predbase16}, while the right edge has the highest convergence of five diagnostic subtypes on position \ref{chovpos:tempaspdisc40}.

\begin{sidewaystable}
    \caption{Morphosyntactic wordhood diagnostics}
    \label{tab:chor:key:4}
    \begin{tabularx}{\textwidth}{QQQrrrr}
         \lsptoprule
\textbf{Abstract type} & \textbf{Subtype} & \textbf{Language specific fracture} & \textbf{Left-edge} & \textbf{Right-edge} & \textbf{Size} & \textbf{Convergence} \\
\midrule
\textbf{Freedom} & Minimal & ~ & \ref{chovpos:predbase16} & \ref{chovpos:predbase16} & 1 & 1\\
& Maximal & ~ & \ref{chovpos:complementizer4} & \ref{chovpos:tempaspdisc40} & 37 & 2\\
\textbf{Non-Interruption} & Single free interruptor & ~ & \ref{chovpos:activenonactive14} & \ref{chovpos:remotepastperfect39} (or -\ref{chovpos:pluractionaliterative38}) & 26 (or 25) & 1\\
& Multiple free interruptor & ~ & \ref{chovpos:mirativereportative7} &\ref{chovpos:locatives41} & 35 & 1\\
& Nonfixed interruptor & ~ & \ref{chovpos:activenonactive14} & \ref{chovpos:momentary22} & 9 & 2\\
\textbf{Deviation} & ~ & Inflectional class & \ref{chovpos:activenonactive14} & \ref{chovpos:antipassvblz18} & 5 & 1/2\\
& ~ & Fossilization & \ref{chovpos:predbase16} & \ref{chovpos:antipassvblz18} & 3 & 1\\
& Extended Exponence & ~ & \ref{chovpos:activenonactive14} & \ref{chovpos:concord29} & 16 & 1\\
\textbf{Non-permutability} & Rigid &  & \ref{chovpos:activenonactive14} & \ref{chovpos:irrealis23} & 9 & 1\\
\textbf{Subspan repetition} & Minimal & Coordination & \ref{chovpos:prospectiveja8} & \ref{chovpos:pluractionaliterative38} & 33 & 1\\
& Maximal & Coordination & \ref{chovpos:prospectiveja8} & \ref{chovpos:remotepastperfect39} & 34 & 1\\
\textbf{Ciscategorial selection} & Strict minimal & ~ & \ref{chovpos:activenonactive14} & \ref{chovpos:concord20} & 7 & 1\\
& Strict maximal & ~ & \ref{chovpos:activenonactive14} & \ref{chovpos:pluractionaldownwards37} & 24 & 1\\
& Lax maximal & ~ & \ref{chovpos:activenonactive14} & \ref{chovpos:momentary22} & 9 & 2\\
& Lax minimal & ~ & \ref{chovpos:prospectiveja8} & \ref{chovpos:tempaspdisc40} & 30 & 1\\
\lspbottomrule
    \end{tabularx}
\end{sidewaystable}


\begin{sidewaystable}
    \caption{Phonological wordhood diagnostics}
    \label{tab:chor:key:5}
    \begin{tabularx}{\textwidth}{QQQrYYY}
         \lsptoprule
\textbf{Abstract} \textbf{type} & \textbf{Subtype} & \textbf{Language} \textbf{specific} \textbf{fracture} & \textbf{Left-edge} & \textbf{Right-edge} & \textbf{Size} & \textbf{Convergence}\\ \midrule
\textbf{Accent} & Minimal-minimal &  & \ref{chovpos:rflxantipassive15} & \ref{chovpos:predbase16} & 2 & 1\\
& Minimal-maximal &  & \ref{chovpos:predbase16} & \ref{chovpos:applipcntloc34} & 19 & 2\\
& Maximal-maximal &  & \ref{chovpos:complementizer4} & \ref{chovpos:tempaspdisc40} & 37 & 2\\
\textbf{y-insertion} & Minimal &  & \ref{chovpos:predbase16} & \ref{chovpos:distaley32} & 17 & 1\\
& Maximal-minimal &  & \ref{chovpos:predbase16} & \ref{chovpos:applipcntloc34} & 19 & 2\\
& Maximal-maximal &  & \ref{chovpos:predbase16} & \ref{chovpos:adpositionsappl44} & 28 & 1\\
\textbf{Palatalization} & Minimal-A & With k\textsuperscript{(}'\textsuperscript{)}y & \ref{chovpos:activenonactive14} & \ref{chovpos:predbase16} & 3 & 2\\
& Maximal-A & With k\textsuperscript{(}'\textsuperscript{)}y & \ref{chovpos:activenonactive14} & \ref{chovpos:tempaspdisc40} & 27 & 2\\
& Minimal-B & Without k\textsuperscript{(}'\textsuperscript{)}y & \ref{chovpos:activenonactive14} & \ref{chovpos:antipassvblz18}/\ref{chovpos:indirectevidential25} & 5/12 & 2/1\\
& Maximal-B & Without k\textsuperscript{(}'\textsuperscript{)}y & \ref{chovpos:activenonactive14} & \ref{chovpos:tempaspdisc40} & 27 & 2\\
& Minimal-C & Coronals only & \ref{chovpos:activenonactive14} & \ref{chovpos:predbase16} & 3 & 2\\
& Maximal-C & Coronals only & \ref{chovpos:activenonactive14} & \ref{chovpos:heavyDP46} & 32 & 1\\
\lspbottomrule
    \end{tabularx}
\end{sidewaystable}

If one takes morphosyntactic and phonological diagnostics separately, looking for separate grammatical and phonological words, the results are not very different. The left edge of a possible grammatical word could reasonably be position \ref{chovpos:activenonactive14} (convergence of eight subtypes), but the right edge could be the positions \ref{chovpos:antipassvblz18}, \ref{chovpos:momentary22}, \ref{chovpos:remotepastperfect39} (or -\ref{chovpos:pluractionaliterative38}) or \ref{chovpos:tempaspdisc40}, with two subtypes converging in each case; from these, only position \ref{chovpos:momentary22} shows convergence in position \ref{chovpos:activenonactive14} (for both subtypes) in the left edge as well.

As for a possible phonological word, the left edge has two candidates: again position \ref{chovpos:activenonactive14}, with a convergence of six subtypes, and position \ref{chovpos:predbase16}, with four subtypes. But all six subtypes converging in position \ref{chovpos:activenonactive14} correspond to the palatalization diagnostic, while the ones converging in position \ref{chovpos:predbase16} correspond to accent (one) and \textit{y}{}-insertion (three) diagnostics. If we take into account the iambic type conjectured for Proto- or Pre-Chorote, there would be a seventh subtype converging in position \ref{chovpos:activenonactive14}, which would not be related to palatalization. As for the right edge, positions \ref{chovpos:predbase16} and \ref{chovpos:tempaspdisc40} show the highest number of subtype convergences - three, belonging each to two different diagnostic types. However, position \ref{chovpos:predbase16} is very problematic as a candidate for the right edge of the phonological word if we consider that it is the position of verbal root and the left edge for several other diagnostics. In turn, position \ref{chovpos:tempaspdisc40}, filled by adverbial particles that can encliticize to the predicate head, appears as a more reasonable candidate. Two subtypes that give position \ref{chovpos:tempaspdisc40} as right edge also give position \ref{chovpos:activenonactive14} as a left edge, which makes the span \ref{chovpos:activenonactive14}{}-\ref{chovpos:tempaspdisc40} the only `candidate' for the phonological word, but with only two diagnostic subtypes converging in it.

Finally, Chorote is interesting regarding the typology of transcategoriality because it shows features of different types. On the one hand, there is a distinct verb word class. Concord morphology can be selected only by certain stems, \textit{verbal} stems. Other words are not ``transcategorialized'' into verbs when they function as predicates. In those cases, the subject is cross{}-referenced by oblique markers, as seen in (\ref{bkm:Ref89694538}). In this respect, Chorote is not different from languages with heavy morphology and limited transcategoriality \citep{Robert2003}, except for the fact that it lacks a copula.

On the other hand, in other respects the language seems to make extensive use of transcategoriality, something which has been correlated with the isolating type (i.e. the type of languages with weak morphology; \citealt{Robert2003}). For example, an inflected verb can perform a referential function -i.e. head a noun phrase- without any overt transcategorial morphology, as (\ref{bkm:Ref90302758}) shows. Moreover, many TAME markers can be bound to NPs, clearly taking nominal rather than clausal scope (\citealt{Carol2014,Carol2015}). All this gives Chorote some properties of an `omnipredicative language' \citep{Launey1994}, which in turn underscores that tests based on cis-/transcategoriality deserve further discussion.


\printglossary

\sloppy\printbibliography[heading=subbibliography,notkeyword=this]

\end{document} 
