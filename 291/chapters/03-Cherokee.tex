\documentclass[output=paper]{langscibook}
\ChapterDOI{10.5281/zenodo.13208544}
\author{Hiroto Uchihara\affiliation{Tokyo University of Foreign Studies}}
\title{Constituency in Oklahoma Cherokee}
\abstract{This chapter provides a fine-grained description of the result of constituency diagnostics applied to Oklahoma Cherokee, a Southern Iroquoian language spoken in Northeastern Oklahoma. The case of Oklahoma Cherokee is especially intriguing, due to its polysynthetic nature. As is claimed in \citet{Bickel2017} on constituency in polysynthetic languages, more than one constituent need to be posited. On the other hand, unlike what they report for other polysynthetic languages, the method employed here shows that language-internally there is a strong wordhood candidate; this also reflects the general intuitions about wordhood among speakers (see below) and linguists working on Cherokee and Iroquoian languages.}

\IfFileExists{../localcommands.tex}{
  \addbibresource{../localbibliography.bib}
  \addbibresource{../collection_tmp.bib}
  % add all extra packages you need to load to this file

\usepackage{tabularx,multicol}
\usepackage{url}
\urlstyle{same}

\usepackage{listings}
\lstset{basicstyle=\ttfamily,tabsize=2,breaklines=true}

\usepackage{langsci-basic}
\usepackage{langsci-optional}
\usepackage{langsci-lgr}
\usepackage{langsci-osl}
% \usepackage{./langsci/styles/langsci-lgr}
% \usepackage{./langsci/styles/langsci-osl}
% \usepackage{langsci-gb4e}

\usepackage{tikz}
\usetikzlibrary{patterns,calc}
\pgfdeclarepatternformonly{south east lines}{\pgfqpoint{-0pt}{-0pt}}{\pgfqpoint{3pt}{3pt}}{\pgfqpoint{3pt}{3pt}}{
    \pgfsetlinewidth{0.6pt}
    \pgfpathmoveto{\pgfqpoint{0pt}{3pt}}
    \pgfpathlineto{\pgfqpoint{3pt}{0pt}}
    \pgfpathmoveto{\pgfqpoint{.2pt}{-.2pt}}
    \pgfpathlineto{\pgfqpoint{-.2pt}{.2pt}}
    \pgfpathmoveto{\pgfqpoint{3.2pt}{2.8pt}}
    \pgfpathlineto{\pgfqpoint{2.8pt}{3.2pt}}
    \pgfusepath{stroke}}
    
\usepackage{stmaryrd}
\usepackage{wasysym}
\usepackage{multirow}
\usepackage{caption}
\usepackage{subcaption}
\usepackage{mathrsfs}
\usepackage{qtree}

\usepackage{linguex}


  %pminos do not split footnotes
% \interfootnotelinepenalty=10000 %Footnote in Laporte chapters has to be split SN


%\DeclareIndexNameFormat{default}{%
%\nameparts{#1}%
%\usebibmacro{index:name}%
%{\index[names]}%
%{\namepartfamily}%
%{\namepartgiveni}%
% {}% L1
% {}% L2
%{\namepartprefix}% generates spurious space L3
%{\namepartsuffix}% generates spurious space L4
%}

%  {\DeclareIndexNameFormat{default}{%
%     \usebibmacro{index:name}{\index[names]}{#1}{#3}{#5}{#7}}}

%\DeclareIndexNameFormat{default}{%
%  \usebibmacro{index:name}{\sindex[nom]}{#1}{#3}{#5}{#7}}

%\DeclareIndexNameFormat{default}{%
%  \usebibmacro{index:name}{\sindex[person]}{#1}{#3}{#5}{#7}}
%\DeclareIndexNameFormat{default}{%
%\nameparts{#1} \usebibmacro{index:name}{\sindex[person]]}{\namepartfamily}{‌​\namepartgiven}{\nam‌​epartprefix}{\namepa‌​rtsuffix}}

%\newcommand{\smiley}{:)}

%\renewbibmacro*{index:name}[5]{%
%\usebibmacro{index:entry}{#1}%
%{\iffieldundef{usera}{}{\thefield{usera}\actualoperator}\mkbibindexname{#2}{#3}{#4}{#5}}}

% \newcommand{\noop}[1]{}

%remove for final
%\overfullrule=1mm

\newcommand{\tobi}[2]}}
\renewcommand{\S}[1]{\tobi{#1}{\textsc{*}}}

% this volume references
% puts: [this volume]
% already defined: \citetv
%\newcommand{\citepv}[1]{(\citeauthor{#1} \citeyear*{#1} [this volume])}
\newcommand{\citealtv}[1]{\citeauthor{#1} \citeyear*{#1} [this volume]}

%parentheses around example number
\newcommand{\pref}[1]{(\ref{#1})}

% in-text examples

\newcommand{\lnex}[1]{\textit{#1}} %target lang word
\newcommand{\lnlit}[1]{(lit.: `#1')} %literal reading
\newcommand{\lnlat}[1]{(#1)} % latinization
\newcommand{\lntrans}[1]{`#1'} %translation
\newcommand{\lnexl}[2]%
{\lnex{#1}{} \lnlat{#2}} % ex with latinization
\newcommand{\lnexlat}[3]{\lnex{#1}{} \lnlat{#2}{} \lntrans{#3}} % ex with latinization and tranl.

%ch01
\newcommand{\co}[1]{\mbox{\textbf{#1}}}

%ch09

\newcommand{\cyrbulg}[1]{\begin{otherlanguage*}{bulgarian}#1\end{otherlanguage*}}


%ch10
\newcommand{\nlp}{{\small NLP}}
\newcommand{\mwe}{{\small MWE}}
\newcommand{\rae}{{\small RAE}}
\newcommand{\lvc}{{\small LVC}}
\newcommand{\pos}{{\small P}o{\small S}}
%\newcommand{\todo}[1]{ \textcolor{red}{#1} }

%\renewcommand{\labelenumi}{\theenumi}
%\ainamefmt{{vv}{ll}{, ff}{, jj}} % fullname

\newcommand{\biberror}[1]{{\color{red}#1}}

\newcommand{\osenovaitem}{--~} 
  %% hyphenation points for line breaks
%% Normally, automatic hyphenation in LaTeX is very good
%% If a word is mis-hyphenated, add it to this file
%%
%% add information to TeX file before \begin{document} with:
%% %% hyphenation points for line breaks
%% Normally, automatic hyphenation in LaTeX is very good
%% If a word is mis-hyphenated, add it to this file
%%
%% add information to TeX file before \begin{document} with:
%% %% hyphenation points for line breaks
%% Normally, automatic hyphenation in LaTeX is very good
%% If a word is mis-hyphenated, add it to this file
%%
%% add information to TeX file before \begin{document} with:
%% \include{localhyphenation}
\hyphenation{
    Beck-man
    Ngu-yen
    back-chan-nel
    back-chan-nels
    mo-not-o-nous
    ste-reo-typ-i-cal
}

\hyphenation{
    Beck-man
    Ngu-yen
    back-chan-nel
    back-chan-nels
    mo-not-o-nous
    ste-reo-typ-i-cal
}

\hyphenation{
    Beck-man
    Ngu-yen
    back-chan-nel
    back-chan-nels
    mo-not-o-nous
    ste-reo-typ-i-cal
}
 
  \togglepaper[3]%%chapternumber
}{}

\begin{document}
\maketitle 
%\shorttitlerunninghead{}%%use this for an abridged title in the page headers

\section{Introduction}

This chapter provides a fine-grained description of the result of constituency diagnostics applied to Oklahoma Cherokee, a Southern Iroquoian language spoken in Northeastern Oklahoma. The chapter is divided into four sections after this introductory section. First, \sectref{sec:key:2} discusses the planar structures in the verb and noun complex, followed by §\ref{bkm:Ref87347154} and §\ref{bkm:Ref87352909} which provide a description of each of such constituency diagnostics: phonological diagnostics in §\ref{bkm:Ref87347154}, and the morphosyntactic diagnostics in §\ref{bkm:Ref87352909}. §\ref{bkm:Ref87347176} summarizes the result of application of various diagnostics to the Cherokee verb complexes and concludes with some typological and theoretical implications. 

Oklahoma Cherokee, a Southern Iroquoian language spoken in North Carolina and Oklahoma, the United States, is a polysynthetic language, and as in other such languages, poses a question with regard to the definition of `word': ideas conveyed by phrases or sentences in languages such as English, Spanish or Japanese can be conveyed by a `word' in Cherokee, as illustrated in (\ref{bkm:Ref72229011}) and (\ref{bkm:Ref72229021}); in the examples the plus sign indicates that the morphemes connected with this sign are synchronically no longer analyzable:

\ea\label{bkm:Ref72229011}  d\v{v}:ní:ne:giʔe:li\protect\footnotemark\\
    \gll ta-anii-nee+kiʔ-ee-l-i \\
     \Cisl{}-\Third\Sg{}.\Aarg{}-liquid+take-\Dat{}-\Prf{}-\Mot{} \\
     \glt `They will take it (liquid) from him.'  \citep[206]{FeelingEtAl2003}
\z

\footnotetext{In the examples, the first line shows the surface forms as pronounced by speakers and the second line shows segmented forms. The numbers in the third line, which is shown after examples in \REF{ex:cher:key:3}, correspond to the slot numbers in the table on the third page. These are followed by glosses and free translations.}


\ea\label{bkm:Ref72229021}(hla) yigv:n\^{v}:tlo:híha \\
    \gll hla yi-kvv-nv́v(ʔ)+:(ʔ)tlhoo-híh-a\\
     \Neg{} \Irr{}-\First{}/\Second\Sg{}-leg+strap-\Prs{}-\Ind{}\\
     \glt `I'm not tying up your leg.' (EJ2011)
\z

The case of Oklahoma Cherokee is especially intriguing, due to the number of morphemes a `word' can contain. This chapter attempts to answer questions such as how many constituents are needed, whether there are any convergences, and whether a word can be defined in such a language.

\section{Planar structures} 
\label{sec:key:2}
\subsection{Verbal planar structure}

The planar structures for the verb, noun and adjective complexes are provided in \tabref{tab:cher:planv}-\tabref{tab:cher:plana} below. They are based on flattening out and elaborating template representations and/or phrase structure rules across morphological and syntactic domains.

First, \tabref{tab:cher:planv} shows the planar structure for the verb complex. The positions 1, 18, 23 and 24 are zones, while the others are slots. Zones are where variable elements can occur in free order, while slots are where only one element can occur at a time. Prefix order is fixed, while there is some uncertainty with respect to the suffix order, especially of derivational suffixes in positions 14 – 20. This is because co-occurrence of more than one derivational suffix is relatively uncommon in natural speech, and I have no elicitation data to confirm if alternative orders are possible with or without scope differences. Most of the orders in \tabref{tab:cher:planv} are motivated based on the attested data in my corpus.\footnote{The data in this chapter comes from my fieldnotes and recordings collected during 2011--2013 (in field) and since 2020 (with Christian Koops), as well as a set of recordings collected by Durbin Feeling and William Pulte in the late 1970s, and various interviews recorded and provided by the Cherokee Nation, including Cherokee Nation Radio Show (CNRS). In addition, some data comes from published materials by a speaker-linguist Durbin Feeling, especially \citet{Feeling1975} and \citet{FeelingEtAl2003}. The initials in the sources are abbreviations of the speakers' names.}

In position 18 within the verbal complex, dative and ambulative suffixes can co-occur without any apparent scope difference (cf. §\ref{bkm:Ref72229055}). Word order in Cherokee or in Iroquoian in general is not fixed and is mostly determined by information structure (\citealt{Scancarelli1987}: \S 3.7; \citealt{Mithun1995}). It is still unknown if clitic order is fixed or not. 

\begin{table}[htp]
    \caption{Planar structure for verb in Oklahoma Cherokee}
    \label{tab:cher:planv}
    \centering 
    \begin{tabular}{Srp{9cm}}\lsptoprule
\multicolumn{1}{l}{Positions} & Type & Elements \\ \midrule
\label{chnp1} & zone & NP\{A, S, P\}; PP; Adv\\
\label{chirr2} & slot & Irrealis \textit{y(i)}{}-; relative \textit{c(i)}{}-\\
\label{chtl3} & slot & Translocative \textit{w(i)}{}-\\
\label{chpart4} & slot & Partitive \textit{n(i)-/ii- {\textasciitilde} iy-}\\
\label{chdist5} & slot & Distributive \textit{tee-/ti- {\textasciitilde} c-/too-} \\
\label{chcl6} & slot & Cislocative \textit{ta(y)-/ti(y)- {\textasciitilde} c-}\\
\label{chit7} & slot & Iterative \textit{vv- {\textasciitilde} v́ʔ{}-/ii- {\textasciitilde} íʔ{}-}\\
\label{chneg8} & slot & Negative \textit{ka(y)-/kee-}\\
\label{chpron9} & slot & Pronominal prefixes\\
\label{chmid10} & slot & Middle \textit{ata(a)-/ ali-/ at}{}-; reflexive \textit{ataat-/ ata(a)-/ at-}\\
\label{chinc11} & slot & Incorporated noun root, compounded verb root\\
\label{chbase12} & slot & Verb root\\
\label{chasp13} & slot & Aspectual (perfective, only to host the following derivational suffixes in positions 14 - 20)\\
\label{chdup14} & slot & Duplicative -\textit{iis}{}-\\
\label{chrep15} & slot & Repetitive -\textit{iiloo}{}-\\
\label{chcaus16} & slot & Causative (can be repeated)\\
\label{chcomp17} & slot & Completive -\textit{o}{}-\\
\label{chdat18} & zone & Dative -\textit{e(e)-;} ambulative -\textit{iit}{}-\\
\label{chven19} & slot & Venitive -\textit{ii}{}-; andative -\textit{ee}{}-\\
\label{chinc20} & slot & Inceptive -\textit{iit-}\\
\label{chasp21} & slot & Aspectual (present; imperfective; perfective; punctual; infinitive)\\
\label{chmod22} & slot & Modal (indicative -\textit{a}; assertive -\textit{v\'{v}ʔi}; reportative -\textit{ééʔi;} habitual -\textit{óóʔi}; future imperative -\textit{vvʔi;} participial; nominal -\textit{i})\\
\label{chclit23} & zone & Clitics (interrogative, discursive)\\
\label{chnp24} & zone & NP\{A,S,P\}; PP; Adv\\
\lspbottomrule
\end{tabular}
\end{table}

The following is an example of a verb containing some of the morphemes in \tabref{tab:cher:planv}:

\ea\label{ex:cher:key:3} {nidayú:go:whtv́hdi} \\
\glll ni-tay-uu-koohwahth-v́ht-i \\ 
 v:\ref{chpart4}-\ref{chcl6}-\ref{chpron9}-\ref{chbase12}-\ref{chasp21}-\ref{chmod22} \\
\Part{}-\Cisl{}-\Third\Sg.\Barg{}-see-\Inf{}-\Nom{}\\ 
\glt `for him to see it (looking this way).' (\citealt{PulteFeeling1975}: 246) 
\z 


Some issues that were encountered during the development of the verbal planar structure are as follows. First, `aspectual' suffixes are found in two positions in the planar structure, 13 and 21. The (perfective) aspectual suffix in position 13 is required only when one of the derivational suffixes in positions 14 – 20 is present. Moreover, when there is more than one derivational suffix, all but the last have to have the aspectual suffix in position 13. Otherwise, the aspectual suffixes are not filled out in both positions. Secondly, Oklahoma Cherokee, as other Iroquoian languages, is rich in fusional morphology: some morphemes are portmanteau, and some morphemes manifest complex allomorphy conditioned by phonological and morphological factors (\citealt{BarrieUchihara2019}). This sometimes makes segmentation challenging, especially in positions 2 - 21,  which might result in more than one planar structure that could be posited. Non-concatenative morphological processes are also robust, including two stem alternation processes, Laryngeal Alternation and tonicity\footnote{Laryngeal Alternation is triggered by certain pronominal prefixes, where the stem-initial \textit{h} alternates with a glottal stop \citep{Munrolaryngeal}. Tonicity is conditioned by various morphosyntactic factors and reflected in the tonal effects of a glottal stop and whether a vowel-initial pronominal prefix has a lowfall tone or not (\citealt{Cook1979}: 92; \citealt{uchihara2016tone}: Appendix A).} , and superhigh accent that has some morphosyntactic functions (\citealt{uchihara2016tone}: Ch. 11). These are not reflected in the planar structure in \tabref{tab:cher:planv}.


\subsection{Nominal and adjectival planar structures}

\tabref{tab:cher:plann} and \tabref{tab:cher:plana} show the planar structures for the noun and adjective complexes. They share some positions with the verbal planar structure presented above; for instance, all of them share partitive, distributive, pronominal and middle/reflexive prefixes. However, as can be observed, the number of positions for the nominal and adjectival planar structures is significantly reduced compared to verbs. That is, like other languages spoken in North America, Oklahoma Cherokee is a heavily `verbal' language.

\begin{table}[htp]
    \caption{Planar structure for noun in Oklahoma Cherokee}
    \label{tab:cher:plann}
    \begin{tabular}{Trp{9cm}} 
    \lsptoprule   
    \multicolumn{1}{l}{Positions} & Type & Elements\\ \midrule
\label{chnnp1}     & zone & NP\{A,S,P\}, PP, Adv\\
     & slot & Partitive \textit{ii- {\textasciitilde} iy-}\\
     & slot & Distributive \textit{ti- {\textasciitilde} c-}\\
\label{chnpron4}     & slot & Pronominal prefixes\\
     & slot & Middle \textit{ata(a)-/ ali-/ at}{}-, reflexive \textit{ataat-/ ata(a)-/ at-}\\
     & slot & Compounded noun root\\
\label{chnbase7}     & slot & Noun root\\
\label{chndim8}     & zone & {}-\textit{ya} `real', diminutive -\textit{(uu)ca}, adjectivizer -\textit{haaʔi}\\
\label{chnloc9}     & slot & Locative\\
    \label{chncl10} & zone & Clitics (interrogative, discursive)\\
     & zone & NP\{A,S,P\}, PP, Adv\\
    \lspbottomrule
    \end{tabular}
\end{table}

Again, the orders in \tabref{tab:cher:plann} are justified by the attested forms in my corpus. Thus, the order of -\textit{ya} `real' or the diminutive \textit{–(uu)ca} in position 8 followed by the locative in position 9 is justified by the following examples:

\ea\label{ex:cher:key:4} {kuwa:y\H{o}:ʔi} \\
\glll kuwaa-y(a)-o\H{o}ʔi\\
n:\ref{chnbase7}-\ref{chndim8}-\ref{chnloc9} \\  
mulberry-real-\Loc{}\\ 
\glt `Pryor (a town in Oklahoma).' \citep{Feeling1975}
\z 

\ea\label{ex:cher:key:5} {ani:ge:hyu:j\H{o}} \\
\glll anii-keehy(a)-uuc-o\H{o}ʔi\\
n:\ref{chnpron4}-\ref{chnbase7}-\ref{chndim8}-\ref{chnloc9}\\ 
\Third\Pl.\Aarg{}-woman-\Dim{}-\Loc{}\\
\glt `Female (Seminary).' (CED-EJ2010)
\z 


Adjectives have been argued to constitute an independent lexical category (\citealt{LindseyScancarelli1985}), but \citet{BarrieUchihara2019} argue that they are hard to distinguish from nouns (especially derived nominals) in many cases. The adjectival planar structure does resemble the nominal planar structure as can bee seen in \tabref{tab:cher:plana}, unlike in Northern Iroquoian languages where adjectives are indistinguishable from verbs \citep{Chafe2012}. The only difference between the nominal and the adjectival planar structures is the intensifiers in zone 8, instead of the nominal suffixes in position 8 and the locative suffix in position 9 in the nominal planar structure.

\begin{table}[htp]
    \caption{Planar structure for adjective in Oklahoma Cherokee}
    \label{tab:cher:plana}
    \centering
    \begin{tabular}{Urp{9cm}}
\lsptoprule
\multicolumn{1}{l}{Positions} & Type & Elements\\ \midrule
 & zone & NP\{A, S, P\}, PP, Adv\\
\label{chatl2} & slot & Translocative \textit{w(i)-}\\
 & slot & Partitive \textit{ii- {\textasciitilde} iy-}\\
 & slot & Distributive \textit{ti- {\textasciitilde} c-}\\
\label{chapron5} & slot & Pronominal prefixes\\
 & slot & Middle \textit{ata(a)-/ ali-/ at}{}-, reflexive \textit{ataat-/ ata(a)-/ at-}\\
\label{chabase7} & slot & Adjective root\\
\label{chaint8} & slot & Intensifier\\
 & zone & Clitics (interrogative, discursive)\\
 & zone & NP\{A,S,P\}, PP, Adv\\
\lspbottomrule
\end{tabular}
\end{table}

The following is an example of an adjective containing some of the positions in \tabref{tab:cher:plana}.

\ea\label{ex:cher:key:6} {wǔ:sdî:k\H{v}:ʔi} \\
\glll w-uu--astíi-khvv\H{}ʔi \\
a:\ref{chatl2}-\ref{chapron5}-\ref{chabase7}-\ref{chaint8}\\ 
\Trnsl{}-\Third\Sg.\Barg{}-small-\Int{}\\ 
\glt `smallest.' \citep[337]{Feeling1975}
\z 

Cherokee has its own writing system, the Cherokee Syllabary devised in the early 1800s by Sequoya \citep{Foreman1938}. When writing in syllabary, speakers usually write as one orthographic word from position 2 to 22 or 23 in the verbal planar structure, from position 2 to 9 or 10 in the nominal and 2 to 8 or 9 in the adjectival planar structures, and a space or a period is inserted between the orthographic words. This is illustrated in (\ref{bkm:Ref72230009}), taken from a collection of Cherokee stories collected by a speaker-linguist Durbin Feeling (\citealt{FeelingEtAl2018}), written in the Cherokee Syllabary. As can be seen, enclitics (in position 23 in the verbal planar structure, and positions 10 and 9 in the nominal and adjectival planar structures)\footnote{Here they are connected with = and boldfaced in the syllabary.}, are written together with the preceding hosts.

\ea\label{bkm:Ref72230009} {\cherokeefont ᏦᏍᏓᏓᏅᏟ\textbf{Ꮓ} ᎠᏴ\textbf{Ꮓ} ᎡᏙᏓ\textbf{Ꮄ} ᏃᏊ ᎣᏥᏍᏓᏩᏛᏒ ᎩᏟ} \\
    jo:sdada:hn\H{v}:hli=hno ayv́=hno e:do:dá=lé nǒ:gwu o:ji:sdâ:wadv́:sv́ gi:hli \\ 
\gllll c- oost- ataa- hnv\H{v}hli =hno ayv́ =hno ee- toota =lé noókwu oocii- stá(ʔ)wat -vv(ʔ)s -vv́ʔi kiihli \\ 
    v:1 - - - - 1 - 1 -   \\
    n:3- 4- 5- 7 =10 7 =10 5- 7 =10 1 9- 12 -21 -22 24  \\ 
     \Dist{}- \First\Du.\Excl.\Aarg{}- \Refl{}- brother =and \First\Sg{}/\Pl{} =and \First\Sg.\Barg{}- father =or then \First\Pl{}.\Excl.\Aarg{}- follow -\Prf{}{} -\Asr{} dog\\
 \glt `So, my dad, my brother and I followed the dog.' (\citealt{FeelingEtAl2018}: 13) 
 \z 

\section{Phonological domains}
\label{bkm:Ref87347154}

This and the following sections look at each of the diagnostics applied to the verbal complexes in Oklahoma Cherokee. In this section, I present the phonological diagnostics: Domain of \textsc{H1} \textsc{Spreading} (§\ref{bkm:Ref87347547}), Domain of \textsc{H3} \textsc{Assignment} (§\ref{bkm:Ref72232692}), Domain of \textsc{Superhigh} \textsc{Assignment} (§\ref{bkm:Ref72236002}), \textsc{Final} \textsc{apocope} (§\ref{bkm:Ref87347578}), \textsc{Syllabification} (§\ref{bkm:Ref87347583}) and \textsc{h{}-metathesis + vowel deletion} (§\ref{sec:3.6}).

\subsection{Domain of H1 spreading (11--21)}
\label{bkm:Ref87347547}
H1 is a class of high tone which has been induced by a glottal stop (\citealt{Uchihara2009, uchihara2016tone}: Ch.7). H1 spreads leftward to the preceding mora, as long as it satisfies complex phonological conditions, such as that the preceding syllable is long and does not carry a marked tone (\citealt{uchihara2016tone}: \S 6.5). In (\ref{bkm:Ref77151623}), the high tone on the syllable \textit{dó} spreads to the preceding mora on the syllable \textit{we:}, forming a low-high rising tone on this vowel:

\ea\label{bkm:Ref101099995}\label{bkm:Ref77151623} {à:tawě:dóʔvsga} \\
\glll a-thaweetó-ʔvsk-a\\
v:\ref{chpron9}-\ref{chbase12}-\ref{chasp21}-\ref{chmod22}\\
\Third\Sg.\Aarg{}-kiss-\Prs{}-\Ind{}\\  
\glt `He is kissing her.' \citep[58]{Feeling1975}
\z 

\figref{fig:bkm:Ref101099995} is an autosegmental representation of \ref{bkm:Ref101099995}, visualizing the spreading process.
\begin{figure}
    \centering
    %\includegraphics[scale=.6]{figures/cherokee-fig1.png}
    \caption{Autosegmental representation of \textit{à:tawě:dóʔvsga}}
    \label{fig:bkm:Ref101099995}
    a- thaw\tikzmarknode{e}{ee}t\tikzmarknode{o}{ó} -ʔvsk -a\\\bigskip

    ~~~~~\tikzmarknode{H}{H}
    \begin{tikzpicture}[remember picture, overlay]
      \draw[dashed](e)--(H.north);
      \draw(o)--(H.north);
    \end{tikzpicture}
\end{figure}

Crucially, H1 which is lexically linked somewhere between positions 11 to 21 cannot spread to a syllable which belongs to the pronominal prefix in position 9 as in (\ref{bkm:Ref101100012}) or the reflexive/middle prefixes in position 10 as in (\ref{bkm:Ref101100203}), even if the other conditions for spreading are met (i.e. the preceding syllable is long and does not carry a marked tone). That is, the domain of H1 \textsc{Spreading} is the subspan that extends from position 11 to 21. Here, the domain of H1 \textsc{Spreading} is indicated by square brackets.


\ea\label{bkm:Ref101100012} {ji:[nâ:wi:díh]a  (*jǐ:nâ:wi:díha)} \\
\glll cii-ná(ʔ)wiit-íh-a\\ 
v:\ref{chpron9}-\ref{chbase12}-\ref{chasp21}-\ref{chmod22}\\ 
\First\Sg{}>\An{}-carry.\Fl{}-\Prs{}-\Ind{}\\ 
\glt `I am taking him somewhere.' \citep[104]{Feeling1975}
\z 

\ea\label{bkm:Ref101100203} {à:da:[sdâ:yv:h\'{v}sg]a}\\
\glll Ø-ataa-stá(ʔ)yvv-h\'{v}sk-a\\ 
v:\ref{chpron9}-\ref{chbase12}-\ref{chasp21}-\ref{chmod22}\\
\Third\Sg.\Aarg{}-\Refl-cook.meal-\Prs{}-\Ind{}\\ 
\glt `He is cooking a meal.' \citep[7]{Feeling1975}
\z 

If the morpheme boundary (between the verb base in position 12 and the prefixes in positions  9 and 10) in fact is the conditioning factor, one would expect that the same morpheme with H1 (with a historical glottal stop) would show different realizations depending on whether the preceding morpheme is a pronominal (or reflexive/middle) prefix or part of the verb base. This prediction is born out. Compare the form -\textit{k-íʔ}{}- `eat-\textsc{prs}' with a pronominal prefix \textit{oostii}{}- `\textsc{1du.\Excl.a'} in (\ref{bkm:Ref72230824}) and -\textit{stiik-íʔ-} `eat.\textsc{lg-prs'} in (\ref{bkm:Ref72230835}), both of which clearly have in common the morpheme -\textit{k-íʔ}{}- `eat-\textsc{prs}'. Both in (\ref{bkm:Ref72230824}) and (\ref{bkm:Ref72230835}), the preceding syllables are long and thus the phonological environment is the same. However, in (\ref{bkm:Ref72230824}), the element -\textit{kíʔ}{}- is preceded by a pronominal prefix \textit{oostii}{}- in position 9 to which H1 cannot spread. In (\ref{bkm:Ref72230835}), on the other hand, the element -\textit{kíʔ}{}- is preceded by a stem-internal long vowel \textit{ii} to which H1 can spread:

\ea\label{bkm:Ref72230824}ò:sdi:[gíʔ]a (*ò:sdǐ:gíʔ]a) \\
\glll oostii-k-íʔ-a\\
v:\ref{chpron9}-\ref{chbase12}-\ref{chasp21}-\ref{chmod22}\\
\First\Du{}.\Excl.\Aarg{}-eat-\Prs{}-\Ind{}\\ 
\glt `He and I are eating it.' (DFJuly2013)
\z  

\ea\label{bkm:Ref72230835}à:sdǐ:[gíʔ]a \\
\glll aa-stiik-íʔ-a\\
v:\ref{chpron9}-\ref{chbase12}-\ref{chasp21}-\ref{chmod22}\\
\Third\Sg.\Aarg{}-eat.\Lg{}-\Prs{}-\Ind{}\\
\glt `He is eating it (something long).' \citep[47]{Feeling1975}
\z 

We have seen above that the left-edge of H1 \textsc{Spreading} is at position 11, since H1 fails to spread to the preceding pronominal prefix in position 9 or the reflexive/middle prefixes in position 10. The right-edge of the domain of H1 \textsc{Spreading} is at position 21, that is the aspectual suffix: H1 in the aspect suffix can spread to the verb base, as can be seen in (\ref{bkm:Ref72230835}) above. 

The modal suffix in position 22, which follows the aspect suffix, is outside of the domain of H1 \textsc{Spreading}. This is because H1 in the modal suffix is never observed to spread to the span of positions 11 - 22. Among the modal suffixes, two suffixes, the habitual -\textit{óʔi {\textasciitilde} -óóʔi}, and the reportative -\textit{éʔi {\textasciitilde} -ééʔi}, have H1. However, these suffixes conspire to avoid their H1 to spread to the preceding morpheme. These suffixes have two allomorphs, one with a short vowel and another with a long vowel. The length alternation of these suffixes is conditioned by the tone of the last vowel of the verb stem (verb base in position  + aspect suffixes in position) (\citealt{Cook1979}: 129; \citealt{Montgomery-Anderson2008}: 271). That is, the allomorph with a short vowel is selected after a high tone on the final mora of the verb stem, as in (\ref{bkm:Ref72231039}), while the allomorph with the long vowel is selected otherwise as in (\ref{bkm:Ref72231072}). (\ref{bkm:Ref72231078}) shows that this verb lexically has a high tone on \textit{i} in the imperfective suffix -\textit{híh}, and that the high tone on the penultimate syllable is not due to spreading of the H1 of the habitual suffix -\textit{óʔi} (in boldface).

\ea\label{bkm:Ref72231039}à:[dlo:hyíh]óʔi  \\
\glll Ø-atlooy-híh-\textbf{óʔi}\\
v:\ref{chpron9}-\ref{chbase12}-\ref{chasp21}-\ref{chmod22} \\ 
\Third\Sg.\Aarg{}-cry-\Impf{}-\Hab{}\\
\glt `He habitually cries.' \citep[13]{Feeling1975}
\z 

\ea\label{bkm:Ref72231078}à:[dlo:hyíh]a \\
\glll Ø-atlooy-híh-a\\
v:\ref{chpron9}-\ref{chbase12}-\ref{chasp21}-\ref{chmod22} \\ 
\Third\Sg.\Aarg{}-cry-\Prs{}-\Ind{}\\
\glt `He is crying.' \citep[13]{Feeling1975}
\z 

\ea\label{bkm:Ref72231072}à:[di:tasg]ó:ʔi \\
\glll Ø-atiihtha-sk-oóʔi\\
v:\ref{chpron9}-\ref{chbase12}-\ref{chasp21}-\ref{chmod22} \\
\Third\Sg.\Aarg{}-drink-\Impf{}-\Hab{}\\
\glt `He habitually drinks it.' \citep[11]{Feeling1975}
\z 
     
H1 of these modal suffixes have the possibility of spreading to the preceding morpheme only when the modal suffix has an allomorph with a short vowel, as in (\ref{bkm:Ref72231039}), but in all such instances the final vowel of the verb stem has a high tone, and thus H1 of these modal suffixes cannot spread. Thus, since H1 \textsc{Spreading} is never be observed in this sequence, the modal suffixes in position 22 are outside of the domain of H1 \textsc{Spreading}. 

\subsection{Domain of H3 assignment ({7}-21; {5}-21)}
\label{bkm:Ref72232692}
Certain pre-pronominal prefixes (positions 2 - 8) in Oklahoma Cherokee assign a high tone (henceforth H3, represented with the acute accent diacritic as in H1, since their pitch levels are the same) somewhere within the initial three syllables of the verb (\citealt{Lindsey1987}, \citealt{Wright1996}; \citealt{uchihara2016tone}: Ch.10). In (\ref{bkm:Ref72231865}), the iterative pre-pronominal prefix \textit{v:}{}- assigns H3 to the syllable \textit{hi}; this tone is absent from the form without the pre-pronominal in (\ref{bkm:Ref72231871}): 

\newpage
\ea\label{bkm:Ref72231865}v:h\textbf{í}:gò:wáhta \\
\glll vv-hii-koohwahth-Ø-a \\
v:\ref{chit7}-\ref{chpron9}-\ref{chbase12}-\ref{chasp21}-\ref{chmod22} \\
\Iter{}-\Second\Sg>\An{}-see-\textsc{pnc}-\Ind{}\\ 
\glt `You just saw him again.' (EJ2011)
\z 

\ea\label{bkm:Ref72231871}hi:gò:wáhta \\
\glll hii-koohwahth-Ø-a\\
v:\ref{chpron9}-\ref{chbase12}-\ref{chasp21}-\ref{chmod22}\\ 
\Second\Sg>\An{}-see-\textsc{pnc}-\Ind{}\\ 
\glt `You just saw him.' (EJ2011)
\z

H3 is not only found on the second syllable of the verb as in (\ref{bkm:Ref72231865}), but also on the third syllable of the verb:

\ea\label{bkm:Ref72231881}tla yigin\textbf{í}:gowhtǐ:ha \\
\glll tlha yi-kinii-koohw(a)hth-iíh-a\\
v:1 \ref{chirr2}-\ref{chpron9}-\ref{chbase12}-\ref{chasp21}-\ref{chmod22}\\ 
not \Irr{}-\textsc{1du.in.b}-see-\Prs{}-\Ind{}\\
\glt `He is not seeing you and me.' (EJ2011)\\
\z 
     
\citet[ch.10]{uchihara2016tone} argues that the H3 is essentially an iambic pitch-accent rather than a floating tone, and that the difference between prefixes such as iterative \textit{v:}{}- in (\ref{bkm:Ref72231865}) where the H3 is assigned to the second syllable on the one hand, and prefixes such as irrealis \textit{yi}{}- in (\ref{bkm:Ref72231881}) where the H3 is assigned to the third syllable on the other, can be accounted for by considering that the latter type of prefixes are extrametrical. That is, prefixes such as the irrealis are excluded from syllable counting in the assignment of the iambic pitch accent. In the current method with the verbal planar structure in \tabref{tab:cher:planv}, the prefixes after position 7 (iterative) are always within the domain of H3 \textsc{Assignment}, while the prefixes before that can be outside of its domain, as we will see below.

The aspectual suffixes in position 21 are also within the domain of H3 \textsc{Assignment}. This is evident from the following example, where the H3 is assigned to the vowel of the aspectual suffix /i/ (and then spreads leftward by one mora). Here again the domain of H3 \textsc{Assignment} is indicated by square brackets. 

\ea\label{ex:cher:key:19}{hla yi[g\v{v}:hn\textbf{í}]ha} \\ 
\glll tlha   yi-k-vvn-hih-a\\
v:\ref{chnp1} \ref{chirr2}-\ref{chpron9}-\ref{chbase12}-\ref{chasp21}-\ref{chmod22} \\
not \Irr{}-\Third\Sg.\Aarg{}-hit-\Prs{}-\Ind{}\\
\glt `He is not hitting him.' (\citealt{PulteFeeling1975}: 345)
\z 

The modal suffixes in position 22 always have a high tone, either lexically or due to the boundary H tone (\citealt{Lindsey1985}: 125, 168, \citealt{Haag2002}: 414, \citealt{Johnson2005}: 17), and thus one cannot tell if they are within the domain of H3 \textsc{Assignment} or not, since a high tone could be the lexical high tone or due to the H3. Thus, the discussion so far defines the minimal domain of H3 \textsc{Assignment}: positions {7--21}. 

On the other hand, the pre-pronominal prefixes in position 5 (distributive) and 6 (cislocative) may or may not be within the domain of H3 \textsc{Assignment}, depending on their allomorphy and whether they combine with other pre-pronominal prefixes in positions {2--5} or not. 

First, the distributive prefix in position 5 has allomorphs \textit{tee- {\textasciitilde} ti- {\textasciitilde} c}{}-, the distribution of which is determined by complex phonological and morphosyntactic factors (\citealt{uchihara2016tone}: Appendix A). With the first allomorph \textit{tee}{}-, this prefix is included in the domain of H3 \textsc{Assignment}, and thus the H3 is assigned to the second syllable of the word:

 \ea\label{ex:cher:key:20} {[de:h\textbf{í}go:whtíh]a} \\
\glll tee-hi-koohw(a)hth-íh-a\\
v:\ref{chdist5}-\ref{chpron9}-\ref{chbase12}-\ref{chasp21}-\ref{chmod22} \\
\textsc{dist}-\Second\Sg.\Aarg{}-see-\Prs{}-\Ind{}\\
\glt `You see them.' (\citealt{PulteFeeling1975}: 248)
\z 


On the other hand, when the allomorphs \textit{ti- {\textasciitilde} c}{}- occur, this prefix is outside of the domain of H3 \textsc{Assignment,} and thus the H3 is assigned to the third syllable of the word, as in (\ref{bkm:Ref87349871}):

\ea\label{bkm:Ref87349871}di[jad\textbf{û}:g]a\footnote{The high-low tone on the penultimate syllable, instead of the expected high tone, is due to the underlying glottal stop.}  \\
\glll ti-c-at-u(ʔ)k-a\\
v:\ref{chdist5}-\ref{chpron9}-\ref{chbase12}-\ref{chasp21}-\ref{chmod22} \\
\textsc{dist}-\Second\Sg.\Barg{}-throw-\textsc{pnc}-\Ind{}\\ 
\glt `Throw it!' (\citealt{PulteFeeling1975}: 247)
\z 


When the cislocative prefix in position 6 occurs by itself without other pre-pronominal prefixes in positions 2-5, it behaves as other prefixes in positions  2-5  in that it is outside of the domain of H3 \textsc{Assignment}, and thus the H3 is assigned to the second syllable of the word:

\newpage
\ea\label{ex:cher:key:22} {da[yo:j\textbf{é}:dò:l]i} \\
\glll tay-ooc-eet-oo(ʔ)l-i\\
v:\ref{chcl6}-\ref{chpron9}-\ref{chbase12}-\ref{chasp21}-\ref{chmod22} \\ 
\Cisl{}-\textsc{1pl.\Excl.a}{}-walk.around-\Prf{}-\Mot{}\\
\glt `They and I will come here.' (EJ2011)
\z 


When the cislocative prefix is preceded by another prefix in positions 2--5, it falls within the domain of H3 \textsc{Assignment}, and the H3 is assigned to the syllable immediately after the syllable of the cislocative prefix \citep[204]{uchihara2016tone}: 

\ea\label{ex:cher:key:23} {ni[day\textbf{ú}:go:whtv́hd]i} \\
\glll ni-tay-uu-koohwahth-v́ht-i\\
 v:\ref{chpart4}-\ref{chcl6}-\ref{chpron9}-\ref{chbase12}-\ref{chasp21}-\ref{chmod22} \\
\Part{}-\Cisl{}-\Third\Sg.\Barg{}-see-\Inf{}-\Nom{}\\ 
\glt `for him to see it (looking this way).' (\citealt{PulteFeeling1975}: 246)
\z 

The morphemes outside of this domain are never within the domain of H3 \textsc{Assignment}. Thus, this defines the largest domain of H3 \textsc{Assignment}: positions {5--21}.  


\subsection{Domain of superhigh assignment (7--22; 5--22)}
\label{bkm:Ref72236002}
For another type of an accent in Cherokee, superhigh accent, the pre-pronominal in positions  2–6  are outside of its domain, as in the case of the H3 \textsc{Assignment} discussed above. However, the right edge of the \textsc{Superhigh} \textsc{Assignment} is at position 22 (modal suffixes), and not position  21 as in the case of the H3 \textsc{Assignment}. That is, modal suffixes are within the domain of \textsc{Superhigh} \textsc{Assignment}.

Superhigh accent is carried by a verb in a subordinate clause, by deverbal nouns, and by adjectives (\citealt{Cook1979}: 92, \citealt{Lindsey1985}: 125; \citealt[Ch 11.2]{uchihara2016tone}). Although its occurrence is morphosyntactically conditioned, it manifests some properties common to `accentual' systems: it is culminative (one per word), and its assignment is a `default-to-opposite' footing pattern (\citealt{Wright1996}: 21; \citealt[296--299]{Hayes1995}; \citealt{Kager2012}; \citealt{Kager1995}: 384): namely, the prominence is assigned to the last non-final long vowel in the word, while the prominence is assigned to the first syllable of the word when there is no long vowel in the word. 

Superhigh accent is found only on a long vowel, and is characterized by a gradual rise in pitch that rises to a point above the normal high tone register (\citealt{Wright1996}: 21, \citealt{Johnson2005}: 10). In (\ref{bkm:Ref72232270}), the penultimate syllable has the superhigh accent:

\ea\label{bkm:Ref72232270}[gv:jalhán\H{v}:hi] \\
\glll k-vvcal-áhn-vvhi\\
v:\ref{chpron9}-\ref{chbase12}-\ref{chasp21}-\ref{chmod22}\\ 
\Third\Sg.\Aarg{}-fry-\Prf{}-ppl/\Sh{}\\ 
\glt `fried.' \citep[127]{Feeling1975}
\z 

Extrametricality plays a role when there is no long vowel within the word. If there is no long vowel in the word, a high tone (H4 henceforth, represented with the acute accent diacritic, the same as H1 and H3 above, highlighted in boldface) is assigned to the first vowel of the phonological word, instead of a superhigh accent (\citealt{Lindsey1985}: 127, \citealt{Wright1996}: 21; \citealt{uchihara2016tone}: Ch. 11):

\ea\label{ex:cher:key:25} [\textbf{á}kisdi]\\
\glll a-khi-st-i\\
v:\ref{chpron9}-\ref{chbase12}-\ref{chasp21}-\ref{chmod22}\\
\Third\Sg.\Aarg{}-swallow-\Inf-\Nom/\Sh{}\\
\glt `pill' (lit. thing to swallow) \citep[33]{Feeling1975}
\z 

There is a systematic exception to this generalization stated above; that is, the H4 cannot be assigned to the prefixes in positions 2--6. In (\ref{bkm:Ref72232378}) and (\ref{bkm:Ref72232385}), H4 is assigned to the second syllable rather than the expected first syllable, which belongs to the pre-pronominal prefix:

\ea\label{bkm:Ref72232378}ji[g\textbf{á}hliha] (*jígahliha) \\
\glll ci-ka-lh-ih-a\\
v:\ref{chirr2}-\ref{chpron9}-\ref{chbase12}-\ref{chasp21}-\ref{chmod22}\\
\Rel{}-\Third\Sg.\Aarg{}-sleep-\Prs{}-\Ind{}/\Sh{}\\ 
\glt `the one who is sleeping.' (DJM2012)
\z 

\ea\label{bkm:Ref72232385}yi[ch\textbf{á}wasa] (*yíchawasa) \\
\glll yi-ca-hwa-s-a\\
v:\ref{chirr2}-\ref{chpron9}-\ref{chbase12}-\ref{chasp21}-\ref{chmod22}\\
\Irr{}-\Second\Sg.\Barg{}-buy-\Prf{}-\Ind{}/\Sh{}\\
\glt `If you buy it, ...' (JRS2012)
\z 

The right edge of \textsc{Superhigh} \textsc{Assignment} is the modal suffixes in position 22. This is illustrated in (\ref{bkm:Ref72235388}), where the superhigh accent is assigned to the vowel of the habitual modal suffix in position 22. 

\ea\label{bkm:Ref72235388}win[agíʔluhj\H{o}:ʔi] \\
\glll wi-n-aki-lʔu-hc-óóʔi\\
 v:\ref{chtl3}-\ref{chpart4}-\ref{chpron9}-\ref{chbase12}-\ref{chasp21}-\ref{chmod22}\\
\Trnsl{}-\Part{}-\First\Sg.\Barg{}-arrive-\Prf{}-\Hab{}/\Sh{} \\
\glt `After I arrived there, ...' (\citealt{PulteFeeling1975}: 351)
\z 
     
The superhigh accent cannot be assigned to the enclitics in position 23, even if they have a long vowel, as can be observed in the following example. Here, the enclitic =\textit{hééhnv} in position 23 has a long vowel, but the superhigh accent is not assigned here but rather on the vowel of the negative participle suffix -\textit{vvna} in position 22. Thus, the minimal domain of \textsc{Superhigh} \textsc{Assignment} consists of positions 7--22. 

\ea\label{ex:cher:key:29} {n[v:gáwò:nǐ:sg\H{v}:n]=hé:hn yíg} \\
\glll n-vv-ka-woo(ʔ)ni-:sk-vvna=hééhnv yi-ki\\
 v:\ref{chpart4}-\ref{chit7}-\ref{chpron9}-\ref{chbase12}-\ref{chasp21}-\ref{chmod22}=\ref{chclit23} \ref{chirr2}-12\\
\Part{}-\Iter{}-\Third{}\Sg{}.A-speak-\Impf{}-\Neg.\Pp{}/\Sh{}=because \Irr{}-\Cop/\Sh{}\\
\glt `If you don't speak, ...' (DF2012)
\z 
     
The distributive pre-pronominal prefix in position 5 may or may not be within the domain of \textsc{Superhigh} \textsc{Assignment}, again depending on its allomorphy, as in the case of the \textsc{H3} \textsc{Assignment} discussed above. The distributive prefix has the allomorphs \textit{tee- {\textasciitilde} ti- {\textasciitilde} c}{}-, the distribution of which being conditioned by complex phonological and morphosyntactic factors. When the allomorph \textit{tee}{}- occurs, this prefix can carry the superhigh accent, thus it is within the domain of superhigh assignment:

\ea\label{ex:cher:key:30} {ji[d\H{e}:kdladiʔi]} \\
\glll ci-tee-k-vhtlat-iʔ-i\\
v:\ref{chirr2}-\ref{chdist5}-\ref{chpron9}-\ref{chbase12}-\ref{chasp21}-\ref{chmod22}\\
\Rel{}-\textsc{dist}-\Third\Sg.\Aarg{}-put.out.fire-\Prs{}-\Nom/\Sh{}\\
\glt `the one who is putting out fire.' (DJM2012) 
\z 

On the other hand, when the allomorph \textit{ti}{}- occurs, the high variant of the superhigh accent (H4) cannot be assigned to this syllable and is instead assigned to the following syllable; in other words, it is outside of the domain of \textsc{Superhigh} \textsc{Assignment}: 

\newpage
\ea\label{ex:outsidedomainsuperhigh} {di[j\textbf{á}lhdohdi]} \\
\glll ti-ca-loht-oht-i\\
v:\ref{chdist5}-\ref{chpron9}-\ref{chbase12}-\ref{chasp21}-\ref{chmod22}\\ 
\textsc{dist}-\Second\Sg.\Barg{}-put.\Cmpl{}.into.container-\Inf{}-\Nom/\Sh{}\\
\glt `the one who is putting out fire.' (JRS2012) 
\z 

The morphemes outside of this domain are never within the domain of \textsc{Superhigh} \textsc{Assignment}. Thus, the discussion so far defines the largest domain of \textsc{Superhigh} \textsc{Assignment}: positions {5--22}.


\subsection{Final apocope (2--23)}
\label{bkm:Ref87347578}
The final underlying short vowel of the domain that contains positions \ref{chirr2}-\ref{chclit23} is deleted, and this apocope is not applied to any other vowels within this domain (\citealt{Bender1946}: 17; \citealt{Feeling1975}: xii; \citealt{Scancarelli1987}: 22, 46; \citealt{Montgomery-Anderson2008}: 58ff., \citealt[Ch 2.3]{Uchihara2013}). Thus, even in an elicitation setting, speakers usually give a form without the final vowel, and only occasionally give the `longer', `full' forms:

\ea\label{ex:cher:key:32} {[jà:lsdâ:y\H{v}:hvsk]} \\
\glll c-Ø-al(i)stá(ʔ)yvv-hvsk-(a) \\
v:\ref{chirr2}-\ref{chpron9}-\ref{chbase12}-\ref{chasp21}-\ref{chmod22}\\ 
\Rel{}-\Third\Sg.\Aarg{}-have.meal-\Prs{}-\Ind/\Sh{}\\ 
\glt `the one who is having a meal.' (JRS2012)
\z 


Enclitics in position 23 are within the domain of \textsc{Final} \textsc{Apocope} (cf. \citealt{Haag1997, Haag1999}). When an enclitic is attached, the word-final vowels (before the enclitic) are obligatory, even for speakers for whom deletion of the final vowels is the norm \citep[139]{Lindsey1985}. (\ref{bkm:Ref72241361}) is a form without an enclitic and the final vowel is deleted, while (\ref{bkm:Ref72241368}) has an enclitic =\textit{t\'{v}\'{v}} in position 23 and thus the final vowel of the verb is retained: 

\ea\label{bkm:Ref72241361}tlá=s [yà:go:hwáht] \\
\glll tlha=s y-a-koohwáhth-Ø-(a) \\
v:\ref{chnp1}=\ref{chnp1} \ref{chirr2}-\ref{chpron9}-\ref{chbase12}-\ref{chasp21}-\ref{chmod22}\\ 
not=\Q{} \Irr{}-\Third\Sg.\Aarg{}-see-\textsc{pct}-\Ind{}\\
\glt `Didn't he see it?' (DF1972)
\z 

\newpage
\ea\label{bkm:Ref72241368}v:, [à:go:hwáhtá=d\'{v}:] \\
\glll vv a-koohwáhth-Ø-a=tv́v́\\
v:\ref{chnp1} \ref{chpron9}-\ref{chbase12}-\ref{chasp21}-\ref{chmod22}=\ref{chclit23}\\
yes \Third\Sg.\Aarg{}-see-\textsc{pct}-\Ind{}=\Emph{}\\
\glt `Yes, he saw it.' (DF1972)
\z 

When the enclitic has a final short vowel, this final vowel of the enclitic is deleted instead. (\ref{bkm:Ref72241422}) is a form without an enclitic and the final vowel (as well as the onset \textit{ʔ}) is deleted, while (\ref{bkm:Ref72241446}) has a clitic =\textit{sk(o)} (interrogative), and thus the final vowel of the word is retained, but the final vowel of this clitic, \textit{o}, is deleted instead. The presence of the underlying final vowel \textit{o} of this clitic is evident when this clitic itself is followed by another clitic, as in (\ref{bkm:Ref72241456}):

\ea\label{bkm:Ref72241422}[hi:nâ:hlâ] \\
\glll hii-ná(ʔ)hlá(-ʔ-a) \\
v:\ref{chpron9}-\ref{chbase12}-\ref{chasp21}-\ref{chmod22}\\ 
\Second\Sg>\An{}-own.\An{}-\Prs{}-\Ind{}\\ 
\glt `You own it (AN).' (JRS2013)
\z 

\ea\label{bkm:Ref72241446}[hi:nâ:hláʔa=sk] \\
\glll hii-ná(ʔ)hlá-ʔ-a=sk(o)\\
v:\ref{chpron9}-\ref{chbase12}-\ref{chasp21}-\ref{chmod22}=\ref{chclit23}\\
\Second\Sg>\An{}-own.\An{}-\Prs{}-\Ind{}=\Q{}\\
\glt `Do you own it (AN)?' (JRS2013)
\z 

\ea\label{bkm:Ref72241456}[gawó:nihá=sgò:=hv]\footnote{The vowel of \textit{=skò} is lengthened before the enclitic \textit{=hvv} and is assigned a lowfall tone for an unknown reason.} \\
\glll ka-woó(ʔ)n-ih-a=skò=:hvv\\
v:\ref{chpron9}-\ref{chbase12}-\ref{chasp21}-\ref{chmod22}=\ref{chclit23}=\ref{chclit23}\\ 
\Third\Sg.\Aarg{}-speak-\Prs{}-\Ind{}=\Q{}=\Cntr{}\\
\glt `But is he speaking?' (\citealt{PulteFeeling1975}:294)
\z 

The left edge of this span is at position 2; when present, the final vowel of an NP in position 1 can undergo \textsc{Final} \textsc{Apocope}, as can be seen in (\ref{bkm:Ref72241596}). Here, the final vowel /o/ of \textit{kááko} `who' undergoes \textsc{Final} \textsc{Apocope}:

\newpage
\ea\label{bkm:Ref72241596}gá:g [sdalhno:hé ji:yò:sě:hv́ ] \\
\gllll káák(o) st-ali-hnoo-hé(h-a) ciiy-ooʔs-eéh-vv́ʔi\\
v:\ref{chnp1} \ref{chpron9}-\ref{chmid10}-\ref{chbase12}-\ref{chasp21}-\ref{chmod22} \ref{chpron9}-\ref{chbase12}-\ref{chasp21}-\ref{chmod22}\\
n:\ref{chnbase7} - -\\  
who \Second\Du{}-\textsc{mid}-tell-\Prs{}-\Ind{} \First\Sg>\An{}-say-\Impf{}-\Asr{}\\
\glt ```Who are you talking to?'' I said to him.' (CNRS)
\z 


\subsection{Syllabification (2--23)}
\label{bkm:Ref87347583}
The span that extends from position 2 to 23 is syllabified according to the following maximal syllable template (O = onset, R = Rhyme, N = nucleus, C = coda, and V = vowel), which is also subject to phonotactics constraints (see \figref{fig:cher-syll}). Such a syllable template is justified by the \textsc{Maximal} \textsc{Onset} \textsc{Principle} \citep{Selkirk1982}, \textsc{Closed} \textsc{Syllable} \textsc{Shortening} which applies only in certain contexts, and native speaker judgments. Here the syllabification is mostly based on the judgement by speaker-linguist Durbin Feeling (see (\citealt[Ch. 3]{uchihara2016tone}) for more detail).

\begin{figure}
    \centering
    %\includegraphics[scale=.6]{figures/cherokee-fig2.png}
    \begin{forest}
      [σ
        [O, tier=ONC
            [x]
            [x]
            [x]
            [x]
        ]
        [R
            [N, tier=ONC
                [x,name=xN1]
                [~,no edge
                    [~,no edge,name=empty]
                ]
                [x,name=xN2]
            ]
            [C, tier=ONC
                [x]
                [~,no edge]
                [x]
            ]
        ]
      ]
      \draw(xN1)--(empty.north);
      \draw(xN2)--(empty.north);
    \end{forest}

    \caption{Maximal Syllable Template in Oklahoma Cherokee}
    \label{fig:cher-syll}
\end{figure}

(\ref{bkm:Ref72241694}) shows that syllabification is applied regardless of the morpheme boundaries within the domain of positions \ref{chirr2}-\ref{chclit23} . Note that the syllable boundaries (marked with dots) are placed within the base in position 12 and the aspect suffix in position 21 :

\newpage
\ea\label{bkm:Ref72241694}[gà:.ni.gí.ʔa] \\
\glll k-a:hnik-íʔ-a\\
v:\ref{chpron9}-\ref{chbase12}-\ref{chasp21}-\ref{chmod22}\\
\First\Sg.\Aarg{}-start-\Prs{}-\Ind{}\\
\glt `I'm starting (to walk).' \citep[25]{Feeling1975}
\z 

The pre-pronominal prefixes in positions  2–8 are also parsed into syllables, again confirming their status as part of the domain of syllabification:

\ea\label{x:key:41} {hla [ya.gwá:nh.ta]} \\
\glll hla y-akw-aanht-h-a \\
v:\ref{chnp1} \ref{chirr2}-\ref{chpron9}-\ref{chbase12}-\ref{chasp21}-\ref{chmod22}\\
not \Irr{}-\First\Sg.\Barg{}-know-\Stat{}-\Ind{}\\
\glt `I don't know.' (\citealt{PulteFeeling1975}: 242)
\z 

Enclitics in position 23 also form part of the domain of syllabification, even though in most cases it is not observable since most of the clitics begin with a consonant, and form a separate syllable on their own. However, Durbin Feeling's transcription (he writes the tonal superscript after the syllable boundary in his 1975 dictionary, \citealt{PulteFeeling1975}) below suggests that he analyzes the interrogative clitic =\textit{s} as forming a syllable along with the preceding sequence \textit{ha:}

\ea\label{ex:cher:key:42} {[gạ\textsuperscript{2}wo\textsuperscript{3}nị\textsuperscript{2}has\textsuperscript{3}]} \\
\glll {ka-woó(ʔ)n-ih-a=s}\\
v:\ref{chpron9}-\ref{chbase12}-\ref{chasp21}-\ref{chmod22}=\ref{chclit23}\\
\Third\Sg.\Aarg{}-speak-\Prs{}-\Ind{}=\Q{}\\
\glt `Is he speaking?' (\citealt{PulteFeeling1975}: 293)
\z 


Moreover, the enclitic =\textit{éekv} `also' is syllabified with the preceding morphemes. 

\ea\label{ex:cher:key:43} {[ù:.nv:.ke:w.sgê:.gv́]} \\
\glll {uun-vvkheew(i)-sk-(a)=éekv}\\
v:\ref{chpron9}-\ref{chbase12}-\ref{chasp21}-\ref{chmod22}=\ref{chclit23}\\
\Third\Pl.\Barg{}-forget-\Prs{}-\Ind{}=also\\
\glt `They are forgetting.' (CNRS)
\z 

Syllabification does not apply across orthographic word boundaries (i.e. between position 1 and what follows, and between position 23 and 24), as the following examples show. In (\ref{bkm:Ref72241837}), the final \textit{n} of the first orthographic word (which results from \textsc{Final} \textsc{Apocope}) does not constitute the onset of a syllable with the initial vowel of the following verb. Thus, the discussion so far shows that the left edge of the domain of syllabification is the position 2. 

\ea\label{bkm:Ref72241837}jí:.sdv:n [à:.wa.du:.lí] (*jí:.sdv:.nà:.wa.du:.lí) \\
\gllll {cíístvvn(a) akw-atuul-í(h-a)}\\
v:\ref{chnp1} \ref{chpron9}-\ref{chbase12}-\ref{chasp21}-\ref{chmod22}\\
n:\ref{chnbase7} -\\  
crawdad \First\Sg.\Barg{}-want-\Prs{}-\Ind{}\\ 
\glt `I want a crawdad.' (JRS2013)
\z 

In (\ref{bkm:Ref72241947}), the interrogative enclitic =\textit{s} in position 23 is not syllabified as the onset of the following vowel which belongs to another morpheme which occupies the position 24; thus, this defines that the right edge of the domain of syllabification is the position 23:

\ea\label{bkm:Ref72241947} gv́:n nǒ:=hv́ [aně:=s] áhan e:sga̋:hn \\
    \glll {} kv́v́na noókwu=hv́v́ an-eé(h-a)=s áhani	eeskaa̋hni  \\
    v: \ref{chnp1} \ref{chnp1} \ref{chpron9}-\ref{chbase12}:\ref{chasp21}-\ref{chmod22}=\ref{chclit23} \ref{chnp24} \ref{chnp24} \\  
    {} turkey now=and \Third{}\Pl{}.A-live:\Stat{}-\Nom{}/\Sh{}=\Q{} here nearby \\
    \glt `And turkeys, do they live here?' (CNRS)
\z 


\subsection{h-Metathesis and vowel deletion (2--23)} \label{sec:3.6}

The span that extends from position 2 to 23 is also the domain of a set of segmental processes, \textit{h}{}-\textsc{Metathesis} and \textsc{Vowel} \textsc{Deletion.} These two process are motivated by the dispreference of a \textit{CVh} sequence in Oklahoma Cherokee; when such a sequence occurs, it is remedied by deleting the vowel when \textit{h} is followed by a plosive/affricate or by another vowel (henceforth `\textsc{Vowel} \textsc{Deletion}') as in (\ref{bkm:Ref72242050}), or `metathesizing' \textit{V} and \textit{h} when \textit{h} is followed by a resonant, as in (\ref{bkm:Ref72242059}) (henceforth `\textit{h}{}-\textsc{Metathesis}'; \citealt{Cook1979}, \citealt{Flemming1996}, \citealt{Uchihara2007}, \citealt{Uchihara2013}: Ch.3). Note that the \textit{C} in the dispreferred \textit{CVh} sequence is not also an \textit{h}. The phonemic transcriptions are provided in // so that the behavior of \textit{h} is more visible, which is obscured by the surface representations.

\ea\label{bkm:Ref72242050}[kdíha] /khtíha/ \\
\glll {k-vht-í(h-a)}\\
v:\ref{chpron9}-\ref{chbase12}-\ref{chasp21}-\ref{chmod22}\\
\Third\Sg.\Aarg{}-use-\Prs{}-\Ind{}\\ 
\glt `He is using it.' \citep[142]{Feeling1975}
\z 

\ea\label{bkm:Ref72242059}[kanalu:sga] /khanalu:ska/ \\
\glll {ka-hnaluu-sk-a}\\
v:\ref{chpron9}-\ref{chbase12}-\ref{chasp21}-\ref{chmod22}\\
\Third\Sg.\Aarg{}-ascend-\Prs{}-\Ind{}\\
\glt `He is ascending.' \citep[138]{Feeling1975}
\z 

Deletion is also triggered by an \textit{s}. From this fact, we can propose that Oklahoma Cherokee has a constraint against \textit{CVh} or \textit{CVs} sequences, which is remedied as in (\ref{bkm:Ref72242123})\footnote{Here, \textit{C} = any consonant, \textit{T} = plosives and affricates, and \textit{R} = resonants.}.

\ea \label{bkm:Ref72242123}*\textit{CVh} remedies \\ 
  \begin{tabular}{ll}
       a. Deletion: &  \textit{C(V)hT → ChT} \\ 
       & \textit{T(V)hV → ThV} \\ 
       & \textit{C(V)sT → CsT} \\
       & \textit{C(V)sV → TsV}  \\  
       b. Metathesis:  &  \textit{CVhR → ChVR} \\
       \end{tabular}
\z 

\textsc{Vowel} \textsc{Deletion} or \textit{h}{}-\textsc{Metathesis} applies regardless of the morpheme boundary, as long as the target sequence is within the span of positions \ref{chirr2}-\ref{chclit23}. This test is not fractured since the minimal domain, where these processes are known to apply, and the maximal domain, outside of which these processes never apply, coincide. (\ref{bkm:Ref72242050}) and (\ref{bkm:Ref72242059}) above illustrate cases where \textsc{Vowel} \textsc{Deletion} or \textit{h}{}-\textsc{Metathesis} applies between the pronominal prefix in position 9 and the verb base in position 12. (\ref{bkm:Ref72242809}) shows that \textsc{Vowel} \textsc{Deletion} applies between the cislocative pre-pronominal prefix in position 6 and a pronominal prefix, confirming that the cislocative \textit{t(a)}{}- is within the domain of this process:

\ea\label{bkm:Ref72242809}[tíʔgi] /thíʔki/ \\
\glll {t(a)-hi-k-ʔ-i}\\
v:\ref{chcl6}-\ref{chpron9}-\ref{chbase12}-\ref{chasp21}-\ref{chmod22}\\
\Cisl{}-\Second\Sg.\Aarg{}-eat-\Prf{}-\Mot{}\\
\glt `You will eat it.' (JRS2012)
\z 

Similarly, the irrealis prefix \textit{y(i)-} in position 2 can undergo \textsc{Vowel} \textsc{Deletion}:

\ea\label{ex:cher:key:50} {go:hű:sdi [yhi:yádu:lvʔ\H{e}] kilő} \\
\gllll {} koohuűsti y(i)-hiiy-atuul-vvh-ʔeh-a khiloő \\
v: \ref{chnp1} \ref{chirr2}-\ref{chpron9}-\ref{chbase12}-\ref{chasp13}-\ref{chdat18}:\ref{chasp21}-\ref{chmod22} \ref{chnp24}\\
n: \ref{chnbase7} - \ref{chnbase7}\\  
{} something \Irr{}-\Second\Sg>\An{}-want-\Prf{}-\Dat:\Prs{}-\Ind{} someone\\
\glt `If you want something from someone.' \citep{Montgomery-Anderson2015}
\z 

The following example illustrates a case where \textit{h}{}-\textsc{Metathesis} is applied between the verb base -\textit{asest-} in position 12 and the aspect suffix -\textit{áhn}{}- in position 21.

\ea\label{ex:cher:key:51} {[ù:sestán\v{v}:ʔi] /ù:sesthán\v{v}:ʔi/} \\
\glll {uu-(a)sest-áhn-v\'{v}ʔi}\\
v:\ref{chpron9}-\ref{chbase12}-\ref{chasp21}-\ref{chmod22}\\
\Third\Sg.\Barg{}-include-\Prf{}-\Ind{}\\ 
\glt `He included him.' \citep[49]{Feeling1975}
\z 


\textit{h}{}-\textsc{Metathesis} or \textsc{Vowel} \textsc{Deletion} never apply beyond the span of positions \ref{chirr2}-\ref{chclit23}. On the left side, an element from position 1 cannot participate in these processes, as can be observed in (\ref{bkm:Ref72242969}); here, the sequence \textit{kwa + h} satisfies the condition for \textsc{Vowel} \textsc{Deletion}, but it is not applied, since the sequence includes an element from position 1. 

\ea\label{bkm:Ref72242969}jí:sgwa [hihye:lí:ʔa] (*jí:skwihye:lí:ʔa) \\
\gllll cíískwa  hi-hyeel-iíʔ-a\\
v:\ref{chnp1} \ref{chpron9}-\ref{chbase12}-\ref{chasp21}-\ref{chmod22}\\
n:\ref{chnbase7} -\\ 
bird \Second\Sg.\Aarg{}-imitate-\Prs{}-\Ind{}\\
\glt `You are imitating a bird.' (EJ2011)
\z 

On the right side, an element from position 24 cannot participate in \textit{h}{}-\textsc{Metath\-e\-sis} or \textsc{Vowel} \textsc{Deletion}, as can be observed below. Here, the sequence \textit{ti} and \textit{h} satisfy the structural requirement for these processes to be applied, but they are not, since the \textit{h} belongs to an element in position \ref{chnp24}.

\ea\label{ex:cher:key:53} { \H{o}:sd [yú:lsdohdí] hawi:yá  (*yú:lsdohtawi:yá)} \\
\glllll oősta iy-uu-alist-oht-i hawiiya \\
v:\ref{chnp1}  \ref{chpart4}-\ref{chpron9}-\ref{chbase12}-\ref{chasp21}-\ref{chmod22} \ref{chnp24}\\
n:- - \ref{chnbase7}\\ 
a:7 - - \\ 
good \Part{}-\Third\Sg.\Barg{}-become-\Inf{}-\Nom{} meat\\
\glt `So that the meat becomes well.' (RK2012)
\z 


\section{Morphosyntactic domains}
\label{bkm:Ref87352909}
In this section, I present seven morphosyntactic (and indeterminate) diagnostics applied to the Oklahoma Cherokee verbs: deviations from biuniqueness (§\ref{bkm:Ref72232968}), ciscategorial selection (§\ref{bkm:Ref87350401}), minimum free form (§\ref{bkm:Ref87350406}), non-permutability (§\ref{bkm:Ref72229055}), non-interruption (§\ref{bkm:Ref87350419}), repeated subspan (§\ref{bkm:Ref87350432}) and nominalization (§\ref{bkm:Ref87350422}). Nominalization is a type of subspan repetition, but it is treated here separately for convenience.\footnote{For the purposes of this chapter `indeterminate' domains such as free occurrence are classified as `morphosyntactic.'}

\subsection{Deviations from biuniqueness (4--13, 4--22)}
\label{bkm:Ref72232968}
A deviation from biuniqueness refers to the lack of a one-to-one relation between forms and their meanings. Cases of (non-automatic) allomorphy, suppletion, multiple exponence etc. represent deviations from biuniqueness. 

All positions within the span that extends from position 4 to 13 manifest allomorphy that is not automatic (that is, alternations due to productive phonological processes, as in the processes discussed in §\ref{bkm:Ref87347154}). The minimal domain of deviations from biuniqueness is therefore positions 4 to 13. For instance, the partitive prefix in position 4 shows allomorphy between \textit{ni}- and \textit{i(y)-} conditioned by the presence of the nominal modal suffix in position  \citep[64]{Cook1979}; the distributive prefix in position 5 alternates between \textit{tee- {\textasciitilde} ti- {\textasciitilde} c}{}-, conditioned by complex phonological and morphosyntactic factors (\citealt{uchihara2016tone}: Appendix A); the allomorphy of the \textsc{1sg} agentive prefix in position 9 between \textit{k- {\textasciitilde} ci}{}- is conditioned by the following sound. In most of the cases the allomorphs are predictable from the phonological and morphological contexts, except for the \textsc{3sg} agentive pronominal prefix, which shows allomorphy of \textit{k(a)- {\textasciitilde} a- {\textasciitilde} Ø}{}- that is partially lexically conditioned.

However, the morphemes outside of the domain of positions 4-22 do not show any (non-automatic) allomorphy: the NPs in position 1 (that is, there is no non-automatic allomorphy at the junctures between NPs and other positions); the irrealis and the relative pre-pronominal prefixes in position 2; the translocative prefix in position 3; the enclitics in position 23; and the NPs in position 24. This defines the \textsc{maximal} domain of \textsc{deviations} \textsc{from} \textsc{biuniqueness}.

Between the minimal and maximal domain (namely positions 14 - 21), there are some positions where the morphemes show non-automatic allomorphy. Unlike in the case of the allomorphy within the minimal domain, where the distribution of the allomorphs is mostly predictable from phonological and morphological environments, in the case of the maximal domain the allomorph selection is mostly lexically conditioned. Thus, the causative suffix in position 16 shows various allomorphs -\textit{oht-, -iʔst-, -st}{}-, etc., which are lexically conditioned (cf. \citealt{Mithun2000}); the dative suffix in position  shows allomorphs -\textit{hééh-{\textasciitilde} -ʔééh}{}-, where the conditioning factor is still unknown. Especially the aspectual suffixes in this position manifest complex allomorphy, the combination of which results in no fewer than 67 inflectional classes.

\subsection{Ciscategorial selection (12--22; 2--22)}
\label{bkm:Ref87350401}
Ciscategorial selection refers to a span where all of the elements are strictly modifiers or dependents with a certain part of speech, in this case verbs. A morpheme is ciscategorial if it can only occur with verbs, while it is transcategorial if it can also occur with other parts of speech. This test is fractured into minimal and maximal tests as follows:

\ea \label{ex:cher:key:54} Ciscategorial selection (minimal): all the morphemes in this span are unique to verbs. 
\z 
 \ea\label{ex:cher:key:55} Ciscategorial selection (maximal): all the morphemes outside of this span can not only occur with verbs but also with other parts of speech.
 \z

All the morphemes in the domain that extends from position 12 to 22 are ciscategorial; that is, they are unique to verbs. Thus, to the right side of the verb root in position 12, all positions up to 22 are unique to verbs, while position 23 elements (enclitics) can attach to nouns and adjectives in addition to verbs. 

To the left of the verb root in position 12, not all the morphemes are ciscategorial; that is, while morphemes in positions 8 (negative), 7 (iterative), 6 (cislocative) are unique to verbs, other morphemes are transcategorial. The incorporated noun root in position 11 can occur with an adjectival root,\footnote{As mentioned above, adjectives are more like nouns than verbs, in contrast to Northern Iroquoian \citep{Chafe2012}.}  as in \textit{a-sgù:-sd\H{a}:y} [\textsc{3sg.a}{}-head-hard] `stubborn'. The reflexive prefix in position 10 can occur with nouns, as in \textit{di:-(a)n-ada:-hn\H{v}:hli} [\textsc{dist-3pl.a-refl}{}-brother] `(they are) brothers'\footnote{\citet{Konig2010} argue that kinship terms like this constitute independent parts of speech in Oneida, a Northern Iroquoian language.}  as well as with verbs as in \textit{à:-(a)da:-go:whtíha} [\textsc{3sg.a-refl}{}-see] `he sees himself'. Pronominal prefixes in position 9 can also occur with nouns to express possessors or the copula subject as in \textit{jì:-sgaya} [\textsc{1sg.a}{}-man] `I'm a man' as well as with verbs as in \textit{ji-gíʔa} [\textsc{1sg.a}{}-eat] `I eat'. The distributive prefix in position 5 can occur with a noun as in \textit{di:-(a)sgwag\H{e}:ni} [\textsc{dist}{}-side] `sides' as well as with verbs as in \textit{di-chano:gî:sdi} [\textsc{dist}{}-for.you.to.sing] `for you to sing'. The partitive prefix (position 4) can be found with a noun as in \textit{i:-n\H{v}:d} [\textsc{part}{}-month] `months' as well as with a verb as in \textit{iy-ú:d\`{v}:nhdi} [\textsc{part}{}-for.him.to.do] `for him to do it'. The translocative prefix in position 3 can occur with an adjective as in \textit{w-ǔ:sdǐ:k\H{v}:ʔi} [\textsc{trnsl}{}-small-\Int{}] `smallest', so can the relative prefix in position 2 as in \textit{ji-ganiyè:gv̋ } [\textsc{rel}{}-dangerous] `when he was dangerous'. 

All elements outside of the span of positions 2--22  are transcategorial. This defines the \textsc{maximal} domain of \textsc{ciscategorial} \textsc{selection}. That is, the morphemes in positions 1 (NPs), 23 (enclitics) and 24 (NPs) can attach to any parts of speech. For instance, the enclitics in position 23 can attach to any parts of speech as long as they occupy the first `position' in the clause, as can be observed in the following examples; in (\ref{bkm:Ref72236786}) the interrogative enclitic =\textit{sk(o)} attaches to a verb, while in (\ref{bkm:Ref72236839}) it attached to a noun. 

\ea\label{bkm:Ref72236786}jadu:lí:=sk kanu:n \\
\gllll {c-atuul-ií(h-a)=sk(o) khanuuna}\\
v:\ref{chpron9}-\ref{chbase12}-\ref{chasp21}-\ref{chmod22}=\ref{chclit23} \ref{chnp24}\\
n:- \ref{chnbase7}\\ 
\Second\Sg.\Barg{}-want-\Prs{}-\Ind{} bullfrog\\
\glt `Do you want a bullfrog?' (JRS2013)
\z 

\ea\label{bkm:Ref72236839}kanu:ná=sk jadu:lí \\
\gllll khanuuna=sk(o) c-atuul-iíh-a \\
v:\ref{chnp1}=\ref{chnp1} \ref{chpron9}-\ref{chbase12}-\ref{chasp21}-\ref{chmod22}\\
n:\ref{chnbase7}=\ref{chncl10} -\\ 
bullfrog \Second\Sg.\Barg{}-want-\Prs{}-\Ind{}\\
\glt `Do you want a bullfrog?' (JRS2013)
\z 


\subsection{Minimum free form (9--22; 2--23)} 
\label{bkm:Ref87350406}
\citet[18]{Tallman2020} states that free occurrence identifies a span that contains contiguous positions whose elements can be uttered as a complete utterance. This test is fractured into two:

\ea\label{ex:cher:key:58} Minimum free form (minimal): the shortest span overlapping the verb core that is a complete utterance. It is felicitous to answer a question with that form (e.g. Q: \textit{When did you go to the store?} A: \textit{Early}).
\z 
\ea\label{ex:cher:key:59} Minimum free form (maximal): the longest span overlapping the verb core that can be a single free form.
\z 

A minimal verb form in Cherokee consists of a pronominal prefix (position 9), root (position 12), aspectual suffix (position 21) and a modal suffix (position 23). Thus, the domain of the \textsc{minimal} \textsc{minimum} \textsc{free} \textsc{form} is the span that extends from position 9 to 22. This is illustrated in (\ref{bkm:Ref87351947}):

\ea\label{bkm:Ref87351947}galo:sga\\
\glll {ka-loo-sk-a}\\
v:\ref{chpron9}-\ref{chbase12}-\ref{chasp21}-\ref{chmod22}\\
\Third\Sg.\Aarg{}-pass-\Prs{}-\Ind{}\\
\glt `He is passing it.' \citep[102]{Feeling1975}
\z 

There are a few apparent exceptions to this generalization. First, the copula \textit{iíki/ -ki/ keeʔs}{}- and \textit{ciíy}{}- `it (something long) is lying' do not take any pronominal prefix, unless they contain a fossilized 3\textsc{sg} agentive prefix \textit{k}{}- or \textit{c}{}-. Secondly, some verbs do not have any segmental exponents for the aspectual suffixes in the punctual or stative forms. In such cases I consider them to have a zero suffix; such an analysis is justified by the fact that other allomorphs of such suffixes have segmental exponents.

The span of \textsc{maximal} \textsc{minimum} \textsc{free} \textsc{form}, which is the maximal form that can stand alone and cannot be separated, covers positions \ref{chirr2}-\ref{chclit23}. If one wishes to add elements beyond a 2--23 span, the resulting utterance will no longer be a single free form. Thus, the utterance in (\ref{bkm:Ref72243097}) has elements in position 1 and 24 from the verbal planar structure, each of which constitutes single free forms.

\ea\label{bkm:Ref72243097}hawâ: ga:n\`{v}:dadî:sgó:=d\'{v}: u:gò:dí=w \\
\gllll {hawa k-aanvhtat-íʔsk-óóʔi=tv́v́  uu-kòòti=kwúú}\\
v:\ref{chnp1} \ref{chpron9}-\ref{chbase12}-\ref{chasp21}-\ref{chmod22}=\ref{chclit23} \ref{chnp24}\\
a:- - 5-7=9\\ 
okay \First\Sg.\Aarg{}-remember-\Impf{}-\Hab{}=\Emph{} \Third\Sg.\Barg{}-be.more=DT\\
\glt `Of course I remember a lot.' (CNRS)
\z 

\subsection{Non-permutability (2--17; 2--22)}
\label{bkm:Ref72229055}
Non-permutability, or fixed order, identifies spans where the ordering of elements is fixed \citep[23]{Tallman2020}. Cherokee affix order is fairly rigid within the span of positions 2--17, except that the dative and the ambulative suffixes in position 18 are attested with a variable order, as shown in (\ref{bkm:Ref72232545}) and (\ref{bkm:Ref72232554}). As can be noted in the translations, there does not seem to be any scope differences. Thus, the minimal domain of non-permutability extends from position 2 to 17, where the affix order is rigid. 

\ea\label{bkm:Ref72232545}[dà:kgi:ló:ʔe:l]ǐ:dô:ha \\
\glll {t-ak-vhkiiloó-ʔ-eel-iit-óo(ʔ)h-a}\\
v:\ref{chdist5}-\ref{chpron9}-\ref{chbase12}-\ref{chasp13}-\ref{chdat18}:\ref{chasp13}-\ref{chdat18}-\ref{chasp21}-\ref{chmod22}\\
\Dist-\First\Sg.\Barg{}-wash.\Fl{}-\Prf{}-\Dat:\Prf{}-\Amb{}-\Prs{}-\Ind\\
\glt `He goes around washing for me.' (PA1971)
\z 


\ea\label{bkm:Ref72232554}[gawó:ni:his]ǐ:dô:leha \\
\glll {ka-woó(ʔ)ni-:his-iit-óo(ʔ)l-eh-a}\\
v:\ref{chpron9}-\ref{chbase12}-\ref{chasp13}-\ref{chdat18}-\ref{chasp13}-\ref{chdat18}:\ref{chasp21}-\ref{chmod22}\\
\Third\Sg.\Aarg{}-speak-\Prf{}-\Amb{}-\Prf{}-\Dat{}:\Prs-\Ind{}\\ 
\glt `He is going around speaking for him.' \citep[319]{Feeling1975}
\z 

All elements outside of the span of 2--22 have no fixed order: this concerns the NPs in position 1 as well as enclitics in position 23. This is the maximal domain of non-permutability. First, constituent order in Cherokee is free (\citealt{Scancarelli1987}: \S 3.7; \citeyear{Montgomery-Anderson2015}: \S 11.1 and references therein). \citet{Scancarelli1987} states that “most word orders in Cherokee are variable: not just major constituent orders, but also order within constituents” (ibid.). Thus, any order of S, V and O is possible when the pronominal prefix unambiguously distinguishes the subject from the object \citep[189]{Scancarelli1987}, as in (\ref{bkm:Ref72236939}) – (\ref{bkm:Ref72236947}), which all describe the same situation, even though many speakers prefer not to have the verb appear sentence initially as in (\ref{bkm:Ref72236976}) or (\ref{bkm:Ref72236947}).

\ea\label{bkm:Ref72236939}gi:hli u:sgala achu:ja \\
\gllll kiihli uu-skal-Ø{}-a a-chuuca \\
v:\ref{chnp1} \ref{chpron9}-\ref{chbase12}-\ref{chasp21}-\ref{chmod22} \ref{chnp24}\\
n:\ref{chnbase7} -  \ref{chnpron4}-\ref{chnbase7}\\ 
dog \Third\Sg.\Barg{}-bite-\textsc{pnc}-\Ind{} \Third\Sg.\Aarg{}-boy\\
\glt `The dog bit the boy.' \citep[189]{Scancarelli1987}
\z 

\ea\label{ex:cher:key:65}{gi:hli achu:ja u:sgala} 
\z
\ea\label{ex:cher:key:66}{achu:ja u:sgala gi:hli}
\z
\ea\label{ex:cher:key:67}{achu:ja gi:hli u:sgala}
\z
\ea\label{bkm:Ref72236976}{u:sgala gi:hli achu:ja}
\z
\ea\label{bkm:Ref72236947}{u:sgala achu:ja gi:hli}
\z


At the same time, \citet[173ff.]{Scancarelli1987} remarks that certain orders are not variable; for instance, determiners, numbers and genitives must precede nouns; postpositions always occur after the nouns; and the standard of comparison must follow the comparative adjective in comparative constructions; copula may not precede a predicate nominal or adjective.

Secondly, the order of enclitics in position 23, at least some of them, also seems to be free. Thus, the delimiter enclitic =\textit{kwúú} (`only, just') and the conjunctive enclitic =\textit{hnóó} (`and') can occur in either order. 

\ea\label{ex:cher:key:70} { à:waksestan\'{v}:=wú=hnó}  \\
\glll akw-akasest-ahn-vv́ʔi=kwúú=hnóó \\
v:\ref{chpron9}-\ref{chbase12}-\ref{chasp21}-\ref{chmod22}=\ref{chclit23}=\ref{chclit23}\\ 
\First\Sg.\Barg{}-watch-\Prf{}-\Asr{}=\Dt{}=and\\
\glt `I just looked at (it).' (CNRS)
\z 

\ea\label{ex:cher:key:71} {e:jí=hna=wú}\\
\glll ee-ci=hnóó=kwúú \\
n:\ref{chnpron4}-\ref{chnbase7}=\ref{chnp1}0=\ref{chnp1}0\\ 
\First\Sg.\Barg{}-mother=and=\Dt{}\\
\glt `and mom (watched).' (CNRS)
\z 

More work is needed to determine the precise ordering of the enclitics.


\subsection{Non-interruptability (2--22)}
\label{bkm:Ref87350419}
Non-interruptability identifies a span of positions that cannot be interrupted by some interrupting element \citep[20]{Tallman2020}. Here I use the diagnostic of whether two positions can be interrupted by the second position enclitics. The domain which spans from position 2 to 22 cannot be interrupted with other elements, whether free or bound. Position 1 and the following morpheme can be interrupted by an enclitic as in (\ref{bkm:Ref72236223}), as well as the position 24 and the preceding morpheme as in (\ref{bkm:Ref72236233}):

\ea\label{bkm:Ref72236223}agv:y\H{i}=hé:hn di:wátvs\H{v} gè:hv \\
\gllll a-kvvyii̋=hééhnv ti-akw-athv-s-vv́ʔi kèès-vv́́ʔi\\
v:\ref{chnp1}=\ref{chnp1} \ref{chcl6}-\ref{chpron9}-\ref{chbase12}-\ref{chasp21}-\ref{chmod22} 7-\ref{chclit23}\\
a:5-7=9 - -\\ 
\Third\Sg.\Aarg{}-first=because \Cisl{}-\First\Sg.\Barg{}-grow.up-\Prf{}-\Asr/\Sh{} \Cop{}-\Asr{}\\
\glt `As for where I first grew up.' (CNRS) 
\z 

\ea\label{bkm:Ref72236233}jì:wát yaw\H{e}:lì:sá=hé:hn kil\H{o} \\
\gllll cii-hwahth-Ø-(a)  y-akw-eel-i(ʔ)s-a=hééhnv khilo\H{o}ʔi\\
v:\ref{chpron9}-\ref{chbase12}-\ref{chasp21}-\ref{chmod22} \ref{chirr2}-\ref{chpron9}-\ref{chbase12}-\ref{chasp21}-\ref{chmod22}=\ref{chclit23} \ref{chnp24}\\
n:- - 7\\ 
\First\Sg>\An{}-find-\textsc{pnc}-\Ind{} \Irr{}-\First\Sg.\Barg{}-think-\Prf{}-\Ind/\Sh{}=because someone\\ 
\glt `Because when I think I find someone...' (CNRS)
\z 

The enclitics in position 23 can also be interrupted by other enclitics:

\ea\label{ex:cher:key:74} {yáni:gà:lsdi=wú=lé} \\
\glll y-´-anii-ka(ʔ)l-st-i=kwúú=léé \\
v:\ref{chirr2}\ref{chit7}-\ref{chpron9}-\ref{chbase12}-\ref{chasp21}-\ref{chmod22}=\ref{chclit23}=\ref{chclit23}\\
\Irr{}-\Iter{}-\Third\Pl.\Aarg{}-cut.\Fl{}-\Inf{}-\Nom{}=\Dt{}=or \\
\glt `They can cut it out.' (DC2012)
\z 


\subsection{Repeated subspan (2--23; 1--24)}
\label{bkm:Ref87350432}
According to \citet[30]{Tallman2020}, the \textsc{minimal} \textsc{repeated} \textsc{subspan} is “the subspan of positions whose elements cannot be interpreted unless they are present in the subspan itself. The elements of the positions in the subspan cannot be elided under co-/subordination or the positions of the subspan cannot have wide scope over the repeated subspans.” Within repeated subspans, only position 1 or 24 can be elided. For instance, in (\ref{bkm:Ref72243228}), the NP in position 1 can be elided, but the pronominal prefixes in position 9, the aspectual suffixes in position 21 and the modal suffixes in position 22 are coreferential but none of them can be elided:

\ea\label{bkm:Ref72243228}gi:hli ù:dl\'{v}:gi (gi:hli) galihwó:gi=hn\'{v}: \\
\gllll kiihli uu-htlvv́-(ʔ)k-i kiihli ka-lihwoó-(ʔ)k-i=hn\'{v}\'{v}\\
v:\ref{chnp1} \ref{chpron9}-\ref{chbase12}-\ref{chasp21}-\ref{chmod22} 1 \ref{chpron9}-\ref{chbase12}-\ref{chasp21}-\ref{chmod22}=\ref{chclit23}\\
n:\ref{chnbase7} - \ref{chnbase7} -\\  
dog \Third\Sg.\Barg{}-be.sick-\textsc{pnc}-\Ind{} dog \Third\Sg.\Aarg{}-die-\textsc{pnc}-\Ind{}=and\\
\glt `A dog got sick and died.' (DF1972)
\z 
 
The following example illustrates that the element in position 24 \textit{sgwu} `also' has scope over the two coordinated infinitive verbs, \textit{digigo:lǐ:yê:dí} `to read' and \textit{digo:hwě:lô:dí} `to write' (because the speaker is contrasting `speaking' with `reading' or `writing', neither of which he knew how to). Thus, this confirms that the position 24 is also outside of the subspan of the minimal repeated subspan. 

\ea\label{ex:cher:key:76} {agv:y\H{i}=hé:hn jijiwó:ni:h\H{v}, hlá yagwá:nhté di:gigo:lǐ:yê:dí digo:hwě:lô:dí=lé: \textbf{sgwu}, hla} \\
\glll akvvyii\H{}ʔi=hééhnv ci-ci-woó(ʔ)ni-:h-vv\'{}ʔi hla y-akw-aanvht-h-ééʔi ti-aki-kooliiy-é(ʔ)t-i ti-k-oohweel-óʔt-i=léé skwu hla \\
v:\ref{chnp1}=\ref{chnp1} \ref{chirr2}-\ref{chpron9}-\ref{chbase12}-\ref{chasp21}-\ref{chmod22} 1 \ref{chirr2}-\ref{chpron9}-\ref{chbase12}-\ref{chasp21}-\ref{chmod22} \ref{chdist5}-\ref{chpron9}-\ref{chbase12}-\ref{chasp21}-\ref{chmod22} \ref{chdist5}-\ref{chpron9}-\ref{chbase12}-\ref{chasp21}-\ref{chmod22}=\ref{chclit23} \ref{chnp24} \ref{chnp24}\\
first=because \Rel{}-\First\Sg.\Aarg{}-speak-\Impf{}-\Asr/\Sh{} {} \Irr{}-\First\Sg.\Barg{}-know-\Stat{}-\Rep{} \First\Sg.\Barg{}-read-\Inf{}-\Nom{} \Third\Sg.\Aarg{}-write-\Inf{}-\Nom{}=or also \Neg{}\\
\glt `When I first talked, I didn't know how to read or to write.' (EJ2012)
\z 


In the following example, the translocative prefix in position 3 has to be repeated so that each verb conveys the translocative meaning (`away'); if the second occurrence of the translocative is omitted, the second verb no longer has the `away' meaning:

\ea\label{ex:cher:key:77} {kò:sd wu:dánv:liyeʔé: wu:nó:hi:l\`{v}:sé:} \\
\gllll khòòstu w-uu-ata-nvvliy-eʔ-ééʔi w-uu-noohiil-vv(ʔ)s-ééʔi\\
v:\ref{chnp1} \ref{chtl3}-\ref{chpron9}-\ref{chmid10}-\ref{chbase12}-\ref{chasp21}-\ref{chmod22} \ref{chtl3}-\ref{chpron9}-\ref{chbase12}-\ref{chasp21}-\ref{chmod22}\\
n:\ref{chnbase7} - -\\ 
dust \Trnsl{}-\Third\Sg.\Barg{}-\Refl{}-rub.on-\Prf{}-\Rep{} \Trnsl{}-\Third\Sg.\Barg{}-fly-\Prf{}-\Rep{}\\
\glt `She put dust on her and she flew.' (CNRS)
\z 

Derivational suffixes such as the ambulative in position 18 cannot be elided either and need to be repeated so that each verb conveys the ambulative meaning (`here and there'):

\ea\label{ex:cher:key:78} {aksu:hní:dà:sdí no:lé agino:halǐ:dâ:sdí agil\H{v}:kwdi gè:sv́} \\
\glll akw-asuu-hn-iit-a(ʔ)st-i nooléé aki-noohal-iit-á(ʔ)st-i aki-lvvkwoht-i kèès-vv́ʔi\\
v:\ref{chpron9}-\ref{chbase12}-\ref{chasp13}-\ref{chdat18}-\ref{chasp21}-\ref{chmod22} 1 \ref{chpron9}-\ref{chbase12}-\ref{chdat18}-\ref{chasp21}-\ref{chmod22} \ref{chpron9}-\ref{chbase12}-\ref{chmod22} \ref{chbase12}-\ref{chmod22}\\
\First\Sg.\Barg{}-fish-\Prf{}-\Amb{}-\Inf{}-\Nom{} and \First\Sg.\Barg{}-hunt-\Amb{}-\Inf{}-\Nom{} \First\Sg.\Barg{}-like/\Sh{} \Cop{}-\Asr{}\\
\glt `I liked to fish and hunt.' (CNRS)
\z 

According to \citet[30]{Tallman2020}, the \textsc{maximal} \textsc{repeated} \textsc{subspan} is “the subspan of positions whose elements can occur in each of the coordinated constituents without reference to whether some of these elements can be elided or interpreted via widescope of one element over the repeated subspans”. In Oklahoma Cherokee, this corresponds to the entire planar structure (positions 1--24). The following example shows that elements from position 1 to position 22 can occur in each of the coordinated constituents

\newpage
\ea\label{ex:cher:key:79} {ach\H{u}:ja gawó:niha agě:hyá=hno dě:káno:gíʔa} \\
\gllll {a-chu\H{u}ca ka-woó(ʔ)n-ih-a a-keéhya=hno tee-ka-hnook-íʔ-a}\\
v:\ref{chnp1} \ref{chpron9}-\ref{chbase12}-\ref{chasp21}-\ref{chmod22} 1 \ref{chdist5}-\ref{chpron9}-\ref{chbase12}-\ref{chasp21}-\ref{chmod22}\\
n:\ref{chnbase7} - \ref{chnbase7}=\ref{chncl10} -\\ 
\Third\Sg.\Aarg{}-boy \Third\Sg.\Aarg{}-speak-\Prs{}-\Ind{} \Third\Sg.\Aarg{}-girl=and \Dist-\Third\Sg.\Aarg{}-sing-\Prs{}-\Ind{}\\
\glt `A boy is speaking and a woman is singing.' (\citealt{PulteFeeling1975}: 343)
\z 

\subsection{Nominalization (2--20; 1--21)}
\label{bkm:Ref87350422}
Nominalization can be considered a type of subspan repetition. When Cherokee verbs are nominalized, all the elements between slots 1 and 21 can be inherited, including an NP patient \textit{aciíla} `fire' as in (\ref{bkm:Ref72237101}) or a pronominal agent as in (\ref{bkm:Ref101421871}). This then is the maximal span of nominalization. Positions after 22 are excluded since all the nominalized forms have the modal suffix -\textit{i} in position 22.\footnote{An aspectual suffix Position 21 can also be inherited in the nominalized form when it is the imperfective suffix.} 

\ea\label{bkm:Ref72237101}ajǐ:lá g\H{o}:tlvhdi \\
\gllll aciíla k-oohtlhvv-ht-i\\
v:\ref{chnp1} \ref{chpron9}-\ref{chbase12}-\ref{chasp21}-\ref{chmod22}\\
n:\ref{chnbase7} - \\ 
fire \Third\Sg.\Aarg{}-make-\Inf{}-\Nom/\Sh{}\\
\glt `match.' (EJ2011)
\z 


Within the span of positions 1 –21 , the subspan between positions 2 and 20 cannot be elided, thus constituting the minimal subspan. Thus, in (\ref{bkm:Ref101421871}), the 3\textsc{sg} pronominal agent \textit{k(a)}- (position 9) in the infinitive forms of the first two verbs (`speak' and `write') is coreferential with the 3\textsc{sg} pronominal agent (here with the allomorph zero) of the verb `get ready', but they cannot be elided.

\ea\label{bkm:Ref101421871}gawò:n\H{i}:hisd digo:hwě:lô:dí yadv:n\H{v}:wstan \\
\glll ka-woo(ʔ)ni-:hist-(i) ti-k-oohweel-óʔt-i y-Ø-atvvnvv(ʔ)wist-ahn-(a)\\
v:\ref{chpron9}-\ref{chbase12}-\ref{chasp21}-\ref{chmod22} \ref{chdist5}-\ref{chpron9}-\ref{chbase12}-\ref{chasp21}-\ref{chmod22} \ref{chirr2}-\ref{chpron9}-\ref{chbase12}-\ref{chasp21}-\ref{chmod22}\\
\Third\Sg.\Aarg{}-speak-\Inf{}-\Nom/\Sh{} \Dist-\Third\Sg.\Aarg{}-write-\Inf{}-\Nom/\Sh{} \Irr{}-\Third\Sg.\Aarg{}-get.ready-\Prf{}-\Ind/\Sh{}\\
\glt `when you get ready to write your language.' (CNRS)
\z 

Likewise, (\ref{bkm:Ref101359093}) shows that the distributive prefix in position 5 cannot be elided even though it occurs in the matrix verb.

\ea\label{bkm:Ref101359093}de:jád\H{e}:hlgwaʔ dijago:lǐ:yê:dí dijo:hwě:lô:dí \\
\glll {tee-c-ateehlohkw-aʔ-a ti-ca-kooliiy-éʔt-i ti-c-oohweel-óʔt-i}\\
v:\ref{chdist5}-\ref{chpron9}-\ref{chbase12}-\ref{chasp21}-\ref{chmod22} \ref{chdist5}-\ref{chpron9}-\ref{chbase12}-\ref{chasp21}-\ref{chmod22} \ref{chdist5}-\ref{chpron9}-\ref{chbase12}-\ref{chasp21}-\ref{chmod22}\\
\Dist-\Second\Sg.\Barg{}-learn-\Prf{}-\Ind/\Sh{} \Dist-\Second\Sg-\Barg{}-read-\Inf{}-\Nom/\Sh{} \Dist-\Second\Sg-\Barg{}-write-\Inf{}-\Nom/\Sh{}\\
\glt `when you learn to read and write.' (CNRS)
\z 

\section{Conclusion}
\label{bkm:Ref87347176}
In this chapter, I have shown how 8 phonological and 13 morphosyntactic constituency diagnostics are applied to the verbal planar structure with 24 positions to see whether any convergence of diagnostics is observed, and if so, in which layers. \figref{fig:pooled} provides an overview of the results of the constituency variables applied to Cherokee in terms of layers.\footnote{The figures were created by Sandra Auderset. Four tests that were classified as morphosyntactic are labelled as ``indeterminate''.}  The numbers refer to the position numbers in the verbal planar structure laid out in \tabref{tab:cher:planv}. From this display we can see that a span from position 2 to position 22  (layer 13) and the other from position 2 to position 23 (layer 14) show high convergences.

\begin{figure}
    \centering
    \includegraphics[width=\textwidth]{figures/cherokee_pooled_plot.png}
    \caption{Constituency domains organized by converging layers in Cherokee}
    \label{fig:pooled}
\end{figure}


\figref{fig:boundary} displays the results in terms of edges, where the y-axis refers to the number of times a constituency result hits a specific edge, and the x-axis refers to position in the planar structure. The green columns is for the left edge and the purple columns are for the right edge. As we can observe, position 2 at the left edge and position 22 at the right edge are where more constituency results have an edge.

\begin{figure}
    \centering
    \includegraphics[width=.9\textwidth]{figures/cherokee_boundaries.png}
    \caption{Constituency domain edges organized by count and type in Oklahoma Cherokee}
    \label{fig:boundary}
\end{figure}

The following observations can be made from this result. First, as can be seen, convergences are not found except for layer 13 (positions 2--22 ), where three diagnostics converge, and layer 14 (positions \ref{chirr2}-\ref{chclit23} ), where five diagnostics converge, which are the best `wordhood' candidates in Oklahoma Cherokee. That there are convergences shows that there is more structure than just word vs. sentence. What is noteworthy about this latter constituent (layer 14) – which could be the principal candidate for a `word' in Oklahoma Cherokee – is the size of this domain: this domain contains up to 22 morpheme slots. A comparison with other languages in the volume confirms that the size of this domain is indeed significantly larger than average; the only language with a comparably large domain of convergences is C'upik.

Partly due to the large size of the wordhood candidate, and since this candidate can contain an incorporated noun in Northern Iroquoian,\footnote{There is not much consensus on whether compounds should be treated morphologically or syntactically as there is more of a cline in this domain (cf. \citealt{Tallman2021}).} some recent studies on Iroquoian languages propose that an Iroquoian word corresponds to the phonological phrase \citep{Dyck2009} or that the word-internal structure is a phrase rather than a head (\citealt{BarrieMatthieu2016}). The methodology employed in this chapter allows us to abstract away from arbitrary labels such as `phrase' or `word', but in light of such analyses, one might argue that the layer 13 (positions 2--22) is the `word' while the layer 14 (position \ref{chirr2}-\ref{chclit23}) is the `phrase' in Oklahoma Cherokee, the two sole layers with any convergences, assuming that any number of convergences automatically provide word-hood candidates. However, as mentioned above, the only difference between these two layers is the incorporation of the enclitics; if anything, the group that consists of a word + enclitics should correspond to the clitic group (\citealt{Nespor1986}: Ch. 5) or the prosodic word group \citep{Vigario2010}, rather than a phrase. Neither layer 12 nor layer 14 have any characteristics that we would expect of a phrase:\footnote{Unlike Northern Iroquoian languages, Cherokee does not have productive noun incorporation.}  “a set of the form \{γ, \{α,β\}\}, where α and β are syntactic objects, be they lexical items (heads) or other phrases” (\citealt{BarrieMatthieu2010}: 10). The result obtained in this chapter indicates that the Cherokee `word' is a `word' after all, assuming that convergence is the correct criterion for wordhood \citep{Matthews2002}\footnote{Adam Tallman suggests that an alternative is to consider that words are non-extractable or non-coordinable elements following \cite{Bruening2018}.}, and not a `phrase', despite its large size. 

Secondly, looking at the phonological and morphosyntactic diagnostics separately, the best phonological wordhood candidate is the span from position 2 to 23, with the convergence of three phonological diagnostics (\textsc{Final} \textsc{Apocope;} \textsc{Syllabification;} \textsc{h-metathesis/vowel} \textsc{deletion}), while the best morphosyntactic wordhood candidate is the span from positions 2 to 22, with the convergence of three morphosyntactic diagnostics (\textsc{non-interruptibility,} \textsc{fixed} \textsc{order} \textsc{(maximal),} \textsc{ciscategorial} \textsc{selection} \textsc{(maximal)}). This is shown in \figref{fig:cherms} and \figref{fig:cherphon} below.\footnote{Note that for the purposes of this chapter I assume that indeterminate domains are tests for morphosyntactic wordhood.} The only difference between them is that the enclitics in position 23 are incorporated in the phonological wordhood candidate while they are not in the morphosyntactic wordhood candidate. This more or less supports the `word bisection thesis' \citep[7]{Dixon2009}, which states that `phonological word' and `grammatical word' can be recognized.

\begin{figure}
    \centering
    \includegraphics[width=\textwidth]{figures/cherokee_ms_mixed_plot.png}
    \caption{Morphosyntactic and indeterminate domains organized in terms of converging layers}
    \label{fig:cherms}
\end{figure}
  
\begin{figure}
    \centering
    \includegraphics[width=\textwidth]{figures/cherokee_phon_plot.png}
    \caption{Phonological domains organized in terms of converging layers}
    \label{fig:cherphon}
\end{figure}

As \citet{Bickel2017} claim on constituency in polysynthetic languages, more than one constituent needs to be posited and convergence is uncommon except for a couple of layers. On the other hand, unlike what they report for other polysynthetic languages, the method employed here shows that there is a strong wordhood candidate language-internally; this also reflects the general intuitions about wordhood among speakers and linguists working on Cherokee and Iroquoian languages. Future research might find that convergences such as those found in Cherokee (see \citetv{chapters/02-Cupik} on C'upik and \citetv{chapters/08-Chatino} on Zenzontepec Chatino) are not so uncommon even when a larger sample of candidate diagnostics are considered. If this ends up being the case, it would demand an explanation, and such an explanation is not obviously available in current ``emergentist'' approaches.\footnote{I thank Adam Tallman for this idea.}

In sum, the only peculiarity of Cherokee wordhood is its size, but otherwise it is `well behaved', in that the convergences are found only in two layers, each of which correspond to morphosyntactic and phonological words, respectively. 


\section*{Acknowledgements}
I would first and foremost thank the speakers of Oklahoma Cherokee, especially: Durbin Feeling, Ed Jumper, Junior Scraper, Ida Scraper, DJ McCarter, Anna Sixkiller, David Crawler, and Dennis Sixkiller. I'm also indebted to Adam Tallman, Sandra Auderset, Karin Michelson, Marianne Mithun, Naonori Nagaya and Shun Nakamoto for their insights. The previous versions of this chapter were presented at the International Workshop Constituency, Wordhood, and the Morphol\-ogy-Syntax Distinction: Description and Typology (online) and a meeting group organized by Naonori Nagaya, both in 2021. I thank the participants in both of these events. This project has been funded by the University at Buffalo, Jacobs Fund (University of Washington) and Phillips Grant (American Philosophical Society) and the Instituto de Investigaciones Filológicas, UNAM.


\printglossary

\sloppy\printbibliography[heading=subbibliography,notkeyword=this]
\end{document} 
