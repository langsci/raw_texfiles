\documentclass[output=paper]{langscibook}
\ChapterDOI{10.5281/zenodo.13208570}
\author{Cristian R. Juárez\affiliation{The University of Texas at Austin}}
\title{Constituency in Northern Chaco Mocoví (Guaycuruan, Argentina)}
\abstract{This chapter explores 22 different constituency tests applied to the verbal planar structure in Northern Chaco Mocoví, the northern-most Mocoví speech community in Argentina. It is shown that the verbal subspans selected by different tests display divergences and convergences within and across phonological and morphosyntactic domains. Regardless of the divergences, there are recurrent convergences, identifying  minimal and maximal subspans. I argue then that those recurrent subspans can be interpreted as the best exemplars of a word-like constituent in this language.}

\IfFileExists{../localcommands.tex}{%hack to check whether this is being compiled as part of a collection or standalone
   % add all extra packages you need to load to this file

\usepackage{tabularx,multicol}
\usepackage{url}
\urlstyle{same}

\usepackage{listings}
\lstset{basicstyle=\ttfamily,tabsize=2,breaklines=true}

\usepackage{langsci-basic}
\usepackage{langsci-optional}
\usepackage{langsci-lgr}
\usepackage{langsci-osl}
% \usepackage{./langsci/styles/langsci-lgr}
% \usepackage{./langsci/styles/langsci-osl}
% \usepackage{langsci-gb4e}

\usepackage{tikz}
\usetikzlibrary{patterns,calc}
\pgfdeclarepatternformonly{south east lines}{\pgfqpoint{-0pt}{-0pt}}{\pgfqpoint{3pt}{3pt}}{\pgfqpoint{3pt}{3pt}}{
    \pgfsetlinewidth{0.6pt}
    \pgfpathmoveto{\pgfqpoint{0pt}{3pt}}
    \pgfpathlineto{\pgfqpoint{3pt}{0pt}}
    \pgfpathmoveto{\pgfqpoint{.2pt}{-.2pt}}
    \pgfpathlineto{\pgfqpoint{-.2pt}{.2pt}}
    \pgfpathmoveto{\pgfqpoint{3.2pt}{2.8pt}}
    \pgfpathlineto{\pgfqpoint{2.8pt}{3.2pt}}
    \pgfusepath{stroke}}
    
\usepackage{stmaryrd}
\usepackage{wasysym}
\usepackage{multirow}
\usepackage{caption}
\usepackage{subcaption}
\usepackage{mathrsfs}
\usepackage{qtree}

\usepackage{linguex}


   %pminos do not split footnotes
% \interfootnotelinepenalty=10000 %Footnote in Laporte chapters has to be split SN


%\DeclareIndexNameFormat{default}{%
%\nameparts{#1}%
%\usebibmacro{index:name}%
%{\index[names]}%
%{\namepartfamily}%
%{\namepartgiveni}%
% {}% L1
% {}% L2
%{\namepartprefix}% generates spurious space L3
%{\namepartsuffix}% generates spurious space L4
%}

%  {\DeclareIndexNameFormat{default}{%
%     \usebibmacro{index:name}{\index[names]}{#1}{#3}{#5}{#7}}}

%\DeclareIndexNameFormat{default}{%
%  \usebibmacro{index:name}{\sindex[nom]}{#1}{#3}{#5}{#7}}

%\DeclareIndexNameFormat{default}{%
%  \usebibmacro{index:name}{\sindex[person]}{#1}{#3}{#5}{#7}}
%\DeclareIndexNameFormat{default}{%
%\nameparts{#1} \usebibmacro{index:name}{\sindex[person]]}{\namepartfamily}{‌​\namepartgiven}{\nam‌​epartprefix}{\namepa‌​rtsuffix}}

%\newcommand{\smiley}{:)}

%\renewbibmacro*{index:name}[5]{%
%\usebibmacro{index:entry}{#1}%
%{\iffieldundef{usera}{}{\thefield{usera}\actualoperator}\mkbibindexname{#2}{#3}{#4}{#5}}}

% \newcommand{\noop}[1]{}

%remove for final
%\overfullrule=1mm

\newcommand{\tobi}[2]}}
\renewcommand{\S}[1]{\tobi{#1}{\textsc{*}}}

% this volume references
% puts: [this volume]
% already defined: \citetv
%\newcommand{\citepv}[1]{(\citeauthor{#1} \citeyear*{#1} [this volume])}
\newcommand{\citealtv}[1]{\citeauthor{#1} \citeyear*{#1} [this volume]}

%parentheses around example number
\newcommand{\pref}[1]{(\ref{#1})}

% in-text examples

\newcommand{\lnex}[1]{\textit{#1}} %target lang word
\newcommand{\lnlit}[1]{(lit.: `#1')} %literal reading
\newcommand{\lnlat}[1]{(#1)} % latinization
\newcommand{\lntrans}[1]{`#1'} %translation
\newcommand{\lnexl}[2]%
{\lnex{#1}{} \lnlat{#2}} % ex with latinization
\newcommand{\lnexlat}[3]{\lnex{#1}{} \lnlat{#2}{} \lntrans{#3}} % ex with latinization and tranl.

%ch01
\newcommand{\co}[1]{\mbox{\textbf{#1}}}

%ch09

\newcommand{\cyrbulg}[1]{\begin{otherlanguage*}{bulgarian}#1\end{otherlanguage*}}


%ch10
\newcommand{\nlp}{{\small NLP}}
\newcommand{\mwe}{{\small MWE}}
\newcommand{\rae}{{\small RAE}}
\newcommand{\lvc}{{\small LVC}}
\newcommand{\pos}{{\small P}o{\small S}}
%\newcommand{\todo}[1]{ \textcolor{red}{#1} }

%\renewcommand{\labelenumi}{\theenumi}
%\ainamefmt{{vv}{ll}{, ff}{, jj}} % fullname

\newcommand{\biberror}[1]{{\color{red}#1}}

\newcommand{\osenovaitem}{--~}
   %% hyphenation points for line breaks
%% Normally, automatic hyphenation in LaTeX is very good
%% If a word is mis-hyphenated, add it to this file
%%
%% add information to TeX file before \begin{document} with:
%% %% hyphenation points for line breaks
%% Normally, automatic hyphenation in LaTeX is very good
%% If a word is mis-hyphenated, add it to this file
%%
%% add information to TeX file before \begin{document} with:
%% %% hyphenation points for line breaks
%% Normally, automatic hyphenation in LaTeX is very good
%% If a word is mis-hyphenated, add it to this file
%%
%% add information to TeX file before \begin{document} with:
%% \include{localhyphenation}
\hyphenation{
    Beck-man
    Ngu-yen
    back-chan-nel
    back-chan-nels
    mo-not-o-nous
    ste-reo-typ-i-cal
}

\hyphenation{
    Beck-man
    Ngu-yen
    back-chan-nel
    back-chan-nels
    mo-not-o-nous
    ste-reo-typ-i-cal
}

\hyphenation{
    Beck-man
    Ngu-yen
    back-chan-nel
    back-chan-nels
    mo-not-o-nous
    ste-reo-typ-i-cal
}

    \bibliography{../localbibliography}
    \togglepaper[16]
}{}


\begin{document}
\maketitle

\section{Introduction}
The empirical definition of a cross-linguistically valid notion of ``word'' has been shown to be a theoretically and descriptively challenging task. For Mocoví, a Guaycuruan language spoken in northeastern Argentina, the definition of ``word'' has not received a specific study to show the necessary and sufficient criteria that identify such a category \citep[cf.][]{Carrio2009,Gualdieri1998,Grondona1998}.

Following the methodology delineated in this edited volume, this chapter explores ten different constituency tests applied to the verbal planar construction, i.e., ``an entire sentence headed by a verb where no distinction between word-internal and sentence-level structure is presupposed'' \citep[][52--53]{Tallman2020}, in Northern Chaco Mocoví. Both phonological and morphosyntactic domains are tested to determine the extent to which a verbal word constituent can be identified by the planar structure position/s that each test targets. 

Results presented in this study suggest that a word constituent in the verbal domain is not categorical. Convergences and divergences are observed in both phonological and morphosyntactic domains. When comparing both domains, however, convergences that identify minimal and maximal subspans of the planar structure emerge, which suggest that those constituents might represent the best exemplars of non-arbitrarily defined verbal words.   

In \S \ref{sec:Planarstructure} of this chapter, I describe the Mocoví planar structure in which all elements of a clause are organized in a linear fashion. Then, phonological and morphosyntactic tests are applied to the planar structure in \S \ref{sec:phonologicaltest} and \S\ref{sec:morphsyntactictests}, respectively. Major findings on constituency and word detection are presented in \S \ref{sec:finalremarks}. 


\subsection{Mocoví: geographic location, linguistic family and data}
Mocoví is spoken in northeastern Argentina and belongs to the Guaycuruan linguistic family, along with Toba, Pilagá, Kadiweo and Abipón (extinct). While current estimates of the Mocoví population are not up to date, the last Argentine national census in 2010 recognized more than 16,000 Mocoví members. 

So far, it has been argued that there are at least two distinct variants of Mocoví within the Mocoví dialectal chain in Argentina. One is spoken in the province of Chaco and has been initially documented by \citet{Buckwalter1995} and described by \citet{Grondona1998} and \citet{Gualdieri1998}. These three studies were primarily based on data from Colonia El Pastoril in Southern Chaco. The other variant is spoken in the province of Santa Fe and is nowadays mainly studied by Cintia Carrió and her students \citep[see, for instance,][]{Carrio2009,Carrio2011, Carrio2015a, Carrio2019, Rabasedas2017}, covering different morphosyntactic and lexical domains of Mocoví grammar \citep[but see also][]{Gualdieri2006}. To the best of my knowledge, none of these scholars working on Mocoví has produced specific publications dealing with wordhood issues. 

In this chapter, the Mocoví data come from the Northern Chaco community, a cluster of socially connected families located specifically in the indigenous territory known as Colonia Aborigen, a multilingual region that has served as the home for Mocoví and Toba communities for more than hundred years  (see \citealt[][vol II]{Hermitte1995a}, \citealt{Salamanca2008}). Map \ref{map:mocovilocalization} zooms in on the sites in Colonia Aborigen where I carried out fieldwork, and displays the current locations of Mocoví communities in Santa Fe and Chaco.

\begin{figure}
\captionsetup{name=Map}
    \centering
   \includegraphics[width=\textwidth]{figures/mapachaco2.png}
    \caption{Localization of Northern Chaco Mocoví in Colonia Aborigen in the context of the Mocoví communities in Argentina}
    \label{map:mocovilocalization}
\end{figure}

 Northern Chaco Mocoví has not received much attention from Mocoví studies until recent years \citep[e.g.,][]{Juarez2013}. Since then a series of studies have been produced, advancing the current knowledge on the northern-most Mocoví community of the dialectal continuum. The corpus for this study combines naturally-occurring and elicited data collected via original fieldwork from 2011 to 2021. Part of this corpus is available at the fully open-access Mocoví collection at ELAR \citep{Juarez2019}.\footnote{To access the collection, visit \url{https://elar.soas.ac.uk/Collection/MPI1314056}.} Each individual example comes with an identifier, indicating its source. Examples taken from the Mocoví corpus are accompanied by short abbreviations of the language and community names where they were recorded in addition to the year, month, day and the exact start time of an utterance in the recording session. Natural discourse examples are differentiated from elicited ones in that the former are preceded and followed by three dots in the first and fourth line of transcription.  

%\begin{table}
%\caption{Frequencies of word classes}                                            
%\label{tab:1:frequencies}
% \begin{tabular}{l rrrr}
%  \lsptoprule
 %           & nouns & verbs  & adjectives & adverbs\\
 % \midrule
 % absolute  &   12  &    34  &    23      & 13\\
 % relative  &   3.1 &   8.9  &    5.7     & 3.2\\
 % \lspbottomrule
 %\end{tabular}
%\end{table}

\section{Verbal planar structure}
\label{sec:Planarstructure}
This chapter explores multiple individual tests that allow us to empirically assess the extent to which a verbal constituent exists in Northern Chaco Mocoví. To do so, a methodological implementation to capture verbal constituency is in order. Following recent work by \citet{Tallman2020,Tallman2021}, I propose the following planar structure in \autoref{tab:verbplanar}. A planar structure captures all the elements that are part of a specific domain; in this chapter I only focus on the verbal domain. 

All elements in a planar structure are part of a flat structure that does not presuppose any type of constituency structure or morphological organization, i.e., class/templatic system or layered system  \citep[for a recent overview of these morphological systems, see][]{Mithun2016}. Within a planar structure, elements occupy individual positions arranged in two different types: slots and zones. The application of tests described in Sections \ref{sec:phonologicaltest}-\ref{sec:morphsyntactictests} will identify different spans of positions, whose alignment might single out what the best candidate for a word constituent is.\footnote{The labels S, A, P in the planar structure are based on \citet {Comrie1981}, and T and R on \citet{Malchukov2010}.}
  

\begin{table}
    \caption{Verbal planar structure of Mocoví}
    \label{tab:verbplanar}
\begin{tabular}{Sclll} \lsptoprule
	\multicolumn{1}{c}{Pos.}    & Type  & Elements  & Forms \\ \midrule
	\label{linkers} & slot &  clause linkers & \textit{nakoʔ}, \textit{maʔ}, \textit{kan}, \textit{qaʔ} \\
\label{sapadv}                & zone          & S/A, adverbs                    &  \\

\label{p}           & slot          & 1\textsuperscript{st}, 2\textsuperscript{nd} P             & \textit{ajim}, \textit{qomiʔ}, \textit{qamiʔ}, \textit{qamiɾi} \\

\label{moc:neg}                & slot          & negation           & \textit{sV-}, \textit{sqa-}                  \\

\label{sdef}               & slot         & subject defocusing  & \textit{qa-} $\sim$ \textit{qo-}             \\


\label{sap}                & slot          & bound person forms  & Set I,  Set II and Set III \\

\textbf{\label{moc:core}}       & slot          & \textbf{verb core}       &    \\ 

\label{vmI}               & slot          & valence modifier I     & \textit{-ɢat}, \textit{-ɢan}, \textit{-n} \\

\label{causee}               & slot      & causee    & \textit{-it}     \\

\label{vmII}              & slot          & valence modifier II    & \textit{-ɢan}     \\ 

\label{2and1pl}              & slot          &  1\textsuperscript{st}\textsc{pl}, 2\textsuperscript{nd} S/A, 3\textsuperscript{rd}\textsc{`pl'} P    & \textit{-ɢ},\textit{-iɾ},\textit{-i} $\sim$ \textit{-iː}, \textit{-eʔ}        \\

\label{moc:asp}               & slot          & aspect             & \textit{-ta}, \textit{-tak}, \textit{-sa}, \textit{-sak}, \textit{-teɡ}        \\
\label{des}               & slot          &  desiderative & \textit{-ake} \\ 
\label{rr}             & slot          & reflexive, reciprocal    & \textit{-taʔ}, \textit{-ltaʔ}          \\
 \multirow{3}{*}{\label{trlocdir}}          & {zone}       &  transitivizer & \textit{-a} $\sim$ \textit{-aʔ} $\sim$ \textit{-aʔa}\\
\multicolumn{2}{l}{}& {directionals} & \textit{-ʃiɡem}, \textit{-ɲi}, \textit{-wek}, \textit{-o}\\
\multicolumn{2}{l}{}& {locatives} & \textit{-ge}, \textit{-gi}, \textit{-lek}, \textit{-ɡit}  \\
\label{pllo}             & slot          & non-subject plural    & \textit{-lo}    \\
\label{moc:dim}             & slot          & diminutives    & \textit{-okiʔ}, \textit{-oʎiʔ}    \\
\label{encl}             & slot          & temporal/ evid?    & \textit{=oʔ}     \\
\label{moc:parg}             & slot          & 3\textsuperscript{rd} P or A  &        \\
\label{moc:obl}             & slot          &  oblique nominals     & \textit{ke}-nouns      \\

\lspbottomrule
\end{tabular}
\end{table}
%\end{landscape}

The relative order of elements as well as their occurrence in the same position of the structure are based on two principles: mutual exclusivity and co-occurrence. Thus, individual elements will occupy different positions if they co-occur. For example, in the positions preceding the verb core, a bound person form, e.g., \textit{i-} `\textsc{\Third.\II}' can co-occur with other preceding elements such as the \textsc{subject defocusing} marker \textit{qa-} and the \textsc{negative} prefix \textit{sa-} in the following order, i.e., \textit{sa-qa-i-aʔden} `they don't know it'. In contrast, different elements will occupy the same position if they exclude one another. In positions following the verb core, for example, the suffixes \textit{-lek} `\LocOne' and \textit{-ge} `\LocTwo' exclude each other and the sequences *\textit{-lek-ge} or *\textit{-ge-lek} do no exist in the language.

%There is a general (implicit) assumption to interpret that all the elements falling within the negative marker and the enclitic are a single word (this is, generally speaking, true for the other Guaycuruan languages), but was never tested empirically. This work represents then a contribution to a better characterization of the verbal domain in Mocoví and presents further implications for understanding the verbal structure of other Guaycuruan languages as well.  

%\section{Nominal structure}

%\subsection{Determiners beyond the nominal domain}
%In natural discourse, the occurrence of nominal determiners is attested beyond the nominal domain. It is rather common finding them in the context of verbal clauses where the position determiners occupy resembles the structure of the nominal structure.

%\ea

%\glll ...nagi \textbf{na} s-we-ta-aʔ qajka qajka ka i-oɢoɾ-ik...\\
%        \ref{sapadv} \ref{sapadv} \ref{sapadv}-\ref{moc:core}-\ref{moc:asp}-\ref{loctr}  \\
 %   {}{ahora} {\DetThree} {\Fsg-\Exist-\Dur-\Trvz}  {\Exist.\Neg}  {\Exist.\Neg}  {\DetOne} {\First\Poss}-be.poor-{\Sg.\M}{} \\
%\glt `... now I'm in a place without suffering...'
%\glt `... y ahora estoy en un lugar sin sufrimiento.. kljlkjflkajsfdljasldkfjalskjdflasjdflkasdf'\\
%\glt \hfill{(mocCA190924\_1: 00:06:13)}
%\z
%\il{Mocoví}

%Nested structure for examples:
%\ea
 %\ea \label{ex:vaverb}
%	\gll 	\\
%					\\
%	\glt `' \hfill SMD-0020-Huauzontle
 %   \ex \label{ex:vaadv}
    
%	\gll  \\
%			\\
%	\glt `' \hfill SMD-0020-Huauzontle
 %   \ex \label{ex:vanoun}
    
%	\gll  	\\
%		\\
%	\glt `' \hfill SMD-0009-Jardin
 %\z
%\z

\section{Phonological diagnostics} 
\label{sec:phonologicaltest}

This section focuses on phonological diagnostics applied to the Mocoví planar structure. The diagnostics are stress (\S \ref{sec:stress}), palatalization (\S\ref{sec:palatalization}), epenthesis (\S \ref{sec:epenthesis}) and vowel harmony (\S\ref{sec:vowel harmony}). The description of these segmental and suprasegmental properties of Northern Chaco Mocoví  builds on previous works of Southern Chaco Mocoví phonology (see \citealt[chap.2]{Gualdieri1998} and \citealt[21--42]{Grondona1998}). Examples are represented following a standard practice of an input-output schema, in which the output is produced by rules in the course of a phonological derivation \citep{Hayes, Gussenhoven2011, Kenstowicz1979}. Thus, the first line represents the output, whereas the second line provides the underlying representation of sounds.   

\subsection{Stress: spans \ref{sap}-\ref{moc:core} and \ref{moc:neg}-\ref{encl}}\label{sec:stress}
Mocoví has been described as a stress language at the lexical level, although the acoustics of stress remain to be studied in detail. I limit myself here to reporting on the position of stress and the spans of elements that stress identifies.

The position of stress is fixed and thus predictable. It falls on the last syllable of a single element or a string of elements of the planar structure. Stress can identify minimal and maximal spans of positions. The minimal span includes positions \ref{sap}-\ref{moc:core}, as in (\ref{ex:stressmin}).  

\ea \label{ex:stressmin}
 [so ja:ˈleɣ jaˈβik]\\
\glll so i-jaːle-ɡ i-awik\\
\ref{sapadv} {} \ref{sap}-\ref{moc:core}\\
{\DetTwo} {\First\Sg.\Poss}-descendent-{\M} {\Third.\I}-get.burn\\
\glt `My son is burnt.' \hfill(mocCA191111: 00:02:16)
\z

The maximal span, on the other hand, includes positions \ref{moc:neg}-\ref{encl}, as in (\ref{ex:stressencl}). The stress regularly moves to the right edge of a constituent regardless of the number of elements added to the verb structure. 
 

\ea \label{ex:stressencl}
 [soja:ˈleɣ jaβiˈkoʔ]\\
\glll so i-jaːle-ɡ i-awik=oʔ\\
\ref{sapadv} {} \ref{sap}-\ref{moc:core}-\ref{encl}\\
{\DetTwo} {\First\Sg.\Poss}-descendent-{\M} {\Third.\I}-get.burn={\Evid/\Tprl}?\\
\glt `My son got burnt.' \hfill(mocCA191111: 00:03:46)
\z



 %Compare, for instance, the position of stress on the verbs [nojaɲiːˈsak] (\ref{ex:stressb}) and [nojantaʔˈpe] (\ref{ex:stressc}), as well as, on the independent pronouns   [qaˈmiʔ] and [qamiˈɾi] in (\ref{ex:stressa})-(\ref{ex:stressb}).

%\ea
%\ea \label{ex:stressa}
%[qaˈmiʔ nojaɲiʔˈsak]\\
%	\glll qamiɾ n-ojan-iɾ-sak\\
%            \ref{sapadv} \ref{sap}-\ref{moc:core}-\ref{2and1pl}-\ref{moc:asp}\\
%			{\Second.\Pron} {\Second\Sg.\III}-cry-{\Second\Sg.\III}-{\Prog}\\
%	\glt `You are crying' or  `Are you crying?' \hfill (mocCA210717:01:25:14)
%	\ex \label{ex:stressb}
% [qamiˈɾi nojaɲiːˈtak]  \\
%       	\glll    qamiɾ-i n-ojan-iː-tak\\
%            \ref{sapadv} \ref{sap}-\ref{moc:core}-\ref{2and1pl}-\ref{moc:asp}\\
%			{\Second.\Pron}-{\Pl}  {\Second\Pl.\III}-cry-{\Second\Pl.\III}-{\Prog}\\
%	\glt `You (pl) are crying' or  `Are you (pl) crying?' \hfill (mocCA210717: 01:27:30)
%	\ex\label{ex:stressc}
%	 [sojaˈleʔ nojaɲtaʔˈpe]  \\
%      \glll so jale-ɾ n-ojan-taʔ-pe\\
%            \ref{sapadv} {}  \ref{sap}-\ref{moc:core}-\ref{2and1pl}-\ref{moc:asp}\\
%			{\DetTwo} man-{\Pl}  {\Second\Sg.\III}-cry-{\Pl}-{\Prog}\\
%	\glt `The two man are crying.' \hfill (mocCA210717: 01:28:55)
%\z
%\z 

 The occurrence and position of stress aligns with what speakers intuitively identify as a word. Thus, speakers consider this suprasegmental property as a strong clue to recognize a word in the language. Not only the span of positions \ref{moc:neg}-\ref{encl} and \ref{sap}-\ref{moc:core} are considered independent words, but also other elements outside these spans, such as adverbial elements in \ref{sapadv} and independent pronouns in \ref{p}, among others. However, since other individual elements or spans of positions that also carry main stress do not include the verbal core, they are not considered here.


\subsection{Palatalization: spans \ref{moc:core}-\ref{2and1pl} and \ref{sap}-\ref{trlocdir}} \label{sec:palatalization}

A pervasive phonological process in Mocoví is palatalization. Alveolar consonants such as /d, t, l, n, s/ become [dʒ, tʃ, ʎ, ɲ, ʃ] when followed by /i/. 
In (\ref{ex:palatizationt})-(\ref{ex:palatizationl}), palatalization occurs at morpheme boundaries. Alveolar consonants in the right edge of core element \ref{moc:core} are affected by the high front vowel [i] that is part of the suffix in position \ref{2and1pl}. However, palatalization not only affects those consonants at the right edge of the core element, but also alveolar consonants on its left edge. Observe the change [d] $\rightarrow$ [dʒ] in (\ref{ex:palatizationnb}).
The span where palatalization occurs thus runs from \ref{moc:core} to \ref{2and1pl}.

\ea\label{ex:palatizationt}
    /t/ $\rightarrow$ [tʃ] / \_ [i] \\
    \ea{} [qoˈpa\textbf{t}] \\
    \glll ∅-qopat  \\
        \ref{sap}-\ref{moc:core}\\
      {\Third.\I}-be.hungry\\
    \glt `She/He is hungry.'  \hfill (mocCA120713: 00:03:48)

    \newpage
    \ex{} [ɾoqopaˈ\textbf{tʃ}iʔ] \\
    \glll       ɾ-qopat-iɾ  \\
    \ref{sap}-\ref{moc:core}-\ref{2and1pl}\\
    {\Second\Sg.\I}-be.hungry-{\Second\Sg.\I}\\
    \glt `You are hungry.' or `Are you hungry?'  \hfill (mocCA120713: 00:03:38)
    \z
\z 

\ea\label{ex:palatizationn}
/n/ $\rightarrow$ [ɲ] / \_ [i] \\
\ea\label{ex:palatizationna}
[jaʔˈ\textbf{d}e\textbf{n}]\\
\glll    i-ʔden  \\
    \ref{sap}-\ref{moc:core}\\
      {\Third.\II}-know\\
\glt `She/He knows it.'  \hfill (mocCA120706: 01:24:28)
\ex\label{ex:palatizationnb}
 [\textbf{dʒ}iˈ\textbf{ɲ}iʔ] \\
\glll        ʔden-iɾ \\
    \ref{moc:core}-\ref{2and1pl}\\
      know-{\Second\Sg.\II}\\
\glt `You know it.'  \hfill (mocCA120713: 00:01:12)
\z
\z

\ea\label{ex:palatizationl}
    /l/ $\rightarrow$ [ʎ] / \_ [i] \\
    \ea{} [neˈsa\textbf{l}] \\
    \glll    n-sal  \\
    \ref{sap}-\ref{moc:core}\\
      {\Third.\I}-vomit\\
\glt `She/He vomits.'  \hfill (mocCA110711: 00:10:29)
\ex{}  [ɾesaˈ\textbf{ʎ}iʔ] \\
    \glll  ɾ-sal-iɾ  \\
    \ref{sap}-\ref{moc:core}-\ref{2and1pl}\\
      {\Second\Sg.\I}-vomit-{\Second\Sg.\I}\\
\glt `You vomit.' or `Did you vomit?' \hfill (mocCA110705: 02:18:14)      
\z
\z

The element in position \ref{sap} can also palatalize. Compare the articulation of the first-person bound form \textit{s-} before [a] and [i], as illustrated in (\ref{ex:palatprecore}). 

\ea\label{ex:palatprecore} 
    [\textbf{s}aβotaˈke \textbf{ʃ}ikʃimˈɡe aɾaqoˈpaq] \\
    \glll s-wo-ta-ake s-k-ʃimɡe a-ɾa qopaq\\
    \ref{sap}-\ref{moc:core}-\ref{moc:asp}-\ref{des} \ref{sap}-\ref{moc:core}-\ref{trlocdir} \ref{moc:parg}\\
    {\First.\II}-want-{\Dur}-{\Des} {\First.\II}-move-{\Dir:\Up} {\F}-{\DetFour} tree\\
    \glt `I want to climb the tree.' \hfill(mocCA120626: 00:15:24)
\z

However, the palatalization of alveolar consonants in position \ref{sap} is rare and seems to be subject to variation within and across speakers. Typically, morphological elements that contain an alveolar consonant in that position, i.e., certain bound person forms and the negative marking, do not palatalize.  For example, unlike (\ref{ex:palatprecore}), the first-person bound form \textit{s-} in (\ref{ex:nonpalatsa}) does not change its articulation in the context of [i]. The same holds for the negative prefix \textit{s-} in (\ref{ex:nonpalatsb}).\footnote{ Syllable structure constraints and other phonological processes directly interact with palatalization in (\ref{ex:nonpalats}), e.g., vowel harmony and vowel epenthesis. In both examples, the segment [i] is inserted to maintain the syllable structure CV. Additionally, in (\ref{ex:nonpalatsb}) that vowel harmonizes in height with the semiconsonant /j/, which underlyingly correspond to the bound person form \textit{i-}, but re-syllabifies as [j] to produce the structure CVC with this specific verb root.}     

\ea \label{ex:nonpalats}
    Left-edge non-patalization: \textit{s-} `\First.\II' and \textit{s-} `\Neg'
    \ea\label{ex:nonpalatsa}
    [\textbf{s}iˈkin ɾajaqaˈja]\\
    \glll s-kin ɾa i-aqaja\\
    \ref{sap}-\ref{moc:core} \ref{moc:parg} \\
    {\First.\II}-greet {\DetFour} {\First.\Poss}-brother\\
    \glt `I greeted my brother.' \hfill(mocCA160725: 00:37:46)
    \ex\label{ex:nonpalatsb} 
    [\textbf{s}ijaʔˈden] \\
    \glll s-i-ʔden \\
    \ref{moc:neg}-\ref{sap}-\ref{moc:core} \\
    {\Neg}-{\Third.\II}-know \\
    \glt `(They) don't know it.' \hfill(mocCA180807: 00:04:04)
    \z
\z

Similarly, the third-person bound form \textit{n-} does not palatalize before [i], as shown in (\ref{ex:nonpalatn}). The contrast between second and third person shows that \textit{n-} in (\ref{ex:nonpalatn}) does not change its place of articulation in position \ref{sap} before the epenthetic [i].


\ea\label{ex:nonpalatn}
    Left-edge non-patalization: \textit{n-} `\Third.\I'
    \ea\label{ex:nonpalatnb} [ɾijaˈtʃiʔ]\\
    \glll ɾ-jat-iɾ\\
    \ref{sap}-\ref{moc:core}-\ref{2and1pl}\\
    {\Second.\I}-be.worried-{\Second.\I}\\
    \glt `You are worried' or `Are you worried?' \hfill(mocCA120717: 01:33:51)
    \ex\label{ex:nonpalatna} [niˈjat]\\
    \glll n-jat\\
    \ref{sap}-\ref{moc:core} \\
    {\First.\I}-be.worried\\
    \glt `He/She is worried.' \hfill(mocCA120717: 01:34:03)
    \z
\z

As one of the reviewers pointed out, palatalization might not affect alveolar consonants on the left edge of the verb complex because they precede a non-underlying [i]. This explanation might be valid, but it should also be considered that the non-palatalization of a segment like \textit{n-} for the third person preserves a meaningful paradigmatic distinction in the language. In Set III bound person form, for example, the contrast \textit{ɲ-} vs. \textit{n-} distinguishes first and third-person arguments, as illustrated in (\ref{ex:paradgmdist}).  

\ea\label{ex:paradgmdist}
\ea\label{ex:paradgmdista}
 [ɲenaːɲi]\\
\gll ɲ-naːn-ɲi\\
{\First.\III}-lay.down-{\Dir:\Down}\\
\glt `I lay down.' \hfill(mocCA120717: 01:07:14)
\ex\label{ex:paradgmdistb}
[nenaːɲi]\\
\gll n-naːn-ɲi\\
{\Third.\III}-lay.down-{\Dir:\Down}\\
\glt `He/She lays down.' \hfill(mocCA120717: 01:10:03)
\z
\z

Although examples (\ref{ex:nonpalats}) and (\ref{ex:nonpalatn}) show that palatalization is ruled out with \textit{n-} and (optionally) \textit{s-} on the left edge of core element in \ref{moc:core}, we do not find sequences such as /t + i/ or /l + i/ in verbs to empirically prove that this process is blocked across the same consonants attained by palatalization on the right edge. The segment \textit{t-}, for example, is a third-person bound form, lexically restricted to predicates that do not contain [i] in their phonological structure, e.g., \textit{\textbf{t}aʔwe} `S/he goes to a place'. The segment \textit{l-}, on the other hand, is a third person possessive bound form, e.g., \textit{\textbf{l}awa} `His/her arm', and does not occur with verbs.\footnote{In Mocoví, unlike its sister language Western Formosa Toba \citep[see][]{Carpio2012a}, there is no verbal marker \textit{l-} indicating third-person arguments.}    


\subsection{Vocalic epenthesis: spans \ref{sap}-\ref{moc:core} and \ref{sap}-\ref{trlocdir}} 
\label{sec:epenthesis}

As generally assumed in phonology literature \citep[e.g.,][263]{Hayes}, epenthetic processes are motivated by syllabification constraints. In other words, epenthetic elements are added to create valid syllable types within a language. In Mocoví, the simplest syllable type is V, but we also find CV, CVC and CCVC as other syllable types. The two most frequent syllable types are CV, as in (\ref{ex:posteriorharmonyc}) and (\ref{ex:highharmonya}), and CVC, as in (\ref{ex:epenthesisleft}), which are also the cross-linguistically least marked types of syllables \citep[84--85]{Gordon2016}. 
This section shows the application of vocalic epenthesis as another diagnostic that identifies a span of positions in the planar structure. This diagnostic targets minimally positions \ref{sap}-\ref{moc:core} and maximally positions \ref{sap}-\ref{trlocdir}.

 Epenthesis occurs on both sides of the core element in position \ref{moc:core}. When epenthetic vowels are inserted to the left of position \ref{moc:core}, the minimal sub-span is identified. In  (\ref{ex:epenthesisleft}), for example, the vowel [e] is inserted between the bound person form \textit{n-} and the verb root \textit{-tʃaq} `cut'. 

\ea\label{ex:epenthesisleft}
 [neˈtʃaq]\\
\glll n-tʃaq\\
\ref{sap}-\ref{moc:core}\\
{\Third.\III}-cut\\
\glt `She/He cuts himself/herself.' \hfill(mocCA210803\_1: 00:08:38)
\z

On the other hand, the maximal span of positions can be seen on the right side of position \ref{moc:core}, as in (\ref{ex:epenthesisright}). The epenthetic vowel [a] is required between the verb base and the locative suffix \textit{-ɡe} `{\LocTwo}'. 

\ea\label{ex:epenthesisright}
 [itʃaq\textbf{a}ˈɣe]\\
\glll i-tʃaq-ɡe\\
\ref{sap}-\ref{moc:core}-\ref{trlocdir}\\
{\Third.\III}-cut-{\LocTwo}\\
\glt `She/He cuts it completely.' \hfill(mocCA210803\_1: 00:32:55)
\z 

Outside the domain of positions \ref{sap}-\ref{trlocdir}, the epenthetic process does not apply. Morphological elements preceding position \ref{sap}, for example, already include a vowel in their underlying form, e.g., \textit{sV} or \textit{sqa} `\Neg' in positions \ref{moc:neg} and \textit{qa-} or \textit{qo-} `\Def.\Sbj' in position \ref{sdef}. Likewise, morphological elements occurring between positions \ref{trlocdir}-\ref{encl} come in a syllabic structure that does not require the addition of an extra segment to conform a valid syllable type. These elements add another CV syllable to the verb structure or create close syllables of the type CVC. 

% It seems that the vowel [e] before the progressive is part of the verb as -ɢan does not include [a] at the beginning like in other examples.  
%An interesting fact about epenthetic vowels is that their selection is not always predictable. With some predicates, the insertion of a vowel or another seems to be conditioned by the bound person paradigm. Observe the contrast between [dʒaˈpil] vs. [soˈpil]. 

\subsection{Vowel harmony: spans \ref{moc:neg}-\ref{moc:core},  \ref{sap}-\ref{2and1pl} and \ref{moc:core}-\ref{2and1pl}}
\label{sec:vowel harmony}

 Vowel harmony is another common phonological process in Mocoví. In many cases, vowel harmony  interacts with vocalic epenthesis, in that the inserted vowels must harmonize with the vowel or the semi-consonant [j] that is part of the closest syllable to the verb.  %vowels that are part of the verb. Another example of vowel harmony has to do with changes in the verb root vowels, which are conditioned by the second-person bound form \textit{-i} in position \ref{2and1pl}. This test is fractured according to these two types of instances of vowel harmony.
 
 Two classes of vowels usually condition one another; on the one hand, the front vowels /e/ and /i/, and on the other hand, the non-front vowels /a/ and /o/. One vowel of each group can trigger the assimilatory process, while the other is the target of the change.  The directionality of vowel harmony can be progressive or regressive \citep[in the sense of][82]{Katamba1992}.

\subsubsection{Verb-conditioned vowel harmony}
Verb-conditioned vowel harmony identifies the span of positions \ref{moc:neg}-\ref{moc:core}, as shown in (\ref{ex:minimalharmony}). This example illustrates a case of progressive assimilation, i.e., the inserted sound is similar to the sound that follows it.  The vowel included in the negative prefix \textit{sV-} changes its articulation depending on the structure of the following syllable. Preceding a syllable of the type CV, where the consonant does not correspond to the semi-consonant [j], the inserted vowel harmonizes with the root vowel [a], as in (\ref{ex:minimalharmonya}). However, when the syllable after the negative marker includes [j], the vowel articulates as [i] like in (\ref{ex:minimalharmonyb}). Recall that in this latter example, the underlying [i] surfaces as [j] due to the preference of the CV syllable structure. 

\ea \label{ex:minimalharmony}
\ea \label{ex:minimalharmonya}
 [s\textbf{a}saʔˈden]\\
\glll sa-s-aʔden\\
\ref{moc:neg}-\ref{sap}-\ref{moc:core}\\
{\Neg}-{\First.\II}-know\\
\glt `I don't know (something).' \hfill(mocCA120706: 00:15:50)
\ex \label{ex:minimalharmonyb}
 [s\textbf{i}jaʔˈden]\\
\glll si-i-aʔden\\
\ref{moc:neg}-\ref{sap}-\ref{moc:core}\\
{\Neg}-{\Third.\II}-know\\
\glt `She/He does not know (something).' \hfill(mocCA180807: 00:09:06)
\z
\z 

On positions to the right side of \ref{moc:core}, vowel harmony identifies the span of positions \ref{sap}-\ref{vmII} and the directionality of the harmonic process is regressive, i.e., a vowel becomes similar to the preceding vowel. In (\ref{ex:posteriorharmony}), vowel harmony involves the mid back vowel /o/. Observe that the verb base in (\ref{ex:posteriorharmonya}) does not include a final vowel on its right edge structure. The epenthethic vowel /o/, inserted at the juncture of positions \ref{moc:core} and \ref{vmII}, is identical to the vowel of the preceding syllable, as in (\ref{ex:posteriorharmonyb}). Furthermore, that inserted vowel conditions the realization of /a/ in the valence modifier /-ɢan/. The example in  (\ref{ex:posteriorharmonyc}) shows that vowel harmony does not affect other elements beyond position \ref{vmII}. 

\newpage
\ea \label{ex:posteriorharmony}
\ea\label{ex:posteriorharmonya}
 [qajpeˈlok]\\
\glll qa-i-peloɡ\\
\ref{sdef}-\ref{sap}-\ref{moc:core}\\
{\Sbj.\Def}-{\Third.\II}-rake\\
\glt `Someone raked it.' \hfill(mocCA160725: 00:02:03)
\ex\label{ex:posteriorharmonyb}
 [ɾepeloɣ\textbf{o}ˈʁ\textbf{o}n]\\
\glll ɾ-peloɡ-ɢan\\
\ref{sap}-\ref{moc:core}-\ref{vmII}\\
{\Third.\II.\Intr}-rake-{\Vm:\Intr}\\
\glt `He/She rakes.' \hfill(mocCA160725: 00:09:32)
\ex\label{ex:posteriorharmonyc}
 [peloɣ\textbf{o}ˈtaʔ]\\
\glll ∅-peloɡ-taʔ\\
\ref{sap}-\ref{moc:core}-\ref{rr}\\
{\Third.\II}-rake-{\Recp}\\
\glt `He/She rakes them.' \hfill(mocCA160725: 00:03:07)
\z
\z 

 %The vowel change described above, on the contrary, does not occur when \textit{-ɢan} attaches to a verb base that does not contain the mid back vowel /o/, as illustrated in the contrast of (\ref{ex:nonposteriorharmony}).\footnote{The reader probably notes the alternation [k] vs. [ɣ] at the end of the verb base ([ɣ] is a non-distinctive phonetic variant of /ɡ/). The voiceless/voiced alternation is not clear yet, but it does not seem to be predictable on pure phonological grounds, as the alternation occurs in the same environment, e.g., [e]\_[e]. The alternation seems to be rather conditioned by the inflectional or derivational morphological characteristics of the elements that are part of the verb complex. Thus, [k] is found with inflectional elements such as the progressive marker \textit{-tak}, whereas [ɣ] occurs with a derivational-like element such as \textit-{ɢan}.} 

%\ea \label{ex:nonposteriorharmony}
%\ea\label{ex:nonposteriorharmonya}
% [s\textbf{e}ʔlekeˈtak]\\
%\glll s-ʔlek-tak\\
%\ref{sap}-\ref{moc:core}-\ref{moc:asp}\\
%{\First.\II}-sweep-{\Prog}\\
%\glt `I'm sweeping.' \hfill(mocCA130614: 00:08:45)
%\ex\label{ex:nonposteriorharmonyb}
% [ɾeʔleɣe\textbf{ʁan}ˈtak]\\
%\glll ɾ-ʔlek-ɢan-tak\\
%\ref{sap}-\ref{moc:core}-\ref{vmI}-\ref{moc:asp}\\
%{\First.\II}-sweep-\Vm-\Prog\\
%\glt `He/She is sweeping.' \hfill(mocCA160725: 00:06:54) % It seems that the vowel [e] before the progressive is part of the verb as -ɢan does not include [a] at the beginning like in other examples.  
%\z
%\z 

\subsubsection{Affix-conditioned vowel harmony}
Unlike examples above, vowel harmony is also produced by the second-person bound form \textit{-i}. The expression of the second person triggers progressive assimilation in the verb vowels, as in (\ref{ex:highharmony}). This process identifies the span \ref{moc:core}-\ref{2and1pl}. Observe the inflectional contrast of the verb \textit{-keʔe} `eat$_1$' and note that the verb vowels change when the verb inflects for the second-person A argument. Since the second person is expressed by the suffix \textit{-i}, the verb vowels change to [i], assimilating in height to the following vowel suffix, as in (\ref{ex:highharmonyb}). 

\ea \label{ex:highharmony}
\ea \label{ex:highharmonya}
 [sek\textbf{e}ʔ\textbf{e}ˈtak]\\
\glll s-keʔe-tak\\
\ref{sap}-\ref{moc:core}-\ref{moc:asp}\\
{\First.\II}-eat\textsubscript{1}-{\Prog}\\
\glt `I'm eating (something).' \hfill(mocCA120706: 00:21:03)
\ex \label{ex:highharmonyb}
 [k\textbf{i}ʔ\textbf{i}ˈsak]\\
\glll keʔe-i-sak\\
\ref{moc:core}-\ref{2and1pl}-\ref{moc:asp} \\
eat\textsubscript{1}-{\Second.\II}-{\Prog}\\
\glt `You are eating (something).', `Eat.' or `Are you eating?'\\\hfill (mocCA120706: 00:22:44)
\z
\z 

This type of vowel harmony rule occurs across verbs that include the vowel /e/ as part of their vocalic elements. As shown in many other other examples throughout the chapter, vowel harmony feeds another phonological process, such as palatalization. This phenomenon was observed earlier in (\ref{ex:palatizationn}), but the same examples are repeated here in (\ref{ex:harmonyandpalat}) for convenience. 
\ea\label{ex:harmonyandpalat}
\ea\label{ex:harmonyandpalata}
[jaʔˈ\textbf{d}e\textbf{n}]\\
\glll    i-ʔden  \\
    \ref{sap}-\ref{moc:core}\\
      {\Third.\II}-know\\
\glt `He/She knows it.'  \hfill (mocCA120706ː 01ː24ː28)
\ex\label{ex:harmonyandpalatb}
 [\textbf{dʒ}iˈ\textbf{ɲ}iʔ] \\
\glll        ʔden-iɾ \\
    \ref{moc:core}-\ref{2and1pl}\\
      know-{\Second\Sg.\II}\\
\glt `You know it.'  \hfill (mocCA120713: 00:01:12)
\z
\z

Underlyingly, the verb `know' is \textit{ʔden}, but when the verb inflects for the second person, the vowel /e/ changes to [i]. Since the inflected verb form only contains the high vowel /i/, the palatalization of alveolar consonants follows, as in (\ref{ex:harmonyandpalatb}). 

\subsection{Phonological tests: interim results}
\autoref{fig:phonresults} summarizes the results of the phonological tests applied to the planar structure. The graph shows that individual tests identify different spans of positions and these subspans do not align with the same constituent. 

\begin{figure}
    \centering
   \includegraphics[width=\textwidth]{figures/mocovi_phon_plot.png}
    \caption{Results from phonological tests}
    \label{fig:phonresults}
\end{figure}

There are certain subspans that are selected more than once, \ref{sap}-\ref{moc:core}, \ref{moc:core}-\ref{2and1pl}, \ref{sap}-\ref{trlocdir}, but still other results add more variation to the selection of constituents. Thus, the phonological domain does not provide strong evidence for selecting a subspan of positions and defining it as the best exemplar of a phonological word. 


\section{Morphosyntactic diagnostics}\label{sec:morphsyntactictests}
The following sections analyze morphosyntactic diagnostics applied to the Mocoví verbal planar structure and show the different spans of positions that are selected by each diagnostic. The diagnostics explored are free occurrence (\S\ref{sec:mocovi:freeoccurrence} ), non-interruptability (\S\ref{sec:noninter} ), sub-span repetition (\S\ref{sec:subspan}), (non)-permutability (\S\ref{sec:nonperm}), deviation from bi-uniqueness (\S\ref{sec:deviation}) and (non)-ciscategorial selection (\S\ref{sec:nonciscategorial}). Each of these diagnostics is described in turn. 


\subsection{Free occurrence: span \ref{sap}-\ref{moc:core} and \ref{moc:neg}-\ref{encl}}\label{sec:mocovi:freeoccurrence}
Free forms, i.e., a combination of elements that can occur as a complete utterance \citep[][18--19]{Tallman2020}, can be divided into minimal and maximal free forms. Minimal free forms identify the smallest span of positions that can stand alone, whereas maximal free forms represent the largest number of positions, corresponding to a full utterance.   

A minimal free form in Mocoví is commonly identified by the combination of the core element in \ref{moc:core} and the expression of core arguments via bound person forms. Three bound person form paradigms are the main morphological coding devices of core arguments.  The complete paradigms of bound person forms are presented in  \autoref{tab:bound person forms paradigmsI}. Typically, only one argument is expressed in each verb, S or A, via one of the bound person forms.  The span of positions identified by the verb core and the expression of core arguments varies. Two different minimal free forms can be recognized depending on the grammatical person that is expressed and the bound person form paradigm that is (lexically) selected.

\begin{table}
\captionof{table}{Bound person form paradigms in positions \ref{sap} and \ref{2and1pl}}
\label{tab:bound person forms paradigmsI}
\begin{tabular}{cccc}
\lsptoprule
\textbf{Person \& Number} &  \textbf{Set I }&   \textbf{Set II} & \textbf{Set III }         \\\midrule
\textsc{1sg} &   \textit{dʒ-} &  \textit{s-} & \textit{ɲ-}\\
\textsc{1pl} &   \textit{qaɾ-} & \textit{s-...-ɢ}& \textit{ɲ-...-ɢ}\\
\textsc{2sg} &\textit{ɾ-...-iɾ} &\textit{-iɾ} & \textit{n-...-iɾ}\\
\textsc{2pl} &  \textit{ɾ-...-i} $\sim$\textit{ɾ-...-iː} &  \textit{-i} $\sim$ \textit{-iː} & \textit{n-...-i} \\
\textsc{3} & \textit{i-} & \textit{i-} &\textit{n-} \\
 & \textit{ɾ-} &  \textit{ɾ-} &\\
 & \textit{∅-} &   \textit{∅-} &\\
  & \textit{n-}&  \textit{t-} &\\
\lspbottomrule
\end{tabular}
\end{table}


For example, a minimal free form includes only two positions, \ref{sap}-\ref{moc:core}, with verbs that take Set II to express first-person singular and third-person arguments. Consider the verb \textit{iwaɢan} `S/he hits him/her/it' in (\ref{ex:minimal1}) inflected for the third-person A. The predicate in position \ref{moc:core} takes the bound person \textit{i-} from the Set II in position \ref{sap}. As a free form, this verb, furthermore, can be used as a complete answer to a question, such as \textit{What happened?} or \textit{What did he do?}. Neither verbs nor bound person forms can occur by themselves. 

\ea \label{ex:minimal1}
\ea 
	\glll i-waɢan \\
	      \ref{sap}-\ref{moc:core}\\               
	  {\Third.\II}-hit  \\
	\glt `She/He hit him/her.' \hfill (mocCA160725: 01:23:11)
\ex	[*] {
\glll waɢan\\
      \ref{moc:core}\\               
	    hit\\
	\glt `She/He hit him/her.'}
\z
\z

The default interpretation of a verb form like \textit{iwaɢan} is that the P argument corresponds to a third person. However, if the P argument corresponds to non-speech act participants, independent pronouns are employed, e.g., \textit{ajim} \textsc{`\First\Sg.\Pron'} or \textit{qamiɾ} \textsc{`\Second\Sg.\Pron'} and their corresponding plural counterparts, to explicit the argument's referentiality. The same holds for nominal elements, in that they are employed to establish  the specific extra linguistic referent in the discourse. 

On the other hand, the maximal free form identifies the span of positions \ref{moc:neg}-\ref{encl}, including the negative prefix to the left edge of the verb structure and the enclitic \textit{=oʔ} `\Evid/\Tprl?' to the right.  Almost all elements of that string of positions are illustrated in (\ref{ex:maximum}). None of the elements in the maximal free form span can occur by themselves.   

\ea\label{ex:maximum}
\glll  \textbf{qa}-n-oʔteɡ-ʃiɡem=\textbf{oʔ}\\
 \ref{sdef}-\ref{sap}-\ref{moc:core}-\ref{trlocdir}-\ref{encl}\\
 {\Sbj.\Def}-{\Third.\III}-pull.out-{\Dir:\Up}={\Evid/\Tprl}?\\
	\glt `Somebody has pulled it out.'
	 \hfill (mocCA160720: 00:17:58)
\z

Adding other elements to the span \ref{moc:neg}-\ref{encl} is possible, but the clause structure would include multiple free forms that do not necessarily overlap with the verb core. Natural discourse examples illustrate cases of this kind.     

\ea \label{ex:maxmimum1}
	\glll ... maʔ s-wiɾ-lek=oʔ a-so i-kome:na... \\
	       {} \ref{linkers}   \ref{sap}-\ref{moc:core}-\ref{trlocdir}-\ref{encl} \ref{moc:parg}\\ 
	   {} {because} {\First.\II}-arrive-{\LocOne}={\Evid}/{\Tprl}? {\F}-{\DetTwo} {\First\Sg.\Poss}-grandmother \\
	\glt `... (I'm talking about this) because I reached my grandmother back then...'\\
\glt	\hfill (mocCA191010\_1: 00:14:02)
\z
In (\ref{ex:maxmimum1}), for instance, more elements of the planar structure are included, e.g., the adverbial marker \textit{maʔ} in position \ref{linkers} and the nominal constituent \textit{aso ikomeːna} in position \ref{moc:parg}.\footnote{I mark the position of large nominal constituents under the first element that corresponds to that constituent. In the case of (\ref{ex:maxmimum1}), the first element of the nominal structure is the prefix \textit{a-} `\F'.}


\subsection{Non-interruptability: span \ref{moc:neg}-\ref{encl}}
\label{sec:noninter}

The span of positions \ref{moc:neg}-\ref{encl} cannot be interrupted by simple or complex free forms. Simple free forms can be, for example, adverbial elements in position \ref{sapadv}, whereas complex free forms can correspond to nominal constituents, such as those in positions \ref{sapadv}, \ref{moc:parg}  or \ref{moc:obl}. For the sake of brevity, interruptability will be tested here by taking an adverbial free form as the interrupting element. If the interrupting element was a nominal constituent, the result provided by this test would not be different. 

As mentioned earlier, the span of positions \ref{moc:neg}-\ref{encl} cannot be interrupted by any type of free form. It does not matter whether the verbal free form is minimal, as in (\ref{ex:minimal1}), or maximal, like in (\ref{ex:maximum}), no other free form can interrupt that string of positions. An example of the interruptability test is given in (\ref{ex:non-interrupt}). The span of positions that goes from \ref{sap}-\ref{vmII} can not be interrupted by the free form \textit{nagi} `now'. 

\ea \label{ex:non-interrupt}
\ea[] {
	\glll so i-taʔa ajim i-kin-aɢan-aɢan ke-so qaɾ-piɾ \\
	       \ref{sapadv} {} \ref{p} \ref{sap}-\ref{moc:core}-\ref{vmI}-\ref{vmII} \ref{moc:obl} \\
	   {\DetTwo} {\First.\Poss}-father {\First\Sg.\Pron} {\Third.\II}-greet-{\Vm:\Intr}-{\Vm:\Tr}   {\Obl}-{\DetTwo} {\First\Pl.\Poss}-grandfather\\
	\glt `My father makes me greet my grandfather.' \hfill (mocCA160725: 01:11:22)} \label{ex:uninterrupt1}
\ex[*] {
	\glll  so i-taʔa ajim i-kin-\textbf{naɡi}-aɢan-aɢan ke-so qaɾ-piɾ \\
	         \ref{sapadv} {} \ref{p} \ref{sap}-\ref{moc:core}-naɡi-\ref{vmI}-\ref{vmII} \ref{moc:obl} \\
	    {\DetTwo} {\First.\Poss}-father {\First\Sg.\Pron} {\Third.\II}-greet-now-{\Vm:\Intr}-{\Vm:\Tr}   {\Obl}-{\DetTwo} {\First\Pl.\Poss}-grandfather\\
	\glt `My father makes me greet my grandfather.' \hfill (Fieldnotes 2021)}\label{ex:uninterrupt2}
	\ex[] {
	\glll  so i-taʔa \textbf{nagi} ajim i-kin-aɢan-aɢan ke-so qaɾ-piɾ \\
	       \ref{sapadv} {} {} \ref{p} \ref{sap}-\ref{moc:core}-\ref{vmI}-\ref{vmII} \ref{moc:obl} \\
	   {\DetTwo} {\First.\Poss}-father now {\First\Sg.\Pron} {\Third.\II}-greet-{\Vm:\Intr}-{\Vm:\Tr}   {\Obl}-{\DetTwo} {\First\Pl.\Poss}-grandfather\\
	\glt `My father makes me greet my grandfather now.' \hfill (Fieldnotes 2021)}\label{ex:uninterrupt3}
\z
\z

 It is quite frequent in natural discourse to find adverbial units placed before or after the verb structure formed by elements \ref{moc:neg}-\ref{encl} (see, for instance, \textit{naʔɢa} `day' in (\ref{ex:fixedend}) below). In the case of \textit{naɡi}, for example, it  can occur relatively freely between other positions of the planar structure. It can occur before the P argument in position \ref{p}, as illustrated in (\ref{ex:uninterrupt3}), or after position \ref{encl}.  In (\ref{ex:non-interruptadv}), for example, \textit{naɡi} occurs between two predicates occupying different positions of the planar structure. 
    
 
 \ea\label{ex:non-interruptadv}
 \glll ... qaʔ s-kin-ɢan-tak \textbf{naɡi} s-kin-tak na-ʔe qaja-ɾ-pi...\\
      {}  \ref{linkers} \ref{sap}-\ref{moc:core}-\ref{vmII}-\ref{moc:asp}  \ref{sapadv} \ref{sap}-\ref{moc:core}-\ref{moc:asp} \ref{moc:parg}\\
      {}  {\LinkOne} {\First.\II}-greet-{\Vm:\Intr}-{\Prog} now {\First.\II}-greet-{\Prog} {\DetThree}-ʔe sibling-{\Pl}-{\Coll}\\
 \glt `...thus, I'm greeting, now I'm greeting my siblings...' \\
  \glt \hfill (mocCA180709: 00:21:12)
 \z  
 
 It is not clear yet what pragmatic or semantic factors control the distribution of adverbial elements in the clause structure. No studies have explored this topic in detail yet, therefore further research is needed to gain a better understanding of the  variable positioning of temporal elements in the language.


\subsection{Subspan repetition}
\label{sec:subspan}
This test evaluates the subspans of the planar structure combined in specific constructions. The constructions analyzed here involve different types of clause linkage, including overt and non-overt clause linkers. The repeated sub-spans are defined according to the elements that have local scope and, therefore, must occur in each combined portion of the planar structure. Omitting elements with local scope can lead to changes in meaning and ungrammatical utterances. Similar the previously discussed tests, the conjunction of repeated subspans can also be divided into minimal and maximal subspans of positions.

\subsubsection{Non-overt clause combination: span \ref{sap}-\ref{moc:core}, \ref{p}-\ref{encl} and \ref{sap}-\ref{moc:parg}}
The maximal span of repeated elements needs to be devided further according to the grammatical person of the P argument. When the P argument corresponds to a third person, the repeated span is  \ref{sap}-\ref{moc:parg}, as in (\ref{ex:minsubspan1}). A combination of two predicates of this kind is employed to syntactically express caused events. Both predicates form a complex clause and are integrated by the joint function of \textit{ajim} `\First\Sg.\Pron', which is usually not repeated with the second predicate of this type of clauses. This independent pronoun functions as P in the first clause, and as A in the second. There is no overt linker in the sentence and the logical sequence of events mirrors the linear occurrence of predicates. 

\ea \label{ex:minsubspan1}
	\glll ajim \textbf{i}-laɾ \textbf{so} \textbf{i}-\textbf{taʔa} \textbf{s}-alawat \textbf{a}-\textbf{so} \textbf{waqaeʔ} \\
	       \ref{p}   \ref{sap}-\ref{moc:core} \ref{moc:parg} {} \ref{sap}-\ref{moc:core} \ref{moc:parg}\\
	    {\Fsg.\Pron} {\Third.\II}-order/send {\DetTwo} {\First\Sg.\Poss}-father {\First.\II}-kill  {\F}-{\DetTwo} chicken   \\
	\glt `My dad ordered me to kill the chicken.' \\
	\glt  (Lit. My dad ordered me, I kill the chicken.) \hfill(mocCA191025: 01:05:21)
\z

The other fracture of maximally repeated subspans is \ref{p} to \ref{encl}, which is only observed when the P argument corresponds to first or second-person. Elements within this subspan of positions do not display wide scope and they are repeated in the syntactically combined subspans.  I do not have examples that illustrate all the positions of the subspan \ref{p} to \ref{encl}, but finding  examples in which individual elements within that span can be repeated is not hard. For the sake of brevity, here I only provide three examples that support this argument. I focus on examples that illustrate the edges \ref{p} and \ref{encl}, which delimit the maximal repeated subspan, and one example illustrating an itermediate subspan of repeated positions. 

The left edge of the subspan \ref{p}-\ref{encl} is illustrated in (\ref{ex:maxspan0}). As shown in this example, the independent pronoun \textit{qamiɾ} `\Second\Sg.\Pron' in position \ref{p} must occur in each of the combined clauses and commonly precedes two-place predicates (or any minimal and maximal free form with a core element of that valence, see \S\ref{sec:mocovi:freeoccurrence}). If, for instance, the independent pronoun does not occur in the second clause, that event would naturally entail a third-person P argument, which commonly follows two-place predicates (cf. (\ref{ex:minsubspan1}) above). 

\ea\label{ex:maxspan0}
\glll ... \textbf{qamiɾ} s-alawat-q \textbf{qamiɾ} s-ʔɡin-q...\\
{} \ref{p} \ref{sap}-\ref{moc:core}-\ref{2and1pl} \ref{p} \ref{sap}-\ref{moc:core}-\ref{2and1pl}\\
{} {\Second\Sg.\Pron} {\First.\II}-kill-{\First.\II.\Pl} {\Second\Sg.\Pron} {\First.\II}-kill-{\First.\II.\Pl} \\
\glt `... we are going to kill you and we are going to eat you...'\\
\glt \hfill (mocCA191025: 00:38:35)
\z

Other elements can be added to the right side of the core element and they also show local scope. This is the case, fro example, of the directional \textit{-ɲi}, which repeats in each subspan of (\ref{ex:subspanmid}). Note that the element in position \ref{sapadv} can be omitted but all the other elements of the combined predicates are obligatory to have grammatical clauses. 

\ea \label{ex:subspanmid}
	\glll ... so peget \textbf{n}-ah-\textbf{ɲi} \textbf{ɾ}-amoɢoja-\textbf{ɲi} \\
	        {} \ref{sapadv} {}   \ref{sap}-\ref{moc:core}-\ref{trlocdir} \ref{sap}-\ref{moc:core}-\ref{trlocdir} \\
	  {} {\DetTwo} {plate} {\Third.\III}-fall-{\Dir:\Down}    {\Third.\II}-get.broken-{\Dir:\Down}\\
	\glt `The plate fall and broke in many parts.' \hfill (mocCA170803: 08:52:00)
\z


The right edge of the maximally repeated  subspan is delimited by the element in position \ref{encl}. As shown in (\ref{ex:maxsubspan1}), the enclitic \textit{=oʔ} repeats in each of these combined predicates, which also indicates the local scope of this element. 

\ea \label{ex:maxsubspan1}
	\glll ... \textbf{n}-owiɾ-ta=\textbf{oʔ} \textbf{ɾ}-pil-ta=\textbf{oʔ}... \\
	      {}  \ref{sap}-\ref{moc:core}-\ref{moc:asp}=\ref{encl} \ref{sap}-\ref{moc:core}-\ref{moc:asp}=\ref{encl} \\
	    {} {\Third.\III}-come-{\Dur}={\Evid/\Tprl}?  {\Third.\II}-come.back-{\Dur}={\Evid/\Tprl}?  \\
	\glt `...she then came and went back (just like that)...' \\
	\hfill (mocCA191010\_1: 00:14:22)
\z

Finally, the minimal repeated subspan of positions is \ref{sap}-\ref{moc:core}. This type of subspan is found in a lexical causative construction, as in (\ref{ex:minimalspan}). 

\ea\label{ex:minimalspan}
\glll jim s-tʃiko qaʔen \\
\ref{sapadv} \ref{sap}-\ref{moc:core} \ref{sap}.\ref{moc:core} \\
{\First\Sg.\Pron} {\First.\II}-be/get.sad make.{\Third}\\
\glt `She/He made me sad.'\\
\glt  `Lit: I am sad, She/He makes (it).' 	\hfill(mocCA160705: 00:53:15)
\z

Both predicates only include one core argument each and they are obligatory elements with local scope. Syntactically,  the first predicate expresses the causee argument via the bound person form \textit{s-}, whereas the second predicate expresses the causer argument via the suppletive verb form of `make'. 

\subsubsection{Overt clause combination: span \ref{sap}-\ref{rr} and \ref{sap}-\ref{moc:parg}}   
Clause combination by overt linking elements also identify maximal and minimal subspans of repeated positions.  The precise number and function of clause linkers has yet to be studied for the different Mocoví variants, but see \citet[170--175]{Grondona1998} and \citet[109--113]{Gualdieri1998} for some preliminary information about clause linkers in Southern Chaco Mocoví.
Here I focus on two clause linkers, \textit{kan} and \textit{nakoʔ}, present in Northern Chaco Mocoví. These linkers are similar in form to some of the clause linkers found in Southern Chaco Mocoví. 

The clause linker \textit{kan}, which can be translated as `and then', identifies the span of positions  \ref{sap}-\ref{moc:parg}, as in (\ref{ex:linking1}).  This span of positions represents the maximal number of repeated elements, which are linked by \textit{kan}. 

\ea \label{ex:linking1}
\glll ... n-atʃel=oʔ ʔaːlo kan i-ʔwe-lek=oʔ so l-lo... \\
 {} \ref{sap}-\ref{moc:core}=\ref{encl} \ref{moc:parg} \ref{linkers} \ref{sap}-\ref{moc:core}-\ref{trlocdir}=\ref{encl} \ref{moc:parg} \\ 
 {} {\Third.\III}-shower-{\Evid/\Tprl}? {woman} {\LinkTwo} {\Third.\II}-mount-{\LocOne}-{\Evid/\Tprl}? {\DetTwo} {\Third.\Poss}-{\Poss.\Clf}\\
\glt `... the woman showered herself and mounted her animal (horse)...'\\
\glt \hfill (mocCA180807: 00:04:51)
\z 

 Note that elements in position \ref{moc:parg} differ in their syntactic function. Both predicates share the participant \textit{ʔaːlo} `woman', which expresses the S argument for the predicate \textit{natʃeloʔ} and A for the predicate \textit{iʔwelekoʔ}. On the other hand, the participant \textit{so lalo} `his/her animal' in position \ref{moc:parg} after the second predicate corresponds to the argument P.     

A minimal span of repeated positions is \ref{sap}-\ref{rr}, but this subspan is linked by \textit{nakoʔ}. Like (\ref{ex:linking1}), the combination of clauses in (\ref{ex:linking2}) deals with a same subject construction. Each predicate requires its own bound person form \textit{n-} and the reciprocal marker \textit{-taʔ}. Both of these elements have a local scope restricted to each predicate.  

\ea \label{ex:linking2}
\glll n-wiɾ-taʔ nakoʔ n-qaʔal-taʔ\\
\ref{sap}-\ref{moc:core}-\ref{rr} \ref{linkers} \ref{sap}-\ref{moc:core}-\ref{rr}\\
    {\Third.\III}-come-{\Recp} {\LinkOne} {\Third.\III}-hug-{\Recp}\\
     \glt `They met each other and hug each other.' \hfill (mocCA191017\_2: 00:35:19)
\z

Like asyndetic clause combination, we could have expected that predicates having just a distance of one between their elements, e.g., \ref{sap}-\ref{moc:core}, can naturally be overtly linked. I have not tested that possible combination yet, thus, further research is needed. 


\subsection{(Non)-permutability: span \ref{moc:neg}-\ref{encl} and \ref{moc:neg}-\ref{trlocdir}} \label{sec:nonperm}
This test explores the extent to which elements overlapping with the core element in \ref{moc:core}  can be flexibly ordered within the planar structure.  Elements in the span of positions \ref{moc:neg}-\ref{encl} have fixed positions, i.e., each element is assigned to a single unique position. However, some elements within that span of position have a flexible order, which is constrained by verb types. Thus, I fracture this test  according to such internal organization of elements. First, I show the rigid positioning of elements, and then turn to the examples where permutability of affixes is observed. % This flexible behaviour of elements indicates the presence of a zone rather than a slot in the planar structure (see position \ref{trlocdir} in the \autoref{tab:verbplanar}). In the following subsections, I first show cases of non-permutability and then .  

%\subsubsection{Non-permutability: span \ref{moc:neg}-\ref{rr} and \ref{pllo}-\ref{encl}}
The non-permutability of elements can be observed on both sides of the core element in position \ref{moc:core}. On the left side, for example, the negative marker \textit{sV-} and the third-person bound form \textit{i-} in (\ref{ex:fixneg}) cannot switch positions. The reverse order of these elements produces an ungrammatical sentence, as in  (\ref{ex:fixnegungr}). Furthermore, I did not document examples in which the negative marker nor the third-person bound forms are placed somewhere else within the planar structure.

\ea
\ea[]{ \label{ex:fixneg}
\glll ... si-i-aʔden mi{\Sploan} pensamiento{\Sploan}...  \\
     {}  \ref{moc:neg}-\ref{sap}-\ref{moc:core} \ref{moc:parg}\\
     {}  {\Neg}-{\Third.\II}-know my thought\\
\glt `...he doesn't know my thought...' \hfill (mocCA191025: 00:32:33)}

\ex[*]{\label{ex:fixnegungr}
\glll  i-s-aʔden mi{\Sploan} pensamiento{\Sploan}  \\
     \ref{sap}-\ref{moc:neg}-\ref{moc:core} \ref{moc:parg}\\
      {\Third.\II}-{\Neg}-know my thought\\
\glt `he doesn't know my thought.'\\
}
\z
\z

 On the right side, all elements up to position \ref{encl} have a rigid order as well. This can be observed by looking at different subspans including the core element in addition to other positions; see, for instance, the positions of the progressive marker \textit{-teɡ} and the reciprocal marker \textit{-taʔ} in (\ref{ex:fixedpos1a}). These two elements cannot occur in any other order or position within the verb structure. Thus, the progressive must precede the reciprocal and the reverse order produces an ungrammatical clause, as in (\ref{ex:fixedpos1b}). 
 
 \ea\label{ex:fixedpos1}
  \ea\label{ex:fixedpos1a}
    \glll n-meːn-\textbf{teɡ}-\textbf{taʔ} na l-owen-ek\\
    \ref{sap}-\ref{moc:core}-\ref{moc:asp}-\ref{rr} \ref{moc:parg}  \\
     {\Third.\III}-exchange-{\Prog}-{\Recp} {\DetThree} {\Third.\Poss}-property-{\M}\\
    \glt `They are exchanging their belongings.' \hfill (mocCA191017: 00:44:56)\\ 
     \ex\label{ex:fixedpos1b}
     \glll *n-meːn-\textbf{taʔ}-\textbf{teɡ} na l-owen-ek \\
    \ref{sap}-\ref{moc:core}-\ref{rr}-\ref{moc:asp} \ref{moc:parg}  \\
    {\Third.\III}-exchange-{\Prog}-{\Recp} {\DetThree} {\Third.\Poss}-property-{\M}\\
     \glt `They are exchanging their belongings.' \\
    \z
 \z
 
 Similarly, the non-permutable property of elements is observed up to the enclitic \textit{=oʔ} {`\Tprl/\Evid?'}. Here I focus on this latter element, which indicates the edge of the larger spans \ref{moc:neg}-\ref{encl}. This element occupies the final position of the predicate, like other examples analyzed before (e.g., (\ref{ex:maxmimum1}), (\ref{ex:maxsubspan1}), and (\ref{ex:linking1})), and it is not found in any other position within the verbal planar structure. The marker \textit{=oʔ} is analyzed as an enclitic, as it is one of the two elements in Mocoví's grammar, showing a great deal of flexibility to combine with different hosts, e.g., nouns, verbs and noun-like units (see  \citealt{Juarez2022} for a recent analysis of this suffix in Northern Chaco Mocoví). Regardless of the host, the enclitic occurs always to the right end of the host constituent. Its meaning, however, expresses a fuzzy notion linking the domains of evidentiality and temporality. For example, in (\ref{ex:fixedend}) \textit{=oʔ} is related to the temporal domain, as the enclitic indicates that events occurred in a remote time compared to the moment of speech. 
 
 \ea\label{ex:fixedend}
 \glll ... so tapiɲik i-waɢan-git=\textbf{oʔ} naʔɢa qopaq so l-jaːle-qa...\\
 {} \ref{sapadv} {} \ref{sap}-\ref{moc:core}-\ref{trlocdir}-\ref{encl} {} \ref{moc:parg} \\
 {} {\DetTwo} armadillo {\Third.\II}-hit-{\LocThree}-{\Evid/\Tprl}? day three {\DetTwo} {\Third.\Poss}-descendent-{\Pl}\\
 \glt `...the armadillo hit his sons against the tree...' 
 \hfill (mocCA191025: 00:17:21)
\z 

The main fracture of this test is motivated by elements in position \ref{trlocdir}, which display permutability properties and delimit the subspan \ref{moc:neg}-\ref{trlocdir}. A property that distinguishes this fracture is that one can observe the flexible order of elements by comparing the behaviour of the same element with different types of predicates and other elements. Let us consider the suffix \textit{-a}, one of the multiple forms of the transitivizer \textit{-aʔa}, \textit{-aʔ}. This element shows a range of functions. With intransitive predicates, it increases the verb valency by one, like an applicative, adding a P argument. The semantic role of the added argument varies, including \textsc{location}, \textsc{goal}, \textsc{instrument}, etc. With transitive predicates, the transitivizer appears to indicate that the verb base is transitive and does not increase the verb valence. In these two contexts, the common denominator is a broad notion of transitivity.
 
 The transitivizer can occupy two different positions and combine with, at least, two grammatically different suffixes, directionals and locatives. One position is illustrated in the contrast of (\ref{ex:trvz0}). Here the suffix \textit{-a} occurs next to the root, preceding the locative \textit{-lek}, as in (\ref{ex:trvz2}). While \textit{-a} entails the presence of another argument, \textit{-lek} indicates the path of the movement, which in English can be rendered as `over'.\footnote{I analyze \textit{-a} as an instance of the transitivizer rather than as an epenthetic vowel. This vowel here does not follow any of the vowel insertion rules described for Mocoví and is coincidentally the same vowel that occurs in other contexts, where \textit{-a} clearly adds a P argument.}
 
\ea \label{ex:trvz0}
\ea \label{ex:trvz1}
\glll jim ɲ-noɾ-ʃiɡem \\
    \ref{sapadv} \ref{sap}-\ref{moc:core}-\ref{trlocdir}\\
      {\First\Sg.\Pron} {\First.\III}-jump-{\Dir}:{\Up}\\
\glt `I jump (in the same place).'
\hfill (mocCA120626: 01:18:21)
\ex \label{ex:trvz2}
\glll ajim s-not-\textbf{a}-lek so n-po-ɢan-ɢat\\     
 \ref{sapadv} \ref{sap}-\ref{moc:core}-\ref{trlocdir}-\ref{trlocdir} \ref{moc:parg}\\
      {\First\Sg.\Pron} {\First.\II}-jump-{\Trvz}-{\LocOne} {\DetTwo} {\Poss.\Ind}-close-{\Vm:\Intr}-{\Ins}\\
\glt `I jumped over the fence.'
\hfill (mocCA120626: 01:20:16)
\z
\z

The other possible position that \textit{-a} can occupy is after the directional \textit{-ʃiɡem}, as in the contrast of (\ref{ex:trvz01}). The specific function of \textit{-a} in these examples is not easy to delimit. It indicates, on the one hand, the end point of the climbing event, as noted in the translation, but, on the other hand, also increases the transitivity of the clause. Note that the nominal phrase \textit{so qopaq} `the tree' can no longer receive the oblique marking \textit{ke-}, as in (\ref{ex:trvz4}).    

\ea \label{ex:trvz01}
\ea \label{ex:trvz3}
\glll qamiɾ k-ir-ʃiɡem ke-na qopaq \\
    \ref{sapadv} \ref{moc:core}-\ref{2and1pl}-\ref{trlocdir} \ref{moc:obl}\\
      {\Second\Sg.\Pron} move-{\Second\Sg.\II}-{\Dir:\Up} {\Obl}-{\DetThree} tree\\
\glt `You climbed the tree.'
\hfill (mocCA120717: 00:36:28)
\ex \label{ex:trvz4}
\glll qamiɾ k-iɾ-ʃiɡem-\textbf{a} so qopaq \\
  \ref{sapadv} \ref{moc:core}-\ref{2and1pl}-\ref{trlocdir}-\ref{trlocdir} \ref{moc:parg}\\
      {\Second\Sg.\Pron} move-{\Second\Sg.\II}-{\Dir:\Up}-{\Trvz} {\DetTwo} tree\\
\glt `You climbed to the top of the three.'
\hfill (mocCA120717: 00:37:45)
\z
\z

As shown in these previous examples, the two possible orders of the transitivizer are found with (i) two different predicates and (ii) two different suffixes, i.e., a locative marker in (\ref{ex:trvz2}) and a directional in (\ref{ex:trvz4}). It is not clear yet whether the predicate or each of these suffixes constraint the position of the transitivizer element. What is empirically observable, nevertheless, is that the relative order of the transitivizer is not possible with the same predicate. Thus, only one position is possible for the transitivizer depending on those two internal variables, predicate and locative/directional types.

Other elements with a variable order are also found on the right side of the core element. This is the case for the locative \textit{-ɡi} and the directional \textit{-ɲi}. The basic locative meaning of \textit{-gi} can be rendered as `inside', or `the interior part of an object'. However, as shown in the following examples, the meaning of this element can be broader and less clearly related to its locative component. The meaning of \textit{-ɲi}, on the other hand, can be translated as `down', but this suffix also shows some idiosyncratic meanings with certain verbs. The two possible combinations of these suffixes are \textit{-ɲi}-\textit{ɡi} and \textit{-ɡi}-\textit{ɲi}. Each of these combinations, however, is linked  to specific types of verbs, as both of them are not attested with the same predicate. 

In (\ref{ex:cutlocdir}), the suffixes \textit{-ɡi} and \textit{-ɲi} combine with the predicate \textit{-tʃaq} `cut'. Note that they can occur independently as the last segments of the verb structure, as in (\ref{ex:endloc}) or (\ref{ex:enddir}). The combination of both elements is also possible, but only in the order \textit{-ɲi-ɡi}, as in (\ref{ex:enddirloc}).\footnote{Note that the combination of these suffixes also changes the verb meaning. The locative and directional appear to describe metaphorically the denoted meaning. Thus, the event of fornication could be roughly described as the act of cutting somebody internally in a lower part of somebody's body.} The reverse order is ungrammatical (\ref{ex:endlocdirb}).

\ea {\label{ex:cutlocdir}}
\ea[] {\label{ex:endloc}
\glll i-tʃaq-\textbf{ɡi}  \\
    \ref{sap}-\ref{moc:core}-\ref{trlocdir}\\
      {\Third.\II}-cut-{\LocThree}\\
\glt `She/He cut it accidentally.' \hfill (mocCA210803\_1: 00:34:32) }
\ex[]{\label{ex:enddir}
\glll i-tʃaq-\textbf{ɲi}  \\
    \ref{sap}-\ref{moc:core}-\ref{trlocdir}\\
      {\Third.\II}-cut-{\Dir:\Down}\\
\glt `She/He cut it deeply.'  \hfill (mocCA210803\_1: 00:35:12)}
\ex[]{\label{ex:enddirloc}
\glll i-tʃaq-\textbf{ɲi}-\textbf{ɡi}  \\
    \ref{sap}-\ref{moc:core}-\ref{trlocdir}-\ref{trlocdir}\\
      {\Third.\II}-cut-{\Dir:\Down}-{\LocThree}\\
\glt `She/He fornicates with him/her.'  \hfill (mocCA210803\_1: 00:36:09)}
\ex[*]{\label{ex:endlocdirb}
\glll  i-tʃaq-\textbf{ɡi}-\textbf{ɲi}  \\
     \ref{sap}-\ref{moc:core}-\ref{trlocdir}-\ref{trlocdir}\\
       {\Third.\II}-cut-{\LocThree}-{\Dir:\Down}\\
\glt `She/He fornicates with him/her.'  \hfill (mocCA210803\_1: 00:36:42)}
\z
\z

Although the sequence \textit{-gi-ɲi} is ungrammatical with a verb like \textit{-tʃaq} `cut', it is possible with a verb like \textit{iʔmaqata} `feel good', as in (\ref{ex:seqlocdir}). Like the example above, the suffixes can occur independently, as in (\ref{ex:endlocb}) or (\ref{ex:enddirb}), but also in combination (\ref{ex:enddirlocb}). The meanings that each of these suffixes convey with this predicate are not appropriately captured by the translation of these examples. These suffixes appear to serve as elements that help to describe the stative meaning of the verb root, but when they are combined, the verb meaning is modified, expressing a gradable reading that emphazises the root meaning similar to `very' or `really' in English.   


\ea \label{ex:seqlocdir}
\ea \label{ex:endlocb}
\glll i-ʔmaqata-\textbf{ɡi}  \\
    \ref{sap}-\ref{moc:core}-\ref{trlocdir}\\
      {\Third.\II}-feel.good-{\LocThree}\\
\glt `She/He is happy, feels optimistic.'  \hfill \citep[211]{Buckwalter1995}\\
\ex \label{ex:enddirb}
\glll i-ʔmaqata-\textbf{ɲi}  \\
    \ref{sap}-\ref{moc:core}-\ref{trlocdir}\\
      {\Third.\II}-feel.good-{\Dir:\Down}\\
\glt `She/He is ready, gets better.'  \hfill \citep[][211]{Buckwalter1995}\\
\ex\label{ex:enddirlocb}
\glll i-ʔmaqata-\textbf{ɡi}-\textbf{ɲi}  \\
    \ref{sap}-\ref{moc:core}-\ref{trlocdir}-\ref{trlocdir}\\
      {\Third.\II}-feel.good-{\LocThree}-{\Dir:\Down}\\
\glt `She/He feels really good.' \hfill (mocCA160711\_3: 00:36:36) \\
\z
\z

I have not yet test whether the sequence \textit{-ɲi-ɡi} is ungrammatical with this verb, but data suggest that such a sequence might not be possible. For instance,  \citet[]{Buckwalter1995} did not  document the sequence \textit{-ɲi-gi} with `cut', although he listed other verb forms combined with locative and directionals. Also, the sequence \textit{ɲi-ɡi} is not present in my corpus, including elicited and natural speech data.  

\subsection{Deviation from bi-uniqueness: span \ref{sap}-\ref{moc:core} and \ref{sap}-\ref{encl} \label{sec:deviation}}
Deviation from bi-uniqueness, i.e., a deviation from a one-to-one correspondence between form and meaning, is a common property of morphological and, to some extent, of lexical elements in Mocoví. 

A common example of multiple forms related to one meaning is found in the third-person bound forms, e.g., \textit{n-} (\ref{ex:manytoone1}), \textit{i-} (\ref{ex:manytoone2}), \textit{∅-} (\ref{ex:manytoone3}) and \textit{ɾ-} (\ref{ex:manytoone4}) (examples are re-adapted from \citealt[322]{Juarez2021}; see also \citealt{Juarez2013}). These third-person bound forms are part of the Set I paradigm and the selection of almost all of these prefixes is lexically conditioned. Only the prefix \textit{ɾ-} is conditioned by the transitivity of the base, as it occurs with intransitive predicates and is required for detransitivized verbs \citep[see][]{Juarez2017}. The multiple forms of the third person occur in position \ref{sap} and identify a minimal span of positions of two, \ref{sap}-\ref{moc:core}. 

\ea \label{ex:manytoone}
\ea \label{ex:manytoone1}
\glll \textbf{n}-esal  \\
    \ref{sap}-\ref{moc:core}\\
      {\Third.\I}-vomit\\
\glt `She/He vomits.'  \hfill (mocCA110711: 00:10:29)
\ex \label{ex:manytoone2}
\glll \textbf{i}-ʔloɢol  \\
    \ref{sap}-\ref{moc:core}\\
      {\Third.\I}-tremble\\
\glt `She/He trembles.'  \hfill (mocCA120716: 01:16:03)\\
\ex \label{ex:manytoone3}
\glll ∅-qopat  \\
    \ref{sap}-\ref{moc:core}\\
      {\Third.\I}-be.hungry\\
\glt `She/He is hungry.'  \hfill (mocCA120713: 00:03:48)
\ex \label{ex:manytoone4}
\glll \textbf{ɾ}-apil  \\
    \ref{sap}-\ref{moc:core}\\
      {\Third.\Intr.\I}-come.back\\
\glt `She/He comes back.'  \hfill (mocCA120711: 00:14:00)
\z
\z

Another example of deviation from bi-uniqueness is illustrated with the multiple forms of aspectual markers. As \citet[246--255]{Gualdieri1998} noted for Southern Chaco Mocoví, aspectual markers show variation conditioned by the grammatical person of the subject or the object. Here I only focus on the progressive marking conditioned by the subject person, but readers are referred to Gualdieri's work for more details on this topic. 

The progressive aspect in Northern Chaco Mocoví also shows two forms for one meaning, the suffixes \textit{-tak} and \textit{-sak}, as in (\ref{ex:aspectsalter}). They identify a subspan of the planar structure that includes positions \ref{sap}-\ref{moc:asp}. The distribution of these two suffixes is as follows. The suffix \textit{-tak} occurs when the A or S arguments are first (\ref{ex:asp1st}) or third person (\ref{ex:asp3rd}), but \textit{-sak} only occurs when A or S are a second person (\ref{ex:asp2nd}).    

\ea\label{ex:aspectsalter}
\ea \label{ex:asp1st}
\glll s-keʔe-\textbf{tak}\\
     \ref{sap}-\ref{moc:core}-\ref{moc:asp}\\
      {\First.\II}-eat\textsubscript{1}-{\Prog}\\
\glt `I'm eating something.' \hfill (mocCA120706: 00:21:03)
\ex \label{ex:asp2nd}
\glll qamiɾ keʔe-i-\textbf{sak}\\
     \ref{sapadv} \ref{moc:core}-\ref{2and1pl}-\ref{moc:asp}\\
     {\Second\Sg.\Pron} eat\textsubscript{1}-{\Second.\Sg.\II}-{\Prog}\\
\glt `You are eating something.'/ `Are you eating?'/ `Eat.' \\
\glt \hfill (mocCA120706: 00:22:43)
\ex \label{ex:asp3rd}
\glll so noɡot-okiʔ ∅-keʔe-\textbf{tak} pan \\
     \ref{sapadv} {} \ref{sap}-\ref{moc:core}-\ref{moc:asp}\\
     {\DetTwo} adolescent-{\Dim.\M} {\Third.\II}-eat\textsubscript{1}-{\Prog} bread\\
\glt `The child is eating bread.' \hfill (mocCA160726: 00:03:02)
\z
\z 

Other elements also show deviation from bi-uniqueness and identify different subspans. Thus, the maximum subspan of positions defined according to this test is \ref{sap}-\ref{encl}. As mentioned earlier, the tansitivizer in position \ref{trlocdir} has multiple forms for different but semantically related meanings. Similarly, the enclitic \textit{=oʔ} represents a single formative with fuzzy meanings related to temporal discourse and evidentiality.  

\subsection{(Non)-ciscategorial selection: span \ref{moc:core}-\ref{2and1pl}}
\label{sec:nonciscategorial}

Ciscategorial selection refers to elements that exclusively combine with one part of speech, e.g., nouns or verbs. In other words, we explore the extent to which elements of the planar structure are constrained by a specific part of speech. Some elements in the planar structure can be categorized as ciscategorial whereas other are transcategorial, which means that they occur with both verbs and nouns. The minimal/maximal fracture identifies the span overlapping the verb core that only contains verb-specific elements. 

In this test, positions \ref{moc:core} to \ref{vmII} represent both the minimal and maximal span of positions that only include ciscategorial elements. The reason for this analysis is that other positions preceding and following that subspan can include both ciscategorial and transcategorial elements. For example, on the left edge of such subspan, elements such as the negative and the subject defocusing markers in positions \ref{moc:neg}-\ref{sdef} do not occur with nouns. However, some bound person forms in position \ref{sap} can be categorized as transcategorial whereas others are clearly ciscategorial. Compare, for instance, the bound person forms from \autoref{tab: bound person forms paradigms} and the possessive markers in \autoref{tab:NCMpossessiveparadigms}.

\begin{table}
\captionof{table}{Bound person form paradigms in positions \ref{sap} and \ref{2and1pl}}
\label{tab: bound person forms paradigms}
\begin{tabular}{cccc}
\lsptoprule
\textbf{Person \& Number} &  \textbf{Set I }&   \textbf{Set II} & \textbf{Set III }         \\\midrule
\textsc{1sg} &   \textit{dʒ-} &  \textit{s-} & \textit{ɲ-}\\
\textsc{1pl} &   \textit{qaɾ-} & \textit{s-...-ɢ}& \textit{ɲ-...-ɢ}\\
\textsc{2sg} &\textit{ɾ-...-iɾ} &\textit{-iɾ} & \textit{n-...-iɾ}\\
\textsc{2pl} &  \textit{ɾ-...-i} $\sim$\textit{ɾ-...-iː} &  \textit{-i} $\sim$ \textit{-iː} & \textit{n-...-i} \\
\textsc{3} & \textit{i-} & \textit{i-} &\textit{n-} \\
 & \textit{ɾ-} &  \textit{ɾ-} &\\
 & \textit{∅-} &   \textit{∅-} &\\
  & \textit{n-}&  \textit{t-} &\\
\lspbottomrule
\end{tabular}
\end{table}
%

\begin{table}
\captionof{table}{Northern Chaco Mocoví possessive paradigms}
\label{tab:NCMpossessiveparadigms}
\begin{tabular}{ccccccc} 
\lsptoprule
\textbf{Person \& Number}& \multicolumn{4}{c}{\textbf{Inalienable}} & \textbf{Alienable} &\\\midrule
\textsc{\textbf{1sg}}	&	 \multicolumn{4}{c}{\textit{i-}}	&	ɲ- &\\
\textsc{\textbf{1pl}}	&\textit{	qod-} &			&	&		\textit{qaɾ- }		& \textit{qan-} &\\
\textsc{\textbf{2sg}}	&	\textit{qad-...-iɾ}& 	&& 	\textit{ɾ-...-iɾ}		&	 \textit{n-...-iɾ	} &\\
\textsc{\textbf{2pl}} 	& \textit{qad-...-iː} 	& & & 	\textit{ɾ-...-i}		&	\textit{n-...-i} &\\
\textsc{\textbf{3}}		&	\multicolumn{4}{c}{\textit{l-}}	&\textit{ n-} &\\
\textsc{\textbf{indet}} & \multicolumn{4}{c}{\textit{n-}}&  &\\	
\lspbottomrule
\end{tabular}
\end{table}

 We have bound person prefixes that only occur with verbs in position \ref{sap}, e.g., \textit{r-} `\Third.\Intr' or \textit{s-} `\First.\II', but other bound person forms are formally similar to nominal morphology. The prefix \textit{i-} `\Third.\II', for instance, is identical to the possessive marker \textit{i-} `\First\Sg.\Poss' in nouns like \textit{i-komeena} `my grandmother' and \textit{i-taʔa} `my father'. However, they clearly encode two different grammatical persons, i.e., third and first, and neither of them can be used to encode both nominal or verbal arguments. That is, the verbal \textit{i-} cannot encode third-person possessor nor the nominal \textit{i-} can encode first-person A or S. Thus, both of these elements are transcategorial and delimit the left edge of the transcategorial span.


Furthermore, note the paradigmatic similarity between the Set III and the alienable paradigm. These two paradigms only differ in the expression of the first-person plural. To illustrate the similarity between these two paradigms consider, for example, the uses of the prefix \textit{n-}. It encodes a third-person S or A argument (\ref{ex:policverb}) with verbs, but an undetermined possessor with some derived and non-derived nouns (\ref{ex:policnoun}). Semantically, these two meanings are not exactly identical, as they only share the third-person referentiality. There is no implication of an `indeterminate' argument when \textit{n-} is used with verbs. 

\ea
\ea \label{ex:policverb}
\glll \textbf{n}-oɢon-ʃiɡem=oʔ so iːmek\\
\ref{sap}-\ref{moc:core}-\ref{trlocdir}-\ref{encl} \ref{moc:parg} \\
{\Third.\III}-construct-{\Dir:\Up}={\Tprl/\Evid}? {\DetTwo} house\\
\glt `She/He built the house.' \hfill (mocCA120706ː 01ː01ː57)
\ex\label{ex:policnoun}
\gll na \textbf{n}-oʔwen-aɢa ʔwe ɾa \textbf{n}-waɡ-aɢa\\
{\DetThree} {\Poss.\Ind}-work-{\Nmlz} {\Exist} {\DetFour} {\Poss.\Ind}-be.calm-{\Nmlz}\\
\glt `The field is quiet.' \hfill(mocCA120706ː 01ː17ː55)
\z
\z

The right edge of the ciscategorial span is at position \ref{vmII}, because the person markers in \ref{2and1pl} are transcategorial. Consider the bound person form \textit{r-...-iɾ} in (\ref{ex:non-cis}). The same marker encodes the second person possessor in (\ref{ex:non-cisposs}) and the S\textsubscript{\textsc{p}} argument in (\ref{ex:non-cisverb}).\footnote{Northern Chaco Mocoví displays a split at the encoding of S arguments: S\textsubscript{\textsc{A}}=A and S\textsubscript{\textsc{P}}=P. The S\textsubscript{\textsc{P}}=P alignment is, however, restricted to intransitive Speech Act Participants and P arguments in a transitive scenario such as 3 $\rightarrow$ SAP (see further details in \citealt[chap.3]{Juarez2013}).}

\ea \label{ex:non-cis}
\ea \label{ex:non-cisposs}
\glll wet-iɾ na \textbf{ɾ}-aqaiɡ-\textbf{iɾ}\\
    \ref{moc:core}-\ref{2and1pl}  \ref{moc:parg} \\
    feel.pain-{\Second.\II} {\DetThree} {\Second\Sg.\Poss}-head-{\Second\Sg.\Poss}\\
\glt `Does your head hurt?' or `Your head hurts.'\\
\hfill (mocCA120626: 01:33:42)
\ex\label{ex:non-cisverb}
\glll qamiɾ  \textbf{ɾ}-oqopat-\textbf{iɾ}\\
    \ref{sapadv} \ref{sap}-\ref{moc:core}-\ref{2and1pl}  \\
    {\Second\Sg.\Pron} {\Second\Sg.\I}-be.hungry-{\Second\Sg.\I}\\
\glt `Are you hungry?' or `Your are hungry.'
\hfill (mocCA120713: 00:03:28)
\z
\z

After position \ref{2and1pl}, the planar structure includes ciscategorial and transcategorial elements.  In positions \ref{moc:asp}-\ref{des} elements are only attested with verbs, however, the suffix \textit{-taʔ} in position \ref{rr} represents another transcategorial element. This suffix occurs with both verbs and nouns. Its general meaning could be captured under the synchronic rubric of `interaction among plural participants', a function that is cross-linguistically associated with pluractional constructions (\citealt[21--40]{Mattiola2019}; see also \citealt[chap. 6]{Juarezb}). In the verbal domain, \textit{-taʔ} is part of reciprocal constructions, as in (\ref{ex:policattaʔvb}), whereas in the nominal domain, it simply indicates nominal plurality, as in (\ref{ex:policattaʔnb}).
However, this suffix is not obligatory. 

\ea
\ea \label{exːpolicattaʔva}
\glll so jale jim i-waɢan\\
    \ref{sapadv} {} \ref{sap}-\ref{moc:core}\\
   {\DetTwo} man {\First.\Sg.\Pron} {\Third.\II}-hit\\
\glt `The man hit me.' \hfill (mocCA160725: 01:23:11)
\ex \label{ex:policattaʔvb} 
 \glll dʒi jale-ɾ n-waɢan-teɡ-\textbf{taʔ}\\
    \ref{sapadv} {} \ref{sap}-\ref{moc:core}-\ref{moc:asp}-\ref{rr}\\
   {\DetTwo} man-{\Pl} {\Third.\III}-hit-{\Prog}-{\Recp}\\
\glt `The men  are hitting each other.' \hfill (mocCA191017\_3: 00:37:59) 
\z
\z

\ea
\ea \label{exːpolicattaʔna}
\glll ∅-ʔom na naʔɢa\\
    \ref{sap}-\ref{moc:core} \ref{moc:parg} \\
   {\Third.\II}-be.cold {\DetThree} day\\
\glt `The day is cold.' \hfill (mocCA160712: 01:26:10)
\ex \label{ex:policattaʔnb} 
 \gll na naʔɢa-\textbf{taʔ} naɡi n-ʔom-aɢa\\
  {\DetThree} day-{\Pl} now {\Ind.\Poss}-be.cold-{\Nmlz} \\
\glt `These are the days of coldness.' \hfill (mocCA160712: 01:27:06)
\z
\z

 Further down in the planar structure, elements in positions \ref{trlocdir} and \ref{pllo} have  been identified only with verbs. Finally, elements in positions \ref{moc:dim} and \ref{encl} are transcategorial. In a recent chapter, \citet{Juarez2022} showed the different nominal and verbal contexts in which these two elements occur, indicating that \textit{=oʔ} is one of the most flexible elements in the language (in the sense of \citealt{Bisang2013,VanLier2013a}).      


\subsection{Morphosyntactic tests: interim results}
\begin{figure}[b]
    \centering
   \includegraphics[width=\textwidth]{figures/mocovi_ms_plot.png}
    \caption{Results from morphosyntactic tests}
    \label{fig:morphresults}
\end{figure}
\autoref{fig:morphresults} summarizes the results of morphosyntactic tests applied to the planar structure. Like the phonological domain, tests do not identify a single and unique (sub)span of positions. 



We do find test results that overlap in selecting a subspan, \ref{sap}-\ref{moc:parg} or \ref{moc:neg}-\ref{encl}, but not all the tests select the same span. Therefore, in this particular domain the emergence of a unique string of positions as a word is not clearly  defined. 


\section{Major findings and implications} \label{sec:finalremarks}

\begin{figure}[b]
    \centering
   \includegraphics[width=\textwidth]{figures/mocovi_pooled_plot.png}
    \caption{Results from phonological and morphosyntactic tests applied to Northern Chaco Mocoví}
    \label{fig:mocovítests}
\end{figure}

The main motivation of this chapter was to assess whether the category of ``word'' can be defined for Norhthern Chaco Mocoví. Building on the novel methodology for constituency delineated by \citet{Tallman2020}, I analyzed the verbal structure by testing properties of the phonological and morphosyntactic domains.

\autoref{fig:mocovítests}, presents all the diagnostics applied in this work, organized by span size and color-coded for domains. Not all tests select the same span of positions, which indicates that a verbal word cannot be categorically recognized. There are, however, convergences between the minimally and maximally selected subspans as well as the edges of multiple tests. The span \ref{sap}-\ref{moc:core} aligns across diagnostics and domains. These two positions include essential elements of the verbal structure and discourse information (namely, who does what), as they include the verb root and one of the core arguments of the clause. The other subspans that are recurrently selected include \ref{moc:neg}-\ref{encl} and \ref{sap}-\ref{trlocdir}, but they show less preference compared to the minimum subspan \ref{sap}-\ref{moc:core}.



There are also recurrent positions that are taken as edges of different tests. For example, twelve diagnostics align in position \ref{sap} and six end at position \ref{moc:core} and \ref{encl}. These results indicate a preference for specific limits that verbal constituents can reach and reinforce the presence of minimal and maximal ends of constituents.  

If the notion of ``word'' would be understood as a gradient constituent, the minimally and maximally  selected subspans, \ref{sap}-\ref{moc:core} and \ref{moc:neg}-\ref{encl},  can be interpreted as the best exemplars of such continuum, whereas the other subspans represent in-between instances of  a word-like constituent.


\section*{Acknowledgements}
My deepest gratitude to all Mocoví and Toba community members in Colonia Aborigen, especially to Héctor José, Nieves José, Ramón Tomás, Dario José, Nanci Oliva, José Oliva, Andrea José, and many more who kindly receive me in the field every year I visit Colonia Aborigen. This chapter would not have been possible without the support from institutions and colleagues. Special thanks go to Adam Tallman, Sandra Auderset and Hiroto Uchihara for their encouragement and patience throughout the writing of this chapter. This research was also supported by ELDP (IGS0365) and NSF-DEL-DDRIG (1918028) grants to Cristian Juárez.

\largerpage[-2]
\printglossary

\printbibliography[heading=subbibliography,notkeyword=this]

\end{document}
