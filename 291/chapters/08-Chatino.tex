\documentclass[output=paper]{langscibook}
\ChapterDOI{10.5281/zenodo.13208554}
\author{Eric W. Campbell\affiliation{University of California Santa Barbara}}
\title{Constituency in Zenzontepec Chatino}
\abstract{This chapter presents morphosyntactic and phonological constituency tests in the Zenzontepec Chatino language of southern Mexico, with data and analysis based almost entirely on a corpus of language use of varied genres. The language displays a notably wide range of segmental and suprasegmental sound patterns, some of which have strong convergences around the verbal lexical core, and which in turn align with some of the morphosyntactic tests. In light of recent typological and descriptive work on wordhood (including many contributions to this volume) that shows constituency tests do not tend to converge as they are emergent in dynamic language use and change, Zenzontepec Chatino presents a notable case as a language with a relatively strongly motivated word constituent.}
\IfFileExists{../localcommands.tex}{
  \addbibresource{../collection_tmp.bib}
  \addbibresource{../localbibliography.bib}
  % add all extra packages you need to load to this file

\usepackage{tabularx,multicol}
\usepackage{url}
\urlstyle{same}

\usepackage{listings}
\lstset{basicstyle=\ttfamily,tabsize=2,breaklines=true}

\usepackage{langsci-basic}
\usepackage{langsci-optional}
\usepackage{langsci-lgr}
\usepackage{langsci-osl}
% \usepackage{./langsci/styles/langsci-lgr}
% \usepackage{./langsci/styles/langsci-osl}
% \usepackage{langsci-gb4e}

\usepackage{tikz}
\usetikzlibrary{patterns,calc}
\pgfdeclarepatternformonly{south east lines}{\pgfqpoint{-0pt}{-0pt}}{\pgfqpoint{3pt}{3pt}}{\pgfqpoint{3pt}{3pt}}{
    \pgfsetlinewidth{0.6pt}
    \pgfpathmoveto{\pgfqpoint{0pt}{3pt}}
    \pgfpathlineto{\pgfqpoint{3pt}{0pt}}
    \pgfpathmoveto{\pgfqpoint{.2pt}{-.2pt}}
    \pgfpathlineto{\pgfqpoint{-.2pt}{.2pt}}
    \pgfpathmoveto{\pgfqpoint{3.2pt}{2.8pt}}
    \pgfpathlineto{\pgfqpoint{2.8pt}{3.2pt}}
    \pgfusepath{stroke}}
    
\usepackage{stmaryrd}
\usepackage{wasysym}
\usepackage{multirow}
\usepackage{caption}
\usepackage{subcaption}
\usepackage{mathrsfs}
\usepackage{qtree}

\usepackage{linguex}


  %pminos do not split footnotes
% \interfootnotelinepenalty=10000 %Footnote in Laporte chapters has to be split SN


%\DeclareIndexNameFormat{default}{%
%\nameparts{#1}%
%\usebibmacro{index:name}%
%{\index[names]}%
%{\namepartfamily}%
%{\namepartgiveni}%
% {}% L1
% {}% L2
%{\namepartprefix}% generates spurious space L3
%{\namepartsuffix}% generates spurious space L4
%}

%  {\DeclareIndexNameFormat{default}{%
%     \usebibmacro{index:name}{\index[names]}{#1}{#3}{#5}{#7}}}

%\DeclareIndexNameFormat{default}{%
%  \usebibmacro{index:name}{\sindex[nom]}{#1}{#3}{#5}{#7}}

%\DeclareIndexNameFormat{default}{%
%  \usebibmacro{index:name}{\sindex[person]}{#1}{#3}{#5}{#7}}
%\DeclareIndexNameFormat{default}{%
%\nameparts{#1} \usebibmacro{index:name}{\sindex[person]]}{\namepartfamily}{‌​\namepartgiven}{\nam‌​epartprefix}{\namepa‌​rtsuffix}}

%\newcommand{\smiley}{:)}

%\renewbibmacro*{index:name}[5]{%
%\usebibmacro{index:entry}{#1}%
%{\iffieldundef{usera}{}{\thefield{usera}\actualoperator}\mkbibindexname{#2}{#3}{#4}{#5}}}

% \newcommand{\noop}[1]{}

%remove for final
%\overfullrule=1mm

\newcommand{\tobi}[2]}}
\renewcommand{\S}[1]{\tobi{#1}{\textsc{*}}}

% this volume references
% puts: [this volume]
% already defined: \citetv
%\newcommand{\citepv}[1]{(\citeauthor{#1} \citeyear*{#1} [this volume])}
\newcommand{\citealtv}[1]{\citeauthor{#1} \citeyear*{#1} [this volume]}

%parentheses around example number
\newcommand{\pref}[1]{(\ref{#1})}

% in-text examples

\newcommand{\lnex}[1]{\textit{#1}} %target lang word
\newcommand{\lnlit}[1]{(lit.: `#1')} %literal reading
\newcommand{\lnlat}[1]{(#1)} % latinization
\newcommand{\lntrans}[1]{`#1'} %translation
\newcommand{\lnexl}[2]%
{\lnex{#1}{} \lnlat{#2}} % ex with latinization
\newcommand{\lnexlat}[3]{\lnex{#1}{} \lnlat{#2}{} \lntrans{#3}} % ex with latinization and tranl.

%ch01
\newcommand{\co}[1]{\mbox{\textbf{#1}}}

%ch09

\newcommand{\cyrbulg}[1]{\begin{otherlanguage*}{bulgarian}#1\end{otherlanguage*}}


%ch10
\newcommand{\nlp}{{\small NLP}}
\newcommand{\mwe}{{\small MWE}}
\newcommand{\rae}{{\small RAE}}
\newcommand{\lvc}{{\small LVC}}
\newcommand{\pos}{{\small P}o{\small S}}
%\newcommand{\todo}[1]{ \textcolor{red}{#1} }

%\renewcommand{\labelenumi}{\theenumi}
%\ainamefmt{{vv}{ll}{, ff}{, jj}} % fullname

\newcommand{\biberror}[1]{{\color{red}#1}}

\newcommand{\osenovaitem}{--~} 
  %% hyphenation points for line breaks
%% Normally, automatic hyphenation in LaTeX is very good
%% If a word is mis-hyphenated, add it to this file
%%
%% add information to TeX file before \begin{document} with:
%% %% hyphenation points for line breaks
%% Normally, automatic hyphenation in LaTeX is very good
%% If a word is mis-hyphenated, add it to this file
%%
%% add information to TeX file before \begin{document} with:
%% %% hyphenation points for line breaks
%% Normally, automatic hyphenation in LaTeX is very good
%% If a word is mis-hyphenated, add it to this file
%%
%% add information to TeX file before \begin{document} with:
%% \include{localhyphenation}
\hyphenation{
    Beck-man
    Ngu-yen
    back-chan-nel
    back-chan-nels
    mo-not-o-nous
    ste-reo-typ-i-cal
}

\hyphenation{
    Beck-man
    Ngu-yen
    back-chan-nel
    back-chan-nels
    mo-not-o-nous
    ste-reo-typ-i-cal
}

\hyphenation{
    Beck-man
    Ngu-yen
    back-chan-nel
    back-chan-nels
    mo-not-o-nous
    ste-reo-typ-i-cal
}
 
  \togglepaper[1]%%chapternumber
}{}

\begin{document}
\maketitle 
%\shorttitlerunninghead{}%%use this for an abridged title in the page headers


\section{Introduction}
\largerpage
This chapter investigates constituency in the Zenzontepec Chatino language, a Zapotecan language of the Otomanguean stock of Mesoamerica (\citealt{Mechling1912}; \citealt{Boas1913}). Following the methodology outlined in Tallman (\citeyear{Tallman2021}, and this volume), the structure of verbal predications is flattened out into a verbal planar structure. No constituency is assumed \textit{a priori}, and language specific constituency tests are applied, and fractured as necessary, to identify possible constituents in the language where the tests might converge.

Zenzontepec Chatino presents an interesting and important case in the empirical study of constituency in human language, for several reasons. First of all, in a previous pilot study that looked at 17 languages of the Americas \citep{Tallman2019Talk}, Zenzontepec Chatino was one of only a few that displayed evidence that could motivate both phonological and grammatical words beyond chance, and additionally, a word-like constituent substantiated by pooling together morphosyntactic and phonological constituency tests. Zenzontepec Chatino also stands out in this project for the plethora of phonological tests -- both segmental and suprasegmental -- that can be observed. Of additional interest is special and illuminating evidence for constituency from a play language, or \textit{ludling} \citep{Laycock1972}, which converges with other sound patterns. Finally, the language is tonal but displays a low tonal density (\citealt{Campbell2014,Campbell2016}), meaning that many tone-bearing units are not specified for tone, and this affords tonal processes that may operate across wide spans in the planar structure, interacting with intonational factors such as declination and pitch reset at the start of new intonational units. Due to the large number of tests identified for the verbal domain and considerations of space, constituency in the nominal domain is not treated here.

The chapter is organized as follows. Some background on the language community and the data are provided in \sectref{bkm:Ref90318129}. The verbal planar structure is introduced in \sectref{bkm:Ref90318196}. Morphosyntactic tests are presented in \sectref{bkm:Ref90318239}, and (morpho)phonological tests in \sectref{bkm:Ref90318262}. Evidence for constituency from the play language is provided in \sectref{bkm:Ref90496744}, followed by a discussion of the findings (\sectref{bkm:Ref90326435}) and a short conclusion (\sectref{bkm:Ref113366717}).

\largerpage
\section{The language and the data}
\label{bkm:Ref90318129}
The Zenzontepec Chatino language is spoken widely in the municipality of Santa Cruz Zenzontepec, and by a few elders in the municipality of Santa María Tlapanalquiahuitl, both located in the district of Sola de Vega in southwestern Oaxaca, Mexico. The language is also spoken in diaspora communities in other parts of Mexico and the United States, especially California. According to the 2020 Mexican national census \citep{INEGI2020}, the municipality of Santa Cruz Zenzontepec has about 19,000 inhabitants, of whom roughly 12,000 people report speaking an indigenous language (presumably Zenzontepec Chatino for most).

The data presented in this study are drawn from a corpus of about 21 hours of transcribed and translated language use \citep{Campbell2012}, supplemented by an analytical lexical database consisting of about 10,000 entries, some with limited elicited or offered examples, from collaborative language documentation and description that is ongoing since 2007. About 20 participants from a handful of villages and of varied age and gender have contributed to the corpus in a range of speech genres, including but not limited to personal narrative, description, conversation, advice-giving, and folklore. While the data do not reflect the full range of genres and uses of the language, they nevertheless present a reasonably broad picture of its structure and use. Following the free translations in interlinear examples are reference words and time positions that registered users of the Endangered Language Archive (ELAR) can use to find most of the passages in their larger discourse context \citep{Campbell2012}.


It is important to highlight that the documentary corpus and usage-based nature of this study limit the amount of evidence that is brought to bear on the analysis via ungrammaticality judgments about constructed examples. This is especially relevant for the morphosyntactic constituency tests, which by nature require testing in addition to observation for determining finer details. Nonetheless, the study shows that with substantial documentation and rigorous analysis even domains of morphosyntactic tests can be fairly precisely delimited. Phonological patterns, on the other hand, lend themselves more to observation in language use and depend less on grammaticality judgements for outlining the details.


\largerpage[2]
\section{Verbal planar structure}
\label{bkm:Ref90318196}
This section presents the verbal planar structure of Zenzontepec Chatino. For the distinction between verbs and other word classes in the language, see \citet{Campbell2014}, but for present purposes, verbs are defined as forms that directly and obligatorily inflect for aspect-mood. Nouns do not inflect for aspect-mood, and they may be followed by forms expressing property concepts that attributively modify them (adjectives), and/or demonstratives. Nouns, but not verbs, may be preceded (and quantified) by numerals and/or articles. If a noun phrase is followed by another noun phrase with a distinct referent (in some cases with prepositional \textit{hiʔį̄}) then that following form (or span) is understood as referring to a possessor of the preceding form, and not as subject or object. On the other hand, a noun or noun phrase that immediately follows a verb will encode the verb's subject, an object, an oblique participant, or an adverbial expression involved in the clause.

The Zenzontepec Chatino verbal planar structure is presented in \tabref{tab:zenz:key:1}.\footnote{The main orthography used in this chapter is an IPA-based phonemic orthography.
Deviations from the IPA are as follows: r = [ɾ]; V̄ = M tone, V́ = H tone, V = no tonal specification on a mora, V̨ = phonologically nasal vowel, VV = phonologically long (bimoraic) vowel, =V(ʔ) = demonstrative enclitic with unspecified vowel quality. As needed, fine phonetic transcriptions that display allophonic realizations, or the lack of such, are provided in square brackets. Textual examples consist of the following lines and conventions. First line: surface phonemic form, showing results of elision and contraction but not allophonic alternations such as H tone spreading, downstep, nasality spreading or palatalization of coronals. Second line: planar structure position numbers (v: verbal planar structure). Third line: underlying phonological representation without allophonic alternations, contractions, or fusion; hyphen (-) represents a prefix, equals sign (=) indicates an enclitic, plus sign (+) precedes the second stem in a compound stem. Fourth line: glosses and grammatical abbreviations, which follow the Leipzig Glossing Rules. Fifth line: free English translation including key words and time stamp of the source media file.} 
The simplex verbal root occurs in the slot in position 13; a compounded stem that derives another verbal lexeme may occur in position 14. Other details about the core of the verbal planar structure near the lexical root are outlined in \sectref{bkm:Ref90318239} via morphosyntactic constituency tests.

\begin{table}
    \centering
    \caption{Zenzontepec Chatino verbal planar structure}
    \label{tab:zenz:key:1}
         \begin{tabular}{Slp{9cm}}
\lsptoprule
\multicolumn{1}{l}{Positions} & Type & Elements \\ \midrule
\label{zen1con} & Slot & Conjunction\\
\label{zen2np} & Zone & NP \{A, S, P\}\\
\label{zen3adv} & Zone & Adverbial (\textit{niī}, others)\\
\label{zen4con} & Slot & Mood (Conditional \textit{tī}, Interrogative \textit{ʔā})\\
\label{zen5md} & Zone & Mood (Negation \textit{ná}, \textit{naʔā}, \textit{nītsáʔ}, \textit{wílā} Hypothetical \textit{tu}, Assertive \textit{tala}, \textit{ta})\\
\label{zen6adv} & Zone & Adverbial (\textit{tíʔ}, \textit{tʲāʔ})\\
\label{zen7am} & Slot & Aspect-mood\\
\label{zen8tr} & Slot & Transitivity\\
\label{zen9aux} & Slot & Auxiliary\\
\label{zen10am} & Slot & Aspect-mood\\
\label{zen11cau} & Slot & Causative \textit{u-}, Iterative \textit{i-}\\
\label{zen12tr} & Slot & Transitivity \textit{s-}, \textit{t-}, \textit{j-}\\
\label{zen13base} & Slot & Verb root\\
\label{zen14com} & Slot & Compound stem (any lexical class), Applicative \textit{lóʔō }\\
\label{zen15adv} & Zone & Adverbial\\
\label{zen16ess} & Slot & Essence \textit{=tīʔ}, \textit{=rīké}\\
\label{zen17sub} & Slot & Subject NP (A, S)\\
\label{zen18np} & Zone & NP \{P, T, R\}\\
\label{zen19adv} & Zone & Adverbial\\
\label{zen20ten} & Slot & Tense\\
\label{zen21dm} & Zone & Discourse marker, Adverbial\\
\lspbottomrule
    \end{tabular}
\end{table}


\hspace*{-2.5pt}Establishing the verbal planar structure of Zenzontepec Chatino in this methodology posed some challenges. First of all, aspect-mood inflection is partly prefixal and partly expressed by tone melody alternations (or lack thereof) on verb stems \citep{Campbell2019}. This nonconcatenativity and deviation from biuniqueness does not fit neatly into a discrete linear model. Second, most verbs of emotion and cognition are idiosyncratically formed by combining a verb stem in position(s) 13(+14) with an “essence” element (\citealt{CruzStump2018}), like a body part, in position 16: \textit{=tīʔ} `living core' or \textit{=rīké} `chest' \citep{Campbell2015}. However, the two components of these verbal lexemes are not contiguous in the planar structure when optional adverbial elements intervene, while other multi-stem verbal lexemes consist of rigidly contiguous lexical elements in positions 13+14. Thus, the verbal “lexeme” does not have a consistent and non-interruptible span in the planar structure. Third, there is some variable ordering between the adverbial zone in position 15 and the “essence” elements in position 16. The details of this ordering and its relevance for constituency are discussed in \sectref{bkm:Ref90326435}. Fourth, subject person marking occurs in position 17, but there are nonconcatenative processes that do not fit neatly and discretely into our linear planar structure model: (i) 2\textsuperscript{nd}{}-person singular is marked by tonal ablaut on the preceding element, which could be in position 13, 14, 15 (adverbial) or 16 (essence), and (ii) vowel-initial subject markers in position 17 fuse with some verb stems (or the immediately preceding element in the same positions just mentioned).

Although the language displays fairly flexible constituent order \citep{Campbell2021a}, this plays out largely in the zones in positions 2 and 18, where nominal spans occur in varied orders or varied positions with respect to the verb. Finally, some finer detailed work remains to be done with respect to any different meanings or implicatures that arise from the variable placement of adverbials in positions 3, 15, 19, and 21.

An important fact about the language is that there is only one set of pragmatically neutral short pronouns, and a parallel set of longer, mostly bimoraic pronouns, whose use is for emphasis or disambiguation. Setting aside contractions in natural speech that can be unpacked \citep{Campbell2014}, there are no distinct forms of pronouns for different grammatical relations: the role of a participant expressed by a pronoun or any NP is conveyed by context and by its position in the clause with respect to other elements, or unexpressed for topical 3rd persons (zero anaphora).


\section{Morphosyntactic tests}
\label{bkm:Ref90318239}
The morphosyntactic constituency tests that have so far been identified for the Zenzontepec Chatino verbal domain include free occurrence (\sectref{bkm:Ref90386630}), non-per\-mu\-tability (\sectref{bkm:Ref90386650}), non-interruptability (\sectref{bkm:Ref113137532}), and coordination (or subspan repetition, \sectref{bkm:Ref113137621}).

\subsection{Free occurrence}
\label{bkm:Ref90386630}
In this study “free occurrence” in the verbal domain is defined as the minimal span of positions (in the verbal planar structure) that must occur to form a complete utterance (i.e., to be produced in isolation) and which includes the verb root. However, the span that may function holophrastically depends on the person and topicality of the subject of the verb, and also whether the verb occurs with a lexicalized auxiliary. Thus, this constituency test is fractured in Zenzontepec Chatino.

\subsubsection{Free occurrence (minimal), positions 10{}--13} 
\label{bkm:Ref113307769}
Zenzontepec Chatino verbs obligatorily inflect for aspect-mood via prefixes and tonal alternations that are largely independent of one another \citep{Campbell2019}. Third person referents that are highly topical in discourse are often omitted in non-ambiguous contexts (\citealt{Campbell2015,Campbell2021a}), regardless of syntactic function, which I refer to as anaphoric zero \citep{Givon1983}. Thus, the (smallest) minimum free occurrence form that includes a verb root (and can express a proposition) consists of the root in position 13 and aspect-mood inflection in position 10, as shown in (\ref{bkm:Ref78641297}), but note that these positions are fused for certain verbs (see e.g., (\ref{chatino:ex:key:4a}). 


\ea\label{bkm:Ref78641297} Minimum free occurrence (small), positions 10{}--13 \\
\textit{jaku} \\
\glll j-aku \\
v:\ref{zen10am}-\ref{zen13base} \\  
\Pfv{}-eat.\Third{} \\ 
\glt `(They) ate (him).' (kwiten7 nkatzen 6:24) 
\z 

\subsubsection{Free occurrence (large), positions 7{}--17}
\label{bkm:Ref113184795}
Zero anaphora is not possible for first- or second-person reference, so pronouns must occur, and a set of several third-person pronouns are also used for maintaining referential continuity or for stylistic purposes. If a subject occurs, it occurs in position 17, delimiting the final edge of a maximal free occurrence span, as shown in example (\ref{bkm:Ref78653292}). Some auxiliaries have lexicalized with certain verbs; these are bound to (and precede) the main verb and are not able to stand as free forms themselves, as also illustrated in (\ref{bkm:Ref78653292}). The causative auxiliary and its associated positions (positions 7{}--9) occur obligatorily in order to express the intended meaning. Therefore, the minimum free occurrence test is fractured to include this larger span based on person and lexicalized auxiliaries.


\ea\label{bkm:Ref78653292} Free occurrence (large), span 7{}--17\\
\textit{laaɁ laa nkʷēkūtūɁúūɁ hiɁį̄} \\ 
\glll {} laaɁ l- aa \textbf{nkʷ-} \textbf{ē+} k- ū- t- ūɁú \textbf{=ūɁ} hiɁ\={\k{ı}}\\
v: \ref{zen3adv} \ref{zen3adv}- \ref{zen3adv} \ref{zen7am}{}- \ref{zen9aux}+ \ref{zen10am}{}- \ref{zen11cau}{}- \ref{zen12tr}{}- \ref{zen13base} =\ref{zen17sub} \ref{zen18np}  \\
{} like.so \Stat{}- be \Pfv{}- \Caus{}+ \Pot{}- \Caus{}- \Trvz{}- be.inside =\Third\Pl{} \Obj.(\Third{})\\
\glt `That's how they dressed her.' (4 bailes 6:11)
\z

\subsection{Non-permutability}
\label{bkm:Ref90386650}
Most Otomanguean languages have adverbial elements that follow the verb and precede the subject (\citealt[21--22]{Campbell2017}), modifying the event expressed in the clause. In Zenzontepec Chatino, multiple adverbs may co-occur, in variable order in the position 15 zone, as in related languages (\citealt[41]{Gutierrez-Lorenzo2014}). Since the positions of the verb root (and compound stem if present) are fixed, as is the position of the subject, we can speak of a rigid non-permutability span (\sectref{bkm:Ref78989028}), and a non-rigid non-permutability span (\sectref{bkm:Ref78989083}) around the core of the verbal planar structure.

\subsubsection{Non-permutability (rigid), positions 7{}--14}
\label{bkm:Ref78989028}
Elements from the aspect-mood marker of an auxiliary (position 7) to a postpound stem of a compound verb (position 14) occur in a rigidly fixed order and are non-permutable. The textual example in (\ref{ex:key:3a}) becomes ungrammatical if the order of the auxiliary and (compound) main verb is switched or if the order of the two stems within the (compound) main verb is reversed (\ref{ex:key:3b}). However, the order of elements outside of this span is more flexible.


\ea\label{bkm:Ref78658419}Non-permutability, span 7{}--14 \\ 
        \ea\label{ex:key:3a}{
        \textit{nkʷitakakūɁwí tī na kʷiniɁ lahaaɁ} \\ 
        \glll {} nkʷi- ta+ k -a +kūɁwí tī na kʷiniɁ laha =VɁ\\
        v: \ref{zen7am}- \ref{zen9aux}+ \ref{zen10am} -\ref{zen13base} +\ref{zen14com} \ref{zen17sub} \ref{zen17sub} \ref{zen17sub} \ref{zen17sub} =\ref{zen17sub} \\ 
        {} \Pfv{}- finish+ \Pot{} -become +drunk \Tplz{} \Def{} person wild =\Ana{}\\
        \glt `The devils finished getting drunk.' (amigo borracho 3:55)
        } 
        \ex\label{ex:key:3b} *{ k- a +kūɁwí \textbf{+nkʷi-} \textbf{ta} tī na kʷiniɁ laha =VɁ \\ 
        *nkʷi- ta+ \textbf{kūɁwí} +k- a tī na kʷiniɁ laha =VɁ \\
        } 
        \z 
\z 


\subsubsection{Non-permutability (scopal), positions 7{}--17}
\label{bkm:Ref78989083}\label{bkm:Ref113307808}
Following position 7, it is not until the adverbial zone in position 15 that the ordering of any elements may be manipulated. The examples in (\ref{bkm:Ref78660310}) display alternate orders of the adverbial elements \textit{=ri} `only' and \textit{=kāʔá} `again, also' following the verb root. Available evidence suggests that there are subtle scopal differences expressed by the alternate orders, with \textit{=ri} having scope over \textit{=kāʔá} `again' in (\ref{chatino:ex:key:4b}).

\ea\label{bkm:Ref78660310} Permutability in position 15 \\
            \ea \label{chatino:ex:key:4a}{
            \textit{tʲaa}\textbf{\textit{rikāʔá}} \textit{tsaka somanā nʲāʔā} \\ 
            \glll tʲaa \textbf{=ri} \textbf{=kāʔá} tsaka somanā nʲāʔā\\ 
            v:\ref{zen10am}.\ref{zen13base} =\ref{zen15adv} =\ref{zen15adv} \ref{zen17sub} \ref{zen17sub} \ref{zen21dm} \\
            \Pot{}.finish =only =again one week see.\Second\Sg{}\\ 
            \glt `A week just finishes, again, you see.' (chu ti7yu 10:36)
            }
            \ex\label{chatino:ex:key:4b} {
            \textit{jaku}\textbf{\textit{tsoʔōkāʔári}}\textit{ju} \\
            \glll j- aku \textbf{=tsoʔō} \textbf{=kāʔá} \textbf{=ri} =ju\\ 
              v:\ref{zen10am}- \ref{zen13base} =\ref{zen15adv} =\ref{zen15adv} =\ref{zen15adv} =\ref{zen17sub} \\ 
            \Pfv{}- eat =well =again =only =\Third\Sg{}.\M{}\\
            \glt `He only ate well again.' (offered)
            }
            \z
\z 


The offered (not elicited from a contact language) example in (\ref{chatino:ex:key:4b}) contains three elements in the adverbial zone in position 15. The initial one, adjective \textit{=tsoʔō} `good' (functioning adverbially as `well') may also occur later, in the zone in position 19, as shown in (\ref{bkm:Ref78994068}). If there is a meaning difference conveyed by these different positions for adverbials, it is not yet clear to me.


\ea\label{bkm:Ref78994068}Adverb in position 19 \\
  \textit{nu ntēʔjákʷentāą tī hnuwēʔ} \textbf{\textit{tsoʔō}} \textit{tsa hnʲāʔá}  \\
\glll {} nu ntē- ʔjá +kʷentā =ą tī h- nuwēʔ \textbf{tsoʔō} tsa.hnʲāʔá\\
v: \ref{zen1con} \ref{zen10am}{}- \ref{zen13base} +\ref{zen14com} =\ref{zen17sub} \ref{zen18np} \ref{zen18np}{}- \ref{zen18np} \ref{zen19adv} \ref{zen21dm} \\
{} \Sub{} \Prog{}- buy +account =\First\Incl{} \Tplz{} \Obj{}- \Third\Ana{} well truly\\
\glt `We are watching over this very well.' (lukwi proceso 6:01)
\z

What is clear is that an element that occurs in position 18, like the direct object \textit{leta} `path' in (\ref{ex:key:6a}), cannot be positioned anywhere among positions 7{}--17. The example in (\ref{ex:key:6b}) has the same root in position 14, but this is a compound verb with a conventionalized and very distinct meaning.


\ea\label{bkm:Ref90309480}Direct objects cannot interrupt the span of positions 7{}--17\\
            \ea\label{ex:key:6a} {nkatūɁúūɁ \textbf{leta} wá\\
            \glll  nka- t- ūɁú =ūɁ \textbf{leta} wá\\ 
            v:\ref{zen10am}.\ref{zen11cau}{}- \ref{zen12tr}{}- \ref{zen13base} =\ref{zen17sub} \ref{zen18np} \ref{zen18np} \\ 
            \Pfv{}.\Caus{}- \Trvz{}- be.inside =\Third\Pl{} path \Dist{}\\
            \glt `They put in that path.' (offered, verb examples5 19:22) 
            }
            \ex \label{ex:key:6b}{
            kutūɁú\textbf{leta}ą̄Ɂ hiɁį̄wą\\
            \glll k- u- t- ūɁú +\textbf{leta} =ą̄Ɂ hiɁ\={\k{ı}} =wą\\
            v:\ref{zen10am}{}- \ref{zen11cau}{}- \ref{zen12tr}{}- \ref{zen13base} +\ref{zen14com} =\ref{zen17sub} \ref{zen18np} =\ref{zen18np} \\
            \Pot{}- \Caus{}- \Trvz{}- be.inside +path =\First\Sg{} \Obj{} =\Second\Pl{}\\ 
            \glt `I will guide you (pl.).' (offered, verb examples5 21:34)
            }
            \z     
\z

Likewise, if an element in the position 15 adverbial zone, as shown in (\ref{ex:key:7a}), is placed after the subject (position 17), as in (\ref{ex:key:7b}), the meaning is significantly different, as the adverb modifies the subject instead of the verb.


\ea\label{bkm:Ref90312746}Adverbials modifying verb, or subject \\
            \ea\label{ex:key:7a} {
            tāká\textbf{kāʔá} migū nāáʔ jākʷá nakʷę\\ 
            \glll  tāká \textbf{=kāʔá} migū nāáʔ jākʷá nakʷę\\ 
            v:\ref{zen13base} =\ref{zen15adv} \ref{zen17sub} \ref{zen17sub} \ref{zen19adv} \ref{zen21dm} \\ 
            exist =also friend \First\Sg{} there say.\Third{}\\ 
            \glt `“I \textbf{also} \textbf{have} a friend there”, he said.' (escarabajo 3:10)
            }
            \ex \label{ex:key:7b}{
            tāká migū\textbf{kāʔá} nāáʔ nakʷẽ\\ 
            \glll tāká migū \textbf{=kāʔá} nāáʔ nakʷẽ\\ 
            v:\ref{zen13base} \ref{zen17sub}  =\ref{zen17sub} \ref{zen17sub} \ref{zen21dm} \\
            exist friend =also \First\Sg{} say.\Third{}\\
            \glt `“I have \textbf{another} \textbf{friend}”, he said.' (piedra rajada 1:43)
            }
            \z 
\z

The domain of scopal non-permutability in positions 7{}--17 is referred to as the Verbal Complex. This span deserves a special name because lexicalization of verbal meanings can involve in some cases the auxiliary span (7{}--9), and in emotion and cognition verbs, it includes the essence element in position 16. Moreover, subjects in position 17 are obligatory for certain persons, and if vowel-initial, they lengthen or fuse with the element that precedes them (see \sectref{bkm:Ref98189638}).

\subsection{Non-interruptability, positions 7{}--14}
\label{bkm:Ref113137532}\label{bkm:Ref113307818}
Non-interruptability constituency tests are used to identify the span of positions that cannot be interrupted by any free occurrence form. The main elements for testing this in Zenzontepec Chatino are the adverbials just discussed, which can interrupt the Verbal Complex in position 15. A select set of these adverbials can freely occur. Looking back at the example in (\ref{bkm:Ref78994068}), the adverbial \textit{tsoʔō} `well' modifies the verb but occurs in position 19, following the verb's arguments. This form (from the adjective `good') can occur freely on its own, and is commonly used to express agreement with one's interlocutor.


\ea\label{chatino:ex:key:8}{Adjective-adverbial \textit{tsoʔō} as free occurrence}\\
\textit{tsoʔō} \\ 
\glll tsoʔō\\
v:- \\
     good\\
\glt `It's good.' (nkoon lisu 2:13)
\z

The example in (\ref{bkm:Ref113123783}) illustrates that the adverb can interrupt the verbal complex. The verb `dance' is a compound verb (positions 13 +14, `make.music +foot') that occurs in both clauses in the construction. In the second clause, the adverbial \textit{tsoʔō} `well' occurs in position 15 between the verb stem and the subject.


\ea\label{bkm:Ref113123783}Adverb in position 15 \\
\textit{hnii jūlákija\={ą}ʔ tʲāʔ hjánā nt͡ʃūlákijaʔ}\textbf{\textit{tsoʔō}}\textit{\'{ǫ}ʔ} \\ 
\glll {} hnii j- ūlá +kijaʔ =\={ą}ʔ tʲāʔ hjánā nt͡ʃ- ūlá +kijaʔ \textbf{=tsoʔō} =ą̄ʔ\\
v: \ref{zen2np} \ref{zen10am}{}- \ref{zen13base} +\ref{zen14com} =\ref{zen17sub} \ref{zen19adv} \ref{zen19adv} \ref{zen10am}{}- \ref{zen13base} +\ref{zen14com} =\ref{zen15adv} =\ref{zen17sub} \\
{} song \Pfv{}- make.music +foot =\First\Sg{} still year.ago \Prog{}- make.music +foot =well =\First\Sg{}\\
\glt `The dance that I performed a year ago, I am dancing it better.' (dos cuentos raton 2:17)
\z

As shown in (\ref{bkm:Ref113132580}), the same adverb could just as well occur in position 19 -- which, as mentioned earlier, may have a slightly different meaning -- but it cannot interrupt the two stems of the compound, as shown in example (\ref{bkm:Ref113138975}). If we add an auxiliary in positions 7{}--9, the adverb cannot occur between the auxiliary span and the main verb (\ref{bkm:Ref113206476}).

\ea\label{bkm:Ref113132580}Adverb in position 19\\
\textit{hnii j-ūlá +kijaʔ =\={ą}ʔ tʲāʔ hjánā nt͡ʃ- ūlá +kijaʔ =\={ą}ʔ} \textbf{\textit{tsoʔō}} \\
`The dance that I performed a year ago, I am dancing it better.'\\
\z 

\ea\label{bkm:Ref113138975}Adverb between stems of a compound, ungrammatical \\
\textit{ *hnii j- ūlá +kijaʔ =\={ą}ʔ tʲāʔ hjánā nt͡ʃ- ūláʔ \textbf{tsoʔō} \textbf{+}kijaʔ\={ą}ʔ}\\
`The dance that I performed a year ago, I am dancing it better.'\\ 
\z

\ea\label{bkm:Ref113206476}Adverb cannot interrupt auxiliary `begin' and main verb\\
\textit{ *hnii j-ūlá +kijaʔ =\={ą}ʔ tʲāʔ hjánā ntē -tá + \textbf{tsoʔō} j-ūlá +kijaʔ =\={ą}ʔ}\\
`The dance that I performed a year ago, I am starting to dance it better.'\\
\z 

The examples above have shown that the span of non-interruptability that includes the verb root is positions 7{}--14.

\subsection{Coordination (subspan repetition)}
\label{bkm:Ref113137621}
In Zenzontepec Chatino, spans of varied lengths in the verbal planar structure may be coordinated. Following \citet[350]{Tallman2021}, this is a type of subspan repetition whose informativeness for constituency requires fracturing the test into subtests: a maximal (repeated) subspan and a minimal repeated subspan, where the minimal subspan refers to the span of elements in which none can be elided or have wide scope over both subspans.

\subsubsection{Minimal (repeated) subspan positions 5{}--16}
\label{bkm:Ref113307833}
The example in (\ref{bkm:Ref112926731}) illustrates asyndetic coordination of two verbs functioning as adverbial purpose clauses modifying a matrix clause. The first-person exclusive subject is elided on the first verb and occurs on the second verb, where it has scope over both subspans of positions 10{}--13. The dative oblique argument in position 18 likewise occurs only once, following the subject in the second subspan, and it also has scope over both subspans. Therefore, the subject in position 17 and non-subject arguments in position 18 are not within the minimal repeated subspan domain.


\ea\label{bkm:Ref112926731} Coordination of positions 10{}--18\\
\textit{ nteē nihjaą jakʷa \textbf{kikʷiʔ} \textbf{ketsāʔja} hjū}\\ 
\glll {} nteē nihjaą jakʷa \textbf{ki-} \textbf{akʷiʔ} \textbf{k-} \textbf{etsāʔ} \textbf{=ja} hiʔ\={\k{ı}} =ju\\
v: \ref{zen3adv} \ref{zen10am}.\ref{zen13base} \ref{zen17sub}  [\ref{zen10am}{}- \ref{zen13base}] [\ref{zen10am}{}- \ref{zen13base}] =\ref{zen17sub} \ref{zen18np} =\ref{zen18np} \\
{} here \Pot{}.come \First\Excl{} \Pot{}- speak \Pot{}- inform =\First\Excl{} \Dat{} =\Third\Sg.\M{}\\
\glt `We shall come here to speak and inform him.' (ntelinto itza7 4:18)
\z

The example in (\ref{bkm:Ref90395805}), illustrates that the minimal subspan includes position 14 (for both verbs in this case) and position 16, neither of which can be elided and neither of which may have scope over both subspans. Again, the subject NP in position 17 has scope over both subspans and is elided following the first verb. If the essence form in position 16 were omitted, this would yield a different verbal lexeme altogether, `have sexual relations.' Thus, the final edge of the minimal repeated subspan is position 16.


\newpage

\ea\label{bkm:Ref90395805}Verbal coordination \\ 
\textit{ ʔniléē ʔnikíʔjūrīké tī kʷaʔą tīkʷá =wą}\\
\glll [ʔni +léē] [ʔni +kíʔjū =rīké] tī kʷaʔą tīkʷá =wą\\
v:[\ref{zen10am}.\ref{zen13base} +\ref{zen14com}] [\ref{zen10am}.\ref{zen13base} +\ref{zen14com} =\ref{zen16ess}] \ref{zen17sub} \ref{zen17sub} \ref{zen10am}.\ref{zen13base} =\ref{zen17sub} \\
\Pot{}.do +strong \Pot{}.do +male =chest \Tplz{} \Second\Pl{} \Pot{}.sit =\Second\Pl{}\\
\glt `You all will make efforts and feel manly so that you sit (in power).' (ntelinto itza7 3:07)
\z

Since aspect-mood inflection is obligatory for Zenzontepec Chatino verbs, a smaller subspan excluding position 10 (or lack of inflectional tone on a verb for which positions 10 and 13 are fused) would be unutterable. However, the initial edge of the minimal repeated subspan domain remains to be demonstrated. As discussed in \sectref{bkm:Ref113184795}, for some verbs the auxiliary positions 7{}--9 are lexicalized and obligatory, which is relevant for free occurrence. For most verbs, however, auxiliaries are not obligatory, and they compositionally add their meaning to the clause. The example in (\ref{bkm:Ref113186492}) shows two repeated subspans of auxiliary, verb and subject, with the same auxiliary occurring in both. The free translation provided by a collaborator includes the meaning of the motion auxiliary in each clause.

\ea\label{bkm:Ref113186492}Subspan repetition\\
\textit{lēʔ janaʔaūʔ jatūkʷá kāʔáūʔ ike niʔi} \\
\glll {} lēʔ j- a+naʔa =ūʔ j- a+tūkʷá =kāʔá =ūʔ ike niʔi\\
v: \ref{zen1con} \ref{zen7am} \ref{zen9aux}+\ref{zen13base} =\ref{zen17sub} \ref{zen7am}- \ref{zen9aux}+\ref{zen13base} =\ref{zen15adv} =\ref{zen17sub} \ref{zen18np} - \\ 
{} then \Pfv{}- go+see =3\Pl{} \Pfv{}- go+\Caus{}.be.placed =again =3\Pl{} head house  \\
\glt `Then they went to see and they went to climb on top of the roof of the house again.' (nkwitzan tiʔi 15:58)
\z

To exclude the auxiliary on the first verb, the verb would take perfective aspect inflection, but the meaning changes, as shown in (\ref{ex:key:16a}) (elided material is crossed out). If the auxiliary is omitted from the second span, perfective aspect again provides the closest meaning, but the meaning changes slightly in (\ref{ex:key:16b}). Although the structure in (\ref{ex:key:16b}) can be interpreted as having the original meaning in (\ref{bkm:Ref113186492}), due to context and inference, it also has a subtly different literal meaning and can have other interpretations. Therefore, the auxiliary span of positions 7{}--9 is best treated as part of the minimal repeated subspan.

\newpage
\ea\label{bkm:Ref113187071} Subspan repetition with auxiliaries omitted \\
            \ea\label{ex:key:16a} {
            \textit{lēʔ \sout{ja}\textbf{nka}naʔaūʔ jatūkʷá kāʔáūʔ ike niʔi}\\
            `Then they \textbf{saw} and they went to climb on top of the roof of the house again.'
            }
            \ex\label{ex:key:16b} { 
            \textit{lēʔ janaʔaūʔ \sout{ja}\textbf{nkā }tūkʷá kāʔáūʔ ike niʔi} \\
            `Then they went to see and they \textbf{climbed} on top of the roof of the house again.'
            }
            \z
\z

In example (\ref{bkm:Ref112675242}), subspans including the adverbial particle in position 6 are repeated, and the particle cannot be elided in one subspan or the other without yielding a different interpretation, as shown in (\ref{bkm:Ref112678895}), and thus position 6 is also part of the minimal span.


\ea\label{bkm:Ref112675242}Verbal coordination, positions 10{}--14\\
\textit{ tʲāʔ kīkʷīʔ tʲāʔ tʲāá+tíʔ ʃī kīkʷīʔ}\\
\glll tʲāʔ kī- akʷīʔ tʲāʔ tʲāá +tíʔ ʃī kī- akʷīʔ\\
v:[\ref{zen6adv} \ref{zen10am}{}- \ref{zen13base}] [\ref{zen6adv} \ref{zen10am}.\ref{zen13base} +\ref{zen14com}] \ref{zen1con} \ref{zen10am}{}- \ref{zen13base} \\
still \Pot{}- speak.\Second\Sg{} still \Pot.\Iter{}.give +living.core.\Second\Sg{} \Conj{} \Pot{}- speak.\Second\Sg{} \\
\glt `Speak and still remember what you want to speak about.' (ntetakan7 jute7 1:37)
\z


\ea\label{bkm:Ref112678895}Verbal coordination, positions 10--16 \\
    \textit{ \sout{tʲāʔ} kīkʷīʔ tʲāʔ tʲāá+tíʔ ʃī kīkʷīʔ} \\ 
    \glll {} \sout{tʲāʔ} kī- kʷīʔ tʲāʔ tʲāá +tíʔ ʃī kī- akʷīʔ\\
    v: {} [\ref{zen10am}{}- \ref{zen13base}] [\ref{zen6adv} \ref{zen10am}.\ref{zen13base} +\ref{zen14com}] \ref{zen1con} \ref{zen10am}{}- \ref{zen13base} \\
    {} {} \Pot{}- speak.\Second\Sg{} still \Pot.\Iter{}.give +essence.\Second\Sg{} \Conj{} \Pot{}- speak.\Second\Sg{} \\
    \glt `Speak and still remember what you want to speak about.'
\z 

The example in (\ref{bkm:Ref112691082}) illustrates repeated subspans with adverbial-modal particles in position 5 that cannot be elided and neither can have scope over the other, as shown by the unacceptable translations beneath the free translation. Thus, the minimal repeated subspan includes at least positions 5{}--16.

\ea\label{bkm:Ref112691082}Verbal coordination, positions 16\\
\textit{ta tākárúʔ nkʷítsą hjā ná t͡ʃáʔąja}\\
\glll ta tāká =rúʔ nkʷítsą hiʔį̄ =jā ná t͡ʃáʔą =ja\\
v:[\ref{zen5md} \ref{zen10am}.\ref{zen13base} =\ref{zen16ess}] \ref{zen17sub} \ref{zen17sub} =\ref{zen17sub} [\ref{zen5md} \ref{zen10am}.\ref{zen13base}] =\ref{zen17sub} \\
already exist =even child \Gen{} =\First\Excl{} \Neg{} \Pot{}.get.accustomed =\First\Excl{} \\
\glt `We already have kids and we still don't get along.' (historia medicina 2:40)
\glt *`We \textit{still} already have kids and we still don't get along.'
\glt *`We already have kids and we \textit{already} still don't get along.'
\z

\subsubsection{Maximal (repeated) subspan, positions 2{}--20}
\label{bkm:Ref113226455}
Having defined the minimal repeated subspan as the span of positions in which no element can be elided or have scope over the other, the maximal repeated subspan is defined as the span of positions that can be repeated without reference to whether elements of one can be elided or have scope over both. Just as the preceding examples have shown that an elided subject in position 17 may have scope over multiple repeated subspans, thus delimiting the final edge of the minimal repeated subspan domain, some other less immediate elements may likewise be omitted from one subspan and have scope over both.

Example (\ref{bkm:Ref113187697}) shows that a significant span of pre-verbal elements may also be included in repeated subspans; the two clauses have fronted subjects in position 2. Also, the locative NP in position 18 only occurs following the second subspan, but it has scope over the first span as well.


\ea\label{bkm:Ref113187697}Subspan repetition, positions 2{}--18\\
\textit{hā maʃi nuʔu tsāā maʃi ahentē tsaaju nanēʔ kīkʷ\'{ą}}\\
\glll {} hā [maʃi nuʔu ts- āā] [maʃi ahentē ts- aa =ju] nanēʔ kīkʷ\'{ą}\\
v: \ref{zen1con} [\ref{zen2np} \ref{zen2np} \ref{zen10am}{}- \ref{zen13base}] [\ref{zen2np} \ref{zen2np} \ref{zen10am}{}- \ref{zen13base} =\ref{zen17sub}] \ref{zen18np} \ref{zen18np} \\
{} \Sub{} even \Second\Sg{} \Pot{}- go.\Second\Sg{} even agente \Pot{}- go =\Third\Sg.\M{} stomach metal\\
\glt `...because even you may go and even the \textit{agente} may go to jail.' (ntelinto itza7 4:01)
\z

Another example shows that adverbial particles that function as tense markers in position 20 also occur in repeated subspans, as shown in example (\ref{bkm:Ref112960117}). 


\ea\label{bkm:Ref112960117}Subspan repetition, positions 2{}--20\\
\textit{ nkatāká koʔma nkā ntuhwi kʷaa koʔma nkā}\\
\glll {} nk- a +tāká koʔma nkā ntu- hwi kʷaa koʔma nkā\\
v: [\ref{zen10am}{}- \ref{zen13base} +\ref{zen14com} \ref{zen17sub} \ref{zen20ten}] [\ref{zen10am}{}- \ref{zen13base} \ref{zen17sub} \ref{zen18np} \ref{zen20ten}] \\
{} \Pfv{}- be +exist macaw \Pst{} \Pfv{}- kill \Second\Excl{} macaw before\\
\glt `There were many macaws before and we would kill the macaws before.' (no hay brujos 9:45)
\z

The discourse markers in position 21 do not express information that contributes to the lexical interpretation of a proposition, but rather express a spea\-ker's appeal to an interlocutor. No examples were found in the corpus in which any of these could be understood as having scope over repeated subspans. Thus, the maximal repeated subspan domain includes positions 2{}--20.

Although the domains of subspan repetition are already defined, one more example serves to illustrate a ternary subspan repetition, but more importantly, it serves to make a point that is relevant later for tonal processes, which interact with intonation. Since the preferred strategy for coordinating clauses in Zenzontepec Chatino is asyndesis, that is, without any overt segmental marker indicating the clausal relation, one can wonder if these are cases of coordination at all or whether they are simply sequences of sentences. How real and strong of a difference is this anyway? The key cues are intonational. The example in (\ref{bkm:Ref90416693}) involves three instances of the same verb, each with a distinct lexical noun subject, and all of these VS spans (positions 10{}--17) are coordinated asyndetically. \figref{fig:chat:key:1} shows that all three clauses fall into one pitch contour with no pitch reset among them, with stylistic and expressive emphasis added to the final clause, especially on its subject.\footnote{All pitch tracks were made by Adam J.R. Tallman using a script developed by José Elias-Ulloa.}

\ea\label{bkm:Ref90416693}Ternary coordination V S + V S + V S\\
\textit{ nt͡ʃat\={ę} ketǫʔ nt͡ʃat\={ę} ʃikaʔ nt͡ʃat\={ę} tatījá}\\
\glll [n- t͡ʃat\={ę} ketǫʔ] [n- t͡ʃat\={ę} ʃikaʔ] [n- t͡ʃat\={ę} tatījá]\\
v:[\ref{zen10am}{}- \ref{zen13base} \ref{zen17sub}] [\ref{zen10am}{}- \ref{zen13base} \ref{zen17sub}] [\ref{zen10am}{}- \ref{zen13base} \ref{zen17sub}] \\
\Hab{}- get.washed pot \Hab{}- get.washed gourd \Hab{}- get.washed all\\
\glt `The pots get washed, the gourds get washed, \textit{everything} gets washed.' (ntelinto itza7 32:16)
\z

\begin{figure}[t]
    \centering
    \includegraphics[width=\textwidth]{figures/chat_1_nchate.jpg}
    \caption{Ternary asyndetic V S coordination}
    \label{fig:chat:key:1}
\end{figure}

\largerpage
Finally, syndetic coordination is also possible with the form \textit{lóʔō} occurring between the repeated subspans. The example in (\ref{bkm:Ref90396903}) illustrates coordination of two verbs with a coreferential subject expressed by a short pronoun on both verbs (position 17) and coreferential direct objects also overtly expressed after each verb in position 18. 



\ea\label{bkm:Ref90396903}Coordination of V=A O spans\\
\textit{ nkajúnēja hnā jooʔ \textbf{lóʔō} nkāhnʲája hnā jooʔ}\\
\glll [nkaj- únē =ja h- nā joo =Vʔ] \textbf{lóʔō} [nkā- hnʲá =ja h- nā joo =Vʔ]\\
v:[\ref{zen10am}{}- \ref{zen13base} =\ref{zen17sub} \ref{zen18np}{}- \ref{zen18np} \ref{zen18np} =\ref{zen18np}] \ref{zen1con} [\ref{zen10am}{}- \ref{zen13base} =\ref{zen17sub} \ref{zen18np}{}- \ref{zen18np} \ref{zen18np} =\ref{zen18np}] \\
\Pfv{}- dig =\First\Excl{} \Obj{}- \Def{} oven =\Ana{} and \Pfv{}- make =\First\Excl{} \Obj{}- \Def{} oven =\Ana{}\\
\glt `We dug the oven and we made the oven.' (historia maguey 2:44)
\z

More work is needed on exploring syndetic coordination, but it is relatively rare in discourse. It may be the preferred strategy when elidable positions (like subjects and objects) are restated in full form in each subspan, or perhaps it explicitly encodes temporal subsequence.

\section{(Morpho-)Phonological constituency tests}
\label{bkm:Ref90318262}
There are many distributional sound patterns, morphological alternations involving tone, and phonological processes that can be observed in Zenzontepec Chatino. Lexical and inflectional tone melodies are presented in \sectref{bkm:Ref90474687}, suprasegmental culminativity constraints in \sectref{bkm:Ref90474927}, segmental phonological processes in \sectref{bkm:Ref90475064}, and tonal phonological processes in \sectref{bkm:Ref109738706}.

\subsection{Lexical and inflectional tone melodies}
\label{bkm:Ref90474687}\subsubsection{Paradigmatic lexical tone melodies (positions 10{}-13)} 
\label{bkm:Ref113307854}\label{bkm:Ref112962511}
The tone bearing unit (TBU) in Zenzontepec Chatino is the mora, on which there is a three-way tonal specification contrast: High tone (H), Mid tone (M), or no tone ∅ (\citealt{Campbell2014,Campbell2016}). Unless affected by tonal processes, sequences of toneless TBUs are realized as mid-to-low gradually falling pitch (intonational declination, see \sectref{bkm:Ref90284807}). There is a bimoraic minimality preference for lexical forms, which bear one of five \textsc{basic} \textsc{tone} \textsc{melodies}: ∅∅, ∅M, MH, HM, H∅. Monomoraic forms display ∅, M, or H. In trimoraic forms, the bimoraic melodies align to the end of the form, and the antepenultimate mora tone is predictable. Note that word class or root class matters, so antepenultimate tone fill-in is not a strictly phonological process:

\begin{itemize}
\item For the ∅∅ and ∅M melodies, the antepenultimate mora is always tonally unspecified (∅) 
\item For the MH melody, it is ∅ for forms based on noun roots (∅MH) and M for forms based on verb roots (MMH)
\item For the HM melody, it is ∅ if the (derived) form is a verb (∅HM) and M if the form is not a verb (MHM)
\item For the H∅ melody, the antepenultimate mora tone is always M (MH∅)
\end{itemize}

The basic tone melodies are exemplified in \tabref{tab:zenz:key:2}, on bimoraic monosyllables, bimoraic disyllables, and trimoraic forms (disyllables or trisyllables). 

\begin{table}
    \centering
    \caption{Basic tonal melodies on lexical forms}
    \label{tab:zenz:key:2}
    \begin{tabular}{p{1.3cm}p{0.8cm}p{1.3cm}p{1.4cm}p{1.4cm}p{1.7cm}p{1.4cm}}
         \lsptoprule
Melody & \multicolumn{2}{p{2.1cm}}{Bimoraic monosyllable} & \multicolumn{2}{p{2.8cm}}{Bimoraic disyllable} & \multicolumn{2}{p{3.1cm}}{Trimoraic di- or tri-syllable}\\\midrule
(∅)∅∅   & j-aa  & `went'        & t͡ʃano         & `will stay'       & kukehę       & `will scratch'\\
(∅)∅M   & j-oō  & `ground it'   & n-tʲehnā      & `begins'          & nka-lʲaʔā       & `smelled it'\\
(∅/M)MH & tāá   & `will give'   & nk-jātę́       & `entered'         & nkū-tsāɁą́       & `changed'\\
        &       &               &               &                   & kʷi-līʃí      & `butterfly'\\
(∅/M)HM & kʷ-íī & `morning star' & j-únẽ̄        & `dug it'          & nka-wíī      & `cleaned it'\\
        &       &               &               &                   & lā-wíī        & `clean (adj.)'\\
(M)H∅   & tíi   & `ten'         & nkʲáku        & `got eaten'       & nkū-tákʷi    & `flew'\\
\lspbottomrule
    \end{tabular}
\end{table}

In the verbal planar structure, the five basic tone melodies span a domain that includes the verbal root and any derivational and inflectional material immediately preceding it (altogether, positions 10{}--13), as illustrated in (\ref{bkm:Ref84684092}).


\ea\label{bkm:Ref84684092}(∅)∅M tone melody in positions 10{}--13\\
\textit{ \textbf{kuʃik\={ą}ʔ}ja hį̄ laaʔ laa}\\
\glll \textbf{k-} \textbf{u-} \textbf{ʃi-} \textbf{k\={ą}ʔ} =ja hiʔ\={\k{ı}} laaʔ.laa\\
v:[\ref{zen10am}{}- \ref{zen11cau}{}- \ref{zen12tr}{}- \ref{zen13base}] =\ref{zen17sub} \ref{zen18np} \ref{zen19adv} \\
\Pot{}- \Caus{}- \Trvz{}- tie.up =\First\Excl{} \Obj.(\Third{}) like.so\\
\glt ʻthat we tie it up like so.ʼ (naten7 michen 5:04)
\z

As shown in example (\ref{bkm:Ref90483115}), the basic tone melodies occur separately and independently in the auxiliary span of positions 7{}--9 (MH melody). Also in (\ref{bkm:Ref90483115}), the main verb root in position 13 (HM) and disyllabic adverbial element in position 15 (MH) bear their own basic tone melodies.

\ea\label{bkm:Ref90483115}Basic tonal melodies in a verb with auxiliary and adverbial\\
\textit{lēʔ nu \textbf{nkūtá}ʃáʔā\textbf{kāʔá} na la kʷajūuʔ nʲāʔā}\\
\glll {} lēʔ.nu \textbf{nkū-} \textbf{tá}+ ʃáʔā \textbf{=kāʔá} na la.kʷajū =Vʔ nʲāʔā\\
v: \ref{zen1con} [\ref{zen7am}{}- \ref{zen9aux}+] [\ref{zen13base}] =\ref{zen15adv} \ref{zen17sub} \ref{zen17sub} =\ref{zen17sub} \ref{zen21dm} \\
{} then \Pfv{}- begin+ scream =also \Def{} horse =\Ana{} see.\Second\Sg{}\\
\glt ʻThen the horse also began to scream, you see.ʼ (rey david 0:56)
\z

Moreover, as example (\ref{bkm:Ref90483210}) shows, the basic tone melodies occur separately and independently on the second stem of a compound verb in position 14 (HM) and on the essence element `chest' in position 16 (MH). The issue of the reversed order of positions 15 and 16 is discussed further in \sectref{bkm:Ref90326435}.


\ea\label{bkm:Ref90483210}Basic tonal melodies on verb stem, postpound stem and essence element\\
\textit{ntesuʃíīrīkékāʔá naa}\\
\glll nte- su +\textbf{ʃíī} =rīké =kāʔá naa\\
v:[\ref{zen10am}{}- \ref{zen13base}] [+\ref{zen14com}] [=\ref{zen16ess}] [=\ref{zen15adv}] \ref{zen17sub} \\
\Prog{}- lie +light =chest =also \First\Incl{}\\
\glt `We are also awakening.' (la familia 23:52)
\z

What we observe here is that the domain within which contrastive tonal melodies occurs that includes the verb root is the span of positions 10{}--13. Basic tonal melodies also occur independently over the spans of positions 7{}--9, 14, 15 and 16. 

\subsubsection{Deviation from biuniqueness: TAM tonal alternations (positions 10{}--13)}
\label{bkm:Ref113308071}
Aspect-mood exponence consists of two largely orthogonal types of morphological expression, each of which displays significantly unpredictable allomorphy: prefixes (typically position 10, but position 7 when the verb occurs with an auxiliary; \citealt{Campbell2011}), and tonal alternations (\citealt{Campbell2016,Campbell2019}). Single-stem (non-compound) inflected verb forms bear one of the five basic tone melodies just discussed in \sectref{bkm:Ref112962511}, spanning positions 10{}--13.

Although some verbs have an invariant tone melody in all aspect-mood forms, such as the verb `pass' in \tabref{tab:zenz:key:3}, which bears the ∅M melody in all of its forms, other verbs display tonal melody alternations among the aspect-mood forms. For example, the verb `shell, degrain' is toneless in the potential mood and habitual aspect forms but bears the HM tone melody in the progressive and perfective aspect forms. The verb `get cooked' displays another tonal alternation pattern: it bears the MH melody in the potential, habitual, and perfective forms and the ∅M melody only in its progressive form. These unpredictable tonal alternations are an integral part of the exponence of TAM inflection, a deviation from biuniqueness and in some ways acting as a phonological test while in others ways a morphosyntactic one. The domain of TAM tonal melody alternations that includes the verb root is the span of positions 10--13.

\begin{table}
    \caption{Tonal alternations in TAM inflection}
    \label{tab:zenz:key:3}
    \fittable{
    \begin{tabular}{lllllll}
    \lsptoprule
                        & \multicolumn{2}{p{2.5cm}}{-tehę̄} & \multicolumn{2}{p{2.5cm}}{-u-s-úkʷāʔ}  & \multicolumn{2}{p{2.5cm}}{-ākéɁ}   \\ 
                        & \multicolumn{2}{p{2.5cm}}{`pass'} & \multicolumn{2}{p{2.5cm}}{`shell, degrain'}  & \multicolumn{2}{p{2.5cm}}{`get cooked'}   \\ \midrule
    Potential mood      & tʲehę̄     & ∅M & k-u-s-ukʷaʔ     & ∅∅ & k-ākéɁ   & MH\\
    Habitual aspect     & n-tʲehę̄   & ∅M & nt-u-s-ukʷaʔ    & ∅∅ & ntī-kéɁ  & MH\\
    Progressive aspect  & nte-tehę̄  & ∅M & nte-s-úkʷāʔ     & HM & nt͡ʃ-akēɁ   & ∅M\\
    Perfective aspect   & nku-tehę̄  & ∅M & nka-s-úkʷāʔ     & HM & nkū-kéɁ  & MH\\
    \lspbottomrule
    \end{tabular}
    }
\end{table}

Beyond position 13, a new tone melody domain begins. In the example in (\ref{ex:key:27a}), the main verbal inflectional tone melody is (M)MH, spanning positions 10{}--13, and it can be observed that the preceding demonstrative is toneless and the following stem in the compound is monomoraic and toneless (position 14). The example in (\ref{ex:key:27b}) illustrates the non-compound verb upon which the compound verb in (\ref{ex:key:27a}) is based, inflected for the same aspect; what follows is a bare noun as direct object NP with its own lexical tone domain (HM).


\ea\label{bkm:Ref83134430}Tonal aspect-mood inflection in positions 10{}--13, not 14 \\
            \ea \label{ex:key:27a} {
            {} \textit{tsúna hakʷa na lítʲúu nkáʔāaʔ \textbf{nkātēʔę́}tsaūʔ wiʔ} \\
            \glll {} tsúna hakʷa na lí.tʲúu nkáʔā =Vʔ \textbf{nkā-} \textbf{tēʔę́ } +tsa =ūʔ wiʔ \\
            v: \ref{zen1con} - - - - =- [\ref{zen10am}{}- \ref{zen13base}] +\ref{zen14com} =\ref{zen17sub} \ref{zen19adv} \\
             {} three four \Def{} adobe red =\Ana{} \Pfv.\Caus{}- be.located +placed =\Third\Pl{} there\\
            \glt `Three or four red (adobe) bricks, they put in there.' (ni7 rosa 3:24)
            }

            \ex \label{ex:key:27b}{ 
            \textit{\textbf{nkātēʔ\'{ę}} hú\={\k{u}} janeju} \\ 
            \glll {} \textbf{nkā-} \textbf{tēʔ\'{ę}} hú\={\k{u}} jane =ju\\
            v: [\ref{zen10am}{}- \ref{zen13base}] \ref{zen18np} \ref{zen18np} =\ref{zen18np} \\
            {} \Pfv{}- \Caus{}.be.located(.\Third{}) rope neck =\Third\Sg.\M{}\\
            \glt ʻHe put rope around his neck.ʼ (offered)
            } 
            \z 
\z 

As a further demonstration, consider the TAM inflectional paradigms of the three verbs shown in \tabref{tab:zenz:key:4} that all contain the verbal root \textit{{}-ūlá} `make music', which has aspect-mood tone alternation ∅M {\textasciitilde} MH. In the compound verb \textit{{}-ūlá+tuɁwa} `sing' the compounded element \textit{+tuɁwa} `mouth' in position 14 is a toneless noun whose (lack of) lexical tone is independent of the verbal inflectional stem and does not participate in the aspect-mood tonal alternation. In the verb \textit{{}-ūlá=rīké} `long for' the final element \textit{=rīké} `chest' in position 16 displays its own invariant MH tone melody that does not alternate with the TAM melodies in positions 10{}--13. Thus, tonal alternations as part of TAM inflection only identify the span of positions 10{}--13.


\begin{table}
    \caption{Tonal alternations in complex verbal lexemes}
    \label{tab:zenz:key:4}
    \fittable{
    \begin{tabular}{p{1.6cm}llllll}
    \lsptoprule
         & \multicolumn{2}{l}{\textit{{}-ūlá}}   & \multicolumn{2}{l}{\textit{{}-ūlá+tuɁwa}}      & \multicolumn{2}{l}{\textit{{}-ūlá=rīké}}\\
         & \multicolumn{2}{l}{`make music'}      & \multicolumn{2}{l}{`sing'}      & \multicolumn{2}{l}{`long for'}\\ \midrule
    Potential Mood      & k-ulā     & ∅M & k-ulā+tuɁwa      & ∅M+∅∅ & k-ulā=rīké    & ∅M=MH\\
    Habitual Aspect     & nt-ulā    & ∅M & nt-ulā+tuɁwa     & ∅M+∅∅ & nt-ulā=rīké    & ∅M=MH\\
    Progressive Aspect  & nt͡ʃ-ūlá & MH & nt͡ʃ-ūlá+tuɁwa  & MH+∅∅ & nt͡ʃ-ūlá=rīké  & MH=MH\\
    Perfective Aspect   & k-ūlá     & MH & k-ūlá+tuɁwa      & MH+∅∅ & k-ūlá=rīké    & MH=MH\\
    \lspbottomrule
    \end{tabular}
    }
    
\end{table}

Finally, in auxiliary constructions, the TAM inflection of the entire auxiliary construction with main verb occurs on -- and only on -- the auxiliary in positions 7{}--9. The verb `be afraid' with the causative auxiliary illustrates: in (\ref{chatino:ex:key:28a}), the progressive aspect form displays the ∅M melody, and in (\ref{chatino:ex:key:28b}) the potential mood form is monomoraic and toneless ∅ (the monomoraic correlate of the ∅∅ melody), realized on positions 7{}--9 in both cases.


\ea\label{bkm:Ref90484055}Tonal aspect-mood inflection on positions 7{}--9\\
                \ea\label{chatino:ex:key:28a} { 
                \textit{maʃi \textbf{ntekē}kutsęūʔ hį̄ laaʔ} \\  
                \glll {} maʃi \textbf{nte-} \textbf{k-} \textbf{ē}+ k- utsę =ūʔ h\={\k{ı}} laaʔ\\
                v: \ref{zen1con} [\ref{zen7am}{}- \ref{zen8tr}{}- \ref{zen9aux}+] [\ref{zen10am}{}- \ref{zen13base}] =\ref{zen17sub} \ref{zen18np} \ref{zen19adv} \\
                {} even.if \Prog{}- \Pot{}- \Caus{}+ \Pot{}- be.afraid =\Third\Nspec{} \Obj{} like.so\\ 
                \glt `Even if they are frightening her like so.' (nino chiquito 1:35)
                }
                \ex\label{chatino:ex:key:28b} { 
                \textit{\textbf{ke}kutsę nūw\'{ą} nkʷítsą}\\
                \glll \textbf{k-} \textbf{e}+ k- utsę nūw\'{ą} nkʷítsą\\
                V:[\ref{zen7am}{}- \ref{zen9aux}+] [\ref{zen10am}{}- \ref{zen13base}] \ref{zen17sub} \ref{zen18np} \\
                \Pot{}- \Caus{}+ \Pot{}- be.afraid \Third\Dist{} child\\
                \glt `That one is going to frighten the children.' (juan oso 7:02)
                }
                \z 
\z 

To illustrate the locus of aspect-mood inflection in the auxiliary span, \tabref{tab:zenz:key:5} presents the paradigm of aspect-mood inflection for the verb in the examples in (\ref{bkm:Ref90484055}).

\begin{table}
    \caption{Aspect-mood tonal alternation in an auxiliary construction}
    \label{tab:zenz:key:5}
     {
    \begin{tabular}{lll}
         \lsptoprule
                        & \multicolumn{2}{c}{\textit{{}-}\textbf{\textit{ē}}\textit{+ k-utsę} `frighten'} \\ \midrule
    Potential Mood      & k-e+k-utsę    & \textbf{∅}+∅∅\\
    Habitual Aspect     & nt-e+k-utsę   & \textbf{∅}+∅∅\\
    Progressive Aspect  & nte-k-ē+k-utsę    & \textbf{∅M}+∅∅\\
    Perfective Aspect   & nkʷ-ē+k-utsę      & \textbf{M}+∅∅\\
\lspbottomrule
    \end{tabular}
    }
\end{table}



\subsubsection{Deviation from biuniqueness: 2sg tone melodies (positions 10{}--13)}
\label{bkm:Ref113308088}
As already discussed, subjects occur in position 17 of the verbal planar structure. The pragmatically-neutral, short pronouns encoding subject immediately follow the last element in the range of positions 13{}--16. However, 2sg person is encoded by replacing the basic lexical tone melody of the preceding element or the TAM-inflected Verbal Core in positions 10{}--13 with one of two specialized 2sg tone melodies: (∅)(∅)H and (M)(M)M. These melodies are only and always found in 2sg inflection. The process works as follows.

\begin{itemize}
\item 
If the preceding element bears the (∅)(∅)M tone melody, the 2sg inflected form will bear (∅)(∅)H
\item
If the final element bears any other lexical tone melody (i.e.; ∅∅, MH, HM, H∅ or their trimoraic counterparts), then the 2sg inflected form will bear (M)MM: a Mid tone on each mora.
\end{itemize}

Some examples of minimal free form verbs (positions 10{}--13) with 2sg subject inflection illustrate the patterns in \tabref{tab:zenz:key:6}.


\begin{table}
    \caption{2sg tone melodies on verbs}
    \label{tab:zenz:key:6}
    \fittable{
    \begin{tabular}{lllllll}
    \lsptoprule
        \multicolumn{3}{l}{Uninflected for person} &  & \multicolumn{3}{l}{Inflected for 2sg} \\ \midrule
        nt͡ʃ-uhwīɁ      & ∅M    & `is selling'  & → & nt͡ʃ-uhwíɁ    & ∅H & {`you are selling'} \\
        k-u-nakʷā ̨     & ∅∅M   & `will bless'  & → & k-u-nakʷą́    & ∅∅H & {`you will bless'}\\
        k-ōó           & MH    & `will grind'   & → & k-ōō         & MM & {`you will grind'}\\
        nk-j-ánō       & HM    & `stayed'       & → & nk-j-ānō     & MM & {`you stayed'}\\
        nku-líhī   & ∅HM   & `got lost'     & → & nkūlīhī       & MMM & {`you got lost'}\\
         \lspbottomrule
    \end{tabular}
    }
\end{table}

 
The locus of the tonal 2sg inflection is always the element that immediately precedes where any NP or short pronoun in the same grammatical function would occur. In example (\ref{bkm:Ref83388735}), the 2sg tone melody MM is found on the adverbial element \textit{kāɁá} `again' in position 15 in the second clause, which otherwise bears the MH melody.


\ea\label{bkm:Ref83388735}Second-person tonal inflection on adverbial in position 15\\
\textit{ta nkʷitaa na ja nt͡ʃakę nʲāʔā hā tsa+k-iʔja\textbf{=kāʔā}}\\
\glll {} ta nkʷi- taa na ja.nt͡ʃakę nʲāʔā hā ts- a+ k- iʔja \textbf{=kāʔā}\\
V: \ref{zen5md} \ref{zen10am}{}- \ref{zen13base} \ref{zen17sub} \ref{zen17sub} \ref{zen21dm} \ref{zen1con} \ref{zen7am}{}- \ref{zen9aux}+ \ref{zen10am}{}- \ref{zen13base} [=\ref{zen15adv}] \\ 
{} already \Pfv{}- finish \Def{} firewood see.\Second\Sg{} \Conj{} \Pot{}- go+ \Pot{}- transport =again.\Second\Sg{}\\
\glt `If the firewood has been used up, well, you have to go and bring more.' (juan oso 9:29)
\z 

The examples in \tabref{tab:zenz:key:7} illustrate the 2sg tone melody on positions 10{}--13, 14, 15, and 16, following the corresponding structures with the 3\textsc{sg.f} pronoun \textit{=t͡ʃūɁ} in position 17.

\begin{table}
    \caption{2sg tone melodies on varied positions and spans}
    \label{tab:zenz:key:7}
    \centering
    \begin{tabular}{llll}
        \lsptoprule
    \multicolumn{4}{l}{Inflected for 3\textsc{sg.f}} \\ \midrule
    nt-e+k-\textbf{ū}{}-l\textbf{í}h\textbf{i}=t͡ʃūɁ  & V: \ref{zen10am}-\ref{zen13base} & `she loses (tr.)'    & → \\
    nka-Ɂni+tsoɁ\textbf{ō}=t͡ʃūɁ                      & V: \ref{zen14com}                & `she fixed it'       & →  \\
    t͡ʃ-uɁu=tsoɁ\textbf{ō}=t͡ʃūɁ                       & V: \ref{zen15adv}                & `she will live well' & →  \\
    {}Ɂne+tii=r\textbf{ī}k\textbf{é}=t͡ʃūɁ            & V: \ref{zen16ess}                & `she can guess'      & →  \\\midrule
    \multicolumn{4}{l}{Inflected for 2sg} \\ \midrule
     nt-e+\textbf{k-ū-līhī}  & MH∅→MMM & `\textbf{you} lose (tr.)'  & \\
     nka-Ɂni+\textbf{tsoɁó}  & ∅M→∅H & `\textbf{you} fixed it'  & \\
     t͡ʃ-uɁu=\textbf{tsoɁó}   & ∅M→∅H & `\textbf{you} will live well'  & \\
     Ɂne+tii=\textbf{rīkē}   & MH→MM & `\textbf{you} can guess'  & \\
    \lspbottomrule
    \end{tabular}
\end{table}


\subsection{Suprasegmental culminativity}
\label{bkm:Ref90474927}
A range of suprasegmental phonotactic restrictions can be observed around the Zenzontepec Chatino verbal core. These involve culminativity of H tone (\sectref{bkm:Ref113308111}), glottal stop (\sectref{bkm:Ref113372574}), contrastive vowel nasality (\sectref{bkm:Ref113372594}) and contrastive vowel length (\sectref{bkm:Ref15745289}).

\subsubsection{Culminative H tone constraint (positions 10{}--13)}
\label{bkm:Ref113308111}
None of the basic (∅∅, ∅M, MH, HM, H∅) or second-person (MM, ∅H) tone melodies just presented, or their trimoraic extensions, contain more than one H tone. The same is true for the tonal alternations in aspect-mood inflection; they never contain multiple H tones. This distributional pattern is referred to as culminative H tone. However, multiple M tones (or unspecified moras ∅), may occur in the lexical tone melody domain, and therefore, the restriction on multiple H tones is best explained by a culminativity constraint. It should be noted that this constraint operates at the phonological level, but due to H tone spreading in language use (\sectref{bkm:Ref112148431}), we find significant stretches of high-pitch plateaus. Furthermore, one may wonder if culminative H tone is really a distinct test from the inflectional tone alternations. However, tonal alternations could imaginably operate on larger spans, but they do not (for example, including the auxiliary span as well, 7{}--13), and the alternations can't simply be derived from positing a culminative H constraint. Forms that have more than one H tone reflect different tonal domains. Consider the analyzable compound verb in (\ref{bkm:Ref90485763}), in which the main, inflected verb stem (positions 10{}--13) has the H∅ melody while the second stem (position 14) has monomoraic H (the result here is downstep of the second H tone; see \sectref{bkm:Ref96876657}).


\ea\label{bkm:Ref90485763}Two H tones, in different tone melody domains\\
\textit{nku\textbf{túʔuhná} tī nāáʔ nakʷę}\\
\glll nku- \textbf{túʔu} \textbf{+hná} tī nāáʔ nakʷę\\
V:\ref{zen10am}{}- \ref{zen13base} +\ref{zen14com} \ref{zen18np} \ref{zen18np} \ref{zen21dm} \\ 
\Pfv{}- leave +flee \Tplz{} \First\Sg{} say.\Third{}\\
\glt `Well, I ran, he said.' (nagual tigre 1:43)
\z

One can review example (\ref{bkm:Ref90483115}) and see that the auxiliary span (positions 7{}--9) bears the MH melody, the main verb stem (position 13) has HM, and the adverbial (position 15) has the MH melody. Another example is (\ref{bkm:Ref90483210}), in which the first, inflected stem in the compound verb (positions 10{}--13) is toneless ∅∅, the second stem in the compound (position 14) bears the HM melody, and the adverbial form in position 15 and essence form in position 16 each independently bear the MH melody.

\subsubsection{Culminative glottal stop} 
\label{bkm:Ref113372574}
No lexical tone melody domain ever contains more than one glottal stop. Roots that historically did have multiple glottals in proto-Zapotecan have all reduced them to maximally one, in all Chatino languages \citep{Campbell2021b}. 

\subsubsubsection{Culminative glottal stop (minimal) (positions 10{}--13)}
\label{bkm:Ref113308189}
One can peruse the example sentences throughout this chapter and note the lack of multiple glottal stops in any domain of the lexical tone melodies, such as the verb root and its derivational and inflectional prefixes (positions 10{}--13). A more interesting fact can be appreciated when this test is fractured and we look for a maximal domain.

\subsubsubsection{Culminative glottal stop (maximal), positions 7{}--13}
\label{bkm:Ref113308202}
No aspect-mood formatives (positions 7 and 10), derivational formatives (positions 8, 11, 12), or auxiliary verbs (position 9) contain a glottal stop, a generally very frequent consonant in the language. If this is not due to chance -- as the inventories of elements in these categories are small -- then there is a limit of a maximum of one glottal stop in the combined auxiliary and main verb span (positions 7{}--13). The auxiliary construction in (\ref{bkm:Ref83400183}) has a main verb that is a compound and both compounded stems contain a glottal stop (positions 13 and 14) showing that this culminativity domain does not reach past position 13. Another auxiliary construction is shown in (\ref{bkm:Ref83399628}). The main verb contains a glottal stop, and the following glottal stop is part of the subject pronoun in position 17.


\ea\label{bkm:Ref83400183}Glottal stops in positions 13 and 14\\
\textit{ t͡ʃajuɁuseɁęna}\\
\glll {} t͡ʃa+ j{}- \textbf{uɁu} +\textbf{seɁę} =na \\
v: \ref{zen7am}.\ref{zen9aux}+ \ref{zen12tr}{}- \ref{zen13base} +\ref{zen14com} =\ref{zen17sub}\\
{} \Pot{}.go+ \Itr{}- \textbf{be.inside} \textbf{+place} =\First\Incl{}\\
\glt `We are going to go rest.' (historia1 4:40)
\z


\ea\label{bkm:Ref83399628}Glottal stops in positions 13 and 17\\
\textit{ kenaɁa tī kʷítī ta jakīɁjáą̄Ɂ}\\
\glll {} kenaɁa tī kʷítī ta j- a+ k- īɁjá =ą̄Ɂ\\
v: \ref{zen2np} \ref{zen2np} \ref{zen2np} \ref{zen5md} \ref{zen7am}{}- \ref{zen9aux}+ \ref{zen10am}{}- \ref{zen13base} =\ref{zen17sub}\\
{} a.lot \Tplz{} remedy already \Pfv{}- go+ \Pot{}- transport =\First\Sg{}\\
\glt `I have gone to get a lot of medicine already.' (historia medicina 47:33)
\z

The example in (\ref{bkm:Ref90505554}) illustrates the proximative aspectual particle that occurs in position 6; it contains a glottal stop, as does the verb root and subject pronoun in positions 13 and 17, respectively.


\ea\label{bkm:Ref90505554}Glottals in positions 6, 13, and 17\\
\textit{ tíʔ kikʷeʔ\={ę}ʔ}\\
\glll {} tíʔ ki- kʷiʔ =\={ę}ʔ\\
v: \ref{zen6adv} \ref{zen10am}{}- \ref{zen13base} =\ref{zen17sub}\\
{} \Prx{} \Pot{}- speak =\First\Sg{}\\
\glt `I am just about to speak...' (medicina2 4:17)
\z

Thus, the maximal domain around the verb root in which glottal stop culminativity holds does not include position 6 and includes only positions 7{}--13.

\subsubsection{Culminative and final-position vowel nasality}
\label{bkm:Ref113372594}\label{bkm:Ref84412913}
Contrastive vowel nasality only occurs in the final syllable of lexical roots, long and short pronouns, and bimoraic function words. There is a minimal span in which the constraint can be observed, and a larger (maximal) span that includes positions for which evidence is not available.

\subsubsubsection{Culminative vowel nasality (minimal), positions 7{}--13}
\label{bkm:Ref113308220}
The verb \textit{yaą} `come' of the temporal adverbial clause in (\ref{ex:key:34a}) has a nasal vowel. The same verb, reduced as an auxiliary (positions 7{}--9) lacks vowel nasality, as shown in (\ref{ex:key:34b}), which is strong evidence of this distributional restriction on nasal vowels. 

\ea\label{bkm:Ref84262486} Loss of vowel nasality in auxiliary position\\
                \ea \label{ex:key:34a}{
                \textit{ná nteʔękāʔā  \textbf{nkjaą}kāʔája}\\
                \glll {} ná n- teʔę =kāʔā  \textbf{nk-} \textbf{jaą} =kāʔá =ja\\
                v: \ref{zen5md} \ref{zen10am}{}- \ref{zen13base} =\ref{zen15adv} \ref{zen10am}{}- \ref{zen13base} =\ref{zen15adv} =\ref{zen17sub} \\ 
                {} \Neg{} \Stat{}- be.located =again.\Second\Sg{} \Pfv{}- come =again =\First\Excl{}\\
                \glt `You weren't here when we came the other time.' (historia1 29:25)
                }
                \ex\label{ex:key:34b} 
                \textit{tsáʔ wiʔ laa \textbf{nkja}hnáʔ huteę̄ʔ hją́ʔ \\  
                \glll {} tsáʔ.wiʔ laa \textbf{nk-} \textbf{ja}+ hnáʔ huti =ą̄ʔ hiʔ\={\k{ı}} =ą̄ʔ\\
                v: \ref{zen3adv} \ref{zen3adv} \ref{zen7am}{}- \ref{zen9aux}+ \ref{zen13base} \ref{zen17sub} =\ref{zen17sub} \ref{zen18np} =\ref{zen18np}\\
                {} word.\Ana{} be \Pfv{}- come+ throw.away father =\First\Sg{} \Obj{} =\First\Sg{}\\
                \glt `Because of that my father came to throw me away.' (nkwitzan ti7i 7:08)
                }
                \z 
\z 

Vowel nasality may occur in both stems of a compound verb, showing that the culminative nasality restriction applies separately to positions 13 and 14, as illustrated in (\ref{bkm:Ref84262642}).


\ea\label{bkm:Ref84262642}Vowel nasality in both stems of a compound verb\\ 
\textit{nkalātíʔ tī na tukalāaʔ niī lēʔ ja\textbf{saʔąseʔę}ju} \\ 
\glll nka- lātíʔ tī na tukalā =Vʔ niī lēʔ j- a+ \textbf{saʔą} \textbf{+seʔę} =ju\\
v:\ref{zen10am}{}- \ref{zen13base} \ref{zen17sub} \ref{zen17sub} \ref{zen17sub} =\ref{zen17sub} \ref{zen21dm} \ref{zen1con} \ref{zen7am}{}- \ref{zen9aux}+ \ref{zen13base} +\ref{zen14com} =\ref{zen17sub}\\ 
\Pfv{}- stop \Tplz{} \Def{} cloudiness =\Ana{} now then \Pfv{}- go+ be.attached +place =\Third\Sg.\M{}\\
\glt `The cloudiness ceased, and then he went to rest.'  (muchacha ixtayutla 6:33)
\z

\subsubsubsection{Culminative vowel nasality (maximal), positions 4{}--13}
\label{bkm:Ref113308229}

There are no modal particles that occur in positions 4--5 or adverbials of position 6 that contain contrastive nasal vowels. The inventories of elements that occur in these positions are small, so this may be due to chance, but nonetheless the test can be fractured: a maximal span for culminative vowel nasality is positions 4{}--13.

\subsubsection{Culminative and final-position vowel length} 
\label{bkm:Ref15745289}
Similar to vowel nasality, contrastive vowel length occurs mostly in final syllables of lexical roots. It also occurs in independent pronouns and some bimoraic function words. This test must also be fractured because evidence for the initial point of the span may be lacking due to chance.

\subsubsubsection{Culminative vowel length (minimal), positions 7{}--13}
\label{bkm:Ref113308246}
Within the auxiliary and main verb span of positions 7{}--13, long vowels only occur in final syllables of position 13. The pair of examples in (\ref{bkm:Ref84262486}) above illustrate, in part, this distribution. In (\ref{ex:key:34a}) the verb \textit{\nobreakdash-yaą} `come' in position 13 contains a long vowel, but as an auxiliary in position 9 in (\ref{ex:key:34b}) it lacks its original vowel length (and nasality, as discussed in \sectref{bkm:Ref84412913}).

The example in (\ref{bkm:Ref83937179}) shows a compound verb in which both stems of the compound (positions 13 and 14) have long vowels. Thus, position 14 is beyond the domain of culminative final-position vowel length.


\ea\label{bkm:Ref83937179}Distribution of long vowels\\
\textit{nkjánō nteē lēʔ nka\textbf{lōónaa}ūʔ saperū}\\
\glll {} nk- j- ánō nteē lēʔ nka- \textbf{lōó} \textbf{+naa} =ūʔ saperū\\
v: \ref{zen10am}{}- \ref{zen12tr}{}- \ref{zen13base} \ref{zen19adv} \ref{zen1con} \ref{zen10am}{}- \ref{zen13base} +\ref{zen14com} =\ref{zen17sub} \ref{zen18np}\\ 
{} \Pfv{}- \Itr{}- stay(.\Third{}) here then \Pfv{}- take.out +name =\Third\Pl{} San.Pedro\\
\glt `Here it remained, and they named it San Pedro.' (medicina1 38:23)
\z

\subsubsubsection{Culminative vowel length (maximal) (positions 4{}--13)}
\label{bkm:Ref113308255}
Like vowel nasality, vowel length does not occur in modal particles or adverbials in positions 4{}--6. The example in (\ref{bkm:Ref90505575}) illustrates the adverb \textit{niī} `now' in the position 3 zone, which is the last position before the verb complex in which long vowels occur.


\ea\label{bkm:Ref90505575}Long vowel in adverbial position 3\\
\textit{wī laaʔ laa \textbf{niī} tʲāʔ ntikʷiʔntakǫʔ}\\
\glll wī laaʔ.laa \textbf{niī} tʲāʔ nti- kʷiʔ =ntakǫʔ \\
\Conj{} like.so.be now still \Hab{}- speak =a.lot(\Third{})\\
v:1 \ref{zen3adv} \ref{zen3adv} \ref{zen6adv} \ref{zen10am}{}- \ref{zen13base} =\ref{zen15adv}\\
\glt `And like so, at the time he still spoke a lot.' (santa maria1 4:28)
\z

 %longdistance
It should be noted that the presence of bimoraic \textit{disyllabic} forms in position 5 (e.g.; \textit{tala} `for sure') suggests that the lack of forms with long vowels in position 5 and perhaps position 6 could be due to chance, since the inventory of forms that occur in those positions is limited.

\subsection{Segmental processes}
\label{bkm:Ref90475064}
Several segmental phonological processes are observable in Zenzontepec Chatino, especially in verbal aspect-mood inflection.

\subsubsection{Vowel elision}
\label{bkm:Ref90494223}
Vowel hiatus is not permitted in several spans of the verbal planar structure, and vowel elision occurs due to this constraint. There are minimal and maximal domains to distinguish because some positions do not provide observable evidence.

\subsubsubsection{Vowel elision (minimal), positions 7{}--13}
\label{bkm:Ref113308269}
Within the span of positions 10{}--13, where hiatus would occur, one of two vowels elides; the details are not quickly formalizable in rule notation but are explained in more depth elsewhere (\citealt{Campbell2011,Campbell2019}). This is illustrated in the aspect-mood inflection of vowel-initial verb stems that combine with vowel-final aspect-mood formatives. In the following examples, each verb belongs to a distinct inflectional class based on the allomorphy of aspect-mood prefixes and tonal alternations. In (\ref{bkm:Ref14616697}) the /a/ of the stem \textit{{}-akʷiɁ} `speak' in position 13 elides in contact with the vowel /i/ of the potential mood and habitual aspect prefixes in position 10.


\ea\label{bkm:Ref14616697}Aspect/mood inflection for verb \textit{{}-akʷiɁ} `speak'\\ 

Potential mood \hspace{1.05cm}  /k\textbf{i-a}kʷiɁ/ \hspace{1.15cm}[k\textbf{i}{}-kʷiɁ] \hspace{1.15cm}`will speak' \\

Habitual aspect\hspace{1cm}     /nt\textbf{i-a}kʷiɁ/ \hspace{1cm}  [nd\textbf{i}{}-kʷiɁ] \hspace{1cm}`speaks' \\

Progressive aspect\hspace{0.55cm} /nt͡ʃ{}-akʷiɁ/ \hspace{1cm} [nd͡ʒakʷiɁ]  \hspace{0.9cm}`is speaking' \\

Perfective aspect\hspace{0.9cm} /j-akʷiɁ/\hspace{1.33cm} [jakʷiɁ] \hspace{1.25cm} `spoke' \\

\z 


The causative prefix \textit{u-} in position 11 elides when following the vowels /e/ and /a/ of the progressive and perfective aspect markers, respectively (\ref{bkm:Ref84415676}). No derivational prefixes that occur in position 12 contain a vowel that would illustrate the process at the juncture 11{}--12.


\ea\label{bkm:Ref84415676}Aspect/mood inflection for verb \textit{{}-u-lukʷā} `sweep (tr.)'\\

Potential mood\hspace{1.05cm}  /ki-u-lukʷā/                 \hspace{1cm}[kulukʷā]  \hspace{1cm}`will sweep' \\
Habitual aspect\hspace{1cm}   /nti-u-lukʷā/                 \hspace{0.88cm}[ndulukʷā]  \hspace{0.8cm}`sweeps' \\
Progressive aspect\hspace{0.55cm} /nt\textbf{e-u}{}-lukʷā/  \hspace{0.8cm}[nd\textbf{e}lukʷā]  \hspace{0.8cm}`is sweeping'\\
Perfective aspect\hspace{0.8cm} /nk\textbf{a-u}{}-lukʷā/  \hspace{0.7cm}[nɡ\textbf{a}lukʷā]  \hspace{0.8cm}`swept'\\
\z

Vowel elision is also observed in the auxiliary span of positions 7{}--9. For example, the vowel /i/ in the potential mood, habitual aspect and perfective aspect prefixes (position 7) on the verb `feed, make eat' in (\ref{bkm:Ref84416360}) is elided by the following vowel /e/ of the causative auxiliary -\textit{ē+} in position 9. Note that the velar /k/ palatalizes automatically when preceding /e/, and only /e/ (/kee/ `stone' → [kʲee]), so the phonetic palatalization we see in potential mood and progressive aspect forms is due to that process.


\ea\label{bkm:Ref84416360}Aspect/mood inflection for verb \textbf{{}-}\textbf{\textit{e}}\textit{+k-aku} `feed' \\ 
    \begin{tabular}{llll}
        Potential mood      & /k\textbf{i-e}+k-aku/     & [kʲ\textbf{e}kaku]    & `will feed' \\
        Habitual aspect     & /nt\textbf{i-e}+aku/      & [nd\textbf{e}kaku]    & `feeds' \\
        Progressive aspect  & /nte-k-ē+k-aku/           & [ndekʲēkaku]     & `is feeding' \\
        Perfective aspect   & /nkʷ\textbf{i-ē}+k-aku/   & [nɡʷ\textbf{ē}kaku]  & `fed' \\
    \end{tabular}
\z

Vowel elision \textit{does not} occur when the second element in a sequence is a vowel-initial short pronoun in subject function in position 17, as shown in (\ref{bkm:Ref84417539}). The same example illustrates the lack of vowel elision when the same pronoun functions as inalienable possessor following the head noun in position 18.


\ea\label{bkm:Ref84417539}Lack of vowel elision in a short pronoun \\ 
\textit{[nde.Ɂnẽ\textbf{.\={\~{u}}}Ɂ  ......hut\textbf{i.ū}Ɂ]}\\
\glll {} $/$nte- Ɂne \textbf{=ūɁ} hnʲá lóʔō  huti \textbf{=ūɁ}$/$ \\
v: \ref{zen10am}{}- \ref{zen13base} =\ref{zen17sub} \ref{zen18np} \ref{zen18np} \ref{zen18np} =\ref{zen18np} \\
{} \Prog{}- do \textbf{=\Third\Pl{}} work \textsc{with} father \textbf{=\Third\Pl{}} \\
\glt `They were working with their father.' (michen 1:49)  
\z

\subsubsubsection{Vowel elision (maximal) (positions 3{}--16)}
\label{bkm:Ref113308340}
The vast majority of realized vowel sequences in Zenzontepec Chatino occur at the juncture between short pronouns in position 17 and a preceding element, which in the verbal planar structure may be any position from 13 through 16. Otherwise, the lack of vowel elision is not easily observable because Zenzontepec Chatino phonotactics strongly prefer syllable onsets. Only a handful of native lexical forms begin with a vowel /i/, and these present the main examples of domains in which elision does not occur where it imaginably could. We can fracture the vowel elision domain because there are positions around the verbal complex in which the elements that may occur do not provide instances where vowels could occur in sequence. No adverbial or modal elements in positions 3{}--6 and no aspect-mood prefixes in position 7 begin with vowels. The example in (\ref{bkm:Ref113100526}) shows a fronted subject noun phrase in the zone in position 2, where the form \textit{ītsáʔ} `word', `thing' is vowel-initial and not elided despite the final vowel of the preceding quantifier.


\ea\label{bkm:Ref113100526}Lack of vowel elision \\
\textit{kenaʔa ītsáʔ ntetaʔą tī hiʔ\={\k{ı}} tselā juu}\\
\glll kenaʔa ītsáʔ nte- taʔą tī hiʔ\={\k{ı}}  tselā.juu\\
v:2  \ref{zen2np} \ref{zen10am}{}- \ref{zen13base} \ref{zen18np} \ref{zen18np} \ref{zen18np} \\
many thing \Prog{}- pass(.\Third{}) \Tplz{} \Dat{} world\\
\glt `Many things pass in the world.' (lengua tlaco 58:06)  
\z

No post-verbal adverbials or essence elements in positions 15 or 16 begin with vowels. Therefore, the maximal domain of vowel elision spans positions 3{}--16.

\subsubsection{Palatalization of non-sibilant coronals} 

Non-sibilant, non-rhotic coronal consonants /t/, /n/, /l/ palatalize when they follow [i] in certain contexts.

\subsubsubsection{Palatalization (minimal), positions 10{}--13}
\label{bkm:Ref113308359}
In the verb \textit{\nobreakdash-nāá} `get cleared (field)' the initial /n/ of the stem (position 13) palatalizes only in the potential mood and habitual aspect forms, whose prefixes (position 10) end in /i/ (\ref{bkm:Ref8680410}).


\ea\label{bkm:Ref8680410}Aspect/mood inflection for verb \textit{{}-nāá} `get cleared (field)'\\
\begin{tabular}{llll}
     Potential mood     & /k\textbf{i}{}-nāá/   & [k\textbf{īnʲ}\={\~{a}}\'{\~{a}}]     & `will get cleared' \\
    Habitual aspect     & /nt\textbf{i}{}-nāá/  & [nd\textbf{īnʲ}\={\~{a}}\'{\~{a}}]    & `gets cleared' \\
    Progressive aspect  & /nte-nāá/             & [ndē\textbf{n}\={\~{a}}\'{\~{a}}]     & `is getting cleared' \\
    Perfective aspect   & /nku-nāá/             & [ŋgū\textbf{n}\={\~{a}}\'{\~{a}}]     & `got cleared' \\
\end{tabular}
\z 

Palatalization also occurs between the iterative prefix \textit{i-} (position 11) and a stem-initial coronal, as shown in (\ref{chatino:ex:key:44a}), while the same consonant of the same stem does not palatalize in the absence of the iterative prefix (\ref{chatino:ex:key:44b}).


\ea\label{bkm:Ref84682903}Palatalization of coronal, and lack thereof \\
            \ea\label{chatino:ex:key:44a} { 
            \textit{nkʷ\textbf{ītʲ}ākǫ́Ɂ tī na kūɁwíiɁ hnā tuɁwa na lometāaɁ} \\ 
            \glll nkʷ- \textbf{i-} \textbf{tʲ-} ākǫ́Ɂ tī na kūɁwí =VɁ hiɁ\={\k{ı}} nā tuɁwa na lometā =VɁ\\
            v:\ref{zen10am}{}- \ref{zen11cau}{}- \ref{zen12tr}{}- \ref{zen13base} \ref{zen17sub} \ref{zen17sub} \ref{zen17sub} =\ref{zen17sub} \ref{zen18np} \ref{zen18np} \ref{zen18np} \ref{zen18np} \ref{zen18np} =\ref{zen18np}\\ 
            \Pfv{}- \Iter{}- \Trvz{}- close \Tplz{} \Def{} drunk =\Ana{} \Obj{} \Def{} mouth \Def{} bottle =\Ana{}\\
            \glt `The drunk closed the opening of the bottle again.' (amigo borracho 5:02)
            }
            \ex \label{chatino:ex:key:44b}{
            \textit{nte\textbf{t}ākǫ́Ɂwą niɁií hiɁ\={\k{ı}}ja} \\ 
            \glll nte- \textbf{t-} ākǫ́Ɂ =wą niɁi =\'{V} hiɁ\={\k{ı}} =ja\\
            v:\ref{zen10am}{}- \ref{zen12tr}{}- \ref{zen13base} =\ref{zen17sub} \ref{zen18np} =\ref{zen18np} \ref{zen18np} =\ref{zen18np}\\ 
            \Prog{}- (\Caus{})\Trvz{}- close =\Second\Pl{} house =\Dist{} \Obj{} =\First\Excl{}\\
            \glt `You (pl.) are closing our house there.' (amigo borracho 2:44)
            }
            \z 
\z 


Although the inventory of auxiliaries is limited, the completive auxiliary \textit{{}-ta+} `finish' has aspect-mood inflection that allows palatalization to be observed in the auxiliary span as well, as shown in (\ref{bkm:Ref90496872}).


\ea\label{bkm:Ref90496872}Palatalization in auxiliary span (positions 7{}--9)\\
\textit{nkʷ\textbf{itʲa}jālú kitsąʔ ke} \\ 
\glll {} nkʷ\textbf{i}{}- \textbf{tʲa}+ j- ālú kitsąʔ ke\\
v: \ref{zen7am}- \ref{zen9aux}+ \ref{zen12tr}{}- \ref{zen13base} \ref{zen17sub} \ref{zen17sub} \\ 
{} \Pfv{} finish+ \Itr{}- spill hair head(.\Third{})\\
\glt `Her hair finished falling out.' (mateya 3:41)
\z

Palatalization does not occur between two stems in a compound verb (positions 13 and 14), as shown in (\ref{bkm:Ref84690089}) nor does it occur between a verb stem and a short pronoun, as shown in (\ref{bkm:Ref90492981}).


\ea\label{bkm:Ref84690089}No palatalization at compound juncture \\
$[$kátī kʲee ŋɡaʔn\textbf{ĩt}ēʔẽ́hnáʔ$]$ (*[kátī kʲee ŋɡaʔn\textbf{ĩtʲ}ēʔẽ́hnáʔ])  \\
\glll {} kátī kee nka- ʔn\textbf{i} \textbf{+t}ēʔę́ =hnáʔ  \\
v: \ref{zen2np} \ref{zen2np} \ref{zen10am}{}- \ref{zen13base} +\ref{zen14com} =\ref{zen15adv} \\ 
{} seven stone \Pfv{}- hit +\Tr{}.be.located =forcefully(.\Third{})\\
\glt `seven stones he forcefully threw.' (no hay brujos 1:19)
\z



\ea\label{bkm:Ref90492981}No palatalization at subject pronoun juncture\\
$[$...ŋguhnĩĩ\textbf{n}ã$]$ (*[ŋguhnĩĩ\textbf{nʲ}ã])\\
\glll {} Tī nāʃíʔi laaʔ nku- hnii =na nkā    \\
v: \ref{zen3adv} \ref{zen3adv} \ref{zen3adv} \ref{zen10am}{}- \ref{zen13base} =\ref{zen17sub} \ref{zen20ten} \\ 
{} \Tplz{} \Neg{} like.so \Pfv{}- grow =\First\Incl{} \Pst{}\\
\glt `We (incl.) did not grow up like that in the past.' (antes aparatos 41:23)
\z

The preceding discussion shows that positions 10{}--13 are the minimal domain of palatalization that includes the verb root.

\subsubsubsection{Palatalization (maximal), positions 7{}--13}
\label{bkm:Ref113308369}
Among the inventory of auxiliaries, four of them end in /a/ (GO, COME, START, FINISH) and one ends in /e/ (CAUS). There are a handful of verbs for which the iterative marker occurs in the auxiliary position 9 instead of the usual prefixal position 11, such as \textit{nkʷ}\textbf{\textit{\nobreakdash-i}}\textit{+k-ikʷą} `restitch'; however, none of these cases display an initial coronal consonant in the main verb span that could undergo palatalization if it were to apply at the juncture [9]-[10]. Thus, we can speak of a maximal domain of palatalization that includes positions 7{}--13 since palatalization cannot be observed to fail to apply within positions 7{}--10. Otherwise, in examples such as (\ref{bkm:Ref90492981}), palatalization does not occur among the particles in position \ref{zen3adv}.


\subsubsection{Nasality spreading, positions 13{}--17} 
\label{bkm:Ref113308377}
Vowel nasality in vowel-initial person markers spreads regressively to a stem, if only a laryngeal consonant, or no consonant, intervenes. Such nasality will further regressively spread within a stem across a medial laryngeal consonant but not across a non-laryngeal consonant (\ref{bkm:Ref85012114}). The example in (\ref{bkm:Ref85012114}) illustrates that the spreading within the main verb span does not reach position 10. This is notable, since in the discussion so far, this is the only pattern observed that includes position 13 but not also position 10.

\ea\label{bkm:Ref85012114}Regressive spreading of vowel nasality \\
\ea  /ki-ʃaʔa/  \rightarrow  [kiʃaʔa]   `will scream' \\ 
\ex /ki-ʃaʔa=ą/  \rightarrow  [kiʃãʔãã]   `we will scream' \\
\ex /nka-húʔū/  \rightarrow  [ŋɡahúʔū]  `got embarrassed' \\
\ex /nka-húʔū=ą̄Ɂ/  \rightarrow  [ŋɡah\textbf{ṍ}ʔ\textbf{ȭȭ}ʔ]   `I got embarrassed' \\
\ex /k-alaʔ/   \rightarrow [kalaʔ]    `will hold'  \\
\ex /k-alaʔ=ą̄Ɂ/  \rightarrow  [kal\textbf{ã}ʔã̄Ɂ]  `I will hold'
\z 
\z 

% \begin{tabularx}{0.9\textwidth}{XXXXXX}
% /ki-ʃaʔa/   & [kiʃaʔa]  & `will scream'     & /ki-ʃaʔa=ą/    & [kiʃãʔãã]     & `we will scream' \\
% /nka-húʔū/  & [ŋɡahúʔū] & `got embarrassed'  & /nka-húʔū=ą̄Ɂ/  & [ŋɡah\textbf{ṍ}ʔ\textbf{ȭȭ}ʔ]  & `I got embarrassed' \\
% /k-alaʔ/    & [kalaʔ]   & `will hold'     & /k-alaʔ=ą̄Ɂ/    & [kal\textbf{ã}ʔã̄Ɂ]  & `I will hold' \\
% \end{tabularx}

The verbal examples in (\ref{bkm:Ref85012114}) illustrate regressive nasality spreading from position 17 to 13, and we can also observe nasal spreading from position 17 to position 15 in (\ref{bkm:Ref90496894}).  


\ea\label{bkm:Ref90496894}Nasality spreading from position 17 to 15\\
$[$\textit{nt͡ʃuwetīʔk\textbf{ãʔã́ã}}$]$ \\ 
\glll tī nu nālá nkʷítsą hiɁ\={\k{ı}} -na nt͡ʃ- uwe =tīʔ \textbf{=kāʔá} =\textbf{ą}\\
v:- - - - - - \ref{zen10am}{}- \ref{zen13base} =\ref{zen16ess} =\ref{zen15adv} =\ref{zen17sub} \\
\Cond{} \Sub{} \Neg{}.exist child \Gen{} \First\Incl{} \Prog{}- get.ground =living.core =also =\First\Incl{}\\
\glt `If we don't have children then we are also sad.' (ntetakan7 jute7 4:44) 
\z

We can observe vowel nasality spread in the other direction, from a stem to a following vowel-initial person marker that otherwise has no nasal vowel, as in the 3\textsuperscript{rd} person plural/nonspecific pronoun \textit{=ūɁ} shown in the second example in (\ref{bkm:Ref85014225}b).


\ea\label{bkm:Ref85014225}Vowel nasality spreading and not spreading from position 13 to 17\\
    \ea /k-alaʔ=ūɁ/ \hspace{1cm} [kalaʔūɁ] \hspace{1cm} `they will hold'\\
    \ex /nakʷę=ūɁ/ \hspace{1cm} [nakʷẽ\textbf{ũ̄}Ɂ] \hspace{1cm} `they said' \\
    \z
\z 


% \ea\label{ex:key:51} Nasality spreading from position 17 to 15 \\
% $[$ \textit{nt͡ʃuwetīʔk\textbf{ã̄}ʔ\textbf{á̃}ã} $]$ \\ 
% \glll tī nu nālá nkʷítsą hiɁį̄ -nā nt͡ʃ- uwe =tīʔ =\textbf{kāʔá} {=}\textbf{ą}\\
% v:- - - - - - \ref{zen10am}{}- \ref{zen13base} =\ref{zen16ess} =\ref{zen15adv} =\ref{zen17sub} \\
% \Cond{} \Sub{} \Neg{}.exist child \Gen{} \First\Incl{} \Prog{}{}- get.ground =essence =also =\First\Incl{}\\
% \glt `If we don't have children then we are also sad.' (ntetakan7 jute7 4:44) 
% \z

In the example in (\ref{bkm:Ref113107741}), we observe that the vowel nasality of the final mora of the object marker \textit{hiʔį̄} in position 18 does not progressively spread through the initial syllable of the following form with initial glottal fricative. Thus, nasality spreading is not operative beyond position 17.


\ea\label{bkm:Ref113107741}Progressive nasality spreading not applying in position 18 \\
\textit{nkāsāʔ\'{ą}tsǫʔju hiʔ\={\k{ı}} hutiju } (*$[$ \textit{nkāsāʔ\'{ą}tsǫʔju hiʔ\={\k{ı}}  hũtiju} $]$) \\ 
    \glll {} nkā- sāʔ\'{ą} +tsǫʔ =ju hiʔ\={\k{ı}} huti =ju \\
    v: \ref{zen10am}{}- \ref{zen13base} +\ref{zen14com} =\ref{zen17sub} \ref{zen18np} \ref{zen18np} =\ref{zen18np} \\ 
{} \Pfv.\Caus{} be.attached +back =\Third\Sg\M{} \Obj{} father =\Third\Sg\M{} \\
    \glt `He carried his father (on his back) ...' (santa maria2 5:16)  
\z

\subsubsection{Vowel fusion, positions 13{}--17} 
\label{bkm:Ref98189638}
The first-person singular \textit{=ą̄ʔ} and first-person inclusive \textit{=ą} short pronouns elongate or undergo fusion of vowel quality with the final vowel of a preceding element (in positions 13, 14, 15, or 16), as shown in (\ref{bkm:Ref90494342}). 


\ea\label{bkm:Ref90494342}Vowel fusion at position 17\\ 
/nte-Ɂn\textbf{e=ą̄}Ɂ/\hspace{0.53cm} [nde.Ɂn\textbf{\~e\={\~e}}Ɂ] 
\hspace{2.75cm} `I am doing'\\
/ts-a+lóɁ\textbf{ō=ą}/\hspace{0.44cm}  [tsal\'{\~{o}}Ɂ\textbf{\={õ}õ}] \hspace{3cm} `We'll go to leave it'\\
/nku-hw\textbf{ī=ą̄}Ɂ/\hspace{0.39cm}  [ŋɡu.hɸ\textbf{\={\~{e}}\'{\~{e}}}Ɂ]  {\textasciitilde}  [ŋɡu.hɸ\textbf{\={\~{i}}\'{\~{i}}}Ɂ] 
\hspace{0.6cm}`I got'\\
/ki-is\textbf{u=ą}/\hspace{1cm}  [kis\textbf{õõ}]  {\textasciitilde}  [kis\textbf{ũũ}] 
\hspace{1.92cm} `We (incl.) will pay'\\
\z 


Note that this process is in contrast to what occurs between positions 10 or 11 and a verb root's initial vowel in position 13, where vowel elision, not fusion, is observed (\sectref{bkm:Ref90494223}).

\subsection{Tonal processes} 
\label{bkm:Ref109738706}
As discussed earlier, Zenzontepec Chatino has a privative tone system, in which a mora may be specified for H tone, M tone or no tone (∅). Toneless strings display a default intonational declination from mid to low pitch; no tones are inserted on toneless moras, and no intonational boundary tones have been encountered. Zenzontepec Chatino has a relatively low tonal density, with about 60\% of basic vocabulary bearing no lexical tone \citep{Campbell2014}. Thus, whole utterances may be toneless, gradually descending from mid to low pitch within a speaker's range. The main tonal processes are H tone spreading, H and M tone downstep, and M tone replacement (\citealt{Campbell2014,Campbell2016}). 

\subsubsection{H tone spreading (positions 1{}--21)}
\label{bkm:Ref112148431}
H tone spreads progressively through subsequent toneless moras until another tone, or pitch reset (see \sectref{bkm:Ref90284807}), occurs. In example (\ref{bkm:Ref90242254}) the H tone of the final mora of the existential predicate \textit{nk-ā+tāká} spreads until it reaches the M tone of the form \textit{nʲatę̄} `person' (\figref{fig:chat:key:2}).


\ea\label{bkm:Ref90242254}H tone spreading\\
\textit{nkātāká tsaka nʲatę̄} \\ 
\glll nk- ā +tāká tsaka nʲatę̄\\
v:\ref{zen10am}{}- \ref{zen13base} +\ref{zen14com} \ref{zen17sub} \ref{zen17sub} \\ 
\Pfv{}- be +exist one person\\
\glt `There was a person.' (cotita 0:19)
\z

\begin{figure}[p]
    \centering
    \includegraphics[height=.45\textheight]{figures/chat_2_nkata.png}
    \caption{H tone spreading interrupted by M tone}
    \label{fig:chat:key:2}
\end{figure}

\begin{figure}[p]
    \centering
    \includegraphics[height=.45\textheight]{figures/chat_3_nkuju_u.png}
    \caption{Declination and rise to M tone}
    \label{fig:chat:key:3}
\end{figure}

To demonstrate that the intervening moras are in fact toneless, consider the example in (\ref{bkm:Ref90242544}), which has much the same meaning as example (\ref{bkm:Ref90242254}) but contains a different, toneless existential predicate. Intonational declination (from mid-level pitch slowly descending) is observed throughout the tonelessness of the clause until the rise to the final M tone of \textit{nʲatę̄} `person' (\figref{fig:chat:key:3}).


\ea\label{bkm:Ref90242544}Declination and M tone target\\
\textit{nkjuɁu tsaka nʲatę̄} \\
\glll nk- j- uɁu tsaka nʲatę̄\\
v:\ref{zen10am}{}- \ref{zen12tr}{}- \ref{zen13base} \ref{zen17sub} \ref{zen17sub} \\ 
\Pfv{}- \Itr{}- be.inside one person\\
\glt `There was a person.' (ketu kela7 china7 0:43)
\z

Finally, pitch reset at the start of another intonational unit is shown to block H tone spreading in example (\ref{bkm:Ref90497515}). The final H tone of the first relative clause would spread through the following toneless subordinator of the second relative clause, but instead, pitch reset occurs and declination from a mid-level pitch is observed to reinitiate on the subordinator (\figref{fig:chat:key:4}). Pitch reset occurs here because non-restrictive relative clauses have their own prosodic packaging apart from their matrix clause \citep{Campbell2021a}. In this case, there is also a pause, but the pause does not trigger the reset. There are examples with longer pauses beyond which tone spreading continues.


\ea\label{bkm:Ref90497515}Pitch reset interrupts H tone spreading\\
\textit{tatījá nʲatę̄ nu t͡ʃu ʔne \textbf{hnʲá} \textbf{nu} n-tūkʷá tī ntsukʷāʔ} \\ 
\glll {} tatījá nʲatę̄ nu t͡ʃu ʔne \textbf{hnʲá} \textbf{nu} n- tūkʷá tī ntsukʷāʔ\\
v: \ref{zen2np} - - - \ref{zen10am}{}.\ref{zen13base} \ref{zen18np} \ref{zen1con} \ref{zen10am}{}- \ref{zen13base} \ref{zen18np} \ref{zen18np} \\ 
{} all person \Sub{} \Hum{} \Hab{}.do work \Sub{} \Hab{}- plant \Tplz{} corn\\
\glt `All of the people who work, those who plant corn ... ' (luna y siembra 0:28)
\z

\begin{figure}
    \centering
    \includegraphics[height=.45\textheight]{figures/chat_4_tatija.png}
    \caption{Pitch reset blocks H tone spreading}
    \label{fig:chat:key:4}
\end{figure}

In contrast, pitch reset does not occur at the relative clause boundary in (\ref{bkm:Ref90243618}), where the H tone of the noun \textit{kʷējáʔ} `time' spreads through the subordinator and verb of the following relative clause (\figref{fig:chat:key:5}). This is a restrictive relative clause, and H tone spreading is not blocked at restrictive relative clause junctures.


\ea\label{bkm:Ref90243618}H tone spreading into restrictive relative clause\\ 
\textit{tala tāká kʷējáʔ nu nti- ʔnʲa nʲatę̄} \\ 
\glll {} tala tāká kʷējáʔ nu nti- ʔnʲa nʲatę̄\\
v: \ref{zen5md} \ref{zen10am}.\ref{zen13base} \ref{zen17sub} \ref{zen1con} \ref{zen10am}{}- \ref{zen13base} \ref{zen17sub} \\ 
{} for.sure exist time \Sub{} \Hab{}- clear.field person\\
\glt `For sure there are times when people clear fields.' (kuna7a kusu7 4:07)
\z

\begin{figure}
    \centering
    \includegraphics[height=.45\textheight]{figures/chat_5_tala.png}
    \caption{H tone spreading into restrictive relative clause}
    \label{fig:chat:key:5}
\end{figure}


\subsubsection{Declination and pitch reset, positions 1{}--21}
\label{bkm:Ref90284807}\label{bkm:Ref113308417}
Now that the intonational pattern of declination along a string of tonally unspecified moras has been introduced, a question arises: What is the domain of this process? We already saw that pitch reset occurs at the start of a parenthetical remark (\figref{fig:chat:key:4}).

In toneless declination the pitch will continue to decline until a tone occurs, or until the pitch is reset to a phonetically-mid level. Pitch reset between two utterances is illustrated by the example in (\ref{bkm:Ref90285149}). In the first clause, after the M tone at the end of the form \textit{laʔā}, the pitch begins to decline through the following, toneless form \textit{nikʷę=ą} (see \figref{fig:chat:key:6}). After a pause, the next clause is entirely toneless, but instead of continuing the declination from the previous clause, the pitch is reset to a mid level, whence it begins to decline through the entire toneless utterance.


\ea\label{bkm:Ref90285149}Declination, pitch reset, more declination\\
\textit{nkʷítsą tiʔi laʔā nikʷęę} \\
\glll nkʷítsą tiʔi laʔā nikʷę =ą\\
v:2 \ref{zen2np} \ref{zen3adv} \ref{zen10am}.\ref{zen13base} =\ref{zen17sub} \\ 
     child poor like.so \Irr{}.say =\First\Incl{}\\
\glt `An orphan, so we say.' \\

\textit{ntetaʔãju laha niʔi} \\ 
\glll nte- taʔã =ju laha niʔi\\
v:\ref{zen10am}{}- \ref{zen13base} =\ref{zen17sub} \ref{zen19adv} \ref{zen19adv} \\ 
\Prog{}{}- go.around =\Third\Sg{}.\M{} between house\\
\glt `He was walking around in the street.' (juan oso 0:11)
\z

\begin{figure}[p]
    \centering
    \includegraphics[height=.45\textheight]{figures/chat_6_nkwitsa.png}
    \caption{Declination and pitch reset}
    \label{fig:chat:key:6}
\end{figure}


Further study is needed on the information structural and discursive factors that determine the domains of declination and pitch reset. However, we can posit that declination may maximally span the entire verbal planar structure, positions 1{}--21, but likely beyond that in coordination, because coordinated spans may fall together within one intonational contour as shown in \sectref{bkm:Ref113226455}.

\subsubsection{Downstep, positions 1{}--21}
\label{bkm:Ref96876657}\label{bkm:Ref113308426}
H tone causes a following H tone to downstep to a slightly lower pitch and a following M tone to sharply downstep to low pitch; the process is allotonic and downstepped H and M tones still behave phonologically as such. In (\ref{bkm:Ref90286148}), the final H tone of the adverbial \textit{=kāʔá} that follows the first verb causes the initial H tone of the second verb to downstep (\figref{fig:chat:key:7}). The second, downstepped H tone recovers to a high pitch as it spreads through the following entirely toneless restrictive relative clause.

\begin{figure}[p]
    \centering
    \includegraphics[height=.45\textheight]{figures/chat_7_tsaaka.png}
    \caption{H tone downstep}
    \label{fig:chat:key:7}
\end{figure}


\ea\label{bkm:Ref90286148}H tone downstep and spreading of the downstepped H\\
\textit{tsaakāʔá tʲánaja ʃaaʔ nu t͡ʃu tsaa} \\
\glll ts- aa =kāʔá tʲána =ja ʃaaʔ nu t͡ʃu ts- aa\\
v:\ref{zen10am}{}- \ref{zen13base} =\ref{zen15adv} \ref{zen10am}.\ref{zen13base} =\ref{zen17sub} \ref{zen18np} \ref{zen18np} \ref{zen18np} \ref{zen18np}{}- \ref{zen18np} \\
\Pot{}{}- go =again \Pot{}.look.for =\First\Excl{} other \Sub{} \Hum{} \Pot{}{}- go\\
\glt `Another can go; we're going to look for another who can go.' (ku7wi lo jo7o 2:34)
\z



Example (\ref{bkm:Ref90505608}) illustrates two instances of M tone downstep. The final H tone of the initial first-person singular independent pronoun \textit{nāáʔ} downsteps the M tone of the following vocative particle. The H tone of the negator particle spreads through the following compound verb and downsteps the M tone of the first person singular pronominal enclitic  (\figref{fig:chat:key:8}).

% Like H tone spreading and declination, downstep is blocked at pitch reset. In the example in (\ref{bkm:Ref90505608}), the final H tone of the first-person singular independent pronoun \textit{nāáʔ} does not cause downstep of the H tone of the following negation particle (\figref{fig:chat:key:8}). Instead, the pitch resets there. This is an emphatic topicalized pronoun followed by a pause. Syntactically, it pertains to the second clause, but it is intonationally packaged with the preceding clause. This same example also serves to illustrate two instances of M tone downstep.

\ea\label{bkm:Ref90505608} M tone downstep \\
\textit{nāáʔ nī t͡ʃoō nakʷę nāáʔ ná ntsuʔuntoǭʔ hiʔ\'{\k{ı}}} \\
\glll {} nāáʔ nī t͡ʃoō nakʷę nāáʔ ná n- tsuʔu +ntoo =ą̄ʔ hiʔ\'{\k{ı}}\\
v: - - - - \ref{zen2np} \ref{zen5md} \ref{zen10am}{}- \ref{zen13base} +\ref{zen14com} =\ref{zen17sub} \ref{zen18np} \\
{} \First\Sg{} \Voc{} friend say(.3) \First\Sg{} \Neg{} \Stat{}- be.inside +face =\First\Sg{} \Obj{}.\Second\Sg{}\\
\glt `“Me, friend', he said. Me ... I don't know you.'  (ku7wi lo jojo 4:39)
\z

\begin{figure}
    \centering
    \includegraphics[height=.45\textheight]{figures/chat_8_naa.png}
    \caption{M tone downstep}
    \label{fig:chat:key:8}
\end{figure}


\subsubsection{Mid tone replacement, positions 10{}--17}
\label{bkm:Ref113308436}
A mid tone on a monomoraic element in position 17 (short pronoun) is replaced by a H tone if and only if the preceding element has only a M tone on its final mora. The example in (\ref{bkm:Ref90108098}) illustrates this alternation on the third person plural dependent pronoun \textit{=ūɁ}. The sharp fall in pitch at the end of the first clause is due to the M tone of the pronoun being sharply downstepped by the final H tone of the essence form \textit{=rīké} `chest' (\figref{fig:chat:key:9}). The sharp pitch rise in the second clause is due to the H tone that has replaced the pronoun's M tone because the preceding verb bears only a final M tone.


\ea\label{bkm:Ref90108098}Alternation showing M tone replacement\\
{} \textit{lēɁ nkjalarīké\textbf{ūɁ} nkahnʲā\textbf{úɁ} jaą} \\ 
    \glll {} lēɁ nk- jala =rīké \textbf{=ūɁ} nka- hnʲā \textbf{=ūɁ} jaą \\
    v: \ref{zen1con} \ref{zen10am}{}- \ref{zen13base} =\ref{zen16ess} =\ref{zen17sub} \ref{zen10am}{}- \ref{zen13base} =\ref{zen17sub} \ref{zen18np} \\ 
    {} then \Pfv{}- fill =chest =\Third\Pl{} \Pfv{}- make \textbf{=\Third\Pl{}}  sweat.bath \\
    \glt `They made a plan and built a sweat bath.' (ni7 rosa 2:24)
\z

\begin{figure}
    \centering
    \includegraphics[height=.45\textheight]{figures/chat_9_le.png}
    \caption{M tone replacement on third-person plural pronoun}
    \label{fig:chat:key:9}
\end{figure}

Mid tone replacement occurs in position 16 as well if the specific tonal and moraic conditions apply. The essence element \textit{=tīʔ} `living core' displays its basic M tone following most tone melodies (\ref{bkm:Ref90239685}), and it bears H tone when the preceding element bears only a final M tone (\ref{bkm:Ref90239700}). 


\ea\label{bkm:Ref90239685}Essence predicates with no Mid tone replacement on \textit{=tīʔ}\\
j-a+toǫ=tīʔ \hspace{1cm} `like'\hspace{2.2cm}[lit. `go be standing living core']\\
nk-j-uʔu=tīʔ \hspace{0.77cm} `remember'\hspace{1.1cm}  [lit. `be inside living core']\\
nka-tāá=tīʔ \hspace{0.94cm} `think'\hspace{1.84cm}  [lit. `give living core']\\
\z 

\ea\label{bkm:Ref90239700}Essence predicates with Mid tone replacement on \textit{=tīʔ}\\
nku-teh\={ę}\textbf{=tíʔ} \hspace{0.7cm} `fall in love with' \hspace{0.1cm} [lit. `pass living core']\\
nkʷi-tsā\textbf{=tíʔ} \hspace{0.85cm} `forget' \hspace{1.64cm} [lit. `mistake living core']\\
\z 

The example in (\ref{bkm:Ref90505645}) illustrates that the element triggering Mid tone replacement of an essence form may occur as a postpound in position 14. 

\ea\label{bkm:Ref90505645}Mid tone replacement between positions 14 and 16\\
    \textit{tī ʔnehlʲū\textbf{tíʔ} nuʔu seʔju hā tsaa nāáʔ lóʔō nuʔu} \\
    \glll tī ʔne +hlʲū \textbf{=tīʔ} nuʔu seʔju hā ts- aa nāáʔ lóʔō nuʔu\\
    v:\ref{zen4con} \ref{zen10am}.\ref{zen13base} +\ref{zen14com} =\ref{zen16ess} \ref{zen17sub} \ref{zen17sub} \ref{zen1con} \ref{zen10am}{}- \ref{zen13base} \ref{zen17sub} \ref{zen18np} \ref{zen18np} \\ 
    \Cond{} \Pot{}.do +big =essence \Second\Sg{} sir \Conj{} \Pot{}{}- go \First\Sg{} \textsc{with} \Second\Sg{}\\
    \glt `Forgive me sir but I will go with you.' (muchacha ixtayutla 1:20)
\z

The example in (\ref{bkm:Ref90308462}) illustrates Mid tone replacement on the feminine singular pronoun \textit{=t͡ʃūʔ} in position 17 and also shows that the process does not occur between the two stems in a compound verb that occupy positions 13 and 14, which in this case display the moraic and tonal requisites that should trigger the process.


\ea\label{bkm:Ref90308462}Lack of M tone replacement between positions 13 and 14 \\
\textit{nkāhlʲūt͡ʃ\textbf{ú}ʔ tī niī} \\ 
    \glll {} nk- ā +hlʲū =t͡ʃ\textbf{ū}ʔ tī niī \\
    v: \ref{zen10am}{}- \ref{zen13base} +\ref{zen14com} =\ref{zen17sub} \ref{zen19adv} \ref{zen19adv} \\ 
    {} \Pfv{}- be +big =3\textsc{sg.f} \Tplz{} now \\
    \glt `Now yes, she has developed well.' (offered)
\z

Mid tone replacement thus affects only monomoraic elements in positions 16 and 17. No adverbial elements that occur in position 15 are monomoraic with a M tone, so we are not able to observe if the process applies there. Nevertheless, the domain in which this process occurs is positions 10{}--17, when a monomoraic element with M tone in position 16 or 17 immediately follows a form within the same span whose only tone is a M tone on its final mora.

\section{Play language and constituency, positions 10{}--13}
\label{bkm:Ref90496744}\label{bkm:Ref113308447}
There is a play language in which speakers transpose the initial syllable of a form to the end of the form \citep{Campbell2020}. The examples in (\ref{bkm:Ref90505658}) show a numeral, some basic nouns, and some inflected verbs (positions 10{}--13).

\newpage
\ea\label{bkm:Ref90505658}Play language basic forms\\
\begin{tabular}{llll}
\textbf{ká}tī & → & tí\textbf{ka} & `seven'\\
\textbf{kʷe}tǫ & → & tǫ́\textbf{kʷe} & `bee'\\
\textbf{kū}náɁa & → & nāɁá\textbf{ku} & `woman'\\
\textbf{kʷi}līʃí & → & lʲīʃí\textbf{kʷi} & `butterfly'\\
\textbf{nte}{}-lákʷi & → & lākʷí\textbf{nte} & `is boiling'\\
\textbf{nka}j-ūná & → & jūná\textbf{nka} & `cried'\\
\textbf{nt͡ʃ-ū}ná & → & ná\textbf{nt͡ʃu} & `is crying'\\
\textbf{k-a}ku & → & kú\textbf{ka} & `will eat'\\
\end{tabular}
\z 

The play language forms in (\ref{bkm:Ref90505481}) show that dependent pronouns expressing subject (position 17) do not fall in the target domain of the play language, as they are added after transposition has applied. The final example in (\ref{bkm:Ref90505481}) illustrates that a second stem in a compound verb (position 14) forms its own transposition domain apart from the first stem of the compound in the span 10{}--13.


\ea\label{bkm:Ref90505481}Short pronouns do not participate in the transposition\\
\begin{tabular}{llll}
\textbf{nkā}{}-sāɁą́=ju & → & sāɁą́\textbf{nka}=ju & `he wrote'\\
\textbf{nka}{}-ʃiti=t͡ʃūɁ & → & ʃītí\textbf{nka}=t͡ʃūɁ & `she laughed'\\
\textbf{nka}{}-ʃiti=ja & → & ʃītí\textbf{nka}=ja & `we (excl.) laughed'\\
\textbf{j-a}ku=wą & → & kú\textbf{ja}=wą & `you (pl.) ate'\\
\textbf{nt͡ʃ-ū}lá+\textbf{tu}Ɂwa & → & lá\textbf{nt͡ʃu}+Ɂwá\textbf{tu} & `(he) is singing'\\
\end{tabular}
\z 


The example in (\ref{bkm:Ref90501458}) shows a complete utterance in the play language. Note that the transposition occurs in the tonal melody domain of each form of a lexical class: it excludes function words. Also of note is that a glottal stop is inserted in the bimoraic monosyllabic vocative form \textit{t͡ʃoō} `friend' so that it can undergo transposition.


\ea\label{bkm:Ref90501458}Play language utterance\\
$[$kʷí\textbf{\textup{tu}} \textup{kʷę́}\textbf{\textup{na}} \textup{hiɁ\'{\k{ı}} laɁā Ɂó}\textbf{t͡ʃo}$]$\\
    \glll {} /tukʷi nakʷę hiɁį́ laɁā t͡ʃoō/\\
    v: 2 13 18 19 21 \\ 
    {} who \Pfv{}.say \Dat{}.\Second\Sg{} like.so friend.\Voc{}\\
    \glt `Who told you that, friend?'
\z

The play language, thus, applies to basic tonal melody domains (positions 10{}--13), and like many other phonological patterns, treats bound (pronominal) forms in position 17 very differently from positions in the tonal domain that includes the lexical verb root.

\section{Discussion}
\label{bkm:Ref90326435}
This study of constituency in Zenzontepec Chatino, following the methodology outlined by \citet{Tallman2021} produces interesting results.
\figref{fig:zenz:poooled_plot} is a convergence plot for the domains discussed in this chapter, showing phonological domains in yellow, morphosyntactic domains in blue and indeterminate domains in red.

\begin{figure}
    \includegraphics[width=\textwidth]{figures/chatino_pooled_plot.png}
    \caption{Zenzontepec Chatino constituency tests}
    \label{fig:zenz:poooled_plot}
\end{figure}

\subsection{A phonological word in Zenzontepec Chatino?} 
 
One can see from \figref{fig:zenz:poooled_plot} that the first layer contains eight converging tests: one morphosyntactic (minimal free occurrence) and seven which are phonological or indeterminate which might be more traditionally classified as ``morphophonological'' (paradigmatic tone melody; TAM tone alternation; 2sg tonal inflection; culminative H tone; culminative glottal stop [minimal]; palatalization [minimal]; and play language). Since seven of the tests are (morpho)phonological, the span of positions from 10 to 13 is a strong candidate for a phonological word in Zenzontepec Chatino. In light of recent research on phonological wordhood, it is somewhat surprising to find such a strong convergence of tests. \citet{Schiering2010}, for example, find that in Vietnamese (Austroasiatic) no single domain emerges as a best candidate for phonological word, and in Limbu (Sino-Tibetan) multiple constituents emerge as equally plausible phonological word domains. They argue, therefore, that phonological words are not a universal feature of human language as assumed by the prosodic hierarchy (e.g.; \citealt{Nespor1986}) and in other works \citep{Hall1999}, but rather they are emergent in language structure in the dynamics of language use and change (see also \citetv{chapters/10-Hup}).

As \citet[6]{dixonaikhenvald02} point out, a challenge in the study of wordhood is that the best candidate for grammatical word is often not isomorphic with the best candidate for phonological word in a language and the two types of tests should be distinguished. That the minimal free occurrence test in Zenzontepec Chatino aligns with the strongest phonological word candidate is also striking. Moreover, the patterns of deviation from biuniqueness we find in TAM tonal inflection and 2sg person tone are just as much morphosyntactic as they are phonological in nature \citep{Woodbury2019}. Such strong convergence points to general (not strictly phonological or grammatical) wordhood in the language and cries out for an explanation. The Verbal Core, as an inflectable lexical domain, is a fitting name for this constituent.

\subsection{Other convergences}

The second strongest layer in terms of convergences spans position 7 to 13 and includes the following: culminative glottal stop (maximal); culminative V nasality (minimal); culminative vowel length (minimal); vowel elision; and palatalization. This span in turn consists of the auxiliary span in positions 7{}--9 and the Verbal Core 10--13. Due to the fact that the second stem in compound verbs (position 14) is a separate domain for several sound patterns (tone melodies; culminativity constraints), the closest span in morphosyntactic tests is identified by the tests of non-permutability (rigid) and non-interruptability, which refer to positions 7{}--14.

Next we find a convergence of three unique tests over positions 1{}--21, that is, the entire verbal planar structure: H tone spreading, declination-pitch reset, and downstep. These tests are all prosodic in nature and point to a possible phonological constituent of \textit{utterance}. The autosegmental nature of lexical tone in Zenzontepec Chatino and its low tonal density afford for such large domains to be active in the prosodic structure. 

On the morphosyntactic side, the presence of essence elements as obligatory parts of verbal lexemes in position 16 and the obligatoriness of subject expression for speech act participants (1\textsuperscript{st} and 2\textsuperscript{nd} persons) in position 17 lead to the misalignments of the final edges of salient morphosyntactic and phonological domains. Since subjects can have scope over repeated subspans that precede them, the minimal repeated subspan test does not align with minimal free occurrence, non-permutability, or non-interruptability, and we find something more like a patchwork of not-quite-aligning domains of morphosyntactic tests. Therefore, positing a most promising candidate for morphosyntactic word in Zenzontepec Chatino is less straightforward than for a phonological word, and such conflicting evidence has been reported in other studies that challenge the notion of wordhood (\citealt{Evans2008}; \citealt{Bickel2017}).



% \begin{table}
%     \caption{Constituency tests and spans}
%     \label{tab:zenz:key:8}
%     \centering
%     \begin{tabularx}{\textwidth}{XXX}  \\
% \lsptoprule
% \textbf{Morphosyntactic} \textbf{tests}  & Span & Section\\ \midrule
% Free occurrence (min.) & 10{}--13  & \ref{bkm:Ref113307769}\\
% Free occurrence (large) & 7{}--17  & \ref{bkm:Ref113184795}\\
% Non-permutability (rigid) & 7{}--14  & \ref{bkm:Ref78989028}\\
% Non-permutability (scopal) & 7{}--17  & \ref{bkm:Ref113307808}\\
% Non-interruptability & 7{}--14  & \ref{bkm:Ref113307818}\\
% Min. repeated subspan & 5{}--16  & \ref{bkm:Ref113307833}\\
% Max. repeated subspan & 2{}--20  & \ref{bkm:Ref113226455}\\
%  & \\ \midrule
% \textbf{Morphophonological} \textbf{tests}  & Span & Section\\ \midrule
% Paradig. tone melody & 10{}--13  & \ref{bkm:Ref113307854}\\
% TAM tone alternations & 10{}--13  & \ref{bkm:Ref113308071}\\
% 2sg tone inflection & 10{}--13  & \ref{bkm:Ref113308088}\\
% Culminative H tone & 10{}--13  & \ref{bkm:Ref113308111}\\
% Culm. glottal stop (min.) & 10{}--13  & \ref{bkm:Ref113308189}\\
% Culm. glottal stop (max.) & 7{}--13  & \ref{bkm:Ref113308202}\\
% Culm. V nasality (min.) & 7{}--13  & \ref{bkm:Ref113308220}\\
% Culm. V nasality (max.) & 4{}--13  & \ref{bkm:Ref113308229}\\
% Culm. V length (min.) & 7{}--13  & \ref{bkm:Ref113308246}\\
% Culm. V length (max.) & 4{}--13  & \ref{bkm:Ref113308255}\\
% Vowel elision (min.) & 7{}--13  & \ref{bkm:Ref113308269}\\
% Vowel elision (max.) & 3{}--16  & \ref{bkm:Ref113308340}\\
% Palatalization (min.) & 10{}--13  & \ref{bkm:Ref113308359}\\
% Palatalization (max.) & 7{}--13  & \ref{bkm:Ref113308369}\\
% Nasality spreading & 13{}--17  & \ref{bkm:Ref113308377}\\
% Vowel fusion & 13{}--17  & \ref{bkm:Ref98189638}\\
% H tone spreading & 1{}--21  & \ref{bkm:Ref112148431}\\
% Declin. and pitch reset & 1{}--21  & \ref{bkm:Ref113308417}\\
% Downstep & 1{}--21  & \ref{bkm:Ref113308426}\\
% M tone replacement & 10{}--17  & \ref{bkm:Ref113308436}\\
% Play language & 10{}--13  & \ref{bkm:Ref113308447}\\
% \lspbottomrule
% \end{tabularx}
% \end{table}


\subsection{Essence elements and adverbials}

The main issue that arises in laying out the verbal planar structure is the variable ordering of (transcategorial) adverbials in position 15, a zone in which they may be iterated, and the essence element slot in position 16.

In some cases, the essence element precedes the adverbial (\ref{bkm:Ref90383583}), and in other cases, the order is reversed (\ref{bkm:Ref90383592}). However, the cases in which the essence element precedes the adverbial(s) display fusion and/or suppletion, and the essence element \textit{tīʔ} does not alternate freely with \textit{=rīké} in these lexemes as it does in others with the alternate order. 



\ea\label{bkm:Ref90383583}Essence element preceding adverbial\\
\textit{ná ntʲāá\textbf{tīʔtsoʔō}ǫ́ʔ hį̄} \\  
\glll {} ná n- tʲāá \textbf{+tīʔ} \textbf{=tsoʔō} =ą̄ʔ hiʔ\={\k{ı}}\\
v: \ref{zen5md} \ref{zen10am}{}- \ref{zen13base} +\ref{zen14com} =\ref{zen15adv} =\ref{zen17sub} \ref{zen18np} \\
{} \Neg{} \Hab{}- \Iter{}.give +living.core =good =\First\Sg{} \Obj{}(.\Third{})\\
\glt `I don't remember it well.' (leonardo 10:02)
\z

\newpage
\ea
\label{bkm:Ref90383592}Essence element following adverbial\\
\textit{ nkʷeja\textbf{kāʔátīʔ} tī na kojotē} \\  
\glll nkʷ -eja \textbf{=kāʔá} \textbf{=tīʔ} tī na kojotē\\
v:\ref{zen10am} {}-\ref{zen13base} =\ref{zen15adv} =\ref{zen16ess} \ref{zen17sub} - - \\
\Pfv{} {}-lie =also =living.core \Tplz{} \Def{} coyote\\
\glt `The coyote believed (him) again ...' (500 toads 5:45)
\z

In trying to resolve this issue, one must watch out for adverbial elements that appear to occur in an unexpected place, but are actually in the nominal domain. In example (\ref{bkm:Ref98078162}), the adverbial enclitic \textit{=kāʔá} is likely modifying the light-headed relative clause, a topical NP coreferential with the subject of the matrix clause.


\ea\label{bkm:Ref98078162} Adverbial \textit{=kāʔá} in the nominal domain\\
\textit{nu t͡ʃu ná ntejatīʔ\textbf{kāʔá} nʲāʔā ʔnetsǫʔūʔ hiʔ\={\k{ı}}} \\ 
\glll {} [nu t͡ʃu ná nte- ja =tīʔ] \textbf{=kāʔá} nʲāʔā ʔne +tsǫʔ =ūʔ hiʔ\={\k{ı}}\\
v: \ref{zen2np} - - - - - =\ref{zen2np} \ref{zen2np} \ref{zen10am}.\ref{zen13base} +\ref{zen14com} =\ref{zen17sub} \ref{zen18np} \\
{} \Sub{} \Hum{} \Neg{} \Prog{}- lie =living.core =also see.\Second\Sg{} \Pot{}.do +back =\Third\Pl{} \Dat{}(.\Third{})\\
\glt `Those who do not believe as well, they turn their backs on it.' (4bailes 11:37)
\z

A few verbs of cognition, like `know' display stem suppletion, reduced aspectual inflection (stative semantics), and what appears to be an essence element adjacent to an otherwise opaque verb root, as shown in (\ref{bkm:Ref90503744}).

\ea\label{bkm:Ref90503744}Fossilized essence element in verb of cognition\\
\textit{lēʔ ntʲō\textbf{tíʔ}kāʔá tī nāáʔ tula ʔneęʔ} \\ 
\glll {} lēʔ n- tʲō.\textbf{tíʔ} =kāʔá tī nāáʔ tula ʔne =\={ą}ʔ \\
v: \ref{zen1con} \ref{zen10am}{}- \ref{zen13base} =\ref{zen15adv} \ref{zen17sub} \ref{zen17sub} \ref{zen18np} \ref{zen18np} =\ref{zen18np} \\ 
{} then \Hab{}- \textbf{?.living.core} =again \Tplz{} \First\Sg{} what \Pot{}.do =\First\Sg{} \\
\glt `Then I do also know what I'm going to do.' (medicina2 2:43)
\z

In cases such as (\ref{bkm:Ref90503744}) adverbial elements can no longer occur between the original verb root and the essence element. All of these facts suggest that such cases are now best analyzed as compounds, with the essence element having been reanalyzed as a postpound stem and now occurring in position 14. Therefore, the decision was made to treat the default order as adverbial followed by essence element as in (\ref{bkm:Ref90383592}) above, despite the discontinuity of verbal lexemes that it entails. 

\section{Conclusion}
\label{bkm:Ref113366717}
Zenzontepec Chatino presents an interesting case in the cross-linguistic study of constituency. There is a multitude of observable sound patterns, both segmental and suprasegmental, and some convergence around a bi- or trimoraic constituent that includes a verb root and its derivational and inflectional prefixes. The study also illustrates that play language sheds useful light on constituency, and which, in this case, mostly aligns with other evidence. The method of using almost entirely naturalistic language use for this study has not very much limited how much detail could be provided for some of the morphosyntactic tests. On the other hand, the structure of the verbal lexical core and verbal complex is informed by extensive lexicographic work on the language and a resulting analytical database that includes roughly 10,000 lexemes, including the inflectional paradigms of roughly 1,500 simplex, derived, compound and phrasal verbs. Such lexicographic work informs the analysis of language structure in discourse. Finally, because of the nature of tone in Zenzontepec Chatino, its larger-domain behavior is able to, and in fact does, align with intonational patterns such as declination and pitch reset and discourse-grammatical distinctions such as that between restrictive vs. non-restrictive relative clauses (and perhaps other parenthetical remarks). Despite the recent challenges to notions of wordhood after long having presupposed that word constituents were manifested in all languages, Zenzontepec Chatino presents a counter case study in which constituency tests show some striking convergences and plausible candidates for phonological and general word constituents.


\section{Acknowledgements}
I am very grateful to the many Chatino collaborators from Zenzontepec who have shared some of their knowledge and insights with me over the years, especially Tranquilino Cavero Ramírez, Flor Cruz Ortiz, and Alfonso Merino Pérez. I am also grateful to Emiliana Cruz, Hilaria Cruz, Terrence Kaufman, and John Justeson, without whom I would not have had the opportunity to begin this work. Sincere thanks to Adam Tallman, Sandra Auderset, and Hiroto Uchihara for including me in this project and for lots of patience. The chapter was greatly improved by their comments as well as by insightful feedback from other reviewers, including at least Tony Woodbury and Enrique Palancar. I acknowledge ELDP for crucial support for documenting usage of Zenzontepec Chatino with award IGS0080 to the University of Texas at Austin and for analysis, UCSB Academic Senate Faculty Research Grant 8-584204-19900-7.


\printglossary

\sloppy\printbibliography[heading=subbibliography,notkeyword=this]

\end{document} 
