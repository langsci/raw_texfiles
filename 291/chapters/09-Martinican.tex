\documentclass[output=paper]{langscibook}
\ChapterDOI{10.5281/zenodo.13208556}
\author{Minella Duzerol\affiliation{CNRS-DDL}}
\title{Constituency in Martinican (creole, Martinique)}
\abstract{In this corpus-based study, fourteen morphosyntactic and two phonological word diagnostics are applied to the Martinican predicative planar structure to investigate whether any grammatical and phonological words are identified. In so doing, I contribute to Tallman’s research on the empirical groundings of the distinction between grammatical and phonological words. I also provide linguistic-based arguments to nourish the ongoing debate on the Martinican orthographic system and more specifically its words’ boundaries.}
\IfFileExists{../localcommands.tex}{
  \addbibresource{../localbibliography.bib}
  \addbibresource{../collection_tmp.bib}
  % add all extra packages you need to load to this file

\usepackage{tabularx,multicol}
\usepackage{url}
\urlstyle{same}

\usepackage{listings}
\lstset{basicstyle=\ttfamily,tabsize=2,breaklines=true}

\usepackage{langsci-basic}
\usepackage{langsci-optional}
\usepackage{langsci-lgr}
\usepackage{langsci-osl}
% \usepackage{./langsci/styles/langsci-lgr}
% \usepackage{./langsci/styles/langsci-osl}
% \usepackage{langsci-gb4e}

\usepackage{tikz}
\usetikzlibrary{patterns,calc}
\pgfdeclarepatternformonly{south east lines}{\pgfqpoint{-0pt}{-0pt}}{\pgfqpoint{3pt}{3pt}}{\pgfqpoint{3pt}{3pt}}{
    \pgfsetlinewidth{0.6pt}
    \pgfpathmoveto{\pgfqpoint{0pt}{3pt}}
    \pgfpathlineto{\pgfqpoint{3pt}{0pt}}
    \pgfpathmoveto{\pgfqpoint{.2pt}{-.2pt}}
    \pgfpathlineto{\pgfqpoint{-.2pt}{.2pt}}
    \pgfpathmoveto{\pgfqpoint{3.2pt}{2.8pt}}
    \pgfpathlineto{\pgfqpoint{2.8pt}{3.2pt}}
    \pgfusepath{stroke}}
    
\usepackage{stmaryrd}
\usepackage{wasysym}
\usepackage{multirow}
\usepackage{caption}
\usepackage{subcaption}
\usepackage{mathrsfs}
\usepackage{qtree}

\usepackage{linguex}


  %pminos do not split footnotes
% \interfootnotelinepenalty=10000 %Footnote in Laporte chapters has to be split SN


%\DeclareIndexNameFormat{default}{%
%\nameparts{#1}%
%\usebibmacro{index:name}%
%{\index[names]}%
%{\namepartfamily}%
%{\namepartgiveni}%
% {}% L1
% {}% L2
%{\namepartprefix}% generates spurious space L3
%{\namepartsuffix}% generates spurious space L4
%}

%  {\DeclareIndexNameFormat{default}{%
%     \usebibmacro{index:name}{\index[names]}{#1}{#3}{#5}{#7}}}

%\DeclareIndexNameFormat{default}{%
%  \usebibmacro{index:name}{\sindex[nom]}{#1}{#3}{#5}{#7}}

%\DeclareIndexNameFormat{default}{%
%  \usebibmacro{index:name}{\sindex[person]}{#1}{#3}{#5}{#7}}
%\DeclareIndexNameFormat{default}{%
%\nameparts{#1} \usebibmacro{index:name}{\sindex[person]]}{\namepartfamily}{‌​\namepartgiven}{\nam‌​epartprefix}{\namepa‌​rtsuffix}}

%\newcommand{\smiley}{:)}

%\renewbibmacro*{index:name}[5]{%
%\usebibmacro{index:entry}{#1}%
%{\iffieldundef{usera}{}{\thefield{usera}\actualoperator}\mkbibindexname{#2}{#3}{#4}{#5}}}

% \newcommand{\noop}[1]{}

%remove for final
%\overfullrule=1mm

\newcommand{\tobi}[2]}}
\renewcommand{\S}[1]{\tobi{#1}{\textsc{*}}}

% this volume references
% puts: [this volume]
% already defined: \citetv
%\newcommand{\citepv}[1]{(\citeauthor{#1} \citeyear*{#1} [this volume])}
\newcommand{\citealtv}[1]{\citeauthor{#1} \citeyear*{#1} [this volume]}

%parentheses around example number
\newcommand{\pref}[1]{(\ref{#1})}

% in-text examples

\newcommand{\lnex}[1]{\textit{#1}} %target lang word
\newcommand{\lnlit}[1]{(lit.: `#1')} %literal reading
\newcommand{\lnlat}[1]{(#1)} % latinization
\newcommand{\lntrans}[1]{`#1'} %translation
\newcommand{\lnexl}[2]%
{\lnex{#1}{} \lnlat{#2}} % ex with latinization
\newcommand{\lnexlat}[3]{\lnex{#1}{} \lnlat{#2}{} \lntrans{#3}} % ex with latinization and tranl.

%ch01
\newcommand{\co}[1]{\mbox{\textbf{#1}}}

%ch09

\newcommand{\cyrbulg}[1]{\begin{otherlanguage*}{bulgarian}#1\end{otherlanguage*}}


%ch10
\newcommand{\nlp}{{\small NLP}}
\newcommand{\mwe}{{\small MWE}}
\newcommand{\rae}{{\small RAE}}
\newcommand{\lvc}{{\small LVC}}
\newcommand{\pos}{{\small P}o{\small S}}
%\newcommand{\todo}[1]{ \textcolor{red}{#1} }

%\renewcommand{\labelenumi}{\theenumi}
%\ainamefmt{{vv}{ll}{, ff}{, jj}} % fullname

\newcommand{\biberror}[1]{{\color{red}#1}}

\newcommand{\osenovaitem}{--~} 
  %% hyphenation points for line breaks
%% Normally, automatic hyphenation in LaTeX is very good
%% If a word is mis-hyphenated, add it to this file
%%
%% add information to TeX file before \begin{document} with:
%% %% hyphenation points for line breaks
%% Normally, automatic hyphenation in LaTeX is very good
%% If a word is mis-hyphenated, add it to this file
%%
%% add information to TeX file before \begin{document} with:
%% %% hyphenation points for line breaks
%% Normally, automatic hyphenation in LaTeX is very good
%% If a word is mis-hyphenated, add it to this file
%%
%% add information to TeX file before \begin{document} with:
%% \include{localhyphenation}
\hyphenation{
    Beck-man
    Ngu-yen
    back-chan-nel
    back-chan-nels
    mo-not-o-nous
    ste-reo-typ-i-cal
}

\hyphenation{
    Beck-man
    Ngu-yen
    back-chan-nel
    back-chan-nels
    mo-not-o-nous
    ste-reo-typ-i-cal
}

\hyphenation{
    Beck-man
    Ngu-yen
    back-chan-nel
    back-chan-nels
    mo-not-o-nous
    ste-reo-typ-i-cal
}
 
  \togglepaper[1]%%chapternumber
}{}

\begin{document}

\maketitle 
%\shorttitlerunninghead{}%%use this for an abridged title in the page headers

\section{Introduction}

This chapter provides a fine-grained description of the results of constituency diagnostics applied to the predicate complex of Martinican (Glottolog: mart1259), a French-based creole language of Martinique spoken by about 600.000 speakers according \citet{Colot2013}. Martinique is a Lesser Antilles island first inhabited by Amerindian peoples which became a French colony in 1635\footnote{\url{https://www.zananas-martinique.com/histoire}}. During the triangular trade era that extended from the 16\textsuperscript{th} to the 18\textsuperscript{th} centuries, the need for human resources was fulfilled by successive deportation waves coming from the coast of African countries. Martinican (creole, Martinique) arose in this heterogeneous social and linguistic context in which communication between the French settlers and the slaves was crucial. After the abolition of slavery in 1848, low-cost human resources were brought from India and took part in the evolution of the creole language. After Martinique became a French department on March 19\textsuperscript{th} 1946, French became the official language of administration and instruction in Martinique, according to the French Constitution. This institutional support of French plus its age-old international prestige contributed to a hierarchical distribution and perception of French and Martinican by the speakers. Nowadays, both languages still coexist in a ``dominant contact'' setting (\citealt{GadetPfaender2009} cited by \citealt{Colot2013}) that perpetuates the linguistic asymmetry. Today, a substantial majority of Martinican speakers are bilingual. However, there is an ongoing debate on whether Martinicans are really bilingual, sustained by the controversial decreolization thesis.\footnote{See \citet{Siegel2010} and \citet{Degraff2005} for example.} While this chapter does not address this question, the study is based on the understanding that Martinican, like every other language, shows variation. Little corpus-based work on Martinican varieties has been done so far. Therefore, no statement on the categorization of Martinican varieties will be made here. Still, the corpus used in this study was built to target different sociolinguistic profiles regarding age, geographic origin, and professional status%
% \citep{Duzerolinprep}%
. Twenty bilingual speakers participated, aged from 21 to 75 years old.

This chapter is based on 12 hours of spontaneous speech (of which 110 minutes were fully transcribed with ELAN-CorpA software 2022) recorded by the author and a number of elicited sentences. For the spontaneous speech, 18 speakers were asked to either talk about a subject of their choosing or to describe one of the five set of pictures proposed by the linguist.\footnote{For a more detailed presentation of the methodology, refer to \citealt[17--22]{Duzerol2021}.}

First, I discuss the planar structure of the predicate complex. Second, I deal with the constituency diagnostics applied to Martinican, starting with the morphosyntactic constituency diagnostics and ending with the phonological constituency diagnostics. For each type of diagnostics, I identify the convergent and divergent tests as well as wordhood candidates. The orthographic system for Martinican is still subject to ongoing debate (\citealt{Zribi-Hertz2017}; \citealt{Bernabe2013}). The results of the constituency diagnostics in Martinican are thus of particular interest with respect to the ‘word’ in the current orthographic system and whether the convergences match this orthographic word or other candidates emerge.

\section{Martinican predicative planar structure}
\label{bkm:Ref102580696}
As seen in the introduction of the volume, \citeauthor{chapters/01-Introduction}’s approach questions the empirical justification for postulating a distinction between morphology and syntax. Thus, the structure to which the constituency tests are applied puts morphological and syntactic elements at the same level.

Tallman \citep[10]{tallman2021constituency} specifies that planar structures are composed of elements, ``a formative, morpheme, affix, clitic, root, stem, phrase, clitic, or compound'', that occupy positions in these structures. These positions are numbered ``to account for relative ordering of its elements within the planar structure.''

While Tallman’s methodology distinguishes between nominal versus verbal planar structures, I oppose predicative versus non-predicative planar structure for the Martinican case. Given the widespread transcategoriality in Martinican (\citealt[75--76]{Colot2002}), I refer to any element that expresses “the semantic content of a predication' (\citealt[111]{Payne1997}) as predicate, regardless of its part of speech.

Based on the corpus used for this chapter, the Martinican predicative planar structure consists of 27 positions. As defined by \citet{Tallman2020b} there are two types of positions in a planar structure: zones where “all elements can occur and in any order" and slots where “all elements are mutually exclusive and only one can occur". Martinican predicative planar structure counts 11 zones and 15 slots as \tabref{tab:mart:key:1} shows. This structure starts and ends with sentence adverbs (positions \ref{mtsad1} and \ref{mtsad27}). The predicate base occupies position \ref{mtbase18}. Since Martinican is an SVO language, subjects and unique arguments precede the predicate base, they occur at position \ref{mtsub5}. Tense, aspect and modality are also encoded by markers placed before the predicate. Object pronouns and their phrasal and clausal equivalents occur after the predicate base. So far, the behavior of adverbials is not clear: their integration into an adverbial phrase is not certain.

\begin{table}[t]
\caption{Martinican predicative planar structure}
\label{tab:mart:key:1}
\begin{tabular}{Sll}
\lsptoprule
\multicolumn{1}{l}{\textbf{Position}} & \textbf{Type} & \textbf{Elements}\\ \midrule
\label{mtsad1} & zone & Sentence adverbs\\
\label{mtadv2} & zone & Adverbial clauses\\
\label{mtint3} & slot & Interrogative marker\\
\label{mtobl4} & slot & Obligation \textit{fok}\\
\label{mtsub5} & zone & NP, subject pronouns (A, S)\\
\label{mtadv6} & zone & Adverbial \protect\footnotemark\\
\label{mtcop7} & slot & Copula \textit{sé}\\
\label{mtneg8} & slot & Negative marker \textit{pa} (\textsc{neg}), \textit{pé} (\textsc{neg}), \textit{poko} (\textsc{not.yet})\\
\label{mtten9} & slot & Tense marker \textit{té} (\textsc{pst})\\
\label{mtadv10} & zone & Adverbial\\
\label{mtten11} & slot & Tense marker \textit{ka} (\textsc{ipfv}), \textit{ké} (\textsc{fut}), \textit{key} (\textsc{fut}), \textit{kay} (\textsc{fut})\\
\label{mtmod12} & slot & Modal \textit{pé} ʻcanʼ\\
\label{mtneg13} & slot & Negative marker \textit{pa} (\textsc{neg})\\
\label{mtcau14} & slot & Causative \textit{fè}\\
\label{mtnpc15} & slot & NP (causee); causee pronouns\\
\label{mtadv16} & zone & Adverbial\\
\label{mtder17} & slot & Derivational morpheme\\
\label{mtbase18} & \textbf{slot} & \textbf{V} \textbf{base}\\
\label{mtobj19} & slot & Object pronouns (recipient)\\
\label{mtobj20} & slot & Object pronouns (patient)\\
\label{mtadv21} & zone & Adverbial\\
\label{mtnpc22} & zone & NP, complement clauses\\
\label{mtadv23} & zone & Adverbial\\
\label{mtpp24} & slot & PP\\
\label{mtneg25} & slot & Negative marker …\textit{ankò}\\
\label{mtadv26} & zone & Adverbial clauses\\
\label{mtsad27} & zone & Sentence adverbs\\
\lspbottomrule
\end{tabular}
\end{table}


\footnotetext{The label \textit{adverbial} was used as convenient way of naming “any word with semantic content that is not cleary a noun, a verb or an adjective" (\citealt[69]{Payne1997}).}

The following example in (\ref{bkm:Ref105162975}) illustrates how an affirmative declarative sentence is constructed in Martinican.

\ea\label{bkm:Ref105162975}
\glll yo té ka sèvi gomié -a pou alé chèché pwason\\
    \ref{mtsub5} \ref{mtten9} \ref{mtten11} \ref{mtbase18} \ref{mtnpc22} {} \ref{mtadv26} {} {} {} \\ 
\Third\Pl{} \Pst{} \Impf{} use gommier -\Def.\Art{} \Sub.\Purp{} go look.for fish\\
\glt `they used to use the gommier to look for fishes.' (Descrip\protect\footnotemark{} REU 016)
\footnotetext{\textit{Descrip} stands for `description'.}
\z


In this corpus-based predicative planar structure, some morphemes can occur in multiple positions. This is the case for adverbial clauses (positions \ref{mtadv2} and \ref{mtadv26}), noun phrases (positions \ref{mtsub5} and \ref{mtnpc22}), negative marker \textit{pa} (positions \ref{mtneg8} and \ref{mtneg13}), adverbials (positions \ref{mtadv6}, \ref{mtadv10}, \ref{mtadv16}, \ref{mtadv21} and \ref{mtadv23}) and sentence adverbs (positions \ref{mtsad1} and \ref{mtadv26}).

\largerpage
Adverbial clauses, that is to say temporal, purpose and reason clauses, can occur at the beginning of the planar structure in position \ref{mtadv2}, before the predicate base or at the end of the planar structure position \ref{mtadv26} after the predicate base. Their position does not convey any syntactic nor semantic difference but an emphatic effect that has to do with information structure. In example (\ref{bkm:Ref100253992}), the purpose clause \textit{pou kay-la pwop} appears in position \ref{mtadv26}, i.e. in one of the last positions of the planar structure.


\ea \label{bkm:Ref100253992} 
\glll tout moun -nan ka fè menm bagay -la \textbf{pou} \textbf{kay} \textbf{-la} \textbf{pwop}\\
    \ref{mtsub5} {} {} \ref{mtten11} \ref{mtbase18} \ref{mtnpc22} {} {} \ref{mtadv26} {} {} {} \\
 every person -\Def.\Art{} \Impf{} do same thing -\Def.\Art{} \textbf{\Sub.\Purp{}} \textbf{home} \textbf{-}\textbf{\Def.\Art{}} \textbf{clean}\\
\glt `every person does the same thing for the house to be clean.' (Narr LOR\_part\_1 074)
\z

Example (\ref{bkm:Ref100254034}) shows that it is possible for a purpose clause to be placed in one of the first positions of the planar structure i.e. position \ref{mtadv2}.


\ea \label{bkm:Ref100254034}
\glll apré \textbf{pou} \textbf{stabilizé} \textbf{hm} \textbf{yol} \textbf{-la} nou ka itilizé dé bwa drésé \\
\ref{mtsad1} \ref{mtadv2} {} {} {} {} \ref{mtsub5} \ref{mtten11} \ref{mtbase18} \ref{mtnpc22} {} {} \\ 
after \textbf{\Sub.\Purp{}} \textbf{stabilize} \textbf{hum} \textbf{yole} \textbf{-}\textbf{\Def.\Art{}} \First\Pl{} \Impf{} use \Art.\Indf.\Pl{} bwa drésé \\
\glt `then, to stabilize hum the yole, we use \textit{bwa drésé}s.'\protect\footnotemark{} (Descrip HAT 037)
\z
\footnotetext{\textit{hm} = Hesitation phenomenon. \\
\textit{Bwa drésé}s are wood pieces maneuvered by humans for \textit{yole}’s balance. \textit{Yole} in Martinican and \textit{yole} in French is a Martinican boat that pertains to the Intangible Cultural Heritage of UNESCO.}


Noun phrases are also elements that occupy different positions in the planar structure. When noun phrases occupy position \ref{mtsub5}, they are the subject or the single argument of the predicate base. In example (\ref{bkm:Ref100222327}), the definite noun phrase \textit{paran-an} is the subject of \textit{pati}.

\ea \label{bkm:Ref100222327}
\glll \textbf{paran} \textbf{-an} ka pati\\
\ref{mtsub5} {} \ref{mtten11} \ref{mtbase18} \\ 
\textbf{parent} -\textbf{\Def.\Art{}} \Impf{} leave\\
\glt `the parent leaves' (Narr\footnote{\textit{Narr} stands for `narration'.} TYR\_part\_1 052)
\z

Noun phrases and their pronominal equivalent (position \ref{mtnpc15}) occur in causative constructions and encode the causee as in example (\ref{bkm:Ref89770232}) where the causee is the first person singular pronoun \textit{mwen}.

\ea \label{bkm:Ref89770232}
\glll papa -mwen té ka fè \textbf{mwen} dansé bèlè, danmié\\
\ref{mtsub5} {} \ref{mtten9} \ref{mtten11} \ref{mtcau14} \ref{mtnpc15} \ref{mtbase18} \ref{mtnpc22} {} \\
father -\First\Sg{} \Pst{} \Impf{} make \textbf{\First\Sg{}} dance bèlè danmié\\
\glt `my father used to make me dance \textit{bèlè}, \textit{danmié}'\footnote{\textit{Bèlè} and \textit{danmié} are Martinican traditional dances.} (LUI Descrip\_part\_1\_013)
\z

Finally, noun phrases can also appear at position \ref{mtnpc22}. In that case they are objects of the predicate base. In example (\ref{bkm:Ref100254088}), \textit{bannann} is the direct object of the predicative base \textit{livré}.

\ea\label{bkm:Ref100254088} 
\glll alò i té ka livré \textbf{bannann} ba lakopérativ\\
\ref{mtsad1} \ref{mtsub5} \ref{mtten9} \ref{mtten11} \ref{mtbase18} \ref{mtnpc22} \ref{mtpp24} {}\\
so \Third\Sg.\Sarg{} \Pst{} \Impf{} deliver \textbf{banana} \Prep{}.for cooperative\\
\glt `so he used to deliver bananas to the cooperative.' (LAU Descrip 052)
\z

There is another morpheme that occupies several positions in the planar structure: the negative marker \textit{pa}. \textit{Pa} occupies positions \ref{mtneg8} and \ref{mtneg13}. Standard negation is encoded by the negative marker \textit{pa} occurring in position \ref{mtneg8}, as in example (\ref{bkm:Ref100254113}).


\ea\label{bkm:Ref100254113}
\glll avan ou \textbf{pa} té ni sa\\
\ref{mtsad1} \ref{mtsub5} \ref{mtneg8} \ref{mtten9} \ref{mtbase18} \ref{mtnpc22} \\ 
before \Second\Sg{} \textbf{\Neg{}} \Pst{} have \Dem{}.\Pr{}\\
\glt `you did not have that before.' (LAU Descrip~056)
\z

However, when the modal \textit{pé} (position \ref{mtmod12}) is used without any overtly expressed TAM marker, the negative marker is not placed in position \ref{mtneg8} but in position \ref{mtneg13}, which is illustrated by example (\ref{bkm:Ref89770968}).

\ea\label{bkm:Ref89770968}
\glll man pé \textbf{pa} fè -y épi sè -mwen\\
\ref{mtsad1} \ref{mtmod12} \ref{mtneg13} \ref{mtbase18} \ref{mtobj20} \ref{mtpp24} {} {} \\
\First\Sg.\Sarg{} can \textbf{\Neg{}} do -\Third\Sg.\Obj{} \Prep{}.with sister -\First\Sg{}\\
\glt `I cannot do it with my sister.' (Narr MUR 079)
\z

When the modal \textit{pé} is used an overtly expressed TAM marker, the negative marker is placed in position \ref{mtneg8} as in example (\ref{bkm:Ref89774220}).

\ea\label{bkm:Ref89774220}
\glll man \textbf{pa} té pé alé antréné\\
\ref{mtsad1} \ref{mtneg8} \ref{mtten9} \ref{mtmod12} \ref{mtbase18} \ref{mtnpc22} \\
\First\Sg.\Sarg{} \textbf{\Neg{}} \Pst{} can go exercise\\
\glt `I could not go exercise.' (ELO Narr\_part\_1 022)
\z

In the corpus, adverbials can appear in multiple positions: \ref{mtadv6}, \ref{mtadv10}, \ref{mtadv16}, \ref{mtadv21} and \ref{mtadv23}. Although the question of how these various positions work has not been solved yet\footnote{I still need to evaluate how speakers perceive each adverbial placement to see if they are judged to be more Martinican-like or more French-like. I do not know yet if Martinican allows a free positioning of any adverbial, if the speakers’ bilingualism (Martinican-French) interacts with the placement of the adverbials or leads to change in the modern Martinican structure.}, it seems that the placement depends on the adverbial used. Some adverbials have a single position while others appear at several places. In the corpus, the adverbial \textit{vit} only appears in position \ref{mtadv23} as in example (\ref{bkm:Ref89962845}).\footnote{In the original data, this construction was in a subordinate clause. Example (\ref{bkm:Ref89962845}) is the corresponding independent clause.}

\ea\label{bkm:Ref89962845}
\glll pay -la pa ka izé \textbf{vit}\\
\ref{mtsub5} {} \ref{mtneg8} \ref{mtten11} \ref{mtbase18} \ref{mtadv23} \\ 
straw -\Def.\Art{} \Neg{} \Impf{} wear.down \textbf{quickly}\\
\glt `straw does not wear down quickly.' (Narr LOR\_part\_1 060)
\z

On the other hand, the adverbial \textit{vréman} is placed in positions \ref{mtadv16} or \ref{mtadv21}. There may be a bias in the corpus data in that \textit{vréman} only occur with nominal and adjectival predicates including either the copula \textit{sé} or the predicate \textit{ni} used as light predicative bases. In example (\ref{bkm:Ref89962847}), the nominal predicate is an existential construction involving the predicate base \textit{ni}. The adverbial \textit{vréman} appears in position \ref{mtadv16}. 

\ea\label{bkm:Ref89962847}
\glll pa \textbf{vréman} ni kouw\\
\ref{mtneg8} \ref{mtadv16} \ref{mtbase18} \ref{mtnpc22} \\ 
\Neg{} \textbf{really} have class\\
\glt `there is not really class.' (Narr HAT 007)
\z

Example (\ref{bkm:Ref89977050}) illustrates another existential construction involving the predicate base \textit{ni}. This time, the adverbial \textit{vréman} occupies position \ref{mtadv21}. The meaning of \textit{vréman} does not change.

\ea\label{bkm:Ref89977050}
\glll pa ni \textbf{vréman} non kréyol ba sa\\
\ref{mtneg8} \ref{mtbase18} \ref{mtadv21} \ref{mtnpc22} {} \ref{mtpp24} {} \\ 
\Neg{} have \textbf{really} noun creole \Prep{}.for \Dem{}.\Pr{}\\
\glt `there is not really a creole noun for this.' (Narr HAT 002)
\z

Last come the sentence adverbs. They appear in the very first or the very last positions of the planar structure. It seems that each sentence adverb has a preferential position without being restricted to this placement. In the corpus, the sentence adverb \textit{donk} is mainly placed in position \ref{mtsad1} as in (\ref{bkm:Ref89964096}).

\ea\label{bkm:Ref89964096}
\glll\textbf{donk} yo chayé -nou an fon bato -a\\
\ref{mtsad1} \ref{mtsub5} \ref{mtbase18} \ref{mtobj20} \ref{mtpp24} {} {} {} \\
\textbf{so} \Third\Pl{} carry -\First\Pl{} \Prep{}.in bottom boat -\Def.\Art{}\\
\glt `so they carried us in the bottom of the boat.' (Descrip MAU 054)
\z

The sentence adverb \textit{aprézan} is mainly used in position \ref{mtsad27}.

\ea\label{ex:mart:key:14'}
\glll bagay -la ka entérésé tout moun \textbf{aprézan}\\
\ref{mtsub5} {} \ref{mtten11} \ref{mtbase18} \ref{mtnpc22} {} \ref{mtsad27}\\
thing -\Def.\Art{} \Impf{} interest all person \textbf{nowadays}\\
\glt `the thing interests everyone nowadays.' (Descrip ELO 028)
\z

With respect to the orthographic word, it is defined by the academic writing system, the GEREC, named after the Martinican research center GEREC-F, (\textit{Groupe de Recherches en Espace Créolophone et Francophone}, Research Group in Creole-speaking and French-speaking area), in which its authors worked. This writing system has three versions namely GEREC-1, GEREC-2 and GEREC-3 (see \citealt{Zribi-Hertz2017}). For Martinican, it is the GEREC-2 that is mainly used. According to this version of the writing system, the orthographic predicative word gathers positions \ref{mtder17} to \ref{mtobj20} with some writing specificities for positions \ref{mtobj19} and \ref{mtobj20}. The following example (\ref{bkm:Ref105162940}) shows how a predicate receiving a preposed derivational morpheme is written. According to GEREC-2, no space is to be put between the derivational morpheme \textit{ri-} and the predicate base \textit{fè}.


\ea\label{bkm:Ref105162940}
\gllll man rifè lapenti-a\\
man \textbf{ri-} \textbf{fè} lapenti -a\\ 
\ref{mtsub5} \ref{mtder17} \ref{mtbase18} \ref{mtnpc22} \\
\First\Sg.\Sarg{} \textbf{again-} \textbf{do} paint -\Def.\Art{}\\
\glt `I did the painting again.' (Elicitation)
\z


Positions \ref{mtobj19} and \ref{mtobj20} are part of the orthographic predicative word only when they are pronouns of second and third persons singular. These pronouns have two phonetic realizations for both positions: \textit{ou} [u] and \textit{w} [w] for the second person singular, \textit{li} [li] and \textit{y} [j] for the third person singular. Only \textit{w} [w] and \textit{y} [j] are part of the orthographic predicative word. However, they have to be preceded by an apostrophe as examples (\ref{bkm:Ref100254163}) and (\ref{bkm:Ref100254748}) show. In (\ref{bkm:Ref100254163}), it is the patient pronoun \textit{y} that appears in the orthographic predicative word \textit{woti’y.}\\


\ea\label{bkm:Ref100254163} 
i di man pé pa menm \textbf{\textit{woti'y}}\\
\glll i di man pé pa menm \textbf{woti} \textbf{-y}\\
\ref{mtsub5} \ref{mtbase18} \ref{mtsub5} \ref{mtmod12} \ref{mtneg13} \ref{mtadv16} \ref{mtbase18} \ref{mtobj20} \\
\Third\Sg.\Sarg{} say \First\Sg.\Sarg{} can \Neg{} even \textbf{roast} -\textbf{\Third\Sg.\Obj{}}\\
\glt `she\protect\footnotemark{} said: ``I cannot even roast it.''' (Descri OTA 1 110)
\z

\footnotetext{The pronoun \textit{i} is not gender-specified. In example (\ref{bkm:Ref100254163}), the gender is identified thanks to the context. The speaker was talking about a woman.}

In example (\ref{bkm:Ref100254748}) the orthographic predicative word \textit{ba’y} is composed of the predicate base \textit{ba} and the recipient pronoun \textit{y}.

\ea\label{bkm:Ref100254748}
{...} \textit{nou} \textbf{\textit{ba'y}} \textit{do-nou}\\
\glll {...} nou \textbf{ba} \textbf{-y} do -nou\\
{...} \ref{mtsub5} \ref{mtbase18} \ref{mtobj19} \ref{mtnpc22} \ref{mtnpc22} \\
{...} \First\Pl{} \textbf{give} \textbf{-}\textbf{\Third\Sg.\Obj{}} back -\First\Pl{}\\
\glt `we turned our back to it.' (Descrip MAU 082)
\z

This presentation of the Martinican predicative planar structure and the predicative word according to the GEREC-2 writing system allows us to move on to the constituency diagnostics and the subspans of the planar structure that these diagnostics identify.  I start by considering the morphosyntactic diagnostics and their results in \sectref{bkm:Ref77669555}. Then, in \sectref{bkm:Ref77669705}, I focus on the phonological diagnostics and their results. I end by considering questions related to wordhood in Martinican from a typological perspective.


\section{The morphosyntactic diagnostics}
\label{bkm:Ref77669555}
In this section, I present in detail the fourteen morphosyntactic tests that have been applied to the Martinican predicative planar structure. They were established on the basis of six of the abstract morphosyntactic constituency tests present in \citet[16]{tallman2021constituency}'s taxonomy. Then, I present their results. 

\subsection{Free occurrence (18-18, 17-20)}

The free occurrence test identifies “a well-defined contiguous subspan of positions whose elements can be uttered as a minimal free form'' (\citealt[16]{tallman2021constituency}). It was fractured into smallest and biggest free occurrence tests:

\begin{enumerate}
\item free occurrence (smallest) identifies the smallest span of structure that can be a single free form;
\item free occurrence (largest) identifies the largest span of structure that can be a single free form.
\end{enumerate}

In the corpus, the predicate base (position \ref{mtbase18}) is the smallest subspan of structure to be a single free form. This single free form appears in constructions such as imperative, as in example (\ref{bkm:Ref100254817}).


\ea\label{bkm:Ref100254817}
\glll Sòti!\\
    \ref{mtbase18} \\ 
     get.out\\
\glt `Get out!' (\citealt[462]{Bernabe1983}; elicitation)
\z

Such a construction is described by \citet[1:462]{Bernabe1983} as a ``positive imperative exhortative''.\footnote{Translated from French : «\textit{impératif exhortatif positif}» \citep[462]{Bernabe1983}.}

The largest span of structure to be a single free form is subspan \ref{mtder17}-\ref{mtobj20}. This subspan gathers the predicate base (position \ref{mtbase18}), preposed derivational morphemes (position \ref{mtder17}), recipient object pronouns (position \ref{mtobj19})  and patient object pronouns (position \ref{mtobj20}). The largest single free form is also specific to imperatives. Only ditransitive predicates that have a double direct object construction can show such a structure. This corresponds to what \citet[49]{PinalieBernabe1999} call ``attribution constructions''. In example (\ref{bkm:Ref90043242}) the predicate base \textit{ba} has two pronominal direct objects: the recipient pronoun -\textit{mwen} and the patient pronoun \textit{{}-y}.

\ea\label{bkm:Ref90043242}Ba mwen’y\\
\glll Ba -mwen -y!\\
 \ref{mtbase18} -\ref{mtobj19} -\ref{mtobj20} \\
give -\First\Sg{} -\Third\Sg.\Obj{}\\
\glt `Give it to me!' (Elicitation)
\z

\subsection{(Non-)interruptibility (17-20, 17-20)}

The non-interruptibility test targets “a well-defined contiguous subspan of positions whose elements cannot be interrupted by element(s) of class I\footnote{Class I elements are those identified by the free occurrence diagnostics.}'' (\citealt[16]{tallman2021constituency}). This test was fractured into two subtests:

\begin{enumerate}
\item (non)-interruptibility (one free form) where the subspan of structure cannot be interrupted by an element that is a single free form;
\item (non)-interruptibility (more than one free form) where the subspan of structure cannot be interrupted by an element of more than one free form.
\end{enumerate}

However, in the corpus, these tests identify the same result. Since these tests share the same result, there is no need for a fracture based on simple vs. multiple free form interruption. The span identified by free form interruption is  \ref{mtder17} to \ref{mtobj20}. Positions \ref{mtadv16} and \ref{mtadv21} contain adverbial elements that are single free forms. It also seems that adverbial elements can be composed of more than one free form as in an adverbial phrase. However, more investigation needs to be done to confirm this last statement.

Examples (\ref{bkm:Ref90742879}) and (\ref{bkm:Ref100254851}) illustrate cases where positions \ref{mtadv16} and \ref{mtadv21} are filled by a free adverbial form, \textit{za} and \textit{vréman} respectively. 


\ea\label{bkm:Ref90742879}
\glll man té \textbf{za} ba -w -li\\
\ref{mtsub5} \ref{mtten9} \ref{mtadv16} \ref{mtbase18} -\ref{mtobj19} -\ref{mtobj20} \\ 
\First\Sg.\Sarg{} \Pst{} \textbf{already} give -\Second\Sg.\Obj{} -\Third\Sg.\Obj{}\\
\glt `I had already given it to you.' (Elicitation)
\z


\ea\label{bkm:Ref100254851}
\glll man fè -y \textbf{vréman}\\
\ref{mtsub5} \ref{mtbase18} -\ref{mtobj19} \ref{mtadv21} \\ 
\First\Sg.\Sarg{} do -\Third\Sg.\Obj{} \textbf{really}\\
\glt `I really did it.' (Elicitation)
\z

\subsection{(Non-)permutability (17-20, 17-20, 4-25, 3-25)}

The (non)-permutability test determines “a well-defined contiguous subspan of positions that cannot be variably ordered with one another (if a-b, then b-a must not occur)'' \citep[16]{tallman2021constituency}. This test was fractured following two criteria: whether (non)-permutability is considered in a strict or a scopal way and whether the construction is interrogative or declarative/imperative. Rigid (non)-permutability means that the elements cannot be variably ordered. Scopal (non)-permutability is when the variable ordering of an element goes with a difference in scope for this element. Consequently, there are four subtests:

\begin{enumerate}
\item 
(Non)-permutability - rigid - (declarative/imperative)
\item 
(Non)-permutability - rigid - (interrogative)
\item 
(Non)-permutability - scopal - (declarative/imperative)
\item 
(Non)-permutability - scopal - (interrogative)
\end{enumerate}

In the corpus, declarative/imperative clauses and interrogative clauses show the same results for the rigid (non)-permutability test. The subspan identified in the corpus extends from position \ref{mtder17} to position \ref{mtobj20}. Within this subspan as well as at its margin, declarative/imperative clauses and interrogative clauses do not have any structural differences. Thus, they share the same result. The elements occurring in positions \ref{mtadv16} and \ref{mtadv21} can be variably ordered. As mentioned in the presentation of the Martinican predicative planar structure, the adverbial \textit{vréman} can occur either in positions \ref{mtadv16} or \ref{mtadv21} in the corpus. Examples (\ref{bkm:Ref100746865}) and (\ref{bkm:Ref100654963}) were cited to illustrate it.


\ea\label{bkm:Ref100746865}
\glll pa \textbf{vréman} ni kouw\\
\ref{mtneg8} \ref{mtadv16} \ref{mtbase18} \ref{mtnpc22} \\
\Neg{} \textbf{really} have class\\
\glt `there is not really class.' (Narr HAT 007)
\z


\ea\label{bkm:Ref100654963}
\glll pa ni \textbf{vréman} non kréyol ba sa\\
\ref{mtneg8} \ref{mtbase18} \ref{mtadv21} \ref{mtnpc22} {} \ref{mtpp24} {} \\ 
\Neg{} have \textbf{really} noun creole \Prep{}.for \Dem{}.\Pr{}\\
\glt `there is not really a creole noun for this.' (Narr HAT 002)
\z

The subspan that extends from position \ref{mtobl4} to position \ref{mtneg25} is identified by the scopal (non)-permutability test for declarative and imperative sentences. The left edge of this subspan is preceded by position \ref{mtint3} that contains interrogative markers. These markers are not part of declarative/imperative sentences. The position following the right edge of the subspan, i.e. position \ref{mtadv26}, has elements that can be variably ordered without any scopal change. The adverbial clauses can equally occupy positions \ref{mtadv2} and \ref{mtadv26} without any scopal change. Adverbial clause will still depend on the predicate base. Its placement before or after the predicate base is matter of stylistic choices and information ordering. In example (\ref{bkm:Ref100254896}), the preposed purpose clause \textit{pou twouvé gwo lanmè-a} is in position 2.


\ea\label{bkm:Ref100254896}
\glll\textbf{pou} \textbf{twouvé} \textbf{gwo} \textbf{lanmè} \textbf{-a} fok ou alé lwen lwen lwen déwò\\
\ref{mtadv2} {} {} {} {} \ref{mtobl4} \ref{mtsub5} \ref{mtbase18} \ref{mtadv23} {} {} {} \\ 
\textbf{\Sub.\Purp{}} \textbf{find} \textbf{big} \textbf{sea} -\textbf{\Def.\Art{}} \Obl{} \Second\Sg{} go far far far outside\\
\glt `\textbf{to} \textbf{find} \textbf{the} \textbf{deep} \textbf{sea}, you have to go far far far away, outside.' (Descrip MAU 040)
\z

\textit{Pou twouvé gwo lanmè-a} could equally appear in position \ref{mtadv26} as in the elicited example (\ref{bkm:Ref90384114}).


\ea\label{bkm:Ref90384114}
\glll fok ou alé lwen lwen lwen déwò \textbf{pou} \textbf{twouvé} \textbf{gwo} \textbf{lanmè} \textbf{-a}\\
4 \ref{mtsub5} \ref{mtbase18} \ref{mtadv23} {} {} {} \ref{mtadv26} {} {} {} {} \\
\Obl{} \Second\Sg{} go far far far outside \textbf{\Sub.\Purp{}} \textbf{find} \textbf{big} \textbf{sea} -\textbf{\Def.\Art{}}\\
\glt `you have to go far far far away, outside \textbf{to} \textbf{find} \textbf{the} \textbf{deep} \textbf{sea}.' (Elicitation)
\z 

The scopal (non)-permutability test for interrogative sentences points to the subspan \ref{mtint3}-\ref{mtneg25}. The only difference between the results for interrogative sentences and the declarative/imperative sentences is that subspan \ref{mtint3}-\ref{mtneg25} includes interrogative markers (position 3), elements that cannot be variably ordered either. Consequently, a similar reasoning justifies the identification of the subspan \ref{mtint3}-\ref{mtneg25}. The right and left edges of subspan \ref{mtint3}-\ref{mtneg25} are occupied by adverbial clauses. As previously illustrated, adverbial clauses are elements that can be variably ordered. Furthermore, this variable ordering does not condition any scopal change. The following examples (\ref{bkm:Ref100254949}) and (\ref{bkm:Ref90398090}) show that the temporal clause \textit{lè ou té piti} can equally be placed in position \ref{mtadv2} or \ref{mtadv26}.

\ea\label{bkm:Ref100254949}
\glll \textbf{Lè} \textbf{ou} \textbf{té} \textbf{piti}, es ou té ka gadé kous yol?\\
\ref{mtadv2} {} {} {} \ref{mtint3} \ref{mtsub5} \ref{mtten9} \ref{mtten11} \ref{mtbase18} \ref{mtnpc22} {} \\ 
\textbf{\Sub{}.when} \textbf{\Second\Sg{}} \textbf{\Pst{}} \textbf{little} \Q{} \Second\Sg{} \Pst{} \Impf{} look race yole\protect\footnotemark\\
\glt `\textbf{When} \textbf{you} \textbf{were} \textbf{young}, would you look at yole races?' (Elicitation)
\z
\footnotetext{Yoles are traditional Martinican boats that appear on the UNESCO Register of Good Safeguarding Practices since 2020.}

\ea\label{bkm:Ref90398090}
\glll Es ou té ka gadé kous yol \textbf{lè} \textbf{ou} \textbf{té} \textbf{piti} ?\\
\ref{mtint3} \ref{mtsub5} \ref{mtten9} \ref{mtten11} \ref{mtbase18} \ref{mtnpc22} {} \ref{mtadv26} {} {} {} \\
\Q{} \Second\Sg{} \Pst{} \Impf{} look race yole \textbf{\Sub{}.when} \textbf{\Second\Sg{}} \textbf{\Pst{}} \textbf{little}\\
\glt `Would you look at yole races \textbf{when} \textbf{you} \textbf{were} \textbf{young}?' (Elicitation)
\z


\subsection{Ciscategorial selection (17-18, 2-26)}

Ciscategorial selection tests target “a well-defined contiguous subspan of positions whose elements can only semantically combine with one part of speech class'' (\citealt[16]{tallman2021constituency}). To obtain unambiguous results this test was fractured into two subtests:

\begin{enumerate}
\item 
ciscategorial selection (predicate only): a test that considers elements combining with predicative bases that do not combine with non-predicative bases;
\item 
ciscategorial selection (with the predicate): a test that considers elements combining with predicative bases that could also combine with non-predic\-a\-tive bases.
\end{enumerate}

The ciscategorial selection (predicate only) test identifies a subspan composed of positions \ref{mtder17} to \ref{mtbase18} according to the speakers’ productions. Preposed derivational morphemes (position \ref{mtder17}) are the only elements that only combine with predicates. In the corpus, the derivational morpheme \textit{ri-} only occur with predicates. In (\ref{bkm:Ref100254964}), \textit{ri-} combines with the predicate \textit{fè} ‘do’.

\ea\label{bkm:Ref100254964}
\textit{man} \textbf{\textit{rifè}}\textit{’y}\\
\glll man \textbf{ri-} \textbf{fè} {}-y\\ 
\ref{mtsub5} \ref{mtder17}- \ref{mtbase18} -\ref{mtobj20} \\ 
\First\Sg.\Sarg{} \textbf{again} -do -\Third\Sg.\Obj{}\\
\glt `I did it again.' (Elicitation)
\z

Adverbials come in position \ref{mtadv16}. Adverbials are elements that can combine with parts of speech other than predicates, such as adjectives and other adverbials. Examples (\ref{bkm:Ref100222334}) and (\ref{bkm:Ref100222335}) show that the adverbial \textit{bien} can combine with the predicate \textit{enmen} but also with the adverbial \textit{lwen}.

\ea\label{bkm:Ref100222334}\textit{man té} \textbf{\textit{bien}} \textit{enmen’y}\\
\glll man té \textbf{bien} enmen -y\\
\ref{mtsub5} \ref{mtten9} \ref{mtadv16} \ref{mtbase18} -\ref{mtobj20} \\ 
\First\Sg.\Sarg{} \Pst{} \textbf{well} like -\Third\Sg.\Obj{}\\
\glt `I really liked it.' (Descrip ELO 055)
\z


\ea\label{bkm:Ref100222335}
\glll \textup{[…]}nou té ka garé \textbf{bien} lwen\\
\ref{mtsub5} \ref{mtten9} \ref{mtten11} \ref{mtbase18} \ref{mtadv23} {} \\
\First\Pl{} \Pst{} \Impf{} park \textbf{well} far\\
\glt `We used to park really far.' (Descrip TUO 056)
\z

After the left-edge of this subspan, there are recipient object pronouns (position \ref{mtobj19}). These forms are transcategorial since they combine with nouns as possessive determiners (\citealt{Colot2013}). Examples (\ref{bkm:Ref102489799}) and (\ref{bkm:Ref102489800}) show that \textit{mwen} ‘\textsc{1sg’} functions as an object pronoun and a possessive marker respectively.

\ea\label{bkm:Ref102489799}
\glll yo di \textbf{mwen} sa sé pa kréyol sa\\
\ref{mtsub5} \ref{mtbase18} \ref{mtobj19} \ref{mtnpc22} {} {} {} {} \\
\Third\Pl{} tell \textbf{\First\Sg{}} \Dem{}.\Pr{} be \Neg{} Creole \Dem.\Pr{}\\
\glt `They told me this, this is not Creole.' (Narr AUG 110)
\z

\ea\label{bkm:Ref102489800}
\textit{manman-}\textbf{\textit{mwen}} \textit{té ka fè’y osi}\\
\glll manman -\textbf{mwen} té ka fè -y osi\\
\ref{mtsub5} {} \ref{mtten9} \ref{mtten11} \ref{mtbase18} \ref{mtobj20} \ref{mtadv23} \\
mother \textbf{\First\Sg{}} \Pst{} \Impf{} do -\Third\Sg.\Obj{} too\\
\glt `My mother used to do it too.' (Narr PRU 039)
\z

Subspan \ref{mtadv2}-\ref{mtadv26} is the result of the ciscategorial selection (with the predicate) test. In the predicative planar structure, sentence adverbs (positions \ref{mtsad1} and \ref{mtsad27}) are the only elements that do not combine with the predicate since they have scope over the whole sentence. In example (\ref{bkm:Ref100254978}) the sentence adverb \textit{efektivman} expresses the speaker’s position regarding the whole sentence \textit{yo di mwen atann} attesting that the predication really did happen.

\ea\label{bkm:Ref100254978}
\glll \textbf{efektivman} yo di mwen atann\\
\ref{mtsad1} \ref{mtsub5} \ref{mtbase18} \ref{mtobj19} \ref{mtnpc22} \\ 
\textbf{indeed} \Third\Pl{} tell \First\Sg{} wait\\
\glt `they told me to wait indeed.' (Elicitation)
\z

\subsection{Biuniqueness deviation domain: negation \textit{pa} \textit{ankò} 'no more' (8-25)}

A devation from biuniqueness domain is ``a well-defined contiguous subspan of positions whose elements display deviations from biuniqueness (one meaning-one form)'' (\citealt[16]{tallman2021constituency}). In the corpus, to express that the predication does not hold at the time of the event but was true before the time of the event, a negative discontinuous morpheme is used. This discontinuous morpheme is also identified in \citet[41]{PinalieBernabe1999}. \textit{Pa} \Neg{} occupies position \ref{mtneg8} and \textit{ankò} `again' occupies position \ref{mtneg25}. The positions of these morphemes do not vary in the corpus. Thus, the biuniqueness deviation domain for the negation \textit{pa ankò} is subspan \ref{mtneg8}-\ref{mtneg25}. The speaker of example (\ref{bkm:Ref100222676}) explains that following the Martinican traditional \textit{yole} racing was a custom before but is not one anymore.

\ea\label{bkm:Ref100222676}
\glll donk atjelman man \textbf{pa} ka suiv touw -la \textbf{ankò}\\
 \ref{mtsad1} {} \ref{mtsub5} \ref{mtneg8} \ref{mtten11} \ref{mtbase18} \ref{mtnpc22} {} \ref{mtneg25} \\
so nowadays \First\Sg.\Sarg{} \textbf{\Neg{}}… \Impf{} follow race -\Def.\Art{} …\textbf{anymore}\\
\glt `so, nowadays, I do not follow the race anymore.' (Descrip BEL 051)
\z

\subsection{Biuniqueness deviation domain: second and third singular object pronouns allomorphy (18-20; 18-19)}

The second and third singular persons of the object pronouns (positions \ref{mtobj19} and \ref{mtobj20}) display a morpheme-specific allomorphy. The diagnostic has been split into two subtests to target each position:

\begin{enumerate}
\item Biuniqueness deviation domain: second and third singular object pronouns (patient) allomorphy 
\item Biuniqueness deviation domain: second and third singular object pronouns (recipient) allomorphy
\end{enumerate}

The patient object pronouns (position \ref{mtobj20}) of second and third singular persons display a morpheme-specific allomorphy (\citealt[23--24]{PinalieBernabe1999}). Both second and third singular persons have two allomorphs. Their distribution is summarized in \tabref{tab:mart:key:2} below and does not show any variation in the corpus.

\begin{table}
    \centering
     \caption{Second and third patient object pronouns allomorphy}
    \label{tab:mart:key:2}
    \begin{tabular}{lll}
    \lsptoprule
    \multicolumn{2}{c}{{\bfseries Patient object pronoun realizations}} & {\bfseries Preceding syllable type}\\ \midrule
    2\textsuperscript{nd} person singular & [u] & Closed\\
        & [w] & Open\\
    3\textsuperscript{rd} person singular & [li] & Closed\\
     & [j] & Open\\
    \lspbottomrule
\end{tabular}
\end{table}


Examples (\ref{bkm:Ref100255103}) and (\ref{bkm:Ref100255112}) illustrate the allomorphy for the second person singular patient pronoun (position \ref{mtobj20}). In (\ref{bkm:Ref100255103}), the pronoun comes after a closed syllable [tãn] and is realized [u].


\ea\label{bkm:Ref100255103}
\textit{man pa ka tann} \textbf{\textit{ou}}\\
 $[$mãpakatãn\textbf{u}$]$\\
\glll man pa ka tann -\textbf{ou}\\
\ref{mtsub5} \ref{mtneg8} \ref{mtten11} \ref{mtbase18} -\ref{mtobj20} \\ 
\First\Sg.\Sarg{} \Neg{} \Impf{} hear -\textbf{\Third\Sg{}}\\
\glt `I do not hear you.' (Elicitation)
\z

In (\ref{bkm:Ref100255112}), the pronoun comes after an open syllable [ʁe] and is realized [w].

\ea\label{bkm:Ref100255112}
\textit{nou ka préparé’}\textbf{\textit{w}}\\
$[$nukapʁepaʁe\textbf{\textup{w}}$]$\\
\glll  nou ka préparé -\textbf{w}\\
\ref{mtsub5} \ref{mtten11} \ref{mtbase18} -\ref{mtobj20} \\
\First\Pl{} \Impf{} prepare -\textbf{\Second\Sg.\Obj{}}\\
\glt `we are preparing you.' (Elicitation)
\z

(\ref{bkm:Ref100680806}) and (\ref{bkm:Ref100680808}) exhibit the allomorphy of the third person singular patient pronoun (position \ref{mtobj20}). In (\ref{bkm:Ref100680806}), the pronoun is preceded by a closed syllable [diw] and is realized [li].


\ea\label{bkm:Ref100680806}
\textit{man té di’w} \textbf{\textit{li}}\\
$[$mãtediw\textbf{li}$]$\\
\glll man té di -w -\textbf{li}\\
\ref{mtsub5} \ref{mtneg8} \ref{mtbase18} -\ref{mtobj19} -\ref{mtobj20} \\
\First\Sg.\Sarg{} \Pst{} tell -\Second\Sg.\Obj{} -\textbf{\Third\Sg.\Obj{}}\\
\glt `I had told you that.' (Elicitation)
\z

In (\ref{bkm:Ref100680808}), it is the open syllable [di] that precedes the pronoun that is realized [j].

\ea\label{bkm:Ref100680808}
\textit{man rédi’}\textbf{\textit{y}}\\
$[$mãredi\textbf{j}$]$\\
\glll man rédi -\textbf{y}\\
\ref{mtsub5} \ref{mtbase18} -\ref{mtobj20} \\
\First\Sg.\Sarg{} take -\Third\Sg.\Obj{}\\
\glt `I took it.' (Elicitation)
\z

Since it is the syllable that immediately precedes the pronoun that conditions the form of the pronoun, the deviation from biuniqueness domain for second and third singular patient object pronoun allomorphy is subspan \ref{mtbase18}-\ref{mtobj20}; that is to say the predicate base, the object pronouns (recipient) and the object pronouns (patient). Examples (\ref{bkm:Ref100255103}) to (\ref{bkm:Ref100680808}) show cases where the patient pronoun is preceded by the predicate base. In the next example (\ref{bkm:Ref100680932}), the patient pronoun comes after a recipient pronoun (position \ref{mtobj19}). The patient pronoun realized [li] is post-posed to the closed syllable [baw].

\newpage
\ea\label{bkm:Ref100680932}
\textit{man ba’w} \textbf{\textit{li}}\\
$[$mãbaw\textbf{li}$]$\\
\glll man ba -w -\textbf{li}\\ 
\ref{mtsub5} \ref{mtbase18} -\ref{mtobj19} -\ref{mtobj20} \\ 
\First\Sg.\Sarg{} give -\Second\Sg.\Obj{} -\textbf{\Third\Sg.\Obj{}}\\
\glt `I gave it to you.' (Elicitation)
\z

Recipient object pronouns of the second and third singular persons (position \ref{mtobj19}) undergo the same morpheme-specific allomorphy rules as the patient object pronouns. These rules are summarized in \tabref{tab:mart:key:3}.

\begin{table}
    \centering
    \caption{Second and third recipient object pronouns allomorphy}
    \label{tab:mart:key:3}
    \begin{tabular}{lll}
     \lsptoprule
    \multicolumn{2}{c}{{\bfseries Recipient object pronoun realizations}} & {\bfseries Preceding syllable type}\\ \midrule
    2\textsuperscript{nd} person singular & [u] & Closed\\
    & [w] & Open\\
    3\textsuperscript{rd} person singular & [li] & Closed\\
    & [j] & Open\\
\lspbottomrule
    \end{tabular}  
\end{table}

In the data transcribed, no context where recipient pronouns were post-posed to closed syllable was found. Examples (\ref{bkm:Ref100681053}) and (\ref{bkm:Ref100681044}) illustrate the phonetic realizations of the second and third recipient object pronouns when they follow the open syllable [ba]. They are realized [w] and [j] respectively.


\ea\label{bkm:Ref100681053}
\textit{lanné pasé, man ba’}\textbf{\textit{w}} \textit{fos pou vansé}\\
$[$lãnepasemãba\textbf{w} fɔspuvãse$]$\\
\glll lanné pasé, man ba -\textbf{w} fos pou vansé\\
\ref{mtsad1} {} \ref{mtsub5} \ref{mtbase18} \ref{mtobj19} \ref{mtnpc22} \ref{mtadv26} \\ 
year pass \First\Sg.\Sarg{} give -\textbf{\Second\Sg.\Obj{}} strength \Prep{} move.forward\\
\glt `last year, I gave you strenght to move forward.' (Elicitation)
\z
 
\ea\label{bkm:Ref100681044}
\textit{lanné pasé, man ba’}\textbf{\textit{y}} \textit{fos pou vansé}\\
$[$lãnepasemãba\textbf{j}fɔspuvãse$]$\\
\glll lanné pasé, man ba -\textbf{y} foss pou vansé\\
\ref{mtsad1} {} \ref{mtsub5} \ref{mtbase18} -\ref{mtobj19} \ref{mtnpc22} \ref{mtadv26} \\
year pass \First\Sg.\Sarg{} give -\textbf{\Third\Sg.\Obj{}} strength \Prep{} move.forward\\
\glt `last year, I gave him/her strenght to move forward.' (Elicitation)
\z

As a result, the deviation from biuniqueness domain for second and third singular recipient object pronouns' allomorphy is subspan \ref{mtbase18}-\ref{mtobj19} which includes the pronouns themselves (position \ref{mtobj19}) and the predicate base (position \ref{mtbase18}). Indeed, the predicate base is the position that comes right before the second and third singular recipient object pronouns and it is its last syllable that conditions the phonetic variation.

\subsection{Subspan repetition test: finite declarative complement clauses (4-27)}

A subspan repetition test identifies ``a well-defined contiguous subspan of positions that occurs more than once for a given construction'' (\citealt[16]{tallman2021constituency}). The subspan repetition test presented in this section only considers finite declarative complement clauses. Based on the data available, only the largest repeated subspan has been identified.

Finiteness receives a language specific definition (Duzerol in prep) to take into account relevant features that do not necessarily correspond to the traditional morphological ones as \citet{Migge2018} pointed out. Thus, in this chapter, finiteness is considered as a continuum with two poles corresponding to the prototypical finite predicate and the prototypical non-finite predicate. They are identified according to three criteria: the presence of TAM markers, the presence of an overtly expressed subject, and negation. These features are presented in \tabref{tab:mart:key:4}.

\begin{table}
    \centering
    \caption{Finiteness in Martinican \citep[3]{Duzerol2021b}}
    \label{tab:mart:key:4}
    \begin{tabularx}{\textwidth}{lQQl}
        \lsptoprule
        & Presence of TAM markers & Subject overtly expressed & Negation\\ \midrule
         Prototypical finiteness & + & + & +\\
         Prototypical non finitess & - & - & -\\
    \lspbottomrule
    \end{tabularx}
\end{table}

In the corpus, when the predicate’s object is a finite declarative complement clause, the largest subspan of structure that is repeated goes from position \ref{mtobl4} to position \ref{mtsad27}. In fact, the syntactic difference between an independent finite declarative clause and a subordinate finite declarative clause relies on the dependent syntactic status of the subordinate clause -- it saturates the valency of the main clause predicate -- and the possible presence of a non-mandatory complementizer between the two clauses \citep{Duzerol2021b}\footnote{When the main clause predicate is an utterance predicate, pronominal shift in the subordinate clauses also indicates that the clause is subordinate.}. However, because the complement clause is a declarative clause, there is no interrogative marker (position \ref{mtint3}). That is why the left edge of the subspan is position 4. In (\ref{bkm:Ref100255067}), \textit{sa ka entérésé tout moun aprézan} is the declarative clause identified by the recursion-based test for finite declarative complement clauses. There, position \ref{mtobl4} is empty because there is no obligation marker.

\ea\label{bkm:Ref100255067}
\glll man sav \textbf{sa} \textbf{ka} \textbf{entérésé} \textbf{tout} \textbf{moun} \textbf{aprézan}\\
\ref{mtsub5} \ref{mtbase18} \ref{mtsub5} \ref{mtten11} \ref{mtbase18} \ref{mtnpc22} {} \ref{mtsad27}\\
\First\Sg.\Sarg{} know \textbf{\Dem{}}.\textbf{\Pr{}} \textbf{\Impf{}} \textbf{interest} \textbf{all} \textbf{person} \textbf{nowadays}\\
\glt `I know it interests everyone nowadays.' (Elicitation)
\z


\subsection{The grammatical predicative word candidate}

\tabref{tab:mart:key:5} summarizes the spans identified by each of the fourteen morphosyntactic diagnostics.

\begin{table}[t]
    \centering
    \caption{Results of the morphosyntactic diagnostics}
    \label{tab:mart:key:5}
    \fittable{
    \begin{tabular}{r>{\raggedright}p{6.2cm}rrrr}
         \lsptoprule
{\bfseries  N°~} & {\bfseries Test ID} & {\bfseries Left} & {\bfseries Right} & {\bfseries Size} & {\bfseries Layer ID}\\\midrule
1 & Free occurrence (smallest)              & 18 & 18 & 1 & 1\\
2 & Free occurrence (largest)               & 17 & 20 & 4 & 2\\
3 & (Non)-interruptibility (one free form)  & 17 & 20 & 4 & 2\\
4 & (Non)-interruptibility (more than one free form)                    & 17 & 20 & 4 & 2\\
5 & (Non)-permutability - rigid - (declarative/imperative)   & 17 & 20 & 4 & 2\\
6 & (Non)- permutability - rigid - (interrogative)                      & 17 & 20 & 4 & 2\\
7 & (Non)- permutability - scopal - (declarative/imperative)            & 4 & 25 & 22 & 3\\
8 & (Non)- permutability - scopal - (interrogative)                     & 3 & 25 & 23 & 4\\
9 & Ciscategorial selection (predicate only)            & 17 & 18 & 2 & 5\\
10 & Ciscategorial selection (with the predicate)       & 2 & 26 & 25 & 6\\
11 & Biuniqueness deviation domain: negation pa \textit{ankò} 'no more'                                 & 8 & 25 & 18 & 7\\
12 & Biuniqueness deviation domain: second and third singular object pronouns (patient) allomorphy      & 18 & 20 & 3 & 8\\
13 & Biuniqueness deviation domain: second and third singular object pronouns (recipient) allomorphy    & 18 & 19 & 2 & 9\\
14 & Recursion-based test: finite declarative complement clauses (largest)                              & 4 & 27 & 24 & 10 \\
\lspbottomrule
    \end{tabular}
    }
\end{table}

Fundamentally, these corpus-based results of the morphosyntactic diagnostics used in this study do not converge with one another consistently. Ten subspans are identified in total. There is one subspan which is a good candidate for the grammatical predicative word candidate: subspan \ref{mtder17}-\ref{mtobj20}. This word candidate would be comprised of the preposed derivational morphemes, the predicate base and the object pronouns. The other spans are less likely candidates since they do not converge with any other span (as the smallest free occurrence) or isolate a part of a clause or entire clauses what would make Martinican an extremely polysynthetic language. Therefore, on the basis of this study, free occurrence (largest), non interruptibility (one free form), (non)-interruptibility (more than one free form), (non)-permutability - rigid (declarative/imperative), (non)-permutability - rigid (interrogative) would be relevant predicative morphological wordhood tests for Martinican.

Interestingly, this morphological predicative word candidate partially corresponds to what the GEREC-2 defines as a word. Indeed, as mentioned in the presentation of the corpus-based Martinican planar structure used in this chapter (\sectref{bkm:Ref102580696}), the predicative word according to the GEREC-2 always comprises positions \ref{mtder17} and \ref{mtbase18}, namely the preposed derivational morphemes and the predicate base. The GEREC-2 word contains positions \ref{mtobj19} or \ref{mtobj20} only when they are filled by the forms \textit{{}-y} and \textit{{}-w}. Besides, when included in the word, the elements of positions \ref{mtobj19} and \ref{mtobj20} are submitted to a specific writing rule: they have to be preceded by an apostrophe. On the basis of the convergence of the corpus-based morphological results, one could question why all the elements of positions \ref{mtobj19} and \ref{mtobj20} are not included in the predicative word. This possibility is precisely brought up by \citet{Zribi-Hertz2017}.

After examining the morphosyntactic diagnostics, it is time to move on to the phonological ones to investigate the divergences and convergences of their results.

\section{The phonological diagnostics}
\label{bkm:Ref77669705}

This section is dedicated to the presentation of the phonological diagnostics applied to the Martinican predicative planar structure and the spans of structure defined by the aforesaid diagnostics. One phonological diagnostic is presented. It was established on the basis of one of the phonological abstract constituency tests presented in Tallman (\citeyear[16]{tallman2021constituency})'s taxonomy, which is the stress domain.\footnote{There might be other phonological domains that are not identified yet due few phonological and prosodic corpus-based litterature on Martinican.}

\subsection{Stress domain (17-20)}
\largerpage[2]
According to \citet[16]{tallman2021constituency}, the stress domains identifies ``a well-defined contiguous subspan of positions that define the domain for the application of a stress rule''. Little exhaustive corpus-based work has been done on Martinican’s prosody. Thus, I present here preliminary results.

\citet{Colot2013} argue that ``word stress is always on the last syllable'' and ``phrase and sentence stress is also in final position''. The core question for the predicative planar structure is to know if the dependent elements filling positions \ref{mtder17}, \ref{mtobj19} and \ref{mtobj20} bear stress. The elements of positions \ref{mtadv16} and \ref{mtadv21}, namely adverbials, are considered free forms. Thus, one would expect them to be in line with the rule stated by \citet{Colot2013}.\footnote{This investigation falls outside the scope of this chapter.}

A preliminary analysis of twelve elicited clauses and one spontaneous clause where positions \ref{mtder17}, \ref{mtobj19} and \ref{mtobj20} are filled was used to investigate the stress domain. I collected the elicited data with three speakers. They were asked to repeat the clauses three times. Examples (\ref{bkm:Ref105163185}) to (\ref{bkm:Ref105163197}) are the transcriptions of these clauses. Examples (\ref{bkm:Ref105163185}) and (\ref{bkm:Ref105166070}) were previously mentioned in the chapter.

\ea\label{bkm:Ref105163185}
\textit{man rifè’y}\\
$[$mãʁifɛj$]$\\
\glll man ri- fè -y\\
\ref{mtsub5} \ref{mtder17}- \ref{mtbase18} -\ref{mtobj20} \\
\First\Sg.\Sarg{} again- do -\Third\Sg.\Obj{}\\
\glt `I did it again.' (Elicitation)
\z
 
\ea\label{bkm:Ref105166070}
\textit{mwen ba’w li}\\
$[$mwɛ̃bawli$]$\\
\glll mwen ba -w -li\\
\ref{mtsub5} \ref{mtbase18} -\ref{mtobj19} -\ref{mtobj20} \\ 
\First\Sg{} give -\Second\Sg.\Obj{} -\Third\Sg.\Obj{}\\
\glt `I gave it to you.' (Elicitation)
\z

\ea\label{bkm:Ref105166218}
\textit{man bat ou}\\
$[$mãbatu$]$\\
\glll man bat -ou\\
\ref{mtsub5} \ref{mtbase18} -\ref{mtobj20} \\
\First\Sg.\Sarg{} hit -\Second\Sg.\Obj{}\\
\glt `I hit you.' (Elicitation)
\z

\ea\label{bkm:Ref105166237}
\textit{man té di’y sa}\\
$[$mãtedijsa$]$\\
\glll man té di -y sa\\
\ref{mtsub5} \ref{mtten9} \ref{mtbase18} -\ref{mtobj19} \ref{mtnpc22} \\
\First\Sg.\Sarg{} \Pst{} tell -\Third\Sg.\Obj{} \Dem{}.\Pr{}\\
\glt `I had told her/him that.' (Elicitation)
\z

\ea\label{bkm:Ref105167438}
\textit{mwen té di’y sa}\\
$[$mwɛ̃tedijsa$]$\\
\glll mwen té di -y sa\\
\ref{mtsub5} \ref{mtten9} \ref{mtbase18} -\ref{mtobj19} \ref{mtnpc22} \\
\First\Sg{} \Pst{} tell -\Third\Sg.\Obj{} \Dem{}.\Pr{}\\
\glt `I had told her/him that.' (Elicitation)
\z

\ea\label{bkm:Ref105166251}
\textit{i té di mwen monté épi’y}\\
$[$itedimwɛ̃mõteepij$]$\\
\glll i té di -mwen monté épi -y\\
\ref{mtsub5} \ref{mtten9} \ref{mtbase18} \ref{mtobj19} \ref{mtnpc22} {} {} \\
\Third\Sg.\Sarg{} \Pst{} tell -\First\Sg{} go.up \Prep{}.with -\Third\Sg.\Obj{}\\
\glt `(S)he had told me to come with her/him.' (Elicitation)
\z

\ea\label{bkm:Ref105166274}
\textit{man té di zot monté épi mwen}\\
$[$mãtedizɔtmõteepimwɛ̃$]$\\
\glll man té di -zot monté épi -mwen\\
  \ref{mtsub5} \ref{mtten9} \ref{mtbase18} \ref{mtobj19} \ref{mtnpc22} {} {} \\
\First\Sg.\Sarg{} \Pst{} tell -\Second\Pl{} go.up \Prep{}.with -\First\Sg{}\\
\glt `I had told you to come with me.' (Elicitation)
\z


\ea\label{bkm:Ref105166580}
\textit{man té di yo sa}\\
$[$mãtedijosa$]$\\
\glll man té di -yo sa\\
\ref{mtsub5} \ref{mtten9} \ref{mtbase18} \ref{mtobj19} \ref{mtnpc22}\\
\First\Sg.\Sarg{} \Pst{} tell -\Third\Pl.\Obj{} \Dem{}.\Pr{}\\
\glt `I had told them that.' (Elicitation)
\z


\ea\label{bkm:Ref105167067}
\textit{i té di mwen mwen té épi’y}\\
$[$itedimwɛ̃mwɛ̃teepij$]$\\
\glll i té di -mwen mwen té épi -y\\
\ref{mtsub5} \ref{mtten9} \ref{mtbase18} \ref{mtobj19} \ref{mtnpc22} {} {} {} \\
\Third\Sg.\Sarg{} \Pst{} tell -\First\Sg{} \First\Sg{} \Pst{} \Prep{}.with -\Third\Sg.\Obj{}\\
\glt `(S)he had told I was with her/him.' (Elicitation)
\z

 
\ea\label{bkm:Ref105167130}
\textit{mwen té di zot sa}\\
$[$mwɛ̃tedizɔtsa$]$\\
\glll mwen té di -zot sa\\
\ref{mtsub5} \ref{mtten9} \ref{mtbase18} \ref{mtobj19} \ref{mtnpc22} \\
\First\Sg.\Sarg{} \Pst{} tell -\Second\Pl.\Obj{} \Dem{}.\Pr{}\\
\glt `I had told you that.' (Elicitation)
\z

\ea\label{bkm:Ref105167256}
\textit{man té di yo monté épi mwen}\\
$[$mãtedijotmõteepimwɛ̃$]$\\
\glll man té di -yo monté épi -mwen\\
\ref{mtsub5} \ref{mtten9} \ref{mtbase18} \ref{mtobj19} \ref{mtnpc22} {} {} \\
\First\Sg.\Sarg{} \Pst{} tell -\Third\Pl{} go.up \Prep{}.with --\First\Sg{}\\
\glt `I had told them to come with me.' (Elicitation)
\z


\ea\label{bkm:Ref105168971}
\textit{mwen té di zot monté épi mwen}\\
$[$mwɛ̃tedizɔtmõteepimwɛ̃$]$\\
\glll mwen té di -zot monté épi -mwen\\
  \ref{mtsub5} \ref{mtten9} \ref{mtbase18} \ref{mtobj19} \ref{mtnpc22} {} {} \\
\First\Sg{} \Pst{} tell -\Second\Pl{} go.up \Prep{}.with -\First\Sg{}\\
\glt `I had told you to come with me.' (Elicitation)
\z


\ea\label{bkm:Ref105163197}
{...} \textit{nou pa menm konet} \textbf{\textit{li}}\\
$[$nupamɛmkonɛtli$]$\\
\glll {...} nou pa menm konet -\textbf{li}\\
{...} \ref{mtsub5} \ref{mtneg8} \ref{mtadv10} \ref{mtbase18} \ref{mtobj20}\\ 
{...} \First\Pl{} \Neg{} even know -\textbf{\Third\Sg.\Obj{}}\\
\glt `[…] we do not even know it.' (Narr MUR 067)
\z

Discussions with bilingual speakers of Martinican and French seem to show that the concept of stress does not make much sense out of a specific discursive context. In this case, the judgments collected do not deal with stress domains but with intonation choices motivated by pragmatic concerns. Therefore, I collaborated with four linguists to investigate Martinican’s prosody. They were sent the audio files and asked to identify the syllable(s) where the stress occur. Based on these files, they all agreed on the fact that the position of a possible stress varies and that it seems that Martinican would not have a stress system. The salience of some syllables over others would be intonation instead. Two linguists indicated where they perceive the stress. \tabref{tab:mart:key:6} shows that their analyses do not always converge.

\begin{table}
    \centering
    \caption{Stress analysis of examples (\ref{bkm:Ref105163185}) to (\ref{bkm:Ref105163197}) by two linguists}
    \label{tab:mart:key:6}
    \begin{tabular}{llp{4cm}p{4cm}}
         \lsptoprule
Example & Speaker & Linguist 1 & Linguist 2\\ \midrule
(\ref{bkm:Ref105163185}) & LOR & [mãʁiˈfɛj] & [mãʁiˈfɛj]\\
(\ref{bkm:Ref105166070}) & LOR & [mwɛ̃ˈbawli] & [mwɛ̃ˈbawli]\\
(\ref{bkm:Ref105166218}) & LOR & [mãbaˈtu] & [mãbaˈtu]\\
(\ref{bkm:Ref105166237}) & JUV & [mãteˈdijsa] & [mãteˈdijsa]\\
(\ref{bkm:Ref105167438}) & PAT & [mwɛ̃tedijˈsa] & [mwɛ̃tedijˈsa]\\
(\ref{bkm:Ref105166251}) & JUV & [itediˈmwɛ̃mõˈteeˈpij] & [iteˈdimwɛ̃mõteeˈpij]\\
(\ref{bkm:Ref105166274}) & JUV & [mãtedizɔtmõˈteepiˈmwɛ̃] & [mãteˈdizɔtmõteepimwɛ̃]\textasciitilde{} [mãtediˈzɔtmõteeˈpimwɛ̃]\\
(\ref{bkm:Ref105166580}) & JUV & [mãtedijoˈsa] & [mãteˈdijosa]\\
(\ref{bkm:Ref105167067}) & PAT & [itediˈmwɛ̃mwɛ̃ˈteeˈpij] & [itediˈmwɛ̃mwɛ̃teeˈpij]\\
(\ref{bkm:Ref105167130}) & PAT & [mwɛ̃ˈtedizɔtˈsa] & [mwɛ̃ˈtedizɔtsa]\\
(\ref{bkm:Ref105167256}) & PAT & [mwɛ̃tediˈjomõteeˈpiˈmwɛ̃] & [mwɛ̃ˈtedijomõˈteeˈpimwɛ̃]\\
(\ref{bkm:Ref105168971}) & PAT & [mwɛ̃tediˈzɔtmõteepiˈmwɛ̃] & [mwɛ̃ˈtedizɔtmõˈteeˈpimwɛ̃]\\
(\ref{bkm:Ref105163197}) & MUR & [nupamɛmkonɛˈtli] & [nupamɛmkonɛˈtli]\\
\lspbottomrule
    \end{tabular}
\end{table}


The linguists’ results suggest that the salient syllables do not always occur on the same position of the planar structure. For some examples, no salient syllable is identified in positions \ref{mtder17} to \ref{mtobj20}. For these cases, it would mean that there is not a subspan of structure that contains the predicative base and that has a salient syllable. It seems to be the case for examples (\ref{bkm:Ref105167438}), (\ref{bkm:Ref105166580}), and (\ref{bkm:Ref105167130}) where the two linguists either identify the TAM marker \textit{té} [te] (position \ref{mtten9}) or the demonstrative pronoun \textit{sa} [sa] (position \ref{mtnpc22}) as the salient syllable. When the linguists identify a salient syllable within a subspan of structure that contains the predicative base (position \ref{mtbase18}), the salient syllable is not consistent throughout the data. For instance, the syllable \textit{di -y} [dij] (positions \ref{mtbase18}-\ref{mtobj19}) is either salient, as in (\ref{bkm:Ref105166237}), or not salient, as in (\ref{bkm:Ref105167438}). \textit{Di -zot} [dizɔt] (positions \ref{mtbase18}-\ref{mtobj19}) is another example of this inconsistency as in examples (\ref{bkm:Ref105166274}), (\ref{bkm:Ref105167130}) and (\ref{bkm:Ref105168971}).

Consequently, more investigation needs to be pursued to address the question of the stress system of Martinican, if there is one.

Nevertheless, these preliminary thoughts shed a new light on the GEREC-2 writing system. As was explained in the introductory section dedicated to the presentation of the planar structure (\sectref{bkm:Ref102580696}), according to this writing system, object pronouns undergo different orthographic treatments. One could wonder if this convention is motivated by prosodic differences. The object pronouns that are preceded by a hyphen and integrated into the predicative word, in the GEREC-2 system, the third person singular \textit{{}-y} [j] for instance, do not seem to have prosodic features distinct from the pronouns which are not integrated into the predicative word, as -\textit{zot} [zɔt]. Hence the fact that they are not written with the same orthographic conventions cannot be justified on the grounds of a difference in their prosodic features.

\section{Conclusion}

Based on this corpus-based investigation on constituency, it has been highlighted that, within the morphosyntactic domain, the tests do not systematically converge. Out of fourteen morphosyntactic tests, five diagnostics converge namely free occurrence (largest), non-interruptibility (one free form), non-interrupti\-bil\-i\-ty (more than one free form), (non)-permutability - rigid (declarative/imperative), (non)-permutability - rigid (interrogative). Hence, dissociating between morpho\-syntactic and phonological tests does not solve the misalignments observed between the tests' results, as \citet[2]{tallman2021constituency} suggested.

The word candidate identified by the five converging morphosyntactic diagnostics comprises the predicate base, predicative derivational morphemes and object pronouns. This word candidate differs from the word defined by the GEREC-2 writing system. It is interesting to note that the word candidate based on constituency tests corresponds to how I have seen some speakers separate orthographic domains when writing in Martinican without using the official conventions. It would be of major interest to look at written corpora to see how Martinican speakers discriminate between words since most of Martinican speakers have not been initiated into the GEREC-2 writing system.

To sum up, if there is a convergence between morphosyntactic and phonological domains is still to be found. Further work on adverbials, derivational morphemes, and phonology will help to enhance the predicative planar structure and with it the investigation on constituency in Martinican. 

\section{Acknowledgments}

I am truly grateful to the 20 Martinican speakers who contributed to the data collection. I thank the editors Sandra Auderset, Hiroto Uchihara and Adam Tallman, the reviewers and my PhD supervisor Françoise Rose for their numerous comments and suggestions. I thank Denis Creissels, Magdalena Lemus Serrano and Gérard Philippson that participated in the analysis on the stress domain.


\printglossary

\sloppy\printbibliography[heading=subbibliography,notkeyword=this]

\end{document}  

