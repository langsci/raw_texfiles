\documentclass[output=paper]{langscibook}
\ChapterDOI{10.5281/zenodo.13208546}
\author{Taylor L. Miller \affiliation{State University of New York at Oswego}}
\title{Constituency and Wordhood in Kiowa}
\abstract{This chapter builds on previous work \citep{Miller:2015,Miller:2018,Miller:2020} and investigates wordhood in Kiowa, a polysynthetic Tanoan language spoken in Oklahoma, with a focus on the verbal predicate and clause. Using the Planar-Fractal Method \citep{Tallman:ur}, five candidates for wordhood are identified using twelve diagnostics (five morphosyntactic, six phonological, and deviations from biuniqueness). The candidates are identified by the convergence of both morphosyntactic and phonological criteria, and they are largely expected given previous analysis of the prosodic structure of Kiowa \citep[][]{Miller:2015,Miller:2018,Miller:2020}.}


\IfFileExists{../localcommands.tex}{%hack to check whether this is being compiled as part of a collection or standalone
   \usepackage{langsci-optional}
\usepackage{langsci-gb4e}
\usepackage{langsci-lgr}

\usepackage{listings}
\lstset{basicstyle=\ttfamily,tabsize=2,breaklines=true}

%added by author
% \usepackage{tipa}
\usepackage{multirow}
\graphicspath{{figures/}}
\usepackage{langsci-branding}

   
\newcommand{\sent}{\enumsentence}
\newcommand{\sents}{\eenumsentence}
\let\citeasnoun\citet

\renewcommand{\lsCoverTitleFont}[1]{\sffamily\addfontfeatures{Scale=MatchUppercase}\fontsize{44pt}{16mm}\selectfont #1}
  
   %% hyphenation points for line breaks
%% Normally, automatic hyphenation in LaTeX is very good
%% If a word is mis-hyphenated, add it to this file
%%
%% add information to TeX file before \begin{document} with:
%% %% hyphenation points for line breaks
%% Normally, automatic hyphenation in LaTeX is very good
%% If a word is mis-hyphenated, add it to this file
%%
%% add information to TeX file before \begin{document} with:
%% %% hyphenation points for line breaks
%% Normally, automatic hyphenation in LaTeX is very good
%% If a word is mis-hyphenated, add it to this file
%%
%% add information to TeX file before \begin{document} with:
%% \include{localhyphenation}
\hyphenation{
affri-ca-te
affri-ca-tes
an-no-tated
com-ple-ments
com-po-si-tio-na-li-ty
non-com-po-si-tio-na-li-ty
Gon-zá-lez
out-side
Ri-chárd
se-man-tics
STREU-SLE
Tie-de-mann
}
\hyphenation{
affri-ca-te
affri-ca-tes
an-no-tated
com-ple-ments
com-po-si-tio-na-li-ty
non-com-po-si-tio-na-li-ty
Gon-zá-lez
out-side
Ri-chárd
se-man-tics
STREU-SLE
Tie-de-mann
}
\hyphenation{
affri-ca-te
affri-ca-tes
an-no-tated
com-ple-ments
com-po-si-tio-na-li-ty
non-com-po-si-tio-na-li-ty
Gon-zá-lez
out-side
Ri-chárd
se-man-tics
STREU-SLE
Tie-de-mann
}
    \bibliography{../localbibliography}
    \togglepaper[4]
}{}



\glsresetall


\begin{document}
\maketitle

\section{Introduction} \label{sec:kiowaintroduction}

The definition of the word has been a longstanding focus of debate shaping multiple areas within linguistics \citep[e.g.][]{dixon:2002,dixon:2010,bruening:2018}. Polysynthesis has been a driving catalyst of the debate since first described by \citet{duponceau:1819}. Characteristic ``sentence words", or single words that encode all necessary information to be a free-standing utterance, challenge traditional understandings of the ``word" in all areas of grammar \citep[e.g.]{mithun:1983,fortescue:1994,evans:2002}. Thus, polysynthetic languages must play a central role in determining a definition of wordhood. Complicating matters, \citet{haspelmathword:2011} suggests the lack of uniform criteria and methods across studies precludes a viable definition of the word. Additionally, \citet{bickel:2017} argue defining the word may be beyond reach due to substantial variation across and within languages. \citegen{Tallman2021} Planar-Fractal Method offers a set of criteria that may be applied uniformly across languages, and this volume allows us to examine many languages (including a number of polysynthetic languages) while holding the methods constant. 

This chapter investigates wordhood in Kiowa, a polysynthetic Tanoan language spoken in Oklahoma, with a focus on the verbal predicate and clause. Building on previous work \citep{Miller:2015,Miller:2018,Miller:2020}, I use the Planar-Fractal Method to identify five candidates for wordhood using twelve diagnostics (five morphosyntactic, six phonological, and deviations from biuniqueness). The candidates are identified by the convergence of both morphosyntactic and phonological criteria, and they are largely expected given previous analysis of the prosodic structure of Kiowa \citep{Miller:2015,Miller:2018,Miller:2020}. 

\sectref{sec:language} provides an overview of the Kiowa language and its speakers. A brief grammatical sketch includes the phoneme inventories (\ref{sec:phonemes}), basic syllable structure (\ref{sec:syllable}), tone inventory (\ref{sec:tone}), the structure of the verb complex (\ref{sec:verbcomplex}), and syntactic information relevant to the present analysis (\ref{sec:syntaxclause}). I discuss how the data in this chapter is sourced and how it is presented in \sectref{sec:data}. In \sectref{sec:planar}, I present the flattened planar structure of the Kiowa clause. \sectref{sec:morphosyntactic} introduces five morphosyntactic constituency diagnostics to be applied to the Kiowa verbal planar structure: Free Occurrence (\ref{sec:free}), Non-interruptability (\ref{sec:non-interruptability}), Non-permutability (\ref{sec:non-permutability}), Subspan Repetition (\ref{sec:subspanrep}), and Ciscategorial Selection (\ref{sec:ciscategorial}). \sectref{sec:phonological} introduces the six phonological processes which will be examined with respect to the verbal planar structure. Segmental domains are considered first: Syllabification and sensitive phenomena (\ref{sec:syllab}), Cluster Devoicing (\ref{sec:cluster}), Vowel Truncation (\ref{sec:VT}), and Dental-Velar Switch (\ref{sec:DV}). The section concludes with an examination of Tone Lowering (\ref{sec:lowering}) and Pausing (\ref{sec:pausing}). Finally I evaluate Deviations from Biuniqueness in \sectref{biuniqueness}. All results are summarized and discussed in \sectref{sec:discussion}.

\section{The language and its speakers} \label{sec:language}

Kiowa is a North American language spoken in southwestern Oklahoma. Though originally classified as a linguistic isolate by \citet{powell:1891}, later work found a close relationship between Kiowa and the Tanoan languages of New Mexico and Arizona \citep{harrington:1910,harrington:1928,miller:1959,trager:1959}. \citet{hale:1962} showed that Kiowa should be classified as a Tanoan language, an affiliation which has since been adopted in subsequent work \citep[e.g.][]{Watkins:1984,harbour:2003,adger:2009,McKenzie:2012,sutton:2014,Miller:2015,Miller:2018,Miller:2020}.

The Kiowa Tribal Complex is located in Carnegie, Oklahoma. While tribal membership is in the thousands, local administrators and activists estimate there are approximately ten expertly fluent native speakers and fifty proficient speakers of the language \citep{Miller:2018}. The most fluent elders are over 90 years old. Efforts to bolster language use and awareness are beginning to see results thanks in large part to outreach events and teacher training through an Association for Native Americans (ANA) education grant awarded to the Kiowa tribe in 2016. Beginning at the same time, Dane Poolaw digitized and expanded upon work by Parker McKenzie, David Paddlety, Alecia Gonzales, and William Meadows, and compiled the Kiowa Language Student Glossary \citep{poolaw:up}. A large-scale online dictionary project is also underway including full entries in four orthography systems, audio, story analyses, and grammatical sketches \citep{miller:2019}. The four orthography systems will be presented and discussed in \sectref{sec:data}.

\subsection{Grammatical sketch} \label{sec:grammatical}

This subsection provides a basic overview of relevant aspects of Kiowa grammar to be referenced in the present analysis. The phoneme inventory is presented in \ref{sec:phonemes}, followed by syllable structure in \ref{sec:syllable}, and tone in \ref{sec:tone}. The final subsection (\ref{sec:morphosyntax}) concludes with a summary of the morphophonological structure of the Kiowa verb complex, as well as the basic order of a Kiowa clause. The descriptions are intended to be very brief, as these topics are to be presented, justified, and modified when necessary in later sections.

\subsubsection{Phoneme inventory} \label{sec:phonemes}

Kiowa’s phoneme inventory has been established in earlier work \citep[see][]{wonderly:1954,sivertsen:1956,merrifield:1959,Watkins:1984}. See \tabref{tab:consonants} for the consonant inventory. In traditional Kiowa literature, the affricate [t͡s] is transcribed as [c], but I am adopting the IPA conventions here. The phoneme /l/ is noteworthy, as it is only realized as [l] syllable-initially. Otherwise, it is affricated as [\textsuperscript{d}l]. Note, also, that the phonemic status of the glottal stop is controversial. Some work has concluded that the glottal stop in Kiowa is problematic and unpredictable and therefore phonemic \citep[][]{wonderly:1954,trager:1960}, while other work has explained its distribution as entirely predictable and thus not phonemic \citep[][]{sivertsen:1956,merrifield:1959,Watkins:1984}. The present analysis assumes the glottal stop is not a phoneme (adopting \citealt{Watkins:1984}’s analysis, but the phonemic status of the glottal stop is not relevant to the arguments made here. It is included in \tabref{tab:consonants} between parentheses, as this is an unresolved issue.)

\begin{table}
\begin{tabular}{lllllllllllll}
        \lsptoprule
         & \multicolumn{2}{l}{Labial} & \multicolumn{2}{l}{Dental} & \multicolumn{2}{l}{Alveolar} & \multicolumn{2}{l}{Palatal} & \multicolumn{2}{l}{Velar} & \multicolumn{2}{l}{Laryngeal} \\
         \midrule
         Stops &&&&&&&&&&& \\
         \hspace{.5cm} \textit{Plain} & p & b & t & d & & & &  & k & g & (ʔ)\\
         \hspace{.5cm} \textit{Ejective} & p' & & t' & & & & & & k' & & \\
         \hspace{.5cm} \textit{Aspirated} & pʰ & & tʰ & & & & & & kʰ & &  \\
         Affricates &&&&&&&&&&& \\
         \hspace{.5cm} \textit{Plain} &&&&& t͡s &&&&&& \\
         \hspace{.5cm} \textit{Ejective} &&&&& {t͡s'} &&&&&& \\
         Fricatives &&&&& s & z &&& & & h \\
         Nasals & & m & & n &&&&&&& \\
         Liquids &&&& l &&&&&&& \\
         Glides &&&&&&&& j &&& \\
         \lspbottomrule
    \end{tabular}
    \caption{Consonants (adapted from \citealt{Watkins:1984})}
    \label{tab:consonants}
\end{table}

Kiowa's vowel inventory may be found in \tabref{tab:vowels}. Monophthongs may be underlyingly short or long and oral or nasal. Diphthongs may be oral or nasal. Length is marked with the IPA symbol [:], and nasality is marked with the Polish hook (e.g. \k{a}). The Polish hook is used extensively in the existing research on Kiowa, and that usage is continued here in place of the more modern tilde in order to avoid conflict with tonal diacritics.

\begin{table}
    \begin{tabular}{l cccc}
        \lsptoprule
         &\multicolumn{2}{c}{Monophthongs} &  \multicolumn{2}{c}{Diphthongs}\\
         \cmidrule(r){2-3}\cmidrule(l){4-5}
         & Front & Back &  Front & Back \\
         \midrule
         High & i & u &    & uj \\
         Mid  & e & o &    & oj \\
         Low  & a & ɔ & aj & ɔj \\
         \lspbottomrule
    \end{tabular}
    \caption{Vowels (adapted from \citealt{Watkins:1984})}
    \label{tab:vowels}
\end{table}

\subsubsection{Syllable} \label{sec:syllable}

The basic syllable in Kiowa consists of a vocalic nucleus, optionally preceded by one consonant (or Cj cluster), and optionally followed by one consonant from the set /p, t, m, n, l, j/ \citep[][]{Watkins:1984}. The syllable may be schematized as (C)V(C). Thus, depending on the boundaries of syllabification, VCV sequences may be ambiguous in terms of syllabification. For example, a CVCV sequence may be syllabified as CV.CV as in the noun [m\`a:.j\k{í}] `woman' or as CVC.V as in the verb [b\`at.\^{ɔ}m] `You make it'. This ambiguity forms the crucial test for syllabification domains in Kiowa, which will be discussed in detail in \sectref{sec:syllab}.

\subsubsection{Tone} \label{sec:tone}

Pitch is contrastive in Kiowa (high, low, and falling). High tone (H) is marked with acute accent (e.g. á), low tone (L) is marked with a grave accent (e.g. \`a), and falling tone (HL) is marked with a circumflex (e.g. â). Only H and L are permitted on short vowels, while all three tones are permitted on long vowels or VC sequences when C is from the set /m, n, l, j/ \citep[][]{Watkins:1984}. A minimal triplet is provided below in (\ref{triplet}).

\ea H-L-F Minimal Triplet \label{triplet} \\
tʰ\'{ɔ}: `hunger' \\
tʰ\`{ɔ}: `sit, seat' \\
tʰ\^{ɔ}: `beyond'
\z 

Tones are modified through morphologically-conditioned (e.g. compound raising and lexically-specified tone lowering morphemes) and phonologically-condi\-tioned processes (tone lowering). The present analysis focuses entirely on phonological tone lowering, as it is not restricted to specific morphemes or morphological structures. Interested readers are directed to \citet{Watkins:1984}'s discussion of morphologically-conditioned tone processes.

\subsubsection{The verb complex} \label{sec:morphosyntax}

This subsection introduces previous accounts of the Kiowa verb and relevant morphophonological and syntactic information for the present analysis. This information, in particular, is expanded upon and updated in Sections \ref{sec:planar}-\ref{sec:phonological} within the present methodology. A linear organization of the verb complex in Kiowa is provided in \ref{verblinear}, which combines Watkins' \citet{Watkins:1984} and McKenzie's \citet{McKenzie:2012} analyses. Watkins refers to the extensive verb as the most complex word class in Kiowa.\footnote{Note that Watkins does not refer to any diagnostics for wordhood and is likely referring to traditional lexical categories and what could be considered an X\textsuperscript{0}.} With up to ten slots, the verb can form an independent clause through inflection, agreement, and the incorporation of verbs, nouns, and adverbs.

\ea \label{verblinear}
{\Pronom} - ({\Adv}) - ({\N}) - ({\V}) - {\Stem} - {\Asp} - ({\Neg}) - ({\Mods}) - ({\Hsy}) - (\Synt) 
\z 

Only three elements above are obligatory: a pronominal (\Pronom), the stem (\Stem), and a suffix indicating aspect (\Asp), which is sometimes pronounced (e.g. imperfective /-m\`a/), sometimes phonetically null (e.g. perfective /-$\varnothing$/), and sometimes collapsed with the \textsc{stem} via stem allomorphy or alternations (e.g. adding falling tone to indicate the imperative). Therefore, a verb complex in Kiowa may be very short as in (\ref{shortverb}) or extremely long as in (\ref{longverb}).

\ea \label{shortverb}
\glll 
h\'{ɔ}n $\varnothing$- tʰép -$\varnothing$ \\
{\Neg} [\Third\Sg]- go.out -\Pfv \\
\textcolor{white}{.} {\Pronom}- {\Stem} -{\Asp}  \\
\trans `He didn't go out.' \citep[][44]{Miller:2018}
\z

\ea \label{longverb}
\glll 
\`an \`a- bô:- pòl\`a:j\`i- \k{è}:- b\`an -m\`a \\
{\Hab} [\First\Sg]- always- rabbit- hunt- go -\Ipfv \\
\textcolor{white}{.} {\Pronom}- {\Adv}- {\N}- {\V}- {\Stem} -{\Asp} \\
\trans `I am always going rabbit hunting.' \citep[][44]{Miller:2018}\footnote{The verb stem is incorrectly transcribed as [bá:] in \citealt{Miller:2018}. This is corrected here.}
\z

Historically, the pronominal is a complex morphological element \citep{merrifield:1959,Watkins:1984,watkins:1993,adger:2007,miller:prep}. Previous research traditionally calls the pronominal a ``pronominal prefix", but this is modified here as ``prefix" is a misnomer. \citet{Watkins:1984} argues that the pronominal was composed of a tightly knit cluster of morphemes, which indicate the semantic role of the primary animate participant (agent or patient), that participant’s person and number, and the number of any third person object. Each piece of information is encoded as a sub-syllabic segment (C or V) or tone (H or L) in the form C\'VVC or C\`VVC. The semantic interpretations of each segmental slot and tone are provided in (\ref{pronominal}).

\eabox{\label{pronominal}
\begin{tabular}{lllll}
     C & -V & -V & -C & (L/H) \\
     Person & -Person No. & -Object & -Object No. & (Agent/Patient) \\
\end{tabular}
}

For example, consider the pronominal in (\ref{pronominalex}) below. Watkins glosses pronominals as bracketed strings containing primary role information (Agent, Patient, Object) like {[\Aarg:\Parg:\Obj]}, so the pronominal below is glossed as {[(x/\Aarg):\Second\Pl/\Parg:\Pl/\Obj]}. In this case, there is an implied agent of unspecified person, a second person plural patient is the primary participant, and there is a plural object. Implied agents are not marked explicitly, so the segmental and tonal information comes from the Patient and the Object. Because the patient is second person, the first morpheme slot for Person is filled with /b/. The second slot for Person Number is then filled with /ɔ/, since the patient is plural.\footnote{/ia/ actually indicates that the number of the patient is non-dual, non-inverse, and non-singular \citep[118]{Watkins:1984}. I have abbreviated this as `plural' here for clarity.} The third slot for Object is filled with /ia/, since the object is plural. The last morpheme slot is then filled with /d/, since a plural object is non-singular. Finally, the pronominal is marked with a high tone, since the primary animate participant is a patient.\footnote{Interested readers are directed to \citet{Watkins:1984} and to \citet{miller:prep} for a discussion of the pronominal prefixes and all of the possibilities for each of the slots.} The analysis is complex and abstract, but it is the best analysis of the patterns observed in Kiowa pronominals to date.

\eabox{ \label{pronominalex}
\begin{tabular}{llll}
     b & -\'{ɔ} & -ia & -d \\
     {\Second} & -{\Pl} & -{\Pl} & -{\Nonsg} \\
\end{tabular}
}

Each slot is then subject to a series of phonological processes yielding a surface form that can be quite different from the underlying form. All but one process (nasalization) are productive and seen outside of pronominals in Kiowa.\footnote{\citet{Watkins:1984} argues that the nasalization process may have been more widespread historically. Interested readers are directed to her discussion beginning on page 48.} As seen below, the underlying form is subject to four processes: Vowel Truncation, Glide Formation, Glide Deletion, and Final Devoicing.

\ea {[(x/\Aarg):\Second\Pl/\Parg:\Pl/\Obj]} \citep[][41---42]{Watkins:1984} \label{pronominalderiv} \\
\begin{tabular}{ll}
     {/b-\'{ɔ}-ia-d/} & \\
     biád & Vowel Truncation \\
     bjád & Glide Formation \\
     bád & Glide Deletion \\
     bát & Final Devoicing \\
     {[bát]} & \\
\end{tabular}
\z

This decomposition is not active synchronically. Speakers are not aware of meaningful segmental morphemes, and they instead focus on the complex meanings of the pronominals. Therefore, I treat them as single elements in the present analysis.

The verb stem may consist of a simple root or a root combined with derivational or inflectional endings resulting in several different kinds of stems, including derived transitives, intransitives, and thematic stems \citep[][]{Watkins:1984}. The verb obligatorily marks Aspect (e.g. perfective vs. imperfective) via suffixation (e.g. /-má/ `imperfective'), stem allomorphy (e.g. imperfective stems are marked by final -n, -l, or a falling tone on the root vowel), zero allomorphs (e.g. perfective stems are sometimes marked by -$\varnothing$), or a combination of the three. All other verb endings are optional but must occur in the order Aspect - Negative - Modality - Hearsay. The two modality suffixes (imperative and future) may co-occur in that order when modifying an imperfective stem.\footnote{The negative suffix only adds to perfective stems. Additionally, of modality suffixes (imperative and future), only future may co-occur with the other suffixes in this string unless the imperative and future co-occur together \citep[][]{Watkins:1984}.} Consider the Stem /bá:/ `go' in (\ref{infl}). In (\ref{A}), the stem is inflected as perfective. Because the root ends in a long vowel, a zero allomorph attaches, and the stem appears unchanged. When imperfective in (\ref{I}), the stem ends in [n] and the suffix \textit{-má}. In (\ref{AN})-(\ref{ANMH}), all suffixes attach to the perfective stem from (\ref{A}).

\ea Inflections of /bá:/ `go' \label{infl} \\
\ea Stem-Aspect (Perfective) \label{A}\\
\gll bá: -$\varnothing$\\
go -{\Pfv}\\
\trans `went'

\ex Stem-Aspect (Imperfective) \label{I}\\
\gll bán -m\`a\\
go -{\Ipfv}\\
\trans `went'

\ex Stem-Aspect-Negative \label{AN} \\
\gll bá: -m\^{ɔ}\\
go.{\Pfv} -{\Neg}\\
\trans `not go'

\ex Stem-Aspect-Negative-Modality \label{ANM} \\
\gll bá: -m\^{ɔ}: -t'\`{ɔ}:\\
go.{\Pfv} -{\Neg} -{\Fut}\\
\trans `will not go'

\ex Stem-Aspect-Negative-Modality-Hearsay \label{ANMH} \\
\gll bá: -m\^{ɔ} -t'\`{ɔ}: -dê:\\
go.{\Pfv} -{\Neg} -{\Fut} -{\Hsy}\\
\trans `will not go (it was said)'
\z

\z 

Preceding the stem but following the pronominal are optionally incorporated adverbs, nouns, and verbs (\ref{incorp}). Incorporated stems are bare (without suffixes) and are typically phonologically identical to their inflected perfective stems.\footnote{A notable exception to this is that incorporated verbs beginning in an underlying voiced obstruent or /h/ demonstrate a stem-initial ablaut rule. Interested readers are directed to \citet[60]{Watkins:1984} for a discussion of this process.}

\ea Incorporated Elements \citep[][46---47]{Miller:2018} \label{incorp}
\ea Adverb \\
\gll \`a- k\`{ɔ}ét- bá:\\
[\First\Sg]- fearfully- go.\Pfv\\
\trans `I fearfully went.'
\ex Verb \\
\gll \`a- d\k{è}:- hê:m -\`a\\
[\First\Sg]- sleep- die -{\Ipfv}\\
\trans `I'm sleepy/I'm about to sleep.'
\ex Noun \\
\gll bé- t͡sát- h\k{è}:dè \\
[\Second\Sg/\Aarg:\Inv/\Obj]- door- remove.\Ipfv\\
\trans `Open the door.'
\z
\z

\hspace*{-.9pt}Finally, syntactic markers indicate clausal relationships such as relative clauses, subordinating conjunctions, and switch-reference markers \citep[][]{Watkins:1984}.\footnote{\citet{Watkins:1984} calls these ``syntactic suffixes", but this is only true of the locatives. All others are clitics. Thus, I have chosen the more neutral term ``syntactic markers'' here.} A complete list of Kiowa’s syntactic markers is provided in (\ref{syntsuff}), and (\ref{relative}) shows the nominal basic suffix /-dè/ used in the relativization of the first verb complex referring back to the noun [kút] `book'.

\ea Syntactic Markers \citep[][230--244]{Watkins:1984} \label{syntsuff} \\
\begin{tabular}{lll}
    Nominal & /-dè/ & `basic' \\
    & /-g\`{ɔ}/ & `inverse' \\
    Locative & /-èm/ & `here/away' \\
    & /-òj/ & `at/generally' \\
    & /-\k{è}/ & `here' \\
    Switch-Reference & /-g\`{ɔ}/ & `and/same'\\
    & /-n\`{ɔ}/ & `and/different' \\
    & /-t͡sè/ & `when, if/same'\\
    & /-\k{è}/ & `when, if/different'\\
    & /-k'\`{ɔ}t/ & `yet, anyway/same' \\
    & /-\`{ɔ}t/ & `yet, anyway/different'\\
    Other & /-\`al/ & `although, even though'\\
    & /-dò/ & `because' \\
\end{tabular}
\z

\ea \label{relative}
\gll kút gjá- tót -dè j\k{á}- \k{\'ɔ}: \\
book [(\First\Sg/\Aarg):\Second,\Third\Sg/\Parg:\Sg/\Obj]- {send.\Pfv} -{\Nom} [(\Second,\Third\Sg/\Aarg):\First\Sg/\Parg:\Pl/\Obj]- give.\Imp\\
\trans `Give me the book that was sent.' \citep[][47]{Miller:2018}
\z 

\subsubsection{Relevant syntax} \label{sec:syntaxclause}

Kiowa demonstrates a basic SOV word order \citep[e.g.][]{Watkins:1984,watkins:1990,harbour:2003,adger:2007,adger:2009,McKenzie:2012} as seen in (\ref{dog}), though it is subject to change due to discourse factors. For example, topics may be left-dislocated and given nouns may be right-dislocated after the verb. When two objects are present, the indirect object precedes the direct object. Kiowa is also a pro-drop language, and any argument can be left out. In fact, most Kiowa sentences consist only of a verb and its pronominal. 

\ea \label{dog}
\gll t͡ség\`un s\`a:né $\varnothing$- hân\\
dog snake [\Third\Sg/\Aarg:\Sg/\Obj]- eat.\Pfv\\
\trans `The dog ate the snake.' \citep[][48]{Miller:2018}
\z 

Determiner Phrases consist of Quantifier - Demonstrative - Noun. Demonstratives are the only overt determiners in Kiowa (\ref{DP}). There are no adjectives in Kiowa. Instead, adjectival modification occurs through compounds (\ref{compounding}) or relative clauses (recall \ref{relative}). 

\ea \label{DP}
\gll té: új -g\`{ɔ} t͡s\k{ê}: -g\`{ɔ}\\
all that -{\Inv} horse -{\Inv}\\
\trans `All those horses' \citep[adapted from][35]{McKenzie:2012}
\z

\ea \label{compounding}
\gll k'j\k{á}:h\k{î}: + ét\\
man + be.big\\
\trans `big man' \citep[][48]{Miller:2018}
\z

Relative clauses are head-internal and marked with a clause-final nominalizer that agrees in number with the head noun (/-dè/ or /-g\`{ɔ}/). They are optionally preceded by a subordinating particle /\'{ɔ}g\`{ɔ}/ to provide clarity as in (\ref{subparticle}), and the relative anaphoric particle /ám/ is used when the relativized noun has been mentioned previously or the speaker assumes the addressee has it in mind (\ref{subanph}). When both particles co-occur, the subordinating particle precedes the anaphoric particle.

\ea \label{subparticle}
\gll {\{}\'{ɔ}g\`{ɔ} sôl b\`at- tá:- {\k{\`ɔ}}:m -è -dè{\}} gj\`a- ból- d\`{ɔ}:\\
{\{}{\Sub} onion [\Second\Sg/\Aarg:\Pl/\Obj]- cook- do -{\Pfv} -{\Nom}/{\Bas}{\}} [\Pl]- rotten- be  \\
\trans `The onions that you cooked are rotten.' \citep[][231]{Watkins:1984}
\z

\ea \label{subanph}
\gll {\{}\'{ɔ}g\`{ɔ} ám kút b\`at- h\'{ɔ}: -gj\`a -dè{\}} j\k{á}- \k{\'ɔ}: \\
{\{}{\Sub} {\Anph} book [\Second\Sg/\Aarg:\Pl/\Obj]- get -{\Pfv} -{\Nom}/{\Bas}{\}} [(\Second,\Third\Sg/\Aarg):\First\Sg/\Parg:\Pl/\Obj]- give.\Ipfv\\
\trans `Give me that book that you bought.' \citep[][231]{Watkins:1984}
\z

\citet{McKenzie:2012} shows that relative clauses are embedded using scope facts and center-embedding, which I also assume here. In a neutral order, relative clauses occur in place of the relativized noun. In questions, the relative clauses are left-dislocated (\ref{RCleft}). To indicate new information or contrast, the head itself can be left-dislocated from the relative clause as in (\ref{lefthead}). Finally, like overt DPs, the relative clause can also be right-dislocated to indicate that it is old information. 

\ea \label{RCleft}
\gll {\{}\'{ɔ}g\`{ɔ} k'j\k{á}:h\k{î}: {$\varnothing$}- p\k{ó}:- t͡sán -dè{\}} h\'{ɔ} Lawton-g\`u {$\varnothing$}- bá: \\
{\Sub} man [\Third\Sg]- see- {arrive.\Pfv} -{\Nom} {\Q} Lawton-to [\Third/\Sg]- {go.\Pfv}\\
\trans `Did the man who came to see you go to Lawton?' \citep[][212]{Watkins:1984} %17b
\z 

\ea \label{lefthead}
\gll Gene {$\varnothing$}- t\k{ó}: -t\'{ɔ}: tógúl {\{}\'{ɔ}g\`{ɔ} t͡ség\`un \`a- p'\^{ɔ}j -dè{\}}. \\
Gene [\Third\Sg/\Aarg:\Sg/\Obj]- talk.to -{\Ipfv} boy  {\{}{\Sub} dog [(\Second,\Third\Sg/\Aarg):\Third\Sg/\Parg:\Sg/\Obj]- lose.{\Pfv} -{\Nom}/{\Bas} \\
\trans `Gene is talking to the boy who lost his dog.' \citep[][234]{Watkins:1984}
\z

Questions use a sentence-initial yes/no question particle [h\`{ɔ}] as in (\ref{yn}). Wh-words are obligatorily fronted as in (\ref{wh}).

\ea \label{yn}
\gll á- jój -g\`{ɔ} h\`{ɔ} bèt- k\'{ɔ}j- t\k{ò}- hájgjá- d\'{ɔ}:\\
your- child.{\Inv} -{\Inv} {\Q} [\Second\Pl/\Aarg:\Pl/\Obj]- Kiowa- speak- know- be\\
\trans `Do your children speak Kiowa?' \citep[][48]{Miller:2018}
\z

\ea \label{wh}
\gll h\^{ɔ}ndé $\varnothing$- d\'{ɔ}:\\
what [\Third\Sg]- be \\
\trans `What is it?' \citep[][48]{Miller:2018}
\z

\subsection{Data presentation and sources} \label{sec:data}

All data presented in this chapter comes from previously published sources on Kiowa or my own fieldwork on the language in 2016 and 2019. It is provided in the International Phonetic Alphabet (IPA) rather than a Kiowa orthographic system. There is no standard Kiowa orthography, though there are four systems currently in use: the Original Parker McKenzie system (OPM), two Modified McKenzie systems (MMB uses a bracket notation and MMS uses a strike-through notation), and the Gonzales Phonic System (GPS).\footnote{Another system of note is the Summer Institute of Linguistics (SIL) system used to publish Kiowa hymns (\citealt{gibson:1962}; reprinted as sleeve notes in \citealt{kotay:2005}), which is still well-liked.} 

Parker McKenzie was a Kiowa leader and linguist who devoted the majority of his life to the study of the language and the development of an orthographic system. The system is a phonetic transcription system, aiming for a one-to-one relationship between symbols and sounds much like the IPA. The system is summarized and published in \citet{mckenzie:2001}. It is praised for its phonetic accuracy in \citet{watkins:2010}. The system has also been used extensively in various works on Kiowa \citep[e.g.]{palmer:2003,meadows:2010,mckenzie:2010,McKenzie:2012,mckenzie:2015,sutton:2014}. Though the most popular orthography amongst language learners (e.g. at University of Oklahoma) and linguists for its marking of vowel length, nasality, and tone, older native speakers tend to find it difficult to understand. Language learners also struggle with how non-English sounds are transcribed, and it is difficult to use his diacritic system on a computer without complex unicode combinations or using typesetting systems like LaTeX.

Alecia Gonzales, a Kiowa speech language pathologist, used much of Parker McKenzie’s work as a guide when creating a more user-friendly orthography for pedagogical purposes \citep{gonzales:2001}. The GPS is a transphonic system, and it is decidedly closer to English orthography. It bypasses marking tone entirely, while marking nasalization and non-English sounds with a series of digraphs and trigraphs. It is also largely written in monosyllabic or monomorphemic chunks. Though it is successfully used in the classroom, it can be confusing without certain phonemic properties listed and is not well-suited to linguistic study. \citet{neely:2009} offer a comparison between the GPS and OPM systems, as well as examining the larger context of language ideologies. 

The final two systems are closely related to the OPM system. The Modified McKenzie Bracket and Modified McKenzie Strike-through systems update OPM to include more intuitive symbols. The MMS was largely created at University of Oklahoma by Kiowa teachers and activists involved in language classes, and it is the orthography used in the Kiowa Student Language Glossary \citep{poolaw:up}. The MMB was adapted by the Kiowa Language \& Culture Revitalization Program in an effort to turn the MMS into a more ``texting-friendly" system that does not require any special or conditional formatting like a strike-through. They almost exclusively use the MMB system now in their language materials. 

A side-by-side comparison of all four systems are presented in \tabref{tab:04:sidebyside} alongside the IPA.

\begin{table}
\caption{\label{tab:04:sidebyside} Kiowa orthography comparison. The translation between systems is my own. \label{taorthography} }
\begin{tabular}{llll} \lsptoprule
     & `come here' & `one' & `man' \\ \midrule
    IPA & èm-\k{á}: & pá:g\`{ɔ} & k'j\k{á:}h\k{î}: \\
    OPM & èm \uline{{\strut}\'{\={a}}} & f\'{\={a}}g\`au & q\uline{{\strut}\'{\={a}}}h\uline{{\strut}\^{\={i}}} \\
    MMB & èm {á}:]n & ]bá:g\`au & k'já:]nhî:]n \\
    MMS & èm {á}:\sout{n} & \sout{b}á:g\`au & k'já:\sout{n}hî:\sout{n} \\
    GPS & aim ahn & pbah gaw & kxai-hehn \\ \lspbottomrule
\end{tabular}
\end{table}

It is worth noting that the use of spaces to connote word boundaries varies widely between speakers of Kiowa. Using GPS, most spaces occur between monosyllables or simple morphemes. Dashes are sometimes used, though, this seems to be dependent on who is writing. Most language learners use OPM or one of the Modified McKenzie systems. Though word boundaries in those systems are considered to be more along the lines of what a linguist would assume (grouping bound morphemes together into complexes), language learners often default to spaces between syllables at first. This is likely due to language learners not yet understanding the meanings associated with each morpheme. Instead, they focus on individual syllables at a time. In my experience, native speakers who use an orthography can agree on the meaning of individual morphemes but vary in identifying where words are. This is particularly interesting for this chapter, as it raises questions about the psychological reality of any wordhood candidates for native speakers and language learners alike.  

\section{Planar structure} \label{sec:planar} 

For this analysis, I adopt the Planar-Fractal Method first introduced in \citealt{Tallman:ur}. All morphological and syntactic information is flattened and presented as a planar structure to eliminate as many a priori assumptions about structural relationships or constituency as possible.\footnote{Interested readers are directed to \citealt{Tallman:ur} for an in depth discussion of the motivation behind the Planar-Fractal Method. Such a discussion is beyond the purview of the present chapter.} Planar structures include elements, positions, slots, and zones.

\ea {Planar Structure Properties \citep[][10--11]{Tallman:ur}}
\ea \textbf{Element}: A formative, morpheme, affix, clitic, root, stem, phrase, clitic, or
compound. Or more generally any simplex element or definable subspan
of the planar structure. An element can refer to a whole paradigm of categories
(e.g. associated motion) or a single morpheme (e.g. \textit{=yó} ‘completive’)
which may not enter into paradigmatic relations.
\ex \textbf{Position}: Planar structures are made up of positions. Each position in a
template has a number that is used to account for relative ordering of its
elements within the planar structure. Each position is either a slot or a
zone.
\ex \textbf{Zone}: A type of position where more than one element can occur, and the
elements are not constrained with respect to their ordering. For example,
a zone with the elements {a, b} can output five possible strings: ∅, ab, ba,
a or b.
\ex \textbf{Slot}: A type of position where only one element can occur at a time. If elements
are listed as potentially occupying a slot, they are mutually exclusive.
For example, a slot with elements {a, b} can output three possible
strings: ∅, a or b.
\z
\z

\begin{table}
\caption{Kiowa verbal planar structure}
\label{tab:verbalplanarkiowa}
\small
\begin{tabularx}{\textwidth}{Srlll}
\lsptoprule
	\multicolumn{1}{r}{Pos.}    & Type  & Elements  & Forms \\ \midrule
	\label{VLeftRC}  &   Slot & \textsc{Left-Dislocated RC} & \\
    \label{VQuestionParticles}  &   Slot & \textsc{Question Particles/WH Words} & \textit{h\'{ɔ}, h{â}:t{ê}l, h\^{ɔ}ndé, etc.} \\
    \label{VClauseIntroducers} & Slot & \textsc{Clause Introducers} & \textit{hét\'{ɔ}, hèg\'{ɔ}} \\
    \label{VM1} & Zone & \textsc{Modal Particles} & \textit{p\`ah\k{í}:, bèth{ê}ndè, m\'{ɔ}n, etc.} \\
    \label{VTA1} & Zone & \textsc{Tense/Aspect Particles} & \textit{s\'{ɔ}t, mîn, \`an, etc.} \\
    \label{VAdverbs1} & Slot & \textsc{Adverbs} (place, manner, time) & \\ 
    \label{VNL1} & Slot & \textsc{Noun-Locative Adverbials} & \\ 
    \label{VM2} & Zone & \textsc{Modal Particles} & \textit{p\`ah\k{í}:, bèth{ê}ndè, m\'{ɔ}n, etc.} \\
    \label{VTA2} & Zone & \textsc{Tense/Aspect Particles} & \textit{s\'{ɔ}t, mîn, \`an, etc.} \\
    \label{VNegation} & Slot & \textsc{Negation} & \textit{h{\'{ɔ}n}, pòj, hę́:} \\
    \label{VM3} & Zone & \textsc{Modal Particles} & \textit{p\`ah\k{í}:, bèth{ê}ndè, m\'{ɔ}n, etc.} \\
    \label{VTA3} & Zone & \textsc{Tense/Aspect Particles} & \textit{s\'{ɔ}t, mîn, \`an, etc.} \\
    \label{VAgent} & Slot & DP \{A, S\} or RC & \\ 
    \label{VNL2} & Slot & \textsc{Noun-Locative Adverbials} & \\
    \label{VM4} & Zone & \textsc{Modal Particles} & \textit{p\`ah\k{í}:, bèth{ê}ndè, m\'{ɔ}n, etc.} \\
    \label{VTA4} & Zone & \textsc{Tense/Aspect Particles} & \textit{s\'{ɔ}t, mîn, \`an, etc.} \\
    \label{VPatient} & Slot & DP \{P, i.o.\} or RC & \\ 
    \label{VNL3} & Slot & \textsc{Noun-Locative Adverbials} & \\
    \label{VM5} & Zone & \textsc{Modal Particles} & \textit{p\`ah\k{í}:, bèth{ê}ndè, m\'{ɔ}n, etc.} \\
    \label{VTA5} & Zone & \textsc{Tense/Aspect Particles} & \textit{s\'{ɔ}t, mîn, \`an, etc.} \\
    \label{VDirectObject} & Slot & DP \{d.o.\} or RC & \\  
    \label{VNL4} & Slot & \textsc{Noun-Locative Adverbials} & \\
    \label{VM6} & Zone & \textsc{Modal Particles} & \textit{p\`ah\k{í}:, bèth{ê}ndè, m\'{ɔ}n, etc.} \\
    \label{VTA6} & Zone & \textsc{Tense/Aspect Particles} & \textit{s\'{ɔ}t, mîn, \`an, etc.} \\
    \label{Vpro} & Slot & {\textsc{Pronominal}} & \\
    \label{VIncorpAdv} & Slot & \textsc{Incorp. Adverb} & \\
    \label{VIncorpN} & Slot & \textsc{Incorp. Noun} & \\
    \label{VIncorpV} & Slot & \textsc{Incorp. Verb} & \\ 
    \label{VStem} & Slot & \textsc{Verb Stem} (Root-Deriv) & \\
    \label{VAsp} & Slot & \textsc{Aspect Suffix} & \textit{-m\`{ɔ}, -g\`u, -(m)i\`a} \\
    \label{VNegativeSuffix} & Slot & \textsc{Negative Suffix} & \textit{-\^{ɔ}: allomorphs} \\
    \label{VMod} & Zone & \textsc{Modality Suffix} & \textit{-t\'{ɔ}:, -t'\'{ɔ}:, -î} \\
    \label{VHearsay} & Slot & \textsc{Hearsay Suffix} & \textit{-h{ê}l, etc. allomorphs}\\
    \label{VNominal} & Slot & \textsc{Nominalizer/Relativizer Suffix} & \textit{-dè, -g\`{ɔ}, -n\`{ɔ}, etc.} \\
    \label{VLocative} & Slot & \textsc{Locative/Directional Suffix} & \textit{-èm, -òj, \k{è}:}, etc. \\
    \label{VSubordinate} & Slot & \textsc{Subordinate Markers} & \textit{switch-reference markers, etc.}  \\ 
    \label{VAdverbs2} & Slot & \textsc{Adverbs} (place, manner, time) & \\ 
    \label{VNL5} & Slot & \textsc{Noun-Locative Adverbials} & \\
    \label{VDislocatedDP} & Slot & \textsc{Right-Dislocated DP or RC} & \\
\lspbottomrule
\end{tabularx}
\end{table}


The Kiowa verbal planar structure is presented in \tabref{tab:verbalplanarkiowa}. The structure expands upon the brief explanation of the Kiowa verb and syntactic information of the larger clause in \ref{sec:grammatical}. As mentioned before, the only required elements in a clause are the pronominal (Position \ref{Vpro}), the verb stem (simple or derived in Position \ref{VStem}), and some Aspectual marking (Position \ref{VAsp} when a suffix). Note that overt DPs are included in their neutral pre-verbal position, but arguments are encoded via the pronominal.

Discontinuity is common in Kiowa. I have attempted to account for it as much as possible by indicating all places in the planar structure where certain elements may appear. As mentioned earlier, overt DPs and relative clauses may be right-dislocated due to new/old information or to avoid clashes with similar words. Relative clauses may also left-dislocate in questions, which is indicated in Position (\ref{VLeftRC}). These positions are included in the planar structure, but do not affect any diagnostics and therefore will not be discussed much further. Finally, the subordinating and anaphoric particles in relative clauses mentioned in \sectref{sec:syntaxclause} are assumed as possible initial positions within any ``RC" below but are not included in the overall planar structure.


\noindent Before turning to any constituency tests, let us examine each of the positions in \tabref{tab:verbalplanarkiowa}. The remainder of this section is divided into the following subsections: Clause-Initial Elements (\ref{sec:clauseinitial}), Adverbials and Negation (\ref{sec:adverbials}), Modal and Tense/Aspect Particles (\ref{sec:MTA}), and the Verb Complex (\ref{sec:verbcomplex}).

\subsection{Clause-initial elements} 
\label{sec:clauseinitial}

Questions are introduced with a question particle (\textit{h\'{ɔ}}) or {\textsc{wh}}-word in Slot \ref{VQuestionParticles} as in (\ref{questions}). Questioned relative clauses are the only elements which may occur earlier in the clause, which will be discussed in \sectref{sec:verbcomplex}.

\ea Questions \label{questions} \\
\ea
\glll \uline{{\strut}h\'{ɔ}} mén- gút\\
\uline{{\strut}\ref{VQuestionParticles}} {\ref{Vpro}}- \ref{VStem}.\ref{VAsp} \\
\uline{{\strut}\Q} [(\X/\Aarg):\Third\Du/\Parg:\Pl/\Obj]- write.\Pfv\\
\trans `Did you write to them?' \citep[][212]{Watkins:1984}

\ex
\glll \uline{{\strut}h\^{ɔ}ndé} $\varnothing$- d\`{ɔ}:\\
\uline{{\strut}\ref{VQuestionParticles}} {\ref{Vpro}}- \ref{VStem} \\
what [\Third\Sg]- be\\
\trans `What is it?' \citep[][48]{Miller:2018}
\z
\z

\noindent Clause introducing particles (\textit{hèg\'{ɔ}} `now, then'\footnote{The particle \textit{hèg\'{ɔ}} is commonly used as a filler word in Kiowa. It is also often truncated or reduced, sometimes only pronounced as [g] (Andrew Robert McKenzie, p.c.). For this chapter, I will focus on its non-filler use, distribution, and restrictions.} or \textit{hét\'{ɔ}} `still') follow in position \ref{VClauseIntroducers} as in (\ref{clauseintroducer}) and (\ref{Qintroduce}).

\ea Clause Introducer \label{clauseintroducer} \\
\glll \uline{{\strut}hèg\'{ɔ}} ját dè- kò:dó- pè:tòp\\
\uline{{\strut}\ref{VClauseIntroducers}} \ref{VAdverbs1} {\ref{Vpro}}- \ref{VIncorpAdv}- \ref{VStem}.\ref{VAsp}\\
\uline{{\strut}now} right.now [\First\Sg/\Refl]- very- try.\Ipfv\\
\trans `I'm really trying right now.' \citep[][218]{Watkins:1984}
\z

\newpage
\ea Question and Clause Introducer \label{Qintroduce} \\
\glll \uline{{\strut}h\'{ɔ}} \uline{{\strut}hèg\'{ɔ}} g\'{ɔ}- tʰét -kjá\\
\uline{{\strut}\ref{VQuestionParticles}} \uline{{\strut}\ref{VClauseIntroducers}}  {\ref{Vpro}}- \ref{VStem} -\ref{VAsp}\\
\uline{{\strut}{\Q}} \uline{{\strut}now} [(\First\Sg/\Aarg):\Second\Sg/\Parg:\Inv/\Obj]- cut.open -\Det/\Pfv\\
\trans `Did you manage to get it cut open?' \citep[][143]{Watkins:1984}
\z

\subsection{Adverbials and negation} \label{sec:adverbials}

Some elements are possible in multiple positions within the clause (adverbs and noun-locative adverbials) and are included at each location they may occur. For example, adverbs are possible in pre- and post-verbal Slots \ref{VAdverbs1} and \ref{VAdverbs2} as in (\ref{adverbs}).

\ea \label{adverbs}
\ea Pre-Verbal Adverb \\
\glll \uline{{\strut}g\k{í}:g\'{ɔ}:} \`an dé- kʰî:pòp \\
\uline{{\strut}\ref{VAdverbs1}} \ref{VTA6} {\ref{Vpro}}- \ref{VStem}.\ref{VAsp}\\
\uline{{\strut}early/morning} {\Hab} [\First\Sg/{\Refl}]- fly.up/{\Ipfv}\\\
\trans `I pop up early in the morning.' \citep[][209]{Watkins:1984}

\ex Post-Verbal Adverb \\
\glll jí:dè ójdè mátʰ\`{ɔ}n d\'{ɔ}- k'\'{ɔ}:t -é \uline{{\strut}kʰí:dêl}\\
{\ref{VDirectObject}} {\ref{VDirectObject}} {\ref{VDirectObject}} {\ref{Vpro}}- \ref{VStem} -\ref{VAsp} \uline{{\strut}\ref{VAdverbs2}}\\
both that girl {[(\X/\Aarg):\First\Pl/\Parg:$\varnothing$/\Obj]}- meet -{\Pfv} \uline{{\strut}yesterday}\\
\trans `Both those girls met us yesterday.' \citep[][210]{Watkins:1984}\footnote{The DP [jí:dè ójdè mátʰ\`{ɔ}n] `both those girls' forms a single preverbal direct object DP slot \ref{VDirectObject}. As DP structure is not within the scope of this chapter, I have chosen to mark each element within the DP as Slot \ref{VDirectObject}. This method will be adopted throughout the rest of the chapter whenever a multi-part DP is present in the clause.}
\z
\z

\noindent Noun-Locative Adverbials' neutral positions are post pre-verbal adverb (Slot \ref{VNL1}) as in (\ref{cooccur}) or after overt Nouns (Slots \ref{VNL2}, \ref{VNL3}, and \ref{VNL4}) as in (\ref{afterN}).\footnote{Note that in (\ref{afterN}c) the direct object - noun-locative sequence occurs within a relative clause. I have indicated the relative clause with braces.} 


\ea \label{cooccur}
Noun-Locative after Pre-verbal Adverb \\
\glll t'á:gj\`aj \uline{{\strut}m\'{ɔ}n-tò} gjá- pʰátt\`{ɔ}\\
\ref{VAdverbs1} \uline{{\strut}\ref{VNL1}} {\ref{Vpro}}- \ref{VStem}.\ref{VAsp}\\
carefully \uline{{\strut}hand-with} [\First\Sg/\Aarg:\Sg/\Obj]- smooth.\Ipfv\\
\trans `I was carefully smoothing it with my hands.' \citep[][210]{Watkins:1984}
\z

\newpage
\ea \label{afterN}
\ea Noun-Locative After Overt Agent (Slot \ref{VAgent}) \\
\glll tʰ\`aljóp \uline{{\strut}t͡sát-kj\`a} ét- m\'{ɔ}b\'{ɔ}tt\`{ɔ}\\
\ref{VAgent} \uline{{\strut}\ref{VNL2}} {\ref{Vpro}}- \ref{VStem}.\ref{VAsp}\\
boy/{\Inv} \uline{{\strut}door-at} [\Third/\Refl]- crowd.\Ipfv\\
\trans `The boys were crowding at the door.' \citep[][210]{Watkins:1984}

\ex Noun-Locative After Overt Patient (Slot \ref{VPatient})\\
\glll k'ɔnkʰ\k{í}:-g\`{ɔ} \uline{{\strut}tʰ\k{ó}:-kj\`a} è- jî: -j\`a\\
\ref{VPatient} \uline{{\strut}\ref{VNL3}} {\ref{Vpro}}- \ref{VStem} -\ref{VAsp}\\
turtle-{\Inv} \uline{{\strut}water-in} [\Third\Inv]- disappear -\Ipfv\\
\trans `The turtles are disappearing into the water.' \citep[][159]{Watkins:1984}

\ex Noun-Locative After Overt Object (Slot \ref{VDirectObject})\footnote{Note that the relative clause itself fills the direct object's Slot \ref{VDirectObject} in the matrix clause. This is indicated with a subscript outside the braces.}\\
\glll {\{}k'í: \uline{{\strut}k'ɔdá:l-\^{ɔ}:} $\varnothing$- òl- s\'{ɔ}l -dè{\}} gj\`a- p'étt\`{ɔ}\\
{\{}\ref{VDirectObject} \uline{{\strut}\ref{VNL4}} {\ref{Vpro}}- \ref{VIncorpV}- \ref{VStem} -\ref{VSubordinate}{\}}{\textsubscript{\ref{VDirectObject}}} {\ref{Vpro}}- \ref{VStem}.\ref{VAsp}\\
{\{}wood \uline{{\strut}wagon-on} [\Third\Sg]- load- be.in -{\Nom/\Bas}{\}} [\First\Sg/\Aarg:\Sg/\Obj]- take.down/\Ipfv \\
\trans `I am unloading wood that was loaded in the wagon.' \\ \citep[][230]{Watkins:1984}
\z
\z

\noindent If two Adverbs or Noun-Locative Adverbials are present, they may co-occur in Slots \ref{VAdverbs1} and \ref{VNL1} respectively as in (\ref{cooccur}) above. The second element tends to shift to the post-verbal Slots \ref{VAdverbs2} and \ref{VNL5} due to discourse factors (i.e. new/old information). Noun-locatives, for example, are right-dislocated to Slot \ref{VNL5} in (\ref{clausefinal}).

\ea \label{clausefinal}
Right-Dislocated Noun-Locative \\
\glll kʰí:dêl páj $\varnothing$- jâj \uline{{\strut}m\'{ɔ}s\'{ɔ}-j\`{ɔ}}\\
\ref{VAdverbs1} \ref{VDirectObject} {\ref{Vpro}}- \ref{VStem}.\ref{VAsp} \uline{{\strut}\ref{VNL5}}\\
yesterday sun [\Third\Sg]- disappear/{\Pfv} \uline{{\strut}six-at}\\
\trans `The sun set at six yesterday.' \citep[][210]{Watkins:1984}
\z

Negation is marked by a pre-verbal particle (\textit{h\'{ɔ}n} in most cases; negative imperatives are marked with \textit{pòj} and existential negatives are marked with \textit{h\k{é}:}) and a negative suffix on the verb (-{\^ɔ:}). The negative particle occurs in Slot \ref{VNegation}, and the negative suffix occurs in Slot \ref{VNegativeSuffix} after the verb stem. The negative particle is typically clause-initial (\ref{negation}), but it is optionally preceded by Question Particles/{\textsc{wh}}-Words and/or Clause Introducers (\ref{qintroduce}). In addition, adverbs and non-locatives in contrastive focus or introducing new information may occur before a negative particle (\ref{adverbnegation}).

\ea \label{negation}
\glll \uline{{\strut}h\'{ɔ}n} mátʰ\`{ɔ}n $\varnothing$- t͡s\k{á}:n -\uline{{\strut}\^{ɔ}:} kʰí:dêl-g\`{ɔ}:\\
\uline{{\strut}\ref{VNegation}} \ref{VPatient} {\ref{Vpro}}- \ref{VStem} -\uline{{\strut}\ref{VNegativeSuffix}} \ref{VNL4}\\
\uline{{\strut}{\Neg}} girl [\Third\Sg]- arrive -\uline{{\strut}{\Neg}} yesterday-since\\
\trans `The girl hasn't come since yesterday.' \citep[][214]{Watkins:1984}
\z

\ea Negation with Questions and Clause Introducers \label{qintroduce}\\
\ea 
\glll h\'{ɔ} \uline{{\strut}h\'{ɔ}n} k'j\k{á}:h\k{î}: \`a- b\k{ó}: -\uline{{\strut}m\^{ɔ}}\\
\ref{VQuestionParticles} \uline{{\strut}\ref{VNegation}} \ref{VDirectObject} {\ref{Vpro}}- \ref{VStem} -\uline{{\strut}\ref{VNegativeSuffix}}\\
{\Q} \uline{{\strut}{\Neg}} man [\Second\Sg/\Aarg:\Sg/\Obj]- see -\uline{{\strut}{\Neg}}\\
\trans `Didn't you see the man?' \citep[][215]{Watkins:1984}

\ex
\glll hét\'{ɔ} \uline{{\strut}h\'{ɔ}n} gj\`a- tʰáp- \'{ɔ}m -\uline{{\strut}g\^{ɔ}:}\\
\ref{VClauseIntroducers} \uline{{\strut}\ref{VNegation}} {\ref{Vpro}}- \ref{VIncorpV}- \ref{VStem} -\uline{{\strut}\ref{VNegativeSuffix}}\\
still \uline{{\strut}{\Neg}} [\Pl]- dry- become -\uline{{\strut}{\Neg}}\\
\trans `It still hasn't dried.' \citep[][215]{Watkins:1984}
\z
\z

\ea Preposed Adverbials and Negation \label{adverbnegation} \\
\glll hèg\'{ɔ} k\'{ɔ}j-d\`{ɔ}m-gj\`a \uline{{\strut}h\'{ɔ}n} m\`a- t͡s\k{á}:n -\uline{{\strut}\^{ɔ}:} -hèl h\`aótè-s\`aj\\
\ref{VClauseIntroducers} \ref{VNL1} \uline{{\strut}\ref{VNegation}} {\ref{Vpro}}- \ref{VStem} -\uline{{\strut}\ref{VNegativeSuffix}} -\ref{VHearsay} \ref{VAdverbs2}\\
now Kiowa-land-at \uline{{\strut}{\Neg}} [\Second\Du]- arrive -\uline{{\strut}{\Neg}} -{\Hsy} several-year\\
\trans `So (I hear) you haven't been in Kiowa country for several years.' \citep[][216]{Watkins:1984} 
\z

\subsection{Modal and tense/aspect particles} \label{sec:MTA}

Modal and tense/aspect particles are the most freely ordered elements in the Kiowa clause, as they are only required to occur pre-verbally, though they do occur in the relative order with modal followed by tense/aspect particles when they co-occur.\footnote{\citet{Watkins:1984} presents them as occurring in the opposite order, yet all data I have studied suggest otherwise. Therefore, I propose the order with modal particles occurring first unless future research shows otherwise.} There are eleven modal particles, which are listed in (\ref{tab:modal}). As seen in (\ref{modal}), \textit{hájáttò} translates to `maybe' and indicates uncertainty as to whether the event will happen. While \citet{Watkins:1984} argues modal particles occur in complementary distribution, one example has been found which shows two modal particles co-occuring (\ref{modal2}). Given this, I have indicated modal particles as a Zone, and exactly what may co-occur and in what order is left to future research.

\ea Modal Particles \citep[][221--223]{Watkins:1984}\label{tab:modal} \\
    \begin{tabular}{ll}
        p\`ah\k{í}: & `clearly' \\
        bèthêndè & `never, unlikely' \\
        m\'{ɔ}n & `probably' \\
        hájáttò & `maybe, might' \\
        h\`agj\`a & `maye, might' \\
        mágjá & `was going to, might (have)' \\
        dá & `must' \\
        j\`al & `hope' \\
        hét & `let's, let me' \\
        béth\`{ɔ}: & `unknowing' \\
        m\`{ɔ}\'{ɔ}jdèl & `fortunately not, if by ill fate' \\
    \end{tabular}
\z

\ea \label{modal}
\glll \uline{{\strut}hájáttò} h\'{ɔ}n ján- t͡sá:- \'{ɔ}mdé -t'\`{ɔ}:\\
\uline{{\strut}\ref{VTA2}} \ref{VNegation} {\ref{Vpro}}- \ref{VIncorpV}- \ref{VStem} -\ref{VMod} \\
\uline{{\strut}maybe} {\Neg} [(\First\Sg/\Aarg):\Second,\Third\Sg/\Parg:\Pl/\Obj]- go- become -{\Fut}\\
\trans `You might not be able to get there.' \citep[][221]{Watkins:1984}
\z 

\ea \label{modal2}
\glll \uline{{\strut}hét} \uline{{\strut}h\`agj\`a} \k{é}:dè kút ján- hájdé -t'\`{ɔ}:\\
\uline{{\strut}\ref{VM4}}\textsubscript{1} \uline{{\strut}\ref{VM4}}\textsubscript{2} \ref{VDirectObject} \ref{VDirectObject} \ref{Vpro}- \ref{VStem} -\ref{VMod}\\
\uline{{\strut}let's} \uline{{\strut}maybe} this letter [(\First\Sg:\Aarg):\Second,\Third\Sg/\Parg:\Pl/\Obj]- learn -\Fut\\
\trans `Let's see if maybe you can understand this letter.' \citep[][222]{Watkins:1984}
\z  

There are five tense/aspect particles which indicate immediate time (\textit{s\'{ɔ}t} `immediate/recent past', \textit{ját} `immediate present', \textit{mîn} `immediate/near future'), not-yet-achieved future events (\textit{mí:} `almost'), or habitual acts (\textit{\`an} `habitual'). For example, in (\ref{habitual}), the habitual particle \textit{\`an} indicates that the act of rabbit hunting is a repeated process.

\ea \label{habitual} 
\glll \uline{{\strut}\`an} \`a- bô:- pòl\`a:j\`i- \k{è}:- b\`an -má\\
\uline{{\strut}\ref{VTA6}} {\ref{Vpro}}- \ref{VIncorpAdv}- \ref{VIncorpN}- \ref{VIncorpV}- \ref{VStem} -\ref{VAsp}\\
\uline{{\strut}{\Hab}} [\First\Sg]- always- rabbit- hunt- go -\Ipfv\\
\trans `I'm always going rabbit hunting.' \citep[][44]{Miller:2018}
\z

\hspace*{-2.8pt}Just like modal particles, more than one tense/aspect particle is possible, though the first must be either \textit{hét\'{ɔ}} `still' or \textit{hèg\'{ɔ}} `now, then'. The same two particles were seen earlier as clause introducers (Slot \ref{VClauseIntroducers}), and if they occur clause-initially before another tense/aspect particle it is ambiguous if they are acting as clause introducers or tense/aspect particles. They do pattern more freely as part of the tense/aspect particle zone later in the clause, though, and that is unambiguously a case of two tense/aspect particles co-occurring. Consider, for example, the following example where \textit{hèg\'{ɔ}} occurs before another tense/aspect particle indicating the continuation of an event from the past to the present as in (\ref{TAZone}).

\ea \label{TAZone}
\glll á:kʰ\k{\`i}:gj\`a \uline{{\strut}hèg\'{ɔ}} \uline{{\strut}mîn} gjá- kʰ\k{î}: -m\`a\\
{\ref{VPatient}} \uline{{\strut}{\ref{VTA6}}$_{1}$} \uline{{\strut}{\ref{VTA6}}$_{2}$} {\ref{Vpro}}- \ref{VStem} -\ref{VAsp}\\
flowers \uline{{\strut}now} \uline{{\strut}about.to} [\Pl]- bloom -\Ipfv\\
\trans `The flowers are about to bloom.' \citep[][159]{Watkins:1984}%147a p. 159
\z

As mentioned earlier, modal and tense/aspect particles may also co-occur and in that order. See (\ref{mta}) as an example.

\ea \label{mta}
\glll \uline{{\strut}m\'{ɔ}n} \uline{{\strut}mîn} g\'{ɔ}- átt\`{ɔ}\\
\uline{{\strut}{\ref{VM1}}} \uline{{\strut}{\ref{VTA1}}} {\ref{Vpro}}- {\ref{VStem}.\ref{VAsp}}\\
\uline{{\strut}probably} \uline{{\strut}about.to} [(\X/\Aarg):\Second\Sg/\Parg:$\varnothing$/\Obj]- chase.\Ipfv\\
\trans `It (a bull) is probably about to chase you.' \citep[][221]{Watkins:1984}
\z

As they are the most freely ordered elements in the Verbal Planar Structure, the modal and tense/aspect particle zones are included in \tabref{tab:verbalplanarkiowa} in six possible positions prior to the verb complex. While complete data sets for each position are yet to be found (i.e. at least one modal particle, one tense/aspect particle, both a modal and tense/aspect particle), the present data are sufficient to indicate five of the six positions. The sixth position is assumed based on other patterns until data suggest otherwise. This will be discussed below.

The earliest position for both zones is after Clause Introducers (Slot \ref{VClauseIntroducers}) and before Adverbs (Slot \ref{VAdverbs1}) as Zones \ref{VM1} and \ref{VTA1} as in (\ref{ModalPosition1}) and (\ref{TAPosition1}) below. 

\ea Modal Particle in Zone \ref{VM1} \label{ModalPosition1} \\
\glll hét\'{ɔ} \uline{{\strut}m\'{ɔ}n} \k{é}:h\`{ɔ}: \'{ɔ}jh\`{ɔ}: èm- t'\'{ɔ}:\\
{\ref{VClauseIntroducers}} \uline{{\strut}{\ref{VM1}}} {\ref{VAdverbs1}} {\ref{VAdverbs1}} {\ref{Vpro}}- {\ref{VStem}.\ref{VAsp}}\\
still \uline{{\strut}probably} now there [\Second\Sg]- stay\\
\trans `You are probably still there now.' \citep[][219]{Watkins:1984}%31a p.219
\z

\ea Tense/Aspect Particle in Zone \ref{VTA1} \label{TAPosition1} \\
\glll hèg\'{ɔ} \uline{{\strut}ját} kóttè dè- pʰótt\`{ɔ}\\
\ref{VClauseIntroducers} \uline{{\strut}\ref{VTA1}} \ref{VAdverbs1} {\ref{Vpro}}- {\ref{VStem}.\ref{VAsp}}\\
now \uline{{\strut}right.now} hard [\First\Sg/\Refl]- blow.\Ipfv\\
\trans `I am really blowing hard.' \citep[][218]{Watkins:1984}%27b p. 218
\z

\noindent Both zones may also occur immediately before negation in Zones \ref{VM2} and \ref{VTA2}. For example, the modal particle \textit{hájáttò} `maybe' occurs in this position in (\ref{ModalPosition2}) below.

\ea Modal Particle in Zone \ref{VM2} \label{ModalPosition2} \\
\glll \uline{{\strut}hájáttò} h\'{ɔ}n ján- t͡s\k{á}:- \'{ɔ}mdé -t'\`{ɔ}:\\
\uline{{\strut}\ref{VM2}} \ref{VNegation} {\ref{Vpro}}- \ref{VIncorpV}- \ref{VStem} -\ref{VMod}\\
\uline{{\strut}maybe} {\Neg} [(\First\Sg/\Aarg):\Second,\Third\Sg/\Parg:\Pl/\Obj]- go- become -\Fut\\
\trans `You might not be able to get there.' \citep[][221]{Watkins:1984}% 36b p. 221
\z

\noindent The third position immediately precedes an overt Agent DP in Zones \ref{VM3} and \ref{VTA3} as in (\ref{ModalPosition3}) and (\ref{TAPosition3}). 

\ea Modal Particle in Zone \ref{VM3} \label{ModalPosition3} \\
\glll \uline{{\strut}dá-\`al} ám jí:dè k\^{ɔ}l p\k{í}:gjá gját- b\'{ɔ}:\\
\uline{{\strut}{\ref{VM3}}} {\ref{VAgent}} {\ref{VAgent}} {\ref{VDirectObject}} {\ref{VDirectObject}} {\ref{Vpro}}- {\ref{VStem}.\ref{VAsp}}\\
\uline{{\strut}must-also} you both some food [(\X/\Aarg):\First\Pl/\Parg:\Pl/\Obj]- bring.\Ipfv\\
\trans `You (dual) must also bring some food for us.' \citep[][222]{Watkins:1984}% 39a p. 222
\z

\ea Tense/Aspect Particle in Zone \ref{VTA3} \label{TAPosition3} \\
\glll h\'{ɔ}n \uline{{\strut}\`an} t͡sój gj\`a- th\k{ó} -m\^{ɔ}:\\
{\ref{VNegation}} \uline{{\strut}{\ref{VTA3}}} {\ref{VDirectObject}} {\ref{Vpro}}- {\ref{VStem}} -\ref{VNegativeSuffix}\\
{\Neg} \uline{{\strut}{\Hab}} coffee [\First\Sg/\Aarg:\Sg/\Obj]- drink -\Neg\\
\trans `I never drink coffee.' \citep[][223]{Watkins:1984} % 20d p. 215
\z

\noindent In the fourth position, both zones (\ref{VM4} and \ref{VTA4}) precede an overt Patient DP as in (\ref{ModalPosition4}) and in (\ref{TAPosition4}).

\ea Modal Particle in Zone \ref{VM4} \label{ModalPosition4} \\
\glll \uline{{\strut}béth\`{ɔ}:} ám èm- d\k{\'ɔ} -m{ê}:\\
\uline{{\strut}{\ref{VM4}}} {\ref{VPatient}} {\ref{Vpro}}- {\ref{VStem}} -{\ref{VHearsay}}\\
\uline{{\strut}unknowing} you [\Second/\Sg]- be -{\Hsy}\\
\trans `I didn't know it was you (standing behind the door).' \citep[][223]{Watkins:1984} % 42b p. 223
\z

\ea Tense/Aspect Particle in Zone \ref{VTA4} \label{TAPosition4} \\
\glll \uline{{\strut}\`an} t'ól $\varnothing$- sô: -j\`a\\
\uline{{\strut}{\ref{VTA4}}} {\ref{VPatient}} {\ref{Vpro}}- {\ref{VStem}} -{\ref{VAsp}} \\
\uline{{\strut}{\Hab}} snow [\Third\Sg]- descent -{\Ipfv}\\
\trans `...it snows.' \citep[adapted from][218]{Watkins:1984}\footnote{This clause originally appears in a subordinate clause in \citet{Watkins:1984} in the sentence `When it gets really cold here, it snows.'} % 30c p. 218
\z

\noindent In the fifth position, both zones (\ref{VM5} and \ref{VTA5}) precede a Direct Object DP as in (\ref{ModalPosition5}) and (\ref{TAPosition5}).

\ea Modal Particle in Zone \ref{VM5} \label{ModalPosition5} \\
\glll \uline{{\strut}hét} \uline{{\strut}h\`agj\`a} \k{é}:dè kút ján- hájdé -t'\`{ɔ}\\
\uline{{\strut}{\ref{VM5}}$_{1}$} \uline{{\strut}{\ref{VM5}}$_{2}$} {\ref{VDirectObject}} {\ref{VDirectObject}} {\ref{Vpro}}- {\ref{VStem}} -{\ref{VMod}}\\
\uline{{\strut}let's} \uline{{\strut}maybe} this letter [(\First\Sg/\Aarg):\Second,\Third\Sg/\Parg:\Pl/\Obj]- learn -{\Fut}\\
\trans `Let's see if maybe you can understand this letter.' \citep[][222]{Watkins:1984} % 41b p. 222
\z

\ea Tense/Aspect Particle in Zone \ref{VTA5} \label{TAPosition5} \\
\glll h\'{ɔ} kôl \uline{{\strut}s\'{ɔ}t} kút ján- gút \\
{\ref{VQuestionParticles}} {\ref{VAdverbs1}} \uline{{\strut}{\ref{VTA5}}} {\ref{Vpro}}- {\ref{VStem}.\ref{VAsp}}\\
{\Q} some \uline{{\strut}just} letter [(\First\Sg/\Aarg):\Second\Sg/\Parg:\Pl/\Obj]- write.{\Pfv}\\
\trans `Did I recently write you a letter?' % 26b p. 217  
\z

As mentioned earlier, the sixth position for both zones (\ref{VM6} and \ref{VTA6}) is assumed in the planar structure above. It is the last logically possible position for both zones prior to the verb complex (i.e. after any overt DPs and noun-locatives but before the verb complex), even though I have yet to find unambiguous evidence that either zone occurs in this location. Given clear confirmation of the other five locations within the planar structure, however, I will assume that both zones may occur in this position until data suggests otherwise.\footnote{Note there is no evidence to suggest modal or tense/aspect particles can occur between a DP and a noun-locative. This is assumed not to be the case, as it is not observed in the present data.} 

\subsection{The verb complex} 
\label{sec:verbcomplex}

The verb complex, as previously discussed in \sectref{sec:grammatical}, consists of a pronominal (Slot \ref{Vpro}), the stem (Slot \ref{VStem}), and an aspect marker (Slot \ref{VAsp}). Syntactic markers occur after inflections as in (\ref{verblinear}), repeated below as (\ref{verblinear2}) \citep{Watkins:1984}.

\ea \label{verblinear2}
{\Pronom} - ({\Adv}) - ({\N}) - ({\V}) - {\Stem} - {\Asp} - ({\Neg}) - ({\Mods}) - ({\Hsy}) - (\Synt) 
\z 

These syntactic markers include nominal, locative, switch-reference, and other subordinating conjunctions (see \ref{syntsuff} again for the full list). At closer inspection, however, it seems that it is too simplistic to treat them identically and in the same position. As expected, the nominalizing/relativizing suffix (/-dè/ `basic' or /-g\`{ɔ}/ `inverse' depending on the head noun) occurs at the end of the verb complex in Slot \ref{VNominal} as in (\ref{nominalizer}).

\ea \label{nominalizer}
\glll {\{}p\k{í}á:d\`{ɔ} è- ét -\uline{{\strut}g\`{ɔ}}{\}} dé- h\'{ɔ}: -gj\`a\\
{\{}\ref{VDirectObject} {\ref{Vpro}}- \ref{VStem} -\uline{{\strut}\ref{VNominal}}{\}}\textsubscript{\ref{VDirectObject}} {\ref{Vpro}}- \ref{VStem} -\ref{VAsp}\\
{\{}{table.\Inv} [\Third\Inv]- {big.\Sg} -\uline{{\strut}{\Nom.\Inv}}{\}} [\First\Sg:\Aarg:\Inv/\Obj]- get -{\Pfv}\\
\trans `I bought a big table/table that is big.' \citep[][230]{Watkins:1984} %58c p.230
\z 

\noindent Relative clauses may also be accompanied by locative suffixes just like the noun-locative adverbials in Slots \ref{VNL1}, \ref{VNL2}, \ref{VNL3}, \ref{VNL4}, and \ref{VNL5}. As seen below in (\ref{locative}), the locative suffix /-òj/ `at/generally' occurs immediately following the nominalizing suffix /-dé/ in Slot \ref{VNominal}.

\ea \label{locative}
\glll {\{}hèg\'{ɔ} m\'{ɔ}n mîn \k{é}- p'\'{ɔ}jdép -dé{\}} -\uline{{\strut}òj} ján- gút \\
{\{}\ref{VClauseIntroducers} \ref{VM1} \ref{VTA1} {\ref{Vpro}}- \ref{VStem} -\ref{VNominal}{\}}\textsubscript{\ref{VNL1}} -\uline{{\strut}\ref{VNL1}} {\ref{Vpro}}- {\ref{VStem}.\ref{VAsp}}\\
{\{}now probably about.to [(\Second,\Third\Sg/\Aarg):\Second\Sg/\Parg:$\varnothing$/\Obj]- {forget.\Ipfv} -{\Nom}{\}} -\uline{{\strut}at.generally} [(\First\Sg/\Aarg):\Second\Sg/\Parg:\Pl/\Obj]- write.\Pfv\\
\trans `You were probably about to forget me around the time I wrote you.' \citep[][235]{Watkins:1984}\footnote{The relative clause structure is not immediately clear in \citet{Watkins:1984}'s translation. An alternative translation is `At the time of your probable forgetting me, I wrote you.'} %66b p.235
\z 

As with adverbials, focus and new/old information can lead to dislocation of relative clauses. In cases of contrastive focus, the contrasted relative clause moves to the left and is the first element of the clause. As seen in (\ref{contrastivefocus}), the second person singular \textit{ám} is left-dislocated to precede the Question Particle \textit{h\'{ɔ}} in the second clause.

\ea \label{contrastivefocus}
\glll gját- hájgjá- d\`{ɔ}: {...} n\`{ɔ} \uline{{\strut}ám} h\'{ɔ} ján- hájgjá- d\`{ɔ}:\\
[(\X/\Aarg):\First\Pl/\Parg:\Pl/\Obj]- {learn.\Det}- be {...} {and/\Diff} \uline{{\strut}you} {\Q} [(\First\Sg/\Aarg):\Second\Sg/\Parg:\Pl/\Obj]- {learn.\Det}- be \\
{\ref{Vpro}}- \ref{VIncorpV}- \ref{VStem} {...} \ref{VSubordinate} \uline{{\strut}\ref{VPatient}} \ref{VQuestionParticles} {\ref{Vpro}}- \ref{VIncorpV}- \ref{VStem} \\
\trans `We know... do \textit{you} know?' \citep[][212]{Watkins:1984} %17a
\z 

\noindent As mentioned earlier, in questions where the questioned element is a relative clause, the full relative clause moves to the left and is the first element of the clause (\ref{RCleft} is repeated here as \ref{questionRC}).\footnote{As mentioned earlier, the subordinating marker at the beginning is not provided a position in the planar structure. It is understood to be part of the RC, which itself fills a slot in the matrix clause.}

\ea \label{questionRC}
\glll {\{}\'{ɔ}g\`{ɔ} k'j\k{á}:h\k{î}: {$\varnothing$}- p\k{ó}:- t͡sán -dè{\}} h\'{ɔ} Lawton-g\`u {$\varnothing$}- bá: \\
{\{}{\Sub} \ref{VAgent} {\ref{Vpro}}- \ref{VIncorpV}- \ref{VStem}.\ref{VAsp} -\ref{VNominal}{\}}\textsubscript{\ref{VAgent}} \ref{VQuestionParticles} \ref{VNL1} {\ref{Vpro}}- \ref{VStem}.\ref{VAsp}\\
{\{}{\Sub} man [\Third\Sg]- see- {arrive.\Pfv} -{\Nom}{\}} {\Q} Lawton-to [\Third/\Sg]- {go.\Pfv}\\
\trans `Did the man who came to see you go to Lawton?' \citep[][212]{Watkins:1984} %17b
\z 

The remaining syntactic markers are morphemes that may be used in subordination or coordination structures. There are three pairs of switch-reference markers (\ref{SR}) and three subordinate markers (\ref{conjunctions}). Switch-reference markers in Kiowa are most often ambiguous as to whether they are being used in a subordinate or coordinate structure, but the difference does not seem to affect speaker intuitions. \citet{Watkins:1984} mentions that Kiowa linguist Parker McKenzie could not easily decide if switch-reference markers cohered to the preceding word as suffixes or clitics or if they were independent particles in the clause. She observed that he typically cliticized the switch-reference markers to the preceding word when clearly part of a subordinate clause instead of a coordinate construction (endnote 11, p. 245). \citet{McKenzie:2012,mckenzie:2015} posits that switch-reference markers are pronominal heads in Kiowa, as opposed to grouping with T or C in traditional syntactic analyses. In my experience, I have found Kiowa speakers to even vary in the prosodification of switch-reference markers. Sometimes they cohere to the left, and sometimes they cohere to the following clause/pause group. 

\ea Switch-Reference Markers \citep[][236]{Watkins:1984}\label{SR} \\
\begin{tabular}{lll}
     \uline{{\strut}Same} & \uline{{\strut}Different} &  \\
     g\`{ɔ} & n\`{ɔ} & `and, if' (neutral, sequential, conditional) \\
     t͡s\k{è}: & \k{è}: & `when, while' (simultaneous) \\
     k'\`{ɔ}t & \`{ɔ}t & `yet, anyway' (contrary to expectation) \\
\end{tabular}
\z 

\ea Subordinate Morphemes \label{conjunctions} \\
\begin{tabular}{ll}
    -ál &  `although, even though'\\
    né & `but' \\
    -dò & `because' (with clause initial particle \textit{bót})\\
\end{tabular}
\z 

I will assume there is a verb-complex final position for Subordinate Markers (switch-reference and other subordinate suffixes). Research is split between a flat or compositional analysis of coordinate structures (see \citealt{wagner:2010} for an overview of the discussion). In a flat structure, the coordinating head projects to a new structure and therefore is defined by occurring outside of the clause (joining the two together with no clear head). A compositional structure is more obviously similar to subordinate constructions (a clause within a clause). \citet{bickel:2010} argues that cross-linguistic variation blurs the line between coordination and subordination, suggesting a more continuum-like understanding of clause-linkage. For the present analysis, I remain as agnostic as possible. I adopt a flat structure and leave the coordinating switch-reference markers out of the planar structure pending future research. 

\section{Morphosyntactic diagnostics} 
\label{sec:morphosyntactic}

This section provides an overview of the results of five morphosyntactic constituency diagnostics applied to \tabref{tab:verbalplanarkiowa}: Free Occurrence (Minimal and Maximal), Non-interruptability (Free Simplex and Free Complex), Non-permutability (Rigid and Flexible), Subspan Repetition (Minimal and Maximal), and Ciscategorial Selection. Note that most tests are fractured into two sub-tests corresponding to different interpretations of the overarching test (cf. \citealt{Tallman:ur}). A contiguous subspan of planar positions is considered a candidate for wordhood if two or more diagnostics converge to identify it. Interested readers are directed to \citet{Tallman:ur} or to the introduction of this volume for more information on each test. Overall, eight subspans are identified using morphosyntactic information.

\subsection{Free occurrence (\ref{Vpro}-\ref{VAsp}; \ref{Vpro}-\ref{VSubordinate})} \label{sec:free}

{\textsc{Free Occurrence}} identifies a subspan of the planar structure that may be uttered as a minimal free form. That is, the subspan may form its own utterance or be a grammatical sentence-fragment answer to a question (e.g. Q: \textit{What did the children do?} A: \textit{Play}). This test may be fractured to two sub-tests: minimal and maximal. The {\textsc{Minimal Free Occurrence}} is the smallest subspan whose elements can be uttered as a free form. In Kiowa, the smallest possible verb complex consists of a pronominal (Positions \ref{Vpro}), stem (Position \ref{VStem}), and aspectual marking (Position \ref{VAsp}) as in (\ref{minimalfree}). Incorporated elements can intervene and by definition are included in the identified subspan (Positions \ref{Vpro}-\ref{VAsp}).

\ea \label{minimalfree}
\glll gját- gút -kjá\\
{\ref{Vpro}}- \ref{VStem} -\ref{VAsp}\\
[\First\Sg/\Aarg:\Pl/\Obj]- write -{\Pfv}\\
\trans `I wrote it/it was written.' \citep[][85]{Miller:2018}
\z

\noindent Recall, however, that both the pronominal and aspectual marker can be a zero morpheme as in (\ref{minimalfree2}). If those were not actively present in the interpretation and agreement within the clause, one could argue it is only the verb stem itself that is required (Position \ref{VStem}-\ref{VStem}). As both have semantic interpretations playing a role in the clause, and there are multiple forms of morphemes like the perfective (See \sectref{biuniqueness} for further discussion), I assume that they are indeed present. Future research may suggest a better analysis, though.

\ea \label{minimalfree2}
\glll $\varnothing$- tʰép -$\varnothing$\\
{\ref{Vpro}}- \ref{VStem} -\ref{VAsp}\\
[\Third\Sg]- go.out -{\Pfv}\\
\trans `He went out.' \citep[adapted from][44]{Miller:2018}
\z

\noindent The {\textsc{Maximal Free Occurrence}} is the largest subspan whose elements may be uttered as a free form. Since there are additional suffixes and verb endings possible, the largest subspan that forms a minimal free form consists of the maximal verb complex. It spans from the pronominal through subordinate markers. Thus, the {\textsc{Minimal Free Occurrence}} in Kiowa is Positions \ref{Vpro}-\ref{VAsp}. The {\textsc{Maximal Free Occurrence}} is Positions \ref{Vpro}-\ref{VSubordinate}.

\subsection{Non-interruptability (\ref{VStem}-\ref{VSubordinate}; \ref{VM6}-\ref{VSubordinate})} \label{sec:non-interruptability}

{\textsc{Non-interruptability}} identifies a subspan of the planar structure that cannot be interrupted. Again, this test may be fractured into two sub-tests: simplex and complex. {\textsc{Non-interruptability (Simplex)}} identifies the subspan that cannot be interrupted by any free form (e.g. any morpheme, particle, phrase, etc.). As bare stems are possible free forms, incorporated elements are ruled out. This subspan is therefore much more restricted and includes only the Verb Stem (Position \ref{VStem}) through the subordinate markers (Position \ref{VSubordinate}). 

{\textsc{Non-interruptability (Complex)}} identifies a subspan that cannot be interrupted by anything larger than a free form (e.g. a phrase). In Kiowa, this means examining where full DPs may occur/interrupt elements. It is also reasonable to assume Noun-Locative Adverbials form some type of adjuncted phrase themselves. Whatever that phrase is (i.e. Adverbial Phrase or a subset of DPs) is left to future research. Thus, the subspan that does not involve a full phrase intervening at any point is from the Modal and Tense/Aspect Particle zones immediately preceding the pronominal (Positions \ref{VM6} and \ref{VTA6}) through to the subordinate markers (Position \ref{VSubordinate}) before any post-verbal adverbials. Thus, The {\textsc{Non-interruptability (Simplex)}} subspan is Positions \ref{VStem}-\ref{VSubordinate}. The {\textsc{Non-interruptability (Complex)}} subspan is Positions \ref{VM6}-\ref{VSubordinate}.

\subsection{Non-permutability (\ref{Vpro}-\ref{VNegativeSuffix}; \ref{Vpro}-\ref{VNominal})} \label{sec:non-permutability}

{\textsc{Non-permutability}} identifies subspans of elements which cannot be variably ordered. This test is fractured into two sub-tests: rigid and flexible. A subspan demonstrates {\textsc{Rigid Non-permutability}} if its elements always occur in a fixed order with respect to one another. The majority of the verb complex is rigidly ordered in Kiowa and does not allow for any other orders from the pronominal (Position \ref{Vpro}) to through the Hearsay suffix (Position \ref{VNegativeSuffix}). As discussed earlier, Adverbials occur in different positions to indicate differences in discourse factors like new vs. old information. Prior to the verb complex, Modal and Tense/Aspect Particles are the most freely ordered elements in the clause and thus ruled out. To the right, it is possible to reorder due to scope differences.

A subspan demonstrates {\textsc{Flexible Non-permutablity}} if its elements are rigid\-ly ordered but may re-order with respect to one another to condition differences in scope. Relative clauses may left-dislocate and move out of the scope of Negation in Position \ref{VNegation}. Other variable orders (e.g. adverbials) are due to non-scope discourse factors like new versus old information and are thus disregarded here. The subspan identified by this sub-test is the minimal relative clause, or the verb complex from the pronominal (Position \ref{Vpro}) through to the nominalizer suffix (\ref{VNominal}). Thus, {\textsc{Rigid Non-permutability}} and {\textsc{Flexible Non-permutablity}} identify the subspans Positions \ref{Vpro}-\ref{VNegativeSuffix} and Positions \ref{Vpro}-\ref{VNominal}, respectively.

\subsection{Subspan repetition (\ref{VLeftRC}-\ref{VDislocatedDP})} \label{sec:subspanrep}

{\textsc{Subspan Repetition}} identifies subspans of the verbal planar structure that are repeated in specific constructions (e.g. compounds, serial verbs, reduplication, coordination, subordination, etc.). In Kiowa, we may test for this in coordination and/or subordination constructions. As mentioned earlier, it is almost always ambiguous in Kiowa if a given structure is truly coordinating or subordinating \citep[][]{Watkins:1984}. There is a difference between coordination and subordination when it comes to the placement of switch-reference markers, though, for some speakers. In a truly subordinate structure, the switch-reference marker may cliticize to the right-edge of the verb complex. Otherwise, they act as independent particles between clauses. Let us focus only on the instances where subordinate markers are attached to the complex. Specifically, consider the subordinating marker /-\`al/ `although, even though' which is always found verb complex-finally. 

As seen below, full clauses may be repeated in the construction. In (\ref{repeated}), negation (Position \ref{VNegation}) and the modal particle /\`an/ `habitual' (in any post-negation position (i.e. Positions \ref{VM3}, \ref{VM4}, \ref{VM5}, or \ref{VM6}) are permitted. In (\ref{repeated2}) a pre-verbal adverbial is permitted in the subordinate clause (Position \ref{VAdverbs1}). In both cases, the second clause has been marked with braces.

\ea \label{repeated}
\glll \`a- d\k{è}:- k'\'{ɔ}: -\`al {\{}h\'{ɔ}n \`an \`a- d\k{è}:- h\k{é}:m -\^{ɔ}:{\}}\\
{\ref{Vpro}}- \ref{VIncorpV} \ref{VStem} -\ref{VSubordinate} {\{}\ref{VNegation} \ref{VM3} {\ref{Vpro}}- \ref{VIncorpV}- \ref{VStem} -\ref{VNegativeSuffix}{\}}\\
[\First\Sg]- sleep- be.lying -although {\{}{\Neg} {\Hab} [\First\Sg]- sleep- die -{\Neg}{\}} \\
\trans `Although I lie down, I can't fall asleep.' \citep[][242]{Watkins:1984}%83a p. 242
\z 

\ea \label{repeated2}
\glll bîndè gj\`at- p\'{ɔ}:l -î: -t'\`{ɔ}: -\`al {\{}bòtʰêndè \`a- tón- â: -j\`i: -t'\`{ɔ}:{\}}\\
\ref{VAdverbs1} {\ref{Vpro}}- \ref{VStem} -\ref{VAsp} -\ref{VMod} -\ref{VSubordinate} {\{}\ref{VAdverbs1} {\ref{Vpro}}- \ref{VIncorpN}- \ref{VStem} -\ref{VAsp} -\ref{VMod}{\}}\\
much [\First\Sg/\Aarg:\Pl/\Obj]- eat -{\Ipfv} -{\Fut} -although {\{}unlikely [\First\Sg]- fat- grow -{\Ipfv} -{\Fut}{\}} \\
\trans `Even if I should eat a lot, I can't/don't get fat.' \citep[][242]{Watkins:1984} %83b p.242
\z 

I have yet to find an example which includes the earliest positions of the planar structure (i.e. Question Particles or Clause Introducers) in previous work or in my own corpus of Kiowa data. There is no reason, however, to think that the coordinated/subordinated clauses cannot span the entire planar structure. Unless future analysis suggests otherwise, then, I assume that the entire Kiowa verbal planar structure is the {\textsc{Repeated Subspan}}. Additionally, I have found no data showing an element can take wide-scope of a coordinated conjunct. Therefore, there is not a need to fracture this test in Kiowa at this time. 

\subsection{Ciscategorial selection (\ref{VStem}-\ref{VHearsay})} \label{sec:ciscategorial}

{\textsc{Ciscategorial Selection}} identifies a subspan where all the elements are modifiers or dependents of a particular syntactic category (i.e. are ciscategorial). This can be fractured two ways: minimal and maximal. A subspan is {\textsc{Minimally Ciscategorial}} if all elements in the subspan are ciscategorial (only pertaining to the verb in this case). A subspan is {\textsc{Maximally Ciscategorial}} if all elements outside of this span are transcategorial (may occur with more than one category or at least in non-verbal constructions). For Kiowa, both sub-tests identify the same subspan. Since incorporated stems are bare and not restricted to verbal predicates, they are ruled out. The subspan identified is from the Verb Stem (Position \ref{VStem}) through the hearsay suffix (Position \ref{VHearsay}). Incorporated elements are added to modify the understanding of the verbal predicate, but they are not strictly modifiers or dependent on the verb. The same suffixes used as nominalizers to mark relative clauses are used to mark number on nouns more generally. Thus, the subspan identified by {\textsc{Ciscategorial Selection}} is Position \ref{VStem}-\ref{VHearsay}.

\section{Phonological domains} \label{sec:phonological}

This section provides an overview of the results of the phonological domains identified in \tabref{tab:verbalplanarkiowa}: Syllabification (Minimal and Maximal), Cluster Devoicing (Minimal and Maximal), Vowel-Truncation (Minimal and Maximal), Dental-Velar Switch (Minimal and Maximal), Tone Lowering (Minimal and Maximal), and Pausing. As with morphosyntactic diagnostics, a contiguous subspan in \tabref{tab:verbalplanarkiowa} is considered a candidate for wordhood if two or more diagnostics converge to identify it. Overall, nine phonological domains are identified. %Be sure to check this.

\subsection{Segmental domains} \label{sec:segmental}

Let us first consider the processes which result in changed segmental forms within a particular subspan. In Kiowa, there are seven such processes: two syl\-lab\-le-sensitive phenomena (Syllable-Final Devoicing and Closed Syllable Shortening), Cluster Devoicing, Glide Formation, Glide Deletion, Vowel Truncation, and the Dental-Velar Switch. In all cases, the phonological diagnostics will be fractured to form minimal and maximal subspans. A minimal subspan is that which there is positive evidence that the process in question applies. A maximal subspan is that which there is no counterevidence against the process in question applying across that subspan.

\subsubsection{Syllabification and sensitive processes (\ref{VStem}-\ref{VSubordinate}; \ref{VIncorpAdv}-\ref{VSubordinate})} \label{sec:syllab}

Syllabification in Kiowa is characterized by two phonological processes: Syllable-Final Devoicing (devoicing syllable-final obstruents) and Closed-Syllable Shortening (shortening underlying long vowels in closed syllables). \citet{Miller:2018} identified the domains for syllabification within the verb complex and the larger clause. Syllabification spans the junctures between the verb stem (Position \ref{VStem}) and suffixes: aspect, negative, modality, and hearsay.

The data in (\ref{syll:stemasp}) shows syllabification spanning the juncture between the verb stem /t͡s\k{â}:/ `arrive', the aspectual suffix /-n/ `imperfective', and the imperative (modality) suffix /-\`i:/. The underlying long vowel in the stem does not need to shorten because /n/ can form the onset of the syllable with the imperative suffix.

\ea \label{syll:stemasp}
\glll pá:tʰ\k{\`a}:-tʰ\`{ɔ}p $\varnothing$- tʰó:gjáj -t'\`{ɔ}: -\k{è}: \`a- \textbf{t͡s\k{â}:} -\textbf{n} -\textbf{\`i:} -t'\`{ɔ}:\\
\ref{VNL1} {\ref{Vpro}}- \ref{VStem} -\ref{VMod} -\ref{VSubordinate} {\ref{Vpro}}- \ref{VStem} -\ref{VAsp} -\ref{VMod}$_{1}$ -\ref{VMod}$_{2}$\\
eleven-beyond [\Third\Sg]- pass -{\Fut} -{\When.\Diff} [\First\Sg]- arrive -{\Ipfv} -{\Imp} -{\Fut}\\
\trans `I'll be coming (regularly) at eleven.' \citep[][173]{Watkins:1984}%178b p. 173
\z 

\noindent In (\ref{syll:stemneg}), syllabification spans the juncture between the verb stem and the negative suffix. Again, the underlying long vowel of the stem need not shorten because stem-final /d/ may form the onset of the syllable with the negative suffix. Compare to the same stem when not suffixed in (\ref{syll:stemdev}). Because syllabification must end, the underlying long vowel shortens and the final /d/ devoices and surfaces as [t]. 

\ea \label{syll:stemneg}
\glll h\'{ɔ}n \`an p\k{í}:gjá gj\`a- \textbf{tó:d} -\textbf{\^{ɔ}} (*gj\`a-tót-\^{ɔ})\\
\ref{VNegation} \ref{VM3} \ref{VDirectObject} {\ref{Vpro}}- \ref{VStem} -\ref{VNegativeSuffix}\\
{\Neg} {\Hab} food [\Third\Sg/\Aarg:\Sg/\Obj]- send -{\Neg}\\
\trans `They do not send the food.' \citep[][83]{Miller:2018}\footnote{The stem in (\ref{syll:stemneg}) and (\ref{syll:stemdev}) is incorrectly transcribed as low in \citet{Miller:2018}. This has been corrected here.} %96 p. 83 in diss.
\z 

\ea \label{syll:stemdev}
\glll p\k{í}gjá gj\`a- \textbf{tót}\\
\ref{VDirectObject} {\ref{Vpro}}- \ref{VStem}.\ref{VAsp}\\
food [\Third\Sg/\Aarg:\Sg/\Obj]- send.{\Pfv}\\
\trans `They sent the food.' \citep[][83]{Miller:2018} %96 p. 83 in diss.
\z 

\noindent Syllabification also spans the Stem-Hearsay juncture in (\ref{syll:stemhsy}) below. Just like above, the underlying long vowel surfaces unchanged and stem-final /n/ syllabifies as the onset of the syllable with the hearsay suffix /-ê/.

\ea \label{syll:stemhsy}
\glll èm- \textbf{g\k{ú}:n} -\textbf{ê} (*èm-g\k{ú}n-ê) \\
{\ref{Vpro}}- \ref{VStem}.\ref{VAsp} -\ref{VHearsay}\\
[\Third\Sg/\Refl]- dance.{\Ipfv} -{\Hsy}\\
\trans`I heard they were dancing.' \citep[][93]{Miller:2018} %dance example from ET
\z 

It is impossible to determine if syllabification spans the junctures across to the nominalizer or locative suffixes in the verb complex. Because nominalizer suffixes are consonant-initial and thus have onsets (e.g. /-dè/ and /-g\`{ɔ}/), any preceding syllable will be self-contained and thus untestable. Even though there are vowel-initial locative suffixes (e.g. /-èm/ `here, away'), they only co-occur with a nominalized relative clause. A nominalizer suffix is always short vowel-final and thus also irrelevant for both diagnostics. Subordinate markers are the only complex-final element that \textit{can} be tested, but I have yet to find the relevant environments to conduct the test (e.g. obstruent-final preceding morpheme before a vowel-initial subordinate marker like /-\k{è}:/ `when, different' or /-\`al/ `although'). Until there is such evidence, the subspan up to and including subordinate markers are included.

Finally, syllabification is restricted to the pronominals and blocked from spanning across the rest of the verb complex. In (\ref{syll:pro}), for example, the final obstruent /d/ in the pronominal /b-i\`a-ia-d/ devoices to [t] rather than syllabifying as the onset of the following syllable. 

\ea \label{syll:pro}
\glll \textbf{b\`at}- \^{ɔ}m (*b\`ad-\^{ɔ}m)\\
{\ref{Vpro}}- \ref{VStem}.\ref{VAsp}\\
[\Second\Sg/\Aarg:\Pl/\Obj]- do.{\Ipfv} \\
\trans `You make it.' \citep[][82]{Miller:2018}
\z 

As for incorporated elements, Watkins includes discussions of /d/-final noun roots that devoice and also undergo Closed Syllable Shortening (e.g. /t͡sá:d/ `door'). I have found no evidence of any relevant alternations in my own work, though, so these are set aside. Similarly, any obstruent-final adverbs already end in a voiceless sound (e.g. /kòét/ `fearfully'). Therefore, the only possible test is an incorporated verb that is consonant- or obstruent-final so that syllabification may be confirmed. I have yet to find such an example. Through other phonological diagnostics, though, we will confirm that incorporated elements form individual phonological domains.

Therefore, the \textsc{Minimal Syllabification} domain is Slots \ref{VStem}-\ref{VSubordinate} (Stem to the subordinate marker). Given that there is only one possible test (an incorporated verb that is consonant- or obstruent-final), and it is left to future research to find such an example, we must conclude the \textsc{Maximal Syllabification} domain is Slots \ref{VIncorpAdv}-\ref{VSubordinate} (Incorporated elements through the subordinate marker). Stems tend to cross-linguistically form individual phonological words and thus are expected to form separate domains from the rest of the verb complex (see \citealt{Miller:2018} and the discussion therein). Thus, I suspect future research will rule this out. Without such evidence though, I include the identified maximal subspan. 

\subsubsection{Cluster devoicing (\ref{VStem}-\ref{VNegativeSuffix}; \ref{VStem}-\ref{VHearsay}) } \label{sec:cluster}

Cluster Devoicing is an assimilation process which devoices stops after a voiceless obstruent. As seen in below, the process applies across the Stem-Aspect boundary. In (\ref{cluster:stemasp}), the initial /g/ of the perfective suffix devoices after the final [t] in `write.'\footnote{The underlying form of `write' is /gú:l/. It first undergoes Lateral Obstruentization (l {$\rightarrow$} d) before the initial obstruent of perfective /-gjá/. Then the final /d/ devoices via Syllable-Final Devoicing thereby triggering Cluster Devoicing of the /g/ in /-gjá/.}

\ea \label{cluster:stemasp}
\glll gját- \textbf{gút} -\textbf{kjá} (*gját-gút-gjá) \\
{\ref{Vpro}}- \ref{VStem} -\ref{VAsp} \\
[\First\Sg/\Aarg:\Pl/\Obj]- write -{\Pfv} \\
\trans `I wrote it/it was written.' \citep[][85]{Miller:2018} %101 p. 85 
\z 

Cluster Devoicing also applies across the Stem-Negative boundary. \citet[177]{Watkins:1984} lists the negative form for `be lying pl.' as [kóp-k\^{ɔ}] (cf. \textit{k'úl $\thicksim$ kóp} `be lying pl'). It is impossible to test whether the process applies across the Stem-Modality juncture, as no modality suffix begins with a voiced stop. The stative and modal hearsay form /-dê:/ provides the necessary environment to test across the Stem-Hearsay juncture (i.e. after a stative verb ending in a voiceless obstruent), but I have yet to find such an example.

There is clear evidence, however, that Cluster Devoicing is blocked from applying across the Stem-Nominalizer juncture in (\ref{cluster:stemrc}). The nominalizer suffix /-g\`{ɔ}/ surfaces unchanged after a /t/-final verb stem. All identified locative suffixes are vowel-initial and thus irrelevant for this test. There is a potential test for subordinate markers (e.g. when /-g\`{ɔ}/ follows a [t]). There is no such example in the current corpus, though, leaving this to future research.

\ea \label{cluster:stemrc}
\glll {\{}p\k{í}á:d\`{ɔ} è- \textbf{ét} -\textbf{g\`{ɔ}}{\}} dé- h\'{ɔ}: -gjá (*... è-ét-k\`{ɔ} ...)\\
{\{}\ref{VPatient} {\ref{Vpro}}- \ref{VStem} -\ref{VNominal}{\}}\textsubscript{\ref{VDirectObject}} {\ref{Vpro}}- \ref{VStem} -\ref{VAsp} \\
{\{}table.{\Inv} [\Third\Inv]- be.big -\Nom{\}} [\First\Sg/\Aarg:\Inv/\Obj]- get -{\Pfv} \\
\trans `I bought a big table/table that is big.' \citep[][230]{Watkins:1984}  %58c p. 230 
\z 

Cluster Devoicing does not apply prior to the stem in the verb complex. As seen in (\ref{cluster:pro}), the process does not apply across a pronominal's juncture. The stem /gú:l/ `write' surfaces unchanged after [t]. Similarly, the process is blocked across an incorporated element's juncture. In (\ref{cluster:incorp}), the final [t] of the incorporated adverb /k\`{ɔ}ét/ `scared' does not trigger the devoicing of /b/ in /bá:/ `go'.

\ea \label{cluster:pro}
\glll \textbf{gját}- \textbf{gúl} -t\`{ɔ} (*gját-kúl-t\`{ɔ})\\
{\ref{Vpro}}- \ref{VStem} -\ref{VMod} \\
[\First\Sg/\Aarg:\Pl/\Obj]- write -{\Fut} \\
\trans `I will write.' \citep[][85]{Miller:2018}
\z

\ea \label{cluster:incorp}
\glll \`a- \textbf{k\`{ɔ}ét}- \textbf{bá:} (*\`a-k\`{ɔ}ét-pá:) \\
{\ref{Vpro}}- \ref{VStem}.\ref{VAsp} \\
[\First\Sg]- scared- go.{\Pfv} \\
\trans `I fearfully went.' \citep[][85]{Miller:2018}
\z

When fractured into minimal and maximal sub-tests, Cluster Devoicing identifies two subspans. The \textsc{Minimal Cluster Devoicing} domain is from the stem to the negative suffix (Slots \ref{VStem}-\ref{VNegativeSuffix}). The \textsc{Maximal Cluster Devoicing} domain spans from the stem through the hearsay suffix where there is clear evidence that the process is blocked across to the nominalizer (Slots \ref{VStem}-\ref{VHearsay}).

\subsubsection{Vowel truncation (\ref{VStem}-\ref{VAsp})} \label{sec:VT}

In vowel hiatus, the first vowel deletes via Vowel Truncation (a vowel is considered any monophthong, diphthong, or /ia/ sequence). The process applies across the Verb Stem-Aspect juncture. In (\ref{VT:StemAsp}), the verb root forms a derived intransitive (considered together the Verb Stem here) and combines with the perfective suffix -iá. Closed-Syllable Shortening, Vowel Truncation, and Glide Formation apply and yield the surface form [tʰémgjá]. This surface form is observed in (\ref{VT:aspect}).\footnote{The underlying form of the verb root is /tʰê:m/ `break' with a falling tone, but it changes to a high tone via detransitivization.}


\ea Derivation of /tʰê:m-gé-iá/ {`break-{\Itrd}-{\Pfv}'} \label{VT:StemAsp} \\
\begin{tabular}{ll}
    {/tʰê:m-gé-iá/} &  \\
    tʰémgeiá & Closed-Syllable Shortening \\
    tʰémgiá& Vowel Truncation \\
    tʰémgjá & Glide Formation \\
    {[tʰémgjá]} & \\
\end{tabular} 
\z 

\ea \label{VT:aspect}
\glll è- tʰémgjá\\
{\ref{Vpro}}- \ref{VStem}.\ref{VAsp}\\
[\Third\Sg/\Aarg:\Inv/\Obj]- {break.{\Intr}.{\Pfv}} \\
\trans `It's broken.' \citep[adapted from][91]{Miller:2018} %124 p. 91 diss
\z 

Vowel Truncation does not apply, though, across any other morpheme junctures in the verbal planar structure. Instead, a gliding process ($\varnothing$ $\rightarrow$ [j] / V$\_$V) is observed across the Stem and negative, modality, and hearsay junctures. For example, a glide is inserted between vowels spanning the Stem-Negative juncture in (\ref{gliding}). \citet{Miller:2018} first identified this gliding process. As it is restricted to these junctures and not seen elsewhere, it is excluded from the present results pending further research. 

\ea \label{gliding}
\glll {...} á- \textbf{g\^u:} -\textbf{j\^{ɔ}:} {...} (*á-g\^{ɔ}:)\\
{...} {\ref{Vpro}}- \ref{VStem} -\ref{VNegativeSuffix} {...}\\
{...} [\Third\Pl]- get.well -{\Neg} {...}\\
\trans `They don't get better.' \citep[from][216]{Watkins:1984} %195c p. 216
\z 

Since the nominalizer suffix is always consonant-initial, it is impossible to test for Vowel Truncation's application. There is clear evidence that the process is blocked from applying at the locative juncture, though. In (\ref{VT:loc}), the locative suffix /-èm/ `where' attaches to the relative clause but does not undergo Vowel Truncation when adjacent to the vowel-final nominalizer.

\ea \label{VT:loc}
\glll {\{}\'{ɔ}:k\'{ɔ} $\varnothing$- tʰón- d\'{ɔ}: -\textbf{dé}{\}} -\textbf{èm} \`a- t͡sán -gòm \\
{\{}\ref{VDirectObject} {\ref{Vpro}}- \ref{VIncorpV} \ref{VStem} -\ref{VNominal}{\}}\textsubscript{\ref{VDirectObject}} -\ref{VLocative} {\ref{Vpro}}- \ref{VStem} -\ref{VAsp}\\
{\{}well [\Third\Sg]- dig- be -{\Nom}{\}} -where [\First\Sg]- arrive -\Distr/\Pfv \\
\trans `I got around to places where wells had been dug.' \citep[][180]{Watkins:1984} %195 p. 180
\z 

\noindent The process is blocked at the Stem-Subordinate marker juncture in (\ref{VT:sub}). As seen below, the future and switch reference marker join together and form vowel hiatus. Vowel Truncation does not apply, and both endings surface unchanged.

\ea \label{VT:sub}
\glll gjá- tʰént͡s'ò \textbf{t\`{ɔ}} -\textbf{\k{è}:} èm- bá:\\
{\ref{Vpro}}- \ref{VStem} -\ref{VMod} -\ref{VSubordinate} {\ref{Vpro}}- \ref{VStem}.\ref{VMod}\\
[(\First\Sg/\Aarg):\Second,\Third\Sg:/\Parg:\Sg/\Obj]- allow -{\Fut} -{\When}.{\Diff} [\Second\Sg]- go.{\Imp}\\
\trans `When I allow it, you will go.' \citep[][128]{Miller:2018}  %128 p. 92 diss
\z 

\noindent Vowel Truncation applies within a pronominal but not across its juncture. Similarly, the process is blocked from applying across incorporated elements' junctures. Both instances can be seen in (\ref{VT:proincorp}) below.

\ea \label{VT:proincorp}
\glll \k{é}:- \'{ɔ}:- \k{\'{ɔ}}:\\
{\ref{Vpro}}- \ref{VIncorpAdv}- \ref{VStem}.\ref{VMod}\\
[(\Second,\Third\Sg/\Aarg):\First\Sg/\Parg:$\varnothing$/\Obj]- temporarily- give.{\Imp}\\
\trans `(You) loan it to me.' \citep[adapted from][]{Miller:2018} %126 p. 92 diss
\z 

\noindent Even when fractured, the minimal and maximal domains identify a single subspan. The \textsc{Vowel Truncation} domain spans from the verb stem to the aspectual marker (Slots \ref{VStem}-\ref{VAsp}). 

\subsubsection{Dental-velar switch (\ref{VAsp}-\ref{VHearsay}; \ref{VIncorpAdv}-\ref{VSubordinate})} \label{sec:DV}

The final segmental process we will consider is the Dental-Velar Switch, an interesting process in Kiowa where dental and velar stops switch before certain front vowels (i.e. /ge/ $\rightarrow$ [de] and /di/ $\rightarrow$ [gi]). There is evidence that the process applies across the Aspect-Modality juncture and the Aspect-Hearsay juncture. In (\ref{DV:AspMod}), the initial /d/ in the imperfective suffix switches to [g] before the imperfective [-î:] following Vowel Truncation. In (\ref{DV:AspHsy}), the /g/ in the imperfective suffix switches to [d] before the hearsay [-ê:].

\ea Derivation of /há:-dè-î:/ `shout-\Ipfv-\Imp' \label{DV:AspMod}
\begin{tabular}{ll} 
     /há:-dè-î:/ &  \\
     há:-dî: & Vowel Truncation \\
     há:-gî: & DV Switch \\
     {[há:-gî:]} & \\
\end{tabular}
\z 

\ea Derivation of /má:-dè-ê:/ `feed-\Ipfv-\Hsy' \label{DV:AspHsy}
\begin{tabular}{ll} 
     /má:-gè-ê:/ &  \\
     má:-gê: & Vowel Truncation \\
     má:-dê: & DV Switch \\
     {[má:-dê:]} & \\
\end{tabular}
\z 

A combination of factors disallow testing of other morphemes and junctures. First, a phonotactic constraint bans /g/ as a coda thereby requiring that any test focus on /d/-final morphemes. Second, /i/-initial morphemes are rare in Kiowa. In order to test between the Stem-Aspect juncture, we need a /d/-final verb stem before an /i/-initial aspectual marker. No such sequence has been found in the current corpus. Additionally, there is no possible test for the negative suffix, nominalizers, or locative suffixes, as none of them begin with /i/ or /e/. While there are /e/-initial subordinate markers, it is not possible to test since there is no reason a final [g] would ever precede the subordinate marker. 

Like Vowel Truncation, Dental-Velar Switch is attested within pronominals but not across their juncture. Because pronominals form their own syllabification domain, final /d/ always devoices to [t] thereby bleeding the application of Dental-Velar Switch. Additionally, there are very few /i/-initial morphemes reported in Kiowa (e.g. /îl/ ‘warn,’ /í:/ ‘baby’). In fact, the current corpus and surveys of the literature do not include the necessary constructions to test across junctures between incorporated elements before the stem.

When fractured, the \textsc{Minimal Dental-Velar Switch} domain identifies a subspan of the aspect, modality, and hearsay suffixes (Slots \ref{VAsp}-\ref{VHearsay}). Since there is very little that could be tested, we must say the \textsc{Maximal Dental-Velar Switch} domain is much larger. Though it is clear the process cannot apply from the right edge of the pronominal, there has been no counterevidence throughout the remainder of the verb complex. Thus, the domain spans from the first incorporated element through the subordinate marker (Slots \ref{VIncorpAdv}-\ref{VSubordinate}).

\subsection{Tone lowering (\ref{Vpro}-\ref{VHearsay}; \ref{Vpro}-\ref{VSubordinate})} \label{sec:lowering}

While there are several reported tone processes in Kiowa, most are morpholog\-i\-cal\-ly-conditioned and thus irrelevant to the present analysis.\footnote{There
    is a tone raising rule found only in compounds, and there is a morphological tone lowering rule. Watkins cannot find a systematic analysis other than to lexically specify each verb root underlyingly as tone-lowering or non-tone-lowering. Interested readers are directed to \citet{Watkins:1984} for more information on these processes.}
The only phono\-log\-i\-cal\-ly-conditioned tonal modification is observed in Tone Lowering (lower tones after a falling tone), a type of L-spreading. As seen in (\ref{low:pro}), the process is triggered by the falling tone on the pronominal and lowers the underlying high tone on both verb stems (cf. /p\k{ó}:/ `look' and /\'{ɔ}:/ `give').\footnote{For maximal clarity, I have provided underlying forms for each example in this subsection. They are found in the first line between slashes.}

\ea \label{low:pro}
\gllll /kút bágî:- p\k{ó}:- \'{ɔ}:/ \\
kút \textbf{bágî:}- \textbf{p\k{ò}:}- \textbf{\`{ɔ}:}\\
\ref{VDirectObject} {\ref{Vpro}}- \ref{VIncorpV}- \ref{VStem}.\ref{VMod} \\
book [\Second\Pl/\Aarg:(\First,\Third\Sg/\Parg):\Pl/\Obj]- look- give.{\Imp}\\
\trans `(You pl.) show me the book.' \citep[adapted from][92]{Miller:2018} %142 p. 92 diss
\z 

\noindent The process does not occur across the verb complex's left edge, though. As seen in (\ref{low:complex}), the falling tone in /k'j\k{á}:h\k{î}:/ `man' does not lower anything in the verb complex. In fact, the same verb stem for `look' as above appears here unchanged with its underlying high tone /p\k{ó}:/.

\ea \label{low:complex}
\gllll /k'j\k{á}:h\k{î}: $\varnothing$- p\k{ó}:- \k{\`a}:/ \\
k'j\k{á}:h\k{î}: $\varnothing$- p\k{ó}:- \k{\`a}:\\
\ref{VDirectObject} {\ref{Vpro}}- \ref{VIncorpV}- \ref{VStem}.\ref{VAsp} \\
man [\Third\Sg]- look- come.{\Pfv} \\
\trans `The man came to see (you).' \citep[][98]{Miller:2018} %148 p. 98 diss.
\z 

An incorporated element can also trigger lowering of the remainder of the verb complex. As seen in (\ref{low:incorp}), the falling tone on the incorporated noun /s\k{â}/ `child' lowers the incorporated verb stem and verb stem.

\ea \label{low:incorp}
\gllll /\`a- s\k{â}- p\k{ó}:- \k{\`a}:/\\
\`a- \textbf{s\k{â}}- \textbf{p\k{ò}}:- \textbf{\k{\`a}:}\\
{\ref{Vpro}}- \ref{VIncorpN}- \ref{VIncorpV}- \ref{VStem}.\ref{VAsp}\\
[\First\Sg]- child- look- come.\Pfv\\
\trans `I came to see the child.' \citep[][92]{Miller:2018} %143 p. 92 diss
\z 

Low tone spreads throughout the verb suffixes like the imperative modality suffix, the negative suffix, and the hearsay suffix. In (\ref{low:mod}), the stem's falling tone triggers the imperative suffix /-î:/ to lower. In (\ref{low:neg}), the falling tone on the incorporated noun lowers the negative suffix /-m\^{ɔ}/. Finally, the negative suffix's falling tone triggers the hearsay suffix /-hêl/ to lower in (\ref{low:hsy}).

\ea \label{low:mod}
\gllll /hóld\`a b\`at- \k{\^{ɔ}}:m -î:/ \\
hóld\`a b\`at- \textbf{\k{\^{ɔ}}:m} -\textbf{\`i:} \\
\ref{VDirectObject} {\ref{Vpro}}- \ref{VStem}.\ref{VAsp} -\ref{VMod}\\
dress [\Second\Sg/\Aarg:\Pl/\Obj]- make.{\Ipfv} -{\Imp}\\
\trans `Keep on making the dress.' \citep[adapted from][92]{Miller:2018} %144 p. 92 diss
\z 

\ea \label{low:neg}
\gllll /\`a- s\k{â}- p\k{ó}:- \k{\`a}: -m\^{ɔ}:/ \\
\`a- \textbf{s\k{â}}- \textbf{p\k{ò}:}- \textbf{\k{\`a}:} -\textbf{m\`{ɔ}:}\\
{\ref{Vpro}}- \ref{VIncorpN}- \ref{VIncorpV}- \ref{VStem}.\ref{VAsp} -\ref{VNegativeSuffix}\\
[\First\Sg]- child- look- come.{\Pfv} -{\Neg}\\
\trans `I came to see the child.' (Andrew Robert McKenzie, p.c.)
\z 

\ea \label{low:hsy}
\gllll /hèg\'{ɔ} k\'{ɔ}j-d\`{ɔ}m-gj\`a h\'{ɔ}n m\`a:- t͡s\k{\`a}:n -\^{ɔ} -hêl háòtè-sáj/ \\
hèg\'{ɔ} k\'{ɔ}j-d\`{ɔ}m-gj\`a h\'{ɔ}n m\`a:- t͡s\k{\`a}:n -\textbf{\^{ɔ}} -\textbf{hèl} háòtè-sáj\\
\ref{VClauseIntroducers} \ref{VNL1} \ref{VNegation} {\ref{Vpro}}- \ref{VStem} -\ref{VNegativeSuffix} -\ref{VHearsay} \ref{VAdverbs2}\\
now Kiowa-land-at {\Neg} [\Second\Du]- arrive -{\Neg} -{\Hsy} several-year\\
\trans `You (dual) reportedly haven't been in Kiowa country for several years.' \citep[][178]{Watkins:1984}
\z 

All nominalizer suffixes and subordinate markers have an underlying low tone, so it is not possible to test for the process's application. There are occasional examples where their underlying tones change, but it is not due to Tone Lowering. Tonal modification in Kiowa is relatively understudied and other tonal processes are left to future research. 

Finally, the process is blocked at the right-edge of the verb complex (e.g. an adverb or right-dislocated element) just like the left. For example, (\ref{low:right}) two verb complexes occur next to one another. The first ends in falling tone on the negative suffix, but that does not trigger Tone Lowering across into the next verb complex. 

\ea \label{low:right}
\gllll /h\^{ɔ}ndó h\'{ɔ}n \k{é}:- há:d -\^{ɔ} \k{é}:- b\k{ó}: -t͡s\k{è}:/\\
h\^{ɔ}ndó h\'{ɔ}n \k{é}:- há:d -\textbf{\^{ɔ}} \textbf{\k{é}:}- b\k{ó}: -t͡s\k{è}:\\
\ref{VQuestionParticles} \ref{VNegation} {\ref{Vpro}}- \ref{VStem} -\ref{VNegativeSuffix} {\ref{Vpro}}- \ref{VStem}.\ref{VAsp} -\ref{VSubordinate}\\
why/{\Q} {\Neg} [(\Second,\Third\Sg/\Aarg):\First\Sg/\Parg:$\varnothing$/\Obj]- call.to -{\Neg} [(\Second,\Third\Sg/\Aarg):\First\Sg/\Parg:$\varnothing$/\Obj]- see.{\Pfv} -{\When}.{\Same} \\
\trans `Why didn't you call to me when you saw me?' \citep[][240]{Watkins:1984}
\z 

When fractured, the \textsc{Minimal Tone Lowering} Domain is from the pronominal through the hearsay suffix (Slots \ref{Vpro}-\ref{VHearsay}). The \textsc{Maximal Tone Lowering} domain continues through to the subordinate markers that cannot be tested (Slots \ref{Vpro}-\ref{VSubordinate}). 

\subsection{Pausing (\ref{VLeftRC}-\ref{VDislocatedDP})} \label{sec:pausing}

Finally, Kiowa uses pausing to mark grammatical information between clauses much like English (e.g. to indicate a conditional statement).\footnote{I did not test for where a speaker \textit{could} pause within a clause. I only examined cases of clause marking and disambiguation. Where exactly speakers are comfortable including pauses not directly related to clause-marking or grammatical information is left to future work.} In my fieldwork, I have found that it is a consistent diagnostic of junctures between clauses. For example, a brief pause has been indicated by the IPA pause symbol (.) in (\ref{food}) below. It occurs between the first and second clause, separating the conditional statement from the rest of the sentence. Thus, the domains for grammatical pausing is the full Kiowa verbal planar structure (Positions \ref{VLeftRC}-\ref{VDislocatedDP}). 

\ea \label{food}
\glll j\k{\`a}n- p\k{í}:- \^{ɔ}:m -\k{è}: (.) b\`at- pô:\\
{\ref{Vpro}}- \ref{VIncorpN}- \ref{VStem}.\ref{VMod} -\ref{VSubordinate} (.) {\ref{Vpro}}- \ref{VStem}.\ref{VMod}\\
[(\Second,\Third\Sg/\Aarg):\First\Sg/\Parg:\Pl/\Obj]- food- make.{\Imp} -{\When}.{\Diff} (.) [\Second\Sg/\Aarg:\Pl/\Obj]- eat.{\Imp}  \\
\trans `If I make food for you, you must eat it.' \citep[][100]{Miller:2018}
\z 

\section{Deviations from biuniqueness (\ref{VStem}-\ref{VNominal})} \label{biuniqueness}

The final diagnostic we will consider is {\textsc{Deviations from Biuniqueness}}. Biuniqueness is the requirement that formatives display a one-to-one relation with meaning. Kiowa deviates from biuniqueness when inflecting verb stems with aspect, negation, and when forming a relative clause (Positions \ref{VStem}-\ref{VNominal}). For each of the morphemes involved, there are forms that do not appear to be phonologically conditioned. 

Consider, for example, the perfective suffix (\tabref{extab:04:perfective}), which \citet{Watkins:1984} references as the most morphologically complex of any verb inflection categories. In all cases except for intransitive stems ending in basic verb suffixes \textit{-bé}, \textit{-dé}, or \textit{-gé}, the perfective has multiple forms associated with the same meaning. First, stems ending in /m, n, j, V:/ may either surface seemingly unchanged (a zero allomorph) or with the suffix \textit{-é}. There is no way to predict which one surfaces. Second, /l/-final stems undergo obstruentization (/l/ $\rightarrow$ [t]) but the form may also optionally include -é. Again, there is no way to predict when this suffix surfaces and when it does not. For those stems with the basic verb suffixes, transitive stems are suffixed with -\'{ɔ} or -é. Intransitive stems are only suffixed with -iá(j). Finally, some vowel-final stems include no \textit{-é} but instead end in one of series of consonants (/m, n, j, p/). This choice is not phonologically predictable. Thus, a vowel-final stem inflected for the perfective may involve a zero morpheme (i.e. no surface change), an \textit{-é} suffix, or end with one of four consonants (/m, n, j, p/), thereby deviating from Biuniqueness clearly.

\begin{table}
\caption{Perfective endings \citep[160--164]{Watkins:1984} \label{extab:04:perfective}}
\begin{tabularx}{\textwidth}{Xll}
\lsptoprule
    Stems ending in & Allomorph(s) & Examples \\
    \midrule
    m, n, j, V: & $\varnothing$ or -é: & tʰêm `break.{\Pfv}' (cf. /tʰ\k{ê}:m/)\\
    & & \k{\'{ɔ}}:m-é `make-{\Pfv}' (cf. /\k{\'{ɔ}}:m/)\\
    \tablevspace
    l & t or -é & gút `write.{\Pfv}' (cf. /gú:l/)\\
    & & k'\'{ɔ}:l-é: `bite.{\Pfv}' (cf. /k'\'{ɔ}:l/)\\
    \tablevspace
    -bé, -dé, -gé & -\'{ɔ} or -é: (\Tr) & hé:b-\^{ɔ} `bring in-{\Pfv}' (cf. /hé:-bé/)\\
     & & k'\'{ɔ}:t-é: `meet.{\Pfv}' (cf. /k'\'{ɔ}:té/) \\
     & -iá(j) ({\Intr}) & kʰút-kjá `get pulled off-{\Pfv}' (cf.  /kʰú:l/)\\
    V: & -C (m, n, j, p) & tʰóm `drink.{\Pfv}' (cf. /tʰ\k{ó}:/)\\
    \lspbottomrule
\end{tabularx}
\end{table}

Other aspect markers show similar patterns, though not nearly as complicated. The transitive imperfective, for example, has three forms: -m\`{ɔ}, -t\`{ɔ}, and -g\`u. The first two forms could arguably be grounded in phonology. The first form -m\`{ɔ} occurs after /m, n, j, V:/. The second -t\`{ɔ} occurs after l-final obstruentization and therefore exhibits a type of stop assimliation. One could argue that /-m\`{ɔ}/ is underlying and the default form. The third form /-g\`u/, however, is not predictable in any way. There is no phonological explanation for why the first consonant needs to be [g] or why the vowel is different in that form (\tabref{tab:04:imperfective}).

\begin{table}
\caption{\label{tab:04:imperfective}Imperfective endings \citep[][164--167]{Watkins:1984} }
\begin{tabularx}{\textwidth}{Xll} \lsptoprule
    Stems ending in & Allomorph(s) & Examples \\ \midrule
    m, n, j, V: & -m\`{ɔ} & kʰ\k{î}n-m\`{ɔ} `cough-\Ipfv' (cf. /kʰ\k{î}:n/)\\
    l & -t\`{ɔ} & ót-t\`{ɔ} `drop/fall-\Ipfv' (cf. /ó:l/)\\
    j, V: (\Tr) & -g\`u & sô:-g\`u `sew-\Ipfv' (cf. /sô:/)\\
    \lspbottomrule
\end{tabularx}
\end{table}

The negative suffix also shows deviation from biuniqueness when attached to vowel-final stems. Though they are predictably patterned in terms of transitivity and whether or not the verb is stative/active, the only thing connecting the three endings is a falling tone. The vowels and initial consonants differ with no obvious reason (\tabref{tab:04:negative}).

\begin{table}
\caption{\label{tab:04:negative}  Negative endings \citep[][176--178]{Watkins:1984}}
\begin{tabularx}{\textwidth}{Xll}\lsptoprule
    Stems ending in & Allomorph(s) & Examples \\ \midrule
    m, n, l, j & -\^{ɔ} & tʰ\k{é}:m-\^{ɔ}: `break-\Neg' (cf. /tʰ\k{é}:m/)\\
    \k{V} & -m\^{ɔ}  & \k{á}:-m\^{ɔ}: `come-\Neg' (cf. /\k{á}:/)\\
    V & -g\^u (\Tr/\Act) & kʰí:-g\^u: `carry.out-\Neg' (cf. /kʰî:/)\\
     & -j\^{ɔ} (\Intr/\Act) & á:-j\^{ɔ}: `grow-\Neg' (cf. /á:/)\\
     & -g\^{ɔ} (\Intr/\Stat) & dé:-g\^{ɔ}: `be.standing-\Neg' (cf. /dé:/)\\
     \lspbottomrule
\end{tabularx}
\end{table}

Finally, the nominalizing suffix -- specifically the inverse suffix /-g\'{ɔ}/ -- shows deviations from biuniqueness. The nominalizing suffix references the head noun in a relative clause. In Kiowa, all nouns have an inherent or implicit number when unsuffixed. They may be singular/dual or dual/plural. The inverse suffix \textit{-g\'{ɔ}} indicates the non-inherent number. A noun that is inherently singular/dual, for example, is plural when the inverse suffix is added. A noun that is inherently dual/plural is singular when the inverse suffix is added. The inverse suffix demonstrates numerous allomorphs that are not phonologically conditioned (\tabref{extab:04:inverse}).

\begin{table}
\caption{\label{extab:04:inverse}Inverse endings \citep[][80]{Watkins:1984}}
\begin{tabularx}{\textwidth}{XXl} \lsptoprule
    Stems Ending In... & Allomorph(s) & Examples \\ \midrule
    \k{V}j & -m\'{ɔ} & t'\k{á}j/t'\k{á}j-m\`{ɔ} `egg'\\
    m & -b\'{ɔ} & kóm/k\k{ó}:-b\`{ɔ} `friend'\\
    n & -d\'{ɔ} & k'\^{ɔ}n/k'\k{\^{ɔ}}:-d\`{ɔ} `tomato'\\
    {\`V}l & -d\'{ɔ} & tógúl/tógú:-d\'{ɔ} `young man'\\
    \^{V}l & -t\'{ɔ} & tâl/tát-t\`{ɔ} `skunk'\\
    j & -gú & k\'{ɔ}j-/k\'{ɔ}j-gú `Kiowa'\\
    i & -ój & p'í:/p'j-ój `female's sister'\\
    e & -óp & s\`a:né/s\`a:n-óp `snake'\\
    elsewhere & -g\'{ɔ} & t͡s\k{ê}:/t͡s\k{ê}:-g\`{ɔ} `horse'\\
    \lspbottomrule
\end{tabularx}
\end{table}

To summarize, \textsc{Deviation from Biuniqueness} identifies the subspan from the verb stem to the nominalizer (Slots \ref{VStem}-\ref{VNominal}). Outside of this subspan, Kiowa is pretty consistently and transparently agglutinative and predictable. 

\section{Discussion} \label{sec:discussion}

In this section, I briefly summarize the results and wordhood candidates identified by convergence of diagnostics. I then discuss the implications of these results. I focus first on the success of the Planar Fractal Method for Kiowa and then how these results are situated within the larger wordhood discussion. I conclude by outlining further questions and future directions. 

\subsection{Summary}

Together, morphosyntactic and phonological diagnostics converge and identify five candidates for wordhood. I have included the subspans in increasing size and which identifying diagnostics converged in (\tabref{extab:04:candidates}) below. Candidates 1, 2, 3, and 5 are characterized by a mix of morphosyntactic and phonological diagnostics, strengthening any proposals including them as candidates for wordhood, while Candidate 4 relies exclusively on phonological diagnostics.

\begin{table}
\caption{\label{extab:04:candidates}  Wordhood candidates in Kiowa }
\fittable{
\begin{tabular}{lll}
\lsptoprule
    Candidate & Positions & Convergence\\ \midrule
   (1) \Stem-\Hsy & \ref{VStem}-\ref{VHearsay} &  Ciscat. Select.; Cluster Devoicing (Max.)\\
   (2) \Stem-\Sub & \ref{VStem}-\ref{VSubordinate} & Nonint. (Simplex); Syllab (Min.) \\
   (3) \Pronom-\Sub & \ref{Vpro}-\ref{VSubordinate} & Free Occur. (Max.), Tone Lowering (Max.) \\
   (4) {\textsc{IncorpAdv}}-\Sub & \ref{VIncorpAdv}-\ref{VSubordinate} &  Syllab. (Max.), D-V Switch (Max.)\\
   (5) {\textsc{full clause}} & \ref{VLeftRC}-\ref{VDislocatedDP} & Subspan Repitition, Pausing\\
   \lspbottomrule
\end{tabular}
}
\end{table}

Candidate 1 corresponds with what most interface theories would call a phonological word (verb and inflectional suffixes). This is, in fact, one of the phonological words identified in \citet{Miller:2015,Miller:2018,Miller:2020}. Candidate 2 adds the remainder of the verb complex to Candidate 1 (i.e. Nominalizer, Locatives, and Subordinating Markers). Though this does not correspond to a previously proposed prosodic constituent in Kiowa, it is not surprising that there may be an intermediate constituent between the phonological word and phonological phrase.  Candidate 3 corresponds to the verb complex itself, which is not surprising since it is a complex V\textsuperscript{0} and thus identified as a phonological word under some theories. In \citet{Miller:2018,Miller:2020}, however, this is identified as a phonological phrase.  Candidate 4 is interesting, since it is the full verb complex without the pronominal. As it is identified by phonological criteria only, perhaps it is an artifact of the phonological separation of the pronominal clitic and the remainder of the verb. Finally, Candidate 5 consists of the entire Kiowa clause or verbal planar structure corresponding with an intonational phrase in \citet{Miller:2018,Miller:2020}.

\subsection{Situating the results}

In \citealt{Miller:2018}, I adopted a similarly structured method to the Planar-Fractal Method but focused entirely on phonological processes. Any domains that were identified by overlap (i.e. convergence of more than one process) were compared to theoretical predictions for prosodic constituents of different size. I concluded that there were three different sizes of phonological domains, and those domains correspond to the phonological word, phonological phrase, and intonational phrase. I am able to correctly predict the Kiowa domains using Tri-P Mapping (or Phase-based Prosodic Phonology) referencing cycles in the syntax to map prosodic structure \citep[][]{Miller:2018,Miller:2020,miller:2021}. It is interesting that the Planar-Fractal Method 1) successfully replicated the results of this previous analysis (Candidates 1, 3, and 5 correspond to the phonological word, phrase, and intonational phrase, respectively) and 2) did so with minimal theoretical assumptions and machinery. The fact that it does so is impressive confirmation of the domains active in Kiowa and of Tri-P's analysis of the language. 

These candidates correspond to prosodic constituents, though, and I hesitate to call them "words". If anything, I think these results suggest that the idea of the ``word" is tangential to successful analysis. As mentioned earlier, it is the verb complex itself that is arguably a complex head V\textsuperscript{0} and -- by many scholars -- would be called a word (see \citealt{selkirk:2011} and the discussion therein). This is not a meaningful distinction, though, without further extrapolation about the properties of this word and what that means. In this, the Planar-Fractal Method is a successful method for stripping away unnecessary assumptions and may be helpful in confirmation of theoretical proposals in the future. I would not go as far as \citet{bickel:2017} to say that a clear word definition (at least in phonology) is out of reach, though. Tri-P Mapping offers such a definition, and it is showing early success.

\subsection{Remaining questions and future directions}

In this section, I conclude with a list of questions to pursue in future research.

\begin{enumerate}
    \item While previous research admits that more than one tense/aspect particle can occur, it is a novel analysis to allow modal particles to form a zone in the planar structure above. I have found only one example of two modals co-occurring, and this merits further interest. Which particles can co-occur? For both zones, is it possible for more than two to co-occur?
    \item What is the difference between coordinate and subordinate structure in Kiowa, and how does that affect the prosodification of switch-reference markers? 
    \item What is the precise nature of the gliding process that seems to subvert Vowel Truncation? 
    \item What is the precise nature of the other tonal modification processes at play in the data? 
    \item Address the gaps in testing mentioned in the phonological analysis (i.e. those whose environments are indicated as crucial but no such example exists in the current corpus). 
\end{enumerate}


\section*{Acknowledgements}

My thanks go first and foremost to the Kiowa people for welcoming me into their community and trusting me with their knowledge. I am forever grateful and honored. Thanks to the editors of this volume (Adam Tallman, Sandra Auderset, and Hioroto Uchihara) and to two outside reviewers (Andrew McKenzie and Anthony Woodbury). Your comments and suggestions throughout the writing and editing process were invaluable. Many thanks to Natalie Weber for their helpful comments, sanity checks, and countless Zoom co-working sessions. Finally, thank you to the audience at the Workshop on Constituency and Convergence in the Americas for their helpful comments on an earlier version of this work. The fieldwork referenced in this work was partially funded by the Jacobs Research Fund, University of Delaware, and SUNY Oswego. 

\newpage
%Print the abbreviations
\printglossary

\printbibliography[heading=subbibliography,notkeyword=this]

\newpage
\section*{Appendix: Complete diagnostic results}

All results from the morphosyntactic and phonological constituency diagnostics throughout this analysis are summarized below:

    \begin{table}
    \caption{Diagnostic results for verbal planar structure: morphosyntactic diagnostics}
    \label{tab:verbresults}
      \begin{tabularx}{\textwidth}{>{\raggedright}p{4cm}rrrrQ}
        \lsptoprule
        & L & R & Size & Conv. & \\ \midrule
        Free Occurrence (Minimal) & \ref{Vpro} & \ref{VAsp} & 6 & 1 & The smallest possible span that can be a minimal free form  \\
        Free Occurrence (Maximal) & \ref{Vpro} & \ref{VSubordinate} & 11 & 2 & The largest possible span that can be a minimal free form  \\
        Non-interruptability (Simplex) & \ref{VStem} & \ref{VSubordinate} & 8 & 2 & Elements in this span cannot be interrupted by any free form \\
        Non-interruptability (Complex) & \ref{VM6} & \ref{VSubordinate} & 17 & 1 &  Elements in this span cannot be interrupted by anything larger than a free form \\
        Non-permutability (Rigid) & \ref{Vpro} & \ref{VNegativeSuffix} & 7 & 1 &  Elements in this span cannot be permuted or variably ordered \\
        Non-permutability (Flexible) & \ref{Vpro} & \ref{VNominal} & 10 & 1 & Elements in this span can only be permuted to change scope \\
        Subspan Repetition & \ref{VLeftRC} & \ref{VDislocatedDP} & 39 & 2 & This is the smallest subspan which may be coordinated or subordinated. \\
        Ciscategorial Selection & \ref{VStem} & \ref{VHearsay} & 5 & 2 &  Elements in this span can only semantically combine with one part of speech class. \\
        \lspbottomrule
        \end{tabularx}
        \end{table}

\begin{table}
\caption{Diagnostic results for verbal planar structure: Phonological domains}
\begin{tabularx}{\textwidth}{>{\raggedright}p{2.5cm}r@{~~}r@{~~}r@{~~}rQ}
\lsptoprule  & L & R & Size & Conv. & \\
\midrule
Syllabification (Minimal) & \ref{VStem} & \ref{VSubordinate} & 8 & 2 & A span where there is positive evidence that elements of adjacent positions interact in syllabification. \\
Syllabification (Maximal) & \ref{VIncorpAdv} & \ref{VSubordinate} & 11 & 2 & The largest possible span where there is no evidence against elements of adjacent positions interact in syllabification. \\
Cluster Devoicing (Minimal) & \ref{VStem} & \ref{VNegativeSuffix} & 3 & 1 & A span where there is positive evidence that elements of adjacent positions interact in Cluster Devoicing. \\
Cluster Devoicing (Maximal) & \ref{VStem} & \ref{VHearsay} & 5 & 2 & The largest possible span where there is no evidence against the elements interacting in Cluster Devoicing. \\
Vowel-Truncation & \ref{VStem} & \ref{VAsp} & 2 & 1 & The span where elements of adjacent positions interact in Vowel Truncation. \\
Dental-Velar Switch (Minimal) & \ref{VAsp} & \ref{VHearsay} & 4 & 1 & The span where there is positive evidence that elements of adjacent positions interact in Dental-Velar Switch. \\
Dental-Velar Switch (Maximal) & \ref{VIncorpAdv} & \ref{VSubordinate} & 11 & 2 & The largest possible span where there is no evidence against the elements interacting in Dental-Velar Switch.\\
Tone Lowering (Minimal) & \ref{Vpro} & \ref{VHearsay} & 9 & 1 & The span where there is positive evidence that elements of adjacent positions interact in Tone Lowering \\
Tone Lowering (Maximal) & \ref{Vpro} & \ref{VSubordinate} & 12 & 2 & The largest possible span where there is no evidence against the elements interacting in Tone Lowering \\
Pausing & \ref{VLeftRC} & \ref{VDislocatedDP} & 39 & 2 & The span where elements of adjacent positions interact in Pausing \\
\lspbottomrule
\end{tabularx}
\end{table}

\begin{table}
\caption{Diagnostic results for verbal planar structure: Other Diagnostics}
\begin{tabularx}{\textwidth}{>{\raggedright}p{2.5cm}r@{~~}r@{~~}r@{~~}rQ}
\lsptoprule
 & L & R & Size & Conv. & \\
 \midrule
Deviations from Biuniqueness & \ref{VStem} & \ref{VNominal} & 6 & 1 & The span where forms in adjacent positions do not display a one-to-one relation with meaning, and the differences are not phonologically conditioned \\
\lspbottomrule
\end{tabularx}
\end{table}


\end{document}
