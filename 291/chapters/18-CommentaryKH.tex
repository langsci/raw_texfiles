\documentclass[output=paper]{langscibook}
\ChapterDOI{10.5281/zenodo.13208574}
\author{Kristine Hildebrandt\orcid{}\affiliation{Southern Illinois University Edwardsville}}
\title{Word domains, and what comes after}
\abstract{This commentary summarizes the work done by in the Word Domains module within the Autotyp initiative, including scholarship on prosody-morphology interfaces and the Prosodic
Hierarchy Hypothesis. The commentary includes the methods and findings from the Word Domains module, as well as ongoing initiatives and methodological challenges. The commentary then turns to how the case studies include in this volume
expand/deepen/improve upon the work started by Word Domains, also including some commentary on challenges highlighted by this work and some possible directions for future initiatives.}
\IfFileExists{../localcommands.tex}{
  \addbibresource{../localbibliography.bib}
  \addbibresource{../collection_tmp.bib}
  % add all extra packages you need to load to this file

\usepackage{tabularx,multicol}
\usepackage{url}
\urlstyle{same}

\usepackage{listings}
\lstset{basicstyle=\ttfamily,tabsize=2,breaklines=true}

\usepackage{langsci-basic}
\usepackage{langsci-optional}
\usepackage{langsci-lgr}
\usepackage{langsci-osl}
% \usepackage{./langsci/styles/langsci-lgr}
% \usepackage{./langsci/styles/langsci-osl}
% \usepackage{langsci-gb4e}

\usepackage{tikz}
\usetikzlibrary{patterns,calc}
\pgfdeclarepatternformonly{south east lines}{\pgfqpoint{-0pt}{-0pt}}{\pgfqpoint{3pt}{3pt}}{\pgfqpoint{3pt}{3pt}}{
    \pgfsetlinewidth{0.6pt}
    \pgfpathmoveto{\pgfqpoint{0pt}{3pt}}
    \pgfpathlineto{\pgfqpoint{3pt}{0pt}}
    \pgfpathmoveto{\pgfqpoint{.2pt}{-.2pt}}
    \pgfpathlineto{\pgfqpoint{-.2pt}{.2pt}}
    \pgfpathmoveto{\pgfqpoint{3.2pt}{2.8pt}}
    \pgfpathlineto{\pgfqpoint{2.8pt}{3.2pt}}
    \pgfusepath{stroke}}
    
\usepackage{stmaryrd}
\usepackage{wasysym}
\usepackage{multirow}
\usepackage{caption}
\usepackage{subcaption}
\usepackage{mathrsfs}
\usepackage{qtree}

\usepackage{linguex}


  %pminos do not split footnotes
% \interfootnotelinepenalty=10000 %Footnote in Laporte chapters has to be split SN


%\DeclareIndexNameFormat{default}{%
%\nameparts{#1}%
%\usebibmacro{index:name}%
%{\index[names]}%
%{\namepartfamily}%
%{\namepartgiveni}%
% {}% L1
% {}% L2
%{\namepartprefix}% generates spurious space L3
%{\namepartsuffix}% generates spurious space L4
%}

%  {\DeclareIndexNameFormat{default}{%
%     \usebibmacro{index:name}{\index[names]}{#1}{#3}{#5}{#7}}}

%\DeclareIndexNameFormat{default}{%
%  \usebibmacro{index:name}{\sindex[nom]}{#1}{#3}{#5}{#7}}

%\DeclareIndexNameFormat{default}{%
%  \usebibmacro{index:name}{\sindex[person]}{#1}{#3}{#5}{#7}}
%\DeclareIndexNameFormat{default}{%
%\nameparts{#1} \usebibmacro{index:name}{\sindex[person]]}{\namepartfamily}{‌​\namepartgiven}{\nam‌​epartprefix}{\namepa‌​rtsuffix}}

%\newcommand{\smiley}{:)}

%\renewbibmacro*{index:name}[5]{%
%\usebibmacro{index:entry}{#1}%
%{\iffieldundef{usera}{}{\thefield{usera}\actualoperator}\mkbibindexname{#2}{#3}{#4}{#5}}}

% \newcommand{\noop}[1]{}

%remove for final
%\overfullrule=1mm

\newcommand{\tobi}[2]}}
\renewcommand{\S}[1]{\tobi{#1}{\textsc{*}}}

% this volume references
% puts: [this volume]
% already defined: \citetv
%\newcommand{\citepv}[1]{(\citeauthor{#1} \citeyear*{#1} [this volume])}
\newcommand{\citealtv}[1]{\citeauthor{#1} \citeyear*{#1} [this volume]}

%parentheses around example number
\newcommand{\pref}[1]{(\ref{#1})}

% in-text examples

\newcommand{\lnex}[1]{\textit{#1}} %target lang word
\newcommand{\lnlit}[1]{(lit.: `#1')} %literal reading
\newcommand{\lnlat}[1]{(#1)} % latinization
\newcommand{\lntrans}[1]{`#1'} %translation
\newcommand{\lnexl}[2]%
{\lnex{#1}{} \lnlat{#2}} % ex with latinization
\newcommand{\lnexlat}[3]{\lnex{#1}{} \lnlat{#2}{} \lntrans{#3}} % ex with latinization and tranl.

%ch01
\newcommand{\co}[1]{\mbox{\textbf{#1}}}

%ch09

\newcommand{\cyrbulg}[1]{\begin{otherlanguage*}{bulgarian}#1\end{otherlanguage*}}


%ch10
\newcommand{\nlp}{{\small NLP}}
\newcommand{\mwe}{{\small MWE}}
\newcommand{\rae}{{\small RAE}}
\newcommand{\lvc}{{\small LVC}}
\newcommand{\pos}{{\small P}o{\small S}}
%\newcommand{\todo}[1]{ \textcolor{red}{#1} }

%\renewcommand{\labelenumi}{\theenumi}
%\ainamefmt{{vv}{ll}{, ff}{, jj}} % fullname

\newcommand{\biberror}[1]{{\color{red}#1}}

\newcommand{\osenovaitem}{--~} 
  %% hyphenation points for line breaks
%% Normally, automatic hyphenation in LaTeX is very good
%% If a word is mis-hyphenated, add it to this file
%%
%% add information to TeX file before \begin{document} with:
%% %% hyphenation points for line breaks
%% Normally, automatic hyphenation in LaTeX is very good
%% If a word is mis-hyphenated, add it to this file
%%
%% add information to TeX file before \begin{document} with:
%% %% hyphenation points for line breaks
%% Normally, automatic hyphenation in LaTeX is very good
%% If a word is mis-hyphenated, add it to this file
%%
%% add information to TeX file before \begin{document} with:
%% \include{localhyphenation}
\hyphenation{
    Beck-man
    Ngu-yen
    back-chan-nel
    back-chan-nels
    mo-not-o-nous
    ste-reo-typ-i-cal
}

\hyphenation{
    Beck-man
    Ngu-yen
    back-chan-nel
    back-chan-nels
    mo-not-o-nous
    ste-reo-typ-i-cal
}

\hyphenation{
    Beck-man
    Ngu-yen
    back-chan-nel
    back-chan-nels
    mo-not-o-nous
    ste-reo-typ-i-cal
}
 
  \togglepaper[1]%%chapternumber
}{}

\begin{document}
\maketitle 
%\shorttitlerunninghead{}%%use this for an abridged title in the page headers

\title{Commentary}

\section{Introduction}

As articulated in the introduction to this volume, attempts at modeling the pho\-nol\-ogy-syntax interface have given rise to various ways of defining prosodic and morphological constituency, or more generally, ``words." Proposals range from the tradition of invoking boundaries and junctures in describing constraints and patterns that map over morphological or syntactic structure (\citealt{chomsky_sound_1968}, \citealt{mccawley_phonological_1968}), on to prosodic phonology, where domains or phonology-grammar mapping are part of a larger prosodic hierarchy (\citealt{nespor_prosodic_1986}{\slash}\citeyear{nespor2007prosodic}, \citealt{truckenbrodt_relation_1999}, \textit{inter alia}), and also re-casting of these as violable constraints in the tradition of prosodic morphology (\citealt{mccarthy_prosodic_1986, mccarthy_prosodic_2001}, applied cross-linguistically in \citealt{kager1999prosody}). All of these take as their underlying assumption that the word is universal, including the tradition of basic linguistic theory, for example, \citet{dixon_basic_2010} claim that phonological and grammatical words can be recognized for all languages, and whose word bisection thesis attempts to account for prosodic and grammatical misalignments by separating a single notion of word into two potentially misaligning constituents.

These modeling attempts have run into problems in cross-linguistic applications, with repeated instances of languages that display a proliferation of misaligned constituents, or with constituents defined by differentially defined and sometimes conflicting diagnostics, or else with a lack of any evidence motivating word domains altogether (\citealt{schieringetal:2010}, \citealt{Haspelmath2011}, as covered in \citealt{Tallman2021}). As such, capturing a cross-linguistically viable notion of wordhood has remained elusive, a challenge taken up most recently in this collection of language-specific treatments with modified diagnostic methods.

In this section, I summarize attempts to typologize on prosody-morphology interfaces in the Word Domains module, and then turn to how the methods and languages included in this volume expand/deepen/improve upon the work started by the Autotyp group. I also consider some ongoing challenges highlighted by this work and some possible directions for future initiatives.

\section{The AUTOTYP Word Domains module: A recap and ongoing questions}

The original project was proposed by Balthasar Bickel and Tracy Hall in 2002 and their ideas were situated primarily in the context of Prosodic Phonology (\citealt{nespor_prosodic_1986}/\citeyear{nespor2007prosodic}), more specifically the predictions made within the Prosodic Hierarchy Hypothesis:

\begin{itemize}
\item 
Prosodic domains cluster on a single universal set of domains (‘Clustering’), and,

\item 
No level or node is skipped in the building of prosodic structure unless this is required by independently motivated higher ranking principles or constraints (‘Strict Succession’).

\end{itemize}

The focus in this project was to catalogue prosodic words, recast as ``domains" in which phonological generalizations are mapped onto morphological structure, for example, a stem and its affixes. While the database was primarily aimed at tracking prosodic processes that mapped morphological material, other domains were also defined and tracked on a language-specific basis, including syllable, foot, and when the language provided evidence for these, phonological phrase, intonation phrase, and phonological utterance.

The Word Domains module investigated the challenges summarized above. Working with an original sample of 70 languages, the researchers who participated in this project discovered that domains proliferate in number and type across languages, or else in some circumstances are not motivated at all. Detailed illustrations of domain proliferation or undergeneration are detailed in \citet{hildebrandt2007grammatical} and \citet{schieringetal:2010}, although it should be noted that other scholars have documented similar challenges in either a proliferation of prosodic word types (\citealt{post_phonology_2009}, \citealt{dunn_grammar_1999}, \citealt{hall_phonological_2008}, \citealt{mcdonough_bipartite_nodate}), or data that fail to identify lexically generalizable prosodic words \citep{bickel2009distribution}. The challenges are usually accounted for by including the exceptions in a finite list, by positing recursive domains, or by factoring out prosodic domains to different phonological tiers. Or, they have motivated a `weak layering' of the Prosodic Hierarchy and this has been cast within the tradition of Optimality Theory.

The Word Domains module is part of the larger AUTOTYP network of typological linguistic databases (\citealt{bickel_balthasar_autotyp_2017}, \citealt{witzlack2022managing}). The network seeks statistical universals by coding language-specific phenomena ``from the bottom up" to help understand how a probable system might look. In the case of the word domains module, we are interested in how a probable prosodic system might look.. Breaking this more general goal down, each module in AUTOTYP, including Word Domains, is constructed based on the following basic principles of:\footnote{A fifth AUTOTYP principle, “Exemplar-based method” is not discussed here.}

\begin{itemize}
\item 
Modularity and Connectivity: AUTOTYP is a network of thematically defined and connected modules (including the Word Domains Module) with shared infrastructure \& design principles;

\item 
Autotypology: Like other modules, Word Domains avoids pitfalls of theoretical positioning or a-priori intuition that can influence database design by building modules that dynamically expand lists of possible values during data input;

\item 
A database structure consisting of data files and definition files: Data files contain data on individual languages and Definition files are lists of possible values for each coded variable;

\item 
Late data aggregation: During data entry, we choose the lowest-level, most exhaustive model that is appropriate to the data domain \& purpose of data collection. Data filtering and aggregation are done outside of the database to avoid pitfalls connected to Principle ii.

\end{itemize}

These principles, and the resulting database structure allowed us to undertake an empirical investigation of the presumptions behind the Prosodic Hierarchy Hypothesis, namely a set of predictions contained within the larger Hypothesis:

\begin{itemize}
\item 
Some kinds of domains are recurrently larger than others, and that larger domains properly contain the smaller ones;

\item 
These hierarchies of domains will tend to cluster on universal “attractors” that are defined by some shared property. For example, vowel harmony processes might tend to cluster on certain domain sizes, while stress patterns cluster on another.

\end{itemize}

\citet{schieringetal:2010} asked whether probabilistic clusters may be identified, perhaps echoing what \citet{hyman_luganda_1987} suggested, namely that such patterns should be rather understood as a probabilistic trend rather than universal categorical constraint.


\hspace*{-3pt}In fact, the multidimensional scaling analysis employed in \citet{schieringetal:2010} did not significantly demonstrate this, other than showing an increased proportion of stress-related prosodic word-patterns in one cluster. This gave rise to one probabilistic universal: stress-related domains tend to be universally larger than other domains. Their investigation of this universal across three families (Austroasiatic, Indo-European, Sino-Tibetan) supported this prediction, and they also observed that non-stress pw-patterns do vary across the families, a trend of stress domains aligning with genealogical affiliation.

Of course, the methods and the findings in \citet{schieringetal:2010} were met with a variety of critiques. Most related to this volume is that the Word Domains dataset focuses largely on morphologically defined domains to the exclusion of syntactically defined ones (\citealt{bennett2019syntax}, \citealt{miller2018phonology}). The issue raised by these responses is that our database focuses primarily on so-called “word-level” prosodic units, without deeper consideration of larger morphosyntactic domains, leaving open larger questions of constituency that recognize larger grammatical units. On the one hand, this is a justified critique. On the other hand, the goal of the Word Domains project was always to survey (primarily) prosodic domains at sub-phrasal and sub-clausal levels, in line with specific predictions within the Prosodic Hierarchy Hypothesis regarding phonological words and the domains that are contained within it. The Word Domains project also had always recognized a lack of consistency in cross-linguistic descriptive accounts in terms of how ``words", and larger syntactic units, were defined in different treatments. This required a decomposition of grammatical units such as ``affix", ``clitic" and ``particle" into theory-neutral elements. For example, for the purpose of the Word Domains project, these units would be differentiated by means of application of a number of diagnostics, including the element’s categorical type, its host restrictions, its behavior and position in the relevant domain, its degree of prosodic coherence, its gapability, its position with respect to the host, and so on (\citealt{bickel_diversity_2005}, \citealt{Bickel2017}). This greatly increased the time it took to enter data comprehensively, and therefore had a constraining effect on the number of languages and the types of domain-related phenomena that could be tracked beyond the word level.

It is therefore refreshing to see this question of ``wordhood" (and of constituency more generally) taken up again, with different methods, and with a sample of languages that were not included in the original Word Domains project. This volume represents a typological investigation of 16 languages of the Americas, including a French-based creole \parencitetv{chapters/09-Martinican}. The studies employ controlled terms and methods, including larger morphosyntactic domains along with prosodic diagnostics, and tests constituency, rather than assuming it a-priori. The planar structures first illustrated in \citet{Tallman2021}, and employed here, building a bottom-up multivariate typology, and avoiding some of the assumptions and pitfalls inherent to the universalist models noted above. This makes Tallman et al.’s (ed.) approach similar to principle ii of AUTOTYP, while allowing for a greater range of diagnostics to be included in word-hood evaluations than allowed for in the Word Domains project.

\section{Strengths and challenges of this volume}
\subsection{The planar-fractal method}

The contributions in this volume all make use of (and in many cases, provide evaluative comments on) the planar-fractal method. In this approach, the morphosyntax of a language is rendered (``flattened") onto a templatic structure that represents all elements of some (verbal or nominal) domain, regardless of constituency structure. Planar structures thus conflate morphology and syntax, allowing for a more comprehensive application of constituency diagnostics.

A clear benefit of this approach is that ``fracturing" such planes of constituency allows for a much finer-grained detail in constituency variables on a language-specific basis, and for more nuanced portraits of aligning (or mis-aligning) phonological and grammatical domains.

However, one potential challenge is that the planar structure by necessity and by design conflates morphology and syntax. While there are those who argue that word formation is intrinsically linked to syntactic operations (\citealt{Baker1988}, \citealt{marantz_no_1997}), under other views (for example, \citealt{jackendoff_architecture_1997}, \citealt{ackema_competition_2001, ackema_morphology_2007}, and in this volume), this homogenization could be seen as problematic. Rather than building a database based on an a-priori assumption of the distinctiveness of morphological and syntactic modules, the goal here is to discover (via empirical evidence provided on language-specific bases) whether these two components can justifiably be defended as distinct modules or not.

This approach also provides evidence for multiple constituencies even within grammatical or phonological components. For example, several treatments at least distinguish between verbal and nominal planar structures and least one contribution finds evidence for an adverbial planar structure. For example, Nakamo\-to's treatment of Ayautla Mazatec \parencitetv{chapters/05-Mazatec}. On the other hand, the analysis of Cherokee \parencitetv{chapters/03-Cherokee} provides further evidence for adjectives and nouns as a single constituent type.

Another important potential takeaway from this approach, one that can fuel further research, is a different way of thinking about what morphology is. Rather than a view in which morphology is simply a set of word-level alternations and operations, it can be viewed instead as referring to paradigmatic dimensions of structuring. The approaches as currently formulated in this volume unfortunately do not expand on this potential, as they necessarily underdescribe interesting cross-linguistic morphological variation (which makes this approach different, for example, from \citealt{baerman_covert_2014}, \citealt{corbett_morphosyntactic_2015}, and other projects run by members of the Surrey Morphology group). As such, the planar-fractal method would need to be amended to further this view.

One potential challenge to this approach comes from languages that have so-called ``root-and-pattern" or templatic morphological systems. While this is most famously described for Semitic languages, some languages in the Americas might be candidates for inclusion due to their templatic systems, for example Yowlumni Yokuts (\citealt{kuroda_yawelmani_1967}, \citealt{archangeli_syllabification_1992}). For these languages, the planar-fractal method would result in their CV skeletons represented on the same morphosyntactic level, complicating attempts to tease out prosodic and morphological diagnostics. Other approaches to such languages suggest this is not a problem, and that aspects of the phonology point to syntactic structures (e.g., \citealt{faust_how_2009} account of Hebrew and Italian non-concatenative morphology).

The case of Mẽbêngôkre \parencitetv{chapters/12-Mebengokre}, which displays more fusional and non-concatenative processes than the other languages in this volume, presents similar potential complications for a planar model. Salanova illustrates ambiguities in distinguishing simple and complex structures in the language, e.g., in nominal quantification and modification, and in sentence-level modification, with a continuum of more or less grammaticalized elements. Salanova decides to treat such cases as revealing a complex structure, and these elements as part of a single independent clause template.

Another area for future work is constituency and convergence in creoles. The one creole in this study is Martinican. The ways in which constituency cues may overlap with those found in the contributing languages is not explicitly considered but is worthy of future study in this approach (see e.g. \citealt{good_tone_2004} analysis of a phonological split in Saramaccan creole).

One of the biggest strengths of this project is the active participation and criticism by the fieldworkers who engaged in the data collection and analysis for these chapters. Often working in tandem with the speech community (as evidenced by the many comments on speaker intuitions about constituency), they know the fine details, which can be left out in even the ``thickest" of reference grammatical descriptions. They also can introduce new ways of thinking about diagnostics and domain, as I comment on in \sectref{sec:diagnostics}.

\subsection{Fracturing}
\label{sec:KH:fracturing}

If a given test is ambiguous and delimits different spans according to the interpretation test fracturing is applied following \citet{Tallman2021}. For example, if the positive evidence and the negative evidence of a phenomenon define different domains, they are treated as two constituency diagnostics. This helps to identify minimal and maximal domains for free occurrence and for certain diagnostics (e.g. floating tone placement in Yukuna). It also allows for nuances in the description of diagnostic sub-types, for example, types and sub-domains of interruptions in a span of otherwise non-interruptible material in Chorote \parencitetv{chapters/15-Chorote}.

Compared with the coding decisions made in the Word Domains project, fracturing is an important methodological advancement. In the Word Domains data\-base, distinctions between “edge” and “span” processes and constraints were encoded, but there were times when this distinction was fuzzy (for example, how to encode a syllable-onset constraint and its resolution that applies between a prefix and a stem, and optionally includes the stem and all postposed inflectional{\slash}derivational material). Additional fields in our database (including examples) helped to disambiguate domain boundaries, but the fracturing approach here ensures that every constituency diagnostic is well-defined, including specific reference to a beginning position and an ending position. Similarly to the Autotyp principles stated above, this approach attenuates bias and a-priori assumptions about what/how many elements may be assumed or excepted from relevance in a diagnostic \citep{Tallman2021}.

\subsection{Diagnostics}
\label{sec:diagnostics}

A common concern in prior treatments of wordhood focus on the diagnostics for wordhood. Either they are too vague, there is uncertainty as to whether the tests identify words specifically, there is concern as to whether the tests themselves are reliable, or there is disagreement as to whether the tests identify prosodic or grammatical domains, rather than a unified notion of ``word" \parencitetv{chapters/01-Introduction}. One way around this, taken on by both the Word Domains module and by the Convergence and Constituency group, is to apply multiple tests and to see if and how they converge around a domain that could be considered a word in the language (and then potentially comparing that to native speaker intuitions, which itself can be conflicting and problematic).

\largerpage
In this volume, some diagnostics are appropriate in all (or most cases), for example minimal and maximal domains, while other diagnostics are modified and applied in language-appropriate ways. This is the case in the analysis of Hup grammatical constituency (Epps), where non-interruptability (defined) is sub-grouped as non-interruptability by a full NP and non-interruptability by a a promiscuous element, resulting in two sub-tests with different sizes of interrupting element. 
In the case of Chorote (Carol), ciscategoriality is sub-grouped into ``strict" (specific to verbs) and ``lax" (referencing the “main predicate of the clause”, whether verbal or not) versions. This accounts for the fact that in Chorote, both the verb, other word classes, as well as some inflectional markers (negation) may head the predicate in certain cases. Also in Chorote, NPs and DPs can take some of the ``verbal" TAME markers even when they function as arguments.

In the same spirit, conflicting results are embraced, rather than discarded or ignored, responding to critiques of diagnostic fishing or methodological opportunism voiced by \citet{Croft2001}, \citet{Haspelmath2011}. This is illustrated in the case of Zenzontepec Chatino. Campbell notes that establishing the verbal planar structure of Zenzontepec Chatino is challenged in the diagnostic of ``biuniqueness", defined as a deviation from a one-to-one form-meaning correspondence. In the case of Z. Chatino, aspect-mood inflection is partly prefixal and partly expressed by tone melody alternations (or lack thereof) on verb stems. Such cases of nonconcatenativity and deviation from biuniqueness do not fit neatly into a discrete linear model. Similarly, in Cup’ik, biuniqueness reveals what Woodbury terms as two ``patches" of constituency behavior outside of the verb core. In Chorote, Carol notes that the distinction between lexical classes is not always clean, which may be viewed as a challenge for the diagnostic of ``ciscategorial selection", where the domain refers to elements that exclusively combine with one part of speech.

Again, this project does not start with an a-priori assumption about what should (and will) converge, or even if a singular notion of ``word" is relevant{\slash}useful. Rather, the goal is to cross-linguistically survey the distribution of these diagnostic results, including how they might support or not support traditional understandings of words as grammatical and phonological constituents correlating with semantic relations, to use these data to test claims about the morphosyn\-tax-phonology interface, and to then move into the ``why" dimension (diachronic and cognitive forces) of contemporary typological inquiry (\citealt{bickel_typology_2007}, \citealt{levinson_original_2012}).

\section{Convergence, and what remains}

Some languages in this volume demonstrate very little evidence for convergence of any kind of word-like unit, such as Mixtec (Auderset et al.), or else, strong convergence signals reference only smaller domains, as with Mẽbêngôkre (Salanova). But, roughly half of the languages in this study do show patterns in line with the working assumption of the volume, namely that domains of high-constituency convergence are candidates for what we might think of as ``wordhood" (see \citealt{Matthews2002}; \citealt{Tallman2021}). However, this assumption can result in anomalies because in many of these cases, the largest constituency domains emerge as the most convergent. This is seen with Hup (Epps), Cherokee (Uchihara), Cup’ik (Woodbury), Araona (Tallman), and arguably Chatino (Campbell), although lar\-ger domain convergences reveal prosodic word candidates more so than morphosyntactic words. One solution to this anomaly is to propose some mix of convergence tests and then take into an account whether the fractured test is specifically a minimal or a maximal domain. With this approach, the difference between a minimal and maximal version of a constituency test would reflect degrees of freedom in the interpretation of a test and provide a clearer picture of domain trends.

In other cases, the lack of clearly converging diagnostics may also be an artifact of a domain not having the adequate morphosyntactic or phonological context for the constraint to be tested in the first place, as discussed in Salanova's treatment of Mẽbêngôkre's verb complex. Other cases of non-convergence can be attributed to diachronic forces, as nicely quoted in the Epps’ contribution, that ``every language is more or less a ruin." It is therefore not surprising that heterogeneous sets of diagnostics that are really the result of diachronic processes do not necessarily converge on a uniform domain akin to a ``word." These issues also echo what was reported in \citealt{schieringetal:2010}, (at least synchronically), that domains are often language-particular, intrinsic, highly specific, and contra to a proposed universal hierarchy of aligning and strictly layered domains.

As such, is there futility in searching for a unified cross-linguistic notion of ``word"? Even when attempts at morphosyntactic and phonological convergence are made in this volume, in many cases, even when there is a strong clustering signal, that convergence is still partial, as in the case of Quechua \parencitetv{chapters/14-Quechua} or Ayautla Mazatec \parencitetv{chapters/05-Mazatec} or the signal reveals a domain alignment in only one component of the grammar, as in the case of prosodic word domain convergence in Chatino \parencitetv{chapters/08-Chatino} and in Kiowa \parencitetv{chapters/04-Kiowa} to the exclusion of morphosyntactic convergence. These recurring challenges are an opportunity to remind ourselves that the label ``domains" was a deliberate decision made by \citet{schieringetal:2010} to recognize multiple, non-aligning span-units of constraints and processes.

Perhaps one way to go about identifying convergent notions of ``word" is to adopt different methods of data collection and analysis. Some contributions to this volume have referenced speaker intuitions, orthographic word comparisons, and patterns from language games (e.g., \textit{ludling} in Chatino). Tallman  suggests that a combination of language documentation collaboration (already on display in this volume) and a larger corpus of spontaneous speech data with annotated lexical and clause-level phonetic phenomena will contribute to a more empirically rich planar structure analysis across languages \citep{tallman2023phonology/phonetics}. This data set would potentially reveal more diagnostics as candidates for convergence. These approaches could serve to unlock the potential for the planar-fractal approach to yield more empirically robust results in the search for wordhood.


\sloppy\printbibliography[heading=subbibliography,notkeyword=this]


\end{document} 
