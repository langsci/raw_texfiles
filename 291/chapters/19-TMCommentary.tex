\documentclass[output=paper]{langscibook}
\ChapterDOI{10.5281/zenodo.13208576}
\author{Taylor L. Miller \affiliation{State University of New York at Oswego}}
\title{Diagnosing phonological constituency}
\abstract{The planar-fractal method is meant to provide a theory-neutral way to evaluate linguistic theories. In this commentary, I do this for the Combined Model, a new phonology-syntax interface theory which combines Tri-P Mapping and Cophonologies by Phase. The model successfully predicts and accounts for the patterns in Araona and Ayautla Mazatec, highlighting several strengths of the planar-fractal method and opening issues for future direction.}

\IfFileExists{../localcommands.tex}{%hack to check whether this is being compiled as part of a collection or standalone
   % add all extra packages you need to load to this file

\usepackage{tabularx,multicol}
\usepackage{url}
\urlstyle{same}

\usepackage{listings}
\lstset{basicstyle=\ttfamily,tabsize=2,breaklines=true}

\usepackage{langsci-basic}
\usepackage{langsci-optional}
\usepackage{langsci-lgr}
\usepackage{langsci-osl}
% \usepackage{./langsci/styles/langsci-lgr}
% \usepackage{./langsci/styles/langsci-osl}
% \usepackage{langsci-gb4e}

\usepackage{tikz}
\usetikzlibrary{patterns,calc}
\pgfdeclarepatternformonly{south east lines}{\pgfqpoint{-0pt}{-0pt}}{\pgfqpoint{3pt}{3pt}}{\pgfqpoint{3pt}{3pt}}{
    \pgfsetlinewidth{0.6pt}
    \pgfpathmoveto{\pgfqpoint{0pt}{3pt}}
    \pgfpathlineto{\pgfqpoint{3pt}{0pt}}
    \pgfpathmoveto{\pgfqpoint{.2pt}{-.2pt}}
    \pgfpathlineto{\pgfqpoint{-.2pt}{.2pt}}
    \pgfpathmoveto{\pgfqpoint{3.2pt}{2.8pt}}
    \pgfpathlineto{\pgfqpoint{2.8pt}{3.2pt}}
    \pgfusepath{stroke}}
    
\usepackage{stmaryrd}
\usepackage{wasysym}
\usepackage{multirow}
\usepackage{caption}
\usepackage{subcaption}
\usepackage{mathrsfs}
\usepackage{qtree}

\usepackage{linguex}


   %pminos do not split footnotes
% \interfootnotelinepenalty=10000 %Footnote in Laporte chapters has to be split SN


%\DeclareIndexNameFormat{default}{%
%\nameparts{#1}%
%\usebibmacro{index:name}%
%{\index[names]}%
%{\namepartfamily}%
%{\namepartgiveni}%
% {}% L1
% {}% L2
%{\namepartprefix}% generates spurious space L3
%{\namepartsuffix}% generates spurious space L4
%}

%  {\DeclareIndexNameFormat{default}{%
%     \usebibmacro{index:name}{\index[names]}{#1}{#3}{#5}{#7}}}

%\DeclareIndexNameFormat{default}{%
%  \usebibmacro{index:name}{\sindex[nom]}{#1}{#3}{#5}{#7}}

%\DeclareIndexNameFormat{default}{%
%  \usebibmacro{index:name}{\sindex[person]}{#1}{#3}{#5}{#7}}
%\DeclareIndexNameFormat{default}{%
%\nameparts{#1} \usebibmacro{index:name}{\sindex[person]]}{\namepartfamily}{‌​\namepartgiven}{\nam‌​epartprefix}{\namepa‌​rtsuffix}}

%\newcommand{\smiley}{:)}

%\renewbibmacro*{index:name}[5]{%
%\usebibmacro{index:entry}{#1}%
%{\iffieldundef{usera}{}{\thefield{usera}\actualoperator}\mkbibindexname{#2}{#3}{#4}{#5}}}

% \newcommand{\noop}[1]{}

%remove for final
%\overfullrule=1mm

\newcommand{\tobi}[2]}}
\renewcommand{\S}[1]{\tobi{#1}{\textsc{*}}}

% this volume references
% puts: [this volume]
% already defined: \citetv
%\newcommand{\citepv}[1]{(\citeauthor{#1} \citeyear*{#1} [this volume])}
\newcommand{\citealtv}[1]{\citeauthor{#1} \citeyear*{#1} [this volume]}

%parentheses around example number
\newcommand{\pref}[1]{(\ref{#1})}

% in-text examples

\newcommand{\lnex}[1]{\textit{#1}} %target lang word
\newcommand{\lnlit}[1]{(lit.: `#1')} %literal reading
\newcommand{\lnlat}[1]{(#1)} % latinization
\newcommand{\lntrans}[1]{`#1'} %translation
\newcommand{\lnexl}[2]%
{\lnex{#1}{} \lnlat{#2}} % ex with latinization
\newcommand{\lnexlat}[3]{\lnex{#1}{} \lnlat{#2}{} \lntrans{#3}} % ex with latinization and tranl.

%ch01
\newcommand{\co}[1]{\mbox{\textbf{#1}}}

%ch09

\newcommand{\cyrbulg}[1]{\begin{otherlanguage*}{bulgarian}#1\end{otherlanguage*}}


%ch10
\newcommand{\nlp}{{\small NLP}}
\newcommand{\mwe}{{\small MWE}}
\newcommand{\rae}{{\small RAE}}
\newcommand{\lvc}{{\small LVC}}
\newcommand{\pos}{{\small P}o{\small S}}
%\newcommand{\todo}[1]{ \textcolor{red}{#1} }

%\renewcommand{\labelenumi}{\theenumi}
%\ainamefmt{{vv}{ll}{, ff}{, jj}} % fullname

\newcommand{\biberror}[1]{{\color{red}#1}}

\newcommand{\osenovaitem}{--~}
   %% hyphenation points for line breaks
%% Normally, automatic hyphenation in LaTeX is very good
%% If a word is mis-hyphenated, add it to this file
%%
%% add information to TeX file before \begin{document} with:
%% %% hyphenation points for line breaks
%% Normally, automatic hyphenation in LaTeX is very good
%% If a word is mis-hyphenated, add it to this file
%%
%% add information to TeX file before \begin{document} with:
%% %% hyphenation points for line breaks
%% Normally, automatic hyphenation in LaTeX is very good
%% If a word is mis-hyphenated, add it to this file
%%
%% add information to TeX file before \begin{document} with:
%% \include{localhyphenation}
\hyphenation{
    Beck-man
    Ngu-yen
    back-chan-nel
    back-chan-nels
    mo-not-o-nous
    ste-reo-typ-i-cal
}

\hyphenation{
    Beck-man
    Ngu-yen
    back-chan-nel
    back-chan-nels
    mo-not-o-nous
    ste-reo-typ-i-cal
}

\hyphenation{
    Beck-man
    Ngu-yen
    back-chan-nel
    back-chan-nels
    mo-not-o-nous
    ste-reo-typ-i-cal
}

    \addbibresource{../collection_tmp.bib}
  \addbibresource{../localbibliography.bib}
    \togglepaper[19]
}{}


\begin{document}

\maketitle

\section{Introduction} 
\label{sec:introduction}  

As mentioned in the Introduction \parencitetv{chapters/01-Introduction}, the planar-fractal method is meant to provide a theory-neutral way to compare constituency across languages and may be used to evaluate competing linguistic theories. A model of the phonology-syntax interface, for example, should successfully predict prosodic constituents that align with and explain the phonological patterns and convergences in a given language. In Chapter 4, I identified five wordhood candidates in Kiowa (Tanoan) and found the results neatly coincide with constituents predicted by a new phase-based model \citep{millersande:2021,millersande:2023} which combines Tri-P Mapping \citep{Miller:2018,Miller:2020} and Cophonologies by Phase (\citeauthor{Sande:2017} \citeyear{Sande:Language}, \citeyear{Sande:Phonology}; \mbox{\citealt{Sande&Jenks:2018}}; \citealt{Sande:2020}). Here, I test and confirm the Combined Model's success for two other languages from this volume: Araona (Takanan) and Ayautla Mazatec (Popolocan). The results highlight several strengths of the planar-fractal method and open issues for future directions.

In Section 2, I introduce the details of the Combined Model. In Section 3, I test Tri-P's predictions against the wordhood candidates identified in each language. In each language, there is evidence for a Phonological Word ($\omega$), Phonological Phrase ($\varphi$, and Intonational Phrase ($\iota$). There is also evidence of the Prosodic Stem ($\rho$) in both languages and Constituent $\chi$ in Ayautla Mazatec, though neither constituent has been precisely defined yet under the Combined Model. In Section 4, I discuss the results and conclude.

\section{The Combined Model} 
\label{sec:models}

\subsection{Tri-P Mapping}
\label{sec:trip}

Tri-P Mapping (or Phase-based Prosodic Phonology)\footnote{The three Ps of Phase-based, Prosodic, and Phonology are abbreviated as Tri-P.} is a model of the phonology-syntax interface, which builds on the findings of \citet{Miller:2018} that current interface models \mbox{(i.e., Relational} Mapping as in \citealt{NesporVogel:1986}; \citealt{Vogel:2019}, Syntax-Driven Mapping as in \citealt{Selkirk:2011}, and Syntactic Spell-Out Approaches as in \citealt{sato:2006,Pak:2008}; \mbox{\citealt{samuels:2011}}) fall short when tested against data from languages with extreme morpho-syntactic complexity. Relational Mapping and (Direct Reference) Syntactic-Spell Out Approaches alone correctly predicted verb-internal domains in languages like Kiowa and Saulteaux Ojibwe, but neither provided full accounts for either language. Arguing a combined approach with assumptions from both models is necessary, and \citet{Miller:2018, Miller:2020} advanced such a model in Tri-P Mapping.

Tri-P Mapping uses an Indirect Reference strategy for mapping prosodic constituents from morpheme- and clause-level phases (\citeauthor{Miller:2018} \citeyear{Miller:2018}, \citeyear{Miller:2020}). Phonology may reference any spelled-out phase to map to prosodic structure, but phonology itself does not apply cyclically. This allows for domains of smaller sizes, as opposed to work like {\citet{cheng2016}} which assumes phonology applies after all Spell-Out operations. As in other Indirect Reference Spell-Out accounts (\citealt{ahn2015}; \mbox{\citealt{cheng2007}}; \citealt{compton2007}; \citeauthor{dobashi2003}  \citeyear{dobashi2003}, \citeyear{dobashi2004a}, \citeyear{dobashi2004restructuring}; \citealt{ishihara2007}; \mbox{\citealt{kratzer2007}}; \citealt{piggott2006}), mor\-pheme-level phases (those headed by a categorizing head) map to $\omega$ and clause-level phases little \textit{v}/\textsc{voice} map to $\varphi$. C's phase maps to $\iota$. Phonologically motivated restructuring may then occur including or excluding various elements within the tree.

Recursion is banned below $\varphi$, as in \citegen{Vogel:2019}'s Composite Prosodic Model. This suggests at least one intermediate constituent between $\omega$ and $\varphi$ is necessary: Constituent $\chi$. This constituent is not yet formally defined, but it is expected to be mapped referencing prosodic and not syntactic structure.

\subsection{Cophonologies by Phase}\label{sec:F}
Cophonologies by Phase (CbP) is a model of the interface between morphosyntax and phonology, which assumes late insertion of vocabulary items, spell-out at syntactic phase boundaries, and a constraint-based phonology (\citeauthor{Sande:Language} \citeyear{Sande:Language}, \citeyear{Sande:Phonology}; \mbox{\citealt{Sande&Jenks:2018}}; \citealt{Sande:2020}). The innovation of CbP is in the content of vocabulary items, or lexical items. Specifically, in addition to their phonological feature content ($\mathcal{F}$), vocabulary items also contain a prosodic subcategorization frame $\mathcal{P}$ \citep{Inkelas:1990, Paster:2006}, and a morpheme-specific constraint ranking adjustment $\mathcal{R}$ (\ref{vi}).

\ea
Example CbP vocabulary entry\label{vi}\\ 
{[$n$]} $\longleftrightarrow  \begin{Bmatrix}
  \mathcal{F}: & \text{in}  \\
 \mathcal{P}: &   [_{\omega} \ \text{X-}  \\
\mathcal{R}: & \textsc{NasalPlaceAssimilation} \gg \textsc{Ident-Place}\\
 \end{Bmatrix}$
\z

The segmental and suprasegmental content of the plural marker in (\ref{vi}) is /in-/, the prosodic subcategorization frame says it should be a prefix within a prosodic word, and the constraint adjustment tells the phonological grammar to rank \textsc{NasalPlaceAssimilation} above \textsc{Ident-Place}. In the spell-out domain containing the morpheme in (\ref{vi}), the default ranking of \textsc{Ident-Place} $\gg$ \textsc{NasalPlaceAssimilation} will be reversed, resulting in assimilation in this domain, even if assimilation is not a general process in the language. That is, similar to traditional Co-Phonology Theory \citep{Orgun:1996, Anttila:2002, Inkelas&Zoll}, there are multiple phonological rankings of constraints within the same language, which vary with the specific morphemes present in a spell-out domain. The key difference is that, in CbP, phonological evaluation applies at phase boundaries, rather than on the addition of each morpheme.

The result of adding morpheme-specific constraint ranking adjustments to vocabulary items is a specific mechanism of communication between the morphology and phonology, such that the phonology knows which grammar or cophonology to apply in a given instance of phonological evaluation. Additionally, the fact that CbP assumes spell-out at syntactic phase boundaries means that morpheme-specific effects are predicted to apply within the phase in which they are introduced, but they are not predicted to affect morphemes introduced in higher phase boundaries (\ref{pcp}).

\ea
Phase containment principle\label{pcp} \citep{Sande&Jenks:2018, Sande:2020}: \\
Morphophonological operations conditioned internal to a phase cannot affect the phonology of phases that are not yet spelled out.
\z

\noindent The phase containment principle, which is related to previous predictions of level-ordering theories and cophonologies (cf. \citealt{Inkelas&Orgun:2002}) holds of morpheme-specific constraint rankings, but also of morpheme-specific prosodic subcategorization effects.

Previous work in CbP has shown that this framework can account for mor\-pheme-specific phonological effects that apply in domains smaller than a word \citep{Sande:Language}, larger than a word \citep{Sande&Jenks:2018, Sande:2020}, competing morpheme-specific specifications within a phase \citep{Sande:2020}, category-specific effects (\mbox{\citealt{Sande&Jenks:2018}}; \citealt{Sande:2020}), and morpheme-specific phonology conditioned by two simultaneous morphological triggers within a phase domain \citep{Sande:Phonology}.

\section{Analysis}

The two languages presented and analyzed below were selected for no other reason than they were first alphabetically from the list of languages discussed in the present volume (\tabref{tab:summarytable}). The languages are unrelated genetically and aerially and thus offer an interesting test for the Combined Model. In the following subsections, I will present analyses for Aranoa (Takanan) as first analyzed by Adam Tallman in Chapter 12 and Ayautla Mazatec (Popolocan) as first analyzed by Shun Nakamoto in Chapter 5. Both languages are argued to present challenges for any prosodic analysis, but the Combined Model provides a principled account for both. I have included my own chapter's results for Kiowa (Tanoan) in the table below, though interested readers are directed to that chapter itself for the relevant analysis and discussion.

\begin{table}
    \begin{tabular}{lll}
    \lsptoprule Language & Domain & Reanalysis \\ \midrule
           Araona & Pos. 6 \textsc{Verb Core} & $\rho$? \\
            & Pos. 4--15 \textsc{Prefixes--tam} & $\omega$ \\
            & Pos. 4--17 \textsc{Prefixes--Linkage} & $\varphi$ \\
            & Pos. 1--17 \textsc{Full Clause} & $\iota$ \\
            \midrule
           Ayautla Mazatec & Pos. 19 \textsc{Stem} & $\rho$? \\
            & Pos. 15--19 \textsc{Prog--Stem} & $\omega$ \\
            & Pos. 15--28 \textsc{Prog--Pronom.} & $\chi$ \\
            & Pos. 6--28 \textsc{Ant./Post.--Pronom.} & $\varphi$ \\
            & Pos. 1--31 \textsc{Full Clause} & $\iota$ \\
             \midrule
           Kiowa\footnote{In the original chapter, there are a total of five wordhood candidates identified via convergence. The fifth candidate is not listed here, as it consists of everything but the initial pronominal in the verb complex. This seems to be a reflex of the phonological separation of the pronominal from the rest of the verb complex and is therefore unrelated to the structure itself.} & Pos. 30--34 \textsc{Stem--hsy} & $\omega$ \\
            & Pos. 30--37 \textsc{Stem--sub} & $\chi$ \\
            & Pos. 26--37 \textsc{Pronom--sub} & $\varphi$ \\
            & Pos. 2--40 \textsc{Full Clause} & $\iota$ \\ \lspbottomrule
    \end{tabular}
    \caption{Summary of phonological results}
    \label{tab:summarytable}
\end{table}

\subsection{Araona}

Tallman identifies six phonological domains that show no convergence at all. He, however, finds some convergence when including constituency tests which are indeterminate as to whether they fall under phonology or morphosyntax like {\textsc{Free Occurrence}}, {\textsc{Subspan Repetition}}, and {\textsc{Extended Exponence}}. In the end, Tallman only finds two domains that show some convergence: Pos. 4--17 ``Prefix''--Connector which is the domain for {\textsc{Maximal Pitch Accent}} and {\textsc{Maximal Free Occurrence}} domain and Pos. 4--14 ``Prefix''--TAM which is the domain for {\textsc{Minimal Subspan Repetition}} (\textit{tso} `prior'), {\textsc{Extended Exponence (Negation)}}, and {\textsc{E-selection}}. Tallman posits that we may need to ignore span convergence and instead examine the strongest structural edges. In Araona, this is the ``Prefix'' (Pos. 4) and the Core Verb Root (Pos. 6).

Tallman ultimately argues for a gradient and more fine-grained view of phonological patterns in the language  itself as well as cross-linguistically. Therefore, we should move past formalist terminology and constituents used in the literature like ``phonological word'' or the rest of the Prosodic Hierarchy. While I agree that the results do look unclear at first glance, Tri-P's independent mapping criteria give us a much clearer picture with three predicted constituents confirmed in the analysis: $\omega$, $\varphi$, and $\iota$. There is also evidence for a Prosodic Stem ($\rho$) constituent, which is yet to be formally defined in Tri-P Mapping.

First, consider the $\omega$ domain. Tri-P Mapping predicts categorial heads' phases map to their own $\omega$ and may adjust phonologically to include or exclude elements that phonologically cohere or not. For verb complexes, this typically means that the verb stem and any suffixes tend to map to a $\omega$. In Araona, there has been an apparent phonological adjustment to also include material preceding the verb stem. Inflectional prefixes, incorporated noun stems, and inflectional TAM suffixes join the verb core in the $\omega$ (Pos. 4--15 as seen in \ref{AraonaWord} below). This is the domain for E-Selection and Minimality, and it is the Maximal Subspan Repetition (Auxiliary). There is convergence with one morphosyntactic constituency diagnostic; the same subspan is the {\textsc{Maximal Ciscategorial Selection}} domain. None of these are surprising as $\omega$-level processes and properties.

\ea
    \label{AraonaWord} Araona $\omega$ Domain\footnote{This is a simplified template provided for ease of understanding. The abbreviations used combine and adjust Tallman's verbal planar structure and Pitman's analysis of the Araona verb.} \\ 
        \begin{tabular}{lllllll}
       ``Prefix''- & N- & Root & -Aspect & -Margins & -TAM \\
       4 & 5 & 6 & 7--9 & 10--13 & 14--15
    \end{tabular}
\z

It's interesting that Araona includes the ``prefixes'', which are reportedly complex morphological elements in and of themselves. Cross-linguistically, prefixes are often phonologically separate from the rest of the verb complex due to boundaries of the verb stem's $\omega$ and any intervening incorporated stems that also form $\omega$s. These boundaries don't appear to be happening in this case. Though incorporated nouns are typically not included in an $\omega$ with another root/stem, bare roots coming together into a single $\omega$ are not unattested. In Greek, for example, compounds do not form two $\omega$s to make a new, larger constituent \citep{athanasopoulou:2014}. The inflectional prefixes and bare noun roots thus seem to be included in the same domain as the verb core. Both modify the verb (part-to-whole) but are not semantically transparent for transitivity or any other syntactic process. Thus, I am comfortable assuming that the incorporated noun is included in the $\omega$ via phonological adjustment. The details of that adjustment are left to future research. %future research

The verb core itself is clearly a domain as well (Pos. 6). I posit that it forms a Prosodic Stem (abbreviated here as $\rho$), but this constituent has not been formally defined within the framework of Tri-P Mapping. Let us adopt an analysis in the spirit of \citet{downing:2015} and \citet{downing:2020}. The $\rho$ in Araona is the {\textsc{Minimal Vowel Syncope}} domain, as well as the {\textsc{Minimal Free Occurrence}} and {\textsc{Minimal Subspan Repetition (Auxiliary)}} domain. There is convergence with two syntactic constituency diagnostics: {\textsc{Minimal Non-interrup\-tability}} and {\textsc{Minimal Non-permutability}}.

\ea
 \label{AraonaPStem} Araona $\rho$ Domain = Core Verb Root (6) \\ 
\z

Tri-P Mapping predicts that a $\varphi$ will minimally consist of the little v/{\textsc{voice}} phase head's spelled-out phase. In Araona, this domain spans from the prefixes through to the auxiliary and connector at the end of the verb complex (Pos. 4--17). As expected, the language's rather free constituent order means the following XP in Position 18 is not included in the $\varphi$ domain. The $\varphi$ in Araona is the Maximal Pitch Accent Domain. There is convergence with two other constituency diagnostics: {\textsc{Maximal Free Occurrence}} and {\textsc{Maximal Non-interruptability}}.

\ea
    Araona $\varphi$ Domain \\
    \begin{tabular}{llllllllll}
       ``Prefix''- & N- & Root & -Asp. & -Margins & -TAM & -Endings \\
       4 & 5 & 6 & 7--9 & 10--13 & 14--15 & 16--17
    \end{tabular}
\z

Finally, Tri-P Mapping predicts that the entire clause will map to an $\iota$ because it is the C's phase. There is no positive evidence for the full clause (Pos. 1--17) forming a phonological domain, but it is the domain for {\textsc{Maximal Subspan Repetition}} (-tso-) and {\textsc{Maximal Ciscategorial Selection}} (broad). Though empty categories with no clear explanation are undesirable, I suspect future research will find $\iota$-level phonological patterns. This is likely a result of the types of phonological processes documented and analyzed rather than a sign there is no $\iota$ in Araona.  

\subsection{Ayautla Mazatec}

Nakamoto identifies six wordhood candidates via convergence. Candidate 1\footnote{Nakamoto refers to Candidates 1--6 and Layer 1, 4, 5, 6, 9, and 11, respectively.} consists of the verb root itself (Pos. 19). Three diagnostics converge to identify the domain, all of which are phonological ({\textsc{Minimal *ε.j}} and {\textsc{Minimal *3.24}}) or indeterminate ({\textsc{Minimal Minimum Free Form}}). Candidate 2 is comprised of all prefixes and the verb root (Pos. 15--19), and it is identified by 5 diagnostics: one is phonological ({\textsc{Minimal Sandhi-Blocking Tone Sequences}}) and two are indeterminate ({\textsc{Reduplication}} and {\textsc{Minimal Deviation from Biuniqueness}}).

Candidate 3 spans from the prefixes through to the comitative suffix (Pos. 15--20). In other words, this domain spans all non-clitic elements in the verb complex. Of the two diagnostics that converge, only one is phonological. This is the domain for {\textsc{Maximal Stress Assignment}}. Candidate 4 is just one position larger and includes the focus tonal marker (Pos. 15--21). Two phonological diagnostics converge to identify this domain: {\textsc{Maximal Sandhi-Blocking Tone Sequences}} and {\textsc{Maximal *ε.j}}. Candidate 5 spans from prefixes through all enclitics (Pos. 15--28), and it shows the highest level of convergence with 7 diagnostics; two are phonological ({\textsc{Obligatory Sandhi}} and {\textsc{Minimal Possible Sandhi}}) and two are indeterminate ({\textsc{Maximal Deviation from Biuniqueness}} and {\textsc{Maximal Minimal Free Form}}). Candidate 6 (Pos. 6--28) consists of virtually the entire verb complex. The only position excluded is the focus marker in Position 5. This domain is only identified by two morphsyntactic diagnostics, though.

Because most convergences in Ayautla Mazatec are morphosyntactic and not phonological, Nakamoto concludes that prosodic domains must not be universal as in \citet{schiering:2010}. He notes that the fine-grained differences between Candidates 1--6 often hinge on the tonal focus markers in Positions 5 and 21. Their tonal nature poses challenges for most phonological diagnostics. It is therefore separate and forms an incrementally larger domain (e.g. Candidate 4 versus Candidate 3) or left out entirely as in Candidate 6. As in the previous section, however, the Combined Approach (Tri-P Mapping and Cophonologies by Phase) provides a principled account of what we observe in Ayautla Mazatec. 

First, Tri-P Mapping predicts that the $\omega$ will coincide with the categorial verb head's phase (i.e. stem and cohering suffixes) with optional phonological adjustment. In Ayautla Mazatec, there is clear phonological adjustment as the $\omega$ consists of Pos. 19 (the Stem) and its \textit{preceding} phase (i.e. the inflectional prefixes (Pos. 15)) instead of the phase below like expected. Thus, the $\omega$ coincides with Nakamoto's Candidate 2, and it is the domain for phonological processes like {\textsc{Minimal Sandhi Blocking Tone Sequences}} and {\textsc{Minimal Deviation from Biuniqueness}}. 

\ea
    Ayautla Mazatec $\omega$ Domain \\
        \begin{tabular}{lllll}
            Prog.- & Asp./Mode- & Assoc. Motion- & Caus., Incoh.- & root(s)  \\
            15 & 16 & 17 & 18 & 19  
        \end{tabular}
\z

Recall that Nakamoto identified Candidate 3, which includes the $\omega$ plus an additional position: the comitative suffix \textit{-ko\textsuperscript{13}} in Position 20. There is indeed a clear separation between the Stem (Pos. 19) and the Comitative (Pos. 20) for {\textsc{Minimum Deviation from Biuniqueness}}, {\textsc{Total Reduplication}}, and {\textsc{Verbal Paralellism}}. In all three cases, the comititative is blocked from being involved. Additionally, Candidate 3 is identified as the domain for {\textsc{Maximal Stress Assignment}} and {\textsc{Non-permutability}}. The only phonological diagnostic here is stress assignment, but a re-analysis is possible.

Stress is predictably assigned to the verb root, but it will shift to the comitative suffix if it is present. While Nakamoto identifies the root and comitative as the minimal domain for stress assignment, the maximal domain proceeds backward until the next element that may exhibit stress (i.e. independent pronouns in Pos. 14). It is possible to re-analyze stress as a $\rho$-final process where stress is applied to a verb root. The comitative's special nature can then be captured by a morpheme-specific overwriting stress assignment or a re-bracketing process. There is no need for an additional prosodic domain.

\ea
    {Ayautla Mazatec $\rho$ Domain = Verb Root (Pos. 19)}
\z

Next, Nakamoto's Candidate 4 consists of the $\omega$, the comitative, and the focus marker in Pos. 21. This domain is associated with two phonological constituency tests: {\textsc{Maximal Sandhi Blocking Tone Sequences}} and {\textsc{Maximal *3.(2)4}}. In both cases, this domain is established by only negative evidence and no other convergence. While there is clear separation of the other enclitics, there is no way to tell whether the comitative and focus are also separated as they will never participate in either process. Thus, Candidate 4 is not actually a viable candidate and will be excluded from the present analysis. Sandhi Blocking Tone Sequences and *3.(2)4 are assumed to be restricted to $\rho$. 

Turning again to Tri--Mapping, $\chi$ is not yet formally defined but seems to support spanning processes between the $\omega$ and $\varphi$ \citep{millersande:2021}. In this language, this domain spans from the progressive prefix (Pos. 15) through the pronominal clitics (Pros. 28). Of the reported processes, only Obligatory Sandhi shows a spanning process across this domain. The remaining processes can be reanalyzed as edge-based phenomena that can be accounted for with a boundary-requirement or constraint instead of appealing to prosodic structure (e.g. pausability is likely referencing the right edge of $\varphi$). 

\ea
    Ayautla Mazatec $\chi$ Domain \\
        \begin{tabular}{llll}
            prefixes-root(s) & comitative & focus & enclitics  \\
            15--19 & 20 & 21 & 22--28 
        \end{tabular}
\z

Next, Tri-P Mapping defines the $\varphi$ as the little v or VOICE phase, which typically maps to the full verb complex.  In Ayautla Mazatec, this domain spans from the anterior/posterior prefix (Pos. 6) to the pronominal clitics (Pos. 28). The adjacent positions are an NP's focus marker on the left, and another NP is on the right. Like the comitative suffix, these positions are mentioned as ``prosodically variable" because there are few to no phonological contexts to check the domains of relevant phonological phenomena based on the shapes of the relevant morphemes. At this point, only morphosyntactic evidence converges on this domain (non-interruptability 1< and coordination min.), but that does not rule it out as a phonological domain. Future research may find a phonological phenomenon that takes place at this level of the prosodic structure. In fact, given the reanalysis above, pausability references the right edge of $\varphi$.

\ea
    Ayautla Mazatec $\varphi$ Domain \\
        \begin{tabular}{llllll}
            proclitics & adv.,pro. & prefixes.-root(s) & comitative & focus & enclitics  \\
            6--13 & 14 & 15--19 & 20 & 21 & 22--28 
        \end{tabular}
\z

Finally, the $\iota$ consists of the full clause, and it is the maximal Possible Sandhi domain. Nakamoto initially lists this as only Pos. 2--31 but there is no reason not to include position 1. It is simply never going to take part in the process, as it never includes tone 4. This is not identified by convergence, but phonological evidence may yet be found.

\section{Discussion}
The Combined Model's success in Araona and Ayautla Mazatec is only possible because of the fine-grained and comprehensive analysis via the planar-fractal method. First, justifying the planar structures and identifying each element as a zone or slot strips away theoretic decisions like morpheme type. Second, constituency diagnostics are defined precisely and may be fractured to formally account for different types of evidence that may identify subspans (e.g. positive vs. negative evidence). \citet{Miller:2018} offered a rudimentary attempt to do this by color-coding different types of evidence, but the final results became unwieldy and hard to follow. This, on the other hand, is quite elegant!

The above analysis raises issues related to convergence, however. Though the Combined Model successfully predicts the subspans in Ayautla Mazatec, most of the convergence is syntactic. In most cases, only one phonological diagnostic identifies each constituent. The fact that the Combined Model still successfully predicts the subspans provides support for convergence alone mattering, but I can see arguments against accepting such lean evidence. If two or more diagnostics of a particular type are required, we would also see issues of insufficient phonological diagnostics in order to satisfy the convergence requirement. Next, a subspan was identified in Ayautla Mazatec by two maximal fractures of tests. In other words, the subspan was identified entirely be negative evidence. This can be handled with a simple constraint that a subpsan cannot be exclusively identified by maximal fractures of diagnostics. 

In all, the planar-fractal method successfully enables cross-linguistic comparison and is suitable for testing models of the phonology-syntax interface. Future research should focus on what exactly is expected for convergence across languages.


\printbibliography[heading=subbibliography,notkeyword=this]


\end{document}
