\chapter{Appendix}

\section{RPT Experiment}


\subsection{Recorded speaker information}\label{sec:rec_spkr-information}

\tabref{rec_Speaker_RPT} shows the speaker information of the RPT experiment discussed in  \sectref{sec:RPT}.



\begin{table}
	\caption{Information of recorded speakers for the stimulus of the RPT experiment discussed in  \sectref{sec:RPT}}
	\label{rec_Speaker_RPT}
	\begin{tabular}{ccccc}
							\lsptoprule
		\textsc{id} &	\textsc{origin} & \textsc{gender} & \textsc{year of birth} &\begin{tabular}[c]{@{}l@{}}\textsc{n speech}\\ \textsc{samples}\end{tabular} \\
						\midrule
FAH     & Nalu &m   & 1964&14               \\
IRN        & Pinjan &m&1967&4                \\
RSTM      &Dadakitan & m&1966&17               \\
SNG        & Nalu &m&1940&15               \\
SP         & Nalu &m &1958&21                \\
		\lspbottomrule
	\end{tabular}
\end{table}


\subsection{Participant information}\label{sec:participant-information}

\tabref{Speaker_RPT} shows the speaker information of the RPT experiment discussed in  \sectref{sec:RPT}. The third column in the table refers to the participants' place of residence at the time of data collection, which almost always corresponds to the location where they grew up.

\begin{table}
\caption{Speaker information of participants of RPT experiment discussed in  \sectref{sec:RPT}}
\label{Speaker_RPT}
	\begin{tabular}{cccc}
						\lsptoprule
		\textsc{speaker}   & \textsc{year of birth}       & \textsc{place of living} & \textsc{gender} \\
				\midrule
		AKR  & 1990 & Nalu         & m \\
BSTN & 1976 & Nalu         & m \\
DHL  & 1988 & Nalu         & f \\
DT   & 1989 & Nalu         & m \\
DDN  & 1988 & Nalu         & m \\
FSL  & 1994 & Binontoan    & m \\
FTR1 & 1994 & Binontoan    & f \\
FTR2 & 1991 & Binontoan    & f \\
IFS  & 1986 & Binontoan    & m \\
IM   & 1972 & Kalangkangan & f \\
MG   & 2000 & Pinjan       & m \\
NRBT & 1983 & Nalu         & f \\
OCH  & 1994 & Binontoan    & f \\
RMDN & 1994 & Nalu         & m \\
RID  & 1998 & Binontoan    & f \\
RST  & 1983 & Nalu         & m \\
RDT  & 1981 & Nalu         & m \\
SRN  & 1985 & Nalu         & f \\
STDI & 1988 & Kalangkangan & m \\
WN   & 1979 & Kalangkangan & m\\
		\lspbottomrule
	\end{tabular}
\end{table}


\subsection{Instructions of boundary marking task}
\label{Instructions boundaries}

\subsubsection{Indonesian original}
\label{Instructions_b_IND}


Ketika seseorang berbicara, dia akan membagi ucapan mereka menjadi potongan-potongan. Potongan-potongan tersebut membentuk kelompok kata-kata yang memudahkan pendengar untuk memahami ucapan pembicara. Potongan-potongan tersebut penting terutama saat pembicara memproduksi ucapan yang panjang. \newline 

Contoh potongan yang mungkin Anda ketahui adalah potongan nomor ketika Anda memberi tahu nomor telepon Anda kepada orang lain. Biasanya, Anda tidak setiap kali memberi satu nomor (0, 8, 1, 3 …), tetapi Anda akan memotong nomor hp tersebut menjadi kelompok-kelompok yang terdiri atas dua, tiga, atau empat angka (081, 358, 772 …). \newline 

Untuk rekaman yang akan Anda dengar, Anda diminta untuk menandai bagian yang terdengar sebagai satu potongan. Dengan mengklik kata terakhir dari ucapan tersebut, Anda dapat menetapkan batas, yang kemudian muncul di belakang kata yang diklik. Batas antara dua potongan tidak harus sama dengan lokasi tempat Anda akan menulis tanda koma, titik, atau tanda baca lainnya. Jadi, Anda harus benar-benar hati-hati mendengar ujaran dan tandai batas yang Anda dengar sebagai akhir sebuah potongan.\newline 

Jawaban yang Anda berikan tidak ada yang salah atau benar karena semuanya bergantung pada rasa bahasa.\newline 

Jika Anda mau memperbaiki pilihan Anda, Anda dapat mengklik kata tersebut untuk kedua kalinya, dan btas yang menjadi pilihan awal Anda akan lenyap.\newline 

Sebuah potongan mungkin saja berupa satu kata, atau mungkin terdiri atas beberapa kata, dan ukuran (jumlah kata) dalam setiap potongan dari para pembicara bisa saja berbeda-beda dalam satu ujaran. Beberapa ujaran mungkin Anda dengar konsisten, yaitu terdiri atas satu potongan saja. Jika demikian, Anda tidak perlu menandai batas potongan.\newline

Anda dapat memutar setiap rekaman kalimat sebanyak dua kali. Akan tetapi, tidak memungkinkan untuk menghentikan rekaman pada saat contoh kalimat sedang diputar.\newline 
Contoh:  
\begin{center}
	
	081|358|772… \newline
	0813|5877|2… \newline
	Bapak saya | sudah datang \newline
	Bapak | saya sudah datang \newline
\end{center}
Selamat mengikuti eksperimen ini!\newline
Silahkan menandai bagian yang Anda dengar sebagai satu potongan. 
Dengan mengklik kata terakhir salah satu potongan, batas akan muncul di belakang kata yang diklik.

\subsubsection{English translation}
\label{Instructions_b_ENG}




When someone speaks, they divide their speech into segments. These segments form groups of words that make it easier for the listener to understand the speaker's message. These segments are especially important when the speaker produces long utterances. \newline 


An example of segments that you may be familiar with is the number segments when you give your phone number to someone else. Usually, you do not give the number one digit at a time (0, 8, 1, 3...), but you break up the phone number into groups consisting of two, three, or four digits (081, 358, 772...).
\newline 

For the recordings that you will hear, you are asked to mark the sections that sound like one segment. By clicking on the last word of the speech segment, you can set a boundary, which will then appear behind the clicked word. The boundary between two segments does not necessarily have to be in the same location as where you would write a comma, period, or other punctuation marks. Therefore, you must listen carefully to the speech and mark the boundary that you hear as the end of a segment.
\newline 

The answer you provided is neither right nor wrong, as it all depends on one's sense of language.
\newline 

If you wish to revise your selection, you can click on the word again, and the boundary that was your initial choice will disappear.\newline 


A segment may consist of a single word or may be made up of several words, and the size (number of words) of each segment from the speakers may vary within one utterance. Some utterances may sound consistent, consisting of only one segment. If so, you do not need to mark any segment boundaries.\newline

You can play each recorded sentence twice. However, it is not possible to stop the recording while the sample sentence is playing.\newline 
Example:  
\begin{center}
	
	081|358|772… \newline
	0813|5877|2… \newline
	Bapak saya | sudah datang (approx. ``My father already came home.") \newline
	Bapak | saya sudah datang (approx. ``Father, I already came home.")\newline
\end{center}

Enjoy the experiment!\newline
Please mark the chunks you hear as one unit.
By clicking on the last word of one of the chunk, a boundary will appear behind the clicked word.


\subsection{Instructions of prominence marking task}
\label{Instructions prominences}

\subsubsection{Indonesian original}
\label{Instructions_p_IND}
Dalam berbicara seseorang akan mengucapkan beberapa atau banyak kata dalam sebuah kalimat dengan nada yang lebih menonjol dibandingkan dengan kata-kata lain 
yang terdapat dalam kalimat tersebut. 
Kata-kata dengan nada yang menonjol ini biasanya dapat dirasakan oleh pendengarnya. 
Tugas Anda adalah menandai (mewarnai) kata-kata yang nadanya Anda dengar lebih menonjol dibandingkan dengan kata-kata lain dalam rekaman kalimat yang akan Anda putar. 
\newline

Berikut ini Anda akan diputarkan 71 kalimat. Setiap kalimat juga akan disajikan dalam bentuk tertulis. \newline

Tugas Anda adalah mewarnai semua kata yang nadanya Anda anggap lebih menonjol (mis. lebih tinggi) dibandingkan dengan kata-kata lain pada setiap rekaman kalimat yang Anda dengarkan. 
Untuk mewarnai kata, silakan menklik kata tersebut dan warnanya akan berubah menjadi merah:\newline

\begin{center}
	Dia melihat \BracksEmphasis{sapi}
\end{center}
Dalam hal ini, Anda dimungkinkan untuk memilih lebih dari satu kata pada setiap rekaman kalimat! 

\begin{center}
	Dia melihat \BracksEmphasis{sapi} dan kuda \BracksEmphasis{makan} rumput
	
\end{center}
Jika Anda mau memperbaiki pilihan Anda, Anda dapat mengklik kata tersebut untuk kedua kalinya, dan kata yang menjadi pilihan awal anda akan kembali berubah warna menjadi hitam.\newline

Anda dapat memutar setiap rekaman kalimat sebanyak dua kali. Akan tetapi, tidak memungkinkan untuk menghentikan rekaman pada saat contoh kalimat sedang diputar.\newline

Selamat mengikuti eksperimen ini!\newline

Tandai kata-kata yang terdengar  \BracksEmphasis{lebih menonjol}  untuk Anda.



\subsubsection{English translation}
\label{Instructions_p_ENG}


When speaking, individuals often emphasize certain words in a sentence through variations in tone. These prominent words are typically noticeable to the listener. Your objective is to identify and highlight (using color) the words in a recorded sentence where the speaker's tone stands out in comparison to the other words. \newline

You will hear 71 sentences. You will also be provided with each sentence as a written transcript.\newline

Your task is to color all the words that you deem to to stand out more (e.g. higher tone) compared to the other words in each recorded sentence that you listen to. You will also be provided with a written transcript of each sentence. To color a word, please click on the word and it will turn red: \newline

\begin{center}
	S/he sees \BracksEmphasis{a cow}
\end{center}

In this case, it is possible for you to choose more than one word in each recorded sentence!

\begin{center}
	S/he sees \BracksEmphasis{a cow} and a horse \BracksEmphasis{eating} grass
	
\end{center}
If you need to revise your selection, click on the word again, and it will revert back to its original color black.\newline

You can play each recording twice. It will not be possible to stop the recording while it is playing.\newline

Enjoy the experiment!\newline

Mark the words that sound   \BracksEmphasis{more prominent}   to you.


\section{Focus marking}\label{sec:prosodic-prominence}

\subsection{Recorded speaker information}\label{sec:foc_rec-information}

 \tabref{foc_rec_Speaker_RPT} shows the speaker information of the RPT experiment discussed in  \sectref{sec:Fokus}.


\begin{table}
	\caption{Information of recorded speakers for the stimulus of the Focus Marking experiment discussed in  \sectref{sec:Fokus}}
	\label{foc_rec_Speaker_RPT}
	\begin{tabular}{cccc}
			\lsptoprule
					\textsc{speaker} & 			\textsc{origin} & \textsc{gender} & \textsc{year of birth} \\
		\midrule
		AKR        & Nalu       &m&     1990   \\
		DT        & Nalu            &m&1989\\
		FTR       & Binontoan       &f&  1994   \\
		IFS        & Binontoan            &m&  1986 \\
		SP         &   Nalu              &m&1958\\
		ZHRM        & Tambun         &m&  1965     \\
			
		\lspbottomrule
	\end{tabular}
\end{table}


\subsection{Participant information}\label{sec:foc_participant-information}

 \tabref{speakers-foucs} shows the participant information of the focus marking  experiment discussed in  \sectref{sec:Fokus-perception}. The third column in the table refers to the participants' place of residence at the time of data collection, which almost always corresponds to the location where they grew up.

\begin{table}
\caption{Participant information of focus marking perception experiment discussed in  \sectref{sec:Fokus-perception}}
\label{speakers-foucs}
	
	\begin{tabular}{cccc}
		 \lsptoprule
			\textsc{speaker}   & 	\textsc{year of birth}       & 	\textsc{place of living} & 	\textsc{gender} \\
		 \midrule
		AM   & 1978 & Binontoan    & m \\
ANDR & 1997 & Dapalak      & f \\
AAL  & 1988 & Nalu         & m \\
BLW  & 1994 & Binontoan    & m \\
DAT  & 1989 & Nalu         & m \\
DWS  & 1994 & Binontoan    & f \\
EKW  & 1989 & Binontoan    & f \\
HLM  & 1986 & Binontoan    & f \\
IFS  & 1986 & Binontoan    & m \\
IRM  & 1972 & Laulalang    & f \\
IWRM & 1978 & Binontoan    & m \\
ISRW & 1999 & Gio          & f \\
JMTR & 1993 & Binontoan    & f \\
SRMN & 1986 & Nalu         & f \\
NSK  & 1980 & Binontoan    & f \\
NRM  & 1981 & Binontoan    & f \\
MRB  & 1994 & Nalu         & m \\
SRM  & 1985 & Nalu         & f \\
WN   & 1979 & Kalangkangan & m \\
YK   & 1985 & Binontoan    & m \\
		 \lspbottomrule
	\end{tabular}
\end{table}






\subsection{Stimuli}
\label{Prosodyic Prominence Stimuli}

Examples \REF{appendix-Focus Question 1}--\REF{appendix-Focus Question 9} are the QA pairs that were used in  \sectref{sec:Fokus}. 

\ea
\label{appendix-Focus Question 1}
\ea{
	\textit{inaŋ nɔŋinum sɔpa?} \\
	\gll  inaŋ nɔN-inum sɔpa \\
		  mother \textsc{av.rls-}drink what\\
	\glt `What does the mother drink?' 
}

{
	
	\ex
	\textit{inaŋ nɔŋinum \textbf{ɔɡɔ}!} \\
	\gll   inaŋ nɔN-inum ɔɡɔ\\
		   mother \textsc{av.rls-}drink water\\
	\glt `The mother drinks water.' 
}


\z
\z




\ea
\label{appendix-Focus Question 2}
\ea{
	\textit{isɛi nɔŋinum ɔɡɔ?} \\
	\gll  isɛi nɔN-inum ɔɡɔ \\
	who \textsc{av.rls-}drink water\\
	\glt `Who drinks the water?' 
}

{
	
	\ex
	\textit{\textbf{inaŋ} nɔŋinum ɔɡɔ!} \\
	\gll   inaŋ nɔN-inum ɔɡɔ\\
	mother \textsc{av.rls-}drink water\\
	\glt `The mother drinks water.' 
}


\z
\z




\ea
\label{appenix-Focus Question 3}
\ea{
	\textit{sɔpa niinum inaŋ?} \\
	\gll  sɔpa ni-inum-0 inaŋ \\
	what \textsc{rls-}drink\textsc{-uv} mother\\
	\glt `What does the mother drink?' 
}

{
	
	\ex
	\textit{\textbf{ɔɡɔ} niinum inaŋ!} \\
	\gll   ɔɡɔ ni-inum-0 inaŋ\\
	water \textsc{rls-}drink\textsc{-uv} mother\\
	\glt `The mother drinks water.' 
}


\z
\z



\ea
\label{appendix-Focus Question 4}
\ea{
	\textit{ɔɡɔ niinum isɛi?} \\
	\gll  ɔɡɔ ni-inum-0 isɛi \\
	water \textsc{rls-}drink\textsc{-uv} who\\
	\glt `Who drinks the water?' 
}

{
	
	\ex
	\textit{isɛi niinum \textbf{inaŋ}!} \\
	\gll   ɔɡɔ ni-inum-0 inaŋ\\
	water \textsc{rls-}drink\textsc{-uv} mother\\
	\glt `The mother drinks water.' 
}


\z
\z



\ea
\label{appendix-Focus Question 5}
\ea{
	\textit{sɔpa nɔlugud sɛsɛŋ?} \\
	\gll  sɔpa nɔN-lugud sɛsɛŋ \\
	what \textsc{av.rls-}chase cat\\
	\glt `Who/What chases the cat?' 
}

{
	
	\ex
	\textit{\textbf{dɛuk} nɔlugud sɛsɛŋ!} \\
	\gll   dɛuk nɔN-lugud sɛsɛŋ\\
	dog \textsc{av.rls-}chase cat\\
	\glt `The dog chases the cat.' 
}


\z
\z

\ea
\label{appendix-Focus Question 6}
\ea{
	\textit{dɛuk nɔlugud sɔpa?} \\
	\gll  dɛuk nɔN-lugud sɔpa \\
	dog \textsc{av.rls-}chase what\\
	\glt `What does the dog chase?' 
}

{
	
	\ex
	\textit{dɛuk nɔlugud \textbf{sɛsɛŋ}!} \\
	\gll   dɛuk nɔN-lugud sɛsɛŋ\\
	dog \textsc{av.rls-}chase cat\\
	\glt `The dog chases the cat.' 
}


\z
\z

\ea
\label{appendix-Focus Question 7}
\ea{
	\textit{maŋana dɔlaɡɔ nɛmɛɛnan buuk dɛi isɛi?} \\
	\gll  maŋana dɔlaɡɔ nɔN-bɛɛn-an buuk dɛi isɛi \\
	child girl \textsc{av.rls-}give-\textsc{appl} book \textsc{loc} who\\
	\glt `Who does the girl give the book to?' 
}

{
	
	\ex
	\textit{maŋana dɔlaɡɔ nɛmɛɛnan buuk \textbf{dɛi inaŋna!}} \\
	\gll   maŋana dɔlaɡɔ nɔN-bɛɛn-an buuk dɛi inaŋ=na\\
	child girl \textsc{av.rls-}give-\textsc{appl} book \textsc{loc} mother\textsc{=3.sg}\\
	\glt `The girl gives the book to her mother.' 
}


\z
\z



\ea
\label{appendix-Focus Question 8}
\ea{
	\textit{maŋana dɔlaɡɔ nɛmɛɛnan sɔpa dɛi inaŋna?} \\
	\gll  maŋana dɔlaɡɔ nɔN-bɛɛn-an sɔpa dɛi inaŋ=na \\
	child girl \textsc{av.rls-}give-\textsc{appl} what \textsc{loc} mother\textsc{=3.sg}\\
	\glt `What does the girl give to her mother?' 
}

\newpage
{
	
	\ex
	\textit{maŋana dɔlaɡɔ nɛmɛɛnan \textbf{buuk} dɛi inaŋna!} \\
	\gll   maŋana dɔlaɡɔ nɔN-bɛɛn-an buuk dɛi inaŋ=na\\
	child girl \textsc{av.rls-}give-\textsc{appl} book \textsc{loc} mother\textsc{=3.sg}\\
	\glt `The girl gives the book to her mother' 
}


\z
\z



\ea
\label{appendix-Focus Question 9}
\ea{
	\textit{isɛi nɛmɛɛnan buuk dɛi inaŋna?} \\
	\gll  isɛi nɔN-bɛɛn-an buuk dɛi inaŋ=na \\
	who  \textsc{av.rls-}give-\textsc{appl} what \textsc{loc} mother\textsc{=3.sg}\\
	\glt `Who gives the book to the mother?' 
}

{
	
	\ex
	\textit{\textbf{maŋana dɔlaɡɔ} nɛmɛɛnan buuk dɛi inaŋna!} \\
	\gll   maŋana dɔlaɡɔ nɔN-bɛɛn-an buuk dɛi inaŋ=na\\
	child girl \textsc{av.rls-}give\textsc{-appl} book \textsc{loc} mother\textsc{=3.sg}\\
	\glt `The girl gives the book to her mother' 
}


\z
\z








\subsection{Instructions}



\subsubsection{Indonesian original}

Please listen carefully.\newline You will hear two question-answer pairs. \newline Only one of them is correct!\newline Your task is to choose a compatible pair.
\newline You will hear each question-answer pair twice. \newline After the second time, you should choose the one that sounds more compatible.
\newline
Which pair is more compatible?

\subsubsection{English translation}

Tolong mendengar dengan seksama.\newline  Anda akan mendengar dua pasangan pertanyaan-jawaban. \newline  Hanya salah satunya adalah yang benar!\newline Tugas Anda adalah memilih pasangan yang cocok.
\newline Anda akan mendengar setiap pasangan pertanyaan-jawaban dua kali. \newline  Setelah kedua kalinya, Anda harus pilih salah satu yang kedengarannya lebih cocok.
\newline
Pasangan yang mana lebih cocok?

\newpage
\section{The corpus}
\label{sec:the-corpus}
~
\vspace*{-\baselineskip}
%  \tabref{Overview_conv} gives an overview of the conversational recordings.

\begin{table}[h!]
	\caption{Overview of conversational recordings}
	\label{Overview_conv}
	\begin{tabular}{lcccc}
		\lsptoprule
			\textsc{filename} & 	\textsc{speaker} & 	\textsc{gender} & 	\textsc{n IUs} & 	\textsc{duration} \\
		\midrule
		QUIS-animalgame\ & RSM, AKR & m, m & 112 & 00:03:08 \\
		spacegames  & KSR, SP & m, m & 1215 & 00:37:00 \\\midrule
		& & & 1327 & 00:40:08 \\ 
		\lspbottomrule
	\end{tabular}
\end{table}

%  \tabref{Overview_monol} gives an overview of the monological recordings.

\begin{table}[h!]
	\caption{Overview of monological recordings}
	\label{Overview_monol}
	\begin{tabular}{lcccc}
		\lsptoprule
		\textsc{filename} & \textsc{speaker} & \textsc{gender} & \textsc{n IUs} & \textsc{duration} \\
		\midrule
		explanation\_lelegesan & SYNO & m & 164 & 00:06:51 \\
		explanation\_wedding-tradition & ZBR & m & 321 & 00:16:13 \\
		pearstory & SP & m & 46 & 00:02:03 \\
		pearstory & RSTM & m & 192 & 00:06:23 \\
		pearstory & RD & f & 72 & 00:03:11 \\
		pearstory & SP & m & 51 & 00:02:33 \\
		pearstory & IRN & m & 31 & 00:01:27 \\
		pearstory & MLI & f & 44 & 00:04:22 \\
		pearstory & SNG & m & 131 & 00:05:57 \\
		pearstory & SELP & f & 70 & 00:02:38 \\
		pearstory & FAH & m & 74 & 00:03:09 \\
		QUIS-animalgame & SP & m & 89 & 00:11:09 \\
		QUIS-focus & SP & m & 41 & 00:04:37 \\
		lifestory & RDA & m & 198 & 00:07:35 \\
		story-monkey-butterfly & RSM & m & 69 & 00:02:25 \\
		story-monkey-crocodile & RSM & m & 47 & 00:01:35 \\
		story-monkey-python & RSM & m & 27 & 00:02:33 \\
		story-monkey-turtle & RSM & m & 72 & 00:02:14 \\
		story-session & MMN & f & 88 & 00:03:13 \\ \midrule
		& & & 1899 & 01:30:08 \\ 
		\lspbottomrule
	\end{tabular}
\end{table}
