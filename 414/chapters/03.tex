\chapter[Corpus-based approaches]{Corpus-based approaches to the intonation of Totoli}
\label{sec:The IU in Totoli}




%What remains to be seen is the nature of these units with regard to the grammatical structures they contain.


In this chapter, I investigate the segmentable prosodic units as they occur in the corpus of (semi-)spontaneous speech of Totoli. Based on the analysis of tonal patterns in \sectref{IU-model}, I conclude that in Totoli we have to assume recursive embedding \is{recursivity} of IUs into complex Compound IUs \is{Compound Intonation Unit} rather than IUs that are parsed into lower-level prosodic units. Hence,  the label CIU here. The CIU in my analysis is hence equivalent to the label IU in other studies  \citep[]{Croft_1995, Croft_2007, Tao_1996, Park_2002, Schuetze-Coburn1991, Schuetze-Coburn1994, Matsumoto_2003, Iwasaki1996Thai, IwasakiTao1993comparative, Wouk_2008}.  For the sake of clarity and readability, I use the label CIU for both complex CIUs that consist of several embedded IUs and also for singleton IUs (see \figref{model IU Tolitoli}).



In  \sectref{IU-Props}, I describe some of the fundamental properties of the CIU in the corpus, including their categorization, distribution, and length. Section \sectref{IU-model} presents an intonational model \is{intonational model} and discusses the tonal specifications of  boundaries of prosodic units. Finally, in \sectref{sec:the-syntax-of-intonation-units}, I investigate the prosody-syntax interface \is{prosody-syntax interface} by examining the syntactic content of CIUs as a whole (\sectref{sec:grammatical-units-and-the-intonation-unit}) and the embedded IUs of CIUs in particular (\sectref{sec:grammatical-units-and-intermediary-phrases}).



\section[Properties of the CIU in Totoli in a cross-linguistic perspective]{Properties of the Compound Intonation Unit in Totoli in a cross-linguistic perspective}
\label{IU-Props}



From the discussion in \sectref{The units of spoken speech}, it is clear that the IU is  “the spoken-language analyst’s most popular unit of choice for analysis”  \citep[841]{Croft_1995} and it is considered the basic unit structuring discourse.  Furthermore, it is locally managed and ``different sizes  of IUs are used in different interactional contexts" \citep[674]{Park_2002}. This section explores some properties of the 3226 CIUs in the corpus of Totoli  and compares these with data reported for other languages. 



\subsection{The corpus}
\label{Corpus}

\is{corpus of Totoli|(}

The corpus consists of 21 selected recordings that were segmented and  annotated by me according to the criteria described in \sectref{The units of spoken speech}. All recordings were  made using head-mounted microphones worn by the consultants, with an additional recording on the built-in camera microphone. Video and audio were recorded with a Zoom Q8 audio/video recorder with two external AKG C520 head-mounted condenser microphones at a sampling rate of 48 kHz.



I distinguish between conversational and monological data, as they are often theorized to differ substantially in various ways, such as ``information pressure" \citep[836]{du1987discourse}.
In this seciton, frequency distributions of various phenomena are displayed for the entire corpus, as well as for conversational and monological data separately. Inclusion of both types of data  ensures that certain prosodic phenomena that may occur infrequently, if at all, in either genre are accounted for. Analyzing these data categories individually allows us to investigate how various types of genres influence the frequency of tonal events and syntactic structures. By taking this approach, we can draw more nuanced and detailed conclusions about the relationship between prosodic and syntactic phenomena and discourse genre.



The corpus includes recordings from 15 speakers: 11 male and 4 female. Further information on the speakers are given in  \tabref{Overview_monol} and 	\tabref{Overview_conv} in the Appendix.  \tabref{Overview_recordings} below gives an overview of the recordings.

\begin{table}
	\caption{Overview of recordings}
	\label{Overview_recordings}
	\begin{tabular}{lcc}
		\lsptoprule
		interactivity type & $n$ CIUs & duration \\\midrule
		conversational & 1327  & 00:40:08 \\  
		monological    & 1899  & 01:30:08 \\ \midrule
		total          & 3226  & 02:10:16 \\
		\lspbottomrule
	\end{tabular}
\end{table}


%The nature of the data is described in more detail in the following sections.

%\section{Conversational data}

Read, elicited, or otherwise highly planned speech such as data obtained from a laboratory setup are not included in the corpus. I used this type of data only in the experiments on focus marking, which are described in \sectref{sec:Fokus}.

Obtaining natural conversational data while speakers are wearing head-mounted microphones is rather difficult, especially when working with endangered languages such as Totoli. To overcome this difficulty, I used two different elicitation games to obtain the conversational data:



\begin{description}
	\item [The Animal Game] is an elicitation game described in \citet[111--117]{Skopeteas.2006b} with the original purpose of eliciting narrow/contrastive focus. \is{Animal Game \citep{Skopeteas.2006b}} \is{narrow focus} \is{contrastive focus}
	A stack of cards with photos is divided equally between two speakers who take turns in describing the different pictures on the cards. I also used the game  in a monological setting with one speaker only.
	\item [Man and Tree \& Space Games] is a classic elicitation game, originally designed to explore spatial reference \is{language of space} in field settings \citep{Levinson.1992c}. \is{Man and Tree \& Space Games \citep{Levinson.1992c}} It has proven to be a very interactive task where two participants are each presented with identical sets of 12 cards displaying different items. Without seeing the interlocutor's stack of photos, both participants have to describe the relevant details of a certain card to find an exact match. Once they are certain they have found the matching one, they put the card aside. After all cards have been described, they are checked to see whether or not the cards match. Participants usually, but not obligatorily, take turns in describing. The game involves four rounds.
\end{description}



The monological data comprise recordings of various genres. All data were recorded in a face-to-face situation with a local member of the Totoli documentation team, with both parties wearing head-mounted microphones. The recordings were of different types:


\begin{description}
	\item [The Pear Story] is a short movie, designed by \citet{chafe1980pear}. \is{Pear Story \citep{chafe1980pear}} The corpus contains several recordings of retellings of the film.  Participants watch the movie first and are then asked to narrate the story-line. 
	\item [Anima] is an elicitation game described in   \citet[99--107]{Skopeteas.2006b} with the original purpose being to elicit different focus types. \is{Anima game \citep{Skopeteas.2006b}} In the game, participants are asked a set of questions about the photos they are seeing. 
	\item [The Animal Game] is an elicitation game described in \citet[111--117]{Skopeteas.2006b} for eliciting narrow/contrastive focus.\is{Animal Game \citep{Skopeteas.2006b}} \is{narrow focus} \is{contrastive focus}
	The speaker receives a stack of cards with photos on them and then simply describes the different pictures on the cards. 
	\item [Stories and folktales]  \is{folk tales} were recorded with three different speakers. These include, firstly, a recording of a lengthy account of a particularly memorable period of the speaker’s life, and, secondly, five folk tales. 
	\item [Explanatory texts] \is{explanatory texts} were obtained from two different speakers, each describing an important cultural event, namely wedding traditions and a special ritualized singing game called Lelegesan. In this game, two or more singers “spontaneously produce as many rhyming two-liners as possible”  \citep[83]{riesberg2019fern}.  \is{Lelegesan}
\end{description}










It is important to note that the language studied in this research is endangered  \is{language endangerment}  and has only few remaining speakers. In such circumstances, opportunities for data collection are limited, and an optimal  setup may not be attainable. In this study, this led to, e.g., an unavoidable imbalance between conversational and monological data, which must be taken into account when interpreting the results presented in this chapter.






\subsection{Distribution of Chafeian IU types in the corpus}
\label{sec:distribution-of-chafean-iu-types}

\is{Chafeian Intonation Unit types|(}

\citet[63--64]{chafe1994discourse} proposes a categorization scheme of IUs into different types:

\begin{description}
	\item [Substantive IUs] are those which convey events, states or referents \newline (e.g. examples	\REF{ex:dɛi dalan babi}, 	\REF{ex:douamo anu nopool}, \REF{ex:kaasikan nogiigitai mangana dolago itu sapedana nollumpak}, \REF{ex:dello enggaenggat anu dɛi ulin wi}). \is{Substantive Intonation Unit}
	\item [Regulatory IUs] regulate interaction and the flow of information  \newline (e.g. examples \REF{ex:hm}, 	\REF{ex:aa},	\REF{ex:bali}, 	\REF{ex:interactional_anu2}). \is{Regulatory Intonation Unit}
	\item [Fragmentary IUs] are unsuccessful IUs that are truncated or abandoned (e.g. \REF{ex:nisakena dɛi sapena danna <ipoa>}). \is{Fragmentary Intonation Unit}
\end{description}


Regulatory IUs are further grouped  into textual (e.g., \textit{and, then, well}), interactional (e.g., \textit{mhm, you know}),  cognitive (e.g., \textit{let me see, oh}) and validational (e.g., \textit{maybe, I thin}k) \citep[65]{chafe1994discourse}. In Totoli, frequent items of this category are discourse particles and interjections such as \textit{mh}, \textit{io}, \textit{aih}. Other items included are connectors such as \textit{bali} `so/then' and the filler element \textit{anu}.

\tabref{Distribution_Chafean} shows the distribution of the three  types in the corpus of Totoli. The category ``Rest" includes uncodable or unintelligible utterances. 



\begin{figure}
	\includegraphics
	[
	width=1\textwidth
	]
	{figures/Distribution_Chafean.png}
	\caption{Frequency distribution of CIU types}
	\label{Distribution_Chafean}
\end{figure}




The frequency distribution indicates that 71.9\% of the 3226 CIUs in the corpus are of the substantive type, while 15.9\% are regulatory and 6.6\% are fragmentary CIUs. The majority of CIUs in both settings are of the substantive type. However, the distribution of CIU types within each setting differs: the proportion of regulatory CIUs is 12.8 percentage points higher in conversational data than in monological data, and the proportion of substantive CIUs is 12.7 percentage points lower.

The frequency distribution of Chafeian IU types is rarely reported. However, in a study on Japanese  \il{Japanese} two-party conversations, \citet[50]{Matsumoto_2003} found that 81\% of the IUs were substantive, 17\% were regulatory, 0.9\% were fragmentary, and 1.1\% were uncodable.




\citet[63]{chafe1994discourse}  has suggested that this categorization ``is useful because certain aspects of an analysis can be directed at one of these types to the exclusion of the others." However, the distinction between regulatory and substantive IUs is not always clear-cut. While \citet{Croft_1995, Croft_2007} agrees that the IU is the basic discourse unit, he does not adopt the Chafeian categorization and provides his own set of criteria. Similarly, \citet[59]{Tao_1996} has stated that ``in order to avoid any arbitrary decisions, I have chosen not to discriminate between the two types of IUs but instead provide a detailed grammatical taxonomy of all IUs." For the Totoli corpus, I have utilized the Chafeian classification. Overall, I found that the vast majority of IUs were easily classifiable.


\is{Chafeian Intonation Unit types|)}

\subsection{The length of Intonation Units of the corpus}
\label{sec:the-length-of-intonation-units-of-the-corpus}

\is{length of Intonation Units|(}

The length of IUs has received a lot of attention in the literature and has been used as a central argument for the IU as cognitive unit (see  \sectref{sec:the-syntax-of-intonation-units}). Yet, little effort has been made to measure the length of an IU, as it is not a straightforward task. Several ways of measurement are conceivable. \citet[64]{chafe1994discourse} discusses the number of words an IU contains as the “simplest and most obvious measure.”


With regard to \ili{English}, one is left with several figures:

\begin{itemize}
	\item In an early publication, \citet[14]{chafe1980pear} states the average length of all IUs taken together to be about 6 words.
	\item In \citet[42]{CHAFE_1988}, he suggests that ”(t)he intonation units of ordinary spoken language show a relatively constant length [...]. In English the mean length of an intonation unit is between 5 and 6 words.”
	\item In \citet[39]{chafe1993prosodic}, he states that “one finds the modal length of regulatory units to be one word and that of substantive units to be five words.”
	\item In his account of the English Pear Story corpus, \is{Pear Story \citep{chafe1980pear}} \citet[64]{chafe1994discourse} finds Regulatory IUs to have a mean length of 1.36 words, and Substantive IUs have a mean of 4.84 words with a modal length of 4.
	\item \citet{Croft_1995}, referencing \citet[282]{altenberg1987prosodic} and \citet[256]{crystal1969prosodic}, reports counts of 4 and 6 words, respectively.
\end{itemize}


Disregarding the counts for regulatory IUs, we can summarize that an IU in English roughly contains about four to six words.  Importantly, these counts hold for \ili{English} only and languages vary with regard to the average number of words an IU contains.  



With regard to typologically diverse languages, mean numbers of words per IU vary from two to five. 

\begin{itemize}
	\item \citet[52--54]{Tao_1996} reports an average of 3.5  and a modal length of 3 words for \ili{Mandarin Chinese}.
	\item \citet[148]{chafe1994discourse} reports a modal and an average length of 2 words for \ili{English}.
	\item \citet[224]{himmelmann_IP_Universal} report average numbers of words per IU in 4 typologically unrelated languages: 5.13 for \ili{German}, 3.73 for \ili{Papuan Malay}, 3.37 for \ili{Wooi}, and 3.44 for \ili{Yali}.\footnote{Only the consensus values of  expert annotators are given here.}
\end{itemize}



Commenting on the difference in length of IUs in the four investigated languages, \citet[222]{himmelmann_IP_Universal} note that the main difference lies in the grammatical structure, but that the orthographic conventions of the languages also play a role.

\figref{dist length} shows their distribution of substantive CIUs in the Totoli corpus and gives information on their length, measured in number of words. Both conversational and monological CIUs are included. The distribution of CIUs  shows that the modal length  is 2 words in conversational data and 3 words in monological data. Monological data is  less skewed and CIUs with more than one word are much more common. 


\begin{figure}
	\includegraphics[
	width=1\textwidth]{figures/n_word_in_IU.png}
	\caption{Distribution of substantive CIUs of the corpus according to lengths in words}
	\label{dist length}
\end{figure}









Measuring the length of IUs in words has its drawbacks. As the data for Totoli show, the length of CIUs varies substantially according to the data type used, even within the same language.
Furthermore, the grammatical structure and orthographic conventions of any given language will result in different counts that may render a comparison difficult.


As \citet[222]{Fenk_Oczlon_2002} put it: 

\begin{quotation}
	In languages with a pronounced tendency to synthetic (agglutinative or
	fusional) morphology we have to expect a lower number of words per intonation
	unit (and in polysynthetic and incorporating languages even one long word that we
	would encode in a sentence comprising 5 or 6 words.)
\end{quotation}


This touches on topics of wordhood  \citep{Tallman} and alternative measures of the length of an IU have been suggested. Research on language acquisition has long focused on the length of “utterances” and on the question of what to measure 
\citep[for an overview of different measures, see][]{Allen_Dench_MLU_2015}. 




A possible alternative is to measure the number of syllables an IU contains \citep[161]{Schuetze-Coburn1994}. \citet{himmelmann_IP_Universal} propose another alternative for measuring the length of IUs in terms of content words. Probably the most straightforward way, yet also the one most susceptible to speaking style and individual speaker differences, is to measure the length of IUs in terms of duration in seconds. \citet[221]{Fenk_Oczlon_2002} cite \citet{maatta1993prosodic}, who reports an average length of a “breath group” \is{Breath Group} of about 2.1–2.2 seconds. \citet{chafe1980pear} reports a mean length of 2 seconds (incl. pauses).


I include descriptive statistics of the CIU in Totoli in terms of the different measurements – number of words, syllables and total duration – for each of the Chafeian (\citeyear{chafe1994discourse}) IU types; see  \tabref{description of IUs}.




\begin{table}
	\caption{Description of CIUs in the corpus in terms of number of words, syllables and total duration in seconds, divided over Chafeian (\citeyear{chafe1994discourse}) IU-types and data types}
	\label{description of IUs}
	\begin{tabular}{l@{~~}l *9{S[table-format=1.2]}}
		\lsptoprule
		& & \multicolumn{3}{c}{all} & \multicolumn{3}{c}{conversational} & \multicolumn{3}{c}{monological} \\\cmidrule(lr){3-5}\cmidrule(lr){6-8}\cmidrule(lr){9-11}
		& & \multicolumn{1}{c}{\begin{tabular}[c]{@{}c@{}}$n$\\ words\end{tabular}} & \multicolumn{1}{c}{\begin{tabular}[c]{@{}c@{}}$n$\\ syll\end{tabular}} & \multicolumn{1}{c}{dur} & \multicolumn{1}{c}{\begin{tabular}[c]{@{}c@{}}$n$\\ word\end{tabular}} & \multicolumn{1}{c}{\begin{tabular}[c]{@{}c@{}}$n$ \\ syll\end{tabular}} & \multicolumn{1}{c}{dur} & \multicolumn{1}{c}{\begin{tabular}[c]{@{}c@{}}$n$\\ words\end{tabular}} & \multicolumn{1}{c}{\begin{tabular}[c]{@{}c@{}}$n$\\ syll\end{tabular}} & \multicolumn{1}{c}{dur} \\ \midrule
		\multicolumn{11}{l}{substantive}  \\
		& mean & 3.81 &8.90 &1.48 &3.43 & 7.80 &1.26 &4.04 &9.54 & 1.61\\
		& median &3 &8 &1.24 &3 & 7&1.07 &3 &8 &1.35 \\
		& sd & 2.35&5.31 &0.92 &2.35 & 4.39&0.77 &1.99 &5.68 & 0.97\\ \addlinespace
		\multicolumn{11}{l}{regulatory}  \\
		& mean &1.24 &1.86 &0.52 &1.2 &1.85 &0.50 &1.3 &1.87 &0.57 \\
		& median & 1& 1&0.40 &1 &2 &0.37 &1 &1 & 0.48\\
		& sd &0.77 &1.68 & 0.61&0.55 &1.19 &0.70 &1.05 &2.31 & 0.43\\ \addlinespace
		\multicolumn{11}{l}{all}  \\
		& mean & 3.24&7.31 &1.26 &2.75 & 5.91&1.02 &3.60 &8.35 &1.45 \\
		& median & 3& 6&1.04 &2 &5 &0.84 &3 &7 &1.21 \\
		& sd & 2.32& 5.46&0.93 &1.95 &4.52 &0.80 &1.50 &5.86 &0.98 \\
		\lspbottomrule
	\end{tabular}
\end{table}


In this section, I have described some of the fundamental properties of the IU in general and the CIU in Totoli specifically. In the next section, I analyze the tonal specifications of CIUs in the corpus and propose an intonational model thereof. 



In order to conduct a corpus-based analysis, I segmented the corpus into Compound Intonation Units, for which I described the criteria in \sectref{The units of spoken speech}. Whether native and naive listeners actually perceive the segmented Compound Intonation Units as such can be answered by revisiting the results from the RPT experiment (\sectref{sec:RPT}). \tabref{Overview stimuli} in \sectref{Materials} provides an overview of the stimuli used in the RPT experiment. In the experiment, participants rated 71 speech samples, of which 26 consisted of 2 CIUs and 4 consisted of 3 CIUs, according to the segmentation criteria applied to the corpus. Hence, I collected boundary judgments for 105 words in final position of a CIU. The 71 boundary judgments for words in final position of a speech sample were discarded, resulting in 34 boundary judgments for CIU-final but not stimulus-final words.

\figref{bscore_IU} shows the b-scores (the proportion of speakers who marked the respective word as prominent or as preceding a boundary, cf.  \sectref{Analysis}) for words in CIU-final position in comparison to b-scores for words in CIU-internal position. 


\begin{figure}
	\includegraphics[width=\textwidth]{figures/IU_boundary_bscores.png}
	\caption{Comparison of b-scores of words in CIU-final and in CIU-internal position: median = 10, mean = 20.91 for words in CIU-internal position; median = 90, mean = 85.44 for words in CIU-final position; words in final position of a speech sample are excluded}
	\label{bscore_IU}
\end{figure}


The boundary scores for words that  occur in CIU-final position are substantially higher (median = 90; mean = 85.44) than those for words in CIU-internal position (median = 10; mean = 20.91). This correlation shows that what I considered a Compound Intonation Unit in the segmentation of the corpus is actually perceived as a unit, and thus confirms the viability of units obtained from the corpus segmentation. 

\is{corpus of Totoli|)}

\is{length of Intonation Units|)}

\section[Intonational model]{An intonational model of the Compound Intonation Unit in Totoli}
\label{IU-model}

\is{intonational model|(}
\is{phrase tone|(}
\is{boundary tone|(}


Having described the corpus in detail, I will now turn to the study of Totoli intonation.  This study focuses on the Compound Intonation Unit, which is -- to repeat \citeauthor{Chafe_1987}'s (\citeyear[22]{Chafe_1987})  definition of an Intonation Unit -- a ``sequence of words combined under a single, coherent intonation contour, usually preceded by a pause."	




In this section, I propose an intonational model of the CIU in Totoli. The model is couched in  the autosegmental-metrical  framework and the ToBI framework  \citep{arvaniti2020autosegmental, jun2006, jun_2014, Ladd_2008} and assumes  singleton Intonation Units or Compound Intonation Units. The model takes up \citeauthor{Ladd_2008}'s (\citeyear[297; chapter 8.2]{Ladd_2008}) notion of the Compound Prosodic Domain (CPD), \is{Compound Prosodic Domain} i.e. strings of IUs  which are recursively embedded and together form a CIU.  Singleton IUs or CIUs  are strings of words that are combined under a coherent intonation contour, usually preceded by a pause \citep[22]{Chafe_1987}. They are perceived as such by listeners due to a complex interplay of cues mainly pertaining to pitch, rhythm and voice quality (\citealt[93--155]{Schuetze-Coburn1994}, \citealt[260--270]{himmelmann2006challenges}, \citealt{dubois1992, dubois1993}, and \citealt[29--39]{cruttenden1997intonation}). Tonal specifications are assigned at the level of the IU, and they are associated with their right-edge boundary and consist of a bitonal edge-tone complex. The right-edge of  an IU -- a singleton IU or an embedded IU of a CIU -- is demarcated by one of the three proposed boundary tones (see  \sectref{sec:tonal-events-at-the-boundaries-of-ius} and \sectref{sec:tonal-events-at-the-boundaries-of-ips}). In a CIU, only the last embedded IU is followed by typical final cues, such as e.g. pause and pitch reset \citep[217]{Schuetze-Coburn1991}. Tonal specifications of singleton IUs and embedded IUs are equal and vary only with regard to their frequency distribution (see   \sectref{sec:discussion}). 

\is{recursivity}

The intonational model for  singleton IUs and CIUs in Totoli is shown in  \figref{model IU Tolitoli}. It brings together insights from the two experiments described in \chapref{sec:Experiments} and findings from a large scale investigation of tonal events and syntactic content of the prosodic units presented in the remainder of this chapter (\sectref{sec:tonal-events-at-the-boundaries-of-ius}--\sectref{sec:prosodic-clitics}).



\begin{figure}
	\caption{The CIU in Totoli: A singleton IU on the left-hand side, recursively embedded IUs into a Compound Intonation Unit (CIU) on the right-hand side.}
	\label{model IU Tolitoli}
	\begin{tikzpicture}[scale=0.7]
		%\draw[style=help lines] (0,0) grid (15,5);
		
		%L-H%
		\draw  [very thick] (-4.5,-2) --  (-4.5,5.5) --  (12.5,5.5) --  (12.5,-2) --  (-4.5,-2);
		
		%syllable signs
		\node (CIUa) at (7.1,5) {CIU};
		
		\node (IU1a) at (-2.8,3.5) {IU};
		\node (CIU1a) at (4,3.5) {IU};
		\node (CIU2a) at (7.1,3.5) {IU};
		\node (CIU3a) at (10.2,3.5) {IU};
		
		\node (IUqb) at (-2.8,1.5) {{[}σ σ σ σ{]}\#};
		\node (CIU1b) at (4,1.5) {{[}σ σ σ σ σ{]}};
		\node (CIU2b) at (7.1,1.5) {{[}σ σ σ σ σ{]}};
		\node (CIU3b) at (10.2,1.5) {{[}σ σ σ σ σ{]}\#};
		
		\draw[] (7.1,4.7) -- (4,3.8);
		\draw[] (7.1,4.7) -- (7.1,3.8) ;
		\draw[] (7.1,4.7) -- (10.2, 3.8);
		
		\node (T0) at (-2.2,2.2) {T-T\%}; 
		\node (T1) at (4.9,2.2) {T-T\%}; 
		\node (T2) at (8,2.2) {T-T\%};	
		\node (T3) at (11.1,2.2) {T-T\%};	
		\node [right] (1st) at (0,-0.7) {T\%};	
		\node [right] (1st) at (0,-1.4) {T-};	
		\node  [right] (1st) at (1,-0.7)  {= boundary tone};	\is{phrase tone}
		\node [right] (1st) at (1,-1.4) {= phrase tone};	\is{phrase accent}
	\end{tikzpicture}
\end{figure}





The model analyzes IU-final pitch events as bitonal boundary-tone complexes, consisting of a phrase tone (T-) and a boundary tone (T\%). In  \sectref{sec:tonal-events-at-the-boundaries-of-ips}, I show that the right-edge boundary of an embedded IU of a CIU is not merely classified by a single boundary tone such as an H\% or T\%.    Instead, I propose a set of three different final boundary-tone complexes that are essentially the same as those occurring at the right-edge boundary of the last IU of a CIU or a singleton IU respectively.

In the model, I use the more theory-neutral label \textit{phrase tone} instead of \textit{\isi{phrase accent}} that is typically anchored to a metrically strong syllable \citep{grice_ladd_arvaniti_2000}. The alignment of this tone is roughly the prefinal syllable but lacks near-constant timing  \citep[see][356]{Maskikit_Essed_2016}. This will become obvious from the  discussion in the following sections \sectref{sec:tonal-events-at-the-boundaries-of-ius} and \sectref{sec:tonal-events-at-the-boundaries-of-ips}. 


The IU -- singleton IUs and embedded IUs of CIUs -- regularly maps onto syntactic or grammatical units, such as a subject or object NP, a verb or a VP, an adverbial phrase or a complement clause. This observation will be discussed in  \sectref{sec:the-syntax-of-intonation-units}.



The only tonal event in singleton IUs is the obligatory final boundary-tone complex. In CIUs, each embedded IU is marked by a final boundary-tone complex. The syntactic content is decisive in whether a construction is uttered as a singleton IU or a CIU consisting of several embedded IUs (see  \sectref{sec:the-syntax-of-intonation-units}).  Hence, the difference between embedded IUs of CIUs and non-embedded IUs -- singleton IUs or final IUs of CIUs -- lies in their co-occurrence with other boundary phenomena such as pitch reset, final lengthening, pauses and glottalization. In the remainder of the work, I will use CIU as a cover term to refer to both singleton IUs and Compound IUs  -- hence, those which co-occur with other boundary phenomena -- as opposed to embedded IUs of CIUs. 



An important observation is that the tonal marking at the right-edge boundary of singleton IUs, embedded non-final IUs of CIUs, and final IUs of CIUs is essentially the same, as demonstrated in the following sections \sectref{sec:tonal-events-at-the-boundaries-of-ius} and \sectref{sec:tonal-events-at-the-boundaries-of-ips}.






\subsection{Tonal events at the right-edge boundaries of CIUs}
\label{sec:tonal-events-at-the-boundaries-of-ius}

In this section, I discuss tonal events occurring at the right-edge boundary of CIUs, i.e., the final IUs of CIUs and singleton IUs respectively, as exemplified in \figref{singleton and finals of CIUS}.

\begin{figure}[h]
	\caption{Visualization of final IUs of CIUs and singleton IUs discussed in this section}
	\label{singleton and finals of CIUS}
	\begin{tikzpicture}[scale=0.7]
		%\draw[style=help lines] (0,0) grid (15,5);
		
		%L-H%
		\draw  [very thick] (-4.5,1) --  (-4.5,5.5) --  (12.5,5.5) --  (12.5,1) --  (-4.5,1);
		
		%syllable signs
		\node (CIUa) at (7.1,5) {CIU};
		
		\node  (IU1a) at (-2.8,3.5) {IU};
		\node [lightgray] (CIU1a) at (4,3.5) {IU};
		\node [lightgray] (CIU2a) at (7.1,3.5) {IU};
		\node (CIU3a) at (10.2,3.5) {IU};
		
		\node (IUqb) at (-2.8,1.5) {{[}σ σ σ σ{]}\#};
		\node [lightgray] (CIU1b) at (4,1.5) {{[}σ σ σ σ σ{]}};
		\node [lightgray] (CIU2b) at (7.1,1.5) {{[}σ σ σ σ σ{]}};
		\node (CIU3b) at (10.2,1.5) {{[}σ σ σ σ σ{]}\#};
		
	%	\draw[] (-2.8,4.7) -- (-2.8,3.8);
		\draw[] [lightgray] (7.1,4.7) -- (4,3.8);
		\draw[] [lightgray] (7.1,4.7) -- (7.1,3.8) ;
		\draw[] (7.1,4.7) -- (10.2, 3.8);
		
		\node (T0) at (-2.2,2.2) {T-T\%}; 
		\node [lightgray] (T1) at (4.9,2.2) {T-T\%}; 
		\node [lightgray] (T2) at (8,2.2) {T-T\%};	
		\node (T3) at (11.1,2.2) {T-T\%};	
		
		
	\end{tikzpicture}
\end{figure}



Tonal events at the right-edge boundary of CIUs and singleton IUs can be classified into a set of three different tonal contours. To classify and annotate their pitch contours, I visually inspected all IUs. The summarizing plot of these is shown in \figref{spaghetti plot-IU-final}. The three contours are explained in the following.

\begin{figure}
	\includegraphics[
	%height=0.3\textheight,
	width=0.6\textwidth]{figures/smoothed_IU_final.png}
	\caption{Spaghetti plot showing the  intonation contours of the final three syllables of each of the three final boundary-tone complexes with the items superimposed on each other and with an average contour produced by \textit{Loess} smoothing in R \is{R software} \citep{R_manual}. Vertical bars indicate syllable boundaries; only CIUs with final $CV.CV.CV]_{CIU}\#$ syllable structure are displayed; values are z-transformed for CIU.}
	\label{spaghetti plot-IU-final}
\end{figure}


In the model proposed here, the final boundary-tone complex   is analyzed as consisting of a phrase tone (T-) and a boundary tone (T\%). The phrase tone can be a low tone L-, a high tone H- or a rising tone LH-.  The boundary tone can be either a low tone L\% or a high tone H\%.  \figref{boundary tones} depicts schematic versions of the three possible combinations of a phrase tone and a boundary tone, including two rising patterns and one falling pattern.









\begin{figure}
	\caption{Schematic representation of CIU-final boundary-tone complexes: vertical bars indicate syllable boundaries.}
	\label{boundary tones}
	
	\begin{tikzpicture}[scale=0.7]


		
		\node [draw] at (2.75,4.5) 	 [above] {\textcolor{white}{(}L-H\%};
		\node [draw] at (8.75,4.5) 	[above] {(L)H-L\%};
		\node [draw] at (14.75,4.5)	[above] {\textcolor{white}{(}LH-H\%};
		
		%L-H%
		\draw (0,-1) -- (0,4.5) -- (5.5,4.5) -- (5.5,-1) -- (0,-1);
		
		\draw [color={rgb:black,1;white,10}] (1,3) -- (1,1) ;
		\draw [color={rgb:black,1;white,10}] (2,3) -- (2,1) ;
		\draw [color={rgb:black,1;white,10}] (3,3) -- (3,1) ;
		\draw [color={rgb:black,1;white,2}] (4,3) -- (4,1) ;
		\draw [color={rgb:black,1;white,2}] (4.05,3) -- (4.05,1) ;
		
		%Nodes:
		\node (start) at (0,2) {};
		\node (L) at (3,0.7) {L-};
		\node (H) at (4.2,3.4) {H\%};
		\draw [ultra thick] plot [smooth, tension=0.6] coordinates {(0,1.8)  (3,1.2)  (3.7,2.9) (4,3.1)}; 
		
		
		
		%syllable signs
		\node (1st) at (0.5,0) {...σ};
		\node (1st) at (1.5,0) {σ};
		\node (1st) at (2.5,0) {σ};
		\node (1st) at (3.5,0) {σ};
		\node (1st) at (4.7,0) {]\textsubscript{CIU}\#}; 

		
		%L(H)-L% 
		\draw (6,-1) -- (6,4.5) -- (11.5,4.5) -- (11.5,-1) -- (6,-1);
		
		\draw [color={rgb:black,1;white,10}] (7,3) -- (7,1) ;
		\draw [color={rgb:black,1;white,10}] (8,3) -- (8,1) ;
		\draw [color={rgb:black,1;white,10}] (9,3) -- (9,1) ;
		\draw [color={rgb:black,1;white,2}] (10,3) -- (10,1) ;
		\draw [color={rgb:black,1;white,2}] (10.05,3) -- (10.05,1) ;
		
		%Nodes:
		\node (start) at (6,2) {};
		\node (H) at (8.5,3.4) {\textcolor{gray}{(L)}H-};
		\node (L) at (9.9,0.7) {L\%};    
		\draw [ultra thick] plot [smooth, tension=0.6] coordinates {(6,1.5) (6.5,1.7)  (9.1,3)  (10,1)}; 
		\draw [gray, dashed, thick] plot [dashed,smooth,tension=0.6] coordinates {(6,1.5) (8,1.5) (8.6,2.8) (9.2,3)};   
		
		
		%syllable signs
		\node (1st) at (6.5,0) {...σ};
		\node (1st) at (7.5,0) {σ};
		\node (1st) at (8.5,0) {σ};
		\node (1st) at (9.5,0) {σ};
		\node (1st) at (10.5,0) {]\textsubscript{CIU}\#}; 
		

		%H-H% 
		\draw (12,-1) -- (12,4.5) -- (17.5,4.5) -- (17.5,-1) -- (12,-1);
		
		\draw [color={rgb:black,1;white,10}] (13,3) -- (13,1) ;
		\draw [color={rgb:black,1;white,10}] (14,3) -- (14,1) ;
		\draw [color={rgb:black,1;white,10}] (15,3) -- (15,1) ;
		\draw [color={rgb:black,1;white,2}] (16,3) -- (16,1) ;
		\draw [color={rgb:black,1;white,2}] (16.05,3) -- (16.05,1) ;
		
		
		%Nodes:
		\node (start) at (12,2) {};
		\node (H) at (14.5,3.4) {LH-};
		\node (H) at (16.3,3.4) {H\%};
		
		\draw  [ultra thick] plot [smooth,tension=0.6] coordinates {(12,1.5) (14,1) (14.6,2.8) (15.2,3) (16,2.8)};   
		
		
		%syllable signs
		\node (1st) at (12.5,0) {...σ};
		\node (1st) at (13.5,0) {σ};
		\node (1st) at (14.5,0) {σ};
		\node (1st) at (15.5,0) {σ};
		\node (1st) at (16.6,0) {]\textsubscript{CIU}\#}; 
		

		
	\end{tikzpicture}
	
\end{figure}




The two rising patterns are rather similar and their main difference is the domain of the pitch rise, or the low tone L- respectively. However, they are clearly distinct in their function, as I show below (cf. \tabref{functions tonal patterns IU}).

In the following section, I will provide a brief summary of each contour and illustrate them with examples from the corpus. After that, I will discuss their functions and distribution in the corpus. For the purpose of exemplification, only singleton IUs will be displayed in the figures below.

\subsubsection{The L-H\% boundary-tone complex}

\is{boundary tone complex|(}

An instance of the L-H\% boundary-tone complex is given in the periogram   of example \REF{ex:daan tooka nemenek isia lau memenek naasyik lau monipu} in  \figref{pitch:daan tooka nemenek isia lau memenek naasyik lau monipu}. 

Pitch starts around the middle of the speaker's current range. Over the initial 15 syllables, pitch remains near level and then  drops 4 st   towards the low target of the L- phrase tone located at the boundary between the penultimate (\textit{.ni.}) and the ultimate syllable (\textit{.pu\#}) of the IU. On the last syllable, pitch rises almost 19 st to the high target of the H\% boundary tone.  

\begin{figure}
	\includegraphics[
	height=0.3\textheight
	%,
	%width=1\textwidth
	]{figures/pearstory_36_SELP_extSELP_daan_tooka_nemenek_isia_lau_memenek_naasyik_lau_monipu_plot.png}
	\caption{Periogram with pitch track (in st) for example \REF{ex:daan tooka nemenek isia lau memenek naasyik lau monipu}, with final boundary-tone complex L-H\%, speaker SELP}
	\label{pitch:daan tooka nemenek isia lau memenek naasyik lau monipu}
\end{figure}


\newpage
\ea
\label{ex:daan tooka nemenek isia lau memenek naasyik lau monipu}
\textit{daan tɔɔka nɛmɛnɛk isia lau mɛmɛnɛk naaʃik lau mɔnipu} \\
\gll daan tɔɔka nɔN-pɛnɛk isia lau mɔN-pɛnɛk nɔ-aʃik lau mɔN-tipu \\
later finished \textsc{av.rls}-climb 3\textsc{s} presently \textsc{av-}climb \textsc{st.rls}-busy presently \textsc{av}-pick\\
\glt `after he climbed; eagerly picking (pears)'\hfill(pearstory\_36\_SELP.015)
	\osflink{65a55062a246ff0ac2dd3e1b}{pearstory_36_SELP_extSELP_daan_tooka_nemenek_isia_lau_memenek_naasyik_lau_monipu.wav}
\z

Speaker SELP shows a very high pitch range, especially on the final syllable (\textit{.pu\#}). This is not uncommon in the corpus and is frequently observable when speakers are very engaged in their conversation or narration (cf. example \REF{ex:kaasikan nogiigitai mangana dolago itu sapedana nollumpak}) /  \figref{pitch:kaasikan nogiigitai mangana dolago itu sapedana nollumpak}, 
example \REF{ex:molitenggean wi} /
 \figref{pitch:molitenggean wi},
example \REF{ex:dello enggaenggat anu dɛi ulin wi} / \figref{pitch:dello enggaenggat anu dɛi ulin wi}).



\subsubsection{The (L)H-L\% boundary-tone complex}

I analyze the combinations of an LH- or H- phrase tone with the boundary tone L\% as variations of the same tonal complex, referred to here as (L)H-L\%. The difference resides in the domain of the pitch rise, as indicated by the dashed line in   \figref{boundary tones}. In the LH-L\% variant, the main domain for the pitch rise is the penultimate syllable. In the {H-L\%} variant, on the other hand, the rise in pitch may extend over several syllables. 

Consider first the LH-L\% variant, which is exemplified by the pitch contour of example  \REF{ex:tauna dɛi anu baduna} in  \figref{pitch:tauna dɛi anu baduna}. Pitch begins around the middle of the speaker’s current range and remains near flat with a slight downtrend over the first 7 syllables of the IU. Starting at the beginning of the penultimate syllable (\textit{.du.}), pitch rises 10 st to the high target of the  LH- phrase tone at the beginning of the vowel of the last syllable (\textit{.na\#}). Pitch then drops 12 st to the low target of the  L\%  of the IU boundary tone.





\begin{figure}
	\includegraphics[
	height=0.3\textheight
	%,
	%width=1\textwidth
	]{figures/tau_na_dei_anu_baduna_PearStory_12_RSTM_plot.png}
	\caption{Periogram with pitch track (in st) for example \REF{ex:tauna dɛi anu baduna}, with final boundary-tone complex LH-L\%, speaker RSTM}
	\label{pitch:tauna dɛi anu baduna}
\end{figure}

\ea
\label{ex:tauna dɛi anu baduna}
\textit{tauna dɛi anu baduna} \\
\gll tau-0=na dɛi anu badu=na \\
put\textsc{-uv}\textsc{=3s.gen} \textsc{loc} \textsc{fill} shirt=3\textsc{s.gen}\\
\glt ‘it is fixed at his whatchamacallit shirt’ \hfill(pearstory\_12\_RSTM.055)
	\osflink{65a5507c1cba3e0c1c0183db}{tau_na_dei_anu_baduna_PearStory_12_RSTM.wav}
\z




The H-L\% boundary-tone complex is illustrated by   \REF{ex:saasalu koloannako}, for which the periogram is shown in   \figref{pitch_saasalu koloannako}. The rise to the high target does not occur on the penultimate syllable exclusively, but instead happens gradually.  Hence, the contour is labeled H-L\%.


The pitch contour in  \figref{pitch_saasalu koloannako} shows that pitch begins mid-level of the speaker’s current range and gradually rises about 4 st over the initial 6 syllables of the IU. The pitch rise reaches the high target  of the H- phrase tone at the beginning of the vowel of the ultimate syllable (\textit{.kɔ\#}) and then drops about 15 st to the low target of the L\% boundary tone. 





\begin{figure}
	\includegraphics[
	height=0.3\textheight
	%,
	%width=1\textwidth
	]{figures/spacegames_sequence4_KSR-SP_extSP_saasalu_kololannako_plot.png}
	\caption{Periogram with pitch track (in st) for example \REF{ex:saasalu koloannako}, with final boundary-tone complex H-L\%, speaker SP}
	\label{pitch_saasalu koloannako}
\end{figure}

\ea
\label{ex:saasalu koloannako}
\textit{saasalu kɔlɔannakɔ} \\
\gll \textsc{rdp-}salu kɔlɔan=na=kɔ \\
\textsc{rdp-}to.face right=3s\textsc{.gen}\textsc{=and}\\
\glt ‘it is facing to the right side’ \hfill(spacegames\_sequence4\_KSR-SP.035)
	\osflink{65a550761cba3e0c180185c6}{spacegames_sequence4_KSR-SP_extSP_saasalu_kololannako.wav}
\z



\subsubsection{The LH-H\% boundary-tone complex}

An instance of the LH-H\% boundary-tone complex is found in example  \REF{ex:cremonies}, for which the periogram shown in  \figref{pitch_ceremonies}. Speaker ZBR enumerates several cultural events and festivities. Similar to example \REF{ex:saasalu koloannako}, the domain for the pitch rise is the penultimate syllable. On the last syllable, pitch remains (near) high. Similar to the L-H\% above, the pitch contour shows a slight dip towards the end of the IU. This even visible in the summarizing spaghetti plot in  	\figref{spaghetti plot-IU-final} and can be interpreted as an anticipation of the following low tone. 

\begin{figure}
	\includegraphics[
	height=0.3\textheight
	%,
	%width=1\textwidth
	]{figures/explanation-wedding-tradition_ZBR_extZBR_mangabing_monggulang_ballamate_plot.png}
	\caption{Periogram with pitch track (in st) for example \REF{ex:cremonies}, with final boundary-tone complexes LH-H\%, speaker ZBR}
	\label{pitch_ceremonies}
\end{figure}



\ea
\label{ex:cremonies}
\ea{
	\label{ex:mangabing}
	\textit{maŋabing} \\
	\gll mɔN-kabing \\
	\textsc{av-}marry\\
	\glt `to marry'
}

\ex{
	\label{ex:monggulang}
	\textit{mɔŋɡulaŋ} \\
	\gll mɔN-ɡulaŋ \\
	\textsc{av-}to.cradle \\
	\glt `to cradle'
}


{
	\ex
	\label{ex:ballamate}
	\textit{ballamatɛ} \\
	\gll  ballamatɛ \\
	funeral.ceremony \\
	\glt `the funeral ceremony'
	\begin{flushright}(explanation-wedding-tradition\_ZBR.265-267)
		\osflink{65a5500d1c92110a81abea57}{explanation-wedding-tradition_ZBR_extZBR_mangabing_monggulang_ballamate.wav}\end{flushright}
}
\z
\z




While the LH-H\% boundary-tone complex and the L-H\% boundary-tone complex share some similarities, the main difference lies in the pitch rise domain.  The two patterns can be easily differentiated from each other.


\subsubsection{Distribution}

The three boundary-tone complexes are the main tonal events occurring at the right-edge boundary of CIUs, i.e. singleton IUs and final IUs of CIUs. In addition to final boundary-tone complexes, there are rarely occurring discourse particles which attach to one of the boundary-tone complexes. They are not included in  	\figref{Freq_IU_type} but are described in   \sectref{sec:prosodic-clitics}. 


The frequency distribution of the different final boundary-tone complexes over the different data types is shown in  \figref{Freq_IU_type}. Clearly evident is the fact that the (L)H-L\% and the L-H\% boundary-tone complexes are the two major patterns. The LH-H\% pattern is a minor pattern but is very distinct in its function, as I will show in the next section (\sectref{sec:function}). 

\begin{figure}
	\includegraphics[
	%height=0.3\textheight, 
	width=1\textwidth]{figures/freq_IU_types.png}
	\caption{Frequency distribution of tonal events at the right edge of substantive CIUs within conversational and monological recordings, numbers are rounded to  one decimal place.}
	\label{Freq_IU_type}
\end{figure}



The frequency distribution within the two data types is similar, with both types attesting more (L)H-L\% boundary tones than L-H\%. The distribution is more heavily skewed in conversational data, in which 60.1\% of CIUs occur with the (L)H-L\% boundary tone and 29.3\% with the L-H\%. This trend is less pronounced in the monological data, where there is a difference of only 10 percentage points. Furthermore, monological data show more final LH-H\%.

\subsubsection{Function}
\label{sec:function}

The difference in distribution can be explained by the different functions of CIU-final boundary-tone complexes. These are summarized in  \tabref{functions tonal patterns IU}.

\begin{table}
	\caption{Summary of functions of CIU-final boundary-tone complexes}
	\label{functions tonal patterns IU}
	\begin{tabular}{cc}
		\lsptoprule
		\begin{tabular}[c]{@{}l@{}}CIU-final boundary-tone  complex\end{tabular} & function                                                              \\
		\midrule
		L-H\%                                                                      & continuation                                                          \\
		(L)H-L\%                                                                   & finality                                                              \\
		LH-H\%                                                                     & \begin{tabular}[c]{@{}l@{}}non-final elements of lists\end{tabular} \\
		\lspbottomrule
	\end{tabular}
\end{table}


Monologues yield a higher proportion of CIUs with the LH-H\% pattern because, in descriptive texts in the corpus, speakers frequently enumerate referents, events or entities; see example \REF{ex:cremonies} above.
In lists, especially for non-final elements of lists, the LH-H\% is the preferred intonation pattern. However, the pattern is not exclusively reserved for such CIUs but can also occur in other environments where the speaker wants to express a high degree of continuation. Similarly, (non-final) elements of lists can also be uttered with the L-H\% when signaling less strong continuation.



The distributional differences between the boundary-tone complexes observed in \figref{Freq_IU_type} support the proposed functions in the following way: The higher proportion of CIUs with final (L)H-L\% in conversations as compared to monologues reflects the fact that in conversations a paragraph may consist of a question and an answer only, while in monologues narrations are organized into longer paragraphs, containing several CIUs.


The function of (L)H-L\% as signaling finality and L-H\% as signaling continuation  is nicely illustrated by \textit{Tail-Head Linkage} (THL) \is{Tail-Head Linkage} constructions in narrations. \citeauthor{THLVries2005} (\citeyear[262]{THLVries2005}) defines THL as 

\begin{quote}
	[...] a way to connect clause chains in which the last clause of a chain is partially or completely repeated in the first clause of the next chain.
\end{quote}  


In Totoli, instances of THL occur mostly in unplanned narratives and contribute to what \citeauthor{THLVries2005} (\citeyear[378]{THLVries2005}) calls \textit{processing} \textit{ease}, as it links paragraphs and maintains event coherence. At the same time, they serve as a planning device, allowing speakers more time to plan the next paragraph. In the corpus, recordings of retellings of the Pear Story \is{Pear Story \citep{chafe1980pear}} yield a considerable number of instances of THL constructions, as speakers are given the task to extemporize a coherent account of the story-line of a previously unknown story. \is{Tail-Head Linkage}
In the Pear Story retellings, CIUs are usually grouped into higher-level units above the CIU which may be termed ``paragraphs"  \citep[see][251]{himmelmann2008prosodic}. A paragraph consists of a series of CIUs ending on L-H\% and concludes with a final CIU marked by the (L)H-L\% pattern. In a THL construction, the final CIU -- the Tail of a paragraph -- is repeated in full or in part as Head of the subsequent paragraph. The excerpt of a Pear Film retelling in example \REF{bali tau pagauan}--\REF{ingga daan noosa kaddanmai mangngana saasapeda} provides an illustration.

\is{Tail-Head Linkage}

\ea
\label{ex:THL}
\ea{
	\label{bali tau pagauan}
	\textit{bali tau paɡauan} \hfill L-H\%\\
	\glt `So a gardener ...'
}

\ex{
	\label{kononipu alpukaatna}
	\textit{<na> kɔnɔnipu alpukaatna} \hfill L-H\%\\
	\glt `... is picking the avocados,'
}
\ex{
	\label{nipenekanna dɛi batangna}
	\textit{nipɛnɛkanna dɛi bataŋna} \hfill L-H\% \\
	\glt `he is climbing up the trunk,'
}
\ex{
	\label{niambinnamai uliai babo}
	\textit{niambinnamai uliai babɔ}\hfill L-H\%\\
	\glt ‘he is getting (the avocados down) from the top,’
}

\ex{
	\label{sagaat nadabumai dɛi buta}
	\textit{saɡaat nadabumai dɛi buta} \hfill L-H\%\\
	\glt ‘half (of the avocados) fell to the ground,’
}
\ex{
	\label{bai injan nakaalamai}
	\textit{bai inʤan nakaalamai} \hfill L-H\%\\
	\glt ‘and then after (he) picked (them) up,’
}
\ex{
	\label{ninauna poniai <moi> nitauna dɛi karanjang}
	\textit{ninauna pɔniai <mɔi> nitauna dɛi karanʤaŋ} \hfill L-H\%\\
	\glt ‘he brought them there and put them in the basket,’
}
\ex{
	\label{dɛi llengget}
	\textit{dɛi llɛŋɡɛt} \hfill L-H\%\\
	\glt ‘in the hamper,’
}
\ex{
	\label{kaddaan tau}
	\textit{kaddaan tau} \hfill L-H\%\\
	\glt ‘(then) there was a person,’
}
\ex{
	\label{notumalibko}
	\textit{nɔtumalibkɔ} \hfill L-H\%\\
	\glt ‘(he) passed by,’
}
\ex{
	\label{biibindas toolang1}
	\textit{biibindas tɔɔlaŋ} \hfill \textbf{H-L\%}\\
	\glt ‘(and he) pulled a goat.’
}
\ex{
	\label{biibindas toolang2}
	\textit{biibindas tɔɔlaŋ} \hfill \textbf{L-H\%}\\
	\glt ‘(Though the person) pulled a goat,’
}
\ex{
	\label{tapi ganega tumalibko}
	\textit{tapi ɡanɛɡa tumalibkɔ} \hfill H-L\%\\
	\glt ‘he only passed by.'
}
\ex{
	\label{ingga daan noosa kaddanmai mangngana saasapeda}
	\textit{iŋɡa daan noosa kaddaanmai maŋŋana saasapɛda} \hfill L-H\%\\
	\glt ‘Not long after that, there came a child, cycling.'
	\begin{flushright}(pearstory\_11\_SP.001-014) \osflink{65a55052c585fd0c3b9ce2d9}{pearstory_11_sp_THL.wav}\end{flushright}
}
\z
\z


Examples \REF{bali tau pagauan}--\REF{biibindas toolang1} form one paragraph and are each uttered with the L-H\% boundary-tone complex, marking non-finality. The final CIU of the paragraph -- the Tail of the paragraph -- is \REF{biibindas toolang1} \textit{biibindas} \textit{tɔɔlaŋ} ‘pulling a goat’, which is uttered with the H-L\% boundary-tone complex. It is repeated  as  the Head of the subsequent paragraph, this time bearing the L-H\% boundary-tone complex. The realization of the two CIUs of the THL construction in \REF{biibindas toolang1}--\REF{biibindas toolang2} above are displayed in  \figref{biibindas toolang}.







\begin{figure}
	\includegraphics[
	height=0.3\textheight
	]{figures/pearstory_11_SP_extSP_biibindas_toolang_plot.png}
	\caption{Periogram with pitch track (in st) for example \REF{biibindas toolang1}--\REF{biibindas toolang2}, realization of the two segmentally identical CIUs of the THL construction; speaker SP}
	\label{biibindas toolang}
\end{figure}




These examples support the hypothesis that (L)H-L\% signals finality, while L-H\% serves as ``continuer", signaling non-finality of a CIU with regard to a higher-level discourse unit. Note that both contours occur in interrogative as well as declarative sentences. 





\is{boundary tone complex|)}


\subsection{Tonal events at the boundaries of non-final, embedded IUs of CIUs}
\label{sec:tonal-events-at-the-boundaries-of-ips}

\is{embedded Intonation Units|(}

This section deals with tonal events at the right-edge boundaries of non-final IUs of CIUs. This is exemplified in  \figref{non-final of CIUS} below.

\begin{figure}
	\caption{Visualization of non-final IUs of CIUs  being discussed in this section}
	\label{non-final of CIUS}
	\begin{tikzpicture}[scale=0.7]
		%\draw[style=help lines] (0,0) grid (15,5);
		
		%L-H%
		\draw  [very thick] (-4.5,1) --  (-4.5,5.5) --  (12.5,5.5) --  (12.5,1) --  (-4.5,1);
		
		%syllable signs
	%	\node [lightgray](IUa) at (-2.8,5) {IU};
		\node (CIUa) at (7.1,5) {CIU};
		
		\node  [lightgray] (IU1a) at (-2.8,3.5) {IU};
		\node (CIU1a) at (4,3.5) {IU};
		\node  (CIU2a) at (7.1,3.5) {IU};
		\node [lightgray](CIU3a) at (10.2,3.5) {IU};
		
		\node [lightgray](IUqb) at (-2.8,1.5) {{[}σ σ σ σ{]}\#};
		\node  (CIU1b) at (4,1.5) {{[}σ σ σ σ σ{]}};
		\node (CIU2b) at (7.1,1.5) {{[}σ σ σ σ σ{]}};
		\node [lightgray](CIU3b) at (10.2,1.5) {{[}σ σ σ σ σ{]}\#};
		
		%\draw[] [lightgray](-2.8,4.7) -- (-2.8,3.8);
		\draw[] (7.1,4.7) -- (4,3.8);
		\draw[]  (7.1,4.7) -- (7.1,3.8) ;
		\draw[] [lightgray](7.1,4.7) -- (10.2, 3.8);
		
		\node [lightgray](T0) at (-2.2,2.2) {T-T\%}; 
		\node  (T1) at (4.9,2.2) {T-T\%}; 
		\node (T2) at (8,2.2) {T-T\%};	
		\node [lightgray](T3) at (11.1,2.2) {T-T\%};	
		
		
	\end{tikzpicture}
\end{figure}


Tonal events at the right-edge boundary of non-final, embedded IUs of CIUs can be equally classified into three different tonal contours. The summarizing plot of the three boundary-tone complexes is shown in   \figref{spaghetti plot-ip-final}.

\begin{figure}
	\includegraphics[
	height=0.3\textheight
	%width=0.6\textwidth
    ]{figures/smoothed_ip_final.png}
	\caption{Spaghetti plot showing the  intonation contours of the final three syllables of each of the three final boundary-tone complexes of embedded and non-final IUs with the items superimposed on each other and with an average contour produced by \textit{Loess} smoothing in R \citep{R_manual}: vertical bars indicate syllable boundaries; only final $CV.CV.CV]_{IU}[CV...$ syllable structure are displayed; values are z-transformed for IU.}
	\label{spaghetti plot-ip-final}
\end{figure}

The final boundary-tone complex consists of a phrase tone (T-) and a boundary tone (T\%): the phrase tone is a low tone L-, a high tone H- or a rising tone LH-, while the boundary tone can be either a low tone L\% or a high tone H\%.   \figref{ip boundary tones} depicts a schematic version of the three final boundary-tone complexes of non-final, embedded IUs of CIUs. In the following, I briefly summarize each boundary-tone complex. Subsequently, I illustrate them with examples and then describe their distribution in the corpus.


\begin{figure}
	\caption{Schematic representation of final boundary-tone complexes of embedded IUs: vertical bars indicate syllable boundaries}
	\label{ip boundary tones}
	
	\begin{tikzpicture}[scale=0.7]

		\node[draw] at (-3.5,4.5) [above] {\textcolor{white}{(}L-H\%};
		\node[draw] at (8.5,4.5) [above] {(L)H-L\%};
		\node[draw] at (2.5,4.5) [above] {(L)H-H\%};

		%L(H)-L% 
		\draw (6,-1) -- (6,4.5) -- (11,4.5) -- (11,-1) -- (6,-1);
		
		\draw [color={rgb:black,1;white,10}] (7,3) -- (7,1) ;
		\draw [color={rgb:black,1;white,10}] (8,3) -- (8,1) ;
		\draw [color={rgb:black,1;white,2}] (9,3) -- (9,1) ;
		\draw [color={rgb:black,1;white,2}] (9.05,3) -- (9.05,1) ;
		\draw [thick, color={rgb:black,1;white,10}] (10,3) -- (10,1) ;
		
		%Nodes:
		\node (start) at (6,2) {};
		\node (H) at (7.8,3.4) {\textcolor{gray}{(L)}H-};
		\node (L) at (9,0.7) {L\%};    

		\draw  [ultra thick]  plot [smooth,tension=0.6] coordinates {(6,1.5)   (7.8,3) (9.2,1) (11,2.3)};   
		\draw  [dashed, gray, thick] plot [smooth,tension=0.6] coordinates {(6,1.5) (7,1.5)   (7.7,2.9) (8,2.9) };  	
		
		%syllable signs
		\node (1st) at (6.5,0) {...σ};
		\node (1st) at (7.5,0) {σ};
		\node (1st) at (8.5,0) {σ};
		\node (1st) at (9.15,0) {]\textsubscript{IU}[};	
		\node (1st) at (9.7,0) {σ};
		\node (1st) at (10.5,0) {σ...}; 
		
		
		
		
		
		
		
		%L(H)-L% 
		\draw (0,-1) -- (0,4.5) -- (5,4.5) -- (5,-1) -- (0,-1);
		
		\draw [color={rgb:black,1;white,10}] (1,3) -- (1,1) ;
		\draw [color={rgb:black,1;white,10}] (2,3) -- (2,1) ;
		\draw [color={rgb:black,1;white,2}] (3,3) -- (3,1) ;
		\draw [color={rgb:black,1;white,2}] (3.05,3) -- (3.05,1) ;
		\draw [color={rgb:black,1;white,10}] (4,3) -- (4,1) ;
		
		%Nodes:
		\node (start) at (0,2) {};
		\node (H) at (1.8,3.4) {\textcolor{gray}{(L)}H-};
		\node (L) at (3,3.4) {\textcolor{white}{(L)}H\%};    
		\draw   [ultra thick] plot [smooth,tension=0.6] coordinates {(0,1.5)   (1.8,3) (3.2,3) (5,1.6)};   
		\draw  [dashed, gray, thick] plot [smooth,tension=0.9] coordinates {(0,1.5) (1,1.5)   (1.8,3) (3.2,3) };  	
		
		%syllable signs
		\node (1st) at (0.5,0) {...σ};
		\node (1st) at (1.5,0) {σ};
		\node (1st) at (2.5,0) {σ};
		\node (1st) at (3.15,0) {]\textsubscript{IU}[};	
		\node (1st) at (3.7,0) {σ};
		\node (1st) at (4.5,0) {σ...}; 
		
		
		

		\draw (-6,-1) -- (-6,4.5) -- (-1,4.5) -- (-1,-1) -- (-6,-1);
		
		\draw [color={rgb:black,1;white,10}] (-5,3) -- (-5,1) ;
		\draw [color={rgb:black,1;white,10}] (-4,3) -- (-4,1) ;
		\draw [color={rgb:black,1;white,2}] (-3,3) -- (-3,1) ;
		\draw [color={rgb:black,1;white,2}] (-2.95,3) -- (-2.95,1) ;
		\draw [color={rgb:black,1;white,10}] (-2,3) -- (-2,1) ;
		
		
		%Nodes:
		\node (start) at (-6,2) {};
		\node (H) at (-4,0.6) {L-};
		\node (H) at (-3,3.4) {H\%};
		
		\draw   [ultra thick] plot [smooth,tension=0.6] coordinates {(-6,1.5) (-4,1) (-3.4,2.8) (-2.8,3) (-1,2.3)};   
		
		
		%syllable signs
		\node (1st) at (-5.5,0) {...σ};
		\node (1st) at (-4.5,0) {σ};
		\node (1st) at (-3.5,0) {σ};
		\node (1st) at (-2.85,0) {]\textsubscript{IU}[};	
		\node (1st) at (-2.3,0) {σ};
		\node (1st) at (-1.6,0) {σ...};

	\end{tikzpicture}
	
\end{figure}





The three final boundary-tone complexes of embedded IUs of CIUs are illustrated with examples from the corpus. For the purpose of exemplification, only CIUs consisting of two embedded IUs are used. Hence, they only have a single CIU-internal tonal event, i.e., that of the non-final, embedded IU.

\subsubsection{The L-H\% boundary-tone complex}

\is{boundary tone complex|(}


An instance of the final boundary-tone complex L-H\% is found in example  \REF{ex:sapeda nollumpakmoko dɛi batu}, for which the periogram is shown in   \figref{pitch:sapeda nollumpakmoko dɛi batu}. 

The initial word of the CIU, \textit{sapɛda} ‘bicycle’, forms its own embedded IU, demarcated by the final boundary-tone complex L-H\%. Pitch starts around the middle of the speaker’s current range and it reaches the low target of the L- phrase tone located at the boundary between the penultimate (\textit{.pɛ.}) and the ultimate syllable (\textit{.da}) of the IU. On the ultimate syllable, pitch rises 6 st to the high target of the H\% boundary tone.




\begin{figure}
	\includegraphics[
	height=0.3\textheight
	%,
	%width=1\textwidth
	]{figures/pearstory_14_SP_extSP_sapeda_nollumpakmoko_dei_batu_plot.png}
	\caption{Periogram with pitch track (in st) for example \REF{ex:sapeda nollumpakmoko dɛi batu}, example of final boundary-tone complex L-H\% on embedded IU \textit{sapɛda},  speaker SP}
	\label{pitch:sapeda nollumpakmoko dɛi batu}
\end{figure}

\ea
\label{ex:sapeda nollumpakmoko dɛi batu}
\textit{sapɛda | nɔllumpakmɔkɔ dɛi batu} \\
\gll sapɛda  nɔ-\textsc{rdp-}lumpak=mɔ=kɔ  dɛi batu  \\
bicycle \textsc{st.rls}-\textsc{rdp}-hit.against=\textsc{cpl}=\textsc{and} \textsc{loc} stone\\ 
\glt ‘the bicycle hit the stone’ \hfill(pearstory\_14\_SP.028)
	\osflink{65a550551cba3e0c1d01857d}{pearstory_14_SP_extSP_sapeda_nollumpakmoko_dei_batu.wav}
\z


\subsubsection{The (L)H-H\% boundary-tone complex}

The (L)H-H\% is a summarizing label designating the two variants LH-H\% and H-H\%.  The difference is that in the LH-H\% variant the main domain for the pitch rise is the penultimate syllable. In the H-H\% variant, on the other hand, the rise in pitch may extend over several syllables. The two  variants  are illustrated below.

An instance of the LH-H\% boundary-tone complex is found in example  \REF{ex:anu moga ulai sei kakaita ai bakeleta}, for which the periogram is given in  \figref{pitch:anu moga ulai sei kakaita ai bakeleta}. The initial 5 words form a separate IU, the final word of which is \textit{kakaita} ‘your grandfather’.

Pitch starts around the middle of the speaker's current range and gradually drops about 3 st over the first 8 syllables. On the prefinal syllable (\textit{.kai.}) of the final word  \textit{kakaita} `your grandfather' of the first IU, pitch rises 5 st towards the high target of the LH- phrase tone at the beginning of the final syllable (\textit{.ta}). Pitch then remains high over the final syllable of the IU as the IU ends on an H\% boundary tone. 







\begin{figure}
	\includegraphics[
	height=0.3\textheight
	%,
	%width=1\textwidth
	]{figures/explanation-lelegesan_SYNO_extSYNO_anu_moga_ulai_sei_kakaita_ai_bakeleta_plot.png}
	\caption{Periogram with Pitch track (in st) for example \REF{ex:anu moga ulai sei kakaita ai bakeleta}, example of LH-H\% boundary-tone complex on  \textit{kakaita}, speaker SYNO}
	\label{pitch:anu moga ulai sei kakaita ai bakeleta}
\end{figure}

\newpage
\ea
\label{ex:anu moga ulai sei kakaita ai bakeleta}
\textit{anu mɔɡa ulai sɛi kakaita | ai bakɛlɛta} \\
\gll anu mɔɡa uli=ai sɛi kakai=ta ai bakɛlɛ=ta \\
\textsc{rel} only from=\textsc{ven} \textsc{hon} grandfather=1\textsc{pi}.\textsc{gen} and grandmother=1\textsc{pi}.\textsc{gen}\\ 
\glt ‘which (only) comes from our grandfathers and grandmothers’ \begin{flushright}(explanation-lelegesan\_SYNO.002)
	\osflink{65a550071c92110a77abe72b}{explanation-lelegesan_SYNO_extSYNO_anu_moga_ulai_sei_kakaita_ai_bakeleta.wav}\end{flushright}
\z



The H-H\% variant of the boundary-tone complex is exemplified in   \REF{ex:ana moita kami dɛi lipulipu giigii ana},  with its corresponding periogram depicted in  \figref{pitch:ana moita kami dɛi lipulipu giigii ana}. The first three words form a separate IU, the final word of which is \textit{kami} `1pe'. Here, the domain of the pitch rise to the high target of the H\% boundary tone is not the prefinal syllable (\textit{ka.})
exclusively, but extends over several syllables (\textit{mɔ.i.ta.ka}.).








\begin{figure}
	\includegraphics[
	height=0.3\textheight
	%,
	%width=1\textwidth
	]{figures/explanation-wedding-tradition_ZBR_extZBR_ana_moita_kami_dei_lipu_lipu_giigii_ana_plot.png}
	\caption{Periogram with pitch track (in st) for example \REF{ex:ana moita kami dɛi lipulipu giigii ana}, example of H-H\% boundary-tone complex  on  \textit{kami}, speaker ZBR}
	\label{pitch:ana moita kami dɛi lipulipu giigii ana}
\end{figure}


\ea
\label{ex:ana moita kami dɛi lipulipu giigii ana}
\textit{ana mɔita kami | dɛi lipulipu ɡiiɡii ana} \\
\gll ana mɔ-ita kami dɛi \textsc{rdp}-lipu \textsc{rdp}-ɡii ana \\
if \textsc{pot}-see 1\textsc{pe} \textsc{loc} \textsc{rdp}-country \textsc{rdp}-different \textsc{med}\\ 
\glt ‘when we look at that in other countries’ \begin{flushright}(explanation-wedding-tradition\_ZBR.258)
	\osflink{65a5500af2240f0b5032e6d2}{explanation-wedding-tradition_ZBR_extZBR_ana_moita_kami_dei_lipu_lipu_giigii_ana.wav}\end{flushright}
\z



\subsubsection{The (L)H-L\% boundary-tone complex}
Lastly, the (L)H-L\% boundary-tone complex and its two realizations H-L\% and LH-L\% are exemplified. Similar to the above, in the LH-L\% variant the main domain for the pitch rise is the penultimate syllable. In the H-L\% variant, on the other hand, the rise in pitch extends over several syllables. The two variants are illustrated below.

An instance of the final boundary-tone complex H-L\% is displayed in the periogram of example  \REF{ex:moane ana lau monuludan oto ana} in   \figref{pitch:moane ana lau monuludan oto ana}. The initial word \textit{moanɛ} ‘man’ forms a separate IU, demarcated by the IU-final boundary-tone complex H-L\%.

Pitch starts around the middle of the speaker’s current range. Pitch then rises 10 st over the first two syllables until it reaches the high target of the H- phrase tone located between the penultimate (\textit{.a.}) and the ultimate syllable (\textit{.nɛ}) of the IU. On the ultimate syllable, pitch drops 7 st to the low target of the IU-final L\% boundary tone.






\begin{figure}
	\includegraphics[
	height=0.3\textheight
	%,
	%width=1\textwidth
	]{figures/QUIS-focus_SP_extSPmoane_ana_lau_monuludan_oto_ana_plot.png}
	\caption{Periogram with pitch track (in st) for example \REF{ex:moane ana lau monuludan oto ana}, example of final boundary-tone complex H-L\% on \textit{mɔanɛ}, speaker SP}
	\label{pitch:moane ana lau monuludan oto ana}
\end{figure}


\ea
\label{ex:moane ana lau monuludan oto ana}
\textit{mɔanɛ | ana lau mɔnuludan ɔtɔ ana} \\
\gll mɔanɛ     ana lau       mɔN-sulud-an    ɔtɔ   ana  \\
man  \textsc{med} presently \textsc{av-}push\textsc{-appl} car   \textsc{med}\\ 
\glt ‘it is a man who is pushing that car’ \hfill(QUIS-focus\_SP.010)
	\osflink{65a550651c92110a7babe96f}{QUIS-focus_SP_extSPmoane_ana_lau_monuludan_oto_ana.wav}
\z




Example \REF{ex:moane ana lau monuludan oto ana}  is particularly interesting. 
In this example, the word \textit{mɔanɛ} `man' is informationally important and therefore constructed as the first element of an cleft construction. The  construction marks the focus on  \textit{mɔanɛ} `man', which  is then followed by \textit{ana lau mɔnuludan ɔtɔ ana} `this (one) is pushing the car'. 
The pitch contour on \textit{mɔanɛ} `man'  could potentially be interpreted as a prominence-lending pitch movement on a word that is in focus. However, in Chapter  \ref{sec:Experiments}, and in particular in  \sectref{sec:Fokus}, I showed that Totoli does not make use of such means to mark focus. Focus is expressed by means of a cleft construction in this case. The focus constituent is then prosodically uttered in its own IU with a  H-L\% boundary tone that signals finality. The prosodic realization of syntactic constructions is further discussed in  \sectref{sec:the-syntax-of-intonation-units}.






Finally, the LH-L\% variant is illustrated by example \REF{ex:io kita poni maanu ai} and its periogram in  \figref{pitch:io kita poni maanu ai}. The first three words form a separate IU, the last word of which is \textit{pɔni} ‘again’.


Pitch begins around the middle and initially drops about 4 st. On the prefinal syllable (\textit{pɔ.}), pitch rises 9 st towards the high target of the H- phrase tone, located at the beginning of the penultimate syllable of the IU-final word \textit{pɔni}. On the final syllable, pitch drops about 7 st.


\begin{figure}
	\includegraphics[
	height=0.3\textheight
	%,
	%width=1\textwidth
	]{figures/spacegames_sequence1_KSR-SP_extSP_io_kita_poni_maanuai_plot.png}
	\caption{Periogram with pitch track (in st) for example \REF{ex:io kita poni maanu ai}, LH-L\% boundary-tone complex on \textit{pɔni}, speaker SP}
	\label{pitch:io kita poni maanu ai}
\end{figure}


\ea
\label{ex:io kita poni maanu ai}
\textit{io kita pɔni | maanuai} \\
\gll io kita pɔni mɔ-anu=ai  \\
\textsc{ok} 2\textsc{s} again \textsc{av}-\textsc{fill}\textsc{=ven}\\ 
\glt ‘ok, now you do one again’ \hfill(spacegames\_sequence1\_KSR-SP.252)
	\osflink{65a5506c1cba3e0c1101880e}{spacegames_sequence1_KSR-SP_extSP_io_kita_poni_maanuai.wav}
\z



\subsubsection{Distribution}

The distribution of the different final boundary-tone complexes of non-final, embedded IUs is shown in  \figref{Freq_ip_type}.



\begin{figure}
	\includegraphics[%height=0.3\textheight,
	width=1\textwidth]{figures/freq_ip_types.png}
	\caption{Frequency distribution of tonal events at the right-edge boundary of embedded, non-final IUs, within conversational and monological recordings}
	\label{Freq_ip_type}
\end{figure}



The distribution shows that the (L)H-H\% and the L-H\% are the major final tonal patterns of non-final, embedded IUs of CIUs. The LH-H\% pattern is only marginally attested. The distribution is similar over the different data types. 


\is{boundary tone complex|)}
\is{embedded Intonation Units|)}

\subsection{Phonetic variability}
\label{sec:phonetic-variability}



The segmental material  influences the realization of the pitch contours in that a lack of sonorant material in the final and prefinal syllable results in the pitch rises or falls to be only partly realized.



For the purpose of illustration, I use examples of singleton IUs or final IUs of CIUs with final boundary tones (L)H-L\% and L-H\%   respectively, as they are the major final boundary tones. The effects are the same for the final LH-H\% boundary-tone complex and also for final boundary-tone complexes of embedded, non-final  IUs of CIUs.



Consider  \figref{pitch_anu ampi koloanan saasaluai kita1} and \figref{pitch_anu ampi koloanan saasaluai kita2}, which are the periograms of  another example of a THL given in  example \REF{ex:THLampi koloanan saasaluai kita}. As is specified for all THL constructions, the first CIU bears the final (L)H-L\% boundary-tone complex and the second one the L-H\%. These are only partly realized, due to the voiceless plosives [k] and [t] in the onset of the final (\textit{ki.}) and the prefinal syllable (\textit{.ta\#}).

\is{Tail-Head Linkage}

In  \figref{pitch_anu ampi koloanan saasaluai kita1}, the rise in pitch on the penultimate syllable of the CIU (\textit{ki.}) is interrupted  and is only visible on the short vowel of the penultimate syllable. The drop in pitch of about 11 st of towards the low boundary tone L\% is realized in full on the final syllable. In  \figref{pitch_anu ampi koloanan saasaluai kita2},  the rise to the high target of the H\% boundary tone is interrupted  and hence results in a jump in pitch of about 6 st on \textit{ki.ta\#}.

\begin{figure}
	\includegraphics[height=0.3\textheight
	%,
	%width=1\textwidth
	]{figures/spacegames_sequence4_KSR-SP_extSPanu_ampi_koloanan_saasaluai_kita_plot.png}
	\caption{Periogram with pitch track (in st) for example \REF{ex:anu ampi koloanan saasaluai kita} with CIU-final boundary-tone complex LH-L\%, speaker SP}
	\label{pitch_anu ampi koloanan saasaluai kita1}
\end{figure}






\begin{figure}
	\includegraphics[height=0.3\textheight
	%,
	%width=1\textwidth
	]{figures/spacegames_sequence4_KSR-SP_extSP__ampi_koloanan_sasaluai_kita_plot.png}
	\caption{Periogram with pitch track (in st) for example \REF{ex:ampi koloanan saasaluai kita} with CIU-final boundary-tone complex L-H\%, speaker SP}
	\label{pitch_anu ampi koloanan saasaluai kita2}
\end{figure}


\ea
\label{ex:THLampi koloanan saasaluai kita}
\ea{
	\label{ex:anu ampi koloanan saasaluai kita}
	\textit{anu ampi kɔlɔanan | saasaluai kita} \\
	\gll anu ampi kɔlɔanan \textsc{rdp-}salu=ai kita\\
	\textsc{rel} side right \textsc{rdp-}to.face\textsc{=ven} \textsc{2s}\\
	\glt `So the (one) on the right-hand side is facing you'
}

\ex{
	\label{ex:ampi koloanan saasaluai kita}
	\textit{ampi kɔlɔanan | saasaluai kita} \\
	\gll ampi kɔlɔanan \textsc{rdp-}salu=ai kita\\
	side right \textsc{rdp-}to.face\textsc{=ven} \textsc{2s}\\
	\glt `the (one) on the right-hand side is facing you' \begin{flushright}(spacegames\_sequence4\_KSR-SP.231 \& 233)
		\osflink{65a54ff3f2240f0b5032e6cc}{anu_ampi_koloanan_saasaluai_kita_ampi_koloanan_sasaluai_kita_spacegames_sequence4_KSR-SP_extSP.wav}\end{flushright}}

\z
\z



\is{Tail-Head Linkage}

If the segments of the final and the prefinal syllable are fully sonorant, the boundary-tone complexes are fully realized, as can be seen in the two realizations of example  \REF{ex:molitenggean} in  \figref{pitch_molitenggean1} and \figref{pitch_molitenggean2}. 




\begin{figure}
	\includegraphics[
	height=0.3\textheight
	%,
	%width=1\textwidth
	]{figures/spacegames_sequence4_KSR-SP_extSP_molitenggean_plot.png}
	\caption{Periogram with pitch track (in st) for example \REF{ex:molitenggean} with IU-final boundary-tone complex L-H\%, speaker SP}
	\label{pitch_molitenggean1}
\end{figure}



\begin{figure}
	\includegraphics[
	height=0.3\textheight
	%,
	%width=1\textwidth
	]{figures/spacegames_sequence4_KSR-SP_extSP__molitenggean_plot.png}
	\caption{Periogram with pitch track (in st) for example \REF{ex:molitenggean} with IU-final boundary-tone complex LH-L\%, speaker SP}
	\label{pitch_molitenggean2}
\end{figure}




\ea
\label{ex:molitenggean}
\textit{mɔlitɛŋɡɛan} \\
\gll mɔli-tɛŋɡɛ-an  \\
\textsc{rcp-}back\textsc{-rcp} \\ 
\glt ‘(they are) back to back’ \hfill(spacegames\_sequence4\_KSR-SP.071 \& 105)
	\osflink{65a55074a246ff0abcdd3d8b}{spacegames_sequence4_KSR-SP_extSP_molitenggean.wav}, 
	\osflink{65a550731c92110a80abea41}{spacegames_sequence4_KSR-SP_extSP__molitenggean.wav}
\z


For final syllables with a short vowel, the main domain of the rise or fall to the T- phrase tone is the penultimate syllable. Segmental material affects the shape of the tonal contours but not the location of tonal targets.
However, if the IU-final syllable involves a long vowel, the tonal targets of both the phrase tone and the boundary tone are realized in full on that syllable.



\largerpage
For an illustration of the LH-L\% and the L-H\% on a final syllable with a long vowel, see examples \REF{ex:ramean tau pomoo} and \REF{ex:geipo sallo pomoo}, which both end on the word \textit{pɔmɔɔ} ‘back then/first’. Two different realizations are given in  \figref{pitch:ramean tau pomoo} and  \figref{pitch:geipo sallo pomoo}.  

%Figure \ref{pitch:ramean tau pomoo} shows the realization of IU-final \textit{pomoo} with the LH-L\% boundary-tone complex. Figure \ref{pitch:geipo sallo pomoo} shows the realization of the same word with a final L-H\% boundary-tone complex.


The pitch contour of example \REF{ex:ramean tau pomoo} in  \figref{pitch:ramean tau pomoo} shows that the rise to the high target of the LH- phrase tone and the subsequent fall to the low target of the L\% boundary tone are both realized on the ultimate syllable (\textit{.mɔɔ}).

\begin{figure}
	\includegraphics[
	height=0.3\textheight
	%,
	%width=1\textwidth
	]{figures/ramean_tau_pomoo_explanation-lelegesan_SYNO_plot.png}
	\caption{Periogram with pitch track (in st) for example \REF{ex:ramean tau pomoo} with IU-final boundary-tone complex LH-L\% on a final syllable with a  long vowel, speaker SYNO}
	\label{pitch:ramean tau pomoo}
\end{figure}


\ea
\label{ex:ramean tau pomoo}
\textit{ramɛan tau pɔmɔɔ} \\
\gll ramɛ-an tau pɔmɔɔ  \\
lively-\textsc{nmlz} person back.then \\ 
\glt ‘The crowd/amusement of the people back then’ \begin{flushright}(explanation-lelegesan\_SYNO.085)
	\osflink{65a550681cba3e0c180185bd}{ramean_tau_pomoo_explanation-lelegesan_SYNO.wav}\end{flushright}
\z




Similarly, the realization of example \REF{ex:geipo sallo pomoo} in  \figref{pitch:geipo sallo pomoo} shows that the drop in pitch towards the low target of the L- phrase tone and the subsequent rise towards the high target of the H\% boundary tone are also both realized on the final long syllable (\textit{.mɔɔ}).



\begin{figure}
	\includegraphics[
	height=0.3\textheight
	%,
	%width=0.8\textwidth
	]{figures/geipo_sallo_pomoo_pearstory_13_RD_plot.png}
	\caption{Periogram with pitch track (in st) for example \REF{ex:geipo sallo pomoo} with CIU-final boundary-tone complex L-H\% on a final syllable with a  long vowel, speaker RD}
	\label{pitch:geipo sallo pomoo}
\end{figure}


%\begin{dont-break}
\ea
\label{ex:geipo sallo pomoo}
\textit{geipɔ | sallɔ pɔmɔɔ} \\
\gll geip=pɔ sallɔ pɔmɔɔ \\
\textsl{neg}\textsc{=incpl} basket first \\ 
\glt ‘no, but the basket first’ \hfill(pearstory\_13\_RD.015)
	\osflink{65a5500f1c92110a80abe9d2}{geipo_sallo_pomoo_pearstory_13_RD.wav}
\z
%\end{dont-break}





For the purpose of exemplification, I discussed the different boundary-tone complexes on rather short CIUs above. However, many CIUs contain several embedded IUs. Therefore, I discuss two examples of such rather long CIUs.  Example \REF{ex:kaasikan nogiigitai mangana dolago itu sapedana nollumpak} consists of a long CIU that begins with an embedded IU spanning 5 words and is demarcated by the final boundary-tone complex L-H\% realized on the final word \textit{itu} ‘\textsc{dist}’. The rise in pitch of about 5 st to the high target of the H\% boundary tone is realized merely as a jump, due to voiceless plosive [t] in the onset of the final syllable (\textit{.tu}). Pitch then drops 8 st to the low target of the L-H\% boundary-tone complex of the following IU containing the word \textit{sapɛda} ‘bicycle’. Pitch then drops towards the low target of the IU-final boundary-tone complex L-H\% of the final IU in the CIU. The final rise on the last syllable (\textit{.mpak}) is again realized only as a jump in pitch due to the voiceless syllable onset.







\begin{figure}
	\includegraphics[
	height=0.3\textheight
	%,
	%width=1\textwidth
	]{figures/pearstory_11_SP_extSP_kaasikan_nogiigitai_mangana_dolago_itu_sapedana_nollumpak_plot.png}
	\caption{Periogram with pitch track (in st) for example \REF{ex:kaasikan nogiigitai mangana dolago itu sapedana nollumpak} with CIU-final boundary-tone complex L-H\%, speaker SP}
	\label{pitch:kaasikan nogiigitai mangana dolago itu sapedana nollumpak}
\end{figure}



\ea
\label{ex:kaasikan nogiigitai mangana dolago itu sapedana nollumpak}
\textit{kaasikan mɔɡiiɡitai maŋana dɔlaɡo itu | sapɛda  | nɔollumpak} \\
\gll kɛasikan mɔɡ-\textsc{rdp}-ita-i maŋana dɔlaɡɔ itu sapɛda nɔ-\textsc{rdp}-lumpak \\
excitement \textsc{av.nrls}-\textsc{rdp}-watch-\textsc{appl} child girl \textsc{dist} bicycle \textsc{av.rls}-\textsc{rdp}-hit.against\\
\glt ‘because of his excitement in looking at the girl, his bicycle crashed (against the stone)’ \hfill(pearstory\_11\_SP.025)
	\osflink{65a55050f2240f0b4d32e5d6}{pearstory_11_SP_extSP_kaasikan_nogiigitai_mangana_dolago_itu_sapedana_nollumpak.wav}
\z

Consider example \REF{ex:siritana ia geimo daan lau mokodoong maaling ia barang ia} and its visualization in  \figref{pitch:siritana ia geimo daan lau mokodoong maaling ia barang ia}. The CIU  consists of several embedded IUs.  The first two bear the final  boundary-tone complexes LH-H\%, realized on \textit{siritana} `this story' and \textit{daan} `\textsc{exist}'. The word \textit{maaling} `to get lost' bears the IU-final LH-L\% boundary-tone  complex.  The final IU of the CIU ends on an  L-H\% boundary-tone complex. 

\newpage
\begin{figure}
	\includegraphics[
	height=0.3\textheight
	%,
	%width=1\textwidth
	]{figures/explanation-lelegesan_SYNO_extSYNO_siritana_ia_geimo_daan_lau_mokodoong_maaling_ia_barang_ia_plot.png}
	\caption{Periogram with pitch track (in st) for example \REF{ex:siritana ia geimo daan lau mokodoong maaling ia barang ia} with IU-final boundary-tone complex L-H\%, speaker SYNO}
	\label{pitch:siritana ia geimo daan lau mokodoong maaling ia barang ia}
\end{figure}



\ea
\label{ex:siritana ia geimo daan lau mokodoong maaling ia barang ia}
\textit{siritana | ia ɡɛimɔ daan | lau mɔkɔdɔɔng maaliŋ | ia baraŋ ia} \\
\gll sirita=na ia ɡɛimɔ daan lau mɔkɔ-dɔɔŋ mo-aliŋ ia baraŋ ia \\
story=3\textsc{s}.\textsc{gen} \textsc{prx} not \textsc{exist} presently \textsc{st.av}-want \textsc{st}-disappear \textsc{prx} goods \textsc{prx}\\
\glt ‘This story will never again get lost; this thing’ \begin{flushright}(explanation-lelegesan\_SYNO.007)
	\osflink{65a5500af2240f0b5132e95a}{explanation-lelegesan_SYNO_extSYNO_siritana_ia_geimo_daan_lau_mokodoong_maaling_ia_barang_ia.wav}\end{flushright}
\z


The question is whether the tonal contours can be explained as involving  IU-final H\% boundary tones only, rather than the combination of a LH- or L- phrase tone with an H\% boundary tone. Such an analysis would not capture the fact that the pitch rises occur on one syllable only. Analyzing the IU-final rise as an H\% boundary tone  alone would not explain why the pitch contour does not remain high after the initial high target towards the end of the first embedded IU and the high targets of the high boundary tones of the following IUs.  

In considering example \REF{ex:siritana ia geimo daan lau mokodoong maaling ia barang ia} and its visualization in  \figref{pitch:siritana ia geimo daan lau mokodoong maaling ia barang ia}, it becomes evident that we must assume an L- or LH- phrase tone to account for the drop in pitch before the final rises on the ultimate syllables. 
The alternative would be to assume that embedded IUs begin with an IU-initial low tone. This, however, would not explain why pitch drops steadily over the entire IU towards the final or prefinal syllable. 



To conclude the discussion of phrase-final pitch events, a note on discourse particles is due.


\subsection{Discourse particles}
\label{sec:prosodic-clitics}




In addition to one of the boundary-tone complexes, a \isi{discourse particle} can optionally occur at the end of a singleton IU or CIU. The two prosodic clitics \textit{wi} and \textit{ɛɛ} are the most frequently attested in the corpus. Other discourse particles are not frequent enough to allow for any generalizations. The discourse particles \textit{wi} and \textit{ɛɛ} are uttered under a coherent pitch contour together with the host CIU -- either singleton IU or a CIU -- and no pause  occurs between them.  Impressionistically, most  CIUs sound complete if the `prosodic clitic' is cut off using any annotation software. These encliticized discourse markers are tonally specified as either rising or falling, independent of the boundary-tone complex of the host CIU. They are  tonally specified for either H\%, to signal continuation or L\%, to signal finality.

A frequently occurring  discourse particle in the corpus is \textit{wi}. Similar to \ili{Indonesian} \textit{kan}, the Totoli discourse marker \textit{wi} is used as “a request of verification or confirmation, or it may be a marker of conjoint knowledge” \citep[403]{Wouk1998}. In the corpus, this discourse particle frequently occurs in recordings of the Space Game task \citep{Levinson.1992c}. \is{Animal Game \citep{Skopeteas.2006b}} \is{narrow focus} \is{contrastive focus}  In this task, two participants are each given an identical set of photos and must find matching photos in a memory game. As one participant has to identify the photo being described by the second participant without seeing the latter’s stack of photos, the consultants frequently ask for verification or confirmation of whether the photo selected indeed matches the intended image. In these contexts, the discourse marker \textit{wi} is used. It is tonally specified for H\%.

\figref{pitch:dello enggaenggat anu dɛi ulin wi} shows the realization of example \REF{ex:dello enggaenggat anu dɛi ulin wi}, with  \textit{wi} in CIU-final position. 


\begin{figure}
	\includegraphics[
	height=0.3\textheight
	%,
	%width=1\textwidth
	]{figures/dello_enggaenggat_anu_dei_ulin_wi_spacegames_sequence3_ext2_SumitroP_plot.png}
	\caption{Periogram with pitch track (in st) for example \REF{ex:dello enggaenggat anu dɛi ulin wi} with CIU-final boundary-tone complex L-H\%, followed by the discourse particle \textit{wi} with H\% tone, speaker SP}
	\label{pitch:dello enggaenggat anu dɛi ulin wi}
\end{figure}

\ea
\label{ex:dello enggaenggat anu dɛi ulin wi}
\textit{dɛllɔ ɛŋɡaɛŋɡat | anu dɛi ulin | wi} \\
\gll  dɛllɔ \textsc{rdp-}ɛŋgat anu dɛi ulin wi\\
like \textsc{rdp-}lift.up \textsc{rel} \textsc{loc} back \textsc{intj}\\ 
\glt ‘like being lifted up, the one at the back, right?’ \begin{flushright}(spacegames\_sequence3\_KSR-SP.225)
	\osflink{65a55003c585fd0c479ce099}{dello_enggaenggat_anu_dei_ulin_wi_spacegames_sequence3_ext2_SumitroP.wav}\end{flushright}
\z



The L-H\% boundary-tone complex is realized on the word \textit{ulin} ‘back’. After a high target on the last syllable \textit{ɛŋɡaɛŋɡat} ‘being lifted’ of the first IU, pitch drops towards the low target of the L- phrase tone located at the boundary between the penultimate and ultimate syllables of the IU-final word \textit{ulin} ‘back’. On the final syllable, pitch rises 6 st towards the high target of the H\% boundary-tone complex. The discourse marker \textit{wi} occurs after the rise of the L-H\% boundary-tone complex, extending it by another 15 st.

\is{discourse particle}


Taken from the same recording, example \REF{ex:molitenggean wi} is an instance of the same prosodic clitic realized after an IU-final LH-L\% boundary-tone complex, as shown in  \figref{pitch:molitenggean wi}.


\begin{figure}
	\includegraphics[
	height=0.3\textheight
	]{figures/molitenggean_wi_spacegames_sequence4_Kaiser_SumitroP_extSumitroP_plot.png}
	\caption{Periogram with pitch track (in st) for example \REF{ex:molitenggean wi} with final boundary-tone complex L-H\%, followed by the discourse particle \textit{wi} with H\% tone, speaker SP}
	\label{pitch:molitenggean wi}
\end{figure}

\ea
\label{ex:molitenggean wi}
\textit{mɔlitɛŋɡɛan | wi} \\
\gll mɔli-tɛŋɡɛ-an wi  \\
\textsc{rcp-}back\textsc{-rcp}  \textsc{intj}\\ 
\glt ‘(they are) back to back, right?’ \hfill(spacegames\_sequence4\_KSR-SP.124)
	\osflink{65a5502c1c92110a7babe95d}{molitenggean_wi_spacegames_sequence4_Kaiser_SumitroP_extSumitroP.wav}
\z


\figref{pitch:molitenggean wi} shows the pitch rising 6 st towards the high target of the phrase tone, located at the beginning of the syllable $.an$. The subsequent final major drop in pitch is only partially realized, as the rise to the H\% tone of the discourse particle extends into the coda of the preceding syllable.

Note the two examples with a slight dip on the final discourse particle \textit{wi} in  \figref{pitch:dello enggaenggat anu dɛi ulin wi}  and even more pronounced so in  	\figref{pitch:molitenggean wi}. This is parallel to the realization of the L-H\% and the (L)H-H\% pattern described above and appears to be a characteristic of the rising patterns in Totoli (see  \cite{dombrowski2010shaping} for a discussion on convex vs. concave rising patterns).

\is{discourse particle}

Another frequently occurring final discourse marker is \textit{ɛɛ}. The discourse marker is prosodically realized as a clitic, tonally specified for L\%. It is used as an emphasizer, asserting the validity of the question or, as in example \REF{ex:molitenggean ee}, reaffirming the correctness of the statement of the host CIU. Often it is a request for action. In example \REF{ex:molitenggean ee}, the speaker urges the interlocutor to find the intended photo.  \figref{pitch:molitenggean ee} shows the pitch contour of example \REF{ex:molitenggean ee}.







\begin{figure}
	\includegraphics[
	height=0.3\textheight
	%,
	%width=0.8\textwidth
	]{figures/spacegames_sequence4_KSR-SP_extKSR_ia_molitenggean_ee_plot.png}
	\caption{Periogram with pitch track (in st) for example \REF{ex:molitenggean ee} with final boundary-tone complex L-H\%, followed by a discourse particle, speaker KSR}
	\label{pitch:molitenggean ee}
\end{figure}

\ea
\label{ex:molitenggean ee}
\textit{ia | mɔlitɛŋɡɛan | ɛɛ} \\
\gll ia mɔli-teŋɡɛ-an ɛɛ \\
yes \textsc{rcp-}back\textsc{-rcp}  \textsc{intj}\\ 
\glt ‘yes, (they are) back to back!’ \hfill(spacegames\_sequence4\_KSR-SP.278)
	\osflink{65a5506fc585fd0c3f9ce687}{spacegames_sequence4_KSR-SP_extKSR_ia_molitenggean_ee.wav}
\z


\is{discourse particle}

The CIU ends on the L-H\% boundary-tone complex followed by the L\% boundary tone of the prosodic clitic. The boundary-tone complex is realized on the final two syllables of \textit{mɔlitɛŋɡɛan}, to which the prosodic clitic is added. On the vowel of the syllable \textit{.ŋɡɛ.}, the low target of the L- phrase tone is located somewhat earlier than in contexts without a prosodic clitic. Pitch then rises 5 st towards the high target of the H\% boundary tone, reaching its peak at the boundary of the syllable \textit{.an} and the following prosodic clitic. Pitch then drops 9 st towards the low target of the L\% boundary tone of the prosodic clitic. The combination of an (L)H-L\% boundary-tone complex followed by a prosodic clitic specified for L\% is realized either as a sustained pitch plateau following the L\% of the boundary-tone complex, or is integrated into the major final drop in pitch. This can be seen in  \figref{pitch:moliulunan ee}, which depicts the pitch contour of example \REF{ex:moliulunan ee}. The major final fall to the L\% of the H-L\% boundary-tone complex is realized on the last syllable of \textit{mɔliulunan} ‘being in a row’. The prosodic clitic is then added, resulting in a further fall at the bottom of the speaker’s range.


\is{discourse particle}






\begin{figure}
	\includegraphics[
	height=0.3\textheight
	%,
	%width=0.7\textwidth
	]{figures/moliulunan_eh_spacegames_sequence3_ext1_Kaiser_plot.png}
	\caption{Periogram with pitch track (in st) for example \REF{ex:moliulunan ee} with final boundary-tone complex H-L\%, followed by a discourse particle, speaker KSR}
	\label{pitch:moliulunan ee}
\end{figure}


\ea
\label{ex:moliulunan ee}
\textit{mɔliulunan | ɛɛ} \\
\gll  mɔli-ulun-an ɛɛ  \\
\textsc{rcp-}row\textsc{-rcp}  \textsc{intj}\\ 
\glt ‘(they are) in a row!’ \hfill(spacegames\_sequence3\_KSR-SP.017)
	\osflink{65a55030a246ff0ac2dd3de1}{moliulunan_eh_spacegames_sequence3_ext1_Kaiser.wav}
\z


\is{discourse particle}

If the preceding syllable ends on a vowel, the final prosodic clitic tends to be realized as part of the final major fall in pitch, as shown in  \figref{pitch:ia poniga ee} from example \REF{ex:ia poniga ee}.


\begin{figure}
	\includegraphics[
	height=0.3\textheight
	%,
	%width=0.7\textwidth
	]{figures/ia_poniga_e_spacegames_sequence3_ext1_Kaiser_plot.png}
	\caption{Periogram with pitch track (in st) for example \REF{ex:ia poniga ee} with final boundary-tone complex H-L\% and a following discourse particle, speaker KSR}
	\label{pitch:ia poniga ee}
\end{figure}

%\begin{dont-break}




\ea 
\label{ex:ia poniga ee}
\textit{ia pɔniɡa | ɛɛ} \\
\gll ia pɔni=ɡa ɛɛ  \\
\textsc{prx} still=\textsc{?} \textsc{intj}\\ 
\glt ‘there is this one still!’ \hfill(spacegames\_sequence3\_KSR-SP.136)
	\osflink{65a55012f2240f0b5032e6d6}{ia_poniga_e_spacegames_sequence3_ext1_Kaiser.wav}
\z



\subsection{Discussion}
\label{sec:discussion}





In this section, I developed a model of the intonation of Totoli (\sectref{IU-model}) on the basis of a corpus of (semi-)spontaneous speech (\sectref{sec:tonal-events-at-the-boundaries-of-ips}--\sectref{sec:prosodic-clitics}) and informed by insights from the  experiments described in \chapref{sec:Experiments}.






In Totoli, a high proportion of singleton IUs are observed, where the final boundary-tone complex is the only pitch event. However, CIUs are also common.  \tabref{Length and Number ips} shows the average number of IUs contained in a substantive CIU. The bins represent the number of IUs contained in a segmented CIU of the corpus, and the height of the bins represents their overall proportion. The proportion and absolute numbers are stated above the bins.


\begin{figure}
	\includegraphics[
	width=1\textwidth]{figures/Prop_of_n_ips_in_IU.png}
	\caption{Frequency distribution of substantive CIUs containing n embedded IUs, 1 equals a singleton IU.}
	\label{Length and Number ips}
\end{figure}




The distribution is skewed, and less than 10\% of CIUs contain four or more embedded IUs. In the entire corpus, 43.4\% of IUs are singletons, with 41.3\% in monological data and 47.0\% in conversational data. CIUs consisting of two embedded IUs occur at a proportion of 35.3\% in the entire corpus, with 36.1\% in conversational data and 34.8\% in monological data.




In sections \sectref{sec:tonal-events-at-the-boundaries-of-ius}--\sectref{sec:prosodic-clitics}, I analyzed pitch events at the right-edge boundaries of singleton IUs, non-final embedded IUs of CIUs, and final IUs of CIUs. I proposed a classification of tonal events, which includes three boundary-tone complexes of each type.  \tabref{Summary tonal complexes} provides a summary of the different tonal complexes.


\begin{table}
	\caption{Summary of proposed IU-final boundary-tone complexes}
	\label{Summary tonal complexes}
	\begin{tabular}{cc}
		\lsptoprule
		\begin{tabular}[c]{@{}c@{}}IU-final\\ boundary-tone complexes\\ of  non-final IUs of CIUs\end{tabular} & \begin{tabular}[c]{@{}c@{}}IU-final\\ boundary-tone complexes\\ of singleton IUs or final IUs of CIUs.\end{tabular} \\ \hline
		&                        \\
		(L)H-H\%                                                                                             & LH-H\%                                                                                                              \\
		L-H\%                                                                                                & L-H\%                                                                                                               \\
		(L)H-L\%                                                                                             & (L)H-L\%            \\
		\lspbottomrule                                                                                               
	\end{tabular}
\end{table}

\tabref{Summary tonal complexes} presents the different boundary-tone complexes, arranged such that those with similar tunes are in the same row. The only difference is between the (L)H-H\% and the LH-H\% boundary-tone complexes. I argue that in CIU-final position, the domain for the pitch rise is exclusively the penultimate syllable, expressed here by an LH- phrase tone. In final position of embedded, non-final IUs of CIUs, however, there is more variation with regard to the domain of the pitch rise, expressed here by the label (L)H-.

So far, the main difference between the final boundary-tone complex of embedded, non-final IUs of CIUs and singleton IUs or final IUs of CIUs is that, by definition, the latter co-occurs with other boundary phenomena which do not occur at the end of embedded, non-final IUs of CIUs or may not be as pronounced. For instance, this could be a reset in pitch and an interruption in the pitch contour, a following pause, final syllable lengthening, and vocal fry (cf. \citealt[93--155]{Schuetze-Coburn1994}, \citealt[260--270]{himmelmann2006challenges}, \citealt{dubois1992, dubois1993}, and \citealt[29--39]{cruttenden1997intonation}). In the exposition above, I focused mainly on a discussion of pitch events. Further investigation of boundary strength may reveal interesting insights into the interplay of boundary type and boundary phenomena in Totoli \citep{schwiertz2009intonation, cho2005prosodic, fougeron1997articulatory}.




Regarding the tonal patterns, the main difference pertains to the distribution of the types of boundary-tone complexes. While the (L)H-H \% pattern is the main pattern at the right-edge boundary of non-final IUs of CIUs, it is the minor pattern in CIU-final position, i.e. of singleton IUs and final IUs of CIUs. Conversely, (L)H-L \% is the main pattern occurring at the right-edge of CIUs but the minor pattern occurring at the right-edge of embedded, non-final IUs of CIUs. In both positions, the L-H \% boundary-tone complexes occur in comparable proportions. This is displayed in \figref{Freq_ip_IU_type}.


\begin{figure}
	\includegraphics[%height=0.3\textheight,
	width=1\textwidth]{figures/freq_ip_IU.png}
	\caption{Frequency distribution of tonal events at the right-edge boundary of embedded IUs and non-final IUs of CIUs (e. IUs) in blue and singleton IUs or final IUs of CIUs  in yellow}
	\label{Freq_ip_IU_type}
\end{figure}









Considering the similarities between tonal events demarcating the right-edge boundaries of CIUs and non-final IUs of CIUs, the question arises as to how one can explain the differences in distribution. 

In section \sectref{sec:function} I described the different patterns as  expressing varying degrees of finality or continuation. The LH-H\% pattern expresses a high degree of non-finality or continuation, the L-H\% pattern expresses a medium degree of continuation or non-finality and the LH-L\% pattern expresses finality. Taking this as the main function of the patterns explains why the LH-H\% pattern is the most frequent pattern for embedded, non-final IUs of CIUs. In a chunk of speech, here the CIU, speakers phrase various grammatical units into separate prosodic units, for which the non-finality-signaling pattern, the LH-H\%, is used to signal full integration. CIU internally, the LH-L\% pattern is infrequent and only used in non-canonical constructions such as cleft constructions to signal less integration. The opposite holds true for singleton IUs and final IUs of CIUs. Here, the non-finality-signaling pattern LH-H\% is used very infrequently, and in fact almost exclusively in lists, to signal non-finality. The finality-signaling pattern LH-L\%, however, is very frequent. In both instances, that is CIU internally or CIU final, the L-H\% pattern, signaling medium finality/non-finality is equally frequent.




The intonational model of the CIU in Totoli is based mainly on the inspection of tonal events. I showed that, internally in CIUs as well as in final position of CIUs, the same tonal patterns occur. This led me to postulate a recursively embedded structure of a Compound IU, rather than a hierarchical structure where higher-level units (i.e., the IU) consist of lower-level units (i.e., Accentual Phrases \is{Accentual Phrase} or intermediate phrases) and where higher-level units are tonally specified differently than lower-level prosodic units (see e.g., \cite{Jun_2000} for such an analysis of \ili{French} and \cite{Turkish_Intonation} for \ili{Turkish}).

 

\citeauthor{Himmelmann_Preliminary_2018}'s (\citeyear{Himmelmann_Preliminary_2018}) model of an IP  in languages of Western Austronesia\il{Austronesian} describes IU-internal tonal events as boundary-marking devices of smaller units, called the intermediate phrase (ip). \is{intermediate phrase} His model proposes that the right-edge boundary of an IU consists of a phrase tone and a boundary tone and that ip-final boundary-tone complexes consist of a high target only. The model is reprinted in  	\figref{Himmelmann_ip}.



\begin{figure}[h]
	\caption{ \citeauthor{Himmelmann_Preliminary_2018}'s (\citeyear[360]{Himmelmann_Preliminary_2018}) model of the IP in Western Austronesian languages: T\$ representing an ipboundary tone, T\% an IP boundary tone, and T- and IP phrase accent.}
	\label{Himmelmann_ip}
	\begin{tikzpicture}

		
		\node (IUqb) at (0,1.5) {
			{[}{[}σσσσσ{]}\textsubscript{ip}
			{[}σσσσσσ{]}\textsubscript{ip}
			σσσσ{]}\textsubscript{IP}
		};

		
		
		\draw[-To] (-3.1,2.6) -- (-3.1,1.9);
		\draw[-To] (-1.8,2.6) -- (-1.8,1.9);
		\draw[-To] (-0.6,2.6) -- (-0.6,1.9);
		\draw[-To] (0.9,2.6) -- (0.9,1.9);
		\draw[-To] (2.9,2.6) -- (2.9,1.9);
		\draw[-To] (3.2,2.6) -- (3.2,1.9);
		
		
		\node at (-3.1,3) {L\$};
		\node at (-1.8,3) {H\$};
		\node at (-0.6,3) {L\$};
		\node at (0.9,3) {H\$};
		\node at (3,3) {T-T\%};



		
	\end{tikzpicture}
\end{figure}



Before discussing the theoretical implications of such an analysis, it is worth examining two further illustrative examples of Tail-Head Linkage (THL) constructions (see  \sectref{sec:function} for an explanation of THLs). These make an interesting case in point. The Tail IU is repeated in part or in full and serves as the Head IU of the subsequent paragraph. The Tail always bears the IU-final boundary-tone complex(L)H-L\%  and the Head always has the L-H\% boundary-tone complex. Two examples are given below in \REF{ex:THL2} and \REF{ex:THL+IU}. Consider first \REF{ex:THL2} and its periogram in  \figref{pitch:THL2}. The first IU is repeated after a long pause and the adverb \textit{inʤan} ‘after’ is added, pointing to the adverbial status the Head of a THL construction holds for the subsequent paragraph.

\is{Tail-Head Linkage}



\begin{figure}
	\includegraphics[
	height=0.3\textheight
	%,
	% width=1\textwidth
	]{figures/THL19_pearstory_38_SUD_ah_injan_nodului_isia_plot.png}
	\caption{Periogram with pitch track (in st) for example \REF{ex:THL2}; speaker SUD}
	\label{pitch:THL2}
\end{figure}

\ea
\label{ex:THL2}
\ea
	\label{ex:nodulu isia}
	\textit{<na> nɔdulu isia} \\
	\gll  nɔ-dulu isia\\
	\textsc{av.rls}-help \textsc{3s}\\
	\glt `helping him'


\ex
	\label{ah injan nodulu isia}
	\textit{ah inʤan nɔdulu isia} \\
	\gll ah inʤan nɔ-dulu isia\\
	\textsc{intj} after \textsc{av.rls}-help \textsc{3s}\\
	\glt `after helping him' \hfill(pearstory\_38\_SUD.056)
		\osflink{65a5507e1cba3e0c1d018698}{THL19_pearstory_38_SUD_ah_injan_nodului_isia.wav}

\z
\z

The final boundary-tone complexes and the subsequent pauses mean that the status of  examples	\REF{ex:nodulu isia} and  	\REF{ah injan nodulu isia} as separate CIUs is unambiguous. However, in some instances, the Head is directly followed by further syntactic material, as in the THL in example \REF{ex:THL+IU}, for which the periogram given in   \figref{pitch:THL+IU}.

\is{Tail-Head Linkage}




\begin{figure}
	\includegraphics[
	height=0.3\textheight
	%,
	% width=1\textwidth
	]{figures/pearstory_23_AT_extAT_48986_55641_plot.png}
	\caption{Periogram with pitch track (in st) for example \REF{ex:THL+IU}, speaker AT}
	\label{pitch:THL+IU}
\end{figure}


\ea
\label{ex:THL+IU}
\ea{
	\label{ex:mangana umbasan dedeng}
	\textit{maŋana umbasan dɛdɛŋ} \\
	\gll maŋana  umbasan dɛdɛŋ\\
	child  young.man small\\
	\glt `(there was) a young boy'
}

\newpage
\ex{
	\label{mangana <um> umbasan dedeng mai nagala anu ia}
	\textit{ma <um> umbasan dɛdɛŋ | mai naɡala anu ia} \\
	\gll maŋana  umbasan dɛdɛŋ mai nɔɡ-ala anu ia\\
	child  young.man small come \textsc{av.rls-}fetch \textsc{fill} \textsc{prx}\\
	\glt `(there was) the boy; he comes to take the thingy' \begin{flushright}(pearstory\_23\_AT.039-40)
		\osflink{65a550611c92110a7babe96d}{pearstory_23_AT_extAT_48986_55641.wav}\end{flushright}}

\z
\z


As expected, the Head \textit{maŋana umbasan dɛdɛŋ} `young boy' bears the final boundary-tone complex LH-L\%, as it is the standard pattern for Tails of THLs. \is{Tail-Head Linkage} The difference pertains to the Heads of the two THLs in example \REF{ah injan nodulu isia} and example \REF{ex:mangana umbasan dedeng}. In  \REF{mangana <um> umbasan dedeng mai nagala anu ia},  the Head -- here in near exact repetition -- is followed by further syntactic material, without any pause, pitch reset, or other  boundary phenomena. In this case, the Head would have to be analyzed as the first ip of the IP, if one applies \citeauthor{Himmelmann_Preliminary_2018}'s (\citeyear{Himmelmann_Preliminary_2018}) model.





\begin{figure}
	\begin{tikzpicture}
		
		\tikzset{level distance=1cm, sibling distance=1cm}
		
		\Tree [.IP [.ip {[[<na> nɔdulu isia]]} ]  ]
	\end{tikzpicture}
	%
	\begin{tikzpicture}
		
		\tikzset{level distance=1cm, sibling distance=1cm}
		
		\Tree [.IP [.ip {[[ah inʤan nɔdulu isia]]} ]  ]
	\end{tikzpicture}
	\caption{Prosodic organization of examples \REF{ex:nodulu isia}--\REF{ah injan nodulu isia}, according to \citeauthor{Himmelmann_Preliminary_2018}'s \citeyear{Himmelmann_Preliminary_2018} model}
	\label{Prosodic organization1}
\end{figure}


\begin{figure}
	\begin{tikzpicture} 
		
		\tikzset{level distance=1cm, sibling distance=1cm}
		
		\Tree [.IP [.ip {[[maŋana umbasan dɛdɛŋ]]} ]  ]
	\end{tikzpicture}
	%
	\begin{tikzpicture}
		
		\tikzset{level distance=1cm, sibling distance=0cm}
		
		\Tree [.IP [.ip {[[ma <um> umbasan dɛdɛŋ]} ] [.ip {[mai naɡala anu ia]]} ] ]
	\end{tikzpicture}
	\caption{Prosodic organization of examples 	\REF{ex:mangana umbasan dedeng}--	\REF{mangana <um> umbasan dedeng mai nagala anu ia}, according to \citeauthor{Himmelmann_Preliminary_2018}'s \citeyear{Himmelmann_Preliminary_2018} model}
	\label{Prosodic organization}
\end{figure}

However, the tonal events at the right edge of the Heads \REF{ah injan nodulu isia}  and \REF{mangana <um> umbasan dedeng mai nagala anu ia} are the same in both of the THL constructions. In  example \REF{ex:THL2}, we would label it L-H\% as it occurs in IP-final position. In the second THL \is{Tail-Head Linkage} construction in example \REF{ex:THL+IU},  we would have to label the final tonal pattern L-H\$, as it occurs at the edge of an ip, integrated into an IP (\$ indicates an ip boundary tone in \citeauthor{Himmelmann_Preliminary_2018}'s \citeyear{Himmelmann_Preliminary_2018} model). However, the model assumes that (non-final) ips and IPs are tonally differentiated.  We could do away with this seeming contradiction by assuming no further intonational level between the phonological word and the IU/IP, and by describing IUs such as those in example  \REF{mangana <um> umbasan dedeng mai nagala anu ia} as recursively parsed into IUs: 


\begin{figure}
	\begin{tikzpicture} 
		
		\tikzset{level distance=2cm, sibling distance=1cm}
		
		\Tree  [.IU {[maŋana umbasan dɛdɛŋ]} ] 
	\end{tikzpicture}
	%
	\begin{tikzpicture}
		
		\tikzset{level distance=1cm, sibling distance=0.1cm}
		
		\Tree [.CIU [.IU {[ma <um> umbasan dɛdɛŋ]} ] [.IU {[mai naɡala anu ia]} ] ]
	\end{tikzpicture}
	\caption{Alternative prosodic organization of example \REF{ex:THL+IU}}
	\label{Alternative prosodic organization}
\end{figure}

One reason for opposing the alternative analysis in  \figref{Alternative prosodic organization} is the Strict Layer Hypothesis  \citep[SLH;][]{selkirk1986, nespor1983prosodic, marina1986prosodic, Vogel_2019}, which predicts that any prosodic structure consists exhaustively of units of the next level down in the prosodic hierarchy, and allows no recursivity. \is{Strict Layer Hypothesis}

Though widely applied, the SLH causes empirical problems \citep[see the discussion in][chapter 8.2]{Ladd_2008}.  With evidence from tone sandhi in \ili{Xiamen Chinese}, \citet{Chen_1987} shows that tone groups  and IUs  regularly intersect  and hence violate the SLH in that a tone group may be associated with two IUs. On the issue of overlapping domains in \ili{Luganda}, \citeauthor{Hyman_1987} comment:

\begin{quotation}
	The alternative in Luganda which we consider in work in progress is that the SLH and some of the claims of its advocates must be significantly weakened to allow cyclic assignments of postlexical domains.  (\citeyear[107]{Hyman_1987})
\end{quotation}



The model proposed  by \citet[369]{Himmelmann_Preliminary_2018} analyzes IU-internal tonal events as boundary tones of ips and leaves open the status of  IU-final  material that follows the last ip-final boundary, see \figref{Himmelmann_ip}.





One possibility is to analyze this as  constituting an ip as well. Himmelmann argues that the Strict Layer Hypothesis  \is{Strict Layer Hypothesis} \citep[SLH;][]{selkirk1986} would demand such an analysis,  but points out that tonal targets are too different (his model assumes simple boundary tones at the right-hand edge of ips and boundary-tone complexes at the end of IUs) and that one would have to assume that the tune of IU-final ips are deleted or overwritten by the IU-final boundary tones. \citet{Himmelmann_Preliminary_2018} notes that IU-final boundary-tone complexes are of a different type and do not include ip-level tones, which in itself is again a violation of the SLH. However, I showed that  final boundary tones are very similar, if not identical; hence, no overwriting rule would have to be postulated.

Abolishing the notion of an intermediate phrase level altogether and assuming recursive parsing of IUs into CIUs,  we can avoid the difficulties in explaining that tunes are essentially the same except for the presence or absence of IU boundary phenomena. In THL \is{Tail-Head Linkage} constructions, one would avoid having to use different labels for essentially the same tonal pattern. 

Also evident from the examples above are the obvious differences to the Accentual Phrase (AP), the postulated prosodic unit below the IU in many of the prosodic descriptions in the two volumes edited by Jun (\citeyear{jun2006, jun_2014}). Not only does the AP \is{Accentual Phrase} differ from the IU in its tonal marking but the ``AP has been typically defined as a tonally marked prosodic unit which contains one word" \citep[532]{Jun_byProm}. Example \REF{ex:kaasikan nogiigitai mangana dolago itu sapedana nollumpak}  in  	\figref{pitch:kaasikan nogiigitai mangana dolago itu sapedana nollumpak} above is a particularly instructive instance of an adverbial phrase realized as embedded IU  that spans  5 words / 15 syllables and has a near-flat level contour except for the right-edge  boundary-tone complex (\textit{kaasikan mɔɡiiɡitai maŋana dɔlaɡɔ itu} `because of his excitement in looking at the girl ...'). The example adverbial phrase is uttered as one prosodic phrase and clearly larger than the AP and the prosodic word. Therefore, we cannot do away with recursive embedding of IUs by assuming simply the level of the non-recursively embedded IU and the phonological word or the AP or both. 




The  prosodic organization in  \figref{Alternative prosodic organization} equates to the model proposed by  \citet[297]{Ladd_2008} which he calls the \isi{Compound Prosodic Domain} (CPD): ``A CPD is a prosodic domain of a given type X whose immediate constituents \textit{are} \textit{themselves} \textit{of} \textit{type} \textit{X}."  


\begin{figure}
	\begin{tikzpicture}		
		\tikzset{level distance=1cm, sibling distance=1cm}
		\Tree [.X [.X ] [.X  ] ]
	\end{tikzpicture}
	\caption{Exemplification of Compound Prosodic Domains, reproduced from \citet[297]{Ladd_2008}}
	\label{x prosodic organization}
\end{figure}

%\citet[297]{Ladd_2008} exemplifies this is prosodic structures of the type \textit{[A and B]}, where \textit{[A]} and \textit{[B]} are roughly of the same syntactic structure. If \textit{[A]}, \textit{[B]} and \textit{[A and B]} are marked by a single boundary tone 


In a more recent account by \citet{selkirk201114} termed \textit{\isi{Match Theory}}, recursivity is permitted and attributed to syntactic constituency-respecting \textit{\isi{Match Constraints}}. 

Evidence obtained from an inspection of the IUs as they occur in  the corpus of Totoli leave no doubt that tonal events at the edges of prosodic units are essentially the same. Therefor, I argue that  complex IUs are best be described as CIUs that consist of a string of embedded IUs. 


I started by assuming that speech is chunked into prosodic units, which are demarcated by a set of boundary phenomena and which is perceived as such by listeners (\sectref{The units of spoken speech} and \sectref{IU-Props}). I then described and categorized the tonal events at the end of such units (\sectref{sec:tonal-events-at-the-boundaries-of-ius}). In a further step, I looked at tonal events within such units and found that they are essentially the same as those that occur at the end (\sectref{sec:tonal-events-at-the-boundaries-of-ips}). I concluded that they are also right-edge boundary tones of prosodic units which regularly match syntactic units. Based on the observation that tonal events of all kinds of prosodic units are essentially the same, I argue for assuming recursive embedding of IUs into Compound IUs. The results here show that tonal contours are engaged at the level of the IU but not the CIU. It is crucial to point to the fact that the argument for recursion is only based on the tonal realization of prosodic units alone.  

Further evidence for recursion come from syntax, as briefly shown above. In section \sectref{sec:the-syntax-of-intonation-units}, I discuss this aspect more thoroughly by comparing the syntactic content of singleton IUs with that of embedded, non-final IUs of CIUs. 



\is{intonational model|)}
\is{phrase tone|)}
\is{boundary tone|(}

\section{Intonation Units and grammatical units in Totoli}
\label{sec:the-syntax-of-intonation-units}

\is{prosody-syntax interface|(}

In the previous Chapter  \ref{IU-model}, I presented an in-depth analysis of the tonal patterns of prosodic units in Totoli. In this section, I will investigate the syntactic content of prosodic units in the Totoli corpus (see \sectref{Corpus}) and analyze the grammatical units they typically contain. Specifically, I will first investigate which structures are usually found in CIUs, whether they are singleton IUs or Compound IUs. Secondly, I will investigate the syntactic structures embedded in CIUs and compare them to those found in singleton IUs.




In this present work, I adopt a discourse-oriented approach based on a corpus of natural, (semi-)spontaneous, and unscripted speech. Working with such data highlights the  flexibility in the syntactic content of prosodic units. The question arises of the type of syntactic content that can exist within a prosodic unit and whether there are any regularities in the relationship between them.



A confounding factor pertains to the concept of CIUs, which I have introduced in this study, referring to either singleton IUs or CIUs, as distinct from embedded IUs of CIUs. In my analysis, the term CIU denotes those prosodic units that are demarcated by typical boundary cues such as pitch reset and/or pause, as well as other criteria mainly related to pitch, rhythm, and voice quality, as mentioned in  \sectref{IU-Props} above. Therefore, CIU is equivalent to IU as reported in the literature below, as compound intonational units are not posited for these languages.




According to \citet[288]{Ladd_2008}, explicit phonetic definitions are necessary for determining the criteria of IU and prosodic domain types in general. One of the confounding factors he identifies in the segmentation of spontaneous speech is the presumption that the division of syntactic units into prosodic ones reflects syntactic criteria, with many assuming that:
\begin{quote}
	[...] the various prosodic domains are defined by descriptions of how syntactic structure is mapped onto prosodic structure. \citep[289]{Ladd_2008}
\end{quote}


One of the significant achievements in prosodic phonology was the realization that prosodic boundaries systematically differ from syntactic boundaries. This was famously discussed by \citet{chomsky1968sound}, who provided the frequently cited example of right-branching relative clauses\il{English}. The syntactic boundaries are reprinted in \REF{ex:this is the dog syntax} and the prosodic boundaries in \REF{ex:this is the dog prosody}.


\ea
\label{ex:this is the dog}
\ea{
	\label{ex:this is the dog syntax}
	This is [the cat that caught [the rat that stole [the cheese]]]
}

\ex{
	\label{ex:this is the dog prosody}
	this is the cat - that caught the rat - that stole the cheese 	\begin{flushright}\citep[372]{chomsky1968sound}\end{flushright}
}

\z
\z


Such systematic misalignment of syntactic and prosodic boundaries is usually interpreted as the result of mapping a complex syntactic structure onto an ``intuitively `flatter' or `shallower' prosodic structure'" \citep[290]{Ladd_2008}.
Within Prosodic Phonology \citep{nespor1983prosodic, selkirk1986, marina1986prosodic}, mapping constraints which describe the relation between syntactic and prosodic units were formalized. As \citet[62--63]{Fery_2016} puts it: 

\begin{quote}
	The basic idea of all models accounting for the syntax-prosody mapping is that the syntactic component is submitted to an algorithm -- a set of rules or constraints -- the aim of which is to map a prosodic structure to it. Theoretical issues relate to the way this correspondence is formulated as well as to the resulting prosodic constituency.  
\end{quote}


In recent years, new alignment constraints have been proposed, including \isi{Wrap Theory} \citep{truckenbrodt1999relation} and \isi{Match Theory} \citep{selkirk201114}. The underlying assumption of such theories is that syntactic constituents correspond to prosodic units. In Match Theory, this assumption is expressed by the Match Clause, Match Phrase and Match Word constraints \citep[5]{selkirk201114}:

\begin{description}
	\item [i. Match Clause:] A clause in syntactic constituent structure must be matched by a	corresponding prosodic constituent, call it ι [intonational phrase], in phonological
	representation.
	\item [ii. Match Phrase:] A phrase in syntactic constituent structure must be matched by a	corresponding prosodic constituent, call it ϕ [phonological phrase], in phonological
	representation.
	\item [iii. Match Word:] A word in syntactic constituent structure must be matched by a corresponding prosodic constituent, call it ω [phonological word], in phonological
	representation. 
\end{description}	


\citet[289]{Ladd_2008} comments on these accounts:

\begin{quote}
	In my view, it makes no sense to treat accounts like Nespor and Vogel's or Selkirk's as definitions; rather, they are hypotheses, predictions about the correspondence between one type of independently definable structure and another. [...] Unless the syntactic and the phonological structures are defined in their own terms, the whole exercise becomes purely circular.
\end{quote}


Focusing on natural data, works by  \citet{IwasakiTao1993comparative, Schuetze-Coburn1994, Croft_1995, Tao_1996, Iwasaki1996Thai} and more recently \citet{Croft_2007, Park_2002,  Matsumoto_2003} and  \citet{Wouk_2008} have provided detailed descriptions of the syntactic content of IUs as they are found in corpora of spontaneous speech from a variety of typologically unrelated languages. These accounts have shown the flexibility of the syntactic content of IUs but have also revealed some regularities.




One tendency found in these studies is that approximately 50\% of all IUs in a corpus consist of a simple clause, e.g. 47.8\% in \ili{English} \citep[849]{Croft_1995}, 50.5\% in \ili{Wardaman} \citep[12]{Croft_2007}, 47.9\% in \ili{Mandarin Chinese} \citep[72]{Tao_1996}, and 51.7\% in \ili{Sasak} \citep[150]{Wouk_2008}. 

Moreover, there seems to be a considerable number of IUs that consist of a single NP (referred to as `lone NP' by \citealt[12]{Croft_2007}); for instance, 13.7\% in \ili{English} \citep[849]{Croft_1995}, 21.1\% in \ili{Wardaman} \citep[11]{Croft_2007}, 25.9\% in \ili{Mandarin Chinese} \citep[72]{Tao_1996}, and 21.0\% in \ili{Sasak} \citep[150]{Wouk_2008}.



However, genre appears to have a substantial influence on the proportions of IU types. This must be taken into account when comparing the results from different studies, as they vary with regard to the types of data used. 

\citet{Tao_1996, Wouk_2008, Matsumoto_2000, Schuetze-Coburn1994} and \citet{Park_2002} use conversations between two or more participants, whereas \citet{Croft_1995, Croft_2007} bases his analysis on monological data. The cross-linguistic comparison in \citet[12]{Croft_2007} conflates both data types. Another reason why these results should be approached with caution is that different coding conventions have been applied. In two different studies on \ili{Japanese}, the difference in coding conventions leads to an 11.6 percentage point difference in the proportion of clausal IUs in \ili{Japanese} (57.0\% in \citealt[58]{Matsumoto_2000} and 45.4\% in \citealt[3]{IwasakiTao1993comparative}).  These factors have to be taken into consideration when comparing the results on the reported data (see also the comments in \citealt[642]{Park_2002}, \citealt[12]{Croft_2007}, and \citealt[139--144]{Wouk_2008}).

Despite the differences, these studies have provided cross-linguistic evidence which confirms the centrality of the (simple) clause and the lone NP with regard to grammatical structures typically found in IUs. The sentence, on the other hand, appears to be rather difficult to identify in spoken speech:




\begin{quotation}
	
	This is not a  problem  found with most  other  grammatical  units,  such  as the  clause or the phrase, which are generally clearly identifiable in spoken  language. Yet  the  sentence  is  generally  taken  to  be  the  basic  unit  of  syntactic analysis.  On  the  other  hand, a sentence cannot be equated with an IU, the spoken-language analyst's most popular unit of choice for analysis.
	An IU does not grammatically correspond to a sentence, since it frequently is a unit smaller than a sentence and sometimes (though quite rarely) is not a full grammatical constituent at all. \citep[841]{Croft_1995}
	
\end{quotation}


As described above, the majority of IUs contain a clause or a phrase. Other types include IUs consisting of a single connective or an interjection. \citet[11]{Croft_2007} specifically argues that these should similarly be thought of as constituting independent grammatical units. He calls these “lexical IUs”. Another central observation is that the number of broken or interrupted IUs, such as (uncorrected) false starts, disjointed IUs and fragmentary IUs, is very small  \citep[2\% in the corpus on \ili{Wardaman}, see][11]{Croft_2007}.

In sum, speakers produce short stretches of speech which are rarely broken or fragmented and which mostly contain a full grammatical unit. Based on these observations,  \citet[845]{Croft_1995} proposes the \textit{\isi{full Grammatical Units condition}}:

\begin{quote}
	The overwhelming preference for IUs to be in the form of full GUs [grammatical units], other things being equal, will be called the \textit{full GU condition}.
\end{quote} 

In the same article, \citet[872]{Croft_1995} offers a possible explanation to account for both (a) the small number of broken IUs and (b) the high number of rather short and syntactically simple IUs. Croft calls it the \textit{IU storage hypothesis}.


Intonation Units are explained as cognitive units and are considered ``linguistic expressions of focuses of consciousness, whose properties apparently belong to our built-in information-processing capabilities" \citep[48]{chafe1980pear}. As there is no inherent constraint on the size of IUs per se, there must be some sort of cognitive limitation. Croft's (\citeyear[873]{Croft_1995}) \textit{\isi{IU storage hypothesis}} suggests that Intonation Units consist of grammatical units that are stored or precompiled in the memory of the speaker. He argues that this accounts for the overwhelming frequency of IUs consisting of a single clause or single NP. More complex structures need to be computed based on the precompiled or stored grammatical units, which is why complex structures, such as multiply embedded NPs, rarely occur in spontaneous speech and are usually broken across several IUs. \citet[873]{Croft_1995} explains this in terms of the cognitive limitations of humans:

\begin{quotation}
Stored/precompiled constructions – and IUs themselves – may be the manifestation of the limitations of short-term memory in processing. The IU storage hypothesis suggests that grammatical structure and organization have evolved to conform to the limitations as well as the capacities of the human mind, specifically those embodied in IU structure.

\end{quotation}


\citet[108]{chafe1994discourse} offers another cognitively grounded explanation to account for the types of structures typically found in IUs,  the  \textit{\isi{One New Idea Constraint}}:

\begin{quotation}
	Conversational language appears subject to a constraint that limits an intonation unit to the expression of no more than one new idea. 
\end{quotation}

Chafe argues that speakers can only activate one concept at a time and the IU is the basic unit used to express this cognitive process. If a simple clause with one predicate and one argument is the typical exponent of an IU, then only one of the two elements expresses a new idea. Chafe acknowledges that counterexamples from spontaneous speech are plentiful and offers a variety of explanations for such structures.  
\citet{himmelmann_IP_Universal} offer a way to measure the information content of an IU by computing the average number of content words an IU typically contains. Their study on four typologically unrelated languages found an average of 1.6-1.8 content words per IU (see also the discussion on various ways of measuring the length of IUs in \sectref{sec:the-length-of-intonation-units-of-the-corpus}). 


The \textit{\isi{One New Idea Constraint}} is limited to those IUs that   \citet[63]{chafe1994discourse} refers to  as \textit{substantive IUs}, \is{Substantive Intonation Unit} that is, those which express ideas of events, states or referents. 
The other major IU type is \textit{regulatory IUs}, \is{Regulatory Intonation Unit} which regulate interaction and \isi{information flow}. A third and minor IU type is \textit{fragmentary IUs}. \is{Fragmentary Intonation Unit}


On that matter, \citet[119]{chafe1994discourse} comments:


\begin{quotation}
	In any case, the finding that people can activate only one new idea at a  time, as well as the insight that finding gives us into what it means to constitute ``one idea," may be at least as important as the finding that short-term memory is limited to seven items plus or minus two \citep{Miller_1994}. The magical number one appears to be fundamental to the way the mind handles the flow of information through consciousness and language. 
\end{quotation}

Yet, cognitive limitations are not the only constraints at work. Research by \citet[674]{Park_2002} has shown that the IU is a “resource that participants in an interaction may use and manipulate to achieve their interactional goals.” He shows that  \textit{substantive IUs}, too, are subject not only to cognitive constraints but also to interactional constraints. In what follows, I contrast different aspects of the IU as they occur in conversation and in monological recordings. I show that genre has a substantial effect on the size of IUs, and detail the proportion of (Chafeian) IU types and the proportion of IUs with regard to their syntactic content, among other results.









In this section, I investigate the grammatical structures that prosodic units typically contain. It is organized into four parts: In the first part, \sectref{sec:grammatical-units-and-the-intonation-unit}, I explore the grammatical structures found in CIUs -- either singleton IUs or CIUs. In the second section, \sectref{sec:grammatical-units-and-intermediary-phrases}, I explore the grammatical structures found in embedded IUs of CIUs. The following section, \sectref{sec:comparing-ius-with-ips}, aims to compare the two from a syntactic point of view. In section \sectref{sec:summary},  I review the findings from the analysis of tonal patterns occurring at the edges of prosodic units and revisit the evidence for recursive embedding of IUs in Totoli with the evidence from syntax.





% which suggest that IUs, whether embedded in CIUs or occuring as singletons, are essentially the same type of units with regard to their intonation. The results from the comparison of grammatical structures  provide further syntactic evidence to that hypothesis. 

%In light of these insights, I propose an alternative model of the Intonation Unit, in which complex IUs are recursively parsed into a string of IUs. A string of IUs together form a larger unit which may be termed a Compound Intonation Unit \citep[after][297]{Ladd_2008}.





\subsection{Syntactic structures of singleton IUs and CIUs}
\label{sec:grammatical-units-and-the-intonation-unit}

I briefly discussed above several studies that have addressed the question of what grammatical structures are typically found in IUs \citep[]{Croft_1995, Croft_2007, Tao_1996, Park_2002, Schuetze-Coburn1991, Schuetze-Coburn1994, Matsumoto_2003, Iwasaki1996Thai, IwasakiTao1993comparative, Wouk_2008}. The studies vary substantially with regard to the data they are based on. Yet, it is to be expected that genre has a substantial influence on the proportions of IU types. The influence of genre has been anticipated by \citet[836]{du1987discourse}:
\begin{quotation}
	It is worth emphasizing that, while conversation may well be the more frequent genre, narrative is especially likely to display conditions of relatively high information pressure (...) The heavy information pressure demands in narrative may well give it significance beyond what it otherwise would have for the adaptive shaping of grammar in response to discourse needs.
\end{quotation}



Despite this obvious fact, \citet[12]{Croft_2007} makes cross-linguistic claims about the syntactic nature of IUs by comparing the proportions of grammatical structures reported in different studies. The present work is based on a corpus of conversational and monological recordings, enabling a comparison of the proportions of different types of CIUs within these two data types. The analysis presented here systematically investigates the influence of genre -- monological versus conversational -- and demonstrates its strong impact on the proportions of grammatical structures found in CIUs. Comparing other, more subtle subtypes of genre is also conceivable and is likely to yield slightly different results concerning the distribution of the syntactic nature of prosodic units. In this chapter, I focus on two broad categories---conversations versus monologues---only, as per \citeauthor{du1987discourse}'s \citeyear{du1987discourse} indication  that the difference in information pressure is most pronounced in these categories.





\subsubsection{Methodology and coding conventions}
\label{sec:methodology-and-coding-conventions}


The analysis presented here examines the CIU, which refers to either a singleton IU or a Compound IU, as the primary unit of investigation. The study aims to determine the typical grammatical structures found within these units. As a result, the CIU is the sole domain for coding. It is a prosodic unit that consists of either a singleton IU or a CIU made up of a sequence of IUs, which is delimited by boundary marking cues such as a pause, a break in pitch contour, and a pitch reset. 

This is illustrated in the three CIUs presented in examples (\ref{ex:nilantumnamo}--\ref{ex:sellengget}). The second singleton IU in \REF{ex:sellengget} contains the noun \textit{sɛllɛŋɡɛt} ‘one basket’, which can be analyzed either as an argument to the verb in the preceding singleton IU in \REF{ex:nilantumnamo} or as part of the following CIU in \REF{ex:nisakena dɛi sapena danna <ipoa>}. However, because it constitutes its own singleton IU, (\ref{ex:sellengget}) is analyzed as a nominal IU, while both \REF{ex:nilantumnamo} and \REF{ex:nisakena dɛi sapena danna <ipoa>} are considered clausal CIUs.



\begin{figure}
	\includegraphics[
	height=0.3\textheight
	]{figures/nilantumnamo_sellengget_nisakena_dei_sapena_danna__ipoa__pearstory_14_SP_extSP_plot.png}
	\caption{Periogram with pitch track (in st) for example \REF{ex:nilantumnamo sellengget}, speaker SP}
	\label{pitch:nilantumnamo sellengget}
\end{figure}


\ea
\label{ex:nilantumnamo sellengget}
\ea{
	\label{ex:nilantumnamo}
	\textit{ilantumnamɔ} \\
	\gll ni-lantum-0=na=mɔ \\
	\textsc{rls-}bring.along\textsc{-uv}=3\textsc{s.gen=cpl}\\
	\glt `he brought (it)'
}

\ex{
	\label{ex:sellengget}
	\textit{sɛllɛŋɡɛt} \\
	\gll so-\textsc{rdp}-lɛŋɡɛt\\
	\textsc{one-rdp}-basket\\
	\glt `a basket'
}


{
	\ex
	\label{ex:nisakena dɛi sapena danna <ipoa>}
	\textit{sakɛna dɛi sapɛda danna <ipoa>} \\
	\gll  sakɛ-0=na dɛi sapɛda danna  \\
	get.on\textsc{-appl=3s}.\textsc{gen} \textsc{loc} cycle then  \\
	\glt `he put (it) on the bicycle and then'\hfill(pearstory\_14\_SP.019-21)
		\osflink{65a5503c1cba3e0c1d018566}{nilantumnamo_sellengget_nisakena_dei_sapena_danna__ipoa__pearstory_14_SP_extSP.wav}
}
\z
\z

The structural relations between CIUs are only examined in relation to specific aspects of their internal distinctions, which will be discussed in the following sections. In order to provide adequate context and facilitate understanding for the reader, I include the CIUs adjacent to the examples being discussed, although they may not always be explicitly elaborated upon.

\subsubsection{Discussion of CIU types}\label{sec:discussion-of-iu-types}

Following \citet{Tao_1996}, I differentiate four  categories: (a) clausal CIUs, (b) nominal CIUs, (c) interactional CIUs and (d) other minor types. I will discuss these in the following sections.



\subsubsubsection{Clausal CIUs}\label{sec:clausal-ius}

In this study, a clausal CIU is defined as one that contains at least one predicate. Two types of clausal CIUs are distinguished: independent and dependent. Moreover, independent CIUs are further classified based on the number of overtly expressed arguments. The definitions of these categories are detailed in the following sections.

\subparagraph{Independent clausal CIUs}
\label{sec:independent-clausal-cius} 

The simplest form of an independent clausal CIU comprises a verbal predicate and a single overtly expressed argument, which may be either a lexical NP or a pronoun. An example of this is provided in \REF{ex:isia nabbabag}, which illustrates a basic independent clause containing a preverbal pronominal argument \textit{isia} `he'.

\begin{figure}
	\includegraphics[
	height=0.3\textheight
	]{figures/isia_nabbabag_pearstory_15_IRN_extIRN_plot.png}
	\caption{Periogram with pitch track (in st) for example \REF{ex:isia nabbabag}, speaker IRN}
	\label{pitch:isia nabbabag}
\end{figure}

\ea
\label{ex:isia nabbabag}
\textit{isia nabbabaɡ} \\
\gll isia nɔ-\textsc{rdp}-babaɡ \\
\textsc{3s} \textsc{st.rls-rdp}-crash.into\\ 
\glt ‘he crashed into (it)’\hfill (pearstory\_15\_IRN.009)
	\osflink{65a5501d1cba3e0c1d01855a}{isia_nabbabag_pearstory_15_IRN_extIRN.wav}
\z



In Totoli, the agent argument of the undergoer voice is often realized as a clitic pronoun on the verb (for an explanation of the voice system, see \cite{Riesberg_2019, Riesberg_2014_Symmetrical_Voice}). In the conventions of this study, such constructions are categorized as simple clauses with one overtly expressed argument. For instance, consider the three IUs in \REF{ex:niuntudnamoko sapeo itu nibeen}. The initial singleton IU \REF{ex:niuntudnamoko} comprises a verb with the agent argument expressed as the enclitic \textit{=na} `\textsc{3s.gen}'. The second CIU 	\REF{ex:niuntudmoko sapeo itu} features the same verb, but the undergoer argument is expressed as the lexical NP \textit{sapɛɔ itu} `this hat', while the agent argument is unexpressed. The third singleton IU \REF{ex:nibeennamai} also contains a verb with the agent argument realized as an enclitic on the verb. According to the coding conventions employed in this study, all three CIUs are classified as simple independent clausal CIUs with one overtly expressed argument.


\begin{figure}
	\includegraphics[
	height=0.3\textheight
	]{figures/niuntudnamoko_niuntudmoko_sapeo_itu_nibeennamai_pearstory_17_SNG_extSNG_plot.png}
	\caption{Periogram with pitch track (in st) for example \REF{ex:niuntudnamoko sapeo itu nibeen}, speaker SNG}
	\label{pitch:niuntudnamoko sapeo itu nibeen}
\end{figure}


\ea
\label{ex:niuntudnamoko sapeo itu nibeen}
\ea{
	\label{ex:niuntudnamoko}
	\textit{niuntudnamɔkɔ} \\
	\gll ni-untud-0=na=mɔ=kɔ \\
	\textsc{rls}-bring\textsc{-uv}=3\textsc{s}.\textsc{gen=cpl=and}\\
	\glt `he brought (it)'
}

\ex{
	\label{ex:niuntudmoko sapeo itu}
	\textit{niuntudmɔkɔ sapɛɔ itu} \\
	\gll ni-untud-0=mɔ=kɔ sapɛɔ itu \\
	\textsc{rls}-bring\textsc{-uv=cpl=and} hat \textsc{dist}\\
	\glt `(he) brought this hat'
}

\newpage

{
	\ex
	\label{ex:nibeennamai}
	\textit{nibɛɛnnamai} \\
	\gll  ni-bɛɛn-0=na=mɔ=ai\\
	\textsc{rls}-give-\textsc{uv}=3\textsc{s}.\textsc{gen}=\textsc{cpl}=\textsc{ven}\\
	\glt `he gave (it)'
	\hfill(pearstory\_17\_Sng.101-103)
		\osflink{65a55041f2240f0b4d32e5d1}{niuntudnamoko_niuntudmoko_sapeo_itu_nibeennamai_pearstory_17_SNG_extSNG.wav}
}
\z
\z



Oblique and core arguments are not distinguished in the analysis. Example \REF{ex:ngga noliitaan takin tau dako} illustrates a clause where the negated verb \textit{nɔliitaan} `meet' is followed by an oblique argument introduced by the preposition \textit{takin} `with'. Despite being an oblique argument, this CIU is still coded as a simple independent clausal CIU with one overtly expressed argument.

\begin{figure}
	\includegraphics[
	height=0.3\textheight
	]{figures/ingga_noliitaan_takin_tau_dakolifestory_RDA_1_extRDA_plot.png}
	\caption{Periogram with pitch track (in st) for example \REF{ex:ngga noliitaan takin tau dako}, speaker RDA}
	\label{pitch:ngga noliitaan takin tau dako}
\end{figure}




\ea
\label{ex:ngga noliitaan takin tau dako}
\textit{ŋɡa nɔliitaan takin tau dakɔ} \\
\gll iŋɡa {nɔli-}ita{-an} takin tau dakɔ\\ 
\textsc{neg} \textsc{rcp.rls-}see\textsc{-rcp.rls} with person big\\
\glt ‘(I) haven't met my parents’ \hfill(lifestory\_RDA\_1.024)
	\osflink{65a55018a246ff0abbdd3dc5}{ingga_noliitaan_takin_tau_dakolifestory_RDA_1_extRDA.wav}
\z



Equational predications are analyzed as simple clauses with one nominal predicate and one argument. In example \REF{ex:siritaku ia sirita tau pomoo}, the CIU consists of an equational clause with two elements, \textit{siritaku ia} `my story' and \textit{sirita tau pɔmɔɔ} `the story of a person from the old times'. It is worth noting that Totoli does not use a copula. The CIU in example \REF{ex:siritaku ia sirita tau pomoo} is also considered a simple independent clausal CIU with one overt argument.


\begin{figure}
	\includegraphics[
	height=0.3\textheight
	]{figures/siritaku_ia_sirita_tau_pomoo_lifestory_RDA_1_extRDA_plot.png}
	\caption{Periogram with pitch track (in st) for example \REF{ex:siritaku ia sirita tau pomoo}, speaker RDA}
	\label{pitch:siritaku ia sirita tau pomoo}
\end{figure}




\newpage
\ea
\label{ex:siritaku ia sirita tau pomoo}
\textit{sirita aku ia sirita tau pɔmɔɔ} \\
\gll sirita aku ia sirita tau pɔmɔɔ\\ 
story 1\textsc{s}\textsc{.gen} \textsc{prx} story person first\\
\glt ‘my story is the story of a person from the old times’ \hfill(lifestory\_RDA\_1.014)
	\osflink{65a5506ba246ff0abcdd3d89}{siritaku_ia_sirita_tau_pomoo_lifestory_RDA_1_extRDA.wav}
\z


Totoli has two existential constructions. One construction involves a form of the existential predicate \textit{daan}/\textit{kaddaan}/\textit{dadaan} `\textsc{exist}'. The other construction involves an existential  prefix  \textit{kɔ=}. \is{existential construction}


Examples of the existential prefix are given in the singleton IUs in  \REF{ex:ngga  kabadu} and \REF{ex:ngga  sampang}. The bases \textit{badu} `shirt' and \textit{sampaŋ} `pants' occur with \textit{kɔ=} and in this case with the negator \textit{ŋɡa}. Each constitutes a full clause.

\begin{figure}
	\includegraphics[
	height=0.3\textheight
	]{figures/ingga_kabadu_ingga_kasampang_lifestory_RDA_1_extRDA_plot.png}
	\caption{Periogram with pitch track (in st) for example \REF{ex:NPx}, speaker RDA}
	\label{pitch:NPx}
\end{figure}



\ea
\label{ex:NPx}
\ea{
	\label{ex:ngga  kabadu}
	\textit{ŋɡa  kabadu} \\
	\gll iŋɡa kɔ=badu \\
	\textsc{neg}   \textsc{exist}-shirt\\
	\glt `there were no shirts'
}


{
	\ex
	\label{ex:ngga  sampang}
	\textit{ŋɡa  kasampaŋ} \\
	\gll iŋɡa kɔ=sampaŋ \\
	\textsc{neg}   \textsc{exist}-pants\\
	\glt `there were no pants'
	\hfill(lifestory\_RDA\_1.032-034)
		\osflink{65a55015c585fd0c479ce0a4}{ingga_kabadu_ingga_kasampang_lifestory_RDA_1_extRDA.wav}
}
\z
\z

An example of a singleton IU containing a construction with an existential predicate is given in  \REF{ex:daan taiso'1}.  \is{existential construction}


\begin{figure}
	\includegraphics[
	height=0.3\textheight
	]{figures/kejadianna_daan_taiso__lalau_monipu_piir_pearstory_9_FAH_extFAH_plot.png}
	\caption{Periogram with pitch track (in st) for example \REF{ex:NPdaan}, speaker FAH}
	\label{pitch:NPdaan}
\end{figure}


\ea
\label{ex:NPdaan}
\ea{
	\label{ex:daan taiso'1}
	\textit{daan taisɔl} \\
	\gll daan taisɔl \\
	\textsc{exist} old.man\\
	\glt `there is an old man'
}


{
	\ex
	\label{ex:laalau monipu pir}
	\textit{laalau mɔnipu piir} \\
	\gll \textsc{rdp}-lau mɔN-tipu piir\\
	\textsc{rdp}-presently \textsc{av}-pick pear\\
	\glt `(he) currently picks pears'
	\hfill(pearstory\_9\_FAH.002-4)
		\osflink{65a55022f2240f0b4d32e5ca}{kejadianna_daan_taiso__lalau_monipu_piir_pearstory_9_FAH_extFAH.wav}
}
\z
\z


The constructions presented in \REF{ex:ngga  kabadu}, \REF{ex:ngga  sampang} and \REF{ex:daan taiso'1} are full clauses. The three singleton IUs are considered simple clausal CIUs that consist of one (existential) predicate and one overtly expressed argument. \is{existential construction}

Clausal CIUs which contain  more than one predicate and/or more than two overtly expressed arguments are referred to here as complex clausal CIUs. This includes CIUs containing a simple clause and a subordinate clause, e.g. a simple clause with a modifying relative clause or with an adverbial clause. Other cases are  two coordinated clauses or a main clause with a complement clause  parsed into one CIU. It is important to note that in Totoli, as well as in the local (Manado) Malay variety, a negated existential predicate is often used to negate entire clauses. In the counts used in this study, such constructions will appear as complex clausal CIUs since they involve two predicates. Such a construction is given  in example  \REF{ex:ha ingga daan parhatikanna}. The clause \textit{parhatikanna tau ipanau ia} `he notices the people below' is negated with the negated existential predicate \textit{daan}. \is{existential construction}


\begin{figure}
	\includegraphics[
	height=0.3\textheight
	]{figures/injan_itai_paapasoonako__pene__baboko_dei_tau_lau_mogipu_togu_pir_ia_ha_ingga_daan_parhatikanna_tau_ipanau_ia_pearstory_12_RSTM_extRSTM_plot.png}
	\caption{Periogram with pitch track (in st) for example \REF{ex:ha ingga daan parhatikanna}, speaker RSTM}
	\label{pitch:ha ingga daan parhatikanna}
\end{figure}




\ea


\label{ex:ha ingga daan parhatikanna}
\textit{ha iŋɡa daan parhatikanna tau ipanau ia} \\
\gll ha iŋɡa daan parhatikan-0=na tau i-panau ia \\
\textsc{intj} \textsc{neg} \textsc{exist} pay.attention\textsc{-uv}=3\textsc{s}.\textsc{gen} person \textsc{loc-}under \textsc{prx}\\
\glt `He didn't notice the people below there'\hfill(pearstory\_12\_RSTM.090-92)
	\osflink{65a550171cba3e0c1c018350}{injan_itai_paapasoonako__pene__baboko_dei_tau_lau_mogipu_togu_pir_ia_ha_ingga_daan_parhatikanna_tau_ipanau_ia_pearstory_12_RSTM_extRSTM.wav}

\z


An instance of  two coordinate clauses parsed into one CIU is given in example  \REF{ex:giigii mellegesan giigii meggegesan}. Note that no coordinating conjunction occurs. 

%\begin{dont-break}
\begin{figure}
	\includegraphics[
	height=0.3\textheight
	%,
	% width=1\textwidth
	]{figures/giigi_mellegesan_giigi_meggegesan_explanation-lelegesan_SYNO_extSYNO_plot.png}
	\caption{Periogram with pitch track (in st) for example \REF{ex:giigii mellegesan giigii meggegesan}, speaker SYNO}
	\label{pitch:giigii mellegesan giigii meggegesan}
\end{figure}



\ea
\label{ex:giigii mellegesan giigii meggegesan}
\textit{ɡiiɡii mɛllɛɡɛsan ɡiiɡii mɛɡɡɛɡɛsan} \\
\gll \textsc{rdp}-ɡii mo-lɛlɛɡɛsan \textsc{rdp}-ɡii mo-\textsc{rdp}-ɡɛɡɛs-an\\ 
\textsc{rdp}-different \textsc{av}-Lelegesasn \textsc{rdp}-different \textsc{av}-\textsc{rdp}-rub-\textsc{appl}\\
\glt ‘Singing Lelegesan is different from rubbing (your body).’ \\ \textit{(lit. ‘Singing Lelegesan is different (and) rubbing  is different.')} \begin{flushright}(explanation-lelegesan\_SYNO.032)
	\osflink{65a550101c92110a81abea5b}{giigi_mellegesan_giigi_meggegesan_explanation-lelegesan_SYNO_extSYNO.wav}\end{flushright}
\z


Example \REF{karena isia nogitai sapeona itu geiga noitana batu dɛi dulak} is another instance of a complex clausal CIU which involves an adverbial clause  and its matrix clause parsed in a single CIU. 

\begin{figure}
	\includegraphics[
	height=0.3\textheight
	]{figures/nadabu_sapeona_karena_____isia_nogitai_sapeona_itu_geiga_noitana_batu_dei_dulak_pearstory_9_FAH_extFAH_plot.png}
	\caption{Periogram with pitch track (in st) for example \REF{ex:AdvCl + Simple Cl}, speaker FAH}
	\label{pitch:AdvCl + Simple Cl}
\end{figure}



\ea
\label{ex:AdvCl + Simple Cl}
\ea{
	\label{ex:nadabu sapeona}
	\textit{nadabu sapɛɔna} \\
	\gll nɔ-dabu sapɛɔ=na \\
	\textsc{st.rls}-fall hat=3s.\textsc{gen}\\
	\glt `his hat fell'
}

{
	\ex
	\label{karena isia nogitai sapeona itu geiga noitana batu dɛi dulak}
	\textit{karɛna isia nɔɡitai sapɛɔna itu ɡɛiɡa nɔitana batu dɛi dulak} \\
	\gll karɛna isia nɔɡ-ita-i sapɛɔ=na itu ɡɛiɡa nɔ-ita-0=na batu dɛi dulak\\
	because 3\textsc{s} \textsc{av.rls}-see\textsc{-appl} hat=3\textsc{s}.\textsc{gen} \textsc{dist} \textsc{neg} \textsc{pot}-see-\textsc{uv=}3\textsc{s}.\textsc{gen} stone \textsc{loc} front\\
	\glt `Because he looks at the hat, he doesn't see the stone in front.'
	\begin{flushright}(pearstory\_9\_FAH.026-27)
		\osflink{65a55036a246ff0abcdd3d78}{nadabu_sapeona_karena_____isia_nogitai_sapeona_itu_geiga_noitana_batu_dei_dulak_pearstory_9_FAH_extFAH.wav}\end{flushright}
}
\z
\z





\subparagraph{Dependent clausal CIUs}
\label{sec:dependent-clausal-cius} 

These  include various adverbial clauses that occur in separate CIUs. An example is provided in  \REF{ex:injan nopulingmo doua llengget itumoko}, where the initial element is a subordinating conjunction  \textit{inʤan} `then' that unambiguously indicates its dependent status. Unlike \REF{karena isia nogitai sapeona itu geiga noitana batu dɛi dulak} above, the adverbial clause and its matrix clause are in two separate CIUs.



\begin{figure}
	\includegraphics[
	height=0.3\textheight
	]{figures/injan_nopulingmo_doua_llengget_itumoko__no-__notumalibmoko_tau_googoot_toalang_itu_pearstory_12_RSTM_extRSTM_plot.png}
	\caption{Periogram with pitch track (in st) for example \REF{ex:free simple AdvCl}, speaker RSTM}
	\label{pitch:free simple AdvCl}
\end{figure}

\ea
\label{ex:free simple AdvCl}
\ea{
	\label{ex:injan nopulingmo doua llengget itumoko}
	\textit{inʤan nɔpuliŋmɔ dɔua llɛŋɡɛt itumɔkɔ} \\
	\gll inʤan nɔ-puliŋ=mɔ dɔua \textsc{rdp}-lɛŋɡɛt itu=mɔ=kɔ \\ after \textsc{st.rls-}full=\textsc{cpl} two \textsc{rdp}-basket \textsc{dist=cpl=and}\\
	\glt `after the two baskets were full'
}

{
	\ex
	\label{ex: notumalibmoko tau goɔɡɔot toalang itu}
	\textit{nɔtumalibmɔkɔ tau ɡɔɡɔɔt tɔalaŋ itu} \\
	\gll  nɔ-t<um>alib=mɔ=kɔ tau \textsc{rdp}-ɡɔɔt tɔalaŋ itu\\
	\textsc{av.rls}-\textsc{<auto.mot>}pass.by\textsc{=cpl=and} person \textsc{rdp}-hold goat \textsc{dist}\\
	\glt `a person passed by holding a goat'
	\hfill(pearstory\_12\_RSTM.064-65)
		\osflink{65a5501af2240f0b5132e96b}{injan_nopulingmo_doua_llengget_itumoko__no-__notumalibmoko_tau_googoot_toalang_itu_pearstory_12_RSTM_extRSTM.wav}
}
\z
\z



In several instances, the subordinated status of a clause is not indicated by a subordinating conjunction. Nonetheless, the intonation and the context clearly indicate its status. This will be discussed further in section \sectref{sec:grammatical-units-and-intermediary-phrases}. An example is provided in the IU in \REF{ex:moniiniligko dɛi dolago terus itu}, which includes an adverbial clause of either temporal or, more likely, causal status. However, no subordinating conjunction specifies one interpretation over the other.



\begin{figure}
	\includegraphics[
	height=0.3\textheight
	]{figures/moniiniligko_dei_dolago_terus_itu_sapeda_nollumpakmoko_dei_batu_pearstory_14_SP_extSP_plot.png}
	\caption{Periogram with pitch track (in st) for example \REF{ex:free AdvCl}, speaker SP}
	\label{pitch:free AdvCl}
\end{figure}

\newpage
\ea
\label{ex:free AdvCl}
\ea{
	\label{ex:moniiniligko dɛi dolago terus itu}
	\textit{mɔniiniliɡkɔ dɛi dɔlaɡɔ terus itu} \\
	\gll mɔN-\textsc{rdp}-siliɡ=kɔ dɛi dɔlaɡɔ tɛrus itu \\
	\textsc{av-rdp-}glance\textsc{=and} \textsc{loc} girl then \textsc{dist}\\
	\glt `he looked at the girl constantly'
}

{
	\ex
	\label{ex:sapeda nollumpakmoko dɛi batu2}
	\textit{sapɛda nollumpakmɔkɔ dɛi batu} \\
	\gll sapɛda nɔ-\textsc{rdp}-lumpak=mɔ=kɔ dɛi batu\\
	bicycle \textsc{st.rls-rdp}-hit.against\textsc{=cpl=and} \textsc{loc} stone\\
	\glt `the bicycle crashes against the stone'
	\hfill(pearstory\_14\_SP.027-28)
		\osflink{65a55030f2240f0b4d32e5cd}{moniiniligko_dei_dolago_terus_itu_sapeda_nollumpakmoko_dei_batu_pearstory_14_SP_extSP.wav}
}
\z
\z


\subsubsubsection{Nominal CIUs}

Nominal CIUs are composed of either a single NP or a relative clause. The latter is included here because it is equally referential, hence the label ``nominal CIUs" rather than ``NP-CIUs" (cf. \citealt[13]{Croft_2007} and \citealt[79]{Tao_1996}).

\citet[13]{Croft_2007} presents a basic categorization of nominal CIUs into three types:



\begin{description}
	\item [{Independent}:] are those nominal CIUs which have no structural relation with any of the adjacent intonation units.
	\item [{Parallel}:] are separate CIUs containing ``conjoined  or  appositive  NPs" \citep[13]{Croft_2007}.
	\item [{Arguments}:] are nominal CIUs that have a structural relationship with a neighboring CIU; i.e. they can be analyzed as an argument to a predicate of an adjacent clausal CIU.
\end{description}


The immediately adjacent CIUs are taken as the domain for category-internal classification of nominal IUs. A nominal CIU is considered an argument CIU if it can be analyzed as an argument of a clausal CIU immediately preceding or following it. Relative clauses with a head noun in the immediately preceding CIU are classified as parallel. Most free relative clauses that constitute their own CIUs can be analyzed as arguments and are classified as such; otherwise, they are classified as independent. Examples of all of these subtypes of nominal CIUs are provided below.





\subparagraph{Nominal argument CIUs}
\label{sec:lone-nps}


Example \REF{ex:memang sistim kokoluargaan} is an instance of a nominal argument CIU. It serves as an argument  to the existential predicate in the subsequent CIU in \REF{ex:musti dadaanpo}.

\begin{figure}
	\includegraphics[
	height=0.3\textheight
	]{figures/memang_sistim_kokoluargaan__musti_dadaanpo_masih_kuat_explanation-wedding-tradition_ZBR_extZBR_plot.png}
	\caption{Periogram with pitch track (in st) for example \ref{ex:lon NPs}, speaker ZBR}
	\label{pitch:lon NPs}
\end{figure}



\ea
\label{ex:lon NPs}
\ea{
	\label{ex:memang sistim kokoluargaan}
	\textit{mɛmaŋ sistim kɔkɔluarɡaan} \\
	\gll mɛmaŋ sistim kɔ-kɔluarɡa-an \\
	in.fact system \textsc{nr-}family\textsc{-nr}\\
	\glt `in fact the family system'
}

\ex{
	\label{ex:musti dadaanpo}
	\textit{musti dadaanpɔ} \\
	\gll musti \textsc{rdp-}daan=pɔ \\
	have.to \textsc{rdp-exist=incpl}\\
	\glt `has to remain'
}


{
	\ex
	\label{ex:mekelegpo}
	\textit{mɛkɛlɛɡpɔ} \\
	\gll mɔ-kɛlɛɡ=pɔ \\
	\textsc{st}-strong=\textsc{incpl}\\
	\glt `stay strong'
	\hfill(explanation-wedding-tradition\_ZBR.249-251)
		\osflink{65a5502df2240f0b4c32e47b}{memang_sistim_kokoluargaan__musti_dadaanpo_masih_kuat_explanation-wedding-tradition_ZBR_extZBR.wav}
}
\z
\z



A nominal CIU may also contain an NP with a modifier. For instance, in \REF{ex:advokad anu nilantumnako ia}, the CIU contains the NP \textit{adfɔkaat} ‘avocado’ and the modifying relative clause \textit{anu nilantumnakɔ ia} ‘which he brought’. This CIU is a nominal argument CIU as it serves as the argument of the verb in the following clausal, singleton IU in  \REF{ex:nakakabmoko}.

\begin{figure}
	\includegraphics[
	height=0.3\textheight
	%,
	% width=1\textwidth
	]{figures/nollumpak_dei_batu_advokad_anu_nilantumnako_ia_nakakabmoko_pearstory_11_SP_extSP_plot.png}
	\caption{Periogram with pitch track (in st) for example \REF{ex:NP with Rel-.}, speaker SP}
	\label{pitch:NP with Rel-.}
\end{figure}



\ea
\label{ex:NP with Rel-.}
\ea{
	\label{ex:nollumpak dɛi batu}
	\textit{nɔllumpak dɛi batu} \\
	\gll nɔ-\textsc{rdp}-lumpak dɛi batu\\
	\textsc{st-rdp}-hit.against \textsc{loc} stone\\
	\glt `(he) crashed against the stone'
}

\ex{
	\label{ex:advokad anu nilantumnako ia}
	\textit{adfɔkaat anu nilantumnakɔ ia} \\
	\gll alpukaat anu ni-lantum-0=na=kɔ ia \\
	avocado \textsc{rel} \textsc{rls}-bring.along-\textsc{uv}=3\textsc{s}\textsc{.gen=and} \textsc{prx}\\
	\glt `the avocados which (he) brought'
}


{
	\ex
	\label{ex:nakakabmoko}
	\textit{nakakabmɔkɔ} \\
	\gll no-kakab=mɔ=kɔ \\
	\textsc{st.rls}-pour=\textsc{cpl=and}\\
	\glt `scattered/poured'
	\hfill(pearstory\_11\_SP.027-28)
		\osflink{65a55042a246ff0ac1dd3d75}{nollumpak_dei_batu_advokad_anu_nilantumnako_ia_nakakabmoko_pearstory_11_SP_extSP.wav}
}
\z
\z


CIUs consisting of a headless relative clause are argument CIUs if they serve as an argument to an adjacent CIU.  Example \REF{ex:anu saasalu dɛi puun kayu} is an instance of a headless relative clause phrased as a separate CIU. It serves as the argument to the verb in the preceding clausal CIU \REF{ex:sukati itaita}.

\begin{figure}
	\includegraphics[
	height=0.3\textheight
	]{figures/sukati_itaita_anu_saasalu_dei_puun_kayu_spacegames_sequence2_KSR-SP_extSP_plot.png}
	\caption{Periogram with pitch track (in st) for example \REF{ex:Rel wo head}, speaker SP}
	\label{pitch:Rel wo head}
\end{figure}




\ea
\label{ex:Rel wo head}
\ea{
	\label{ex:sukati itaita}
	\textit{sukati itaita} \\
	\gll sukat-i ita-i=ta\\
	try-\textsc{uv} see-\textsc{appl=2s.gen}\\
	\glt `try to look for'
}

{
	\ex
	\label{ex:anu saasalu dɛi puun kayu}
	\textit{anu saasalu dɛi puun kaju} \\
	\gll anu \textsc{rdp}-salu dɛi puun kaju \\
	\textsc{rel} \textsc{rdp}-facing \textsc{loc} tree wood\\
	\glt `the one facing the tree'
	\hfill(spacegames\_sequence2\_KSR-SP.198-199)
		\osflink{65a55077f2240f0b4d32e5ec}{sukati_itaita_anu_saasalu_dei_puun_kayu_spacegames_sequence2_KSR-SP_extSP.wav}
}
\z
\z




\subparagraph{Parallel nominal CIUs}
\label{sec:parallel-nominal-cius}

The CIUs in \REF{ex:saakan tau montoliusat} and  \REF{ex:montoliamang} are examples of the parallel type. They are appositives of the argument \textit{sasaakan} `everybody' in the first CIU.


\begin{figure}
	\includegraphics[
	height=0.3\textheight
	]{figures/bali_long_name.png}
	\caption{Periogram with pitch track (in st) for example \REF{ex:Parallel NP IUs}, speaker SP}
	\label{pitch:Parallel NP IUs}
\end{figure}



\ea
\label{ex:Parallel NP IUs}
\ea{
	\label{ex:bali nnea ia mollinjon sasaakan}
	\textit{bali nnɛa ia mɔllinʤɔn sasaakan} \\
	\gll bali nɛnɛa ia mɔ-\textsc{rdp}-linʤɔn sasaakan\\
	so today \textit{prx} \textsc{av-rdp}-gather everybody\\
	\glt `so today everybody gathers'
}

\ex{
	\label{ex:saakan tau montoliusat}
	\textit{ssaakan tau mɔntɔliusat} \\
	\gll sasaakan tau mɔntɔli-usat\\
	all person \textsc{be.related.as}-related\\
	\glt `all the relatives'
}


{
	\ex
	\label{ex:montoliamang}
	\textit{mɔntɔliamaŋ} \\
	\gll mɔntɔli-amaŋ \\
	\textsc{be.related.as}-father\\
	\glt `the relatives of the father'
	\hfill(explanation-wedding-tradition\_ZBR.023-25)
		\osflink{65a54ff8a246ff0abcdd3d61}{bali_nea_ia_mollinjo_sasaakan_explanation-wedding-tradition_ZBR_extZBR.wav}
}
\z
\z




The singleton IU in \REF{ex:anu lau penek tau pagauan ia dɛi alung alpokat ia} is an example of a relative clause with its head in the preceding CIU, shown in \REF{ex:nallakomoko notumalibmoniko dɛi alung puun kayu}. It is analyzed here as a nominal CIU of the parallel type, as it modifies the head NP \textit{puun kaju} ‘the tree’ in the preceding CIU.

\begin{figure}
	\includegraphics[
	height=0.3\textheight
	]{figures/bali_long_name.png}
	\caption{Periogram with pitch track (in st) for example \REF{ex:Rel with head}, speaker SP}
	\label{pitch:Rel with head}
\end{figure}



\ea
\label{ex:Rel with head}
\ea{
	\label{ex:bali singgaian nibeenannako alpukat kalangena itu}
	\textit{bali siŋɡaian nibɛɛnannakɔ alpukaat kalaŋɛna itu} \\
	\gll bali siŋɡaian ni-bɛɛn-an=na=kɔ alpukaat kalangɛna itu\\
	so friend \textsc{rls}-give-\textsc{appl=}3\textsc{s}.\textsc{gen=and} avocado a.moment.ago \textsc{dist}\\
	\glt `so the friends who were given the avocado earlier'
}

\newpage
\ex{
	\label{ex:nallakomoko notumalibmoniko dɛi alung puun kayu}
	\textit{nallakɔmɔkɔ nɔtumalibmɔ nikɔ dɛi aluŋ puun kaju} \\
	\gll nɔ-\textsc{rdp}-lakɔ=mɔ=kɔ nɔ-t<um>alib=mɔ pɔni=kɔ dɛi aluŋ puun kaju \\
	\textsc{av.rls-rdp}-walk=cpl=and \textsc{av.rls}-\textsc{<auto.mot>}pass.by\textsc{=cpl} again\textsc{=and} \textsc{loc} under tree wood\\
	\glt `they walked past, below the tree'
}


{
	\ex
	\label{ex:anu lau penek tau pagauan ia dɛi alung alpokat ia}
	\textit{anu lau pɛnɛk tau paɡauan ia dɛi alung alpukaat ia} \\
	\gll anu lau pɛnɛk-0 tau pɔ-ɡauan ia dɛi alung alpukaat ia \\
	\textsc{rel} presently climb-\textsc{uv} person \textsc{ger-}garden \textsc{prx} \textsc{loc} under avocado \textsc{prx}\\
	\glt `which was just climbed up by the farmer under the avocados'
	\begin{flushright}(pearstory\_11\_SP.043-45)
		\osflink{65a54ffc1cba3e0c1c018348}{bali_singgaian_nibeenannako_alpukat_kalangena_itu_nallakomoko_notumalibmoniko_dei_alung_puun_kayu_anu_lau_penek_tau_pagauan_ia_dei_alung_alpokat_ia_pearstory_11_SP_extSP.wav}\end{flushright}
}
\z
\z

\subparagraph{Independent nominal CIUs}
\label{sec:independent-nominal-cius} 

These are nominal CIUs -- often singleton IUs -- which cannot be analyzed as bearing a structural relation with any adjacent CIU. They often perform a topic-introducing function \citep[13]{Croft_2007}, as in example \REF{ex:kejadianna}. 


\begin{figure}
	\includegraphics[
	height=0.3\textheight
	]{figures/hm_kejadianna_daan_taiso__lalau_monipu_pir_pearstory_9_FAH_extFAH_plot.png}
	\caption{Periogram with pitch track (in st) for example \REF{ex:Independent nom IU}, speaker FAH}
	\label{pitch:Independent nom IU}
\end{figure}




\ea
\label{ex:Independent nom IU}
\ea{
	\label{ex:hm}
	\textit{hm} \\
	\glt `hm'
}

\newpage
\ex{
	\label{ex:kejadianna}
	\textit{kɛʤadianna} \\
	\gll kɛʤadian=na \\
	event=3s.\textsc{gen}\\
	\glt `the situation'
}


{
	\ex
	\label{ex:daan taiso'}
	\textit{daan taiso{\ü}} \\
	\gll daan taisɔ{\ü} \\
	\textsc{exist} old.man\\
	\glt `there is an old man'
	\hfill(pearstory\_9\_FAH.001-3)
		\osflink{65a55023f2240f0b5132e985}{kejadianna_daan_taiso__lalau_monipu_pir_pearstory_9_FAH_extFAH.wav}
}
\z
\z



\subsubsubsection{Interactional CIUs}

This  category includes CIUs -- often singleton IUs -- that  consist of an interjection such as  \textit{eh, mm, io} and other discourse markers; see example \REF{ex:aa}.


\begin{figure}
	\includegraphics[
	height=0.3\textheight
	]{figures/douamo_anu_nopool_aa_sia_nemenek_ulang_magalai_poni_monuangan_dei_sallo_pearstory_13_RD_extRD_plot.png}
	\caption{Periogram with pitch track (in st) for example \REF{ex:Rel with head2}, speaker RD}
	\label{pitch:Rel with head2}
\end{figure}




\ea
\label{ex:Rel with head2}
\ea{
	\label{ex:douamo anu nopool}
	\textit{dɔuamɔ anu nɔpɔɔl} \\
	\gll dɔua=mɔ anu nɔ-pɔɔl\\
	two=\textsc{cpl} \textsc{rel} \textsc{st.rls}-full\\
	\glt `two are already full'
}

\ex{
	\label{ex:aa}
	\textit{aa} \\
	\gll aa \\
	\textsc{intj}\\
	\glt `aa'
}


{
	\ex
	\label{ex:sia nemenek ulang magalai poni monuangan dɛi sallo}
	\textit{sia nɛmɛnɛk ulaŋ maɡalai pɔni mɔnuaŋan dɛi sallɔ} \\
	\gll isia nɔN-pɛnɛk ulaŋ mɔɡ-ala=ai pɔni mɔN-suaŋ-an dɛi sallɔ \\
	3\textsc{s} \textsc{av.rls-}climb repeat \textsc{av-}fetch=\textsc{ven} again \textsc{av}-fill-\textsc{appl} \textsc{loc} basket\\
	\glt `he climbs again, to take again and put it in the basket'
	\begin{flushright}(pearstory\_13\_RD.025-27)
		\osflink{65a55004a246ff0ac1dd3d5b}{douamo_anu_nopool_aa_sia_nemenek_ulang_magalai_poni_monuangan_dei_sallo_pearstory_13_RD_extRD.wav}\end{flushright}
}
\z
\z

The filler element \textit{anu} is considered an interactional singleton IU only if it occurs as a bare root in a separate singleton IU such as in 	\REF{ex:interactional_anu2}. In the presence of verbal morphology, it is coded as clausal, as it is usually smoothly integrated into the clause structure. See  the two CIUs in examples  \REF{ex:a sagaat naanuanmo alpukat ia}--\REF{ex:sagaat naanuanmo tau nanako ia}. 

\begin{figure}
	\includegraphics[
	height=0.3\textheight
	]{figures/kudabuan_dei_ogo_story-monkey_turtle_RSM_plot.png}
	\caption{Periogram with pitch track (in st) for example \REF{ex:interactional anu}, speaker RSM}
	\label{pitch:interactional anu}
\end{figure}



\ea
\label{ex:interactional anu}
\ea{
	\label{ex:interactional_anu1}
	\textit{tutuŋmɔ} \\
	\gll	\textit{tutuŋ=mɔ} \\
	burn\textsc{=compl} \\
	\glt `Burn me!'
}

\ex{
	\label{ex:interactional_anu2}
	\textit{anu} \\
	\gll anu \\
	\textsc{fill}\\
	\glt `...thinggy...'
}

\newpage
{
	\ex
	\label{ex:interactional_anu3}
	\textit{o, tiana, gɛiga, kudabuan dɛi ɔgɔ} \\
	\gll o tiana gɛiga ku=dabu-an dɛi ɔgɔ\\
	\textsc{intj} 	\textsc{quot} \textsc{neg} \textsc{1sg}=throw\textsc{-uv} \textsc{loc} water\\
	\glt `Oh no, she says, I will throw you in water!'
% 	\begin{flushright}(story-monkey-turtle\_RSM\_extRSM.062-64)
% 		\href{run:figures/kudabuan_dei_ogo_story-monkey_turtle_RSM.wav}{\triangleright}\end{flushright}
\osflink{65bdf972435c450666da7db2}{kudabuan_dei_ogo_story-monkey_turtle_RSM.wav}
}
\z
\z




\begin{figure}
	\includegraphics[
	height=0.3\textheight
	]{figures/sagaat_naanuanmo_alpukat_ia__sagaat_naanuanmo_tau_nanako_ia_pearstory_36_SELP_extSELP_plot.png}
	\caption{Periogram with pitch track (in st) for example \REF{ex:a sagaat naanuanmo}, speaker SELP}
	\label{pitch:a sagaat naanuanmo}
\end{figure}




\ea
\label{ex:a sagaat naanuanmo}
\ea{
	\label{ex:a sagaat naanuanmo alpukat ia}
	\textit{a saɡaat naanuanmɔ alpukaat ia } \\
	\gll a sɔ-ɡaat nɔ-anu-an=mɔ alpukaat  ia\\
	a \textsc{one}-part \textsc{av.rls}-\textsc{fill}-\textsc{appl=cpl} avocado \textsc{prx} \\
	\glt `half of the avocados thingied'
}


{
	\ex
	\label{ex:sagaat naanuanmo tau nanako ia}
	\textit{saɡaat naanuanmɔ tau nanakɔ ia} \\
	\gll   sɔ-ɡaat nɔ-anu-an=mo tau nɔn-takɔ ia \\
	\textsc{one}-part \textsc{av.rls-}\textsc{fill}\textsc{-appl=cpl} person \textsc{av.rls-}steal \textsc{prx} \\
	\glt `half of it was thingied by the thief'
	\hfill(pearstory\_36\_SELP.287)
		\osflink{65a550681c92110a80abea2f}{sagaat_naanuanmo_alpukat_ia__sagaat_naanuanmo_tau_nanako_ia_pearstory_36_SELP_extSELP.wav}
}
\z
\z	








\subsubsubsection{Others}

This category includes adverbs and connectives that occur as single CIUs, as well as prepositional phrases. Additionally, it encompasses fragmentary CIUs and instances of code-switching.

Frequently, an adverb or connective itself forms a separate CIU---often a singleton IU. The most commonly used items in Totoli are \textit{bali} `so', \textit{inʤan} `then', \textit{antuknakɔ} `that is', \textit{tapi} `but', and \textit{danna} `then'. An example is provided in \REF{ex:bali}.


\begin{figure}
	\includegraphics[
	height=0.3\textheight
	]{figures/_b__bali_pogitata_anu_batu_kaddaan_buubunga_spacegames_sequence1_KSR-SP_extSP_plot.png}
	\caption{Periogram with pitch track (in st) for example \REF{ex:bali pogitata anu batu bunga}, speaker SP}
	\label{pitch:bali pogitata anu batu bunga}
\end{figure}



\ea
\label{ex:bali pogitata anu batu bunga}
\ea{
	\label{ex:bali}
	\textit{bali} \\
	\gll bali \\
	so\\
	\glt `so'
}

\ex{
	\label{ex:pogitata anu batu}
	\textit{pɔɡitata anu batu} \\
	\gll pɔɡ-ita-0=ta anu batu\\
	\textsc{sf-}look.for\textsc{-uv=2s.gen} \textsc{rel} stone \\
	\glt `look for a stone'
}


{
	\ex
	\label{ex:kaddaan buubunga}
	\textit{kaddaan buubuŋa} \\
	\gll  kɔ=\textsc{rdp}-daan \textsc{rdp}-buŋa \\
	\textsc{exist}-\textsc{rdp}-\textsc{exist} \textsc{rdp}-flower \\
	\glt `(which) has flowers'
	\hfill
% 	(spacegames\_sequence1\_KSR-SP.055-57)
% 		\href{run:figures/_b__bali_pogitata_anu_batu_kaddaan_buubunga_spacegames_sequence1_KSR-SP_extSP.wav}{\triangleright}
		\osflink{65bdec123280d80676a3abc2}{_b__bali_pogitata_anu_batu_kaddaan_buubunga_spacegames_sequence1_KSR-SP_extSP.wav}
}
\z
\z



Prepositional phrases  involve a preposition and a nominal element in adverbial function forming a single CIU, as in example  (\ref{ex:dɛi dalan babi}). 

\begin{figure}
	\includegraphics[
	height=0.3\textheight
	]{figures/niketembaanmoko_paa_namo_nallakoan_baki_tuku_dei_dalan_babi_lifestory_RDA_1_extRDA_plot.png}
	\caption{Periogram with pitch track (in st) for example \REF{ex:namo nallakoan baki tuku}, speaker RDA}
	\label{pitch:namo nallakoan baki tuku}
\end{figure}



\ea
\label{ex:PP}


\ea{
	\label{ex:namo nallakoan baki tuku}
	\textit{namɔ nallakɔan baki tuku} \\
	\gll namɔ nɔ-\textsc{rdp}-lakɔ-an baki tuku\\
	only \textsc{av.rls-rdp}-walk-\textsc{appl} head knee \\
	\glt `only walking on knees'
}


{
	\ex
	\label{ex:dɛi dalan babi}
	\textit{dɛi dalan babi} \\
	\gll  dɛi dalan babi \\
	\textsc{loc} road pig \\
	\glt `on a secret path'
	\hfill(lifestory\_RDA\_1.072-74) 
		\osflink{65a5503b1c92110a81abea90}{niketembaanmoko_paa_namo_nallakoan_baki_tuku_dei_dalan_babi_lifestory_RDA_1_extRDA.wav}
}
\z
\z

As per \citet[72]{Tao_1996} and \citet[150]{Wouk_2008}, the CIUs grouped under ``Other" include oblique arguments and adverbial adjuncts. Differentiating between these two types of CIUs in the corpus is relatively straightforward, and only a few ambiguous cases were encountered. One simple distinguishing characteristic is optionality. \citet[50]{quirk1985comprehensive} argues that while oblique arguments are usually obligatory, adverbial adjuncts 

\begin{quote}
    ``may be regarded, from a structural point of view, largely as ‘optional extras’, which may be added at will, so that it is not possible to give an exact limit to the number of adverbials a clause may contain.'' 
\end{quote}


Various other elements occur as separate CIUs and cannot be classified under any of the categories mentioned earlier. These are also included in the category ``Other". One example is negatives, as shown in example \REF{ex:o ingga ingga}.



\begin{figure}
	\includegraphics[
	height=0.3\textheight
	]{figures/o_ingga_ingga_ingga_daan_kan_mokodoong_maggalimo_ia_ingga_explanation-wedding-tradition_ZBR_extZBR_plot.png}
	\caption{Periogram with pitch track (in st) for example \REF{ex:}, speaker RDA}
	\label{pitch:ex}
\end{figure}




\ea
\label{ex:}
\ea{
	\label{ex:o ingga ingga}
	\textit{ɔ iŋɡa iŋɡa} \\
	\gll ɔ iŋɡa iŋɡa\\
	o \textsc{neg} \textsc{neg}\\
	\glt `oh no no'
}


{
	\ex
	\label{ex:ingga daan kan mokodoong maggalimo ia ingga}
	\textit{iŋɡa daan kan mɔkɔdɔɔng maɡɡalimɔ ia iŋɡa} \\
	\gll  iŋɡa daan kan mɔkɔ-dɔɔng mɔ-\textsc{rdp}-ɡali=mɔ ia iŋɡa \\
	\textsc{neg} \textsc{exist} perhaps \textsc{st.av-}want \textsc{av-rdp-}stop=\textsc{cpl} \textsc{prx} \textsc{neg}\\
	\glt `there is no longing to stop this, no'
	\begin{flushright}(explanation-wedding-tradition\_ZBR.341-342)
		\osflink{65a550441cba3e0c1c018385}{o_ingga_ingga_ingga_daan_kan_mokodoong_maggalimo_ia_ingga_explanation-wedding-tradition_ZBR_extZBR.wav}\end{flushright}
}
\z
\z	

Other units are numerals, shown in  \REF{ex:sabatu}, and quotative elements, such as in \REF{ex:tiana}. 


\begin{figure}
	\includegraphics[
	height=0.3\textheight
	]{figures/isia_kodoong_modumakit_tiana_sabatu_story-monkey-crocodile_RSM_extRSM_plot.png}
	\caption{Periogram with pitch track (in st) for example \REF{ex:nums and QUOTs}, speaker RDA}
	\label{pitch:nums and QUOTs}
\end{figure}



\ea
\label{ex:nums and QUOTs}
\ea{
	\label{ex:isia kodoong modumakit}
	\textit{isia kɔdɔɔŋ mɔdumakit} \\
	\gll isia kɔ=dɔɔŋ mɔ-d<um>akit \\
	3s \textsc{exist}-want \textsc{av-}\textsc{<auto.mot>}across\\
	\glt `he wants to cross'
}

\ex{
	\label{ex:tiana}
	\textit{tiana} \\
	\gll tiŋana \\
	\textsc{quot}   \\
	\glt `he says'
}


{
	\ex
	\label{ex:sabatu}
	\textit{sabatu} \\
	\gll  sabatu\\
	sabatu \\
	\glt `one'
	\hfill(story-monkey-crocodile\_RSM.030-32)
		\osflink{65a5501bc585fd0c409ce965}{isia_kodoong_modumakit_tiana_sabatu_story-monkey-crocodile_RSM_extRSM.wav}
}
\z
\z



\subsubsection{Distribution and discussion}\label{sec:distribution-of-main-iu-types}


As an initial step, I examine the distribution of the four CIU types: (a) clausal CIUs, (b) nominal CIUs, (c) interactional CIUs, and (d) other minor CIU types.   \figref{Freq_GU_IP_type}  displays the frequency distribution of these CIU types in the corpus, showing both the overall distribution and the distribution within the conversational and monological data.


\begin{figure}
	\includegraphics[
	%height=0.3\textheight,
	width=1\textwidth]{figures/freq_GU_IU.png}
	\caption{Frequency distributions of the four broad categories of CIUs within conversational and monological recordings}
	\label{Freq_GU_IP_type}
\end{figure}



In the entire corpus, clausal CIUs account for 52.7\%, nominal CIUs for 15.9\%, and interactional CIUs for 13.7\%. Other structures make up 17.6\% of the corpus. The difference between conversational and monological data primarily concerns clausal and interactional CIUs. In conversational data, the proportion of clausal CIUs is 12.3 percentage points lower, while the proportion of interactional CIUs is 11.4 percentage points higher.

In Totoli, the clause is a major type of construction that constitutes a CIU. To further examine this type, I present the distribution of various types of clausal CIUs in  \figref{freq_clausal_IU_type}. Dependent clausal CIUs (cf. \sectref{sec:dependent-clausal-cius}) are displayed in the right-hand columns. Independent clausal CIUs (cf.  \sectref{sec:independent-clausal-cius}) are further subdivided into simple clausal CIUs (with zero, one, or two overtly expressed arguments), and complex clausal CIUs (involving more than one predicate and/or more than two arguments).




\begin{figure}
	\includegraphics[
	%height=0.3\textheight,
	width=1\textwidth]{figures/freq_clausal_IU_type.png}
	\caption{Frequency distributions of subcategories of clausal CIUs within conversational and monological recordings}
	\label{freq_clausal_IU_type}
\end{figure}

In both conversational and monological data, there is a notable proportion of clausal CIUs with one overtly expressed argument (47.8\% in conversational data, 41.6\% in monological data), as well as a high proportion of elliptical CIUs, with no overtly expressed verb (33.8\% in conversational data, 20.2\% in monological data). It is also noteworthy that there are no CIUs containing a dependent clause in conversational data.

\figref{freq_N_attachable} offers a detailed breakdown of the various types of nominal CIUs. This includes nominal argument CIUs (cf.  \sectref{sec:lone-nps}), parallel nominal CIUs (cf.  \sectref{sec:parallel-nominal-cius}), and independent nominal CIUs (cf.  \sectref{sec:independent-nominal-cius}).


\begin{figure}
	\includegraphics[
	width=1\textwidth]{figures/freq_N_attachable.png}
	\caption{Distributions of argument, independent and parallel nominal CIUs within conversational and monological recordings; numbers are rounded to one decimal place.}
	\label{freq_N_attachable}
\end{figure}


The data show a high number of independent nominal CIUs and substantial differences between conversational and monological data: 61\% independent nominal CIUs in conversations and 35.5\% in monological data. 

\tabref{Summary IU-types} summarizes the results.

\begin{table}
	\caption{Summary of distributions of different CIU types and subcategories: total proportions are given in the left-hand columns, total numbers in the right-hand columns.}
	\label{Summary IU-types}
	\begin{tabular}{l@{~~}l *8{r}}
		\lsptoprule
		&                        & \multicolumn{2}{c}{all} & \multicolumn{2}{c}{conversational} & \multicolumn{2}{c}{monological} \\\midrule
		\multicolumn{2}{l}{clausal}       & 52.7\%       & 1508      & 45.7\%            & 520            & 57.4\%           & 987          \\
		& 0                      & 13.1\%       & 375      & 15.4\%            & 176             & 11.6\%           & 199          \\
		& 1                      & 23.1\%       & 660      & 21.8\%            & 249             & 243.9\%           & 411          \\
		& 2                      & 5.4\%        & 154       & 3.2\%             & 37              & 6.8\%            & 117           \\
		& complex                & 8.7\%       & 248     & 5.2\%            & 59             & 11.0\%           & 189          \\
		& dependent              & 2.5\%        & 71       & 0.0\%             & 0              & 4.31\%            & 71           \\
		\midrule
		\multicolumn{2}{l}{nominal}       & 15.9\%       & 456      & 13.6\%            & 155             & 17.5\%           & 301          \\
		& argument               & 6.0\%       & 172     & 3.4\%            & 39             & 7.7\%           & 133           \\
		& independent            & 7.1\%       & 202      & 8.3\%            & 95             & 6.2\%           & 107           \\
		& parallel               & 2.9\%       & 82      & 1.8\%             & 21             & 3.5\%           & 61           \\
		\midrule
		\multicolumn{2}{l}{interactional} & 13.7\%       & 393      & 20.6\%            & 235            & 9.2\%            & 158          \\
		\midrule
		\multicolumn{2}{l}{others}        & 17.6\%       & 4504      & 20.2\%            & 230            & 15.9\%           & 274          \\
		& adv. \& con.           & 3.4\%       & 96       & 2.4\%            & 27             & 4.0\%           & 69           \\
		& code-switching         & 1.8\%       & 52       & 1.5\%            & 17             & 2.0\%           & 35           \\
		& fragments              & 2.1\%        & 61       & 3.6\%            & 41             & 1.2\%            & 20           \\
		& further                  & 3.8\%        & 108       & 6.4\%             & 73              & 2.0\%            & 35           \\
		& PP                     & 3.1\%       & 89       & 3.2\%            & 37             & 3.0\%           & 52           \\
		& uncodable             & 3.4\%       & 98       & 3.1\%            & 35             & 3.7\%           & 63          \\
		\midrule
		\multicolumn{2}{l}{\textsc{total}}        & 100\%       & 2861      & 100\%            & 1141            & 100\%           & 1720          \\
		\lspbottomrule
	\end{tabular}
\end{table}




It is important to consider that Totoli is an endangered, understudied language. While \citet{Riesberg_2014_Symmetrical_Voice},  \citet{Himmelmann_2013_symm_voice} and \citet{RiesbergMalcherHimmelmann} provide detailed discussions of the major aspects of the verbal morphology, other aspects of its grammar are still not fully worked out. In some cases, I had to make coding decisions that readers may or may not agree with. One example relevant to the current discussion is the coding of negation with a form of the negated existential predicate \textit{ko=}/\textit{daan}/\textit{kaddaan}. \is{existential construction} This form of negation occurs frequently in both Totoli and the local (Manado) Malay\il{Manado Malay} variety. A simple clause that is negated with an existential predicate will appear as a complex clause in the count, as it involves two predicates: the existential predicate and the predicate of the negated clause. Such decisions must be made for each language based on its unique grammar, and need to be considered when comparing reported data. 

The investigation presented above aimed to answer the question of what grammatical structures a CIU typically contains. I will now turn to a discussion of grammatical structures found in embedded IUs of CIUs.





\subsection{Syntactic structures of embedded IUs of CIUs}
\label{sec:grammatical-units-and-intermediary-phrases}




In this section, I will be discussing several aspects related to the grammatical units found in embedded IUs of CIUs. The analysis is straightforward and is not couched in the framework and coding conventions used above in  \sectref{sec:grammatical-units-and-the-intonation-unit}. Here, I use a limited range of grammatical unit types which are sufficient to describe the majority of structures found, such as noun phrases, verbs, prepositional phrases, adverbial clauses, and relative clauses. To briefly illustrate some of the main aspects, I have provided illustrative examples below.



Noun phrases typically constitute their own (embedded) IUs, and for NPs in preverbal position, this is observed with consistency. An example is provided in \REF{ex:sagaat madabumai dɛi buta}, with the periogram shown in  \figref{pitch:sagaat madabumai dɛi buta}. The simple one-word argument NP `\textit{sagaat} meaning `the half' is parsed into a separate embedded IU, as indicated by the pitch rise on the final syllable. It is followed by an IU containing the verb and a following prepositional phrase. The ``|" symbol indicates a boundary of an embedded IU.

\begin{figure}
	\includegraphics[
	height=0.3\textheight
	]{figures/pearstory_14_SP_extSP_sagaat_madabumai_dei_buta_plot.png}
	\caption{Periogram with pitch track (in st) for example \REF{ex:sagaat madabumai dɛi buta}, a CIU consisting of two embedded IUs, speaker SP}
	\label{pitch:sagaat madabumai dɛi buta}
\end{figure}




\ea
\label{ex:sagaat madabumai dɛi buta}
\textit{saɡaat | madabumai dɛi buta} \\
\gll sɔ-ɡaat mɔ-dabu=mɔ=ai dɛi buta  \\
\textsc{one-}part \textsc{st-}fall\textsc{=cpl=ven} \textsc{loc} earth\\ 
\glt ‘half of it fell to the ground’ \hfill(pearstory\_14\_SP.007)
	\osflink{65a55055a246ff0ac2dd3e14}{pearstory_14_SP_extSP_sagaat_madabumai_dei_buta.wav}
\z




Headless relative clauses are consistently parsed as their own IUs when in preverbal position. For example, consider the CIU in example \REF{ex:anu ampi koloanan saasaluai kita1}  and its corresponding pitch contour shown in  \figref{pitch:anu ampi koloanan saasaluai kita}. The initial IU of the CIU comprises the headless relative clause \textit{anu ampi kɔlɔanan} meaning ‘the one on the right side’.  This is then followed by an IU containing both the verb \textit{saasalu} ‘facing’ and the pronominal NP \textit{kita} ‘us’.

\begin{figure}
	\includegraphics[
	height=0.3\textheight
	]{figures/anu_ampi_koloanan_saasaluai_kita_spacegames_sequence4_KSR-SP_extSP_plot.png}
	\caption{Periogram with pitch track (in st) for example \REF{ex:anu ampi koloanan saasaluai kita1}, a CIU consisting of two embedded IUs, speaker SP}
	\label{pitch:anu ampi koloanan saasaluai kita}
\end{figure}

\newpage

\ea
\label{ex:anu ampi koloanan saasaluai kita1}
\textit{anu ampil kɔlɔanan | saasaluai kita} \\
\gll anu ampi kɔlɔanan \textsc{rdp}-salu=ai kita  \\
\textsc{rel} part right \textsc{rdp}-facing=\textsc{ven} 2\textsc{s}\\ 
\glt ‘the one on the right-hand side is facing you’ \begin{flushright}(spacegames\_sequence4\_KSR-SP.231)
	\osflink{65a54ff2c585fd0c3f9ce657}{anu_ampi_koloanan_saasaluai_kita_spacegames_sequence4_KSR-SP_extSP.wav}\end{flushright}
\z




In examples \REF{ex:sagaat madabumai dɛi buta} and \REF{ex:anu ampi koloanan saasaluai kita1} above, the verb is grouped together in one IU with the following constituent, such as the prepositional phrase in \REF{ex:sagaat madabumai dɛi buta} and pronominal argument NP in  \REF{ex:anu ampi koloanan saasaluai kita1}. Such cases are common. However, in many instances, the verb and possible following adverbs are grouped as separate IUs, as seen in example \REF{ex:baguna poni kalibombang}, where the verb \textit{bagu{\ü}na}  meaning `he was beating' and the following adverb \textit{pɔni} `again'  constitute the first IU of the CIU and are grouped separately from the following argument NP \textit{kalibɔmbaŋ} meaning `butterfly'.

\begin{figure}	
	\includegraphics[
	height=0.3\textheight
	]{figures/baguna_poni_kalibombang_story-monkey-butterfly_RSM_extRSM_plot.png}
	\caption{Periogram with pitch track (in st) for example \REF{ex:baguna poni kalibombang}, a CIU consisting of two embedded IUs, speaker RSM}
	\label{pitch:baguna poni kalibombang}
\end{figure}





\ea
\label{ex:baguna poni kalibombang}
\textit{baɡu{\ü}na pɔni | kalibɔmbaŋ	} \\
\gll baɡu{\ü}-0=na pɔni kalibɔmbaŋ	 \\
beat-\textsc{uv=}3\textsc{s}.\textsc{gen} again butterfly\\ 
\glt ‘he was beating  the butterfly again’ \hfill(story-monkey-butterfly\_RSM.053)
	\osflink{65a54ff5c585fd0c3f9ce65f}{baguna_poni_kalibombang_story-monkey-butterfly_RSM_extRSM.wav}
\z




In fact, the same construction can be found with both realizations: with the verb and the following argument parsed together in one IU or separately in one IU each. Below are two instances of a nearly identical CIU which involves a verb and its oblique argument. In the first example \REF{ex:ingga noliitaan takin tau dako}  and its corresponding visualization in  \figref{pitch:ingga noliitaan takin tau dako}, the verb and the oblique argument form separate, embedded IUs. This is clearly visible by the pitch rise on the last syllable of the verb \textit{nɔlitaan} meaning ‘to meet’.


\begin{figure}
	\includegraphics[
	height=0.3\textheight
	]{figures/ingga_noliitaan_takin_tau_dakolifestory_RDA_1_extRDA_plot.png}
	\caption{Periogram with pitch track (in st) for example \REF{ex:ingga noliitaan takin tau dako}, a CIU consisting of two embedded IUs, speaker RDA}
	\label{pitch:ingga noliitaan takin tau dako}
\end{figure}



\ea
\label{ex:ingga noliitaan takin tau dako}
\textit{iŋɡa nɔliitaan | takin tau dakɔ} \\
\gll iŋɡa nɔli-ita-an takin tau dakɔ	 \\
\textsc{neg} \textsc{rcp.rls}-see-\textsc{rcp.rls} with person big\\
\glt ‘(I) didn't meet  (my) parents’ \hfill(lifestory\_RDA\_1.124)
	\osflink{65a55018a246ff0abbdd3dc5}{ingga_noliitaan_takin_tau_dakolifestory_RDA_1_extRDA.wav}
\z





Example \REF{ex:danna noliitaan takin tau dako} features an almost identical construction. However, in this case, the verb and its oblique argument constitute a single (embedded) IU together. There is no pitch rise observed on the last syllable of the verb \textit{nɔlitaan} meaning ‘to meet’, which would typically indicate an IU boundary. The corresponding periogram with pitch track (in st) is provided in  \figref{pitch:ingga noliitaan takin tau dako}.

\begin{figure}
	\includegraphics[
	height=0.3\textheight
	%,
	%width=1\textwidth
	]{figures/danna_noliitan_takin_tau_dakolifestory_RDA_1_extRDA_plot.png}
	\caption{Periogram with pitch track (in st) for example \REF{ex:danna noliitaan takin tau dako}, a CIU consisting of two embedded IUs, speaker RDA}
	\label{pitch:danna noliitaan takin tau dako}
\end{figure}



\ea
\label{ex:danna noliitaan takin tau dako}
\textit{danna | nɔliitaan takin tau dakɔ} \\
\gll danna nɔli-ita-an takin tau dakɔ	 \\
then \textsc{rcp.rls}-see-\textsc{rcp.rls} with person big\\
\glt ‘Then, (I)  met  (my) parents’ \hfill(lifestory\_RDA\_1.115)
	\osflink{65a55001a246ff0ac2dd3d9a}{danna_noliitan_takin_tau_dakolifestory_RDA_1_extRDA.wav}
\z



Other instances involve an NP with a modifying relative clause that can either occur together in one IU or split into two IUs. The latter is more common when in postverbal position. For instance, in example \REF{ex:lau anu lau suludan tau moane ana dɛi dulak ɔɡɔbbunna moitaku} and its visualization in   \figref{pitch:lau anu lau suludan tau moane ana dɛi dulak ɔɡɔbbunna moitaku}, the first word \textit{lau} `currently' appears in a separate, embedded IU at the beginning of the CIU. It is followed by an IU containing the relative clause \textit{anu lau suludan tau mɔanɛ ana } `which is being pushed by the man', another IU with the prepositional phrase \textit{dɛi dulak ɔɡɔbbunna} `in front of the well', and the final IU containing the verb \textit{mɔitaku} `I see'. The IU containing the relative clause spans six words or twelve syllables, and the IU with the prepositional phrase has four words or seven syllables. Such lengthy IUs are not uncommon in Totoli, as demonstrated by the examples in  \sectref{IU-model} (e.g. example \REF{ex:kaasikan nogiigitai mangana dolago itu sapedana nollumpak} in  \figref{pitch:kaasikan nogiigitai mangana dolago itu sapedana nollumpak}, and example \REF{ex:daan tooka nemenek isia lau memenek naasyik lau monipu}  in  \figref{pitch:daan tooka nemenek isia lau memenek naasyik lau monipu}). This indicates that the (embedded) IU in Totoli differs significantly from the prosodic word and Accentual Phrase \is{Accentual Phrase} in Korean, \is{Korean} as discussed in   \sectref{sec:discussion}.

\begin{figure}
	\includegraphics[
	height=0.3\textheight
	]{figures/lau_anu_lau_suludan__tau_moane_ana_dei_dulak_ogobbunna_moitakuQUIS-focus_SP_extSP_plot.png}
	\caption{Periogram with pitch track (in st) for example \REF{ex:lau anu lau suludan tau moane ana dɛi dulak ɔɡɔbbunna moitaku}, a CIU consisting of four embedded IUs, speaker SP}
	\label{pitch:lau anu lau suludan tau moane ana dɛi dulak ɔɡɔbbunna moitaku}
\end{figure}




\ea
\label{ex:lau anu lau suludan tau moane ana dɛi dulak ɔɡɔbbunna moitaku}
\textit{lau | anu lau suludan tau mɔanɛ ana | dɛi dulak ɔɡɔbbunna | mɔitaku} \\
\gll lau anu lau sulud-an tau ana dɛi dulak ɔɡɔbbun=na  mɔ-ita-0=ku\\
presently \textsc{rel} presently push-\textsc{appl} person \textsc{med} \textsc{loc} front well=3\textsc{s}.\textsc{gen} \textsc{st}-see\textsc{-uv}=1s.\textsc{gen}\\ 
\glt ‘what is currently pushed by the man in front of the well, as 
I see it.’ \begin{flushright}(QUIS-focus\_SP.026)
	\osflink{65a550281c92110a81abea83}{lau_anu_lau_suludan__tau_moane_ana_dei_dulak_ogobbunna_moitakuQUIS-focus_SP_extSP.wav}\end{flushright}
\z







Example \REF{ex:bali tau } and its visualization in   \figref{pitch:bali tau } illustrate a long and complex CIU, where the various grammatical units are very regularly chunked into (embedded) IUs. The CIU commences with an IU containing the connective adverb \textit{bali} `then/so', which typically constitutes its own IU. The following IU contains the subject NP \textit{tau} ‘person’  along with its set of modifiers. This is succeeded by an IU containing the verb \textit{nanaumai} ‘to go down’. The subsequent IU contains a prepositional phrase that is repeated twice. The first instance involves the dummy/filler element \textit{anuna}, and the second instance involves the intended/repaired prepositional phrase \textit{ulai puun alpukaat} ‘from the avocado tree’. The CIU concludes with an IU containing a relative clause that further modifies the noun in the prepositional phrase.




\begin{figure}
	\includegraphics[
	height=0.3\textheight
	]{figures/bali_tau__na__nonipu_togu_alpukat_ia_nanaumai_ulai_anuna_ulai__p__e_puun_alpukat_ia_anu_tooka_tipuna_pearstory_36_SELP_extSELP_plot.png}
	\caption{Periogram with pitch track (in st) for example \REF{ex:bali tau }, a CIU consisting of six embedded IUs, speaker SELP}
	\label{pitch:bali tau }
\end{figure}




\ea
\label{ex:bali tau }
\textit{bali | tau <na> nɔnipu tɔɡu alpukaat ia | nanaumai | ulai anuna | ulai <p> ɛ puun alpukaat ia | anu tɔɔka itipuna} \\
\gll bali tau  nɔN-tipu tɔɡu alpukaat ia nɔ-nau=mɔ=ai uli=ai anu=na uli=ai  ɛ puun alpukaat ia anu tɔɔka ni-tipu-0=na  \\
so person  \textsc{av.rls}-pick possession avocado \textsc{prx} \textsc{av.rls}-go.down\textsc{=cpl=ven } from=\textsc{ven} \textsc{fill}=3s.\textsc{gen} from=\textsc{ven}  \textsc{intj} tree avocado \textsc{prx} \textsc{rel} finished \textsc{rls-}pick\textsc{-uv}=3\textsc{s}.\textsc{gen}\\ 
\glt ‘so, the person picking, the owner of the avocados, goes down
from the avocado tree that he was just picking.’ \hfill(pearstory\_36\_SELP.047)
	\osflink{65a54ffdc585fd0c489ce0a8}{bali_tau__na__nonipu_togu_alpukat_ia_nanaumai_ulai_anuna_ulai__p__e_puun_alpukat_ia_anu_tooka_tipuna_pearstory_36_SELP_extSELP.wav}
\z





In   \sectref{sec:discussion}, I discussed the case of Heads in a THL, \is{Tail-Head Linkage} which are immediately followed by additional syntactic material. In this case, the Head of the THL construction constitutes an IU in CIU-initial position. Adverbial and relative clauses that fulfill this role can occasionally be quite long, consisting of five or more words. Two examples are given below in \REF{ex:tooka_monipu_laalau_isia_dei_babo_ondan}--\REF{ex:kaasikan nogiigitai mangana dolago itu sapedana nollumpak2}.
Example \REF{ex:tooka_monipu_laalau_isia_dei_babo_ondan} and its visualization in  \figref{pitch:{ex:tooka_monipu_laalau_isia_dei_babo_ondan}} is a CIU with an extensive initial adverbial clause, phrased in two IUs. The adverb \textit{laalau} `currently' is parsed in a single IU with the remainder of the adverbial clause parsed in a single long IU, containing 6 words / 13 syllables.

\begin{figure}	\includegraphics[
	height=0.3\textheight
	]{figures/pearstory_36_SELP_tooka_monipu_laalau_isia_dei_babo_ondan_plot.png}
	\caption{Periogram with pitch track (in st) for example \REF{ex:tooka_monipu_laalau_isia_dei_babo_ondan}, a CIU consisting of three embedded IUs, speaker SELP}
	\label{pitch:{ex:tooka_monipu_laalau_isia_dei_babo_ondan}}
\end{figure}


\largerpage
\ea
\label{ex:tooka_monipu_laalau_isia_dei_babo_ondan}
\textit{lalau | isia dɛi babo ɔndan lau mɔnipu | nɔtumalib } \\
\gll \textsc{rdp}-lau isia dɛi babo ondan lau mɔ-tipu nɔ-t<um>alib \\
\textsc{rdp}-while 3\textsc{s} \textsc{loc} above ladder while \textsc{av.rls-}pick 	\textsc{av.rls}-\textsc{<auto.mot>}pass.by\\
\glt ‘While he was on the ladder picking (pears),(he) passed by’
% \begin{flushright}(pearstory\_36\_SELP.09)
% 	\href{run:figures/pearstory_36_SELP_tooka_monipu_laalau_isia_dei_babo_ondan.wav}{\triangleright}\end{flushright}
	\osflink{65bdf37a3280d8066ea3ac4b}{pearstory_36_SELP_tooka_monipu_laalau_isia_dei_babo_ondan.wav}
\zlast\clearpage


Example \REF{ex:kaasikan nogiigitai mangana dolago itu sapedana nollumpak2} and its realization in \figref{pitch:kaasikan nogiigitai mangana dolago itu sapedana nollumpak2} is an instance of a CIU with two initial IUs containing a complex adverbial clause.  



\begin{figure}	\includegraphics[
	height=0.3\textheight
	]{figures/pearstory_11_SP_extSP_kaasikan_nogiigitai_mangana_dolago_itu_sapedana_nollumpak_plot.png}
	\caption{Periogram with pitch track (in st) for example \REF{ex:kaasikan nogiigitai mangana dolago itu sapedana nollumpak2}, a CIU consisting of three embedded IUs, speaker SP}
	\label{pitch:kaasikan nogiigitai mangana dolago itu sapedana nollumpak2}
\end{figure}



\ea
\label{ex:kaasikan nogiigitai mangana dolago itu sapedana nollumpak2}
\textit{kaasikan | mɔɡiiɡitai maŋana dɔlaɡɔ itu | sapɛda | nɔllumpak} \\
\gll kɛasikan mɔɡ-\textsc{rdp}-ita-i maŋana dɔlaɡɔ itu sapɛda nɔ-\textsc{rdp}-lumpak \\
excitement \textsc{av.nrls}-\textsc{rdp}-watch-\textsc{appl} child girl \textsc{dist} bicycle \textsc{st}-\textsc{rdp}-hit.against\\
\glt ‘because of his excitement in looking at the girl, his bicycle crashed (against the stone)’ \hfill(pearstory\_11\_SP.025)
	\osflink{65a55050f2240f0b4d32e5d6}{pearstory_11_SP_extSP_kaasikan_nogiigitai_mangana_dolago_itu_sapedana_nollumpak.wav}
\z




The examples demonstrate regularities in chunking. However, more insightful are instances that appear to contradict these regularities.

In many CIUs, the first word forms its own IU, which is also noted by \citet[361]{Himmelmann_Preliminary_2018}. In fact, 31\% of all CIUs in the corpus containing more than one word have their first word phrased as a separate embedded IU. This is often because words in the initial position of a CIU are connectives or one-word noun phrases that are regularly phrased as a single IU. See examples \REF{ex:sagaat madabumai dɛi buta},  \REF{ex:lau anu lau suludan tau moane ana dɛi dulak ɔɡɔbbunna moitaku}, \REF{ex:bali tau }.

As  explained above, adverbial clauses, relative clauses, and prepositional phrases are typically not further chunked, regardless of their length, as demonstrated by the long embedded IUs in examples \REF{ex:tooka_monipu_laalau_isia_dei_babo_ondan} and \REF{ex:kaasikan nogiigitai mangana dolago itu sapedana nollumpak2}. However, there are instances where the initial elements constitute a separate IU. Examples include the initial relative particle \textit{anu} of a relative clause, the initial preposition of a prepositional phrase, and the initial conjunction of an adverbial clause. Consider \REF{ex:tooka_monipu_laalau_isia_dei_babo_ondan} above, with its periogram in  \figref{pitch:{ex:tooka_monipu_laalau_isia_dei_babo_ondan}}. The initial word \textit{laalau}  `while' of the adverbial clause is phrased as its own IU. The following IU contains a pronoun as well as a prepositional phrase and a predicate. These are not further chunked, as is the case for adverbial clauses (see also example \REF{ex:lau anu lau suludan tau moane ana dɛi dulak ɔɡɔbbunna moitaku}).




In many adverbial clauses, such as examples \REF{ex:tooka_monipu_laalau_isia_dei_babo_ondan} and \REF{ex:kaasikan nogiigitai mangana dolago itu sapedana nollumpak2}, the first element is phrased separately. When an initial element of a relative clause, adverbial clause, or prepositional phrase is phrased separately, it often co-occurs with a hesitation pause. Thus, boundary placement appears to serve as a planning device. Two examples are provided below.

The CIU in example \REF{ex:niuntudnako dɛi (.) tau nanako (.) mangana nnako (.) bungo piir itu} contains a verb and a lengthy prepositional phrase. The initial verb is phrased as a separate IU. However, in the following prepositional phrase, the initial preposition \textit{dɛi} is phrased separately from the rest of the CIU and is followed by a short CIU-internal pause. The periograms with pitch tracks (in st) are shown in   \figref{pitch:niuntudnako dɛi (.) tau nanako (.) mangana nnako (.) bungo piir itu}.


\begin{figure}
	\includegraphics[
	height=0.3\textheight
	]{figures/niuntudnako_dei_____tau_nanako_____mangana_nnako_____bungo_piir_ituuntitled_plot.png}
	\caption{Periogram with pitch track (in st) for example \REF{ex:niuntudnako dɛi (.) tau nanako (.) mangana nnako (.) bungo piir itu}, a CIU consisting of three embedded IUs, speaker FAH}
	\label{pitch:niuntudnako dɛi (.) tau nanako (.) mangana nnako (.) bungo piir itu}
\end{figure}






\ea
\label{ex:niuntudnako dɛi (.) tau nanako (.) mangana nnako (.) bungo piir itu}
\textit{niuntudnakɔ | dɛi | tau nanakɔ maŋana nnakɔ  buŋɔ piir itu} \\
\gll ni-untud-0=na=kɔ dɛi  tau nɔN-takɔ  maŋana nɔN-takɔ  buŋɔ piir itu \\
\textsc{rls}-bring-\textsc{uv=3s.gen=and} \textsc{loc}  person \textsc{av.rls-}steal  child \textsc{av.rls-}steal  fruit pear \textsc{dist}\\
\glt ‘he brought (it) to the person who stole, the child who stole the pears’ \begin{flushright}(pearstory\_11\_SP.025)	
	\osflink{65a5503fc585fd0c489ce0dc}{niuntudnako_dei_____tau_nanako_____mangana_nnako_____bungo_piir_ituuntitled.wav}\end{flushright}
\z




Another example is provided in  \REF{ex:untuk panarimaan tau mongouma ia}, and its visualization is shown in  \figref{pitch:untuk panarimaan tau mongouma ia}. In this example, the initial preposition \textit{untuk}  `for' is also phrased separately, followed by a CIU-internal hesitation pause.

\begin{figure}
	\includegraphics[
	height=0.3\textheight
	]{figures/untuk_panarimaan_tau_mongouma_iaexplanation-wedding-tradition_ZBR_extZBR_plot.png}
	\caption{Periogram with pitch track (in st) for example \REF{ex:untuk panarimaan tau mongouma ia}, a CIU consisting of two embedded IUs, speaker SP}
	\label{pitch:untuk panarimaan tau mongouma ia}
\end{figure}

\ea
\label{ex:untuk panarimaan tau mongouma ia}
\textit{untuk | panarimaan tau mɔngɔuma ia} \\
\gll untuk pɔN-tarima-an tau mɔ-ngɔ-uma ia \\
for \textsc{nmlz}-accept\textsc{-nmlz} person \textsc{st-coll-}arrived \textsc{prx}\\
\glt ‘for the reception of the visitors’ \hfill(pearstory\_11\_SP.025)
	\osflink{65a55081f2240f0b4d32e5f0}{untuk_panarimaan_tau_mongouma_iaexplanation-wedding-tradition_ZBR_extZBR.wav}
\z







Instances of an embedded IU containing elements of two separate clauses are very rare. One such instance is presented in example \REF{ex:aan nadabu nolimulas alpukat noongotmoko gaake buludna}, which is visualized in  \figref{pitch:untuk panarimaan tau mongouma ia}. This example comprises an adverbial clause and two coordinated main clauses and involves an embedded IU containing grammatical units of both main clauses. The CIU starts with an IU containing the adverbial clause \textit{daan nadabu}  `after he fell'. The following verb \textit{nɔlimulas} `scatters' is uttered as its own embedded IU. The subsequent argument \textit{alupkaat} `the avocado' is not parsed into its own IU, but the pitch drops continuously despite the clause boundary. An IU-final boundary tone is placed on the adverb \textit{ɡaakɛ} `also'. As a result, this IU contains the argument NP of the first clause and the verb of the second clause. The argument NP of the second clause \textit{buludna} `his shinbone' forms the final IU of the CIU.



\begin{figure}
	\includegraphics[
	height=0.3\textheight
	]{figures/daan_nadabu_nolimulas_alpukat_noongotmoko_gaake_buludna_pearstory_36_SELP_extSELP_plot.png}
	\caption{Periogram with pitch track (in st) for example \REF{ex:aan nadabu nolimulas alpukat noongotmoko gaake buludna}, a CIU consisting of four embedded IUs, speaker SELP}
	\label{pitch:aan nadabu nolimulas alpukat noongotmoko gaake buludna}
\end{figure}

\newpage
\ea
\label{ex:aan nadabu nolimulas alpukat noongotmoko gaake buludna}
\textit{daan nadabu | nɔlimulas | alpukaat nɔɔngɔtmɔkɔ ɡaakɛ | buludna} \\
\gll daan nɔ-dabu nɔ-l<um>ɛlas alpukaat nɔ-ɔngɔt=mɔ=kɔ ɡaakɛ bulud=na	 \\
later \textsc{st.rls-}fall \textsc{av.rls-}\textsc{<auto.mot>}scatter avocado \textsc{st.rls-}hurt\textsc{=cpl=and} too shin=3s.\textsc{gen}\\
\glt ‘after (he) fell, the avocados scatter and his shin hurts too’ \begin{flushright}(pearstory\_36\_SELP.025)
	\osflink{65a55000f2240f0b5032e6ce}{daan_nadabu_nolimulas_alpukat_noongotmoko_gaake_buludna_pearstory_36_SELP_extSELP.wav}
\end{flushright}
\z



In this section, I have illustrated some key aspects regarding the chunking of CIUs into IUs and their regularities. The syntactic content of IUs typically constitutes a complete grammatical unit, but there are rare instances where two independent grammatical units occur within the same IU. Additionally, certain units such as adverbial clauses, prepositional phrases, and relative clauses may have their initial element phrased as a separate IU. The realization of example \REF{ex:aan nadabu nolimulas alpukat noongotmoko gaake buludna} provides an interesting case in point. In most cases, speakers consistently mark the right-edge boundary of a grammatical unit with a prosodic boundary, but boundary placement is optional. In this example, the speaker did not place a boundary at the end of the NP \textit{alpukaat} `avocado', which is the end of the first main clause. Consequently, the IU consists of two grammatical units that belong to two different clauses.


\subsection[Comparing syntax and prosody]{Comparing the syntax and prosody of embedded IUs  of CIUs with singleton IUs}\label{sec:comparing-ius-with-ips}


In this section, I compare the syntactic content of embedded IUs of CIUs with singleton IUs, investigating the differences or similarities between the two with regard to the grammatical structures they contain. The analysis of phrase-final tonal patterns in  \sectref{sec:tonal-events-at-the-boundaries-of-ius} and \sectref{sec:tonal-events-at-the-boundaries-of-ips}  revealed that prosodic patterns at the end of embedded IUs are essentially the same as those that occur in CIU-final position, providing evidence that these prosodic units are essentially of the same type.

To further explore this assumption, I compare grammatical structures typically found in embedded IUs of complex CIUs with those found in singleton IUs. I illustrate this with two examples below. 

Compare example \REF{ex:bali pogitata anu babi} and its visualization in  \figref{pitch:bali pogitata anu babi} with example \REF{ex:bali ...pogitata anu batu} and its visualization in  \figref{ex:pitch ...pogitata anu batu}. Both contain a similar structure: the connective \textit{bali} `so', followed by the verb \textit{pɔɡitata} `look for' and a subsequent headless relative clause in undergoer function.  The difference is that in example \REF{ex:bali pogitata anu babi}, the connective is parsed into one CIU together with the main clause and constitutes the initial embedded IU of the CIU.  In example \REF{ex:bali ...pogitata anu batu}, however, the connective appears in a separate IU, clearly demarcated by further boundary phenomena, such as pitch reset, and final syllable lengthening. Note, however, that tonal targets and the tonal contours of \textit{bali} `so' are identical in both instances.







\begin{figure}
	\includegraphics[
	height=0.3\textheight
	]{figures/bali_pogitata_anu_babi_spacegames_sequence4_KSR-SP_extSP_plot.png}
	\caption{Periogram with pitch track (in st) for example \REF{ex:bali pogitata anu babi}, a CIU consisting of three embedded IUs, speaker SP}
	\label{pitch:bali pogitata anu babi}
\end{figure}

\newpage
\ea
\label{ex:bali pogitata anu babi}
\textit{bali | pɔɡitata | anu babi} \\
\gll bali pɔɡ-ita-0=ta anu babi	 \\
so \textsc{sf-}look.for\textsc{-uv=2s.gen} \textsc{rel} pig\\
\glt ‘so,  look for the pig.’ \hfill(spacegames\_sequence4\_KSR-SP.012)
	\osflink{65a54ff81cba3e0c1801854f}{bali_pogitata_anu_babi_spacegames_sequence4_KSR-SP_extSP.wav}
\z


\begin{figure}
	\includegraphics[
	height=0.3\textheight
	]{figures/bali_pogitata_anu_batu_spacegames_sequence1_KSR-SP_extSP_plot.png}
	\caption{Periogram with pitch track (in st) for example \REF{ex:bali ...pogitata anu batu}, two CIUs, speaker SP}
	\label{ex:pitch ...pogitata anu batu}
\end{figure}



\ea
\label{ex:bali ...pogitata anu batu}
\ea
	\label{ex:bali1}
	\textit{bali} \\
	\gll bali 	 \\
	so \\
	\glt ‘so’

	\ex
	\label{ex:pogitata anu batu1}
	\textit{pɔɡitata | anu batu} \\
	\gll pɔɡ-ita-0=ta anu batu	 \\
	\textsc{sf-}look.for\textsc{-uv=2s.gen} \textsc{rel} stone\\
	\glt ‘look for the stone.’ \hfill(spacegames\_sequence4\_KSR-SP.012)
		\osflink{65a54ffbf2240f0b5132e91b}{bali_pogitata_anu_batu_spacegames_sequence1_KSR-SP_extSP.wav}
\z
\z	




In both examples above, the tonal targets of the connective \textit{bali} are the same. In example \REF{ex:bali pogitata anu babi}, pitch is interpolated between the IU-final H\% boundary tone located on the last syllable of the connective \textit{bali} `so' and the tonal targets of the second IU \textit{pɔɡitata} `look for'. In example \REF{ex:bali ...pogitata anu batu}, the connective forms its own singleton IU and pitch is reset at the beginning of the then CIU-initial word \textit{pɔɡitata} ‘look for’.

A second example is given in \REF{ex:oto ana lau suludanna ana} and \REF{ex:oto anu laalau}. Again, in  \REF{ex:oto ana lau suludanna ana}, the left-dislocated topic NP \textit{ɔtɔ} ‘car’ is parsed into one CIU with the following clause. In  \REF{ex:oto anu laalau}, the same NP \textit{ɔtɔ} ‘car’ occurs as a separate singleton IU, clearly visible by the reset in pitch and final syllable lengthening. Yet, tonal targets and the shape of the pitch contour of the NP \textit{ɔtɔ} ‘car’ are the same in both instances.



\begin{figure}
	\includegraphics[
	height=0.3\textheight
	]{figures/oto_ana_lau_suludanna_anaQUIS-focus_SP_extSP_plot.png}
	\caption{Periogram with pitch track (in st) for example \REF{ex:oto ana lau suludanna ana}, a CIU consisting of two embedded IUs, speaker SP}
	\label{pitch:oto ana lau suludanna ana}
\end{figure}


\ea
\label{ex:oto ana lau suludanna ana}
\textit{ɔtɔ | ana lau suludanna ana} \\
\gll ɔtɔ ana lau sulud-an=na ana	 \\
car \textsc{med} presently push\textsc{-appl}=3\textsc{s}.\textsc{gen} \textsc{med}\\
\glt ‘it is a car that is being pushed by him’ \hfill (QUIS-focus\_SP.012)
	\osflink{65a550481cba3e0c11018805}{oto_ana_lau_suludanna_anaQUIS-focus_SP_extSP.wav}
\z





\begin{figure}
	\includegraphics[
	height=0.3\textheight
	]{figures/oto_anu_laalau_suludan_tau_moane_dei_dulak_ogobbun_anaQUIS-focus_SP_extSP_plot.png}
	\caption{Periogram with pitch track (in st) for example \REF{ex:oto anu laalau}, speaker SP}
	\label{pitch:oto anu laalau}
\end{figure}




\ea
\label{ex:oto anu laalau}
\ea{
	\label{ex:oto}
	\textit{ɔtɔ} \\
	\gll ɔtɔ\\
	car\\
	\glt `(it is a) car'
}

{
	\ex
	\label{ex:anu laalau suludan tau moane dɛi dulak ɔɡɔbbun ana}
	\textit{anu laalau suludan tau mɔanɛ dɛi dulak ɔɡɔbbun ana} \\
	\gll anu \textsc{rdp}-lau sulud-an tau mɔanɛ dɛi dulak ɔɡɔbbun ana\\
	\textsc{rel} \textsc{rdp}-presently push-\textsc{appl} person man \textsc{loc} front well \textsc{med}\\
	\glt `that is currently pushed by the man in front of the well'
	\begin{flushright}(QUIS-focus\_SP.041-42)
		\osflink{65a55048f2240f0b5132e9ce}{oto_anu_laalau_suludan_tau_moane_dei_dulak_ogobbun_anaQUIS-focus_SP_extSP.wav}\end{flushright}
}
\z
\z



These two example pairs illustrate that the same grammatical structures may occur as either singleton IUs or an embedded IU in a complex CIU. In the examples above, the tonal specifications remain the same. 

Based on the corpus, I conducted a quantitative comparison of grammatical structures found in embedded IUs of CIUs with those found in singleton IUs.  \figref{Comparison of syn} illustrates the comparison. Embedded IUs of complex CIUs (exemplified here as  `y\textsubscript{1}', `y\textsubscript{2}', `y\textsubscript{3}') are given on the right-hand side and singleton IUs which consist of a single IU (exemplified here as `x') are displayed on the left-hand side.




\begin{figure}
	\caption{Comparison of syntactic content of embedded IUs and singleton IUs}
	\begin{tikzpicture}
		\node (IUa) at (-2.8,5) {CIU};
		\node (CIUa) at (7.1,5) {CIU};
		
		\node (IU1a) at (-2.8,3.5) {\fbox{\textbf{IU}}};
		\node (CIU1a) at (4,3.5) {\fbox{\textbf{IU}}};
		\node (CIU2a) at (7.1,3.5) {\fbox{\textbf{IU}}};
		\node (CIU3a) at (10.2,3.5) {\fbox{\textbf{IU}}};
		
		\node [color=infocus, fill = gray!20] (IU1a) at (-2.8,2.5) {{\textbf{x}}};
		\node [color=outfocus, fill = gray!60]  (CIU1a) at (4,2.5) {{\textbf{y\textsubscript{1}}}};
		\node [color=outfocus, fill = gray!60] (CIU2a) at (7.1,2.5) {{\textbf{y}\textsubscript{2}}};
		\node [color=outfocus, fill = gray!60] (CIU3a) at (10.2,2.5) {{\textbf{y}\textsubscript{3}}};
		
		
		\draw[] (-2.8,4.7) -- (-2.8,3.9);
		\draw[] (7.1,4.7) -- (4,3.9);
		\draw[] (7.1,4.7) -- (7.1,3.9) ;
		\draw[] (7.1,4.7) -- (10.2, 3.9);
		\draw [<->] (-1,3.5) -- (2,3.5);
		
	\end{tikzpicture}
	
	\label{Comparison of syn}
\end{figure}








 \figref{Freq_GUs_in_ips_IUs} compares the distribution of grammatical units found in singleton IUs with the distribution within embedded IUs of CIUs. The seven syntactic categories account for 82\% of the 3191 embedded IUs and 78\% of all 1005 unchunked IUs in the corpus.

\begin{figure}
	\includegraphics[
	%height=0.3\textheight,
	width=\textwidth]{figures/GUs_in_ips_IUs.png}
	\caption{Frequency distribution of  the 7 major grammatical structures of non-final IUs, within unchunked IUs: numbers are rounded to one decimal place.}
	\label{Freq_GUs_in_ips_IUs}
\end{figure}


The frequency distribution in  \figref{Freq_GUs_in_ips_IUs}  shows that all seven major grammatical units found in embedded IUs (`y\textsubscript{1}', `y\textsubscript{2}', `y\textsubscript{3}') also occur as unchunked singleton IUs (`x'), although with varying distribution. Specifically, 82\% of all embedded IUs  correspond to one of the seven categories of grammatical units, and these seven grammatical units also describe 78\% of all singleton IUs. The difference in distribution mainly pertains to verbs and connectives. Verbs occur  less frequently as singleton IUs than as embedded IUs; 25.6\% of embedded IUs are verbs, but only 22.5\% of singleton IUs are verbs. Connectives, on the other hand, occur considerably less frequently as singleton IUs.

Note that the absolute numbers of IUs and embedded IUs consisting of an NP are lower here than the number of nominal IUs in  \tabref{freq_N_attachable} above. This is because the counts here are more conservative than above. Nominal IUs above include cases where the nominal element co-occurs with a connective or a relative clause, while the counts here consider only those singleton or embedded IUs that consist of a single NP only.

In summary, the results show that the distribution of grammatical units typically found in non-final IUs and singleton IUs is very similar. Elements that regularly constitute a separate IU in a complex CIU are also found as unchunked singleton IUs.






\subsection{Discussion}\label{sec:summary}



In  \sectref{sec:grammatical-units-and-the-intonation-unit} above, I explored the grammatical units typically found in a CIU. I showed that 52.7\% of CIUs in the corpus are clausal, 15.9\% are nominal and 13.7\% are interactional. Crucially, I found that proportions vary   between  conversational and monological data. In  \sectref{sec:grammatical-units-and-intermediary-phrases}, I briefly illustrated some of the major aspects with regard to CIU-internal chunking and the major types of grammatical units found in embedded IUs of CIUs. Finally, in  \sectref{sec:comparing-ius-with-ips}, I compared grammatical structures found in singleton IUs with those found in embedded IUs.  I showed that the same grammatical units found in embedded IUs of complex CIUs also frequently occur as unchunked singleton IUs.  Hence, neither the syntactic content, nor the tonal markings of singleton IUs and embedded IUs of CIUs differ. If they were of a different category, I would expect the units to differ also with regard to their syntactic content. This is clearly the case in the analysis of e.g. \ili{French} and \ili{Korean} intonation \citep{Jun_2000, phonology2005and} for which both the level of the IU and the lower-level   Accentual Phrase \is{Accentual Phrase} are assumed, of which the latter usually only contains one word. In light of these results, I conclude that there is clear evidence that we have to assume recursive embedding of IUs into complex CIUs in order to describe the data of Totoli as proposed in the model presented in  \sectref{IU-model}.


To conclude this discussion, I will briefly address the distribution of IU-final boundary-tone complexes and review some of the factors that may influence the choice of either complex. Consider  \figref{pitch:dis btc}, which provides examples of adverbial clauses, noun phrases, and verbs, and illustrates their occurrence with one of the boundary-tone complexes. These three constituents are significant syntactic components, and I will use them to compare the factors affecting the choice of a boundary tone. The right-hand columns in each pair indicate the distribution of complex CIUs in embedded IUs, while the left-hand columns indicate their distribution in simple, singleton IUs.

\begin{figure}
	\includegraphics[
	%height=0.3\textheight,
	width=\textwidth]{figures/btcs_of_GUs.png}
	\caption{Distribution of the three boundary-tone complexes in singleton IUs and embedded IUs of CIUs, exemplified with the three grammatical categories AdvCl, NP, and VP}
	\label{pitch:dis btc}
\end{figure}


\newpage
The figure illustrates that when adverbial clauses occur as singleton IUs, they usually take the L-H\% boundary-tone complex. However, when embedded in a compound IU, many adverbial clauses take the LH-H\% pattern. This preference also applies to adverbial clauses that are embedded in CIUs, although many also occur with the LH-H\% pattern in this position. Noun phrases and verbs occurring as singleton IUs show a strong preference for the LH-L\% pattern, although other boundary-tone complexes are possible. When occurring as embedded IUs of a complex CIU, they tend towards the LH-H\% pattern. The tendency for a rise pattern in embedded IUs of complex CIUs is not surprising, as these IUs are part of a larger unit, and the final rise pattern indicates non-finality within the Compound IU.

The question then arises as to why there are two different rise patterns, i.e. L-H\% and LH-H\%. One possible explanation is that the difference between the three patterns LH-L\%, L-H\%, and LH-H\% correlates with degrees of integration. The LH-H\% pattern represents ``full integration" and is regularly used with verbs and noun phrases in non-final positions of complex IUs. The L-H\% pattern and the LH-L\% pattern would then be reserved for clause-external constituents. Adverbial clauses may occur with the LH-H\% full-integration pattern; however, about half of them occur with the L-H\% pattern.



Above, I showed an example of a left-dislocated topic, which involved the LH-L\% pattern (cf. example \REF{ex:oto ana lau suludanna ana}). The dislocated element usually involves the LH-L\% pattern. Other instances involve right-dislocation, afterthoughts or appositive NPs. Example \REF{ex:siritana ia geimo daan lau mokodoong maaling ia barang ia2} shows such an instance. The NP \textit{siritana ia} ‘this story’ is taken up again after the verb \textit{maaliŋ} ‘to get lost’, first by the pronoun \textit{ia} `\textsc{prx}' and then by the appositive NP \textit{baraŋ ia} ‘this thing’. The verb bears the LH-L\% boundary-tone complex and is immediately followed by the pronoun \textit{ia} which takes up the NP \textit{siritana}.

In the preceding section, I presented an example of a focused constituent, which involved the LH-L\% pattern (cf. example \REF{ex:oto ana lau suludanna ana}). The focused element typically involves the LH-L\% pattern, while other instances involve right-dislocation, afterthoughts, or appositive NPs. Example \REF{ex:siritana ia geimo daan lau mokodoong maaling ia barang ia2} illustrates such an instance. The NP \textit{siritana ia} `this story' is taken up again after the verb \textit{maaliŋ} `to get lost', first by the pronoun \textit{ia} \textsc{prx}, and then by the appositive NP \textit{baraŋ ia} `this thing'. The verb bears the LH-L\% boundary-tone complex and is immediately followed by the pronoun \textit{ia}, which takes up the NP \textit{siritana} `this story'.


\begin{figure}
	\includegraphics[
	height=0.3\textheight
	%,
	% width=1\textwidth
	]{figures/explanation-lelegesan_SYNO_extSYNO_siritana_ia_geimo_daan_lau_mokodoong_maaling_ia_barang_ia_plot.png}
	\caption{Periogram with pitch track (in st) for example \REF{ex:siritana ia geimo daan lau mokodoong maaling ia barang ia2}, speaker SYNO}
	\label{pitch:siritana ia geimo daan lau mokodoong maaling ia barang ia2}
\end{figure}


\newpage
\ea
\label{ex:siritana ia geimo daan lau mokodoong maaling ia barang ia2}
\textit{siritana | ia ɡɛimɔ daan | lau mɔkɔdɔɔng maaling | ia baraŋ ia} \\
\gll sirita=na ia ɡɛimɔ daan lau mɔkɔ-dɔɔŋ mɔ-aliŋ ia baraŋ ia \\
story=3\textsc{s}.\textsc{gen} \textsc{prx} not \textsc{exist} presently \textsc{st.av}-want \textsc{st}-disappear \textsc{prx} goods \textsc{prx}\\
\glt ‘This story will never again get lost; this thing.’ \begin{flushright}(explanation-lelegesan\_SYNO.007)
	\osflink{65a5500af2240f0b5132e95a}{explanation-lelegesan_SYNO_extSYNO_siritana_ia_geimo_daan_lau_mokodoong_maaling_ia_barang_ia.wav}\end{flushright}
\z


The question of what conditions the choice of either boundary-tone complex  will remain a topic for further  research.

\is{prosody-syntax interface|)}
