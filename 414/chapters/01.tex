\chapter{Introduction}
\label{Guide}



The present book is an investigation into aspects of Totoli's prosody, intonation and the prosody-syntax interface. Totoli is an endangered Austronesian language of the \ili{Malayo-Polynesian} group and this book is the first study of the intonation of Totoli and among the few investigations into the prosody and intonation of \ili{Austronesian} languages in general. The investigation seeks to  uphold maximal \isi{ecological validity} \citep{Cicourel2007}. To this end, the analysis is based on an extensive \isi{corpus} of natural (semi-)spontaneous speech which is accessible through the \textit{Language Archive Cologne} \citep{TotoliLAC}. The study takes the prime structuring unit of speech -- the Intonation Unit (IU) -- as its principal unit of investigation and presents a thorough description of the IU, develops an intonational model thereof and investigates the syntactic units it contains. The proposed intonational model  is supported by experimental evidence of both production and perception. 




The results of the various approaches  taken in this book show that Totoli falls under the category of \isi{Phrase Languages} \citep{Fery_2016}. From what is known so far, Totoli shows no evidence of tonal specifications at the level of the word; the language does not make use of word stress nor of \isi{lexical tone}. Prosodic prominence \is{prosodic prominence} does not play a role in the marking of information-structural \is{information structure} categories, and tonal specifications are assigned exclusively at the level of the intonational phrase and are associated with their right-edge boundary. Based on an investigation of tonal specifications and syntactic content of prosodic units of Totoli, I show that the data is best analyzed by assuming recursive embedding \is{recursivity} of IUs into Compound Intonation Units \is{Compound Intonation Unit} (CIU). 

\is{recursivity}

When working with un(der)researched languages,  one faces the task of finding appropriate tools for tapping into the prosodic system of the language. \citet{himmelmann2006challenges} and \citet{himmelmann2008prosodic} argue that  the study of the prosody of a language should best be supported with evidence obtained through different approaches. At best, it should contain an investigation of a substantial corpus of \mbox{(semi-)}spontaneous speech, the analysis of which  is computer-aided and complemented with experimental evidence from production and perception. 

The study presented here follows this approach. It is a combination of quantitative and qualitative analyses and is based on an extensive dataset collected by the author in the course of a language documentation project \is{language documentation}  \citep{TotoliLAC}.  With a strongly data-driven approach, the study integrates a combination of experimental evidence from both production \is{speech production} and perception \is{speech perception} with corpus-based evidence through descriptive and inferential statistics. 
The data used for this research was collected within the Collaborative Research Center 1252 “Prominence in Language”, funded by the DFG (German Research Foundation), and was supported by the Ministry of Research and Technology of the Republic of Indonesia through the provision of a research permit.\footnote{Research permit number: 197/SIO/FRP/E5/Dit.KI/VII/2018}  During several field trips to Totoli from 2017-2019, 196 hours of video material of various genres were collected, including a 56-hour Child Language Corpus, 85 hours of elicitation recordings, and 31 hours of \mbox{(semi-)}spontaneous speech. At least 20 hours of these recordings are transcribed.  \is{language documentation} The subset used in this study consists of 2h 19min  of recordings of \mbox{(semi-)}spontaneous\slash naturalistic speech which will be further described and discussed in  \sectref{Corpus}.  It is essentially an extension of the first language documentation corpus \citep{Totoli_Dobes} and follows its glossing conventions and grammatical analysis. 

I make a number of analytical proposals which are relevant to prosodic theory and typology in general. This research represents a significant advancement in our understanding of the nature of prosodic systems found in (Western) \ili{Austronesian} languages and intonational systems in general. Additionally, the study adheres to the principle that research should be reproducible.  Thus, all data is explicitly referenced in the text and made available at an online repository  \citep{TotoliLAC}. Furthermore, examples from the corpus in this book are represented by periograms, which utilize automatically smoothed and interpolated pitch contours that are enriched with periodic energy. Periograms are thus  phonetic representations that modulate pitch trajectories with periodic energy, by integrating ``relevant acoustic cues into a perceptually motivated representation of the pitch contour of an utterance" \citep[807]{albert2018using}. I followed  \citeauthor{Albert_cangemi_Ellison_Grice_2022}'s  (\citeyear{Albert_cangemi_Ellison_Grice_2022}) workflow but modified the color code.  Throughout this book, pitch curves are displayed as yellow lines overlaying  blue lines that represent information about periodic energy, as indicated by modifications in the transparency and width of the line. Syllable boundaries in the periograms are represented by thin, gray lines, while thicker lines are used to indicate word boundaries. \is{periogram}




Audio for all examples is provided alongside this book and is indicated by the ``\triangleright"-sign.   The first line (in italics) gives the phonemic transcription, disregarding allophonic realizations. This includes the particular case of word-final /l/ and its allophonic realization as a length-feature in word-final position in a process of word-final compensatory lengthening (see \sectref{subsubsection:Lateral /l/}). The examples include a second line with the segmentable morphemes separated by hyphens. The third line contains translations and abbreviated grammatical category labels in small capitals, and the fourth line provides a free translation to English. Information on the files of recordings in the Totoli archive \citep{TotoliLAC} is given in a final line. The  examples from Totoli in this book follow the Leipzig Glossing Rules \citep{haspelmath2014leipzig} and all glosses, abbreviations and other symbols used are explained above. 




The primary objective of the introductory section of this book is to provide  an overview and essential information necessary to comprehend the main discussions of this book, which are presented in Chapters~\ref{sec:Experiments} and~\ref{sec:The IU in Totoli}.  These two chapters employ two distinct approaches to studying the prosody and intonation of Totoli, with \chapref{sec:Experiments} concentrating on experimental methods. The results of this chapter are subsequently integrated into the analysis presented in \chapref{sec:The IU in Totoli}, which is based on corpus-based evidence and employs Intonation Units (IUs) as the primary unit of analysis. \chapref{sec:The IU in Totoli} is divided into three sections:  \sectref{IU-Props} describes fundamental properties of IUs in the corpus,   \sectref{IU-model} develops an intonational model of IUs based on boundary tone events and the findings of the experiments presented in \chapref{sec:Experiments}, and  \sectref{sec:the-syntax-of-intonation-units} examines the syntactic content of IUs and complex Compound IUs. Finally, \chapref{Summary and Discussion} summarizes the results, explores their implications, and suggests future research avenues.



\section[The Totoli Language]{Information on the Totoli language}\label{sec:data}
\is{sociolinguistic information on Totoli|(}

Totoli is an Austronesian language spoken in the Tolitoli regency (Kabupaten Tolitoli) of the Central Sulawesi province (Sulawesi Tengah) on the Indonesian island of Sulawesi. The linguistic area is divided into a southern region, primarily comprising the city of Tolitoli and surrounding villages, and a northern area consisting of the villages of Diule, Pinjan, Binontoan, Gio, and Lakuan (Tolitoli).  \figref{Map of Totoli} shows the area where Totoli is spoken. The two linguistic areas are encircled. 


\begin{figure}
	\includegraphics[width=0.8\textwidth]{figures/Map_of_TTL_rework.png}
	\caption{Area where Totoli is spoken, adapted from \citet[29]{himmelmann_source_book}}
	\label{Map of Totoli}
\end{figure}


\citet[18]{himmelmann_source_book} calculates the ethnic population of the Totoli people to be approximately 25,000 people, but estimates that only about 30\% of them – ca. 7,000 – are fluent speakers. \citet{Totoli_Dobes} estimate a maximum of 5,000 fluent speakers. This number may have further declined over the last decade. In the city of Tolitoli and other villages in the southern linguistic region, it is safe to say that Totoli is never heard on the streets, and the everyday language of ethnic Totolis is Indonesian. Even in Totoli households, families speak almost exclusively in Indonesian. In contrast to the southern area, Totoli is more resilient in the five northern villages  and is, to varying degrees, still a language of everyday communication. In these villages, the great majority of ethnic Totolis are fluent speakers of Totoli, although in many areas of everyday life, Indonesian is the preferred language. Children are raised entirely in Indonesian and they are not actively taught Totoli by their parents. The only exception is the village of Gio.
Interestingly, \citet[28]{himmelmann_source_book} lists the village of Gio as a Dondo-speaking \il{Dondo} settlement belonging to the village of Binontoan. Many Totoli families from Binontoan and Lakuan (Tolitoli) have moved to Gio since the 1990s, which resulted in a growth in the population of the village. As of 2019, 839 inhabitants live in 189 households in Gio \citep[7]{Statistics_TolUt_2019}. Although it was formerly a Dondo-speaking \il{Dondo} settlement, Dondo \il{Dondo} is now almost never heard there. Some of the original Dondo-speaking \il{Dondo} inhabitants still speak the language; their children, however, speak Totoli. It is now the only village where Totoli is the preferred language in almost all domains, and it is the only place where infants are reared almost exclusively in Totoli. Furthermore, it is now considered by other Totolis as the stronghold of the Totoli language, culture and music, and musicians from Gio practicing the verbal art of Lelegesan \citep{riesberg2019fern, bracks2022totoli} are frequently invited to perform in other Totoli-speaking villages, as well as in the city of Totoli.
The community has become increasingly aware of the endangered state of their language. Some young speakers have successfully promoted Totoli through short films and other content on social media and streaming platforms. Furthermore, the mayor of Binontoan village has established an improvised TV channel that primarily focuses on topics related to Totoli, such as music and festivities. The channel  features recordings of recent events captured on cellphones and other devices.


Totoli is primarily a spoken language and is rarely used in written communication, resulting in the absence of a standardized orthography. However, some community members occasionally write in Totoli on social media or cellphone messenger apps, using the orthography of the Indonesian language. In this book, examples from Totoli are presented in phonemic transcription.\is{sociolinguistic information on Totoli|)}

\section[Segmental Phonology]{Aspects of the segmental phonology of Totoli}\is{segmental phonology of Totoli|(}


As a necessary precursor to the subsequent chapters, I provide a brief description of the fundamental aspects of Totoli's segmental phonology relevant to this study. For a more detailed description, consult \citet{bracksphonol}. The main focus here  is on the \isi{phoneme inventory} (\sectref{subsection: Phneme inventory}), along with a brief commentary on \isi{phonotactics} and general patterns of \isi{word structure} (\sectref{sec:phonotactics}). Additionally, the topic of \isi{vowel length} and related processes in Totoli is explored in greater detail  (\sectref{subsubsection:Lateral /l/}), as it is pertinent to the ensuing exposition of Totoli's prosody and intonation.

\subsection{Phoneme inventory}
\label{subsection: Phneme inventory}

\is{phoneme inventory of Totoli|(}

The phoneme inventory of Totoli consists of 18 consonants and 5 vowels. Seven consonant phonemes have been introduced through loanwords, mainly from Indonesian and Arabic.



The consonant phonemes are shown in  \tabref{Consonant Chart}, with the 7 marginal phonemes  indicated in brackets.


\begin{table}
\caption{Consonant phonemes of Totoli}
\label{Consonant Chart}
\begin{tabular}{lccccccccc}
\lsptoprule
	& \multicolumn{2}{c}{bilabial} & \multicolumn{2}{c}{alveolar} & \multicolumn{2}{c}{palatal} & \multicolumn{2}{c}{velar} & glottal \\\midrule
	{stop} & p & b & t & d & {\textcolor{gray!90}{(ʧ)}}& {\textcolor{gray}{(ʤ)}} & k & g &  \\
	{fricative} &  &  & \multicolumn{2}{c}{s} & &  &  &  & {\textcolor{gray}{(h)}} \\
	{nasal} & \multicolumn{2}{c}{m} & \multicolumn{2}{c}{n} &  \multicolumn{2}{c}{{\textcolor{gray}{(ɲ)}}} & \multicolumn{2}{c}{ŋ} &  \\
	{lateral} &  &  & \multicolumn{2}{c}{l} &  & &  &  &  \\
	{trill} &  &  & \multicolumn{2}{c}{\textcolor{gray}{(r)}} &  &  & &  &  \\
	{approximant} & \multicolumn{2}{c}{\textcolor{gray}{(w)}} &  &  &\multicolumn{2}{c}{{\textcolor{gray}{(j)}}}   &   &  & \\
	\lspbottomrule
\end{tabular}
\end{table}





The vowel phonemes are shown in  \tabref{Simplified vowel chart}. 


\begin{table}
	\caption{Vowel phonemes of Totoli}
	\label{Simplified vowel chart}
	\begin{tabular}{lccc}
		\lsptoprule
		& {front} & {central} & {back} \\ 
		\midrule
		close & i &  & u \\ 
		mid & ɛ &  & ɔ \\ 
		open &  & a &  \\ 
		\lspbottomrule
	\end{tabular}
\end{table}


The degree of allophonic variation in  phoneme realization is generally limited, except for the phoneme /l/. The following section on vowel length contains a detailed explanation of this exception, as it is of  importance to this study.



\subsection{Vowel length}
\label{subsubsection:Lateral /l/}

\is{vowel length|(}

Totoli phonetically distinguishes between long and short vowels. Both can occur word-initially, word-medially and word-finally. Some lexical roots inherently involve a long vowel and other long vowels are the result of affixation.

\tabref{Morphophonological processes, resulting in long vowels} gives examples of long vowels in each position as they occur in roots and through affixation.

\begin{table}
	\caption{Bases with, or processes leading to long vowels}
	\label{Morphophonological processes, resulting in long vowels}
	\begin{tabular}{llll}
		\lsptoprule
		& \textsc{example} & \begin{tabular}[c]{@{}l@{}}\textsc{position of}\\  \textsc{long vowel}\end{tabular} & \begin{tabular}[c]{@{}l@{}}\textsc{translation/} \\ \textsc{gloss}\end{tabular} \\
		\midrule
		\begin{tabular}[c]{@{}l@{}}Lexical bases \\ with long vowel\end{tabular} &  \begin{tabular}[c]{@{}l@{}}tik\textbf{ɔɔ}\\ t\textbf{ii}ŋ\\ \textbf{ɛɛ}ŋ\end{tabular} &\begin{tabular}[c]{@{}l@{}}\#\_\_Vː\#\\ \#\_Vː\_\#\\ \#Vː\_\_\#\end{tabular} & \begin{tabular}[c]{@{}l@{}}`neck'\\  `hear'\\  `saliva'\end{tabular} \\
		\midrule
		\begin{tabular}[c]{@{}l@{}}Long vowels \\ through affixation\end{tabular} &  \begin{tabular}[c]{@{}l@{}}manaŋk\textbf{ii}\\ \\ kulɔb\textbf{aa}nai\\ \\ \textbf{ii}taanna\\\end{tabular} &\begin{tabular}[c]{@{}l@{}}\#\_\_Vː\#\\ \\ \#\_Vː\_\#\\ \\ \#Vː\_\_\#\end{tabular} & \begin{tabular}[c]{@{}l@{}}mɔN-saŋki-i\\ \hspace*{10pt} \textsc{av-}carry\textsc{-appl}\\  ku=lɔba-an =ai\\ \hspace*{10pt} 1s\textsc{.act}=inform\textsc{-appl=ven}\\ i-ita-an=na\\ \hspace*{10pt} \textsc{rls-}see-\textsc{uv}=3s.\textsc{gen}\end{tabular} \\
		\lspbottomrule		
	\end{tabular}
\end{table}




In addition to the above, final long vowels can occur through a process of \isi{compensatory lengthening}.  The lateral phoneme /l/ has three allophonic realizations: [l] after front vowels, [ɺ] after back vowels, and a length feature on the preceding vowel in word-final position.  \tabref{Allophones of lateral /l/} shows examples for the different allophonic realizations of /l/:


\begin{table}
		\caption{Allophones of lateral /l/\label{Allophones of lateral /l/}}
	\begin{tabular}{llll}
				\lsptoprule
		{[}l{]} after front vowels                                                                                      & \begin{tabular}[c]{@{}l@{}}lɛlɛan\\ siisiliɡna\end{tabular}        & \begin{tabular}[c]{@{}l@{}}{[}lɛle̞an{]}\\ {[}si:sɪlɪɡna{]}\end{tabular}              & \begin{tabular}[c]{@{}l@{}}‘bridge’\\ ‘looking at him/her’\end{tabular}  \\\addlinespace
		{[}ɺ{]} after back vowels                                                                                       & \begin{tabular}[c]{@{}l@{}}balɛ\\ nɔlumulas\\ tuutulu\end{tabular} & \begin{tabular}[c]{@{}l@{}}{[}baɺɛ{]}\\ {[}nɔɺumuɺas{]}\\ {[}tuːtuɺu{]}\end{tabular} & \begin{tabular}[c]{@{}l@{}}‘house’\\ `scatter'\\ `sleeping'\end{tabular} \\\addlinespace
		\begin{tabular}[c]{@{}l@{}}compensatory lengthening\\ in word-final position\\ CV{[}l{]}\# → CVː\#\end{tabular} & \begin{tabular}[c]{@{}l@{}}ampil\\ mɔnɔnʤɔl\end{tabular}           & \begin{tabular}[c]{@{}l@{}}{[}ampiː{]}\\ {[}mɔnɔnʤɔː{]}\end{tabular}                 & \begin{tabular}[c]{@{}l@{}}‘side/twin’\\ ‘regret’\end{tabular}          \\
		\lspbottomrule
	\end{tabular}
\end{table}



Evidence for analyzing final lengthening as an allophone of /l/ comes from the ``reappearance'' of [ɺ] or [l] when suffixes are added to such bases. Note that when clitics are added, no [ɺ] or [l] appears and the vowel remains lengthened. Three examples from the corpus are given in \REF{ex:mosumbo' ana}--\REF{ex:nosumboomo}.\largerpage

In  example \REF{ex:mosumbo' ana}, the base \textit{sumbɔ{\ü}} `life' occurs unsuffixed. The phoneme /l/ is realized in its word-final allophone as a length feature on the preceding vowel. 
In  example \REF{ex:nosumboomo}, the same base is followed by the enclitic \textit{=mɔ} `\textsc{cpl}', so /l/ is also realized as a length feature on the preceding vowel.
In example  \REF{ex:mosumbolangai}, the same base is followed by a suffix with initial vowel /a/ and hence the /l/ is realized as [ɺ]. 



\ea
\label{ex:mosumbo' ana}
\textit{[mɔsumb\textbf{ɔː} ana]}\\
\gll mɔ-sumbɔ{\ü} ana \\
\textsc{av-}live   \textsc{med} \\
\glt `(it is) alive'\hfill(lifestory\_RDA\_1.160)
\osflink{65a55033f2240f0b4d32e5cf}{mosumbo__ana_lifestory_RDA_1_extRDA.wav}
\ex
\label{ex:nosumboomo}
\textit{[nɔsumb\textbf{ɔː}mɔ]}\\
\gll nɔ-sumbɔ{\ü}=mɔ \\
\textsc{av.rls-}grow\textsc{=cpl} \\
\glt `alive' \hfill (monkey\_turtle.130)
\osflink{65a550331c92110a81abea87}{monkey_turtle_331620_332990.wav}

\newpage

\ex
\label{ex:mosumbolangai}
\textit{[nɔsumbɔ\textbf{ɺa}nɡai]}\\
\gll nɔ-sumbɔ{\ü}-an=ɡa=ai \\
\textsc{av.rls-}grow\textsc{-appl}\textsc{=?}\textsc{=ven} \\
\glt `they grow again' \hfill (tau\_bentee.033)
\osflink{65a5507d1cba3e0c180185cb}{tau_bentee_88026_89247.wav}
\z


The regular omission of final laterals in loanwords provides further support  for this analysis: Malay \textit{kapal} ‘ship’ > Totoli [\textit{kapaː}] .    
Throughout this work, the first line of examples gives a phonological representation of the examples. Hence, the lateral /l/ in final position is represented as /l/ but phonetically realized as a length feature.


\is{compensatory lengthening}

The case of word-final long vowels is important for the subsequent discussion of tonal patterns, presented in \chapref{sec:The IU in Totoli}. I show that right-edge boundary-tone complexes are usually associated with the final and the prefinal syllable of an IU. If the final syllable involves a long vowel, the tonal pattern is realized on the final long syllable exclusively. This is illustrated and discussed in \sectref{sec:phonetic-variability},  examples \REF{ex:ramean tau pomoo}  and  \REF{ex:geipo sallo pomoo}.



\is{vowel length|)}

\subsection[Phonotactics]{Phonotactics and word structure}
\label{sec:phonotactics}


Most words in Totoli follow a strict CV-pattern. Consonant clusters are rare with the exception of  \isi{homorganic nasal stop cluster}s, a common phenomenon in languages that otherwise exhibit a rather strict CV-structure \citep{Downing_2017, Downing, Reid_ProtoA_NC, riehl2008phonology, Herbert_markedness}. In Totoli, such sequences occur word-initially and word-medially but not in word-final position. Frequently they arise from a process commonly known as ``nasal substitution" \is{nasal substitution} in the Austronesianist \il{Austronesian} literature \citep{Blust2004, Joe.2004}. In the examples, ``nasal substitution" is represented by a capital N on the second line. Furthermore, Totoli makes use of geminates, \is{gemination} which occur word-initially and word-medially, but not in word-final position. Some lexical roots involve geminates but frequently result from \isi{reduction}  processes of \textit{C\textsubscript{x}V\textcolor{gray}{\textsubscript{y}}C\textsubscript{x}V\textcolor{gray}{\textsubscript{y}}} sequences whereby the first vowel is dropped, yielding  \textit{C\textsubscript{x}C\textsubscript{x}V\textcolor{gray}{\textsubscript{y}}}. Other heterorganic consonant sequences are very rare. Only  few lexical bases involve such consonant sequences. Across clitic-boundaries, however, they are allowed but are also very infrequent. 
Another major morphophonological process  in Totoli is \isi{vowel harmony} in prefixation. It is always regressive, being restricted to prefixes containing the vowel /o/ in their citation form, such as \textit{mɔN}-, \textit{nɔN}-, \textit{mɔ}-, \textit{nɔ}-, \textit{mɔɡ}-, \textit{nɔɡ}-, \textit{pɔ}-, \textit{pɔɡ}-, and \textit{kɔ}=. The vowels of these prefixes occur as /ɔ/ when they precede bases containing /ɔ/, /u/, or /i/ in their first syllable. However, when the first syllable of the base contains /ɛ/, the prefix vowel is realized as /ɛ/, and when it contains /a/, the prefix vowel is realized as /a/. 

Additionally, \isi{reduplication} is a common morphophonological process in both verbal and nominal morphology in Totoli. This process encompasses various forms, all of which are represented by a single label, \textsc{rdp}, in the glossing of examples.


The aforementioned discussion provides a concise overview of the fundamental aspects of segmental phonology necessary for comprehending the discussion on Totoli's prosody and intonation in Chapters~\ref{sec:Experiments} and \ref{sec:The IU in Totoli}. 






\is{segmental phonology of Totoli|)}

\section[Prosody of Austronesian languages]{Research on the prosody of Austronesian languages}
\label{Introduction}

\is{prosody of Austronesian languages|(}

Little is known so far about the prosody  of \ili{Austronesian} languages, a fact also acknowledged by \citeauthor{Himmelmann_Kaufmann_Prosody}'s (\citeyear{Himmelmann_Kaufmann_Prosody}) chapter on the state of the research on the prosody and intonation of Austronesian languages. \citet[348]{Himmelmann_Preliminary_2018}  proposes a model of the basic structure of the Intonation Unit (hereafter: IU) in \ili{Austronesian} languages of Indonesia and East Timor.
More thorough phonetic studies that have been conducted on \ili{Austronesian} languages of Indonesia in recent years suggest that many languages in the area may lack \isi{word-level prominence} and that tonal targets are primarily assigned at the phrase level. \is{phrase level prominence} \ili{Indonesian}/\ili{Malay}  as one of the major languages in the region has stirred debate about “stress” placement \is{stress placement} and its existence \citep[for a summary see][28--9]{goedemans2007stress}. For \ili{Indonesian}, as well as for many other \ili{Austronesian} languages, the position of word stress is often claimed to be on the penultimate syllable of a word. Analyzing this claim on the assumption that speakers of \ili{Indonesian} as a second language show a strong L1 influence, \citet[42]{goedemans2007stress} compared \ili{Indonesian} spoken by \ili{Toba Batak} speakers with that of \ili{Javanese} speakers. They found that \ili{Toba Batak} speakers produce the penultimate syllable of IU-final words in focus condition \is{focus} with higher intensity, longer duration and a rise in F0. Speakers of \ili{Indonesian} with \ili{Javanese} as their first language, however, produce the words in the same condition only with a rise in F0, whilst duration and intensity are not affected. They conclude that Indonesian spoken by Toba Batak speakers exhibits prominence on the level of the word as well as the phrase. For speakers with a Javanese background, however, they ``only found evidence for prominence at the phrase level (in the form of pitch movements)" \citep[45]{goedemans2007stress}. 
The results found for the Indonesian of Javanese L1 speakers are in fact similar to what has been reported about the \ili{Indonesian}/\ili{Malay} variety \ili{Ambonese Malay}, spoken in Eastern Indonesia on the Maluku Islands. Analyzing IU-final F0 movements in Ambon Malay, \citet[382]{Maskikit_Essed_2016}  found no association of the timing of IU-final boundary-tone complexes with any syllable. Moreover, focus condition did not reveal any systematic effect on the range and shape of the pitch on IU-final words. Hence the authors opt for analyzing IU-final tone complexes as floating boundary tones, since such an analysis assumes neither word stress nor \isi{pitch accent}, whether associated lexically \is{lexical pitch accent} or postlexically \is{postlexical pitch accent} \citep[356]{Maskikit_Essed_2016}. They conclude that IU-final boundary-tone complexes may instead signal the function of sentences.
The absence of word prosody and the assignment of tone complexes to boundaries of prosodic domains fit the characteristics of what \citet{Fery_2016}  labels Phrase Languages.



In addition to the studies of phonetic \isi{correlates of stress} in Austronesian languages, only a small number of analyses of the intonation of Austronesian languages exist: \citet{himmelmann2010notes} presents a description of the intonation of \ili{Waima’a}, spoken in East Timor; \citet{Maskikit_Essed_2016} describe the two most common IU-final pitch melodies of \ili{Ambonese Malay}; \citet{stoel2006intonation} proposes a concise description of the intonation of \ili{Banyumas Javanese}. These studies are based on a set of target phrases or question-answer pairs, the realization of which has been taken as generalizable over the intonational system of the language as a whole. Such an approach may be suitable for the description of the major aspects of the intonation of a language. However, the frequency distribution of patterns  and also less frequently used intonation patterns may only be observed in a corpus study, covering different communicative events. Possibly the only study conducted on the intonation of an Indonesian language which is primarily based on a corpus of spoken spontaneous discourse is that of \citet{stoel2005focus} on \ili{Manado Malay}.

\is{prosody of Austronesian languages|)}

\section{The units of spoken speech}
\label{The units of spoken speech}

\is{Intonation Unit boundary|(}
\is{Intonation Unit segmentation|(}
\is{Intonation Unit|(}

This research is based on the  analysis of a large corpus of speech. The first hurdle one faces when dealing with  corpora of (semi-)spontaneous speech is the identification and segmentation of the data into tangible units \citep{himmelmann2006challenges, Talking_data}. Speech can be segmented into various units of different sizes, though most studies recognize the \isi{Intonation Unit} (IU) \citep{chafe1994discourse} as the basic unit into which discourse and the flow of speech is structured.

The IU \is{Intonation Unit} has been discussed under a variety of other names such as the \isi{Tone Group} \citep{Halliday_1967}, the \isi{Tone Unit} \citep{crystal1969prosodic},  the \isi{Intonation/Intonational Phrase} \citep{selkirk1986, nespor1983prosodic, pierrehumbert1980phonology, Ladd_2008, 2021_Handbook_pros, jun2006, jun_2014}, and the \isi{Breath Group} \citep{lieberman1966intonation, lieberman1970articulatory}. Details of the definition of the various terms vary. Underlying this study is the basic definition of the IU by  \citet[22]{Chafe_1987}:





\begin{quote}
	An intonation unit is a sequence of words combined under a single, coherent intonation contour, usually preceded by a pause. 	
\end{quote} 


The coherent intonation contour is the defining characteristic of an Intonation Unit. A number of  features have been identified which contribute to the  perceived single, coherent intonation contour. Other features, on the other hand,  delimit a speech segment and indicate the boundary of an IU.  The criteria discussed pertain mainly to pitch, rhythm, and voice quality, but non-prosodic features have also been identified, such as the end of a turn\slash the change of speaker, inhalation, and lexical boundary markers  (\citealt[93--155]{Schuetze-Coburn1994}, \citealt[260--270]{himmelmann2006challenges}, \citealt{dubois1992, dubois1993}, and \citealt[29--39]{cruttenden1997intonation}).


\citet[52]{Tao_1996} mentions that \isi{discourse particle}s also proved a reliable criterion for the identification of IU boundaries in \ili{Mandarin Chinese}, as they correlate highly with IU boundaries.
Strictly speaking, however, prosodic clitics are a syntactic criterion and, as such, should not be  used to identify any prosodic unit (see the discussion in  \sectref{sec:the-syntax-of-intonation-units}). A single IU-boundary feature alone does not suffice to reliably detect an IU boundary, and hardly any IU exhibits all of the boundary cues:

\begin{quotation}
	The relative importance of the cues may differ --- pitch reset, for example, is arguably more central than tempo modulation --- but none alone defines an IU boundary per se; rather, a conjunction of cues is usually required for an IU to be perceived. One can say that the prototypical IU exhibits all of these cues, yet seldom are all actually present in any given instance. That is, most IUs deviate from the prototype to some degree. Thus, a given IU may exhibit pitch reset and a definite contour, but none of the other features. \citep[217]{Schuetze-Coburn1991}
\end{quotation}

Hence, the IU is defined in terms of a prototype and ``the more features that coalesce at any point, the stronger (‘more prototypical’) the boundary will be, but an IU boundary may also be perceived when only one or two features occur" \citep[227]{Schuetze-Coburn1991}.

\is{speech segmentation|(}

Many discourse-oriented linguists report on difficulties with identifying IU boundaries and comment on the sometimes tedious nature of the task. While \citet[46]{brown2015questions} report ``constant difficulty in identifying tone groups in spontaneous
speech'', a great many other linguists working with discourse data admit that, no matter how difficult the task, with ``practice and appropriate guidance, however, one
should be able to attain a reasonably high degree of inter-transcriber reliability'' \citep[112]{dubois1992}. On that matter, \citet{chafe1994discourse} comments that “in a better world they would be as important a part of the training of a linguist as the ability to transcribe vowels and consonants” (see also comments from  \citealt[165]{Schuetze-Coburn1994}, \citealt[206]{crystal1969prosodic}, and \citealt[29]{cruttenden1997intonation}). Based on his hands-on experience of working with various corpora, \citet[261]{himmelmann2006challenges} reports that an estimated proportion of 80-90\% of IUs are rather unproblematically identifiable, which also reflects my own experience. 


Despite the many remarks about the difficult nature of the transcription process, studies have shown that even naive, untrained speakers perform remarkably well in segmenting discourse.  \citet{Kreckel_1981}  conducted an experiment in which untrained, native \ili{English}-speaking, participants were presented with a written transcript and a corresponding (English) audio recording. Participants were asked to mark ‘message’ boundaries on the transcript.  The results showed that the participants segmented speech into Intonation Units (i.e. `tone groups') with a high degree of interrater agreement. Furthermore, participants gave priority to prosodic cues over syntactic ones. 

\is{speech segmentation|)}

In recent years, these findings have been confirmed by a number of studies using the \isi{Rapid Prosody Transcription} method  
\citep[RPT;][]{Cole2016}. Its boundary-marking task is  similar to the method used by \citet{Kreckel_1981}, and the results obtained from a number of typologically unrelated languages show a high degree of interrater agreement on the placement of boundaries. \is{boundary placement} In  \sectref{sec:discussion and comparison}, I give an overview on agreement results from different studies and compare these to  the  results obtained from an RPT  study of Totoli (cf.  \figref{Kappa comparison b}). However, results from the boundary-marking task obtained  from RPT \is{Rapid Prosody Transcription} experiments are usually not discussed as evidence for the perception of IU boundaries. Yet, in  \sectref{sec:the-length-of-intonation-units-of-the-corpus}, I correlate the results from the boundary-rating task of the RPT with IU boundaries which occur within rated speech segments. The results show that naive listeners can indeed reliably identify IU boundaries. While the universality of prosodic units below the IU are subject to debate \citep{Bickel_2009, schiering2010prosodic}, the IU is widely accepted.

The Intonation Unit as a discourse-structuring unit has been successfully employed by studies on a variety of typologically unrelated languages. If prosodic cues that delimit IUs are similar across languages,   then listeners should be able to identify IU boundaries even in languages they are not familiar with. In this regard,  \citet[174]{Ford_Thompson_1996} briefly commented that trained transcribers can reliably identify IU boundaries in an unknown language with a precision of 85-90\%. This observation has been put to the test only by  \citet{himmelmann_IP_Universal}, which investigated the inter-transcriber agreement of IU segmentation done by trained transcribers on familiar and unfamiliar languages from different language families (\ili{German}, \ili{Papuan Malay}, \ili{Wooi}, \ili{Yali}). The results showed statistically significant inter-transcriber agreement on the placement of IU boundaries, \is{boundary placement} which led the authors to postulate the \isi{Universal Phonetic IP Hypothesis} (UPIPH). The hypothesis claims

\begin{quotation}
	[...] that all natural languages make use of the same kinds of phonetic cues for IPs, and that these cues can be perceived by speaker-hearers even in unfamiliar languages. [...] We believe that it is quite likely that phonological IPs are part of the prosodic system of all natural languages. If this is the case, IPs would be a prime example of a universally attested \textit{phonological} category. \citep[239--240]{himmelmann_IP_Universal}
\end{quotation} 

The UPIPH is a strong claim and further data from different languages is needed to substantiate it. Furthermore, an investigation into the comparison of the various cues for IP boundaries may yield interesting cross-linguistic similarities and/or differences. However,  with supporting evidence from a variety of languages, it appears that all speakers organize their speech into Intonation Units, which are perceived as such by the listener. 

As will become evident from the  analysis of tonal patterns of segmented IUs of the corpus in  \sectref{IU-model}, we have to assume recursive \is{recursivity} embedding of Intonation Units into Compound Intonation Units in Totoli. While some segmentable stretches of speech of the corpus occur as simple, singleton IUs, others occur as  complex, Compound IUs, all of which are subsumed under the label CIU.
\is{Intonation Unit boundary|)}
\is{Intonation Unit segmentation|)}
\is{Intonation Unit|)}
