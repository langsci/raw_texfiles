\addchap{\lsAcknowledgementTitle} 




I am profoundly grateful and humbled by the many privileges I enjoy, the most significant of which is the opportunity to encounter numerous individuals who have, in diverse ways, made invaluable contributions to the realization of this book. I owe an immeasurable debt of gratitude to those who have shared their love and friendship, offered unwavering support, provided indispensable counsel and guidance, and so much more. Without them, this book would not have come to fruition. I am forever indebted to them all.


First and foremost, I extend my heartfelt gratitude to the people of Tolitoli who have generously opened their homes and hearts to me, often hosting me for  several months. 

%I am especially indebted to Datra Rusmin, Akram Hasan, and Winarno, for their friendship and for devoting countless hours to help me realize this project, whether by riding alongside me on motorcycles, transcribing data with me on the floor, or visiting  other community members for long hours of recordings.  I would also like to express my appreciation to Ima, whose company and conversations during our many motorcycle rides provided invaluable insights into reproductive health and family planning. Thank you for the many laughs we shared. I am deeply grateful to Ronal Noldi and Augustina Pali, who have always provided a welcoming home for me when I needed a getaway, and whose cups, coins and newspapers were always ready when I was ill. 
%
%
%During my stay in Totoli, several families opened their doors and welcomed me into their homes and offered me a bed. Bu `Aji' Hasnawati accepted me in her home  for many months and I am indebted to her for her kindness and hospitality and for her delicious \textit{Kola Durian}. In Binontoan, Pak Kades Taufiq Hidayat and Bu Samira welcomed me and provided me with invaluable support and even taught me how to take a proper shower. I am grateful for their generosity and hospitality. In Pinjan, Pak Ahmad Laterey and his wive Bu Bahira graciously allowed me to stay in their beautiful village, which remains one of the most memorable places I have visited. In Nalu, I had the privilege of being hosted by Datra Hasan, his mother Arnia Tahir, and his sister Sarni Hasan. I am thankful for their patience and hospitality and for teaching me the most memorable children's song \textit{Cicak Cicak di Dinding}. I also wish to acknowledge the invaluable contributions of many individuals who were involved in the Totoli language documentation project. It is impossible to name them all and all I can do is to thank those who have spend particularly many hours in front of the camera and the microphone. I would like to extend special thanks to Pak Rusmin DJ Sola, Pak Rusmin Datuamas, and Pak Rustam Laterey, who generously shared their wealth of knowledge of folk tales and other stories with me. Finally, I would like to express my deepest appreciation to Sumitro Pogi, who patiently sat down with me for uncountable hours of elicitation sessions and answered all my questions while correcting my Totoli grammar. Without his invaluable assistance, this research would not have been possible.

Pertama-tama, saya mengucapkan terima kasih yang tulus kepada masyarakat Tolitoli yang telah dengan murah hati membuka rumah dan hati mereka untuk saya, menjadi tuan rumah saya selama beberapa bulan. Saya sangat berterima kasih kepada Datra Rusmin, Akram Hasan, dan Winarno, atas persahabatan mereka dan atas waktu yang mereka luangkan untuk membantu saya mewujudkan proyek ini, baik dengan menemani saya berkendara motor, mengetik data bersama saya di lantai, atau mengunjungi anggota komunitas lain untuk merekam data dalam waktu yang lama. Saya juga ingin mengucapkan terima kasih kepada `Ima' Irma, yang bersama saya selama perjalanan berkendara motor, memberikan wawasan yang sangat berharga tentang kesehatan reproduksi dan hal-hal terkait dengan perencanaan keluarga. Terima kasih untuk banyak tawa yang kita bagi bersama. Saya sangat berterima kasih kepada Pak `Onal' Ronal Noldi dan Bu Augustina Pali, yang selalu memberikan rumah yang nyaman untuk saya ketika saya membutuhkan tempat istirahat, dan cangkir, koin, dan koran mereka selalu siap saat saya sakit.	

Selama tinggal di Totoli, beberapa keluarga membuka pintu mereka dan menyambut saya ke rumah mereka serta menawarkan tempat tidur. Bu `Aji' Hasnawati menerima saya di rumahnya selama beberapa bulan dan saya sangat berterima kasih atas kebaikan hati dan keramahannya serta kelezatan \textit{Kola Durian}-nya. Di Binontoan, Pak Kades Taufiq Hidayat dan Bu Kades Samira menyambut saya dan memberikan dukungan yang sangat berharga dan bahkan mengajari saya cara mandi yang benar. Saya bersyukur atas kemurahan hati dan keramahan mereka. Di Pinjan, Pak Ahmad Laterey dan istrinya Bu Bahira dengan ramah mengizinkan saya tinggal di desa mereka yang indah, yang tetap menjadi salah satu tempat yang paling berkesan yang pernah saya kunjungi. Di Nalu, saya berkesempatan tinggal di rumah Datra Hasan,  Bu Arnia Tahir, dan Ibu Sarni Hasan. Saya berterima kasih atas kesabaran dan keramahan mereka karena mengajari saya lagu anak-anak yang paling berkesan \textit{Cicak Cicak di Dinding}. Saya juga ingin mengakui kontribusi berharga dari banyak individu yang terlibat dalam proyek dokumentasi bahasa Totoli. Tidak mungkin untuk menyebutkan semua nama mereka, dan yang dapat saya lakukan hanyalah berterima kasih kepada mereka yang telah menghabiskan banyak waktu di depan kamera dan mikrofon. Saya ingin mengucapkan terima kasih khusus kepada Pak Rusmin DJ Sola, Pak Rusmin Datuamas, dan Pak Rustam Laterey, yang dengan murah hati berbagi pengetahuan mereka tentang dongeng dan cerita lain dengan saya. Terakhir, saya ingin mengungkapkan penghargaan mendalam kepada Pak Sumitro Pogi, yang dengan sabar duduk bersama saya selama sesi elicitation yang tak terhitung jumlahnya dan menjawab semua pertanyaan saya. Tanpa bantuan yang tak ternilai dari Sumitro Pogi, penelitian ini tidak akan mungkin dilakukan.






I would like to express my sincerest gratitude to my supervisor, Nikolaus P. Himmelmann, for his invaluable contribution to my academic growth. Without his unwavering confidence, guidance, and support, I would not be the linguist I am today. In 2017, Nikolaus offered me a PhD position in the Collaborative Research Center 1252 ``Prominence in Language" at the University of Cologne, where I submitted my doctoral dissertation in 2019, of which this book is a revised version. During my time there, I had the privilege of undertaking several months of data collection and language documentation with the Totoli community, an experience that has profoundly shaped me and will remain with me always. Thank you for your confidence in me and for your support!

In particular, I wish to thank Maria Bardají i Farré, who accompanied me  to Tolitoli. The laughter that we shared, while being laughed at, brought so much joy to my time in Tolitoli. Thank you.
Through their friendship and through the many talks over drinks Henrike Frye, Alessia Cassarà, Fahime Same and Aung Si helped me through some of the more challenging moments of my writing process. Furthermore, I am deeply grateful to my colleagues Sonja Riesberg, Volker Unterladstetter, and Kurt Malcher, as well as the numerous colleagues in and around the Collaborative Research Center ``Prominence in Language" and the various institutes at the University of Cologne. In particular, I want to thank my second supervisor Stefan Baumann and also Martine Grice for their invaluable comments, suggestions and support. 
Lastly, I would like to express my gratitude to the two anonymous referees and the editorial team of the Topics in Phonological Diversity series for their invaluable feedback and suggestions, which greatly enhanced the quality of this book. 