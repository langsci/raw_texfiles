\chapter{Summary and Discussion}
\label{Summary and Discussion}



The present study aimed to achieve two primary goals: Firstly, to provide a detailed discussion of the prosody and intonation of Totoli, adopting a comprehensive approach that combined experimental evidence with data obtained from an impressionistic analysis of a wide-ranging corpus of (semi-)spontaneous speech. Secondly, to explore the grammatical structures that are typically found in (Compound) Intonation Units, which include singleton IUs, embedded IUs of CIUs, and complex CIUs as a whole.


Regarding the first objective, I have demonstrated that prominence is not a relevant concept in the  prosody of Totoli, and that focus is not signaled by any prosodic cues. This was supported by evidence obtained from two experiments. The first experiment, which involved an RPT study, showed that naive native speakers generally could not agree on the location of prominences, suggesting that prominence may not be a significant category in Totoli's prosody. The results of the second experiment further supported this hypothesis, as no evidence was found that prosodic means marked focus as an information-structural category.

As \citet[376]{Himmelmann_Kaufmann_Prosody} have noted, narrow focus \is{narrow focus} on a sub-constituent of a clause or noun phrase in \il{Austronesian} languages is typically not signaled by intonation alone but rather by syntactic means. However, narrow focus on a constituent such as the subject NP or the object NP has only been investigated to a limited extent in \ili{Austronesian} languages \citep[cf. for example][]{Focus_Tagalog, Focus_Tagalog_Kaufmann, KALAND2023101200}.\il{Tagalog} \il{Papuan Malay}In this study, I have shown that in Totoli, syntactically equal SVO(O) constructions are not prosodically marked for focus when used as answers to questions that trigger focus on different constituents. This feature may be present in many other Austronesian languages, but additional data from a variety of \ili{Austronesian} languages are required to determine whether it is a common feature or limited to a specific subgroup of Austronesian languages.



In \sectref{IU-model}, I examined the tonal patterns of the entire corpus of \mbox{(semi-)}spontaneous speech, consisting of 2861 Intonation Units. Based on the analysis of the tonal specifications, I presented a model of the Compound Intonation Unit in Totoli. This model assumes either singleton IUs or complex Compound IUs (CIUs). In the former, the only pitch event is the IU-final boundary complex, which occurs on the final two syllables. In the latter, the CIU comprises a series of two or more IUs, each of which bears one of the three IU-final boundary-tone complexes.

My analysis showed that the Totoli prosody is better described by assuming recursive embedding of IUs into CIUs rather than  parsing of IUs into prosodic units at a level below the IU, as \citeauthor{Himmelmann_Preliminary_2018}'s (\citeyear[348]{Himmelmann_Preliminary_2018}) model of the IU in Austronesian languages of Indonesia and East Timor suggests. \citeauthor{Himmelmann_Preliminary_2018}'s model suggests that Intonation Units are further parsed into smaller prosodic units, such as intermediate phrases, and tonal patterns delimiting them consistently occur at the boundaries of major syntactic units. However, I found that the tonal patterns in my data are essentially the same, although with an inverted distribution. I have demonstrated that an embedded IU of a CIU differs substantially from a prosodic word or what is labeled as Accentual Phrase  \is{Accentual Phrase} in \ili{Korean} or \ili{French}. 


The absence of word prosody and the assignment of tone complexes to boundaries of prosodic domains fit the characteristics of what \citet[270]{Fery_2016}  labels Phrase Languages:

\begin{quotation}
	Phrase languages resemble intonation languages in that their tonal specifications are mostly assigned at the level of ɸ-phrases and ι-phrases. But contrary to intonation languages, specifications at the level of the word are sparse, absent or only weakly implemented. Phrase languages do not automatically associate pitch accents with stressed syllables, most tones are nonlexical (or ‘postlexical’). \citep[270]{Fery_2016}
\end{quotation}

\is{Phrase Languages}

\is{prosody of Austronesian languages}

In fact, many Austronesian languages may fall under the category Phrase Languages, following \citeauthor{Himmelmann_Preliminary_2018}'s (\citeyear[347]{Himmelmann_Preliminary_2018}) assertion:

\begin{quote}
	[...] it seems likely
	that prosodic prominence does not have a major role to play in marking information-structural
	categories. If at all, prosodic phrasing may be of relevance in this regard
	inasmuch as it is not determined by syntactic or processing constraints.
\end{quote}



Further evidence for recursive embedding of IUs into CIUs comes from an analysis of the  grammatical structures that IUs typically contain. I found that a small set of categories suffices to describe the majority  of their content. I compared the grammatical units typically occurring in embedded IUs with those that occur in singleton IUs which are not further segmented and I found that they are essentially similar, although again with varying proportions. 


\is{recursivity}


In sum, tonal patterns at the edges of singleton IUs and final IUs of CIUs are similar to those occurring at the right edge of non-final embedded IUs of CIUs. The syntactic structures they contain also occur as simple, singleton IUs which are not further chunked. In light of these results, I concluded that singleton IUs and embedded IUs of CIUs  are essentially of the same nature with the major difference being the presence or absence---i.e. the strength---of further typical boundary phenomena such as pitch reset, final lengthening, pauses and glottalization. A systematic analysis of boundary strength remains an object for future studies. Furthermore,  although the tonal events at the edges of IUs are the same, it might be the case that they vary with regard to tonal scaling. That is, tonal events at the right-edge boundary of a CIU may be essentially the same as at the right-edge boundary of embedded IUs within CIUs but  may vary in their tonal scaling \citep{Riad2018}. This is an aspect which I have not systematically investigated here, and which presents a promising avenue for further research.


\is{prosody of Austronesian languages}



This research also opens many other  questions. First, what does determine the choice of the final boundary tone of those IUs which are part of a CIU? Speakers are consistent in their choice of an IU-final boundary tone, and the grammatical unit contained in an IU appears to trigger the choice of boundary-tone complex. I suggested that the different patterns might be explicable by different degrees of integration, though the explanation for these different patterns requires further research. Second, what does trigger the realization of two grammatical units as either two separate IUs or a single complex one? Verbs followed by an NP often occur as a single IU. This is also observed with verbs followed by a PP, yet the tendency appears to be less strong. 
The analysis of the intonation of Totoli in \chapref{sec:The IU in Totoli}  focused on the tonal patterns exclusively. Further investigations are needed in order to correlate the different boundary-tone complexes with other acoustic phenomena such as, for example, duration, intensity and possibly spectral tilt, voice quality. A particularly fruitful approach may be the description of the intonation of Totoli with the attractor-based model that encompasses  categorical and also continuous components, but also  accounts for the variation of their frequency \citep{journalpone0216859, Roessig2021}. 

Little is known about the prosody and intonation of \ili{Austronesian} languages. The study presented here pertains to one language in the region, and many of the results may well apply to other languages in the area. This study represents one of the most comprehensive investigations into the prosody and intonation of any Austronesian language to date. Further research on other languages is necessary to relate the reported results and insights to other languages in the region, and to determine which features are specific to Totoli and which are common to the region or the language family as a whole.

\is{prosody of Austronesian languages}


